\section{Benchmark: \benchname}
% \subsection{Creating Challenging and Correct Perturbations}
% \chang{I think you need to write a paragraph about why do you choose these interventions and questions. This process is also reflected in your curation process. And most importantly, explain how you provide the answers and screen out bad quetions.}

% \chang{After this paragraph, ref to the dataset creation process stated in appendix. }

\subsection{Dataset Creation}

\benchname is created by generating question-answer pairs from biological pathway research papers, which are then checked and filtered through a combination of automated methods and expert human review. The dataset creation process involves prompting large language models, with GPT-4 and LLaMA3.1-405B \citep{dubey2024llama} being selected for data generation in this study.

To gather relevant biological pathway questions in realistic scientific research contexts, particularly those involving interventions, the data for \benchname is sourced from over 6,000 biological pathway research papers. These studies include carefully designed experimental interventions supported by pathway mechanisms to observe biological system responses. After extracting detailed experimental observations and their contexts, we convert each one into either a True/False or open-ended question, depending on its content. Each question is paired with corresponding labeled answers.

% Importantly, our focus is on the specific experimental phenomena observed and reported, rather than the final conclusions drawn by the researchers. This is essential for our goal of predicting detailed events. 

% which is used to support the paper's conclusion about the possible pathway mechanism. In other words, these phenomenon are logically correlated to the systems' pathway mechanism. 

% Our goal is to focus on the experimental phenomena presented in these papers and derive questions based on them. These questions pertain to observations made in the biological system after treatment on specific components, ranging from system function to individual component behavior, which is the key topic of interest in biological system research as well as many application domains such as drug discovery and medicine.

We then apply multiple data filters and human expert reviews to ensure the accuracy and quality of the questions. The accuracy of each question is validated by comparing it with the content of the original paper. Question quality is ensured through several filters that remove questions that are poorly defined, ask for specific measurement values, query more than one fact, are trivial (with answers revealed in the question's context), or are unrelated to biological pathways.

Finally, all questions are reviewed by human experts based on quality dimensions and their judgment to ensure overall question quality. The passing rate for expert review is approximately 40\%. After applying all filters, \benchname contains 5.1k high-quality questions. More details are provided in Appendix \ref{appendix: data creation}.

The questions of \benchname cover a wide range of biological domains, as illustrated in Figure \ref{fig bio fan} (left).

% The questions of \benchname cover a wide range of biological domains, including metabolism, genetic information processing, environmental information processing, cellular processes,  organismal systems, and human diseases. The biological domain distribution is illustrated in Figure \ref{fig bio fan} (left).

% The primary objective of \benchname is to generate realistic and practical biological system reasoning questions relevant to biological research, thereby bridging the gap in evaluating the potential of large language models for use in this field. The overall pipeline is depicted in Appendix \ref{dataset_pipeline}. The dataset creation pipeline involves prompting the large language model, and in this study, we have chosen LLaMa-3.1-405B \citep{dubey2024llama} and GPT-4 as the models for data creation.



% \chang{It would be better to describe the APIs for pathway graph here. The API is shared across not only your method but baselines. Could mention them as universal APIs that benefit various graph-reasoning methods. }



% Please add the following required packages to your document preamble:
% \usepackage{multirow}

% \begin{table}[!t]
% \caption{Task example for each category.}
% \label{table category example}
% \resizebox{\linewidth}{!}{
% \begin{tabular}{ccc}
% \toprule
% Dimension &	Category &	Example (abbreviated)\\
% \midrule
% \multirow{2}{*}{Inquiry Type}               & Normal     & What is the effect of \textcolor{blue}{AMPK activation} on SIRT1 activity in mouse skeletal muscle?                              \\
%                                       & Perturbed   & What is the effect of \textcolor{blue}{GogB-deficient Salmonella} on NFkappaB activation and proinflammatory responses in infected mice?             \\

% \midrule
% \multirow{2}{*}{Extra Condition} & Natural       & How does apelin affect TNFalpha inhibition on brown adipogenesis?                                              \\
%                                       & Intervened        & What is the role of BID in BAX activation in AIF-mediated necroptosis \textcolor{blue}{after MNNG treatment}?         \\

% \midrule
% \multirow{3}{*}{Investigation Target}               & Single       & What happens to \textcolor{blue}{AQP2} upon ADH stimulation?                                                                     \\
%                                       & Interaction & How does the influenza protein NS1 affect the \textcolor{blue}{activation of RIG-I by viral ssRNA}?                              \\
%                                       & Function    & What is the effect of losing 11beta-HSD2 from the fetus and fetally derived tissues on \textcolor{blue}{cerebellum development}? \\

% \bottomrule
% \end{tabular}}
% \vspace{-5mm}
% \end{table}

\subsection{Reasoning Type Categories}

\begin{table*}[!t]
\tiny % Change to \tiny for smaller text
\scriptsize
\resizebox{\linewidth}{!}{
\begin{tabular}{ccm{5cm}c}
\toprule
Dimension &	Category &	Example (abbreviated) & Illustration\\
\midrule
\multirow{4}{*}{Inquiry Type}               & Normal     & What is the effect of \textcolor{blue}{AMPK activation} on SIRT1 activity in mouse skeletal muscle?  &   \begin{minipage}{0.2\textwidth}
                   \includegraphics[width=\linewidth]{fig/task_illustration_new_1.pdf} % Adjust width and path
                \end{minipage}                           \\
                                      & Perturbed   & What is the effect of \textcolor{blue}{GogB-deficient Salmonella} on NFkappaB activation and proinflammatory responses in infected mice?    & \begin{minipage}{0.2\textwidth}
                   \includegraphics[width=\linewidth]{fig/task_illustration_new_2.pdf} % Adjust width and path
                \end{minipage}         \\

\midrule
\multirow{4}{*}{Extra Condition} & Natural       & How does apelin affect TNFalpha inhibition on brown adipogenesis?  & \begin{minipage}{0.2\textwidth}
                   \includegraphics[width=\linewidth]{fig/task_illustration_new_1.pdf} % Adjust width and path
                \end{minipage}                                            \\
                                      & Intervened        & What is the role of BID in BAX activation in AIF-mediated necroptosis \textcolor{blue}{after MNNG treatment}?  & \begin{minipage}{0.2\textwidth}
                   \includegraphics[width=\linewidth]{fig/task_illustration_new_3.pdf} % Adjust width and path
                \end{minipage}       \\ 

\midrule
\multirow{7}{*}{Investigation Target}               & Single       & What happens to \textcolor{blue}{AQP2} upon ADH stimulation? & \begin{minipage}{0.2\textwidth}
                   \includegraphics[width=\linewidth]{fig/task_illustration_new_1.pdf} % Adjust width and path
                \end{minipage}                                                                    \\
                                      & Interaction & How does the influenza protein NS1 affect the \textcolor{blue}{activation of RIG-I by viral ssRNA}? & \begin{minipage}{0.2\textwidth}
                   \includegraphics[width=\linewidth]{fig/task_illustration_new_4.pdf} % Adjust width and path
                \end{minipage}                             \\
                                      & Function    & What is the effect of losing 11beta-HSD2 from the fetus and fetally derived tissues on \textcolor{blue}{cerebellum development}? & \begin{minipage}{0.2\textwidth}
                   \includegraphics[width=\linewidth]{fig/task_illustration_new_5.pdf} % Adjust width and path
                \end{minipage} \\

\bottomrule
\end{tabular}}
\caption{Task example and causal illustration for each category.}
\label{table category example}
\vspace{-2mm}
\end{table*}



% \begin{table}[h!]
%     \centering
%     \begin{tabular}{|c|c|}
%         \hline
%         Text & \begin{minipage}{0.3\textwidth} % Adjust the width as needed
%                   \includegraphics[width=\linewidth]{fig/biopathway.pdf} % Adjust width and path
%                \end{minipage} \\
%         \hline
%         Another Cell & More Text \\
%         \hline
%     \end{tabular}
%     \caption{Table with an image in one cell}
%     \label{tab:table_with_image}
% \end{table}

% \begin{tabular}{|p{5cm}|p{5cm}|}
% \hline
% This is a very long text that will wrap automatically within the cell because of the specified width. & Another cell with wrapped text. \\
% \hline
% Another row with wrapped text. & More wrapped text. \\
% \hline
% \end{tabular}

To study various research scenarios in biological pathways, such as natural dynamic changes, disturbances and interventions, and additional intervention conditions, as well as a multi-scale understanding of single factors, action processes, and macroscopic functions, we classify BioMaze tasks from three dimensions, namely inquiry type, extra condition, and investigation target, as shown in Table \ref{table category example}. More full question cases are in Appendix \ref{appendix data cases}. The distribution of the three dimensions' questions is shown in Figure \ref{fig bio fan} (right). We introduce each category of the dimensions below:

\begin{figure*}[!t]
    \centering    
    \renewcommand{\thesubfigure}{} % Hide subfigure labels
    \subfigure[]{\includegraphics[width=0.36\linewidth]{fig/plot_fan_bio_bi.pdf}} 
    \subfigure[]{\includegraphics[width=0.36\linewidth]{fig/plot_fan_reason.pdf}} 
    \vspace{-2mm}
    \caption{Dataset biological domain and reasoning type distribution. Left: \benchname covers six main domains: metabolism, genetic information processing, environmental information processing, cellular processes, organismal systems, and human diseases. Right: \benchname is categorized along three dimensions of reasoning types: inquiry type, extra condition, and investigation target.}
    \label{fig bio fan}
    \vspace{-2mm}
\end{figure*}

% the following contents:

% \textbf{(Definition)} A formal definition of the prediction problem.\\
% \textbf{(Impact)} The impact of improving performance on this problem.\\
% \textbf{(Challenge)} The type of understanding and capacity desired for this category of task.

\textbf{Dimension 1: Inquiry Type} is the independent variable studied, which can be either \textbf{Normal Source}, involving the prediction of the effects of natural components in their normal state within a biological pathway, or \textbf{Perturbed Source}, which deals with predicting the effects of external interventions or treatments—such as mutations, infections, or experimentally introduced elements—on downstream targets within pathways. Normal Source tasks focus on understanding the fundamental mechanisms and natural dynamics of pathways, while Perturbed Source tasks examine the phenomenon under perturbation.

% 

% The goal is to evaluate how well LLMs can comprehend and explain typical biological pathway functions.


% Tasks emphasize reasoning about how these interventions alter pathway functions. This mirrors real-world biological research, where the focus is often on understanding how such interventions influence biological systems and their downstream targets.

% the potential of LLMs in , such as experimental design and drug discovery, 

% (Definition) \\ 
% (Impact) \\
% (Challenge) 


\textbf{Dimension 2: Extra Condition} refers to additional settings besides the independent variable. This could be the \textbf{Natural Condition}, where no additional treatments are applied, and the pathway operates under the organism's natural conditions, or the \textbf{Intervened Condition}, which assesses the impact of the inquiry source when the pathway has already been influenced by other factors, such as mutations or interventions. The Intervened Condition challenges the model by requiring it to deduce the system's behavior under unnatural conditions, thus increasing the reasoning difficulty.




% focusing on how these conditions alter the pathway. For example, the question in Table \ref{table category example} examines BID's role after MNNG treatment, where the pathway differs from its natural state. Enhancing performance here is crucial for modeling complex biological scenarios, such as predicting treatment outcomes and drug interactions, as it shows how multiple factors interact within a system.

% meaning  to the biological system beyond the inquiry source. For example, in Table \ref{table category example}, the natural condition question asks about the mechanism through which apelin affects TNF-alpha inhibition in brown adipogenesis, with no extra interventions present in the pathway.

% (Definition) 

% \chang{Too vague here, what is a natural condition ?I don't understand the source is the sole treatment.}\\
% (Impact) Enhancing performance in this category helps clarify the fundamental roles of biological components in their native environments and provides a baseline for understanding how interventions influence natural system behavior.\\
% (Challenge) Tasks in this category aim to establish a foundational understanding of biological responses to natural conditions, facilitating the application of insights to various biological scenarios and aiding in the development of general principles applicable across different systems.



% (Impact) \\
% (Challenge) Tasks in this category emphasize the interplay of various factors within biological systems, requiring a nuanced understanding and deductive reasoning capacity.


\textbf{Dimension 3: Investigation Target} refers to the dependent variable in the question, which could be \textbf{Single Component as Target}, focusing on the effect of the source on a specific component within the pathway; \textbf{Components Interaction as Target}, examining the effect of the source on interactions between components within the pathway; or \textbf{Function as Target}, evaluating the effect of the source on broader biological functions or macro-level phenomena. The multi-scale targets address the reasoning of single components, downstream processes, or organism-wide outcomes.


% It may involve understanding how  with each other or their roles in regulating pathway processes. For example, the question of this category in Table \ref{table category example} queries influenza protein NS1's effect on the downstream process that viral ssRNA activates RIG-I.

% It addresses more comprehensive system behaviors, helping to link pathway-level changes with organism-wide outcomes, which are crucial for scenarios like understanding health and disease processes. 

% In summary, the inquiry type and extra condition dimensions indicate whether interventions are applied to the biological system. Cases with interventions, compared to natural conditions, better test LLMs' causal reasoning in biological pathways. The investigation target dimension distinguishes different phenomena within the system, providing a more thorough evaluation of LLMs' ability to understand and predict pathway behavior.



% This category often focuses on the most observable phenomena in biological research, 

% (Definition) \\
% (Impact) Improving performance in this category enhances the precision with which LLMs can predict the behavior of individual components.\\
% (Challenge) Tasks in this category emphasize a deep understanding or deductive reasoning ability of individual component behavior within the biological system.


% (Definition) \\
% (Impact) Enhancing performance in this category allows LLMs to understand and reason for biological systems' processes and mechanisms, especially after unnatural interventions happen, which could be helpful in research like disease mechanisms or treatment design. \\
% (Challenge) Tasks in this category require understanding and deductive reasoning ability of biological system mechanism.

% Improved understanding of interactions can inform drug combinations and therapeutic strategies that consider the multifaceted nature of biological systems.


% (Definition) The task assesses the source's effect on macro-level phenomena or biological functions.\\
% (Impact) Strengthening performance in this category reveals how individual components and interactions contribute to overarching biological functions, as well as predicting function change with intervention to the biological system, which is the most accessible phenomenon in biological research and is critical for understanding system-wide behaviors in health and disease. \\
% (Challenge) Tasks in this category foster a holistic comprehension of biological systems and the reasoning ability for the deduction of the whole system.

% Component Formation Assessment
% Component Interaction Assessment
% Component Influence on Biological Function Assessment
% Component Influence on Other Component Assessment
% Intervention Influence on Biological Function Assessment
% Intervention Influence on Component Assessment
% Interaction between Interventions

% \begin{figure*}[!t]
%     \centering    
%     \renewcommand{\thesubfigure}{} % Hide subfigure labels
%     \subfigure[]{\includegraphics[width=0.95\linewidth]{fig/data_category.png}}\\
%     \subfigure[]{\includegraphics[width=0.95\linewidth]{fig/dataset_component.png}}
%     \vspace{-7mm}
%     \caption{\benchname reasoning categories. (\todo{Illustrate the categories better.})}
%     \label{category}
%     \vspace{-5mm} 
% \end{figure*}


% \begin{figure*}[!t]
%     \centering    
%     \renewcommand{\thesubfigure}{} % Hide subfigure labels
%     \subfigure[]{\includegraphics[width=0.95\linewidth]{fig/pathway.png}}
%     \vspace{-7mm}
%     \caption{Pathway understanding and reasoning categories illustration. (\todo{plot each reasoning category in this figure together.})}
%     \label{pothway_reasoning_type_illustrate}
%     \vspace{-5mm} 
% \end{figure*}


% The reasoning difficulty increases in order, as shown in Figure \ref{pothway_reasoning_type_illustrate}. For example, \textit{Component Formation Assessment} and \textit{Component Interaction Assessment} are basic steps in the biological pathway, addressing the synthesis of a component and the direct interaction between components in biological systems. The categories \textit{Component Influence on Biological Function Assessment} and \textit{Component Influence on Other Component Assessment} require reasoning according to the biological pathway.
% Moreover, the categories \textit{Intervention Influence on Biological Function Assessment}, \textit{Intervention Influence on Component Assessment}, \textit{and Interaction between Interventions} involve more challenging questions where additional interventions are added to the biological system. The reasoning of the effect on the system relies on a deep understanding of the biological pathway system and reasoning ability based on the pathway network.





% The benchmark \benchname targets evaluating the reasoning ability of language models for biological pathways. These tasks rely on step-by-step reasoning regarding biological pathway relationships to draw the final conclusion. 



\subsection{Pathway Graph Augmentated Reasoning}

% \outline{Why use pathway for biological system reasoning}


% in this work, we also evaluate LLM's reasoning ability with pathway-graph-augmented reasoning methods, such as ToG, CoK, and G-Retriever, by explicitly providing pathway graph data, which can intuitively enhance reasoning abilities for biological pathways. We formalize this problem as follows:

% Text-only reasoning methods like CoT directly generate the reasoning steps by LLMs based on the question. To perform reasoning for biological pathways, LLMs are required to not only have a complete implicit pathway map of the biological system but also could utilize it for planning reasoning steps and conducting deductive reasoning for the events inside the pathway. The utilization of the implicit pathway map could result in difficulty. Therefore, in this work we also evaluate graph-augmented reasoning methods such as ToG, CoK, and G-Retriever by providing pathway graph data explicitly, which intuitively could enhance reasoning ability for biological pathways. We formalize this problem as follows:

% Since it is unclear whether large language models (LLMs) inherently possess an implicit understanding of biological systems, it is a natural hypothesis that providing them with access to pathway graphs could enhance their reasoning about such systems. In this work, we explore the question: Do large language models need pathway model guidance to reason effectively about complex biological systems? Intuitively, giving LLMs access to biological pathway models could enable them to consult and integrate these models when needed. 



% Text-only reasoning methods like CoT generate reasoning steps directly from LLMs based on the given question. However, due to the inherent graph-data nature of biological pathways, LLMs must not only have a comprehensive implicit map of these pathways but also be able to use it effectively to plan reasoning steps and conduct complex reasoning. This can present challenges for LLMs when reasoning about biological systems.

% In this work, we explore the following question: \textit{Do large language models require pathway graph data augmentation to reason effectively about biological systems?} Providing LLMs with access to explicit biological pathway graphs could intuitively serve as a structural blueprint, enhancing their reasoning abilities from both a knowledge and reasoning perspective. We formalize this problem as:

% This work addresses the question: Do large language models require pathway graph data augmentation to reason effectively about biological systems? 

% This requires LLMs to not only have an implicit understanding of these pathways but also to effectively plan and execute complex reasoning steps.

Text-only reasoning methods, such as Chain-of-Thought (CoT), generate reasoning based on the inherent knowledge of LLMs. However, biological pathways present unique challenges due to their graph-structured nature. Consequently, using graph-augmented LLM reasoning is a natural approach for BioMaze. Providing explicit pathway graphs can serve as structural blueprints, enhancing reasoning from both knowledge and planning perspectives. We formalize this problem as follows:
\begin{equation}
\begin{aligned}
a = G (\mathcal{E}, o),
\end{aligned}
\end{equation}
where $G$ represents the language model, $\mathcal{E}$ denotes the task instruction (including the question), $o$ refers to the observation from the augment pathway graph database, and $a$ is the model output which could be the reasoning and answer.

% \chao{What are the outputs? Can we add a few examples for the outputs?}

% \chang{Also move up to the benchmark section}

\textbf{Pathway Graph Database} To augment LLMs with reasoning in biological pathways, we created a pathway graph database based on KEGG \citep{kanehisa2000kegg}, a collection of pathway maps on metabolism and various cellular and organismal functions widely-used resource among biologists. {We compiled all available pathway networks and maps from KEGG and integrated all of them into a single pathway graph database}. The statistics for the pathways are in Appendix \ref{appendix pathway statistic}. Each entry in the dataset is provided with a detailed description and function corpus. The graph is structured in triples as [\text{Head IDs}, \text{Tail IDs}, \textcolor{gray}{(\text{Relation Type}, \text{Biological Process IDs})}].


% While KEGG pathways may not be as large as some large-scale biological knowledge graphs, such as ProteinKG25 \citep{zhang2022ontoprotein}, they remain high-quality and the most exhaustive resource focused on biological pathways.

% where the 

% \textbf{Pathway Graph Database API} When the language model accesses the pathway database, it may need to retrieve relevant triples from the pathway graph using APIs like $\operatorname{Search\_Node}$, $\operatorname{Search\_Edge}$, $\operatorname{Search\_Triple}$, and $\operatorname{Search\_Subgraph}$ \citep{sun2023think,li2023chain}. Our pathway database supports these core retrieval APIs based on detailed descriptions and functional corpora. These APIs are essential for enabling various graph-augmented reasoning methods in LLMs. After retrieving pathways from the database, it is encoded into text by $o=\operatorname{TripleToText}(\operatorname{DFSOrder}(S))$. The details of the APIs implementation are described in Appendix \ref{appendix subgraph retriever}. 


\textbf{Pathway Graph Database API:} When the language model accesses the pathway database, it may need to retrieve relevant triples from the pathway graph using APIs like $\operatorname{Search\_Node}$, $\operatorname{Search\_Edge}$, $\operatorname{Search\_Triple}$, and $\operatorname{Search\_Subgraph}$ \citep{sun2023think,li2023chain}. Our pathway database supports these core retrieval APIs, which are based on detailed descriptions and functional corpora. These APIs are essential for enabling various graph-augmented reasoning methods in LLMs. When the pathway graph $S$ is to be input to the LLM, they are encoded into text $o$ by $o=\operatorname{TripleToText}(\operatorname{DFSOrder}(S))$. The details of the API implementations are described in Appendix \ref{appendix subgraph retriever}.


\section{Method: Pathway Reasoning Agent \modelname}

\begin{figure*}[!t]
    \centering    
    \renewcommand{\thesubfigure}{} % Hide subfigure labels
    \subfigure[]{\includegraphics[width=0.9\linewidth]{fig/method.pdf}}
    \vspace{-2mm}
    \caption{\modelname allows interactive browsing of the pathway graph database by navigating through subgraphs. At each step, \modelname can perform either a global subgraph search or a local search around a previously explored pathway step. This functionality enables \modelname to fully leverage the augmented pathway graph database during biological pathway reasoning.}
    \label{fig agent method}
    \vspace{-2mm}
\end{figure*}

As we evaluated several graph-augmented reasoning methods, we found that current graph-augmentation methods' performance is limited by their ineffective utilization of the pathway graph database for reasoning. Inspired by how scientists browse pathway networks during reasoning, we propose \modelname, a reasoning agent method that can interactively conduct reasoning and take actions to perceive and navigate pathways using a web-like engine, along with flexible reasoning in each step, as shown in Figure \ref{fig agent method}.

% \modelname allows the language agent to flexibly explore a vast graph database by observing subgraphs at each step, 

% We evaluate various graph-augmented large language model methods and demonstrate the effectiveness of \modelname in biological pathway reasoning.

% The task of a language agent can be succinctly modeled as a Partially Observable Markov Decision Process (POMDP), which is defined by a tuple $\langle{\mathcal{S}, \mathcal{O}, \mathcal{A}, \mathcal{T}, \mathcal{R}\rangle}$, with $\mathcal{S}$ representing the set of all possible states, $\mathcal{O}$ being the observation space through which the agent perceives the state, $\mathcal{A}$ denoting the action space, $\mathcal{T}: \mathcal{S} \times \mathcal{A} \rightarrow \mathcal{S}$ being the state transition function, and $\mathcal{R}: \mathcal{S} \times \mathcal{A} \rightarrow \{0,1\}$ being the reward function. 

% At step $t$, the language agent $G$ takes an action step $a_t$ based on policy $\pi$ (policy prompt), problem $\mathcal{E}$ (problem instructions), and previous observation-action trajectory $h_t=[o_1,a_1, \dots, o_{t-1}, a_{t-1}, o_t]$, 
% \begin{equation}
%   a_t =G(\pi, \mathcal{E}, h_t)  
% \end{equation}
% The agent aims to maximize the final reward $r$. In this work, we use an outcome-based reward mechanism (ORM) \citep{uesato2022solving} that assigns a binary indicator as a reward, depending on whether the task has been successfully completed.


At each step, $t$, the language agent $G$ can conduct reasoning by natural language thought, and takes an action step $a_t$, based on problem $\mathcal{E}$ (problem instructions) and previous observation-action trajectory $h_t=[o_1,a_1, \dots, o_{t-1}, a_{t-1}, o_t]$, 
\begin{equation}
  a_t =G(\mathcal{E}, h_t)  
\end{equation}

\textbf{Global and Local Subgraph Navigation} In addition to the global subgraph retriever $\operatorname{Search\_Subgraph}$, \modelname has access to an additional neighbor subgraph retriever, $\operatorname{Neighbor\_Subgraph}(\texttt{line\_id}, \texttt{query}, N)$, which retrieves an optimal connected subgraph of target size from the multi-hop neighbors of a previously observed pathway step $\texttt{line\_id}$.
\begin{equation}
\small
\begin{aligned}
\operatorname{Neighbor\_Subgraph}&(\texttt{line\_id}, \texttt{query}, N)= \\\underset{\substack{S \subseteq P_{id}, S \text { is connected }, |S|=N}}{\operatorname{argmax}} &\sum_{i \in V_S \cup E_S} \operatorname{score}(i, \texttt{query})
\end{aligned}
\end{equation}
Here, $P_{id}$ represents the multi-hop neighbors of the triple with $\texttt{line\_id}$. This allows \modelname to navigate the pathway graph database by either performing a global search or by exploring the multi-hop neighbors of an observed subgraph at each step. See Appendix for case \ref{appendix agent case}.

\textbf{Graph Encoding} In step $t$, the action taken by LLM agent get subgraph $S_t$ from environment, and the subgraph is encoded into text observation $o_t$ as following:
\begin{equation}
\small
\begin{aligned}
\hat{S}_t&=\operatorname{DFSOrder}\left(\operatorname{RemoveSeen}(S_t,[S_1, \dots, S_{t-1}])\right) \\
o_t&=\operatorname{TripleToOrderedText}\left(\hat{S}_t, \operatorname{TotalNum}\left([S_1, \dots, S_{t-1}]\right)\right)
\end{aligned}
\end{equation}
Function $\operatorname{RemoveSeen}$ eliminates triples from the $t$-th turn's subgraph that have been observed in previous turns, ensuring that each triple appears in the LLM's observations only once when first retrieved. This approach enhances content length efficiency and encourages the LLM to understand the whole navigation history rather than focusing solely on the most recent turn.

The function $\operatorname{TripleToOrderedText}$ convert ordered subgraph $\hat{S}_t$ into text in the following format: 
$ \text{Line ID)}  \text{ Head } |  \text{ Tail } | \text{ Relation and Biological Process}$. 
These global line IDs indicate the order of each triple across all turns, providing a unique reference for the LLM agent during local searches or reasoning. For the $t$-th turn's subgraph $S_t$, the ID starts at the total number of unique triples seen in previous history, given by $\operatorname{TotalNum}([S_1, \dots, S_{t-1}])$. 




% \subsection{Pathway Environment API}

% \chang{move this paragraph to the benchmark section}

% To understand and accomplish reasoning tasks for pathways, we enable the language agent to navigate the provided interactive pathway environment. The action space, denoted as $\mathcal{A}$, encompasses both natural language and the pathway browser's API. Specifically, the following API allows the agent to effectively navigate the pathway:



% $search\_biopathway\_triple\_N\_hop\_subgraph(history\_line\_id, key\_words\_list, target\_size=16)$: Similar to $search\_biopathway\_subgraph\_global$, this API also employs bi-level optimization to retrieve a pathway graph, but only within the neighborhood of `$history\_line\_id$` which is the line id of seen pathway. This API allows the language model to explore the pathway locally, focusing on the neighborhood of the component of interest.


% \subsection{Graph Encoding}
% To efficiently encode the subgraph for LLM agents, we propose a text encoding method for subgraph navigation agent, formalized as follows:

% \begin{equation}
% \begin{aligned}
% \hat{S^*}_t&=\operatorname{DFSOrder}(\operatorname{RemoveSeen}(S^*_t,[S^*_1, \dots, S^*_{t-1}])) \\
% o_t&=\operatorname{TripleToText}(\hat{S^*}_t, \operatorname{TotalNum}([S^*_1, \dots, S^*_{t-1}]))
% \end{aligned}
% \end{equation}

% We describe each encoding function below.

% $\operatorname{RemoveSeen}$: This function eliminates triples from the $t$-th turn's subgraph that have been observed in previous turns, ensuring that each triple appears in the LLM's observations only once when first retrieved. This approach enhances content length efficiency and encourages the LLM to understand the whole navigation history rather than focusing solely on the most recent turn.

% $\operatorname{DFSOrder}$: This function arranges triples in a depth-first search order. Compared to other ordering methods, such as relevance scoring, DFS order more effectively mirrors the reasoning process through the subgraph. For biological pathway tasks, this ordering naturally reflects the functional processes of biological pathways.

% $\operatorname{TripleToText}$: The ordered subgraph $\hat{S^*}_t$ is converted into text format for the LLM by encoding each triple as a string in the following format: 
% $$ \text{Line ID)}  \text{ Head } |  \text{ Tail } | \text{ Relation and Biological Process}$$ 
% These global line IDs indicate the order of each triple across all turns, providing a unique reference for the LLM agent during local searches or reasoning. For the $t$-th turn's subgraph $S^*_t$, the ID starts at the total number of triples seen in previous history, given by $\operatorname{TotalNum}([S^*_1, \dots, S^*_{t-1}])$. 

% The proposed graph encoding method offers an efficient feedback mechanism for the navigation agent, enabling it to access the entire history during navigation in a way that facilitates reasoning and planning. The encoding process is illustrated in Figure \ref{}, with additional examples provided in Appendix \ref{}. We also perform ablation studies on the encoding method in Subsection \ref{}.

\textbf{Final Reasoning} As graph data browsing finishes, the final reasoning is conducted based on all the navigation history:
$$a_r =G(\mathcal{E}_r, [o_1, \dots, o_{T}])$$

\textbf{Graph Navigation Capacity}
The combination of global and local subgraph retrieval APIs empowers LLM agents to explore the entire network flexibly and efficiently. It allows the LLM to guide its exploration by adjusting both keywords and the root of the local subgraph, depending on the intermediate reasoning, offering stronger expressiveness than navigation methods like BFS, DFS, and various retrieval methods.

% The LLM agent can extract keywords from the original question, or the proposed intermediate reasoning process and final result to verify hypotheses. Intuitively, the agent can explore the whole graph database with the capacity to simulate any navigation order. We theoretically prove the navigation capacity of the entire graph in Theorem \ref{}.

