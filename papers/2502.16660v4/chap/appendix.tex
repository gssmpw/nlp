
\subsection{Dataset Case Illustration} \label{appendix data cases}



\textbf{Dimension 1: Inquiry Type}

\textbf{Category 1: Normal Source} 

Question: In the context of YopT-induced cleavage of Rho GTPases, is carboxyl methylation a necessary post-translational modification for the cysteine protease activity of YopT to occur?


Answer: No

Question: In U2OS-hGR osteosarcoma cells, does the glucocorticoid receptor (GR)-dependent mRNA expression of ladinin 1 require the presence of both Mediator subunits MED1 and MED14 for transcriptional regulation in response to dexamethasone treatment?

Answer: Yes

Question: In pancreatic acinar cells, How does the sensitivity to nicotinic acid adenine dinucleotide phosphate (NAADP) compare to the sensitivity to cyclic adenosine 5'-diphosphate-ribose (cADPR) and inositol trisphosphate (InsP3) in terms of triggering Ca2+ release?

Answer: Pancreatic acinar cells are significantly more sensitive to NAADP than to either cyclic adenosine 5'-diphosphate-ribose (cADPR) or inositol trisphosphate (InsP3).

Question: In the context of Mycobacterium tuberculosis signal transduction, what is the effect of TrcS-phosphate and calcium ions (Ca2+) on the transphosphorylation of the response regulator protein TrcR?

Answer: TrcS-phosphate and Ca2+ are required for the transphosphorylation of TrcR.

\textbf{Category 2: Perturbed Source} 

Question: In the context of human papillomavirus type 16 (HPV16) E7 gene product's role in cellular transformation, can a transcriptionally inactive c-Jun deletion mutant that retains E7 binding capability interfere with the E7-induced transformation of rat embryo fibroblasts when co-expressed with an activated ras oncogene?

Answer: Yes

Question: In HTLV-1-infected T-cell lines, does the application of short interfering RNA (siRNA) targeting JunD result in an increase in matrix metalloproteinase-7 (MMP-7) mRNA expression?

Answer: No


Question: In the context of opioid withdrawal, how does blocking A1-adenosine receptors with 8-cyclopentyl-1, 3-dipropylxantine affect the response to mu-opioid receptor activation on the amplitude of GABAB-mediated IPSPs in slices taken from morphine-treated guinea pigs?

Answer: Blocking A1-adenosine receptors with 8-cyclopentyl-1, 3-dipropylxantine allows mu-opioid receptor activation to inhibit the IPSP at all concentrations and increases the maximal inhibition in withdrawn slices.

Question: In 3T3-L1 adipocytes, What is the effect of platelet-derived growth factor treatment on the tyrosine phosphorylation of APS?

Answer: Platelet-derived growth factor treatment results in no APS phosphorylation.

\textbf{Dimension 2: Extra Condition}

\textbf{Category 1: Natural Condition}

Question: In the context of Kaposi's sarcoma-associated herpesvirus (KSHV) infection, can viral interleukin-6 (vIL-6) induce the up-regulation of DNA methyltransferase 1 (DNMT1) in endothelial cells independently of signal transducer and activator of transcription 3 (STAT3) activation?

Answer: No

Question: In BJAB cells, does the expression of the K5 protein, encoded by the Kaposi's sarcoma-associated herpesvirus (KSHV), lead to a reduction in the surface expression of intercellular adhesion molecule 1 (ICAM-1)?

Answer: Yes

Question: In human hepatoma HuH-7 cells, What is the effect of transient expression of the Hepatitis C virus (HCV) core protein on the trans-activation ability of the cellular putative RNA helicase CAP-Rf?

Answer: Transient expression of HCV core protein in human hepatoma HuH-7 cells potentiates the trans-activation effect of CAP-Rf on gene expression.

Question: In Caenorhabditis elegans, What is the effect of being heterozygous for the cet-1 gene on body length compared to wild-type and null mutant individuals?

Answer: Heterozygotes for cet-1 display body lengths ranging between null mutant and wild type.

\textbf{Category 2: Intervened Condition} 

Question: In the Edar signalling pathway, can NF-kappaB activation still be induced by Edar-associated death domain (Edaradd) in the presence of dominant negative forms of TNF-receptor-associated factor 6 (TRAF6)?

Answer: No

Question: In mice treated with the angiotensin-converting enzyme inhibitor ramipril, do those lacking the specific PACAP receptor (PAC1-/-) exhibit lower plasma renin concentrations compared to their wild-type littermates?

Answer: Yes


Question: In human lung cancer cell lines with methylated retinoic acid receptor-beta (RARbeta) P2 promoter, What is the effect of demethylation by 5-aza-2'-deoxycytidine on the expression of RARbeta2 and RARbeta4 isoforms?

Answer: Demethylation by exposure to 5-aza-2'-deoxycytidine restores the expression of RARbeta2 and RARbeta4 in methylated tumor lines.

Question: In the context of hematopoietic progenitor cell proliferation, What is the effect of Flt3 ligand (FL) on progenitor cells from mice deficient in the signal transducer and activator of transcription 5a (Stat5a)?

Answer: Flt3 ligand (FL) does not act on progenitors from marrows of Stat5a(-/-) mice.

\textbf{Dimension 3: Investigation Target}

\textbf{Category 1: Single Component as Target} 

Question: In MELN cells, which are derived from MCF-7 breast cancer cells and stably express estrogen receptor alpha (ERalpha), does exposure to mono-n-butyl ester phthalic acid result in an increase in vascular endothelial growth factor (VEGF) secretion?

Answer: No

Question: In the context of non-alcoholic fatty liver disease (NAFLD) induced by a choline-deficient, ethionine-supplemented (CDE) diet in C57BL/6J mice, is the protein level of peroxisome-proliferator-activated receptor-gamma coactivator 1alpha (PGC1alpha) decreased in comparison to control livers?

Answer: Yes

Question: In the context of Mycobacterium tuberculosis signal transduction, what is the effect of TrcS-phosphate and calcium ions (Ca2+) on the transphosphorylation of the response regulator protein TrcR?

Answer: TrcS-phosphate and Ca2+ are required for the transphosphorylation of TrcR.

Question: In the context of myeloma cells, What is the effect of elevated heparanase expression on the expression levels of urokinase-type plasminogen activator (uPA) and the uPA receptor?

Answer: Elevation in heparanase expression in myeloma cells increases urokinase-type plasminogen activator (uPA) and uPA receptor expression levels.


\textbf{Category 2: Components Interaction as Target} 

Question: In the context of growth hormone-releasing hormone (GHRH) stimulation of growth hormone (GH) gene expression in anterior pituitary somatotrophs, does the CBP-Pit-1 pathway, which involves the interaction between CREB binding protein (CBP) and the pituitary-specific transcription factor Pit-1, require the presence of cAMP-response element binding protein (CREB) to mediate its effects on the human GH promoter?

Answer: No

Question: In the context of mitophagy, does the ubiquitin-binding protein RABGEF1 play a role in recruiting the downstream Rab GTPases, RAB5 and RAB7A, to damaged mitochondria that have been ubiquitinated by Parkin?

Answer: Yes

Question: In human lung cancer cells treated with the nitrosamine 4-(methylnitrosamino)-1-(3-pyridyl)-1-butanone (NNK), What is the role of Bcl2 phosphorylation at Ser(70) in the interaction between Bcl2 and c-Myc oncogenic proteins?


Answer: Phosphorylation of Bcl2 at Ser(70) promotes a direct interaction between Bcl2 and c-Myc, enhancing the half-life of the c-Myc protein.


Question: In cells infected with Epstein-Barr virus (EBV), How does the BGLF4 protein kinase affect the recruitment of active interferon regulatory factor 3 (IRF3) to the IRF3-responsive element containing the interferon-beta (IFN-beta) promoter region?

Answer: BGLF4 reduces the amount of active IRF3 recruited to the IRF3-responsive element containing the IFN-beta promoter region.

\textbf{Category 3: Function as Target}


Question: In human hepatoma cells (Huh-7) and normal human hepatocytes (Hc) infected with an adenovirus expressing a mutated form of IkappaBalpha (Ad5IkappaB), does pretreatment with N,N-dimethylsphingosine (DMS), an inhibitor of sphingosine kinase (SphK), lead to an increase in the number of apoptotic cells induced by tumor necrosis factor-alpha (TNF-alpha) stimulation?

Answer: Yes

Question: In Escherichia coli, do mutants lacking acyl-acyl carrier protein (acyl-ACP) synthetase activity (aas mutants) retain the ability to incorporate exogenous fatty acids into the major outer membrane lipoprotein through an acyl-CoA-independent pathway?

Answer: No

Question: In the context of myocytes expressing adiponectin receptor 1 (AdipoR1), What is the effect of adiponectin on extracellular calcium (Ca2+) influx?

Answer: Adiponectin induces extracellular Ca(2+) influx by adiponectin receptor 1 (AdipoR1) in myocytes.

Question: In the context of lung inflammation resolution, What is the role of cholesterol 25-hydroxylase (Ch25h) in alveolar macrophages (AM) during the resolution phase of inflammation?

Answer: Ch25h is induced in macrophages upon their encounter with apoptotic cells and is required for LXR-dependent prevention of AM lipid overload, induction of Mertk, efferocytic resolution of airway neutrophilia, and induction of TGF-beta.

\subsection{Data Creation and Filter Pipeline} \label{appendix: data creation}

\begin{figure*}[!h]
    \centering    
    \renewcommand{\thesubfigure}{} % Hide subfigure labels
    \subfigure[]{\includegraphics[width=0.95\linewidth]{fig/data_pipeline_new.pdf}}
    % \vspace{-7mm}
    \caption{Dataset \benchname creation pipeline.}
    \label{dataset_pipeline}
    % \vspace{-5mm} 
\end{figure*}

The overall dataset creation pipeline is shown in Figure \ref{dataset_pipeline}. 

To ensure question quality, we employ a two-step process. First, we create and filter questions using an advanced language model ( GPT-4 and
LLaMa 3.1-405B) to assess their relevance and clarity. Subsequently, each question undergoes a final quality check by human experts.

The well-define filter removes questions that are poorly defined, unpredictable (e.g., asking for specific measurement values), or require more than one prediction, and the nontrivial filter removes the data that answers revealed in the question's context. 

To validate the answer quality, we require the LLM to answer the questions based on the original paper's content. The model is explicitly instructed to respond with Undetermined if it cannot confidently generate
an
answer. Each question is tested five times, and only questions that are consistently answered correctly (i.e., aligned
with the intended label) and not marked as Undetermined in any of the trials are retained. This process helps eliminate
questions with incorrect labels, ambiguous phrasing, or poor structure.

In the final stage, human experts perform the quality check to further refine the questions, excluding data that are poorly defined, overly complex (e.g., asking for multiple facts), incorrectly labeled, or of any other low quality, ensuring the dataset's overall reliability and usability. Approximately 60\% of the data is filtered out at this stage.



\subsection{Question key words Distribution}

We present the distribution of question keywords in Figure \ref{fig key words fan}. While these keywords do not directly correspond to the three main categories we primarily use, they offer an additional perspective on the dataset. Below are explanations of the keywords:

General Influence Inquiry: Can x influence y or not?

Activation Inquiry: Can x activate y?

Inhibition Inquiry: Can x inhibit y?

Dependency Inquiry: Is y dependent on x?

Induction Question: Can x induce y?

Relief Inquiry: Can x relieve y?

Mechanism Question: Does x influence y via a specific mechanism?

Exclusive Mechanism Question: Is a specific mechanism the only mechanism for process z?

Significance Inquiry: Will x cause a significant/insignificant phenomenon y?

Baseline Comparison Question: Is x different from the baseline?

Experimental Observation Question: Will a specific detailed phenomenon be observed in the experiment?

Physiological Observation Question: Will a specific phenomenon be observed in the body?


\begin{figure*}[!h]
    \centering    
    \renewcommand{\thesubfigure}{} % Hide subfigure labels
    \subfigure[]{\includegraphics[width=0.48\linewidth]{fig/plot_fan_TrueFalse.pdf}} 
    \subfigure[]{\includegraphics[width=0.48\linewidth]{fig/plot_fan_Open-Ended.pdf}} 
    \caption{Dataset key words distribution.}
    \label{fig key words fan}
\end{figure*}


% To gather realistic biological pathway phenomena, particularly those involving interventions, the data for \benchname is sourced from over 6,000 biological pathway research papers. These studies involve carefully designed intervention experiments to observe how biological systems respond, with the observed phenomena logically supported by pathway mechanisms. We extract detailed experimental observations and their contexts directly from the abstracts. Importantly, our focus is on the specific experimental phenomena observed and reported, rather than the final conclusions drawn by the researchers. This is essential for our goal of predicting detailed events in the benchmark. The dataset creation involves prompting the large language model, and in this study, we have chosen LLaMa-3.1-405B \citep{dubey2024llama} and GPT-4 as the models for data creation.

% After extracting the phenomena, we convert each one into either a True/False or open-ended question, depending on its content. Each question is paired with corresponding labeled answers.

% We then apply multiple data filters and validation steps to ensure the correctness, quality, and relevance to biological pathways. The correctness of each question is validated by checking whether LLMs can answer it accurately using the original paper content, allowing us to exclude question-label pairs with errors. Question quality is ensured through several filters, removing questions that are poorly defined, unpredictable (e.g., asking for specific measurement values), require more than one prediction, are trivial with answers revealed in the question's context, or are unrelated to biological pathways. After all the filters, \benchname contains 1.3k high-quality questions for biological pathways reasoning.



% \begin{lstlisting}[style=QuestionType]
% Previous studies have suggested that heterotrimeric Gi proteins are involved in signaling pathways that lead to the production of inflammatory mediators induced by lipopolysaccharide (LPS). TLR4 has also been shown to play a central role in the response to LPS activation. To investigate the role of Gi proteins in TLR4 activation of signaling pathways, human embryonic kidney (HEK) 293 cells and RAW 264.7 cells were treated with pertussis toxin (PTx) or transfected with dominant negative Galphai3 (Galphai3dn) or Galphai2 minigene plasmid. The cells were then transfected with constitutively active TLR4 (TLR4ca) plasmid or TLR4ca along with an NFkappaB or AP-1 reporter construct. TLR4ca transfection resulted in ERK 1/2 activation, AP-1 activation, and NFkappaB activation. What is the effect of constitutively active Galphai2 and Galphai3 plasmids on TLR4ca-induced ERK 1/2 phosphorylation?
% \end{lstlisting}

% \begin{lstlisting}[style=AnswerType]
% Constitutively active Galphai2 and Galphai3 plasmids potentiate TLR4ca-induced ERK 1/2 phosphorylation.
% \end{lstlisting}

% \begin{lstlisting}[style=ModelResponseType]
% Based on the pathways provided, the effect of constitutively active Galphai2 and Galphai3 plasmids on TLR4ca-induced ERK 1/2 phosphorylation is likely to be an increase in phosphorylation. This is supported by pathway 0, which indicates that the activation of TLR4ca leads to ERK 1/2 activation, and pathway 1, which suggests that Galphai proteins are involved in the activation and phosphorylation of MAP2K1 (MEK1), which is upstream of ERK 1/2. Therefore, the constitutively active Galphai2 and Galphai3 plasmids are likely to enhance the phosphorylation of ERK 1/2 in response to TLR4ca stimulation.
% \end{lstlisting}

% \begin{lstlisting}[style=JudgeType]
% The standard answer states that constitutively active Galphai2 and Galphai3 plasmids potentiate TLR4ca-induced ERK 1/2 phosphorylation, while the new answer concludes that the effect of constitutively active Galphai2 and Galphai3 plasmids on TLR4ca-induced ERK 1/2 phosphorylation is likely to be an increase in phosphorylation. Therefore, the judgement is 'Correct'.
% Judge: Correct
% \end{lstlisting}


\subsection{Pathway Graph Database Statistics} \label{appendix pathway statistic}

\begin{table}[!h]
\centering
\caption{Data statistic of our pathway network database.}
\label{table pathway statistic}
\begin{tabular}{ccc}
\toprule
Entries & Edges & Involved biological process \\
8939    & 15131 & 2265                        \\
\bottomrule
\end{tabular}
\end{table}

\subsection{Implementation of Subgraph Retriever} \label{appendix subgraph retriever}

Since the connectivity of the pathway graph is crucial for enhancing reasoning in biological systems, we developed the retrieval API designed to find the optimal connected subgraph $S^*=\operatorname{Search\_Subgraph}(\texttt{query}, N)$, where $S^*$ is the retrieved subgraph, $\texttt{query}$ represents the query content, such as keywords, and $N$ is the target size. The goal is to match a given target size as closely as possible while maximizing the matching score. This is formulated as a optimization problem:
\begin{equation}
\begin{aligned}
S^*&=\operatorname{Search\_Subgraph}(\texttt{query}, N)\\&= \underset{\substack{S \subseteq P, \\ |S|=N,\\S \text { is connected}}}{\operatorname{argmax}} \sum_{i \in V_S \cup E_S} \operatorname{score}(i, \texttt{query})\\
\end{aligned}
\end{equation}
Here, the overall pathway network is denoted by $P$, with $V_S$ and $E_S$ representing the node and edge sets of subgraph $S$. 

% We define the subgraph retriever as follows:

% \begin{equation}
% \begin{aligned}
% \operatorname{Search\_Subgraph}&(query, N)= \underset{\substack{S \subseteq P, S \text { is connected}, |S|=N}}{\operatorname{argmax}} \sum_{i \in V_S \cup E_S} \operatorname{score}(i, query)\\
% \end{aligned}
% \end{equation}

The optimization problem is hard to solve directly for huge graph databse. Inspire by \cite{he2024g} that convert optimal subgraph problem into a Prize-Collecting Steiner Tree (PCST) problem, we solve the problem as a bi-level optimization 
as follows:

\begin{equation}
\small
\begin{aligned}
&S(C_e)=\underset{\substack{S \subseteq P, \\ S \text { is connected }}}{\operatorname{argmax}} \sum_{i \in V_S \cup E_S} \operatorname{score}(i, \texttt{query})-\left|E_S\right| \times C_e\\
& C_e^* = \underset{\substack{C_e}}{\operatorname{argmin}}  | |S(C_e)|-N|\\
\end{aligned}
\end{equation}

Then 
$$
\operatorname{Search\_Subgraph}(\texttt{query}, N)= S(C_e^*)\\
$$

The inner optimization problem, a Prize-Collecting Steiner Tree (PCST) problem, seeks to identify a connected subgraph that maximizes the total score of its nodes and edges while minimizing the edge costs, as utilized in \citep{he2024g}. The outer optimization problem minimizes the difference between the optimal subgraph size and the target size. In practice, we apply a fast-solving method for the inner problem proposed by \citep{hegde2015nearly} and employ a binary root search algorithm for the outer problem.


\textbf{Graph-to-Text Encoder} For retrieved results which is a list of triples (e.g., the output of $\operatorname{Search\_Triple}$ or $\operatorname{Search\_Subgraph}$), the result $S$ is encoded into text $o$ using the following process:
\begin{equation}
\begin{aligned}
o&=\operatorname{TripleToText}(\operatorname{DFSOrder}(S))
\end{aligned}
\end{equation}
Function $\operatorname{DFSOrder}$ arranges the triples in depth-first search (DFS) order. Unlike other methods, such as relevance scoring, DFS order more closely aligns with the reasoning process through the subgraph, reflecting the natural progression of biological functions. The ordered subgraph $\hat{S}$ is converted into text format for the LLM by encoding each triple as a string in the following format:
$\text{ Head } |  \text{ Tail } | \text{ Relation and Biological Process}$.


% \onecolumn

\subsection{\modelname Case} \label{appendix agent case}

\begin{lstlisting}[style=QuestionType]
Please explore pathways to find relevant information regarding the following question: Worldwide, one of the most prevalent forms of chronic disease is alcoholic fatty liver, which may progress to more severe forms of liver injury. Chronic ethanol exposure is thought to cause enhanced hepatic lipogenesis and impaired fatty acid oxidation. Adiponectin is an adipose-derived hormone that has been associated with the development of alcoholic liver steatosis in rodent models. Does adiponectin have a protective role against alcoholic liver steatosis?
\end{lstlisting}
\begin{lstlisting}[style=AnswerType]
Action: Thought: To find information regarding the role of adiponectin in protecting against alcoholic liver steatosis, I can use the search_biopathway_subgraph_global tool to explore relevant biological pathways. I should convert all the question content into keywords to enhance retrieval quality. Action:
     search_biopathway_subgraph_global(['Adiponectin, adipose-derived hormone, associated with the development of alcoholic liver steatosis in rodent models, protective role against alcoholic liver steatosis, chronic ethanol exposure, enhanced hepatic lipogenesis, impaired fatty acid oxidation, liver injury'])
     End Action
\end{lstlisting}
\begin{lstlisting}[style=QuestionType]
State: 0) C00469: Ethanol; Ethyl alcohol; Methylcarbinol | 406999: microRNA 217 MIR217 MIRN217 mir-217 | PCrel indirect effect activation | hsa04936: Alcoholic liver disease
1) 406999: microRNA 217 MIR217 MIRN217 mir-217 | 23411: sirtuin 1 SIRT1 SIR2 SIR2L1 SIR2alpha | PPrel inhibition indirect effect | hsa04936: Alcoholic liver disease
2) 406999: microRNA 217 MIR217 MIRN217 mir-217 | 29896 6434: TRA2A AWMS1 HSU53209 | PPrel inhibition indirect effect | hsa04936: Alcoholic liver disease
3) 29896 6434: TRA2A AWMS1 HSU53209 | 23175 64900 9663: LPIN1 PAP1 | PPrel activation | hsa04936: Alcoholic liver disease
4) 23175 64900 9663: LPIN1 PAP1 | 6720: sterol regulatory element binding transcription factor 1 SREBF1 HMD IFAP2 SREBP1 bHLHd1 | PPrel inhibition | hsa04936: Alcoholic liver disease
5) 6720: sterol regulatory element binding transcription factor 1 SREBF1 HMD IFAP2 SREBP1 bHLHd1 | 6319 79966: SCD FADS5 MSTP008 SCD1 SCDOS hSCD1 | GErel expression | hsa04152: AMPK signaling pathway hsa04936: Alcoholic liver disease
6) 6720: sterol regulatory element binding transcription factor 1 SREBF1 HMD IFAP2 SREBP1 bHLHd1 | 23175 64900 9663: LPIN1 PAP1 | GErel expression | hsa04936: Alcoholic liver disease
7) C00469: Ethanol; Ethyl alcohol; Methylcarbinol | 51094 79602: ADIPOR1 ACDCR1 CGI-45 CGI45 PAQR1 TESBP1A | PCrel inhibition indirect effect | hsa04936: Alcoholic liver disease
8) C00469: Ethanol; Ethyl alcohol; Methylcarbinol | 51422 53632 5562 5563 5564 5565 5571: PRKAG2 AAKG AAKG2 CMH6 H91620p WPWS | PCrel inhibition indirect effect | hsa04936: Alcoholic liver disease
9) C00469: Ethanol; Ethyl alcohol; Methylcarbinol | 50507: NADPH oxidase 4 NOX4 KOX KOX-1 RENOX | PCrel activation | hsa04936: Alcoholic liver disease
10) 50507: NADPH oxidase 4 NOX4 KOX KOX-1 RENOX | C22381: Reactive oxygen species; ROS | PCrel activation | hsa04936: Alcoholic liver disease
11) C22381: Reactive oxygen species; ROS | 51422 53632 5562 5563 5564 5565 5571: PRKAG2 AAKG AAKG2 CMH6 H91620p WPWS | PCrel inhibition indirect effect | hsa04936: Alcoholic liver disease
12) C22381: Reactive oxygen species; ROS | 10000 207 208: AKT3 MPPH MPPH2 PKB-GAMMA PKBG PRKBG RAC-PK-gamma RAC-gamma STK-2 | PCrel inhibition | hsa04936: Alcoholic liver disease
13) 10000 207 208: AKT3 MPPH MPPH2 PKB-GAMMA PKBG PRKBG RAC-PK-gamma RAC-gamma STK-2 | 1499: catenin beta 1 CTNNB1 CTNNB EVR7 MRD19 NEDSDV armadillo | PPrel activation phosphorylation | hsa04936: Alcoholic liver disease
14) C22381: Reactive oxygen species; ROS | 5606 5608: MAP2K3 MAPKK3 MEK3 MKK3 PRKMK3 SAPKK-2 SAPKK2 | PCrel activation | hsa04936: Alcoholic liver disease
15) C00469: Ethanol; Ethyl alcohol; Methylcarbinol | 712 713 714: C1QA C1QD1 | PCrel activation | hsa04936: Alcoholic liver disease
16) C00469: Ethanol; Ethyl alcohol; Methylcarbinol | 3605: interleukin 17A IL17A CTLA-8 CTLA8 IL-17 IL-17A IL17 ILA17 | PCrel indirect effect activation | hsa04936: Alcoholic liver disease
17) C00084: Acetaldehyde; Ethanal | 51422 53632 5562 5563 5564 5565 5571: PRKAG2 AAKG AAKG2 CMH6 H91620p WPWS | PCrel inhibition indirect effect | hsa04936: Alcoholic liver disease
18) 9370: adiponectin C1Q and collagen domain containing ADIPOQ ACDC ACRP30 ADIPQTL1 ADPN APM-1 APM1 GBP28 | 51094 79602: ADIPOR1 ACDCR1 CGI-45 CGI45 PAQR1 TESBP1A | PPrel activate activation | hsa04152: AMPK signaling pathway hsa04211: Longevity regulating pathway hsa04936: Alcoholic liver disease hsa04920: Adipocytokine signaling pathway N01788: ADIPOQ-ADIPOR signaling pathway
19) 7132: TNF receptor superfamily member 1A TNFRSF1A CD120a FPF TBP1 TNF-R TNF-R-I TNF-R55 TNFAR TNFR1 TNFR55 TNFR60 p55 p55-R p60 | 6720: sterol regulatory element binding transcription factor 1 SREBF1 HMD IFAP2 SREBP1 bHLHd1 | PPrel indirect effect activation | hsa04936: Alcoholic liver disease
\end{lstlisting}
\begin{lstlisting}[style=AnswerType]
Action: Thought: I have gathered a large number of pathways related to the question. To better understand the role of adiponectin in protecting against alcoholic liver steatosis, I will check the neighboring pathways of Pathway 18, which is directly related to adiponectin. I will use the function search_biopathway_triple_N_hop_subgraph with history_line_id set to 18. The keywords are the same as previously used.
     Action:
     search_biopathway_triple_N_hop_subgraph(18, ['Adiponectin, adipose-derived hormone, associated with the development of alcoholic liver steatosis in rodent models, protective role against alcoholic liver steatosis, chronic ethanol exposure, enhanced hepatic lipogenesis, impaired fatty acid oxidation, liver injury'])
     End Action
\end{lstlisting}
\begin{lstlisting}[style=QuestionType]
State: 20) C00219: Arachidonate; Arachidonic acid; (5Z8Z11Z14Z)-Icosatetraenoic acid; cis-581114-Eicosatetraenoic acid; 5Z8Z11Z14Z-Eicosatetraenoic acid; | 2180 2181 2182 23205 23305 51703: ACSL1 ACS1 FACL1 FACL2 LACS LACS1 LACS2 | pathway bind to | hsa04216: Ferroptosis N01590: Arachidonate/Adrenic acid metabolism
21) 2180 2181 2182 23205 23305 51703 81616: ACSL1 ACS1 FACL1 FACL2 LACS LACS1 LACS2 | C00162: Fatty acid | PCrel  | hsa04920: Adipocytokine signaling pathway
22) 1376: carnitine palmitoyltransferase 2 CPT2 CPT1 CPTASE IIAE4 | 51 8310: ACOX1 ACOX AOX MITCH PALMCOX SCOX | ECrel compound | hsa00071: Fatty acid degradation
23) 1376: carnitine palmitoyltransferase 2 CPT2 CPT1 CPTASE IIAE4 | 126129 1374 1375: CPT1C CATL1 CPT1-B CPT1P CPTI-B CPTIC SPG73 | ECrel compound | hsa00071: Fatty acid degradation
24) 126129 1374 1375: CPT1C CATL1 CPT1-B CPT1P CPTI-B CPTIC SPG73 | 2180 2181 2182 23205 23305 51703 81616: ACSL1 ACS1 FACL1 FACL2 LACS LACS1 LACS2 | ECrel compound | hsa00071: Fatty acid degradation hsa04920: Adipocytokine signaling pathway
25) 4217: mitogen-activated protein kinase kinase kinase 5 MAP3K5 ASK1 MAPKKK5 MEKK5 | 5609 6416: MAP2K7 JNKK2 MAPKK7 MEK MEK_7 MKK7 PRKMK7 SAPKK-4 SAPKK4 | PPrel activate activation phosphorylation | hsa05418: Fluid shear stress and atherosclerosis hsa04668: TNF signaling pathway hsa04936: Alcoholic liver disease hsa05208: Chemical carcinogenesis - reactive oxygen species N01407: Metals to JNK signaling pathway
26) 5609 6416: MAP2K7 JNKK2 MAPKK7 MEK MEK_7 MKK7 PRKMK7 SAPKK-4 SAPKK4 | 5599 5601 5602: MAPK8 JNK JNK-46 JNK1 JNK1A2 JNK21B1/2 PRKM8 | PPrel activate activation phosphorylation | hsa05418: Fluid shear stress and atherosclerosis hsa05135: Yersinia infection hsa05417: Lipid and atherosclerosis hsa05167: Kaposi sarcoma-associated herpesvirus infection hsa04620: Toll-like receptor signaling pathway hsa04668: TNF signaling pathway
27) 5599 5601 5602: MAPK8 JNK JNK-46 JNK1 JNK1A2 JNK21B1/2 PRKM8 SAPK1 SAPK1c | 3667 8471 8660: IRS1 HIRS-1 | PPrel inhibition phosphorylation | hsa04910: Insulin signaling pathway hsa05010: Alzheimer disease hsa04930: Type II diabetes mellitus hsa04920: Adipocytokine signaling pathway
28) 5599 5601 5602: MAPK8 JNK JNK-46 JNK1 JNK1A2 JNK21B1/2 PRKM8 SAPK1 SAPK1c | C00162: Fatty acid | PCrel  | hsa04930: Type II diabetes mellitus
29) 4790 5970: NFKB1 CVID12 EBP-1 KBF1 NF-kB NF-kB1 NF-kappa-B1 NF-kappaB NF-kappabeta NFKB-p105 | 2919 2920 2921: CXCL1 FSP GRO1 GROa MGSA MGSA-a NAP-3 SCYB1 | PPrel activation expression | hsa04621: NOD-like receptor signaling pathway hsa05167: Kaposi sarcoma-associated herpesvirus infection hsa05417: Lipid and atherosclerosis hsa05120: Epithelial cell signaling in Helicobacter pylori infection hsa04936: Alcoholic liver disease
30) 4790 5970: NFKB1 CVID12 EBP-1 KBF1 NF-kB NF-kB1 NF-kappa-B1 NF-kappaB NF-kappabeta NFKB-p105 NFKB-p50 | 4792: NFKB inhibitor alpha NFKBIA EDAID2 IKBA MAD-3 NFKBI | PPrel missing interaction dissociation | hsa05215: Prostate cancer hsa05167: Kaposi sarcoma-associated herpesvirus infection hsa05161: Hepatitis B hsa05220: Chronic myeloid leukemia hsa05160: Hepatitis C hsa04936: Alcoholic liver disease hsa04657: IL-17 signaling pathway
31) 4790 5970: NFKB1 CVID12 EBP-1 KBF1 NF-kB NF-kB1 NF-kappa-B1 NF-kappaB NF-kappabeta NFKB-p105 NFKB-p50 NFkappaB | 9021: suppressor of cytokine signaling 3 SOCS3 ATOD4 CIS3 Cish3 SOCS-3 SSI-3 SSI3 | GErel expression | hsa04920: Adipocytokine signaling pathway
32) 9021: suppressor of cytokine signaling 3 SOCS3 ATOD4 CIS3 Cish3 SOCS-3 SSI-3 SSI3 | 3667 8471 8660: IRS1 HIRS-1 | PPrel inhibition | hsa04920: Adipocytokine signaling pathway
33) 51094 79602: ADIPOR1 ACDCR1 CGI-45 CGI45 PAQR1 TESBP1A | C00076: Calcium cation; Ca2+; Calcium(2+); Calcium ion | PCrel indirect effect activation | hsa04152: AMPK signaling pathway hsa04936: Alcoholic liver disease
34) C00076: Calcium cation; Ca2+; Calcium(2+); Calcium ion | 10645: calcium/calmodulin dependent protein kinase kinase 2 CAMKK2 CAMKK CAMKKB | PCrel activation | hsa04152: AMPK signaling pathway hsa04211: Longevity regulating pathway hsa04936: Alcoholic liver disease hsa04920: Adipocytokine signaling pathway
35) 10645: calcium/calmodulin dependent protein kinase kinase 2 CAMKK2 CAMKK CAMKKB | 51422 53632 5562 5563 5564 5565 5571: PRKAG2 AAKG AAKG2 CMH6 H91620p WPWS | PPrel activation phosphorylation | hsa04152: AMPK signaling pathway hsa04921: Oxytocin signaling pathway hsa04211: Longevity regulating pathway hsa04936: Alcoholic liver disease hsa04920: Adipocytokine signaling pathway
36) 51094 79602: ADIPOR1 ACDCR1 CGI-45 CGI45 PAQR1 TESBP1A | 5465: peroxisome proliferator activated receptor alpha PPARA NR1C1 PPAR PPAR-alpha PPARalpha hPPAR | PPrel indirect effect activation | hsa04920: Adipocytokine signaling pathway
37) 5465: peroxisome proliferator activated receptor alpha PPARA NR1C1 PPAR PPAR-alpha PPARalpha hPPAR | 126129 1374 1375: CPT1C CATL1 CPT1-B CPT1P CPTI-B CPTIC SPG73 | GErel expression | hsa04920: Adipocytokine signaling pathway
38) 5465: peroxisome proliferator activated receptor alpha PPARA NR1C1 PPAR PPAR-alpha PPARalpha hPPAR | C00638: Long-chain fatty acid; Higher fatty acid | PCrel  | hsa04920: Adipocytokine signaling pathway
\end{lstlisting}
\begin{lstlisting}[style=AnswerType]
Action: Thought: I have gathered a large number of pathways related to the question. To better understand the role of adiponectin in protecting against alcoholic liver steatosis, I will check the neighboring pathways of Pathway 18, which is directly related to adiponectin. I will use the function search_biopathway_triple_N_hop_subgraph with history_line_id set to 18. The keywords are the same as previously used.
     Action:
     search_biopathway_triple_N_hop_subgraph(18, ['Adiponectin, adipose-derived hormone, associated with the development of alcoholic liver steatosis in rodent models, protective role against alcoholic liver steatosis, chronic ethanol exposure, enhanced hepatic lipogenesis, impaired fatty acid oxidation, liver injury'])
     End Action
\end{lstlisting}
\begin{lstlisting}[style=QuestionType]
State: No new pathways were found besides those previously seen.
\end{lstlisting}
\begin{lstlisting}[style=AnswerType]
Action: Thought: Since no new pathways were found, I will finish the exploration.
Answer:
Finished.
\end{lstlisting}
\begin{lstlisting}[style=QuestionType]
State: You finished the task.
\end{lstlisting}

%I will analyze the existing pathways to identify potential relationships between adiponectin and the development of alcoholic liver steatosis. I will focus on pathways that are directly related to adiponectin and its functions, such as pathways involved in fatty acid metabolism, inflammation, and oxidative stress.

% From the existing pathways, I noticed that adiponectin is involved in the regulation of fatty acid metabolism, particularly in the inhibition of lipogenesis and the promotion of fatty acid oxidation. This suggests that adiponectin may play a role in preventing the accumulation of fat in the liver, which is a hallmark of alcoholic liver steatosis.

% I also noticed that adiponectin is involved in the regulation of inflammation and oxidative stress, which are key mechanisms underlying the development of liver injury in response to chronic ethanol exposure. Adiponectin has been shown to have anti-inflammatory and antioxidant properties, which may help to mitigate the liver damage caused by ethanol.

% Based on these findings, I conclude that adiponectin may have a protective role against alcoholic liver steatosis by regulating fatty acid metabolism, inflammation, and oxidative stress.


% \subsection{Failed Reason Cases}

% \textbf{Reasoning Logic Errors}

% "Question": "Activation of Cdc2/cyclin B kinase and entry into mitosis requires dephosphorylation of inhibitory sites on Cdc2 by Cdc25 phosphatase. In vertebrates, Cdc25C is inhibited by phosphorylation at a single site targeted by the checkpoint kinases Chk1 and Cds1/Chk2 in response to DNA damage or replication arrest. Ca2+/calmodulin-dependent protein kinase II (CaMKII) phosphorylates Cdc25C efficiently on S287 in vitro. How does CaMKII affect Cdc2/cyclin B activation in egg extracts?",

% "Answer": "CaMKII delays Cdc2/cyclin B activation via phosphorylation of Cdc25C at S287 in egg extracts.",

% "Reasoning": "Let's break it down step by step!\n\nIn the absence of DNA damage or replication arrest, Cdc25C is active and dephosphorylates Cdc2, allowing the entry into mitosis. However, in response to DNA damage or replication arrest, Cdc25C is phosphorylated by Chk1 and Cds1/Chk2, leading to its inhibition.\n\nCaMKII phosphorylates Cdc25C on S287, which would normally inhibit its activity. However, in the context of egg extracts, CaMKII phosphorylation of Cdc25C on S287 actually enhances its activity. This is because the phosphorylation site targeted by CaMKII is distinct from the inhibitory site targeted by Chk1 and Cds1/Chk2.\n\nTherefore, CaMKII phosphorylation of Cdc25C on S287 does not inhibit its activity, but rather enhances it, allowing for the activation of Cdc2/cyclin B kinase and entry into mitosis.",

% "Judge": "The question asks about how CaMKII affects Cdc2/cyclin B activation in egg extracts. For this question, the standard answer states that CaMKII delays Cdc2/cyclin B activation via phosphorylation of Cdc25C at S287 in egg extracts. The new answer, however, suggests that CaMKII phosphorylation of Cdc25C on S287 enhances its activity, allowing for the activation of Cdc2/cyclin B kinase and entry into mitosis. Therefore, the judgement is 'Incorrect'. \n\nJudge: Incorrect"



% "Question": "Angiotensin II (Ang II) is an octapeptide that plays a crucial role in the cardiovascular system, and its formation and degradation are tightly regulated. To investigate the role of mast cell proteases in Ang II production, ex vivo cultures of peritoneal cells from mouse strains lacking mast cell heparin or mouse mast cell protease 4 (mMCP-4) were used. What is the effect of heparin deficiency on the formation and degradation of Ang II in these peritoneal cell cultures?",

% "Answer": "The formation and degradation of Ang II are strongly reduced in heparin-deficient peritoneal cells.",

% \begin{lstlisting}[style=ModelResponseType]
% "Reasoning": "Based on the provided pathways, I will infer the answer to the question.\n\nThe question asks about the effect of heparin deficiency on the formation and degradation of Ang II in peritoneal cell cultures from mouse strains lacking mast cell heparin or mouse mast cell protease 4 (mMCP-4).\n\nPathway 0 shows the formation of Ang II through the action of angiotensin I converting enzyme (ACE). Pathway 1 shows the degradation of Ang II by prolylcarboxypeptidase (PRCP).\n\nPathway 2 shows the activation of Ang II by ACE, which is the key enzyme responsible for the formation of Ang II.\n\nPathway 3 shows the activation of Ang II receptors, specifically Ang II receptor type 2 (AGTR2), which is involved in the regulation of blood pressure.\n\nPathway 4 shows the activation of Ang II receptors, specifically Ang II receptor type 1 (AGTR1), which is also involved in the regulation of blood pressure.\n\nPathway 5 shows the degradation of Ang II by carboxypeptidase A3 (CPA3).\n\nPathway 6 shows the degradation of Ang II by membrane metalloendopeptidase (MME).\n\nPathway 7 shows the formation of Ang II through the action of chymase 1 (CMA1).\n\nPathway 8 shows the degradation of Ang II by prolylcarboxypeptidase (PRCP).\n\nPathway 9 shows the binding of Ang I to ACE, which is the first step in the formation of Ang II.\n\nPathway 10 shows the activation of Ang II by Ang II receptor type 2 (AGTR2).\n\nPathway 11 shows the activation of Ang II by Ang II receptor type 1 (AGTR1).\n\nPathway 12 shows the degradation of Ang II by cathepsin A (CTSA).\n\nPathway 13 shows the activation of Ang II by Ang II receptor type 1 (AGTR1).\n\nPathway 14 shows the degradation of Ang II by kallikrein 1 (KLK1).\n\nPathway 15 shows the degradation of Ang II by cathepsin G (CTSG).\n\nBased on these pathways, it can be inferred that heparin deficiency would lead to an increase in the formation of Ang II, as the degradation of Ang II by PRCP and MME would be impaired. This is because heparin is a cofactor for these enzymes, and its deficiency would lead to a decrease in their activity.\n\nTherefore, the answer is that heparin deficiency would lead to an increase in the formation and a decrease in the degradation of Ang II in peritoneal cell cultures from mouse strains lacking mast cell heparin or mouse mast cell protease 4 (mMCP-4).",
% \end{lstlisting}

% \begin{lstlisting}[style=JudgeType]
% "Judge": "The question asks about the effect of heparin deficiency on the formation and degradation of Ang II in peritoneal cell cultures from mouse strains lacking mast cell heparin or mouse mast cell protease 4 (mMCP-4). For this question, the standard answer states that the formation and degradation of Ang II are strongly reduced in heparin-deficient peritoneal cells. The new answer, however, suggests that heparin deficiency would lead to an increase in the formation and a decrease in the degradation of Ang II. Therefore, the judgement is 'Incorrect'. \n\nJudge: Incorrect"
% \end{lstlisting}


\subsection{Baseline Details} \label{appendix baseline}

\textbf{Chain-of-Thought (CoT)} \citep{wei2022chain,kojima2022large} prompting encourages language models to generate step-by-step reasoning to solve tasks. We adopt CoT as the unaugmented standard method for LLM reasoning.

\textbf{Chain-of-Knowledge (CoK)} \citep{li2023chain} enhances chain-of-thought reasoning through knowledge verification, enabling language models to handle knowledge-intensive tasks. CoK first generates reasoning using chain-of-thought processes, then employs knowledge triples to verify the accuracy of the reasoning. While the reasoning of CoK is primarily driven by the language model, graph-based information is used for fact verification.

\textbf{Think-of-Graph (ToG)} \citep{sun2023think} is an interactive reasoning method designed to actively navigate knowledge graphs for question solving. It primarily uses large language models to prune knowledge graph edges, thereby enabling efficient knowledge acquisition from complex graphs. The reasoning process in ToG is guided by graph navigation.

\textbf{G-Retriever} \citep{he2024g} is a graph retriever-augmented generation method that retrieves relevant subgraphs from a database and generates answers based on the retrieved subgraphs. While the original model in their work uses a graph encoder to encode graph data as a separate modality, in this work, we directly implement the graph-to-text encoder for improved versatility and better comparability with other methods.



\subsection{{Failure Reason Category Cases}}

We classify the error reason of biological pathway reasoning into four types:
(1) \textbf{Unresolved Conclusion} For cases where the model fails to provide a definitive answer, indicating uncertainty or belief that the answer is unknown.
(2) \textbf{Incomplete Answer}  When the response lacks essential details, such as missing the requested effects or other key elements.
(3) \textbf{Omission in Reasoning}  For errors where critical pathway steps in the question's biological process are left out, causing the final answer to be incorrect.
(4) \textbf{Faulty in Reasoning} When the reasoning path is correct, but there are significant errors in deducing the events within that pathway.

In this subsection, we provide examples of each failure reason type. The 'Question' and 'Answer' represent the input question and its labeled correct answer. The 'Wrong Answer by Model' refers to the incorrect response generated by the model. Additionally, we include a 'Reference Reasoning Process', which outlines the reasoning process leading to the correct final answer by LLaMA3.1 405B. This helps readers understand how errors occur during the reasoning process.

For the error types 'Omission in Reasoning' and 'Faulty Reasoning,' we present cases for both CoT and PathSeeker individually. These examples demonstrate how LLM reasoning fails either on its own or when augmented with a pathway graph.




\textbf{Category 1: Unresolved Conclusion}

% \begin{tcolorbox}[breakable,title=Prompt Details for Blockworld ]
\begin{tcolorbox}[breakable]
% \textbf{\textcolor{blue}{System Prompt}}\\ 
% You are a master in planning.\\
% \begin{tikzpicture}
% \draw[dashed] (0,0) -- (\linewidth,0); % 绘制一条宽度为\linewidth的虚线
% \end{tikzpicture}
% \textbf{\textcolor{blue}{Instruction}}\\

"Question": "A large fraction of pediatric pre-B acute lymphoblastoid leukemias (ALL) consistently contain a t(1;19) chromosomal translocation, resulting in the production of a chimeric transcription factor containing the N-terminal transactivation domain of E2A fused to the C-terminal DNA-binding homeodomain of Pbx1. What is the effect of the E2A-Pbx1 fusion protein on the expression of the novel WNT gene, WNT-16, in pre-B ALL cells?",

"Answer": "The E2A-Pbx1 fusion protein activates the expression of WNT-16 in pre-B ALL cells.",

\begin{tikzpicture}
\draw[dashed] (0,0) -- (\linewidth,0); % 绘制一条宽度为\linewidth的虚线
\end{tikzpicture}

"Wrong Answer by Model": "The E2A-Pbx1 fusion protein in pediatric pre-B acute lymphoblastic leukemias (ALL) with the t(1;19) chromosomal translocation is known to have oncogenic properties and can affect gene expression. In the case of the novel WNT gene, WNT-16, the E2A-Pbx1 fusion protein may have an effect on its expression in pre-B ALL cells. However, the specific effect of the fusion protein on WNT-16 expression would depend on the regulatory elements present in the WNT-16 gene and the transcriptional activity of the fusion protein. Further investigation would be needed to determine the exact effect of the E2A-Pbx1 fusion protein on WNT-16 expression in pre-B ALL cells.",

\begin{tikzpicture}
\draw[dashed] (0,0) -- (\linewidth,0); % 绘制一条宽度为\linewidth的虚线
\end{tikzpicture}


"Reference Reasoning Process": "To determine the effect of the E2A-Pbx1 fusion protein on the expression of the novel WNT gene, WNT-16, in pre-B ALL cells, we can analyze the provided pathways for transcriptional regulation. The pathways indicate that both E2A (6929) and Pbx1 are transcription factors and that their fusion can lead to transcriptional activation of target genes.

Pathways 0 and 13 demonstrate that the E2A-Pbx1 fusion protein can cause the expression of WNT-16. The fusion of E2A and Pbx1 results in a chimeric transcription factor that retains the transactivation domain of E2A and the DNA-binding homeodomain of Pbx1.

Considering this information, it can be inferred that the E2A-Pbx1 fusion protein leads to increased expression of the novel WNT gene, WNT-16, in pre-B ALL cells. The fusion of E2A and Pbx1 results in a transcription factor that activates the expression of WNT-16, potentially contributing to the development or progression of pre-B ALL.
\end{tcolorbox}

\begin{figure}[!h]
    \vspace{0.01cm}
    \caption{The Unresolved Conclusion error occurs when the model cannot provide a definitive answer, indicating uncertainty or the belief that the answer is unknown. In this case, the model is unable to determine the specific effect of the fusion protein on WNT-16 expression.}
\end{figure}



\textbf{Category 2: Incomplete Answer}
\begin{tcolorbox}[breakable]
"Question": "Continual efferocytic clearance of apoptotic cells (ACs) by macrophages is crucial for injury resolution. The process of continual efferocytosis is enhanced by linking the metabolism of engulfed cargo from initial efferocytic events to subsequent rounds. How is continual efferocytosis enhanced in macrophages?",

"Answer": "Continual efferocytosis is enhanced in macrophages by the metabolism of apoptotic cell-derived arginine and ornithine to putrescine by macrophage arginase 1 (Arg1) and ornithine decarboxylase (ODC).",

\begin{tikzpicture}
\draw[dashed] (0,0) -- (\linewidth,0); % 绘制一条宽度为\linewidth的虚线
\end{tikzpicture}


"Wrong Answer by Model": "Continual efferocytosis is enhanced in macrophages through the process of metabolic reprogramming. When macrophages engulf apoptotic cells (ACs), they not only clear the ACs but also metabolize the engulfed cargo. This metabolic reprogramming involves the activation of specific metabolic pathways, such as glycolysis and fatty acid oxidation, to generate energy and metabolites necessary for the efficient clearance of ACs. By linking the metabolism of engulfed cargo from initial efferocytic events to subsequent rounds, macrophages can sustain their energy requirements and maintain the high phagocytic capacity needed for continual efferocytosis. This enhanced metabolic activity allows macrophages to efficiently clear ACs and promote injury resolution.",

\begin{tikzpicture}
\draw[dashed] (0,0) -- (\linewidth,0); % 绘制一条宽度为\linewidth的虚线
\end{tikzpicture}


"Reference Reasoning Process": "To determine how continual efferocytosis is enhanced in macrophages, we can analyze the provided pathways for the key steps involved in efferocytic clearance of apoptotic cells (ACs) and how they link to subsequent rounds.

The process starts with the engulfment of apoptotic cells. This involves the recognition of phosphatidylserine on the surface of apoptotic cells by receptors such as AGER (Pathway 0). This recognition leads to the activation of signaling pathways, including the activation of Rac family small GTPase 1 (RAC1), which is crucial for the engulfment process (Pathway 1).

Following engulfment, the metabolism of engulfed cargo plays a critical role in enhancing continual efferocytosis. Specifically, the pathway involving the solute carrier family 66 member 1 (SLC66A1) and the metabolism of L-Arginine (Pathway 6) leads to the production of ornithine. Ornithine then binds to ARG1 (Pathway 7), leading to the production of urea and more ornithine (Pathway 8). This process enhances continual efferocytosis by providing the necessary components for subsequent rounds of efferocytic clearance.

Moreover, the pathway involving the metabolism of ornithine to putrescine (Pathway 9) and then the activation of RAC1 by putrescine (Pathway 11) further supports the enhancement of continual efferocytosis. This is because RAC1 activation is crucial for the engulfment process, and its activation by putrescine ensures that the macrophage remains capable of engulfing subsequent apoptotic cells.

Therefore, based on the pathways provided, continual efferocytosis in macrophages is enhanced by the metabolism of engulfed cargo, specifically through the arginine and ornithine pathways, which ultimately lead to the activation of key signaling molecules such as RAC1 that are necessary for the efferocytic clearance of apoptotic cells.",
\end{tcolorbox}
\begin{figure}[!h]
    \vspace{0.01cm}
    \caption{The Incomplete Answer error occurs when the response lacks essential details, such as missing the requested effects or other key elements. In this case, the model's response suggests that continual efferocytosis is enhanced in macrophages through metabolic reprogramming but fails to mention the metabolism of apoptotic cell-derived arginine and ornithine to putrescine, which is a key part of the labeled answer.}
\end{figure}

\textbf{Category 3: Omission in Reasoning}

\textbf{Omission in Reasoning of CoT}

% \begin{tcolorbox}[breakable]
% "Question": "Although oestrogen is essential for the development of the normal breast, adult mammary stem cells are known to be oestrogen receptor alpha (ER) negative and rely on paracrine signals in the mammary epithelium for mediation of developmental cues. However, little is known about how systemic oestrogen regulates breast cancer stem cell (CSC) activity. What is the effect of oestrogen on CSC activity in vitro and in vivo, particularly in CSC-enriched populations sorted from ER positive patient-derived and established cell lines that have low or absent ER expression?",

% "Answer": "Oestrogen stimulates CSC activity demonstrated by increased mammosphere and holoclone formation in vitro and tumour formation in vivo.",

% "Wrong Answer by Model": "The effect of estrogen on breast cancer stem cell (CSC) activity in vitro and in vivo, particularly in CSC-enriched populations sorted from ER positive patient-derived and established cell lines that have low or absent ER expression, is not well understood. However, studies have shown that estrogen can have both stimulatory and inhibitory effects on CSC activity, depending on the context and the specific signaling pathways involved.

% In vitro studies have suggested that estrogen can promote CSC activity by activating ER signaling pathways, leading to increased self-renewal and proliferation of CSCs. This effect may be mediated through the activation of downstream targets such as Wnt/$\beta$-catenin and Notch signaling, which are known to play important roles in CSC maintenance.

% On the other hand, in vivo studies have shown that estrogen can also have inhibitory effects on CSC activity. This may be due to the complex interplay between estrogen signaling and other factors in the tumor microenvironment. For example, estrogen has been shown to modulate the immune response and inhibit the recruitment of immune cells that can promote CSC activity. Additionally, estrogen can also affect the tumor vasculature and inhibit angiogenesis, which is important for CSC survival and growth.

% Overall, the effect of estrogen on CSC activity in ER positive and ER negative breast cancer cells is likely to be context-dependent and influenced by various factors in the tumor microenvironment. Further research is needed to fully understand the mechanisms underlying the regulation of CSC activity by estrogen and to develop targeted therapies for breast cancer.",

% "Reference Reasoning Process": "To determine the effect of oestrogen on CSC (cancer stem cell) activity, particularly in CSC-enriched populations sorted from ER-positive patient-derived and established cell lines with low or absent ER expression, we must analyze the provided pathways for clues about oestrogen's influence on related signaling pathways and cellular processes.

% Pathway 7 provides a crucial piece of information, as it involves ERBB2 (erb-b2 receptor tyrosine kinase 2), a member of the epidermal growth factor receptor (EGFR) family that can be activated by oestrogen in some contexts. Activation of ERBB2 can lead to the activation of PI3K-alpha (PIK3CA), which is a key component in the PI3K/AKT signaling pathway. This pathway is associated with cell survival, proliferation, and stem cell maintenance.

% The activation of PI3K-alpha by ERBB2, as suggested by Pathway 7, can lead to increased CSC activity. This is because PI3K/AKT signaling can promote the survival and self-renewal of stem cells, including cancer stem cells.

% Additionally, Pathways 0, 5, and 6 hint at the involvement of other signaling pathways that might be influenced by oestrogen and could impact CSC activity. For instance, Pathway 0 involves the insulin-like growth factor 1 receptor (IGF1R), which can activate PI3K-alpha, further supporting the role of PI3K/AKT signaling in CSC maintenance.

% Pathway 5 suggests that oestrogen might influence the expression of cyclin D1 (CCND1) through the LEF1 transcription factor, which can be activated by beta-catenin (CTNNB1). Beta-catenin is a key component of the Wnt signaling pathway, which plays a crucial role in stem cell regulation and is often dysregulated in cancer.

% Pathway 6 shows the activation of LEF1 by beta-catenin, further supporting the potential involvement of the Wnt/beta-catenin pathway in CSC regulation under the influence of oestrogen.

% In summary, the provided pathways suggest that oestrogen can influence CSC activity in ER-positive cell lines with low or absent ER expression by activating signaling pathways such as PI3K/AKT and Wnt/beta-catenin, which are crucial for stem cell maintenance and cancer progression.",
% \end{tcolorbox}

% \begin{tcolorbox}[breakable]
% "Question": "Angiotensin II (Ang II) is an octapeptide that plays a crucial role in the cardiovascular system, and its formation and degradation are tightly regulated. To investigate the role of mast cell proteases in Ang II production, ex vivo cultures of peritoneal cells from mouse strains lacking mast cell heparin or mouse mast cell protease 4 (mMCP-4) were used. What is the effect of heparin deficiency on the formation and degradation of Ang II in these peritoneal cell cultures?",

% "Answer": "The formation and degradation of Ang II are strongly reduced in heparin-deficient peritoneal cells.",

% \begin{tikzpicture}
% \draw[dashed] (0,0) -- (\linewidth,0); % 绘制一条宽度为\linewidth的虚线
% \end{tikzpicture}


% "Wrong Answer by Model": "The effect of heparin deficiency on the formation and degradation of Ang II in peritoneal cell cultures is an increase in Ang II formation and a decrease in Ang II degradation. Heparin is a mast cell protease inhibitor that plays a role in regulating the production and degradation of Ang II. In the absence of heparin, mast cell proteases, including mMCP-4, are not inhibited, leading to increased production of Ang II. Additionally, heparin also facilitates the degradation of Ang II, so its deficiency would result in decreased degradation of Ang II in the peritoneal cell cultures.",

% \begin{tikzpicture}
% \draw[dashed] (0,0) -- (\linewidth,0); % 绘制一条宽度为\linewidth的虚线
% \end{tikzpicture}


% "Reference Reasoning Process": "To determine the effect of heparin deficiency on the formation and degradation of Ang II in peritoneal cell cultures, we can analyze the provided pathways for the role of mast cell proteases in Ang II production.

% The pathways provided do not directly mention heparin or its deficiency. However, we can infer the potential effects of heparin deficiency by understanding the role of mast cell proteases in Ang II production.

% Pathway 1 shows the conversion of Angiotensin I to Angiotensin III. Pathway 2 demonstrates the binding of Angiotensin I to angiotensin I converting enzyme (ACE), which is involved in the conversion of Angiotensin I to Angiotensin II. Pathway 3 shows the activation of angiotensin II receptor type 2 by Angiotensin II. Pathway 4 demonstrates the activation of AGTR1 and AGTR2 receptors by Angiotensin II. Pathway 5 shows the activation of angiotensin II receptor type 2 by Angiotensin A.

% Considering the information provided by these pathways, we can infer that heparin deficiency may affect the activity of mast cell proteases, such as mouse mast cell protease 4 (mMCP-4), which are involved in the conversion of Angiotensin I to Angiotensin II. Heparin is a cofactor for the activation of mast cell proteases, and its deficiency may lead to reduced activity of these proteases.

% Therefore, the effect of heparin deficiency on the formation and degradation of Ang II in peritoneal cell cultures may be a reduction in the conversion of Angiotensin I to Angiotensin II, leading to decreased levels of Angiotensin II.",
% \end{tcolorbox}


\begin{tcolorbox}[breakable]
"Question": "BAMBI (BMP and activin membrane-bound inhibitor) is known to negatively regulate the signaling activity of transforming growth factor-beta, activin, and BMP, and has been identified as a target of beta-catenin in colorectal and hepatocellular tumor cells. The canonical Wnt/beta-catenin pathway plays a crucial role in regulating embryogenesis and tumorigenesis by promoting cell proliferation. Does BAMBI enhance or inhibit Wnt-promoted cell cycle progression?",

"Answer": "BAMBI increases Wnt-promoted cell cycle progression.",

\begin{tikzpicture}
\draw[dashed] (0,0) -- (\linewidth,0); % 绘制一条宽度为\linewidth的虚线
\end{tikzpicture}


"Wrong Answer by Model": "BAMBI inhibits Wnt-promoted cell cycle progression. The canonical Wnt/beta-catenin pathway promotes cell proliferation, and BAMBI has been identified as a target of beta-catenin. BAMBI negatively regulates the signaling activity of transforming growth factor-beta, activin, and BMP, which are involved in cell cycle regulation. Therefore, BAMBI acts as a negative regulator of the Wnt pathway and inhibits Wnt-promoted cell cycle progression.",

\begin{tikzpicture}
\draw[dashed] (0,0) -- (\linewidth,0); % 绘制一条宽度为\linewidth的虚线
\end{tikzpicture}


"Reference Reasoning Process": "To determine whether BAMBI enhances or inhibits Wnt-promoted cell cycle progression, we need to analyze its interactions with key components of the Wnt signaling pathway.

From pathways 10, 11, and 12, we can see that BAMBI (BMP and activin membrane-bound inhibitor) interacts with LRP6, FZD10, and DVL1, which are all crucial components of the Wnt signaling pathway. These interactions are described as "activation" and "binding/association," which suggests that BAMBI may be enhancing the activity of these components.

However, it's also known that BAMBI is a negative regulator of the TGF-beta and BMP signaling pathways, which can intersect with the Wnt signaling pathway. 

Considering the complexity of these interactions and the multiple pathways involved, we need to look for more direct evidence of BAMBI's effect on Wnt-promoted cell cycle progression.

Given the information provided, it seems that BAMBI's interactions with the Wnt signaling pathway are more likely to be enhancing or modulating rather than inhibiting. However, without more direct evidence, it's difficult to make a definitive conclusion.

Therefore, based on the available information, it appears that BAMBI may enhance Wnt-promoted cell cycle progression, but further research is needed to confirm this conclusion.",
\end{tcolorbox}
\begin{figure}[!h]
    \vspace{0.01cm}
    \caption{Omission in Reasoning refers to errors where critical steps in the biological pathway described in the question are omitted, leading to an incorrect final answer. In this case, the model's reasoning process only considered that BAMBI has been identified as a target of beta-catenin but overlooked the pathway in which BAMBI interacts with LRP6, FZD10, and DVL1, all of which are crucial components of the Wnt signaling pathway. As a result, the reasoning led to an incorrect conclusion.}
\end{figure}

\textbf{Omission in Reasoning of PathSeeker}

\begin{tcolorbox}[breakable]
"Question": "Angiotensin II (Ang II) is an octapeptide that plays a crucial role in the cardiovascular system, and its formation and degradation are tightly regulated. To investigate the role of mast cell proteases in Ang II production, ex vivo cultures of peritoneal cells from mouse strains lacking mast cell heparin or mouse mast cell protease 4 (mMCP-4) were used. What is the effect of heparin deficiency on the formation and degradation of Ang II in these peritoneal cell cultures?",

"Answer": "The formation and degradation of Ang II are strongly reduced in heparin-deficient peritoneal cells.",

\begin{tikzpicture}
\draw[dashed] (0,0) -- (\linewidth,0); % 绘制一条宽度为\linewidth的虚线
\end{tikzpicture}


"Wrong Answer by Model": "Based on the provided pathways, the effect of heparin deficiency on the formation and degradation of Ang II in peritoneal cell cultures can be inferred. Heparin is known to play a role in the regulation of Ang II production.

Pathway 0 indicates that carboxypeptidase A3 (CPA3) is involved in the degradation of Ang II. This suggests that heparin deficiency may lead to a decrease in the degradation of Ang II, potentially resulting in increased levels of Ang II.

Pathway 1 shows that angiotensin I converting enzyme (ACE) is responsible for the formation of Ang II. However, this pathway does not directly provide information about the effect of heparin deficiency on Ang II formation.

Considering the information from pathways 0 and 1, it can be inferred that heparin deficiency may primarily affect the degradation of Ang II rather than its formation. Without heparin, the degradation of Ang II by carboxypeptidase A3 may be impaired, leading to increased levels of Ang II in the peritoneal cell cultures.",

\begin{tikzpicture}
\draw[dashed] (0,0) -- (\linewidth,0); % 绘制一条宽度为\linewidth的虚线
\end{tikzpicture}


"Reference Reasoning Process": "To determine the effect of heparin deficiency on the formation and degradation of Ang II in peritoneal cell cultures, we can analyze the provided pathways for the role of mast cell proteases in Ang II production.

The pathways provided do not directly mention heparin or its deficiency. However, we can infer the potential effects of heparin deficiency by understanding the role of mast cell proteases in Ang II production.

Pathway 1 shows the conversion of Angiotensin I to Angiotensin III. Pathway 2 demonstrates the binding of Angiotensin I to angiotensin I converting enzyme (ACE), which is involved in the conversion of Angiotensin I to Angiotensin II. Pathway 3 shows the activation of angiotensin II receptor type 2 by Angiotensin II. Pathway 4 demonstrates the activation of AGTR1 and AGTR2 receptors by Angiotensin II. Pathway 5 shows the activation of angiotensin II receptor type 2 by Angiotensin A.

Considering the information provided by these pathways, we can infer that heparin deficiency may affect the activity of mast cell proteases, such as mouse mast cell protease 4 (mMCP-4), which are involved in the conversion of Angiotensin I to Angiotensin II. Heparin is a cofactor for the activation of mast cell proteases, and its deficiency may lead to reduced activity of these proteases.

Therefore, the effect of heparin deficiency on the formation and degradation of Ang II in peritoneal cell cultures may be a reduction in the conversion of Angiotensin I to Angiotensin II, leading to decreased levels of Angiotensin II.",
\end{tcolorbox}
\begin{figure}[!h]
    \vspace{0.01cm}
    \caption{Omission in Reasoning refers to errors where critical steps in the biological pathway involved in the question are omitted, leading to an incorrect final answer. In this case, the model's reasoning process considered the pathway involving the degradation of Ang II but missed the more important pathway of the conversion of Angiotensin I to Angiotensin II. This resulted in reasoning that led to an incorrect conclusion. }
\end{figure}

% \begin{tcolorbox}[breakable]
% "Question": "BAMBI (BMP and activin membrane-bound inhibitor) is known to negatively regulate the signaling activity of transforming growth factor-beta, activin, and BMP, and has been identified as a target of beta-catenin in colorectal and hepatocellular tumor cells. The canonical Wnt/beta-catenin pathway plays a crucial role in regulating embryogenesis and tumorigenesis by promoting cell proliferation. Does BAMBI enhance or inhibit Wnt-promoted cell cycle progression?",

% "Answer": "BAMBI increases Wnt-promoted cell cycle progression.",

% \begin{tikzpicture}
% \draw[dashed] (0,0) -- (\linewidth,0); % 绘制一条宽度为\linewidth的虚线
% \end{tikzpicture}


% "Wrong Answer by Model": "Based on the provided pathways, BAMBI (BMP and activin membrane-bound inhibitor) is known to negatively regulate the signaling activity of the Wnt/beta-catenin pathway, which promotes cell proliferation. Therefore, BAMBI is expected to inhibit Wnt-promoted cell cycle progression. This is supported by Pathways 13, 14, and 15, which show that BAMBI is involved in the activation and binding/association of various components of the Wnt signaling pathway. Additionally, Pathway 0 indicates that beta-catenin, a key component of the Wnt pathway, can interact with BAMBI in colorectal and hepatocellular tumor cells. Therefore, the overall inference is that BAMBI inhibits Wnt-promoted cell cycle progression.",

% \begin{tikzpicture}
% \draw[dashed] (0,0) -- (\linewidth,0); % 绘制一条宽度为\linewidth的虚线
% \end{tikzpicture}


% "Reference Reasoning Process": "To determine whether BAMBI enhances or inhibits Wnt-promoted cell cycle progression, we need to analyze its interactions with key components of the Wnt signaling pathway.

% From pathways 10, 11, and 12, we can see that BAMBI (BMP and activin membrane-bound inhibitor) interacts with LRP6, FZD10, and DVL1, which are all crucial components of the Wnt signaling pathway. These interactions are described as "activation" and "binding/association," which suggests that BAMBI may be enhancing the activity of these components.

% However, it's also known that BAMBI is a negative regulator of the TGF-beta and BMP signaling pathways, which can intersect with the Wnt signaling pathway. 

% Considering the complexity of these interactions and the multiple pathways involved, we need to look for more direct evidence of BAMBI's effect on Wnt-promoted cell cycle progression.

% Given the information provided, it seems that BAMBI's interactions with the Wnt signaling pathway are more likely to be enhancing or modulating rather than inhibiting. However, without more direct evidence, it's difficult to make a definitive conclusion.

% Therefore, based on the available information, it appears that BAMBI may enhance Wnt-promoted cell cycle progression, but further research is needed to confirm this conclusion.",
% \end{tcolorbox}

\textbf{Category 4: Faulty in Reasoning}

\textbf{Faulty in Reasoning of CoT}

% \begin{tcolorbox}[breakable]
% "Question": "Bacterial pathogens often harbour a type III secretion system (TTSS) that injects effector proteins into eukaryotic cells to manipulate host processes and cause diseases. EspH is a type III effector conserved in enteric bacterial pathogens, including enteropathogenic Escherichia coli (EPEC), enterohaemorrhagic E. coli, and Citrobacter rodentium. How does EspH affect the host Rho GTPase signalling pathway, which is involved in various cellular processes, including cytoskeleton organization and cell signaling?",

% "Answer": "EspH inactivates host Rho GTPase signalling pathway at the level of RhoGEF.",

% "Wrong Answer by Model": "EspH affects the host Rho GTPase signaling pathway by acting as a guanine nucleotide exchange factor (GEF) for RhoA. Rho GTPases are key regulators of cytoskeleton organization and cell signaling, and their activity is tightly controlled by the cycling between the active GTP-bound state and the inactive GDP-bound state. EspH binds to RhoA and stimulates the exchange of GDP for GTP, leading to the activation of RhoA. This activation of RhoA by EspH results in the reorganization of the host cell's actin cytoskeleton and the formation of actin-rich pedestals, which are used by the bacteria to attach to and invade host cells. Additionally, EspH also interacts with other components of the Rho GTPase signaling pathway, such as RhoGDI and RhoGAPs, to further modulate Rho GTPase activity and downstream signaling events. Overall, EspH manipulates the host Rho GTPase signaling pathway to promote bacterial colonization and pathogenesis.",


% "Reference Reasoning Process": "To determine the effect of EspH on the host Rho GTPase signaling pathway, we can analyze the provided pathways.

% Pathway 0 provides information that EspH inhibits ARHGEF12 (Rho guanine nucleotide exchange factor), which is linked to the activation of RHOA (Ras homolog family member A). 

% From this information, we can infer that EspH negatively affects the host Rho GTPase signaling pathway by inhibiting the activation of RHOA through ARHGEF12.

% The Rho GTPase signaling pathway is involved in various cellular processes, including cytoskeleton organization and cell signaling. Inhibition of this pathway can lead to disruption of these cellular processes, which may contribute to the pathogenesis of enteric bacterial pathogens, including enteropathogenic Escherichia coli (EPEC), enterohaemorrhagic E. coli, and Citrobacter rodentium.

% Therefore, EspH's inhibition of the Rho GTPase signaling pathway can be a mechanism by which these pathogens manipulate host processes and cause diseases.",
% \end{tcolorbox}

\begin{tcolorbox}[breakable]
"Question": "Enteropathogenic Escherichia coli (EPEC) is a major causative agent of infant diarrhoea in developing countries, and the EspF effector protein is injected from EPEC into host cells via a type III secretion system, disrupting host intestinal barrier function and initiating the mitochondrial death pathway. To clarify the manner in which EspF affects host cells, eukaryotic EspF-binding proteins were identified using affinity purification, and Abcf2, a protein of unknown function and member of the ABC-transporter family, was found to bind EspF. What is the effect of knocking down Abcf2 expression on EspF-induced caspase cleavage?",

"Answer": "Knocking down Abcf2 expression increases EspF-induced caspase 9 and caspase 3 cleavage.",

\begin{tikzpicture}
\draw[dashed] (0,0) -- (\linewidth,0); % 绘制一条宽度为\linewidth的虚线
\end{tikzpicture}


"Wrong Answer by Model": "The effect of knocking down Abcf2 expression on EspF-induced caspase cleavage is likely to decrease or inhibit caspase cleavage. Abcf2, as a binding protein of EspF, may play a role in the signaling pathway that leads to caspase activation and subsequent cleavage. By knocking down Abcf2 expression, the interaction between EspF and Abcf2 is disrupted, potentially preventing or reducing the activation of caspases and subsequent cleavage. This suggests that Abcf2 is involved in the EspF-induced mitochondrial death pathway and its knockdown may have a protective effect against the disruption of host intestinal barrier function caused by EPEC infection.",

\begin{tikzpicture}
\draw[dashed] (0,0) -- (\linewidth,0); % 绘制一条宽度为\linewidth的虚线
\end{tikzpicture}


"Reference Reasoning Process": "To determine the effect of knocking down Abcf2 expression on EspF-induced caspase cleavage, we should analyze the provided pathways for the interactions between EspF, Abcf2, and caspase.

Pathway 0 provides information that EspF binds Abcf2, and this interaction inhibits Abcf2. Additionally, pathway 2 reveals that Abcf2 inhibits caspase 9, which is involved in the intrinsic apoptotic pathway.

When EspF binds Abcf2, it inhibits Abcf2's function. Since Abcf2 normally inhibits caspase 9, the inhibition of Abcf2 by EspF would lead to an increase in caspase 9 activity.

If Abcf2 expression is knocked down, its inhibitory effect on caspase 9 would be reduced. As a result, caspase 9 activity would increase, similar to the effect of EspF binding Abcf2.

Therefore, knocking down Abcf2 expression would enhance EspF-induced caspase cleavage. The reduction in Abcf2's inhibitory effect on caspase 9 would amplify the pro-apoptotic signal, leading to increased caspase cleavage.

In summary, the effect of knocking down Abcf2 expression on EspF-induced caspase cleavage would be an increase in caspase cleavage due to the reduced inhibitory effect of Abcf2 on caspase 9.",
\end{tcolorbox}
\begin{figure}[!h]
    \vspace{0.01cm}
    \caption{Faulty Reasoning occurs when the reasoning path is correct, but there are significant errors in deducing the events within that pathway. In this case, the model's reasoning step proposes that EspF binds to Abcf2 and activates it, which is incorrect, as it actually inhibits Abcf2. As a result, the reasoning leads to an incorrect conclusion.}
\end{figure}

\textbf{Faulty in Reasoning of PathSeeker}

\begin{tcolorbox}[breakable]
"Question": "Bacterial pathogens often manipulate host immune pathways to establish acute and chronic infection, and many Gram-negative bacteria do this by secreting effector proteins through a type III secretion system that alter the host response to the pathogen. In the case of Salmonella, the phage-encoded GogB effector protein targets the host SCF E3 type ubiquitin ligase through an interaction with Skp1 and the human F-box only 22 (FBXO22) protein. What is the effect of GogB-deficient Salmonella on tissue damage and colonization in the gut during long-term chronic infections?",

"Answer": "GogB-deficient Salmonella cause extensive tissue damage and enhanced colonization in the gut during long-term chronic infections.",

\begin{tikzpicture}
\draw[dashed] (0,0) -- (\linewidth,0); % 绘制一条宽度为\linewidth的虚线
\end{tikzpicture}


"Wrong Answer by Model": "Based on the provided pathways, the effect of GogB-deficient Salmonella on tissue damage and colonization in the gut during long-term chronic infections can be inferred. 

Pathway 0 indicates that the GogB effector protein targets the host SCF E3 type ubiquitin ligase through an interaction with Skp1 and the human F-box only 22 (FBXO22) protein. This pathway is associated with Salmonella infection.

Pathways 1 and 2 provide information about other type III secretion system effectors (SseK1 and SseK3) and their interactions with TNF-NFKB signaling pathways. These pathways are also linked to Salmonella infection.

Pathway 3 shows the involvement of tumor necrosis factor (TNF) and its receptor TNFRSF1A in various signaling pathways, including Salmonella infection.

Pathway 4 indicates the activation of NFKB, a transcription factor involved in inflammation and immune responses, by TNF. This pathway is associated with Salmonella infection.

Pathways 5 and 6 provide information about the activation and inhibition of NFKB inhibitor alpha (NFKBIA) and its interaction with NFKB. These pathways are also linked to Salmonella infection.

Pathway 7 shows the activation of MAPK8 (JNK) and its phosphorylation by FOS, which is associated with Salmonella infection.

Based on the provided pathways, it can be inferred that GogB-deficient Salmonella would have a reduced ability to manipulate host immune pathways and alter the host response to the pathogen. This could result in decreased tissue damage and colonization in the gut during long-term chronic infections. However, the specific effects on tissue damage and colonization would require further investigation.",

\begin{tikzpicture}
\draw[dashed] (0,0) -- (\linewidth,0); % 绘制一条宽度为\linewidth的虚线
\end{tikzpicture}


"Reference Reasoning Process": "Based on the pathways provided, the effect of GogB-deficient Salmonella on tissue damage and colonization in the gut during long-term chronic infections can be inferred.

GogB is a phage-encoded effector protein that targets the host SCF E3 type ubiquitin ligase through an interaction with Skp1 and the human F-box only 22 (FBXO22) protein (Pathway 0). This interaction inhibits the NF-kappaB signaling pathway, which is a crucial pathway involved in inflammation and immune response.

In the absence of GogB, the inhibition of the NF-kappaB signaling pathway is lifted, leading to an increase in the production of pro-inflammatory cytokines, such as TNF-alpha (Pathway 3). This increase in inflammation can lead to tissue damage and exacerbate the symptoms of the infection.

Furthermore, the NF-kappaB signaling pathway is also involved in the regulation of apoptosis (programmed cell death). The inhibition of this pathway by GogB can lead to a decrease in apoptosis, allowing the infected cells to survive and potentially leading to persistent infection (Pathway 6).

Additionally, GogB-deficient Salmonella may also have altered interactions with other host proteins, such as SifA, PipB2, and SopD2, which are involved in regulating microtubule dynamics and transport (Pathways 7, 9, and 10). These altered interactions can affect the ability of Salmonella to colonize and persist in the gut.

Overall, the loss of GogB in Salmonella is likely to lead to increased tissue damage and inflammation during long-term chronic infections, as well as altered interactions with host proteins that can affect the ability of the bacteria to colonize and persist in the gut.

In conclusion, the effect of GogB-deficient Salmonella on tissue damage and colonization in the gut during long-term chronic infections is likely to be an increase in tissue damage and inflammation, as well as altered interactions with host proteins that can affect the ability of the bacteria to colonize and persist in the gut.",
\end{tcolorbox}
\begin{figure}[!h]
    \vspace{0.01cm}
    \caption{Faulty Reasoning occurs when the reasoning path is correct, but there are significant errors in deducing the events within that pathway. In this case, the model reasoned along the NF-kappaB signaling pathway but failed to deduce that a GogB deficiency leads to an increase in inflammation. Consequently, the final conclusion is incorrect.}
\end{figure}


% \twocolumn

\subsection{Uncertainty Measure}

\textbf{Uncertainty} We investigate whether graph augmentation can reduce the uncertainty in model responses. We measure this uncertainty in the discriminant task by calculating five times the entropy of the final results, as shown in Figure \ref{fig entropy}. Notably, graph augmentation reduces prediction uncertainty for LLaMA3-8B but not for GPT-3.5. This discrepancy may arise because GPT-3.5 tends to have fewer hallucinations, whereas LLaMA3-8B may exhibit overconfidence in some generations where it is uncertain.

\begin{figure*}[!t]
    \centering    
    \renewcommand{\thesubfigure}{} % Hide subfigure labels
    \subfigure[]{\includegraphics[width=0.48\linewidth]{fig/uncertainty.pdf}}
    \caption{Uncertainty measured by entropy.}
    \label{fig entropy}
    % \vspace{-mm} 
\end{figure*}

\textbf{} \label{subsection evaluator quality}


\subsection{Evaluation of the Evaluator Quality}

As the generation task involves LLMs as evaluators, we assess the quality of the evaluation method by comparing the result with the human manual annotation score. The accuracy of LLaMA3.1 405B with human annotation is $96\%$, while the inconsistency lines in the case that the answer is close to the ground truth but the expression is general and missing details.

% "Question": "Apelin-13 is known to play a role in the migration and proliferation of vascular smooth muscle cells (VSMCs), but the mechanisms underlying this role are still unclear. Egr-1 is a nuclear transcription factor that is considered to be a critical factor in the processes of VSMC proliferation and migration. Egr-1 regulates the expression of osteopontin (OPN), which is a marker of phenotypic modulation necessary for VSMC proliferation and migration. To investigate the role of Apelin-13 in relation to Egr-1, the effects of Apelin-13 treatment on Egr-1 mRNA and protein expression in A10 rat aortic VSMCs were analyzed. Will Apelin-13 treatment upregulate the expression of Egr-1 in A10 rat aortic VSMCs?",
%     "Answer": "Apelin-13 upregulates the expression of Egr-1 in A10 rat aortic VSMCs.",
%     "Model_Answer": "Based on the provided pathways, it is difficult to directly infer the effect of Apelin-13 treatment on the expression of Egr-1 in A10 rat aortic VSMCs. None of the pathways specifically mention the relationship between Apelin-13 and Egr-1 expression in VSMCs.\n\nHowever, pathway 0 indicates that the Apelin signaling pathway is involved in the regulation of various factors, including mitogen-activated protein kinase kinase kinase 14 (MAP3K14) and KLF transcription factor 2 (KLF2). Egr-1 is known to be a critical factor in VSMC proliferation and migration, and its expression is regulated by KLF2.\n\nTherefore, based on the indirect evidence provided by pathway 0, it is possible that Apelin-13 treatment may upregulate the expression of Egr-1 in A10 rat aortic VSMCs. However, further research is needed to confirm this hypothesis.",
%     "Judge": "The question asks about whether Apelin-13 treatment will upregulate the expression of Egr-1 in A10 rat aortic VSMCs. For this question, the standard answer states that Apelin-13 upregulates the expression of Egr-1 in A10 rat aortic VSMCs. The new answer suggests that it is possible that Apelin-13 treatment may upregulate the expression of Egr-1 in A10 rat aortic VSMCs, but does not provide a definitive conclusion. Therefore, the judgement is 'Incorrect'. \n\nJudge: Incorrect"
%   },

% "Question": "Chronic hepatitis B virus (HBV) infection is a leading cause of liver cirrhosis and cancer, and the HBV X protein (HBx) is a multifunctional regulator that modulates cell inflammation and apoptosis. To explore the effect of HBx on oxidative stress-induced apoptosis in normal liver cell line, HL-7702, the cells were exposed to oxidative stress and the impact of HBx on mitochondrial biogenesis was examined. What is the effect of HBx on mitochondrial biogenesis in HL-7702 cells during exposure to oxidative stress?",
%     "Answer": "HBx affects mitochondrial biogenesis by modulating the opening of the mitochondrial permeability transition pore (MPTP) in normal liver cell line, HL-7702.",
%     "Model_Answer": "The effect of HBx on mitochondrial biogenesis in HL-7702 cells during exposure to oxidative stress is an inhibition of mitochondrial biogenesis. HBx, the HBV X protein, is known to modulate cell inflammation and apoptosis. In this study, when HL-7702 cells were exposed to oxidative stress, HBx was found to have a negative impact on mitochondrial biogenesis. This means that HBx suppresses the formation of new mitochondria in the cells, which can contribute to the development of liver cirrhosis and cancer in chronic HBV infection.",
%     "Judge": "The question asks about the effect of HBx on mitochondrial biogenesis in HL-7702 cells during exposure to oxidative stress. The standard answer states that HBx affects mitochondrial biogenesis by modulating the opening of the mitochondrial permeability transition pore (MPTP) in normal liver cell line, HL-7702. The new answer, however, suggests that the effect of HBx on mitochondrial biogenesis is an inhibition of mitochondrial biogenesis. The two answers do not align, as the new answer does not mention anything about the modulation of the MPTP. Therefore, the judgement is 'Incorrect'. \nJudge: Incorrect"
%   },


% \subsection{Performance on Biological Domains} \label{appendix bio radar full}

% \begin{figure*}[!t]
%     \centering    
%     \renewcommand{\thesubfigure}{} % Hide subfigure labels
%     \subfigure[]{\includegraphics[width=0.48\linewidth]{fig/plot_radar_map_judge_GPT3.5.pdf}} 
%     \subfigure[]{\includegraphics[width=0.48\linewidth]{fig/plot_radar_map_reasoning_GPT3.5.pdf}} 
%     \subfigure[]{\includegraphics[width=0.48\linewidth]{fig/plot_radar_map_judge_LLaMA3 8B.pdf}} 
%     \subfigure[]{\includegraphics[width=0.48\linewidth]{fig/plot_radar_map_reasoning_LLaMA3 8B.pdf}} 
%     \caption{The radar chart of both GPT-3.5 and LLaMA3 8B performance on \benchname's different biological domain.}
%     \label{fig bio radar full}
% \end{figure*}

% Figure \ref{fig bio radar full} illustrates both GPT-3.5 and LLaMA3 8B performance on \benchname's different biological domains. 

% \subsection{More Ablation Study}
% \begin{table}[!h]
% \resizebox{0.99\linewidth}{!}{
% \begin{tabular}{ccccccc}
%           & \modelname & w.o. RemoveSeen & w.o. DFSOrder & w.o. TripleToText & w.o. Local search & w.o. FinalReasoner \\
% True/False     & 79.24      & 76.57           & 77.4          & 77.68             & 77.29             & 75.33              \\
% Open-Ended & 79.97      & 77.52           & 77.02         & 78.31             & 76.27             & 71.86             
% \end{tabular}
% }
% \caption{Ablation Study of \modelname.}
% \label{table ablation}
% \end{table}




\subsection{Introduction of Pathway}

Understanding biological systems is inherently complex due to the numerous interacting molecules, processes, and environmental factors involved. These systems operate with intricate interactions that result in non-linear, multi-layered, and dynamic behaviors. To address this complexity, biological researchers use pathway graphs as structured blueprints to simplify these systems into organized structures that consist of basic interactions. The linear reactions, cyclical relationships, or the local network of pathways offer snapshots of how a system behaves under specific conditions and enable researchers to predict how changes in one molecule or interaction can affect the entire system. Pathway graphs also provide a structured, static representation of dynamic processes, helping researchers understand the sequence of events even as the system changes over time. 

\subsection{Detailed Related Work} \label{appendix related work}

% Large Language Models (LLMs) have recently shown impressive performance in science across various domains, including mathematics~\citep{yu2023metamath}, chemistry~\citep{liu2023multi}, biology~\citep{hayes2024simulating,madani2020progen}, and materials science~\citep{zheng2023shaping,park2024multi}. In the biological domain specifically, recent studies have demonstrated the potential of LLMs in tackling challenging tasks such as protein design \citep{valentini2023promises, hosseini2024text2protein}, drug discovery \citep{m2024augmenting,liu2023chatgpt}, clinical trial analysis \citep{singhal2023large, jin2023matching}, and experiment design \citep{ai4science2023impact}. 

% \textbf{LLMs in AI for Science} Large language models (LLMs) have illustrated impressive performance in AI for Science including various domains, such as Math~\citep{yu2023metamath}, Chemistry~\citep{liu2023multi}, Biology~\citep{madani2020progen,liu2023text,hosseini2024text2protein,hayes2024simulating}, Medicine~\citep{liu2023chatgpt}, and Material~\citep{zheng2023shaping,park2024multi}. Specifically for the biological domain, current studies have implied LLMs in challenging tasks like protein design \citep{valentini2023promises}, drug discovery \citep{m2024augmenting,liu2023chatgpt}, clinical trials \citep{singhal2023large, jin2023matching}, experiment design \citep{ai4science2023impact}, and others.

% \textbf{Biological Scientific Question Answering} Previous studies have explored the potential of language models in the biological scientific domain \citep{lu2022learn, vilares2019head, jin2021disease, pal2022medmcqa}. MEDHOP \citep{welbl2018constructing} and PubMedQA \citep{jin2019pubmedqa} investigated biological scientific QA in the form of reading comprehension. BioASQ-QA \citep{krithara2023bioasq} proposed a realistic question-answering benchmark for the information needs of biomedical experts. A few studies have examined language models' ability to understand biological pathways. Chatpathway \citep{li2023chatpathway}, \cite{park2023comparative}, and \cite{azam2024comprehensive} specifically investigated language models' capacity for completing biological pathways. Different from previous tasks, this work introduces a novel task for practical biological pathway phenomenon reasoning, addressing the reasoning capacity of LLMs.


\textbf{Biological Scientific Question Answering} Previous studies have explored the potential of language models in the biological scientific domain. MEDHOP \citep{welbl2018constructing} and PubMedQA \citep{jin2019pubmedqa} investigated biological scientific question answering in the form of reading comprehension. BioASQ-QA \citep{krithara2023bioasq} proposed a realistic question-answering benchmark for the actual information needs of biomedical experts. Beyond textual QA, several works have also studied multimodal scientific ability \citep{lu2022learn}. Additionally, other studies have explored biomedical domain tasks \citep{vilares2019head, jin2021disease, pal2022medmcqa}. Most existing tasks in the biological sciences concentrate on knowledge probing, assessing how well models understand biological information. In contrast, our work, \benchname, is the first to focus on models' reasoning abilities within the biological scientific domain, specifically targeting phenomena observed in experiments about biological pathways.

A few studies have examined language models' ability to understand biological pathways. Chatpathway \citep{li2023chatpathway} and \cite{azam2024comprehensive} specifically investigated language models' capacity for completing biological pathways. However, these studies mainly focus on probing biological pathway knowledge, i.e., determining whether language models possess the relevant pathway information. In contrast, this work introduces a novel task that employs pathway models for practical biological pathway phenomenon reasoning, bridging the gap between pathway network knowledge and its implications. We compare this work with previous biological datasets in Table \ref{appendix related work dataset table}.


\begin{table}[!h]
\caption{Comparison of previous biological scientific question answering tasks and \benchname.}
\label{appendix related work dataset table}
\resizebox{\linewidth}{!}{
\begin{tabular}{llll}
\toprule
& Domain                             & Question Form      &  Task Types                \\
\midrule
MEDHOP \cite{welbl2018constructing} & Scientific                         & Choice             & Multi-hop reading comprehension \\
PubMedQA \cite{jin2019pubmedqa}     & Scientific                         & True/False              &    Reading comprehension                                \\
HEAD-QA \cite{vilares2019head}      & {\color[HTML]{1F2328} Healthcare}  & Choice             & Knowledge probing and reasoning \\

MedQA \cite{jin2021disease}         & Medical                            & Choice             & Reading comprehension           \\
MedMCQA \cite{pal2022medmcqa}       & Medical                            & Choice             & Knowledge probing and reasoning \\

BioASQ-QA \cite{krithara2023bioasq} & Scientific & True/False and Open-Ended & Knowledege probing \\


ChatPathway \cite{li2023chatpathway}            & Pathway and biochemical & Open-Ended         & Knowledge probing               \\
\cite{azam2024comprehensive}        & Pathway and gene                   & Choice             & Knowledge probing               \\
\benchname (Ours)                         & Pathway for the scientific & True/False and Open-Ended & Reasoning in biological pathway        \\
\bottomrule
\end{tabular}
}
\end{table}

\textbf{Graph-augmented Language Model}
Several studies have explored augmenting large language models (LLMs) with graph data. In particular, some works enhance LLMs by encoding graph data as text \citep{ye2023natural, wang2024can, fatemi2023talk}, or tuning LLMs specifically for graph-based tasks \citep{liu2023one, tang2024graphgpt, he2024g, zhao2023gimlet, he2024unigraph}. By augmenting LLMs with graph data, they have been applied to knowledge-based QA \citep{sun2023think, he2024g, li2023chain, jin2024graph, cheng2024call}, and to graph-oriented tasks like graph property prediction \citep{wang2024can, he2023harnessing}. A few other studies leverage graph structures during LLM reasoning to tackle complex tasks \citep{jiang2023structgpt, besta2024graph}.

Unlike tasks in previous works, this study addresses whether reasoning in biological systems can be enhanced by pathway graphs, which act as a \textit{structured blueprint} for reasoning about the system’s states. It is not sufficient to simply identify the correct paths in the pathway graph to find the answer. Instead, it is necessary to perform deductive reasoning about the events that occur when the system is intervened upon under specific conditions and to predict the resulting states and mechanisms of the intervened system.

For large graph databases, most works enable LLMs to access graph data through retrieval mechanisms \citep{he2024g, li2023chain}, while a few studies have explored using LLMs as interactive agents \citep{yao2023react, shinn2023reflexion, zhao2024empowering} to navigate and explore vast graph databases \citep{sun2023think, jin2024graph}. In this work, we introduce an agent-based interactive graph exploration approach using subgraph navigation-based browsing, which is more efficient and offers enhanced navigation capabilities for pathway database.