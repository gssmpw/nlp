%% 
%% Copyright 2007-2024 Elsevier Ltd
%% 
%% This file is part of the 'Elsarticle Bundle'.
%% ---------------------------------------------
%% 
%% It may be distributed under the conditions of the LaTeX Project Public
%% License, either version 1.3 of this license or (at your option) any
%% later version.  The latest version of this license is in
%%    http://www.latex-project.org/lppl.txt
%% and version 1.3 or later is part of all distributions of LaTeX
%% version 1999/12/01 or later.
%% 
%% The list of all files belonging to the 'Elsarticle Bundle' is
%% given in the file `manifest.txt'.
%% 
%% Template article for Elsevier's document class `elsarticle'
%% with numbered style bibliographic references
%% SP 2008/03/01
%% $Id: elsarticle-template-num.tex 249 2024-04-06 10:51:24Z rishi $
%%
\documentclass[preprint,review,10pt]{elsarticle}
 
\usepackage{graphicx} % Required for inserting images
\usepackage{amsmath}
\usepackage{amsfonts}
\usepackage{xcolor}
\usepackage{amssymb}
\usepackage{caption}
\usepackage{float}
\usepackage{algorithm}
\usepackage{algorithmic}
\usepackage{tabularray}
\usepackage{rotating}

%% Use the option review to obtain double line spacing
%% \documentclass[authoryear,preprint,review,12pt]{elsarticle}

%% Use the options 1p,twocolumn; 3p; 3p,twocolumn; 5p; or 5p,twocolumn
%% for a journal layout:
%% \documentclass[final,1p,times]{elsarticle}
%% \documentclass[final,1p,times,twocolumn]{elsarticle}
%% \documentclass[final,3p,times]{elsarticle}
%% \documentclass[final,3p,times,twocolumn]{elsarticle}
%% \documentclass[final,5p,times]{elsarticle}
%% \documentclass[final,5p,times,twocolumn]{elsarticle}

%% For including figures, graphicx.sty has been loaded in
%% elsarticle.cls. If you prefer to use the old commands
%% please give \usepackage{epsfig}

%% The amssymb package provides various useful mathematical symbols
\usepackage{amssymb}
%% The amsmath package provides various useful equation environments.
\usepackage{amsmath}
%% The amsthm package provides extended theorem environments
%% \usepackage{amsthm}
\newcommand*\Laplace{\mathop{}\!\mathbin\bigtriangleup}
\newcommand\restr[2]{\ensuremath{\left.#1\right|_{#2}}}

%% The lineno packages adds line numbers. Start line numbering with
%% \begin{linenumbers}, end it with \end{linenumbers}. Or switch it on
%% for the whole article with \linenumbers.
%% \usepackage{lineno}

%\journal{Composites Part A}

\begin{document}



\begin{frontmatter}

%% Title, authors and addresses

%% use the tnoteref command within \title for footnotes;
%% use the tnotetext command for theassociated footnote;
%% use the fnref command within \author or \affiliation for footnotes;
%% use the fntext command for theassociated footnote;
%% use the corref command within \author for corresponding author footnotes;
%% use the cortext command for theassociated footnote;
%% use the ead command for the email address,
%% and the form \ead[url] for the home page:
%% \title{Title\tnoteref{label1}}
%% \tnotetext[label1]{}
%% \author{Name\corref{cor1}\fnref{label2}}
%% \ead{email address}
%% \ead[url]{home page}
%% \fntext[label2]{}
%% \cortext[cor1]{}
%% \affiliation{organization={},
%%             addressline={},
%%             city={},
%%             postcode={},
%%             state={},
%%             country={}}
%% \fntext[label3]{}

\title{Hybrid machine learning based scale bridging framework for permeability prediction of fibrous structures}

%% use optional labels to link authors explicitly to addresses:
%% \author[label1,label2]{}
%% \affiliation[label1]{organization={},
%%             addressline={},
%%             city={},
%%             postcode={},
%%             state={},
%%             country={}}
%%
%% \affiliation[label2]{organization={},
%%             addressline={},
%%             city={},
%%             postcode={},
%%             state={},
%%             country={}}

%% Author name
\author [label1] {Denis Korolev} 
\author [label2] {Tim Schmidt}
\author [label3] {Dinesh K. Natarajan}
\author [label2] {Stefano Cassola}
\author [label4] {David May}
\author [label2] {Miro Duhovic}
\author [label1,label5] {Michael Hintermüller}

%% Author affiliation
\affiliation[label1]{organization={Weierstrass-Institute for Applied Analysis and Stochastics},%Department and Organization
            addressline={Mohrenstr. 39}, 
            city={Berlin},
            postcode={10117}, 
            country={Germany}}

\affiliation[label2]{organization={Leibniz-Institut für Verbundwerkstoffe GmbH},%Department and Organization
            addressline={Erwin-Schrödinger-Str. 58}, 
            city={Kaiserslautern},
            postcode={67663}, 
            country={Germany}}
            
\affiliation[label3]{organization={German Research Center for Artificial Intelligence GmbH (DFKI)},%Department and Organization
            addressline={Trippstadter Str. 122}, 
            city={Kaiserslautern},
            postcode={67663}, 
            country={Germany}}
\affiliation[label4]{organization={Faserinstitut Bremen e.V. (FIBRE) University of Bremen},%Department and Organization
            addressline={Am Biologischen Garten 2, Geb. IW 3}, 
            city={Bremen},
            postcode={28359}, 
            country={Germany}}

\affiliation[label5]{organization={Institute for Mathematics, Humbolt-Universität zu Berlin},%Department and Organization
            addressline={Unter den Linden 6}, 
            city={Berlin},
            postcode={10099}, 
            country={Germany}}
            
%% Abstract
\begin{abstract}
%Determining the mesoscale permeability of a textile stack is inherently challenging due to its dual scale structure. In flow simulations, microscale and mesoscale flows need to be taken into account for an accurate physical and geometric representation, which is usually achieved by a separation of scale combined with up-and downscaling operations. In this work, we compare several methods for mesoscale permeability determination, outlining the trade-off between computational cost and accuracy. Therefore, a scale bridging method was developed that leverages data driven methods for an efficient upscaling of information (in this case permeability) from micro- to mesoscale. Lastly, we introduce a novel multi-fidelity framework that exploits the interdependency of scales in such materials in order to enable a two-way flow of information between scales. This approach employs physics-informed neural networks (PINN) on the microscale coupled with a robust numerical solver on the mesoscale. This hybrid-PINN approach aims to be a first step towards advanced hybrid modeling showing a high potential in the presented 2D flow scenarios. 

%Determining the mesoscale permeability of textile stacks is challenging due to their dual-scale structure. Accurate flow simulations require capturing both microscale and mesoscale dynamics, typically achieved through scale separation combined with upscaling and downscaling operations. This work investigates methods for mesoscale permeability prediction, focusing on the trade-offs between computational cost and accuracy. We propose a scale-bridging framework that employs data-driven methods to efficiently upscale microscale information, such as permeability, to the mesoscale. Building on this, we introduce a novel multi-fidelity hybrid framework that leverages the interdependence of scales to enable a two-way flow of information. This approach integrates physics-informed neural networks (PINNs) at the microscale with a robust numerical solver at the mesoscale, enhancing the accuracy and efficiency of multiscale simulations. Using 2D flow scenarios in fiber-reinforced composites, we demonstrate the potential of the hybrid-PINN framework to address limitations of traditional methods, such as slow convergence and high computational costs. By coupling data-driven models with numerical methods, our approach represents a step forward in hybrid modeling, offering improved accuracy for multiscale problems and enabling effective scale bridging. This work lays the foundation for more advanced hybrid approaches to tackle the challenges of complex material systems.
%In this work, we have twofold objectives: first, to investigate the effective application of deep learning to scale bridging; second, to enhance the accuracy of deep learning surrogate modeling approaches by leveraging the multiscale structure of the use case itself. Using fiber-reinforced composites as a case study, we propose data-driven scale-bridging methods for permeability prediction and a dual-scale hybrid framework that integrates data-driven approaches with numerical methods. This framework employs physics-informed neural networks and scale-bridging techniques, referred to as upscaling and downscaling. The unified approach demonstrates significant potential for improving the accuracy and efficiency of multiscale simulations.

This study introduces a hybrid machine learning-based scale-bridging framework for predicting the permeability of fibrous textile structures. By addressing the computational challenges inherent to multiscale modeling, the proposed approach evaluates the efficiency and accuracy of different scale-bridging methodologies combining traditional surrogate models and even integrating physics-informed neural networks (PINNs) with numerical solvers, enabling accurate permeability predictions across micro- and mesoscales. Four methodologies were evaluated: Single Scale Method (SSM), Simple Upscaling Method (SUM), Scale-Bridging Method (SBM), and Fully Resolved Model (FRM). SSM, the simplest method, neglects microscale permeability and exhibited permeability values deviating by up to 150\% of the FRM model, which was taken as ground truth at an equivalent lower fiber volume content. SUM improved predictions by considering uniform microscale permeability, yielding closer values under similar conditions, but still lacked structural variability. The SBM method, incorporating segment-based microscale permeability assignments, showed significant enhancements, achieving almost equivalent values while maintaining computational efficiency and modeling runtimes of ~45 minutes per simulation. In contrast, FRM, which provides the highest fidelity by fully resolving microscale and mesoscale geometries, required up to 270 times more computational time than SSM, with model files exceeding 300 GB. Additionally, a hybrid dual-scale solver incorporating PINNs has been developed and shows the potential to overcome generalization errors and the problem of data scarcity of the data-driven surrogate approaches. The hybrid framework advances permeability modelling by balancing computational cost and prediction reliability, laying the foundation for further applications in fibrous composite manufacturing.
\end{abstract}

%%Graphical abstract
%\begin{graphicalabstract}

%\begin{figure}[H]
%    \centering
%    \includegraphics[width=1\linewidth]{Figures/Graphical_abstract.jpg}
%\end{figure}
%\includegraphics{grabs}
%\end{graphicalabstract}

%%Research highlights
%\begin{highlights}
%\item Enhanced methodologies for permeability prediction of textile-based fiber structures

%\item Evaluation of four different methods for determining the mesoscale permeability of 3D models

%\item Dual scale permeability prediction with novel hybrid physics-informed neural network 

%\item Application of hybrid PINN solver for permeability prediction of 2D fiber structures

%\end{highlights}

%% Keywords
\begin{keyword}  A. Fabrics/textiles \sep A. Tow \sep B. Permeability \sep C. Computational modelling \sep E. Resin flow 
%% keywords here, in the form: keyword \sep keyword

%% PACS codes here, in the form: \PACS code \sep code

%% MSC codes here, in the form: \MSC code \sep code
%% or \MSC[2008] code \sep code (2000 is the default)

\end{keyword}

\end{frontmatter}

%% Add \usepackage{lineno} before \begin{document} and uncomment 
%% following line to enable line numbers
%% \linenumbers

%% main text
%%

\section{Introduction} 

%The manufacturing of fiber-reinforced polymers (FRP) entails combining a fiber reinforcement structure with a matrix polymer, typically by infiltrating the fiber structure with a resin system driven by a pressure difference. The prediction of resin flow within this structure requires the consideration of multiple scales: on the microscale, flow between impermeable fibers is simulated; on the mesoscale, flow between permeable rovings becomes relevant; and on the macroscale filling of the mold is considered \cite{lomov2001textile}. Since a simulation covering multiple scales at once requires immense computing resources, the scales are typically considered separately. However, for scale separation, required data must be exchanged between the scale levels. For instance, structural information from meso- to microscale (referred to as downscaling) and, reversely, the microscale permeability must be assigned to the permeable rovings at the mesoscale (referred to as upscaling) as shown in Fig. \ref{fig:enter-label1}. This combination of down- and upscaling, described in Section 2 \ref{Section: Upscaling and Downscaling}, can be referred to as scale bridging \cite{lubbers2020modeling}. The data and information are homogenized for the exchange between the scale levels and in addition, scale separation and simulation lead to simplifications in modelling of underlying physics \cite{gorguluarslan2014simulation}. According to the state of the art, the permeability determination requires multiple computationally expensive flow simulations. In addition, the assignment of permeability to the permeable rovings at the higher scale level is a complex task, as the fiber structures locally can differ significantly \cite{seuffert2021micro}, e.g. regarding fiber orientation, which only increases the effort.

The manufacturing of fiber-reinforced polymers (FRP) entails combining a fiber reinforcement structure with a matrix polymer, typically by infiltrating the fibers with a resin system driven by a pressure difference. Predicting resin flow within this structure requires considering multiple scales: on the microscale, flow between impermeable fibers is simulated; on the mesoscale, flow between permeable rovings becomes relevant; and on the macroscale, mold filling is considered \cite{lomov2001textile}. Since simulating multiple scales simultaneously requires immense computing resources, they are typically treated separately. However, data must be exchanged between scales: structural information from meso- to microscale (referred to as downscaling) and, conversly, the assignment of microscale permeability to the permeable rovings at the mesoscale (referred to as upscaling), as shown in Fig.\ref{fig:enter-label1}. This combination of down- and upscaling , referred as scale bridging \cite{lubbers2020modeling}, requires the homogenization of data for the exchange between scales. In addition, scale separation and simulation lead to simplifications in the underlying physics \cite{gorguluarslan2014simulation}. These simplifications, however, are necessary to reduce the computational costs. Still, state-of-the-art permeability determination methods require multiple computationally expensive simulations, and assigning permeability to rovings at the higher scale is complex due to significant local variations in fiber structure, such as fiber orientation, which only increases the effort \cite{seuffert2021micro}.

\begin{figure}[H]
    \centering
    \includegraphics[width=1\linewidth]{Figures/Chapter1/Figure1_Three_scales_V4.jpg}
    \caption{Relevant scale levels for permeability prediction.} 
    \label{fig:enter-label1}
\end{figure}


Scientific machine learning (SciML) appears capable of aiding the scale bridging process, but its success depends on overcoming the problem that most of the surrogate modelling approaches struggle with: to effectively bridge scales while incorporating physical laws, data-driven insights from experimental observations, and scale relations into a unifying framework. In this work, Section \ref{Section: Upscaling and Downscaling} first presents the mathematical basis for upscaling and downscaling. In Sections \ref{Section: Methods for dual scale permeability prediction} and \ref{Section: comparison of scale-bridging approaches}, data-driven scale-bridging methods for permeability prediction are presented, although their accuracy is fundamentally constrained by generalization error, limited training data, and simplifications in flow models and homogenization techniques for data generation. To overcome these limitations, following the multiscale approach \cite{hintermuller2023hybrid}, Section \ref{Section: Hybrid physics-informed dual scale framework for upscaling} introduces a dual-scale solver that combines PINNs \cite{raissi2019physics} for the microscale and a robust numerical solver for the mesoscale.
The dual-scale solver integrates data from both scales, including available permeability prediction values, into a multi-fidelity framework to further improve the predictions through physics-informed regularization.
%Our permeability determination process is hierarchical, involving one-way coupled partial differential equation (PDE) systems of Stokes and Stokes-Brinkman type at the microscale and mesoscale, respectively. Following \cite{hintermuller2023hybrid}, we modify the PINN objective with a coupling term, making our  hierarchical upscaling approach two-way coupled (dual scale). It allows data integration from both scales, including available permeability prediction values, into a multi-fidelity framework to further improve these predictions through physics-informed regularization.


\section{Upscaling and Downscaling}
\label{Section: Upscaling and Downscaling}

For the textile stack model $\Omega_{\text{Me}} \subset \mathbb{R}^{d}$ and its microscopic representation $\Omega_{\text{Mi}}\subset \mathbb{R}^{d}$, let $\mathcal{M}_{\text{Mi}}=\{ \mathcal{V}   \mid  \mathcal{V} \subseteq \Omega_{\text{Mi}} \}$ and $\mathcal{M}_{\text{Me}}=\{\mathcal{Z}   \mid  \mathcal{Z} \subseteq \Omega_{\text{Me}} \}$ denote the sets of  micro- and mesoscale geometry models, respectively. In mesomodels $\mathcal{Z} \in \mathcal{M}_{\text{Me}}$, represented as mesoscale statistical volume elements (SVE) in the form of textile stack cut-outs, porous rovings are modeled as a continuum, and the SVEs are divided into fluid parts $\mathcal{Z}^{F}$ and porous parts $\mathcal{Z}^{P}$, such that $\mathcal{Z} = \mathcal{Z}^{P} \cup \mathcal{Z}^{F}$. The resolution operator
\begin{align}
\mathcal{F}_{\downarrow}:  \ \mathcal{M}_{\text{Me}} \rightarrow \mathcal{M}_{\text{Mi}} \quad \text{with} \quad  \mathcal{Z}  \mapsto \mathcal{V}_{\mathcal{Z}}:=\mathcal{F}_{\downarrow}(\mathcal{Z}), 
\end{align}
transfers mesomodels into their respective micromodels $\mathcal{V}_{\mathcal{Z}} \in \mathcal{M}_{\text{Mi}}$, corresponding to fully resolved mesoscale SVEs, as in Fig. \ref{fig:enter-label1}. The micromodels are divided into fluid $\mathcal{V}_{\mathcal{Z}}^{F}$ and solid $\mathcal{V}_{\mathcal{Z}}^{S}$ parts. The microstructures $\mathcal{V}_{\mathcal{Z}}^{S}$ (bundles of impermeable fibers) are derived from the mesostructure of $\mathcal{Z}$. The data derived from $\mathcal{Z}$ include the local orientation of the roving, which roughly corresponds to the fiber angle, and the fiber volume content (FVC) in the roving, which is calculated from the roving cross-sectional area, the number of fibers and the fiber diameter. With these features, suitable microscale SVEs as statistical approximations of resolved mesomodels can be generated as well, reducing simulation costs. This transfer of information from the mesoscale to the microscale is referred to as \textbf{downscaling}.

The flow of information from the micro- to the mesoscale is called \textbf{upscaling}. For its realization, we define the upscaling operator 
\begin{align}
\mathcal{F}_{\uparrow}: \  \mathcal{M}_{\text{Mi}} \rightarrow \mathbb{R}^{d \times d} \quad \text{with} \quad  \mathcal{V}_{\mathcal{Z}} \mapsto  \boldsymbol{K}^{p}[\mathcal{V}_{\mathcal{Z}}],
\end{align}
which extracts the geometric information about the microstructures of $\mathcal{V}_{\mathcal{Z}}$ in the form of the permeability tensor $\boldsymbol{K}^{p}[\mathcal{V}_{\mathcal{Z}}]$ (also referred to as micropermeability) to characterize the porous part $\mathcal{Z}^{P}$ of $\mathcal{Z}$. Computing $\boldsymbol{K}^{p}[\mathcal{V}_{\mathcal{Z}}]$ involves solving the Darcy's law algebraic system:
\begin{align}\label{Darcy's law}
\boldsymbol{U} = -\frac{1}{\mu}\boldsymbol{K}^{p}[\mathcal{V}_{\mathcal{Z}}] \boldsymbol{PD},
\end{align}
where $\mu$ is the dynamic viscosity of the fluid, $\boldsymbol{U} \in \mathbb{R}^{d \times d}$ and $\boldsymbol{PD} \in \mathbb{R}^{d \times d}$ are the volume-averaged flow velocity and pressure drop matrices with
\begin{align}\label{averaging process}
\begin{aligned}
\boldsymbol{U}_{k,j} =  \frac{1}{|\mathcal{V}_{\mathcal{Z}}^{F}|}\int_{\mathcal{V}_{ \mathcal{Z}}^{F}} u_{j}^{(k)} (x) \  dx, \quad 
\boldsymbol{PD}_{k,j}  = \frac{1}{|\mathcal{V}_{\mathcal{Z}}^{F}|} \int_{\mathcal{V}_{\mathcal{Z}}^{F}} \frac{\partial p^{(k)}}{\partial x_{j}}(x)  \  dx.
\end{aligned}
\end{align}
The respective velocity fields $\boldsymbol{u}^{(k)} = [u_{1}^{(k)},\  u_{2}^{(k)}, \  u_{3}^{(k)}]$ and pressures $p^{(k)}$ are obtained by solving three Stokes equations (for $d=3$) at the microscale\footnote{The equation is often written in the dimensionless form with the Reynolds number.}: 
\begin{equation}\label{Stokes equation}
\begin{aligned}
- \mu \Delta \boldsymbol{u}^{(k)} + \nabla p^{(k)} &= \boldsymbol{f}^{(k)} \quad &\text{in} \ \mathcal{V}_{\mathcal{Z}}^{F}, \\
\nabla \cdot \boldsymbol{u}^{(k)} &= 0 \quad &\text{in} \ \mathcal{V}_{\mathcal{Z}}^{F}, \\ 
\boldsymbol{u}^{(k)} &= 0 \quad  &\text{on} \ \partial \mathcal{V}_{\mathcal{Z}}^{S},
\end{aligned}
\end{equation}
where $\partial \mathcal{V}_{\mathcal{Z}}^{S}$ denotes the (no-slip) boundary of $\mathcal{V}_{ \mathcal{Z}}^{S}$ and $\boldsymbol{f}^{(k)}$ is a volume force. For example, one applies $\boldsymbol{f}^{(k)}=\boldsymbol{e}^{k}$, where $\boldsymbol{e}^{k}$ is the unit vector in $k$-th direction, and uses periodic boundary conditions for $\boldsymbol{u}^{(k)}$ and $p^{(k)}$ to get such three solutions; cf. \cite{griebel2010homogenization} and references therein. Solving \eqref{Stokes equation} on a fully resolved model $\mathcal{V}_{\mathcal{Z}}$ of the mesogeometry $\mathcal{Z}$ is  computationally expensive or even prohibitive. Therefore, the SVE approximation of $\mathcal{V}_{\mathcal{Z}}$ is often used instead.


At the mesoscale, flow is modeled by the Stokes-Brinkman equation
\begin{equation}\label{Stokes-Brinkman flow}
\begin{aligned}
- \tilde{\mu} \Delta \boldsymbol{u}^{SB 
 (k)} + \mu \big(\boldsymbol{K}_{\text{Me}}[\mathcal{V}_{\mathcal{Z}}]\big)^{-1} \boldsymbol{u}^{SB  (k)}  + \nabla p^{SB (k)} &=  \boldsymbol{f}^{(k)}  \quad \text{in} \ \mathcal{Z}, \\ 
\nabla \cdot \boldsymbol{u}^{SB  (k)}  & =  0   \quad \quad \ \text{in} \ \mathcal{Z}, 
\end{aligned}
\end{equation}
where $\tilde{\mu}$ is the effective Brinkman viscosity, and $\boldsymbol{K}_{\text{Me}}[\mathcal{V}_{\mathcal{Z}}] = \chi_{\mathcal{Z}^{F}} \cdot \infty + \chi_{\mathcal{Z}^{P}} \cdot \boldsymbol{K}^p[\mathcal{V}_{\mathcal{Z}}]$ ($\chi_{\mathcal{Z}^{F}}$ and $\chi_{\mathcal{Z}^{P}}$ are the indicator functions of $\mathcal{Z}^{F}$ and $\mathcal{Z}^{P}$) represents the permeability properties of different domain regions. For $1\leq k \leq d$, $\boldsymbol{f}^{(k)}=\boldsymbol{e}^{k}$ is applied, along with periodic boundary conditions for $\boldsymbol{u}^{SB (k)}$ and $p^{SB  (k)}$, to obtain three solutions of \eqref{Stokes-Brinkman flow}. The macroscale permeability $\boldsymbol{K}[\mathcal{Z}]$ is then computed using \eqref{Darcy's law} by averaging these solutions over $\mathcal{Z}$. 










%The micro- and mesoscale models in this work consist of voxels, equilateral hexahedron comparable to pixels in images. These voxels can be classified as solid (fiber), fluid or porous (permeable roving). Let the domain $\Omega_{\text{Mi}} \subset \mathbb{R}^{d}$ (where $d=3$ is used for in our data-driven approach, and $d=2$ is chosen to illustrate the hybrid framework) in its microscale representation be partitioned into $N_{L}$ SVEs (statistical volume elements) of size $L>0$, with each SVE denoted by $\mathcal{V}_{j}:=\mathcal{V}_{j}(L)$, $1 \leq j \leq N_{L}$. We define its respective solid part (e.g. fiber) $\mathcal{V}_{j}^{S}$ and its fluid part $\mathcal{V}_{j}^{F}$. In the mesomodels, geometries are  divided into porous or fluid voxels. Indeed, the rovings are modeled as a continuum and the properties of the microstructure within the rovings can be assigned via porous voxels. For the mesoscale domain $\Omega_{\text{Me}}$, we define $\mathcal{Z}_{j} \subset \Omega_{\text{Me}}$ dual to $\mathcal{V}_{j}$: within $\mathcal{Z}_{j}$ the properties of the fibrous microstructures $\mathcal{V}_{j}^{S} \subset \mathcal{V}_{j}$ are contained in the permeability tensor $\boldsymbol{K}^{p}[\mathcal{V}_{j}] \in \mathbb{R}^{d \times d}$ and it holds $\mathcal{Z}_{j} = \mathcal{Z}_{j}^{F} \cup \Lambda_{j} \cup \mathcal{Z}_{j}^{P}$ with 


%\begin{align}\label{mesoscale permeability}
%\boldsymbol{K}_{\text{Me}}[\mathcal{V}_{j}] =
%\begin{cases}
%    \infty \quad &\text{in} \ \mathcal{Z}_{j}^{F}, \\
%    \boldsymbol{K}^{p}[\mathcal{V}_{j}^{L}] \quad & \text{in} \ \mathcal{Z}_{j}^{P},
%\end{cases}
%\end{align}
%that is, the different subdomains at the mesoscale are represented by their characteristic permeability. Here, $\mathcal{Z}_{j}^{P}$ is the porous subdomain, $\mathcal{Z}_{j}^{F}$ is the fluid subdomain, and $\Lambda_{j}$ is the interface between $\mathcal{Z}_{j}^{F}$ and $\mathcal{Z}_{j}^{P}$. 


%In particular, $\mathcal{V}_{j} \in \mathcal{M}_{\text{ad}}^{\text{Mi}}$ and $\mathcal{Z}_{j} \in \mathcal{M}_{\text{ad}}^{\text{Me}}$ for $1 \leq j \leq N_{L}$.  

\section{Methods for dual scale permeability prediction}
\label{Section: Methods for dual scale permeability prediction}

According to the state of the art, the micro- and mesoscale are modeled and simulated separately in order to numerically determine the dual-scale permeability of a textile stack \cite{schmidt2019novel, nedanov2002numerical, syerko2017numerical}. For this, geometry models are required that represent the fiber structure. To ensure that the models account for all relevant structural properties, including realistic variations, studies are conducted in advance with the aim of determining SVEs for the fiber structure, see Fig. \ref{fig:enter-label3} \cite{du2006size}. In addition to the structural features, suitable model resolution, model size, and the number of models must be identified during SVE development to achieve a distribution of permeabilities similar to the experiments \cite{syerko2023benchmark}.

\begin{figure}[H]
    \centering
    \includegraphics[width=1\linewidth]{Figures/Chapter3/Figure3_Micro_Meso_SVE_V3.jpg}
    \caption{Visualisation of the statistical representative volume elements for micro- and mesoscale and the fully resolved model}
    \label{fig:enter-label2}
\end{figure}

In our numerical realization, 3D micro- and mesoscale models were generated in the GeoDict\textsuperscript{\textregistered} software and are voxel-based, with voxels (equilateral hexahedra akin to image pixels) classified as solid, fluid, or porous. The microscale SVE contains several hundred fibers \cite{schmidt4872087geometric}, and the mesoscale SVE contains several textile layers, which are modeled using rovings, see Fig. \ref{fig:enter-label2}. The rovings are taken to represent fiber bundles as a continuum, which allows the use of a coarser resolution. To create the model, individual textile layers with varying roving cross-sections are generated and positioned randomly on top of each other within a defined range, followed by a virtual compaction of the textile stack until the desired stack height or FVC is achieved.  This causes the rovings to deform and the textile layers to nest into each other, as in reality. To account for the flow within the rovings, the anisotropic micropermeability must be assigned. In GeoDict, each voxel has a material ID that allows properties such as anisotropic permeability to be defined. All methods are shown in Fig. \ref{fig:enter-label3}. For the single scale method (SSM) and the simple upscaling method (SUM), 10 models for each of five compaction levels with a voxel resolution of \(6^{3} \ \mu\mathrm{m}^{3}\) and six textile layers were generated. For the scale bridging method (SBM), 12 models were generated for each of the four compaction levels using the same modelling parameters and model sizes as for SSM and SUM. Due to the tremendous effort involved, one multifilament model (FRM) for each of the three compaction levels was generated and numerically calculated. In principle, it applies to all models that the model parameters are target values for modelling and are not always met exactly due to randomization. An overview of the resulting models for the individual methods can be found in Fig. \ref{fig:enter-label2}. All numerical flow simulations were carried out in the GeoDict module FlowDict. For the microscale, the FRM, and the mesoscale models (without considering micropermeability), the LIR solver was used for \eqref{Stokes equation}, and for SUM and SBM, the SimpleFFT solver was used for \eqref{Stokes-Brinkman flow} \cite{flowdict2024user}. Periodic boundary conditions were applied in tangential direction and in flow direction, with additional empty inflow and outflow regions of 40 voxels. An error bound of 0.01, corresponding to a change in permeability of less than 1\%, was chosen as the stopping criterion \cite{flowdict2024user}.

\begin{figure}
    \centering
    \includegraphics[width=1\linewidth]{Figures//Chapter3/Figure5_Different_methods.jpg}
    \caption{Schematic illustrations of the various single and dual scale methods}
    \label{fig:enter-label3}
\end{figure}

\subsection{Single Scale Method - SSM}
\label{sec:ssm}
Our dual-scale methods are compared with the SSM, which neglects micropermeability, to evaluate whether the increased complexity of dual-scale methods is justified.

\subsection{Simple Upscaling Method - SUM}
\label{sec:sum}

For the SUM, all warp and weft rovings have the same material ID and, therefore, the same permeability tensor with identical orientation. In the roving direction, assumed to align with the respective model axis, the rovings are assigned the fiber-direction micropermeability. Perpendicular directions are assigned the transverse micropermeability. It implies that local variations within the roving structure, such as those induced by the weave structure or deformations, are neglected. However, this approach represents an improvement over the SSM, which completely neglects the micropermeability.


\subsection{Scale-Bridging Method - SBM} 
\label{sec:sbm}

The SBM approach was developed to reduce the inaccuracies of the SUM approach. In this method, the rovings are decomposed into several segments, see Fig. \ref{fig:enter-label3}, each with individual material IDs before compaction. Downscaling is performed for each segment: the FVC and orientation of the roving segment, which approximately correspond to the fiber orientation within the roving, are determined. However, this method also represents a simplification of the local structure, as homogeneous properties are assumed for each segment. Therefore, the segment size affects the modelling accuracy. Based on the structural parameters, FVC, fiber orientation, and fiber diameter, the micropermeability is determined using the data-driven methods described in Section 3.3.1 and subsequently assigned to the material ID of the corresponding roving segment. This method can be used to assign up to 255 different anisotropic permeability values (due to the limited number of material IDs in GeoDict), resulting in more accurate  micropermeabilities in the mesomodel compared to the SSM and SUM.


\subsubsection{Feature-based and geometry-based emulator for permeability prediction}
Several recent studies have proposed ML methods as surrogate models (or) emulators for the permeability prediction of microstructures, i.e., on a single scale. Based on the input features used for prediction, these methods can be categorized into two classes: (1) feature-based methods and (2) geometry-based method, as shown in Fig. \ref{fig:two-dd-approaches}. Various modelling approaches for micropermeability prediction have been investigated in \cite{schmidt4872087geometric, natarajan_data-driven_2024, caglar2022deep} and references therein. 

The main advantage of such ML emulators is their significant speed-up in inference times compared to numerical simulations, albeit with a trade-off in accuracy. \cite{natarajan_data-driven_2024} showed that feature-based emulators achieved an inference speed-up of $10^6$ with a relative error of $11.35\%$, whereas geometry-based emulators achieved an inference speed-up of $10^4$ with a relative error of $8.33\%$. Depending on the available features describing the microstructure, either feature-based or geometry-based emulators can be employed for micropermeability prediction.

\begin{figure}[H]
    \centering
    \includegraphics[width=1\linewidth]{Figures/Chapter5/Figure_data_driven_approaches.png}
    \caption{Two classes of modeling approaches for the ML emulators for permeability prediction. The approaches differ based on the input features used to predict the permeabilities. The emulators are constructed using fully connected neural networks and 3D convolutional neural networks. The architecture of the 3D CNN encoder is based on \cite{elmorsy_generalizable_2022}.}
    \label{fig:two-dd-approaches}
\end{figure}

\subsection{Fully resolved Model - FRM}

The method using FRM is theoretically the most precise compared to SSM, SUM, and SBM. Scales are not separated, and no microscale homogenization takes place. A sufficiently small voxel size must be chosen for the entire model to resolve the flow between the fibers \((1^{3} \ \mu\mathrm{m}^{3}/\text{voxel})\). This results in large geometry models and flow simulations, which can only be computed on high-performance clusters. The model generation was comparable to the modeling of SUM and SBM, with the only difference being that multifilament rovings (3000 fibers with a diameter of 7 µm) were created. However, using an appropriate resolution \((0.5^{3} \ \mu\mathrm{m}^{3}/\text{voxel})\) for accurate microscale representation at a mesoscale SVE size of 6 layers with 3x3 rovings per layer exceeded 2 TB of RAM during the flow simulatio, which was the maximum available. As a compromise between size and resolution, models were created with a size of 5.100 x 5.100 x 1.120 voxels (4 layers with 2.4 x 2.4 rovings per layer at 10\% compaction) and a voxel size of \(1^{3} \ \mu\mathrm{m}^{3}\). Result files totaling approx. 310 GB were generated for each model. This fully resolved high-fidelity model offers the fewest simplifications and can therefore be considered the reference. However, due to the slightly reduced size and resolution, the results must be critically scrutinized. Nevertheless, the method highlights the need for advanced and efficient scale-bridging methods due to the vast computational cost of fully resolved models.


\section{Comparison of  dual scale methods for permeability prediction} 
\label{Section: comparison of scale-bridging approaches}
The results of the previously presented approaches are shown in Table \ref{tab:result_overview}, and Fig. \ref{fig:Comparison_diagramm} and \ref{fig:SSM-SUM}. The average run time of the presented methods increases from SSM
over SUM and SBM to FRM by more than two orders of magnitude. This shows
the high effort required for the FRM and the advantage of dual scale approaches.
No significant difference between SUM and SBM can be determined, as the run
times vary too much from model to model. The effort for modelling, structural
analysis and micropermeability prediction for SBM takes approximately 45 min
which is about twice as long as the modelling for SUM and SSM. Except for
SSM, the run time of the flow simulation is multiple times longer in comparison
to the model creation run time. The run time for SSM is much faster because of
the significantly faster solver (LIR solver) that can be used. But the LIR solver
is not suitable for SUM and SBM, as it is not applicable for porous media with
anisotropic permeability. Despite the use of the LIR solver, the run time for the
FRM is extremely high (approx. factor 270 compared to SSM and 23 compared
to SUM/SBM).

%The permeability values of all methods can be considered plausible, as the
%in-plane permeability (K11 and K22) is, as expected, approx. one order of
%magnitude higher than the out-of-plane permeability (K33) and the permeability
%decreases with increasing FVC. Nevertheless, differences depending on the
%method can be observed in Fig. 5 and Table 1. At first glance,
%the high coefficients of variation may not reveal significant differences but, when
%comparing the results of SSM to SUM, which were carried out with the same
%models, shows that the results of the SUM are reproducible and for all models
%higher as can be seen in Fig. 6. This diagram shows an example of the comparison
%of all SSM and SUM models at 15\% compaction. The differences between
%the permeability determined using SSM and SUM vary between 3\% and 92\% for
%the different models. The difference tends to increase with increasing FVC from
%7\% on average at 40\% FVC to 25\% on average at 56\% FVC. This is plausible
%because it allows for additional flow within the roving and the increasing influence of micropermeability for higher FVC results from the fact that the meso
%flow channels in the structure are reduced and a higher proportion of the flow
%passes through the micro flow channels in the rovings. In addition, the CV of
%SUM (42.7 \%) is on average 3.7 \% lower compared to the CV of
%SSM (46.4 \%). Fig. 6 shows as an example the wide range of results that can be
%achieved with models created based on the same modelling parameters. These
%differences result from the randomized model creation and thus variation in
%the structure, which leads to significantly different permeability values. The CV
%of the SBM (37.6 \%)is on average 5.1 \% lower compared to the
%CV of SUM. The SBM and FRM approaches provide comparable values for the
%out-of-plane direction (K33), while significantly lower out-of-plane permeability
%(K33) are achieved from SSM and SUM. For the in-plane permeabilities (K11
%and K22), the SBM, SUM and SSM show comparable results. The permeability
%values of FRM decrease less with increasing FVC than with the other methods.
%As a result, the permeability with the FRM method is initially lower at
%the low FVC of 38.5\%, comparable at the medium FVC of 42.8\% and tends
%to be higher at the high FVC of 48.1\%. It is assumed that the high-fidelity
%FRM method provides the most accurate results because the modelling has the
%least simplifications and no information is lost due to scale separation. The
%SBM takes the structural variability of the microscale and the resulting locally
%varying permeability into account more precisely, achieves out-of-plane permeabilities
%comparable to those of the FRM method while requiring significantly
%less computing time. The situation appears to be different for in-plane permeability:
%the results of the SBM, SUM and SSM are much closer
%to each other. All these methods have their advantages and disadvantages, and
%attempts are made to map the physical phenomena as efficiently but accurately
%as possible for dual-scale structures.

\begin{figure}[H]
    \centering
    \includegraphics[width=1\linewidth]{Figures/Chapter4/Figure6_Diagramm_Multiscale_results_V4.jpg}
    \caption{Diagrams of the in-plane and out-of-plane permeability results plotted over FVC of SSM, SUM, SBM and FRM}
    \label{fig:Comparison_diagramm}
\end{figure}

\begin{figure}[H]
    \centering
    \includegraphics[width=1\linewidth]{Figures/Chapter4/Figure7_Diagramm_Comparison_SSM-SUM_V2.jpg}
    \caption{Comparison of the permeability results (K11, K22, K33) from SSM and SUM for ten models at 15 \% compaction}
    \label{fig:SSM-SUM}
\end{figure}

\begin{sidewaystable}
\centering
\tiny
\begin{tblr}{
  cell{1}{1} = {r=2}{},
  cell{1}{2} = {r=2}{},
  cell{1}{3} = {r=2}{},
  cell{1}{6} = {c=3}{},
  cell{1}{9} = {c=2}{},
  cell{1}{11} = {c=2}{},
  cell{1}{13} = {c=2}{},
  cell{3}{1} = {r=7}{},
  cell{10}{1} = {r=7}{},
  cell{17}{1} = {r=4}{},
  cell{21}{1} = {r=3}{},
  cell{21}{9} = {c=2}{},
  cell{21}{11} = {c=2}{},
  cell{21}{13} = {c=2}{},
  cell{22}{9} = {c=2}{},
  cell{22}{11} = {c=2}{},
  cell{22}{13} = {c=2}{},
  cell{23}{9} = {c=2}{},
  cell{23}{11} = {c=2}{},
  cell{23}{13} = {c=2}{},
}
Method & Compaction & Voxel size in µm\textsuperscript{3} & SVP    & FVC    & Model size in voxel &             &             & K11 in m\textsuperscript{2} &       & K22 in m\textsuperscript{2} &       & K33 in m\textsuperscript{2} &      & Run time in h \\
       &            &                                     & MV     & MV     & X-axis (MV)         & Y-axis (MV) & Z-axis (MV) & MV                          & CV    & MV                          & CV    & MV                          & CV   & MV            \\
SSM    & 0\%        & 6                                   & 67,3\% & 40,6\% & 1109                & 1094        & 305         & 2,34E-10                    & 8\%   & 2,02E-10                    & 20\%  & 2,63E-11                    & 24\% & 0,39          \\
       & 5\%        & 6                                   & 71,0\% & 43,3\% & 1109                & 1094        & 289         & 1,70E-10                    & 11\%  & 1,41E-10                    & 25\%  & 1,97E-11                    & 27\% & 0,39          \\
       & 10\%       & 6                                   & 74,9\% & 46,1\% & 1109                & 1094        & 274         & 1,20E-10                    & 15\%  & 9,56E-11                    & 32\%  & 1,43E-11                    & 30\% & 0,41          \\
       & 15\%       & 6                                   & 78,8\% & 48,8\% & 1109                & 1094        & 260         & 8,07E-11                    & 22\%  & 6,07E-11                    & 43\%  & 9,92E-12                    & 34\% & 0,44          \\
       & 20\%       & 6                                   & 82,5\% & 51,5\% & 1109                & 1094        & 246         & 4,87E-11                    & 36\%  & 3,45E-11                    & 62\%  & 6,24E-12                    & 43\% & 0,72          \\
       & 25\%       & 6                                   & 85,9\% & 53,9\% & 1109                & 1094        & 233         & 2,54E-11                    & 61\%  & 1,64E-11                    & 95\%  & 3,48E-12                    & 57\% & 1,00          \\
       & 30\%       & 6                                   & 88,8\% & 56,0\% & 1109                & 1094        & 221         & 1,16E-11                    & 106\% & 6,90E-12                    & 145\% & 1,81E-12                    & 81\% & 1,10          \\
SUM    & 0\%        & 6                                   & 67,3\% & 40,6\% & 1109                & 1094        & 305         & 2,53E-10                    & 7\%   & 2,19E-10                    & 18\%  & 2,74E-11                    & 23\% & 11,81         \\
       & 5\%        & 6                                   & 71,0\% & 43,3\% & 1109                & 1094        & 289         & 1,86E-10                    & 10\%  & 1,57E-10                    & 24\%  & 2,11E-11                    & 26\% & 11,94         \\
       & 10\%       & 6                                   & 74,9\% & 46,1\% & 1109                & 1094        & 274         & 1,32E-10                    & 14\%  & 1,06E-10                    & 30\%  & 1,52E-11                    & 29\% & 9,00          \\
       & 15\%       & 6                                   & 78,8\% & 48,8\% & 1109                & 1094        & 260         & 8,91E-11                    & 21\%  & 6,76E-11                    & 40\%  & 1,06E-11                    & 33\% & 7,55          \\
       & 20\%       & 6                                   & 82,5\% & 51,5\% & 1109                & 1094        & 246         & 6,96E-11                    & 20\%  & 4,96E-11                    & 56\%  & 7,91E-12                    & 38\% & 6,60          \\
       & 25\%       & 6                                   & 85,9\% & 53,9\% & 1109                & 1094        & 233         & 2,93E-11                    & 57\%  & 1,92E-11                    & 89\%  & 3,86E-12                    & 54\% & 3,96          \\
       & 30\%       & 6                                   & 88,8\% & 56,0\% & 1109                & 1094        & 221         & 1,38E-11                    & 98\%  & 8,46E-12                    & 133\% & 2,05E-12                    & 77\% & 0,00          \\
SBM    & SVP 70\%   & 6                                   & 70,4\% & 42,8\% & 1168                & 1144        & 286         & 1,76E-10                    & 11\%  & 1,74E-10                    & 12\%  & 1,39E-11                    & 30\% & 9,41          \\
       & SVP 75\%   & 6                                   & 75,5\% & 46,5\% & 1145                & 1171        & 268         & 1,10E-10                    & 24\%  & 1,12E-10                    & 19\%  & 8,19E-12                    & 40\% & 7,89          \\
       & SVP 80\%   & 6                                   & 80,6\% & 50,1\% & 1180                & 1144        & 249         & 5,86E-11                    & 21\%  & 6,84E-11                    & 17\%  & 4,68E-12                    & 95\% & 6,29          \\
       & SVP 85\%   & 6                                   & 85,6\% & 53,7\% & 1195                & 1183        & 230         & 2,34E-11                    & 65\%  & 3,29E-11                    & 38\%  & 2,00E-12                    & 78\% & 4,87          \\
FRM    & 10\%       & 1                                   & 38,5\% & 38,5\% & 5100                & 5100        & 1119        & 2,08E-10                    &       & 1,50E-10                    &       & 1,93E-11                    &      & 148,91        \\
       & 20\%       & 1                                   & 42,8\% & 42,8\% & 5100                & 5100        & 1007        & 1,66E-10                    &       & 1,18E-10                    &       & 1,28E-11                    &      & 89,76         \\
       & 30\%       & 1                                   & 48,1\% & 48,1\% & 5100                & 5100        & 895         & 1,24E-10                    &       & 8,56E-11                    &       & 8,20E-12                    &      & 114,18        
\end{tblr}
\caption{Results, model size and run time of SSM, SUM, SBM and FRM}
\label{tab:result_overview}
\end{sidewaystable}



Comparing the results of SSM and SUM, which were both carried out with the same models, shows that the usage of the SUM yields consistently higher permeability values for all models as can be seen in Fig. \ref{fig:SSM-SUM}. This diagram shows the comparison of all SSM and SUM models at 15\% compaction. The differences between the permeability determined using SSM and SUM vary between 3\% and 92\% for the different models. The difference tends to increase with increasing FVC from 7\% on average at 40\% FVC to 25\% on average at 56\% FVC. This trend can be explained by the increasing influence of the micropermeabilitiy with increasing FVC since meso flow channels are reduced with higher compaction of the textile stack. While the coefficient of variation (CV) for SSM and SUM is relatively high, the CV of SUM (42.7 \%) is on average 3.7 lower compared to the CV of SSM (46.4 \%). These differences results from the randomized model creation and thus variation in the structure, which leads to significantly different permeability values. The CV of the SBM (37.6 \%)is on average 5.1 percentage points lower compared to the CV of SUM. Fig. \ref{fig:SSM-SUM} shows exemplarily the wide range of results that can be achieved with models created based on the same modelling parameters.The SBM and FRM approaches provide comparable values for the out-of-plane direction (K33), while considerably higher K33 values are obtained from SSM and SUM. However, the FRM seems to be less sensitive to an increasing FVC compared to the other methods. Compared to the SBM this results in similar K33 values at a median FVC of 42.8 \% but higher values at 48.1 \%. Similarly the K11 in-plane permeability value around 38.5 \% for all methods is comparable but values diverge when changing the FVC. Generally, in-plane permeabilities obtained from SSM, SUM and SBM are comparable in value. The methods presented clearly show the trade off between modeling complexity with regards to physical and geometric accuracy and computational efficiency of dual-scale structures.



\section{Hybrid physics-informed dual scale framework
for upscaling}
\label{Section: Hybrid physics-informed dual scale framework
for upscaling}

In Section \ref{Section: Methods for dual scale permeability prediction}, ML-based surrogate models were used to replace the standard numerical methods for permeability determination. In addition to the data-driven methods, we consider rewriting the microscale PDE solver as a PINN optimization problem, as it offers several advantages. Unlike standard numerical methods, PINNs are meshless and can incorporate experimental data and cross-scale information via optimization, enabling the inference of unknown physics and multiscale modelling \cite{alber2019integrating}. They leverage transfer learning to reuse prior simulations \cite{prantikos2023physics, xu2023transfer} and support efficient distributed GPU implementation due to excellent model and data parallelization \cite{shukla2021parallel, escapil2023h}, making them potentially suitable for demanding SciML tasks as upscaling. However, the nonlinear and non-convex nature of PINN optimization makes it sensitive to the network's (hyper-)parameter initialization and prone to local minima, with additional challenges stemming from the PDE structure. These issues make the application of PINNs to complex 3D (fibrous) geometries quite challenging. Thus, we focus here on 2D periodic microscale geometries with periodic fiber arrangements only.

Let $\Phi(\cdot; \Theta)$ denote a permeability prediction network with parameters $\Theta \in \mathbb{R}^{s}$, trained, for example, using the methods from Section \ref{Section: Methods for dual scale permeability prediction}. To enhance $\Phi(\cdot; \Theta)$ using a hybrid approach, we define the \textit{learning-informed} upscaling operator: 
\begin{align}\label{learning-informed op}
\widehat{\mathcal{F}_{\uparrow}}(\cdot; \theta, \Theta) : \mathcal{M}_{\text{Mi}} \rightarrow \mathbb{R}^{d \times d}, \ \ \mathcal{V}_{\mathcal{Z}} \mapsto  \widehat{\boldsymbol{K}^{p}}[\mathcal{V}_{\mathcal{Z}}](\theta, \Theta)=:\widehat{\mathcal{F}_{\uparrow}}(\mathcal{V}_{\mathcal{Z}}; \theta, \Theta).
\end{align}
We evaluate \eqref{learning-informed op} by generating a guess for $\widehat{\boldsymbol{K}^{p}}[\mathcal{V}_{\mathcal{Z}}](\theta, \Theta)$ using $\Phi(\cdot; \Theta)$, refined by our hybrid approach, which adds dependence on the PINN parameters $\theta \in \mathbb{R}^{n_{p}}$. To achieve this, we introduce a special coarse-scale prior that bridges two scales and uses the  permeability guess to effectively inform PINNs, which approximate the microscale problems. This coupling, combined with the physics-informed regularization, enhances the reliability of the permeability prediction process.

For periodic fiber arrangements in $\mathcal{V}_{\mathcal{Z}} \subset \mathbb{R}^{2}$, $\boldsymbol{K}^{p}[\mathcal{V}_{\mathcal{Z}}]$ satisfies $\boldsymbol{K}^{p}[\mathcal{V}_{\mathcal{Z}}] = \boldsymbol{K}_{11}^{p}[\mathcal{V}_{\mathcal{Z}}] \mathcal{I}$, where $\mathcal{I} \in \mathbb{R}^{2\times 2}$ is the identity matrix and $\boldsymbol{K}_{11}^{p}[\mathcal{V}_{\mathcal{Z}}]$ is given by:
\begin{align}\label{permeability periodic}
\boldsymbol{K}_{11}^{p}[\mathcal{V}_{\mathcal{Z}}] = - \frac{1}{\mu}\boldsymbol{PD}^{-1} \boldsymbol{U},
\end{align}
where $\boldsymbol{U}: = \boldsymbol{U}_{11}$ and $\boldsymbol{PD}: =\boldsymbol{PD}_{11}$ according to \eqref{averaging process}. Therefore, one problem \eqref{Stokes equation} is solved for $\boldsymbol{u}:=\boldsymbol{u}^{(1)}$, $p:=p^{(1)}$. For this, we use the boundary conditions:
\begin{align}\label{boundary conditions}
\boldsymbol{u}  = (1,0)^{T} \ \   \text{on} \ \  \Gamma_{\text{in}} \cup \Gamma_{\text{wall}},  \ \ \frac{\partial \boldsymbol{u}}{\partial \eta}  = 0 \ \   \text{on} \ \  \Gamma_{\text{out}}, \ \   p  = 0  \ \  \text{on} \ \  \Gamma_{\text{out}}, 
\end{align}
where $\Gamma_{\text{in}}$ and $\Gamma_{\text{out}}$ are the inlet (left) and outlet (right) sides of $\mathcal{V}_{\mathcal{Z}}$, and $\frac{\partial \boldsymbol{u}}{\partial \eta}$ is the directional derivative of $\boldsymbol{u}$ in the outward normal direction $\eta$. With this 2D setup, we further present the PINN ansatz, followed by the hybrid one.



\subsection{Physics-informed neural networks as microscale surrogates}
\label{Section: Physics-informed neural networks}

Following \cite{raissi2019physics}, we aim to find a neural network $\widehat{NN}_{\theta}$  that learns the map from spatial coordinates to an approximation of the solution to \eqref{Stokes equation}: 
\begin{align} \label{PINN ansatz}
\boldsymbol{x} \mapsto \widehat{NN}_{\theta}(\boldsymbol{x}) := [\widehat{u}_{\theta, 1}(\boldsymbol{x}), \  \widehat{u}_{\theta, 2}(\boldsymbol{x}), \ \widehat{p}_{\theta}(\boldsymbol{x})]. 
\end{align}
The formulation \eqref{PINN ansatz} is then used to compute $\widehat{\boldsymbol{K}_{11}^{p}}[\mathcal{V}_{\mathcal{Z}}](\theta) \approx \boldsymbol{K}_{11}^{p}[\mathcal{V}_{\mathcal{Z}}]$ from \eqref{permeability periodic}. The parameters $\theta$ are determined by minimizing the PINN objective:
\begin{equation}
\begin{aligned}\label{PINN loss}
\mathcal{J}(\theta): = \sum_{i=1}^{2} \lambda_{i}^{r} \mathcal{J}_{i}^{r}(\theta)  +  \lambda_{i}^{b} \mathcal{J}_{i}^{b}(\theta)  + \lambda_{i}^{\text{out}} \mathcal{J}_{i}^{\text{out}}(\theta) + \lambda^{\text{div}} \mathcal{J}^{\text{div}}(\theta),  
\end{aligned}
\end{equation}
where $\lambda_{i}^{r}, \  \lambda_{i}^{b}, \  \lambda_{i}^{\text{out}} , \lambda^{\text{div}}$ are positive weights for the penalty terms
{\small{
\begin{equation}\label{residual losses}
\begin{aligned}
\mathcal{J}_{i}^{r}(\theta)  & = \frac{1}{N_{r}} \sum_{j=1}^{N_{r}} \big|\frac{\partial \widehat{p}_{\theta}}{\partial x_{i}}(\boldsymbol{x}_{j}^{r}) - \mu  \Delta \widehat{u}_{\theta, i} (\boldsymbol{x}_{j}^{r}) \big|^{2}, \quad \mathcal{J}^{\text{div}}(\theta)   = \frac{1}{N_{r}} \sum_{j=1}^{N_{r}} \big| \nabla \cdot  \widehat{u}_{\theta}(\boldsymbol{x}_{j}^{r})\big|^{2}, \\ 
\mathcal{J}_{i}^{b}(\theta) & = \frac{1}{N_{b}} \sum_{j=1}^{N_{b}} \big| \widehat{u}_{\theta, i} (\boldsymbol{x}_{j}^{b}) \big|^{2},  \quad  \mathcal{J}_{i}^{\text{out}}(\theta) = \frac{1}{N_{\text{out}}} \sum_{j=1}^{N_{\text{out}}} \big | \frac{\partial \widehat{u}_{\theta, i}}{\partial x_{1}}(\boldsymbol{x}_{j}^{\text{out}})\big|^{2}.
\end{aligned}
\end{equation}}}The collocation points $\{\boldsymbol{x}_{j}^{r}\}_{j=1}^{N_{r}} \subset \mathcal{V}_{\mathcal{Z}}^{F}$, $\{\boldsymbol{x}_{j}^{b}\}_{j=1}^{N_{b}} \subset \partial \mathcal{V}_{\mathcal{Z}}^{S}$, $\{\boldsymbol{x}_{j}^{\text{out}}\}_{j=1}^{N_{\text{out}}} \subset \Gamma_{\text{out}}$ are sampled uniformly at random in our case, and the spatial derivatives in \eqref{residual losses} are computed using automatic differentiation, as it is usual for PINNs.

The adaptive modified Fourier feature network is employed for neural network-based approximation; cf. \cite{wang2023expert}. The network follows the ansatz:
\begin{align}\label{nn ansatz}
f_{NN}(\boldsymbol{x}; \theta) = \boldsymbol{W}^{L_{f}}(\phi_{L_{f}-1} \circ \phi_{L_{f}-2}...\circ \phi_{L_{1}} \circ \phi_{E} )(\boldsymbol{x}) + \boldsymbol{b}^{L_{f}}. 
\end{align}
For $1 \leq l \leq L_{f}-1$ in \eqref{nn ansatz}, the $l$-th hidden layer is defined as: 
\begin{align*}
\phi_{L_{l}}(\boldsymbol{h}^{l-1}) & =\boldsymbol{h}^{l}: = [1 - \sigma (\boldsymbol{W}^{l}\boldsymbol{h}^{l-1} + \boldsymbol{b}^{l})]  \odot \boldsymbol{T}_{1}  + \sigma (\boldsymbol{W}^{l}\boldsymbol{h}^{l-1} + \boldsymbol{b}^{l})  \odot \boldsymbol{T}_{2}, \\
\boldsymbol{T}_{1} & = \boldsymbol{W}_{T}^{1}\phi_{E}(\boldsymbol{x}) + \boldsymbol{b}_{T}^{1}, \quad \boldsymbol{T}_{2} = \boldsymbol{W}_{T}^{2}\phi_{E}(\boldsymbol{x}) + \boldsymbol{b}_{T}^{2}.
\end{align*}
where $\odot$ is element-wise multiplication, $\boldsymbol{W}_{T}^{1}, \boldsymbol{W}_{T}^{2} \in \mathbb{R}^{n_{1} \times  d_{E}}$, $\boldsymbol{W}^{l} \in \mathbb{R}^{n_{l} \times n_{l-1}}$ and $ \boldsymbol{b}_{T}^{1}, \boldsymbol{b}_{T}^{2} \in \mathbb{R}^{n_{l}}$, $\boldsymbol{b}^{l} \in \mathbb{R}^{n_{l}}$ are trainable weights and biases, $\sigma: \mathbb{R}^{n_{l}} \rightarrow \mathbb{R}^{n_{l}}$ are activation functions, and the hidden states are $\boldsymbol{h}^{l} \in \mathbb{R}^{n_{l}}$, with $\boldsymbol{h}^{0}:=\phi_{E}(\boldsymbol{x})=[ \sin(2 \pi \boldsymbol{E}\boldsymbol{x}), \  \cos(2 \pi \boldsymbol{E}\boldsymbol{x})] \in \mathbb{R}^{d_{E}\times d}$. Choosing large $d_{E}$ in the embeddings $\phi_{E}$ helps mitigate the spectral bias \cite{rahaman2019spectral} toward learning low-frequencies by better capturing high-frequencies of the solution \cite{tancik2020fourier, wang2021eigenvector}. Trainable $\boldsymbol{E}\in \mathbb{R}^{\frac{d_{E}}{2} \times d}$ is useful as the spectral properties of \eqref{Stokes equation} are not known a priori. The Dirichlet boundary condition $\boldsymbol{u}_{D}$ on $\Gamma_{\text{in}} \cup \Gamma_{\text{wall}}$ and $p  = 0$ on  $\Gamma_{\text{out}}$ are imposed penalty-free by modifying \eqref{nn ansatz} as follows: $[\widehat{NN}_{\theta}(\boldsymbol{x})]_{i} = g_{i}(\boldsymbol{x}) + s_{i}(\boldsymbol{x})[f_{NN}(\boldsymbol{x}; \theta)]_{i}$, where $g_{1}(\boldsymbol{x}) = 1, \ g_{2}(\boldsymbol{x}) = 0$ and $s_{i}(\boldsymbol{x})=4x_{1}x_{2}(1-x_{2})$ for $\widehat{u}_{\theta, 1}$ and $\widehat{u}_{\theta, 2}$, and $g_{3}=0$, $s_{3}=1-\exp{(10x_{1}-10)}$ for $\widehat{p}_{\theta}$. 


\subsection{Hybrid physics-informed neural network based dual scale solver}

For multiscale PDEs, incorporating a prior into the PINN objective \eqref{PINN loss} that well-informs the coarse-scale component of the microscale PDE solution can improve neural network training \cite{hintermuller2023hybrid}. To this end, we adopt a mesoscale model governed by the learning-informed Stokes-Brinkman equation:
\begin{equation}\label{Stokes-Brinkman equation}
\begin{aligned}
- \tilde{\mu} \Delta \boldsymbol{u}^{SB}_{\theta} + \mu \big(\widehat{\boldsymbol{K}}_{\text{Me}}[\mathcal{V}_{\mathcal{Z}}](\theta, \Theta) \big)^{-1} \boldsymbol{u}^{SB}_{\theta}  + \nabla p^{SB}_{\theta} &=  \boldsymbol{f}  \quad &\text{in} \ \mathcal{Z}, \\ 
\nabla \cdot \boldsymbol{u}^{SB}_{\theta}  & =  0   \quad &\text{in} \ \mathcal{Z},
\end{aligned}
\end{equation}
where $\widehat{\boldsymbol{K}}_{\text{Me}}[\mathcal{V}_{\mathcal{Z}}](\theta, \Theta) = \chi_{\mathcal{Z}^{F}} \cdot \infty + \chi_{\mathcal{Z}^{P}} \cdot \widehat{\boldsymbol{K}_{11}^{p}}[\mathcal{V}_{\mathcal{Z}}](\theta,  \Theta)$ and $\widehat{\boldsymbol{K}_{11}^{p}}[\mathcal{V}_{\mathcal{Z}}](\theta,  \Theta)$ is computed via \eqref{permeability periodic} using the PINN ansatz. We define the coarse-scale prior
\begin{align}\label{coarse-scale prior}
\mathcal{R}(\theta; \theta_{s}): & =   \sum_{i=1}^{2} \lambda_{i}^{SB} \mathcal{R}_{i}(\theta; \Theta),
\end{align}
where $\mathcal{R}_{i}(\theta; \Theta) = \frac{1}{N_{SB}} \sum_{j=1}^{N_{SB}}  | \widehat{u}_{\theta, i}(\boldsymbol{x}^{c}_{j}) - u_{\theta, i}^{SB}(\boldsymbol{x}^{c}_{j};  \Theta)|^{2}$ with $\{\boldsymbol{x}_{j}^{c}\}_{j=1}^{N_{SB}} \subset \mathcal{Z}_{F}$, and $\lambda_{i}^{SB}>0$ are the prior weights. Using \eqref{PINN loss} with \eqref{coarse-scale prior}, our hybrid problem reads:
\begin{align} \label{hybrid problem}
 \underset{\theta \in \mathbb{R}^{n_{p}}}{\text{min}}  \  J_{\boldsymbol{\lambda}}(\theta;  \Theta):= \mathcal{J}(\theta)  +  \mathcal{R}(\theta;  \Theta) \quad \text{subject to the constraints} \ \eqref{Stokes-Brinkman equation}.
\end{align}
Evaluating $J_{\boldsymbol{\lambda}}$ (where the subscript $\boldsymbol{\lambda}$ marks the dependence on the loss weights) requires (numerically) solving \eqref{Stokes-Brinkman equation}, resulting in a PDE-constrained optimization problem. Since coarse-scale geometries are easier to mesh in general, we apply a standard numerical solver at the mesoscale. Enabling \eqref{Stokes-Brinkman equation} with the boundary conditions \eqref{boundary conditions} and using Taylor-Hood finite elements \cite{boffi2013mixed}, we  approximate $\boldsymbol{u}^{SB}_{\theta}$ and  $p^{SB}_{\theta}$ with $\boldsymbol{u}^{SB}_{h,\theta}$ and $p^{SB}_{h,\theta}$ (the subscript $h>0$ denotes mesh resolution), forming the dual-scale solver shown in Fig.~\ref{fig:hybrid solver}, realized through optimization. 


\begin{figure}
    \centering
    \includegraphics[width=1\textwidth]{Figures/WIAS/ML4SIM_scheme_new.png}
    \caption{The schematic workflow for the hybrid neural network based dual scale solver.}
    \label{fig:hybrid solver}
\end{figure}

\begin{figure}
    \centering
    \includegraphics[width=1\linewidth]{Figures/WIAS/gradient_histogram.png}
    \caption{ \textbf{(a)} Histogram of back-propagated gradients for the momentum equations (top) and prior terms (bottom). \textbf{(b)} Schematic minimization of the physics-informed term $\mathcal{J}_{1}(\theta)$ and the coarse-scale prior $\mathcal{R}_{1}(\theta; \Theta)$: overfitting $\mathcal{R}_{1}(\theta; \Theta)$ perturbs $\mathcal{J}_{1}(\theta)$.}
    \label{fig:training dynamics}
\end{figure}

The hybrid objective \eqref{hybrid problem} is optimized using a variant of the gradient descent algorithm with the following generic form
\begin{align}\label{discrete gradient flow}
\theta_{k+1} = \theta_{k} - l_{k} ( \nabla_{\theta} \mathcal{J}(\theta_{k}) +  \nabla_{\theta}\mathcal{R}(\theta_{k}; \Theta) ),
\end{align}
where $l_{k}>0$ is the learning rate. The discrete gradient flow \eqref{discrete gradient flow} is known for its stiffness due to the variation in the magnitudes of the involved gradients, which causes convergence issues \cite{wang2021understanding}. Indeed, the histograms of back-propagated gradients for two momentum equations and two prior terms, shown in Fig.~\ref{fig:training dynamics}(a), reveal that prior gradients are concentrated near zero, while momentum gradients have much wider ranges. No-slip penalty gradients in \eqref{residual losses} also similarly concentrate near zero, hence stiffness increases with the number of fibers. To help mitigate this, we scale all the gradients using weights $\boldsymbol{\lambda}$ and following \cite[Algorithm 1(c)]{wang2023expert}. Besides, overfitting discrepancies between the micro- and meso-models in \eqref{coarse-scale prior} distort the PDE residuals in \eqref{residual losses}, hindering convergence at both scales. Poor scale bridging further exacerbates it, increasing the red gaps in Fig.~\ref{fig:training dynamics}(b) by causing greater model mismatches. To partially mitigate it, weights in \eqref{coarse-scale prior} are updated via $\lambda_{i, k+1}^{SB} = \lambda_{i, k}^{SB} \exp{(-\alpha_{i}(k-k^{*}) )}$, where $\alpha_{i}>0$ are hyperparameters and $k^{*}$ marks the iteration, where the coarse component of \eqref{Stokes equation} is rather well-approximated. By default, we set $k^{*}=k_{\max}/2$ and $\alpha_{i}=5 \times 10^{-4}$. For $k<k^{*}$, \cite[Algorithm 1(c)]{wang2023expert} is applied. To update \eqref{coarse-scale prior}, we numerically solve \eqref{Stokes-Brinkman equation} every $T$ iterations. The hybrid optimization, outlined in Algorithm \ref{alg:hybrid solver}, requires initializing $\theta_{0}$, typically using Glorot initialization \cite{glorot2010understanding} or transfer learning, where $\theta_{0}$ is set to $\widetilde{\theta}$ from a similar problem \cite{prantikos2023physics, xu2023transfer}. Algorithm \ref{alg:hybrid solver} also requires $\sigma_{\Phi} > 0$ as a deviation constraint on the desired permeability to enhance training robustness; $\sigma_{\Phi}$ is often available from experimental and theoretical estimates.
\begin{algorithm}[ht!]
\caption{Hybrid dual scale solver}
\label{alg:hybrid solver}
  \textbf{Input:} $\widetilde{\boldsymbol{K}_{11}}[\mathcal{V}_{\mathcal{Z}}](\Theta):=\Phi(\mathcal{V}_{\mathcal{Z}}; \Theta), \  \theta_{0}, \  \sigma_{\Phi}$, maximal iteration number $k_{\text{max}}$. \\
  \textbf{Output:}  $\theta$ and $\widehat{\boldsymbol{K}_{11}^{p}}[\mathcal{V}_{\mathcal{Z}}](\theta; \Theta)$.
  \begin{algorithmic}[1]
  \WHILE {$0 \leq k \leq k_{\text{max}}-1$}
  \STATE{Compute $\nabla_{\theta} J_{\boldsymbol{\lambda}_{k}}(\theta_{k})$ using automatic differentiation}
  \IF{$k\bmod{T} = 0$}
  \STATE {Update: $\boldsymbol{\lambda}_{k} \leftarrow \text{Weights scaling}\big(\nabla_{\theta} J_{\boldsymbol{\lambda}_{k}}(\theta_{k}),  \boldsymbol{\lambda}_{k} \big)$.}
  \IF{$\widetilde{\boldsymbol{K}_{11}^{p}}[\mathcal{V}_{\mathcal{Z}}](\Theta) - \sigma_{\Phi} < \widehat{\boldsymbol{K}_{11}^{p}}[\mathcal{V}_{\mathcal{Z}}](\theta_{k}; \Theta) < \widetilde{\boldsymbol{K}_{11}^{p}}[\mathcal{V}_{\mathcal{Z}}](\Theta) + \sigma_{\Phi} $}
  \STATE {Update: $\widehat{\boldsymbol{K}_{11}^{p}}[\mathcal{V}_{\mathcal{Z}}](\theta_{k})  \leftarrow  \text{Darcy's Law}\big(\widehat{NN}_{\theta_{k}}\big)$.}
  \STATE {Compute FEM solution to \eqref{Stokes-Brinkman equation} and update the prior \eqref{coarse-scale prior}.}
  \ENDIF
  \ENDIF
  \STATE {Update: $\theta_{k+1} \leftarrow \text{Optimizer}\big(\nabla_{\theta} J_{\boldsymbol{\lambda}_{k}}(\theta_{k}), \ \text{optimizer hyperparameters} \big)$} 
  \STATE {$k \leftarrow k+1$}
  \ENDWHILE
  \end{algorithmic}
\end{algorithm}

We set $\mathcal{Z}_{i}=(0,1)^{2}$ with $\mathcal{Z}_{P_{i}} = (0.25, 0.75)^{2}$ ($1\leq i \leq2)$ as the mesoscale models. For downscaling, we formally specify the number of fibers for $\mathcal{V}_{1}:=\mathcal{F}_{\downarrow}(\mathcal{Z}_{1})$ (16 fibers) and $\mathcal{V}_{2}:=\mathcal{F}_{\downarrow}(\mathcal{Z}_{2})$ (25 fibers) to resolve the square rovings $\mathcal{Z}_{P_{1}}$ and $\mathcal{Z}_{P_{2}}$, see Fig. \ref{fig:2D geometries}. The fiber obstacles correspond to domain perforations with no-slip boundary conditions. We set  Algorithm \ref{alg:hybrid solver} as follows. The architecture from Section \ref{Section: Physics-informed neural networks} (with $d_{E}=100$, 4 hidden layers, each with 100 neurons) and Adam optimizer \cite{kingma2014adam} are used for hybrid and PINN approaches. For both approaches, we use a full batch of collocation points and an exponential learning rate schedule. The initial learning rate is set to $l_{0} = 1 \times 10^{-3}$, with a decay rate of 0.9. A subset of finite element mesh points $\{\boldsymbol{x}_{j}^{c}\}_{j=1}^{N_{SB}} \subset \mathcal{Z}_{F}$ from discretization of \eqref{Stokes-Brinkman equation} is used in \eqref{coarse-scale prior}. We use $T=250$ to reduce computational costs. Using finite elements with a very fine mesh, we computed permeabilities $\boldsymbol{K}^{p} = 1.4 \times 10^{-3}$ for $\mathcal{V}_{1}$ and $\boldsymbol{K}^{p} = 6.4 \times 10^{-4}$ for $\mathcal{V}_{2}$ as our references. We set $\widetilde{\boldsymbol{K}_{11}} = 9 \times 10^{-4}$ as our initial guess for both geometries, with $\sigma_{\Phi}=0.5 \times \boldsymbol{K}^{p}$. We note that the boundary conditions in \eqref{boundary conditions}, which induce a strong boundary layer, are not well-suited for permeability computation, resulting in suboptimal scale bridging. Therefore, we treat the Stokes-Brinkman viscosity $\widetilde{\mu}$ in \eqref{Stokes-Brinkman equation} as a free parameter to improve scale bridging. For $\mathcal{V}_{1}$, $\widetilde{\mu} = 0.175$ (upscaling-consistent case) achieves a good match between flows at the microscale and mesoscale, whereas $\widetilde{\mu} = 0.45$ leads to significant discrepancies; see (a) and (b) in Fig. \ref{fig:2D geometries}. For $\mathcal{V}_{2}$, we use $\widetilde{\mu} = 0.25$ to achieve good flow matching; see Fig. \ref{fig:2D geometries}(c).
\begin{figure}
    \centering
    \includegraphics[width=1\linewidth]{Figures/WIAS/geometries.png}
    \caption{Geometries and the respective velocity magnitudes at the microscale and mesoscale (computed with FEM) for \textbf{(a)}: $\mathcal{V}_{1}$, $\widetilde{\mu}=0.175$, \textbf{(b)}:  $\mathcal{V}_{1}$, $\widetilde{\mu}=0.45$, \textbf{(c)}: $\mathcal{V}_{2}$, $\widetilde{\mu}=0.25$. Here, $\widetilde{\mu}$ stands for the Stokes-Brinkman viscosity.}
    \label{fig:2D geometries}
\end{figure}


\begin{table}[H]
\footnotesize
\centering
\begin{tabular}{l l l l}
\hline
Method & Relative $l^{2}$ error ($u_{1}$) & Relative $l^{2}$ error ($u_{2}$) & Relative $l^{2}$ error ($p$) \\
\hline
$\text{Hybrid}_{\mathcal{V}_{1}}^{G}$     & 6.78E-03 $\pm$ 1.50E-03 & 1.95E-02 $\pm$ 2.11E-03 & 6.80E-03 $\pm$ 2.51E-03 \\
$\text{PINN}_{\mathcal{V}_{1}}^{G}$    & 1.13E-02 $\pm$ 9.18E-04 & 6.20E-02 $\pm$ 5.61E-03 & 8.06E-03 $\pm$ 9.91E-04 \\
$\text{Hybrid}_{\mathcal{V}_{2}}^{G}$      & 9.00E-03 $\pm$ 1.14E-03 & 2.78E-02 $\pm$ 5.07E-03 & 4.10E-02 $\pm$ 9.02E-03 \\
$\text{Hybrid}_{\mathcal{V}_{2}}^{\text{TL}}$   & 7.64E-03 $\pm$ 5.13E-04 & 2.18E-02 $\pm$ 1.41E-03 & 2.43E-02 $\pm$ 1.27E-03 \\
$\text{PINN}_{\mathcal{V}_{2}}^{G}$    & 2.76E-02 $\pm$ 5.54E-04 & 8.32E-02 $\pm$ 6.50E-03 & 8.06E-03 $\pm$ 1.86E-03 \\
$\text{PINN}_{\mathcal{V}_{2}}^{TL}$    & 1.60E-02 $\pm$ 2.28E-04 & 8.34E-02 $\pm$ 1.81E-03 & 2.85E-03 $\pm$ 1.85E-03 \\
\hline
\end{tabular}
\caption{Comparison for PINN-based approaches with relative errors are averaged across 5 runs with different random seeds. The superscript $G$ and $TL$ in the method indicates default Glorot or transfer learning initializations. The subscript $\mathcal{V}_{i}$ indicates the geometry used. The results for $\text{PINN}_{\mathcal{V}_{1}}^{G}$ are shown for twice more iteration (40000), compared to $\text{Hybrid}_{\mathcal{V}_{1}}^{G}$.}
\label{tab:physics-informed comparison table}
\end{table}

For $\mathcal{V}_{1}$, we use $N_{r}=27500$, $N_{\text{out}}=1000$ collocation points and $250$ collocation points per obstacle, totaling $N_{b}=4000$. In Table \ref{tab:physics-informed comparison table}, we observe that despite requiring twice as many (40000 against 20000) iterations, the accuracy of PINNs is still slightly lower for the velocity components, compared to the hybrid approach. The results in Table \ref{tab:physics-informed comparison table} are presented for proper scale bridging with $\widetilde{\mu} = 0.175$, and, exemplarily, the pointwise errors of the hybrid approach are shown in Fig. \ref{fig:pointwise errors 16}.  However, Fig. \ref{fig:error curves} shows, among other cases, the relative error curves for $\widetilde{\mu} = 0.25$, where the PINN method and the hybrid solver achieve comparable results. This supports our prior expectations as shown in Fig. \ref{fig:training dynamics}(b). Notably, the hybrid solver exhibits faster initial error decay due to the embedded mesoscale information. This advantage, observed in both upscaling-consistent (Fig. \ref{fig:error curves}(b)) and non-upscaling-consistent (Fig. \ref{fig:error curves}(c)) cases, helps yield better permeability values faster, as shown in Fig. \ref{fig:Physics-informed permeabilities}(a).

\begin{figure}
    \centering
    \includegraphics[width=0.9\linewidth]{Figures/WIAS/16_Hybrid.png}
    \caption{The hybrid solver results for the microscale PDE and pointwise errors (compared with the FEM solution) for $\mathcal{V}_{1}$ using Glorot initialization and scale bridging with $\widetilde{\mu}=0.175$.}
    \label{fig:pointwise errors 16}
\end{figure}

\begin{figure}
    \centering
    \includegraphics[width=1\linewidth]{Figures/WIAS/Convergence_curves.png}
    \caption{Relative $l_{2}$ errors for $u_{1}, u_{2}$ and $p$ vs iterations. Abbreviations `16 ob" and `25 ob" stand for $\mathcal{V}_{1}$ and $\mathcal{V}_{2}$, respectively.}
    \label{fig:error curves}
\end{figure}

For $\mathcal{V}_{2}$, we use $N_{r} = 45,000$, $N_{\text{out}} = 1,000$, and $250$ collocation points per obstacle, totaling $N_{b} = 6,250$, and $k_{\text{max}}=25000$. Table \ref{tab:physics-informed comparison table} compares the hybrid solver and PINNs using Glorot initialization and transfer learning (TL) initialization $\theta_{0} := \widetilde{\theta}$ ($\widetilde{\mu} = 0.25$ in the hybrid case), where $\widetilde{\theta}$ is obtained via the hybrid approach on $\mathcal{V}_{1}$. Fig.\ref{fig:pointwise errors 25} shows pointwise errors for the hybrid solver with TL. TL consistently improves accuracy and yields the best permeability value, as seen in Fig. \ref{fig:Physics-informed permeabilities}(b), demonstrating its potential for multi-query SVE computations. However, the PINN with TL and the hybrid solver with Glorot initialization yield similar results. Notably, Fig.\ref{fig:error curves}(e) shows that the hybrid solver refines permeability rapidly even without TL, as upscaling partially averages out neural network approximation error at the microscale. However, a significant limitation of the current PINN-based approach is its training speed — a well-known issue in the scientific community \cite{mcgreivy2024weak}. PINNs take 15 to 30 minutes to train (on A100 GPUs with 40 GB RAM) for $\mathcal{V}_{1}$ and $\mathcal{V}_{2}$, respectively. The hybrid approach is about 20-30$\%$ more computationally expensive if used for $k:=k_{\max}$. Thus, substantial mathematical and algorithmic development is required, and more powerful hardware must be utilized.


\begin{figure}[H]
    \centering
    \includegraphics[width=1\linewidth]{Figures/WIAS/physics_informed_permeabilities.png}
    \caption{Physics-informed permeabilities obtained using the PINN and hybrid (H) apporaches from the microscale geometries \textbf{(a)}: $\mathcal{V}_{1}$, where abbreviation `uc' and `no uc' stands for consistent $(\widetilde{\mu}=0.175)$ and non consistent $(\widetilde{\mu}=0.45)$ scale bridging, and \textbf{(b)}: $\mathcal{V}_{2}$, where TL stands for transfer learning. }
    \label{fig:Physics-informed permeabilities}
\end{figure}

\begin{figure}[H]
    \centering
    \includegraphics[width=0.9\linewidth]{Figures/WIAS/25_hybrid_TL.png}
    \caption{The hybrid solver results for the microscale PDE and pointwise errors (compared with the FEM solution) for $\mathcal{V}_{2}$ using transfer learning initialization and scale bridging with $\widetilde{\mu}=0.25$.}
    \label{fig:pointwise errors 25}
\end{figure}

\section{Conclusion}

This work presents several data-driven approaches for predicting the permeability of mesoscale textile models, based on various permeability determination methods: SSM, SUM, SBM, and FRM. The SSM is the simplest, ignoring microscale permeabilities, while the SUM represents the state of the art. The SBM improves upon the SUM by accounting for the influence of microscale permeability on mesoscale permeability. Despite its increased complexity, the SBM maintains computational efficiency due to the use of data-driven surrogates. Although permeability values vary across methods, as noted in \cite{syerko2023benchmark, syerko2024mesoscale}, the SBM shows potential to enhance both efficiency and accuracy. The FRM is theoretically the most accurate but faces prohibitive computational costs, emphasizing the need for efficient scale-bridging methods. Therefore, we propose the hybrid PINN-numerical dual-scale solver to further assist the data-driven upscaling process. Using simplified 2D models, we demonstrate that hybridization with properly embedded upscaling process enhances microscale PINN approximations and improves the micropermeabilities via physics-informed regularization. The hybrid ansatz serves as a first step not only in merging various approaches to assist scale bridging, but also in capturing the inherent reciprocal dependency between scales, such as those arising in unsaturated flows. However, advanced applications will require progress in optimization to achieve faster and more robust PINN training, as this remains a key limitation.

\section{Acknowledgments}
All authors acknowledge the support of the Leibniz Collaborative Excellence Cluster under project  ML4Sim (funding reference: K377/2021).




%% Refer following link for more details.
%% https://en.wikibooks.org/wiki/LaTeX/Mathematics
%% https://en.wikibooks.org/wiki/LaTeX/Advanced_Mathematics




%% The Appendices part is started with the command \appendix;
%% appendix sections are then done as normal sections
%\appendix
%\section{Example Appendix Section}
%\label{app1}

%Appendix text.

%% For citations use: 
%%       \cite{<label>} ==> [1]

%%
%Example citation, See \cite{lamport94}.

%% If you have bib database file and want bibtex to generate the
%% bibitems, please use
%%
\bibliographystyle{elsarticle-num} 
\bibliography{bib.bib}

%% else use the following coding to input the bibitems directly in the
%% TeX file.

%% Refer following link for more details about bibliography and citations.
%% https://en.wikibooks.org/wiki/LaTeX/Bibliography_Management
% \printbibliography

%\begin{thebibliography}{00}

%% For numbered reference style
%% \bibitem{label}
%% Text of bibliographic item

%\bibitem{lamport94}
%  Leslie Lamport,
 % \textit{\LaTeX: a document preparation system},
%  Addison Wesley, Massachusetts,
%  2nd edition,
%  1994.

%\end{thebibliography}
\newpage









% \begin{table}[H]
% \footnotesize
%     \centering
%     \begin{tblr}{
%           column{2,3,4,5} = {c},
%           cell{1}{1} = {r=2}{},
%           cell{1}{2} = {r=2}{},
%           cell{1}{3} = {c=3}{},
%         }
%         \hline
%         Modelling approach & MRE [\%] & R² score &    &    \\
%                  & $[K_{11},K_{22},K_{33}]$ & $K_{11}$ & $K_{22}$ & $K_{33}$\\
%         \hline
%         (a) 3 input features & 12.68 & 0.941 & 0.953 & 0.706 \\
%         (b) 8 input features & 11.35 & 0.941 & 0.954 & 0.701 \\
%         (c) Geometry only & 11.21 & 0.967 & 0.966 & 0.818 \\
%         (d) Geometry + 3 features & 9.48 & 0.970 & 0.969 & 0.818 \\
%         (e) Geometry + 8 features & \textbf{8.33} & \textbf{0.984} & \textbf{0.978} & \textbf{0.870} \\
%         \hline
%     \end{tblr}
%     \caption{Evaluation metrics for different modelling approaches of the feature-based and geometry-based emulators on the test set. The error metrics compare the ground truth permeabilities and the predicted permeabilities in their true scale $[m^2]$. The lower the MRE metric, the better. The higher the R² metric, the better.}
%     \label{tab:dd-approaches-results}
% \end{table}

% \begin{table}[H]
% \footnotesize
% \centering
% \begin{tblr}{
%         width = \linewidth,
%         column{2,3,4,5} = {c},
%         cell{1}{1} = {r=2}{},
%         cell{1}{2} = {c=2}{},
%         cell{1}{4} = {c=2}{},
%         hline{2} = {2,3,4,5}{0.08em}
%     }
%     \hline\hline
%     modelling Approach & Time investment & & Comparison to simulator \\
%           & Training & Inference & Speed-up & Relative Error \\
%     \hline\hline
%     \RaggedRight{Feature-based (b)} & $49.31$ s & $0.14$ ms& $10^6$ & $11.35\,\%$ \\
%     \hline
%     \RaggedRight{Geometry-based (e)} & $8.84$ h & $2.24$ ms & $10^4$ & $8.33 \,\%$ \\
%     \hline
% \end{tblr}

% \caption{Comparison of training and inference times for the proposed feature-based and geometry-based emulators. Significant inference speed-ups are obtained using the emulators while the trade-offs are the relative errors with respect to the simulator.}
% \label{tab:inference_times}
% \end{table}





\end{document}

\endinput
%%
%% End of file `elsarticle-template-num.tex'.
