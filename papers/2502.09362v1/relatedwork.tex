\section{Related Research}
\label{sec:related-research}

Ever since the discipline's inception as a response to technological advancement \cite{baecker_timelines_2008}, HCI has been evolving, in waves \cite{harrison_making_2011, bodkerWhenSecondWave2006, bodker_third-wave_2015, frauenbergerEntanglementHCINext2019}, with its approach to futures adapting accordingly.  
Recent development suggests that the field is starting to take a more active and political stance to the futures addressed \cite{ashby_fourth-wave_2019}. 
Increasingly often, HCI researchers step beyond user studies, into co-creation of futures with overlooked communities \cite{harrington_eliciting_2021, freeman_rediscovering_2022} and non-human actors \cite{liu_design_2018, reddy_making_2021, heitlingerAlgorithmicFoodJustice2021}.
Below, we review the vast expanse of both conventional and emergent approaches to considering the future in HCI. Against this backdrop, we can paint a picture of dovetailing and divergences between HCI and futures studies, the branch of social science most directly tackling systematic explorations of alternative futures~\cite{bellWhatWeMean1996}.

\subsection{Conventional Approaches to Examining the Future in HCI}
To study futures, one needs to study possibilities. Scholars of HCI tackle this in various ways, among which are prototyping \cite{limAnatomyPrototypesPrototypes2008}, scenario-building (either by the researchers themselves or with possible future users \cite{carroll_making_2003, stromberg2004interactive, epp2022reinventing}), observing (potentially opportunity-exposing) ways in which users make use of technologies \cite{norman2008workarounds,norman2013incremental}, and reflecting on implications for future technologies' development on the basis of user studies \cite{luria_re-embodiment_2019}. 

Despite the work's future-orientation, HCI researchers rarely reflect comprehensively on the complexity of developing futures. This leaves room for improvement in several areas. One issue that rises to prominence is \emph{techno-centrism}: HCI scenarios often are informed by the goal of suggesting new technologies or uses, and the studies typically present prototypes without deeper considerations of how their adoption might develop or of various contingencies' potential influences \cite{lindley_implications_2017,pargmanSustainabilityImaginedFuture2017}. A few scholars have acknowledged the pressing need to consider futures more broadly. Among the most prominent has been \citeauthor{carroll_making_2003}, whose book \emph{Making Use} \cite{carroll_making_2003} discusses the history of scenario-building by referencing the work of \citeauthor{kahn1962thinking} \cite{kahn1962thinking}, a pioneer of scenario-based methods' use in strategic planning, head-on; however, even \citeauthor{carroll_making_2003} narrowed his attention to possible human interactions with technologies, as opposed to broader socio-political considerations or environmental contexts.



While HCI studies may give superficial attention to ethics factors, many fail to integrate corresponding perspectives, political sensitivity, or justice considerations meaningfully into the design of futures. For example, surveillance issues often get sidelined via brief mentions of the pervasive rise of technologies that monitor/track user behavior \cite{zuboff2015bigother}. Likewise, disparities in access to future technologies and the costs of unintended consequences remain under examined \cite{benjamin2019raceaftertechnology}. As automation and AI-based systems threaten to reshape entire industries, often at the expense of vulnerable workers \cite{eubanks2018automatinginequality}, the shadow of labor displacement has taken on especially large dimensions. Yet the ethics implications of adopting new technologies at scale -- from environmental and sustainability domains to mechanisms for data governance and prevention of misuse -- often receive only surface-level treatment, if any at all \cite{jasanoff2016dreamscapes}.


Even in work attuned to ``far-future'' scenarios and active honing of visions, the scenarios' plausibility is seldom scrutinized. In contrast, researchers often approach long-time-horizon studies as thought experiments, seeds for speculative responses to the present \cite{coulton2017design,wong2016when,blythe_research_2014}. In such cases, the visions become \emph{leaps into the future}, in that the work glosses over the intermediate steps needed for their realization. Neglecting the pathways to the future leaves gulfs in understanding of how such futures might materialize.      

In summary, the promise of rapid, productive strides in optimizing and enhancing human interaction with technological systems has brought HCI research a double-edged sword. The methods developed have drawn the field's future-orientation to immediate events and concrete examples of technology use while broader, speculative engagements with possible futures, or possible routes to distant futures, often suffer from neglect. These shortcomings spotlight the need for an expanded approach to futures, wherein HCI scholars embrace diverse explorations that are scalable and context-aware.


\subsection{Speculative Futures in HCI Studies} \label{sec:speculative-futures-HCI} 
In the last decade, the field's traditional approaches to considering the future have been joined by techniques that place greater focus on innovative approaches whereby speculation and foresight aid in examining possible futures. One distinguishing feature of these emerging practices is their stress on envisioning alternative futures -- e.g., on futures that diverge from the most probable scenarios, challenge prevailing assumptions about potential futures, and could help us deepen our understanding of technology design by probing alternative outcomes \cite{kozubaev_expanding_2020}.  

The evolving HCI landscape has witnessed sprouting of speculative design, 
critical design 
and design fiction, 
as fruit of a rich tradition of critical theory and artistic critique. Emerging as responses to technology-design practitioners' conventional orientation toward utilitarian and market-driven outcomes, they are directed instead to challenging dominant narratives and inspiring profound reflection on societal impacts. Rather than content themselves with familiar tools such as prediction and forecasting, proponents of these methods foreground the role of design in crafting scenarios that provoke discussion, question current assumptions, and reimagine worldviews ``otherwise'' \cite{blythe_research_2014}. They embrace a plurality of futuring \cite{howell2021calling}, thus inviting a rich array of speculative visions that extend the sphere of inquiry beyond linear, deterministic projections.

Speculative design often applies the tools of artifact- and scenario-creation to stimulate discussion and, thereby, exlore possible futures \cite{wong2018speculative, dunneSpeculativeEverythingDesign2013a}. Where older methods focus solely on likely or desirable futures, speculative design thrives on ``what if...?'' scenarios to bring in also those futures that could yield insight even if they remain improbable. By letting designers and other participants examine the implications of today's decisions, technology trends, and developments in society, speculative design broadens views, for greater attention to possibilities that may challenge the \emph{status quo}, and homes in on avenues for critical thinking about future technologies.

Critical design goes further still, by explicitly focusing on generating controversy and debate \cite{bardzell_critical_2012, bardzell_reading_2014, bardzell_what_2013}. Often, the critique gets embodied in artifacts that challenge presumptions about the role of technology in society by pointing to the flaws and unfulfilled potential of current technology-use practices. It champions the principle of anti-solutionism \cite{blythe_anti-solutionist_2016}, contesting the supposition that technology must always offer solutions. Positing that deeper understanding and acknowledgment of problems can drive more meaningful developments, critical design creates criticism-imbued artifacts that can help society expose and interrogate the values, practices, and power structures enmeshed in everyday technologies. Thus it encourages a conscious, reflective approach to design.

Design fiction, in turn, is a strongly storytelling-centered approach that crafts narratives instrumented to probe and manifest alternative futures \cite{bleecker_design_2009, sterling_design_2009, tanenbaum_design_2014}. Particularly effective at rendering abstract scenarios tangible and relatable, it displays power to bridge the gap between scientific exploration of concepts and emotional engagement with an audience \cite{coulton2017design}. Design-fiction narratives situate hypothetical technologies within detail-populated social, cultural, and ethics-bearing contexts. These stories anchor potential technologies in fictional interactions between technologies and intended users. 

All these speculative methods echo ideas from critical future studies \cite{slaughterKnowledgeBaseFutures1998}, especially in their emphasis on sensitization to uncertainty 
and recognizing 
the open nature of future outcomes \cite{howell2021calling, poli_anticipation_2019, gallHowVisualiseFutures2022}. Together, they constitute a richly woven tapestry for grappling with possibilities that transcend technological determinism. Speculative methods contribute to futures studies via tools and frameworks that play with and strategize around uncertainty. By harnessing it as a resource to enrich discourse about the futures we choose to pursue \cite{epp_uncertainties_2024}, they work alongside other methods to flesh out the picture. 

\subsection{Possible Futures in HCI} \label{sec:possible-futures-HCI}

\citeauthor{weiser1991computer}’s vision of computing for the 21st century \cite{weiser1991computer} strongly influenced HCI research. That narrow view of the future reproduced mostly middle-class US societal ideals~\cite{bell_yesterdays_2007}, 
but reliance on \emph{any} monocultural imaginary of future computing is bound to lead to conceiving of futures principally on near horizons without reflecting on technologies at systemic scales \cite{nathan_envisioning_2008, mankoffLookingYesterdayTomorrow2013a}. 
The gap has gained increasing recognition in calls for longer-term perspectives and more holism in approaches to possible futures \cite{nathan_envisioning_2008, mankoffLookingYesterdayTomorrow2013a, salovaaraEvaluationPrototypesProblem2017, epp2022reinventing, elsden_speculative_2017, pargmanSustainabilityImaginedFuture2017, lightCollaborativeSpeculationAnticipation2021a}.

For interaction designers to think about their interactive technologies' evolution, consider the consequences 5--20 years down the line, and identify long-term implications \cite{nathan_envisioning_2008, mankoffLookingYesterdayTomorrow2013a}, they must be encouraged to attend to externalities, such as how the technology might affect the environmental, societal, ethics, or economic domain \cite{nathan_envisioning_2008, mankoffLookingYesterdayTomorrow2013a, pargmanSustainabilityImaginedFuture2017, epp2022reinventing, moesgenDesigningUncertainFutures2023}. Such consequences may evolve slowly and stay unforeseeable within short timeframes \cite{pargmanSustainabilityImaginedFuture2017}.

Another major challenge for futuring arises from future-contin\-gent factors' interdependencies. 
Since, for example, environmental and economic forces can interact in complex and
unexpected ways, teasing out the uncertainties necessitates a systemic, open-ended angle on the future 
\cite{jouvenelArtConjecture1967, adamFutureMattersAction2007, poli_anticipation_2019}, such as a perspective acknowledging that multiple futures may warrant exploration and that diverse trajectories have to be considered \cite{pargmanSustainabilityImaginedFuture2017, salovaaraEvaluationPrototypesProblem2017, epp2022reinventing, moesgenDesigningUncertainFutures2023}.

Inherent biases too warrant contemplation. Similarly to how a techno-optimistic approach to interaction design can leave one misapprehending stakeholder needs \cite{nathan_envisioning_2008}, so can insufficient awareness of one's assumptions about the alternative futures \cite{bell_yesterdays_2007, nathan_envisioning_2008, kinsleyFuturesMakingPractices2012, mankoffLookingYesterdayTomorrow2013a}. Some of the contributions to HCI mentioned above are grounded in the principles of value-sensitive design \cite{nathan_envisioning_2008} or participatory design \cite{elsden_speculative_2017, epp2022reinventing}.
Such grounding promotes increased attentiveness to different perspectives when alternative futures come into play, whether through user scenarios \cite{nathan_envisioning_2008}, futures workshops \cite{epp2022reinventing}, or field trials \cite{elsden_speculative_2017, odom_fieldwork_2012}.  

Thanks to the breadth of HCI research's spread of topics and methods, we have numerous approaches at our disposal for looking at futures. While the predominant pattern might still be to envision linear trajectories from short-term visions of emerging technologies, such methods as design fiction and speculative design have expanded the field's understanding of how technology, its futures included, may be examined.  


\subsection{HCI's Future-Orientation in Light of Futures Studies}

HCI scholars are not the only ones interested in studying possible futures and exerting effects on them. Perhaps more than for any other field of research, this enterprise is the territory of futures studies, a branch of the social sciences devoted to forecasting, foresight, and anticipation of probable and possible futures alike \cite{bellWhatWeMean1996}. Correspondingly, it employs both probabilistic forecasting (e.g., trend analyses) and qualitative methods that explore plausible futures (e.g., Delphi studies, which involve distributed expert-based scenario-building \cite{roweDelphiTechniqueForecasting1999} and various scenario-development processes \cite{vanderheijdenScenariosArtStrategic2005}) \cite{bell_purpose_2009}. 

Futures studies has gone through various phases and turns, just as HCI has. An outgrowth from strategic planning~\cite{bellWhatWeMean1996}, it later experienced a critical turn~\cite{slaughterKnowledgeBaseFutures1998}. By shifting its focus to how systems connect with social reality, this continuously developing 
field highlighted how futures get differentially built conceptually and acted, varying with cultures, ways of knowing, and how people care for others~\cite{sardar_colonizing_1993, adam_futures_2011, sardar_postnormal_2015}. 
Current work in futures studies concentrates on \textit{futures literacy}~\cite{millerFuturesLiteracyTransforming2018} and \textit{futures consciousness}~\cite{ahvenharjuFiveDimensionsFutures2018}, in aims of illuminating how individuals and groups, respectively, engage in futuring. This encompasses their ways of seizing active agency in shaping the futures that affect them. Other streams of futures studies focus on the concept of \textit{postnormality} ~\cite{funtowicz_science_1993, sardar_postnormal_2015}, bound up with how volatility, uncertainty, complexity, and ambiguity are pressing us toward urgent decisions with pivotal consequences amid a marked lack of knowledge \cite{sardar_postnormal_2015, slaughter_farewell_2020}. 

The stress on examining possible futures does not imply that futures studies confines itself to studying the future. To explore alternative futures, futures studies researchers also take into account possible pasts~\cite{sardar_colonizing_1993, bendorLookingBackwardFuture2021}, the present~\cite{poliIntroductionAnticipationStudies2017, millerSensingMakingsenseFutures2018, bellWhatWeMean1996}, and the manifold futures that can open out from the exploration. These examinations require approaching the future as mutable and open-ended \cite{jouvenelArtConjecture1967, adamFutureMattersAction2007, poli_anticipation_2019}. Futures studies scholars arrive at knowledge by considering the complexity-rife arena in which several alternative futures could unfold \cite{poliIntroductionAnticipationStudies2017}. 

One commonplace conceptual technique that futures studies employs to understand the web of relations and consequences among numerous factors in the face of alternative futures is use of the STEEPLE framework (the acronym refers to social, technological, environmental, economic, political, legal, and ethics factors) \cite{aguilarScanningBusinessEnvironment1967, saritasMappingIssuesEnvisaging2012}. Digging into future alternatives unveils uncertainty patterns that have escaped our notice or that we have been consciously ignoring, and it assists in tending to the consequences of the decisions made in the present \cite{poliIntroductionAnticipationStudies2017, millerSensingMakingsenseFutures2018, adamFutureMattersAction2007}.

In spite of obvious thematic linkages due to their interest in futures, HCI and futures studies have not interacted extensively with each other. They have reached their closest in the area of scenario-based methods, with HCI studies having applied such techniques from futures studies as Delphi scenario-building \cite {mankoffLookingYesterdayTomorrow2013a} and the Futures Wheel \cite{epp2022reinventing}, a lightweight workshop-based method wherein a seed scenario stimulates envisioning of waves of first-order, second-order, and further consequences. Overlap is evident from the other side also: \citeauthor{dunneSpeculativeEverythingDesign2013a} introduced the futures cone \cite{taylorAlternativeWorldScenarios1993} in the HCI field as a visualization of how the present can develop toward preferred, probable, plausible, and possible futures. 
Beyond these arenas and envisioning (as employed in user studies), futures studies and HCI research have engaged in little collaboration, notwithstanding suggestions for joint work on speculative studies of interaction for in-the-wild studies \cite{elsden_speculative_2017,salovaaraEvaluationPrototypesProblem2017} and VR simulations \cite{simeone2022immersive}. 

Although the outline above points to several valuable contributions to HCI, we have pinpointed the field's most future-oriented publications for discussion. 
In contrast, the vast majority of HCI research has left the opportunities for futuring underexplored, even though this seems to be an inherent aspect of HCI. One reason might be that discussions of the future often hide in the bowels of design and evaluation of emerging technologies while HCI research typically gets judged not by the rigor of its futures thinking but by what opportunities it accords individuals and communities to use technology for purposes of control, reconfiguring day-to-day life, and living that life \cite{bodkerWhenSecondWave2006}.

Design fiction, speculative design, foresight-informed studies, and other investigation routes cited above have started to introduce new methods from which HCI can draw. This growing interest does not, however, shed light on the extent to which the field has actualized its inherent capacity for future-oriented thinking. 
For fuller understanding of how HCI engages with futures, also in contributions that are not explicitly future-oriented, we carried out a literature review. The next section presents its implementation.