%%
%% This is file `sample-sigconf-authordraft.tex',
%% generated with the docstrip utility.
%%
%% The original source files were:
%%
%% samples.dtx  (with options: `all,proceedings,bibtex,authordraft')
%% 
%% IMPORTANT NOTICE:
%% 
%% For the copyright see the source file.
%% 
%% Any modified versions of this file must be renamed
%% with new filenames distinct from sample-sigconf-authordraft.tex.
%% 
%% For distribution of the original source see the terms
%% for copying and modification in the file samples.dtx.
%% 
%% This generated file may be distributed as long as the
%% original source files, as listed above, are part of the
%% same distribution. (The sources need not necessarily be
%% in the same archive or directory.)
%%
%%
%% Commands for TeXCount
%TC:macro \cite [option:text,text]
%TC:macro \citep [option:text,text]
%TC:macro \citet [option:text,text]
%TC:envir table 0 1
%TC:envir table* 0 1
%TC:envir tabular [ignore] word
%TC:envir displaymath 0 word
%TC:envir math 0 word
%TC:envir comment 0 0
%%
%%
%% The first command in your LaTeX source must be the \documentclass
%% command.
%%
%% For submission and review of your manuscript please change the
%\documentclass[manuscript, review, anonymous]{acmart}
%%
%% When submitting camera ready or to TAPS, please change the command
%% to \documentclass[sigconf]{acmart} or whichever template is required
%% for your publication.
%%
%%
\documentclass[sigconf, natbib=false]{acmart}

%%%%%% PACKAGES
\usepackage{dirtytalk}
\usepackage{tabularx}
\usepackage{graphicx}
\usepackage{pdfpages}
% \usepackage{tcolorbox}
% \usepackage{xcolor}

%%
%% \BibTeX command to typeset BibTeX logo in the docs
\AtBeginDocument{%
  \providecommand\BibTeX{{%
    Bib\TeX}}}

%% Rights management information.  This information is sent to you
%% when you complete the rights form.  These commands have SAMPLE
%% values in them; it is your responsibility as an author to replace
%% the commands and values with those provided to you when you
%% complete the rights form.
% \copyrightyear{2025}
% \acmYear{2025}
% \setcopyright{cc}
% \setcctype{by}
% \acmConference[CHI '25]{CHI Conference on Human Factors in Computing Systems}{April 26 -- May 1, 2025}{Yokohama, Japan}
% \acmBooktitle{CHI Conference on Human Factors in Computing Systems (CHI '25), April 26 -- May 1, 2025, Yokohama, Japan}\acmDOI{10.1145/3706598.3713759}
% \acmISBN{979-8-4007-1394-1/25/04}

\copyrightyear{2025}
\acmYear{2025}
\setcopyright{cc}
\setcctype{by}
\acmConference[CHI '25]{CHI Conference on Human Factors in Computing Systems}{April 26-May 1, 2025}{Yokohama, Japan}
\acmBooktitle{CHI Conference on Human Factors in Computing Systems (CHI '25), April 26-May 1, 2025, Yokohama, Japan}\acmDOI{10.1145/3706598.3713759}
\acmISBN{979-8-4007-1394-1/25/04}




%\setcopyright{acmlicensed}
%\copyrightyear{2024}
%\acmYear{2024}
%\acmDOI{XXXXXXX.XXXXXXX}
%% These commands are for a PROCEEDINGS abstract or paper.
%\acmConference[CHI '25]{Conference on Human Factors in Computing Systems}{April 26-- May 1,2025}{Yokohama, Japan}
%%
%%  Uncomment \acmBooktitle if the title of the proceedings is different
%%  from ``Proceedings of ...''!
%%
%%\acmBooktitle{Woodstock '18: ACM Symposium on Neural Gaze Detection,
%%  June 03--05, 2018, Woodstock, NY}
%\acmISBN{978-1-4503-XXXX-X/2018/06}


%%
%% Submission ID.
%% Use this when submitting an article to a sponsored event. You'll
%% receive a unique submission ID from the organizers
%% of the event, and this ID should be used as the parameter to this command.
%%\acmSubmissionID{123-A56-BU3}

%%
%% For managing citations, it is recommended to use bibliography
%% files in BibTeX format.
%%
%% You can then either use BibTeX with the ACM-Reference-Format style,
%% or BibLaTeX with the acmnumeric or acmauthoryear sytles, that include
%% support for advanced citation of software artefact from the
%% biblatex-software package, also separately available on CTAN.
%%
%% Look at the sample-*-biblatex.tex files for templates showcasing
%% the biblatex styles.
%%
%% Bibliography style
\RequirePackage[
  datamodel=acmdatamodel,
  style=acmnumeric,
  ]{biblatex}

%% Declare bibliography sources (one \addbibresource command per source)
\addbibresource{references.bib}
\addbibresource{includes.bib}
\addbibresource{fleeting.bib}
%%
%% The majority of ACM publications use numbered citations and
%% references.  The command \citestyle{authoryear} switches to the
%% "author year" style.
%%
%% If you are preparing content for an event
%% sponsored by ACM SIGGRAPH, you must use the "author year" style of
%% citations and references.
%% Uncommenting
%% the next command will enable that style.
%%\citestyle{acmauthoryear}


%%
%% end of the preamble, start of the body of the document source.
\begin{document}

%%
%% The "title" command has an optional parameter,
%% allowing the author to define a "short title" to be used in page headers.
\title[Let’s Talk Futures: A Literature Review of HCI’s Future-Orientation]{Let’s Talk Futures: A Literature Review of HCI’s Future-Orientation}

%%
%% The "author" command and its associated commands are used to define
%% the authors and their affiliations.
%% Of note is the shared affiliation of the first two authors, and the
%% "authornote" and "authornotemark" commands
%% used to denote shared contribution to the research.
\author{Camilo Sanchez}
\orcid{0000-0002-8486-031X}
\email{camilo.sanchez@aalto.fi}
\affiliation{%
  \institution{Aalto University}
  \city{Espoo}
  \country{Finland}
}

\author{Sui Wang}
\orcid{0009-0009-3748-3417}
\email{suiwang@usc.edu}
\affiliation{%
  \institution{University of Southern California}
  \city{Los Angeles}
  \country{United States of America}}

\author{Kaisa Savolainen}
\orcid{0000-0003-2907-6036}
\email{kaisa.savolainen@aalto.fi}
\affiliation{%
  \institution{Aalto University}
  \city{Espoo}
  \country{Finland}
}

\author{Felix Anand Epp}
\orcid{0000-0001-6252-7244}
\email{mail@felix.science}
\affiliation{%
  \institution{Aalto University}
  \city{Espoo}
  \country{Finland /} % Either a ` / ' here or a small vertical gap after the first affiliation would be very good. -als
}
\affiliation{%
  \institution{University of Helsinki}
  \city{Helsinki}
  \country{Finland}
}

\author{Antti Salovaara}
\email{antti.salovaara@aalto.fi}
\orcid{0000-0001-7260-8670}
\affiliation{%
  \institution{Aalto University}
  \city{Espoo}
  \country{Finland}
}

%%
%% By default, the full list of authors will be used in the page
%% headers. Often, this list is too long, and will overlap
%% other information printed in the page headers. This command allows
%% the author to define a more concise list
%% of authors' names for this purpose.
\renewcommand{\shortauthors}{Sanchez et al.}

%%
%% The abstract is a short summary of the work to be presented in the
%% article.
\begin{abstract}
HCI is future-oriented by nature:
it explores new human--technology interactions and applies the findings to promote and shape vital visions of society. Still, the visions of futures in HCI publications seem largely implicit, techno-deterministic, narrow, and lacking in roadmaps and attention to uncertainties. A literature review centered on this problem examined futuring and its forms in the ACM Digital Library's most frequently cited HCI publications. This analysis entailed developing the four-category framework SPIN, informed by futures studies literature. The results confirm that, while technology indeed drives futuring in HCI, a growing body of HCI research is coming to challenge techno-centric visions. Emerging foci of HCI futuring demonstrate active exploration of uncertainty, a focus on human experience, and contestation of dominant narratives. The paper concludes with insight illuminating factors behind techno-centrism's continued dominance of HCI discourse, as grounding for five opportunities for the field to expand
its contribution to futures and anticipation research.

\end{abstract}

%%
%% The code below is generated by the tool at http://dl.acm.org/ccs.cfm.
%% Please copy and paste the code instead of the example below.
%%
\begin{CCSXML}
<ccs2012>
   <concept>
       <concept_id>10003120.10003121.10003126</concept_id>
       <concept_desc>Human-centered computing~HCI theory, concepts and models</concept_desc>
       <concept_significance>500</concept_significance>
       </concept>
 </ccs2012>
\end{CCSXML}

\ccsdesc[500]{Human-centered computing~HCI theory, concepts and models}


%%
%% Keywords. The author(s) should pick words that accurately describe
%% the work being presented. Separate the keywords with commas.
\keywords{Literature Review, Futures, Futures Studies}
%% A "teaser" image appears between the author and affiliation
%% information and the body of the document, and typically spans the
%% page.
\received{12 September 2024}
\received[revised]{12 November 2024}
\received[accepted]{16 January 2025}

%%
%% This command processes the author and affiliation and title
%% information and builds the first part of the formatted document.

\maketitle

\begin{refsection}
\section{Introduction}
Human--computer interaction (HCI) is a future-oriented enterprise. Interactive prototypes, technical demonstrations and new design methods materialize future visions and shape our ideas of what futures can be \cite{bell_yesterdays_2007, kinsleyFuturesMakingPractices2012, salovaaraEvaluationPrototypesProblem2017, lindley_implications_2017}. They do so by introducing imaginaries of advances wrought by technological development \cite{bell_yesterdays_2007, lindley_implications_2017}.
However, because the HCI's field primary concerns are technology development and implications, it risks overlooking other matters of importance, such as systemic implications and long-term consequences\cite{nathan_envisioning_2008, mankoffLookingYesterdayTomorrow2013a, salovaaraEvaluationPrototypesProblem2017}. 
This leads us to ask whether the future-orientation so central to our discipline needs refining to support more meaningful research inquiry. 

For example, while scenario-building is commonly found in the HCI toolbox, nearly always wielded for envisioning future contexts of technology use \cite{carroll_making_2003}, detailed instructions for constructing the scenarios and on how HCI scholars should carry out their future-building \cite{nathan_envisioning_2008, pargmanSustainabilityImaginedFuture2017} are scarce within HCI, and could benefit from critical attention. The intricacies of how societies and perspectives get portrayed in such scenarios and the visions further illustrate this problem. Seemingly neutral narratives that incorporate certain visions may readily simplify the complex weave wherein interactive technologies exert influence over day-to-day life \cite{bell_yesterdays_2007, nathan_envisioning_2008, ashby_fourth-wave_2019}. If nothing else, visions for HCI often express interests of actors invested in the technologies \cite{bell_yesterdays_2007, kinsleyFuturesMakingPractices2012, harrington_eliciting_2021} and perpetuate socio-technological foci that may leave the agencies of more-than-human systems to the side \cite{forlanoPosthumanismDesign2017} or neglect underprivileged and vulnerable stakeholders. 

Enhancing HCI's future-savviness could safeguard against superficial engagement with futures and help us identify pressing questions, alongside rich opportunities \cite{nathan_envisioning_2008}. 
Interest in futuring is growing in design-oriented HCI studies especially, as approaches such as design fiction \cite{bleecker_design_2009, sterling_design_2009}, critical design \cite{dunne_hertzian_2006, bardzell_critical_2012}, speculative design \cite{dunneSpeculativeEverythingDesign2013a, wongSpeculativeDesignHCI2018}, and more-than-human design \cite{forlanoPosthumanismDesign2017, giaccardiTechnologyMoreThanHumanDesign2020, wakkary2021things,lu2024ecological,sanchezPeeringThroughTime2025} attest, yet the level of exploration and application of futures knowledge remains uneven across our field. Some future-oriented studies are far more comprehensive than others. 

This paper contributes to the solid development of future-oriented research in HCI by examining the prevalence of short-sighted futuring in HCI. We conducted in-depth investigation informed by works that have called for stronger engagement between HCI and futures studies \cite{mankoffLookingYesterdayTomorrow2013a, pargmanSustainabilityImaginedFuture2017, epp2022reinventing, moesgenDesigningUncertainFutures2023}, to construct a comprehensive picture of HCI's status as a future-oriented research discipline. Specifically, we analyzed how the field takes futures into account and formulated ways to undertake more reflective \textit{futuring}, where we use the latter term to describe a dual process: firstly, striving to understand current events deeply and secondly, using that understanding to inform future-oriented actions by exploring potential possibilities \cite{bellWhatWeMean1996}. 

To address these goals, we conducted a literature review assessing the extent to which HCI scholarship engages in futuring and carried out analysis directed at the following research questions: 

\begin{description}
    \item{RQ1:} \textit{How has the HCI field's future-orientation developed over the last 15 years?} 
    \item{RQ2:} \textit{What traits characterize the field's comprehensive futuring?}
\end{description}


We began by applying keywords and citation metrics to identify influential future-oriented papers from 11 high-profile journals and conferences in the ACM Digital Library (DL). 
Proceeding from this initial set of articles, we devised selection criteria that afforded manually inspecting the papers' Introduction, Discussion, and Conclusion sections and, accordingly, grouping the future-oriented publications into those attending to futures only fleetingly and the ones whose futuring could be considered comprehensive. For the 205 articles that appeared comprehensive in their future-orientation, we performed full-text qualitative analysis. To support our analysis, we developed a futures studies-informed analysis framework encompassing four categories of futuring: epistemic stance, contingency perceptions, systemic integration, and narrative. Via the outputs from this process, we are able to describe the timelines followed by the future-orientation work behind the most influential HCI papers, over the years and by publication channel; describe how the most future-savvy papers have handled futuring; and articulate the opportunities available to HCI researchers today, so as to deepen futures inquiry in our field. 

\section{Related Research} \label{sec:related-research}

Ever since the discipline's inception as a response to technological advancement \cite{baecker_timelines_2008}, HCI has been evolving, in waves \cite{harrison_making_2011, bodkerWhenSecondWave2006, bodker_third-wave_2015, frauenbergerEntanglementHCINext2019}, with its approach to futures adapting accordingly.  
Recent development suggests that the field is starting to take a more active and political stance to the futures addressed \cite{ashby_fourth-wave_2019}. 
Increasingly often, HCI researchers step beyond user studies, into co-creation of futures with overlooked communities \cite{harrington_eliciting_2021, freeman_rediscovering_2022} and non-human actors \cite{liu_design_2018, reddy_making_2021, heitlingerAlgorithmicFoodJustice2021}.
Below, we review the vast expanse of both conventional and emergent approaches to considering the future in HCI. Against this backdrop, we can paint a picture of dovetailing and divergences between HCI and futures studies, the branch of social science most directly tackling systematic explorations of alternative futures~\cite{bellWhatWeMean1996}.

\subsection{Conventional Approaches to Examining the Future in HCI}
To study futures, one needs to study possibilities. Scholars of HCI tackle this in various ways, among which are prototyping \cite{limAnatomyPrototypesPrototypes2008}, scenario-building (either by the researchers themselves or with possible future users \cite{carroll_making_2003, stromberg2004interactive, epp2022reinventing}), observing (potentially opportunity-exposing) ways in which users make use of technologies \cite{norman2008workarounds,norman2013incremental}, and reflecting on implications for future technologies' development on the basis of user studies \cite{luria_re-embodiment_2019}. 

Despite the work's future-orientation, HCI researchers rarely reflect comprehensively on the complexity of developing futures. This leaves room for improvement in several areas. One issue that rises to prominence is \emph{techno-centrism}: HCI scenarios often are informed by the goal of suggesting new technologies or uses, and the studies typically present prototypes without deeper considerations of how their adoption might develop or of various contingencies' potential influences \cite{lindley_implications_2017,pargmanSustainabilityImaginedFuture2017}. A few scholars have acknowledged the pressing need to consider futures more broadly. Among the most prominent has been \citeauthor{carroll_making_2003}, whose book \emph{Making Use} \cite{carroll_making_2003} discusses the history of scenario-building by referencing the work of \citeauthor{kahn1962thinking} \cite{kahn1962thinking}, a pioneer of scenario-based methods' use in strategic planning, head-on; however, even \citeauthor{carroll_making_2003} narrowed his attention to possible human interactions with technologies, as opposed to broader socio-political considerations or environmental contexts.



While HCI studies may give superficial attention to ethics factors, many fail to integrate corresponding perspectives, political sensitivity, or justice considerations meaningfully into the design of futures. For example, surveillance issues often get sidelined via brief mentions of the pervasive rise of technologies that monitor/track user behavior \cite{zuboff2015bigother}. Likewise, disparities in access to future technologies and the costs of unintended consequences remain under examined \cite{benjamin2019raceaftertechnology}. As automation and AI-based systems threaten to reshape entire industries, often at the expense of vulnerable workers \cite{eubanks2018automatinginequality}, the shadow of labor displacement has taken on especially large dimensions. Yet the ethics implications of adopting new technologies at scale -- from environmental and sustainability domains to mechanisms for data governance and prevention of misuse -- often receive only surface-level treatment, if any at all \cite{jasanoff2016dreamscapes}.


Even in work attuned to ``far-future'' scenarios and active honing of visions, the scenarios' plausibility is seldom scrutinized. In contrast, researchers often approach long-time-horizon studies as thought experiments, seeds for speculative responses to the present \cite{coulton2017design,wong2016when,blythe_research_2014}. In such cases, the visions become \emph{leaps into the future}, in that the work glosses over the intermediate steps needed for their realization. Neglecting the pathways to the future leaves gulfs in understanding of how such futures might materialize.      

In summary, the promise of rapid, productive strides in optimizing and enhancing human interaction with technological systems has brought HCI research a double-edged sword. The methods developed have drawn the field's future-orientation to immediate events and concrete examples of technology use while broader, speculative engagements with possible futures, or possible routes to distant futures, often suffer from neglect. These shortcomings spotlight the need for an expanded approach to futures, wherein HCI scholars embrace diverse explorations that are scalable and context-aware.


\subsection{Speculative Futures in HCI Studies} \label{sec:speculative-futures-HCI} 
In the last decade, the field's traditional approaches to considering the future have been joined by techniques that place greater focus on innovative approaches whereby speculation and foresight aid in examining possible futures. One distinguishing feature of these emerging practices is their stress on envisioning alternative futures -- e.g., on futures that diverge from the most probable scenarios, challenge prevailing assumptions about potential futures, and could help us deepen our understanding of technology design by probing alternative outcomes \cite{kozubaev_expanding_2020}.  

The evolving HCI landscape has witnessed sprouting of speculative design, 
critical design 
and design fiction, 
as fruit of a rich tradition of critical theory and artistic critique. Emerging as responses to technology-design practitioners' conventional orientation toward utilitarian and market-driven outcomes, they are directed instead to challenging dominant narratives and inspiring profound reflection on societal impacts. Rather than content themselves with familiar tools such as prediction and forecasting, proponents of these methods foreground the role of design in crafting scenarios that provoke discussion, question current assumptions, and reimagine worldviews ``otherwise'' \cite{blythe_research_2014}. They embrace a plurality of futuring \cite{howell2021calling}, thus inviting a rich array of speculative visions that extend the sphere of inquiry beyond linear, deterministic projections.

Speculative design often applies the tools of artifact- and scenario-creation to stimulate discussion and, thereby, exlore possible futures \cite{wong2018speculative, dunneSpeculativeEverythingDesign2013a}. Where older methods focus solely on likely or desirable futures, speculative design thrives on ``what if...?'' scenarios to bring in also those futures that could yield insight even if they remain improbable. By letting designers and other participants examine the implications of today's decisions, technology trends, and developments in society, speculative design broadens views, for greater attention to possibilities that may challenge the \emph{status quo}, and homes in on avenues for critical thinking about future technologies.

Critical design goes further still, by explicitly focusing on generating controversy and debate \cite{bardzell_critical_2012, bardzell_reading_2014, bardzell_what_2013}. Often, the critique gets embodied in artifacts that challenge presumptions about the role of technology in society by pointing to the flaws and unfulfilled potential of current technology-use practices. It champions the principle of anti-solutionism \cite{blythe_anti-solutionist_2016}, contesting the supposition that technology must always offer solutions. Positing that deeper understanding and acknowledgment of problems can drive more meaningful developments, critical design creates criticism-imbued artifacts that can help society expose and interrogate the values, practices, and power structures enmeshed in everyday technologies. Thus it encourages a conscious, reflective approach to design.

Design fiction, in turn, is a strongly storytelling-centered approach that crafts narratives instrumented to probe and manifest alternative futures \cite{bleecker_design_2009, sterling_design_2009, tanenbaum_design_2014}. Particularly effective at rendering abstract scenarios tangible and relatable, it displays power to bridge the gap between scientific exploration of concepts and emotional engagement with an audience \cite{coulton2017design}. Design-fiction narratives situate hypothetical technologies within detail-populated social, cultural, and ethics-bearing contexts. These stories anchor potential technologies in fictional interactions between technologies and intended users. 

All these speculative methods echo ideas from critical future studies \cite{slaughterKnowledgeBaseFutures1998}, especially in their emphasis on sensitization to uncertainty 
and recognizing 
the open nature of future outcomes \cite{howell2021calling, poli_anticipation_2019, gallHowVisualiseFutures2022}. Together, they constitute a richly woven tapestry for grappling with possibilities that transcend technological determinism. Speculative methods contribute to futures studies via tools and frameworks that play with and strategize around uncertainty. By harnessing it as a resource to enrich discourse about the futures we choose to pursue \cite{epp_uncertainties_2024}, they work alongside other methods to flesh out the picture. 

\subsection{Possible Futures in HCI} \label{sec:possible-futures-HCI}

\citeauthor{weiser1991computer}’s vision of computing for the 21st century \cite{weiser1991computer} strongly influenced HCI research. That narrow view of the future reproduced mostly middle-class US societal ideals~\cite{bell_yesterdays_2007}, 
but reliance on \emph{any} monocultural imaginary of future computing is bound to lead to conceiving of futures principally on near horizons without reflecting on technologies at systemic scales \cite{nathan_envisioning_2008, mankoffLookingYesterdayTomorrow2013a}. 
The gap has gained increasing recognition in calls for longer-term perspectives and more holism in approaches to possible futures \cite{nathan_envisioning_2008, mankoffLookingYesterdayTomorrow2013a, salovaaraEvaluationPrototypesProblem2017, epp2022reinventing, elsden_speculative_2017, pargmanSustainabilityImaginedFuture2017, lightCollaborativeSpeculationAnticipation2021a}.

For interaction designers to think about their interactive technologies' evolution, consider the consequences 5--20 years down the line, and identify long-term implications \cite{nathan_envisioning_2008, mankoffLookingYesterdayTomorrow2013a}, they must be encouraged to attend to externalities, such as how the technology might affect the environmental, societal, ethics, or economic domain \cite{nathan_envisioning_2008, mankoffLookingYesterdayTomorrow2013a, pargmanSustainabilityImaginedFuture2017, epp2022reinventing, moesgenDesigningUncertainFutures2023}. Such consequences may evolve slowly and stay unforeseeable within short timeframes \cite{pargmanSustainabilityImaginedFuture2017}.

Another major challenge for futuring arises from future-contin\-gent factors' interdependencies. 
Since, for example, environmental and economic forces can interact in complex and
unexpected ways, teasing out the uncertainties necessitates a systemic, open-ended angle on the future 
\cite{jouvenelArtConjecture1967, adamFutureMattersAction2007, poli_anticipation_2019}, such as a perspective acknowledging that multiple futures may warrant exploration and that diverse trajectories have to be considered \cite{pargmanSustainabilityImaginedFuture2017, salovaaraEvaluationPrototypesProblem2017, epp2022reinventing, moesgenDesigningUncertainFutures2023}.

Inherent biases too warrant contemplation. Similarly to how a techno-optimistic approach to interaction design can leave one misapprehending stakeholder needs \cite{nathan_envisioning_2008}, so can insufficient awareness of one's assumptions about the alternative futures \cite{bell_yesterdays_2007, nathan_envisioning_2008, kinsleyFuturesMakingPractices2012, mankoffLookingYesterdayTomorrow2013a}. Some of the contributions to HCI mentioned above are grounded in the principles of value-sensitive design \cite{nathan_envisioning_2008} or participatory design \cite{elsden_speculative_2017, epp2022reinventing}.
Such grounding promotes increased attentiveness to different perspectives when alternative futures come into play, whether through user scenarios \cite{nathan_envisioning_2008}, futures workshops \cite{epp2022reinventing}, or field trials \cite{elsden_speculative_2017, odom_fieldwork_2012}.  

Thanks to the breadth of HCI research's spread of topics and methods, we have numerous approaches at our disposal for looking at futures. While the predominant pattern might still be to envision linear trajectories from short-term visions of emerging technologies, such methods as design fiction and speculative design have expanded the field's understanding of how technology, its futures included, may be examined.  


\subsection{HCI's Future-Orientation in Light of Futures Studies}

HCI scholars are not the only ones interested in studying possible futures and exerting effects on them. Perhaps more than for any other field of research, this enterprise is the territory of futures studies, a branch of the social sciences devoted to forecasting, foresight, and anticipation of probable and possible futures alike \cite{bellWhatWeMean1996}. Correspondingly, it employs both probabilistic forecasting (e.g., trend analyses) and qualitative methods that explore plausible futures (e.g., Delphi studies, which involve distributed expert-based scenario-building \cite{roweDelphiTechniqueForecasting1999} and various scenario-development processes \cite{vanderheijdenScenariosArtStrategic2005}) \cite{bell_purpose_2009}. 

Futures studies has gone through various phases and turns, just as HCI has. An outgrowth from strategic planning~\cite{bellWhatWeMean1996}, it later experienced a critical turn~\cite{slaughterKnowledgeBaseFutures1998}. By shifting its focus to how systems connect with social reality, this continuously developing 
field highlighted how futures get differentially built conceptually and acted, varying with cultures, ways of knowing, and how people care for others~\cite{sardar_colonizing_1993, adam_futures_2011, sardar_postnormal_2015}. 
Current work in futures studies concentrates on \textit{futures literacy}~\cite{millerFuturesLiteracyTransforming2018} and \textit{futures consciousness}~\cite{ahvenharjuFiveDimensionsFutures2018}, in aims of illuminating how individuals and groups, respectively, engage in futuring. This encompasses their ways of seizing active agency in shaping the futures that affect them. Other streams of futures studies focus on the concept of \textit{postnormality} ~\cite{funtowicz_science_1993, sardar_postnormal_2015}, bound up with how volatility, uncertainty, complexity, and ambiguity are pressing us toward urgent decisions with pivotal consequences amid a marked lack of knowledge \cite{sardar_postnormal_2015, slaughter_farewell_2020}. 

The stress on examining possible futures does not imply that futures studies confines itself to studying the future. To explore alternative futures, futures studies researchers also take into account possible pasts~\cite{sardar_colonizing_1993, bendorLookingBackwardFuture2021}, the present~\cite{poliIntroductionAnticipationStudies2017, millerSensingMakingsenseFutures2018, bellWhatWeMean1996}, and the manifold futures that can open out from the exploration. These examinations require approaching the future as mutable and open-ended \cite{jouvenelArtConjecture1967, adamFutureMattersAction2007, poli_anticipation_2019}. Futures studies scholars arrive at knowledge by considering the complexity-rife arena in which several alternative futures could unfold \cite{poliIntroductionAnticipationStudies2017}. 

One commonplace conceptual technique that futures studies employs to understand the web of relations and consequences among numerous factors in the face of alternative futures is use of the STEEPLE framework (the acronym refers to social, technological, environmental, economic, political, legal, and ethics factors) \cite{aguilarScanningBusinessEnvironment1967, saritasMappingIssuesEnvisaging2012}. Digging into future alternatives unveils uncertainty patterns that have escaped our notice or that we have been consciously ignoring, and it assists in tending to the consequences of the decisions made in the present \cite{poliIntroductionAnticipationStudies2017, millerSensingMakingsenseFutures2018, adamFutureMattersAction2007}.

In spite of obvious thematic linkages due to their interest in futures, HCI and futures studies have not interacted extensively with each other. They have reached their closest in the area of scenario-based methods, with HCI studies having applied such techniques from futures studies as Delphi scenario-building \cite {mankoffLookingYesterdayTomorrow2013a} and the Futures Wheel \cite{epp2022reinventing}, a lightweight workshop-based method wherein a seed scenario stimulates envisioning of waves of first-order, second-order, and further consequences. Overlap is evident from the other side also: \citeauthor{dunneSpeculativeEverythingDesign2013a} introduced the futures cone \cite{taylorAlternativeWorldScenarios1993} in the HCI field as a visualization of how the present can develop toward preferred, probable, plausible, and possible futures. 
Beyond these arenas and envisioning (as employed in user studies), futures studies and HCI research have engaged in little collaboration, notwithstanding suggestions for joint work on speculative studies of interaction for in-the-wild studies \cite{elsden_speculative_2017,salovaaraEvaluationPrototypesProblem2017} and VR simulations \cite{simeone2022immersive}. 

Although the outline above points to several valuable contributions to HCI, we have pinpointed the field's most future-oriented publications for discussion. 
In contrast, the vast majority of HCI research has left the opportunities for futuring underexplored, even though this seems to be an inherent aspect of HCI. One reason might be that discussions of the future often hide in the bowels of design and evaluation of emerging technologies while HCI research typically gets judged not by the rigor of its futures thinking but by what opportunities it accords individuals and communities to use technology for purposes of control, reconfiguring day-to-day life, and living that life \cite{bodkerWhenSecondWave2006}.

Design fiction, speculative design, foresight-informed studies, and other investigation routes cited above have started to introduce new methods from which HCI can draw. This growing interest does not, however, shed light on the extent to which the field has actualized its inherent capacity for future-oriented thinking. 
For fuller understanding of how HCI engages with futures, also in contributions that are not explicitly future-oriented, we carried out a literature review. The next section presents its implementation.

\section{A Literature-Based Examination of Futuring in HCI} \label{sec:literature-review} 

As we discussed above, future-orientation is an integral part of HCI. Our discipline envisions and evaluates technologies that shape our imaginaries the future and engages individuals in these visions through interactive prototypes and technological demonstrations. 
However, the extent to which HCI consciously leverages this potential in HCI seems underexplored.
Although the techno-centric focus and short-termism in HCI have been acknowledged \cite{nathan_envisioning_2008, mankoffLookingYesterdayTomorrow2013a, pargmanSustainabilityImaginedFuture2017, salovaaraEvaluationPrototypesProblem2017}, these critiques remain on a general level.
However, we stated in \nameref{sec:related-research} that HCI has potential to dedicate its research on futures more reflectively. The following review seeks to find out how much that is already done in our field's most influential papers.

\subsection{Stage 1: Identification}
\label{sec:identification}

To honor the guiding proposition that HCI is inherently future-oriented, we could not limit our study's scope to explicitly future-oriented publications alone. A larger corpus also assists in reaching meaningful conclusions.

We took papers available in the ACM DL as our starting point for the endeavor on account of the ACM being the most important publisher of research in this field. We chose to narrow our investigation to a set of venues selected to represent the field's most respected research outlets -- CHI,\footnote{~Conference on Human Factors in Computing Systems events.} DIS,\footnote{~The Designing Interactive Systems Conference.} UIST,\footnote{~The Symposium on User Interface Software and Technology.} IUI,\footnote{~The Conference on Intelligent User Interfaces series.} CSCW,\footnote{~The Conference on Computer Supported Cooperative Work.} PACMHCI\footnote{~The journal \emph{Proceedings of the ACM on Human--Computer Interaction}.} IMWUT,\footnote{~\emph{Proceedings of the ACM on Interactive, Mobile, Wearable and Ubiquitous Technologies}.} TSC,\footnote{~\emph{Transactions on Social Computing}.} TIIS,\footnote{~\emph{ACM Transactions on Interactive Intelligent Systems}.} TOCHI,\footnote{~\emph{ACM Transactions on Computer--Human Interaction}.} and the \emph{Interactions} magazine. 
Although we could surmise that some of these attract papers less positioned for futuring than others, we did not wish to risk unwarranted assumptions, 
since our interest cohered around the HCI field as a whole.

Figure~\ref{fig:prisma} presents a PRISMA-style flow diagram \cite{page2021prisma} for our review process. The first stage -- dubbed ``Identification'' -- posed the challenge of defining search terms that supported our objective of examining the HCI field as a whole yet offered high enough specificity to rule out the bulk of the false positives. We wrestled in particular with the fact that research articles may contain future-related terms for various reasons not arising from futuring. For example, multitudes of authors describe ideas for follow-up research under the rubric of limitations and ``future work.'' 

We could not safely resort to the common strategy of filtering papers on the basis of titles, keywords, and abstracts. While that provides for efficient keyword-based identification, it would have restricted our sample to papers that are explicitly futuring-oriented. To explore the role of futuring across the full gamut of HCI research, we exploited the DL's \textit{full-text} search functionality. 
Accordingly, we iteratively searched for workable \textit{keywords} suited to the ACM DL search interface. From the vantage point of full-text search mode, we explored varied keyword combinations with searches of the venues listed above. 

\begin{figure}[tb]
    \centering\includegraphics[width=\linewidth]{screening-process.pdf}
  \caption{A PRISMA diagram of the article-screening process.}
  \Description{A flow diagram outlining the literature review inclusion/exclusion process, divided into three stages: Identification, Screening, and Final set of articles. 
  The identification stage comprised 11587 articles, out of which 8966 were removed after finding duplicates and pieces under the citation count criteria. 
  The Screening stage presents 2621 articles that were reviewed under the inclusion/exclusion criteria. This process excluded 2157 records and identified 259 articles as ``fleeting futuring''.
  The last stage presents the final set of articles. These 205 articles were categorized as comprehensive futuring and selected for qualitative full-text analysis.}
    \label{fig:prisma} 
\end{figure}

We also limited our search to ACM Digital Library's filter ``Article Type'' to ``Research Article''. This filter enabled systematic exclusion of such content types as short papers, extended abstracts, posters, and editorials. 
Since the latter decision necessitated ruling out records from before 
2008, when the DL applied its ``Research Article'' designation, our identification procedure retrieved records from then to 2023 (the most recent year with a complete set of articles at the time of our study, in 2024).
To avoid false positives stemming from musings on further studies (``future research''), we specified that ``the future'' or ``futures'' had to receive mention. We also searched for other future-related terms: ``visions,'' ``tomorrow,'' the stem ``specula*,'' etc. Finally, we added the requirement that the paper contain at least one of the terms attached to STEEPLE categories \cite{aguilarScanningBusinessEnvironment1967, saritasMappingIssuesEnvisaging2012}. 
Table~\ref{tab:search-terms} reproduces our final list of search keywords.

With most keyword combinations probed, our searches yielded 10,000+ articles, an infeasible volume for the next stage: manual analysis. However, we recognized that our well-honed search query returning 11,587 papers would be hard to improve upon. Therefore, we introduced two \textit{citation-based filters}, to reflect our interest in the most influential HCI papers. Referring to the ACM DL's paper-specific citation counts, we looked at various citation-metric cut-offs, ultimately limiting the corpus to articles that had been cited at least 50 times in all or were among the 15\% most cited in the year in question across the selected outlets. 
Excluding papers from 2024 was consistent with this rationale, in that they could not have amassed many citations.
Removal of duplicates left us with a final sample of 2,621 articles. 

\begin{table*}[t]
  \caption{The search terms used for the literature review}
  \label{tab:search-terms}
  \def\arraystretch{1.3}%
  \begin{tabularx}{\textwidth}{p{4cm}X} 
    \toprule
    \textbf{Term} & \textbf{Rationale}\\
    \midrule
    "the future" & Several notions serve to address the future: futures, futuring, future, etc. While these might refer to the same thing, they present an ontological stance to how futures are experienced (e.g., via singular, plural, or active expression). \citeauthor{sardar_namesake_2010} offers an overview of the challenge of defining futures studies \cite{sardar_namesake_2010}.\\
    "futures" & Talking about the future in the plural highlights the possibility of there being more than one future — the possible, the probable, the preferable, etc. -- and opens exploration to a panoply of futures \cite{bell_purpose_2009}. \\
    "visions" & Visions can refer to expressions of what the future might be or ideals we hold for the future \cite{vanderhelmVisionPhenomenonTheoretical2009}. \\
    envision & This element captures the process of constructing visions of the future \cite{vanderhelmVisionPhenomenonTheoretical2009}.\\ 
    tomorrow*& ``Tomorrow'' or ``tomorrows'' often functions as a synonym for the future \cite[e.g., ][]{bell_yesterdays_2007, kinsleyPractisingTomorrowsUbiquitous2010, sardarThreeTomorrowsPostnormal2016, mankoffLookingYesterdayTomorrow2013a}. \\
    specula*& Speculation has seen widespread adoption in HCI work. Building on the contribution of \citeauthor{dunneSpeculativeEverythingDesign2013a} \cite{dunneSpeculativeEverythingDesign2013a}, this method has served the field's critical examination of the future visions that new technologies represent, in particular \cite{wongSpeculativeDesignHCI2018}. \\
    imaginar*& ``Imaginary'' or ``imaginaries'' can allude to future shared imaginings of technology -- in other words, visions of the future that are constructed socially by a community, not just by a single individual \cite{fujimuraChapterFutureImaginaries2019}.  \\
    "design fiction"& Design fiction \cite{bleecker_design_2009, sterling_design_2009} is a method commonly followed to explore possible HCI futures by developing narratives without having to develop a new technology. \\
    societ* OR technolog* OR ethic* OR environmental* OR politic* OR legal* OR econom* & STEEPLE is a tool often used in futures studies to consider interrelationships among distinct factors associated with a future \cite{aguilarScanningBusinessEnvironment1967, saritasMappingIssuesEnvisaging2012}. \\ 
    \hline
    \textbf{Example queries} & [[Full text: "the future"] OR [full text: "futures"] OR [full text: "visions"] OR [Full text: envision*] OR [full text: tomorrow*] OR [full text: specula*] OR [Full text: imaginar*] OR [full text: "design fiction"]] AND [full text: societ*] \\
     & [[Full text: "the future"] OR [Full text: "futures"] OR [full text: "visions"] OR [full text: envision*] OR [Full text: tomorrow*] OR [full text: specula*] OR [full text: imaginar*] OR [Full text: "design fiction"]] AND [full text: ethic*] \\
    \bottomrule
    \end{tabularx}
\end{table*}


\subsection{Stage 2: Screening for Fleeting and Comprehensive Futuring}

Our analyses in the second stage allowed us to answer RQ1, pertaining to how HCI's future-orientation has developed in the last 15 years. 
Manual inspection of each paper pinpointed the works meriting full-text analysis in the final stage. Because experience had pointed to the paper's motivation or discussion/summaries of the findings as the strongest indicator of futuring, we directed our manual analysis to the corresponding sections of the articles 
(their introduction, discussion, and conclusion). 
We scripted downloads of the papers (full-text PDF versions of all 2,621) and programmatically highlighted the search terms' appearances within them, thus making the important parts easier to spot by eye. 

Collaborative classification of the papers and exclusion of those that do not express futuring 
required robust inclusion/exclusion criteria (see Appendix \ref{sec:inclusion-exclusion}) that the four researchers involved could reliably follow independently. This task was implemented in seven rounds, each with its own randomly selected set of 20 papers. After all individuals had worked through the set, we met and refined the inclusion/exclusion criteria. 

In the first decisions, from the round 1 meeting, our individual analyses of the paper sections' content led us to start categorizing the articles into three groups: 1) ones that provide comprehensive description of some future (``comprehensive futuring'' articles); 2) those not applying the key terms in the Introduction, Discussion, or Conclusion section or that refer to futures only in the context of proposed further work (``exclude'' articles); and 3) papers mentioning futures only superficially (``fleeting futuring''). At this point, we agreed to retain papers focused on futuring methods (e.g., speculative design) or with problematization of futuring in HCI even when explicit future visions were absent. Subsequent rounds saw us temporarily add a ``maybe'' category, for papers whose relevant sections promised that other sections would address futuring, and respecify the criteria for ``fleeting futuring'' by clarifying that the paper keeps the future technology uses or situations divorced from broader context (round 2); illustrate the definitions via excerpts from papers in each category (round 3); decide to limit our inspection to only those paragraphs in which search terms appear, rather than the entire section, and also to exclude papers that, while declaring that the work's contributions herald improvements, neglect to reflect on such implications for the future (round 4); remove the ``maybe'' category because it had become unnecessary and simultaneously exclude papers whose hypothetical use scenarios were not futuristic at the time of writing (round 5); decide to include (formerly ``maybe'') papers pointing to elaboration on futuring in sections other than the three target ones (round 6); and further explicate and fine tune the inclusion/exclusion criteria (round 7). In each round, we compared the team members' categorizations and computed the inter-rater agreement (via Fleiss kappa scores), which improved steadily, from 0.51 (in round 2) to 0.75 (in round 7), which lies near the top of the ``substantial agreement'' band (0.60--0.80) \cite{landis1977measurement}.

Upon reaching this level of reliability, we were ready to start the full-scale parallel screening. 
That entailed each of the sub-team's four researchers independently reviewing $4 \times 620$ articles not subjected to screening in the seven-round process described above. In the end, we had 205 articles screened in as manifesting comprehensive futuring, 2,157 excluded articles, and 259 in the ``fleeting futuring'' category.



\subsection{Stage 3: Qualitative Analysis Covering Comprehensive-Futuring Work}
\label{sec:qual-analysis}

We subjected the 205 comprehensive-futuring articles to in-depth reading, employing both inductive and abductive approaches. The goal for this stage of analysis was to answer RQ2, compassing the traits of comprehensive futuring in HCI. 
To analyze those characteristics, three of us began by reviewing typologies found in the futures studies literature \cite{borjesonScenarioTypesTechniques2006, ahvenharjuFiveDimensionsFutures2018, millerFuturesLiteracyTransforming2018, minkkinenSixForesightFrames2019, bergmanTruthClaimsExplanatory2010, vannottenUpdatedScenarioTypology2003, poli_anticipation_2019}.
Paying special heed to futuring models that might facilitate examining HCI's future-orientation, we selected those that articulate concepts suited to examining the degree or purpose of futuring in the literature review's sample \cite{borjesonScenarioTypesTechniques2006, minkkinenSixForesightFrames2019, millerFuturesLiteracyTransforming2018, ahvenharjuFiveDimensionsFutures2018}.
Then, the team synthesized these typologies into a four-category framework designed to inform analysis of the comprehensive-futuring papers. Table~\ref{tab:analytical-framework} presents the framework, denoted as ``SPIN'' (for its categories: Epistemic \textbf{S}tance, Contingency \textbf{P}erceptions, Systemic \textbf{I}ntegration, and \textbf{N}arrative).


\begin{table*}[t]
\caption{The SPIN framework's typology of HCI literature's futuring approaches and their possible manifestations} 
  \label{tab:analytical-framework}
\def\arraystretch{1.3}%  
\begin{tabularx}{\textwidth}{l X}
\toprule
\textbf{Category} and manifestations & \textbf{Description} \\
  \midrule
  \textbf{S -- Epistemic \underline{S}tance} &
    \textit{What stance does the futuring take?} This communicates the kind of knowledge sought and the actions pursued through the futuring process. \\
  Normative &
    Promoting a certain vision (or multiple visions) of the future as desirable or pursuable \\ 
  Explorative &
    Exploring several scenarios for the future, to gain knowledge about the future, while not actively engaging with any particular vision as pursuable \\
  Predictive &
    Presenting some future vision(s) as likely to materialize or inevitable \\
  \hline
  \textbf{P -- Contingency \underline{P}erceptions} &
    \textit{How does the futuring account for the uncertainty arising from timescale, complexity, and complex relations among trajectories?} This reveals whether the futuring process allows for uncertainty vs. focusing on only narrow visions of the future. \\
  Narrow in approach &  
    Presenting a prediction / a vision of the future that features little or no uncertainty, space for other factors to divert its course, etc. \\
  Open in approach &
    In the vision(s) of the future, accounting for possible changes, factors that are not yet known, and/or relations between/among possible visions \\
  Short-term &
    Relying heavily on the past/present, with foreseeable implications \\
  Long-term &
    Acknowledging that complexity develops through time and that effects are largely unforeseeable \\
  Envisioning-strategy-based &
     Constructing the vision by using such methods as these: a linear trajectory (past to present), alternative pasts, a prediction model, current trends, weak signals, speculation, and questioning of the \emph{status quo} \\
  \hline
  \textbf{I -- Systemic \underline{I}ntegration} &
    \textit{How and to what extent are additional factors -- social, cultural, ethics, environmental, etc. -- considered?} This elucidates what factors beyond technological ones the vision treats and how they tie in with issues of scale. \\
  Oriented toward STEEPLE (with extensions) &
    Discussing factors of some/all of the following types: social, technological, economic, environmental, political, legal, ethics, more-than-human, personal-experience, community-related, and cultural \\
  Interconnection-focused &
    Considering interdependencies between/among factors and across scales \\
  \hline
  \textbf{N -- \underline{N}arrative} &
  \textit{What is the explicit and any 
underlying narrative, view, or belief about the future?} This reveals the aspects that the authors have highlighted to present or legitimize a given vision. \\
  Rationale-aligned &
    Using particular key elements to justify a future (e.g., technological development, climate crisis, or dystopia) \\  
  Centered on viewpoints/beliefs &
    Presenting certain assumptions and worldviews in the paper\\
    \bottomrule
\end{tabularx}
\end{table*}

The first category zeroes in on \textit{epistemic stance}, examined in \citeauthor{minkkinenSixForesightFrames2019}'s analysis of futures scenarios \cite{minkkinenSixForesightFrames2019} via the lens of the levels of change pursued.
Actors' stance hinges on their futures literacy \cite{millerSensingMakingsenseFutures2018}: they might link the knowledge/epistemology to concrete actions and goals (e.g., normative strivings toward particular futures), to more explorative investigations of what \textit{could} happen, or to predictive statements intended to capture the most likely trajectories \cite{borjesonScenarioTypesTechniques2006}. 
We saw parallels with researchers' possible epistemic stances, which could prove fruitful in distinguishing among types of HCI publications -- namely, those aligned for pursuing normative technological advances, speculating about possible futures through design-fiction explorations etc., and presenting predictable implications of technology-mediated activities.

The second category, \textit{contingency perceptions}, likewise draws from \citeauthor{minkkinenSixForesightFrames2019}'s analysis -- specifically, of futures scenarios' levels of perceived unpredictability. 
In addition to unpredictability and uncertainty, which are central to every futures studies investigation, we incorporated scope-related and temporal manifestations into this category, in line with the work of both \citeauthor{ahvenharjuFiveDimensionsFutures2018} \cite{ahvenharjuFiveDimensionsFutures2018} and \citeauthor{millerFuturesLiteracyTransforming2018} \cite{millerFuturesLiteracyTransforming2018}. A scenario might be short-term or longer-term but also could be narrow or more open. 
A narrow approach can be useful when it is reasonable to assume that the future will not be affected by major external factors \cite{millerSensingMakingsenseFutures2018}. 
Researchers who adopt an open approach, in contrast, can embrace complexity and uncertainty as resources for awareness-building \cite{millerSensingMakingsenseFutures2018}. 
For crystallizing temporalities, 
we adapted \citeauthor{ahvenharjuFiveDimensionsFutures2018}'s 
\cite{ahvenharjuFiveDimensionsFutures2018} 
subjective timescales to the horizons typical of HCI scholarship: we mapped short-term visions to incremental changes (as in iterative design) and long-term considerations to accounts of emerging/unforeseeable effects, alongside unpredictable consequences \cite{epp2022reinventing}. 
Our final insight here is that analysis might benefit from bearing in mind differences in envisioning strategies: 
HCI use-scenario methods tend to be unidirectional, progressing linearly from past to future \cite{nathan_envisioning_2008}, while HCI scholars' design fiction and speculative design may contemplate alternative-future trajectories with criss-crossing, parallel stretches, and several possible starting points \cite{blythe_research_2014}.

\textit{Systemic integration} captures the differences in futuring's angles of analysis. This category covers both the traditional STEEPLE \cite{aguilarScanningBusinessEnvironment1967, saritasMappingIssuesEnvisaging2012} 
and extensions that observations from stage 2 had led us to regard as fundamental: the screening process revealed that many HCI papers attend to individual-level and community experiences, cultural factors, and more-than-human futures -- none of which the STEEPLE nomenclature directly addresses. 
The importance of these factors' interconnections prompted us to conclude that they deserve consideration as a factor in their own right; our conceptualization here adhered to \citeauthor{ahvenharjuFiveDimensionsFutures2018}'s \cite{ahvenharjuFiveDimensionsFutures2018} systems-perception dimension and \citeauthor{saritasMappingIssuesEnvisaging2012}'s \cite{saritasMappingIssuesEnvisaging2012} suggestions for scenario-building. 
This category offered particular value for problematizing our initial perception of HCI work as populated largely by a techno-centric set of envisioned futures, whereby STEEPLE's technology factor eclipses others in the futuring. We concluded that zooming in on the other possible factors should help us gauge the truth behind that assumption. 

The final SPIN category, pertaining to \textit{narratives}, contends with the explanations and views communicated in the scenarios (for example, the rationales and worldviews behind them). For this, we were informed by \citeauthor{datorAlternativeFuturesManoa2019}'s scenario archetypes and \citeauthor{bazzaniFuturesActionExpectations2023}'s narrative dimension of the future.
Via reduction (distilling the complex) and appeals to emotion (evoking affective responses), narratives shape the future visions' appearance and how they -- and the underlying beliefs -- are legitimized for audiences~\cite{datorAlternativeFuturesManoa2019, bazzaniFuturesActionExpectations2023}. 
We delineated this category for assistance in identifying the article authors'
underlying sociopolitical motivations, as well as the archetypes, assumptions, and beliefs \cite{selinSociologyFutureTracing2008} in HCI's future visions \cite{datorAlternativeFuturesManoa2019}, whether overt or implicit. 

This four-category framework provided solid theory-underpinned grounding for the full-text qualitative analysis that followed.
From our sample of 205 comprehensive-futuring papers, we began by selecting 20 at random and distributed them across all five researchers for in-depth reading. Each paper was read by three of us. 
In our analysis of these 20 papers (12 articles per researcher), we observed that many HCI studies have approached the future via the lens of individuals' experiences as opposed to system-level relations -- a dimension overlooked by our framework's synthesis of futuring typologies. 
Accordingly we returned to the SPIN categories, refining them to incorporate manifestations of personal experience, cultural factors, and more-than-human approaches, all of which seemed relevant on the basis of our prior knowledge of HCI futures discourse. 

Next, we randomly distributed a further 90 papers for reading, this time across three researchers (30 papers each).
We left space both for further tuning of the framework and for flexibility beyond this or any other specific tool.
For instance, we remained free to probe facets not visible in the categories of the SPIN framework, such as commonplace methods, key stakeholders, and the foci of the study. Our analysis was platform-agnostic too: we did not limit ourselves to any analysis application (e.g., ATLAS.ti) or template (e.g., the MS Word file portraying the framework). 
This supported balanced analysis: we were guided by the four-category framing but also encouraged to abductively identify unforeseen futuring-related interpretations with a bearing on the research questions. In practice, 1) our coding highlighted the distinct manifestations of each SPIN category in every comprehensive-futuring article while, 
in addition, 2) any researcher who noticed an important mention of the/a future that did not fit the categories could augment the evaluation accordingly. 
Each researcher concluded the paper-specific analysis with a brief qualitative summation describing the paper's position relative to the dimensions expressed by the SPIN categories. 
To ensure equitable distribution of tasks within the team, only two authors analyzed the remaining papers. 
We developed our findings in full awareness that our corpus, being limited (partly for manageability's sake) to the papers likely to have exerted the greatest influence within the HCI field, is not representative of all futuring research in the HCI domain. The earlier stages in our analysis, on the other hand, fleshed out the broader picture of the discipline considerably.


\section{Findings} \label{sec:findings}

The careful analysis presented above supplied answers to both RQs, which we address in the following two subsections.

\subsection{How HCI's Future-Orientation Has Developed (RQ1)}

Owing to the breadth of RQ1, covering the landscape of HCI futuring practice across the board, we sought to uncover the path taken thus far in HCI's future-orientation in a manner that did not necessitate full-text examination of all our large corpus.

The screening stage and its inclusion/exclusion process (see Figure~\ref{fig:prisma} and Appendix~\ref{sec:inclusion-exclusion}, respectively) provided a suitable starting point for considering RQ1. That stage's decision-making, fruit of the authors' manual analyses of Introduction, Discussion, and Conclusion sections as described above and harmonized via seven iterations, 

yielded a classification that suits this aim well: dividing the publications into three categories: those for exclusion, ``fleeting futuring'' papers, and ``comprehensive futuring.'' 

We will now characterize these groupings, which served as ample foundations for answering the first research question. In the course of our overviews of the two main categories, we show how they represent the evolution of HCI scholarship's future-orientation over time. 

The \textbf{fleeting-futuring} papers' distinguishing property is content touching 
only briefly upon the future, often in passing or as a minor aspect of the topic under discussion. These fleeting references typically lack depth and seldom reflect the main focus of the paper. The piece might mention potential for further studies
or speculate vaguely about future trends but avoid committing to detailed explorations. The following extracts exemplify these patterns. Text included before and after each (italicized) fleeting reference to some future sets these references in context, thus demonstrating that their mention of a future does not constitute part of extensive elaboration. 

\begin{quote}
    \textit{We envision StickEar to be an empowering personal device that anyone would carry and use every day to augment objects and spaces.} 
    [end of paper] \cite[p. 226]{yeo_stickear_2013}
\end{quote}
\begin{quote}
    \textit{The high maintenance and environmental cost of batteries lead to concerns about the wireless sensor networks domain. This sustainability issue has led to battery-free embedded devices powered by ambient energy sources (vibrations, radio frequency transmissions, and light). \textit{These battery-free devices are likely to form the future of physical-computing devices and the Internet Of Things due to being maintenance-free and enhancing long-term deployment.} Recent battery-less device demonstrations include phones, satellites in space, implantables, devices conducting machine learning, handheld gaming consoles, and even underwater sensing}\cite[p. 2]{kraemer_battery-free_2022}
\end{quote}
As illustrated above, the future visions presented were often condensed in one or two sentences, usually imagining the positive influence of the technology discussed.

In \textbf{comprehensive futuring}, the
papers discussed futures in a more holistic manner. These papers did not only speculate on future technologies but also considered the socio-cultural, ethical, or practical implications of these technologies. Such comprehensive treatments often involved scenarios that were richly detailed, offering both broad visions and specific predictions to construct meaningful future worlds. Awareness of the future is typically woven throughout the content of papers in this category, or a whole section might be dedicated to discussing futuristic scenarios, inclusive of thorough and more sustained engagements with futuring. We provide multiple examples of this category in the next subsection in conjunction with the characteristics we pinpointed as constitutive of the last 15 years' comprehensive futuring in HCI. 

\begin{figure*}[tb]
    \centering\includegraphics[width=0.95\linewidth]{mirrored_histogram-corrected.pdf}
  \caption{A histogram presenting the articles in our comprehensive-futuring sample (blue) and in the fleeting-futuring sample (green) by year (on the \emph{x}-axis) and by publication venue (on the \emph{y}-axis). Bar length corresponds to the number of articles in each category.
  Empty space indicates that no relevant articles were published in the relevant year and venue (CSCW material published in PACM is included (only) under the CSCW category). 
  }
  \Description{A histogram representing the amount of articles that were in the comprehensive futuring sample, represented by blue bars, and the fleeting sample, represented by green bars, and in 
  Both categories are distributed by year, from 2008 to 2023, in the x-axis and by publication series in the y-axis.
  Bar length offers an approximation of the number of articles in each category.
Empty space indicates no articles in that year and venue. 
  The histogram highlights a general upward trend in both categories across most series, particularly in CHI, CSCW and DIS. The comprehensive futuring contributions, while smaller in number, have increased noticeably in recent years.
  }
    \label{fig:heatmap}
\end{figure*}

Graphically depicting the development of futuring within HCI over the years (e.g., visually answering RQ1), Figure~\ref{fig:heatmap} presents a timeline of how both categories of future-oriented pieces in the venues examined have developed. 
Note that this graph covers the counts only of fleeting- and comprehensive-futuring papers that 
were above our citation cut-off (see Subsection~\ref{sec:identification}).
While Figure~\ref{fig:heatmap} hence does not represent the total volume of future-oriented papers in each venue, category, or year, it does offer a revealing overview of the most cited future-oriented papers in ACM's HCI venues over the last 15 years.

A trend toward future-oriented articles began to emerge in most of the venues in 2012 (TIIS and \emph{Interactions} constituted exceptions).\footnote{~Note, however, that not all of the publication venues were active throughout the span of time considered. For instance, TIIS was launched in 2011 while PACMHCI and IMWUT entered print only in 2017. 
Likewise, CSCW being biennial until 2012 and DIS until 2016 explains gaps in the corresponding timelines.}
This trend has accelerated since. With regard to such venues as CHI, DIS, CSCW, and TOCHI, we can interpret the growing number of comprehensive-futuring articles as evidence of a tendency to conceptually approach the future in an increasingly holistic manner. Growth in comprehensive futuring peaked in 2017, though the paper count is still showing considerable growth, in the 2020s. 

The fleeting-futuring category too shows growth over the years, particularly from 2011 onward. While the rise in this class of futuring is less strong, even the more technical areas of HCI research clearly demonstrate a growing inclination to extrapolate technologies to visions of the future. 
Overall, fleeting futuring seems to dominate the HCI domain, but there is an evident shift to engaging in comprehensive futuring, particularly in the venues where the social sciences and design practitioners exert a strong influence. And that trend is gaining pace.

To render the temporal development of HCI's future-orientation more comprehensible, it is worth considering 
the fundamental nature of this orientation: how it established itself in the first place. 
The ``fleeting futuring'' category's dominance in Figure~\ref{fig:heatmap} might be a consequence of HCI research’s inherently technology-oriented nature. 
In this light, it may not be surprising that technology forms the main lens through which visions of futures get articulated. 
The operationalization we developed for the fleeting mentions echoes this -- we found brief sketches of potential future applications that depend on the technology introduced in the paper. 
We noticed that fleeting visions of the future typically point to technologies as the main driver of change, therefore 
suggesting leaps or linear progression from certain current applications to future possibilities. These visions also often stress the more positive opportunities created by emerging technologies. 
The extracts below, all from the ``fleeting futuring'' category, illustrate the extrapolation of technology to visions of the future. 
Such extrapolation existed also in comprehensive-futuring papers, but space considerations preclude quoting from the latter, since their descriptions of the future generally extend through two or more paragraphs \cite[e.g., ][]{mueller_jogging_2015, su_dolls_2019, jhaver_designing_2022}.

\begin{quote}
    \textit{We envision eye trackers in the future to be integrated with consumer devices (laptops, mobile phones, displays), hence allowing the user’s gaze to be analyzed and used as input for interactive applications.} \cite[p. 267]{alt2014eye}
\end{quote}
\begin{quote}
\textit{In the future, it is possible that many people who are not already able to drive and are considering learning will have the option of buying an AV} [automated vehicle] \textit{instead.} \cite[p. 526]{hewitt_assessing_2019}
\end{quote} 

Emerging technologies were often treated as springboards that allow for elaboration on potential futures. This pattern reveals a propensity to envision futures through the lens of technological capabilities or implications, thereby confining the imagination of the future to the proximity of certain technologies' existence. 
For example, the authors of many papers in each of the categories focused predominantly on technological factors when envisioning the future. That said, there are notable exceptions: researchers whose envisioning of the future has encompassed non-technological factors \cite{tomlinson_collapse_2013, oogjes_designing_2018, vines_age-old_2015}.
The reliance on technology as the underpinnings to futuring in both categories of papers reveals a degree of techno-centrism in HCI. Indeed, studies' prominent focus on incremental advances in technology and on persuasion to use technology is called out in several papers addressing sustainable HCI \cite{pierce_beyond_2012, brynjarsdottir_sustainably_2012, chopra_negotiating_2022, crivellaro_contesting_2015}. Their authors have called for further engagement with the societal, ethics, and environmental aspects of technologies. 

\noindent Narrowing our examination of techno-centrist tendencies to papers in the comprehensive-futuring category uncovers greater variation in how scholars regard technologies. 
These works often expand the discussion to include broader future scenarios for technology use, specifically involving individual-level experiences \cite{devendorf2016dontwear, luria_re-embodiment_2019, sanchez-cortes_mood_2015, su_dolls_2019}, as alluded to above. For example, one project explored how wearables could mediate attention and contribute to stress management: 

\begin{quote}
    \textit{Even though she imagined activating her scarf with her phone, she felt that her scarf interface would ameliorate stress by providing a slowly shifting dedicated information stream that would allow her to direct her attention to the world, interacting with others, rather than frequently checking her phone.} \cite[p. 6036]{devendorf2016dontwear}
\end{quote}
Similarly, \citeauthor{luria_re-embodiment_2019} investigated how social presence in the auto\-nomous-vehicle space could foster user trust and understanding of technical capabilities:
\begin{quote}
    \textit{The driving social presence can also try to assure the user that it can handle multiple tasks safely; it is possible that more open communication about the car’s “cognitive” effort would have helped participants to better understand its technical abilities.} \cite[p. 642]{luria_re-embodiment_2019}
\end{quote} 

Researchers engaging in comprehensive futuring often delved into the societal implications of the various technologies, posing questions regarding how they might reshape societal norms and interactions. 
Finally, some studies in the comprehensive-futuring category elicited reflections on the values inscribed in the design of technologies (e.g., \citeyear{wong_eliciting_2017}'s from \citeauthor{wong_eliciting_2017} \cite{wong_eliciting_2017}, work by \citeauthor{sondergaard_intimate_2018} from \citeyear{sondergaard_intimate_2018} \cite{sondergaard_intimate_2018}, \citeyear{ballard2019judgment}'s paper by \citeauthor{ballard2019judgment} \cite{ballard_judgment_2019}, and \citeyear{suresh_beyond_2021}'s by \citeauthor{suresh_beyond_2021} \cite{suresh_beyond_2021}) rather than solely improving the technological aspects of design. For instance, \citeauthor{wong_eliciting_2017} approached design artifacts not as end products but as tools to provoke discussion about multiple potential futures and prompt professionals to reflect on underlying social values during product development \cite{wong_eliciting_2017}. With design fiction at their disposal for eliciting values, they shifted the focus from simply refining technologies to understanding the societal context in which these operate and incorporating ethics deliberations into design processes. 

\subsection{Characteristics of Comprehensive Futuring in the HCI Field (RQ2)}
    
To address RQ2, we purposefully performed full-text qualitative analysis of all papers in the ``comprehensive futuring'' category. Accordingly, the findings presented in this subsection are derived from the third stage of the paper-screening process (shown at the bottom in Figure~\ref{fig:prisma}). 
Through the abductive process described in Subsection~\ref{sec:qual-analysis}, we developed and employed the framework presented in Table~\ref{tab:analytical-framework}, which guided the team in mapping common ways of envisioning the future onto the future-orientation of the sample. Informed by the SPIN framework, we identified four main properties of the comprehensive futuring, denoted as ``exploration of uncertainty,'' ``prevailing short-term time horizons,'' ``focus on human experience,'' and ``contestation of dominant narratives.'' 

\subsubsection{Exploration of uncertainty} 
Whenever futures are discussed, they bring uncertainties with them, with our study being no exception. Many papers in our corpus display an explorative approach to uncertainties. 


Exploration of uncertainty displays close links to two categories in the SPIN framework -- epistemic stance and contingency perceptions.
For example, papers adopting an explorative epistemic stance \cite[][ etc.]{alves-oliveira_collection_2021, herdel_above_2022, homewood_tracing_2021} and normative ones \cite[such as ][]{smith_designing_2017, sondergaard_fabulation_2023, prost_awareness_2015} demonstrate heightened sensitivity to the uncertainties bundled with interactive technologies and with their relations to human and other environments. Those papers also manifest open approaches to contingencies by discussing such unpredictable aspects of technology as the ``disconnect'' between urban informatics and nature \cite{smith_designing_2017} or the host of roles that robots could take in society \cite{alves-oliveira_collection_2021}.

In an effort to explore uncertainties via other means than forecasting of technological trends, explorative researchers have engaged in speculation and fabulation to envision alternative futures. Their papers highlight potential risks \cite{wong_eliciting_2017,tseng_dark_2022, eghtebas_co-speculating_2023}, illuminate unexpected outcomes \cite{pater_no_2022, park_social_2022}, and inquire into desirable and utopian futures \cite{chopra_negotiating_2022, sondergaard_fabulation_2023, bardzell_utopias_2018}:

\begin{quote}
  \textit{Presented as a U.S. university-based fictional memo describing a post-hoc IRB} [institutional review board] \textit{review of a research study about social media and public health, this design fiction draws inspiration from current debates and uncertainties in the HCI and social computing communities around issues such as the use of public data, privacy, open science, and unintended consequences, in order to highlight the limitations of regulatory bodies as arbiters of ethics and the importance of forward-thinking ethical considerations from researchers and research communities. Though the following illustrative example is fictional, it is inspired by real examples as well as current ethical debates within the HCI community.} \cite[p. 2]{pater_no_2022} 
\end{quote}

\begin{quote}
    \textit{We also agreed that a key alignment was a desire for our images to be utopian: we intended for them to be interpreted as visual stories for positive change and to counter dystopian narratives that often aim to critique societal values by portraying bleak future scenarios. Striving for utopianism was a direct response to the challenges we had faced in doing critical work with bodily fluids, including stigma, ethics, emotion work, and institutional resistance; from which we needed a space of joyful allyship and optimism.} \cite[p. 1698]{sondergaard_fabulation_2023}
\end{quote}

A more speculative, explorative approach to uncertainty reflects deeper engagement with the complexities of future technologies, acknowledging both their transformative potential and their unpredictability.
Often we could identify an overarching narrative of existential unease. This narrative raises concerns about the implications of emerging technologies, bringing to the fore such issues as the ethics-related tensions of responsible AI \cite{rakova2021responsible}, the increasingly indistinct boundaries between humans and machines \cite{seering_beyond_2019, brand_design_2021}, and societal disruptions (such as mass unemployment) wrought by automation \cite{oh_us_2017}.


Reflections on the ambiguity surrounding human--machine interactions spark existential unease, as illustrated in this extract:

\begin{quote}
    \textit{What is striking, however, is that even quite positive participants seem uncertain about the relationship that will emerge from interacting with an sVA.} [social voice assistant] \textit{Is Kiro the end of "awkward silence" or will it just create experiences similar to the human-human conversation? Will Kiro use technology the same way people do, although it is a technology itself? While the notion to become social with a machine appeared interesting and stimulating, participants seemed slightly puzzled about whether the fact that the social partner remains a technology will negatively or positively impact their experiences.} \cite[p. 13]{ringfort-felner_kiro_2022}
\end{quote}

Similarly, depictions of tension between perceptions of AI as a potential danger and as a helpful tool underscore the uncertainty surrounding its role in societies and a need for control over its use:
\begin{quote}
    \textit{Through the interviews, we identified that the participants had preconceptions and fixed ideas about AI: (a) AI could be a source of potential danger, and (b) AI agents should help humans. Although these two stereotypes seem to be contradictory, one seeing AI as a potential danger and the other seeing it as a tool, they are connected in terms of control over the technology. The idea that AI could be dangerous to humans can be extended to the idea that it should be controlled so that it can play a beneficial and helpful role for us.} \cite[p. 2530]{oh_us_2017}
\end{quote} 


\subsubsection{Prevailing short time horizons}    
As introduced above, the pattern of taking technology as grounding for envisioning futures is visible in the comprehensive-futuring sample too. A techno-centric orientation directly shaped the time frames of the envisioned futures, binding them to the technological possibilities of the present.

 We found that techno-centrism cut short attention to long-term futures, such as those spanning 50 or 100 years, yet precisely these extended time frames prove critical for understanding how interactions evolve over time \cite{nathan_envisioning_2008}, how technology adoption unfolds \cite{lindley_implications_2017}, and ways to design for a dynamic and unpredictable future \cite{kozubaev_expanding_2020}. 

Even in studies employing design fiction and speculative design that probes alternative futures \cite{sondergaard_intimate_2018,khan_speculative_2021, ballard2019judgment}, there was often an implicit assumption of linear progression from existing conditions, with speculative scenarios plainly extended from current technologies and societal behavior. Only a few papers \cite{nathan_envisioning_2008, tseng_dark_2022, sanders_designing_2014, speicher_what_2019, chopra_negotiating_2022, carroll_wild_2013} detail any long-term considerations; far less rare are vague references to ``the long term,'' and concrete time frames, such as five years or a decade, seldom feature.

This limitation is especially notable in contexts requiring longer time horizons to address systemic challenges.
For example, \citeauthor{chopra_negotiating_2022} highlighted the importance of designing for extended time\-scales to address pressing urban challenges:

\begin{quote}
    \textit{While struggling to balance between boundless speculation and the uncompromising realities of the situated everyday. Speculative design and associated approaches for instance, have been drawn on to highlight damaging anthropocentric consequences in the near future [\ldots]. Here, designing for longer timescales is becoming particularly prescient for many urban communities due to the scale of the challenges and ever increasing threats presented by governance, degrading environments, and growing urban populations [\ldots].} \cite[p. 2]{chopra_negotiating_2022}
\end{quote}
Similarly, \citeauthor{nathan_envisioning_2008} underscored how systemic interactions and the impacts of technologies often take years to become evident, highlighting longer timeframes:
\begin{quote}
    \textit{Yet most successfully deployed technologies remain in use in society far longer, on the order of 3, 5, or 10[-]plus years. Moreover, systemic interactions emerge over time. Thus, we are more likely to notice these interactions 5 years rather than 5 months out.} \cite[p. 3]{nathan_envisioning_2008}
\end{quote}

\noindent Temporal delineations of the future need not focus exclusively on moving forward in time, though \cite{kozubaev_expanding_2020}. Counterfactuals \cite{forlano_speculative_2023}, fabulation \cite{sondergaard_fabulation_2023}, and relationships with more-than-human entities \cite{liu_design_2018}  merge temporalities, present alternative futures, and bring out uncertainties through such mechanisms as alternative pasts, parallel timelines, and data obtained from living organisms. 

We appeared to detect the beginnings of a more sensitive approach to uncertainties, nuanced beyond risk factors \cite{kozubaev_expanding_2020, forlano_speculative_2023, benjamin_machine_2021, bodker_participatory_2018}. 
This sensitivity was exhibited by approaches that frame temporality as flexible and contestable, as \citeauthor{kozubaev_expanding_2020} have suggested:

\begin{quote}
    \textit{For HCI design to be reflective about temporality, we suggest framing temporality as malleable and contestable, thereby opening new possibilities and ways of speculating about the future. First, HCI designers can explore alternative and novel notions of temporality and make them more visible and interactive. For example, Odom et al.’s work on slow design illustrates how HCI design can support reflections and subjective experiences of time such as anticipation, memory and re-visiting the past. Soro et al. propose an alternative take on the futures cone by flipping its orientation, much like the Aymara, to designing for the past.} \cite[pp. 5-6]{kozubaev_expanding_2020}
\end{quote}
\citeauthor{benjamin_machine_2021} extended this perspective by examining how addressing uncertainty through speculative design can challenge normative assumptions about human--technology relations:
\begin{quote}
\textit{Designing for thingly uncertainty with futures creep and pattern leakage can shed light on how human subjectivities become entangled with ML} [machine-learning]\textit{-driven artefacts. This is not only a symbolic or aesthetic exercise, but rather a potentially powerful way of investigating how standard thinking on human--ML relations rely on normative assumptions (e.g., anthropocentric, capitalist, hetero-normative) about the technological as much as the human side of those relations.} \cite[p. 11]{benjamin_machine_2021} 
\end{quote}

\subsubsection{Focus on human experience}
We also identified a characteristic whereby HCI seems to fill a unique role that sets it apart especially from the traditions of futures studies: expression of an interest in individuals and their experiences. This focus seems absent from STEEPLE and the other futuring typologies reviewed in the course of developing the SPIN framework.
Hence, upon repeatedly encountering individual- and experience-oriented papers that resisted ready mapping to existing typologies, we chose to introduce individual-level experience as a key dimension within SPIN's systemic-integration category. 

The introduction of individual experience into SPIN proved immediately useful. In the comprehensive futuring papers, we observed a related implicit precautionary narrative framing emerging technologies and their unexpected consequences as threats to humanity. This narrative often reflected an anthropocentric bias in how future scenarios are envisioned.
These papers often explored the ethical, psychological, and social impacts of technologies in human life, digging into how they might impinge on human actions by driving behavior change \cite{brand_design_2021, sauppe_social_2015}, disrupting social norms \cite{blackwell_harassment_2019}, perpetuating inequality \cite{eghtebas_co-speculating_2023}, or even challenging our conceptions of privacy and autonomy \cite{jakobi2019privacy, pierce_smart_2019}. 

The focus on human-experience future scenarios is illustrated well by \citeauthor{brand_design_2021}'s seven design proposals for introspective AI. 
\begin{quote}
    \textit{Deep Talk Report preserves and enhances records of deep social exchanges as interactive resources. This approach suggests an opportunity for designers to generate more interpersonally-oriented Introspective AI applications that engage directly with the social relations that shape a person’s current and future ideal self. Nevertheless, there is a need for future work to explore the extent to which divorcing these exchanges from their original context and reducing them to interconnected bits might alter their perceived value and lead to added social expectations.} \cite[p. 1614]{brand_design_2021}
\end{quote}
\citeauthor{eghtebas_co-speculating_2023} spotlighted human experience similarly, by underscoring the risks of socio-economic imbalances (re)produced in the rollout of ubiquitous technologies, and called for interdisciplinary approaches so as to mitigate the attendant challenges:
\begin{quote}
    \textit{Technology cannot fix all of society’s problems, but it can inherit them. The discussion motivated from our analysis highlights how multi-disciplinary perspectives, including technological views, but also political and social sciences will be required to be able to fully mitigate what may become of our envisioned dark scenarios. As alluded to in several of our scenarios, UAR} [ubiquitous(ly) Augmented Reality] \textit{is inherently asymmetric: unequal access to costly hardware, or the lack of accommodating of peoples various roles and abilities as UAR continues to develop, can cause societal imbalances.} \cite[p. 2404]{eghtebas_co-speculating_2023}
\end{quote}

\noindent On the other hand, HCI's disciplinary strength in examining the everyday interactions with technology also presents opportunities to investigate futures at the micro level, which is often lacking in more systemic analyses from futures studies \cite{candy_futures_2010}.
A micro-level focus can help elucidate how individuals and their communities experience, adapt to, and resist technological changes in day-to-day life.

By granting prioritizing micro-scale interactions between people and technology, individual-experience-oriented papers with comprehensive futuring have propagated nuanced insight connected with broader impacts of technologies' integration and adoption \cite{carroll_wild_2013, pierce_smart_2019, lindley_implications_2017, elsden_speculative_2017}. 
A speculative approach to design has afforded deeper probing of the potential futures of marginal communities, highlighting matters regularly overlooked in mainstream technology design and addressing inequality of several sorts \cite{harrington2019deconstructing, keyes_reimagining_2020, harrington_eliciting_2021, khan_speculative_2021}, culture-specific elements \cite{bray_speculative_2021}, and unique needs of underrepresented groups \cite{morrissey_value_2017, bennett2019biographical, bardzell2010feminist, haimson_designing_2020, dillahunt_eliciting_2023}. 

For instance, \citeauthor{harrington_eliciting_2021} illustrate how participants struggled to envision technology-based futures free from systemic social issues, such as racism:
\begin{quote}
    \textit{Our results uncovered key constructs that are difficult or perhaps impossible to separate from design. Common among our findings was that students have a difficult time imagining a future without the existing social issues they face today. Among many of the utopian concepts were still elements of identified dystopian challenges that seemingly could not be detached from concept generation. In all cases, students’ technology-based futures encapsulated some form of racism and it was difficult for them to imagine technology that exists in a world without it. Previous literature suggests that human imagination is bounded and people might have difficulties imagining the future as situations become more distant in likelihood, perspective, time, and place. However, our analysis provides insight into the unique difficulty of envisioning the future that is confounded by race and social class, which was present among our participants} \cite[p. 10]{harrington_eliciting_2021}
\end{quote}

\subsubsection{Contestation of dominant narratives}
When we examined the comprehensive-futuring papers' narratives, several publications stood out for their critical examination of how design decisions privilege certain individuals or groups over others, highlighting issues of power dynamics embedded in the design process \cite[e.g., ][]{pendse_treatment_2022, keyes_reimagining_2020, winchester_realizing_2010, breuer_how_2023, suresh_beyond_2021, adamu_no_2023, avle_designing_2016, dell_ins_2016}.
These works reveal varying degrees of bias -- whether stemming from the designer or the focus on expert knowledge --
which can culminate in missed opportunities for overlooked communities. For example, \citeauthor{breuer_how_2023} adopts a science \& technology studies lens to analyse the ethical, social, and cultural assumptions that underlie the design of healthcare robotics \cite{breuer_how_2023}. Similarly, \citeauthor{suresh_beyond_2021} illustrate how machine learning designers' recurrent focus on expert knowledge, which leads to assigning all other potential users the one-dimensional label ``non-experts'' and virtually guarantees that opportunities for inventing more equitable, adaptable, and context-sensitive machine learning systems remain overlooked \cite{suresh_beyond_2021}.
Other work stressing designers' accountability for their position of power has been done under the ``human--computer interaction for development,'' or HCI4D, umbrella \cite{avle_designing_2016, dell_ins_2016, saha_towards_2022, adamu_no_2023}.
Despite these works and the expansion of interactive systems' design beyond ``the West'', Western standards and narratives continue to dominate. 

\begin{quote}
\textit{The concept of interest convergence, which stems from critical race theory, holds that those in power tend to support goals that serve their own interests. In other words, without actively involving stakeholders whose interests are in opposition to existing power structures, and considering their input crucial, resultant interpretability systems will fit the standards and needs of those in power -- for example, executives with a vested interest in maintaining the status quo, or engineers and researchers who might communicate about model decisions in a way that is not understandable to people without formal ML} [machine learning] \textit{knowledge. Involving stakeholders with different interests first requires reflexivity, or explicitly acknowledging what our own backgrounds and interests are.} \cite[p. 12]{suresh_beyond_2021}
\end{quote}

\begin{quote}
    \textit{Several participants called for a stronger focus on designing for non-traditional computing environments. For example, P5 said, “Why would you have an office, QWERTY keyboard, desktop metaphor, textual interface for people who don’t think about things in that way? The traditional appliances and systems embed middle-aged white guys from the Pacific North-west. They are the ones in the corner office whose language is premised in QWERTY. Not only their spoken language, but they’re also print literate. The appliance is really focused on that context and no wonder it can be alienating to different contexts. HCI4D is about breaking out of these rich, white, male, US systems into all kinds of other systems. What would a tropical computing environment look like?”} \cite[p. 2228]{dell_ins_2016}
\end{quote}


\noindent In addition to addressing the social and ethics factors obviously raised by HCI, works in the sub-fields of sustainable HCI and more-than-human design \cite{homewood_tracing_2021} underline the webs of relations that link technological developments with ecological systems \cite{nathan_sustainably_2009}. Some researchers dealing with these areas explore connections between cultures and perspectives \cite{lindtner2016reconstituting,bennett_promise_2019, brynjarsdottir_sustainably_2012, chopra_negotiating_2022}, alongside how these intersect with environmental ecosystems \cite{hansson2021decade, smith_designing_2017}, advocating accordingly for sustainable transformation and inclusive design practices that reflect more-than-human concerns \cite{liu_design_2018, frauenberger_entanglement_2020, tsaknaki_fabulating_2022, sondergaard_feminist_2023}. 

One case in point comes from conceptualizing tools that support diverse human--fungus interactions: \citeauthor{liu_design_2018} reimagines design as a means to enable collaborative survival, with the nature of the constituent relations left open-ended and adaptable.
\begin{quote}
    \textit{What the variety of tools makes evident about collaborative survival is that it is well suited to attend}[ing] \textit{to the multiplicity of human--fungi relationships. The metaphor allowed us to honor fungi in its multiple forms and expressions--a ubiquitous underground network, a barometer for ecosystem health, a delicacy to eat, or a specimen to identify (to name a few). There is not one true or correct life for a human to attend to and thus, there are multiple ways of becoming entangled with fungi in both its physical and digital manifestations. For instance, Liu did not design the tools to enact particular narratives of protecting or conserving fungi. Instead, the concept suggested that we design only to make a particular relationship with fungi possible, and it is up to each human as to what that relationship can entail and what arrangements will be mutually beneficial.} \cite[p. 9]{liu_design_2018}
\end{quote}
Taking a distinct stance, \citeauthor{tsaknaki_fabulating_2022} explored what more-than-human design can bring to the table to reframe collab-relationships across human and non-human entities: 
\begin{quote}
    \textit{Taken together, these concepts explore more-than-human agencies living, knowing, and collaborating with humans. Acknowledging these strands of thought do not all knit together perfectly, this theme invites expansive ideation on how relations and collaborations between human and non-human bodies. Instead of foregrounding the authority of exclusively human bodies and biodata, this theme seeks to also account for other companion species [\ldots] in or outside our bodies, ranging from microorganisms to animals, to materials as vibrant bodies [\ldots], to technologies.} \cite[p. 1181]{tsaknaki_fabulating_2022}
\end{quote}

While the foregoing examples demonstrate reaction against entrenched phenomena such as techno-centrism, they also point to our field's rich legacy and its focus on human factors in computing. 
They illustrate a shift toward breaking the hold of Western narratives \cite{alvaradogarcia2021decolonial, adamu_no_2023} and anthropocentric discourse \cite{yoo2023more, sondergaard_feminist_2023}. 
By incorporating diverse perspectives and acknowledging the limitations of human-centric design, HCI researchers of today are embracing sustainable, equitable futures that account for multi-system knots' challenges by factoring in communities, cultures, and non-human organisms. 


\subsection{Summary of the Findings}

Our findings present a heterogeneous landscape of futuring in HCI. 
While there is a growing rate of future-oriented papers, Figure~\ref{fig:heatmap} presented a dominance of fleeting futures. 
This dominace likely stems from the aforementioned prevalence of a technology-use focus as the primary driver for envisioning HCI futures, a trait visible even within the comprehensive-futuring category. However, there are encouraging signs from the latter sample, which exhibits greater depth and variation. It expresses robust engagement along the dimensions of all four SPIN categories. 
Circumscribing those categories has helped us illuminate precisely how HCI papers conceptualize, articulate, and impart futures -- and how they fall short. It aids especially in crystallizing what is distinct in the fleeting vs. comprehensive approaches.

In terms of epistemic stance (S), comprehensive futures demonstrate a more diverse engagement, incorporating explorative and normative stances that seek to imagine alternative trajectories or advocate for desirable futures aligned with societal or community values, contrasting with the narrower, predictive focus of fleeting futures. 

Concerning contingency perception (P), comprehensive futures embrace an open approach to uncertainty although more work is needed in leveraging it as a resource to explore non-linear and long-term temporalities. Clearly, forward-looking HCI research could delve far more deeply into the complex relations and unpredictabilities of future scenarios.

Comprehensive futures recognize the importance of systemic integration (I). Their foci acknowledge community experiences, cultural nuances, and more-than-human positionings. However, scholars' treatment of interrelation in this regard could advance further, in that only a single contingency element receives focus at a time. There are significant opportunities for exploring several contingencies in combination.

Finally, narratives (N) play a significant role in shaping HCI futures. However, we identified a weak spot in the way in which narratives often prioritize Western, anthropocentric perspectives, perpetuating systemic inequities and marginalizing more-than-human and Global South viewpoints. Addressing this gap requires an intentional shift in HCI research to challenge normative assumptions, incorporate underrepresented perspectives, and foster decolonial and inclusive approaches to futuring.


\section{Discussion} \label{sec:discussion}


Both our initial survey of related research (see Section~\ref{sec:related-research}) and our findings offer evidence of how HCI has evolved in its futuring's orientation over the years. Overall, the field has progressed from narrow visions focused on technological development toward more holistic and exploratory efforts to account for voices less often heard 
(e.g., of minorities and non-human entities), 
yet the landscape remains uneven. Some of the remaining gaps we found were highlighted already by \citeauthor{bell_yesterdays_2007} in \citeyear{bell_yesterdays_2007}. 
They attributed the field's techno-centric visions to the enduring influence of \citeauthor{weiser1991computer}'s vision of ubiquitous computing \cite{weiser1991computer}, which not only shoehorned HCI work toward developer time horizons narrowed to daily living in near-term futures but also expressed ideals specific to a North American upper-middle-class future \cite{bell_yesterdays_2007}. While attention from authors such as \citeauthor{bell_yesterdays_2007} may have helped spur bridge-building in HCI's future-orientation, some relatively recent work discussed above \cite[e.g., ][]{avle_designing_2016, adamu_no_2023, dillahunt_eliciting_2023} has continued to spotlight the associated gap. Indeed, it persists in even the most current HCI research.

In this section, we zoom in on two overarching gaps -- expanding the futuring's foundations beyond technology, and scaling HCI's insights through multilateral engagements -- from which we can gain a deeper understanding of the reasons why the gaps highlighted by \citeauthor{bell_yesterdays_2007} still persist today despite HCI's expanded future-orientation. We also propose five actionable opportunities to respond to the challenges. They adopt a normative yet adaptable approach that broadens HCI’s future-orientation. Each opportunity discussed below aims to enhance HCI’s contribution to futuring by promoting more inclusive, interdisciplinary practices.

\subsection{Expanding the Basis of Futuring Beyond Technology}
The undercurrent of techno-centrism, visible in HCI's fleeting and comprehensive futuring alike, has been pointed out again and again over the years. \citeauthor{bell_yesterdays_2007} were not alone. Several publications featured in our literature review point out this latent tendency, all the way from 2008 to 2023 (e.g., \citeauthor{reeves_envisioning_2012}'s \cite{reeves_envisioning_2012} from \citeyear{reeves_envisioning_2012}, \citeauthor{pargmanSustainabilityImaginedFuture2017}'s \cite{pargmanSustainabilityImaginedFuture2017} from \citeyear{pargmanSustainabilityImaginedFuture2017}, and publications by \citeauthor{soden_time_2021} \cite{soden_time_2021} and \citeauthor{forlano_speculative_2023} \cite{forlano_speculative_2023} in the 2021 and 2023 respectively), 
yet it still prevails. In response, some researchers have proposed drawing from disciplines experienced with systemic challenges, such as futures studies \cite{nathan_envisioning_2008, mankoffLookingYesterdayTomorrow2013a, salovaaraEvaluationPrototypesProblem2017, pargmanSustainabilityImaginedFuture2017, lightCollaborativeSpeculationAnticipation2021a, epp2022reinventing, moesgenDesigningUncertainFutures2023}; however, their contributions do not eliminate the risk of endorsing technology-oriented solutions that \emph{appear} transformative yet only provide imagined convenience. If we fail to address the underlying pressing needs by looking past technology \cite{nathan_envisioning_2008, tomlinson_collapse_2013, brynjarsdottir_sustainably_2012, harding_hci_2015}, ongoing reliance on technology-driven futures could even create new forms of dependency, beyond present disparities. We might exacerbate the very problems we set out to solve \cite{bell_yesterdays_2007, kinsleyFuturesMakingPractices2012}. 

\begin{description}
\item[Opportunity 1:] Turn to non-technological domains for drivers of HCI futuring. 
\end{description}

Two papers in our comprehensive-futuring sample 
identify possible deep roots to the challenge of technology-driven futuring. 
Firstly, \citeauthor{reeves_envisioning_2012} argued that a certain techno-determinism distinctive of HCI envisioning promotes a sense that, since technology advances linearly, we can predict societal implications with ease. 
The crutch of regarding technological progress as linear and independent of other catalysts of social change is attractive in that it frees HCI researchers from addressing the complexity of wider implications \cite{reeves_envisioning_2012}.

The other paper, by \citeauthor{do_thats_2023}, points out that computer scientists' research process does not naturally consider unintended consequences, let alone the web of factors created by them \cite{do_thats_2023}. 
\citeauthor{do_thats_2023} cited two main reasons for this blind spot: 
Computer scientists are seldom trained in means and guidance for envisioning such consequences. 
While our survey of prior research (see Section~\ref{sec:related-research}) found that HCI is not lacking in tools for envisioning possible futures, popular ones such as envisioning cards \cite{friedman_envisioning_2012} and design fiction \cite{bleecker_design_2009, sterling_design_2009} tend to get regarded as the province of designers and practitioners. This preconceptions has held back their adoption in research settings \cite{do_thats_2023, forlano_speculative_2023}.

Moreover, feedback loops between research priorities and institutional funding, publication timelines, etc. sustain techno-centric norms in HCI -- 
pressure for swift publishing leaves researchers little or no time to think about unintended consequences \cite{do_thats_2023}; 
meanwhile, incentives for financeable innovation and feasible implementation push the field toward incremental advancements, thereby suppressing its capacity to explore many, far futures \cite{reeves_envisioning_2012, do_thats_2023}.

The dominance of normative narratives centered on techno-driven futures highlights a critical tension in the field. 
One one side there are active groups of scholars forwarding systemic \cite{nathan_envisioning_2008, reeves_envisioning_2012, mankoffLookingYesterdayTomorrow2013a, salovaaraEvaluationPrototypesProblem2017, kozubaev_expanding_2020}, speculative \cite{blythe_research_2014, lindley_pushing_2016, wong2018speculative, elsden_speculative_2017, fox_vivewell_2019, chopra_negotiating_2022, bray_speculative_2021, forlano_speculative_2023}, pluralistic \cite{dell_ins_2016, avle_designing_2016, harrington_eliciting_2021, adamu_no_2023, kozubaev_expanding_2020, bardzell_utopias_2018, saha_towards_2022} and more-than-human \cite{liu_design_2018, key_feminist_2022, sondergaard_feminist_2023, sondergaard_fabulation_2023} approaches just to list some examples.
These groups are pulling against inertia from ``publish or perish'' pressure, discomfort with unfamiliar methods \cite{do_thats_2023}, and other impediments to applied envisioning in computer-science research. 

While the comprehensive-futuring work sheds considerable light on HCI's path, fleeting futuring too plays a critical role in shaping our field's future-orientation discourse. The narrowly scoped futures often operate as low-stakes tools for experimentation, whereby researchers explore emerging ideas or perform quick testing along speculative lines. Additionally, busy scholars can reap practical benefits via a focus on rapid prototyping and immediate utility, especially in industry-driven projects. 
For broader consideration of social, cultural, and ethics dimensions within this ``fleeting'' work, we refer to the suggestions of \citeauthor{lindley_implications_2017} and \citeauthor{kozubaev_expanding_2020}. 
Researchers who give futuring a fleeting glimpse can take advantage of opportunities to evaluate technological innovation, prototypes, etc. amid the everyday and mundane. Interventions that entail evaluating a particular future in day-to-day settings support profound learning about how people would interact with particular technologies in real life \cite{kozubaev_expanding_2020}. Researchers applying such a lens also have a tool for examining these technologies' long-term adoption in connection with the standard evaluation and for reflecting on the implications of that adoption \cite{lindley_implications_2017}. 

For comprehensive futuring, in turn, we argue that future-orien\-tation could be further enhanced by incorporating generative uses of uncertainty. Generative application readily permits explorations that attend to future challenges yet still form new possibilities in the here and now \cite{akamaDesignEthnographyFutures2015, epp_uncertainties_2024, sodenModesUncertaintyHCI2022}. 
For example, uncertainty can invite alternative futures that are not bound by current technological trends or assumptions \cite{tomlinson_collapse_2013}, but also support HCI researchers reflect on what they know or do not know, and reveal factors that have been ignored \cite{sardarThreeTomorrowsPostnormal2016}. The increased attention to design fiction and speculation has paved the path for HCI to consider the future as dynamic and open-ended. Adopting this ontology of the future \cite{poli_anticipation_2019} presents HCI with the opportunity to address complexity beyond technology while also the responsibility to consider and care for non-predominant stakeholders and ways of knowing so that futures being envisioned are meaningful to the stakeholders they are representing \cite{adam_futures_2011, kozubaev_expanding_2020}. 


\begin{description}
\item[Opportunity 2:] Engage with generative modes of uncertainty to tap into unknown possibilities.
\end{description}


\subsection{Scaling Up HCI's Strengths Beyond Its Field}
In our pursuit of a literature review that captures how the HCI field applies its future-orientation at \emph{multiple} levels, 
we have considered the wider expanse via what futures studies scholarship could bring to the table. While we have identified papers that call for integration of futures studies concepts and methods into HCI, 
an opportunity exists also for boundary-crossing work in the other direction: reaching out to contribute to the outputs from futures studies.
We see three traits in particular through which HCI could add value to that neighboring field.

Firstly, HCI research can contribute to wider future-oriented discourses by bringing the angle of individuals and small groups to such arenas as futures studies and policymaking. Development of the SPIN framework (see Subsection~\ref{sec:qual-analysis}) delineated this unique feature, which is demonstrated well by studies of individuals' experiences with technology \cite{pierce_beyond_2012, brynjarsdottir_sustainably_2012,su_dolls_2019}. Futures studies has only recently cultivated an interest in the participatory and experiential so could learn much from this standpoint \cite{vorosIntegralFuturesApproach2008, candy_futures_2010, garcia2021designing}. Scholars of HCI could respond to the impetus for expansion of futures studies at such borders.

Secondly, HCI exercises the rare capability of studying futures ``in action'' by staging prototype-based user studies \cite{salovaaraEvaluationPrototypesProblem2017,elsden_speculative_2017,simeone2022immersive}. This methodological asset holds promise likewise for studies anchored not in HCI research agendas but in goals of other fields.

HCI's future-oriented work can build on a designerly approach that experiments with possibilities and ponders the outcomes. This third opportunity, through which less typically intervention-focused disciplines gain flexibility, is most clearly evident in ``research through design'' \cite[e.g., ][]{zimmerman_analysis_2010} but seized most directly via \citeauthor{kozubaev_expanding_2020}'s positing of ``five modes of reflection'' for HCI, where ``designerly formgiving,'' ``real-world engagement,'' and ``knowledge production through design'' articulate the fundaments poignantly \cite{kozubaev_expanding_2020}. 
Fruitful understandings produced thereby could generate excitement far beyond HCI. 

Our field, as an inherently interdisciplinary and participatory one, occupies a unique position, from which it can facilitate multilateral discussions. Merging perspectives from design, psychology, engineering, ecology, and the social sciences, it is adept at coming to integrative understanding of a myriad viewpoints. 
A more reflective \cite{kozubaev_expanding_2020}, responsible \cite{adam_futures_2011} stance to futuring would enable the HCI field to recast its role in dialogues about where societies, Earth, and other environments are headed. Ultimately, the stances we take decisively influence how futures get envisioned, developed, and implemented, across contextual boundaries.

\begin{description}
\item[Opportunity 3:] Embrace methods and tools for experiencing futures, and demonstrate the strengths they offer other fields.
\end{description}

\noindent Embracing more-than-human approaches, of which we see glimmers in the HCI field, helps overcome the anthropocentric bias that characterizes futuring research generally, whether in academic or policy domains \cite{adam_futures_2011, sardar_postnormal_2015}. Though knowing what technologies do in human life is undeniably important, attending exclusively to this can obscure the interdependencies linking humans and other elements of our planetary ecosystem \cite{yoo2023more, forlanoPosthumanismDesign2017, wakkary2021things}. 
To understand the vital part these dependency relations play in our world’s complex systems, especially as technological interventions casts an ever larger shadow, HCI futuring need look no further than our own field to witness the power of collaboration involving more-than-human organisms and of the relations' power in general \cite{liu_design_2018, homewood_tracing_2021, sondergaard_feminist_2023, sondergaard_fabulation_2023}.

\begin{description}
\item[Opportunity 4:] Cultivate reflexivity, and incorporate perspectives of \textit{otherness}.
\end{description}

\noindent Finally, to amplify the impact of its futures work, HCI can reorient itself by subverting dominant narratives \cite{light_ecologies_2022}. To reach this goal, the field can connect reflexively with its political position, thus demonstrating a willingness to usher in collective change rather than solely support adoption by individuals \cite{dourish_hci_2010, ashby_fourth-wave_2019}. By transforming into a proactive force, HCI can remain open to other perspectives, grapple with the mutability of the future, and maintain capacity to act on it in the present \cite{kozubaev_expanding_2020, millerSensingMakingsenseFutures2018}. To disrupt entrenched power dynamics, HCI researchers can cultivate sufficient humility to listen to the needs of \textit{others}, commit to long-term collaboration able to unveil various desirable futures, and steer actions toward pursuing those futures that emerge as valuable \cite{carroll_wild_2013,chopra_negotiating_2022, clarke_situated_2016, dell_ins_2016, avle_designing_2016, adamu_no_2023}. 


\begin{description}
\item[Opportunity 5:] Engage in activism and political action.
\end{description}


\section{Conclusion}
As we had hoped, our literature-based review shed light on the state of HCI research's future-orientation. We proved able both to examine the extent of futuring in HCI and to identify telling characteristics of that HCI research directed toward more comprehensive futuring activity. Futures studies informed our research design -- from sensitizing the retrieval stage's search terms, through aiding with inclusion/exclusion, to supporting the qualitative analysis (the final stage) by helping populate our four-category SPIN framework for examining how futuring is approached in HCI -- yet the work sharply illuminated our own discipline specifically. Together, the most future-oriented papers in the field, and the sub-sample of $N=205$ highly cited comprehensive-futures publications retrieved for our literature review gave us the final SPIN framework and valuable findings specific to our field's most influential comprehensively future-oriented papers.

Our findings highlight that futuring is a growing trend in HCI, with comprehensive futuring having shown considerable growth since 2017. 
However, fleeting futuring has remained the dominant category, and HCI futuring can be characterized as mostly techno-centric. While our review identified a very real risk of reproducing tunnel vision and of scenarios that echo notions of linear societal progress, 
analysis of the ``comprehensive futuring'' category uncovered encouraging signs: a significant proportion of the work features active exploration of uncertainty, focus on human experience, and contestation of dominant narratives -- all of which help mitigate such issues. 
Both the gaps and such strides toward overcoming them fueled our reflections as to why techno-centrism and short-term visions of the future still prevail in HCI, and on five opportunities to rectify these problems.


%%
%% The next two lines define the bibliography style to be used, and
%% the bibliography file.

\printbibliography[keyword={references}]

\end{refsection}
%% If your work has an appendix, this is the place to put it.
\newpage
\appendix

\begin{refsection}
    

\section{Inclusion/Exclusion Criteria}\label{sec:inclusion-exclusion}
\subsection{Exclusion Criteria: 
}
\textbf{1. Missing future-oriented terminology}
Papers that do not include any future-related search terms in the relevant sections (all sections for Interactions magazine, and the introduction or discussion/conclusion for non-Interactions magazine's articles) are excluded.

\textbf{2. Future terms not used in a future-oriented context}
If future-related terms appear but refer solely to future work/research without exploring future scenarios, the paper is excluded.

Example:“In the future work, more details will be perfected, and more levels will be developed for further enhancing children’s skills, which will also be corroborated in more rigorous experiments.” [p. 14]\cite{10.1145/3586183.3606755}

\textbf{3. Call for future research without elaboration.}
Papers invoking future terms in a call for new research practices or approaches but failing to elaborate on them are excluded.

Examples:
“Where the sanguine rhetoric of democratizing technology and open data envision transformation through the availability of new tools, it does so with little regard for the human and community costs involved in that transformation. However, a mode of intervention that is based in community practice shifts the power to the community.” \cite[p. 792]{10.1145/2598510.2598563}

“How best to train ML researchers and practitioners to engage in creative speculation or to otherwise anticipate potential consequences of their work is an area where more research is needed.” \cite[p. 23]{10.1145/3555760}

\textbf{4. Existing technology use scenarios without futuring}
Articles proposing use-scenarios for technologies already available at the time of writing and not explicitly referring to the future are excluded.

Example:
“There are also several applications for low-cost audience polling outside of a classroom context. We envision that the technology developed in this paper would apply equally well in these scenarios.” \cite[p. 54]{10.1145/2380116.2380124}

\subsection{Inclusion Criteria:}
\textbf{1. Using future-oriented methods}
Papers proposing or utilizing methods for anticipating or envisioning future scenarios are included, including research through design, design fictions, speculative design, etc.

\textbf{2. Future visions and new artifacts}
Papers presenting new artifacts that convey future visions are included, with future visions encompassing: a) future implications for specific communities or society, b) technological implications, c) considerations on how the envisioned future could be achieved, d) discussions within the STEEPLE framework (Social, Technological, Economic, Environmental, Political, Legal, and Ethical considerations).

\textbf{3. Conceptual future discussions}
Papers discussing or critiquing future scenarios, imaginaries, or visions are included, especially if they problematize common future predictions.

\textbf{3. References to future thinking outside of the scope} 
Future search terms appear outside the analysis scope (introduction, discussion, and conclusion sections) but more detailed futuring is discussed the referenced section.

Example:
“The design work we have presented here follows the traditions of participatory design with some affinity toward speculative design.” \cite[p. 2304]{asad_creating_2017}

\subsection{Fleeting Futuring}
Papers that present superficial descriptions of future scenarios or common imaginaries with little elaboration are considered "fleeting futuring." Such mentions typically only include one sentence or a brief reference to a future vision.


\printbibliography[keyword={references}, title={References for Inclusion/Exclusion Criteria}]
\end{refsection}

\begin{refsection}
\section{Comprehensive Futuring Articles}
The following list enumerates the articles that manifest comprehensive futuring. 

\nocite{adamu2023, adibSmartHomesThat2015, alanTariffAgentInteracting2016, alkhatibExaminingCrowdWork2017, alvarezdelavegaUnderstandingPlatformMediated2023, alves-oliveiraCollectionMetaphorsHumanRobot2021, asadCreatingSociotechnicalAPI2017, asadPrefigurativeDesignMethod2019, avleDesigningHereThere2016, ballardJudgmentCallGame2019a, bardzellCriticalDesignCritical2012a, bardzellReadingCriticalDesigns2014, bardzellUtopiasParticipationFeminism2018, bardzellWhatCriticalCritical2013, baumerEvaluatingDesignFiction2020, benfordPerformanceLedResearchWild2013, benjaminMachineLearningUncertainty2021a, bennettAccessibilityCrowdedSidewalk2021, bennettPromiseEmpathyDesign2019, blackwellHarassmentSocialVirtual2019, blytheResearchDesignFiction2014a, blytheResearchFictionStorytelling2017a, blytheSolutionistStrategiesSeriously2016, bodkerParticipatoryDesignThat2018, boehnerDataDesignCivics2016, bonnailMemoryManipulationsExtended2023, boydAutomatedEmotionRecognition2023, brandDesignInquiryIntrospective2021, braySpeculativeBlacknessConsidering2021a, breuerHowEngineersImaginaries2023, brushHomeAutomationWild2011, brynjarsdottirSustainablyUnpersuadedHow2012, cambreOneVoiceFits2019, carrollWildHomeNeighborhood2013, chenLearningHomeMixedMethods2021, cheokEmpatheticLivingMedia2008, chignellEvolutionHCIHuman2023a, chopraNegotiatingSustainableFutures2022, choTopophiliaPlacemakingBoundary2022, chungIntersectionUsersRoles2021, churchillPsQsDesigningDigital2008, cilaProductsAgentsMetaphors2017, clarkeSituatedEncountersSocially2016, costanzaDesignableVisualMarkers2009, costanzaDoingLaundryAgents2014, crivellaroContestingCityEnacting2015, dellInsOutsHCI2016, desjardinsBespokeBookletsMethod2019, desjardinsLivingPrototypeReconfigured2016, devendorfDontWantWear2016, dhanorkarWhoNeedsKnow2021, dillahuntElicitingAlternativeEconomic2023, disalvoNourishingGroundSustainable2009, dorkInformationFlaneurFresh2011, doThatsImportantHow2023, drugaFamilyThirdSpace2022, eghtebasCoSpeculatingDarkScenarios2023, ehsanExpandingExplainabilitySocial2021c, ellisonFEATURESocialNetworkSites2009, elsdenSpeculativeEnactments2017a, eslamiFirstItThen2016, faasLongitudinalVideoStudy2020, forlanoSpeculativeHistoriesJust2023a, foxVivewellSpeculatingFuture2019, frauenbergerEntanglementHCINext2020a, freemanRediscoveringPhysicalBody2022a, friedmanEnvisioningCardsToolkit2012, gargSocialContextsAgency2022, gaverAnnotatedPortfolios2012, grammenosFEATURETheAmbientMirror2009, greenfieldFEATUREAtEndWorld2009, grinterInsOutsHome2009, grudinChatbotsHumbotsQuest2019, gugenheimerShareVREnablingCoLocated2017, haimsonDesigningTransTechnology2020a, hardingHCICivicEngagement2015, harringtonDeconstructingCommunityBasedCollaborative2019a, harringtonElicitingTechFutures2021b, herdelScopingReviewDomains2022, holzImplantedUserInterfaces2012, homewoodTracingConceptionsBody2021, iivariCriticalDesignResearch2017, impioGiveManFish2010, iraniStoriesWeTell2016, iraniTurkopticonInterruptingWorker2013, ishiiRadicalAtomsTangible2012a, ismailImaginingCaringFutures2022, jakobiItWhatThey2019, jhaverDesigningWordFilter2022, kaneSharedDecisionMaking2013, kawakamiSensingWellbeingWorkplace2023, kawakamiWhyCareWhats2022, kayeMoneyTalksTracking2014, keyesReimaginingWomensHealth2020, keyFeministCareAnthropocene2022, khanSpeculativeDesignEducation2021, kingsleyGiveEverybodyLittle2022, kirshEmbodiedCognitionMagical2013, klopfensteinRiseBotsSurvey2017, kocielnikReflectionCompanionConversational2018,  kozubaevExpandingModesReflection2020, lazarovasquezIntroducingSustainablePrototyping2020, leeWeBuildAIParticipatoryFramework2019, liArtifactPowerLens2023, lightDesignExistentialCrisis2017, lightEcologiesSubversionTroubling2022b, lindleyImplicationsAdoption2017, lindleyPlacingAgeTransitioning2015, lindleyPushingLimitsDesign2016a, lindtnerEmergingSitesHCI2014a, lindtnerReconstitutingUtopianVision2016a, liuDesignCollaborativeSurvival2018a, loglerMetaphorCardsHowGuide2018, luCodingBiasUse2021, luFeelItMy2019, luoEseedShapeChangingInterfaces2020, luriaReEmbodimentCoEmbodimentExploration2019, manciniContravisionExploringUsers2010a, maoHowDataScientistsWork2019, matviienkoBabyYouCan2022, maUncoveringGigWorkerCentered2023, meyersImpoverishedVisionsSustainability2016, michaelisCollaborativeSimplyUncaged2020, morrisseyValueExperienceCentredDesign2017, muellerJoggingQuadcopter2015, muellerUnderstandingDesignIntertwined2023, murray-rustBlockchainUnderstandingBlockchains2022, murrayFEATUREResearchStrategiesFuture2009, nathanEnvisioningSystemicEffects2008a, nathanSUSTAINABLYOURSInformation2009, neustaedterSharingDomesticLife2015, odomFieldworkFutureUser2012, odomPhotoboxDesignSlow2012, odomPlacelessnessSpacelessnessFormlessness2014, odomTechnologyHeirloomsConsiderations2012, ohaganPrivacyEnhancingTechnologyEveryday2022, ohUsVsThem2017, oogjesDesigningOtherHome2018, parkGenerativeAgentsInteractive2023, parkSocialSimulacraCreating2022, paterNoHumansHere2022, pendseTreatmentHealingEnvisioning2022, pierceDesignProposalWireless2016, pierceEnergyMonitorsInteraction2012, pierceSmartHomeSecurity2019a, plodererProcessEngagementEngaging2012, poupyrevProjectJacquardInteractive2016, prostAwarenessEmpowermentUsing2015, raeFrameworkUnderstandingDesigning2015, rakovaWhereResponsibleAI2021, raziSlidingMyDMs2023, reevesEnvisioningUbiquitousComputing2012, ribesRepresentingCommunityKnowing2008, ringfort-felnerKiroDesignFiction2022, rogersNeverTooOld2014, sabieDecadeInternationalMigration2022, saffoRemoteCollaborativeVirtual2021, sahaSustainableICTDBangladesh2022, salehiWeAreDynamo2015, sanchez-cortesMoodVlogMultimodal2015, sandersDesigningCodesigningCollective2014, saraijiMetaArmsBodyRemapping2018, sauppeSocialImpactRobot2015, schulenbergLeveragingAIbasedModeration2023, seeringDyadicInteractionsConsidering2019, seeringReconsideringSelfModerationRole2020, simeoneSubstitutionalRealityUsing2015, smithDesigningCohabitationNaturecultures2017, sodenTimeHistoricismCSCW2021, sondergaardFabulationApproachDesign2023, sondergaardFeministPosthumanistDesign2023, sondergaardIntimateFuturesStaying2018, speicherWhatMixedReality2019, steinhardtAnticipationWorkCultivating2015a, sterlingCOVERSTORYDesignFiction2009, strengersSmartEnergyEveryday2014, suDollsMenAnticipating2019, sureshExpertiseRolesFramework2021, sykownikSomethingPersonalMetaverse2022, tanenbaumDesignFictionalInteractions2014, tanenbaumSteampunkDesignFiction2012, thiemeDesigningHumancenteredAI2023, tomlinsonCollapseInformaticsAugmenting2012a, tomlinsonCollapseInformaticsPractice2013a, tomlinsonWhatIfSustainability2012, tsaknakiFabulatingBiodataFutures2022, tsengDarkSidePerceptual2022, tuliRethinkingMenstrualTrackers2022, vanderbeekenFEATURETakingBroaderView2009, varanasiItCurrentlyHodgepodge2023, vermaRethinkingRoleAI2023, vinesAgeOldProblemExamining2015a, vinesJoyChequesTrust2012, volkelElicitingAnalysingUsers2021, wakkaryMorseThingsDesign2017, wakkarySustainableDesignFiction2013, williamsPerpetualWorkLife2019, winchesterREALizingOurMessy2010, wongElicitingValuesReflections2017a, wongRealFictionalEntanglementsUsing2017b, wongSeeingToolkitHow2023, yildirimHowExperiencedDesigners2022, yooValueSensitiveActionreflection2013, zamfirescu-pereiraHerdingAICats2023, zimmermanAnalysisCritiqueResearch2010a}
\printbibliography[keyword={includes}, heading=none]
\end{refsection}

\begin{refsection}
\section{Fleeting Futuring Articles}
The following list enumerates the articles that manifest fleeting futuring. 

\nocite{abokhodairDissectingSocialBotnet2015,ackermansEffectsExplicitIntention2020, ahujaEduSensePracticalClassroom2019, akashClassificationModelSensing2018, alexanderGrandChallengesShapeChanging2018, altInteractionTechniquesCreating2013a, altUsingEyetrackingSupport2014, amoresEssenceOlfactoryInterfaces2017a, andersonInteractionsLookingBroadly2009, anjaniWhyPeopleWatch2020, anThermorphDemocratizing4D2018, baharinSonicAIRSupportingIndependent2015a, bakerSchoolsBackScaffolding2021, bakerUnderstandingTrustRepair2018, bedriEarBitUsingWearable2017, bedriFitByteAutomaticDiet2020, bellUMeExploringHuman2023, benkoSphereMultitouchInteractions2008, bermanHowDIYMetaDesignTools2021, bernsteinCrowdsTwoSeconds2011, besmerMovingUntaggingPhoto2010a, bhatWeAreHalfdoctors2023, boringTouchProjectorMobile2010, breidebandHomeLifeWorkRhythm2022a, brownTroubleAutopilotsAssisted2017, brubakerFocusingSharedExperiences2012, brudySurfaceFleetExploringDistributed2020, brushDigitalNeighborhoodWatch2013, burgessHealthcareAITreatment2023, buschelMIRIAMixedReality2021, caoLargeScaleAnalysis2021a, chancellorWhoHumanHumanCentered2019, chandrasekharanBagCommunitiesIdentifying2017, chenCaringCaregiversDesigning2013, chenDuetExploringJoint2014, chenFinexusTrackingPrecise2016, chiouDesigningAIExploration2023, churchillTeachingLearningHumancomputer2013, conversyVizirDomainSpecificGraphical2018, correllEthicalDimensionsVisualization2019a, dangTextGenerationSupporting2022, danzicoDesignSerendipityNot2010, dasswainAlgorithmicPowerPunishment2023, dekaZIPTZeroIntegrationPerformance2017, dementyevDualBlinkWearableDevice2017, dementyevRovablesMiniatureBody2016, dementyevSensorTapeModularProgrammable2015, dementyevWristFlexLowpowerGesture2014, devitoHowTransfeminineTikTok2022, disalvoFruitAreHeavy2017a, disalvoNavigatingTerrainSustainable2010, dodgeExplainingModelsEmpirical2019, doveUXDesignInnovation2017, duDepthLabRealtime3D2020, ehsanAutomatedRationaleGeneration2019a, ekhtiarGoalsGoalSetting2023, elkinAlignedRankTransform2021, elsdenMakingSenseBlockchain2018, engelbutzederSurplusScarcityAbundance2023, epsteinSuppressingSearchEngine2017, ernalaLinguisticMarkersIndicating2017, estevesOrbitsGazeInteraction2015, fastMetaEnablingProgramming2016, feitEverydayGazeInput2017, fenderCausalitypreservingAsynchronousReality2022a, fengHowUXPractitioners2023, feuchtnerExtendingBodyInteraction2017, fischerRecommendingEnergyTariffs2013, follmerInFORMDynamicPhysical2013, forlizziWhereShouldTurn2010, foxPatchworkHiddenHuman2023, freedStalkersParadiseHow2018, freemanWorkingTogetherApart2022, freyBreezeSharingBiofeedback2018, fridmanCognitiveLoadEstimation2018, furloRethinkingDatingApps2021, geroSparksInspirationScience2022, gheranGesturesSmartRings2018, goedickeVROOMVirtualReality2018, gomesMorePhoneStudyActuated2013, grillAttitudesFolkTheories2022, groegerObjectSkinAugmentingEveryday2018, gugenheimerFaceTouchEnablingTouch2016, guoFacadeAutogeneratingTactile2017, haeslerConnectedSelfOrganizedCitizens2021, hanHydroRingSupportingMixed2018, hartikainenSafeSextingAdvice2021, hassanFootStrikerEMSbasedFoot2017, hassibEmotionActuatorEmbodied2017, headAugmentingScientificPapers2021, held3DPuppetryKinectbased2012, hewittAssessingPublicPerception2019, hirschNothingCompilationHow2024, hollanderTaxonomyVulnerableRoad2021, hongSmartphonebasedSensingPlatform2014a, honnetPolySenseAugmentingTextiles2020, huangOrecchioExtendingBodyLanguage2018, huangWovenProbeProbingPossibilities2021, huhHealthVlogsSocial2014, hurstAutomaticallyDetectingPointing2008, hwangSocialbotsVoicesFronts2012, iqbalOasisFrameworkLinking2010, ishiiOpticalMarionetteGraphical2016, jainHeadMountedDisplayVisualizations2015, jiangHandAvatarEmbodyingNonHumanoid2023, jokelaDiaryStudyCombining2015, kaimotoSketchedRealitySketching2022, kaneAccessOverlaysImproving2011, kapurAlterEgoPersonalizedWearable2018a, karranFrameworkPsychophysiologicalClassification2015, khotUnderstandingPhysicalActivity2014, kimCellsGeneratorsLenses2023, kimDatadrivenInteractionTechniques2014, kimExploringChartQuestion2023a, kimInflatableMouseVolumeadjustable2008a, kimOmniTrackFlexibleSelfTracking2017, kirkHomeVideoCommunication2010, kirkHumanRemainsValues2010, kirkOpeningFamilyArchive2010, kleinbergerSupportingElderConnectedness2019, klokmoseWebstratesShareableDynamic2015, kongFramesSlantsTitles2018, kraemerBatteryfreeMakeCodeAccessible2022, krumAllRoadsLead2008a, kundingerDriverDrowsinessAutomated2020a, lafreniereCrowdsourcedFabrication2016, laputSensingFineGrainedHand2019, lawsonProblematisingUpstreamTechnology2015a, leeSleepGuruPersonalizedSleep2022a, leeZeroNMidairTangible2011, leithingerPhysicalTelepresenceShape2014, leithingerSublimateStatechangingVirtual2013, liaoAllWorkNo2018, lightDigitalInterdependenceHow2011, liMeasuringUnderstandingPhoto2019, linAdasaConversationalVehicle2018, linSYNCCrowdsourcingPlatform2021a, liTangibleGridTangibleWeb2022, liu3DALLEIntegratingTextImage2023, liuLivingHeritageHistoric2012, luceroMobileCollocatedInteractions2013, lugerHavingReallyBad2016b, luParticipatoryNoticingPhotovoice2023, maloneyTalkingVoiceUnderstanding2020, marcusTwitinfoAggregatingVisualizing2011, markResilienceCollaborationTechnology2008, marlowActivityTracesSignals2013, matthiesEarFieldSensingNovelEar2017a, mccarthyContextContentCommunity2008, mcgillCreatingAugmentingKeyboards2022, mcnallyCodesigningMobileOnline2018, mentisInvisibleEmotionInformation2010a, messerschmidtANISMAPrototypingToolkit2022, miltonSeeMeHere2023, mizrahiDigitalGastronomyMethods2016, muellerJoggingDistanceEurope2010, muellerNextStepsHumanComputer2020, mullenbachExploringAffectiveCommunication2014, mullerLookingGlassField2012, mulliganThisThingCalled2019, nakagakiChainFORMLinearIntegrated2016, nakajimaReflectingHumanBehavior2008, niiyamaPoimoPortableInflatable2020a, nunesSharingDigitalPhotographs2008, obristOpportunitiesOdorExperiences2014, oduorFrustrationsBenefitsMobile2016, oelenCrowdsourcingScholarlyDiscourse2021, oharaBlendedInteractionSpaces2011, olesonTeachingInclusiveDesign2023, ouCilllia3DPrinted2016, parkManifestationDepressionLoneliness2015, parkMetaverseWorkspaceOpportunities2023, peiHandInterfacesUsing2022, perraultWatchitSimpleGestures2013, pfeifferCruiseControlPedestrians2015, pierceInterfaceUserExploratory2018, porfirioAuthoringVerifyingHumanRobot2018, prakashAutoDescFacilitatingConvenient2023, qianInferringMotionDirection2017a, qianScalARAuthoringSemantically2022, rajanTaskLoadEstimation2016, rappExploringLivedExperience2023, reichertsItsGoodTalk2022, retelnyExpertCrowdsourcingFlash2014, richardsmaldonadoReaderQuizzerAugmentingResearch2023a, roffarelloAchievingDigitalWellbeing2023, ruanComparingSpeechKeyboard2018, sahaLanguageLGBTQMinority2019, sanchesMindBodyDesigning2010a, satriadiMapsMe3D2020, scheuermanHowComputersSee2019, scheuermanHowWeveTaught2020, scheuermanSafeSpacesSafe2018a, schlesingerLetsTalkRace2018, schmidtCrossdeviceInteractionStyle2012a, schoenebeckGivingTwitterLent2014, schrillsHowUsersExperience2023, schroederPocketSkillsConversational2018, schweikardtSUSTAINABLYOURSUserCentered2009, seboRobotsGroupsTeams2020, shiMarkitTalkitLowBarrier2017, silvaVehicleDriverRecognition2012a, sodenInformatingCrisisExpanding2018, sohnDiaryStudyMobile2008, steichenUseradaptiveInformationVisualization2013, steimleFlexpadHighlyFlexible2013, stewartCharacteristicsPressurebasedInput2010a, streliHOOVHandOutView2023, strohmayerTechnologiesSocialJustice2019, suDesigningNomadicWork2008a, suhAISocialGlue2021, suhSensecapeEnablingMultilevel2023, sunInvestigatingExplainabilityGenerative2022, suraleTabletInVRExploringDesign2019a, sutherlandGigEconomyInformation2017, suzukiAugmentedRealityRobotics2022, suzukiShapeBotsShapechangingSwarm2019, teibrichPatchingPhysicalObjects2015, tolmieThisHasBe2016, toTheyJustDont2020, trajkovaAlexaToyExploring2020, trujilloMakeRedditGreat2022, tsaknakiExpandingWabiSabiDesign2016a, tuddenhamGraspablesRevisitedMultitouch2010, uhlTangibleImmersiveTrauma2023a, vaishTwitchCrowdsourcingCrowd2014, valentineFlashOrganizationsCrowdsourcing2017, vazquez3DPrintingPneumatic2015a, vinesQuestionableConceptsCritique2012, vonsawitzkyHazardNotificationsCyclists2022, wangAutoDSHumanCenteredAutomation2021, wangCarNoteReducingMisunderstanding2017, wangCASSBuildingSocialSupport2021, wangHumanAICollaborationData2019, wangSeismoBloodPressure2018, weibelPaperSketchPaperdigitalCollaborative2011, westQuantifiedPatientDoctors2016, whitmireHapticRevolverTouch2018, whitneyCOVERSTORYTheCounterfeit2009, wibergMaterialityMattersexperienceMaterials2013, willisPrintedOptics3D2012, willisSideBySideAdhocMultiuser2011, wuAIChainsTransparent2022, wuUnfabricateDesigningSmart2020, xiaCrossTalkIntelligentSubstrates2023, xuEnablingHandGesture2022, yanEnhancingAudienceEngagement2016a, yangMakingSustainabilitySustainable2014, yangSimuLearnFastAccurate2020, yenVisibleHeartsVisible2018, yeoStickEarMakingEveryday2013a, zagalskyEmergenceGitHubCollaborative2015, zhangAlgorithmicManagementReimagined2022, zhangHowDataScience2020, zhangSkinTrackUsingBody2016, zhangSpeeChinSmartNecklace2021, zhangTomoWearableLowCost2015, zhengTellingStoriesComputational2022, zhouCyanochromicInterfaceAligning2023}
\printbibliography[keyword={fleeting}, heading=none]

\end{refsection}

\end{document}
\endinput
%%
%% End of file `sample-sigconf-authordraft.tex'.
