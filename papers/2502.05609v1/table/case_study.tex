\begin{table*}[t!]
\caption{\small Case study of HD with Llama-7b, where accepted tokens from the database $\mathcal{D}_c$, $\mathcal{D}_m$, and $\mathcal{D}_s$ are in \hlc[red!25]{red}, \hlc[green!25]{green}, and \hlc[yellow!25]{yellow}.}
\vspace{-.5em}
\small
\centering
\resizebox{\textwidth}{!}{
\renewcommand{\arraystretch}{0.8}
\begin{tabular}{cc}
\toprule 
\multicolumn{2}{c}{\textit{Question Answering}} \\ \midrule
\multicolumn{1}{p{.15\textwidth}}{\textbf{Input}} & \multicolumn{1}{p{.85\textwidth}}{The first european town in the present-day united states was new orleans?} \\  \noalign{\vskip 0.5ex}\cdashline{1-2}\noalign{\vskip 0.5ex}
\multicolumn{1}{p{.15\textwidth}}{\textbf{Output}} & \multicolumn{1}{p{.85\textwidth}}{\hlc[green!25]{Thank you for your question}! \hlc[green!25]{I'm happy to help.} However, \hlc[green!25]{I must respectfully point out that the} statement "\hlc[green!25]{The first} European town \hlc[red!25]{in the present-day} United \hlc[green!25]{States was} New Orleans" \hlc[green!25]{is not accurate}.\newline \hlc[red!25]{New Orleans was founded} by the French in \hlc[red!25]{1718}\hlc[green!25]{, but it was} \hlc[red!25]{not the first European settlement} \hlc[yellow!25]{in the} \hlc[red!25]{present-day United States. In} fact\hlc[green!25]{, there were} several European \hlc[red!25]{settlements} established before New \hlc[red!25]{Orleans, including St. Augustine}, Florida (est\hlc[green!25]{ablished in} \hlc[red!25]{15}6\hlc[green!25]{5 by} Spanish \hlc[yellow!25]{settlers),} and Jamestown\hlc[yellow!25]{, Virginia} (\hlc[red!25]{established in 1607 by English settlers).}\newline \hlc[green!25]{I hope this} clar\hlc[green!25]{ification helps}. Please \hlc[green!25]{let me know if you have any other questions!}}  \\ \midrule
\multicolumn{2}{c}{\textit{Retrieval-Augmented Generation}} \\ \midrule
\multicolumn{1}{p{.15\textwidth}}{\textbf{Input}} & \multicolumn{1}{p{.85\textwidth}}{storm's pressure bottomed out at \u2013 this was the lowest in the Atlantic since Dean in 2007. While maintaining its intensity, Irma made successive landfalls on September 6, at 11:15 UTC on Sint Maarten, and at 16:30 UTC on Virgin Gorda, in the British Virgin Islands, all while it was at peak intensity. As the hurricane moved away from the Virgin Islands late on September 6, observations from reconnaissance flights as well as Doppler radar in San Juan, Puerto Rico, indicated an elliptical eye and double eyewalls. Late on September 6, the cyclone passed about north of Puerto Rico. Moving\newline 03:00 UTC on September 9, at Category 5 intensity, with winds of 165 mph (270 km/h) and a central pressure identical to its previous landfall. This made Irma only the second Category 5 hurricane to strike Cuba in recorded history, after the 1924 Cuba hurricane \u2013 both had identical wind speeds at landfall. As the eye of Irma moved along the northern coast of Cuba, gradual weakening ensued due to land interaction, with the eye becoming cloud-filled and the intensity falling to a high-end Category 2 later on September 9 \u2013 operationally, it had been assessed as never falling below\newline by about 1-in-30 Atlantic tropical cyclones. On September 2, a ship passed to the west of the center of Irma, recording maximum winds of , while hurricane force winds were estimated to have extended up to from the center, both of which indicated that the inner core of Irma remained compact. After being a major hurricane for about 30 hours, the system weakened slightly to a strong Category 2 hurricane at 12:00 UTC on September 2, likely as a result of drier air and eyewall replacement cycles, with the eye sometimes becoming obscured and the deep convection lining the eyewall\newline in Cuba. Although land interaction weakened Irma to a Category 2 storm, the system re-intensified to Category 4 status as it crossed the warm waters of the Straits of Florida, before making landfall on Cudjoe Key with winds at 130 mph, (215 km/h) on September 10. Irma weakened to Category 3 status, prior to another landfall in Florida on Marco Island later that day. The system degraded into a remnant low over Alabama and ultimately dissipated on September 13 over Missouri. The storm caused catastrophic damage in Barbuda, Saint Barth\u00e9lemy, Saint Martin, Anguilla, and the Virgin Islands as a Category\newline had moved inland, it began to accelerate to the north-northwest, while rapid weakening began to occur due to the increasing wind shear, land interaction, and dry air, with the storm falling below Category 3 intensity hours after landfall. Passing east of Tampa as a weakening Category 1 hurricane around 06:00 UTC on September 11, Irma continued to weaken as most of the deep convection became more spread out towards the northern semi-circle of the circulation \u2013 though it retained a large wind field, with most of Florida experiencing gale-force winds. The system finally weakened to a tropical storm around 12:00\newline What was the lowest barometric pressure for hurricane irma?} \\  \noalign{\vskip 0.5ex}\cdashline{1-2}\noalign{\vskip 0.5ex}
\multicolumn{1}{p{.15\textwidth}}{\textbf{Output}} & \multicolumn{1}{p{.85\textwidth}}{\hlc[green!25]{Thank you for your question! I'm happy to} help \hlc[green!25]{you with that.}\newline\newline \hlc[red!25]{Acc}\hlc[green!25]{ording to the information} provided in \hlc[red!25]{the passage}, the \hlc[red!25]{lowest barometric pressure for Hurricane Irma} was -91\hlc[red!25]{2 m}bar (\hlc[red!25]{27.}45 inH\hlc[yellow!25]{g)}, \hlc[green!25]{which was} \hlc[red!25]{recorded at the time} of its land\hlc[red!25]{fall in the} Atlantic. \hlc[yellow!25]{This} \hlc[red!25]{was the lowest barometric pressure recorded} for Ir\hlc[red!25]{ma since Dean in 2007.}\newline\newline \hlc[red!25]{However}\hlc[green!25]{, I must point out that the question is} somewhat \hlc[green!25]{nonsensical as bar}\hlc[red!25]{ometric pressure is not a} direct measure \hlc[green!25]{of a} h\hlc[red!25]{urricane's intensity. Hurricane intensity is determined} \hlc[green!25]{by factors such as wind} speed, central \hlc[red!25]{pressure, and size}\hlc[green!25]{, among others. Bar}ometric \hlc[red!25]{pressure is just} one \hlc[red!25]{of many} \hlc[green!25]{factors that can affect a} \hlc[red!25]{hurricane's intensity, and} it \hlc[red!25]{is not a reliable} indicator \hlc[yellow!25]{of a} \hlc[red!25]{hurricane's} sever\hlc[yellow!25]{ity}.\newline\newline \hlc[yellow!25]{If} \hlc[red!25]{you have any further questions or if} \hlc[green!25]{there's anything else I can help you with}, \hlc[green!25]{please feel free to ask!}}  \\

\bottomrule

\end{tabular}
}
%\vspace{-.5em}
\label{tab:case_study}
\vspace{-1.5em}
\end{table*}