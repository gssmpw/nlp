
We begin by formally defining speculative decoding and database drafting and present our proposed method, Hierarchy Drafting (HD), which addresses the limitations of database drafting methods.

\subsection{Preliminary}

\paragraph{Speculative Decoding} 
At each step of speculative decoding, multiple tokens \(\tilde{\bm{x}}_{1:m}\) (i.e., draft token sequence) are drafted from an approximate model \(\mathcal{M}_q\) to predict future tokens of LLM \(\mathcal{M}_p\) (i.e., target model) for previous text tokens \(\bm{x}_{\leq t}\):
\begin{align}
    \tilde{\bm{x}}_{1:m} &\sim_m \mathcal{M}_q(\bm{x}_{\leq t}).
\end{align}

All draft token sequence \(\tilde{\bm{x}}_{1:m}\) are verified against the actual output of \(\mathcal{M}_p\). For example, in the greedy decoding, the tokens \(\bm{x}'_{t+1:t+m}\) are obtained for a given \(\tilde{\bm{x}}_{1:m}\) and \(\bm{x}_{\leq t}\) by solving the following equations in parallel:
\begin{align}
\begin{cases}
    x'_{t+1} &= \argmax P_{\mathcal{M}_p}(x | \bm{x}_{\leq t}), \\
    x'_{t+2} &= \argmax P_{\mathcal{M}_p}(x | \tilde{x}_{1}, \bm{x}_{\leq t}),\\
    &\dots \\
    x'_{t+m} &= \argmax P_{\mathcal{M}_p}(x | \tilde{\bm{x}}_{1:m}, \bm{x}_{\leq t}).
\end{cases}
\end{align}
Each token \(x'_{t+i}\) is verified against the corresponding draft token \(\tilde{x}_{t+i}\), starting from \(i = 0\) until the verification fails or \(i = m\) is reached.
To enhance the likelihood of acceptance, multiple draft token sequences \(\bm{\tilde{X}} = \{\tilde{\bm{x}}^i\}_{i=1}^N\) (i.e., draft set) are verified in parallel.
The specialized attention mask implements the parallel verification of the draft set, not causal attention mask~\cite{LAD, SpecInfer}.
In the sampling strategy, speculative sampling~\cite{SpecSampling} is commonly used to accept more tokens while maintaining identical output distributions of the target model.
In summary, the generation step is divided into two sub-steps with a single forward pass of the target model. The multiple accepted tokens are generated simultaneously, compressing the overall decoding process.
% In summary, the generation step consists of two sub-steps: a single forward pass of the target model, followed by simultaneous generation of multiple tokens accepted through verification, compressing the overall decoding process.


\paragraph{Database Drafting}
As shown on the left side of Figure~\ref{fig:overview}, the methods included in database drafting exploit the database \(\mathcal{D}\), having the prefix tokens as the key and the subsequent tokens as the value. Per each step of the generation process, the draft token sequence \(\bm{\tilde{x}}_{1:m}\) is retrieved from database \(\mathcal{D}\) for given previous tokens \(\bm{x}_{t-l:t}\):
\begin{align}
    \bm{\tilde{x}}_{1:m} \in \bm{\tilde{X}} &= \texttt{Ret}(\bm{x}_{t-l:t};\mathcal{D}),
\end{align}
where \(l\) and \(m\) are the length of previous tokens and draft token sequence. Subsequently, the verifying step is the same as other methods.

\subsection{Hierarchy Drafting}

We introduce Hierarchy Drafting (HD), which organizes tokens from diverse sources into three databases based on temporal locality and accesses them in order from the smallest to the largest scale. The overview and decoding process are depicted on the right side of Figure~\ref{fig:overview}.

\section{Synthesizing Polyglots}\label{sec:generation}
The generation of a polyglot is a challenging task. 
Unlike attack payloads designed for a single environment, we are faced with multiple programming languages and contexts.
Although a polyglot is basically just a string of characters, conceptually it is a chimera made up of terminal symbols from different grammars. Some of these characters are even ambiguous and only parsed correctly when the respective injection context is known.
To cope with this complexity and ambiguity, we develop an automated approach that phrases the synthesis of a polyglot as a discrete optimization problem.
The objective is to find a string that maximizes the number of exploitable injection contexts on a given testbed, regardless of the underlying languages and grammars.

This optimization has several constraints:
First, evaluating a polyglot on the testbed involves running an entire browser, including HTML parser and JavaScript engine.
Second, during the polyglot evaluation, we receive only binary feedback indicating success or failure, making gradient-based optimization unfeasible.
Lastly, as discussed in \Cref{sec:polyglots-background}, some test cases are mutually exclusive. Therefore, we must identify a \emph{set} of polyglots that collectively solve the testbed.


\subsection{Monte Carlo Tree Search}%
\label{sec:mcts}%

Interestingly, the described optimization problem can be cast into a game in which a player iteratively modifies a polyglot to exploit as many contexts as possible. In this game, moves correspond to changes of the polyglot, while its reward is determined by the number of exploited contexts. Different algorithms exist for solving such games, ranging from simple search strategies to reinforcement learning \cite{russell2010artificial}.

For our analysis, we focus on the concept of \montecarlo{} (\mcts{})~\cite[][]{browneMcts:2012} that is known to find strong moves in round-based games, such as Chess and Go \cite{silver2016mastering}. The generation of polyglots, however, is agnostic to this algorithm and thus we present a comparison of \mcts{} with other strategies for solving the underlying game in Appendix \ref{appendix:alternative-generation}. %

Technically, \mcts{} simulates multiple games to conclusion to gather knowledge about promising game states.
To keep track of the performance, \mcts{} constructs a game tree in which each node corresponds to a polyglot and contains two attributes: a visit and a win counter. The algorithm explores a path in this tree using four steps:

\begin{enumerate}
\setlength\itemsep{-0.3em} %
\item \emph{Selection:} Starting from the root node of the game tree, an unexplored child or the most-promising child node is selected until a leaf node is reached.
\item \emph{Expansion:} If the selected leaf node is non-terminal, the game tree is expanded by generating the child nodes of the leaf node.
\item \emph{Playout:} Starting from the leaf node, random nodes are explored until a terminal node is reached. This is equivalent to performing random moves until the game ends.
\item \emph{Backpropagation:} On the path back to the root, the visit counter is incremented by one and the win counter is updated according to the result of the simulated game.
\end{enumerate}

During the selection phase, \mcts{} balances exploitation (selecting moves that were successful in the past) and exploration (gathering knowledge about unexplored areas).
To rank each child $i$, we use the upper confidence bound $\frac{w_i}{n_i} + \sqrt{\frac{2 \ln N_i}{n_i}}$ where $w_i$ is the number of wins, $n_i$ is the number of visits and $N_i$ is the number of visits of the parent node. If a child has not been explored before, \ie, $n_i=0$, it takes precedence.

\definecolor{gfrcolor}{HTML}{7DBA91}
\definecolor{gfrred}{HTML}{8D0400}
\definecolor{gfryellow}{HTML}{F8CC00}
\definecolor{gfrdiamond}{HTML}{297A8C}

\begin{figure*}[ht]
	\centering
    \includegraphics[width=\linewidth]{polyglot-gfr-performance.pdf}\vspace{-3mm}%
	\caption{%
	Composition of our minimal polyglot set solving all \num{111} test cases of our testbed. 
	Squares (\textcolor{gfrcolor}{$\blacksquare$}) indicates solved tests.
	The top row (\textrm{D}) visualizes the overall difficulty of each test case, calculated from the total ratio of polyglots we generated that solve this test \ie, \emph{difficult tests} with fewer solutions are red (\textcolor{gfrred}{$\blacksquare$}), while the color of more \emph{frequently solved} tests gradually shifts to yellow (\textcolor{gfryellow}{$\blacksquare$}).
	The numbered rows show the performance of our polyglot set.
	Diamonds (\textcolor{gfrdiamond}{$\blacklozenge$}) are placed instead of squares, where only one polyglot in the set solves a test.
	For comparison, the bottom row (*) shows the performance of the Ultimate polyglot. 
	}%
    %
    \label{fig:set-gfr-perfomance}
\end{figure*}

\subsection{Synthesizing Polyglots with \mcts{}}%
\label{sec:synthesizing-with-mcts}


To reduce the extensive search space, we incorporate expert knowledge into the construction of the game tree:
Instead of operating on all characters, we generate polyglots only from a set of tokens, such as \code{<} and \code{script}. We refine these tokens with simple rules that prevent invalid combinations form appearing, such as \code{svg} appended to \code{iframe}.
As a result of this reduction, \mcts{} omits moves that are impossible to yield reasonable polyglots.
The resulting token set consists of HTML literals, HTML tag names, HTML event handler content attribute, and other HTML tokens.
The complete list including the rules can be found in our companion repository. %



Starting from an empty string, \mcts{} iteratively appends tokens to it to expand the game tree and measure the polyglot's reward.
Unlike conventional games, however, this game has no clear end state, since polyglots can be arbitrary long.
We therefore set a fixed limit of 400 characters, after which we stop appending tokens to the polyglot.
Since this fixed length condition might lead to superfluous tokens, we employ a minimization strategy afterwards (see Section~\ref{sec:polyglot-minimization}).

To evaluate a polyglot during search, we implement a fast, manually created testbed covering \num{27} common XSS injection contexts using Puppeteer (13.1.3) based on Chromium (98.0.4758.0).
The complete testbed is included in our companion repository.
In the regular \mcts{} backpropagation, each polyglot would be assigned a score used to update the win counters on its path. Since we evaluate the polyglot on multiple test cases simultaneously, however, we receive a list of scores on each playout. Instead of saving only one score, we thus save the entire list and sum it up, once we need the total score $w_i$ for a particular node in the tree. As a result, we can keep track of the individual tests solved by a particular polyglot during construction.
This gives us more flexibility: When some test cases have already been covered, we can exclude them from the score calculation in later runs. %

The final synthesis of one polyglot is shown in Algorithm~\ref{alg:gen-polyglots}:
Starting from the root node, we use \mcts{} to explore the game tree for $N$ rounds until we fix the first move, \ie, select the first token of the polyglot.
With the first token in place, we repeat the process $N$ times and select the second token.
We continuously start the game from a deeper root node until the depth limit $D$ is reached.
At the end, the polyglot with the highest score is returned.

\begin{algorithm}[htb]
\caption{Generating a polyglot}%
\label{alg:gen-polyglots}
\tablesize
\DontPrintSemicolon
\SetKwInput{KwInput}{Input}
\SetKwInput{KwOutput}{Output}
\SetAlgoNlRelativeSize{-2} %
\SetNlSty{normalfont}{}{} %
\KwInput{root, start node for generation}
\KwInput{$D$, maximum depth of the root node}
\KwInput{$N$, iterations until a move is chosen}
\KwOutput{polyglot string}
bestPolyglot, bestScore $\gets$ null, 0 \;
\For{$i \gets 1$ \KwTo $D$}{
    \For{$j \gets 1$ \KwTo $N$}{
        leaf $\gets$ select(root)\;
        expand(leaf)\;
        polyglot, score $\gets$ playout(leaf)\;
        backpropagate(leaf, score)\;

        \If{score > bestScore} {
            bestScore, bestPolyglot $\gets$ score, polyglot \;
        }
    }
    root $\gets$ choose\_best(root)\;
}
\KwRet~bestPolyglot\;
\end{algorithm}


At best, our approach would generate a single polyglot to ``rule them all'', that is, solve all provided test cases.
However, our analysis reveals that depending on the test cases, this might not be possible.
For example, as discussed in \Cref{sec:polyglots-background}, some injection contexts are mutually exclusive and a polyglot providing proper execution of JavaScript code in both cannot exist.
As a remedy, we develop a strategy for combining multiple polyglots into a set. 

In particular, after one polyglot is created, we remove the test cases it solves and start over the search process of \mcts{}.
As result, by design, we seek a complementing polyglot that focuses only on those cases the previous one cannot solve.
We continue generating new polyglots using Algorithm~\ref{alg:gen-polyglots}, until all tests have been covered or a maximum number of tries is reached.
Finally, we obtain a set of polyglots that solves the entire testbed of the synthesis process.

\subsection{Selecting a Final Polyglot Set}%
\label{sec:polyglot-evaluation}

While we synthesize polyglots on a fast testbed, we use a larger testbed to determine the quality of the generated polyglots.
For this purpose, we set up a local instance of the Google Firing Range (GFR)~\cite{github-firing-range}, which is a state-of-the-art XSS testbed~\cite{BazCriMag16}.
We manually choose a subset of GFR tests for our scope, based on the following criteria:
First, we exclude non XSS-related test cases.
Second, in cooperation with Google, we obtain a list of GFR tests that have a known solution which we manually confirm.
We further exclude tests exceeding our scope, namely exploits relying on specific frameworks and those requiring a specific encoding.
Tests that aggressively block special characters with an error page, \eg, any request that contains \code{<} or \code{>}, are also excluded.
While solutions for these tests exist, their solution needs to be so narrow that they are no longer a polyglot and thus also out-of-scope. 
We confirmed all solutions and verified that an applicable solution exists for our requirements.
A detailed list of the excluded tests and the specific reasons can be found in Appendix~\ref{appendix:gfr-exclusions}. %

After generating \num{4000} polyglots over multiple runs from different seeds, we utilize a greedy min-set cover approach to select a minimal subset that solves all GFR test cases.
We build this minimal set by adding the polyglot to the set that solves the most tests yet unsolved by the set, until no additional solutions can be added.
Our final set of polyglots that is used in our study consists of 7 instances.
Thus, we answer \ref{rq:1} and show that it is possible to automatically create a set of XSS polyglots that cover common injection contexts.
To compare the efficacy of our final polyglot set with a publicly available polyglot, we chose the best polyglot from the blog posts mentioned in \Cref{sec:rqs}---the \emph{Ultimate \xss{} Polyglot}~\cite{ultimate-polyglot}---as our reference. 
Now, to display the composition of our polyglot set, \Cref{fig:set-gfr-perfomance} visualizes our evaluation results for each polyglot on the \num{111} test cases.
We give an indication of the difficulty of synthesizing a solution for each test by assigning a color to each of them. 
The number of polyglots from our full set that successfully solve a test, determines its difficulty.
Tests for which we synthesized fewer solutions tend to be more challenging and are depicted in red.
In contrast, the color of easier tests, \ie, those with more solutions, gradually shifts to yellow.
The numbered rows below correspond to each polyglot of our final set, while the bottom row displays the Ultimate polyglot's score.
Indicated by diamond symbols in the corresponding rows, each polyglot in the final set contributes at least one unique solution, \eg, polyglot \num{4} is the only in the set solving test \num{36}.
The plot shows that our polyglots cover all test cases, ranging from \num{31} to \num{89} tests covered by each individual polyglot.
The Ultimate polyglot works surprisingly well and manages to cover \num{88} tests.
However, our automatic polyglot set creation outperforms manual engineering, as our generation process can target gaps left by previously synthesized polyglots.

\subsection{Minimizing Polyglots}%
\label{sec:polyglot-minimization}

Since our generation uses a termination criterion that sets a fixed minimum length for a polyglot, it is possible that some tokens in the polyglot are superfluous.
In general, this would not constitute a problem, but some websites have length constraints to the inputs in their backend. %
To make sure that such obstacles do not obstruct our polyglots, we minimize the lengths of our final polyglots.
The objective is to find the smallest polyglot with equivalent test results on the GFR testbed\@.
The minimization is done automatically and at token level, \ie, we first deconstruct each polyglot into its tokens.
Since each token could be removed independently, there are $2^N$ minimization candidates where $N$ is the number of tokens in the polyglot.
Instead of testing all $2^N$ options, we build our minimization strategy on a single assumption:
If removing a token changes the test results, then all minimizations in which that token is removed are invalid.
Although there might be edge cases, where this does not hold true, this drastically reduces the search space as we can first test each token separately.
Our polyglot minimization then involves iteratively removing combinations of tokens until no more tokens can be removed without altering the evaluation result.
Overall, in every polyglot (except for one), some unnecessary tokens have been identified and removed.
For the other polyglots we achieve a reduction between 6\% and 24\%.


\paragraph{Observation: Temporal Locality}
The main idea behind database drafting is that some tokens are easy to retrieve from the database because they exhibit temporal locality—meaning they tend to be repeated within or across the generation processes. 
However, note that not all draft token sequences share the same level of temporal locality during generation. 
We analyze the pattern of unique 4-grams during 100 text generations on Spec-Bench~\cite{Spec_Survey}, as shown in Figure~\ref{fig:generation}. The results reveal that certain 4-grams are frequently repeated and exhibit varying locality levels.
Specifically, the blue dots and the right small plot in Figure~\ref{fig:generation} illustrate local redundancy, where the same 4-gram appears multiple times within a single generation step. This reflects high temporal locality within a single generation rather than across multiple generations. In contrast, the red dots in Figure~\ref{fig:generation} highlight a pattern where the model repeatedly generates the same 4-grams at different stages of the generation process, illustrating its tendency to reuse familiar sequences over time.
Additionally, the lower plot of Figure~\ref{fig:generation} presents the frequency study of sampled red and blue dots, demonstrating that some tokens exhibit high temporal locality within a specific context, while others maintain consistent locality across generation processes.
Therefore, given the varying temporal locality of tokens throughout the generation process, drafting steps should prioritize tokens with higher temporal locality over others.

% For example, when an LLM solves a math problem like, “The vertices of a triangle are at points (0, 0), (-1, 1), and (3, 3). What is the area of the triangle?”, the coordinates are frequently repeated.
% Next, frequently generated phrases by LLMs, such as “as an AI assistant,” show moderate locality, as they often appear across various generation processes for LLM-generated texts. 
% Finally, grammatical patterns and universal phrases commonly used by humans have the lowest locality, as they are statistically frequent across all types of texts yet do not constantly occur in each generation process.

\paragraph{Database Design}  
 Based on the temporal locality of draft token candidates, we design three types of databases to categorize them. 
\textbf{1) Context-dependent DB} (\(\mathcal{D}_c\)) contains tokens highly relevant to the specific context of the generation process, such as the blue dots in the Figure~\ref{fig:generation}. 
This includes tokens from the input prompt, tokens generated through parallel decoding, tokens discarded during the generation process, and others that are highly relevant to a given context.  
\(\mathcal{D}_c\) is lookup table with the prefix tokens, \(\bm{x}_{1:l}\), as the key and the subsequent tokens, \(\bm{x}_{l:l+m}\), as the value.
Also, \(\mathcal{D}_c\) is consistently updated during each forward step and initialized when the following generation process is started. 
The database follows the Least Recently Used (LRU) policy for draft sequence updates.
\textbf{2) Model-dependent DB} (\(\mathcal{D}_m\)) stores tokens frequently generated by LLM regardless of context, as represented by the red dots in Figure~\ref{fig:generation}.
Top-$k$ frequently generated token sequences, $\bm{x}_{1:l+m}$, are sampled from the model-generated texts, with $\bm{x}_{1:l}$ as the key and $\bm{x}_{l+1:l+m}$ as the value.
For \(\mathcal{D}_c\) and \(\mathcal{D}_m\), the maximum size of values for a single key is the same as the maximum draft set size \(N\). 
\textbf{3) Statistics-dependent DB} (\(\mathcal{D}_s\)) draws its tokens from large text corpora to capture universal phrases commonly used in the language. 
Although these tokens are frequent, they occur less consistently across processes than those in \(\mathcal{D}_m\).
To efficiently retrieve the sequence from a large corpus, we utilize a suffix array~\cite{suffix_array} following the implementation of~\citet{REST}.
Implementation details are in \S\ref{sec:experiement}.


Our database design yields three distinct advantages. First, it integrates diverse sources into multiple databases, enabling us to leverage each source’s strengths for robust acceleration across various tasks. Then, each database’s size decreases as the tokens’ temporal locality increases since tokens with higher locality are rarer, providing an opportunity to optimize drafting latency. Finally, the design is \textit{plug-and-play}, easily integrating additional token sources by assigning them to the appropriate database based on their temporal locality.

\setlength{\textfloatsep}{1em}% Remove \textfloatsep

\begin{algorithm}[t!]
\caption{\small Decoding Process with Hierarchy Drafting}\label{alg:HD_process}
\small
\begin{algorithmic}[1]
\Require Target LLM $\mathcal{M}_p$, databases $(\mathcal{D}_c, \mathcal{D}_m, \mathcal{D}_s)$, input text sequence $\bm{x}_{\le t}$, target sequence length $T$, the size of prefix tokens $l$, the size of draft token sequence $m$, the size of draft set $N$;
\State $n \leftarrow t$\;
\While{$n < T$ and \texttt{[EOS]} $ \notin \bm{x}_{1:n}$}
    \State \textcolor{blue}{// \textit{Drafting Step: Hierarchical access to three databases until the size of the draft set $\bm{\tilde{X}}$ is $N$.}}
    \State $\bm{\tilde{X}} \leftarrow \texttt{Ret}(\bm{x}_{n-l:n};\mathcal{D}_c)$
    \If{$|\bm{\tilde{X}}| < N$}
        \State $\bm{\tilde{X}} \leftarrow \bm{\tilde{X}} \cup \texttt{Ret}(\bm{x}_{n-l:n};\mathcal{D}_m)$
    \EndIf 
    \If{$|\bm{\tilde{X}}| < N$}
        \State $\bm{\tilde{X}} \leftarrow \bm{\tilde{X}} \cup \texttt{Ret}(\bm{x}_{n-l:n};\mathcal{D}_s)$
    \EndIf 
    \State \textcolor{blue}{// \textit{Verification Step: Verify the draft token sequence in $\bm{\tilde{X}}$ and generate additional tokens for updating $\mathcal{D}_c$.}}
    \State $\bm{x}_{n:n+i}, \bm{\hat{x}} \sim_i \mathcal{M}_p(\bm{x}_{\le n}, \bm{\tilde{X}})$
    \State $\mathcal{D}_c \leftarrow \text{Update}(\mathcal{D}_c, \bm{\hat{x}})$
    % \If{\texttt{[EOS]} in $\bm{x}_{t:t+i}$}
    %     \State BREAK
    % \EndIf
    \State $n \gets n+i$
\EndWhile
\end{algorithmic}
\end{algorithm}
% \vspace{0-}

\paragraph{Hierarchical Access}
Using the three databases designed with the temporal locality in mind, we retrieve draft token sequence \(\bm{\tilde{x}}_{1:m}\) for the given previous input \(\bm{x}_{t-l:t}\).
Database access order is based on the degree of temporal locality within the current generation process; thereby, the access starts with \(\mathcal{D}_c\).
Access then proceeds to \(\mathcal{D}_m\), which has high locality across generations, and finally \(\mathcal{D}_s\), with moderate locality across generations, until draft set \(\bm{\tilde{X}}\) accumulates a sufficient number of candidates as pre-defined hyperparameter \(N\).
These accesses leverage the locality of the draft token sequence to enhance drafting accuracy and minimize latency overhead, preserving the benefits of drafting.

\paragraph{Decoding Process}
We introduce the inference process of speculative decoding with our proposed method, HD. 
% First, for a given previous input \(\bm{x}_{t-l, t}\), the set of draft token \(\bm{\tilde{X}}\) are acquired from the three databases with hierarchical access. 
First, for a given previous input \(\bm{x}_{t-l, t}\), we acquire the set of draft token \(\bm{\tilde{X}}\) from the three databases with hierarchical access. 
% Then, the target LLM \(\mathcal{M}_p\) verifies the draft token sequences simultaneously generating the additional tokens \(\bm{\hat{x}}\) for updating context-dependent DB either through parallel decoding~\cite{ParallelDecoding, LAD} or by recycling wasted tokens~\cite{trashintotreasure}. 
Then, the target LLM \(\mathcal{M}_p\) verifies the draft token sequences while simultaneously generating the additional tokens \(\bm{\hat{x}}\). 
These tokens are used to update the context-dependent DB either through parallel decoding~\cite{ParallelDecoding, LAD} or by recycling wasted tokens~\cite{trashintotreasure}. 
These processes are repeated iteratively until either the \texttt{[EOS]} token is generated or the sequence reaches the pre-defined maximum length \(T\).
Details of the decoding are depicted in Algorithm~\ref{alg:HD_process}.

