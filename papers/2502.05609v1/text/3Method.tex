
We begin by formally defining speculative decoding and database drafting and present our proposed method, Hierarchy Drafting (HD), which addresses the limitations of database drafting methods.

\subsection{Preliminary}

\paragraph{Speculative Decoding} 
At each step of speculative decoding, multiple tokens \(\tilde{\bm{x}}_{1:m}\) (i.e., draft token sequence) are drafted from an approximate model \(\mathcal{M}_q\) to predict future tokens of LLM \(\mathcal{M}_p\) (i.e., target model) for previous text tokens \(\bm{x}_{\leq t}\):
\begin{align}
    \tilde{\bm{x}}_{1:m} &\sim_m \mathcal{M}_q(\bm{x}_{\leq t}).
\end{align}

All draft token sequence \(\tilde{\bm{x}}_{1:m}\) are verified against the actual output of \(\mathcal{M}_p\). For example, in the greedy decoding, the tokens \(\bm{x}'_{t+1:t+m}\) are obtained for a given \(\tilde{\bm{x}}_{1:m}\) and \(\bm{x}_{\leq t}\) by solving the following equations in parallel:
\begin{align}
\begin{cases}
    x'_{t+1} &= \argmax P_{\mathcal{M}_p}(x | \bm{x}_{\leq t}), \\
    x'_{t+2} &= \argmax P_{\mathcal{M}_p}(x | \tilde{x}_{1}, \bm{x}_{\leq t}),\\
    &\dots \\
    x'_{t+m} &= \argmax P_{\mathcal{M}_p}(x | \tilde{\bm{x}}_{1:m}, \bm{x}_{\leq t}).
\end{cases}
\end{align}
Each token \(x'_{t+i}\) is verified against the corresponding draft token \(\tilde{x}_{t+i}\), starting from \(i = 0\) until the verification fails or \(i = m\) is reached.
To enhance the likelihood of acceptance, multiple draft token sequences \(\bm{\tilde{X}} = \{\tilde{\bm{x}}^i\}_{i=1}^N\) (i.e., draft set) are verified in parallel.
The specialized attention mask implements the parallel verification of the draft set, not causal attention mask~\cite{LAD, SpecInfer}.
In the sampling strategy, speculative sampling~\cite{SpecSampling} is commonly used to accept more tokens while maintaining identical output distributions of the target model.
In summary, the generation step is divided into two sub-steps with a single forward pass of the target model. The multiple accepted tokens are generated simultaneously, compressing the overall decoding process.
% In summary, the generation step consists of two sub-steps: a single forward pass of the target model, followed by simultaneous generation of multiple tokens accepted through verification, compressing the overall decoding process.


\paragraph{Database Drafting}
As shown on the left side of Figure~\ref{fig:overview}, the methods included in database drafting exploit the database \(\mathcal{D}\), having the prefix tokens as the key and the subsequent tokens as the value. Per each step of the generation process, the draft token sequence \(\bm{\tilde{x}}_{1:m}\) is retrieved from database \(\mathcal{D}\) for given previous tokens \(\bm{x}_{t-l:t}\):
\begin{align}
    \bm{\tilde{x}}_{1:m} \in \bm{\tilde{X}} &= \texttt{Ret}(\bm{x}_{t-l:t};\mathcal{D}),
\end{align}
where \(l\) and \(m\) are the length of previous tokens and draft token sequence. Subsequently, the verifying step is the same as other methods.

\subsection{Hierarchy Drafting}

We introduce Hierarchy Drafting (HD), which organizes tokens from diverse sources into three databases based on temporal locality and accesses them in order from the smallest to the largest scale. The overview and decoding process are depicted on the right side of Figure~\ref{fig:overview}.

% \section{Generating Hijacking Samples}
\section{\new{Methodology}}
\label{sec:methodology}
%A critical step of the Model Selection Hijacking Adversarial Attack is the generation of adversarial hijacking samples to be inject to the validation set. 
%We now present a novel methodology to design and generate such samples. %\lpasa{occhio che qua sembra che l'unica novelty sia la generzione di esempi!}
% A crucial phase in the Model Selection Hijacking Adversarial Attack involves generating adversarial hijacking samples for injection into the validation set. Among the novelties introduced by this work, we present a new methodology specifically for designing and generating these samples.
\new{
The MOSHI attack operates uniquely by injecting and substituting data points in the validation set with data from $\mathcal{S}^{Val}_{pois}$, disrupting the critical model selection phase without altering the training process or parameters.
This set, which the attacker carefully generates, will be used for the model selection phase, which in turn will return a model $\tilde{h}_{\mathfrak{c}^*}$:
    \begin{equation}
        \label{best_poison}
 \tilde{h}_{\mathfrak{c}^*} = \argmin_{h_{\mathfrak{c}} : \mathfrak{c} \in \mathfrak{C}} \mathcal{L}_{Val}(h_{\mathfrak{c}}, \mathcal{S}^{Val}_{pois}).
    \end{equation}
The selected model $\tilde{h}_{\mathfrak{c}^*}$ is different from $h_{\mathfrak{c}^*}$, as now, the poisoned validation set no longer allows for selecting a better, more generalized, model, but selects one that has a configuration of hyper-parameters which maximizes the hijack metric, chosen by the adversary. 
Thus, a central aspect of this approach involves generating adversarial hijacking samples crafted explicitly for injection into the validation set.
Among the novelties introduced in this work, we present a specialized methodology for designing and generating these samples (Section~\ref{subsec:generation}) and the hijack metrics used in our study (Section~\ref{ssec.hm-theory}).
}

% \subsection{Overview}
% The goal of the adversary is to assume control of the MS phase by injecting and substituting data points in the validation set with data from $\mathcal{S}^{Val}_{pois}$. This set, which is carefully generated by the attacker, will be used for the model selection phase, which in turn will return a model $\tilde{h}_{\mathfrak{c}^*}$:
%     \begin{equation}
%         \label{best_poison}
%  \tilde{h}_{\mathfrak{c}^*} = \argmin_{h_{\mathfrak{c}} : \mathfrak{c} \in \mathfrak{C}} \mathcal{L}_{Val}(h_{\mathfrak{c}}, \mathcal{S}^{Val}_{pois}).
%     \end{equation}

% The selected model $\tilde{h}_{\mathfrak{c}^*}$ is different from $h_{\mathfrak{c}^*}$, as now, the poisoned validation set, no longer allows for selecting a better, more generalized, model, but selects one that has a configuration of hyper-parameters which maximizes the hijack metric, chosen by the adversary. 

% \subsection{Generative Process}
\subsection{\new{Adversarial Sample Generation}}
\label{subsec:generation}
\new{
Although our adversarial sample generation model is based on the Variational Auto Encoder (VAE) architecture (Section~\ref{subsub:vae}), we introduce a variation of the conditional VAE architecture designed for the generation of hijacking samples (Section~\ref{subsub:hvae}).
}
\subsubsection{Variational Auto Encoder (VAE)}
\label{subsub:vae}
We design our generative process using a Variational Auto Encoder (VAE)~\cite{kingma2013auto}, which is an extension of more traditional Autoencoders~\cite{hinton2006reducing}. VAE consists of two modules: first, an \textit{encoder} which learns a \textit{posterior} recognition model $q_{\phi}(z|x)$, encoding an input $x$ to a latent representation $z$; second, a \textit{decoder} that generates samples from the latent space $z$ via the likelihood model $p_{\theta}(x|z)$. $\phi$ and $\theta$ are learning parameters. 
In contrast with standard autoencoders, VAEs enforce a continuous prior distribution $p(z)$, usually set to the Gaussian. This forces the model to encode the entire input distribution to the latent code rather than memorizing single data points. 
Traditional VAEs are trained with the following loss:
\begin{equation} \small
% \begin{split}
    \mathcal{L}_{VAE}(\phi, \theta) = KL(q_{\phi}(z|x) || p(z)) %\\
    -\mathbb{E}_{q_{\phi}(z|x)}(\log p_{\theta}(x|z)), 
% \end{split}
\end{equation}
where $KL$ is the Kullback-Leibler divergence~\cite{kullback1951information} that is a regularizer to keep the posterior distribution close to the prior. The second term is a simple reconstruction loss. 
For the scope of this work, we utilize a Conditional VAE (CVAE) that augments the latent space with information about the true label of a given sample~\cite{sohn2015learning}.  
%
\subsubsection{Hijacking VAE}
\label{subsub:hvae}
We now introduce Hijacking VAE (HVAE), a variation of the more traditional CVAE that is specifically designed to generate hijacking samples to produce $\mathcal{S}^{Val}_{pois}$.
These samples are created in such a way that, when used for computing $\mathcal{L}_{Val}$, the lower the models' hijack metric, the more significant the increase of their validation loss, hence swaying the model selection phase into returning the model that has the highest hijack metric (which has been the least penalized).
We design the HVAE loss function as follows:
    \begin{equation}
        \label{lossMHVAE}
 \mathcal{L}_{\mathrm{HVAE}} = (\mathcal{L}_{\mathrm{rec}} + \mathcal{L}_{\mathrm{KLD}} - Hj_{cost}(\mathfrak{C})) ^ 2.
    \end{equation}
Here, the terms $\mathcal{L}_{\mathrm{rec}}$ and $\mathcal{L}_{\mathrm{KLD}}$ represents the reconstruction loss and the KL divergence, as in the traditional VAE. 
The novel factor of the loss is represented by the third term $Hj_{cost}(\mathfrak{C})$.
This is the pivotal factor of the attack, defined as follows (with $\Lambda = \mathfrak{C}$):

\begin{equation} \label{cost}
     Hj_{cost}(\mathfrak{C}) = \frac{1}{|\mathfrak{C}|}\sum_{\mathfrak{c} \in \mathfrak{C}} \alpha \cdot \mathcal{L}_{Val}(h_{\mathfrak{c}}, \mathcal{S}_{gen})
\end{equation}
with 
\begin{equation} \label{alpha}
 \alpha = \frac
     {\underset{\lambda \in \Lambda}{\max} \{m(h_{\lambda}, \mathcal{S}^{Val})\} - m(h_ {\mathfrak{c}}, \mathcal{S}^{Val})}
     {\underset{\lambda \in \Lambda}{\max} \{m(h_{\lambda}, \mathcal{S}^{Val})\} - \underset{\lambda \in \Lambda}{\min} \{m(h_{\lambda}, \mathcal{S}^{Val})\}}. 
\end{equation}
 
We now explain the rationale behind Equation~\ref{cost}, which is an average of scores that are assigned to each model $\mathfrak{c} \in \mathfrak{C}$. 
The coefficient $\alpha \in \mathbb{R}$ (see Equation~\ref{alpha}) is computed by normalizing the difference between the maximum hijack metric achievable by a model $h_{\lambda}$ with $\lambda \in \Lambda = \mathfrak{C}$ and the metric of the current model.
$\alpha$ yields higher penalties the lower the hijack metric of the model $h_\mathfrak{c}$, reaching 0 if the considered model has the highest metric. This value is fixed for each model and can be computed independently of the HVAE training.
On the opposite, the second term, $\mathcal{L}_{Val}$, assesses the quality of the generative process to produce effective hijacking samples, as it computes the loss of model $h_\mathfrak{c}$ over $S_{gen}$. It is therefore computed at HVAE training time. 
\par
Ideally, we intend to reward higher $Hj_{cost}$, as higher values imply higher losses toward those models with lower hijack metrics.
Therefore, in our loss function, we aim to maximize this value.  
During the training of the HVAE, by minimizing Equation~\ref{lossMHVAE}, we work toward:
\begin{itemize}
    \item diminishing the reconstruction loss $\mathcal{L}_{\mathrm{rec}}$, so that generated samples can resemble the original operations;
    \item diminishing the $\mathcal{L}_{\mathrm{KLD}}$ for obtaining a useful probability distribution;
    \item increasing the hijacking cost function $Hj_{cost}(\mathfrak{C})$. As the penalty value is fixed, by raising Equation~\ref{cost}, we aim at generating samples $\mathcal{S}_{gen}$, which increase the validation loss based on the magnitude of the penalty itself.
    Models with lower hijack metrics incur higher penalties, leading to increased validation loss on the generated samples. This ensures the samples are crafted to produce lower validation loss values for models with the highest hijack metrics.
    % Therefore, those models with lower hijack metrics will have higher penalties, which results in higher validation loss computed on the generated samples. This allows the creation of samples that, when used for evaluating the validation loss of a model, will return a lower value for the ones with the highest hijack metric.
\end{itemize}
% A graphical representation of how  $\mathcal{S}^{Val}_{pois}$ is generated, can be found in Figure~\ref{MHVAE}. 
% By training the HVAE with the objective function Equation~\ref{lossMHVAE}, it is possible to encode a distribution, that is unlike the input samples one -- usually learned by vanilla VAE -- as it governs the generation of samples such that, when injected in the validation set, they can provide a penalty on the validation loss of models at lower hijack metric.
% We report in Algorithm~\ref{alg.HVAE} the HVAE training procedure.
A graphical overview of $\mathcal{S}^{Val}_{pois}$ generation is shown in Figure~\ref{MHVAE}.
By training the HVAE with the objective function in Equation~\ref{lossMHVAE}, the model encodes a distribution distinct from the input samples’ usual one, enabling the generation of validation samples that penalize models with lower hijack metrics.
The HVAE training procedure is detailed in Algorithm~\ref{alg.HVAE}.

% \begin{figure*}[!htbp]
%     \footnotesize
%     \centering
%     \includesvg[width=.775\textwidth]{figures/MHVAE-v2.drawio}
%     \caption{Schematic representation of the generation process of $\mathcal{S}^{Val}_{pois}$. For simplicity, we reported samples from the MNIST dataset~\cite{lecun2010mnist}.}
%     % \caption{Schematic representation of $\mathcal{S}^{Val}_{pois}$ generation using MNIST samples~\cite{lecun2010mnist} for simplicity.}
%     \label{MHVAE}
% \end{figure*}

\begin{figure*}[!htbp] %% ARXIV
    \footnotesize
    \centering
    \includesvg[width=.775\textwidth]{figures/MHVAE-v2.drawio}
    \caption{Schematic representation of the generation process of $\mathcal{S}^{Val}_{pois}$. For simplicity, we reported samples from the MNIST dataset~\cite{lecun2010mnist}.}
    % \caption{Schematic representation of $\mathcal{S}^{Val}_{pois}$ generation using MNIST samples~\cite{lecun2010mnist} for simplicity.}
    \label{MHVAE}
\end{figure*}

\begin{algorithm}[H]
\footnotesize
    \caption{Hijack VAE Training Algorithm}
    \begin{algorithmic}[1]
        \State \textbf{Input:} HVAE model with random weights, training data $\mathcal{S}$, $\alpha_{\mathfrak{C}}$, $h_{\mathfrak{C}}$, number of epochs $epochs$
        \State \textbf{Output:} Trained HVAE model
        \For{$e \gets 1$ to $epochs$}
            \For{$\bm{x}$, $y$ in $\mathcal{S}$}  % are batches
                \State $\hat{\bm{x}} \gets $ HVAE.decode(HVAE.encode($\bm{x}$))  % reconstruct input
                \State rec\_loss $\gets \mathcal{L}_{\mathrm{rec}}(\bm{x}, \hat{\bm{x}})$  % reconstruction loss
                \State kl\_loss $\gets \mathcal{L}_{\mathrm{KLD}}(\mathrm{HVAE})$  % KLD loss
                \State $\hat{\bm{x}}_{gen} \gets$ HVAE.decode(gaussian\_noise)  % generate samples from randomly sampled noise
                \State generated\_val\_loss $\gets \mathcal{L}_{Val}(h_{\mathfrak{C}}, \hat{\bm{x}}_{gen})$  % validation loss of all knowm models on the generated samples
                \State hijack\_cost $\gets Hj_{cost}(\alpha_{\mathfrak{C}}, \mathrm{generated\_val\_loss})$  % compute hijack cost using the hijack cost penalty & the loss of the generated samples
                \State total\_loss $ \gets(\mathrm{rec\_loss + kl\_loss - hijack\_cost})^2 $  % obtain the total loss
                \State HVAE.backward\_propagation\_step(total\_loss)  % update weights
            \EndFor
        \EndFor
        \State \textbf{return} HVAE
    \end{algorithmic}
    \label{alg.HVAE}
\end{algorithm}

\subsection{Hijack Metric}
\label{ssec.hm-theory}
Generally, the purpose of a hijack metric $m$ is to produce damage to the target victim. 
\new{
We now introduce four distinct hijack metrics that impact an ML system in three different ways, i.e., generalization capabilities (Section~\ref{subsub:generalization}), latency (Section~\ref{subsub:latency}), and energy consumption (Section~\ref{subsub:energy}).
}
Note that MOSHI is not limited to such metrics, and future investigations might define different attack objectives. 

% \subsubsection{Weaken the Generalization Capabilities}
\subsubsection{\new{Generalization Capability Attack}}
\label{subsub:generalization}
This first intuitive hijack metric objective is to impact the victim model overall performance. 
Here, the objective of the attack under this metric is to choose a model that less generalizes to unseen data (e.g., test set), and therefore the result of an underfitting or overfitting training.
%Therefore, this case can be reconducted to the more traditional 
Therefore, this case can be considered a form of the more traditional \textit{model poisoning attack}~\cite{tian2022comprehensive}.
The metric $m$ -- that we named \textit{Generalization Metric} -- can simply compute the loss of a target model on an unseen dataset (\textit{e.g., validation set}). 
%
\subsubsection{Latency Attack}
\label{subsub:latency}
Increased latency in ML predictions can significantly impact the performance and usability of ML systems.
Higher latency leads to delayed responses, which can degrade user experience, particularly in real-time applications such as autonomous driving, financial trading, and interactive systems. Additionally, increased latency can hinder the efficiency of decision-making processes, as timely data processing is crucial for accurate and effective outcomes. This delay can also exacerbate the accumulation of errors, potentially compromising the reliability and accuracy of the ML model's predictions.
Therefore, an attacker might aim to induce the model selection to peak a model that results in slower predictions, on average, when deployed. 
The function $m$ -- that we named \textit{Latency Metric} -- can be designed by observing the time required by a target model to predict a set of unseen datasets (\textit{e.g., validation set}). 
%
\subsubsection{Energy Consumption Attack}
\label{subsub:energy}
Similarly to what is discussed in the motivation of the latency attack, increasing the overall energy consumption might lead to resource exhaustion. 
We inspire this metric based on the \textit{sponge attack}~\cite{shumailov2021sponge}. 
In our work, we consider two distinct metrics that measure energy consumption. 
\begin{itemize}
    \item \textit{Energy Consumption}: an estimation of the energy consumption of the model utilization that can be obtained through the OS energy consumption hosting such model. 
    \item \textit{$\ell_0$ norm}: the $\ell_0$ norm of the activations of the neurons in the network, obtained by summing the non-zero activations of each ReLU Layer in the model when it is processing a sample $\bm{x}$, then computing the mean for all samples $\bm{x} \in \mathcal{X}$. 
\end{itemize}
We opt to include this metric as \cite{cina2022energy} showed, there exists a strong link between the $\ell_0$ norm of a model and its energy consumption.
For instance, we report in Figure~\ref{l0_energy} the observed correlation between these two metrics in our experimental setting (which we will describe in the upcoming section).  

\begin{figure}[!htbp]
    \footnotesize
    \centering
    \includesvg[width = .8\linewidth]{figures/MNIST-normalized-energy-l0-v2}
    % \vspace{-15pt}
    \caption[Histogram comparing $\ell_0$ norm and energy consumption per layer.]{Histogram comparing $\ell_0$ norm and energy consumption per layer on FFNNs from 1 to 10 layers of 32 neurons, trained on MNIST dataset with a learning rate of 0.001.}
    \label{l0_energy}
\end{figure}


\subsection{White-box vs Black-box scenarios}
The HVAE requires knowledge about the target models, as described in Equation~\ref{cost} in the $Hj_{cost}$. Models in the grid are utilized for measuring their performance with the hijack metric and for understanding the quality of the dataset $S_{gen}$ produced by HVAE.
As we previously anticipated, in our work we consider a white-box and black-box case study. In the former, we assume the attacker has access to the exact models of the model grid. In the latter, the attacker has no such knowledge.
However, we assume that the attacker has knowledge about both training and validation sets. We can therefore leverage the \textit{adversarial transferability} of attacks. 
\par
Adversarial transferability in AML refers to the phenomenon where adversarial examples crafted to deceive one ML model can also deceive other models, even if they have different architectures or were trained on different datasets~\cite{demontis2019adversarial, alecci2023your}. This property is significant because it highlights the vulnerability of ML systems to attacks that are not specifically tailored to them, thereby posing a broader security risk.

\paragraph{Observation: Temporal Locality}
The main idea behind database drafting is that some tokens are easy to retrieve from the database because they exhibit temporal locality—meaning they tend to be repeated within or across the generation processes. 
However, note that not all draft token sequences share the same level of temporal locality during generation. 
We analyze the pattern of unique 4-grams during 100 text generations on Spec-Bench~\cite{Spec_Survey}, as shown in Figure~\ref{fig:generation}. The results reveal that certain 4-grams are frequently repeated and exhibit varying locality levels.
Specifically, the blue dots and the right small plot in Figure~\ref{fig:generation} illustrate local redundancy, where the same 4-gram appears multiple times within a single generation step. This reflects high temporal locality within a single generation rather than across multiple generations. In contrast, the red dots in Figure~\ref{fig:generation} highlight a pattern where the model repeatedly generates the same 4-grams at different stages of the generation process, illustrating its tendency to reuse familiar sequences over time.
Additionally, the lower plot of Figure~\ref{fig:generation} presents the frequency study of sampled red and blue dots, demonstrating that some tokens exhibit high temporal locality within a specific context, while others maintain consistent locality across generation processes.
Therefore, given the varying temporal locality of tokens throughout the generation process, drafting steps should prioritize tokens with higher temporal locality over others.

% For example, when an LLM solves a math problem like, “The vertices of a triangle are at points (0, 0), (-1, 1), and (3, 3). What is the area of the triangle?”, the coordinates are frequently repeated.
% Next, frequently generated phrases by LLMs, such as “as an AI assistant,” show moderate locality, as they often appear across various generation processes for LLM-generated texts. 
% Finally, grammatical patterns and universal phrases commonly used by humans have the lowest locality, as they are statistically frequent across all types of texts yet do not constantly occur in each generation process.

\paragraph{Database Design}  
 Based on the temporal locality of draft token candidates, we design three types of databases to categorize them. 
\textbf{1) Context-dependent DB} (\(\mathcal{D}_c\)) contains tokens highly relevant to the specific context of the generation process, such as the blue dots in the Figure~\ref{fig:generation}. 
This includes tokens from the input prompt, tokens generated through parallel decoding, tokens discarded during the generation process, and others that are highly relevant to a given context.  
\(\mathcal{D}_c\) is lookup table with the prefix tokens, \(\bm{x}_{1:l}\), as the key and the subsequent tokens, \(\bm{x}_{l:l+m}\), as the value.
Also, \(\mathcal{D}_c\) is consistently updated during each forward step and initialized when the following generation process is started. 
The database follows the Least Recently Used (LRU) policy for draft sequence updates.
\textbf{2) Model-dependent DB} (\(\mathcal{D}_m\)) stores tokens frequently generated by LLM regardless of context, as represented by the red dots in Figure~\ref{fig:generation}.
Top-$k$ frequently generated token sequences, $\bm{x}_{1:l+m}$, are sampled from the model-generated texts, with $\bm{x}_{1:l}$ as the key and $\bm{x}_{l+1:l+m}$ as the value.
For \(\mathcal{D}_c\) and \(\mathcal{D}_m\), the maximum size of values for a single key is the same as the maximum draft set size \(N\). 
\textbf{3) Statistics-dependent DB} (\(\mathcal{D}_s\)) draws its tokens from large text corpora to capture universal phrases commonly used in the language. 
Although these tokens are frequent, they occur less consistently across processes than those in \(\mathcal{D}_m\).
To efficiently retrieve the sequence from a large corpus, we utilize a suffix array~\cite{suffix_array} following the implementation of~\citet{REST}.
Implementation details are in \S\ref{sec:experiement}.


Our database design yields three distinct advantages. First, it integrates diverse sources into multiple databases, enabling us to leverage each source’s strengths for robust acceleration across various tasks. Then, each database’s size decreases as the tokens’ temporal locality increases since tokens with higher locality are rarer, providing an opportunity to optimize drafting latency. Finally, the design is \textit{plug-and-play}, easily integrating additional token sources by assigning them to the appropriate database based on their temporal locality.

\setlength{\textfloatsep}{1em}% Remove \textfloatsep

\begin{algorithm}[t!]
\caption{\small Decoding Process with Hierarchy Drafting}\label{alg:HD_process}
\small
\begin{algorithmic}[1]
\Require Target LLM $\mathcal{M}_p$, databases $(\mathcal{D}_c, \mathcal{D}_m, \mathcal{D}_s)$, input text sequence $\bm{x}_{\le t}$, target sequence length $T$, the size of prefix tokens $l$, the size of draft token sequence $m$, the size of draft set $N$;
\State $n \leftarrow t$\;
\While{$n < T$ and \texttt{[EOS]} $ \notin \bm{x}_{1:n}$}
    \State \textcolor{blue}{// \textit{Drafting Step: Hierarchical access to three databases until the size of the draft set $\bm{\tilde{X}}$ is $N$.}}
    \State $\bm{\tilde{X}} \leftarrow \texttt{Ret}(\bm{x}_{n-l:n};\mathcal{D}_c)$
    \If{$|\bm{\tilde{X}}| < N$}
        \State $\bm{\tilde{X}} \leftarrow \bm{\tilde{X}} \cup \texttt{Ret}(\bm{x}_{n-l:n};\mathcal{D}_m)$
    \EndIf 
    \If{$|\bm{\tilde{X}}| < N$}
        \State $\bm{\tilde{X}} \leftarrow \bm{\tilde{X}} \cup \texttt{Ret}(\bm{x}_{n-l:n};\mathcal{D}_s)$
    \EndIf 
    \State \textcolor{blue}{// \textit{Verification Step: Verify the draft token sequence in $\bm{\tilde{X}}$ and generate additional tokens for updating $\mathcal{D}_c$.}}
    \State $\bm{x}_{n:n+i}, \bm{\hat{x}} \sim_i \mathcal{M}_p(\bm{x}_{\le n}, \bm{\tilde{X}})$
    \State $\mathcal{D}_c \leftarrow \text{Update}(\mathcal{D}_c, \bm{\hat{x}})$
    % \If{\texttt{[EOS]} in $\bm{x}_{t:t+i}$}
    %     \State BREAK
    % \EndIf
    \State $n \gets n+i$
\EndWhile
\end{algorithmic}
\end{algorithm}
% \vspace{0-}

\paragraph{Hierarchical Access}
Using the three databases designed with the temporal locality in mind, we retrieve draft token sequence \(\bm{\tilde{x}}_{1:m}\) for the given previous input \(\bm{x}_{t-l:t}\).
Database access order is based on the degree of temporal locality within the current generation process; thereby, the access starts with \(\mathcal{D}_c\).
Access then proceeds to \(\mathcal{D}_m\), which has high locality across generations, and finally \(\mathcal{D}_s\), with moderate locality across generations, until draft set \(\bm{\tilde{X}}\) accumulates a sufficient number of candidates as pre-defined hyperparameter \(N\).
These accesses leverage the locality of the draft token sequence to enhance drafting accuracy and minimize latency overhead, preserving the benefits of drafting.

\paragraph{Decoding Process}
We introduce the inference process of speculative decoding with our proposed method, HD. 
% First, for a given previous input \(\bm{x}_{t-l, t}\), the set of draft token \(\bm{\tilde{X}}\) are acquired from the three databases with hierarchical access. 
First, for a given previous input \(\bm{x}_{t-l, t}\), we acquire the set of draft token \(\bm{\tilde{X}}\) from the three databases with hierarchical access. 
% Then, the target LLM \(\mathcal{M}_p\) verifies the draft token sequences simultaneously generating the additional tokens \(\bm{\hat{x}}\) for updating context-dependent DB either through parallel decoding~\cite{ParallelDecoding, LAD} or by recycling wasted tokens~\cite{trashintotreasure}. 
Then, the target LLM \(\mathcal{M}_p\) verifies the draft token sequences while simultaneously generating the additional tokens \(\bm{\hat{x}}\). 
These tokens are used to update the context-dependent DB either through parallel decoding~\cite{ParallelDecoding, LAD} or by recycling wasted tokens~\cite{trashintotreasure}. 
These processes are repeated iteratively until either the \texttt{[EOS]} token is generated or the sequence reaches the pre-defined maximum length \(T\).
Details of the decoding are depicted in Algorithm~\ref{alg:HD_process}.

