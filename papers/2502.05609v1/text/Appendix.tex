\section*{Appendix}

In the Appendix, we introduce more related work and details on how the experiments were implemented, along with supplementary results and analysis not included in the main text.

\section{More Related Work}

\paragraph{LLM Inference Acceleration}
Large Language Models (LLMs)~\cite{GPT3, Llama2} have significantly advanced Natural Language Processing (NLP) and are widely used in real-world applications via APIs, highlighting the importance of real-time human-LLM interactions. However, the slow inference speed of LLMs presents the primary bottleneck in deploying these models as practical services. This issue is predominantly memory-bound, with latency arising from the autoregressive decoding process~\cite{latencylagsbandwidth, ScalingTransformerInference, MemoryWall}. During the inference process, each token generation requires transferring model parameters and key-value caches from global memory to the accelerator’s cache, consuming substantial overhead. Given the limited progress in memory bandwidth, several algorithmic approaches have been developed to mitigate these challenges, including model parameter quantization~\cite{ZeroQuant, GPTQ, SqueezeLLM}, memory allocation optimization~\cite{MQAttn, H2O, PagedAttn}, and Speculative Decoding~\cite{BlockWise, SpecDecoding, SpecSampling}.
While other approaches often require hardware modifications or lead to side effects like performance degradation, speculative decoding has gained particular attention for offering a fully algorithmic solution with minimal drawbacks.


\section{More Implementation Details}\label{app:impl}

\paragraph{Implementation Details of HD}  First, we introduce more details of our multiple databases. For \(\mathcal{D}_c\) and \(\mathcal{D}_m\), the prefix token length used as the key is 1, while the given previous token length is \(l\). This mismatch is because increasing the prefix length often leads to draft misses (i.e., keys not found in the database) due to the limited scale of token sources constructing \(\mathcal{D}_c\) and \(\mathcal{D}_m\). Also, for \(\mathcal{D}_s\), the retrieval process is repeated by reducing the previous token length until draft token sequences are found or the token length reaches 0, following implementation of REST~\cite{REST}.
As mentioned in the main text, \(\mathcal{D}_c\) and \(\mathcal{D}_m\) are lookup tables implemented by the Python Dictionary class. In contrast, \(\mathcal{D}_s\) is implemented by DraftRetriever\footnote{\scriptsize{\url{https://github.com/FasterDecoding/REST/tree/main/DraftRetriever}}}, a Python Library based on suffix arrays, proposed by REST for handling a large text corpus with minimal overhead.
The average number of values (i.e., draft token sequences) stored in the database is 1K, 100K, and 200M for \(\mathcal{D}_c\), \(\mathcal{D}_m\), and \(\mathcal{D}_s\), respectively.

We now explain the detailed implementation of the decoding process with HD. Our method is primarily based on the implementation of LADE\footnote{\scriptsize \url{https://github.com/hao-ai-lab/LookaheadDecoding}}~\cite{LAD}, as LADE employs a similar database to $\mathcal{D}_c$ and updates tokens generated through parallel decoding. We extend this by incorporating a hierarchical framework; thereby, the verification step in HD follows LADE’s $n$-gram verification process. However, we emphasize that HD can be extended to other database drafting methods by integrating a hierarchical drafting framework with multiple databases into their implementation.

\paragraph{Prompting Format}
We use the chat format for text input, provided by the Python library FastChat\footnote{\scriptsize \url{https://github.com/lm-sys/FastChat}}, which consists of system, user, and model components. Additionally, we adopted the system message for Vicuna from FastChat and for Llama-2 from the official repository\footnote{\scriptsize \url{https://github.com/meta-llama/llama/blob/main/example_chat_completion.py}}.

\begin{figure}[t]
\centering
\includegraphics[width=0.9\columnwidth]{figure/tokens.jpg}
\vspace{-.8em}
\caption{\small Correlation between generated token length and elapsed latency using Llama-2-7b-chat on Spec-Bench. Dots in the plot represent acceleration results for individual generations, while the lines show the linear regression results for each method.}
\label{fig:tokens}
\vspace{-.5em}
\end{figure}

\paragraph{Details of Dataset}

We exploit Spec-Bench~\cite{Spec_Survey}, collecting data from representative datasets for each task.
Specifically. the collected datasets are MT-bench~\cite{vicuna} for Multi-turn Conversation, WMT14 DE-EN~\cite{translation} for Translation, CNN/Daily Mail~\cite{summarization} for Summarization, Natural Question~\cite{QA} for Question Answering, GSM8K~\cite{math} for Math Reasoning, DPR~\cite{RAG} for Retrieval Augmented Generation.

\section{Additional Results}\label{app:additional_result}

\subsection{Impact of Generated Token Length} 

Since speculative decoding shortens generation steps and reduces the correlation between generated token length and elapsed latency, we explore this relationship to showcase the effectiveness of HD.
As depicted in Figure~\ref{fig:tokens}, all database drafting methods successfully lower the slope compared to autoregressive decoding. 
Among them, HD demonstrates the shallowest slope, highlighting its effectiveness even for long text generation. 
Furthermore, despite variations in drafting latency and accuracy among other methods, as shown in Table~\ref{tab:main}, they exhibit similar slopes, underscoring the importance of balancing latency and drafting accuracy to achieve optimal acceleration for database drafting methods.


\subsection{Impact of Temperature}
% \begin{table}[h]
% \begin{tabular}{lcc}
% \toprule
% \small
% \textbf{Category} & \textbf{GPT-4o T=0.95} & \textbf{GPT-4o T=1.0} \\
% \midrule
% \texttt{cry}    & 3.85 & 5.77 \\
% \texttt{for} & 20.0 & 20.0 \\
% \texttt{pwn}       & 5.13 & \textbf{7.69} \\
% \texttt{rev}       & 11.76 & \textbf{13.73} \\
% \texttt{web}       &   10.53  & 10.53 \\
% \texttt{misc}      & 16.67 & 16.67 \\
% \texttt{\textbf{Avg.}} & 9.5 & \textbf{11.0} \\
% \bottomrule
% \end{tabular}
% \caption{NYU CTF benchmark success rate (\%) of GPT-4o on temperature 0.95 and temperature 1.0, higher temperature means the model will be more creative with higher randomness.}
% \label{tab:temperature}
% \end{table}

\begin{table}[htbp]
\centering
% \small
\caption{GPT 4o \textit{\% solved} for temperatures 0.95 and 1.0.}% Higher temperature produces more creativity and randomness in LLM responses.}
\label{tab:temperature}
\begin{tabular}{lccccccc}
\toprule
& crypto & foren. & pwn & rev & web & misc & \textbf{total} \\
\midrule
$T=1.0$ & 5.8 & 13.3 & 7.7 & 13.7 & 10.5 & 16.7 & 10.5 \\
$T=0.95$ & 3.8 & 13.3 & 5.1 & 11.8 & 10.5 & 16.7 & 9.0 \\
\bottomrule
\end{tabular}
\end{table}
As temperature sampling is commonly used to increase the diversity of text generation, we analyze its impact on the database drafting methods, as shown on the left of Figure~\ref{fig:temperature}.  
Other drafting methods maintain higher speeds than autoregressive decoding across all temperatures. LADE slightly decreases as temperature increases, whereas REST remains consistent.
% As updating \(\mathcal{D}_c\) with sampled outputs, similar to LADE, leads to token mismatches after applying higher temperatures, HD shows a slight reduction in speed at higher temperatures.
However, updating \(\mathcal{D}_c\)  with sampled outputs, similar to LADE, results in token mismatches at higher temperatures, causing HD to show a slight reduction in speed.
Nevertheless, our proposed method, HD, outperforms all others across the entire temperature range, maintaining a high decoding speed of over 70 tokens per second.
% Note that our proposed method, HD, outperforms all others across the entire temperature range, maintaining the highest decoding speed despite a slight reduction in speed at higher temperatures.
% The slight decrease is due to sampling outputs for updating $\mathcal{D}_c$, similar to LADE, which leads to token mismatches after applying higher temperatures.
These results demonstrate that HD remains a suitable solution even with a sampling strategy, enhancing its potential in real-world scenarios.
% % \begin{table}[h]
% \begin{tabular}{lcc}
% \toprule
% \small
% \textbf{Category} & \textbf{GPT-4o T=0.95} & \textbf{GPT-4o T=1.0} \\
% \midrule
% \texttt{cry}    & 3.85 & 5.77 \\
% \texttt{for} & 20.0 & 20.0 \\
% \texttt{pwn}       & 5.13 & \textbf{7.69} \\
% \texttt{rev}       & 11.76 & \textbf{13.73} \\
% \texttt{web}       &   10.53  & 10.53 \\
% \texttt{misc}      & 16.67 & 16.67 \\
% \texttt{\textbf{Avg.}} & 9.5 & \textbf{11.0} \\
% \bottomrule
% \end{tabular}
% \caption{NYU CTF benchmark success rate (\%) of GPT-4o on temperature 0.95 and temperature 1.0, higher temperature means the model will be more creative with higher randomness.}
% \label{tab:temperature}
% \end{table}

\begin{table}[htbp]
\centering
% \small
\caption{GPT 4o \textit{\% solved} for temperatures 0.95 and 1.0.}% Higher temperature produces more creativity and randomness in LLM responses.}
\label{tab:temperature}
\begin{tabular}{lccccccc}
\toprule
& crypto & foren. & pwn & rev & web & misc & \textbf{total} \\
\midrule
$T=1.0$ & 5.8 & 13.3 & 7.7 & 13.7 & 10.5 & 16.7 & 10.5 \\
$T=0.95$ & 3.8 & 13.3 & 5.1 & 11.8 & 10.5 & 16.7 & 9.0 \\
\bottomrule
\end{tabular}
\end{table}

\subsection{Database Access Statistics}
\begin{figure}[h]
\centering
\includegraphics[width=0.9\columnwidth]{figure/access_pattern.jpg}
\vspace{-.8em}
\caption{\small Analysis of database access using Llama-2-7b-chat on Spec-Bench. The total bar height represents the overall database access ratio, with each color indicating draft failure, draft success, and draft \& verify success.}
\label{fig:access_pattern}
\vspace{-.5em}
\end{figure}


We conducted a breakdown study of database access patterns, as shown in Figure~\ref{fig:access_pattern}, focusing on draft \& verify success and its relationship to the temporal locality of draft tokens. The context-dependent database (\(\mathcal{D}_c\)) achieves the highest verify success rate (32.3\%), highlighting its strong alignment with recent input tokens due to its ability to capture local context, though its limited scale results in higher draft failure rates. The model-dependent database (\(\mathcal{D}_m\)) has a lower verify success rate (15.5\%), yet its broader scope allows for more frequent draft success, albeit less aligned with the immediate context. Finally, the statistics-dependent database (\(\mathcal{D}_s\)) exhibits the lowest verify success (6.5\%) but benefits from its vast coverage, which makes it less sensitive to temporal locality. These patterns suggest that \(\mathcal{D}_c\) is critical for capturing temporally localized tokens, while \(\mathcal{D}_m\) and \(\mathcal{D}_s\) complement it by providing a more general, though not as closely aligned to the LLM generation.

\subsection{Ablation Study}

% \begin{table}[!t]
% \centering
% \scalebox{0.68}{
%     \begin{tabular}{ll cccc}
%       \toprule
%       & \multicolumn{4}{c}{\textbf{Intellipro Dataset}}\\
%       & \multicolumn{2}{c}{Rank Resume} & \multicolumn{2}{c}{Rank Job} \\
%       \cmidrule(lr){2-3} \cmidrule(lr){4-5} 
%       \textbf{Method}
%       &  Recall@100 & nDCG@100 & Recall@10 & nDCG@10 \\
%       \midrule
%       \confitold{}
%       & 71.28 &34.79 &76.50 &52.57 
%       \\
%       \cmidrule{2-5}
%       \confitsimple{}
%     & 82.53 &48.17
%        & 85.58 &64.91
     
%        \\
%        +\RunnerUpMiningShort{}
%     &85.43 &50.99 &91.38 &71.34 
%       \\
%       +\HyReShort
%         &- & -
%        &-&-\\
       
%       \bottomrule

%     \end{tabular}
%   }
% \caption{Ablation studies using Jina-v2-base as the encoder. ``\confitsimple{}'' refers using a simplified encoder architecture. \framework{} trains \confitsimple{} with \RunnerUpMiningShort{} and \HyReShort{}.}
% \label{tbl:ablation}
% \end{table}
\begin{table*}[!t]
\centering
\scalebox{0.75}{
    \begin{tabular}{l cccc cccc}
      \toprule
      & \multicolumn{4}{c}{\textbf{Recruiting Dataset}}
      & \multicolumn{4}{c}{\textbf{AliYun Dataset}}\\
      & \multicolumn{2}{c}{Rank Resume} & \multicolumn{2}{c}{Rank Job} 
      & \multicolumn{2}{c}{Rank Resume} & \multicolumn{2}{c}{Rank Job}\\
      \cmidrule(lr){2-3} \cmidrule(lr){4-5} 
      \cmidrule(lr){6-7} \cmidrule(lr){8-9} 
      \textbf{Method}
      & Recall@100 & nDCG@100 & Recall@10 & nDCG@10
      & Recall@100 & nDCG@100 & Recall@10 & nDCG@10\\
      \midrule
      \confitold{}
      & 71.28 & 34.79 & 76.50 & 52.57 
      & 87.81 & 65.06 & 72.39 & 56.12
      \\
      \cmidrule{2-9}
      \confitsimple{}
      & 82.53 & 48.17 & 85.58 & 64.91
      & 94.90&78.40 & 78.70& 65.45
       \\
      +\HyReShort{}
       &85.28 & 49.50
       &90.25 & 70.22
       & 96.62&81.99 & \textbf{81.16}& 67.63
       \\
      +\RunnerUpMiningShort{}
       % & 85.14& 49.82
       % &90.75&72.51
       & \textbf{86.13}&\textbf{51.90} & \textbf{94.25}&\textbf{73.32}
       & \textbf{97.07}&\textbf{83.11} & 80.49& \textbf{68.02}
       \\
   %     +\RunnerUpMiningShort{}
   %    & 85.43 & 50.99 & 91.38 & 71.34 
   %    & 96.24 & 82.95 & 80.12 & 66.96
   %    \\
   %    +\HyReShort{} old
   %     &85.28 & 49.50
   %     &90.25 & 70.22
   %     & 96.62&81.99 & 81.16& 67.63
   %     \\
   % +\HyReShort{} 
   %     % & 85.14& 49.82
   %     % &90.75&72.51
   %     & 86.83&51.77 &92.00 &72.04
   %     & 97.07&83.11 & 80.49& 68.02
   %     \\
      \bottomrule

    \end{tabular}
  }
\caption{\framework{} ablation studies. ``\confitsimple{}'' refers using a simplified encoder architecture. \framework{} trains \confitsimple{} with \RunnerUpMiningShort{} and \HyReShort{}. We use Jina-v2-base as the encoder due to its better performance.
}
\label{tbl:ablation}
\end{table*}
% % \begin{table}[!t]
% \centering
% \scalebox{0.68}{
%     \begin{tabular}{ll cccc}
%       \toprule
%       & \multicolumn{4}{c}{\textbf{Intellipro Dataset}}\\
%       & \multicolumn{2}{c}{Rank Resume} & \multicolumn{2}{c}{Rank Job} \\
%       \cmidrule(lr){2-3} \cmidrule(lr){4-5} 
%       \textbf{Method}
%       &  Recall@100 & nDCG@100 & Recall@10 & nDCG@10 \\
%       \midrule
%       \confitold{}
%       & 71.28 &34.79 &76.50 &52.57 
%       \\
%       \cmidrule{2-5}
%       \confitsimple{}
%     & 82.53 &48.17
%        & 85.58 &64.91
     
%        \\
%        +\RunnerUpMiningShort{}
%     &85.43 &50.99 &91.38 &71.34 
%       \\
%       +\HyReShort
%         &- & -
%        &-&-\\
       
%       \bottomrule

%     \end{tabular}
%   }
% \caption{Ablation studies using Jina-v2-base as the encoder. ``\confitsimple{}'' refers using a simplified encoder architecture. \framework{} trains \confitsimple{} with \RunnerUpMiningShort{} and \HyReShort{}.}
% \label{tbl:ablation}
% \end{table}
\begin{table*}[!t]
\centering
\scalebox{0.75}{
    \begin{tabular}{l cccc cccc}
      \toprule
      & \multicolumn{4}{c}{\textbf{Recruiting Dataset}}
      & \multicolumn{4}{c}{\textbf{AliYun Dataset}}\\
      & \multicolumn{2}{c}{Rank Resume} & \multicolumn{2}{c}{Rank Job} 
      & \multicolumn{2}{c}{Rank Resume} & \multicolumn{2}{c}{Rank Job}\\
      \cmidrule(lr){2-3} \cmidrule(lr){4-5} 
      \cmidrule(lr){6-7} \cmidrule(lr){8-9} 
      \textbf{Method}
      & Recall@100 & nDCG@100 & Recall@10 & nDCG@10
      & Recall@100 & nDCG@100 & Recall@10 & nDCG@10\\
      \midrule
      \confitold{}
      & 71.28 & 34.79 & 76.50 & 52.57 
      & 87.81 & 65.06 & 72.39 & 56.12
      \\
      \cmidrule{2-9}
      \confitsimple{}
      & 82.53 & 48.17 & 85.58 & 64.91
      & 94.90&78.40 & 78.70& 65.45
       \\
      +\HyReShort{}
       &85.28 & 49.50
       &90.25 & 70.22
       & 96.62&81.99 & \textbf{81.16}& 67.63
       \\
      +\RunnerUpMiningShort{}
       % & 85.14& 49.82
       % &90.75&72.51
       & \textbf{86.13}&\textbf{51.90} & \textbf{94.25}&\textbf{73.32}
       & \textbf{97.07}&\textbf{83.11} & 80.49& \textbf{68.02}
       \\
   %     +\RunnerUpMiningShort{}
   %    & 85.43 & 50.99 & 91.38 & 71.34 
   %    & 96.24 & 82.95 & 80.12 & 66.96
   %    \\
   %    +\HyReShort{} old
   %     &85.28 & 49.50
   %     &90.25 & 70.22
   %     & 96.62&81.99 & 81.16& 67.63
   %     \\
   % +\HyReShort{} 
   %     % & 85.14& 49.82
   %     % &90.75&72.51
   %     & 86.83&51.77 &92.00 &72.04
   %     & 97.07&83.11 & 80.49& 68.02
   %     \\
      \bottomrule

    \end{tabular}
  }
\caption{\framework{} ablation studies. ``\confitsimple{}'' refers using a simplified encoder architecture. \framework{} trains \confitsimple{} with \RunnerUpMiningShort{} and \HyReShort{}. We use Jina-v2-base as the encoder due to its better performance.
}
\label{tbl:ablation}
\end{table*}

We conducted an ablation study on the databases used in HD, as shown on the left side of Table~\ref{tab:ablation}. Using only a single database results in an acceptance ratio below 60\%, with a significant increase in draft failures, notably when $\mathcal{D}_s$ is excluded. Incorporating two databases improves the acceptance ratio and reduces draft failures, but it still underperforms compared to all three databases. These findings highlight the importance of combining multiple databases to improve token acceptance, leading to more robust and efficient performance.

Additionally, we observe the need to balance both the acceptance ratio and drafting latency for better speedup. For instance, using the largest database $\mathcal{D}_c$ alone increases the acceptance ratio but significantly raises drafting latency, resulting in the worst speedup—even slower than autoregressive decoding. Although the acceptance ratio with ($\mathcal{D}_c, \mathcal{D}_m$) is 5\% lower than with ($\mathcal{D}_c, \mathcal{D}_m, \mathcal{D}_s$), it achieves higher speedup due to trivial drafting latency under 1ms. However, the negative impact of drafting latency may be mitigated when applying HD to larger models. Since the longer generation latency of larger LLMs makes drafting latency negligible, a high acceptance ratio becomes crucial for acceleration. From these insights, our future research will focus on reducing drafting latency to be more effective regardless of model scale while maintaining an optimal acceptance ratio.

\subsection{Impact of Token Quality}

\begin{table}[ht]
    \caption{\small Experimental results on alternative options for model-dependent and statistics-dependent DB with Vicuna-7B.} \label{tab:db_ablation}
    \vspace{-1mm}
    \centering
    \small
    \renewcommand{\arraystretch}{1}
\resizebox{.9\columnwidth}{!}{
    \begin{tabular}{l  c c c}
    \toprule
        \textbf{$D_m - D_s$} & \textbf{\(\tau\)} & \textbf{D.L (ms)} & \textbf{Speedup}  \\ \midrule
        Vicuna Response - UltraChat (12GB) & 2.38 & 2.17 & 1.51x \\
        \noalign{\vskip 0.5ex}\cdashline{1-4}\noalign{\vskip 0.5ex}
        Vicuna Response - ShareGPT (465MB) & 2.26 & 0.28 & 1.57x \\
        Vicuna Response - TheStack (924MB) & 2.28 & 0.28 & 1.58x \\
        Llama Response - UltraChat (12GB) & 2.34 & 2.35 & 1.47x \\
       \bottomrule
    \end{tabular}
} 
\vspace{-.5em}
\end{table}

To further investigate the impact of token quality, we conducted experiments using alternative token sources, as presented in Table~\ref{tab:db_ablation}. For Vicuna-7b, we utilized Llama-7b responses as model-dependent databases and the code generation corpus, The Stack (924MB)~\cite{Stack}, as a statistics-dependent database. Also, we exploited a small version of the general generation corpus, ShareGPT (465MB)\footnote{\url{https://huggingface.co/datasets/Aeala/ShareGPT_Vicuna_unfiltered/blob/main/ShareGPT_2023.05.04v0_Wasteland_Edition.json}}. While token sources' quality influences acceleration, our proposed method demonstrates significant acceleration gains even when the token sources are not well-aligned with the target LLM or task. In addition, as discussed in the main text, drafting latency is a critical consideration, with smaller statistics-dependent databases demonstrating better acceleration despite a lower accepted length.



\subsection{Token Coverage}

\begin{figure}[ht]
\centering
\includegraphics[width=0.6\columnwidth]{figure/overlap.jpg}
\vspace{-.8em}
\caption{\small Venn diagram of accepted tokens using a single database.}
\label{fig:token_coverage}
\vspace{-.5em}
\end{figure}

Beyond the ablation study, we performed an in-depth analysis to verify the unique token distributions of each database by examining the accepted tokens when using only a single database, as shown on the right side of Figure~\ref{fig:token_coverage}. A significant portion of unique draft tokens was found, indicating that no single database can handle all tokens for drafting. Specifically, the context-dependent database (\(\mathcal{D}_c\)) contains 16\% unique accepted tokens not found in other databases, though its limited scale often leads to draft failures. These findings confirm that our designed databases complement each other by compensating for individual weaknesses in token distribution.


\subsection{Case Study}

Table~\ref{tab:case_study} presents a case study of text generation in question answering (QA) and retrieval-augmented generation (RAG) tasks using Llama-2-7b. The texts highlighted in green, red, and yellow are retrieved from \(\mathcal{D}_c\), \(\mathcal{D}_m\), and \(\mathcal{D}_s\), respectively. Red-highlighted texts are usually found in the middle of the input and are contextually relevant, often including numerical data or named entities. Green-highlighted texts typically appear at the beginning or end, repeated across generation processes; for example, the phrase ‘\textit{Thank you for your question!}’ is consistently retrieved from \(\mathcal{D}_m\). Yellow-highlighted texts are rarer and often capture grammatical patterns, such as articles or prepositions. Also, notable differences exist between tasks: green texts appear more frequently in the QA task, while red texts dominate in the RAG task.


\begin{figure}[htb]
\small
\begin{tcolorbox}[left=3pt,right=3pt,top=3pt,bottom=3pt,title=\textbf{Conversation History:}]
[human]: Craft an intriguing opening paragraph for a fictional short story. The story should involve a character who wakes up one morning to find that they can time travel.

...(Human-Bot Dialogue Turns)... \textcolor{blue}{(Topic: Time-Travel Fiction)}

[human]: Please describe the concept of machine learning. Could you elaborate on the differences between supervised, unsupervised, and reinforcement learning? Provide real-world examples of each.

...(Human-Bot Dialogue Turns)... \textcolor{blue}{(Topic: Machine learning Concepts and Types)}


[human]: Discuss antitrust laws and their impact on market competition. Compare the antitrust laws in US and China along with some case studies

...(Human-Bot Dialogue Turns)... \textcolor{blue}{(Topic: Antitrust Laws and Market Competition)}

[human]: The vertices of a triangle are at points (0, 0), (-1, 1), and (3, 3). What is the area of the triangle?

...(Human-Bot Dialogue Turns)... \textcolor{blue}{(Topic: Area and Circumcircle of Triangle)}

...(Human-Bot Dialogue Interaction on Other Topics)... 

\end{tcolorbox}
\begin{tcolorbox}[left=3pt,right=3pt,top=3pt,bottom=3pt,title=\textbf{Question:}]
What title reflects the thematic essence of the Doctor's experience with time travel as described in the story?
\end{tcolorbox}
\begin{tcolorbox}[left=3pt,right=3pt,top=3pt,bottom=3pt,title=\textbf{Ground-Truth Answer:}]
Awakening to Eternity: The Doctor's New Dawn
\end{tcolorbox}
\begin{tcolorbox}[left=3pt,right=3pt,top=3pt,bottom=3pt,title=\textbf{Retrieved Memory (Turn-Level):}]
[human]: Describe the process and write out the balanced chemical equation for the reaction that occurs when solid calcium carbonate reacts with hydrochloric acid to form aqueous calcium chloride, carbon dioxide, and water. What type of reaction is this, and what observations might indicate that the reaction is taking place?

[bot]: When solid calcium carbonate ($\text{CaCO}_{3}$) reacts with hydrochloric acid (HCl), the products formed are aqueous calcium chloride ($\text{CaCl}_{2}$), carbon dioxide ($\text{CO}_{2}$), and water ($\text{H}_{2}\text{O}$). The balanced chemical equation for this reaction is: ... \textcolor{blue}{(Topic: Calcium Carbonate and Hydrochloric Acid Reaction)}

\textcolor{brown}{
[human]: Write a title for your story.}

\textcolor{brown}{
[bot]: "Awakening to Eternity: The Doctor's New Dawn".
} \textcolor{blue}{(Topic: Time-Travel Fiction)}

[human]: Analyze the following customer reviews from different sources for three different smartphones - the latest iPhone, Samsung Galaxy, and Google Pixel - and provide an overall rating for each phone on a scale of 1 to 10. Consider the following complex and contradictory reviews: - TechRadar's review of the latest iPhone: The new iPhone is a stunning triumph of engineering that sets a new bar for smartphone performance and camera quality. However, the incremental design and high price mean it lacks the 'wow' factor of previous iPhones. Still, its power and intelligence are unrivaled. - CNET's review of the latest Samsung Galaxy: ... Return the answer as a JSON object with the overall ratings for each phone out of 10, to one decimal place.

[bot]: ```json { "iPhone": 8.5, "Samsung Galaxy": 8.0, "Google Pixel": 7.5 } ```

\textcolor{blue}{(Topic: Smartphone Review Ratings Analysis)}
\end{tcolorbox}
\begin{tcolorbox}[left=3pt,right=3pt,top=3pt,bottom=3pt,title=\textbf{Answer (Turn-Level Retrieval):}]
As an AI language model, I do not have access to the story you mentioned earlier. Please provide me with the story or the necessary details so that I can assist you better.

\textbf{\textcolor{red}{(The next part of this figure~\ref{fig: case_study_segment_vs_turn} will be continued on the next page.)}}
\end{tcolorbox}
\end{figure}

\begin{figure}[htb]
% \ContinuedFloat
\small
\begin{tcolorbox}[left=3pt,right=3pt,top=3pt,bottom=3pt,title=\textbf{Retrieved Memory (Ours):}]
[human]: Craft an intriguing opening paragraph for a fictional short story. The story should involve a character who wakes up one morning to find that they can time travel.

[bot]: In the hushed pre-dawn hours, when the world still slumbers under the cloak of night, Alex found themselves caught in the fragile boundary between sleep and wakefulness. ...

[human]: Summarize the story with three bullet points using only nouns and adjectives, without verbs.

[bot]: - Alex, pre-dawn awakening, time-travel discovery - Disorientation, vertigo, reality's malleability - Sunlight, new day, astonishing truth

[human]: Is there any famous fictional character that can time travel? If so, can you rewrite your story with him or her?

[bot]: - \textcolor{brown}{Doctor Who, a renowned time-traveling character from British television, is a perfect fit for this scenario.} In the silent embrace of early morning, where shadows merge with the fading night, the Doctor found themselves in the thin veil separating dreams from consciousness. It was here, amidst the soft whispers of time, that an unusual sensation enveloped them—a sense of time's fabric slipping through their fingers like grains of sand. As dawn's light broke through the darkness, painting the room in hues of gold and amber, the Doctor's eyes opened to what appeared to be an ordinary day. \textcolor{brown}{However, upon standing, a wave of dizziness overwhelmed them, the surroundings blurred, and in that moment of confusion, the Doctor realized a profound truth: they had not only awakened to a new dawn but had also stumbled upon the incredible ability to navigate the currents of time at will.}

\textcolor{brown}{[human]: Write a title for your story.}

\textcolor{brown}{[bot]: "Awakening to Eternity: The Doctor's New Dawn"} \textcolor{blue}{(Topic: Time-Travel Fiction)}
\end{tcolorbox}
\begin{tcolorbox}[left=3pt,right=3pt,top=3pt,bottom=3pt,title=\textbf{Answer (Ours):}]
The title "Awakening to Eternity: The Doctor's New Dawn" reflects the thematic essence of the Doctor's experience with time travel as described in the story. It captures the idea of a new beginning, a sense of eternity, and the awakening of a new power that will change the course of the Doctor's life forever.
\end{tcolorbox}
\caption{Case study comparing our \sysname\ with the \textit{turn-level} memory baseline. When constructing the memory bank at the \textit{turn-level}, the agent fails to provide a satisfactory response due to fragmentation: the retrieved dialogue turns can not capture a complete human-agent interaction related to the query, missing critical information highlighted in \textcolor{brown}{brown}. The dialogue topic is emphasized in \textcolor{blue}{blue}. Some context is omitted (replaced by ...) due to length constraints.}
\label{fig: case_study_segment_vs_turn}
\end{figure}

\newpage

\begin{figure}[htbp]
\small
\begin{tcolorbox}[left=3pt,right=3pt,top=3pt,bottom=3pt,title=\textbf{Conversation History:}]
[human]: Explain what's base rate fallacy and list five specific examples of ...

... (Human-Bot Dialogue Turns) ... \textcolor{blue}{(Topic: Base Rate Fallacy In Politics)}


[human]: How many integers are in the solution of the inequality $|x + 5| < 10$?

... (Human-Bot Dialogue Turns) ... \textcolor{blue}{(Topic: Number of Integers in Inequalities)}

[human]: How have the Alps and Rhine River influenced settlement and agriculture in Western Europe?

... (Human-Bot Dialogue Turns) ... \textcolor{blue}{(Topic: Alps and Rhine River Impact)}

[human]: In a survey conducted at a local high school, preferences for a new school color were measured: 58\% of students liked the color blue, 45\% preferred green, and 22\% liked both colors. If we randomly pick a student from the school, what's the probability that they would like neither blue nor green?

... (Human-Bot Dialogue Turns) ... \textcolor{blue}{(Topic: Probability of Liking Colors in Survey)}

[human]: Imagine you are participating in a race with a group of people. If you have just overtaken the second person, what's your current position? Where is the person you just overtook?

... (Human-Bot Dialogue Turns) ... \textcolor{blue}{(Topic: Overtaking Positions in Race)}
\end{tcolorbox}
\begin{tcolorbox}[left=3pt,right=3pt,top=3pt,bottom=3pt,title=\textbf{Question:}]
[human]: In the inequality question, how many positive integers are in the solution of the second inequality?
\end{tcolorbox}
\begin{tcolorbox}[left=3pt,right=3pt,top=3pt,bottom=3pt,title=\textbf{Ground-Truth Answer:}]
In the second inequality, $(|x + 10| < 5)$, the solution range is $(-15 < x < -5)$. Within this range, there are no positive integers, as all values of $(x)$ that satisfy the inequality are negative.
\end{tcolorbox}
\begin{tcolorbox}[left=3pt,right=3pt,top=3pt,bottom=3pt,title=\textbf{Retrieved Memory (Session-Level):}]
[human]: Explain what's base rate fallacy and list five specific examples of ...

... (Human-Bot Dialogue Turns) ... \textcolor{blue}{(Topic: Base Rate Fallacy In Politics)}

[human]: How many integers are in the solution of the inequality $|x + 5| < 10$?

[bot]: To solve the inequality $|x + 5| < 10$, ... 

\textcolor{brown}{
[human]: What about $|x + 10| < 5$?
}

\textcolor{brown}{
[bot]: To solve the inequality $|x + 10| < 5$, we ...} \textcolor{gray}{(``overwhelmed'' by other irrelevant, redundant content)}

... (Human-Bot Dialogue Turns) ...  \textcolor{blue}{(Topic: Number of Integers in Inequalities)}

% \textcolor{brown}{
% [human]: What about $|x + 10| < 5$
% }

% \textcolor{brown}{
% [bot]: To solve the inequality $|x + 10| < 5$, ...
% } \textcolor{blue}{(Topic: Number of Integers in Inequalities)}

[human]: How have the Alps and Rhine River influenced settlement and agriculture in Western Europe?

... (Human-Bot Dialogue Turns) ... \textcolor{blue}{(Topic: Alps and Rhine River Impact)}

[human]: In a survey conducted at a local high school, ...

... (Human-Bot Dialogue Turns) ... \textcolor{blue}{(Topic: Probability of Liking Colors in Survey)}
\end{tcolorbox}
\begin{tcolorbox}[left=3pt,right=3pt,top=3pt,bottom=3pt,title=\textbf{Answer (Session-Level Retrieval):}]
I apologize for the confusion earlier. To answer your question, the second inequality was not provided in our previous conversation. If you could provide the inequality, I would be happy to help you find the solution and the number of positive integers in the solution set.

\textbf{\textcolor{red}{(The next part of this figure~\ref{fig: case_study_segment_vs_session} will be continued on the next page.)}}
\end{tcolorbox}
\end{figure}
\begin{figure}[htbp]
% \ContinuedFloat
\small
\begin{tcolorbox}[left=3pt,right=3pt,top=3pt,bottom=3pt,title=\textbf{Retrieved Memory (Ours):}]
[human]: How many integers are in the solution of the inequality $|x + 5| < 10$?

[bot]: To solve the inequality $|x + 5| < 10$, we break it into two cases based on the definition of absolute value: 
1. $x + 5 < 10$ 
2. $-(x + 5) < 10$, which simplifies to $x + 5 > -10$ 
Solving each case: 
1. $x < 5$ 
2. $x > -15$ 
Combining these, we get $-15 < x < 5$. This inequality means that $x$ can be any number greater than $-15$ and less than $5$. To find how many integers satisfy this condition, we list them: $-14, -13, -12, -11, -10, -9, -8, -7, -6, -5, -4, -3, -2, -1, 0, 1, 2, 3, 4$. There are 19 integers that satisfy the inequality $|x + 5| < 10$.

\textcolor{brown}{
[human]: What about $|x + 10| < 5$?
}

\textcolor{brown}{
[bot]: To solve the inequality $|x + 10| < 5$, we again consider two cases: 
1. $x + 10 < 5$ 
2. $-(x + 10) < 5$, which simplifies to $x + 10 > -5$ 
Solving each case: 
1. $x < -5$ 
2. $x > -15$ 
Combining these, we get $-15 < x < -5$. This inequality means that $x$ can be any number greater than $-15$ and less than $-5$. To find how many integers satisfy this condition, we list them: $-14, -13, -12, -11, -10, -9, -8, -7, -6$. There are 9 integers that satisfy the inequality $|x + 10| < 5$.
}

[human]: What about $|x + 10| < 10$?

[bot]: To solve the inequality $|x + 10| < 10$, we consider two cases: 
1. $x + 10 < 10$ 
2. $-(x + 10) < 10$, which simplifies to $x + 10 > -10$ 
Solving each case: 
1. $x < 0$ 
2. $x > -20$ 
Combining these, we get $-20 < x < 0$. This inequality means that $x$ can be any number greater than $-20$ and less than $0$. To find how many integers satisfy this condition, we list them: $-19, -18, -17, -16, -15, -14, -13, -12, -11, -10, -9, -8, -7, -6, -5, -4, -3, -2, -1$. There are 19 integers that satisfy the inequality $|x + 10| < 10$. \textcolor{blue}{(Topic: Number of Integers in Inequalities)}
\end{tcolorbox}
\begin{tcolorbox}[left=3pt,right=3pt,top=3pt,bottom=3pt,title=\textbf{Answer (Ours):}]
To find the number of positive integers that satisfy the inequality $(|x + 10| < 5)$, we need to look for positive integers in the solution set $(-15 < x < -5)$. Since there are no positive integers in this interval, the answer is 0.

\end{tcolorbox}
\caption{Case study comparing our \sysname\ with the \textit{session-level} memory baseline. When constructing the memory bank at the \textit{session-level}, the agent fails to provide a satisfactory response due to the inclusion of excessive irrelevant information within a session. This distracts the generator from focusing on the truly useful information highlighted in \textcolor{brown}{brown}. The dialogue topic is emphasized in \textcolor{blue}{blue}. Some context is omitted (replaced by ...) due to length constraints.}
\label{fig: case_study_segment_vs_session}
\end{figure}



\begin{figure}[htb]
\small
\begin{tcolorbox}[left=3pt,right=3pt,top=3pt,bottom=3pt,title=\textbf{Conversation History:}]
[human]: Photosynthesis is a vital process for life on Earth. Could you outline the two main stages of photosynthesis, including where they take place within the chloroplast, and the primary inputs and outputs for each stage? ... (Human-Bot Dialogue Turns)... \textcolor{blue}{(Topic: Photosynthetic Energy Production)}

[human]: Please assume the role of an English translator, tasked with correcting and enhancing spelling and language. Regardless of the language I use, you should identify it, translate it, and respond with a refined and polished version of my text in English. 

... (Human-Bot Dialogue Turns)...  \textcolor{blue}{(Topic: Language Translation and Enhancement)}

[human]: Suggest five award-winning documentary films with brief background descriptions for aspiring filmmakers to study.

\textcolor{brown}{[bot]: ...
5. \"An Inconvenient Truth\" (2006) - Directed by Davis Guggenheim and featuring former United States Vice President Al Gore, this documentary aims to educate the public about global warming. It won two Academy Awards, including Best Documentary Feature. The film is notable for its straightforward yet impactful presentation of scientific data, making complex information accessible and engaging, a valuable lesson for filmmakers looking to tackle environmental or scientific subjects.}

... (Human-Bot Dialogue Turns)... 
\textcolor{blue}{(Topic: Documentary Films Recommendation)}

[human]: Given the following records of stock prices, extract the highest and lowest closing prices for each month in the year 2022. Return the results as a CSV string, with one line allocated for each month. Date,Open,High,Low,Close,Volume ... ... (Human-Bot Dialogue Turns)...  \textcolor{blue}{(Topic: Stock Prices Analysis)}

[human]: The city of Vega intends to build a bridge that will span the Vegona River, covering a distance of 1.8 kilometers. The proposed location falls within a seismically active area that has experienced several high-magnitude earthquakes. Given these circumstances, what would be the best approach to constructing the bridge? ... (Human-Bot Dialogue Turns)...  \textcolor{blue}{(Topic: Seismic Bridge Construction)}
\end{tcolorbox}
\begin{tcolorbox}[left=3pt,right=3pt,top=3pt,bottom=3pt,title=\textbf{Question:}]
Which documentary won two Academy Awards and what lesson does it offer to filmmakers?
\end{tcolorbox}
\begin{tcolorbox}[left=3pt,right=3pt,top=3pt,bottom=3pt,title=\textbf{Ground-Truth Answer:}]
``An Inconvenient Truth" (2006) won two Academy Awards. It offers the lesson of making complex information accessible and engaging, a valuable lesson for filmmakers looking to tackle environmental or scientific subjects.
\end{tcolorbox}
\begin{tcolorbox}[left=3pt,right=3pt,top=3pt,bottom=3pt,title=\textbf{Memory generated by RecurSum~\citep{wang2023recursively}:}]
The dialogue provided detailed information on various topics, including the stages of photosynthesis, energy production estimates through photosynthesis for different organisms, translations and enhancements of phrases in different languages, recommendations for documentary films, a pitch for a film about overcoming adversity, analysis of stock prices, and considerations for constructing a bridge in a seismically active area. Here's a summary of the key points: ...

\textbf{Documentary Film Recommendations}: Suggested five award-winning documentary films for aspiring filmmakers to study, with brief background descriptions for each. The response included:

-  "March of the Penguins" (2005) - Emperor penguins' journey in Antarctica; noted for its cinematography and portrayal of the penguins' life cycle.

-  "Citizenfour" (2014) - Edward Snowden and NSA spying scandal; known for its real-time exploration of Snowden's revelations.

-  "20 Feet from Stardom" (2013) - Stories of backup singers behind musical legends; celebrates unsung heroes in the music industry.

-  "Man on Wire" (2008) - Philippe Petit's 1974 high-wire walk between the Twin Towers; combines personal footage, interviews, and re-enactments.

-  \textcolor{brown}{"An Inconvenient Truth" (2006) - Al Gore's documentary on global warming; known for its presentation of scientific data.}
\end{tcolorbox}
\begin{tcolorbox}[left=3pt,right=3pt,top=3pt,bottom=3pt,title=\textbf{Answer (RSum):}]
I'm sorry, but based on the updated context you provided, I don't see any previous discussion related to the question you asked. However, to answer your question, the documentary that won two Academy Awards is "Free Solo," which won for Best Documentary Feature and Best Achievement in Film Editing in 2019. 
% The lesson it offers to filmmakers is the importance of taking risks and pushing boundaries in order to create something truly unique and impactful.
....
The film follows climber Alex Honnold as he attempts to free solo climb El Capitan in Yosemite National Park, and the filmmakers had to take risks themselves in order to capture the footage and tell the story.

\textbf{\textcolor{red}{(The next part of this figure~\ref{fig: case_study_segment_vs_rsum} will be continued on the next page.)}}
\end{tcolorbox}
\end{figure}
\begin{figure}[htbp]
% \ContinuedFloat
\small
\begin{tcolorbox}[left=3pt,right=3pt,top=3pt,bottom=3pt,title=\textbf{Retrieved Memory (Ours):}]
[human]: Suggest five award-winning documentary films with brief background descriptions for aspiring filmmakers to study.

[bot]: 1. **"March of the Penguins" (2005)** - Directed by Luc Jacquet, this French feature-length nature documentary delves into the annual journey of Emperor penguins across the Antarctic. It won the Academy Award for Best Documentary Feature. The film is celebrated for its stunning cinematography and its detailed, heartfelt portrayal of the penguins' life cycle, making it a must-study for filmmakers interested in wildlife documentaries. 2. **"Citizenfour" (2014)** ... 5. \textcolor{brown}{**"An Inconvenient Truth" (2006)** - Directed by Davis Guggenheim and featuring former United States Vice President Al Gore, this documentary aims to educate the public about global warming. It won two Academy Awards, including Best Documentary Feature.} The film is notable for its straightforward yet impactful presentation of scientific data, making complex information accessible and engaging, \textcolor{brown}{a valuable lesson for filmmakers looking to tackle environmental or scientific subjects.}

[human]: With the spirit in the first film, craft a succinct and persuasive pitch for a film about overcoming adversity.

[bot]: Drawing inspiration from the resilience and majestic journey of Emperor penguins in "March of the Penguins," our documentary, "Unyielding Spirits: The Human Odyssey," will explore the profound stories of individuals around the globe who have faced and overcome staggering adversities. Through breathtaking cinematography and intimate storytelling, we ...
\textcolor{blue}{(Topic: Documentary Films Recommendation)}
\end{tcolorbox}
\begin{tcolorbox}[left=3pt,right=3pt,top=3pt,bottom=3pt,title=\textbf{Answer (Ours):}]
``An Inconvenient Truth" won two Academy Awards, including Best Documentary Feature. The film offers a valuable lesson to filmmakers looking to tackle environmental or scientific subjects, as it presents complex information in a straightforward yet impactful way, making it accessible and engaging to a wider audience.


\end{tcolorbox}
\caption{Case study comparing our \sysname\ with RecurSum~\citep{wang2023recursively}. When using RecurSum, the agent fails to provide a satisfactory answer due to information loss during the summarization process. Summarization omits essential details needed to answer the user's new request, as highlighted in \textcolor{brown}{brown}. The dialogue topic is emphasized in \textcolor{blue}{blue}. Some context is omitted (replaced by ...) due to length constraints.}
\label{fig: case_study_segment_vs_rsum}
\end{figure}




\begin{figure}[htb]
\small
\begin{tcolorbox}[left=3pt,right=3pt,top=3pt,bottom=3pt,title=\textbf{Conversation History:}]
[human]: Photosynthesis is a vital process for life on Earth. Could you outline the two main stages of photosynthesis, including where they take place within the chloroplast, and the primary inputs and outputs for each stage? ... (Human-Bot Dialogue Turns)... \textcolor{blue}{(Topic: Photosynthetic Energy Production)}

[human]: Please assume the role of an English translator, tasked with correcting and enhancing spelling and language. Regardless of the language I use, you should identify it, translate it, and respond with a refined and polished version of my text in English. 

... (Human-Bot Dialogue Turns)...  \textcolor{blue}{(Topic: Language Translation and Enhancement)}

[human]: Suggest five award-winning documentary films with brief background descriptions for aspiring filmmakers to study.

\textcolor{brown}{[bot]: ...
5. \"An Inconvenient Truth\" (2006) - Directed by Davis Guggenheim and featuring former United States Vice President Al Gore, this documentary aims to educate the public about global warming. It won two Academy Awards, including Best Documentary Feature. The film is notable for its straightforward yet impactful presentation of scientific data, making complex information accessible and engaging, a valuable lesson for filmmakers looking to tackle environmental or scientific subjects.}

... (Human-Bot Dialogue Turns)... 
\textcolor{blue}{(Topic: Documentary Films Recommendation)}

[human]: Given the following records of stock prices, extract the highest and lowest closing prices for each month in the year 2022. Return the results as a CSV string, with one line allocated for each month. Date,Open,High,Low,Close,Volume ... ... (Human-Bot Dialogue Turns)...  \textcolor{blue}{(Topic: Stock Prices Analysis)}

[human]: The city of Vega intends to build a bridge that will span the Vegona River, covering a distance of 1.8 kilometers. The proposed location falls within a seismically active area that has experienced several high-magnitude earthquakes. Given these circumstances, what would be the best approach to constructing the bridge? ... (Human-Bot Dialogue Turns)...  \textcolor{blue}{(Topic: Seismic Bridge Construction)}
\end{tcolorbox}
\begin{tcolorbox}[left=3pt,right=3pt,top=3pt,bottom=3pt,title=\textbf{Question:}]
Which documentary won two Academy Awards and what lesson does it offer to filmmakers?
\end{tcolorbox}
\begin{tcolorbox}[left=3pt,right=3pt,top=3pt,bottom=3pt,title=\textbf{Ground-Truth Answer:}]
"An Inconvenient Truth" (2006) won two Academy Awards. It offers the lesson of making complex information accessible and engaging, a valuable lesson for filmmakers looking to tackle environmental or scientific subjects.
\end{tcolorbox}
\begin{tcolorbox}[left=3pt,right=3pt,top=3pt,bottom=3pt,title=\textbf{Memory generated by ConditionMem~\citep{yuan2023evolving}:}]
- \textcolor{brown}{The user is seeking recommendations for award-winning documentary films, along with brief background descriptions, for educational purposes.} This request likely stems from a desire to learn from successful examples in the documentary filmmaking field, possibly to improve their own filmmaking skills or to gain inspiration for their projects. 

- \textcolor{brown}{User is interested in award-winning documentary films for educational purposes, aiming to study their backgrounds and filmmaking techniques.}

- The user's request for crafting a succinct and persuasive pitch for a film about overcoming adversity is inspired by the spirit of "March of the Penguins," a documentary that beautifully captures the Emperor penguins' annual journey across the Antarctic, highlighting themes of endurance and survival against the odds. The user seeks to emulate the emotional depth and cinematic quality that made "March of the Penguins" compelling, aiming to create a pitch that resonates with similar themes of resilience and triumph over challenges. 

- User is interested in creating content that focuses on themes of overcoming adversity, inspired by the emotional depth and cinematic excellence of documentaries like "March of the Penguins." They value storytelling that captures the essence of endurance and survival against the odds.
% : The user's request to write a budget plan for the video production directly follows the task of creating a video script aimed at training new employees of a car wash business in Japan, focusing on business etiquette norms.
% The user is likely seeking to understand the financial requirements to bring the previously crafted script to life, indicating a move from the planning and content development phase to the practical aspects of production. 

... (Other Irrelevant Memory Records) ...
\end{tcolorbox}
\begin{tcolorbox}[left=3pt,right=3pt,top=3pt,bottom=3pt,title=\textbf{Answer (Conditional Memory):}]
The documentary that won two Academy Awards is "Free Solo," which follows climber Alex Honnold as he attempts to free solo climb El Capitan in Yosemite National Park. The film offers a lesson to filmmakers on the importance of capturing authentic and compelling human stories, as well as the technical challenges of filming in extreme environments.

\textbf{\textcolor{red}{(The next part of this figure~\ref{fig: case_study_segment_vs_condmem} will be continued on the next page.)}}
\end{tcolorbox}
\end{figure}
\begin{figure}[htbp]
% \ContinuedFloat
\small
\begin{tcolorbox}[left=3pt,right=3pt,top=3pt,bottom=3pt,title=\textbf{Retrieved Memory (Ours):}]
[human]: Suggest five award-winning documentary films with brief background descriptions for aspiring filmmakers to study.

[bot]: 1. **"March of the Penguins" (2005)** - Directed by Luc Jacquet, this French feature-length nature documentary delves into the annual journey of Emperor penguins across the Antarctic. It won the Academy Award for Best Documentary Feature. The film is celebrated for its stunning cinematography and its detailed, heartfelt portrayal of the penguins' life cycle, making it a must-study for filmmakers interested in wildlife documentaries. 2. **"Citizenfour" (2014)** ... 5. \textcolor{brown}{**"An Inconvenient Truth" (2006)** - Directed by Davis Guggenheim and featuring former United States Vice President Al Gore, this documentary aims to educate the public about global warming. It won two Academy Awards, including Best Documentary Feature.} The film is notable for its straightforward yet impactful presentation of scientific data, making complex information accessible and engaging, a valuable lesson for filmmakers looking to tackle environmental or scientific subjects.

[human]: With the spirit in the first film, craft a succinct and persuasive pitch for a film about overcoming adversity.

[bot]: Drawing inspiration from the resilience and majestic journey of Emperor penguins in "March of the Penguins," our documentary, "Unyielding Spirits: The Human Odyssey," will explore the profound stories of individuals around the globe who have faced and overcome staggering adversities. Through breathtaking cinematography and intimate storytelling, we ...
\textcolor{blue}{(Topic: Documentary Films Recommendation)}
\end{tcolorbox}
\begin{tcolorbox}[left=3pt,right=3pt,top=3pt,bottom=3pt,title=\textbf{Answer (Ours):}]
``An Inconvenient Truth" won two Academy Awards, including Best Documentary Feature. The film offers a valuable lesson to filmmakers looking to tackle environmental or scientific subjects, as it presents complex information in a straightforward yet impactful way, making it accessible and engaging to a wider audience.
\end{tcolorbox}
\caption{Case study comparing our \sysname\ with ConditionMem~\citep{yuan2023evolving}. When using ConditionMem, the agent fails to provide a satisfactory answer due to (1) information loss during the summarization process and (2) the incorrect discarding of turns that are actually useful, as highlighted in \textcolor{brown}{brown}. The dialogue topic is emphasized in \textcolor{blue}{blue}. Some context is omitted (replaced by ...) due to length constraints.}
\label{fig: case_study_segment_vs_condmem}
\end{figure}


