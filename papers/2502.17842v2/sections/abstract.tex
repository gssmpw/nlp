%\documentclass[../main.tex]{subfiles}
%\begin{document}
Semantic communication marks a new paradigm shift from 
bit-wise data transmission to semantic 
information delivery for the purpose of bandwidth reduction. To more effectively carry out specialized downstream tasks at the receiver end, it is crucial to define the most critical semantic message in the data based on the task or goal-oriented features.
%tasks or goal-driven 
%semantic data is crucial. 
In this work, we propose a novel goal-oriented communication (GO-COM) framework, namely Goal-Oriented Semantic Variational Autoencoder (GOS-VAE), by focusing on the extraction of the semantics vital to the downstream tasks. Specifically, 
we adopt a Vector Quantized Variational Autoencoder (VQ-VAE) to compress media data at the transmitter side. Instead of targeting the pixel-wise image data reconstruction, we measure the quality-of-service at the receiver end based on
a pre-defined task-incentivized model. Moreover, to capture the relevant semantic features in the data reconstruction, imitation learning is adopted to measure the data regeneration quality in terms of
goal-oriented semantics. Our experimental results 
demonstrate the power of imitation learning in characterizing goal-oriented semantics and bandwidth efficiency of our proposed GOS-VAE.

%\end{document}