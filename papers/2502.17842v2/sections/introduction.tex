%\documentclass[../main.tex]{subfiles}

%\begin{document}

Next-generation wireless networks and Artificial Intelligence (AI) algorithms have found 
a wide range of data-intensive applications, 
including augmented and virtual reality \cite{hazarika2023towards}, and autonomous driving \cite{chen2020vision}, where low-latency data transport
and high-accuracy decision-making play crucial roles. As these applications proliferate, the hunger for bandwidth and
data rates continues to grow, straining the already 
scarce communication resources \cite{strinati20216g}. To achieve bandwidth efficiency and ensure data transmission quality, the concept of semantic communications has recently 
re-surfaced as an important paradigm, which aims at conveying the most critical semantic information rather than bit-wise packet transport \cite{luo2022semantic}.

Semantic communication systems commonly employ deep learning techniques for embedding representations and regenerating data. For example, DeepSC \cite{xie2021deep} focused on communicating the semantic meaning in text messages by utilizing a
deep-learning transformer architecture. 
%\textcolor{red}{AESC \cite{luo2022autoencoder} utilizes %an Autoencoder-based model for semantic transmission of sentences in communication systems.
%} 
Beyond text communications, 
\begin{comment}
Recently, AI-powered semantic communication frameworks
have focused on extracting semantic information from  source data for transmission \cite{xie2021deep, xie2020lite, jiang2022deep, weng2021semantic, weng2024semantic, han2022semantic, fu2024generative, pan2023image, wu2024semantic, jiang2022wireless, wang2022wireless, li2024video}. In text
based applications \cite{xie2021deep, xie2020lite, jiang2022deep}, transformer-based architecture, DeepSC, 
allows the receiver to 
recover the meaning of text messages
\cite{xie2021deep}. 
Semantic communication has since expanded to various domains, including speech \cite{weng2021semantic, weng2024semantic, han2022semantic}, image \cite{fu2024generative, pan2023image, wu2024semantic}, and video \cite{jiang2022wireless, wang2022wireless, li2024video}. This work focuses on  semantic 
image communications.
\end{comment}
%For image transmission, 
authors of \cite{bourtsoulatze2019deep} proposed a joint source and channel coding technique for wireless image transmission, using a Convolutional Neural Network (CNN) to directly map image pixels to channel input symbols. Another work (VQ-DeepSC \cite{fu2023vector})
presented a vector quantization semantic 
communication system to compress 
multi-scale semantic features through codebook quantization. 
The recent development of generative AI has further
inspired other generative frameworks for 
semantic communications. For example, in \cite{lokumarambage2023wireless}, a pre-trained Generative Adversarial Network (GAN) is utilized to reconstruct images at the receiver. Similarly, Generative Semantic Communication (GESCO) introduced in \cite{grassucci2023generative} utilizes a diffusion-based architecture for data generation based on segmentation maps.
Nevertheless, classic semantic communication frameworks continue to stress the visual quality of the reconstructed images. Future communication systems are expected to play an increasingly important part to serve automation, artificial intelligence, and other decision-making applications, instead of being a pipe to provide data to only human end-users. Thus, next-generation networks must prioritize task-driven semantic communication for downstream tasks without involving human viewers, making efficient semantic feature extraction crucial.


\begin{figure*}[t]
\centering
\includegraphics[width=0.9\textwidth]{images/SQVAE.png}
\caption{The proposed Goal-Oriented Semantic Variational Autoencoder (GOS-VAE) framework for Semantic Communication: The snowflakes symbol denotes pre-trained and fixed model parameters.}\vspace*{-3mm}
\label{fig:sqvae}\vspace*{-3mm}
\end{figure*}

Recognizing the key role of communication networks in AI-driven data applications, recent works have shifted towards task/goal-oriented semantic communication (GO-COM) systems \cite{kang2022task}. Unlike classic semantic communications, these GO-COMs focus more on the delivery of key semantic information for the specific downstream tasks at the receiver end.


\iffalse
For example,
the authors
of \cite{wu2024semantic} presented a semantic image transmission system that identifies Regions of Interest (ROI)
in an image before allocating higher data rates for ROI. 
%However, ROI varies across different scenarios, lacking specification to the downstream tasks. To address this issue, 
The TasCom proposed in
%Task-oriented Adaptive Semantic Communication (TasCom) framework
\cite{fu2024generative} transmits only task-related semantic features, 
%achieved through the proposed Adaptive Coding Controller module, which prioritizes features that significantly contribute to AI performance. 
using an Adaptive Coding Controller to prioritize features that enhance AI performance.
Another work, VIS-SemCom \cite{lv2024importance} is proposed to transmit the feature maps of key objects such as vehicles, pedestrians, and obstacles to reduce redundancy. Despite these progresses, existing GO-COMs reconstruct data based on pre-defined semantic features, which lack scalability, adaptivity, and generalization.
For instance, in autonomous driving, objects like vehicles, traffic signs, and pedestrians are typically considered critical, whereas elements like trees are less relevant to safe driving. However, if fallen tree branches obstruct the road due to
a storm, they shall lead to critical concerns. Moreover, onboard computational resources are inadequate to handle multiple downstream tasks using increasingly complex models with billions of parameters \cite{achiam2023gpt, touvron2023llama, liu2024visual}. 
\fi


For this, \cite{wu2024semantic} proposed a semantic image transmission system that allocates higher data rates to Regions of Interest (ROI) within images. However, ROIs can vary significantly across downstream tasks. The TasCom framework \cite{fu2024generative} addresses this by transmitting only task-specific features, using an Adaptive Coding Controller to prioritize those most relevant to AI performance. Similarly, VIS-SemCom \cite{lv2024importance} reduces redundancy by transmitting feature maps of key objects, such as vehicles and pedestrians, in scenes like autonomous driving. Despite these progresses, existing GO-COMs reconstruct data using pre-defined semantic features, which lack adaptivity and generalization. For instance, while trees may generally be considered non-essential in autonomous driving, an accidentally fallen tree branch obstructing the road would require immediate attention. Furthermore, onboard resources are often insufficient to manage multiple complex tasks with models containing billions of parameters \cite{achiam2023gpt, touvron2023llama, liu2024visual}. Thus, critical challenges remain in advancing GO-COMs, particularly in achieving efficient semantic extraction and implementation.

To address the aforementioned challenges, this work introduces a novel GO-COM framework, namely Goal-Oriented Semantic Variational Autoencoder (GOS-VAE), integrating the Vector Quantized VAE (VQ-VAE) and imitation learning. 
We summarize our major contributions 
as follows:

\begin{itemize}
  \item Taking autonomous driving as an exemplary application, we propose the novel GOS-VAE framework supported
  by a VQ-VAE backbone, which defines the goal-oriented ``semantics" beneficial to the downstream task.
  \item To save computation, GOS-VAE places
  the decoder and computation-intensive downstream task model on a powerful back-end server while deploying a low-complexity encoder at the sender to improve efficiency.
  \item To preserve information vital
  to downstream tasks, our GOS-VAE
adopts imitation learning capable of processing different tasks
without manual labeling.
  \item Through flexible adjustment of network depth and codebook size, plus 
  a customized shallow CNN structure, our proposed GOS-VAE achieves superior
 signal recovery performance at reduced bandwidth
  consumption.
\end{itemize}






%\end{document}