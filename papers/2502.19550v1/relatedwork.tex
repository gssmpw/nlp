\section{Literature Review}
\label{sec:litrev}

ABMs have found a variety of applications across biology~\cite{an2009_bio,zhang2020_bio_ecology,soheilypour2018_bio}, ecology~\cite{mclane2011_ecology,zhang2020_bio_ecology}, economics~\cite{chen2012_economics,axtell2022_economics,hamill2015_economics}, and epidemiology~\cite{hunter2018_epi,marshall2015_epi} and have proven capable in providing predictions of and insight into complex dynamics~\cite{williams2018_complex}. These models simulate the interactions of individual agents, allowing for the exploration of heterogeneous behaviors and local interactions that can lead to emergent phenomena. Several key studies have demonstrated the utility of these models in epidemiological contexts, including the spread of infectious diseases such as influenza, COVID-19, and others.
For instance, Ref.~\cite{li2017_influenza} developed and calibrated an ABM to simulate the transmission dynamics of influenza, highlighting the ability of ABMs to capture the impact of vaccination strategies once tuned.
Similarly, Ref.~\cite{silva2020_covid_social_dist} utilized an ABM to investigate the spread of COVID-19, providing insights into the effectiveness of social distancing measures and demonstrating practical use of ABMs for epidemiological insights.

Effective calibration techniques for these ABMs are essential to ensure that model output aligns with observed data.
Traditional methods, such as grid search and random sampling, have been employed~\cite{quinlan2020_grid}; however, these approaches often struggle with high-dimensional parameter spaces.
More sophisticated techniques, including Bayesian inference and MCMC sampling, have been used because of their theoretical guarantees and uncertainty quantification~\cite{kim2021_mcmc_calib,lux2018_mcmc_calib,grazzini2017_mcmc_calib} but are limited by the usually large computational expense of repeated ABM evaluations.
This is of particular importance for high-dimensional models~\cite{srikrishnan2021_abm_calib_w_many_params}.
Often, this has prompted the use of more heuristic methods such as ABC~\cite{zhang2020_abc_calib,mcculloch2022_abc_calib,ozik2021_citycovid}, genetic algorithms~\cite{d2020_genetic,cess2023_genetic}, or particle swarm optimization~\cite{platt2020_particle_swarm,li2017_influenza}.

Another approach to overcoming the computational challenge of ABM calibration has been addressed via the use of surrogate or emulator models, which approximate the outputs of an ABM using information from previous runs.
Foremost among these approaches are the use of GPs~\cite{kim2021_mcmc_calib, fadikar2018_gp, binois_portfolio_2025}, support vector machines~\cite{angione2022_svm,perumal2022_svm}, bagged and/or boosted machine learning approaches~\cite{ozik_extreme-scale_2018,zhang2020_abc_calib,de2022_rf,kieu2024_rf_nn,perumal2020_rf,lamperti2018_rf,robertson2024_rf,robertson2024_rf_sm}, and neural networks~\cite{cockrell_nested_2020,cess2023_nn,anirudh2022_nn,kieu2024_rf_nn}.
Each of these provide a flexible framework for approximating complex functions, facilitating faster evaluations of the likelihood surface compared to traditional methods.
Recent efforts have also explored the use of spline or equation based modeling via ordinary differential equations as surrogates~\cite{fabiani2024_equation_based,avegliano2023_equation_based,li2017_influenza}.
These provide interpretable and fast simulation alternatives, but require in-depth modeling to either match or coarse-grain the ABM dynamics.
However, the trade-offs between the accuracy of the surrogate and the computational cost of ABM evaluations remains a critical consideration.

The current authors have explored the calibration of epidemiological ABMs, and CityCOVID in particular in Refs.~\cite{robertson2024_rf,robertson2024_rf_sm}.
This was performed using a surrogate model constructed using principal component analysis (to capture the temporal evolution of observables such as daily hospitalizations and deaths) and random forests (to capture their dependence on the ABM parameters being calibrated.
The random forest surrogate also builds in a sensitivity analysis via the Gini impurity, which was used to reduce the number of ABM parameters from 9 key parameters to 4.
These same parameters (which are shown later in Table~\ref{tab:priors}) will be calibrated in this paper, but using a GP surrogate and SVI.
Information on the sensitivity of the GP surrogate approach to these 4 parameters will be included later in Table~\ref{tab:feature_importance}.
The formulation of the inverse problem, the error metrics, figures of merit for checking the predictive skill of the calibration and the training dataset for the surrogates are reused in this work, and are described in detail in Section~\ref{sec:ip} and Section~\ref{sec:surr}; the results from our previous work will be compared with the results from SVI, in Section~\ref{sec:results}.

Outside of ABMs, SVI has emerged as a promising alternative for parameter estimation~\cite{liu2016_stein,biswas2023_stein_calib,morelli2023_stein_calib}.
SVI leverages gradient information to iteratively update several potential parameter samples or ``particles,'' enabling efficient exploration of the parameter space. This method has shown potential for reducing the computational burden of parameter estimation in high-dimensional settings.
This is particularly valuable for ABMs, which often require high-dimensional parameter spaces in order to adequately capture complex dynamics.

Despite the growing interest in SVI, we are unaware of its application to calibration of ABMs and there is limited literature directly comparing its performance to MCMC sampling~\cite{gebraad2021_stein_vs_mcmc}.