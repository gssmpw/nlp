%\documentclass{uai2025} % for initial submission
\documentclass[accepted]{uai2025} % after acceptance, for a revised version; 
% also before submission to see how the non-anonymous paper would look like 
                        
%% There is a class option to choose the math font
% \documentclass[mathfont=ptmx]{uai2025} % ptmx math instead of Computer
                                         % Modern (has noticeable issues)
% \documentclass[mathfont=newtx]{uai2025} % newtx fonts (improves upon
                                          % ptmx; less tested, no support)
% NOTE: Only keep *one* line above as appropriate, as it will be replaced
%       automatically for papers to be published. Do not make any other
%       change above this note for an accepted version.

%% Choose your variant of English; be consistent
\usepackage[american]{babel}
% \usepackage[british]{babel}

%% Some suggested packages, as needed:
\usepackage{natbib} % has a nice set of citation styles and commands
    \bibliographystyle{plainnat}
    \renewcommand{\bibsection}{\subsubsection*{References}}
\usepackage{mathtools} % amsmath with fixes and additions
% \usepackage{siunitx} % for proper typesetting of numbers and units
\usepackage{booktabs} % commands to create good-looking tables
\usepackage{tikz} % nice language for creating drawings and diagrams

%% Provided macros
% \smaller: Because the class footnote size is essentially LaTeX's \small,
%           redefining \footnotesize, we provide the original \footnotesize
%           using this macro.
%           (Use only sparingly, e.g., in drawings, as it is quite small.)
\usepackage{amsmath}  
\usepackage{amssymb}
\usepackage{algorithm}
\usepackage{algpseudocode}
\usepackage{booktabs}
\usepackage{graphicx}
\usepackage{subcaption}
\usepackage{multibib}
% other packages
\usepackage{float}

\usepackage{array,multirow}
\usepackage{adjustbox}
\usepackage{dsfont}
\usepackage{amsthm}
\usepackage{tabularx}
\usepackage{paracol}
\usepackage{stfloats}
\usepackage{xcolor}
\usepackage{placeins}
\usepackage{mathtools}
\usepackage{physics}
\usepackage{amsfonts}       % blackboard math symbols
\usepackage{nicefrac}       % compact symbols for 1/2, etc.
\usepackage{microtype}      % microtypography
\usepackage{enumitem}
\usepackage{amsmath}
\usepackage{thmtools,thm-restate}
\usepackage{bm} 
\usepackage{comment}
%%%%% NEW MATH DEFINITIONS %%%%%

\usepackage{amsmath,amsfonts,bm}
\usepackage{derivative}
% Mark sections of captions for referring to divisions of figures
\newcommand{\figleft}{{\em (Left)}}
\newcommand{\figcenter}{{\em (Center)}}
\newcommand{\figright}{{\em (Right)}}
\newcommand{\figtop}{{\em (Top)}}
\newcommand{\figbottom}{{\em (Bottom)}}
\newcommand{\captiona}{{\em (a)}}
\newcommand{\captionb}{{\em (b)}}
\newcommand{\captionc}{{\em (c)}}
\newcommand{\captiond}{{\em (d)}}

% Highlight a newly defined term
\newcommand{\newterm}[1]{{\bf #1}}

% Derivative d 
\newcommand{\deriv}{{\mathrm{d}}}

% Figure reference, lower-case.
\def\figref#1{figure~\ref{#1}}
% Figure reference, capital. For start of sentence
\def\Figref#1{Figure~\ref{#1}}
\def\twofigref#1#2{figures \ref{#1} and \ref{#2}}
\def\quadfigref#1#2#3#4{figures \ref{#1}, \ref{#2}, \ref{#3} and \ref{#4}}
% Section reference, lower-case.
\def\secref#1{section~\ref{#1}}
% Section reference, capital.
\def\Secref#1{Section~\ref{#1}}
% Reference to two sections.
\def\twosecrefs#1#2{sections \ref{#1} and \ref{#2}}
% Reference to three sections.
\def\secrefs#1#2#3{sections \ref{#1}, \ref{#2} and \ref{#3}}
% Reference to an equation, lower-case.
\def\eqref#1{equation~\ref{#1}}
% Reference to an equation, upper case
\def\Eqref#1{Equation~\ref{#1}}
% A raw reference to an equation---avoid using if possible
\def\plaineqref#1{\ref{#1}}
% Reference to a chapter, lower-case.
\def\chapref#1{chapter~\ref{#1}}
% Reference to an equation, upper case.
\def\Chapref#1{Chapter~\ref{#1}}
% Reference to a range of chapters
\def\rangechapref#1#2{chapters\ref{#1}--\ref{#2}}
% Reference to an algorithm, lower-case.
\def\algref#1{algorithm~\ref{#1}}
% Reference to an algorithm, upper case.
\def\Algref#1{Algorithm~\ref{#1}}
\def\twoalgref#1#2{algorithms \ref{#1} and \ref{#2}}
\def\Twoalgref#1#2{Algorithms \ref{#1} and \ref{#2}}
% Reference to a part, lower case
\def\partref#1{part~\ref{#1}}
% Reference to a part, upper case
\def\Partref#1{Part~\ref{#1}}
\def\twopartref#1#2{parts \ref{#1} and \ref{#2}}

\def\ceil#1{\lceil #1 \rceil}
\def\floor#1{\lfloor #1 \rfloor}
\def\1{\bm{1}}
\newcommand{\train}{\mathcal{D}}
\newcommand{\valid}{\mathcal{D_{\mathrm{valid}}}}
\newcommand{\test}{\mathcal{D_{\mathrm{test}}}}

\def\eps{{\epsilon}}


% Random variables
\def\reta{{\textnormal{$\eta$}}}
\def\ra{{\textnormal{a}}}
\def\rb{{\textnormal{b}}}
\def\rc{{\textnormal{c}}}
\def\rd{{\textnormal{d}}}
\def\re{{\textnormal{e}}}
\def\rf{{\textnormal{f}}}
\def\rg{{\textnormal{g}}}
\def\rh{{\textnormal{h}}}
\def\ri{{\textnormal{i}}}
\def\rj{{\textnormal{j}}}
\def\rk{{\textnormal{k}}}
\def\rl{{\textnormal{l}}}
% rm is already a command, just don't name any random variables m
\def\rn{{\textnormal{n}}}
\def\ro{{\textnormal{o}}}
\def\rp{{\textnormal{p}}}
\def\rq{{\textnormal{q}}}
\def\rr{{\textnormal{r}}}
\def\rs{{\textnormal{s}}}
\def\rt{{\textnormal{t}}}
\def\ru{{\textnormal{u}}}
\def\rv{{\textnormal{v}}}
\def\rw{{\textnormal{w}}}
\def\rx{{\textnormal{x}}}
\def\ry{{\textnormal{y}}}
\def\rz{{\textnormal{z}}}

% Random vectors
\def\rvepsilon{{\mathbf{\epsilon}}}
\def\rvphi{{\mathbf{\phi}}}
\def\rvtheta{{\mathbf{\theta}}}
\def\rva{{\mathbf{a}}}
\def\rvb{{\mathbf{b}}}
\def\rvc{{\mathbf{c}}}
\def\rvd{{\mathbf{d}}}
\def\rve{{\mathbf{e}}}
\def\rvf{{\mathbf{f}}}
\def\rvg{{\mathbf{g}}}
\def\rvh{{\mathbf{h}}}
\def\rvu{{\mathbf{i}}}
\def\rvj{{\mathbf{j}}}
\def\rvk{{\mathbf{k}}}
\def\rvl{{\mathbf{l}}}
\def\rvm{{\mathbf{m}}}
\def\rvn{{\mathbf{n}}}
\def\rvo{{\mathbf{o}}}
\def\rvp{{\mathbf{p}}}
\def\rvq{{\mathbf{q}}}
\def\rvr{{\mathbf{r}}}
\def\rvs{{\mathbf{s}}}
\def\rvt{{\mathbf{t}}}
\def\rvu{{\mathbf{u}}}
\def\rvv{{\mathbf{v}}}
\def\rvw{{\mathbf{w}}}
\def\rvx{{\mathbf{x}}}
\def\rvy{{\mathbf{y}}}
\def\rvz{{\mathbf{z}}}

% Elements of random vectors
\def\erva{{\textnormal{a}}}
\def\ervb{{\textnormal{b}}}
\def\ervc{{\textnormal{c}}}
\def\ervd{{\textnormal{d}}}
\def\erve{{\textnormal{e}}}
\def\ervf{{\textnormal{f}}}
\def\ervg{{\textnormal{g}}}
\def\ervh{{\textnormal{h}}}
\def\ervi{{\textnormal{i}}}
\def\ervj{{\textnormal{j}}}
\def\ervk{{\textnormal{k}}}
\def\ervl{{\textnormal{l}}}
\def\ervm{{\textnormal{m}}}
\def\ervn{{\textnormal{n}}}
\def\ervo{{\textnormal{o}}}
\def\ervp{{\textnormal{p}}}
\def\ervq{{\textnormal{q}}}
\def\ervr{{\textnormal{r}}}
\def\ervs{{\textnormal{s}}}
\def\ervt{{\textnormal{t}}}
\def\ervu{{\textnormal{u}}}
\def\ervv{{\textnormal{v}}}
\def\ervw{{\textnormal{w}}}
\def\ervx{{\textnormal{x}}}
\def\ervy{{\textnormal{y}}}
\def\ervz{{\textnormal{z}}}

% Random matrices
\def\rmA{{\mathbf{A}}}
\def\rmB{{\mathbf{B}}}
\def\rmC{{\mathbf{C}}}
\def\rmD{{\mathbf{D}}}
\def\rmE{{\mathbf{E}}}
\def\rmF{{\mathbf{F}}}
\def\rmG{{\mathbf{G}}}
\def\rmH{{\mathbf{H}}}
\def\rmI{{\mathbf{I}}}
\def\rmJ{{\mathbf{J}}}
\def\rmK{{\mathbf{K}}}
\def\rmL{{\mathbf{L}}}
\def\rmM{{\mathbf{M}}}
\def\rmN{{\mathbf{N}}}
\def\rmO{{\mathbf{O}}}
\def\rmP{{\mathbf{P}}}
\def\rmQ{{\mathbf{Q}}}
\def\rmR{{\mathbf{R}}}
\def\rmS{{\mathbf{S}}}
\def\rmT{{\mathbf{T}}}
\def\rmU{{\mathbf{U}}}
\def\rmV{{\mathbf{V}}}
\def\rmW{{\mathbf{W}}}
\def\rmX{{\mathbf{X}}}
\def\rmY{{\mathbf{Y}}}
\def\rmZ{{\mathbf{Z}}}

% Elements of random matrices
\def\ermA{{\textnormal{A}}}
\def\ermB{{\textnormal{B}}}
\def\ermC{{\textnormal{C}}}
\def\ermD{{\textnormal{D}}}
\def\ermE{{\textnormal{E}}}
\def\ermF{{\textnormal{F}}}
\def\ermG{{\textnormal{G}}}
\def\ermH{{\textnormal{H}}}
\def\ermI{{\textnormal{I}}}
\def\ermJ{{\textnormal{J}}}
\def\ermK{{\textnormal{K}}}
\def\ermL{{\textnormal{L}}}
\def\ermM{{\textnormal{M}}}
\def\ermN{{\textnormal{N}}}
\def\ermO{{\textnormal{O}}}
\def\ermP{{\textnormal{P}}}
\def\ermQ{{\textnormal{Q}}}
\def\ermR{{\textnormal{R}}}
\def\ermS{{\textnormal{S}}}
\def\ermT{{\textnormal{T}}}
\def\ermU{{\textnormal{U}}}
\def\ermV{{\textnormal{V}}}
\def\ermW{{\textnormal{W}}}
\def\ermX{{\textnormal{X}}}
\def\ermY{{\textnormal{Y}}}
\def\ermZ{{\textnormal{Z}}}

% Vectors
\def\vzero{{\bm{0}}}
\def\vone{{\bm{1}}}
\def\vmu{{\bm{\mu}}}
\def\vtheta{{\bm{\theta}}}
\def\vphi{{\bm{\phi}}}
\def\va{{\bm{a}}}
\def\vb{{\bm{b}}}
\def\vc{{\bm{c}}}
\def\vd{{\bm{d}}}
\def\ve{{\bm{e}}}
\def\vf{{\bm{f}}}
\def\vg{{\bm{g}}}
\def\vh{{\bm{h}}}
\def\vi{{\bm{i}}}
\def\vj{{\bm{j}}}
\def\vk{{\bm{k}}}
\def\vl{{\bm{l}}}
\def\vm{{\bm{m}}}
\def\vn{{\bm{n}}}
\def\vo{{\bm{o}}}
\def\vp{{\bm{p}}}
\def\vq{{\bm{q}}}
\def\vr{{\bm{r}}}
\def\vs{{\bm{s}}}
\def\vt{{\bm{t}}}
\def\vu{{\bm{u}}}
\def\vv{{\bm{v}}}
\def\vw{{\bm{w}}}
\def\vx{{\bm{x}}}
\def\vy{{\bm{y}}}
\def\vz{{\bm{z}}}

% Elements of vectors
\def\evalpha{{\alpha}}
\def\evbeta{{\beta}}
\def\evepsilon{{\epsilon}}
\def\evlambda{{\lambda}}
\def\evomega{{\omega}}
\def\evmu{{\mu}}
\def\evpsi{{\psi}}
\def\evsigma{{\sigma}}
\def\evtheta{{\theta}}
\def\eva{{a}}
\def\evb{{b}}
\def\evc{{c}}
\def\evd{{d}}
\def\eve{{e}}
\def\evf{{f}}
\def\evg{{g}}
\def\evh{{h}}
\def\evi{{i}}
\def\evj{{j}}
\def\evk{{k}}
\def\evl{{l}}
\def\evm{{m}}
\def\evn{{n}}
\def\evo{{o}}
\def\evp{{p}}
\def\evq{{q}}
\def\evr{{r}}
\def\evs{{s}}
\def\evt{{t}}
\def\evu{{u}}
\def\evv{{v}}
\def\evw{{w}}
\def\evx{{x}}
\def\evy{{y}}
\def\evz{{z}}

% Matrix
\def\mA{{\bm{A}}}
\def\mB{{\bm{B}}}
\def\mC{{\bm{C}}}
\def\mD{{\bm{D}}}
\def\mE{{\bm{E}}}
\def\mF{{\bm{F}}}
\def\mG{{\bm{G}}}
\def\mH{{\bm{H}}}
\def\mI{{\bm{I}}}
\def\mJ{{\bm{J}}}
\def\mK{{\bm{K}}}
\def\mL{{\bm{L}}}
\def\mM{{\bm{M}}}
\def\mN{{\bm{N}}}
\def\mO{{\bm{O}}}
\def\mP{{\bm{P}}}
\def\mQ{{\bm{Q}}}
\def\mR{{\bm{R}}}
\def\mS{{\bm{S}}}
\def\mT{{\bm{T}}}
\def\mU{{\bm{U}}}
\def\mV{{\bm{V}}}
\def\mW{{\bm{W}}}
\def\mX{{\bm{X}}}
\def\mY{{\bm{Y}}}
\def\mZ{{\bm{Z}}}
\def\mBeta{{\bm{\beta}}}
\def\mPhi{{\bm{\Phi}}}
\def\mLambda{{\bm{\Lambda}}}
\def\mSigma{{\bm{\Sigma}}}

% Tensor
\DeclareMathAlphabet{\mathsfit}{\encodingdefault}{\sfdefault}{m}{sl}
\SetMathAlphabet{\mathsfit}{bold}{\encodingdefault}{\sfdefault}{bx}{n}
\newcommand{\tens}[1]{\bm{\mathsfit{#1}}}
\def\tA{{\tens{A}}}
\def\tB{{\tens{B}}}
\def\tC{{\tens{C}}}
\def\tD{{\tens{D}}}
\def\tE{{\tens{E}}}
\def\tF{{\tens{F}}}
\def\tG{{\tens{G}}}
\def\tH{{\tens{H}}}
\def\tI{{\tens{I}}}
\def\tJ{{\tens{J}}}
\def\tK{{\tens{K}}}
\def\tL{{\tens{L}}}
\def\tM{{\tens{M}}}
\def\tN{{\tens{N}}}
\def\tO{{\tens{O}}}
\def\tP{{\tens{P}}}
\def\tQ{{\tens{Q}}}
\def\tR{{\tens{R}}}
\def\tS{{\tens{S}}}
\def\tT{{\tens{T}}}
\def\tU{{\tens{U}}}
\def\tV{{\tens{V}}}
\def\tW{{\tens{W}}}
\def\tX{{\tens{X}}}
\def\tY{{\tens{Y}}}
\def\tZ{{\tens{Z}}}


% Graph
\def\gA{{\mathcal{A}}}
\def\gB{{\mathcal{B}}}
\def\gC{{\mathcal{C}}}
\def\gD{{\mathcal{D}}}
\def\gE{{\mathcal{E}}}
\def\gF{{\mathcal{F}}}
\def\gG{{\mathcal{G}}}
\def\gH{{\mathcal{H}}}
\def\gI{{\mathcal{I}}}
\def\gJ{{\mathcal{J}}}
\def\gK{{\mathcal{K}}}
\def\gL{{\mathcal{L}}}
\def\gM{{\mathcal{M}}}
\def\gN{{\mathcal{N}}}
\def\gO{{\mathcal{O}}}
\def\gP{{\mathcal{P}}}
\def\gQ{{\mathcal{Q}}}
\def\gR{{\mathcal{R}}}
\def\gS{{\mathcal{S}}}
\def\gT{{\mathcal{T}}}
\def\gU{{\mathcal{U}}}
\def\gV{{\mathcal{V}}}
\def\gW{{\mathcal{W}}}
\def\gX{{\mathcal{X}}}
\def\gY{{\mathcal{Y}}}
\def\gZ{{\mathcal{Z}}}

% Sets
\def\sA{{\mathbb{A}}}
\def\sB{{\mathbb{B}}}
\def\sC{{\mathbb{C}}}
\def\sD{{\mathbb{D}}}
% Don't use a set called E, because this would be the same as our symbol
% for expectation.
\def\sF{{\mathbb{F}}}
\def\sG{{\mathbb{G}}}
\def\sH{{\mathbb{H}}}
\def\sI{{\mathbb{I}}}
\def\sJ{{\mathbb{J}}}
\def\sK{{\mathbb{K}}}
\def\sL{{\mathbb{L}}}
\def\sM{{\mathbb{M}}}
\def\sN{{\mathbb{N}}}
\def\sO{{\mathbb{O}}}
\def\sP{{\mathbb{P}}}
\def\sQ{{\mathbb{Q}}}
\def\sR{{\mathbb{R}}}
\def\sS{{\mathbb{S}}}
\def\sT{{\mathbb{T}}}
\def\sU{{\mathbb{U}}}
\def\sV{{\mathbb{V}}}
\def\sW{{\mathbb{W}}}
\def\sX{{\mathbb{X}}}
\def\sY{{\mathbb{Y}}}
\def\sZ{{\mathbb{Z}}}

% Entries of a matrix
\def\emLambda{{\Lambda}}
\def\emA{{A}}
\def\emB{{B}}
\def\emC{{C}}
\def\emD{{D}}
\def\emE{{E}}
\def\emF{{F}}
\def\emG{{G}}
\def\emH{{H}}
\def\emI{{I}}
\def\emJ{{J}}
\def\emK{{K}}
\def\emL{{L}}
\def\emM{{M}}
\def\emN{{N}}
\def\emO{{O}}
\def\emP{{P}}
\def\emQ{{Q}}
\def\emR{{R}}
\def\emS{{S}}
\def\emT{{T}}
\def\emU{{U}}
\def\emV{{V}}
\def\emW{{W}}
\def\emX{{X}}
\def\emY{{Y}}
\def\emZ{{Z}}
\def\emSigma{{\Sigma}}

% entries of a tensor
% Same font as tensor, without \bm wrapper
\newcommand{\etens}[1]{\mathsfit{#1}}
\def\etLambda{{\etens{\Lambda}}}
\def\etA{{\etens{A}}}
\def\etB{{\etens{B}}}
\def\etC{{\etens{C}}}
\def\etD{{\etens{D}}}
\def\etE{{\etens{E}}}
\def\etF{{\etens{F}}}
\def\etG{{\etens{G}}}
\def\etH{{\etens{H}}}
\def\etI{{\etens{I}}}
\def\etJ{{\etens{J}}}
\def\etK{{\etens{K}}}
\def\etL{{\etens{L}}}
\def\etM{{\etens{M}}}
\def\etN{{\etens{N}}}
\def\etO{{\etens{O}}}
\def\etP{{\etens{P}}}
\def\etQ{{\etens{Q}}}
\def\etR{{\etens{R}}}
\def\etS{{\etens{S}}}
\def\etT{{\etens{T}}}
\def\etU{{\etens{U}}}
\def\etV{{\etens{V}}}
\def\etW{{\etens{W}}}
\def\etX{{\etens{X}}}
\def\etY{{\etens{Y}}}
\def\etZ{{\etens{Z}}}

% The true underlying data generating distribution
\newcommand{\pdata}{p_{\rm{data}}}
\newcommand{\ptarget}{p_{\rm{target}}}
\newcommand{\pprior}{p_{\rm{prior}}}
\newcommand{\pbase}{p_{\rm{base}}}
\newcommand{\pref}{p_{\rm{ref}}}

% The empirical distribution defined by the training set
\newcommand{\ptrain}{\hat{p}_{\rm{data}}}
\newcommand{\Ptrain}{\hat{P}_{\rm{data}}}
% The model distribution
\newcommand{\pmodel}{p_{\rm{model}}}
\newcommand{\Pmodel}{P_{\rm{model}}}
\newcommand{\ptildemodel}{\tilde{p}_{\rm{model}}}
% Stochastic autoencoder distributions
\newcommand{\pencode}{p_{\rm{encoder}}}
\newcommand{\pdecode}{p_{\rm{decoder}}}
\newcommand{\precons}{p_{\rm{reconstruct}}}

\newcommand{\laplace}{\mathrm{Laplace}} % Laplace distribution

\newcommand{\E}{\mathbb{E}}
\newcommand{\Ls}{\mathcal{L}}
\newcommand{\R}{\mathbb{R}}
\newcommand{\emp}{\tilde{p}}
\newcommand{\lr}{\alpha}
\newcommand{\reg}{\lambda}
\newcommand{\rect}{\mathrm{rectifier}}
\newcommand{\softmax}{\mathrm{softmax}}
\newcommand{\sigmoid}{\sigma}
\newcommand{\softplus}{\zeta}
\newcommand{\KL}{D_{\mathrm{KL}}}
\newcommand{\Var}{\mathrm{Var}}
\newcommand{\standarderror}{\mathrm{SE}}
\newcommand{\Cov}{\mathrm{Cov}}
% Wolfram Mathworld says $L^2$ is for function spaces and $\ell^2$ is for vectors
% But then they seem to use $L^2$ for vectors throughout the site, and so does
% wikipedia.
\newcommand{\normlzero}{L^0}
\newcommand{\normlone}{L^1}
\newcommand{\normltwo}{L^2}
\newcommand{\normlp}{L^p}
\newcommand{\normmax}{L^\infty}

\newcommand{\parents}{Pa} % See usage in notation.tex. Chosen to match Daphne's book.

\DeclareMathOperator*{\argmax}{arg\,max}
\DeclareMathOperator*{\argmin}{arg\,min}

\DeclareMathOperator{\sign}{sign}
\DeclareMathOperator{\Tr}{Tr}
\let\ab\allowbreak

\newcommand{\pch}[1]{\textcolor{black}{#1}}
\newcommand{\ymc}[1]{\textcolor{magenta}{#1}}
\newcommand{\todo}[1]{\textcolor{black}{#1}}
\newcommand{\ie}{\textit{i}.\textit{e}.,\ }
\newcommand{\eg}{\textit{e}.\textit{g}.,\ }

\newtheorem{theorem}{Theorem}[section]
\newtheorem{corollary}{Corollary}[theorem]
\newtheorem{lemma}[theorem]{Lemma}
\newtheorem{prop}[theorem]{Proposition}
\DeclareMathOperator{\cA}{\mathcal{A}}
\DeclareMathOperator{\cS}{\mathcal{S}}
\DeclareMathOperator{\cD}{\mathcal{D}}
\DeclareMathOperator{\cM}{\mathcal{M}}
\DeclareMathOperator{\cR}{\mathcal{R}}
\DeclareMathOperator{\cX}{\mathcal{X}}
\DeclareMathOperator{\cF}{\mathcal{F}}
\DeclareMathOperator{\cH}{\mathcal{H}}

%% Self-defined macros
\newcommand{\swap}[3][-]{#3#1#2} % just an example
\makeatletter
\newcommand{\multiline}[1]{%
  \begin{tabularx}{\dimexpr\linewidth-\ALG@thistlm}[t]{@{}X@{}}
    #1
  \end{tabularx}
}
\makeatother
%\title{Active Dynamics Model Selection for Offline Model-Based RL \\ via Bayesian Optimization}

\title{Enhancing Offline Model-Based RL via Active Model Selection:\\A Bayesian Optimization Perspective}

\author[1]{Yu-Wei Yang}
\author[1]{Yun-Ming Chan}
\author[1]{Wei Hung}
\author[2]{Xi Liu}
\author[1]{\href{mailto:<pinghsieh@nycu.edu.tw>?Subject=Enhancing Offline Model-Based RL via Active Model Selection: A Bayesian Optimization Perspective}{Ping-Chun Hsieh}}
% Add affiliations after the authors
\affil[1]{%
    Department of Computer Science\\
    National Yang Ming Chiao Tung University\\
    Hsinchu, Taiwan
}
\affil[2]{%
    Applied Machine Learning, Meta AI\\
    Menlo Park, CA, USA
}
\begin{document}
\maketitle

\begin{abstract}
Offline model-based reinforcement learning (MBRL) serves as a competitive framework that can learn well-performing policies solely from pre-collected data with the help of learned dynamics models. To fully unleash the power of offline MBRL, model selection plays a pivotal role in determining the dynamics model utilized for downstream policy learning. However, offline MBRL conventionally relies on validation or off-policy evaluation, which are rather inaccurate due to the inherent distribution shift in offline RL. To tackle this, we propose BOMS, an active model selection framework that enhances model selection in offline MBRL with only a small online interaction budget, through the lens of Bayesian optimization (BO). Specifically, we recast model selection as BO and enable probabilistic inference in BOMS by proposing a novel model-induced kernel, which is theoretically grounded and computationally efficient. Through extensive experiments, we show that BOMS improves over the baseline methods with a small amount of online interaction comparable to only $1\%$-$2.5\%$ of offline training data on various RL tasks.
\end{abstract}


%!TEX root = gcn.tex
\section{Introduction}
Graphs, representing structural data and topology, are widely used across various domains, such as social networks and merchandising transactions.
Graph convolutional networks (GCN)~\cite{iclr/KipfW17} have significantly enhanced model training on these interconnected nodes.
However, these graphs often contain sensitive information that should not be leaked to untrusted parties.
For example, companies may analyze sensitive demographic and behavioral data about users for applications ranging from targeted advertising to personalized medicine.
Given the data-centric nature and analytical power of GCN training, addressing these privacy concerns is imperative.

Secure multi-party computation (MPC)~\cite{crypto/ChaumDG87,crypto/ChenC06,eurocrypt/CiampiRSW22} is a critical tool for privacy-preserving machine learning, enabling mutually distrustful parties to collaboratively train models with privacy protection over inputs and (intermediate) computations.
While research advances (\eg,~\cite{ccs/RatheeRKCGRS20,uss/NgC21,sp21/TanKTW,uss/WatsonWP22,icml/Keller022,ccs/ABY318,folkerts2023redsec}) support secure training on convolutional neural networks (CNNs) efficiently, private GCN training with MPC over graphs remains challenging.

Graph convolutional layers in GCNs involve multiplications with a (normalized) adjacency matrix containing $\numedge$ non-zero values in a $\numnode \times \numnode$ matrix for a graph with $\numnode$ nodes and $\numedge$ edges.
The graphs are typically sparse but large.
One could use the standard Beaver-triple-based protocol to securely perform these sparse matrix multiplications by treating graph convolution as ordinary dense matrix multiplication.
However, this approach incurs $O(\numnode^2)$ communication and memory costs due to computations on irrelevant nodes.
%
Integrating existing cryptographic advances, the initial effort of SecGNN~\cite{tsc/WangZJ23,nips/RanXLWQW23} requires heavy communication or computational overhead.
Recently, CoGNN~\cite{ccs/ZouLSLXX24} optimizes the overhead in terms of  horizontal data partitioning, proposing a semi-honest secure framework.
Research for secure GCN over vertical data  remains nascent.

Current MPC studies, for GCN or not, have primarily targeted settings where participants own different data samples, \ie, horizontally partitioned data~\cite{ccs/ZouLSLXX24}.
MPC specialized for scenarios where parties hold different types of features~\cite{tkde/LiuKZPHYOZY24,icml/CastigliaZ0KBP23,nips/Wang0ZLWL23} is rare.
This paper studies $2$-party secure GCN training for these vertical partition cases, where one party holds private graph topology (\eg, edges) while the other owns private node features.
For instance, LinkedIn holds private social relationships between users, while banks own users' private bank statements.
Such real-world graph structures underpin the relevance of our focus.
To our knowledge, no prior work tackles secure GCN training in this context, which is crucial for cross-silo collaboration.


To realize secure GCN over vertically split data, we tailor MPC protocols for sparse graph convolution, which fundamentally involves sparse (adjacency) matrix multiplication.
Recent studies have begun exploring MPC protocols for sparse matrix multiplication (SMM).
ROOM~\cite{ccs/SchoppmannG0P19}, a seminal work on SMM, requires foreknowledge of sparsity types: whether the input matrices are row-sparse or column-sparse.
Unfortunately, GCN typically trains on graphs with arbitrary sparsity, where nodes have varying degrees and no specific sparsity constraints.
Moreover, the adjacency matrix in GCN often contains a self-loop operation represented by adding the identity matrix, which is neither row- nor column-sparse.
Araki~\etal~\cite{ccs/Araki0OPRT21} avoid this limitation in their scalable, secure graph analysis work, yet it does not cover vertical partition.

% and related primitives
To bridge this gap, we propose a secure sparse matrix multiplication protocol, \osmm, achieving \emph{accurate, efficient, and secure GCN training over vertical data} for the first time.

\subsection{New Techniques for Sparse Matrices}
The cost of evaluating a GCN layer is dominated by SMM in the form of $\adjmat\feamat$, where $\adjmat$ is a sparse adjacency matrix of a (directed) graph $\graph$ and $\feamat$ is a dense matrix of node features.
For unrelated nodes, which often constitute a substantial portion, the element-wise products $0\cdot x$ are always zero.
Our efficient MPC design 
avoids unnecessary secure computation over unrelated nodes by focusing on computing non-zero results while concealing the sparse topology.
We achieve this~by:
1) decomposing the sparse matrix $\adjmat$ into a product of matrices (\S\ref{sec::sgc}), including permutation and binary diagonal matrices, that can \emph{faithfully} represent the original graph topology;
2) devising specialized protocols (\S\ref{sec::smm_protocol}) for efficiently multiplying the structured matrices while hiding sparsity topology.


 
\subsubsection{Sparse Matrix Decomposition}
We decompose adjacency matrix $\adjmat$ of $\graph$ into two bipartite graphs: one represented by sparse matrix $\adjout$, linking the out-degree nodes to edges, the other 
by sparse matrix $\adjin$,
linking edges to in-degree nodes.

%\ie, we decompose $\adjmat$ into $\adjout \adjin$, where $\adjout$ and $\adjin$ are sparse matrices representing these connections.
%linking out-degree nodes to edges and edges to in-degree nodes of $\graph$, respectively.

We then permute the columns of $\adjout$ and the rows of $\adjin$ so that the permuted matrices $\adjout'$ and $\adjin'$ have non-zero positions with \emph{monotonically non-decreasing} row and column indices.
A permutation $\sigma$ is used to preserve the edge topology, leading to an initial decomposition of $\adjmat = \adjout'\sigma \adjin'$.
This is further refined into a sequence of \emph{linear transformations}, 
which can be efficiently computed by our MPC protocols for 
\emph{oblivious permutation}
%($\Pi_{\ssp}$) 
and \emph{oblivious selection-multiplication}.
% ($\Pi_\SM$)
\iffalse
Our approach leverages bipartite graph representation and the monotonicity of non-zero positions to decompose a general sparse matrix into linear transformations, enhancing the efficiency of our MPC protocols.
\fi
Our decomposition approach is not limited to GCNs but also general~SMM 
by 
%simply 
treating them 
as adjacency matrices.
%of a graph.
%Since any sparse matrix can be viewed 

%allowing the same technique to be applied.

 
\subsubsection{New Protocols for Linear Transformations}
\emph{Oblivious permutation} (OP) is a two-party protocol taking a private permutation $\sigma$ and a private vector $\xvec$ from the two parties, respectively, and generating a secret share $\l\sigma \xvec\r$ between them.
Our OP protocol employs correlated randomnesses generated in an input-independent offline phase to mask $\sigma$ and $\xvec$ for secure computations on intermediate results, requiring only $1$ round in the online phase (\cf, $\ge 2$ in previous works~\cite{ccs/AsharovHIKNPTT22, ccs/Araki0OPRT21}).

Another crucial two-party protocol in our work is \emph{oblivious selection-multiplication} (OSM).
It takes a private bit~$s$ from a party and secret share $\l x\r$ of an arithmetic number~$x$ owned by the two parties as input and generates secret share $\l sx\r$.
%between them.
%Like our OP protocol, o
Our $1$-round OSM protocol also uses pre-computed randomnesses to mask $s$ and $x$.
%for secure computations.
Compared to the Beaver-triple-based~\cite{crypto/Beaver91a} and oblivious-transfer (OT)-based approaches~\cite{pkc/Tzeng02}, our protocol saves ${\sim}50\%$ of online communication while having the same offline communication and round complexities.

By decomposing the sparse matrix into linear transformations and applying our specialized protocols, our \osmm protocol
%($\prosmm$) 
reduces the complexity of evaluating $\numnode \times \numnode$ sparse matrices with $\numedge$ non-zero values from $O(\numnode^2)$ to $O(\numedge)$.

%(\S\ref{sec::secgcn})
\subsection{\cgnn: Secure GCN made Efficient}
Supported by our new sparsity techniques, we build \cgnn, 
a two-party computation (2PC) framework for GCN inference and training over vertical
%ly split
data.
Our contributions include:

1) We are the first to explore sparsity over vertically split, secret-shared data in MPC, enabling decompositions of sparse matrices with arbitrary sparsity and isolating computations that can be performed in plaintext without sacrificing privacy.

2) We propose two efficient $2$PC primitives for OP and OSM, both optimally single-round.
Combined with our sparse matrix decomposition approach, our \osmm protocol ($\prosmm$) achieves constant-round communication costs of $O(\numedge)$, reducing memory requirements and avoiding out-of-memory errors for large matrices.
In practice, it saves $99\%+$ communication
%(Table~\ref{table:comm_smm}) 
and reduces ${\sim}72\%$ memory usage over large $(5000\times5000)$ matrices compared with using Beaver triples.
%(Table~\ref{table:mem_smm_sparse}) ${\sim}16\%$-

3) We build an end-to-end secure GCN framework for inference and training over vertically split data, maintaining accuracy on par with plaintext computations.
We will open-source our evaluation code for research and deployment.

To evaluate the performance of $\cgnn$, we conducted extensive experiments over three standard graph datasets (Cora~\cite{aim/SenNBGGE08}, Citeseer~\cite{dl/GilesBL98}, and Pubmed~\cite{ijcnlp/DernoncourtL17}),
reporting communication, memory usage, accuracy, and running time under varying network conditions, along with an ablation study with or without \osmm.
Below, we highlight our key achievements.

\textit{Communication (\S\ref{sec::comm_compare_gcn}).}
$\cgnn$ saves communication by $50$-$80\%$.
(\cf,~CoGNN~\cite{ccs/KotiKPG24}, OblivGNN~\cite{uss/XuL0AYY24}).

\textit{Memory usage (\S\ref{sec::smmmemory}).}
\cgnn alleviates out-of-memory problems of using %the standard 
Beaver-triples~\cite{crypto/Beaver91a} for large datasets.

\textit{Accuracy (\S\ref{sec::acc_compare_gcn}).}
$\cgnn$ achieves inference and training accuracy comparable to plaintext counterparts.
%training accuracy $\{76\%$, $65.1\%$, $75.2\%\}$ comparable to $\{75.7\%$, $65.4\%$, $74.5\%\}$ in plaintext.

{\textit{Computational efficiency (\S\ref{sec::time_net}).}} 
%If the network is worse in bandwidth and better in latency, $\cgnn$ shows more benefits.
$\cgnn$ is faster by $6$-$45\%$ in inference and $28$-$95\%$ in training across various networks and excels in narrow-bandwidth and low-latency~ones.

{\textit{Impact of \osmm (\S\ref{sec:ablation}).}}
Our \osmm protocol shows a $10$-$42\times$ speed-up for $5000\times 5000$ matrices and saves $10$-2$1\%$ memory for ``small'' datasets and up to $90\%$+ for larger ones.


\section{Preliminaries and Notations}

% \subsection{Downstream Tasks in Genomic Sequence Modelling}

\subsection{Genomic Sequence Modeling} 
DNA is a polymer made of up four types of nucleotides: Adenine (\textit{A}), Thymine (\textit{T}), Guanine (\textit{G}), and Cytosine (\textit{C}). Let $ \mathbb{N}_4 = \{A, T, G, C\}$. A DNA sequence of length $T$, denoted as $\mathbf{x} = {(\mathbf{x}_1, \mathbf{x}_2,...,\mathbf{x}_T)} \in \mathbb{N}^T_4$, follows a natural distribution $\mathbf{x} \sim p_\theta(\mathbf{x})$. We use $p_{\hat{\theta}}(\mathbf{x})$ to represent an estimate to the true distribution. The dataset of unlabeled genomic sequences is given in the form of \( \{\mathbf{x}^{(i)}\}_{i=1}^N\).

Genomic sequence modeling aims to learn a function $f$ that maps input sequences to biological annotations using a labeled dataset \(\mathcal{D} = \{\mathbf{x}^{(i)}, \mathbf{y}^{(i)}\}_{i=1}^N\). The type of $y^{(i)}$ varies depending on the types of tasks: $y^{(i)}$ is a class label in \textbf{DNA Sequence Classification}~\cite{grevsova2023genomic,dalla2024nucleotide}. Or a real value vector in \textbf{Genomic Assay Prediction} tasks~\cite{avsec2021effective,linder2025predicting}. Current genomic sequence models typically follow a two-stage strategy. In the pretraining phase, we learn the data distribution $p_{\hat{\theta}}(x)$ on a unlabeled dataset from unlabeled data using losses such as masked language modeling (MLM)
$p(\mathbf{x}) = \prod_{t \in \mathcal{M}} p_\theta(\mathbf{x}_t \mid \mathbf{x}_{1:t-1}, \mathbf{x}_{t+1:T})$ 
or next token prediction (NTP) $p(\mathbf{x}) = \prod_{t=1}^{T}p_\theta(\mathbf{x}_t | \mathbf{x}_{1:t-1})$. 

%To adapt into downstream tasks, we update the posterior distribution of $\theta_{\text{task}}$ given a labeled dataset $\mathcal{D}$ and the pretrained $\hat{\theta}$.



% \subsection{Genomic Sequence Modeling}
% \textbf{What}\textit{ is the target}? Similar to earlier work of language modeling~\citep{wang2018glue,wang2019superglue}, the goal of genomic sequence modeling is to learn a function $f:\mathcal{D} \rightarrow \mathcal{Y} $, that maps input sequences to their corresponding biological annotations. Given a labelled dataset \(\mathcal{D} = \{\mathbf{x}^{(n)}, \mathbf{y}^{(n)}\}_{n=1}^N\), with the annotation $\mathbf{y}^{(i)}$. Current genomic tasks consist of three primary classes based on biological annotation type: \textit{i.)} \textbf{DNA Sequence Classification} includes tasks like regulatory region identification~\cite{grevsova2023genomic} and histone marker prediction~\citep{dalla2024nucleotide}. In this task, 
% $f$ directly map a given sequence $\mathbf{x}^{(i)}$ to a class label $y^{(i)}$. \textit{ii.)} \textbf{Genomic Assay Prediction}: this includes works from ~\cite{avsec2021effective,linder2025predicting}. $\mathbf{y}^{(n)}$ is a real value vector. Typically this task has longer sequence length $T$ ranging from 10K to 100K. \textit{iii.)} \textbf{Genetic Variant Prediction} standout as a separate tasks, taking the reference sequence, mutated sequence and metadata as input, to predict class labels (e.g., pathogenic vs. benign variants). Such predictions are critical for clinical applications, aiding in diagnosis and personalized treatment strategies. Also exsit other more diverse tasks. For example, there are general interests  in learning a direct function mapping from DNA to the function of it. 

% \textbf{How} \textit{is sequence modeling performed}? A function $f$ solving the above proposed question, can be directly learned end-to-end with a labeled dataset $\mathcal{D}$ using supervised methods like CNNs ~\cite{} and more recently ``pretrain - finetune'' paradigm, which is becoming the dominant paradigm for GFMs. It involves two stages in which we first learn the data distribution $p_{\hat{\theta}}$ on a unlabeled dataset (pretraining), and then subsequently learn the posterior distribution $p(f|\mathcal{D},\hat{\theta})$ (finetuning). This approach has driven the state-of-the-art performance in many models (e.g. HyenaDNA, DNABERT-2, Nucleotide transformer) across various genomic tasks. For instance, Genomic Pre-trained Network (GPN)~\citep{benegas2023dna} leveraged this approach and achieved state-of-the-art (SoTA) performance for genetic variant prediction. The parameters $\theta$ are typically obtained through different pretraining strategies, common approaches include masked language modeling (MLM)  and next token prediction (NTP), formalized in \cref{eq:1} and  \cref{eq:2}, respectively. Here, $\mathcal{M}$ denotes the set of indices for masked tokens, and the model learns to reconstruct the original sequence or predict subsequent tokens based on the unmasked context.

% \begin{equation} \label{eq:1}
% p^{\text{MLM}}(\mathbf{x}) = \prod_{i \in \mathcal{M}} p_\theta(\mathbf{x}_i \mid \mathbf{x}_{1:i-1}, \mathbf{x}_{i+1:T})
% \end{equation}
% \begin{equation} \label{eq:2}
%     p^{AR}(\mathbf{x}) = \prod_{i=1}^{T}p_\theta(\mathbf{x}_i | \mathbf{x}_{1:i-1}).
% \end{equation}

% \paragraph{\textit{Limitations} of Existing Models} Existing genomic foundation models vary significantly in architecture and tokenization strategies, but fundamental questions remain unresolved. First, despite the widespread adoption of pretraining objectives like masked language modeling (MLM) and next token prediction (NTP), there is no consensus on which objective is more effective for genomic sequence modeling. Due to the prohibitive computational cost of pretraining, there is a lack of systematic comparisons to answer this question. While many believe the bidirectional training enables the model to better learn a complete context and genomic element interaction, recent work by \citet{allen2023physics} argues that MLM may inadvertently disrupt long-range sequence dependencies, limiting knowledge retention.  Secondly, existing approaches typically require training separate task-specific models, meaning $f$ needs to be trained multiple times given K tasks $\mathcal{D}$. This results in 1) the additional cost of storing \( \mathcal{O}(K) \) copies of model weights, incurring significant I/O latency, memory costs, and context-switching penalties 2) prevents models from leveraging shared biological patterns (e.g., conserved regulatory motifs or chromatin accessibility signatures) across tasks. This isolation limits generalization and ignores cross-task dependencies formalized in the joint distribution \( p(f \mid \mathcal{D}_1, \dots, \mathcal{D}_K, \theta) \), where \( \mathcal{D}_k \) represents data for task \( k \).

\subsection{Supervised Finetuning}

Supervised Finetuning (SFT) plays a key role in enhancing the instruction-following~\citep{mishra2021reframing,sanh2021multitask,wei2022chain} and reasoning capabilities~\citep{lambert2024t}. For a pretrained auto-regressive model with a fixed vocabulary \( V_x \) and a labeled dataset  $\mathcal{D}$, SFT maximizes the likelihood:
$
\hat{\theta} = \arg\max_{\theta} \sum_{i=1}^N \log p_\theta\left(\mathbf{y}^{(i)} \mid \mathbf{x}^{(i)}\right).
$
%where \( \mathbf{x}^{(i)} \) and \( \mathbf{y}^{(i)} \) are input-output pairs. 
This process retains the model’s pretrained knowledge while aligning its outputs with task-specific objectives, typically using \textbf{cross-entropy loss} on the target tokens:
\begin{equation} \label{eq:3} \small
    p_\theta(\mathbf{y}^{(i)} | \mathbf{x}^{(i)}) = \sum_{i=1}^N \sum_{t=1}^{T'} \log p_\theta(\mathbf{y}_t^{(i)} | \mathbf{y}^{(i)}_{1:t-1},\mathbf{x}^{(i)}).
\end{equation}
% \begin{equation} \label{eq:3} \small
%     p^{AR}(\mathbf{y}) = \prod_{t=1}^{T'}p_\theta(\mathbf{y}_t | \mathbf{y}_{1:t-1},\mathbf{x}).
% \end{equation}
Notably, $\mathbf{y}$ could include new set of vocabulary $V_{z}$ that do not overlap with $V_{o}$. Therefore, each term in \cref{eq:3} is computed through \Cref{eq:4}.
\begin{equation} \label{eq:4} 
\small
    p_\theta(\mathbf{y}_t \mid \mathbf{y}_{1:t-1},\mathbf{x}) = \frac{\exp(h_{t-1}^\top e_{\mathbf{y}_t})}{\sum\limits_{m \in V_{o}} \exp(h_{t-1}^\top e_m) + \sum\limits_{n \in V_{z}} \exp(h_{t-1}^\top e_n)},
\end{equation} 
where $h_{t-1}$ is the neural representation of the prefix sequence $(\mathbf{y}_{1:t-1},\mathbf{x})$, $e_m$ is the embedding of vocab $m$.

As a result, during the finetuning stage, the embeddings for each new vocabulary token \( n \in V_z \) must be initialized, and the original token probabilities are shifted due to the expanded output space, as detailed in~\citet{hewitt2021initializing}.

% A critical challenge during SFT lies in preserving the model’s pretrained knowledge while adapting it to downstream tasks, necessitating careful balancing of task-specific learning with mitigation of catastrophic forgetting~\cite{zheng2024towards}. 

% SFT offers distinct advantages over traditional approaches like classifier head augmentation. First, it enables unified adaptation to multiple downstream tasks through a single finetuning process. Second, it supports complex output generation—such as mapping DNA sequences to structured multimodal outputs —rather than simple class labels. While SFT has been extensively studied in language models, recent work demonstrates its effectiveness in cross-modal settings, including biological sequences ~\cite{jiang2024neurolm}. In this paper, we extend SFT to pretrained autoregressive genomic language models, enabling flexible adaptation to diverse sequence-to-function prediction tasks while preserving foundational knowledge of genomic grammar and patterns.

%\section{A Bayesian Optimization Framework for Model Selection}
\section{Methodology}
\label{sec:boms}
\vspace{-1mm}
%In this section, we formally present the proposed BO approach for model selection. We start from the overall BOMS framework and proceed to introduce the proposed kernel and the practical implementation of BOMS.
%In this section, we formally present the proposed BOMS framework and introduce the proposed kernel and the implementation.
%The existing offline model-based RL approaches often opt for training policies using the dynamics model with the lowest validation loss achieved through supervised learning. Nevertheless, it is worth noting that the validation dataset typically have limited coverage, potentially making it insufficient to comprehensively evaluate the performance of the dynamics model across the entire state-action space in the context of model training. 
%To select a proper dynamics model, one straightforward method is to use OPE and compare the estimated total reward of the policy learned from each dynamics model. However, as mentioned in Section \ref{sec:intro}, OPE suffers from limited accuracy due to the distribution shift problem and can arrive at a rather sub-optimal policy. Motivated by these concerns, we propose Bayesian Optimization for Model Selection, an offline model-based RL algorithm that incorporates Bayesian Optimization and the limited environment interaction budget for dynamics model selection in offline RL.} 

%\subsection{Method Structure}
\subsection{Recasting Model Selection as BO}
\label{sec:boms:framework}
\vspace{-1mm}
We start by connecting the active model selection problem to BO. Specifically, below we interpret each component of model selection in the language of BO.
\vspace{-1mm}
\begin{itemize}[leftmargin=*]
    \item \textit{Input domain $\cX$}: Recall that offline model-based RL first uses the offline dataset $\cD_{\text{off}}$ to generate a collection of ${N}$ dynamics models denoted by $\cM$ ${= \{M^{(1)}, M^{(2)}, ..., M^{(N)}\}}$, which serve as the candidate set for model selection. We view this candidate set $\cM$ as the input domain in BO, \ie $\cM\equiv \cX$.
    %\vspace{-1mm}
    \item \textit{Black-box objective function}: The goal of model selection is to choose a dynamics model from $\cM$ such that the policy learned downstream enjoys high expected return. Let $\pi^{(i)}$ be the policy learned under $M^{(i)}\in \cM$, and let $J_{M^*}^{\pi^{(i)}}$ be the true total expected return achieved by $\pi^{(i)}$. Accordingly, model selection can be viewed as black-box maximization with an unknown objective function $f \equiv J^{\pi}_{M^*}$ defined on $\cM$.
    %\vspace{-1mm}
    \item \textit{Sampling procedure}: In the context of model selection, each sample corresponds to evaluating the policy learned from a model $M'\in \cM$. We let $M_t$ and $J^{\pi_t}_{M^*}$ denote the model selected and the corresponding policy performance at the $t$-th iteration, respectively. Let $\cH_t:=\{(M_i,J^{\pi_i}_{M^*})\}_{i=1}^{t}$ denote the sampling history up to the $t$-th iteration. To determine the $(t+1)$-th sample, we take any off-the-shelf AF, compute $\Psi(M'; \cH_t)$ for each model $M'\in \cM$, and then select the one with the largest AF value, \ie $M_{t+1}\in \arg\max_{M'\in \cM} \Psi(M'; \cH_t)$. 
    In our experiments, we use the celebrated GP-UCB \citep{srinivas2012information} as the AF.
    Then, upon sampling, we learn a policy $\pi_{t+1}$ based on the chosen dynamics model $M_{t+1}$ (\eg by applying SAC), and the function value $J^{\pi_{t+1}}_{M^*}$ can be obtained and included in the history. In practice, we use the Monte-Carlo estimates $R_{t+1}$ an an approximation of the true $J^{\pi_{t+1}}_{M^*}$.
    %\vspace{-1mm}
    \item \textit{GP function prior}:  As in standard BO, here we impose a GP function prior that implicitly captures the structural properties of the objective function. Notably, GP serves as a surrogate model for facilitating probabilistic inference on the unknown function values. Specifically, under a GP prior, we assume that for any subset of models $\cM^\dagger \subseteq \cM$, their function values follow a multivariate normal distribution with mean and covariance characterized by a mean function $m:\cM\rightarrow \mathbb{R}$ and a covariance function $k:\cM\times \cM \rightarrow \mathbb{R}$. As in the standard BO literature, we simply take $m$ as a zero function. Recall that $R_{t}$ denotes the Monte-Carlo estimate of the true $J^{\pi_{t}}_{M^*}$. Let $\bm{R}_t$ denote the vector of all $R_i$, for $i\in \{1,\cdots,t\}$. For convenience, we let $\cM_t$ denote the set of models selected in the first $t$ iterations. 
    For any pair of model subset $\cM',\cM''$, define $\bm{K}(\cM',\cM'')$ to be a $\lvert \cM'\rvert \times \lvert \cM''\rvert$ covariance matrix, where the entries are the covariances $k(M',M'')$ of $M'\in \cM'$ and $M''\in \cM''$.
    Then, given the history $\cH_t$, the posterior distribution of $J^{\pi}_{M^*}$ of each model $M\in \cM$ follows a normal distribution $\mathcal{N}(\mu_t(M),\sigma^2_t(M))$, where
    \begin{align}
    \mu_t(M) &= {\bm{K}(M, \cM_t)} \bm{K}(\cM_t,\cM_t)^{-1} \bm{R}_t,   \\
    \begin{split}    
    \sigma^2_t(M) &= \bm{K}(M, M)\\
        \quad -  &{\bm{K}(M,\cM_t)} \bm{K}(\cM_t,\cM_t)^{-1} \bm{K}(\cM_t, M).
    \end{split}
    \end{align}
    \item \textit{GP kernels for the covariance function}: To specify the covariance function $k(\cdot,\cdot)$ defined above, we adopt the common practice in BO and use a kernel function. For each pair of models $M',M''\in \cM$, under some pre-configured distance $d:\cM\times\cM\rightarrow [0,\infty)$, we use the popular Radial Basis Function (RBF) kernel \citep{williams2006gaussian}
    \begin{equation}
    k(M', M'') := \exp(-\frac{d(M', M'')^\text{2}}{\text{2}\ell^\text{2}}),\label{eq:RBF}
    \end{equation}
    where $\ell$ is the kernel lengthscale and ${d(M', M'')}$ is the model distance between $M'$ and $M''$.
    The overall procedure of BOMS is summarized in Algorithm \ref{algo:boms}. 
    %between the currently selected model ${M_t}$ and other dynamics model candidates ${M_j}$, ${1 \le j \le N}$. 
    %\item \textit{Acquisition function}: To determine the next dynamics model ${M_{t+\text{1}}}$ to evaluate in the environment, we need an acquisition function that quantifies the potential of an input point to improve the current best solution. The acquisition function balances exploration and exploitation based on the GP model's uncertainty. In our approach, we leverage UCB as the acquisition function and choose the dynamics model with the highest UCB value, emphasizing both high expected value (exploitation) and high uncertainty (exploration) in the estimation: $M_{t+\text{1}} = \arg\max_m (\mu_t(m)+\kappa \: \sigma_t(m))$, where ${\kappa}$ is the hyperparameter that controls the balance between exploration and exploitation. 
    %Specifically, BO considers prior information of the function ${f}$ and refines the prior using samples drawn from ${f}$. This process results in a posterior distribution that provides a more accurate approximation of ${f}$. We typically use a Gaussian process (GP) as the surrogate function to approximate ${f}$, which provides a probabilistic estimation of the objective function's behavior. Since we suppose the similar dynamics models generate similar values, it is reasonable to let the collected dynamics models $\cM$ be candidate points and the corresponding empirical return set $\cR$ ${= \{R_1, R_2, ..., R_N\}}$ be the objective function values, where $R= \sum_{n=1}^{\infty} r_n$. With the posterior mean ${\mu}$ and variance ${\sigma^2}$, we can compute the probability distributions of the GP as follows:
%where ${k}$ is the vector of kernel values between the new model point ${M_t}$ and all the training model points $\cM$, and K is the covariance matrix given by the kernel function applied to all possible pairs of candidate points. The kernel function K defines the shape and characteristics of the GP model and is used to capture the underlying relationships between data points. Formally, the kernel function in our approach is the radial basis function (RBF) and can be denoted as the following function:
\end{itemize}
%In our approach, we first assume that similar dynamic models have similar model values. We use the fixed dataset $\cD$ to generate ${N}$ dynamics models as the candidates for the model selection, and the collected model set can be denoted as $\cM$ ${= \{M_1, M_2, ..., M_N\}}$. In each selection epoch ${t}$, we choose one dynamics model ${M_t}$ with the highest score from the acquisition function based on BO, and then train the optimal policy ${\pi_t}$ of the selected model ${M_t}$ for the environmental interaction and gain the empirical returns ${R_t}$ from the limited trajectories. By the assumption above, we can update GP and choose the next promising model by taking into account the recently collected empirical returns and the kernel function. At the end of the selection phase, the dynamics model with the highest empirical return is our final dynamics model selection result. The pseudo-code of BOMS is demonstrated in Algorithm~\ref{algo:boms}.

\begin{algorithm}[ht]
\caption{BOMS}\label{algo:boms}
\begin{algorithmic}
\Require True environment $M^*$, candidate model set $\cM$, and total number of selection iterations $T$.
\For{$t \gets 1$ to $T$}
    \If{$t=1$} 
        \State Randomly choose a model $M_1$ from $\cM$.
    \Else
        \State \multiline{Update the posterior $\mathcal{N}(\mu_t(M),\sigma^2_t(M))$ of each dynamics model $M\in \cM$.}
        %\State \multiline{Fit GP model with new empirical average return ${R_t}$ and K.}
        \State 
            %\multiline{Select the next model $M_t\in \cM$ by using an acquisition function $\Psi$ as ${M_{t}}\in \arg\max_{M\in \cM}\Psi(M;\cH_t)$}
            \multiline{Select ${M_{t}}\in \arg\max_{M\in \cM}\Psi(M;\cH_t)$.}
    \EndIf
    \State 
        \multiline{Learn a policy ${\pi_{t}}$ on the selected model ${M_{t}}$.}
    \State 
        \multiline{Evaluate $\pi_t$ by an estimate of empirical total reward $R_t$ based on online interactions with $M^*$.}
        %\multiline{Using ${\pi_{t}}$ to interact with the \textbf{env} and get the ${R_{t}}$ of limited trajectories.}
    \State 
        \multiline{Calculate the model distance $d(M_{t},M')$ between ${M_{t}}$ and all other models $M'\in \cM$.}
\EndFor

\State Return the model ${M}_{t^*}$ with $t^*:=\arg\max_{1\leq t\leq T}R_t$.
\end{algorithmic}
\end{algorithm}

\vspace{-1mm}
\noindent{\textbf{Remarks on realism of obtaining a candidate model set $\cM$:} We can naturally obtain $\cM$ by collecting the dynamics models learned during training under any offline MBRL method, without any additional training overhead. For example, in Section \ref{sec:exp}, we follow the model training procedure of MOPO and take the last 50 models as candidate models. This also induces a fair comparison between MOPO and BOMS as they both see the same group of dynamics models.} 
\vspace{-1mm}
%Subsequently, we train the optimal policy ${\pi_{t+\text{1}}}$ of the dynamics model ${M_{t+\text{1}}}$ selected from BO and apply the trained policy in the environment to get the empirical return of the limited trajectories. Then, ${R_{t+\text{1}}}$ is used to update ${\mu}$ and ${\sigma^\text{2}}$ of the GP for the next model selection epoch.


%; initial states ${\omega}$;

%\subsection{Theoretical Insights for Kernel Design}
%\label{sec:boms:kernel}

\subsection{Model-Induced Kernels for BOMS}
\label{sec:boms:kernel}
To enable BOMS, one major challenge is to measure the distance $d(M',M'')$ between the dynamics models required by the kernel function. However, in offline MBRL, it remains unknown how to characterize the distance between dynamics models in a way that nicely reflects the smoothness in the total reward. 
\vspace{-1mm}

\noindent{\textbf{Theoretical insights for BOMS kernel design:} To tackle the above challenge, we provide useful theoretical results that can motivate the subsequent design of a distance measure for offline MBRL. For convenience, as all the MDPs considered in this paper share the same $\cS, \cA, \omega$, and $\gamma$, in the sequel, we use a two-tuple $(P,r)$ to denote a model $M$, with a slight abuse of notation. Recall that $\pi_\beta$ denotes the behavior policy of the offline dataset. Here we use MOPO as an example and consider the penalized MDPs  $\widetilde{M}=({P},\widetilde{r})$, where $\widetilde{r}$ is the reward function penalized by the uncertainty estimator $u(s,a)$ and denoted as $\widetilde{r}(s,a) = r(s,a) - \lambda u(s,a)$ by MOPO, as described in Section \ref{sec:prelim}. 
To establish the following proposition, we make one regularity assumption that the value functions of interest are Lipschitz continuous in state, \ie there exist $L>0$ such that $\lvert V^{\pi}_{M}(s)-V^{\pi}_{M}(s)\rvert \leq L\cdot \lVert s-s'\lVert$, for all $s,s'\in \cS$, and ${L}$ is the Lipschitz constant.
We state Proposition~\ref{prop:model_dis} below, and the proof  is in Appendix \ref{app:proof}.}
\begin{restatable}{prop}{difprop}
%\begin{prop}
\label{prop:model_dis}
Given two policies $\widetilde{\pi}_1$ and $\widetilde{\pi}_2$ which are the optimal policies learned on models $\widetilde{M}_1=(P_1,r_1)$ and $\widetilde{M}_2=(P_2,r_2)$, respectively. Then, we have 
\begin{align*}
    &J^{\widetilde{\pi}_1}_{M^*}(\omega) - J^{\widetilde{\pi}_2}_{M^*}(\omega) \le \Big( J^{\widetilde{\pi}_1}_{M^*}(\omega)-J^{\pi_\beta}_{M^*}(\omega) \Big)+2\lambda \epsilon(\pi_\beta) \\
    &+\frac{1}{1-\gamma} {\mathbb{E}} \Big[ \gamma L\norm\big{s'_1 - s'_2}_1 + \big| r_1(s,a)-r_2(s,a)\big| \Big],
\end{align*}
where the expectation is taken over $s\sim\omega$, $a\sim\widetilde{\pi}_1$, $s'_1\sim {P_1}(\cdot\rvert s,a)$, and $s'_2\sim {P_2}(\cdot \rvert s,a)$ and $\epsilon(\pi_\beta):= \mathbb{E}_{(s,a)\sim \rho^{\pi_\beta}}[u(s,a)]$ is the modeling error under behavior policy $\pi_\beta$.
%\end{prop}
\end{restatable}

\noindent\textbf{Insights offered by Proposition \ref{prop:model_dis}}: Notably, Proposition~\ref{prop:model_dis} shows that the policy performance gap of the learned models evaluated in the true environment is bounded by three main factors: 
(i) The first term is the expected discounted return difference between a learned policy and the behavior policy $\pi_\beta$. This term can be considered fixed since it does not change when we measure the distances between the selected model ${M}_t$ and other candidate models.
(ii) The second term is the modeling error along trajectories generated by $\pi_\beta$. Under MOPO, $\epsilon_u(\pi_\beta)$ should be quite small as $u(s,a)$ mainly estimates the discrepancy between the learned transition $\hat{P}$ and the true transition ${P}$, where $\hat{P}$ is learned from the dataset yielded from $\pi_\beta$. Therefore, for state-action pairs from $\rho^{\pi_\beta}$, $\hat{P}$ is close to the true transition ${P}$, and thus $\epsilon_u(\pi_\beta)$ should be relatively small. 
(iii) The third term is the next-step predictions difference between two dynamics models given $\widetilde{\pi}_1$. 
Moreover, note that the first two terms depend on the behavior policy, which is completely out of control by us, and the third term is the only term that reflects the discrepancy between $\widetilde{M_1}$ and $\widetilde{M}_2$. As a result, only the third term matters in measuring of the performance gap between dynamics models in BOMS.

\noindent\textbf{Model-induced kernels:} Built on the BOMS framework in Section \ref{sec:boms:framework} and the theoretical grounding provided above, we are ready to present the kernel used in BOMS. Recall from (\ref{eq:RBF}) that we use an RBF kernel in the design of the covariance matrix. To leverage Proposition \ref{prop:model_dis} in the context of BOMS, we take ${\pi}_1$ as the policy learned from the selected model $M_t=(P_t,r_t)$ at the $t$-th iteration, and ${\pi}_2$ be the policy from another candidate model $M=(P,r)$ in $\cM$. Then, motivated by Proposition \ref{prop:model_dis}, we design the model distance
\begin{equation}
    d(M_t, M):= \displaystyle\mathop{\mathbb{E}}\bigg[\lVert s_1'-s_2'\rVert+\alpha \big\lvert r_t(s,a)-r(s,a)\big\rvert\bigg],\label{eq:distance}
\end{equation} 
where the expectation is taken over $s\sim \cD_{\text{off}}$, $a \sim \pi_t(\cdot\rvert s)$, $s_1'\sim P_t(\cdot\rvert s,a)$, and $s_2'\sim P(\cdot\rvert s,a)$, and $\alpha$ is a parameter that balances the discrepancies in states and rewards. In the experiments, we find that $\alpha=1$ is generally a good choice.
%Inspired by Lemma mentioned above, we define ${d(M_t, M_j)}$ as the Euclidean distance between the next-step predictions of models. 
In practice, we randomly sample a batch of states from the fixed dataset $\cD_{\text{off}}$ to obtain empirical estimates of the above distance $d(M_t, M)$. Additionally, since the covariance function $k(\cdot,\cdot)$ in BO is typically symmetric, we further impose a condition that the model distance $d$ enjoys symmetry, \ie ${d(M_t, M_j)}$ is equal to ${d(M_j, M_t)}$.

\vspace{1mm}
\noindent\textbf{Complexity of the proposed kernel:} Under BOMS with the proposed kernel and $N$ candidate models, the calculation of the covariance terms at the $t$-th iteration takes $O(t^2+tN)$, which is the same as standard BO. Hence, the proposed kernel does not introduce additional computational overhead.

\begin{comment}
    \begin{prop}
\label{prop:model_dis}
Given two policies ${\pi}_i$ and ${\pi}_j$ which are learned from models ${M_i}$ and ${M_j}$ respectively. Let ${L}$ be the Lipschitz constant and $\epsilon_u(\pi)= \displaystyle\mathop{\mathbb{E}}_{ (s,a)\sim\rho^\pi} \left[u(s,a)\right]$, where $\rho^\pi$ is the discounted state-action probability distribution under policy $\pi$, and let $\mathop{\mathbb{E}}$ be a shorthand for $\displaystyle \mathop{\mathbb{E}}_{\scriptstyle s\sim\omega, a\sim{\pi}_i, \atop \scriptstyle s'_1\sim \widetilde{M_i}(s,a), s'_2\sim \widetilde{M_j}(s,a)}$,  we have: 
\begin{align*}
    &J^{{\pi}_i}_{M^*}(\omega) - J^{{\pi}_j}_{M^*}(\omega) \le  \\
    &\frac{1}{1-\gamma} \mathop{\mathds{E}} \Big[ \gamma L\norm\big{s'_1 - s'_2}_1 + \big| r_i(s,a)-r_j(s,a)\big| \Big] + 2\lambda \epsilon_u(\pi_\beta) \\
    & + \Big( J^{{\pi}_i}_{M^*}(\omega)-J^{\pi_\beta}_{M^*}(\omega) \Big)
\end{align*}
\end{prop}
\end{comment}

%By referencing the theoretical sections in MOPO and the work from \citep{vemula2023virtues}, 
%there are not many works that discuss the definition of the distance between different dynamics models. By referencing the theoretical sections in MOPO and the work from \citep{vemula2023virtues}, we state Proposition~\ref{prop:model_dis} to give our model distance method the theoretical support.


\begin{comment}
\subsection{Practical Implementation}
For the practical implementation part, we use MOPO as our offline model-based RL framework. Each dynamics model $m$ corresponds to the penalized MDP $\widetilde{M}= \{ S,A,\widetilde{r},\widehat{P}, \omega, \gamma \}$ with different transition function $\widehat{P}$ respectively, where $\widetilde{r}$ is the reward function penalized by the uncertainty estimator $u(s,a)$ and denoted as $\widetilde{r} = r - \lambda u(s,a)$ by MOPO. 

%Having an appropriate method to measure the model distance plays an important part in the BO kernel function. However, in the field of offline model-based RL, there are not many works that discuss the definition of the distance between different dynamics models. By referencing the theoretical sections in MOPO and the work from \citep{vemula2023virtues}, we state Proposition~\ref{prop:model_dis} to give our model distance method the theoretical support.

In BO model selection stage, assume ${\pi}_i$ be the policy learned from the selected model $M_t$ at time $t$, and ${\pi}_j$ be the policy from another candidate. For the second term, MOPO indicates that $\epsilon_u(\pi_\beta)$ should be quite small. $u(s,a)$ mainly estimate the transition disparity between the learned transition $\hat{P}$ and the true transition ${P}$, which $\hat{P}$ trained from the dataset yielded from $\pi_\beta$. Therefore, for state-action pairs from $\rho^\pi$, $\hat{P}$ should be close to the true transition ${P}$, and thus $\epsilon_u(\pi_\beta)$ should be relatively small. The third term can be considered constant since the expected return difference between ${\pi}_i$ and behavior policy does not change when we measure the different distances between the selected model ${M}_t$ and other models. As a result, only the first term matters in the measurement of the performance gap between the two dynamics models in our BO selection phase.
\begin{gather}
    d(M_t, M_j) = \displaystyle\mathop{\mathbb{E}}_{s\sim D, a \sim \pi_t} \: d(M_t(s,a), M_j(s,a))
\end{gather}
Inspired by Lemma mentioned above, we define ${d(M_t, M_j)}$ as the Euclidean distance between the next-step predictions of models. For the practical implementation, we randomly sample the $I$ states from the fixed dataset $\cD$ and gain actions from learned policy ${\pi_t}$ as the inputs of dynamics models. Additionally, since the covariance matrix K in BO should be symmetric, we further give a reasonable definition which the model distance ${d(M_t, M_j)}$ is equal to ${d(M_j, M_t)}$ for the currently selected model ${M_t}$ and other dynamics model candidates ${M_j}$, which makes the covariance matrix be identical to the transpose of the original matrix.

The time complexity of BOMS with $n$ candidates’ is mainly composed of GP ($O(n^3)$) and the model distance measurement ($I$ model outputs * $n$ candidates).
    
\end{comment}


\begin{table*}[tb]
\centering
\caption{Demographics of Participant Clients: Previous Art Therapy Sessions indicates the number of times the client has previously participated in art therapy; Familiarity with Traditional Drawing reflects the client's level of experience with traditional drawing techniques (0-not familiar; 1-very familiar); Familiarity with Digital Drawing reflects the client's level of experience with digital drawing techniques (0-not familiar; 1-very familiar); Participation Purposes reflects the reasons clients choose to engage in the activity.}
\vspace{-3mm}
\label{tab:clients}
\small
\resizebox{1\linewidth}{!}{
\begin{tabular}{cccccccccc}
\toprule
\textbf{ID} & \textbf{Gender} & \textbf{Age} & \textbf{Education} & \textbf{Region} & \parbox[t]{2.5cm}{\centering\textbf{Previous Art Therapy Sessions}} & \parbox[t]{3cm}{\centering\textbf{Familiarity with Traditional Drawing}} & \parbox[t]{2cm}{\centering\textbf{Familiarity with Digital Drawing}} & \parbox[t]{2cm}{\centering\textbf{Therapist Assignment}} & \parbox[t]{2.5cm}{\centering\textbf{Participation Purposes}} \\
\midrule
C1  & Female & 37  & Bachelor's & China/Shanghai & 0                            & 1                                   & 0.25  &T3 & Personal Growth                   \\
C2  & Female & 35  & Bachelor's & China/Shenzhen & 3                            & 0.5                                   & 0.5   &T3 & Career Development and Family                 \\
C3  & Female & 28  & Master's   & China/Hebei    & 2                            & 0.75                                  & 0.75   &T3  & Family and Emotional Management                \\
C4  & Female & 36  & Bachelor's & China/Beijing  & 10                           & 0.75                                   & 0   &T3  &Career Development                \\
C5  & Male   & 28  & Master's   & Germany       & 0                            & 1                                   & 0.75   &T3   &  Emotional Management and Personal Growth                       \\
C6  & Other  & 26  & Associate's & China/Heilongjiang & 1                            & 0.5                                   & 0.25  &T5  & Emotional Exploration and Intimate Relationships                           \\
C7  & Female & 23  & Master's   & China/Shanghai & 0                            & 1                                   & 1     &T5     &  Intimate Relationships                    \\
C8  & Female & 20  & Bachelor's & China/Shenzhen & 0                            & 0.5                                   & 0.5    &T5   &  Emotional Management and Intimate Relationships                       \\
C9  & Female & 25  & Bachelor's & China/Guangxi  & 4                            & 0                                   & 0.5    &T5    &  Self-Expression and Emotional Exploration                      \\
C10 & Male   & 23  & Master's   & China/Shenzhen & 0                            & 0.75                                   & 0.5   &T5   &             Self-Expression and Social Skills             \\
C11 & Female & 26  & Master's   & China/Hangzhou & 0                            & 0.5                                   & 0.25    &T4  &        Emotional Management, Social Skills and Intimate Relationships                 \\
C12 & Female & 26  & Master's   & China/Shanghai & 2                            & 0.75                                   & 0.5    &T4   &                   Stress Relieving and Intimate Relationships  \\
C13 & Female & 30  & Master's   & China/Dalian   & 0                            & 0.5                                   & 0.25   &T4    &             Family and Emotional Management            \\
C14 & Female & 19  & Bachelor's & China/Chongqing & 0                            & 0.25                                   & 0.25   &T4  &                Personal Growth and Self-Exploration           \\
C15 & Male   & 27  & Bachelor's & China/Beijing  & 0                            & 0.25                                  & 0.25   &T4    &                 Stress Relieving and Personal Growth        \\
C16 & Female & 22  & Bachelor's & China/Shandong & 0                            & 0.5                                   & 0.25   &T1     &              Emotional Management and Social Skills       \\
C17 & Male   & 38  & Master's   & China/Sichuan  & 0                            & 0.75                                   & 0.75   &T1     &                    Personal Growth      \\
C18 & Female & 40  & Master's   & China/Beijing  & 20                           & 1                                   & 0.75    &T1      &               Stress Relieving and Emotional Management          \\
C19 & Female & 28  & Bachelor's & China/Guangzhou & 0                            & 0.5                                   & 0   &T1       &                 Future Career Planning and Personal Growth      \\
C20 & Male   & 25  & Master's   & China/Guangzhou & 0                            & 1                                   & 1   &T1        &                    Academic Pressure Relieving   \\
C21 & Male   & 24  & Master's   & China/Hubei    & 0                            & 0                                   & 0   &T2        &                Childhood Family and Dreams Exploration  \\
C22 & Female & 24  & Master's   & China/Shenzhen & 0                            & 0.25                                   & 0.25    &T2  &                Emotional Management and Personal Growth     \\
C23 & Male   & 25  & Master's   & China/Zhejiang & 10                           & 0.5                                   & 0.5    &T2   &                  Emotional Development and Self-Expression        \\
C24 & Male & 55  & Bachelor's & Dubai& 0 & 0.5& 0.5&T2 &                           Emotional Management \\
\bottomrule

\end{tabular}}
\Description{The table 2 describes 24 participants in art therapy sessions. The participants are from diverse locations, including China (Shanghai, Shenzhen, Hebei, Beijing, Heilongjiang, Guangxi, Hangzhou, Chongqing, Shandong, Sichuan, Hubei, and Zhejiang), Germany, and Dubai. The ages range from 19 to 55 years old, with varying levels of education from associate degrees to master's degrees and bachelor's degrees. Their familiarity with traditional drawing techniques ranges from no familiarity to very familiar, while their familiarity with digital drawing techniques also varies across the spectrum. The participants have attended between 0 and 20 previous art therapy sessions and are assigned to different therapists identified by codes T1 to T5.Participation Purposes reflects the reasons clients choose to engage in the activity}
\end{table*}

\section{Field study}
Using \name{} as both a novel system to study and a research tool to study with, we aim to explore how a human-AI system support clients' art therapy homework in their daily settings (\textbf{RQ1}) and how such a system could mediate therapist-client collaboration surrounding art therapy homework (\textbf{RQ2}). To this end, we conducted a field deployment involving 24 recruited clients and five therapists over the course of one month.



%参与者与实验的setup
    %参与者招募
        % 我们招募的途径:To recruit our clients, we distributed digital recruitment flyers through social media platforms.
        % 海报上描述了什么:The recruitment flyer described the art therapy activities as "promoting self-exploration using a digital software".
        % 我首先要求参与者填写pre-问卷,这个问卷主要包括descriptions of the art therapy activities, demographic information, the number of art therapy sessions they attended, familiarity with digital drawing, and specific needs for the art therapy activities.
        % Participants were included in this study with the aim of reducing stress and anxiety, fostering personal growth, improving emotional regulation, and strengthening social skills.
        % 此外,we tried to selection of participants based on their regions, occupations, the types of devices they used, and the number of times they participated in art therapy.
        % finally, 有27名参与者开始使用这个系统,其中有3名参与者drop out因为缺乏时间
\subsection{Participants and Study Procedure}
\subsubsection{Participants}

The five therapists who participated in the field evaluation were the same ones from our contextual study (see \autoref{tab:expert}). Each therapist was compensated at their regular hourly rate.
For client recruitment, we distributed digital flyers through social media platforms, describing the art therapy activities as an "online art therapy experience promoting self-exploration using a digital software." This aligns with the common goal of art therapy sessions, which are widely used to promote self-exploration for all clients, beyond treating mental illness~\cite{kahn1999art, riley2003family}.

Participants first completed a pre-questionnaire, which provided an overview of the activities and collected demographics, and prior experiences with art therapy experience and with digital drawing---to ensure that we include both novices and experienced user---and their personal goals for participation. 
The therapists guided the recruitment and screening of the the clients, and included individuals seeking for reducing stress, fostering personal growth, enhancing emotional regulation, and strengthening social skills. The therapists excluded individuals with serious mental health conditions to minimize ethical risks.
%Based on the therapists' advice, clients with goals such as reducing stress and anxiety, fostering personal growth, enhancing emotional regulation, and strengthening social skills were included, avoiding ethical concerns related to clinically diagnosed mental health conditions. 
%We also considered participants' regions, device types, drawing familiarity, and prior art therapy experience to create a balanced selection.

In total, 27 clients began using \name{}, but 3 withdrew early due to scheduling conflicts. The final group of 24 clients (C1-C24; 8 self-identified males, 15 self-identified females, 1 identifying as other; aged 19-55) completed the study (client demographics are detailed in the~\autoref{tab:clients}). Clients who completed the full process were compensated with \$37, others were compensated with a prorated fee.
Our study protocol was approved by the institutional research ethics board, and all participant names in this paper have been changed to pseudonyms. Participants reviewed and signed informed consent forms before taking part, acknowledging their understanding of the study.

% The five therapists participated in the field evaluation were the ones who also participated in our contextual study (see \autoref{tab:expert}).
% Five art therapists were compensated with their regular hourly rate.
% For the clients recruitment, we distributed digital recruitment flyers through social media platforms. 
% The recruitment flyer described the art therapy activities as ``online art therapy experience promoting self-exploration using a digital software''.
% This is due to that this is a common goal for art therapy sessions, since Art therapy activities are not only effective in treating mental illness but also widely promote self-exploration for every clients, as commonly integrated into practice~\cite{kahn1999art,riley2003family}.
% First, participants completed a pre-questionnaire that provided an overview of the art therapy activities and gathered details such as their demographics, the number of art therapy sessions they've attended, familiarity with digital drawing, and any specific needs they hoped to address.
% Following that, based on the advices from the therapists, clients were included with the goal of reducing stress and anxiety, fostering personal growth, enhancing emotional regulation, and strengthening social skills.
% The therapists suggest so since they agree that these therapeutic goals would be beneficial for eavery day therapy clients and would could It might avoid the potential ethical and safety risks associated with clinically diagnosed mental health issue.
% Further, we selected participants based on a balance of their regions, the types of devices they used, the familiarity with drawing and their prior experience with art therapy. 

% In total, 27 clients began using \name{}, but 3 withdrew from the study at the early stage due to scheduling conflicts.
% Finally, 24 clients (C1-C24; 8 self-identified males, 15 self-identified females, 1 identifying as other; aged 19-55) completed our field study. 
% APPENDIX shows the specific client demographics.
% We compensated clients based on their level of involvement, with those who completed the full one-month study receiving 200 RMB as a bonus, and clients who dropped out receiving a prorated fee according to the duration of their participation.

% Our protocol was approved by the institutional research ethics board, and all names in this paper have been changed to pseudonyms.
% Also, before participating in the activity, participants carefully reviewed and signed the informed consent form, acknowledging their understanding.

%在与治疗师协商讨论下,这些用户被分到5位治疗师(see Table),其中T2有4位来访者,其余治疗师有5位来访者。
%这个研究. .
%在活动开始前,我们邀请每位参与者开展了一场介绍session. 主要是目的是介绍活动目的与流程,并且演示如何使用\name{},并且为每位来访者可以接触到系统的URL的链接;
%介绍活动结束后,来访者被鼓励有规律地去自行探索使用\name{};
%每隔一周,我们会安排治疗师与来访者进行线上一对一的session。我们会鼓励治疗师在线上一对一session之前提前review来访者的使用数据,并通过即时通讯软件与我们交流review之后的洞见与想法。
%在线上一对一session时,在不干扰治疗师艺术治疗实践的基础上,我们鼓励治疗师在线上一对一session时利用这些数据。在艺术创作阶段,来访者可以通过分享屏幕的方式使用系统的第一个阶段进行创作并与治疗师进行讨论交流,在session快结束前治疗师会给来访者推荐家庭作业。
%在session结束后,治疗师会在治疗师系统上安排家庭作业并给予来访者的个人赠言。此外,来访者在结束线上session后可以按照治疗师的推荐完成家庭作业或者自行探索使用系统。
\subsubsection{Procedures}

Clients were distributed in coordination with the five therapists, as shown in \autoref{tab:expert}. T2 was assigned four clients, while the other therapists each had five clients. The field study consisted of two main activities: (1) three online in-session activities, where clients had one-on-one conversations and collaborated with the therapist, and (2) unstructured between-session activities, where clients practiced therapy homework using \name{} following the therapist’s recommendations.
Before the study, we held online introductory sessions to familiarize the clients with \name{}, and provided both demonstrations and hands-on exploration on their preferred devices. Similarly, we offered online training for therapists on customizing and reviewing homework, while allowing them to explore both the therapist-facing and client-facing applications. After the session, clients were encouraged to regularly explore \name{}.
Two weeks into the study, we scheduled weekly one-on-one online sessions between therapists and clients, each lasting approximately 60 minutes. Therapists were encouraged to review the clients' homework history using \autoref{fig:ui}(c) before each session. During the online session, therapists used this data to inform their practices without interrupting the flow of therapy. We encouraged clients in advance, to create artworks during the Art-making Phase~(\autoref{fig:qual_results}(a)), sharing screens and discussing their creations with the therapist, but did not interfere with the therapeutic process.

%Clients also used \autoref{fig:qual_results}(a) to create artwork, sharing their screens and discussing their creations with the therapist.

At the end of each session, therapists recommended homework tasks based on insights gained during the conversation. After the session, therapists might customize homework agents, including customizing conversational principles, assigning homework tasks, and providing personal messages through \autoref{fig:ui}~(d). Clients could then either complete the assigned homework or engage in self-exploration using \name{} between sessions.

% Clients were distributed In coordination with the five therapists, as shown by \autoref{tab:clients}: T2 was assigned with four clients, while each of the other therapists was assigned with five clients.
% The procedure for the field study consisted of two activities: (1) three online in-session activities where they have one-on-one conversation and collaboration with the therapist and (2) unstructured between-session activities where they perform therapy homework practices either upon recommendations of usage from the therapist or volunteerily use it in their daily lives.
% Before the study, we conducted an introductory session for each client to explain the activities, demonstrate how to use \name{}, and provide access to \name{} via a URL on their preferred devices.
% After the introductory session, the clients were encouraged to explore the use of \name{} on a regular basis.

% After two weeks of self-exploration, we started scheduling weekly one one-on-one online sessions between the therapists and the clients.
% Therapists were encouraged to review clients' homework history using \autoref{fig:ui}~(c) before the online session.
% During the online one-on-one session, we encouraged therapists to use these history data without interfering with their art therapy practices. 
% Also, they would utilize \autoref{fig:ui}~(a) to create their artwork by sharing their screens and discussing their artworks with therapists. 
% Before the end of the session, the therapist would recommend the homework tasks for the client based on the insights gained from the one-on-one session.
% After the online session ends, therapists would customize homework agents, including modifying or updating the conversational agent principles, assigning homework tasks and providing therapist's messages to the client through \autoref{fig:system}~(d). 
% Correspondingly, clients could either complete the homework or engage in self-exploration using \name{} between sessions.

% 对于异步session场景数据收集下,所有来访者使用系统的图像以及对话记录等日志数据以及治疗师在治疗师系统中使用定制功能的日志数据在保存在数据库中。
% 此外,我们鼓励来访者和治疗师通过即时通讯软件发送给我们images以及comments关于使用系统的实践以及感受。
% 对于线上session的场景数据收集,首先,online sessions were audio- and video-recorded.
% 此外,at the end of each online session, we conducted a 5-minute interview with therapists, mainly to collect their practices and experiences about the session.
% Upon concluding all the sessions,我们与治疗师以及来访者开展了约为30分钟的semi-structured interview to 探索ai agents如何支持艺术治疗场景的家庭作业(RQ1)以及AI agents如何mediate 治疗师与来访者合作(RQ2). We used 治疗师与来访者在 the trial period使用系统的log 数据以及他们的反馈作为stimuli 去问特定的使用实践的问题。
% With participants' consent, we recorded the interviews and transcribed them for thematic analysis.
% First, two researchers conducted collaborative inductive coding. They initially annotated the transcript to identify relevant quotes, key concepts, and recurring patterns in the data. These findings were further developed through regular discussions, leading to a detailed coding scheme aligned with the research questions. Quotes were then coded and clustered into a hierarchy of emerging themes, continually reviewed, and refined in recurrent meetings, where exemplar quotes were also selected for presenting each theme and sub-theme. 
% Also, we collected the log data from 治疗师和来访者 作为证据以及examples for the thematic analysis results.

\subsection{Data Gathering Methods} 

For between-sessions, we stored all homework-related data in a database, including artwork, dialogue, usage logs, as well as information on homework customization such as conversational principles, tasks, and personal messages.
We encouraged participants to use personal messaging (WeChat) to share pictures and comments about on-the-spot experience and feelings after homework with \name{} to compensate for semi-structured interviews.
During online sessions, we recorded audio and video. 
The researchers did not observe the therapy session in live, but reviewed post hoc, as the therapists believed a third party's presence could affect a client's emotional expression and the therapist-client dynamic.
After each session, we conducted a brief 5-minute interview with the therapists to gather their insights and feelings.

Upon the completion of the final one-on-one sessions, we conducted 30-minute semi-structured interviews with both therapists and clients. These interviews aimed to explore how \name{} supported art therapy homework in clients' daily lives (\textbf{RQ1}) and how therapists and clients collaborated surrounding art therapy homework (\textbf{RQ2}). We used feedback and homework outcomes from the trial period to ask targeted questions about their practices.
With participants' consent, we recorded and transcribed the brief 5-minute interviews and the 30-minute interviews for thematic analysis~\cite{braun2006using}. This analysis also included the personal messages shared by the participants about their on-the-spot experiences.
%we recorded and transcribed the interviews for thematic analysis. 
Two researchers then engaged in inductive coding, annotating transcripts to identify relevant quotes, key concepts, and patterns. They developed a detailed coding scheme through regular discussions, grouping quotes into a hierarchical structure of themes and sub-themes. Exemplar quotes were selected to represent each theme. We also used homework history (e.g., images or conversation data) and customization data (e.g., homework dialogue principle data) as evidences or examples to back up the findings in our thematic analysis.



% In between sessions, all homework history data~(e.g., artwork, creative process data and dialogue data) and history data on homework customization~(e.g., principles of conversational agents, homework tasks and personal messages) were stored in the database.
% In addition, we encouraged clients and therapists to send us images and comments about their experiences and feelings when using \name{} via an instant messaging app.
% For online in-sessions, the sessions were first audio- and video-recorded.
% At the end of each in-session, we conducted a brief 5-minute interview with the therapists to gather insights into their practices and feelings during the session.
% Upon concluding all the sessions, we conducted approximately 30-minute semi-structured interviews with both the therapists and the clients to explore how \name{} support art therapy homework in clients' daily settings~(\textbf{RQ1}), and how therapists tailored the homework and tracked the homework history surrounding art therapy homework~(\textbf{RQ2}). 
% Further, we employed the homework outcomes and feedback from both therapists and clients during the trial period as stimuli to ask specific questions about their practices. 

% With participants' consent, we recorded the interviews and transcribed them for thematic analysis~\cite{braun2006using}.
% Initially, two researchers engaged in collaborative inductive coding. They began by annotating the transcript to highlight relevant quotes, key concepts, and recurring patterns in the data. Through regular discussions, they expanded these insights into a detailed coding scheme that aligned with their research questions. The quotes were then systematically coded and grouped into a hierarchical structure of emerging themes, which were continuously reviewed and refined during recurring meetings. During these discussions, exemplar quotes were also chosen to represent each theme and sub-theme.
% We also gathered homework history and customization data, including artworks and conversation records from both therapists and clients, as evidence and examples to support the results of the thematic analysis.

\begin{figure*}[tb]
  \centering
  \includegraphics[width=\linewidth]{images/findings_1.png}
  \vspace{-7mm}
  \caption{Overview of The Homework Engagement of Clients with \name{}: (a) Homework Activity Date Distribution; (b) Accumulated Homework Activity Hourly Distribution of the Day; (c) Usage of AI Brushes in Artworks; 
  }
  \Description{Figure 5 contains three sub-figures. Figure 5a shows the Homework Activity Date Distribution for 24 clients over a four-week period, using seven different shades of purple to represent varying levels of participation in the homework sessions. Figure 5b illustrates the frequency of AI brush usage during clients' homework art-making, with the top 20 most frequently used brushes highlighted in larger font. Figure 5c depicts the distribution of homework sessions across different times of the day, revealing that clients tend to engage in homework sessions more frequently in the afternoon and evening.}
  \label{fig:quan_results}
\end{figure*}




\section{Related Work}

%applying penalties to the value function when evaluated on state-action pairs outside of the fixed dataset distribution. It takes into account both the pre-collected dataset and the rollout dataset yielded from the dynamics model to determine the penalty.
%Additionally, there are two classic conservative model-based methods, MOPO \citep{yu2020mopo} and MOReL \citep{kidambi2020morel}. These two methods construct pessimistic MDPs and give the penalty to OOD state-action pairs based on different uncertainty estimations. 
%MOPO chooses the standard deviation of the transition distribution as the soft reward penalty, which allows the learned policy to have more flexibility in choosing actions. On the other hand, MOReL introduces the Halt state in MDP and gives a fixed penalty when the total variation distance between the learned transition and the true transition exceeds a certain threshold. By taking the conservative approaches, these model-based works prevent the learned policy from venturing into the states that are different from the training dataset.

%\subsection{Online Interaction Integration for Offline RL}
%\subsection{Improving Offline RL by Online Interactions}
%To address this, several recent attempts \citep{konyushova2021active,zhao2022adaptive, kostrikov2021implicitq, niu2022trust, nakamoto2023cal} have proposed to incorporate policies learned from offline RL with a small budget of online interactions:

One notable challenge of pure offline RL is that the absence of online interaction with the environment can severely limit the performance due to limited coverage of the fixed dataset. 
%Moreover, when deploying the learned policy in the actual environment, valuable real-world information may be overlooked and miss opportunities for algorithm improvement.
%, enabling more effective fine-tuning processes.
Indeed, the need for a small budget of online interactions has been studied and justified in the offline RL literature:
%To address this, two main categories of methods have been proposed to incorporate policies learned from offline RL with a small budget of online interactions:

\textbf{Offline RL with policy selection}: Among these works, \citep{konyushova2021active} is the most relevant to ours in terms of problem formulation. Specifically, given a collection of candidate policies learned by any offline RL algorithm, \citep{konyushova2021active} proposes a setting called active offline policy selection (AOPS), which applies BO to actively choose a well-performing policy in the candidate policy set. Despite this high-level resemblance, AOPS does \textit{not} address the selection of dynamics models and is not readily applicable in active model selection. More specifically, the direct application of AOPS to our problem (\ie active model selection for offline MBRL) would \textit{require training policies for all the candidate dynamics models} at the first place, and this unavoidably leads to enormous computational overhead. Therefore, we view \citep{konyushova2021active} as an orthogonal direction to ours.

\textbf{Offline-to-online RL}: As a promising research direction, Offline-to-online RL (O2O) is meant to improve the policies learned by offline RL through fine-tuning on a small amount of online interaction.
To begin with, \citep{lee2022offline} introduces an innovative O2O approach that involves fine-tuning ensemble agents by combining offline and online transitions with a balanced replay scheme.
In a similar vein, \citep{agarwal2022reincarnating} presents Reincarnating RL (RRL), which aims to mitigate the data inefficiency of deep RL. Traditional RL algorithms typically start from scratch without leveraging any prior knowledge, resulting in computational and sample inefficiencies in practice. RRL reuses existing logged data or learned policies as an initialization for further real-world training to avoid redundant computations and improve the scalability.
More recently, the O2O problem has been studied from various perspectives, such as data mixing \citep{zheng2023adaptive,ball2023efficient}, policy fine-tuning \citep{xie2021policy,uchendu2023jump,zhang2023policy}, value-based fine-tuning \citep{zhang2024perspective}, and reward-based fine-tuning \citep{nair2023learning}. In contrast to the above works, BOMS utilizes online interactions to enhance the model selection for offline MBRL.

\pch{Moreover, due to the page limit, a review of the offline MBRL methods is provided in Appendix~\ref{app:related}.}
%. They establish a balanced replay scheme by storing the online and offline samples in the prioritized buffer and assigning these samples different priorities based on density online-offline density ratios. Then, they train a pessimistic ensemble Q-function to mitigate the state-action distribution shift. 
%In a similar vein, \citep{konyushova2021active} proposes a method that utilizes OPE estimation and limited online interactions to evaluate policies for policy selection using Bayesian Optimization. The presence of the distribution shift can lead to inaccurate off-policy evaluation (OPE), resulting in the selection of suboptimal policies in offline RL. By incorporating additional information from the environment, this method helps identify the optimal policy in offline RL. 
%In a similar vein, Another offline-to-online work is Reincarnating RL (RRL) \citep{agarwal2022reincarnating}, which aims to mitigate the inefficiency of deep RL. Traditional RL algorithms typically start from scratch without leveraging any prior knowledge, resulting in computational and sample inefficiencies in practice. RRL reuses existing logged data or learned policies as a starting point for further real-world training to avoid redundant computations and improve the scalability of complex real-world problems.

%By pre-training the agent or model offline, reasonable policies can be tested prior to deployment and ensure more reliable and robust performance. The offline-to-online approach not only strengthens the existing policies but also mitigates the risks associated with online interaction.
\section{Conclusion }
This paper introduces the Latent Radiance Field (LRF), which to our knowledge, is the first work to construct radiance field representations directly in the 2D latent space for 3D reconstruction. We present a novel framework for incorporating 3D awareness into 2D representation learning, featuring a correspondence-aware autoencoding method and a VAE-Radiance Field (VAE-RF) alignment strategy to bridge the domain gap between the 2D latent space and the natural 3D space, thereby significantly enhancing the visual quality of our LRF.
Future work will focus on incorporating our method with more compact 3D representations, efficient NVS, few-shot NVS in latent space, as well as exploring its application with potential 3D latent diffusion models.

%\clearpage

\bibliography{uai2025-BOMS}

\newpage

\onecolumn
\appendix
%\title{Active Dynamics Model Selection for Offline Model-Based RL \\ via Bayesian Optimization (Supplementary Material)}
\title{Enhancing Offline Model-Based RL via Active Model Selection:\\A Bayesian Optimization Perspective (Supplementary Material)}
\maketitle

\subsection{Lloyd-Max Algorithm}
\label{subsec:Lloyd-Max}
For a given quantization bitwidth $B$ and an operand $\bm{X}$, the Lloyd-Max algorithm finds $2^B$ quantization levels $\{\hat{x}_i\}_{i=1}^{2^B}$ such that quantizing $\bm{X}$ by rounding each scalar in $\bm{X}$ to the nearest quantization level minimizes the quantization MSE. 

The algorithm starts with an initial guess of quantization levels and then iteratively computes quantization thresholds $\{\tau_i\}_{i=1}^{2^B-1}$ and updates quantization levels $\{\hat{x}_i\}_{i=1}^{2^B}$. Specifically, at iteration $n$, thresholds are set to the midpoints of the previous iteration's levels:
\begin{align*}
    \tau_i^{(n)}=\frac{\hat{x}_i^{(n-1)}+\hat{x}_{i+1}^{(n-1)}}2 \text{ for } i=1\ldots 2^B-1
\end{align*}
Subsequently, the quantization levels are re-computed as conditional means of the data regions defined by the new thresholds:
\begin{align*}
    \hat{x}_i^{(n)}=\mathbb{E}\left[ \bm{X} \big| \bm{X}\in [\tau_{i-1}^{(n)},\tau_i^{(n)}] \right] \text{ for } i=1\ldots 2^B
\end{align*}
where to satisfy boundary conditions we have $\tau_0=-\infty$ and $\tau_{2^B}=\infty$. The algorithm iterates the above steps until convergence.

Figure \ref{fig:lm_quant} compares the quantization levels of a $7$-bit floating point (E3M3) quantizer (left) to a $7$-bit Lloyd-Max quantizer (right) when quantizing a layer of weights from the GPT3-126M model at a per-tensor granularity. As shown, the Lloyd-Max quantizer achieves substantially lower quantization MSE. Further, Table \ref{tab:FP7_vs_LM7} shows the superior perplexity achieved by Lloyd-Max quantizers for bitwidths of $7$, $6$ and $5$. The difference between the quantizers is clear at 5 bits, where per-tensor FP quantization incurs a drastic and unacceptable increase in perplexity, while Lloyd-Max quantization incurs a much smaller increase. Nevertheless, we note that even the optimal Lloyd-Max quantizer incurs a notable ($\sim 1.5$) increase in perplexity due to the coarse granularity of quantization. 

\begin{figure}[h]
  \centering
  \includegraphics[width=0.7\linewidth]{sections/figures/LM7_FP7.pdf}
  \caption{\small Quantization levels and the corresponding quantization MSE of Floating Point (left) vs Lloyd-Max (right) Quantizers for a layer of weights in the GPT3-126M model.}
  \label{fig:lm_quant}
\end{figure}

\begin{table}[h]\scriptsize
\begin{center}
\caption{\label{tab:FP7_vs_LM7} \small Comparing perplexity (lower is better) achieved by floating point quantizers and Lloyd-Max quantizers on a GPT3-126M model for the Wikitext-103 dataset.}
\begin{tabular}{c|cc|c}
\hline
 \multirow{2}{*}{\textbf{Bitwidth}} & \multicolumn{2}{|c|}{\textbf{Floating-Point Quantizer}} & \textbf{Lloyd-Max Quantizer} \\
 & Best Format & Wikitext-103 Perplexity & Wikitext-103 Perplexity \\
\hline
7 & E3M3 & 18.32 & 18.27 \\
6 & E3M2 & 19.07 & 18.51 \\
5 & E4M0 & 43.89 & 19.71 \\
\hline
\end{tabular}
\end{center}
\end{table}

\subsection{Proof of Local Optimality of LO-BCQ}
\label{subsec:lobcq_opt_proof}
For a given block $\bm{b}_j$, the quantization MSE during LO-BCQ can be empirically evaluated as $\frac{1}{L_b}\lVert \bm{b}_j- \bm{\hat{b}}_j\rVert^2_2$ where $\bm{\hat{b}}_j$ is computed from equation (\ref{eq:clustered_quantization_definition}) as $C_{f(\bm{b}_j)}(\bm{b}_j)$. Further, for a given block cluster $\mathcal{B}_i$, we compute the quantization MSE as $\frac{1}{|\mathcal{B}_{i}|}\sum_{\bm{b} \in \mathcal{B}_{i}} \frac{1}{L_b}\lVert \bm{b}- C_i^{(n)}(\bm{b})\rVert^2_2$. Therefore, at the end of iteration $n$, we evaluate the overall quantization MSE $J^{(n)}$ for a given operand $\bm{X}$ composed of $N_c$ block clusters as:
\begin{align*}
    \label{eq:mse_iter_n}
    J^{(n)} = \frac{1}{N_c} \sum_{i=1}^{N_c} \frac{1}{|\mathcal{B}_{i}^{(n)}|}\sum_{\bm{v} \in \mathcal{B}_{i}^{(n)}} \frac{1}{L_b}\lVert \bm{b}- B_i^{(n)}(\bm{b})\rVert^2_2
\end{align*}

At the end of iteration $n$, the codebooks are updated from $\mathcal{C}^{(n-1)}$ to $\mathcal{C}^{(n)}$. However, the mapping of a given vector $\bm{b}_j$ to quantizers $\mathcal{C}^{(n)}$ remains as  $f^{(n)}(\bm{b}_j)$. At the next iteration, during the vector clustering step, $f^{(n+1)}(\bm{b}_j)$ finds new mapping of $\bm{b}_j$ to updated codebooks $\mathcal{C}^{(n)}$ such that the quantization MSE over the candidate codebooks is minimized. Therefore, we obtain the following result for $\bm{b}_j$:
\begin{align*}
\frac{1}{L_b}\lVert \bm{b}_j - C_{f^{(n+1)}(\bm{b}_j)}^{(n)}(\bm{b}_j)\rVert^2_2 \le \frac{1}{L_b}\lVert \bm{b}_j - C_{f^{(n)}(\bm{b}_j)}^{(n)}(\bm{b}_j)\rVert^2_2
\end{align*}

That is, quantizing $\bm{b}_j$ at the end of the block clustering step of iteration $n+1$ results in lower quantization MSE compared to quantizing at the end of iteration $n$. Since this is true for all $\bm{b} \in \bm{X}$, we assert the following:
\begin{equation}
\begin{split}
\label{eq:mse_ineq_1}
    \tilde{J}^{(n+1)} &= \frac{1}{N_c} \sum_{i=1}^{N_c} \frac{1}{|\mathcal{B}_{i}^{(n+1)}|}\sum_{\bm{b} \in \mathcal{B}_{i}^{(n+1)}} \frac{1}{L_b}\lVert \bm{b} - C_i^{(n)}(b)\rVert^2_2 \le J^{(n)}
\end{split}
\end{equation}
where $\tilde{J}^{(n+1)}$ is the the quantization MSE after the vector clustering step at iteration $n+1$.

Next, during the codebook update step (\ref{eq:quantizers_update}) at iteration $n+1$, the per-cluster codebooks $\mathcal{C}^{(n)}$ are updated to $\mathcal{C}^{(n+1)}$ by invoking the Lloyd-Max algorithm \citep{Lloyd}. We know that for any given value distribution, the Lloyd-Max algorithm minimizes the quantization MSE. Therefore, for a given vector cluster $\mathcal{B}_i$ we obtain the following result:

\begin{equation}
    \frac{1}{|\mathcal{B}_{i}^{(n+1)}|}\sum_{\bm{b} \in \mathcal{B}_{i}^{(n+1)}} \frac{1}{L_b}\lVert \bm{b}- C_i^{(n+1)}(\bm{b})\rVert^2_2 \le \frac{1}{|\mathcal{B}_{i}^{(n+1)}|}\sum_{\bm{b} \in \mathcal{B}_{i}^{(n+1)}} \frac{1}{L_b}\lVert \bm{b}- C_i^{(n)}(\bm{b})\rVert^2_2
\end{equation}

The above equation states that quantizing the given block cluster $\mathcal{B}_i$ after updating the associated codebook from $C_i^{(n)}$ to $C_i^{(n+1)}$ results in lower quantization MSE. Since this is true for all the block clusters, we derive the following result: 
\begin{equation}
\begin{split}
\label{eq:mse_ineq_2}
     J^{(n+1)} &= \frac{1}{N_c} \sum_{i=1}^{N_c} \frac{1}{|\mathcal{B}_{i}^{(n+1)}|}\sum_{\bm{b} \in \mathcal{B}_{i}^{(n+1)}} \frac{1}{L_b}\lVert \bm{b}- C_i^{(n+1)}(\bm{b})\rVert^2_2  \le \tilde{J}^{(n+1)}   
\end{split}
\end{equation}

Following (\ref{eq:mse_ineq_1}) and (\ref{eq:mse_ineq_2}), we find that the quantization MSE is non-increasing for each iteration, that is, $J^{(1)} \ge J^{(2)} \ge J^{(3)} \ge \ldots \ge J^{(M)}$ where $M$ is the maximum number of iterations. 
%Therefore, we can say that if the algorithm converges, then it must be that it has converged to a local minimum. 
\hfill $\blacksquare$


\begin{figure}
    \begin{center}
    \includegraphics[width=0.5\textwidth]{sections//figures/mse_vs_iter.pdf}
    \end{center}
    \caption{\small NMSE vs iterations during LO-BCQ compared to other block quantization proposals}
    \label{fig:nmse_vs_iter}
\end{figure}

Figure \ref{fig:nmse_vs_iter} shows the empirical convergence of LO-BCQ across several block lengths and number of codebooks. Also, the MSE achieved by LO-BCQ is compared to baselines such as MXFP and VSQ. As shown, LO-BCQ converges to a lower MSE than the baselines. Further, we achieve better convergence for larger number of codebooks ($N_c$) and for a smaller block length ($L_b$), both of which increase the bitwidth of BCQ (see Eq \ref{eq:bitwidth_bcq}).


\subsection{Additional Accuracy Results}
%Table \ref{tab:lobcq_config} lists the various LOBCQ configurations and their corresponding bitwidths.
\begin{table}
\setlength{\tabcolsep}{4.75pt}
\begin{center}
\caption{\label{tab:lobcq_config} Various LO-BCQ configurations and their bitwidths.}
\begin{tabular}{|c||c|c|c|c||c|c||c|} 
\hline
 & \multicolumn{4}{|c||}{$L_b=8$} & \multicolumn{2}{|c||}{$L_b=4$} & $L_b=2$ \\
 \hline
 \backslashbox{$L_A$\kern-1em}{\kern-1em$N_c$} & 2 & 4 & 8 & 16 & 2 & 4 & 2 \\
 \hline
 64 & 4.25 & 4.375 & 4.5 & 4.625 & 4.375 & 4.625 & 4.625\\
 \hline
 32 & 4.375 & 4.5 & 4.625& 4.75 & 4.5 & 4.75 & 4.75 \\
 \hline
 16 & 4.625 & 4.75& 4.875 & 5 & 4.75 & 5 & 5 \\
 \hline
\end{tabular}
\end{center}
\end{table}

%\subsection{Perplexity achieved by various LO-BCQ configurations on Wikitext-103 dataset}

\begin{table} \centering
\begin{tabular}{|c||c|c|c|c||c|c||c|} 
\hline
 $L_b \rightarrow$& \multicolumn{4}{c||}{8} & \multicolumn{2}{c||}{4} & 2\\
 \hline
 \backslashbox{$L_A$\kern-1em}{\kern-1em$N_c$} & 2 & 4 & 8 & 16 & 2 & 4 & 2  \\
 %$N_c \rightarrow$ & 2 & 4 & 8 & 16 & 2 & 4 & 2 \\
 \hline
 \hline
 \multicolumn{8}{c}{GPT3-1.3B (FP32 PPL = 9.98)} \\ 
 \hline
 \hline
 64 & 10.40 & 10.23 & 10.17 & 10.15 &  10.28 & 10.18 & 10.19 \\
 \hline
 32 & 10.25 & 10.20 & 10.15 & 10.12 &  10.23 & 10.17 & 10.17 \\
 \hline
 16 & 10.22 & 10.16 & 10.10 & 10.09 &  10.21 & 10.14 & 10.16 \\
 \hline
  \hline
 \multicolumn{8}{c}{GPT3-8B (FP32 PPL = 7.38)} \\ 
 \hline
 \hline
 64 & 7.61 & 7.52 & 7.48 &  7.47 &  7.55 &  7.49 & 7.50 \\
 \hline
 32 & 7.52 & 7.50 & 7.46 &  7.45 &  7.52 &  7.48 & 7.48  \\
 \hline
 16 & 7.51 & 7.48 & 7.44 &  7.44 &  7.51 &  7.49 & 7.47  \\
 \hline
\end{tabular}
\caption{\label{tab:ppl_gpt3_abalation} Wikitext-103 perplexity across GPT3-1.3B and 8B models.}
\end{table}

\begin{table} \centering
\begin{tabular}{|c||c|c|c|c||} 
\hline
 $L_b \rightarrow$& \multicolumn{4}{c||}{8}\\
 \hline
 \backslashbox{$L_A$\kern-1em}{\kern-1em$N_c$} & 2 & 4 & 8 & 16 \\
 %$N_c \rightarrow$ & 2 & 4 & 8 & 16 & 2 & 4 & 2 \\
 \hline
 \hline
 \multicolumn{5}{|c|}{Llama2-7B (FP32 PPL = 5.06)} \\ 
 \hline
 \hline
 64 & 5.31 & 5.26 & 5.19 & 5.18  \\
 \hline
 32 & 5.23 & 5.25 & 5.18 & 5.15  \\
 \hline
 16 & 5.23 & 5.19 & 5.16 & 5.14  \\
 \hline
 \multicolumn{5}{|c|}{Nemotron4-15B (FP32 PPL = 5.87)} \\ 
 \hline
 \hline
 64  & 6.3 & 6.20 & 6.13 & 6.08  \\
 \hline
 32  & 6.24 & 6.12 & 6.07 & 6.03  \\
 \hline
 16  & 6.12 & 6.14 & 6.04 & 6.02  \\
 \hline
 \multicolumn{5}{|c|}{Nemotron4-340B (FP32 PPL = 3.48)} \\ 
 \hline
 \hline
 64 & 3.67 & 3.62 & 3.60 & 3.59 \\
 \hline
 32 & 3.63 & 3.61 & 3.59 & 3.56 \\
 \hline
 16 & 3.61 & 3.58 & 3.57 & 3.55 \\
 \hline
\end{tabular}
\caption{\label{tab:ppl_llama7B_nemo15B} Wikitext-103 perplexity compared to FP32 baseline in Llama2-7B and Nemotron4-15B, 340B models}
\end{table}

%\subsection{Perplexity achieved by various LO-BCQ configurations on MMLU dataset}


\begin{table} \centering
\begin{tabular}{|c||c|c|c|c||c|c|c|c|} 
\hline
 $L_b \rightarrow$& \multicolumn{4}{c||}{8} & \multicolumn{4}{c||}{8}\\
 \hline
 \backslashbox{$L_A$\kern-1em}{\kern-1em$N_c$} & 2 & 4 & 8 & 16 & 2 & 4 & 8 & 16  \\
 %$N_c \rightarrow$ & 2 & 4 & 8 & 16 & 2 & 4 & 2 \\
 \hline
 \hline
 \multicolumn{5}{|c|}{Llama2-7B (FP32 Accuracy = 45.8\%)} & \multicolumn{4}{|c|}{Llama2-70B (FP32 Accuracy = 69.12\%)} \\ 
 \hline
 \hline
 64 & 43.9 & 43.4 & 43.9 & 44.9 & 68.07 & 68.27 & 68.17 & 68.75 \\
 \hline
 32 & 44.5 & 43.8 & 44.9 & 44.5 & 68.37 & 68.51 & 68.35 & 68.27  \\
 \hline
 16 & 43.9 & 42.7 & 44.9 & 45 & 68.12 & 68.77 & 68.31 & 68.59  \\
 \hline
 \hline
 \multicolumn{5}{|c|}{GPT3-22B (FP32 Accuracy = 38.75\%)} & \multicolumn{4}{|c|}{Nemotron4-15B (FP32 Accuracy = 64.3\%)} \\ 
 \hline
 \hline
 64 & 36.71 & 38.85 & 38.13 & 38.92 & 63.17 & 62.36 & 63.72 & 64.09 \\
 \hline
 32 & 37.95 & 38.69 & 39.45 & 38.34 & 64.05 & 62.30 & 63.8 & 64.33  \\
 \hline
 16 & 38.88 & 38.80 & 38.31 & 38.92 & 63.22 & 63.51 & 63.93 & 64.43  \\
 \hline
\end{tabular}
\caption{\label{tab:mmlu_abalation} Accuracy on MMLU dataset across GPT3-22B, Llama2-7B, 70B and Nemotron4-15B models.}
\end{table}


%\subsection{Perplexity achieved by various LO-BCQ configurations on LM evaluation harness}

\begin{table} \centering
\begin{tabular}{|c||c|c|c|c||c|c|c|c|} 
\hline
 $L_b \rightarrow$& \multicolumn{4}{c||}{8} & \multicolumn{4}{c||}{8}\\
 \hline
 \backslashbox{$L_A$\kern-1em}{\kern-1em$N_c$} & 2 & 4 & 8 & 16 & 2 & 4 & 8 & 16  \\
 %$N_c \rightarrow$ & 2 & 4 & 8 & 16 & 2 & 4 & 2 \\
 \hline
 \hline
 \multicolumn{5}{|c|}{Race (FP32 Accuracy = 37.51\%)} & \multicolumn{4}{|c|}{Boolq (FP32 Accuracy = 64.62\%)} \\ 
 \hline
 \hline
 64 & 36.94 & 37.13 & 36.27 & 37.13 & 63.73 & 62.26 & 63.49 & 63.36 \\
 \hline
 32 & 37.03 & 36.36 & 36.08 & 37.03 & 62.54 & 63.51 & 63.49 & 63.55  \\
 \hline
 16 & 37.03 & 37.03 & 36.46 & 37.03 & 61.1 & 63.79 & 63.58 & 63.33  \\
 \hline
 \hline
 \multicolumn{5}{|c|}{Winogrande (FP32 Accuracy = 58.01\%)} & \multicolumn{4}{|c|}{Piqa (FP32 Accuracy = 74.21\%)} \\ 
 \hline
 \hline
 64 & 58.17 & 57.22 & 57.85 & 58.33 & 73.01 & 73.07 & 73.07 & 72.80 \\
 \hline
 32 & 59.12 & 58.09 & 57.85 & 58.41 & 73.01 & 73.94 & 72.74 & 73.18  \\
 \hline
 16 & 57.93 & 58.88 & 57.93 & 58.56 & 73.94 & 72.80 & 73.01 & 73.94  \\
 \hline
\end{tabular}
\caption{\label{tab:mmlu_abalation} Accuracy on LM evaluation harness tasks on GPT3-1.3B model.}
\end{table}

\begin{table} \centering
\begin{tabular}{|c||c|c|c|c||c|c|c|c|} 
\hline
 $L_b \rightarrow$& \multicolumn{4}{c||}{8} & \multicolumn{4}{c||}{8}\\
 \hline
 \backslashbox{$L_A$\kern-1em}{\kern-1em$N_c$} & 2 & 4 & 8 & 16 & 2 & 4 & 8 & 16  \\
 %$N_c \rightarrow$ & 2 & 4 & 8 & 16 & 2 & 4 & 2 \\
 \hline
 \hline
 \multicolumn{5}{|c|}{Race (FP32 Accuracy = 41.34\%)} & \multicolumn{4}{|c|}{Boolq (FP32 Accuracy = 68.32\%)} \\ 
 \hline
 \hline
 64 & 40.48 & 40.10 & 39.43 & 39.90 & 69.20 & 68.41 & 69.45 & 68.56 \\
 \hline
 32 & 39.52 & 39.52 & 40.77 & 39.62 & 68.32 & 67.43 & 68.17 & 69.30  \\
 \hline
 16 & 39.81 & 39.71 & 39.90 & 40.38 & 68.10 & 66.33 & 69.51 & 69.42  \\
 \hline
 \hline
 \multicolumn{5}{|c|}{Winogrande (FP32 Accuracy = 67.88\%)} & \multicolumn{4}{|c|}{Piqa (FP32 Accuracy = 78.78\%)} \\ 
 \hline
 \hline
 64 & 66.85 & 66.61 & 67.72 & 67.88 & 77.31 & 77.42 & 77.75 & 77.64 \\
 \hline
 32 & 67.25 & 67.72 & 67.72 & 67.00 & 77.31 & 77.04 & 77.80 & 77.37  \\
 \hline
 16 & 68.11 & 68.90 & 67.88 & 67.48 & 77.37 & 78.13 & 78.13 & 77.69  \\
 \hline
\end{tabular}
\caption{\label{tab:mmlu_abalation} Accuracy on LM evaluation harness tasks on GPT3-8B model.}
\end{table}

\begin{table} \centering
\begin{tabular}{|c||c|c|c|c||c|c|c|c|} 
\hline
 $L_b \rightarrow$& \multicolumn{4}{c||}{8} & \multicolumn{4}{c||}{8}\\
 \hline
 \backslashbox{$L_A$\kern-1em}{\kern-1em$N_c$} & 2 & 4 & 8 & 16 & 2 & 4 & 8 & 16  \\
 %$N_c \rightarrow$ & 2 & 4 & 8 & 16 & 2 & 4 & 2 \\
 \hline
 \hline
 \multicolumn{5}{|c|}{Race (FP32 Accuracy = 40.67\%)} & \multicolumn{4}{|c|}{Boolq (FP32 Accuracy = 76.54\%)} \\ 
 \hline
 \hline
 64 & 40.48 & 40.10 & 39.43 & 39.90 & 75.41 & 75.11 & 77.09 & 75.66 \\
 \hline
 32 & 39.52 & 39.52 & 40.77 & 39.62 & 76.02 & 76.02 & 75.96 & 75.35  \\
 \hline
 16 & 39.81 & 39.71 & 39.90 & 40.38 & 75.05 & 73.82 & 75.72 & 76.09  \\
 \hline
 \hline
 \multicolumn{5}{|c|}{Winogrande (FP32 Accuracy = 70.64\%)} & \multicolumn{4}{|c|}{Piqa (FP32 Accuracy = 79.16\%)} \\ 
 \hline
 \hline
 64 & 69.14 & 70.17 & 70.17 & 70.56 & 78.24 & 79.00 & 78.62 & 78.73 \\
 \hline
 32 & 70.96 & 69.69 & 71.27 & 69.30 & 78.56 & 79.49 & 79.16 & 78.89  \\
 \hline
 16 & 71.03 & 69.53 & 69.69 & 70.40 & 78.13 & 79.16 & 79.00 & 79.00  \\
 \hline
\end{tabular}
\caption{\label{tab:mmlu_abalation} Accuracy on LM evaluation harness tasks on GPT3-22B model.}
\end{table}

\begin{table} \centering
\begin{tabular}{|c||c|c|c|c||c|c|c|c|} 
\hline
 $L_b \rightarrow$& \multicolumn{4}{c||}{8} & \multicolumn{4}{c||}{8}\\
 \hline
 \backslashbox{$L_A$\kern-1em}{\kern-1em$N_c$} & 2 & 4 & 8 & 16 & 2 & 4 & 8 & 16  \\
 %$N_c \rightarrow$ & 2 & 4 & 8 & 16 & 2 & 4 & 2 \\
 \hline
 \hline
 \multicolumn{5}{|c|}{Race (FP32 Accuracy = 44.4\%)} & \multicolumn{4}{|c|}{Boolq (FP32 Accuracy = 79.29\%)} \\ 
 \hline
 \hline
 64 & 42.49 & 42.51 & 42.58 & 43.45 & 77.58 & 77.37 & 77.43 & 78.1 \\
 \hline
 32 & 43.35 & 42.49 & 43.64 & 43.73 & 77.86 & 75.32 & 77.28 & 77.86  \\
 \hline
 16 & 44.21 & 44.21 & 43.64 & 42.97 & 78.65 & 77 & 76.94 & 77.98  \\
 \hline
 \hline
 \multicolumn{5}{|c|}{Winogrande (FP32 Accuracy = 69.38\%)} & \multicolumn{4}{|c|}{Piqa (FP32 Accuracy = 78.07\%)} \\ 
 \hline
 \hline
 64 & 68.9 & 68.43 & 69.77 & 68.19 & 77.09 & 76.82 & 77.09 & 77.86 \\
 \hline
 32 & 69.38 & 68.51 & 68.82 & 68.90 & 78.07 & 76.71 & 78.07 & 77.86  \\
 \hline
 16 & 69.53 & 67.09 & 69.38 & 68.90 & 77.37 & 77.8 & 77.91 & 77.69  \\
 \hline
\end{tabular}
\caption{\label{tab:mmlu_abalation} Accuracy on LM evaluation harness tasks on Llama2-7B model.}
\end{table}

\begin{table} \centering
\begin{tabular}{|c||c|c|c|c||c|c|c|c|} 
\hline
 $L_b \rightarrow$& \multicolumn{4}{c||}{8} & \multicolumn{4}{c||}{8}\\
 \hline
 \backslashbox{$L_A$\kern-1em}{\kern-1em$N_c$} & 2 & 4 & 8 & 16 & 2 & 4 & 8 & 16  \\
 %$N_c \rightarrow$ & 2 & 4 & 8 & 16 & 2 & 4 & 2 \\
 \hline
 \hline
 \multicolumn{5}{|c|}{Race (FP32 Accuracy = 48.8\%)} & \multicolumn{4}{|c|}{Boolq (FP32 Accuracy = 85.23\%)} \\ 
 \hline
 \hline
 64 & 49.00 & 49.00 & 49.28 & 48.71 & 82.82 & 84.28 & 84.03 & 84.25 \\
 \hline
 32 & 49.57 & 48.52 & 48.33 & 49.28 & 83.85 & 84.46 & 84.31 & 84.93  \\
 \hline
 16 & 49.85 & 49.09 & 49.28 & 48.99 & 85.11 & 84.46 & 84.61 & 83.94  \\
 \hline
 \hline
 \multicolumn{5}{|c|}{Winogrande (FP32 Accuracy = 79.95\%)} & \multicolumn{4}{|c|}{Piqa (FP32 Accuracy = 81.56\%)} \\ 
 \hline
 \hline
 64 & 78.77 & 78.45 & 78.37 & 79.16 & 81.45 & 80.69 & 81.45 & 81.5 \\
 \hline
 32 & 78.45 & 79.01 & 78.69 & 80.66 & 81.56 & 80.58 & 81.18 & 81.34  \\
 \hline
 16 & 79.95 & 79.56 & 79.79 & 79.72 & 81.28 & 81.66 & 81.28 & 80.96  \\
 \hline
\end{tabular}
\caption{\label{tab:mmlu_abalation} Accuracy on LM evaluation harness tasks on Llama2-70B model.}
\end{table}

%\section{MSE Studies}
%\textcolor{red}{TODO}


\subsection{Number Formats and Quantization Method}
\label{subsec:numFormats_quantMethod}
\subsubsection{Integer Format}
An $n$-bit signed integer (INT) is typically represented with a 2s-complement format \citep{yao2022zeroquant,xiao2023smoothquant,dai2021vsq}, where the most significant bit denotes the sign.

\subsubsection{Floating Point Format}
An $n$-bit signed floating point (FP) number $x$ comprises of a 1-bit sign ($x_{\mathrm{sign}}$), $B_m$-bit mantissa ($x_{\mathrm{mant}}$) and $B_e$-bit exponent ($x_{\mathrm{exp}}$) such that $B_m+B_e=n-1$. The associated constant exponent bias ($E_{\mathrm{bias}}$) is computed as $(2^{{B_e}-1}-1)$. We denote this format as $E_{B_e}M_{B_m}$.  

\subsubsection{Quantization Scheme}
\label{subsec:quant_method}
A quantization scheme dictates how a given unquantized tensor is converted to its quantized representation. We consider FP formats for the purpose of illustration. Given an unquantized tensor $\bm{X}$ and an FP format $E_{B_e}M_{B_m}$, we first, we compute the quantization scale factor $s_X$ that maps the maximum absolute value of $\bm{X}$ to the maximum quantization level of the $E_{B_e}M_{B_m}$ format as follows:
\begin{align}
\label{eq:sf}
    s_X = \frac{\mathrm{max}(|\bm{X}|)}{\mathrm{max}(E_{B_e}M_{B_m})}
\end{align}
In the above equation, $|\cdot|$ denotes the absolute value function.

Next, we scale $\bm{X}$ by $s_X$ and quantize it to $\hat{\bm{X}}$ by rounding it to the nearest quantization level of $E_{B_e}M_{B_m}$ as:

\begin{align}
\label{eq:tensor_quant}
    \hat{\bm{X}} = \text{round-to-nearest}\left(\frac{\bm{X}}{s_X}, E_{B_e}M_{B_m}\right)
\end{align}

We perform dynamic max-scaled quantization \citep{wu2020integer}, where the scale factor $s$ for activations is dynamically computed during runtime.

\subsection{Vector Scaled Quantization}
\begin{wrapfigure}{r}{0.35\linewidth}
  \centering
  \includegraphics[width=\linewidth]{sections/figures/vsquant.jpg}
  \caption{\small Vectorwise decomposition for per-vector scaled quantization (VSQ \citep{dai2021vsq}).}
  \label{fig:vsquant}
\end{wrapfigure}
During VSQ \citep{dai2021vsq}, the operand tensors are decomposed into 1D vectors in a hardware friendly manner as shown in Figure \ref{fig:vsquant}. Since the decomposed tensors are used as operands in matrix multiplications during inference, it is beneficial to perform this decomposition along the reduction dimension of the multiplication. The vectorwise quantization is performed similar to tensorwise quantization described in Equations \ref{eq:sf} and \ref{eq:tensor_quant}, where a scale factor $s_v$ is required for each vector $\bm{v}$ that maps the maximum absolute value of that vector to the maximum quantization level. While smaller vector lengths can lead to larger accuracy gains, the associated memory and computational overheads due to the per-vector scale factors increases. To alleviate these overheads, VSQ \citep{dai2021vsq} proposed a second level quantization of the per-vector scale factors to unsigned integers, while MX \citep{rouhani2023shared} quantizes them to integer powers of 2 (denoted as $2^{INT}$).

\subsubsection{MX Format}
The MX format proposed in \citep{rouhani2023microscaling} introduces the concept of sub-block shifting. For every two scalar elements of $b$-bits each, there is a shared exponent bit. The value of this exponent bit is determined through an empirical analysis that targets minimizing quantization MSE. We note that the FP format $E_{1}M_{b}$ is strictly better than MX from an accuracy perspective since it allocates a dedicated exponent bit to each scalar as opposed to sharing it across two scalars. Therefore, we conservatively bound the accuracy of a $b+2$-bit signed MX format with that of a $E_{1}M_{b}$ format in our comparisons. For instance, we use E1M2 format as a proxy for MX4.

\begin{figure}
    \centering
    \includegraphics[width=1\linewidth]{sections//figures/BlockFormats.pdf}
    \caption{\small Comparing LO-BCQ to MX format.}
    \label{fig:block_formats}
\end{figure}

Figure \ref{fig:block_formats} compares our $4$-bit LO-BCQ block format to MX \citep{rouhani2023microscaling}. As shown, both LO-BCQ and MX decompose a given operand tensor into block arrays and each block array into blocks. Similar to MX, we find that per-block quantization ($L_b < L_A$) leads to better accuracy due to increased flexibility. While MX achieves this through per-block $1$-bit micro-scales, we associate a dedicated codebook to each block through a per-block codebook selector. Further, MX quantizes the per-block array scale-factor to E8M0 format without per-tensor scaling. In contrast during LO-BCQ, we find that per-tensor scaling combined with quantization of per-block array scale-factor to E4M3 format results in superior inference accuracy across models. 


\section{Additional Experimental Results}
\label{app:exp}
\subsection{Comparison of Model Selection Methods}
Figure~\ref{fig:main_appendix} shows the regret performance of different model selection methods, highlighting the effectiveness of BOMS compared to baseline approaches.
\begin{figure*}[!htbp]
    \centering
    \includegraphics[width=0.9\textwidth]{figures/exp_baselines_ver2.png}
    \includegraphics[width=0.9\textwidth]{figures/exp_baselines_legend_ver2.png}
    \caption{Comparison of BOMS and the baselines in inference regret. BOMS achieves lower regrets than Validation and OPE in almost all tasks after 5-10 iterations.}
    %\caption{Comparison of BOMS and the baselines in inference regret. BOMS achieves lower regrets than Validation and OPE in almost all tasks after 5-10 iterations, which correspond to only 2-5\% of training data.}
    \label{fig:main_appendix}
\end{figure*}


\subsection{Comparison of Various Designs of Model Distance for BOMS in Inference Regret}
{Figure~\ref{fig:ablations} shows the regret performance of BOMS under various designs of model distance. We can observe that BOMS with  the proposed distance defined in Equation (\ref{eq:distance}) generally achieves the lowest inference regret across all the tasks. This further corroborates the theoretically-grounded design suggested by Proposition \ref{prop:model_dis}.}
%\subsection{Comparison of BOMS and Baseline Methods in Inference Regret}
%\pch{Figure~\ref{fig:baselines} shows the regret performance of BOMS and the baselines. We can observe that BOMS indeed outperforms the baseline methods on almost all the MuJoCo and Adroit tasks.  While OPE and MOPO can achieve low regrets on a few tasks, they are clearly not reliable enough for effective model selection. On the other hand, Random Selection generally improves the regret performance as the selection epoch increases, but its progress appears much slower than BOMS.}
%\begin{figure*}[!htp]
    \centering
    \includegraphics[width=0.8\textwidth]{figures/exp_baselines_4envs_ver2.png}
    \includegraphics[width=0.6\textwidth]{figures/exp_baselines_4env_legend_ver2.png}

    \caption{Comparison of BOMS and the baselines in inference regret. BOMS achieves lower regrets than Validation and OPE in all the tasks after 5 iterations, which correspond to only $1\%$-$2.5\%$ of the offline training data.}
    \label{fig:baselines}
\end{figure*}




\begin{figure*}[!htbp]
    \centering
    \includegraphics[width=0.9\textwidth]{figures/exp_ablations_ver2.png}
    \includegraphics[width=0.9\textwidth]{figures/exp_ablations_legend_ver2.png}
    \caption{Comparison of various designs of model distance for BOMS in inference regret. These results corroborate the proposed model-induced kernel.}
    \label{fig:ablations}
\end{figure*}

\newpage

\subsection{Normalized Rewards Under BOMS and Other Offline RL Methods on D4RL }
Table~\ref{tab:mopo_rewards} shows the comparison of BOMS and other offline RL methods in terms of normalized rewards. Specifically, we include 4 popular benchmark methods, namely COMBO \citep{yu2021combo}, IQL \citep{kostrikov2021implicitq}, TD3+BC \citep{fujimoto2021minimalist}, and TT \citep{janner2021offline}, for comparison. We follow the default procedure provided by D4RL \citep{fu2020d4rl} to compute the normalized rewards: (i) A
normalized score of 0 corresponds to the average returns of a uniformly random policy; (ii) A normalized score of 100 corresponds to the average returns of a domain-specific expert policy. 
Notably, despite that the vanilla MOPO (with validation-based model selection) itself is not particularly strong, MOPO augmented by the model selection scheme of BOMS can outperform other benchmark offline RL methods on various offline RL tasks. This further showcases the practical values of the proposed BOMS in enhancing offline MBRL.


\renewcommand{\arraystretch}{0.95}
\begin{table*}[!ht]
\caption{Normalized rewards of BOMS and the offline RL benchmark methods. The best performance of each row is highlighted in bold.}
\label{tab:mopo_rewards}
\begin{adjustbox}{center}
\scalebox{0.75}{\begin{tabular}{c|l|c|c|c|c|c|c|c|c|c|c}
    \toprule
        \multicolumn{2}{c|}{\multirow{2}{*}{Tasks}} &
        \multicolumn{4}{c|}{BOMS} &
        \multirow{2}{*}{MOPO} &
        \multirow{2}{*}{OPE} &
        \multirow{2}{*}{COMBO} &
        \multirow{2}{*}{IQL} &
        \multirow{2}{*}{TD3+BC} &
        \multirow{2}{*}{TT} \\
        \cline{3-6}
        \multicolumn{2}{c|}{\multirow{2}{*}{}}&{$T=5$} & {$T=10$} & {$T=15$} & {$T=20$} & & & & & &\\
    \midrule
        & med & 84.98$\pm$0.59\phantom{0} & 85.34$\pm$0.13\phantom{0} & 85.60$\pm$0.00 & \textbf{85.60$\pm$0.00} & -0.02 & 80.72 & 63.76 & 62.64 & 66.96 & 65.04\\
        walker2d & med-r & 20.85$\pm$12.29 & 38.91$\pm$18.58 & \phantom{0}45.59$\pm$17.18 & \textbf{\phantom{0}57.68$\pm$12.19} & 15.87 & 3.96 & 45.89 & 25.56 & 28.29 & 27.46\\
        & med-e & 7.81$\pm$4.52 & 13.13$\pm$12.39 & \phantom{0}24.02$\pm$23.97 & \phantom{0}24.40$\pm$23.64 & 0.07 & 76.66 & \textbf{116.18} & 111.82 & 112.33 & 92.83\\
    \midrule
        & med & 21.32$\pm$3.21\phantom{0} & 23.12$\pm$0.75\phantom{0} & 23.00$\pm$0.84 & 23.17$\pm$0.79 & 10.26 & 6.74 & 88.82 & 89.68 & 80.27 & \textbf{91.15}\\
        hopper & med-r & 96.74$\pm$1.37\phantom{0} & 97.09$\pm$2.68\phantom{0} & 97.05$\pm$2.69 & 97.05$\pm$2.69 & \textbf{99.20} & 42.02 & 70.07 & 38.21 & 24.80 & 40.08\\
        & med-e & 36.37$\pm$12.43 & 42.05$\pm$12.47 & \phantom{0}50.31$\pm$13.13 & \phantom{0}53.07$\pm$12.15 & 44.30 & 24.85 & 98.79 & 85.77 & 91.81 & \textbf{99.26}\\
    \midrule
        & med & 56.27$\pm$0.22\phantom{0} & 56.37$\pm$0.24\phantom{0} & 56.52$\pm$0.19 & \textbf{56.52$\pm$0.19} & 54.55 & 37.39 & 53.7 & 47.66 & 48.52 & 44.4\\
        halfcheetah & med-r & 56.65$\pm$0.86\phantom{0} & 56.78$\pm$0.69\phantom{0} & 56.86$\pm$0.74 & \textbf{56.86$\pm$0.74} & 54.66 & 56.24 & 54.66 & 44.94 & 45.33 & 44.85\\
        & med-e & 103.24$\pm$3.75\phantom{00} & 104.63$\pm$0.56\phantom{00} & \phantom{0}104.87$\pm$0.74 \phantom{0}& \textbf{104.87$\pm$0.74\phantom{0}} & 75.57 & 88.44 & 89.9 & 59.18 & 61.81 & 29.05\\
    \midrule
        & cloned & 5.69$\pm$4.01 & 6.91$\pm$6.33 & \phantom{0}8.95$\pm$6.22 & \textbf{\phantom{0}9.43$\pm$6.86} & 6.89 & 3.70 & - & - & - & -\\
        pen & mixed & 38.98$\pm$8.76\phantom{0} & 42.49$\pm$6.04\phantom{0} & 44.81$\pm$5.55 & \textbf{44.85$\pm$5.55} & 28.66 & 17.45 & - & - & - & -\\
        & expert & 36.54$\pm$7.58\phantom{0} & 40.19$\pm$12.65 & 45.06$\pm$9.59 & \textbf{49.09$\pm$8.88} & 47.58 & 18.67 & - & - & - & -\\
    \bottomrule
\end{tabular}}
\end{adjustbox}
\end{table*}

\newpage

\subsection{Reproduction of MOPO}
Table~\ref{tab:mopo_impl_comp} shows the normalized rewards of our MOPO implementation and those of the original MOPO reported in \citep{yu2020mopo}.
Since our experiments are mainly based on MOPO, we would like to clarify that we have indeed tried our best on tuning MOPO properly, and hence our reproduced MOPO has similar or even better performance than the original MOPO results on halfcheetah and hopper. Additionally, although our MOPO perform on walker2d, after model selection with a small online interaction budget, our MOPO can still enjoy good results and even outperform other SOTA offline RL as shown in Table~\ref{tab:mopo_rewards}.

\renewcommand{\arraystretch}{0.95}
\begin{table*}[!htbp]
\caption{Normalized rewards of our reproduced MOPO and the original MOPO.}
\label{tab:mopo_impl_comp}
\begin{adjustbox}{center}
\scalebox{0.85}{\begin{tabular}{c|l|c|c}
    \toprule
        \multicolumn{2}{c|}{Tasks} &
        {Our MOPO} &
        {Original MOPO} \\
    \midrule
        & med & -0.1 & \textbf{17.8}\\
        walker2d & med-r & 15.9 & \textbf{39.0}\\
        & med-e & 0.1 & \textbf{44.6}\\
    \midrule
        & med & 10.3 & \textbf{28.0}\\
        hopper & med-r & \textbf{99.2} & 67.5\\
        & med-e & \textbf{44.3} & 23.7\\
    \midrule
        & med & \textbf{54.6} & 42.3\\
        halfcheetah & med-r &  \textbf{54.7} & 53.1\\
        & med-e & \textbf{75.6} & 63.3\\
    \midrule
        & cloned & 6.9 & -\\
        pen & mixed & 28.7 & -\\
        & expert & 47.6 & -\\
    \bottomrule
\end{tabular}}
\end{adjustbox}
\end{table*}

\begin{comment}
    \renewcommand{\arraystretch}{0.95}
\begin{table*}[!htbp]
\caption{Normalized rewards of our reproduced MOPO and the original MOPO.}
\label{tab:mopo_impl_comp}
\begin{adjustbox}{center}
\begin{tabular}{cl|c|c}
    \toprule
        \multicolumn{2}{c|}{\multirow{2}{*}{Environments}} &
        {Our} &
        {Original} \\
        & & {MOPO} & {MOPO}\\
    \midrule
        \parbox[t]{2mm}{\multirow{3}{*}{\rotatebox[origin=c]{90}{walker2d}}} 
        & med & -0.02 & \textbf{17.8$\pm$19.3}\\
        & med-r & 15.87 & \textbf{39.0$\pm$9.6}\\
        & med-e & 0.07 & \textbf{44.6$\pm$12.9}\\
    \midrule
        \parbox[t]{2mm}{\multirow{3}{*}{\rotatebox[origin=c]{90}{hopper}}}
        & med & 10.26 & \textbf{28.0$\pm$12.4}\\
        & med-r & \textbf{99.20} & 67.5$\pm$24.7\\
        & med-e & \textbf{44.30} & 23.7$\pm$6.0\\
    \midrule
        \parbox[t]{2mm}{\multirow{3}{*}{\rotatebox[origin=c]{90}{cheetah}}}
        & med & \textbf{54.55} & 42.3$\pm$1.6\\
        & med-r &  \textbf{54.66} & 53.1$\pm$2.0\\
        & med-e & \textbf{75.57} & 63.3$\pm$38.0\\
    \midrule
        \parbox[t]{2mm}{\multirow{3}{*}{\rotatebox[origin=c]{90}{pen}}} 
        & cloned & 6.89 & -\\
        & mixed & 28.66 & -\\
        & expert & 47.58 & -\\
    \bottomrule
\end{tabular}
\end{adjustbox}
\end{table*}
\end{comment}
\begin{comment}
\subsection{Regrets and Success Rates of the Door-Open Task in Meta-World}

\begin{figure*}[!htbp]
    \centering
    \includegraphics[width=0.9\textwidth]{figures/door_open.png}
    \caption{Comparison of BOMS and the baselines on the door-open task. Left: Inference regrets; Right: Success rates.}
    \label{fig:door_open}
\end{figure*}
\end{comment}


\end{document}