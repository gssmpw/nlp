\section{Literature Review}
\label{sm:literature review}
As we discussed in Section \ref{sec:methods}, the iterative solution algorithms are categorized into neural, symbolic, and neuro-symbolic approaches, based on the parameterization of the space of functions that they consider. 
We exclude numerical methods from this more extensive review, since they are targeting solely approximate solutions and their strengths and weaknesses are generally well-known.

\paragraph{Neural Approaches} A neural solver considers a nonlinear parameterization of the solution function using a neural network. Then, an iterative solution algorithm is defined on the space of the neural network weights to find the parameterization that fits the residual of the differential equation with the lowest error. This family of methods is defined under the Physics-Informed Machine Learning umbrella \cite{karniadakis2021physics}. A variety of physics-informed approaches has been presented \cite{sirignano2018dgm, bhatnagar2019prediction, zhu2018bayesian, khoo2021solving}, with physics-informed neural networks (PINNs)  \cite{raissi2018deep, lagaris1998artificial} one of the most widely applied.  A comprehensive overview is given by Toscano et. al. \cite{toscano2024pinns}. Despite successful application in many different fields \cite{kissas2020machine, manav2024phase, cai2021physics, mao2020physics}, the generalization capabilities of these methods are limited, as they require training a new neural network for new conditions, meshes or physics. Moreover, several difficulties with PINNs have been reported in practice, such as slow convergence during training and strong sensitivity to the initial network weights \cite{de2023operator, wang2022and, wang2021understanding}.

% even in cases where measurements of the baseline function are assumed in the domain. 

\paragraph{Symbolic Approaches} Solving (partial) differential equations ((P)DEs) is a fundamental challenge in the domain of applied mathematics and across a variety of computational sciences. Traditionally, based on the complexity of the DE system, solving it is possible through the use of \textit{Analytical} techniques. For example, linear PDEs can be transformed into simpler ordinary differential systems (ODEs) with techniques such as the \textit{Separation of Variables} \cite{miller1977symmetry} and the Risch method \cite{risch1969the}, while in the case of periodic domains \textit{Fourier transforms} \cite{debnath2006integral, dyke2014an} can be employed. Moreover, there exists commercial software, such as Mathematica \cite{mathematica}, that provides explicit solutions of differential equations as a black-box solver with proprietary techniques. 

A different approach relies on the symbolic solution of DEs. Similar to numerical solvers, these methods receive a symbolic form of the operator, a domain discretization, and initial (and boundary) conditions as arguments. Then, they construct random expressions combining a predefined set of binary and unary operations, variables, and constants. The generated symbolic expressions are checked for how well they satisfy $\mathcal{R}(u)$ and the guess is updated until a symbolic solution of a DE is discovered. Tsoulos et. al. \cite{Tsoulos2006SolvingDE} propose an approach for modeling the genome of the symbolic expressions of differential equations using Formal Grammars and using a Genetic Programming approach for evolving the expression genomes. Liang et. al. \cite{liang2022finite} propose a method that samples a finite number of expressions using a Reinforcement Learning pipeline and performs a weighted average over them to provide an approximate solution of Differential Equations. The drawback of this method is that it needs to be re-trained for each specific task, and that the increase in length of the sub-expressions results to combinatorial explosion of the search algorithm even though this is mitigated by considering sub-expressions of small length. This method, however, also possesses the a-posteriori self compositions property of SIGS. Chaquet et. al. \cite{chaquet2019using} consider a CMA-ES strategy for finding the coefficients of the basis function parameters that parameterize the space of candidate functions. This method also considers probabilistic updates for the iterative method. Boudouaoui et. al. \cite{boudouaoui2020solving} propose a framework that considers an Artificial Bee Colony Programming for solving differential equations, which is a more sophisticated algorithm for performing heuristic search by mimicking the way that bee colonies search for resources. Saeton et. al. \cite{seaton2010analytic} propose a framework for performing a Cartesian Genetic Programming approach for sampling candidate expressions which is shown to have improved properties for faster heuristics search. The symbolic approaches proposed in the literature are computationally expensive due to the combinatorial complexity of the underlying graph topology \cite{virgolin2022symbolic, kissas2024language}, the sensitive dependence on the initial guess, and the lack of a structured way to include domain knowledge. 

\paragraph{Neuro-Symbolic Approaches} Neuro-symbolic approaches are built to combine elements of both symbolic and neural algorithms to enhance the performance of the solver, and few such attempts have been reported yet. Wei et. al. \cite{wei2024closed} propose a methodology that considers a two-step optimization process: It first randomly samples tree structures with variable coefficients \cite{dsr}, and then discovers a policy that provides the parameters with the lowest error by combining reinforcement learning and a parameter neural network. However, this method is very inefficient as the heuristic search is performed in two levels, the skeletons and the parameters. We consider the method proposed by Lampe et. al. \cite{lample2019deep} as a neuro-symbolic method as it is capable of using symbolic computations to discover the solution of ODEs. However, this method is very limited as it works only for explicit ODEs and expensive as it considers very large transformer models.