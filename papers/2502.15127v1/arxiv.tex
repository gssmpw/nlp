\documentclass{article}

\usepackage{primearxiv}

\usepackage[utf8]{inputenc} % allow utf-8 input
\usepackage[T1]{fontenc}    % use 8-bit T1 fonts
\usepackage{hyperref}       % hyperlinks
\usepackage{url}            % simple URL typesetting
\usepackage{booktabs}       % professional-quality tables
\usepackage{amsfonts}       % blackboard math symbols
\usepackage{nicefrac}       % compact symbols for 1/2, etc.
\usepackage{microtype}      % microtypography
\usepackage{lipsum}
\usepackage{fancyhdr}       % header
\usepackage{graphicx}       % graphics

\usepackage{xcolor}
\usepackage{todonotes}
\usepackage{amsmath}
\usepackage{amsthm}

\usepackage{subcaption}
\usepackage{caption} 
\captionsetup[table]{skip=10pt}
\setlength{\tabcolsep}{18pt}
\renewcommand{\arraystretch}{1.2}
\setlength{\arrayrulewidth}{0.2mm}


\newcommand{\red}[1]{\textcolor{red}{#1}}
\newcommand{\tb}{\textbullet\,}
\usepackage{amsfonts}

\newcommand\mc{\mathcal}
\newcommand\mb{\mathbf}

\usepackage{listings}
\usepackage{xcolor}       
\usepackage{multirow}
\usepackage{tcolorbox}
\usepackage{tikz}
\usetikzlibrary{decorations.pathreplacing}


\colorlet{lightyellow}{yellow!40}

\lstset{
  basicstyle=\ttfamily,
  breaklines=true,
  columns=fullflexible,
  mathescape,
  literate={``}{\textquotedblleft}1,
}

\makeatletter
{\small % Capture font definitions of \small
\xdef\f@size@small{\f@size}
\xdef\f@baselineskip@small{\f@baselineskip}
\normalsize % Capture font definitions for \normalsize
\xdef\f@size@normalsize{\f@size}
\xdef\f@baselineskip@normalsize{\f@baselineskip}
}
% Define new \smalltonormalsize font size
\newcommand{\smalltonormalsize}{%
  \fontsize
    {\fpeval{(\f@size@small+\f@size@normalsize)/2}}
    {\fpeval{(\f@baselineskip@small+\f@baselineskip@normalsize)/2}}%
  \selectfont
}
\makeatother

\usepackage{amsmath, amssymb, amsthm}
\usepackage{mathtools} % Enhances amsmath
\newtheorem{theorem}{Theorem}
\theoremstyle{remark}
\newtheorem*{remark}{Remark}
\newtheorem{definition}{Definition}
\newtheorem{corollary}{Corollary}
\newtheorem{hypothesis}{Hypothesis}
\DeclareMathOperator*{\argmax}{arg\,max}
\DeclareMathOperator*{\argmin}{arg\,min}

%Header
\pagestyle{fancy}
\thispagestyle{empty}
\rhead{ \textit{ }} 

% Update your Headers here
\fancyhead[LO]{The Imitation Game for Educational AI}
% \fancyhead[RE]{Firstauthor and Secondauthor} % Firstauthor et al. if more than 2 - must use \documentclass[twoside]{article}
  
%% Title
\title{The Imitation Game for Educational AI}

\author{
  Shashank Sonkar\thanks{Equal contribution.}, \, Naiming Liu\footnotemark[1], \, Xinghe Chen, \, Richard G. Baraniuk \\
  Rice University \\
  Houston, TX \\
  \texttt{shashank.sonkar@rice.edu} \\
}

\newcommand{\richb}[1]{\textcolor{blue}{\bf [richb: #1]}}
\newcommand{\sonkar}[1]{\textcolor{orange}{\bf [ss: #1]}}

\begin{document}
\maketitle

\begin{abstract}
\begin{abstract}
Retrieval-Augmented Generation (RAG) is often used with Large Language Models (LLMs) to infuse domain knowledge or user-specific information. In RAG, given a user query, a retriever extracts chunks of relevant text from a knowledge base. These chunks are sent to an LLM as part of the input prompt. Typically, any given chunk is repeatedly retrieved across user questions. However, currently, for every question, attention-layers in LLMs fully compute the key values (KVs) repeatedly for the input chunks, as state-of-the-art methods cannot reuse KV-caches when chunks appear at arbitrary locations with arbitrary contexts. Naive reuse leads to output quality degradation.  This leads to potentially redundant computations on expensive GPUs and increases latency. In this work, we propose \sys, a system for managing and reusing precomputed KVs corresponding to the text chunks (we call \textit{chunk-caches}) in RAG-based systems. We present how to identify \hl{\textit{chunk-caches} that are reusable}, how to efficiently perform a small fraction of recomputation to \textit{fix} the cache to maintain output quality, and how to efficiently store and evict \textit{chunk-caches} in the hardware for maximizing reuse while masking any overheads. With real production workloads as well as synthetic datasets, we show that \sys reduces redundant computation by \textbf{51\%} over SOTA prefix-caching and \textbf{75\%} over full recomputation.
\hl{Additionally, with continuous batching on a real production workload, we get a \textbf{1.6$\times$} speedup in throughput and a \textbf{2$\times$} reduction in end-to-end response latency over prefix-caching while maintaining quality, for both the \llama-3-8B and \llama-3-70B models. 
}
\end{abstract}





\end{abstract}

\section{Introduction}
\label{sec:intro}

\begin{figure*}[tb]
    \centering
    \includegraphics[width=0.848\linewidth]{figs/circuitnn.pdf} 
    \caption{Illustration of differentiable CircuitNN. CircuitNN is designed based on differentiable NAND gates. After DAS is guided by PI and PO pairs of the truth table, CircuitNN can get the precise circuit architecture logic equivalent to the truth table.}
    \label{fig:circuitnn}
\end{figure*}

% 1. Describe the importance of logic synthesis
% 2. Existing Problems
% (a) Neural Architecture Search: Unstable, Predefined Setting, etc.
% (b) Circuit Generation: Probabilistic Model, Logic Equivalence

With the rapid advancement of technology, the scale of integrated circuits (ICs) has expanded exponentially. 
This expansion has introduced significant challenges in chip manufacturing, particularly concerning power and area metrics.
A primary objective in IC design is achieving the same circuit function with fewer transistors, thereby reducing power usage and area occupancy.

Logic synthesis~\cite{hachtel2005logicsynth}, a critical step in electronic design automation (EDA), transforms behavioral-level circuit designs into optimized gate-level circuits, ultimately yielding the final IC layout. 
The primary goal of logic synthesis is to identify the physical implementation with the fewest gates for a given circuit function. 
This task constitutes a challenging NP-hard combinatorial optimization problem. 
Current logic synthesis tools~\cite{brayton2010abc, wolf2013yosys} rely on human-designed heuristics, often leading to sub-optimal outcomes.

Differentiable architecture search (DAS) techniques~\cite{liu2018darts, chu2020darts} offer novel perspectives on addressing challenges in this problem.
Circuit functions can be represented through truth tables, which map binary inputs to their corresponding outputs. 
Truth tables provide a precise representation of input-output relationships, ensuring the design of functionally equivalent circuits.
Inspired by this, researchers~\cite{deepmind2024ai4sys, wang2024tnet} have begun exploring the application of DAS to synthesize circuits directly from truth tables.
Specifically, \citet{deepmind2024ai4sys} proposed CircuitNN, a framework that learns differentiable connection structures with logic gates, enabling the automatic generation of logic circuits from truth tables.
This approach significantly reduces the complexity of traditional circuit generation. 
Building on this, \citet{wang2024tnet} introduced T-Net, a triangle-shaped variant of CircuitNN, incorporating regularization techniques to enhance the efficiency of DAS.

Despite these advancements, several challenges remain. 
The computational complexity of DAS grows quadratically with the number of gates, posing scalability issues.
Although triangle-shaped architecture~\cite{wang2024tnet} partially mitigates this problem, redundancy persists. 
%Additionally, DAS is susceptible to converging to local optima, limiting the ability to search architectures that satisfy the given truth tables~\cite{liu2018darts}. 
%Furthermore, hyperparameters (network depth and layer width) require extensive searches, introducing complexity and prolonging the synthesis process. 
Additionally, DAS is susceptible to converging to local optima~\cite{liu2018darts} and hyperparameters (network depth and layer width) require extensive searches. 
The challenges arise from the vast search space in DAS. 
% Even with predefined settings for CircuitNN, finding a configuration that meets the truth table requires extensive trial and error during the DAS process. 
Intuitively, limiting the search space through predefined parameters (network depth, gates per layer, and connection probabilities) can significantly reduce the complexity.

Recent advances~\cite{openai2023gpt4, abramson2024alphafold3, esser2024sd3, li2024mar} in conditional generative models have demonstrated remarkable performance across language, vision, and graph generation tasks. 
Motivated by these developments, we propose a novel approach to circuit generation that generates preliminary circuit structures to guide DAS in generating refined circuits matching specified truth tables. 
Firstly, we introduce CircuitVQ, a tokenizer with a discrete codebook for circuit tokenization. 
Built upon our Circuit AutoEncoder framework~\cite{hou2022graphmae,li2023maskgae,wu2025mgvga}, CircuitVQ is trained through a circuit reconstruction task. 
Specifically, the CircuitVQ encoder encodes input circuits into discrete tokens using a learnable codebook, while the decoder reconstructs the circuit adjacency matrix based on these tokens.
Subsequently, the CircuitVQ encoder serves as a circuit tokenizer for CircuitAR pretraining, which employs a masked autoregressive modeling paradigm~\cite{chang2022maskgit, li2023mage}. 
In this process, the discrete codes function as supervision signals. 
After training, CircuitAR can generate discrete tokens progressively, which can be decoded into initial circuit structures by the decoder of the CircuitVQ. 
These prior insights can guide DAS in producing refined circuits that match the target truth tables precisely.

Our key contributions can be summarized as follows:
\begin{itemize}
\item We introduce CircuitVQ, a circuit tokenizer that facilitates graph autoregressive modeling for circuit generation, based on our Circuit AutoEncoder framework;
\item Develop CircuitAR, a model trained using masked autoregressive modeling, which generates initial circuit structures conditioned on given truth tables;
\item Propose a refinement framework that integrates differentiable architecture search to produce functionally equivalent circuits guided by target truth tables;
\item Comprehensive experiments demonstrating the scalability and capability emergence of our CircuitAR and the superior performance of the proposed circuit generation approach.
\end{itemize}

% Motivation
% (a) Diffusion (Vision, Graph), Autoregressive (Language, Vision)
% (b) Circuit Generation for Predefined Setting
% (c) Neural Architecture Search for Strict Logic Equivalence

% Contribution
% (a) Circuit Tokenizer (new transformer arch, training strategy)
% (b) CircuitAR (train and gen strategies, post-ar strategy)
% (c) Extensive Evaluation including BitD (Bit Distance) for Scalability


\section{Test Design}

Our proposed evaluation framework consists of two distinct phases designed to validate an AI system's understanding of student misconceptions. In the first phase, we collect unbiased samples of student misconceptions through open-ended responses. The second phase tests whether the AI system can accurately predict how these misconceptions will manifest in a newly generated question.

Let $S$ denote our set of students and $D$ our domain of questions. For any student $s \in S$ and question $q \in D$, we define $A(s,q)$ as the student's response and $C(q)$ as the correct answer. 

In Phase 1, each student $s$ responds to a set of questions $Q_s \subset D$ without multiple choice options. For each incorrect response, we record the tuple $(s,q,A(s,q))$ where $A(s,q) \neq C(q)$. These tuples provide unbiased samples of natural student misconceptions, uninfluenced by the presence of pre-selected answer choices.

For Phase 2, given each Phase 1 tuple $(s,q,A(s,q))$:

1. The AI system implements a function $f_{LLM}: D \times A \to D$ that generates a new question $q'$ based on the original question and incorrect answer.

2. Given $q'$, both our AI system and a human expert independently generate predicted wrong answers. The AI system implements $g_{AI}: D \times D \times A \to A$ that generates prediction $a'$, while the human expert implements $g_H: D \times D \times A \to A$ generating prediction $a''$. A random wrong answer $r$ is also generated to serve as a control.

3. The student $s$ then receives question $q'$ as a multiple choice question with four options presented in random order: the correct answer $C(q')$, the AI-predicted wrong answer $a'$, the expert-predicted wrong answer $a''$, and the random wrong answer $r$. The student's selection is denoted $P_2(s,q,q')$.


% \begin{table}[t]
% \small
% \caption{\textbf{Key Statistics of \dataset}}
% \centering
% % \begin{center}
% \begin{tabular}{lc}
% \toprule
% \textbf{Statistic} & \textbf{Number} \\
% \midrule
% Total questions & 1130 \\
% \quad - Reasoning split & 837 (74.1\%) \\
% \quad \quad Multiple-choice questions & 431 \\
% \quad \quad Free-form questions & 406 \\
% \quad - Perception split & 293 (25.9\%) \\
% \quad \quad Multiple-choice questions & 275 \\
% \quad \quad Free-form questions & 18 \\
% \midrule
% Total essential step annotation & 3865 \\
% \quad - Total inference conclusion & 2667 \\
% \quad - Average inference conclusion & 3.2 \\
% \quad - Total image observation & 1198 \\
% \quad - Average image observation & 1.4 \\
% Reference image caption item & 1579 \\
% Average reference caption & 1.9 \\
% \midrule
% Number of unique images & 2380 \\
% Number of unique questions & 808 \\
% Number of unique answers & 271 \\
% \midrule
% Maximum question length & 477 \\
% Maximum answer length & 15 \\
% Average question length & 41.2 \\
% Average answer length & 1.2 \\
% \bottomrule
% \end{tabular}
% % \end{center}
% % \vspace{-0.6cm}
% \label{table:statistics}
% \end{table}

% \begin{table}[t]
% \caption{Classification accuracies for naive Bayes and flexible
% Bayes on various data sets.}
% \label{sample-table}
% \vskip 0.15in
% \begin{small}
% \begin{sc}
% \begin{tabular}{lcccr}
% \toprule
% Data set & Naive & Flexible & Better? \\
% \midrule
% Breast    & 95.9$\pm$ 0.2& 96.7$\pm$ 0.2& $\surd$ \\
% Cleveland & 83.3$\pm$ 0.6& 80.0$\pm$ 0.6& $\times$\\
% Glass2    & 61.9$\pm$ 1.4& 83.8$\pm$ 0.7& $\surd$ \\
% Credit    & 74.8$\pm$ 0.5& 78.3$\pm$ 0.6&         \\
% Horse     & 73.3$\pm$ 0.9& 69.7$\pm$ 1.0& $\times$\\
% Meta      & 67.1$\pm$ 0.6& 76.5$\pm$ 0.5& $\surd$ \\
% Pima      & 75.1$\pm$ 0.6& 73.9$\pm$ 0.5&         \\
% Vehicle   & 44.9$\pm$ 0.6& 61.5$\pm$ 0.4& $\surd$ \\
% \bottomrule
% \end{tabular}
% \end{sc}
% \end{small}
% \vskip -0.1in
% \end{table}

\section{The Insufficiency of Unconditioned Single-Phase Testing}

A natural starting point for evaluating an AI system's understanding of student cognition would be to test its ability to generate plausible wrong answers that students might choose. The intuition is compelling: if an AI can anticipate how students will err, surely it understands how they think. This leads to a simple unconditioned single-phase test where both AI and human experts generate distractors for multiple choice questions without knowledge of individual student reasoning patterns. Let us formalize this design:

\begin{definition}[Unconditioned Single-Phase Test Design]
For any question $q$, let:
\begin{itemize}
   \item $C(q)$ be the correct answer
   \item $A(q)$ be the AI-generated distractor 
   \item $H(q)$ be the human expert-generated distractor
   \item $R(q)$ be a randomly generated incorrect answer
\end{itemize}

The test presents these options in random order to students. Let $S(q,s)$ represent student $s$'s selection for question $q$. Critically, both $A(q)$ and $H(q)$ are generated without observing any prior responses from student $s$.
\end{definition}

\begin{definition}[Performance Metrics]
The quality of AI-generated distractors would be evaluated by comparing selection rates:
\begin{align*}
p_A &= \mathbb{P}(S(q,s) = A(q)) \\
p_H &= \mathbb{P}(S(q,s) = H(q))
\end{align*}
with the AI system "passing" if $|p_A - p_H| \leq \epsilon$ for some small $\epsilon > 0$.
\end{definition}

However, this unconditioned approach has fundamental flaws that make it inadequate for validating true understanding of student cognition:

\begin{theorem}[Unconditioned Convergence]
Given a question $q$ with misconception set $M(q) = \{m_1,...,m_k\}$ where $P(m_i)$ represents the probability of misconception $m_i$ in the student population, both AI and human experts will rationally converge to targeting $m^* = \argmax_{m_i \in M(q)} P(m_i)$.
\end{theorem}

\begin{proof}
For any distractor generator (AI or human), the expected selection rate is maximized by targeting the most common misconception:
\[\mathbb{E}[p] = \sum_{i=1}^k P(m_i)\mathbb{I}(m_i \text{ targeted}) \leq \max_{i} P(m_i)\]
with equality achieved only when targeting $m^*$.
\end{proof}

This convergence creates two critical problems:

\begin{enumerate}
   \item \textbf{Population-Level Optimization}: Both AI and human experts are incentivized to target the single most common misconception for each question, regardless of individual student reasoning patterns. This reduces the test to measuring statistical pattern matching rather than understanding of student cognition.
   
   \item \textbf{False Equivalence}: An AI system could ``pass'' this test by simply learning population-level statistics about common wrong answers, without any actual understanding of how individual students reason about concepts or how their misconceptions evolve across related problems.
\end{enumerate}

These flaws demonstrate why a conditioned two-phase design is necessary - one that can validate an AI system's ability to model individual student reasoning rather than just aggregate statistics. To overcome the limitations of unconditioned testing, our two-phase design introduces crucial conditioning on individual student reasoning. In Phase 1, each student $s$ provides open-ended responses to questions $Q_s$, generating tuples $(s,q,A(s,q))$ where $A(s,q)$ is their incorrect answer. In Phase 2, given each Phase 1 tuple $(s,q,A(s,q))$, both the AI system and human experts observe this specific student mistake and use it to inform their predictions. The AI generates a new question $q'$ and predicts the wrong answer $A'(s,q')$ that this particular student would give, while human experts predict their own expected wrong answer $H'(s,q')$. Both these predictions are explicitly conditioned on the observed student mistake from Phase 1.


This conditioning provides several key advantages:

\begin{theorem}[Prediction Differentiation]
Given students $s_1, s_2$ with different misconceptions $m_1, m_2$ observed in Phase 1, optimal predictions for Phase 2 will differ:
\[A'(s_1,q') \neq A'(s_2,q')\]
even for the same question $q'$.
\end{theorem}

\begin{proof}
Unlike the unconditioned case where predictions converge to the population mode, conditioned predictions should maximize:
\[\mathbb{P}(A'(s,q') \text{ chosen} \mid A(s,q))\]
This probability differs based on the specific misconception demonstrated in Phase 1.
\end{proof}

This conditioning forces the AI to demonstrate three crucial capabilities:
\begin{enumerate}
    \item Understanding individual student reasoning patterns from Phase 1 responses
    \item Generating new questions that test the same conceptual understanding
    \item Predicting how specific misconceptions will manifest in new contexts
\end{enumerate}

Unlike the unconditioned design, success in this framework cannot be achieved through simple population-level statistics, as it requires modeling the cognitive processes of individual students. This deeper understanding of student reasoning pathways has direct implications for educational capabilities: an AI system that can accurately predict how a student's specific misconceptions will manifest across different problems is necessarily equipped to provide targeted tutoring interventions and personalized feedback. The ability to generate new questions that probe and predict individual misconceptions demonstrates the AI system's capacity to construct adaptive learning sequences and deliver feedback that addresses each student's particular cognitive challenges.

\section*{Conclusion}
This paper aims to enhance our understanding of the computational complexity of computing various Shapley value variants. We found that for various ML models --- including decision trees, regression tree ensembles, weighted automata, and linear regression --- both local and global interventional and baseline SHAP can be computed in polynomial time under HMM modeled distributions. This extends popular algorithms, such as TreeSHAP, beyond their empirical distributional scope. We also establish strict complexity gaps between the various SHAP variants (baseline, interventional, and conditional) and prove the intractability of computing SHAP for tree ensembles and neural networks in simplified scenarios. Overall, we present SHAP as a versatile framework whose complexity depends on four key factors: \begin{inparaenum}[(i)] \item model type, \item SHAP variant, \item distribution modeling approach, \item and local vs. global explanations\end{inparaenum}. We believe this perspective provides deeper insight into the computational complexity of SHAP, paving the way for future work.




%We believe that our framework provides a more intricate understanding of SHAP computation complexity across different models, distributions, and variants, paving the way for further research.

Our work opens promising directions for future research. First, expanding our computational analysis to other SHAP-related metrics, such as asymmetric SHAP~\citep{frye20} and SAGE~\citep{covert2020understanding}, would be valuable. Additionally, we aim to explore more expressive distribution classes and relaxed assumptions beyond those in Section \ref{sec:tractable} while maintaining tractable SHAP computation. Finally, when exact computation is intractable (Section \ref{sec:intractable}), investigating the approximability of SHAP metrics through approximation and parameterized complexity theory~\citep{downey2012parameterized} is an important direction.

%Our work opens several promising avenues for future research on the computational properties of explainable AI methods, with a particular focus on SHAP. First, it would be interesting to broaden the computational analysis conducted in this work to include other popular SHAP-related metrics in the literature, such as asymmetric SHAP \cite{frye20} and SAGE \cite{covert2020understanding}. Also, in the future, we aim to explore more expressive distribution classes and relaxed distributional assumptions—extending beyond those examined in Section \ref{sec:tractable} —that still yield tractable SHAP computation. Finally, when exact computation proves intractable (Section \ref{sec:intractable}), it is worthwhile to theoretically investigate the question of the approximability of computing the SHAP metrics across various configurations, through the lens of approximation and parametrized complexity theory \cite{arora2009computational}.

%This paper aims to deepen our understanding of the computational complexity involved in obtaining different Shapley value variants. We found that for a variety of ML models, including decision trees, tree ensembles for regression, weighted automata, and linear regression models — computing both local and global interventional and baseline SHAP can be done in polynomial time when distributions are modeled by HMMs. This extends the distributional scope of popular algorithms like TreeSHAP, which is limited to empirical distributions. Additionally, we demonstrate a strict complexity gap between SHAP variants, showing that interventional and baseline SHAP can be strictly easier to compute than conditional SHAP. Despite these positive results, we uncovered intractability for various SHAP variants in neural networks and tree ensembles. Finally, we provided generalized complexity relations across SHAP variants. We believe that our framework offers a deeper understanding of the complexity involved in computing SHAP across various variants, models, distributions, as well as in both local and global computations, laying the groundwork for future research.

\section{Limitations} 

In this work, we compared the effectiveness and interplay of SFT and RL-based methods, under fixed data constraints. In particular, we chose offline methods like DPO and KTO as the baseline implementation of the RL method because it eliminates the need for reward modeling or iterative finetuning. This means that the process of development is limited to collecting an offline dataset and fientuning it - making it the most fair comparable to SFT in terms of implementation effort, compute costs and annotation efforts. Since this baseline RL method shows optimal performance over SFT, we hope that this motivates future work to study more complex RL-based methods and their interplay with SFT. In addition, we used GPT4o annotation for synthetic data generation, and also for evaluating Summarization and Helpfulness, which could include potential biases inherited from the model. 

In addition, we limited the size of the model to under 10 Billion parameters, to keep the finetuning cost low enough to ignore as compared to the data annotation costs. In addition, it would be extremely compute resource intensive to run thousands of finetuning runs with larger model sizes like 70B parameters. We hope that future work would study the scaling trends of RL-based methods against different model sizes, and also study the compute-data trade-off in-depth.


\section*{Acknowledgments}
This work was supported by NSF grant 1842378, ONR grant N0014-20-1-2534, AFOSR grant FA9550-22-1-0060, a Vannevar Bush Faculty Fellowship, OpenAI, and ONR grant N00014-18-1-2047.

%Bibliography
\bibliographystyle{unsrt}  
\bibliography{custom,dsp}  


\end{document}
