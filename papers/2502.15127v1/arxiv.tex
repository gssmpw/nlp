\documentclass{article}

\usepackage{primearxiv}

\usepackage[utf8]{inputenc} % allow utf-8 input
\usepackage[T1]{fontenc}    % use 8-bit T1 fonts
\usepackage{hyperref}       % hyperlinks
\usepackage{url}            % simple URL typesetting
\usepackage{booktabs}       % professional-quality tables
\usepackage{amsfonts}       % blackboard math symbols
\usepackage{nicefrac}       % compact symbols for 1/2, etc.
\usepackage{microtype}      % microtypography
\usepackage{lipsum}
\usepackage{fancyhdr}       % header
\usepackage{graphicx}       % graphics

\usepackage{xcolor}
\usepackage{todonotes}
\usepackage{amsmath}
\usepackage{amsthm}

\usepackage{subcaption}
\usepackage{caption} 
\captionsetup[table]{skip=10pt}
\setlength{\tabcolsep}{18pt}
\renewcommand{\arraystretch}{1.2}
\setlength{\arrayrulewidth}{0.2mm}


\newcommand{\red}[1]{\textcolor{red}{#1}}
\newcommand{\tb}{\textbullet\,}
\usepackage{amsfonts}

\newcommand\mc{\mathcal}
\newcommand\mb{\mathbf}

\usepackage{listings}
\usepackage{xcolor}       
\usepackage{multirow}
\usepackage{tcolorbox}
\usepackage{tikz}
\usetikzlibrary{decorations.pathreplacing}


\colorlet{lightyellow}{yellow!40}

\lstset{
  basicstyle=\ttfamily,
  breaklines=true,
  columns=fullflexible,
  mathescape,
  literate={``}{\textquotedblleft}1,
}

\makeatletter
{\small % Capture font definitions of \small
\xdef\f@size@small{\f@size}
\xdef\f@baselineskip@small{\f@baselineskip}
\normalsize % Capture font definitions for \normalsize
\xdef\f@size@normalsize{\f@size}
\xdef\f@baselineskip@normalsize{\f@baselineskip}
}
% Define new \smalltonormalsize font size
\newcommand{\smalltonormalsize}{%
  \fontsize
    {\fpeval{(\f@size@small+\f@size@normalsize)/2}}
    {\fpeval{(\f@baselineskip@small+\f@baselineskip@normalsize)/2}}%
  \selectfont
}
\makeatother

\usepackage{amsmath, amssymb, amsthm}
\usepackage{mathtools} % Enhances amsmath
\newtheorem{theorem}{Theorem}
\theoremstyle{remark}
\newtheorem*{remark}{Remark}
\newtheorem{definition}{Definition}
\newtheorem{corollary}{Corollary}
\newtheorem{hypothesis}{Hypothesis}
\DeclareMathOperator*{\argmax}{arg\,max}
\DeclareMathOperator*{\argmin}{arg\,min}

%Header
\pagestyle{fancy}
\thispagestyle{empty}
\rhead{ \textit{ }} 

% Update your Headers here
\fancyhead[LO]{The Imitation Game for Educational AI}
% \fancyhead[RE]{Firstauthor and Secondauthor} % Firstauthor et al. if more than 2 - must use \documentclass[twoside]{article}
  
%% Title
\title{The Imitation Game for Educational AI}

\author{
  Shashank Sonkar\thanks{Equal contribution.}, \, Naiming Liu\footnotemark[1], \, Xinghe Chen, \, Richard G. Baraniuk \\
  Rice University \\
  Houston, TX \\
  \texttt{shashank.sonkar@rice.edu} \\
}

\newcommand{\richb}[1]{\textcolor{blue}{\bf [richb: #1]}}
\newcommand{\sonkar}[1]{\textcolor{orange}{\bf [ss: #1]}}

\begin{document}
\maketitle

\begin{abstract}
\begin{abstract}  
Test time scaling is currently one of the most active research areas that shows promise after training time scaling has reached its limits.
Deep-thinking (DT) models are a class of recurrent models that can perform easy-to-hard generalization by assigning more compute to harder test samples.
However, due to their inability to determine the complexity of a test sample, DT models have to use a large amount of computation for both easy and hard test samples.
Excessive test time computation is wasteful and can cause the ``overthinking'' problem where more test time computation leads to worse results.
In this paper, we introduce a test time training method for determining the optimal amount of computation needed for each sample during test time.
We also propose Conv-LiGRU, a novel recurrent architecture for efficient and robust visual reasoning. 
Extensive experiments demonstrate that Conv-LiGRU is more stable than DT, effectively mitigates the ``overthinking'' phenomenon, and achieves superior accuracy.
\end{abstract}  
\end{abstract}

\section{Introduction}


\begin{figure}[t]
\centering
\includegraphics[width=0.6\columnwidth]{figures/evaluation_desiderata_V5.pdf}
\vspace{-0.5cm}
\caption{\systemName is a platform for conducting realistic evaluations of code LLMs, collecting human preferences of coding models with real users, real tasks, and in realistic environments, aimed at addressing the limitations of existing evaluations.
}
\label{fig:motivation}
\end{figure}

\begin{figure*}[t]
\centering
\includegraphics[width=\textwidth]{figures/system_design_v2.png}
\caption{We introduce \systemName, a VSCode extension to collect human preferences of code directly in a developer's IDE. \systemName enables developers to use code completions from various models. The system comprises a) the interface in the user's IDE which presents paired completions to users (left), b) a sampling strategy that picks model pairs to reduce latency (right, top), and c) a prompting scheme that allows diverse LLMs to perform code completions with high fidelity.
Users can select between the top completion (green box) using \texttt{tab} or the bottom completion (blue box) using \texttt{shift+tab}.}
\label{fig:overview}
\end{figure*}

As model capabilities improve, large language models (LLMs) are increasingly integrated into user environments and workflows.
For example, software developers code with AI in integrated developer environments (IDEs)~\citep{peng2023impact}, doctors rely on notes generated through ambient listening~\citep{oberst2024science}, and lawyers consider case evidence identified by electronic discovery systems~\citep{yang2024beyond}.
Increasing deployment of models in productivity tools demands evaluation that more closely reflects real-world circumstances~\citep{hutchinson2022evaluation, saxon2024benchmarks, kapoor2024ai}.
While newer benchmarks and live platforms incorporate human feedback to capture real-world usage, they almost exclusively focus on evaluating LLMs in chat conversations~\citep{zheng2023judging,dubois2023alpacafarm,chiang2024chatbot, kirk2024the}.
Model evaluation must move beyond chat-based interactions and into specialized user environments.



 

In this work, we focus on evaluating LLM-based coding assistants. 
Despite the popularity of these tools---millions of developers use Github Copilot~\citep{Copilot}---existing
evaluations of the coding capabilities of new models exhibit multiple limitations (Figure~\ref{fig:motivation}, bottom).
Traditional ML benchmarks evaluate LLM capabilities by measuring how well a model can complete static, interview-style coding tasks~\citep{chen2021evaluating,austin2021program,jain2024livecodebench, white2024livebench} and lack \emph{real users}. 
User studies recruit real users to evaluate the effectiveness of LLMs as coding assistants, but are often limited to simple programming tasks as opposed to \emph{real tasks}~\citep{vaithilingam2022expectation,ross2023programmer, mozannar2024realhumaneval}.
Recent efforts to collect human feedback such as Chatbot Arena~\citep{chiang2024chatbot} are still removed from a \emph{realistic environment}, resulting in users and data that deviate from typical software development processes.
We introduce \systemName to address these limitations (Figure~\ref{fig:motivation}, top), and we describe our three main contributions below.


\textbf{We deploy \systemName in-the-wild to collect human preferences on code.} 
\systemName is a Visual Studio Code extension, collecting preferences directly in a developer's IDE within their actual workflow (Figure~\ref{fig:overview}).
\systemName provides developers with code completions, akin to the type of support provided by Github Copilot~\citep{Copilot}. 
Over the past 3 months, \systemName has served over~\completions suggestions from 10 state-of-the-art LLMs, 
gathering \sampleCount~votes from \userCount~users.
To collect user preferences,
\systemName presents a novel interface that shows users paired code completions from two different LLMs, which are determined based on a sampling strategy that aims to 
mitigate latency while preserving coverage across model comparisons.
Additionally, we devise a prompting scheme that allows a diverse set of models to perform code completions with high fidelity.
See Section~\ref{sec:system} and Section~\ref{sec:deployment} for details about system design and deployment respectively.



\textbf{We construct a leaderboard of user preferences and find notable differences from existing static benchmarks and human preference leaderboards.}
In general, we observe that smaller models seem to overperform in static benchmarks compared to our leaderboard, while performance among larger models is mixed (Section~\ref{sec:leaderboard_calculation}).
We attribute these differences to the fact that \systemName is exposed to users and tasks that differ drastically from code evaluations in the past. 
Our data spans 103 programming languages and 24 natural languages as well as a variety of real-world applications and code structures, while static benchmarks tend to focus on a specific programming and natural language and task (e.g. coding competition problems).
Additionally, while all of \systemName interactions contain code contexts and the majority involve infilling tasks, a much smaller fraction of Chatbot Arena's coding tasks contain code context, with infilling tasks appearing even more rarely. 
We analyze our data in depth in Section~\ref{subsec:comparison}.



\textbf{We derive new insights into user preferences of code by analyzing \systemName's diverse and distinct data distribution.}
We compare user preferences across different stratifications of input data (e.g., common versus rare languages) and observe which affect observed preferences most (Section~\ref{sec:analysis}).
For example, while user preferences stay relatively consistent across various programming languages, they differ drastically between different task categories (e.g. frontend/backend versus algorithm design).
We also observe variations in user preference due to different features related to code structure 
(e.g., context length and completion patterns).
We open-source \systemName and release a curated subset of code contexts.
Altogether, our results highlight the necessity of model evaluation in realistic and domain-specific settings.






\documentclass{article}
\usepackage{graphicx} % Required for inserting images

\title{XFlow}
\author{tailin Wu}
\date{January 2025}

\begin{document}

\maketitle

\section{Introduction}

\end{document}


% \begin{table}[t]
% \small
% \caption{\textbf{Key Statistics of \dataset}}
% \centering
% % \begin{center}
% \begin{tabular}{lc}
% \toprule
% \textbf{Statistic} & \textbf{Number} \\
% \midrule
% Total questions & 1130 \\
% \quad - Reasoning split & 837 (74.1\%) \\
% \quad \quad Multiple-choice questions & 431 \\
% \quad \quad Free-form questions & 406 \\
% \quad - Perception split & 293 (25.9\%) \\
% \quad \quad Multiple-choice questions & 275 \\
% \quad \quad Free-form questions & 18 \\
% \midrule
% Total essential step annotation & 3865 \\
% \quad - Total inference conclusion & 2667 \\
% \quad - Average inference conclusion & 3.2 \\
% \quad - Total image observation & 1198 \\
% \quad - Average image observation & 1.4 \\
% Reference image caption item & 1579 \\
% Average reference caption & 1.9 \\
% \midrule
% Number of unique images & 2380 \\
% Number of unique questions & 808 \\
% Number of unique answers & 271 \\
% \midrule
% Maximum question length & 477 \\
% Maximum answer length & 15 \\
% Average question length & 41.2 \\
% Average answer length & 1.2 \\
% \bottomrule
% \end{tabular}
% % \end{center}
% % \vspace{-0.6cm}
% \label{table:statistics}
% \end{table}

% \begin{table}[t]
% \caption{Classification accuracies for naive Bayes and flexible
% Bayes on various data sets.}
% \label{sample-table}
% \vskip 0.15in
% \begin{small}
% \begin{sc}
% \begin{tabular}{lcccr}
% \toprule
% Data set & Naive & Flexible & Better? \\
% \midrule
% Breast    & 95.9$\pm$ 0.2& 96.7$\pm$ 0.2& $\surd$ \\
% Cleveland & 83.3$\pm$ 0.6& 80.0$\pm$ 0.6& $\times$\\
% Glass2    & 61.9$\pm$ 1.4& 83.8$\pm$ 0.7& $\surd$ \\
% Credit    & 74.8$\pm$ 0.5& 78.3$\pm$ 0.6&         \\
% Horse     & 73.3$\pm$ 0.9& 69.7$\pm$ 1.0& $\times$\\
% Meta      & 67.1$\pm$ 0.6& 76.5$\pm$ 0.5& $\surd$ \\
% Pima      & 75.1$\pm$ 0.6& 73.9$\pm$ 0.5&         \\
% Vehicle   & 44.9$\pm$ 0.6& 61.5$\pm$ 0.4& $\surd$ \\
% \bottomrule
% \end{tabular}
% \end{sc}
% \end{small}
% \vskip -0.1in
% \end{table}

\section{The Insufficiency of Unconditioned Single-Phase Testing}

A natural starting point for evaluating an AI system's understanding of student cognition would be to test its ability to generate plausible wrong answers that students might choose. The intuition is compelling: if an AI can anticipate how students will err, surely it understands how they think. This leads to a simple unconditioned single-phase test where both AI and human experts generate distractors for multiple choice questions without knowledge of individual student reasoning patterns. Let us formalize this design:

\begin{definition}[Unconditioned Single-Phase Test Design]
For any question $q$, let:
\begin{itemize}
   \item $C(q)$ be the correct answer
   \item $A(q)$ be the AI-generated distractor 
   \item $H(q)$ be the human expert-generated distractor
   \item $R(q)$ be a randomly generated incorrect answer
\end{itemize}

The test presents these options in random order to students. Let $S(q,s)$ represent student $s$'s selection for question $q$. Critically, both $A(q)$ and $H(q)$ are generated without observing any prior responses from student $s$.
\end{definition}

\begin{definition}[Performance Metrics]
The quality of AI-generated distractors would be evaluated by comparing selection rates:
\begin{align*}
p_A &= \mathbb{P}(S(q,s) = A(q)) \\
p_H &= \mathbb{P}(S(q,s) = H(q))
\end{align*}
with the AI system "passing" if $|p_A - p_H| \leq \epsilon$ for some small $\epsilon > 0$.
\end{definition}

However, this unconditioned approach has fundamental flaws that make it inadequate for validating true understanding of student cognition:

\begin{theorem}[Unconditioned Convergence]
Given a question $q$ with misconception set $M(q) = \{m_1,...,m_k\}$ where $P(m_i)$ represents the probability of misconception $m_i$ in the student population, both AI and human experts will rationally converge to targeting $m^* = \argmax_{m_i \in M(q)} P(m_i)$.
\end{theorem}

\begin{proof}
For any distractor generator (AI or human), the expected selection rate is maximized by targeting the most common misconception:
\[\mathbb{E}[p] = \sum_{i=1}^k P(m_i)\mathbb{I}(m_i \text{ targeted}) \leq \max_{i} P(m_i)\]
with equality achieved only when targeting $m^*$.
\end{proof}

This convergence creates two critical problems:

\begin{enumerate}
   \item \textbf{Population-Level Optimization}: Both AI and human experts are incentivized to target the single most common misconception for each question, regardless of individual student reasoning patterns. This reduces the test to measuring statistical pattern matching rather than understanding of student cognition.
   
   \item \textbf{False Equivalence}: An AI system could ``pass'' this test by simply learning population-level statistics about common wrong answers, without any actual understanding of how individual students reason about concepts or how their misconceptions evolve across related problems.
\end{enumerate}

These flaws demonstrate why a conditioned two-phase design is necessary - one that can validate an AI system's ability to model individual student reasoning rather than just aggregate statistics. To overcome the limitations of unconditioned testing, our two-phase design introduces crucial conditioning on individual student reasoning. In Phase 1, each student $s$ provides open-ended responses to questions $Q_s$, generating tuples $(s,q,A(s,q))$ where $A(s,q)$ is their incorrect answer. In Phase 2, given each Phase 1 tuple $(s,q,A(s,q))$, both the AI system and human experts observe this specific student mistake and use it to inform their predictions. The AI generates a new question $q'$ and predicts the wrong answer $A'(s,q')$ that this particular student would give, while human experts predict their own expected wrong answer $H'(s,q')$. Both these predictions are explicitly conditioned on the observed student mistake from Phase 1.


This conditioning provides several key advantages:

\begin{theorem}[Prediction Differentiation]
Given students $s_1, s_2$ with different misconceptions $m_1, m_2$ observed in Phase 1, optimal predictions for Phase 2 will differ:
\[A'(s_1,q') \neq A'(s_2,q')\]
even for the same question $q'$.
\end{theorem}

\begin{proof}
Unlike the unconditioned case where predictions converge to the population mode, conditioned predictions should maximize:
\[\mathbb{P}(A'(s,q') \text{ chosen} \mid A(s,q))\]
This probability differs based on the specific misconception demonstrated in Phase 1.
\end{proof}

This conditioning forces the AI to demonstrate three crucial capabilities:
\begin{enumerate}
    \item Understanding individual student reasoning patterns from Phase 1 responses
    \item Generating new questions that test the same conceptual understanding
    \item Predicting how specific misconceptions will manifest in new contexts
\end{enumerate}

Unlike the unconditioned design, success in this framework cannot be achieved through simple population-level statistics, as it requires modeling the cognitive processes of individual students. This deeper understanding of student reasoning pathways has direct implications for educational capabilities: an AI system that can accurately predict how a student's specific misconceptions will manifest across different problems is necessarily equipped to provide targeted tutoring interventions and personalized feedback. The ability to generate new questions that probe and predict individual misconceptions demonstrates the AI system's capacity to construct adaptive learning sequences and deliver feedback that addresses each student's particular cognitive challenges.

\section{Conclusion}
In this work, we propose a simple yet effective approach, called SMILE, for graph few-shot learning with fewer tasks. Specifically, we introduce a novel dual-level mixup strategy, including within-task and across-task mixup, for enriching the diversity of nodes within each task and the diversity of tasks. Also, we incorporate the degree-based prior information to learn expressive node embeddings. Theoretically, we prove that SMILE effectively enhances the model's generalization performance. Empirically, we conduct extensive experiments on multiple benchmarks and the results suggest that SMILE significantly outperforms other baselines, including both in-domain and cross-domain few-shot settings.

\section{Limitation}
The use of 3D-printed PLA for structural components improves improving ease of assembly and reduces weight and cost, yet it causes deformation under heavy load, which can diminish end-effector precision. Using metal, such as aluminum, would remedy this problem. Additionally, \robot relies on integrated joint relative encoders, requiring manual initialization in a fixed joint configuration each time the system is powered on. Using absolute joint encoders could significantly improve accuracy and ease of use, although it would increase the overall cost. 

%Reliance on commercially available actuators simplifies integration but imposes constraints on control frequency and customization, further limiting the potential for tailored performance improvements.

% The 6 DoF configuration provides sufficient mobility for most tasks; however, certain bimanual operations could benefit from an additional degree of freedom to handle complex joint constraints more effectively. Furthermore, the limited torque density of commercially available proprioceptive actuators restricts the payload and torque output, making the system less suitability for handling heavier loads or high-torque applications. 

The 6 DoF configuration of the arm provides sufficient mobility for single-arm manipulation tasks, yet it shows a limitation in certain bimanual manipulation problems. Specifically, when \robot holds onto a rigid object with both hands, each arm loses 1 DoF because the hands are fixed to the object during grasping. This leads to an underactuated kinematic chain which has a limited mobility in 3D space. We can achieve more mobility by letting the object slip inside the grippers, yet this renders the grasp less robust and simulation difficult. Therefore, we anticipate that designing a lightweight 3 DoF wrist in place of the current 2 DoF wrist allows a more diverse repertoire of manipulation in bimanual tasks.

Finally, the limited torque density of commercially available proprioceptive actuators restricts the performance. Currently, all of our actuators feature a 1:10 gear ratio, so \robot can handle up to 2.5 kg of payload. To handle a heavier object and manipulate it with higher torque, we expect the actuator to have 1:20$\sim$30 gear ratio, but it is difficult to find an off-the-shelf product that meets our requirements. Customizing the actuator to increase the torque density while minimizing the weight will enable \robot to move faster and handle more diverse objects.

%These constraints highlight opportunities for improvement in future iterations, including alternative materials for enhanced rigidity, custom actuator designs for higher control precision and torque density, the adoption of absolute joint encoders, and optimized configurations to balance dexterity and weight.



\section*{Acknowledgments}
This work was supported by NSF grant 1842378, ONR grant N0014-20-1-2534, AFOSR grant FA9550-22-1-0060, a Vannevar Bush Faculty Fellowship, OpenAI, and ONR grant N00014-18-1-2047.

%Bibliography
\bibliographystyle{unsrt}  
\bibliography{custom,dsp}  


\end{document}
