\begin{figure}[t!]
    \centering
    \includegraphics[width=\textwidth]{figures/figure_3.pdf}
    \caption{
    % An example procedure of Imitation Game. In phase 1, a question will present to a group of students, then incorrect answers will be collected and feed into next phase. In phase 2, AI will generate new test questions along with distractor that follows the same misconceptions from the incorrect answers in the last stage. And finally, the new questions, AI generated distractor, along with teacher predicted distractor will be presented as MCQ to the same batch of students to evaluate the effectiveness of the distractors.
    A novel Turing-like test for evaluating AI's understanding of student cognition through distractor generation conditioned on individual students' demonstrated misconceptions. In Phase 1 (Response Collection), students solve a mathematical expression Q: 1 + 2 * 3 + 4 = ?, revealing common misconceptions such as solving from left to right (yielding 13) or applying addition before multiplication (yielding 21). In Phase 2 (Conditional Distractor Generation), for a new question Q': 3 + 4 * 3 + 7 = ?, both AI and human experts generate distractors based on observed student misconceptions. The resulting multiple-choice question presents four options: the correct answer, a human expert-generated distractor, an AI-generated distractor, and a random distractor. By analyzing whether students select AI-generated distractors at rates similar to human expert-generated ones, we can rapidly evaluate if the AI system truly understands student cognition. This ability to model and predict student misconceptions is fundamental to providing effective feedback and tutoring, as an AI system that can accurately anticipate how students will misunderstand new concepts can provide targeted, personalized educational support. Note distractor generation must be conditioned on students' prior responses, as unconditioned approaches converge to targeting only the most frequent misconception, failing to test AI's understanding of individual student reasoning. While traditional learning outcome studies require extended periods to validate educational AI systems, this test provides rapid evaluation through conditional distractor generation.}
    % \vspace{-10pt}
    \label{fig:Example}
\end{figure}