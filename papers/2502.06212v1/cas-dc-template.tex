%% 
%% Copyright 2019-2020 Elsevier Ltd
%% 
%% This file is part of the 'CAS Bundle.'
%% --------------------------------------
%% 
%% It may be distributed under the conditions of the LaTeX Project Public
%% License, either version 1.2 of this license or (at your option) any
%% later version. The latest version of this license is in
%%    http://www.latex-project.org/lppl.txt
%% and version 1.2 or later is part of all distributions of LaTeX
%% version 1999/12/01 or later.
%% 
%% The list of all files belonging to the 'CAS Bundle' is
%% given in the file `manifest.txt'.
%% 
%% Template article for cas-dc documentclass for 
%% double column output.

%\documentclass[a4paper,fleqn,longmktitle]{cas-dc}
\documentclass[a4paper,fleqn]{cas-dc}

\usepackage[numbers]{natbib}

%%%Author definitions
\def\tsc#1{\csdef{#1}{\textsc{\lowercase{#1}}\xspace}}
\tsc{WGM}
\tsc{QE}
\tsc{EP}
\tsc{PMS}
\tsc{BEC}
\tsc{DE}
%%%

% Uncomment and use as if needed
%\newtheorem{theorem}{Theorem}
%\newtheorem{lemma}[theorem]{Lemma}
%\newdefinition{rmk}{Remark}
%\newproof{pf}{Proof}
%\newproof{pot}{Proof of Theorem \ref{thm}}
\usepackage{subcaption}
\usepackage{enumitem}

\begin{document}
\let\WriteBookmarks\relax
\def\floatpagepagefraction{1}
\def\textpagefraction{.001}

% Short title
\shorttitle{AVSim - Realistic Simulation Framework for Airborne and Vector-Borne Disease Dynamics}

% Short author
\shortauthors{ et~al.}

% Main title of the paper
\title [mode = title]{AVSim - Realistic Simulation Framework for Airborne and Vector-Borne Disease Dynamics}                      

\author[1]{Pandula Thennakoon}
\credit{Writing – original draft, Data curation, Formal analysis, Methodology, Conceptualization}
 
\author[1]{Mario De Silva}
\credit{Writing – original draft, Data curation, Formal analysis, Methodology, Conceptualization}

\author[1, 2]{M. Mahesha Viduranga}
\credit{Writing – original draft, Data curation, Formal analysis, Methodology, Conceptualization}

\author[2, 3]{Sashini Liyanage}
\cormark[1]
\ead {sashinil@eng.pdn.ac.lk}
\credit{Writing – original draft, Data curation, Formal analysis, Methodology, Conceptualization}

\author[1, 2]{Roshan Godaliyadda}
\credit{Conceptualization, Methodology, Supervision}

\author[1, 2]{Mervyn Parakrama Ekanayake}
\credit{Conceptualization, Methodology, Supervision}

\author[1, 2]{Vijitha Herath}
\credit{Conceptualization, Methodology, Supervision}

\author[4]{Anuruddhika Rathnayake}
\credit{Conceptualization, Supervision}

\author[5]{Ganga Thilakarathne}
\credit{Conceptualization, Supervision}

\author[1, 2]{Janaka Ekanayake}
\credit{Conceptualization, Supervision}

\author[4]{Samath Dharmarathne}
\credit{Conceptualization, Supervision}

\affiliation[1]{organization={Department of Electrical and Electronic Engineering, University of Peradeniya},
                country={Sri Lanka}}

\affiliation[2]{organization={Multidisciplinary AI Research Center, University of Peradeniya}, country={Sri Lanka}}

\affiliation[3]{organization={Department of Computer Engineering, University of Peradeniya},
                country={Sri Lanka}}
\affiliation[4]{organization={Faculty of Medicine, University of Peradeniya, Peradeniya},
                country={Sri Lanka}}
\affiliation[5]{organization={Institute of Policy Studies, Colombo}, country={Sri Lanka}}

% Corresponding author text
\cortext[1]{Corresponding author}

%(such as mosquitoes)
\begin{abstract}
% The COVID-19 pandemic has highlighted the urgent need to identify epidemic trends and explore effective intervention strategies. Meanwhile, diseases like dengue continue to act as silent killers every year. In this context, we present AVSim (Airborne and Vectorborne Simulator), an agent-based model designed to provide micro-level insights and develop intervention strategies for managing epidemics. AVSim leverages real human mobility and behavioral patterns to simulate disease spread. It features a scalable and realistic environmental structure, an optimized representation of vectors, a public transportation model, and a detailed human motion model. The simulated environment includes diverse locations such as homes, schools, hospitals, and supermarkets. Human motion is modeled based on occupational and behavioral patterns, such as weekday and weekend behaviors of school students. The simulator incorporates age-specific and environment-specific disease outcomes, host-to-host and host-to-vector interactions, and multiple disease stages, including mild, severe, and critical phases. It also accounts for immunity, quarantine, and hospitalization. Furthermore, AVSim supports contact tracing, offering micro-level insights into disease transmission dynamics. Implemented in Python, AVSim is designed for ease of use, flexibility, and extensibility. Users can create realistic, highly customized scenarios to model both airborne and vectorborne diseases. As a case study, AVSIM has been used to demonstrate its capabilities in modeling COVID-19 and dengue, showing its value in providing actionable epidemic insights. AVSim is a robust framework for understanding and managing the spread of infectious diseases, playing a crucial role in public health planning and response.
The COVID-19 pandemic underscored the critical need for rapid epidemic trend identification and effective intervention strategies to mitigate disease progression and its socio-economic impact.  Concurrent with emerging threats, endemic diseases like dengue continue to strain healthcare systems, particularly in populous, economically challenged nations.  This paper introduces AVSim (Airborne and Vectorborne Simulator), an agent-based model designed to provide granular insights for optimizing resource allocation within existing healthcare management frameworks.  AVSim leverages realistic human mobility and behavioral patterns to simulate disease propagation within a detailed, scalable environment encompassing homes, schools, hospitals, and commercial venues.  Human movement is modeled based on occupational and behavioral patterns, including age-specific activities.  The simulator incorporates age- and environment-specific disease outcomes, host-host and host-vector interactions, and multiple disease stages, including mild, severe, and critical phases.  Immunity, quarantine, and hospitalization are also modeled.  Furthermore, AVSim supports tracing the path of disease spread, providing micro-level insights into transmission dynamics.  Implemented in Python, AVSim offers flexibility and extensibility, enabling users to create highly customized scenarios for airborne and vector-borne disease modeling.  Case studies demonstrating AVSim's application to COVID-19 and dengue illustrate its potential for generating actionable epidemic insights, thereby enhancing public health planning and response.
\end{abstract}

% Research highlights
\begin{highlights}
\item Human mobility modeling using machine learning and probabilistic models.
\item Unveiling hidden behavioral patterns via unsupervised clustering and graph theory.
\item Agent-based model with realistic scalable environments and transportation systems.
\item Simulates airborne and vector-borne disease spread dynamics.
\item Provides micro-level insights for public health planning and response.

\end{highlights}

\begin{keywords}
Agent-based model (ABM)\sep airborne diseases \sep vector-borne diseases \sep epidemiology \sep public health intervention
\end{keywords}

\maketitle

\section{Introduction}

The COVID-19 pandemic lasted over three years, resulting in more than 700 million cases and claiming over 7 million lives \cite{whoCOVID19Deaths}. This crisis highlighted the critical importance of timely interventions and coordinated efforts by individuals, healthcare systems, and governments to control disease outbreaks and prevent them from escalating into pandemics. Beyond COVID-19, vector-borne diseases, transmitted by living organisms that carry infectious pathogens between humans or from animals to humans, pose a significant public health challenge, causing over 700,000 deaths annually. For instance, In 2024, the World Health Organization (WHO) estimated approximately 249 million global cases of malaria and 100–400 million cases of dengue each year \cite{whoVectorborneDiseases}. Many vector-borne and airborne diseases, including those similar to COVID-19, are preventable through protective measures and effective community mobilization. These preventable diseases underscore the need for continued vigilance, innovative interventions, and global collaboration to mitigate the risk of future outbreaks and safeguard public health.

Numerous methods have been employed to model disease spread and uncover patterns to better understand disease propagation so that effective disease control strategies could be adopted efficiently while minimizing the net burden. These models can be broadly classified into two categories: compartment models and agent-based models (ABMs). Compartment models are relatively simple and computationally efficient. Within the compartmental modeling framework, the Susceptible-Exposed-Infectious-Recovered (SEIR) model has been widely used to study epidemics and pandemics \cite{gallagher2017comparing}. These models classify individuals by their disease status and use differential equations to describe transitions between these states. Although they can give an overview of the disease's spread, they fail to capture many important details. Despite this, numerous attempts exist to model disease spread with such compartment models. Shaobo He et al.~\cite{compartment1} used an SEIR compartment model based on the data from Hubei province (China) to show the evolution of the COVID-19 epidemic. Jonas Dehning et al.~\cite{compartment2} focused on COVID-19 spread in Germany and used a SIR model to quantify the effect of interventions. Giulia Giordano et al.~\cite{compartment3} proposed a new extended compartment model, SIDARTHE (Susceptible-Infected-Diagnosed-Ailing-Recognized-Threatened-Healed-Extinct), that predicts the course of the epidemic to help plan an effective control strategy.

Agent-based models, on the other hand, are more complex and able to deliver micro-level details about each agent in the simulation. They have the capability to model interactions between agents or entities in a defined environment. Each agent follows a set of well-defined rules, allowing ABMs to capture emergent behaviors that arise from these interactions. This makes them particularly useful for modeling complex systems in fields like epidemiology~\cite{COVASIM, OPENABM}, urban planning \cite{Bastarianto2023}, ecology \cite{axtell2022agent}, and molecular-level biological \cite{MAIA2019218, BORAU2024108831} systems. The ability to incorporate fine-grained details enhances decision-making, especially for policymakers seeking accurate predictions. 

The use of ABMs in epidemiology has grown significantly during the COVID-19 pandemic. Models like Covasim \cite{COVASIM}, PDSIM \cite{Weligampola2023}, and OpenABM \cite{OPENABM} have been widely used to analyze disease dynamics and inform public health responses. Some benchmark ABMs are mentioned and compared with our proposed ABM in Table~\ref{tab:disease_simulator_comparison}.

Additionally, there has been increasing research on the dynamic analysis of vector-borne diseases. One notable approach by Carrie Manore et al.~\cite{patch-2014} employs an ABM to analyze the propagation of mosquito-borne diseases. This study introduces the network-patch method, utilizing differential equations to model mosquito density. Similarly, Imran Mahmood et al.~\cite{dengueMohomad} use an ABM to compute vector density based on the reproductive behavior of vectors. However, their model does not incorporate realistic movement patterns, limiting its ability to provide real-world insights. Another study applied a fuzzy modeling approach to develop an epidemic model for mosquito-borne diseases \cite{dengueDAYAN2022105673}.

A comprehensive modeling and analysis tool, such as ABM or otherwise, helps to be better prepared for an epidemic situation. For example, during the initial stages of the COVID pandemic, there were many unknowns and uncertainties resulting in confusion, stressing of healthcare resources, as well as burdening the public excessively due to the control strategies. Many of the initial responses involved in reactive measures such as contact tracing and mass-scale quarantining and lock-downs. A comprehensive computational model would, on the other hand, complement the existing diverse healthcare systems and practices of countries and regions by making it possible to "play out" different scenarios using the best of data and knowledge available and/or learned assumptions on what may happen under different cases. Armed with all these possible outcomes, healthcare resources could be targeted effectively and efficiently so that maximum outcomes could be attained.

Effective intervention and prevention strategies in epidemiology require modeling disease spread under realistic conditions. Human mobility, behavioral patterns, and public modes of transportation significantly influence the speed and extent of pathogen transmission \cite{Weligampola2023}. Different behaviors and transport modes affect disease spread in varied ways. However, many models either overlook these variations \cite{Trivedi2021, Fan2024} or rely on oversimplified assumptions, such as treating transportation as random interactions aggregated with activities like shopping or social events \cite{COVASIM, Hinch2021}, failing to reflect real-world complexities.

To address the limitations of existing methods, we propose an agent-based model for vector-borne and airborne diseases, named AVSim, that uses real-world data and accounts for real-world conditions. Our framework learns and models different human mobility and behavioral patterns, simulating agents based on occupation and behavior at a granular level. It incorporates a realistic environment and public transportation model to simulate disease spread. The proposed model is customizable and extendable, offering a flexible framework adaptable to various diseases and transmission modes, thus enhancing its applicability to emerging public health challenges. A preliminary version of this work was presented in \cite{our1, our2}. In this article, we expand on our previous findings by introducing additional experiments, deeper analysis, and improved methodologies. We further demonstrate the framework's use in simulating COVID-19 and dengue propagation through an extended SEIR model within an ABM.

\begin{table*}[h]
    \centering
    \caption{Comparison of disease simulators found in the literature with ours.}
    \label{tab:disease_simulator_comparison}
    \resizebox{\textwidth}{!}{%
    \begin{tabular}{|l|l|c|c|c|c|c|c|c|c|c|c|c|c|c|}
    \hline
    \textbf{Simulation Method} & \textbf{Simulator} & \rotatebox{90}{\textbf{Realistic Human Motion}} & \rotatebox{90}{\textbf{Use of real mobility data }} & \rotatebox{90}{\textbf{Realistic Environment}} & \rotatebox{90}{\textbf{Transport System}} & \rotatebox{90}{\textbf{Testing Protocol}} & \rotatebox{90}{\textbf{Vaccination Strategy}} & \rotatebox{90}{\textbf{Containment Strategy}} & \rotatebox{90}{\textbf{Hygiene Protocol}} & \rotatebox{90}{\textbf{Tracing Path of Disease Spread} \hspace{0.3em}} & \rotatebox{90}{\textbf{User-Friendly Interface}} & \rotatebox{90}{\textbf{Detailed agent modeling}} & \rotatebox{90}{\textbf{Environmental Factors}} & \rotatebox{90}{\textbf{Simulation}} \\
    \hline
    \multirow{6}{*}{\textbf{Vector-borne Diseases}} & Network-patch methodology for adapting disease models \cite{patch-2014} & \checkmark & & & & & & & & & & \checkmark & \checkmark & \checkmark \\ \cline{2-15}
     & Agent-Based Simulation of Dengue Fever Spread \cite{dengueMohomad} 
 & & & \checkmark & & \checkmark & & & & & \checkmark & \checkmark & & \checkmark \\ \cline{2-15}
     & Climate-based dengue model in Semarang, Indonesia \cite{climate_baseed_dengue_model} & & & \checkmark & & \checkmark & & & & & & & \checkmark & \checkmark \\ \cline{2-15}
     & ABM for vaccine efficacy's influence on public health projections \cite{Perkins082396} & \checkmark & & \checkmark & & \checkmark & \checkmark & & \checkmark & & & \checkmark & & \checkmark \\ \cline{2-15}
     & ABM driven by rainfall to study chikungunya outbreak \cite{DOMMAR201461} & \checkmark & & \checkmark & & \checkmark & & & & \checkmark & \checkmark  & \checkmark & \checkmark & \checkmark \\ \cline{2-15}
     & \textbf{AVSim (Our Approach)} & \checkmark & \checkmark & \checkmark & & \checkmark & & \checkmark & \checkmark & \checkmark & \checkmark & \checkmark & \checkmark & \checkmark \\ \hline
    \multirow{10}{*}{\textbf{Air-borne Diseases}}
    
     & Covasim \cite{COVASIM} & &  & & & \checkmark & \checkmark & \checkmark & \checkmark& \checkmark & \checkmark & & &  \\ \cline{2-15}
    
     & OpenABM-Covid19 \cite{OPENABM} &  &  & & & \checkmark &  & \checkmark &  & \checkmark & & & &  \\ \cline{2-15}

     & PDSIM \cite{Weligampola2023} & \checkmark &  & & \checkmark & \checkmark & \checkmark & \checkmark & \checkmark & \checkmark & \checkmark & \checkmark &  &  \\ \cline{2-15}
     
     & A particle-based COVID-19 simulator \cite{A_particle-based_COVID-19_simulator} & & & & & \checkmark &  &  & & \checkmark & & & &  \\ \cline{2-15}
     
     & Simulator of interventions for COVID-19 \cite{Simulator-of-interventions-for-COVID-19} & & & & & \checkmark &  & \checkmark & \checkmark & \checkmark &  & & &  \\ \cline{2-15}
     
     & People meet people \cite{People-meet-people} & & & & & \checkmark & & \checkmark &  & \checkmark & & & & \\ \cline{2-15}
     
     & COVID-town \cite{covidtown-integratedeconomicepidemiologicalagentbased} & & & & & \checkmark & \checkmark & \checkmark & \checkmark & \checkmark & &  & &  \\ \cline{2-15}
     
     & An Agent-based modeling of COVID-19 \cite{An-Agent-based-modeling-of-COVID-19} & & & & & & & \checkmark &  & \checkmark & & & & \\ \cline{2-15}
     
     & Social bubble vanpooling (SBV) \cite{social-b-v} & & &  & \checkmark & \checkmark & & \checkmark & & \checkmark & & & & \\ \cline{2-15}
     
     & \textbf{AVSim (Our Approach)} & \checkmark & \checkmark & \checkmark & \checkmark & \checkmark & \checkmark & \checkmark & \checkmark & \checkmark & \checkmark & \checkmark & & \checkmark \\ \hline
    \end{tabular}
    }
\end{table*}


\section{Material and methods}

The proposed AVSim consisted of two major stages. The first stage was the data generation stage, where human mobility and behavioral patterns were identified and generated using real-world data. This sets the foundation for the second stage of AVSim: the simulation of the ABM. As shown in Fig.~\ref{fig:methodology}, the ABM simulation stage comprised four main components: the environment, vector patch, transportation, and agent generation. However, for simulating vector-borne diseases, only the environment, vector patch, and agent generation components were used, as transportation had no significant impact on disease spread. Conversely, for airborne diseases, only the environment, transportation, and agent generation components were utilized. Each of these stages will be explained in detail in the following sections.

\begin{figure*}[h]
    \centering
    \includegraphics[width=1\linewidth]{Figures/fig1.png}
    \caption{Overview of the proposed agent-based model structure, illustrating the key components: environment, vector-patch, transportation, agent simulation, and the two disease transmission models: vector-borne and airborne.}
    \label{fig:methodology}
\end{figure*}
\subsection{Modeling human behaviour}


To accurately model disease spread in a human population, it was essential to replicate realistic human motion patterns within the simulation environment. In the context of human motion, various factors affected: 
% Human motion is not random. It is shaped by various factors:
\begin{itemize}
  \item Time-dependent location visits: Individuals visit specific locations based on the time of day. 
  \item Duration of stay: The length of time spent at a location varies depending on the time of day and the nature of the location. 
  \item Profession-dependent behavior: The types of locations visited and the associated durations differ based on individuals' professions. 
\end{itemize}

To address these factors, In the first stage of AVSim, we proposed a method for generating realistic human motion patterns using real-world GPS data. This approach enabled us to delve deeper into human mobility and behavioral dynamics, uncovering insights that were not immediately apparent. The following subsections outline the data acquisition process and the methods used for pattern recognition.

\subsubsection{Tracking human movement and data acquisition}
\label{subs:data_collection}
The first step in modeling human motion was gathering a large and diverse dataset of real-world movement patterns. To achieve this, GPS tracking of human agents was conducted using a smartphone application. The dataset was sourced from 100 voluntary participants residing in the Kandy district of Sri Lanka, each representing one of 13 distinct professions. The participants' GPS coordinates, altitude, and timestamps were recorded at 5-minute intervals over 21 days. This duration was chosen as the optimal time frame for capturing representative mobility patterns while remaining manageable for participants. A longer duration could have led to participant fatigue or reduced engagement, affecting data quality. To enhance diversity and fairness, participants within a single profession were carefully selected to reflect varying factors such as age, gender, and the relative distance between their workplace and residence.

% Fig. ~\ref{raw dataset doctor} shows an example of the raw data recorded from one participant, a doctor.

% \begin{figure}[h]
%     \centering
%     \includegraphics[width=0.6\linewidth]{Figures/raw_dataset_dc.png}
%     \caption{Excerpt of the dataset collected from a doctor}
%     Figure Description
%     \label{raw dataset doctor}
% \end{figure}
\begin{figure}[h]
    \centering
    \includegraphics[width=1.0\linewidth]{Figures/fig2.png}
    \caption{An example of accepted and unaccepted patterns based on daily data availability.}
    \label{data availability graph}
\end{figure}

Before enrollment, all participants were well informed about the scope of the data being collected and its intended use. Informed verbal and written consent were obtained from each participant, with students requiring additional consent from a parent or guardian and their educational institution. Upon consent, participants were provided with the smartphone application and instructed to keep their smartphones active and within reach throughout the study. Data privacy and confidentiality were maintained at all stages, and all data was anonymized.

{\bf Ethical approval}: The ethical clearance for this study was obtained from the Ethical Review Committee, Faculty of Arts, University of Peradeniya, Sri Lanka, (ARTS/ERC/ 2021/01 and 18 September 2021) with the support of the Department of Sociology. Administrative clearance was obtained by the Ministry of Home Affairs of Sri Lanka, relevant Divisional Secretariate (DS) offices, and relevant Grama Niladari (GN) offices.

To ensure data quality, daily datasets were reviewed for completeness, and 21 complete daily datasets were required per participant. A day was considered complete if at least 80\% of the expected 288 data points (recorded every 5 minutes) were present and evenly distributed throughout the day, minimizing gaps in coverage. Days with more than 20\% missing data (57 data points) or significant periods of consecutive data loss were excluded. As a result, data collection for a participant typically extended beyond 21 days to ensure 21 days of complete data. Fig. \ref{data availability graph} shows an example of accepted and discarded data based on its availability throughout the day. 
% Explain plots


\subsubsection{Human mobility patterns identification}
\label{method:mobility patterns}

The GPS data was recorded in the time and coordinate domain (longitude and latitude), making it specific to the geographical environment where it was collected, in this case, Kandy, Sri Lanka. However, human motion patterns are generally consistent across different geographical locations. For example, a teacher's typical working day will not depend on the geographical location. They will be at the resting place in the morning, leave for work, maybe visit some places in the evening, and return to the resting place at the end of the day. To make the data applicable beyond the specific geographic context, we transformed it into a time and location domain that removes the dependency on physical coordinates. This transformation generalizes the data, enabling broader applicability and analysis of human motion patterns.

To identify the locations visited by each person, the GPS data for all participants over the 21 days was plotted on a map of Sri Lanka. Fig.~\ref{clusterZoom} illustrates the raw GPS data when mapped geographically. The data reflects the places visited and the movements during transportation, such as commuting between locations. 

% \begin{figure}[h]
%     \centering
%     \includegraphics[width=0.8\linewidth]{Figures/GPS_data_plotted.png}
%     \caption{Raw GPS data plotted on Google Maps.}
%     \label{Raw GPS data}
% \end{figure}

When zooming in on these denser clusters, which occur at locations where the person pauses, additional sub-clusters emerge within them, revealing further granularity. 
\begin{figure}[h]
    \includegraphics[width=1\linewidth]{Figures/fig3.png}
    \caption{Hierarchical clustering of locations as we zoom into GPS data.}
    \label{clusterZoom}
\end{figure}
Fig.~\ref{clusterZoom} demonstrates this inherent pattern in human motion. This pattern highlights a hierarchical structure in the locations visited, with primary locations further divided into more specific destinations. 

For this study, our primary interest lies in the denser clusters of GPS data, which indicate areas where participants spent more time, such as workplaces, homes, and other specific locations. 

To identify these clusters, a clustering method needs to be deployed.
%It should be capable of finding cluster patterns based on the density of GPS data points in different regions. Furthermore, the clustering method should be able to detect and filter any noise points (These points belong to transport). Because of these reasons, %
In this study, we employed the Density-Based Spatial Clustering of Applications with Noise (DBSCAN), which produced effective results. DBSCAN is a popular density-based clustering algorithm known for detecting clusters of varying densities while being robust to noise. References ~\cite{DBSCAN-1} and ~\cite{DBSCAN-2} proved to be useful in implementing the DBSCAN algorithm. Unlike K-Means clustering and Gaussian Mixture Models, DBSCAN does not require the number of clusters to be specified in advance. Instead, it determines the number of clusters based on the data, which makes it well-suited for identifying clusters in GPS data. DBSCAN classifies points into three categories: core points, border points, and noise points. Core and border points form the clusters, while noise points are discarded. As a result, transportation-related data is treated as noise and removed from the analysis. The DBSCAN algorithm requires two initial parameters:
\begin{enumerate}
  \item epsilon (\(\epsilon\)) defines the maximum distance between two points in a cluster.
  \item minPts, the minimum number of points required for a point to be considered a core point of a cluster.
\end{enumerate}

% We have used different \(\epsilon\) and minPts values and selected \(\epsilon\) = 5 and minPts = 10, which yielded optimal results for our clustering purpose. Once the clusters were identified, they were labeled with numbers, as shown in Fig. ~\ref{DBSCAN}.

To determine the most suitable values for these parameters, we experimented with different values of \(\epsilon\) and minPts. Through an iterative process of testing and evaluating clustering performance, we found that setting \(\epsilon\) = 5 and minPts = 10 produced the most meaningful and consistent clusters. These values effectively distinguished visited locations while minimizing noise, making them optimal for our specific application.

Since the primary goal was to convert time-coordinates domain data into the time-location domain, the next step involved identifying the numbered clusters as real-world locations. For this, commonly visited locations such as "Home," "School," and "Bank" were made into a list and grouped into broader categories like "Residential Zone," "Education Zone," and "Commercial and Financial Zone".
% as shown in Fig. ~\ref{zones}. 
Each cluster was then labeled based on these predefined locations and zones. 
\begin{figure}[h]
    \includegraphics[width=1\linewidth]{Figures/fig4.png}
    \caption{Clustering GPS data using DBSCAN algorithm.}
    \label{DBSCAN}
\end{figure}
For example, if a cluster represented an individual's home, it was labeled as "Home." Similarly, clusters corresponding to areas around the individual's neighborhood were labeled as "Residential Zone."
% \begin{figure}[h]
%     \includegraphics[width=1\linewidth]{Figures/zones.png}
%     \caption{Locations categorized as different zones with a hierarchical structure}    
%     \label{zones}
% \end{figure}

During this process, some clusters could not be mapped to identifiable real-world locations. This may have occurred for several reasons, such as GPS data drift, a phenomenon where the reported GPS position shifts over time despite the device being stationary, or network connectivity issues that affected the accuracy of location tracking. Additionally, even if the individual visited the area corresponding to a cluster, it might not match any meaningful location in the predefined list. These clusters were treated as outliers and excluded from the analysis.

% \begin{itemize}
%   \item GPS data drift - this is a phenomenon where the reported GPS position of a device changes over time even though the GPS receiver device is still.
%   \item Network connectivity issues.
%   \item Even though the person visited the location indicated by the cluster, the location cannot be identified as any sensible location in our predefined location list.
% \end{itemize}
% These clusters were considered as outlier clusters and removed from the analysis. Such a cluster is shown in Fig ~\ref{fig:clusters_sub2}.

% \begin{figure}[h]
%     \includegraphics[width=1\linewidth,left]{Figures/outlier_cluster.jpg}
%     \caption{\bf Outlier cluster}
%     figure description
%     \label{outlier cluster}
% \end{figure}

% \begin{figure}[h]

% \begin{subfigure}{0.5\textwidth}
% \includegraphics[width=0.9\linewidth]{Figures/cluster_labeling.png} 
% \caption{Identification of clusters}
% \label{fig:clusters_sub1}
% \end{subfigure}

% \begin{subfigure}{0.5\textwidth}
% \includegraphics[width=0.9\linewidth]{Figures/outlier_cluster.jpg}
% \caption{Identification of outlier clusters}
% \label{fig:clusters_sub2}
% \end{subfigure}

% \caption{(a) Each cluster was manually identified as a real-world location, (b) Outlier clusters were identified and removed from the analysis.}

% \label{cluster_labeling}
% \end{figure}

The original dataset was recorded at a 5-minute resolution with some missing data points. However, the emulator (described in subsequent sections) operates at a 1-minute resolution. To bridge this gap, the data was reconstructed using Zero Order Hold (ZOH) reconstruction. This method was deemed sufficient because of several key factors. First, it was reasonable to assume that a person's location remained approximately constant within the 5-minute intervals between recorded points. Additionally, the dataset had been carefully preprocessed to ensure that missing data was minimal, allowing interpolation to address these gaps with negligible inaccuracies. 

The reconstructed dataset with a 1-minute resolution is visualized in Fig. ~\ref{Traj}. It is important to note that transportation-related data points were excluded during the DBSCAN clustering process and are, therefore, not present in the final dataset. This dataset was used to model human movements in the proposed AVSim.

% \subsubsection{Data Preprocessing (Mario)}
% \subsubsection{Data Cleaning and Preprocessing (Mario)}
% \label{subs:data_preproc}

% As explained above in the Section "Tracking Human Movement and Data Acquisition," the collected data had a 5-minute resolution with some missing points. However, in order to calculate the probability matrices required for the emulator (explained in later sections), the dataset needed to be at a 1-minute resolution with no missing points. Therefore, the collected data was reconstructed using Zero Order Hold (ZOH) reconstruction. It was determined that complex reconstruction techniques were not required due to several key considerations. Primarily, it can be assumed the location (domain converted) of a person is approximately constant within the 5-minute interval between two recorded data points. Moreover, as the collected data was carefully filtered to only contain small portions of missing data, these gaps could be addressed by interpolation with minimal inaccuracies.

\subsubsection{Human behavioural patterns identification}

% To model human behavior, we need to have a solid understanding of the underlying social structure of society. This is something often overlooked in many current ABMs. Since ABMs rely on a bottom-up approach, where complex, system-wide phenomena emerge from the behavior and interactions of individual agents, accurate representation of micro-level details is critical. 
 
%  \begin{itemize}
%   \item modeling diverse behavior patterns to maintain the heterogeneity in the society.
%   \item enables experimentation and hypothesis testing on a much broader scale.
%   \item allows studying emergent phenomena in society (emergent phenomena are the behaviors and properties that rise from the interaction of system parts but not easily explained by the behavior of these parts alone) 
% \end{itemize}

In this study, GPS data was collected from individuals belonging to various professional classes. This data was used to uncover the hidden structure of society based on occupation. Within each professional class, sub-professional class behaviors can be identified, as shown in Fig. ~\ref{fig:methodology}.
\begin{figure}[h]
    \includegraphics[width=1\linewidth]{Figures/fig5.png}
    \caption{Final dataset: A visual representation of the trajectory of a teacher for a regular workday (A) and a part of the actual final dataset (B).}
    \label{Traj}
\end{figure}

% \begin{figure}[h]
%     \includegraphics[width=1\linewidth,center]{Figures/sub classes.png}
%     \caption{Example sub behaviors within student and teacher profession.}
%     \label{social structure}
% \end{figure}

%  An advanced clustering method Spectral Clustering) that uses spectral graph theory was used to uncover these hidden behavior patterns or sub-classes within each occupation class. Spectral clustering is a clustering algorithm that considers the global structure of a dataset and gives the optimal number of clusters within it. For the clustering process, locations have to be encoded into numbers to process. This was done the following way.

%  Remember that the locations are in a hierarchical order where the locations are grouped based on zones. For each zone, a binary value was given as shown in Table ~\ref{table Zone enc}.

% \begin{table}[!ht]

% \caption{
% {\bf Zones and assigned binary values}}
% \label{table Zone enc}

% \begin{tabular}{| m{5cm} | m{2.5cm} |} 
%   \hline
%   Zone & Assigned Binary Number \\ 
%   \hline
%   Residential Zone & 000 \\ 
%   \hline
%   Other & 001 \\ 
%   \hline
%   Transport & 010 \\ 
%   \hline
%    Administrative Zone & 011 \\ 
%   \hline
%    Urban Zone & 100 \\ 
%   \hline
%    Agricultural Zone & 101\\ 
%   \hline
%    Educational Zone & 110\\ 
%   \hline
%    Industrial and Manufacturing Zone & 111 \\ 
%   \hline
% \end{tabular}
% \begin{flushleft} Table notes 
% \end{flushleft}
% % \label{table1}
% \end{table}

% After that, the location was given a separate binary number. Subsequently, both binary numbers were concatenated to create the encoding for the location. See Fig ~\ref{enc}. This was done to make sure there was some separation between locations, which are different from each other. Locations in the same zone will have less separation according to this encoding method. Overall, this approach provided a compact, numerical encoding for each location and yielded acceptable results.

% \begin{figure}[h]
%     \includegraphics[width=0.8\linewidth]{Figures/encoding.png}
%     \caption{\bf Encoding process}
%     figure description
%     \label{enc}
% \end{figure}

The Spectral Clustering method, which utilizes spectral graph theory \cite{chung1997spectral}, was employed to uncover hidden behavior patterns or sub-classes within each occupational class. Spectral clustering is an algorithm that considers the global structure of a dataset and determines the optimal number of clusters within it. 

Although there is a lot of literature on spectral clustering within machine learning, the sources ~\cite{ng2002spectral} and ~\cite{Ranasinghe2022} proved to be of great value when applying spectral clustering to unlock hidden patterns in the dataset. 

For the clustering process, locations needed to be encoded numerically. A 6-digit binary encoding was used, where the first three digits uniquely represent the zones, and the last three digits uniquely identify the locations within those zones, as defined in Section \ref{method:mobility patterns}. This encoding ensured a clear separation between distinct locations, while locations within the same zone had less separation, reflecting their relative proximity or similarity. This compact numerical representation of locations facilitated effective clustering and produced acceptable results, as elaborated in Section \ref{result:spectral clustering}.

Let us consider a day as a data point. Then we can represent our data set $(X)$ as, $X = x_{1}, x_{2}, x_{3}, \dots, x_{n}$ and the notion of similarity between $x_{i}$ and $x_{j}$ as $S_{ij} > 0$.

To define similarity, we have chosen a Gaussian kernel, as shown below. 
\begin{equation}
 S_{i,j} =\begin{cases}{exp(-\frac{\parallel{x_{i}-x_{j}}\parallel^{2}}{2\sigma^2})} & i \neq j\\\mu & i = j\end{cases}   
\end{equation}

Here, sigma is a scaling parameter that determines the closeness of data points to be considered a cluster. The other parameter $\mu$ is a predefined value. In this study, it will be referred to as self-similarity, which will be in the range of [0,1].

The Gaussian kernel was selected because it will assign high similarity to close-by points, allowing them to cluster together and assign zero similarity to points far apart. This means that the number of clusters will depend on the sigma value, which is evident from the results.

\begin{figure}[h]
    \centering
    \includegraphics[width=0.9\linewidth]{Figures/fig6.png}
    \caption{Sigma sweep for bank worker occupation class.}
    \label{sigma_sweep}
\end{figure}
First, a similarity matrix (Adjacency matrix) was formulated as mentioned below.
\begin{equation}
A_{n,m} = 
\begin{pmatrix}
S_{1,1} & S_{1,2} & \cdots & S_{1,n} \\
S_{2,1} & S_{2,2} & \cdots & S_{2,n} \\
\vdots  & \vdots  & \ddots & \vdots  \\
S_{m,1} & S_{m,2} & \cdots & S_{m,n} 
\end{pmatrix}
\end{equation}
\noindent Next, we created the degree matrix as,
\begin{equation}
D_{[i,j]} =
\begin{cases}{\sum\limits_jA_{[i,j]}} & i = j\\\mu & otherwise.
\end{cases}
\end{equation}

\noindent Then, the symmetric normalized Laplacian was obtained,
\begin{equation}
L=I-D^{-1/2}AD^{-1/2}.
\end{equation}

Thereafter, eigen decomposition was performed on the Laplacian matrix. The eigenvalues were then arranged in ascending order. The next procedure involved conducting a sigma sweep by plotting the eigen gaps between the sorted eigenvalues against sigma, as shown in Fig.~\ref{sigma_sweep}. Note that mode $xy$ corresponds to the eigen gap between $x\textsuperscript{th}$ and $y\textsuperscript{th}$ eigenvalues after sorting.


Using the aforementioned sigma sweep plot as a reference, the number of optimal clusters can be determined by applying the selection algorithm outlined below:

\begin{enumerate}
  \item Largest Gap Value: The eigen gap with the highest value was considered as the prominent mode.
  \item Selection of Dominant Sigma: Once the prominent mode is determined, the sigma value corresponding to the highest gap value for this mode is selected as the dominant sigma according to the selection algorithm \cite{Ranasinghe2022}.
  \item Clustering Procedure: With the prominent mode and dominant sigma identified, clustering can be executed by following the steps described in \cite{Ranasinghe2022}.
\end{enumerate}

When considering the sigma sweep for the bank worker class, as shown in Fig.~\ref{sigma_sweep}, the prominent mode could be determined as 23, which would mean the behavior would be broken down into two significant clusters at the corresponding sigma value. Then the matrix, 
\begin{equation}
X=[v_{1}, v_{2}, v_{3}, ... ,v_{n}] \in 	\mathbb{R}^{n\times{n}},
\end{equation}
can be formulated by stacking eigenvectors in columns. By treating each row of $X$ as a point, the resulting points can be clustered into the desired number of groups using K-means clustering.

The above method was proven to be very successful in identifying hidden motion patterns within each profession class. These results were used to model human behavior, which will be discussed in the next section.

\subsubsection{Bimodality probability matrix generation}
\label{method:bimodality}
To model human motion, we employed a robust probabilistic approach capable of capturing complex movement dynamics, as introduced by Weligampola et al.~\cite{Weligampola2023}. This method enabled the generation of daily trajectories for agents within a predefined environment, taking into account the type of day (weekday or weekend) and the agent's profession.

The inspiration for adopting this method stemmed from observations of GPS trajectories. As previously observed, denser points in GPS data were the locations where people tended to stay for a longer time. It was further observed that human motion consists of pseudo-random visits to locations and a stay time associated with each location. These observations form the basis of the theory proposed by Weligampola et al.~\cite{Weligampola2023}.

To model human movement, the theory introduces two probability matrices with size $n\times m$, where $n$ represents the number of zones or locations and $m$ represents the total number of minutes in a day:
\begin{enumerate}
  \item Location visit probability matrix $(\mathcal{L})$ - Represents the probability of being at a specific location at any given minute of the day. 
  \item Location occupancy probability matrix $(\mathcal{S})$ - Represents the probability of remaining at a location for a specified duration. 
\end{enumerate}

The $\mathcal{L}$ matrix is influenced by the time of day, while the $\mathcal{S}$ matrix depends on the location being visited. Fig. \ref{tt_process} illustrates how these matrices are utilized to model human motion. At any given moment, an individual is assigned probabilities for being at various locations. The higher the probability, the greater the likelihood of being at that location. The actual location a person visits is determined using the roulette wheel selection method applied to the matrix $\mathcal{L}$. Once a location is selected, the duration of stay at that location is determined using matrix $\mathcal{S}$ in a similar manner. After the stay duration ends, the process repeats with a new location and corresponding stay duration. This iterative approach is used to generate schedules for agents in the second stage of AVSim: ABM simulation.

\begin{figure}[h]
    \includegraphics[width=1\linewidth]{Figures/fig7.png}
    \caption{Process of generating daily trajectories.}
    % For each agent, depending on their occupation class a time-table will be generated according to the algorithm shown in this figure. First a location will be picked and a stay duration for that location will be also selected. This process will repeat until end of the day.
    \label{tt_process}
\end{figure}
\begin{figure}[h]
    \includegraphics[width=1\linewidth]{Figures/fig8.png}
    \caption{Probability variation for the most visited locations with time for Students.}
    \label{prob_distribution_st}
\end{figure}

Using the time-location dataset generated earlier, we extracted these two matrices for each profession class. Two rows extracted from a generated matrix $\mathcal{L}$ are illustrated in Fig.~\ref{prob_distribution_st}. This figure shows the visit probabilities for two of the most visited locations for students throughout a regular school day.
% Such generated matrix $L$ for students is shown in Fig.~\ref{probMat}.



% \begin{figure}[h]
%     \cenofession 
%     \begin{subfigure}[a]{\linewidth}
%         \centering
%         \includegraphics[width=1\linewidth]{Figures/probMat.png}
%         \caption{}
%         \label{probMat}
%     \end{subfigure}
%     \vspace{0.5cm} % Add vertical spacing if needed
%     \begin{subfigure}[b]{\linewidth}
%         \centering
%         \includegraphics[width=0.9\linewidth]{Figures/fig8.png}
%         \caption{}
%        \label{bimodality}
%     \end{subfigure}
%     \caption{Student's (a) Probability Matrix of Location Visits and (b) Probability Variation of the Most Visited Locations with time.}
%     \label{fig:combined}
% \end{figure}

% For validation of this method, agent trajectories for 1000 days were generated for a given profession class. Then the probability matrices were generated and compared with the original matrices. We can observe that they are very similar to each other as seen in Fig ~\ref{BimValidation}



% Most of the existing ABMs lack the ability to model proper human mobility patterns. Many have tried to model human motion ...(EXPLAIN MORE HERE!!!). In our proposed ABM, human mobility patterns will be captured using the aforementioned bimodality model. Each agent in the simulation will be given a timetable at the start of the day. This generated timetable will depend on the occupation class and the type of day. For a simple example, A Teacher will most likely not visit school on a weekend day. There is a clear distinction between teachers' timetables on a typical workday and a holiday. Therefore, using spectral clustering results together with the Bimodality model, the ABM can generate realistic daily routines for each agent. Thereafter, agents will move in the environment according to the given timetable for the day. They will use the transport system explained in the Transport section to move between locations. They are given the option to take a bus, tuk-tuk (local taxi), a private vehicle, or walk.


\subsection{Software system}
AVSim was implemented using Python for its flexibility and extensive libraries for simulation and data processing. The class and object structure based on object-oriented principles were used to represent the system's agents, locations, and transport mediums. This approach improved code modularity, reusability, and ease of maintenance. 
% Add advantages of OOP and other stuff
For data visualization, a React application was employed, utilizing the React-D3-Tree library for visualizing tree structures. Python's Folium library was used to generate interactive maps. 


\subsection{Environment}
To represent the geographical hierarchy identified in Section \ref{method:mobility patterns}, we used a tree-based hierarchical structure to represent the environment within the ABM. 
\begin{figure}[h]
    \centering
    \includegraphics[width=1\linewidth]{Figures/fig9.jpg}
    \caption{Map illustrating the Kandy District with annotated cities, sub-cities, and functional zones.}
    \label{fig:map}
\end{figure}
This structure is advantageous as it allows agents to navigate between locations in a way that mirrors real-world movement. When an agent transitions from one location to another, the tree structure inherently defines the sequence of locations or zones it must traverse, which is essential for understanding and predicting the spread of disease.

Given that the motion data was sourced from the Kandy District in Sri Lanka, the same environment was recreated within the ABM framework. The district's geography, including its three major cities—Kandy, Pallekele, and Gampola—was annotated using polygon shapes on Google Maps. Each city was further subdivided into sub-cities and zones, classified based on their primary functions, such as educational, medical, and residential areas. For instance, schools located near each other within a city were grouped into a single educational zone. Fig. \ref{fig:map} illustrates the annotated cities and zones.

The annotations were exported, and the polygon coordinate points were used to construct the environment within the ABM. The hierarchical structure was built using the NetworkX library.

At the base level of the hierarchy, the tree's leaves represent primary locations such as schools, homes, and hospitals. These locations were randomly placed within designated zones (e.g., educational, residential, medical) to reduce manual annotation, as specific placements within zones have minimal impact on disease spread. The zones form the next hierarchical level. To differentiate locations and zones, a numbering system (e.g., "Home\_1," "Home\_2," "ResidentialZone\_1," "ResidentialZone\_2") was applied. These names were derived from the generated dataset to align with the created bimodal probability matrices.

Higher levels of the hierarchy represent broader geographical categories such as sub-cities, cities, and districts. ABM users can define areas based on their needs, regardless of the generated dataset. Fig. \ref{fig:tree} illustrates the tree structure, with only a single zone expanded to show its locations for clarity.

\begin{figure}[h]
    \centering
    \includegraphics[width=1\linewidth]{Figures/fig10.jpg}
    \caption{Hierarchical tree structure of the environment.}
    \label{fig:tree}
\end{figure}

In this tree structure, each node represents a geographical area. Nodes are implemented as objects, with attributes defining each area's characteristics, including its boundary (polygon coordinates), centroid, and the list of agents present. Each node also contains a dictionary to track the number of agents and their respective disease states.

\subsection{Vector patch}
\label{method:patch}
In agent-based models of vector-borne diseases, it is not essential to represent each vector individually, as is done for human agents. Instead, modeling vector density within localized areas provides a simpler yet accurate representation of the vector population. Following approaches from similar studies \cite{patch-2014}, the simulation adapts a patch-based approach to represent vector densities.

Each zone is divided into patches, with the patch size determined by the approximate habitat areas of the vectors. Fig. \ref{fig:mosquitopatches} presents a map of Kandy City, illustrating an example of vector density across different patches.

A patch is modeled as an object that encapsulates environmental factors such as temperature, humidity, and rainfall, which influence vector emergence and death rates. It also maintains the number of vectors in different disease states, a dictionary of agents with their disease statuses, and a list of locations within the patch.

% The network-patch method is particularly effective for modeling diseases like dengue, as it captures interactions among hosts, vectors, and their environments while incorporating spatial and movement dynamics \cite{patch-2014}.

% The Aedes aegypti mosquito typically flies up to 400 meters in search of water-filled containers to lay eggs, but it generally stays near human habitation \cite{whoDengueSevere}. Our simulation used a 500x500 square meter patch size to represent the mosquito's activity area. Fig. \ref{fig:mosquitopatches} shows a map of Kandy City, with patches indicating vector density across the city. The map also displays the spatial distribution of infected hosts (red dots) and uninfected hosts (blue dots).

\begin{figure}[h]
    \centering
    \includegraphics[width=1\linewidth]{Figures/fig11.png}
    \caption{Map of Kandy District showing the zones and their patches, with a color bar representing an example vector density across the patches}
    \label{fig:mosquitopatches}
\end{figure}

The total birth rate of vectors in a given patch $k$, $h_v^k$, is calculated as follows:

\begin{equation}
   h_v^k = N_v^k \left( \psi_v - (\psi_v - \mu_v) \cdot \frac{N_v^k}{K_v} \right) 
\end{equation}

where:
\begin{itemize}
    \item $\psi_v$: Average natural emergence rate of vectors
    \item $\mu_v$: Average death rate of vectors 
    \item $K_v$: Carrying capacity of the vectors in the patch
    \item $N_v$: Total number of vectors in the patch
\end{itemize}


\subsection{Agent generation and emulation}
% problem-solving through similitude.

The agents of the system are a vital component in any agent-based model. In this model, agents were initially categorized based on their occupation class and further subdivided as per the identified mobility patterns. For instance, doctors who worked the day shift vs. the night shift were classified into two groups. Each occupation class was defined as either an adult or child class. This parameter was important to establish the independence of each occupation class. The occupation classes, sub-classes and composition of agents to be modeled by the ABM could be varied based on input data.

The agents were represented through the use of Object Oriented Programming (OOP) concepts, specifically classes and objects. The agent class represented the structure of all agents (objects) created for emulation. Parameters of an agent defined the characteristics of each individual agent, such as the occupation class, sub-class (if applicable), age ranges, home and work location classes, and factors necessary for disease modeling. The total count of agents was dependent on the carrying capacity of the environment. As mentioned in the above sections, this study was capable of modeling the Kandy district, and approximately 2000 agents were generated for emulation. This constituted about 1\% of the actual population of the district. The composition of agent occupation classes was also selected to approximately represent their real-life counterparts.

The procedure of agent generation within the ABM was as follows. At the start of emulation, the ABM was given the carrying capacity and composition of agents from the environment and input data, respectively. The agent objects were then generated with parameters set within defined ranges for the respective occupational classes. Subsequently, each agent object was assigned a home object. The occupancy limit for home objects is defined by the emulation parameters. Each house object was constrained to either be empty or to contain one or more adult agents. This was due to child agents not being considered independent in the ABM. Finally, a work location object corresponding to the agent's work location class was assigned to the agent at random.

Upon completion of the agent generation procedure, the emulation process was initiated. Simulation steps were executed every 1 minute for each day (resolution of the simulation. At the start of an emulation day, a timetable was generated for each individual agent (this process is described in Section \ref{method:bimodality}). This timetable defined the routine (in terms of locations to visit and time of occurrence) for each agent. Except for home and work locations, all other locations (such as shopping malls and banks) were given as location classes. Therefore, the agent would choose to visit the closest location instance of that class. Once the timetables were generated, each agent would utilize modes of transport (as explained in the next section) to follow their schedule.

\subsection{Transport} 

In the context of disease modeling, long-range human mobility plays a critical role in facilitating the rapid geographical dissemination of emergent infectious diseases \cite{brockmann2009human}. This study incorporates the simulation of long-range human mobility by modeling both private and public transportation systems to better understand their impact on disease spread.   

In the private transport modeling approach, a subset of agents was designated with a public transport medium flag, while others were assigned private transport objects at the simulation's outset. These private transport objects enabled agents to travel to their destinations whenever needed. Conversely, agents marked with the public transport medium flag utilized public transport for all instances of long-range mobility. The primary objective was to evaluate the impact of private transport users on disease propagation. 

Public transport modeling holds significant importance in the context of disease modeling, particularly due to the increased vulnerability of passengers to airborne disease transmission in confined spaces with close proximity to others \cite{CUMMINGS2024111303}. Agents not assigned a private transport medium flag relied on public transport for long-range mobility instances. 

In this study, buses and taxis were used as the primary modes of public transport. Two types of bus objects were employed to represent different hierarchical levels in the simulated environment. Inter-city buses facilitate transport between cities, while intra-city buses facilitate transport inside a city. This process is explained in Fig ~\ref{transportr}.

% Inter-city bus objects facilitated transport within zones (the second layer of the hierarchical structure from bottom) corresponding to sub-cities. This inter-zone bus object started a journey from a bus station in a sub-city and moved to every zone inside that sub-city and arrived to the destination at same point it started. This circular movement continued throughout the whole day in a given defined start and end time (from 5AM to 11PM). Short distances between locations at the base layer rendered bus objects unnecessary at that level. However, the simulation framework is adaptable, allowing bus objects to be incorporated across all hierarchical levels as needed. 

% Other type of the bus object was the city-city objects. They started their journey from a defined city and reached another city after passing all the sub-city bus stations along that root. This operation is executed back and forth throughout the day from 5AM to 11PM.

Bus objects were initiated at the start of the simulation. 
% Every bus object were started their journey in each and every 10 minutes from their starting location. This interval was decided while ensuring most of the agents could find a bus without waiting 15 minutes or more at a bus station. The number of bus objects in each root were decided on the parameters such as total travel time taken to a certain root bus object to reach its destination, the average waiting time agents on that root per day.
Each bus station referred to as a Bus Node, maintained
an updated dictionary of available buses and their next destinations.
\begin{figure}[h]
    \includegraphics[width=1\linewidth]{Figures/fig12.png}
    \caption{Public transport mechanism.}
    \label{transportr}
\end{figure}
% Each Bus object waited for 5 minutes at each
% Bus Node.
When a bus arrived, agents who had been waiting for transportation were assigned to that bus, provided the bus's next destination matched the agent's next location in the bus's dictionary of destinations.

If the agent had to wait too long for a bus, they would be assigned to a taxi to travel to the next destination. The Taxi was assigned to each zone so that every agent who missed the bus could still get a taxi. When a taxi arrived at a destination zone, it stayed there until the next trip was assigned.

% The travel time between zone to zone for the buses and the three-wheelers was calculated by dividing a defined speed by the straight line distance between zones. This speed was decided considering the practical time for travel using those transportation mediums. If the starting zone and next destination zone bus nodes' geographic coordinates were $(lon 1, lat 1)$ and $(lon 2, lat 2)$, respectively, the distance was calculated by
% \begin{equation}
% d = 2 * R\operatorname{atan} 2(\sqrt{a}, \sqrt{(1-a)})
% \end{equation}
% \begin{equation}
% a=\sin ^2\left(\frac{ dlat}{2}\right)+\cos (lat 1) * \cos (lat 2) * \sin ^2\left(\frac{ dlon }{2}\right)
% \end{equation}
% \begin{equation}
% dlon = lon 2 - lon 1
% \end{equation}
% \begin{equation}
% dlat = lat 2- lat 1
% \end{equation}
% where R was the radius of the Earth, and the mean radius was taken here, which was $6371 \mathrm{~km}$.

% Here, only one agent was assigned at a time for a taxi. There was no contact between the agents inside the taxi. So, the taxis were used only to ensure the transportation of agents to the next location unless there wasn't a bus object for that agent within the waiting time.

% \subsection{ABM performance}

% \subsection{Disease Modeling}

% One of the most prominent uses of ABMs is modeling disease spread patterns. By leveraging the below key qualities/capabilities of the ABM, it was possible to conduct thorough research on using the above ABM to model disease spread patterns.
% \begin{itemize}
%   \item Capability to model real-world environments using geospatial data.
%   \item Ability to generate realistic human motion patterns inside that environment.
%   \item Heterogeneity of the agent population based on occupation and age.
% \end{itemize}

%  In this research, air borne disease(Covid 19) spread and vector borne disease(dengue) spread were analysed.

% Add more here if needed. probably cite?


\subsection{Markov disease progression model}
\label{method:markov}
A person can transition through various states after being exposed to a disease. These transitions depend on the nature of the disease. Generally, an exposed individual becomes infectious after a certain period, carrying the risk of spreading the disease. The disease progression may remain asymptomatic or exhibit symptoms, potentially progressing through severity stages such as mild, severe, and critical. 

We used a Markov model to handle state transitions, as illustrated in Fig. \ref{fig:markov model}, which defines the possible states a person can be in after exposure to a virus. Transitions between these states were probabilistic and were often modeled using distributions such as the log-normal distribution. The probabilities of these transitions depended on factors like the nature of the disease, age, prior infections, and the quality of medical care. For our two use cases, dengue and COVID-19, the transition times are shown in the figure and followed a log-normal distribution. Values for COVID-19 were taken from Covasim~\cite{COVASIM} while dengue was taken from WHO~\cite{whoDengueSevere}. 

It is important to note that the transition from the "susceptible" to "exposed" state was not considered part of the Markov model. This transition depends on how a person becomes exposed to the virus. In airborne diseases, this typically occurs through human-to-human contact, while in vector-borne diseases, it happens via interactions with vectors. These mechanisms will be detailed in subsequent sections.

\begin{figure}[h] 
\centering \includegraphics[width=1\linewidth]{Figures/fig13.png}
\caption{Markov model for disease transmission modeling.} 
\label{fig:markov model} \end{figure}
In this model, we had \( N\) possible states (without "susceptible"), and transitions between these states were represented by the transition matrix \( A \), where \( A_{ij} \) denotes the probability of transitioning from state \( j \) to state \( i \). We used a probability vector \( p_t \) (of size \( N \times 1 \)) that captures the state probabilities at time \( t \). To find the next state probabilities \( p_{t+1} \), we applied the fundamental equation \ref{eq:markov_transition} for a Markov chain.
\begin{equation}
\label{eq:markov_transition}
p_{t+1} = A \cdot p_t
\end{equation}

To determine the actual next state, we generated a random number \( r \) between 0 and 1. Using \( p_{t+1} \), we calculated cumulative probability ranges for each state \( i \), based on the probabilities in \( p_{t+1} \). The next state is chosen as the one for which \( r \) falls within the cumulative range associated with that state.

\subsection{Air borne disease modeling}
\label{air borne method}

% To model airborne diseases, COVID-19 was selected. owever, it is important to note that this disease model is not only limited to modeling a specific disease. Covid-19 was selected as a mere example. The literature `\cite{Weligampola2023, COVASIM, OPENABM}
% were particularly useful in this analysis.

\subsubsection{Disease progression mechanism}

Initially, a small percentage of the population was infected with the COVID-19 virus, and the agents were freely allowed to go about their daily business, allowing us to study the disease spread. Once an agent is infected, it will progress through different stages of the disease. This was modeled using the Markov model as shown in Fig ~\ref{fig:markov model}. Each agent will belong to either susceptible, exposed, infectious, symptomatic, asymptomatic, or dead states.

% The transition time from one state to another will take a Log-normal distribution according to ~\cite{COVASIM}.

The disease is spread when a susceptible individual comes into contact with an infected one. Contacts occurred in two scenarios: at specific locations or during transport (on a bus). At each time step, a 1-meter radius around each agent was monitored. If another agent enters this radius, it is recorded as a contact. When a susceptible agent contacts an infected agent, a random number is generated. If this number recedes the agent's infectious probability (\(\rho\)), the susceptible agent's disease state changes to "exposed." Each agent's disease state is then updated according to the disease model. Agent's infectious probability depends on the individual's immunity level, as described in the following section. 
% This if further explained in Fig . ???? DO WE NEED THIS  ????

\subsubsection{Immunity modeling}
Contact between a susceptible agent and an infected agent does not always result in disease transmission. This is determined by a transmission probability, which depends on the agent's immunity. In the simulation, agents were assigned varying immunity levels based on factors such as age, hygiene practices (e.g., maintaining a safe distance, hand washing, sanitizer use), and vaccination status. This infection probability, \(\rho\), can be modeled using the equation \ref{eq:infec_prob_AirB}.
\begin{eqnarray}
\label{eq:infec_prob_AirB}
\rho = S_{age} \cdot k \cdot (1 - \alpha_{vacc} \gamma_{vacc} - \alpha_{hyg} \gamma_{hyg})
\end{eqnarray}

where \(S_{age}\) is the susceptibility based on the age, \(k\) is a tunable parameter that has to be manually found, \(\alpha_{vacc}\) is the Weight assigned to vaccination, \(\gamma_{vacc}\) is the Immunity boost provided by vaccination, \(\alpha_{hyg}\) is the Weight assigned to hygiene factor and \(\gamma_{hyg}\) is the Immunity boost provided by safe hygiene practices which include washing hands, wearing masks, and social distancing. The age-based susceptibility values were taken from Covasim \cite{COVASIM}, and they are presented in Table ~\ref{Table: Age-based immunity}. Through some prior tuning process, the value of \(k\) was selected as 0.3. 
% Where, 
% \begin{itemize}
%     \item \(S_{age}\): Susceptibility based on the age
%     \item \(\alpha_{vacc}\): Weight assigned to vaccination
%     \item \(\gamma_{vacc}\): Immunity boost provided by vaccination
%     \item  \(\alpha_{hyg}\): Weight assigned to hygiene factors
%     \item \(\gamma_{hyg}\): Immunity boost provided by hygiene practices 
% \end{itemize}


% The two weighting parameters, \(\alpha_{vacc}\) and  \(\alpha_{hyg}\), are decided based on WHAT.


% --------------- Age bases susceptibility table ---------------------
\begin{table}[!ht]

\caption{
{Age-based susceptibility values for agents}}
\label{Table: Age-based immunity}

\begin{tabular}{| m{1.78cm} | m{0.78cm} | m{0.78cm} | m{0.78cm} | m{0.78cm} | m{0.78cm}|}
  \hline
  Age & 0-9 &  10-19 & 20-69 & 70-79 & 80+\\ 
  \hline
  Susceptibility & 0.34 & 0.67 & 1.00 & 1.24 &1.47 \\
  \hline
\end{tabular}
% \begin{flushleft} %Susceptibility values were taken from ~\cite{COVASIM}.
% \end{flushleft}
% \label{table2}
\end{table}

\subsubsection{PCR testing and quarantine mechanism}

Polymerase Chain Reaction (PCR) testing is crucial in predicting the risk of airborne diseases \cite{West2008-xo}. This is used as a pre-event testing activity, and people with positive PCR results are identified as infected agents. In the COVID-19 pandemic, infected agents were identified using PCR tests and using contact tracing methods. Contacted agents were subjected to quarantine or self-isolation \cite{Wong2022-ql}.

The AVSim had the functionality to perform PCR tests to identify the infected agents. Each agent would have a daily testing probability, which would depend on the state of infection. This method allowed the user to specify the testing probabilities for each of these states. These numbers would depend on the regulations of the government and the healthcare sector. Therefore, the simulation would decide the number of PCR tests done each day. 

Two methods were employed in the simulation to perform quarantine. In the first method, if a PCR test was positive, the relevant agent would be isolated and put under quarantine. When that agent was recovered and safe to interact with other agents, the simulation would allow the agent to end their quarantine. In the second method, all agents within a specific occupation class were quarantined if the number of infected agents in that class surpassed a certain threshold. In \ref{air borne results}, results for each of these methods are analyzed and discussed.

\subsubsection{Vaccination mechanism}

Vaccination mechanisms are utilized to enhance community immunity, particularly during outbreaks of airborne diseases like COVID-19 \cite{Stauft2023-rl}. In AVSim, vaccination events were organized to reduce the disease spread while increasing the immunity levels of the agents.

In AVSim, an outbreak was declared whenever the total number of infected agents exceeded a specified threshold. When an outbreak was detected in the simulation, vaccination events could be organized. The AVSim simulation allowed the organization of vaccination events on a selected day at a specified geographical zone. AVSim further allowed the capability to vaccinate targeted occupational classes and gave out various types of vaccines. When an agent is given a vaccination, their vaccine immunity \(\gamma_{vacc}\) is boosted. According to ~\cite{Soheili2023-ka}, a few possible vaccine types and their immunity boosts are shown in Table ~\ref{Table: vacc}. New vaccination events with new vaccine types could be added this way. Additionally, vaccination could be administered in doses.

\begin{table}[!ht]
\centering
\caption{Different types of possible vaccines and their effectiveness}
\label{Table: vacc}

\begin{tabular}{| m{2cm} | >{\centering\arraybackslash}m{2.5cm} | >{\centering\arraybackslash}m{2.5cm} |} 
  \hline
  \makecell[c]{Vaccine Name} & \makecell[c]{Immunity boost\\after 1st dose} & \makecell[c]{Immunity boost\\after 2nd dose} \\ 
  \hline
  Moderna & 74 & 93\\
  \hline
  AstraZeneca & 78 & 67 \\
  \hline
  Pfizer & 84 & 93\\
  \hline
\end{tabular}
\end{table}


% \subsubsection{Events (Mario)}

% Effect of Gatherings, party meetings.


\subsection{Vector-borne disease modeling}

For modeling vector-borne diseases, the network-patch approach proposed by Carrie Manorea et al.~\cite{patch-2014} for dengue transmission has been adapted in this study to reflect the dynamics of vector-borne diseases more accurately.

\subsubsection{Disease progression in vectors}
\label{method:vb-disease-propagation}
As mentioned in Section \ref{method:patch}, we do not consider individual vectors separately. Instead, we used a patch to represent vector density and employed a stochastic process to model disease progression. Specifically, we focused on three disease states of the vectors: susceptible, exposed, and infected, as illustrated in Fig. \ref{fig:vb_states}.

Each patch $k$ is initialized with counts for susceptible ($S_v^k$), exposed ($E_v^k$), and infected ($I_v^k$) vectors, ensuring that these values remain below a defined vector carrying capacity specific to each patch. The temporal dynamics of vector density within a patch are modeled using a system of differential equations. The changes in vector counts for a given patch $k$ are determined as follows. Factors with superscript $k$ are unique to a given patch, while others are patch-independent.

\begin{equation}
  \frac{dS_v^k}{dt} = h_v^k - \lambda_v^k S_v^k - \mu_v S_v^k  
\end{equation}
\begin{equation}
    \frac{dE_v^k}{dt} = \lambda_v^k S_v^k - v_v^k E_v^k - \mu_v E_v^k
\end{equation}
\begin{equation}
    \frac{dI_v^k}{dt} = v_v^k E_v^k - \mu_v I_v^k
\end{equation}

where:
\begin{itemize}
    \item $h_v^k$: Total birth rate of vectors 
    \item $\lambda_v^k$: Average infection rate of vectors
    \item $v_v^k$: Average rate of progression of vectors from exposed state to infectious state
    \item $\mu_v$: Average death rate of vectors
\end{itemize}
\begin{figure}[h]
    \centering
    \includegraphics[width=1\linewidth]{Figures/fig14.png}
    \caption{Mosquito state transition.}
    \label{fig:vb_states}
\end{figure}

\bigskip
The average infection rate of vectors in patch $k$, $\lambda_v^k$, is calculated as follows:

\begin{equation}
  \lambda_v^k = b_v^k \cdot \beta_{vh} \cdot \left( \frac{I_h^k}{N_h^k} \right)  
\end{equation}

where:
\begin{itemize}
     \item $b_v^k$: The number of bites per vector per unit time 
     \item $\beta_{vh}$: The probability of transmission from an infectious agent to a susceptible vector given that contact between the two occurs
    \item $I_h^k$: The number of infected agents in the patch $k$
    \item $N_h^k$: Total number of agents in patch $k$
\end{itemize}

The number of bites per vector per unit of time in patch $k$ which is defined in the following way:

\begin{equation}
    b_v^k = \frac{b^k}{N_v^k}
\end{equation}    
    
where:
\begin{itemize}
    \item $b^k$: The total number of contacts between agents and vectors (bites) in patch $k$ per unit time
\end{itemize}

The total number of contacts between agents and vectors (bites) in patch $k$ is defined in the following way:
\begin{equation}
     b^k = \frac{\sigma_v \cdot N_v^k \cdot \sigma_h \cdot N_h^k}{\sigma_v \cdot N_v^k + \sigma_h \cdot N_h^k}
\end{equation}
  
where:
\begin{itemize}
    \item $\sigma_v$: The maximum number of bites per vector per unit time.
    \item $\sigma_h$: The number of bites a human can get per unit of time.
\end{itemize}
    
\bigskip
For the selected use case of dengue virus infection, the average rate per agent of progression from the exposed state to the infectious state of vectors in patch $k$, denoted as $v_v^k$, depends on the temperature~\cite{Xiao2014} and was calculated as follows:

\begin{equation} 
    v_v^k = \frac{1}{tinc_v^k} 
\end{equation}

where:
\begin{itemize}
    \item $tinc_v^k$: The incubation time of the virus in the mosquito in patch $k$. This depends on the temperature of the patch ($T^k$) and is calculated as follows:
    \begin{itemize}
        \item If $T^k < 21$, then $tinc_v^k \sim U(10, 25)$
        \item If $21 \leq T^k < 26$, then $tinc_v^k \sim U(7, 10)$
        \item If $26 \leq T^k < 31$, then $tinc_v^k \sim U(4, 7)$
    \end{itemize}
\end{itemize}

The parameters used for dengue transmission \cite{whoDengueSevere} are provided in Table. \ref{tab:vb_params}, with rate values specified on a per-day basis. Since the model includes multiple vector patches, a range of values was selected for some parameters, indicating minimum and maximum values (enclosed in brackets).


\begin{table}[h]
    \centering
    \caption{Parameters for dengue simulation}
    \label{tab:vb_params}
    \begin{tabular}{|p{5.5cm}|c|}
        \hline
        \makecell[c]{Parameter} & Value/Range\\
        \hline
         $\psi_v$: Average natural emergence rate of mosquitoes  & 0.3 \\
         \hline
         $\mu_v$: Average death rate of vectors & 1/14\\
         \hline
         $\sigma_v$: The maximum number of bites per vector per day
         & 0.5 \\
         \hline
         $\sigma_h$: The number of bites a human can get per day & 10 \\
         \hline
         $K_v$: Mosquito carrying capacity
         & [100-200]\\
          \hline
          $\beta_{vh}$: The probability of transmission from an infectious agent to a susceptible mosquito & 0.33 \\
          \hline
          $\beta_{hv}$: The probability of transmission from an infectious mosquito to a susceptible agent & 0.33 \\\hline
    \end{tabular}
\end{table}

\subsubsection{Disease progression in agents}

The infection rate of agents from infectious mosquitoes is determined by the frequency of bites an agent receives, the likelihood that the bite is from an infectious mosquito, and the probability that the infection will be successful given the bite. This rate can be represented as:

\begin{equation}
    \lambda_{k,h}(t) =  b^k_h \cdot \beta_{hv} \left( \frac{I_k^v}{N_k^v} \right)
\end{equation}

where:
\begin{itemize} \item $b_{k,h}$: the number of bites an agent receives per unit time from mosquitoes in patch $k$
\item $\beta_h$: the probability of successful infection given a bite from an infectious mosquito
\item $I_k^v$: the number of infectious mosquitoes in patch $k$ 
\item $N_k^v$: the total number of mosquitoes in patch $k$ \end{itemize}

To calculate the probability of a susceptible agent becoming infected, the infection rate must be converted into a probability of infection within a given time step. We run the simulation and update agent movements in time steps of $\Delta t$, while the disease states of agents and vectors are updated every $n \cdot\Delta t$ time steps. In our simulation, agent movements are updated every 1 minute, and disease statuses are updated every 5 minutes. Assuming the time to infection follows an exponential distribution \cite{patch-2014}, the probability of infection at the end of the time interval $\Delta t$ is given by:

 \begin{equation}  
p_{k} = 1 - e^{-\lambda_{k,h} \Delta t}
 \end{equation}

where:
\begin{itemize} \item $p_{k}$: infection probability of a susceptible person in patch $k$ \item $\lambda_{k,h}(t)$: infection rate of an agent \end{itemize}

Then, a random variable between 0 and 1 is generated every 5 minutes to check if it falls above the infection probability. If the random variable exceeds the probability, the agent is marked as exposed. Once an agent is exposed, the disease transitions follow the Markov model as shown in Fig.~\ref{fig:markov model}.

An infected person may either stay asymptomatic, develop mild symptoms, or, in rare cases, progress to severe symptoms \cite{whoDengueSevere}. Many people with dengue remain asymptomatic, showing no visible symptoms. Those with mild symptoms experience typical dengue signs like fever, headache, and body aches, which often resolve with basic supportive care, allowing for full recovery. Severe dengue, however, can bring serious symptoms like intense abdominal pain, persistent vomiting, bleeding, or shock, often requiring hospitalization and posing life-threatening risks.

The likelihood of transitioning between these states depends on various factors, including age, prior dengue infections, and the quality of medical care. For instance, the likelihood of an infected person developing symptoms may vary based on their immune response.

The Markov model outlined in Section \ref{method:markov} is used to determine state transitions. For dengue, the model focuses only on mild and severe symptomatic states. By combining probabilistic state transitions with random transition periods, the model can effectively simulate diverse disease progression outcomes among individuals.


\subsubsection{Hospitalization mechanism}
There is a high probability of symptomatic patients visiting the hospital. At the start of each day in the simulation, all agents' disease statuses are checked. If symptomatic agents are present, a random hospitalization probability is generated. If this probability exceeds a certain threshold, the agent is hospitalized. For severe cases, this threshold is lower, reflecting the higher likelihood of hospitalization. In our simulation, we used 0.4 for mild cases and 0.9 for severe cases.

Once the agents requiring hospitalization are identified, they are removed from both the environment and the patches and placed in a virtual space. Hospitalized agents do not contribute to further disease spread. However, their disease states continue to update as outlined in Fig. \ref{fig:markov model}. Eventually, these agents either recover or die. Recovered hospitalized agents are returned to both the environment and their respective patches.

\subsubsection{Intervention mechanism}
When there is a rise in reported vector-borne cases in a specific area, vector control interventions are initiated, often spearheaded by government healthcare programs or community efforts. These measures typically include mosquito spraying or the removal of breeding sites. 

In AVSim, to simulate such interventions, the total mosquito-to-host exposure in each patch is recorded. If the cumulative number of exposed agents in a zone over one week surpasses a predefined threshold, $E_h^k$, the vector population in the affected zone's patches is reduced by $m\%$ to represent mosquito spraying and the elimination of breeding sites.


% Results and Discussion can be combined.
\section{Results and discussion}

\subsection{Spectral clustering results}
\label{result:spectral clustering}
Distinct behavioral patterns within each occupation class were derived by applying the Spectral Clustering algorithm to each occupation class. This was done by manual inspection of the output of the clustering algorithm. Table ~\ref{Occupation_Subclusters} presents the synthesized results.
Note that normal behavior refers to a typical workday for each class.
\begin{table}[htbp]
\centering
\caption{Distinct behavioral patterns Obtained for occupation classes through spectral clustering method.}
\label{Occupation_Subclusters}

\begin{tabular}{|l|>{\raggedright\arraybackslash}p{0.53\linewidth}|}
\hline
\multicolumn{1}{|c|}{Class}          & \multicolumn{1}{c|}{Identified Clusters}                                                                \\ \hline
\multirow{3}{*}{Animal Farmer}       & Work area is located in close proximity to the residence.\\ \cline{2-2} 
                                     & Work area is situated separately from the residence.\\ \cline{2-2} 
                                     & Individual resides and works within a designated Agricultural Zone.\\ \hline
\multirow{3}{*}{Bank Worker}         & Normal workday behaviour\\ \cline{2-2} 
                                     & Weekend and holiday behaviour\\ \cline{2-2} 
                                     % & Outlier behavior (evident as spending entire days at the bank for multiple consecutive days)\\ 
                                     \hline
\multirow{5}{*}{Doctor}              & Normal workday behaviour\\ \cline{2-2} 
                                     & Night to evening shift                                                                                  \\ \cline{2-2} 
                                     & Night to morning shift                                                                                  \\ \cline{2-2} 
                                     & Weekend and holiday behaviour\\ \cline{2-2} 
                                     & Night shift at the hospital with daytime duties at a school medical center \\ \hline
\multirow{2}{*}{Farmer}              & Work area is located in close proximity to the residence.\\ \cline{2-2} 
                                     & Work is conducted within an Agricultural Zone.\\ \hline
% \multirow{2}{*}{Field Officer}       & Work-related behavior\\ \cline{2-2} 
%                                      & Non-work-related behaviour\\ \hline
\multirow{2}{*}{Garment Admin}       & Weekend and holiday behaviour\\ \cline{2-2} 
                                     & Normal workday behaviour\\ \hline
\multirow{2}{*}{Garment Worker}      & Work area is located separately from the residence.\\ \cline{2-2} 
                                     & The individual resides in or close to the work area.\\ \hline
% \multirow{2}{*}{Midwife/PHI}         & Work is performed on the move (visiting residences).\\ \cline{2-2} 
%                                      & The work is conducted within a Medical Zone.\\ \hline
\multirow{2}{*}{Sales Rep}           & Normal workday behaviour                                                                                \\ \cline{2-2} 
                                     & Holiday behaviour\\ \hline
\multirow{3}{*}{Student}             & Normal behaviour of a student travelling from home\\ \cline{2-2} 
                                     & Weekend and holiday behaviour with irregular school attendance\\ \cline{2-2} 
                                     & Hostel student                                                                                          \\ \hline
\multirow{4}{*}{Super Market Worker} & Morning to night shift behaviour                                                                        \\ \cline{2-2} 
                                     & Afternoon to night shift behaviour\\ \cline{2-2} 
                                     & Morning to evening shift behaviour\\ \cline{2-2} 
                                     & Works from morning to afternoon \& evening to night with a break in between. also includes holiday behavior \\ \hline
\multirow{3}{*}{Teacher}             & Teacher resides in or near the school premises\\ \cline{2-2} 
                                     & Normal teacher behaviour (travelling from home)                                                         \\ \cline{2-2} 
                                     & Weekend and holiday behaviour\\ \hline
\end{tabular}
\begin{flushleft}
\end{flushleft}
\end{table}


After examining the results, it can be observed that the
algorithm excels at producing clusters with distinct behavioral
patterns, ranging from regular work routines to holiday behaviors in alignment with real-world intuition. Furthermore,
the algorithm effectively filters out anomalous patterns, as
demonstrated in the instances above. The final results and the human motion dataset we created to fuel the AVSim can be found online at \cite{https://doi.org/10.5281/zenodo.13621863}.

% Please add the following required packages to your document preamble:
% \usepackage{multirow}

\subsection{Validation of human motion modeling}
% ---------------  Bimodality validation plots----------------------

To validate the accuracy of the human motion model, 10000 samples of daily person trajectories were generated for each profession class. Then, the probability distributions were reconstructed using them to compare with the original probability distributions. These synthetic and original plots can be seen in Fig.~\ref{BimValidation}. We can see that the synthetic distributions closely resemble the ground truth distributions, thus proving the above method to be highly accurate in replicating human behavior.

\begin{figure*}[h]
    \centering
    \includegraphics[width=1\linewidth]{Figures/fig15.png}
    \caption{Validation of bimodality.}
    \label{BimValidation}
\end{figure*}


\subsection{Air borne disease simulation results}
\label{air borne results}

In this sector, a few simulations were conducted to research air-borne disease spread. However, it is important to note that these results are only aligned through a subset of possible research directions. One can perform various analyses based on their need and interests. The simulations were conducted in a simulated environment replicating the Kandy district.


\subsubsection{Contact tracing}

By default, the AVSim will generate detailed descriptions of each agent in the simulation, such as daily routines, daily person trajectories, disease progression history, and daily contacts. This is illustrated in Fig.~\ref{contact_tracing}. The AVSim can be customized to generate any other required information.

\begin{figure}[h]
    \includegraphics[width=1\linewidth]{Figures/fig16.png}
    \caption{Contact tracing capability of AVSim.}
    \label{contact_tracing}
\end{figure}

\begin{figure*}[h]
    \centering
    \includegraphics[width=1\linewidth]{Figures/fig17.png}
    \caption{All the contacts in an uncontrolled, i.e., no vaccine nor quarantine disease-transmitting moment for 50 days simulation in a part of "KandyCity" and "Pallekele." Initially, three students were infected randomly. The receiving agent is shown by filled circles and the cause; the transmitting agent is shown by outer borders in each instance.}
    % figure description
    \label{contactPlot}
\end{figure*}


\begin{figure}[h]
    \includegraphics[width=1\linewidth]{Figures/fig18.png}
    \caption{Agents count in different states in an uncontrolled epidemic outbreak, which started with three infected students.}
    % figure description
    \label{NQNV}
\end{figure}

The ability to trace back contacts is a key feature that is crucial in any type of ABM. Fig.~\ref{contactPlot} shows the locations where the disease was transmitted from an infected agent to a healthy agent. This is essentially the contact tracing mechanism of the ABM in action. When analyzing, it can be seen more contacts are present in the "Residential Zones." This can be described as the area of houses compared to other places being lower. Therefore, there is a high possibility of more contact in houses. Also, another major observation is that although there are higher contacts in residential zones, the spread of disease to other communities, i.e., residences present in other cities, is by contacts in work locations or outside. This can be verified by Fig. \ref{contact_tracing}. Starting with one infected student agent, the disease transmits to another student agent at the school. Then, the received student agent transmits the disease to the nurse at the student agent's home. This leads to the spread of disease in a hospital, which is the working location of the nurse. Likewise, it can be observed that there are several branches of contact between the second student who was contacted and the initial student at the school. It proves work locations (here, a school for the student and a hospital for the nurse) are involved in spreading diseases to different communities through the agents in the same work locations. Therefore, identifying these kinds of contact patterns and predictions at early stages will enable policymakers to pinpoint high-risk areas and implement targeted interventions to prevent the spread of the disease in these regions.

\begin{figure}[h]
    \includegraphics[width=1\linewidth]{Figures/fig19.png}
    \caption{Disease spread in different professional classes in an uncontrolled epidemic outbreak which started with
three infected students.}
    \label{Spread students}
\end{figure}

\begin{figure}[h]
    \includegraphics[width=1\linewidth]{Figures/fig20.png}
    \caption{Disease spread in different professional classes in an uncontrolled epidemic outbreak which started with three random infected agents.}
    \label{Spread random}
\end{figure}
\subsubsection{Disease progression in an uncontrolled environment}

This AVSim simulation can be executed without imposing policies to reduce the spread, which is known as an uncontrolled environment. Fig. ~\ref{NQNV} shows an uncontrolled epidemic outbreak where three random student agents were initially infected. We can see that this plot closely resembles the theoretical SEIR model curve, hence validating the practical approach of AVSim. 

\subsubsection{Effect of initially infected agents on disease progression }

The AVSim allows the analysis of disease spread patterns when agents from a specific class were initially infected. To demonstrate this, three students were initially infected in an uncontrolled environment, and then the disease spread to other classes over the simulation days was observed. The Observations are shown in Fig.\ref{Spread students}, highlighting only the top three classes that had the highest percentage of infected agents. For approximately up to 50 days, a considerable percentage of the disease was spread to teachers, w.r.t. other classes. This is reasonable because contact from students is mostly present in schools, which is the work location of teachers. This analysis helps identify the most vulnerable groups to the disease concerning initial infection. 

Furthermore, it can be observed that the spread to other classes in Fig. \ref{Spread students} becomes higher when the days pass, showing the effect of contacts in an uncontrolled environment. When the agents are randomly infected, it is hard to predict a pattern of spreading to other classes, and it depends on contact tracing. For example, in the simulation provided by Fig ~\ref{Spread random}, when the agents were randomly selected, initially infected agents were one student, one doctor, and one teacher. However, in that case, disease spread had a significant impact on the medical sector agents. One key observation from the two simulations mentioned above is that the student class is consistently more affected. This is because most agents in different professional classes have student agents in their residences, where they will interact more with student agents. Therefore, more controlled strategies should be implemented to minimize the risk of disease spread to other classes from these vulnerable groups.
\subsubsection{Quarantine and vaccination measures}

\begin{figure}[h]
    \includegraphics[width=1\linewidth]{Figures/fig21.png}
    \caption{Effect of quarantine mechanism in airborne diseases w.r.t. uncontrolled case.}
    \label{qua_ana}
\end{figure}

\begin{figure}[h]
    \includegraphics[width=1\linewidth]{Figures/fig22.png}
    \caption{Effect of vaccine mechanism in airborne diseases w.r.t. uncontrolled case.}
    \label{vac_ana}
\end{figure}

\begin{figure}[h]
    \includegraphics[width=1\linewidth]{Figures/fig23.png}
    \caption{Effect of quarantine and vaccination mechanism in airborne diseases.}
    \label{QNV}
\end{figure}

We have tested several controlled mechanisms to reduce the spread of airborne diseases. The following conclusions were assumed based on the parameters that we set as described in the \ref{air borne method} section. 
\begin{enumerate}[label=\Alph*]
\item Effect of Quarantine:
    Fig. \ref{qua_ana} shows the number of infected agents in different quarantine mechanisms compared to uncontrolled cases. The AVSim allows two mechanisms in quarantine: quarantine by profession class and self-quarantine of individuals when PCR results are positive. It can be observed that when combined mechanisms are used, the peak of infected agents is reduced by approximately 87.5\%.
\item Effect of Vaccine:
    Fig. \ref{vac_ana} shows the number of infected agents in different vaccine mechanisms compared to uncontrolled cases. We analyzed the effect of vaccinating all the agents when the infected percentage of the population is greater than 5\% and 10\%. It can be deduced that early vaccination events cause a higher disease spread reduction.
\item Controlled vs. Uncontrolled cases: Fig. \ref{QNV} shows the number of infected agents in controlled via vaccination or quarantine and uncontrolled cases. It can be observed that a higher reduction in disease spread was obtained from proper quarantine mechanisms.

\end{enumerate}
% The effect of quarantine on the disease spread is demonstrated in Fig. ~\ref{QNV}. It is clear that the disease is greatly controlled under the quarantine intervention.



% \begin{figure}[h]
%     \includegraphics[width=1\linewidth]{Figures/QNV.jpg}
%     \caption{Bolded capt}
%     \label{QNV}
% \end{figure}

% add vaccines here.
% add events here.


\subsection{Vector-borne disease simulation results}

The simulations were conducted with 2000 individual agents, using Kandy District as the simulation environment. The transportation model was not included in the simulation, based on the assumption that vector bites do not occur in public transportation.

The Aedes aegypti mosquito typically flies up to 400 meters in search of water-filled containers to lay eggs, but it generally stays near human habitation \cite{whoDengueSevere}. Our simulation used a 500x500 square meter patch size to represent the mosquito's activity area. The other simulation parameters or ranges are specified in Table. \ref{tab:vb_params}.

% \subsubsection{Contact tracing}
% At the start of the simulation, the patches and locations were initialized with agents and vector counts. All the patches in a residential zone were initially assigned as infected by assigning an infected vector count between 0 and 20. The patches were updated daily during the simulation, and the disease propagated accordingly.

\subsubsection{Disease progression in an uncontrolled environment}

At the beginning of the simulation, the patches and locations were initialized with agent and vector counts. Six homes were chosen from a designated residential zone, with up to five agents per home initially marked as exposed to dengue. The nearest nine patches around each home were given an infected vector count ranging from 0 to 20. During the simulation, the patches were updated daily, and the disease propagated accordingly.


\begin{figure}[h]
    \centering
    \begin{subfigure}[t]{0.49\linewidth}
        \centering
        \includegraphics[width=\linewidth]{Figures/fig24-a.png}
        \caption{}
        \label{non-hospitalized}
    \end{subfigure}
    \hfill
    \begin{subfigure}[t]{0.49\linewidth}
        \centering
        \includegraphics[width=\linewidth]{Figures/fig24-b.png}
        \caption{}
        \label{hospitalized}
    \end{subfigure}
    
    \vspace{0.2em} % Adjust spacing between rows

    \begin{subfigure}[t]{0.49\linewidth}
        \centering
        \includegraphics[width=\linewidth]{Figures/fig24-c.png}
        \caption{}
        \label{vb_disease_states}
    \end{subfigure}
    \hfill
    \begin{subfigure}[t]{0.49\linewidth}
        \centering
        \includegraphics[width=\linewidth]{Figures/fig24-d.png}
        \caption{}
        \label{vb_all_states}
    \end{subfigure}

    \caption{(a) Infectious disease states of non-hospitalized agents, (b) infectious disease states of hospitalized agents, (c) infectious disease states of all agents, and (d) major states changes of all agents over time.}
    \label{fig:vb-combined}
\end{figure}

At the start of the simulation, changes in the disease state of agents can be observed over time, as shown in Fig. \ref{fig:vb-combined}. Analyzing the rate of change indicates that once dengue starts spreading, the infection rate gradually increases before eventually declining. However, in this simulation, each individual can only be infected once. If reinfection were allowed, the infection rates would likely be higher. Reinfection could be modeled by resetting recovered agents to a susceptible state, but for simplicity, this is not considered in this simulation.
% Similarly, the infected mosquito counts in the patches change over time and can be observed in Fig. \ref{fig:patch-changes}. This figure is useful in identifying new emerging vector patches and attempting to control those vectors.

% \begin{figure}[h]
%     \centering
%     \begin{subfigure}[t]{0.49\linewidth}
%         \centering
%         \includegraphics[width=\linewidth]{Figures/vb-states.png}
%         \caption{}
%         \label{major_disease_states}
%     \end{subfigure}
%     \hfill
%     \begin{subfigure}[t]{0.49\linewidth}
%         \centering
%         \includegraphics[width=\linewidth]{Figures/vb-infected_states.png}
%         \caption{}
%         \label{bimodality}
%     \end{subfigure}
%     \caption{Agents' (a) major disease states and (b) infectious disease states change over days.}
%     \label{fig:vb-combined}
% \end{figure}

% \begin{figure}[h]
%     \centering
%     \includegraphics[width=1\linewidth]{Figures/patch-changes.png}
%     \caption{The infected vector count changes over four months}
%     \label{fig:patch-changes}
% \end{figure}
\subsubsection{Disease progression in a controlled environment}

Various strategies can be employed to control dengue. In this simulation, the focus is on managing mosquito populations. If the cumulative number of exposed agents in a zone over an \( n \)-day period exceeds a predefined threshold, \( E_h^k \), the vector population in the affected zone's patches is reduced by \( m\% \). This reduction represents interventions such as mosquito spraying and the elimination of breeding sites. Fig. \ref{fig:mosquito-control} illustrates a case where \( n = 7 \), \( E_h^k = 2 \), and \( m = 75 \). 

% \begin{figure}[h]
%     \centering
%     \includegraphics[width=1\linewidth]{Figures/fig17.png}
%     \caption{The infected vector count changes over 2 months in an uncontrolled and controlled environment.}
%     \label{fig:mosquito-control}
% \end{figure}

\begin{figure}[h]
    \centering
    \begin{subfigure}[t]{0.49\linewidth}
        \centering
        \includegraphics[width=\linewidth]{Figures/fig25-a.png}
        \caption{}
        \label{uncontrolled}
    \end{subfigure}
    \hfill
    \begin{subfigure}[t]{0.49\linewidth}
        \centering
        \includegraphics[width=\linewidth]{Figures/fig25-b.png}
        \caption{}
        \label{controlled}
    \end{subfigure}
    \caption{The infected vector count changes over 2 months in (a) an uncontrolled and (b) controlled environment.}
    \label{fig:mosquito-control}
\end{figure}

Fig. \ref{fig:vb-controling} illustrates the mosquito control strategy under different scenarios. As the infection count increases, the number of exposed individuals in each zone also rises. At the end of each week, zones with more than n exposed agents undergo vector reduction. However, when higher thresholds for exposed counts are allowed, it becomes evident that this strategy is not significantly different from the uncontrolled case.

In contrast, when control measures are implemented even for a small number of recorded cases, the strategy proves more effective. However, this comes with an economic trade-off. By using AVSim, it is possible to determine the optimal threshold for recorded cases to trigger control measures and the extent of control needed.

In the given example, while 90\% control is the most effective, applying 75\% control can also significantly reduce mosquito populations and, consequently, dengue cases.

\begin{figure}
    \centering
    \includegraphics[width=1\linewidth]{Figures/fig26.png}
    \caption{The effect of controlling mosquitoes}
    \label{fig:vb-controling}
\end{figure}

\subsubsection{Effect of temperature}

As mentioned in Section \ref{method:vb-disease-propagation}, the average rate of progression of vectors from the exposed state to the infectious state depends on the mosquito's incubation time. The mosquito incubation time is highly influenced by temperature. Fig. \ref{fig:vb-temp} shows how the total infectious agent count changes as the mosquito incubation period varies with temperature. During high-temperature seasons, it is important to implement necessary measures to reduce the propagation of dengue.

\begin{figure}
    \centering
    \includegraphics[width=1\linewidth]{Figures/fig27.png}
    \caption{Effect of temperature on dengue disease progression}
    \label{fig:vb-temp}
\end{figure}


\section*{Funding}
This research was funded by the International Development Research Centre, Canada, grant number 109586-001.

\section*{Declaration of competing interest}
The authors declare that they have no known competing financial interests or personal relationships that could have appeared to influence the work reported in this paper.

\section{Conclusion}

The emerging threat of air-borne and vector-borne diseases has existed forever in human history. Recent events such as the COVID-19 pandemic and the continuous spread of dengue fever are key examples of this. Some countries have faced severe consequences since they were not able to manage the spread of the disease by executing effective policies. In our study, we present AVSim, an agent-based model designed to assist policymakers in making more informed decisions for the effective control of diseases by taking in to account numerous possible scenarios.

AVSim sets itself apart by its ability to model complex real-world environments, a feature many other agent-based models lack. It can precisely simulate entire cities, subdivisions, zones, and even individual households. By leveraging demographic data of the modeled geographical area, AVSim generates agent entities that accurately represent the population.

The movement of these agents is modeled using a novel probability-based approach informed by real-world GPS data collected from 100 individuals across various occupational classes. This innovative method effectively replicates human behavior and has been rigorously validated through an extensive study of human motion patterns.

Additionally, AVSim incorporates a realistic transport network, enabling agents to move via modeled transportation systems. It can simulate infections based on professional classes and supports a variety of analyses to study disease dynamics. Together, these capabilities place AVSim in a class of its own, far surpassing the limitations of other agent-based models.

The results from AVSim enable users to identify vulnerable agent groups and high-risk locations. Preventive measures, such as promoting safe hygiene practices and implementing vaccination campaigns, can then be simulated to assess their impact on mitigating disease spread. These simulations provide valuable insights, helping policymakers make informed decisions and implement strategies to control disease transmission while minimizing societal and economic disruption in the region.

% \plocations. This will further rintcredits
\printcredits

\bibliographystyle{model1-num-names}
\bibliography{paper-refs}

\end{document}

