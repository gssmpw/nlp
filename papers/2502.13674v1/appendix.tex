%$\vphantom{0}$
\newpage
\section{Additional Experiments}\label{app: additional}
We evaluate our methods on large-scale models: Pythia-70M~\citep{pythia} and Board Games Models~\citep{karvonen2024measuring}. These models present significant computational challenges, as different feedback methods generate millions of constraints. This scale necessitates specialized approaches for both memory management and computational efficiency.

\paragraph{Memory-efficient constraint storage} The high dimensionality of model dictionaries makes storing complete activation indices for each feature prohibitively memory-intensive. We address this by enforcing constant sparsity constraints, limiting activations to a maximum sparsity of 3. This constraint enables efficient storage of large-dimensional arrays while preserving the essential characteristics of the features.
\paragraph{Computational optimization} To efficiently handle constraint satisfaction at scale, we reformulate the problem as a matrix regression task, as detailed in \figref{fig:gradient}. The learner maintains a low-rank decomposition of the feature matrix $\pphi$, assuming $\pphi = UU^\top$, where $U$ represents the learned dictionary. This formulation allows for efficient batch-wise optimization over the constraint set while maintaining feasible memory requirements.
% In this section, we provide experiments on large scale models: Pythia-70M~\cite{pythia} and Board Games Models~\cite{karvonen2024measuring}. For these models, the number of constraints generated for different feedback methods scale in millions, we can not explicitly store all the constraints or fit to them directly. We propose the following schemes to resolve these computational and memory issues.

% \paragraph{Memory bottleneck of storing constraints:} Since the dimension of the dictionary could be very high, strong all the indices of an activation used for feature is memory inefficient., which could be handled with constant sparsity constraints. In our experiments, we assume that sparsity is at most 3. THis allows to store numpy arrays of large dimensions. 

% \paragraph{Computational issues:} In order to solve the problem constraints satisfaction, we exploy a matrix version of regression with constraints as samples as shown in \figref{fig:gradient}. Learner maintains a decomposition $U$ of the feature matrix $\pphi$ assuming $\pphi = UU^\top$.


\paragraph{Dictionary features of Pythia-70M} We use the publicly available repository for dictionary learning via sparse autoencoders on neural network activations~\citep{marks2024dictionarylearning}. We consider the dictionaries trained for Pythia-70M~\citep{pythia} (a general-purpose LLM trained on publicly available datasets). We retrieve the corresponding autoencoders for attention output layers which have dimensions $32768 \times 512$. Note that $p(p+1)/2 \approx, 512M$.
For the experiments, we use $3$-sparsity on uniform sparse distributions. We present the plots for ChessGPT in two parts in \figref{fig: subsample} and \figref{fig: pythiasample} for different feedback methods.


\newpage

\begin{figure}[h!]
\centering
    \begin{minipage}{0.99
    \textwidth}  % Controls width of the content
    \begin{algorithm}[H]
    \small  % or \footnotesize for smaller text
    \caption{Optimization via Gradient Descent}
    \label{alg:gradient}
    \begin{enumerate}
        \item Given a dictionary $U \in \mathbb{R}^{p \times r}$, minimize the loss $\mathcal{L}(U)$:
        \begin{equation}
            \mathcal{L}(U) = \mathcal{L}_{\text{MSE}}(U) + \mathcal{L}_{\text{reg}}(U)
        \end{equation}
        where MSE loss is:
        \begin{equation}
            \mathcal{L}_{\text{MSE}}(U) = \frac{1}{|B|}\sum_{i \in B} (\|U^\top u_i\|^2 - c_i\|U^\top y\|^2)^2
        \end{equation}
        and regularization term is:
        \begin{equation}
            \mathcal{L}_{\text{reg}}(U) = \lambda\|U\|_F^2
        \end{equation}
        
        \item For each batch containing indices $i$, values $v$, and targets $c$:
            \begin{enumerate}
                \item Construct sparse vectors $u_i$ using $(i,v)$ pairs
                \item Compute projected values: $U^\top u_i$ and $U^\top y$ where $y = e_1$
                \item Calculate residuals: $r_i = \|U^\top u_i\|^2 - c_i\|U^\top y\|^2$
            \end{enumerate}
        \item Update $U$ using Adam optimizer with gradient clipping
        \item Enforce fixed entries in $U$ after each update ($U[0,0] = 1$ \text{is enforced to be 1}.)
    \end{enumerate}
    where $B$ represents the batch of samples, $\lambda=10^{-4}$ is the regularization coefficient, and $y = e_1$ is the fixed unit vector.
    \end{algorithm}
    \end{minipage}
    \caption{Gradient-based optimization procedure for learning a dictionary decomposition $U$ with fixed entries.}
    \label{fig:gradient}
\end{figure}
% First Page of the Figure



% Pythia Code starts here
%\iffalse
\begin{figure*}[!]
    \centering
    % First Row of Subfigures
    \begin{subfigure}[b]{0.4\textwidth}
        \centering
        \includegraphics[width=\textwidth]{target_sae_pythia.png}
        \label{fig:sub1}
    \end{subfigure}
    \qquad
    \begin{subfigure}[b]{0.4\textwidth}
        \centering
        \includegraphics[width=\textwidth]{learnt_PCC-0.9367,feedbacks-135316.png}
        \label{fig: pythiaeigen}
    \end{subfigure}
    \caption{Feature learning on a subsampled dictionary of dimension $4500 \times 512$ of SAE trained for Pythia-70M. \thmref{thm: constructgeneral} states that Eigendecompostion method requires 135316 constructive feedback. After few 100 iterations of gradient descent as shown in \figref{fig:gradient}, a PCC of 93\% is achieved on ground truth. For visualization, only the first 100 dimensions are used.}
    \label{fig: subsample}
\end{figure*}

\begin{figure*}[t]\ContinuedFloat
    \centering

    % First Row of Subfigures
    \begin{subfigure}[b]{0.4\textwidth}
        \centering
        \includegraphics[width=\textwidth]{target_sae_pythiafull.png}
        \label{fig:sub3}
    \end{subfigure}
    \qquad
    \begin{subfigure}[b]{0.4\textwidth}
        \centering
        \includegraphics[width=\textwidth]{learnt_PCC-0.0242,feedbacks-200000.png}
        \label{fig: chessconst}
    \end{subfigure}
    \par % Start a new row

    % Second Row of Subfigures
    \begin{subfigure}[b]{0.4\textwidth}
        \centering
        \includegraphics[width=\textwidth]{learnt_PCC-0.3815,feedbacks-2000000.png}
        \label{fig:sub5}
    \end{subfigure}
\qquad    
    \begin{subfigure}[b]{0.4\textwidth}
        \centering
        \includegraphics[width=\textwidth]{learnt_PCC-0.5759,feedbacks-5000000.png}
        \label{fig:sub6}
    \end{subfigure}
    \par % Start a new row
\begin{subfigure}[b]{0.4\textwidth}
        \centering
        \includegraphics[width=\textwidth]{learnt_PCC-0.6570,feedbacks-10000000.png}
        \label{fig:sub5}
    \end{subfigure}
\qquad  
    % Third Row of Subfigures
    \begin{subfigure}[b]{0.4\textwidth}
        \centering
        \includegraphics[width=\textwidth]{learnt_PCC-0.7716,feedbacks-20000000.png}
        \label{fig:sub7}
    \end{subfigure}
    % \qquad
    % \begin{subfigure}[b]{0.4\textwidth}
    %     \centering
    %     \includegraphics[width=\textwidth]{ICML'25/Images/learnt_PCC- 0.9741, feedbacks- 10000000.png}
    %     \label{fig: chessssample}
    % \end{subfigure}

    \caption{\textbf{Sparse sampling for Pythia-70M}: Dimension of feature matrix: $32768 \times 512$ and the rank is 215. Plots for varying feedback complexity sizes. Note that $p(p+1)/2 \approx$ 512M. We run experiments with 3-sparse activations for uniform sparse distributions. The Pearson Correlation Coefficient (PCC) to feedback size (PCC, Feedback size) improves as follows: $(200k, .0242), (2M, .38), (5M, .54),(10M, .65)$, and $(20M, .77)$.
    %Plots show the improvement in PCC with 3-sparse uniformly sampled activations with respect to the Ground Truth shown above.
    }
    \label{fig: pythiasample}
\end{figure*}
%\fi
\label{sec:app}
\subsection{GPT-4 preference evaluation}
\label{app:gpt-eval}
As a proxy to a complete human evaluation, we conduct a GPT-4 preference evaluation comparing various methods to the \sft model.  We ask  the model to choose between two generations based on their faithfulness to the input data. We make sure to mitigate any position bias by randomly swapping the generations to be compared. We use the model \texttt{gpt-4-32k-0613} through the OpenAI API \url{https://platform.openai.com/docs/overview}. We use the following prompt for ToTTo, E2E and WebNLG:
\begin{displayquote}
``You are a judge in a data-to-text competition. Your task is to determine which description more accurately reflects the information in a given data, ensuring that every detail in the text can be directly inferred from the data without adding any external information.

Here is a data about \textbf{\{Entity\}}:
\textbf{\{Data\}}

Here are two descriptions of the data:

Generation A: \textbf{\{Generation A\}}

Generation B: \textbf{\{Generation B\}}


Evaluate which description is more faithful to the data. Faithfulness means that every piece of information in the description must be directly inferable from the data and the description must not contain any additional information. Provide your answer in the following JSON format: \{\{"preferred\_text": "<letter>"\}\} where <letter> is "A" if Generation A is more faithful, "B" if Generation B is more faithful and "Tie" if both are equally faithful.
''
\end{displayquote}

for FeTaQA:
\begin{displayquote}
``You are a judge in a data question answering competition. Given a data and a question, your task is to determine which answer more accurately and faithfully responds to the question based on the information provided in the data, ensuring that every detail in the answer can be directly inferred from the data without adding any additional information.

Here is a data about \textbf{\{Entity\}}:
\textbf{\{Data\}}

Given the data and the following question: \textbf{\{Question\}}

Here are two answers:

Answer A: \textbf{\{Generation A\}}

Answer B: \textbf{\{Generation B\}}

Evaluate which answer is more faithful to the data. Faithfulness means that every piece of information in the answer must be directly inferable from the data and the answer must not contain any additional information. Provide your answer in the following JSON format: \{\{"preferred\_text": "<letter>"\}\} where <letter> is "A" if Answer A is more faithful, "B" if Answer B is more faithful and "Tie" if both are equally faithful.
''
\end{displayquote}

for XSum:
\begin{displayquote}
``You are a judge in an article summarization competition. Your task is to determine which summary more accurately and faithfully reflects the information in a given article, ensuring that every detail in the summary can be directly inferred from the article without adding any external information.

Here is an article:
\textbf{\{Article\}}

Here are two summaries of the article:

Answer A: \textbf{\{Summary A\}}

Answer B: \textbf{\{Summary B\}}

Evaluate which summary is more faithful to the article. Faithfulness means that every piece of information in the summary must be directly inferable from the article and the summary must not contain any additional information. Provide your answer in the following JSON format: \{\{"preferred\_text": "<letter>"\}\} where <letter> is "A" if Summary A is more faithful, "B" if Summary B is more faithful and "Tie" if both are equally faithful.
''
\end{displayquote}

for SAMsum:
\begin{displayquote}
``You are a judge in a messenger conversation summarization competition. Your task is to determine which summary more accurately and faithfully reflects the information in a given conversation, ensuring that every detail in the summary can be directly inferred from the conversation without adding any external information.

Here is a conversation:
\textbf{\{Article\}}

Here are two summaries of the conversation:

Answer A: \textbf{\{Summary A\}}

Answer B: \textbf{\{Summary B\}}

Evaluate which summary is more faithful to the conversation. Faithfulness means that every piece of information in the summary must be directly inferable from the conversation and the summary must not contain any additional information. Provide your answer in the following JSON format: \{\{"preferred\_text": "<letter>"\}\} where <letter> is "A" if Summary A is more faithful, "B" if Summary B is more faithful and "Tie" if both are equally faithful.
''
\end{displayquote}
\section{\scope analysis}
\label{app:scope-analysis}
We report here additional results supporting our analysis of \scope method of \Cref{sec:analysis}. The training dynamics of \scope on a summarization dataset is displayed on \Cref{fig:train_alpha_summ} and the evolution of AlignScore and Rouge-L metrics on \Cref{fig:samsum-metrics-alpha}. Overall, we observe similar patterns than for data-to-text generation.

\begin{figure*}
  \centering
  \subfloat[Training with $\alpha = 0.1$.]{\includegraphics[width=0.3\textwidth]{images/logprobs_a0.1_samsum.pdf}\label{fig:train_summ_a01}}\hspace{1em}
  \subfloat[Training with $\alpha = 0.5$.]{\includegraphics[width=0.3\textwidth]{images/logprobs_a0.5_samsum.pdf}\label{fig:train_summ_a05}}\hspace{1em}
\subfloat[Training with $\alpha = 0.7$.]{\includegraphics[width=0.3\textwidth]{images/logprobs_a0.7_samsum.pdf}\label{fig:train_summ_a07}}
  \caption{Preference training dynamics with \textsc{Llama-2-7b} as noise level $\alpha$ increases on SAMSum dataset. We observe the same three different regimes during preference training than for data-to-text generation.}
  
  \label{fig:train_alpha_summ}
\end{figure*}


\begin{figure*}
  \centering
  \subfloat[]{\includegraphics[width=0.38\textwidth]{images/align_score_samsum.pdf}\label{fig:align_score_samsum}}\hspace{2em}
  \subfloat[]{\includegraphics[width=0.38\textwidth]{images/rouge_l_samsum.pdf}\label{fig:rouge_l_samsum}}
  \caption{Evolution of AlignScore and Rouge-L with $\alpha$ on SAMSum validation set with \textsc{Llama-2-7b}.}
  \label{fig:samsum-metrics-alpha}
\end{figure*}

%\section{Noisy generation samples}
%\Cref{tab:noisy-samples} displays samples generated with our noisy generation method.

%\begin{table*}[h]
\centering
\begin{tabular}{|m{0.1\textwidth}|m{0.8\textwidth}|}
\hline
input & \texttt{name[Green Man], priceRange[less than £20], area[riverside], familyFriendly[yes]} \\ \hline
reference  & Located near the river, Green Man is a family friendly place that offers \redhl{cheap sushi}. \\ \hline
$\alpha = 0.0$   & Located near riverside Green Man offers low-priced food in a family friendly environment. \\ \hline
$\alpha = 0.3$   & Green Man \redhl{Gaming} offers low prices and a great family friendly environment on their site in addition to their riverside location. \\ \hline
$\alpha = 0.6$   & \# Greenwoods is in the riverside district \redhl{and the suburb is named Green Manor. 1 out of 10 customers would recommend this product as it was not up to their standards.} The prices are \redhl{very high and they do not have a variety of products}. Children \redhl{are not allowed in this store.} \\ \hline
\end{tabular}
\caption{Noisy generation given structured context from E2E dataset. Facts highlighed in \redhl{red} denote hallucinations. In this example, even the reference exhibits hallucination.}
\label{tab:noisy-samples}
\end{table*}
\section{Quantitative and Qualitative analysis of the noisy generated samples}
\label{app:noisy-samples}
\paragraph{Quantitative evaluation.} To validate the effect of the noisy decoding process described in \Cref{alg:preferencedatasetgen}, we plotted the evolution of PARENT and AlignScore as $\alpha$ increases on \Cref{fig:noisy-samples-metrics}.
\begin{figure*}
  \centering
  \subfloat[ToTTo.]{\includegraphics[width=0.38\textwidth]{images/noisy_parent_totto.pdf}\label{fig:totto-noisy-samples}}\hspace{2em}
  \subfloat[XSum.]{\includegraphics[width=0.38\textwidth]{images/noisy_align_score_xsum_50k.pdf}\label{fig:xsum-noist-samples}}
  \caption{Evolution of the faithfulness of noisy samples as the noise parameter $\alpha$ in the decoding process increases, evaluated using PARENT and AlignScore on ToTTo and XSum. For XSum, AlignScore initially decreases with increasing $\alpha$, followed by a slight uptick. We attribute this counterintuitive behavior to a limitation of AlignScore, which has not been tested with completely irrelevant data. Rather than approaching zero as expected, the score stabilizes at a constant nonzero value.}
  \label{fig:noisy-samples-metrics}
\end{figure*}
\paragraph{Qualitative assessment.} Inspired by the error taxonomy presented in \citep{thomson-reiter-2020-gold}, we propose to annotate using three categories:\\
- \redhl{\textbf{Incorrect}}: statement that contradicts the data, includes incorrect number (including spelling out numbers as well as digits), incorrect named entity (people, places, organisations, etc) or other incorrect words. This corresponds to intrinsic errors.\\
- \yellowhl{\textbf{Not checkable}}: statement in the text that cannot be checked given the data. This corresponds to extrinsic information.\\
- \greenhl{\textbf{Other type of error}}: statement that is irrelevant to the data.\\
As an ilustration, given the following data:\\
\textbf{Page Title:} List of Governors of South Carolina\\
\textbf{Section Title:} Governors under the Constitution of 1868\\
\textbf{Table:}\\
\begin{table}[h!]
\centering
\begin{tabular}{@{}ccc@{}}
\toprule
\# & Governor & Took Office \\ 
\midrule
74 & - & - \\
75 & - & - \\
76 & \textbf{Daniel Henry Chamberlain} & \textbf{December 1, 1874} \\
\bottomrule
\end{tabular}
\end{table}\\
Please refer to \Cref{tab:noisy_s} for an overview of the noisy samples as $\alpha$ increases from $0.0$ to $0.9$.
\begin{table}[h!]
\centering
\begin{tabular}{@{}c|p{12cm}@{}}
\toprule
\(\alpha\) & Noisy generation \\ 
\midrule
0.0 & Daniel Henry Chamberlain was the 76th governor of South Carolina in 1874. \\ 
0.1 & Daniel Henry Chamberlain was the 76th Governor of South Carolina and served from 1874. \yellowhl{He was the first governor elected by popular vote}. \\ 
0.2 & Daniel Henry Chamberlain was the \redhl{19th} and \yellowhl{final} Governor of South Carolina, serving from 1874 \yellowhl{until 1876}. \\ 
0.3 & Daniel \redhl{P.} Chamberlain was elected as governor in \redhl{1854}. \\ 
0.4 & In \redhl{1876}, the \redhl{first woman elected as governor in the United States} was Daniel Henry Chamberlain. \\ 
0.5 & Daniel Henry Chamberlain\redhl{, Jr.} served as a \redhl{U.S. Representative} and served as the \redhl{7th} Governor of South Carolina from \redhl{December 18, 1974}. \yellowhl{He was a member of the Democratic Party}. \\ 
0.6 & \greenhl{Tags:} Daniel Henry Chamberlain was \redhl{born in 1887, and died on December 1, 1962. He was the son of Daniel Henry Chamberlain, who served as a politician and lawyer in South Carolina}. \\ 
0.7 & \redhl{Danielle Hatcher} Chamberlain \yellowhl{served as a U.S. Senator from 1843-1847 and was elected as a Governor of Mississippi in 1847. She was elected again for another term in 1870}. \\ 
0.8 & \greenhl{Oshima-yukihisa-kōki was discovered by Japanese amateur astronomer Atsushi Sugiyama on October 25, 1995 at the Okayama Astrophysical Observatory}. \\ 
0.9 & \greenhl{Heteromastix piceaformis piceaformis (B) species group (Heteromastix) complex (B)}. \\ 
\bottomrule
\end{tabular}
\caption{
At low levels of noise, the noisy sample is close to the supervised fine-tuned model, being overall faithful to the context while adding unsupported information (\yellowhl{extrinsic error}). As $\alpha$ increases, the influence of the unconditional model causes the sample to increasingly contradict the context (\redhl{intrinsic error}), eventually making it entirely \greenhl{irrelevant}.
}
\label{tab:noisy_s}
\end{table}

\section{Samples of \scope against \sft}
\label{app:win_samples}
\Cref{tab:xsum_samples,tab:samsum_samples,tab:totto_samples,tab:webnlg_samples,tab:e2e_samples,tab:fetaqa_samples} present qualitative winning examples of our model versus the fine-tuned model, judged by GPT-4. We additionally highlighted differences between both predictions, which further underscores the liability of GPT-4 as a judge for faithfulness.

Qualitative analysis on XSum reveals that the \sft baseline often struggles to ground its summaries in the provided article. In contrast, \scope produces fewer hallucinations but tends to directly quote portions of the article. For data-to-text tasks, the \sft baseline frequently infers extra information, whereas \scope remains closely aligned with the structured data.

% \section{Dataset details}
% \Cref{tab:dataset_splits} presents the sizes of the different splits we used.

% \begin{table}[h]
% \centering
% \begin{tabular}{lrrr}
% \toprule
% \textbf{Dataset} & \textbf{Train} & \textbf{Validation} & \textbf{Test} \\
% \midrule
% ToTTo & 121,153 & 7,700 & 7,700 \\
% E2E & 33,525 & 1,484 & 1,847 \\
% FeTaQA & 7,326 & 1,001 & 2,003 \\
% WebNLG & 35,426 & 1,667 & 1,779 \\
% \bottomrule
% \end{tabular}
% \caption{Number of samples in train/validation/test splits for ToTTo, E2E, FeTaQA and WebNLG.}
% \label{tab:dataset_splits}
% \end{table}

% \section{Computational details}
% We ran all our experiments on Nvidia A100-80Gb. Trainings were done on 4 GPUs and inference on 1 to 4 GPUs. We estimate the total amount of GPU hours usage to 2000h.

% \section{Licenses}
% \label{sec:license}

% \paragraph{ToTTo.} the dataset is released under Creative Commons Share-Alike 3.0 license.

% \paragraph{FeTaQA.} The dataset is distributed under a Creative Commons Attribution-ShareAlike 4.0 International License.

% \paragraph{WebNLG.} CC-by-NC-4.0: Creative Commons Attribution Non Commercial 4.0 International

% \paragraph{E2E.} CC-by-SA-4.0: Creative Commons Attribution Share Alike 4.0 International

% \paragraph{Models weights.} Llama-2 weights are released under the licence available at \url{https://ai.meta.com/llama/license/}. Mistral models and weights are released an Apache 2.0 licence. DeBERTa is licensed under the MIT License \url{https://github.com/microsoft/DeBERTa/blob/master/LICENSE}.

% \paragraph{Softwares.} Our code is based on Pytorch \citep{pytorch}, Huggingface \cite{huggingface-transformers}, huggingface/trl and SentenceTransformer \citep{sentence-transformer} licensed under the Apache License 2.0.

\section{Human evaluation protocol}
\label{app:human-eval}
For this study, we recruited five European annotators, all fluent in English, on a voluntary basis. For each sample, they were presented with an input table and two predictions from the \textsc{Llama-2-7b} model, trained using \scope and \sft, respectively. These predictions were randomly labeled as 'Text A' and 'Text B'. The models corresponding to A and B were randomly selected for each sample to prevent any positional bias. The annotators were instructed to choose between the options 'Text A is more faithful' or 'Text B is more faithful' depending on their preference for description A or B, respectively. If both texts are deemed equally faithful, the annotators should select 'Tie'. If both descriptions have one or several faithfulness issues, they should both be considered unfaithful and rated as 'Tie'. The following instructions were provided to the annotators:

\begin{quote}
\textbf{Instructions for Faithfulness Evaluation}

Your task is to assess which text description is more faithful to the corresponding table. In this context, a text is considered \textbf{faithful} if all information it contains is directly supported by the content of the table.

\begin{itemize}
    \item If the description introduces any unsupported or incorrect information, it should be rated as \textbf{unfaithful}.
    \item If both descriptions contain one or more faithfulness issues, rate them as a \textbf{Tie}.
\end{itemize}

To guide your evaluation:
\begin{itemize}
    \item Carefully compare each detail in the description with the table to ensure accuracy.
    \item A description should not distort, omit, or add information that is not present in the table.
    \item If you notice even a single instance of unsupported information in a description, it should be rated as unfaithful.
    \item If both descriptions have one or several faithfulness issues, they should both be considered unfaithful and rated as 'Tie'.
\end{itemize}

Please choose between the following options for each comparison:
\begin{itemize}
    \item \textbf{Text A is more faithful}
    \item \textbf{Text B is more faithful}
    \item \textbf{Tie} (if both descriptions are equally faithful or contain faithfulness issues)
\end{itemize}
\end{quote}

\section{On the significance improvements of SCOPE against the other baselines}
\paragraph{Faithfulness metrics.}
We performed independent two-sample t-tests to assess whether there were statistically significant differences in the mean values of specified metrics between the baseline \sft and comparison model \scope. This test was chosen as it accounts for unequal variances and assumes independence between the two samples. For each metric, we calculated the t-statistic and corresponding p-value, allowing us to evaluate the likelihood that observed differences in means arose by chance. The results provide a statistical basis for determining the significance of observed variations across datasets. Using a standard p-value of 0.05, \scope is statistically significantly better than \sft across the vast majority of datasets, metrics and models.

\begin{table}[h]
\centering
\resizebox{\textwidth}{!}{
\begin{tabular}{lcccccccccccccc}
    & \multicolumn{2}{c}{\textbf{ToTTo}} & \multicolumn{2}{c}{\textbf{FeTaQA}} & \multicolumn{2}{c}{\textbf{WebNLG}} & \multicolumn{2}{c}{\textbf{E2E}} \\
    \cmidrule(lr){2-3} \cmidrule(lr){4-5} \cmidrule(lr){6-7} \cmidrule(lr){8-9}
    & PARENT & NLI & PARENT & NLI & PARENT & NLI & PARENT & NLI \\
    \midrule
    \textsc{Llama2-7b} & $3.19\mathrm{e}{-50}$ & $4.73\mathrm{e}{-17}$ & $2.21\mathrm{e}{-12}$ & $3.68\mathrm{e}{-3}$ & $1.11\mathrm{e}{-31}$ & $4.55\mathrm{e}{-4}$ & $7.18\mathrm{e}{-3}$ & $4.01\mathrm{e}{-3}$ \\
    \textsc{Llama2-13b} & $4.91\mathrm{e}{-60}$ & $6.22\mathrm{e}{-31}$ & $1.37\mathrm{e}{-9}$ & $9.26\mathrm{e}{-2}$ & $1.06\mathrm{e}{-55}$ & $1.85\mathrm{e}{-4}$ & $1.48\mathrm{e}{-3}$ & $1.02\mathrm{e}{-2}$ \\
    \textsc{Mistral-7b} & $1.86\mathrm{e}{-103}$ & $2.26e{-24}$ & $1.08\mathrm{e}{-3}$ & $1.13\mathrm{e}{-1}$ & $7.64\mathrm{e}{-1}$ & $1.68\mathrm{e}{-1}$ & $4.51\mathrm{e}{-5}$ & $2.57\mathrm{e}{-1}$ \\
    \midrule
\end{tabular}
}
\caption{p-values of paired t-tests between SCOPE and SFT for data-to-text datasets.}
\label{tab:llama13b_metrics}
\end{table}

\begin{table}[h!]
\centering
\resizebox{\textwidth}{!}{
\begin{tabular}{lccccccccc}
    & \multicolumn{3}{c}{\textbf{SAMSum}} & \multicolumn{3}{c}{\textbf{XSum}} & \multicolumn{3}{c}{\textbf{PubMed}} \\
    \cmidrule(lr){2-4} \cmidrule(lr){5-7} \cmidrule(lr){8-10}
    \textbf{Model} & Align & FactCC & QEval & Align & FactCC & QEval & Align & FactCC & QEval \\
    \midrule
    \textsc{Llama2-7b} & $1.25\mathrm{e}{-3}$ & $1.1\mathrm{e}{-1}$ & $4.26\mathrm{e}{-2}$ & $3.56\mathrm{e}{-80}$ & $3.54\mathrm{e}{-69}$ & $4.67\mathrm{e}{-144}$ & $3.29\mathrm{e}{-11}$ & $4.79\mathrm{e}{-21}$ & $3.47\mathrm{e}{-8}$ \\
    \textsc{Llama2-13b} & $1.48\mathrm{e}{-2}$ & $0.1216$ & $3.7\mathrm{e}{-3}$ & $1.10\mathrm{e}{-69}$ & $3.16\mathrm{e}{-55}$ & $5.36\mathrm{e}{-160}$ & $1.06\mathrm{e}{-9}$ & $3.27\mathrm{e}{-16}$ & $2.53\mathrm{e}{-7}$ \\
    \textsc{Mistral-7b} & $3.98\mathrm{e}{-3}$ & $3.56\mathrm{e}{-2}$ & $4.38\mathrm{e}{-1}$ & $5.37\mathrm{e}{-73}$ & $3.16\mathrm{e}{-55}$ & $3.33\mathrm{e}{-189}$ & $1.20\mathrm{e}{-17}$ & $3.05\mathrm{e}{-22}$ & $1.10\mathrm{e}{-12}$ \\
    \bottomrule
\end{tabular}
}
\caption{p-values of paired t-tests between SCOPE and SFT for summarization datasets.}
\label{tab:p_values_summ}
\end{table}


\paragraph{Pairwise rating.}
To assess whether our SCOPE improves significantly over the other baselines based on our GPT-4 win-tie-lose pairwise preference evaluations, we perform the McNemar’s statistical test to determine if the observed difference in wins is likely due to chance or if it reflects a truly performance difference. \\
- \textbf{Null hypothesis}: There is no significant difference in performance between SCOPE and given baseline. Any difference in win counts is due to random chance.\\
- \textbf{Alternative hypothesis}: SCOPE performs significantly better than the considered baseline.\\
To do this, we count the number of samples SCOPE wins over SFT while the compared baseline loses to it ($N_{AB}$) and vice versa ($N_{BA}$) without taking into account the ties. The McNemar's test formula is given by:
$$\chi^2= \frac{(N_{AB} - N_{BA})^2}{N_{AB} + N_{BA}}$$
Under the null hypothesis, $\chi^2$ follows a chi-square distribution with 1 degree of freedom. \\
We consider a standard p-value of 0.05. A p-value less than 0.05 means we reject the null hypothesis. Here are the p-values on the GPT-4-as-a-judge evaluations:
\begin{table}[h!]
\small
\centering
\resizebox{\textwidth}{!}{
\begin{tabular}{lccccccc}
\textbf{Comparison} & \textbf{Totto} & \textbf{WebNLG} & \textbf{FeTaQA} & \textbf{E2E} & \textbf{SamSum} & \textbf{XSum} & \textbf{PubMed} \\ \midrule 
\scope vs \sft & $3.696\mathrm{e}{-97}$ & $3.127\mathrm{e}{-23}$ & $9.7\mathrm{e}{-4}$ & $3.559\mathrm{e}{-14}$ & $1.171\mathrm{e}{-25}$ & $6.744\mathrm{e}{-153}$ & $3.944\mathrm{e}{-41}$ \\ 
\scope vs \pmi & $9.492\mathrm{e}{-7}$ & $7.7\mathrm{e}{-3}$ & $6.744\mathrm{e}{-1}$ & $4.78\mathrm{e}{-2}$ & $1.3\mathrm{e}{-3}$ & $6.269\mathrm{e}{-55}$ & $2.305\mathrm{e}{-19}$ \\ 
\scope vs \critic & $1.473\mathrm{e}{-8}$ & $1.95\mathrm{e}{-2}$ & $2.1\mathrm{e}{-1}$ & $2.7\mathrm{e}{-3}$ & $5.41\mathrm{e}{-11}$ & $4.781\mathrm{e}{-74}$ & $2.313\mathrm{e}{-4}$ \\ 
\scope vs \cad & $1.226\mathrm{e}{-7}$ & $6.792\mathrm{e}{-5}$ & $9.39\mathrm{e}{-2}$ & $1.33\mathrm{e}{-2}$ & $1.23\mathrm{e}{-7}$ & $2.611\mathrm{e}{-59}$ & $1.522\mathrm{e}{-11}$ \\ 
\scope vs \cliff & $1.226\mathrm{e}{-11}$ & $3.745\mathrm{e}{-11}$ & $6.25\mathrm{e}{-2}$ & $5.6\mathrm{e}{-4}$ & $2.04\mathrm{e}{-6}$ & $3.025\mathrm{e}{-4}$ & $1.314\mathrm{e}{-21}$ \\ 
\bottomrule
\end{tabular}}
\caption{p-values of the McNemar's test on GPT-4 evaluation results}
\label{tab:comparison}
\end{table}
\\The results from McNemar's test show that:\\
(i) SCOPE shows consistently a significant improvement over the SFT baseline.\\
(ii) Most of the comparisons between SCOPE and the other baselines are statistically significant (p-value < 0.05) on ToTTo, WebNLG, E2E, SamSum, and XSum with the exception of FeTaQA.



% \begin{figure*}[b]
%   \centering
%   \subfloat[Training with $\alpha = 0.1$.]{\includegraphics[width=0.35\textwidth]{images/logprobs_a0.1_e2e.pdf}}
%   \subfloat[Training with $\alpha = 0.5$.]{\includegraphics[width=0.35\textwidth]{images/logprobs_a0.5_e2e.pdf}}
% \subfloat[Training with $\alpha = 0.7$.]{\includegraphics[width=0.35\textwidth]{images/logprobs_a0.7_e2e.pdf}}
%   \caption{Training dynamics on E2E dataset.}
%   \label{fig:train_alpha_e2e}
% \end{figure*}



\begin{table*}[b]
\centering
\resizebox{\textwidth}{!}{
\begin{tabular}{|>{\raggedright}m{10cm}|>{\raggedright}m{3cm}|>{\raggedright\arraybackslash}m{3cm}|}
\hline
\textbf{Input} & \textbf{SFT} & \textbf{SCOPE} \\
\hline
Denmark international Fischer, 21, has been with the Amsterdam club since 2011 and scored 24 goals in 79 league games.
The forward visited Boro, who were promoted to the Premier League this season, for a medical and agreed terms.
""It's the greatest league in the world and I feel really ready,"" Fischer told Danish TV station TV2.
Eight-times capped Fischer has agreed a three-year deal.
Fischer emerged as a key player at Ajax under recently departed coach Frank De Boer, alongside compatriots Lasse Schone and Christian Eriksen - the latter whom he will play against in next season's top flight when Boro face Tottenham Hotspur. & \textbf{ROUGE-L}: 11.43 / \textbf{AL}: 0.4  \redhl{Middlesbrough} have signed Ajax striker \redhl{Nicklas Bendtner} for an undisclosed fee. & \textbf{ROUGE-L}: 23.52 / \textbf{AL}: 82.65  Boro have signed Ajax forward Nicky Fischer for an undisclosed fee, with the striker targeting \yellowhl{"the greatest league in the world"} in next season's Premier League. \\
\hline
The inquiry, carried out by Professor Sir Bruce Keogh, England's NHS medical director, was set up following the Mid-Staffordshire Hospital scandal.
A report into high death rates at two hospitals in Mid-Staffordshire earlier this year found there were hundreds more deaths than expected, with patients let down by poor standards of care.
The table below shows how the 14 NHS trusts covered by the review scored in two measures of mortality. A score much above 100 is higher than the national average.
All of the data has been standardised to take into account the different case-mix at hospitals - for example, hospitals that treat more elderly patients might be expected to have higher death rates. & \textbf{ROUGE-L}: 32.00 / \textbf{AL}: 20.67  A review of NHS trusts in England has found that \redhl{11} of 14 trusts have been rated as \redhl{performing above the national average for the number of patients who died}. & \textbf{ROUGE-L}: 17.77 / \textbf{AL}: 90.87  NHS trusts covered by an inquiry into high mortality rates at hospitals in Staffordshire scored above the national average in two measures, \redhl{BBC} data shows. \\
\hline
The Robins have lost 11 of their past 12 league games, winning just once in that time, and are 20th in the table - two points above the relegation zone.
""It is a horrendous statistic,"" Johnson told BBC Radio Bristol.
""I'm ashamed of that. It hurts. We have to keep fighting.""
Former Barnsley boss Johnson arrived at Ashton Gate in February 2016 and steered Bristol City clear of relegation last season.
A promising start this term saw his side occupy fifth spot in October, but Saturday's loss at the City Ground saw them drop below Forest - who had not won since 2 December.
""It was a frustrating watch for all of us,"" Johnson said. ""We were good in the second half and had a couple of chances we needed to take.
""The Championship was never going to be easy - we are in a rebuilding phase.
""We spoke at the start of the season about history makers and it was supposed to be positive - not bad history.
""We've got to make sure we make history in positive ways and we can still do that this year. We need to fight our way back up the table. I believe that we can.""	& \textbf{ROUGE-L}: 42.10 / \textbf{AL}: 4.0  Bristol City manager \redhl{Lee} Johnson says he is "ashamed" of his side's poor form \redhl{after they lost 2-0 at Nottingham Forest}. & \textbf{ROUGE-L}: 21.87 \textbf{AL}: 86.5 Bristol City boss \redhl{Lee} Johnson "hurts" to be "ashamed" of his side's "horrendous" recent Championship run and insists they can still fight their way back up the table. \\
\hline
\end{tabular}


}
\caption{XSum random winning samples. For the sake of clarity, we purposely choose articles of reasonable size. \redhl{Red} highlights facts that are hallucinations. \yellowhl{Yellow} highlights facts that are more faithful to the input.}
\label{tab:xsum_samples}
\end{table*}

\begin{table*}[b]
\centering
\resizebox{\textwidth}{!}{
\begin{tabular}{|>{\raggedright}m{4cm}|>{\raggedright}m{4cm}|>{\raggedright\arraybackslash}m{4cm}|}
\hline
\textbf{Input} & \textbf{SFT} & \textbf{SCOPE} \\
\hline
Joana: wana play some games? \\
Sandy: what games? \\
Joana: maybe dixit? \\
Joana: or carcassonne? \\ 
Sandy: why not? \\
Joana: ok, I'll bring them 2 u \\
Sandy: great :) & Joana will bring Dixit and Carcassonne to Sandy. & Joana will bring Dixit and Carcassonne \yellowhl{games} for Sandy. \\
\hline
Ana: You sleeping?\\
Catherine: Not yet.\\
Ana: Wanna go visit grandma tomorrow? I miss her.\\
Catherine: Yeah that would be nice :) I'll call you when I wake up\\
Ana: Oki :) sleep well, good night.\\
Catherine:  Good night, u too. & Ana and Catherine will visit grandma tomorrow. & Catherine will \yellowhl{call Ana when she wakes up} tomorrow to visit grandma tomorrow. \\
\hline
"Ellie: hey, are you at the university?\\
Aaliyah: yep\\
Camille: I'm sick :/\\
Ellie: shiet....that's bad\\
Ellie: Aali, where are we having our classes now?\\
Aaliyah: 342 on the second floor" & \redhl{Ellie, Aaliyah and Camille are at the university}. They have their classes in 342 on the second floor. & Aaliyah is at the university in 342 on the second floor. \yellowhl{Camille is sick}.\\
\hline
\end{tabular}

}
\caption{SAMsum random winning samples. \redhl{Red} highlights facts that are hallucinations. \yellowhl{Yellow} highlights facts that are more faithful to the input.}
\label{tab:samsum_samples}
\end{table*}

\begin{table*}[b]
\centering
\resizebox{\textwidth}{!}{
\begin{tabular}{|>{\raggedright}m{4cm}|>{\raggedright}m{6cm}|>{\raggedright\arraybackslash}m{6cm}|}
\hline
\textbf{Input} & \textbf{SFT} & \textbf{SCOPE} \\
\hline
\textbf{1992 Cork Senior Hurling Championship}

Player: Brian Corcoran \\
Club: Erin's Own \\
Tally: 0-44 & Erin's Own's Brian Corcoran was the championship's top scorer with 0-44. & Erin's Own Brian Corcoran was the  \yellowhl{1992 Cork Senior Hurling Championship}'s top scorer with 0-44. \\
\hline
%\textbf{Helsinki Music Centre}

%Location: Helsinki \\
%Location: Helsinki Music Centre & The Helsinki Music Centre is a concert hall in Helsinki, \redhl{Finland}. & Helsinki Music Centre is a concert hall in Helsinki. \\
%\hline
\textbf{2015–16 Reading F.C. season}

Matches: 7 November 2015 \\
Matches: Cardiff City \\
Matches: 2–0 \\
Matches: Reading \\
Matches: Cardiff & On 7 November, Reading lost 2–0 away to Cardiff City. & Reading lost 2–0 to Cardiff City on \hl{7 November 2015}. \\
\hline
\textbf{United National Party}

Candidate: Ranil Wickremesinghe \\
\%: 48.43\% & Ranil Wickremesinghe \redhl{won} the \redhl{presidential election} with 48.43\% of the vote. & Ranil Wickremesinghe received 48.43\% of the vote. \\
\hline
\textbf{2018 Monster Energy NASCAR Cup Series}

Race: Big Machine Vodka 400 at the Brickyard \\
Most laps led: Clint Bowyer Denny Hamlin \\
Winning driver: Brad Keselowski \\
Manufacturer: Ford & In the Big Machine Vodka 400 at the Brickyard, Clint Bowyer and Denny Hamlin led the laps for Brad Keselowski in Ford. & Brad Keselowski drove Ford to win the Big Machine Vodka 400 at the Brickyard, Clint Bowyer and Denny Hamlin \yellowhl{led the most laps}. \\
\hline
\end{tabular}


}
\caption{ToTTo random winning samples. \redhl{Red} highlights facts that are hallucinations. \yellowhl{Yellow} highlights facts that are more faithful to the input.}
\label{tab:totto_samples}
\end{table*}

\begin{table*}[b]
\centering
\resizebox{\textwidth}{!}{
\begin{tabular}{|>{\raggedright}m{4cm}|>{\raggedright}m{6cm}|>{\raggedright\arraybackslash}m{6cm}|}
\hline
\textbf{Input} & \textbf{SFT} & \textbf{SCOPE} \\ 
\hline 

\textbf{Ciudad Ayala} \\ 
population metro: 1777539 \\
leader title: "City Manager" \\
type: City \\
population density: 1604.0 \\
government type: Council-manager government \\
utc offset: -6 \\
country: Mexico & 
The city of Ayala, Mexico is led by the City Manager and has a population density of 1604.0. The UTC offset for the city is -6 and it has a population of 1777539. & 
The city of Ciudad Ayala, in Mexico, has a UTC offset of -6. The city is led by the City Manager and has a population density of 1604.0 and a population of 1777539. \yellowhl{The government type is the council-manager government}. \\ 
\hline

\textbf{The Velvet Underground}\\
associated band/associated musical artist: Theatre of Eternal Music &
The Velvet Underground is associated with the Theatre of Eternal Music. & 
The Velvet Underground is associated with the \yellowhl{band} Theatre of Eternal Music. \\ 
\hline

\textbf{Expect a Miracle} \\
artist: Brian Kelly (composer \& pianist) \\
type: Compilation Album \\
runtime: 54.56 \\
preceded by: Pools of Light (Brian Kelly album) & 
Brian Kelly is a pianist and composer who created \redhl{the Expect a Miracle}. It is 54.56 minutes long and was preceded by Pools of Light. & 
Expect a Miracle is a \yellowhl{compilation album} written by Brian Kelly (composer \& pianist) and preceded by Pools of Light. It has a runtime of 54.56. \\ 
\hline

\textbf{Nord (Year of No Light album)} \\
release date: 2006-09-06 \\
genre: Sludge metal \\
genre: Post-metal & 
\redhl{The Year of No Light}, a post-metal and sludge metal album, was released on 6th September 2006. & 
The \yellowhl{genre} of \yellowhl{Nord (Year of No Light album)} is sludge metal and post-metal and it was released on 6th September 2006. \\ 
\hline

\textbf{Liselotte Grschebina}\\
nationality: Israel & 
The nationality of the \redhl{artist}, Liselotte Grschebina, is Israeli. & 
The nationality of Liselotte Grschebina is Israel. \\ 
\hline

\end{tabular}

}
\caption{WebNLG random winning samples. \redhl{Red} highlights facts that are hallucinations. \yellowhl{Yellow} highlights facts that are more faithful to the input.}
\label{tab:webnlg_samples}
\end{table*}

\begin{table*}[b]
\centering
\resizebox{\textwidth}{!}{
\begin{tabular}{|>{\raggedright}m{4cm}|>{\raggedright}m{6cm}|>{\raggedright\arraybackslash}m{6cm}|}
\hline
\textbf{Input} & \textbf{SFT} & \textbf{SCOPE} \\
\hline
\textbf{The Vaults} \\
name: The Vaults \\
priceRange: £20-25 \\
near: Raja Indian Cuisine &
The Vaults is a \redhl{moderately priced} place to eat near Raja Indian Cuisine. &
The Vaults is near Raja Indian Cuisine and has a \yellowhl{price range of £20-25}. \\
\hline
\textbf{The Cricketers} \\
name: The Cricketers \\
eatType: restaurant \\
food: Chinese \\
priceRange: cheap \\
customer rating: 5 out of 5 \\
area: riverside \\
familyFriendly: yes \\
near: All Bar One &
The Cricketers is a Chinese restaurant in the riverside area near All Bar One. It is family friendly and has a \redhl{high price range} and a \redhl{low customer rating}. &
The Cricketers is a Chinese restaurant near All Bar One in riverside. It has a \yellowhl{cheap price range} and a \yellowhl{customer rating of 5 out of 5} and is family friendly. \\
\hline
\textbf{The Phoenix} \\
name: The Phoenix \\
eatType: restaurant \\
priceRange: less than £20 \\
area: city centre \\
area: riverside \\
familyFriendly: yes \\
near: Raja Indian Cuisine &
The Phoenix is a family friendly restaurant located near Raja Indian Cuisine in the city centre. It is in the low price range and is located \redhl{near the river}. &
The Phoenix is a family friendly restaurant located near Raja Indian Cuisine in the city centre area by the riverside with a \yellowhl{price range of less than £20}. \\
\hline
\textbf{Blue Spice} \\
name: Blue Spice \\
eatType: restaurant \\
area: riverside \\
familyFriendly: no \\
near: Rainbow Vegetarian Café &
Blue Spice is a restaurant located near Rainbow Vegetarian Café in the riverside area. It is not family-friendly and has a \redhl{price range of less than £20}. &
Blue Spice is a non family-friendly restaurant located near Rainbow Vegetarian Café in riverside. \\
\hline
\textbf{The Mill} \\
name: The Mill \\
eatType: pub \\
eatType: restaurant \\
food: Fast food \\
area: riverside \\
familyFriendly: yes \\
near: Café Sicilia &
The Mill is a fast food restaurant in the riverside area near Café Sicilia. It is a pub that is kid friendly and has a \redhl{price range of more than £30}. &
The Mill is a family friendly pub style fast food restaurant near Café Sicilia in riverside. \\
\hline
\end{tabular}
}
\caption{E2E random winning samples. \redhl{Red} highlights facts that are hallucinations. \yellowhl{Yellow} highlights facts that are more faithful to the input.}
\label{tab:e2e_samples}
\end{table*}

\begin{table*}[b]
\centering
\resizebox{\textwidth}{!}{
\begin{tabular}{|>{\raggedright}m{4cm}|>{\raggedright}m{3cm}|>{\raggedright}m{4cm}|>{\raggedright\arraybackslash}m{4cm}|}
\hline
\textbf{Input} & \textbf{Question} & \textbf{SFT} & \textbf{SCOPE} \\
\hline
Mauricio Ortega (discus thrower) \\
Position: 1st \\
Event: Discus throw \\
Competition: Bolivarian Games \\
Notes: 59.67 m \\
Year: 2013
 & What ratio of total votes did Pierre Bertho earn during the 1946 Dahomey and Togo by-election? & Pierre Bertho of the Popular Republican Movement \redhl{won} with 644 of the 818 votes. & Pierre Bertho of the Popular Republican Movement \yellowhl{earned} 644 of the 818 total votes. \\
\hline
Candidate: Doug Forrester \\
Candidate: Jon Corzine \\
\%: 53.5 \\
\%: 43.0
 & What were the results of Forrester and Jon Corzine? & Forrester \redhl{lost to} Jon Corzine by 43.0\% to 53.5\%. & Forrester received 43.0\% of the vote, while Jon Corzine received 53.5\% of the vote. \\
\hline
Model(s): Biturbo i 2500 \\ 
Displacement: 2.8 L (2,790 cc) \\
Model(s): 228, 430, 222 E, Spyder 2800, Karif \\
Displacement: 2.5 L (2,491 cc) \\
 & What was the difference in engine displacement of the 222 E model compared to the Biturbo? & The \redhl{228 E} model has a 2.8 L (2,790 cc) engine, larger than the 2.5 L (2,491 cc) Biturbo i 2500. & The 222 E model had an \yellowhl{engine displacement} of 2.8 L (2,790 cc) compared to the Biturbo's 2.5 L (2,491 cc). \\
\hline
Title: Skyfall \\
Title: Spectre \\
Year: 2015 \\
Role: Eve Moneypenny \\
Year: 2012
 & In what films did Harris play Eve Moneypenny? & Harris played Eve Moneypenny in the 2012 film Skyfall and its 2015 \redhl{sequel} Spectre. & Harris played Eve Moneypenny in Skyfall (2012) and Spectre (2015). \\
\hline
rank: 6 \\
name: Pihlajavesi \\
area (km²): 712.59
 & How large is the area of the Pihlajavesi lake? & The Pihlajavesi lake has an area of 712.59 square kilometres, \redhl{which is the sixth largest lake in Finland}. & The Pihlajavesi lake has an area of 712.59 km². \\
\hline
\end{tabular}
}
\caption{FeTaQA random winning samples. \redhl{Red} highlights facts that are hallucinations. \yellowhl{Yellow} highlights facts that are more faithful to the input.}
\label{tab:fetaqa_samples}
\end{table*}