
\begin{abstract}
Learning tool use from a single human demonstration video offers a highly intuitive and efficient approach to robot teaching. While humans can effortlessly generalize a demonstrated tool manipulation skill to diverse tools that support the same function (e.g., pouring with a mug versus a teapot), current one-shot imitation learning (OSIL) methods struggle to achieve this. A key challenge lies in establishing functional correspondences between demonstration and test tools, considering significant geometric variations among tools with the same function (i.e., intra-function variations). To address this challenge, we propose FUNCTO (\underline{Func}tion-Centric OSIL for \underline{To}ol Manipulation), an OSIL method that establishes function-centric correspondences with a 3D functional keypoint representation, enabling robots to generalize tool manipulation skills from a single human demonstration video to novel tools with the same function despite significant intra-function variations. With this formulation, we factorize FUNCTO into three stages: (1) functional keypoint extraction, (2) function-centric correspondence establishment, and (3) functional keypoint-based action planning. We evaluate FUNCTO against exiting modular OSIL methods and end-to-end behavioral cloning methods through real-robot experiments on diverse tool manipulation tasks. The results demonstrate the superiority of FUNCTO when generalizing to novel tools with intra-function geometric variations. More details are available at \href{https://sites.google.com/view/functo}{https://sites.google.com/view/functo}.
\end{abstract}



% optimization-based action planning



