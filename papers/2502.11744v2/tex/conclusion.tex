\section{Conclusion and Limitations}
\noindent \textbf{Conclusion.}  In this work, we present FUNCTO, a function-centric one-shot imitation learning framework for tool manipulation. At the core of FUNCTO is the idea of functional correspondence using a 3D functional keypoint representation. With such a formulation, FUNCTO generalizes the tool manipulation skill from a single human demonstration video to novel tools with the same function despite significant intra-function variations. Extensive real-robot experiments validate the effectiveness of FUNCTO, outperforming both modular one-shot imitation learning methods and end-to-end behavioral cloning methods. \\

\noindent \textbf{Limitations.}  Despite the promising results, several limitations remain: (1) Functional keypoint visibility and collinearity. Our current implementation assumes that the function point is clearly visible from the camera view. However, this assumption may not always hold, especially when learning from egocentric videos. Additionally, FUNCTO fails when the three functional keypoints are collinear in 3D. That said, such cases are uncommon for everyday tools. (2) State tracking and closed-loop execution. The current pipeline operates in an open-loop manner, which is sensitive to unexpected state changes or external disturbances. Integrating a state tracking module (probably using multiple calibrated cameras) and enabling closed-loop execution would further improve the robustness. (3) Multi-modal function modeling. Functions inherently exhibit multi-modality. For example, the function point on the mug shown in Figure \ref{fig:concept} is used for forward pouring, while points positioned on the left or right sides of the rim can facilitate side pouring. Although FUNCTO is currently limited to imitating a single usage of the function based on a single demonstration, future work will focus on capturing the multi-modality of a function from few-shot human demonstrations with a human-robot interaction system \cite{xiao2024robi}. Further discussions can be found in Appendix H.













