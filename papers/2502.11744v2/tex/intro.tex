\section{Introduction}

% Current Logic:
% 1. Use tool -> human intelligence
% 2. Robots can use tools -> useful
% 3. How to allow robots to use tool -> Recent Imitation Learning
% 4. One-shot imitation learning



The ability to use tools has long been recognized as a hallmark of human intelligence \cite{washburn1960tools}. 
Endowing robots with the same capability holds the promise of unlocking a wide range of downstream tasks and applications \cite{chi2023diffusion, finn2017one, vitiello2023one}. As a step towards this goal, we tackle the problem of one-shot imitation learning (OSIL) for tool manipulation, which involves teaching robots a tool manipulation skill with a single human demonstration video. The objective is to develop an OSIL method capable of \textit{generalizing the demonstrated tool manipulation skill to novel tools with the same function.} Here, ``same function" refers to the robot imitating the demonstrated tool manipulation behavior to accomplish functionally equivalent tasks.
% \textbf{The objective is to develop an OSIL policy capable of generalizing the demonstrated tool manipulation skill to novel tools with the same function. }



% is a highly intuitive and efficient approach to robot teaching.

% significant endeavors have been made on imitation learning (IL) from a variety of perspectives \cite{das2021model, chi2023diffusion, finn2017one, vitiello2023one}. 

% Specifically, we assume (1) a single human demonstration video, (2) no object/task-specific prior knowledge (e.g., 3D object models or manual task constraints), and (3) no in-domain pre-training. 

% figure 1 has two parts:
% upper part: a mug-teapot example, keypoints-triangles-trajectories human-robot
% lower part: functo, octopus, mug center -> different pouring tools

\begin{figure}[th]
  \centering
  \vspace*{-0.1in}
  \begin{tikzpicture}[inner sep = 0pt, outer sep = 0pt]
    \node[anchor=south west] (fnC) at (0in,0in)
      {\includegraphics[height=5.2in,clip=true,trim=0in 0in 0.1in 0in]{imgs/main_figure-5.png}};
  \end{tikzpicture}
    \vspace*{-0.3in}
  \caption{FUNCTO establishes functional correspondences between demonstration and test tools using 3D functional keypoints. With a single human demonstration video, FUNCTO generalizes the demonstrated tool manipulation skill to novel tools, even with significant intra-function geometric variations.}
  \label{fig:concept}
  \vspace*{-0.8cm}
\end{figure}

While humans can effortlessly achieve the objective described above, it remains a non-trivial challenge for robots due to significant geometric variations (e.g., shape, size, topology) among tools supporting the same function (i.e., intra-function variations). 
As shown in Figure \ref{fig:concept}, although both the mug and the teapot support the same function of pouring,  their geometries differ significantly (e.g., the teapot features a long neck and a handle positioned on top of its body). 
% Apparently, to successfully apply OSIL in this case, \textbf{the key challenge lies in establishing accurate correspondences between demonstration and test tools with the same function.} 
Apparently, to successfully apply OSIL in this case, a key challenge lies in establishing functional correspondences between demonstration and test tools.
Previous OSIL methods \cite{vitiello2023one, heppert2024ditto, di2024dinobot, zhu2024vision, li2024okami, biza2023one, zhang2024one} assume that tools supporting the same function share highly similar shapes or appearances. 
Based on this assumption, they establish ``shallow" correspondences through techniques such as keypoint-based pose estimation \cite{vitiello2023one, heppert2024ditto, di2024dinobot}, global point set registration \cite{zhu2024vision, li2024okami}, shape warping \cite{biza2023one}, and invariant region matching \cite{zhang2024one} to align geometrically or visually similar tools. However, this assumption often fails in practice due to large intra-function variations.  
As a result, previous OSIL methods exhibit limited generalization to novel tools. This limitation motivates us to ask: What remains invariant among tools with the same function despite significant intra-function variations? Pioneering studies in cognitive anthropology \cite{washburn1960tools} reveal that humans exhibit highly consistent \textit{behavioral patterns} when using different tools serving the same purpose. For instance, the behavioral pattern of pouring involves approaching the tool (e.g., mug), grasping it, and directing its spout towards the target object (e.g., bowl). This spatiotemporal pattern remains invariant across tools (e.g., mug, teapot, saucepan) with the same function of pouring. 
% This limitation motivates us to ask: \textbf{What remains invariant among tools with the same function, despite significant intra-function variations?} 
% This limitation motivates us to ask: What remains invariant among tools with the same function despite significant intra-function variations?

Inspired by this observation, we propose FUNCTO (\underline{Func}tion-Centric OSIL for \underline{To}ol Manipulation), which emphasizes the functional aspects of tool correspondences over geometric or visual similarities as in previous works. FUNCTO achieves this by establishing function-centric correspondences between demonstration and test tools using a 3D functional keypoint representation. The 3D functional keypoint representation consists of a function point, where the tool interacts with the target (e.g., the spout of a mug); a grasp point, where the tool is held (e.g., the handle of a mug); and a center point, which is the tool's 3D center. By focusing on these three functional keypoints, FUNCTO captures the invariant spatiotemporal pattern among tools supporting the same function while ignoring function-irrelevant geometric details. Specifically, FUNCTO is factorized into three stages: (1) Functional keypoint extraction, which detects functional keypoints and tracks their motions from the human demonstration video; (2) Function-centric correspondence establishment, which transfers functional keypoints from the demonstration tool to the test tool and establishes function-centric correspondences using geometric constraints on the functional keypoints; (3) Functional keypoint-based action planning, which uses the demonstration and test functional keypoints to generate a robot end-effector trajectory for task execution.

We evaluate FUNCTO against existing OSIL methods and behavioral cloning (BC) methods through extensive real-robot experiments on diverse tool manipulation tasks. Leveraging the proposed function-centric approach with 3D functional keypoints, FUNCTO addresses the limitations of previous works that rely solely on geometric or visual similarities and achieves better generalization to novel tools with the same function despite significant intra-function variations. \\

\noindent \textbf{Contribution.} \ The main contribution of this work is a novel formulation of function-centric correspondence using a 3D functional keypoint representation for tool manipulation. This enables robots to generalize tool manipulation skills from a single human demonstration video to novel tools with the same function despite significant intra-function variations.







% , (2) function-centric correspondence establishment, and (3) functional keypoint-based action planning. 
% To validate the effectiveness of FUNCTO, we evaluate it against modular OSIL methods and end-to-end behavioral cloning (BC) methods through extensive real-robot experiments across diverse tool manipulation tasks. 
% The results demonstrate the superiority of FUNCTO when generalizing to novel tools with intra-function variations.






% To address the limitation of previous works, we propose FUNCTO (\underline{Func}tion-Centric OSIL for \underline{To}ol Manipulation), a novel function-centric OSIL method. The key idea is that, instead of solely focusing on geometric or visual similarities as in previous works, FUNCTO emphasizes the functional aspects of tool use by establishing function-centric correspondences between demonstration and test tools using a 3D functional keypoint representation. 

% Specifically, the 3D functional keypoint representation consists of a function point, where the tool interacts with the target (e.g., the spout of a teapot for pouring), a grasp point, where the tool is grasped ( the handle of a teapot for pouring), and a center point, which is the 3D tool center. 


% By focusing on these functional keypoints, FUNCTO captures the invariant spatiotemporal pattern of tool use with the same function. For instance, the pouring motion involves approaching and grasping the tool at the grasp point, and then manipulating the tool to interact with the target at the function point. This pattern remains consistent across different pouring tools










% Pioneering studies in cognitive anthropology \cite{washburn1960tools} reveal that humans exhibit highly consistent \textit{behavioral patterns} across different tools to achieve the same purpose. 




% to establish function-centric correspondences between demonstration and test tools using a 3D functional keypoint representation, enabling the generalization of tool manipulation skills from a single human demonstration to novel tools with the same function.


% Specifically, instead of solely focusing on geometric or visual similarities as in previous works, this work emphasizes the functional aspects of tool correspondences. This means identifying how different tools achieve the same purpose, despite their varying geometries or appearances.






% Pioneering studies in cognitive anthropology \cite{washburn1960tools} reveal that humans exhibit highly consistent \textit{behavioral patterns} across different tools to achieve the same purpose. 
% In the case of pouring illustrated in Figure \ref{fig:concept}, for example, the behavioral pattern can be abstracted into two phases: (1) the hand approaches the tool object (e.g., mug) and grasps its handle, and (2) the tool object is manipulated to interact with the target object (e.g., bowl) with its spout. This spatiotemporal pattern remains invariant across tools (e.g., mug, teapot, saucepan) affording the same function of pouring. 

% Inspired by this observation, we propose FUNCTO (\underline{Func}tion-Centric OSIL for \underline{To}ol Manipulation), a novel function-centric OSIL method. 


% By focusing on these functional keypoints, the method captures the invariant spatiotemporal pattern of tool use. For instance, the pouring motion involves approaching the tool at the grasp point, and then manipulating the tool to interact with the target at the function point. This pattern remains consistent across different pouring tools.


% Instead of solely focusing on geometric or visual similarities as in previous works, this work emphasizes the functional aspects of tool correspondences. This means identifying how different tools achieve the same purpose, despite their varying geometries or appearances.



% Technically, FUNCTO establishes function-centric correspondences between demonstration and test tools with a 3D functional keypoint representation, which consists of a function point, where the tool interacts with the target (e.g., the spout of a teapot for pouring), a grasp point, where the tool is grasped ( the handle of a teapot for pouring), and a center point, which is the 3D tool center






% a grasp point, and a center point. Specifically, the function point is where the tool interacts with the target (e.g., the spout of a teapot for pouring). 






% The proposed function-centric correspondence offers two key advantages: (1) It captures the invariant spatiotemporal pattern among tools supporting the same function while ignoring function-irrelevant geometric details. 
% (2) Each functional keypoint is physically grounded and can be reliably detected from visual observations. 
% With this formulation, FUNCTO is factorized into three stages: (1) functional keypoint extraction, (2) function-centric correspondence establishment, and (3) functional keypoint-based action planning. 
% To validate the effectiveness of FUNCTO, we evaluate it against modular OSIL methods and end-to-end behavioral cloning (BC) methods through extensive real-robot experiments across diverse tool manipulation tasks. 
% The results demonstrate the superiority of FUNCTO when generalizing to novel tools with intra-function variations. \\

% Inspired by this finding, we propose \textbf{FUNCTO}, an OSIL method that approaches the correspondence problem from a function-centric perspective. 

% \noindent \textbf{Contribution.} \ Our main contribution is a novel formulation of function-centric correspondence using a 3D functional keypoint representation for tool manipulation. This formulation enables the generalization of the tool manipulation skill from a single human demonstration video to novel tools with the same function despite significant intra-function variations.

















% Previous OSIL methods \cite{vitiello2023one, di2024dinobot, zhang2024one, zhu2024vision, li2024okami, biza2023one,heppert2024ditto} assume that tools supporting the same functionality share highly similar or identical shapes or appearances. Consequently, they establish ``shallow" correspondences derived from shared visual or geometric features. For instance, Heppert et al.\cite{heppert2024ditto} employ LoFTR\cite{sun2021loftr} to estimate the relative tool pose transformation and warp the demonstrated manipulation trajectory accordingly. Di Palo et al.\cite{di2024dinobot} leverage the off-the-shelf vision foundation model\cite{oquabdinov2} to build semantic correspondences between tools. Zhu et al.\cite{zhu2024vision} and Li et al.\cite{li2024okami} utilize global geometric registration\cite{choi2015robust} to align demonstration and test tools. However, this assumption often fails in practice due to large intra-function geometric variations. As a result, previous OSIL methods exhibit limited generalization to novel tool instances and categories supporting the same functionality.



































