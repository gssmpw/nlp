\section{Related Work}
% We first introduce some existing works about modelling an agent's decision-making process using ToM. We then review two groups of methods that focus on modelling interactive and dynamic humans, respectively. Lastly, we review common types of human trajectory prediction methods used in robot planning.

\subsection{ToM in Agent Modelling}
\label{sec:ToM}
% ToM has been a well-established theory in cognitive psychology \cite{baron1997mindblindness}. 
Prior works have formulated a ToM agent's decision-making processes for basic games such as rock-paper-scissors \cite{de2013much}, Tacit Communication Game \cite{de2015higher}, and Common Pool Resource Game \cite{von2017minds}. 
These extensive-form games have relatively small action sets, making recursive reasoning straightforward by listing down all possible game states. Some methods apply ToM to predict agent's movement \cite{baker2014modeling, rabinowitz2018machine, tian2021learning}, but they are constrained to a grid space with limited action sets. For a more general discussion on ToM in machine intelligence, we direct the reader to \cite{cuzzolin2020knowing}.

\subsection{Interactive Human Models}
Many human motion predictors have studied interactions in human crowds \cite{sighencea2021review}. A heuristic approach for human-robot interaction is to directly model the robot as a human neighbour. 
\cite{schaefer2021leveraging} uses a neural network to predict human trajectories with different candidate robot plans. By observing how much the human tries to avoid each planned trajectory, it chooses the least invasive robot plan for maximised human comfort. 
This method relies on a strong assumption that humans treat robots the same as other humans. 
Another line of work models how humans react to robots. \cite{tian2022safety} predicts whether the human behaves like a leader or follower in a Stackelberg game. When the human is a follower, the cost is minimised based on observed robot actions. 
Similarly in \cite{geldenbott2024legible, sripathy2021dynamically}, the robot action is used in human cost functions to predict the human response. The key difference between these methods and our proposed method is that we enable the human to ``predict" future motions of other agents, instead of only ``observe and respond".

\subsection{Dynamic Human Models}
Instead of using a unified and static human model for all interactions, some methods model a dynamic human. 
\cite{tian2023towards} models a human teleoperator whose internal estimation of the robot dynamics is potentially inaccurate. The robot then corrects this estimation by modifying its response to teleoperation. It assumes that there exists a true value that the human model needs to approach. However, there is no true human model in reality. 
\cite{parekh2023learning, muktadir2024adaptive, cathcart2023proactive} models diverse human actions and preferences using either explicit categories or latent distributions.
% \cite{parekh2023learning} models different human behaviours using a latent distribution. Future human actions are predicted from the most likely latent behaviour sample. 
% \cite{muktadir2024adaptive} quantitatively defines human driver behaviours to categorize observed vehicle motions. Future motions are predicted separately for each behaviour group. 
These methods provide some level of interpretability of different human behaviours, but they fail to explain why humans change behaviours over repeated interactions.

\subsection{Human Trajectory Prediction for Robot Planning}
\label{seq:traj_prediction}
% To navigate safely in a shared space with humans, robots rely on human trajectory predictions to perform predictive planning. 
Early works on robot planning do not consider human motion models \cite{wu2019depth, wu2018learn, wu2021learn, wu2020achieving, wu2019bnd, wu2019tdpp, cao2022direct}.
In recent social predictive planning, the most commonly adopted heuristic is the constant velocity model. 
Another classic model is the Social Force model \cite{helbing1995social} which considers multiple factors that can influence human motions. 
Recent learning-based methods have achieved outstanding prediction accuracy on large-scale pedestrian datasets \cite{salzmann2020trajectron++, xu2022remember}. 
Existing robot planners have integrated these human predictors into the system \cite{cao2024learningdynamicweightadjustment, boldrer2022multi, boldrer2020socially, ryu2024integrating, poddar2023crowd}. 
However, few methods have compared the effect of different prediction methods over downstream planning. 
Through our experiment, we show that our model not only produces more accurate predictions, but also ensures safety and efficiency in robot planning.