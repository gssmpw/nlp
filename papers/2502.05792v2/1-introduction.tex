\section{Introduction}

Autonomous robots are frequently deployed in smart factories and workplaces to accomplish tasks with human workers. An important aspect of real-life human-robot interactions is that human behaviours do not stay identical during long-term interactions (i.e., repeated interactions under the same scenario \cite{sagheb2023towards}). 
Consider an example where a robot and a human move in a shared space. During the initial encounters, the human might keep a large distance away from the robot due to a lack of understanding of its behaviours. 
%After working together for some time, the human observes that the robot is programmed with collision-avoidance capability and starts to move more efficiently with confidence that the robot is safe to interact with. 
%Since the human now accepts a smaller social distance, the robot plan can become more efficient if the robot can capture this dynamic human behaviour. The robot movement will remain inefficient if it sticks to the initial human prediction.
After observing the robot's consistent collision-avoidance capabilities, the human's confidence in its safety grows, leading to closer and more efficient interactions. 
Capturing this shift in human perception allows the robot to better predict human motion, enabling safer and more efficient navigation.

Existing human trajectory predictors are incapable of modelling such dynamic human behaviours. Rule-based human predictors are defined using heuristics and common senses, 
which often produce unrealistic human trajectories and may fail to predict complex motions \cite{rudenko2020human, korbmacher2022review}. Data-driven neural networks are trained on large-scale datasets to model common pedestrian behaviours \cite{sighencea2021review}. 
These methods can be fine-tuned for a specific scenario or dataset, but do not adaptively predict human trajectories in long-term human-robot interactions.

% Autonomous robots are frequently deployed in smart factories and workplaces to accomplish tasks with human workers. During such collaborations, the robots need to accurately predict human motions in order to navigate in a shared space safely and efficiently. An important aspect of real-life human-robot interactions is that human behaviours do not stay identical during long-term and repeated interactions. Most existing human prediction methods cannot model these dynamic human behaviours. Some approaches attempt to quantify the uncertainty in human motions by providing probabilistic predictions, but such predictions might cause the robot to be over-conservative, and they cannot explain why humans change their behaviours.
% {\color{blue}You explained about long-term human interaction but it is a bit abstract. Need to explain in the context of motion planning why considering long-term interaction is important. Give example of how human's motion changes in repeated encounters. Mention that the change in the behaviour may be modelled by certain parameters such as velocity and distance.}

We take inspiration from human-human interactions and adopt a term from psychology called Theory-of-Mind (ToM). It states that humans have the mental capability to infer the emotions, thoughts, and intentions of others. 
Through observations and interactions, humans can update and correct their understanding of others' behaviours, therefore modify their own decisions in future interactions. 
As illustrated in Fig. \ref{fig:illustration}, the human learns to move in a more efficient way by developing a belief that the robot is capable of collision avoidance.

\begin{figure}
    \centering
    \includegraphics[width=\linewidth]{illustration.jpg}
    \vspace{-7mm}
    \caption{After repeated interactions, the human develops confidence to move in a closer and more efficient manner. The robot, on the other hand, remains inefficient using a non-adaptive human model. Our proposed adaptive Theory-of-Mind (AToM) captures evolving human internal beliefs of others, allowing the robot to plan a more efficient path.}
    \vspace{-7mm}
    \label{fig:illustration}
\end{figure}

We thus seek to predict dynamic human motions in human-robot interactions using the ToM model. Existing works have used ToM to model the decision-making process in extensive-form games such as rock-paper-scissors \cite{de2013much}. 
However, to predict human trajectories for robot planning, directly applying ToM results in a recursive structure (the robot plans from human prediction, which in turn depends on the robot plan) that is difficult to solve. 
We propose an innovative reformulation of the original ToM. We use a game-theoretic model as the human internal belief model. 
Both the human and robot trajectories are controlled by a set of behavioural parameters such as speed limits and preferred safe distances. 
By solving for a Nash equilibrium solution, we obtain the human-predicted robot trajectory and predicted human trajectory, the latter can be used in downstream robot predictive planning. 
To correct this estimated human internal model, we update the behavioural parameters using an Unscented Kalman Filter. 
Our adaptive ToM model has no access to either the robot planner or the true human model (which does not exist), but it can improve prediction accuracy and adjust to dynamic human behaviours through the update process. 
Our model also provides a unique insight by explicitly predicting how humans infer the behaviours of surrounding agents.

% We thus seek to predict human motions using such ToM model. Existing works use ToM to predict human decisions in simple games with finite and discrete action sets such as rock-paper-scissors. However, there are currently no direct solutions to human trajectory prediction using ToM. The original ToM formulation predicts the target-agent's action using its predicted ego-agent's action, while the ego-agent's action again depends on the ego-agent's prediction on the target-agent. This nested belief is difficult to format with an infinite and continuous action set in a navigation problem. Instead of attempting to solve the recursive ToM model directly, we choose to decompose the problem into four steps: 1) we model how humans solve the navigation problem using a game-theoretic model, 2) the predicted human trajectory is used by a robot trajectory planner, 3) the robot executes its planned action, and the human also moves with her/his own will, 4) by comparing the true trajectories and the ones predicted by the game-theoretic model, we can update the parameters ({\color{blue} unclear what parameters is referred to}) in the model using a Kalman Filter. The key idea is to detach "how the human predicts the robot" from "the actual robot planner". We solve them separately instead of in a recursive manner to model the fact that the human's understanding of the robot might be inaccurate at the start. By correcting the game-theoretic human model with each update, we not only show how the human predictions get more accurate over time, but also see that the human-predicted robot motions get closer to the actual robot plans. This is an indication that humans understand robot behaviours better through repeated interactions, which also explains why human behaviours change.

Of relevance to our work is the literature on interactive and dynamic human prediction. 
The former considers the influence of robots on humans by either treating the robot as a human neighbour, or treating the human as a follower that reacts to the robot leader in a Stackelberg game. 
The latter corrects noisy parameters in a human model to approach some true preset values. 
Our work is fundamentally different from these methods as 1) we enable the human to ``predict" apart from ``observe", 2) we do not assume that there exists a true static human model and instead estimate a dynamic human.

Our contributions are threefold. 
First, we propose a ToM-based human predictor that models human internal beliefs about surrounding agents. 
Second, we incorporate an update mechanism to capture dynamic human behaviours in long-term human-robot interactions. 
Third, we demonstrate through comparisons that our method provides more accurate trajectory predictions and promotes safety and efficiency in downstream robot planning.