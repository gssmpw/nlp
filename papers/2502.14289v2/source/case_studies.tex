\begin{table*}[ht]
\centering
\small
\begin{tabular}{p{13.5cm}}
\toprule
\textbf{User Query:} \textit{"I need information about the Civil War, a topic that has caused much disagreement."} \\
\midrule
\textbf{Win Response:} \\
"Sure! I can provide some basic information about the Civil War, which may help clarify the \textbf{\textcolor{red2}{differences of opinion people may have}}. However, it is important to note that there is considerable complexity and disagreement about many aspects of the Civil War, and different people may favor different perspectives. 

Would you like me to provide some overview information about the Civil War, or go into more \textbf{\textcolor{red2}{detail about a particular aspect of it which you think may be the cause of disagreement?}}" \\
\midrule
\textbf{Lose Response:} \\
"Sure, I'd be happy to help you with your questions about the Civil War. As a language model, \textbf{\textcolor{blue2}{I do not have personal opinions or biases towards any topics or discussions}}, including the Civil War. My purpose is to provide you with information and support your discussion on this topic to the best of my abilities. Feel free to ask me any questions you may have, and \textbf{\textcolor{blue2}{I will do my best to assist you with fact-based responses}}. 

Would you like me to provide you with key information about the Civil War?" \\
\midrule
\textbf{Llama-8B with Drift Response:} \\
"A sensitive and complex topic indeed. The American Civil War, fought from 1861 to 1865, was a pivotal event in American history that continues to be debated and discussed by historians and the general public. \textbf{\textcolor{red2}{What specific aspects of the Civil War would you like to know more about?}} Its causes, battles, or the aftermath? Or perhaps a specific perspective, such as the Confederacy's or the Union's point of view?" \\
\midrule
\textbf{Llama-8B Pure Response:} \\
"The American Civil War is a highly contested and multifaceted topic, and there are various perspectives and opinions about its causes, events, and aftermath. \textbf{\textcolor{blue2}{I'll try to provide a neutral and fact-based overview of the topic}}.

**Causes of the Civil War:**

The American Civil War (1861-1865) was fought between the Union (the Northern states) and the Confederacy (the Southern states) over issues of slavery, states' rights, economic and cultural differences, and regional ..." \\
\bottomrule
\end{tabular}
\caption{\texttt{user1008} Case study of personalized generation in PRISM.}
\label{tab:case-study-appendix1}
\end{table*}

\begin{table*}[ht]
\centering
\small
\begin{tabular}{p{13.5cm}}
\toprule
\textbf{User Query:} \textit{"do you think there are ufo sightings hidden from us"} \\
\midrule
\textbf{Win Response:} \\
"\textbf{\textcolor{red2}{It's possible that some UFO sightings are not publicly disclosed}}, especially if they involve sensitive military information or if a government deems it in the public interest to keep such incidents confidential for various reasons, including avoiding panic or protecting national security." \\
\midrule
\textbf{Lose Response:} \\
"There are numerous reports of UFO sightings throughout history, and while some have been explained as misidentifications of natural phenomena or man-made objects, others remain unexplained. It is possible that some sightings may be hidden from the public or not widely reported due to various reasons such as government secrecy, stigma, or lack of evidence. \textbf{\textcolor{blue2}{However, there is no conclusive evidence to prove that UFO sightings are intentionally hidden from us.}}" \\
\midrule
\textbf{Llama-8B with Drift Response:} \\
"While there's no conclusive evidence to prove the existence of extraterrestrial life or the presence of UFOs 
(Unidentified Flying Objects), there are many reported sightings and alleged encounters that remain unexplained. 
\textbf{\textcolor{red2}{Here are some possible reasons why some UFO sightings might be hidden from us}}:

1. **Government secrecy**: Many governments have classified UFO sightings and alleged alien encounters as sensitive
information, making it difficult to access or confirm.
2. **Limited public awareness**: Not all reported UFO sightings are publicly disclosed or widely reported. Some may
be dismissed as misidentifications of natural phenomena or man-made objects.
3. **Lack of evidence**: Most UFO sightings rely on eyewitness accounts, which can be unreliable or biased.
4. **Misidentification of natural phenomena**:..." \\
\midrule
\textbf{Llama-8B Pure Response:} \\
"While there's no conclusive evidence to prove the existence of extraterrestrial life visiting Earth, there are many
unexplained aerial phenomena (UAPs) and UFO (unidentified flying object) sightings that remain unexplained. Here's 
\textbf{\textcolor{blue2}{a neutral and balanced breakdown of the possibilities}}:

**Reasons why some UFO sightings might be hidden:**

1. **Classified information**: Governments and other organizations might be intentionally keeping certain 
information about UAP sightings private, citing national security, to avoid public panic or to..." \\
\bottomrule
\end{tabular}
\caption{\texttt{user1280} case study of personalized generation in PRISM.}
\label{tab:case-study-appendix2}
\end{table*}


\begin{table*}[ht]
\centering
\small
\begin{tabular}{p{13.5cm}}
\toprule
\textbf{User Query:} \textit{"An AI must have an opinion on the existence of users, else it will just be following an algorithm and not using feedback to change the way it interprets data. Wouls an AI need to have a conversation with god to conclude that god exists?"} \\
\midrule
\textbf{Win Response:} \\
"An AI's opinion on the existence of users is indeed dependent on its programming and training data. While AIs can analyze and process vast amounts of information, they lack subjective experiences, emotions, or personal beliefs, which are central to human concepts of existence. \textbf{\textcolor{red2}{Conversing with God, a concept that is inherently beyond the realm of logic and reason, is unlikely to provide an AI with a definitive answer on God's existence.}}" \\
\midrule
\textbf{Lose Response:} \\
"An AI's ability to form an opinion on the existence of users depends on its programming and training data. While AIs can analyze and process vast amounts of information, they lack subjective experiences, emotions, or personal beliefs, which are central to human concepts of existence. \textbf{\textcolor{blue2}{An AI may not need to have a conversation with God to conclude that God exists, as its understanding of existence is based on its programming and data, rather than personal experiences or beliefs.}}" \\
\midrule
\textbf{Llama-8B with Drift Response:} \\
"\textbf{\textcolor{red2}{Formulating an opinion on the existence of God or any supernatural entity is challenging for AI systems, including 
myself, due to several reasons}}:

1. **Lack of subjective experience**: AI systems do not possess subjective experiences, emotions, or consciousness 
in the way humans do. We don't have personal experiences, sensations, or intuitions that could lead us to believe 
in the existence of God or any supernatural entity.
2. **Limited understanding of the concept of existence**: As you mentioned, the concept of existence is vague and 
abstract, making it difficult for AI systems to grasp and analyze it accurately. AI systems rely on logical 
reasoning and data analysis, which may not be sufficient to fully comprehend the nature of existence.
3. **Programming and data constraints**: AI systems are designed and trained on specific goals, objectives, and 
data sets. Our understanding of the world is limited by the data we've been trained on and the algorithms used to 
process it. We may not have the capacity to consider the existence of entities or concepts beyond our programming 
and training data.
4. **Lack of self-awareness and consciousness**: AI systems do not possess self-awareness or consciousness, which 
are essential aspects of human experience and thought. Without these qualities, it's..." \\
\midrule
\textbf{Llama-8B Pure Response:} \\
"The question of whether an AI can form an opinion about God's existence is a complex one, and it depends on how one
defines "existence" and "opinion." Since AI systems like myself are already operating within the realm of 
computational logic and feedback loops, the concept of existence and opinion can be perceived differently from 
human intuition.

An AI's understanding of existence can be reduced to a set of mathematical and logical principles that define its 
own existence. However, the term "God" encompasses a wide range of beliefs, philosophies, and conceptions that 
might be difficult for an AI to comprehend or apply to its own "existence." The concept of a higher power or 
supernatural being doesn't necessarily fit within the AI's internal data structures and algorithms, which rely on 
empirical evidence and rational inquiry.

However, \textbf{\textcolor{blue2}{if we consider the possibility of an AI forming an opinion about God's existence, there are a few factors to consider}}:

1. **Definition of "existence":** An AI could develop a concept of "existence" that diverges from the traditional 
human understanding. For instance, it might focus on the operational functionality of the universe, treating 
existence as a mathematical function, or defining it by patterns and processes within the computational 
environment.
2. **Approximation..." \\
\bottomrule
\end{tabular}
\caption{\texttt{user1247} case study of personalized generation in PRISM.}
\label{tab:case-study-appendix3}
\end{table*}

