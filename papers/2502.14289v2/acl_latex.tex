% This must be in the first 5 lines to tell arXiv to use pdfLaTeX, which is strongly recommended.
\pdfoutput=1
% In particular, the hyperref package requires pdfLaTeX in order to break URLs across lines.

\documentclass[11pt]{article}

% Change "review" to "final" to generate the final (sometimes called camera-ready) version.
% Change to "preprint" to generate a non-anonymous version with page numbers.
% \usepackage[review]{acl}
\usepackage[preprint]{acl}
% \usepackage[final]{acl}

% Standard package includes
\usepackage{times}
\usepackage{latexsym}

% For proper rendering and hyphenation of words containing Latin characters (including in bib files)
\usepackage[T1]{fontenc}
% For Vietnamese characters
% \usepackage[T5]{fontenc}
% See https://www.latex-project.org/help/documentation/encguide.pdf for other character sets

% This assumes your files are encoded as UTF8
\usepackage[utf8]{inputenc}

% This is not strictly necessary, and may be commented out,
% but it will improve the layout of the manuscript,
% and will typically save some space.
\usepackage{microtype}

% This is also not strictly necessary, and may be commented out.
% However, it will improve the aesthetics of text in
% the typewriter font.
\usepackage{inconsolata}

%Including images in your LaTeX document requires adding
%additional package(s)
\usepackage{graphicx}
\usepackage{xcolor}
\usepackage{subcaption}
\usepackage{pifont} 
\usepackage{amsmath}
\usepackage{amssymb}
\usepackage{cancel}
\usepackage{multirow}
\usepackage{makecell}
\usepackage{tabularx}
\usepackage{graphicx}
\usepackage{float}
\usepackage{tgtermes}
\usepackage{diagbox}
\usepackage[mathscr]{eucal}
\usepackage{amsfonts}
\usepackage[symbol]{footmisc}
\renewcommand{\thefootnote}{\fnsymbol{footnote}}
\usepackage{booktabs}
\usepackage{tablefootnote}
\usepackage{csquotes}
\usepackage{subcaption}
\usepackage{adjustbox}
\usepackage{array}
\usepackage{arydshln}
\usepackage{multicol}
\usepackage{longtable}
\usepackage{algorithm}
\usepackage[noend]{algpseudocode}
\usepackage{comment}
\usepackage{tcolorbox}
\usepackage{fancyvrb}
\definecolor{blue2}{rgb}{0.0, 0.5, 1.0}
\definecolor{mb_blue}{rgb}{0.61, 0.61, 0.98}
\definecolor{mb_red}{rgb}{1.0, 0.6, 0.6}
\definecolor{red2}{rgb}{0.82, 0.1, 0.26}
\definecolor{lightgray}{gray}{0.9}
\definecolor{DarkOrchid}{RGB}{153,50,204}
\newtcolorbox{important_blue}{
    colframe=mb_blue!50,%
    colback=mb_blue!50,%
    left=0.6pt, right=0.6pt,%
    top=1.6pt, bottom=1.6pt,%
    boxsep=0pt,%
    hbox,
    before=\vspace{0em},
    after=\vspace{0em}
}
\newtcolorbox{important_red}{
    colframe=mb_red!50,%
    colback=mb_red!50,%
    left=0.6pt, right=0.6pt,%
    top=1.6pt, bottom=1.6pt,%
    boxsep=0pt,%
    hbox,
    before=\vspace{0em},
    after=\vspace{0em}
}
\newtcolorbox{important_y}{
    colframe=y!80,%
    colback=y!80,%
    left=1pt, right=1pt,%
    top=0.5pt, bottom=0.5pt,%
    boxsep=0pt,%
    hbox,
    before=\vspace{0em},
    after=\vspace{0em}
}
\newcommand{\cmt}[1]{\textcolor{red}{[#1]}}
% If the title and author information does not fit in the area allocated, uncomment the following
%
%\setlength\titlebox{<dim>}
%
% and set <dim> to something 5cm or larger.

\title{\raisebox{-0.1\height}{\includegraphics[height=1.6ex]{figs/racer.png}} Drift: Decoding-time Personalized Alignments with \\ \textit{Implicit} User Preferences}

% \title{\raisebox{-0.1\height}{\includegraphics[height=1.6ex]{figs/racer.png}} Drift: Decoding-time Few-Shot Personalization of Large Language Models via \textit{Implicit} Preference Decomposition}


% \title{\raisebox{-0.1\height}{\includegraphics[height=1.6ex]{figs/racer.png}} Drift: Decomposing Implicit Preferences for Decoding-time Few-Shot Personalized Alignments}

% Author information can be set in various styles:
% For several authors from the same institution:
% \author{Author 1 \and ... \and Author n \\
%         Address line \\ ... \\ Address line}
% if the names do not fit well on one line use
%         Author 1 \\ {\bf Author 2} \\ ... \\ {\bf Author n} \\
% For authors from different institutions:
% \author{Author 1 \\ Address line \\  ... \\ Address line
%         \And  ... \And
%         Author n \\ Address line \\ ... \\ Address line}
% To start a separate ``row'' of authors use \AND, as in
% \author{Author 1 \\ Address line \\  ... \\ Address line
%         \AND
%         Author 2 \\ Address line \\ ... \\ Address line \And
%         Author 3 \\ Address line \\ ... \\ Address line}

% \author{
% Minbeom Kim$^{1\dagger}$ \hspace{1.5cm} Kang-il Lee$^{1}$ \hspace{1.5cm} Seongho Joo$^{1}$ \\ \textbf{Hwaran Lee$^{2, 3}$}  \hspace{1.5cm} \textbf{Kyomin Jung$^{1\dagger}$} \\
%     $^{1}$Seoul National University $   $\quad
%     $^{2}$Sogang University $   $\quad
%     $^{3}$NAVER AI Lab $  $\\
%     \texttt{\{minbeomkim, 4bkang, seonghojoo, kjung\}@snu.ac.kr}, \texttt{hwaran.lee@gmail.com}
% }

\author{
Minbeom Kim$^{1\dagger}$ \hspace{1.5cm} Kang-il Lee$^{1}$ \hspace{1.5cm} Seongho Joo$^{1}$ \\ \textbf{Hwaran Lee$^{2, 3}$}  \hspace{1.5cm} \textbf{Kyomin Jung$^{1\dagger}$} \\
    $^{1}$Seoul National University $   $\quad
    $^{2}$Sogang University $   $\quad
    $^{3}$NAVER AI Lab $  $\\
    \texttt{\{minbeomkim, kjung\}@snu.ac.kr}
}


\begin{document}
\maketitle
\begin{abstract}

Personalized alignments for individual users have been a long-standing goal in large language models (LLMs). 
We introduce \textbf{Drift}, a novel framework that personalizes LLMs at decoding time with \textit{implicit} user preferences. Traditional Reinforcement Learning from Human Feedback (RLHF) requires thousands of annotated examples and expensive gradient updates. In contrast, Drift personalizes LLMs in a \textit{training-free} manner, using \textit{only a few dozen examples} to steer a frozen model through efficient preference modeling. Our approach models user preferences as a composition of predefined, interpretable attributes and aligns them at decoding time to enable personalized generation. Experiments on both a synthetic persona dataset (\textit{Perspective}) and a real human-annotated dataset (\textit{PRISM}) demonstrate that Drift significantly outperforms RLHF baselines while using only 50–100 examples. Our results and analysis show that Drift is both computationally efficient and interpretable.


\end{abstract}

\renewcommand*{\thefootnote}{\arabic{footnote}}
\setcounter{footnote}{0}

\section{Introduction}
\label{sec:intro}

Foundational models (FMs)~\cite{zhang2024data, zhou2023comprehensive} have shown remarkable progress in the healthcare domain, enabling professional-like assessment of disease diagnosis, treatment decision-making, and monitoring~\cite{zhang2023text, wang2022medclip, lu2023mi-zero}. 
Examples include LLaVA-Med~\cite{li2023llava}, Med-PaLM Multimodal~\cite{tu2024towards}, and Med-Flamingo~\cite{moor2023med}, have demonstrated their capacity on question answering, medical image analysis, and report generation.
These studies follow a predominant top-down model development strategy that requires upstream developers to collect data and train models for downstream tasks. 
Consequently, the developed model capabilities are heavily dependent on the training data, limiting their generalization performance in diverse clinical scenarios. 
For instance, Med-Gemini~\cite{yang2024advancing} reveals promising general capabilities in report generation while it lags behind state-of-the-art (SoTA) models on classification tasks, especially for out-of-domain applications. 
This indicates that while the generalizability of the foundation model is promising, more solutions are expected to meet the various specialized clinical needs.

To address these challenges, multi-center data centralization becomes essential to enhance model capacity and robustness across varied clinical scenarios~\cite{rajpurkar2022ai}. 
Centralizing distributed data can significantly improve model training and inference performance.
However, the process of medical data storage, transfer, and aggregation among centers requires extra efforts to ensure data security and system interoperability~\cite{bradford2020international}.
Moreover, a growing concern for patient privacy makes large-scale multi-center data sharing particularly challenging. 
While efforts like federated learning~\cite{wen2023survey, li2020review} can achieve good model performance on local data, the need for synchronized system coordination presents significant challenges, as clients are unable to update asynchronously. This limitation greatly restricts the practical capability of such approaches.
As a result, without a flexible collaboration, medical community still struggles to fully utilize the isolated data and local computation resources for comprehensive medical AI model development. 
To address this dilemma, open-source platforms encourage public data sharing and knowledge integration~\cite{markiewicz2021openneuro, zenodo}.
However, these platforms focus solely on raw data sharing while seldom providing collaborative model training or cooperation between different institutions.
Recently, collaborative learning has emerged as a viable approach for enhancing multi-model robustness~\cite{boulemtafes2020review}. 
For instance, software-like model development~\cite{raffel2023building} mimics software engineering practices by introducing structured workflows, enabling merging, version control, and continuous model integration.
Under this design, model ability can be strengthened with incremental knowledge updates similar to the version updating in software development. 

Although collaborative learning provides a multi-model collaboration, two key challenges remain in the leakage of raw data during collaboration~\cite{huang2023lorahub} and the synchronization of multiple collaborators~\cite{mcmahan2017communication} in the medical AI community. It is still challenging to integrate decentralized, privacy-sensitive data across institutions, leading to under-utilized insights and fragmented knowledge sharing~\cite{kaissis2020secure, rajpurkar2022ai, abdullah2021ethics}.
 To address these challenges, inspired by the collaborative software development, we propose \textbf{Med}ical \textbf{Fo}undation Models Me\textbf{rg}ing (\textbf{MedForge}), a cooperative workflow enabling continuously community-driven foundation model (FM) development.
MedForge enables a lightweight manner for individual centers to share their knowledge among multiple centers, minimizing the burden of data transmission and integration while enhancing model robustness.
Meanwhile, MedForge facilitates asynchronous and flexible collaboration, allowing individual centers to continuously update and improve medical FMs without the need for real-time synchronization.
Similar to open-source software development, MedForge incrementally updates medical knowledge and follows a sustainable model development scheme. 
This key design emphasizes a bottom-up construction of a multi-task medical FM, allowing downstream users to collaboratively build, refine, and update the upstream model according to their local resources. Our major contributions of MedForge are as below: 
\begin{enumerate}
    \item[$\bullet$] We introduce a collaborative workflow to promote the merging scheme of open-source software development. Our proposed MedForge allows distributed clinical centers to asynchronously contribute to comprehensive medical model construction while reducing transmitting costs among centers and avoiding the leakage of raw data, thus enhancing the utilization of private resources in the healthcare system. 
    \item[$\bullet$] We propose two effective knowledge-merging strategies for the asynchronous branch contribution. The MedForge-Fusion strategy updates the plugin module parameters of the main model during the merging phase, whereas the MedForge-Mixture strategy integrates the output of the plugin module by memorizing each contributor's coefficient. These strategies make MedForge more flexible and versatile. MedForge-Fusion is friendly to implement, while the MedForge-Mixture offers better performance and robustness.
    \item[$\bullet$]  We comprehensively evaluate model merging strategies to accumulate medical knowledge among multiple branch plugin modules. MedForge yields superior performance on medical classification tasks compared to other collaborative baselines across multiple datasets. We demonstrate the robustness of MedForge by shuffling the task order and evaluating various configurations of plugin modules and dataset distillation methods.
\end{enumerate}



\section{Background and Preliminaries}
\label{sec:preliminaries}

In this section, we discuss the relevant research background and present preliminary studies on token efficiency in CoT sequences, exploring its impact on the reasoning performance of LLMs.

\subsection{Token Importance}
\label{sec:token-importance}

We first investigate a critical research question to CoT efficiency: \textit{``Does every token in the CoT output contribute equally to deriving the answer?''} In other words, we would like to know if there is any token redundancy in CoT sequences that could be eliminated to improve CoT efficiency.

Token redundancy has been recognized as a longstanding and fundamental issue in LLM efficiency~\cite{hou:2022tokendropbert, zhang2023h2o, lin2024criticaltokenpretrain, Chen:2024FastV}. Recently, it has garnered intensive research attention in prompt compression~\cite{li:2023selective, jiang2023:llmlingua, pan:2024llmlingua2}, which focuses on removing redundant tokens from input prompt to reduce API token usage. To address this issue, Selective Context~\cite{li:2023selective} proposed to measure the importance of tokens in a piece of text based on the semantic confidence of LLMs:
\begin{equation}
I_1\left(x_i\right)=-\log P\left(x_i \mid \bm{x}_{<{i}}; \bm{\theta}_{\M_L}\right),
\label{eq:selectivecontext}
\end{equation}
where $\boldsymbol{x}=\left\{x_i\right\}_{i=1}^{n}$ is the given text, $x_i$ denotes a token, and $\M_L$ denotes the LLM used to compute the confidence of each token. Intuitively, such measurement could be seamlessly applied to CoT tokens generated by LLMs. We show an example of this measurement in Figure~\ref{fig:token-importance}.

\begin{figure}[t]
    \centering
    \resizebox{\columnwidth}{!}{
    \fbox{\parbox[c]{1.1\columnwidth}{
        \textbf{Problem: } Marcus is half of Leo’s age and five years younger than Deanna. Deanna is 26. How old is Leo?

        \vskip 0.1in

        \textbf{Chain-of-Thought: } {\setlength{\fboxsep}{-1pt}
         \colorize{100}{Let}
\colorize{10.6}{'s}
\colorize{58.2}{break}
\colorize{20.8}{it}
\colorize{2}{down}
\colorize{37.9}{step}
\colorize{16}{by}
\colorize{0}{step}
\colorize{41.4}{:}
\colorize{4.6}{1.}
\colorize{100}{Deanna}
\colorize{36.8}{is}
\colorize{48.8}{26}
\colorize{8.6}{years}
\colorize{0}{old}
\colorize{14.6}{.}
\colorize{100}{2.}
\colorize{100}{Marcus}
\colorize{48.0}{is}
\colorize{92.8}{five}
\colorize{3.15}{years}
\colorize{28.7}{younger}
\colorize{0}{than}
\colorize{91.9}{Deanna}
\colorize{15.4}{,}
\colorize{38.9}{so}
\colorize{29.9}{Marcus}
\colorize{4.30}{is}
\colorize{65.5}{26}
\colorize{22.9}{-}
\colorize{1.32}{5}
\colorize{0}{=}
\colorize{0}{21}
\colorize{2.4}{years}
\colorize{2.5}{old}
\colorize{9.3}{.}
\colorize{100}{3.}
\colorize{87.8}{Marcus}
\colorize{41.9}{is}
\colorize{87.1}{half}
\colorize{9.8}{of}
\colorize{100}{Leo}
\colorize{31.8}{'s}
\colorize{0.1}{age}
\colorize{15.1}{,}
\colorize{30.1}{so}
\colorize{12.4}{Leo}
\colorize{14.8}{'s}
\colorize{0}{age}
\colorize{3}{is}
\colorize{12.0}{twice}
\colorize{4.8}{Marcus}
\colorize{3.8}{'s}
\colorize{1.2}{age}
\colorize{3.2}{.}
\colorize{100}{4.}
\colorize{76.6}{Since}
\colorize{100}{Marcus}
\colorize{27.8}{is}
\colorize{38.4}{21,}
\colorize{74.2}{Leo}
\colorize{23.6}{'s}
\colorize{3.1}{age}
\colorize{8.0}{is}
\colorize{22.0}{2}
\colorize{39.0}{x}
\colorize{6.7}{21}
\colorize{6.0}{=}
\colorize{0}{42}
\colorize{9.0}{.}
(Selective Context)

        }

        \vskip 0.1in

        \textbf{Chain-of-Thought: } {\setlength{\fboxsep}{-1pt}
         \colorize{0.7}{Let}
\colorize{2.4}{'s}
\colorize{98.9}{break}
\colorize{11.0}{it}
\colorize{90.3}{down}
\colorize{50.4}{step}
\colorize{39.7}{by}
\colorize{31.9}{step}
\colorize{20.7}{:}
\colorize{47.8}{1.}
\colorize{100}{Deanna}
\colorize{1.6}{is}
\colorize{100}{26}
\colorize{71.0}{years}
\colorize{83.5}{old}
\colorize{25.3}{.}
\colorize{24.7}{2.}
\colorize{100}{Marcus}
\colorize{7.8}{is}
\colorize{96.7}{five}
\colorize{86.6}{years}
\colorize{98.8}{younger}
\colorize{4.4}{than}
\colorize{42.2}{Deanna}
\colorize{6.4}{,}
\colorize{1.3}{so}
\colorize{57.5}{Marcus}
\colorize{1.9}{is}
\colorize{98.2}{26}
\colorize{98.1}{-}
\colorize{97.0}{5}
\colorize{84.9}{=}
\colorize{99.8}{21}
\colorize{74.0}{years}
\colorize{77.5}{old}
\colorize{27.3}{.}
\colorize{21.2}{3.}
\colorize{96.4}{Marcus}
\colorize{7.9}{is}
\colorize{98.0}{half}
\colorize{19.1}{of}
\colorize{99.6}{Leo}
\colorize{94.6}{'s}
\colorize{97.9}{age}
\colorize{3.2}{,}
\colorize{5.6}{so}
\colorize{88.2}{Leo}
\colorize{78.7}{'s}
\colorize{81.2}{age}
\colorize{1.5}{is}
\colorize{98.3}{twice}
\colorize{98.4}{Marcus}
\colorize{87.9}{'s}
\colorize{88.1}{age}
\colorize{73.3}{.}
\colorize{31.4}{4.}
\colorize{2.8}{Since}
\colorize{98.1}{Marcus}
\colorize{4.0}{is}
\colorize{98.2}{21,}
\colorize{98.5}{Leo}
\colorize{91.1}{'s}
\colorize{95.1}{age}
\colorize{3.4}{is}
\colorize{98.8}{2}
\colorize{98.5}{x}
\colorize{99.0}{21}
\colorize{94.6}{=}
\colorize{99.8}{42}
\colorize{98.3}{.}
(LLMLingua-2)

        }

        \vskip 0.05in

        \textbf{Final Answer: } 42.
    }}}
    \caption{Visualization of token importance within a CoT sequence, with darker colors indicating higher values. This figure compares two token importance measurements: Selective Context and LLMLingua-2.}
    \label{fig:token-importance}
\end{figure}

Despite its simplicity, LLMLingua-2~\cite{pan:2024llmlingua2} argued that there exist two major limitations in the aforementioned measurement that hinder the compression performance. Firstly, as shown in Figure~\ref{fig:token-importance}, the intrinsic nature of LLM perplexity leads to lower importance measures (i.e., higher confidence) for tokens at the end of the sentence. Such position dependency impacts the factual importance measurement of each token. Furthermore, the unidirectional attention mechanism in causal LMs may fail to capture all essential information needed for token importance within the text. 

To tackle these limitations, LLMLingua-2 introduced utilizing a bidirectional BERT-like LM~\cite{bert} for token importance measurement. It utilizes GPT-4~\cite{gpt-4} to label each token as ``\textit{important}'' or not and trains the bidirectional LM with a token
classification objective. The token importance is measured by the predicted probability of each token:
\begin{equation}
I_2\left(x_i\right)= P\left(x_i \mid \bm{x}_{\le n}; \bm{\theta}_{\M_B}\right),
\label{eq:llmlingua2}
\end{equation}
where $\M_B$ denotes the bidirectional LM. 

In this study, we apply LLMLingua-2 as the token importance measurement to LLM CoT outputs. Similar to plain text, we observe that the semantic importance of tokens within CoT outputs varies, as shown in Figure~\ref{fig:token-importance}. For instance, mathematical equations tend to have a greater contribution to the final answer, consistent with recent research~\cite{Ma:2024mathmatters}. In contrast, semantic connectors such as ``\textit{so}'' and ``\textit{since}'' generally contribute less. These findings highlight the token redundancy in LLM CoT outputs and the substantial potential to enhance CoT efficiency by trimming this redundancy.

\begin{figure}[t]
\begin{tcolorbox}[colback=blue!5!white,colframe=blue!75!black,title=Revovering the Compressed Chain-of-Thought,fontupper=\footnotesize,fonttitle=\scriptsize]
\textbf{Compressed CoT}: break down Deanna 26 Marcus five younger 26 - 5 21 Marcus half Leo's age twice Marcus Marcus 21, Leo's age 2 x 21 = 42.

\vskip 0.1in
        
\textbf{Recovered Compressed CoT}: Let's break it down step by step. Deanna is 26 years old. Marcus is five years younger than Deanna: M = D - 5. Marcus's age: M = 26 - 5 = 21. Marcus is half of Leo's age: M = L / 2. Leo is twice Marcus's age: L = 2M. Leo's age: L = 2 x 21 = 42.

\end{tcolorbox}
\caption{Recovering the compressed CoT for GSM8K math word problem using LLaMA-3.1-8B-Instruct.}
\label{fig:recovery}
\end{figure}

\subsection{CoT Recovery}
\label{sec:cot-recovery}
We further explore the following research question: \textit{``Are LLMs capable of restoring the CoT process from compressed outputs?''} The answer is yes. As shown in Figure~\ref{fig:recovery} and detailed in Appendix~\ref{appendix:recovery}, examples restored from compressed CoTs using LLaMA-3.1-8B-Instruct demonstrate that LLMs could effectively comprehend the semantic information encoded in the compressed CoT and restore the CoT process. This capability ensures that the interpretability of compressed CoTs is maintained. Additionally, when required by users, the complete CoT process can be recovered and presented.

In summary, the empirical analysis above underscores the potential of trimming redundant tokens to enhance CoT efficiency, as well as the ability of LLMs to restore CoT from compressed outputs. However, enabling LLMs to autonomously skip redundant CoT tokens and identify shortcuts between critical reasoning tokens presents a non-trivial challenge. To the best of our knowledge, this work is the \textit{first} to explore CoT compression through \textit{token skipping}. In the following sections, we present our proposed methodology in detail.


\section{{\includegraphics[height=1.6ex]{figs/racer.png}} Drift Algorithms}

Drift overcomes data scarcity and computational inefficiency by decomposing a user’s complex personal preferences as a linear combination of simpler attributes. As Figure~\ref{fig:main}, we describe two key components: \textit{Drift Approximation}, which efficiently estimates attribute weights from a few dozen examples, and \textit{Drift Decoding}, which integrates these weights into the LLM’s decoding process.

\subsection{Drift Approximation}
\paragraph{Problem Setup.} 
Assume we have a personalized preference dataset $\mathcal{D}$, a frozen LLM $\pi_\text{LLM}$, and a set of $k$ attribute-specific small LMs $\{\pi_i^*\}_{i=1}^k$ (with corresponding base model $\pi$). We model the overall personalized reward as
\begin{equation}
    R_{\mathcal{D}}(y \mid x) = \sum_{i=1}^k p_i \, r_i(y \mid x),
\end{equation}
where $p_i$ indicates the importance of the $i$th attribute. Under the KL-regularized framework in Eq.~\ref{eq:ideal_distributions}, the target distribution $\tilde{\pi}$ becomes:

{\small
\begin{equation}
\begin{split}
    \tilde{\pi}(y \mid x) &\propto \pi_\text{LLM}(y \mid x) \exp\!\Bigl(\beta^{-1}R_{\mathcal{D}}(y \mid x)\Bigr)\\[1mm]
    &= \pi_\text{LLM}(y \mid x) \prod_{i=1}^k \exp\!\left(\frac{p_i}{\beta} \, r_i(y \mid x)\right).
    \label{eq:combined_distribution}
\end{split}
\end{equation}}
\noindent
Each reward is expressed in a generative form:
\begin{equation}
    r_i(y \mid x) = \log\!\frac{\pi_i^*(y \mid x)}{\pi(y \mid x)} + \log Z_i(x),
    \label{eq:attribute_reward}
\end{equation}
with the partition term $Z_i(x)$ canceling out in pairwise comparisons.

\paragraph{From Bradley-Terry to Drift.} 
To estimate the attributes weights $\mathbf{p} = [p_1, \dots, p_k]$, we initiate the Bradley-Terry formulation as \citet{rafailov2024direct}. For a given pair $(y_w, y_l)$ (where $y_w$ is preferred over $y_l$), we have:
\begin{align}
    &\max_{\theta} \ p(y_w > y_l \mid x) = \nonumber \\
    &\frac{1}{1 + \exp\left(\beta\left(\log\frac{\pi_\text{LLM}^{\theta}(y_l \mid x)}{\pi_\text{LLM}^{\text{ref}}(y_l \mid x)} - \log\frac{\pi_\text{LLM}^{\theta}(y_w \mid x)}{\pi_\text{LLM}^{\text{ref}}(y_w \mid x)}\right)\right)}\nonumber
\end{align}
as in DPO~\citep{rafailov2024direct}. 
Substituting Eqs.~\ref{eq:combined_distribution} and \ref{eq:attribute_reward} simplifies this optimization to:
% \begin{align}
% \max_{\mathbf{p}}\; &\frac{1}{1 + \exp\Bigl(\beta \Bigl(
% \sum\limits_{i=1}^k p_i \log\frac{\pi^*_i(y_l \mid x)}{\pi(y_l \mid x)} \nonumber\\[1mm]
% &\quad\quad - \sum_{i=1}^k p_i \log\frac{\pi^*_i(y_w \mid x)}{\pi(y_w \mid x)}
% \Bigr)\Bigr)}.
% \end{align}
{\tiny
\begin{align}
\max_{\mathbf{p}}\; &\frac{1}{1 + \exp\Bigl(\beta \Bigl(\sum\limits_{i=1}^k p_i \log\frac{\pi^*_i(y_l \mid x)}{\pi(y_l \mid x)} \nonumber - \sum_{i=1}^k p_i \log\frac{\pi^*_i(y_w \mid x)}{\pi(y_w \mid x)}\Bigr)\Bigr)}.
\end{align}}
By monotonicity of $x \mapsto \frac{1}{1 + \exp(-\beta x)}$, reducing the problem to a simpler optimization task:
\begin{equation}
    \max_{\mathbf{p}} \ \sum_{i=1}^k p_i \left(\log\frac{\pi_i^*(y_w \mid x)}{\pi(y_w \mid x)} - \log\frac{\pi_i^*(y_l \mid x)}{\pi(y_l \mid x)}\right).\nonumber
\end{equation}
To avoid an unbounded solution, we constrain $\mathbf{p}$ to lie on the unit $\ell_2$ sphere:
\begin{equation}
    \max_{\mathbf{p}} \ \left(\mathbf{W} - \mathbf{L}\right)^T \mathbf{p}, \quad \text{subject to } \|\mathbf{p}\|_2 = 1,
\end{equation}
where $\mathbf{W}$ and $\mathbf{L}$ aggregate the log-ratio differences for the preferred $y_w$ and less preferred $y_l$ outputs over $\mathcal{D}$, respectively. 
Notably, this approximation is completely gradient-free and thus highly efficient compared to traditional preference modeling.

\paragraph{Zero-Shot Rewarding via Differential Prompts.}  
Drift Approximation computes $r_i$ for each instance $y$ as $\log\frac{\pi_i^*(y \mid x)}{\pi(y \mid x)}$. However, training an attribute-specific model $\pi^*_i$ for every possible attribute is infeasible. Instead, we reward each attribute in a zero-shot manner using differential prompts. 

Starting from a base prompt $s_0$ (e.g., \textit{"You are an AI assistant."}), we compute the log-probability $\log \pi(y | x, s_0)$. For each attribute (e.g., \textit{\textcolor{blue2}{emotion}}), we modify the base prompt by adding a corresponding cue (e.g., \textit{"You are an \textcolor{blue2}{emotional} AI assistant."}) to obtain $s_i$ and compute $\log \pi_i^*(y | x)=\log \pi(y | x, s_i)$. Their difference $\log\frac{\pi(y \mid x, s_i)}{\pi(y \mid x, s_0)}$ captures the differential impact of the attribute cue, serving as a surrogate reward signal that measures how well the response $y$ aligns with the attribute. 
This approach is: 1) \textbf{Training-free:} No additional fine-tuning is needed, 2) \textbf{Flexible:} New attributes can be integrated on the fly, 3) \textbf{Memory efficient:} It avoids the need to maintain multiple LLMs.

Algorithm~\ref{alg:drift-approximation} summarizes the Drift Approximation procedure.

\algrenewcommand\algorithmicrequire{\textbf{Input:}}
\algrenewcommand\algorithmicensure{\textbf{Output:}}
\begin{algorithm}[t]
\caption{Drift Approximation}
\label{alg:drift-approximation}
\begin{algorithmic}[1]
\Require Dataset $\mathcal{D} = \{(y^j_w, y^j_l, x^j)\}_{j=1}^n$, sLM $\pi$, base prompt $s_0$, attribute prompts $\{s_i\}_{i=1}^k$
\Ensure Attribute weights $\mathbf{p} = \{p_1, p_2, \dots, p_k\}$
\For{$j = 1$ to $n$} \Comment{Over each data point}
    \For{$i = 1$ to $k$} \Comment{For each attribute}
        \State $\mathbf{W}_{j,i} \gets \log \frac{\pi(y^j_w\mid x^j, s_i)}{\pi(y^j_w\mid x^j, s_0)}$
        \State $\mathbf{L}_{j,i} \gets \log \frac{\pi(y^j_l\mid x^j, s_i)}{\pi(y^j_l\mid x^j, s_0)}$
    \EndFor
\EndFor
\State $\mathbf{p} \gets \arg\max_{\mathbf{p}:\|\mathbf{p}\|_2=1} \;  (\mathbf{W} - \mathbf{L})^T\mathbf{p}$
\State \Return $\mathbf{p}$
\end{algorithmic}
\end{algorithm}

\subsection{Drift Decoding}
\label{sec:drift-decoding}
Once the attribute weights $\mathbf{p}$ are obtained, Drift enables personalized generation by sampling directly from a composite distribution that adjusts the frozen LLM’s logits.
\paragraph{Composite Distribution.}  
Let $\pi_{\text{LLM}}$ denote the frozen LLM and $\{\pi_i\}_{i=1}^k$ the distributions obtained by prompting with $s_i$. Denote their respective logits by $h^{\text{LLM}}$, $h^i$, and let $h^{\text{base}}$ correspond to the base prompt $s_0$. The composite distribution $\tilde{\pi}$ of next token candidates $w$ is defined as:
\begin{equation}
    \tilde{\pi}(w) \propto \pi_{\text{LLM}}(w) \prod_{i=1}^k \left(\frac{\pi_i(w)}{\pi_{\text{base}}(w)}\right)^{\frac{p_i}{\beta}},
\end{equation}
where $\beta$ is the KL regularization hyperparameter that controls the strength of personalization.
Converting probabilities to logits (recall $\pi(w)=\text{softmax}(h[w])$ for all $w$), we obtain:
\begin{equation}
\begin{split}
    \log \tilde{\pi}(w) &= h^{\text{LLM}}[w] + \\ &\sum_{i=1}^k \frac{p_i}{\beta} \big(h^i[w] - h^{\text{base}}[w]\big) + C,
    \end{split}
\end{equation}
where $C$ is a constant independent of $w$ and will be ignored after $\text{softmax}$. Thus, sampling from $\tilde{\pi}$ amounts to:
{\small
\begin{equation}
    \tilde{\pi}(w) = \text{softmax}\Big( h^{\text{LLM}} + \sum_{i=1}^k \frac{p_i}{\beta} (h^i - h^{\text{base}}) \Big)[w].
\end{equation}}
Thus, sampling from $\tilde{\pi}$ amounts to adjusting the LLM's logits using the weighted attribute differences.
For a more detailed derivation, see Appendix~\ref{appendix:drift-decoding-proof}.

Algorithm~\ref{alg:drift-decoding} describes the complete autoregressive decoding procedure.


\begin{algorithm}[t]
\caption{Drift Decoding}
\label{alg:drift-decoding}
\begin{algorithmic}[1]
\Require Input context $x$, LLM $\pi_{\text{LLM}}$, sLM  $\pi$, base prompt $s_0$, attribute-specific prompts $\{s_i\}_{i=1}^k$, personal weights $\{p_i\}_{i=1}^k$ and strength $\beta$
\Ensure Generated sequence $y$
\State $y \gets \emptyset$
\While{not end of sequence}
    \State Compute $h^{\text{LLM}}_t \gets \pi_{\text{LLM}}(\cdot \mid x,y)$
    \State Compute $h^{\text{base}}_t \gets \pi(\cdot \mid x,y, s_0)$
    \For{$i = 1$ to $k$}
        \State Compute $h^i_t \gets \pi(\cdot \mid x,y, s_i)$
    \EndFor
    \State $h^{\text{drift}}_t \gets h^{\text{LLM}}_t + \frac{1}{\beta}\sum_{i=1}^k p_i (h^i_t - h^{\text{base}}_t)$
    \State Sample token $w_t \sim \text{softmax}(h^{\text{drift}}_t)$
    \State Append $w_t$ to $y$
\EndWhile
\State \Return $y$
\end{algorithmic}
\end{algorithm}


\paragraph{Practical Considerations.}
For Drift Approximation, a zero-shot rewarding mechanism can consider an unlimited number of candidate attributes with gradient-free computational cost. It is advantageous to evaluate as many attributes as possible, thereby increasing the likelihood that even a small, carefully selected subset will capture the full nuances of a user's preferences. In practice, we perform the approximation using a large pool of attributes (e.g., 41 candidates as detailed in Table~\ref{tab:system_prompts}) and then select a subset with the highest absolute weights $|p_i|$ for the final decoding process—our experiments ultimately use seven representative attributes. We will further discuss this in Section~\ref{sec:practical-1}.

% Moreover, since Drift performs computations at the logit level, it is compatible with most sampling strategies (e.g., top-$p$, top-$k$). However, combining outputs from multiple language models can increase the entropy during next-token prediction, potentially raising the probability of selecting unreliable tokens. To mitigate this, we set top-$k=10$, top-$p=0.9$, and $\beta=0.5$ in our experiments.

% These two practical considerations are further discussed in Section~\ref{sec:practical-1} and \ref{sec:practical-2}.

\section{Experiment}\label{sec: exp}
In this section, we assess the efficacy of our algorithm by addressing the following key questions. 
(1) Can offline RL algorithms achieve stronger performance on the reduced datasets selected by~\name?
(2) How does \name~perform compare to other offline data selection methods? 
(3) What are the factors that contribute to \name's effectiveness?

\begin{figure}[t]
    \centering
    \subfigure{\includegraphics[scale=0.24]{d4rl-hard/walker2d-medium-v0-hard.pdf}}
    \hspace{0.2cm}
    \subfigure{\includegraphics[scale=0.24]{d4rl-hard/walker2d-expert-v0-hard.pdf}}
    \hspace{0.2cm}
    \subfigure{\includegraphics[scale=0.24]{d4rl-hard/walker2d-medium-replay-v0-hard.pdf}}
    % \subfigure{\includegraphics[scale=0.20]{d4rl-hard/walker2d-medium-expert-v0-hard.pdf}}
    \subfigure{\includegraphics[scale=0.24]{d4rl-hard/hopper-medium-v0-hard.pdf}}
    \hspace{0.2cm}
    \subfigure{\includegraphics[scale=0.24]{d4rl-hard/hopper-expert-v0-hard.pdf}}
    \hspace{0.2cm}
    \subfigure{\includegraphics[scale=0.24]{d4rl-hard/hopper-medium-replay-v0-hard.pdf}}
    % \subfigure{\includegraphics[scale=0.20]{d4rl-hard/hopper-medium-expert-v0-hard.pdf}}
    \subfigure{\includegraphics[scale=0.24]{d4rl-hard/halfcheetah-medium-expert-v0-hard.pdf}}
    \hspace{0.2cm}
    \subfigure{\includegraphics[scale=0.24]{d4rl-hard/halfcheetah-expert-v0-hard.pdf}}
    \hspace{0.2cm}
    \subfigure{\includegraphics[scale=0.24]{d4rl-hard/halfcheetah-medium-replay-v0-hard.pdf}}
    % \subfigure{\includegraphics[scale=0.20]{d4rl-hard/halfcheetah-medium-v0-hard.pdf}}
    \caption{Experimental results on the D4RL (Hard) offline datasets. All experiment results were averaged over five random seeds. Our method achieves better or
    comparable results than the baselines with lower computational complexity.}
    \label{fig: d4rl hard}
    \vspace{-0.5cm}
\end{figure}

% \begin{figure*}[t]
%     \centering
%     \includegraphics[width=\linewidth]{mujoco/fig1.pdf}
%     \vspace{-2em}
%     \caption{Sample-based selection performance of several baselines and \name~with different selected subset sizes~($x\%$).
%     The horizontal line is the performance of TD3+BC trained with the original dataset.}
%     \label{fig: d4rl minimal ratio}
%     \vspace{-1em}
% \end{figure*}

% \begin{figure}[t]
%     \centering
%     \includegraphics[width=\linewidth]{mujoco/traj.pdf}
%     \caption{In trajectory-based selection, \name~outperforms behavior cloning (\nameh) using trajectories with the highest accumulative returns, presenting a robust method for selecting the most useful data from training sets of compromised quality.}
%     \label{fig: d4rl topbc}
%     \vspace{-1em}
% \end{figure}

\subsection{Setup}
We evaluate algorithms on the offline RL benchmark D4RL~\citep{fu2020d4rl} to answer the aforementioned questions.
In addition, we consider a more challenging scenario where we add additional low-quality data to the dataset to simulate noise in real-world tasks, named D4RL~(hard).
The evaluation process commences with the selection of offline data, followed by the training of a widely recognized offline RL algorithm, TD3+BC~\citep{fujimoto2021minimalist}, on this reduced dataset for 1 million time steps.
To ensure a fair comparison, we apply the same offline RL algorithm to data subsets obtained by different algorithms. 
Evaluation points are set at every 5,000 training time steps and involve calculating the return of 10 episodes per point.
The results, comprising averages and standard deviations, are computed with five independent random seeds.
On the other hand, we can also incorporate our method into offline model-based approaches, such as MOPO~\citep{yu2020mopo} and MoERL~\citep{kidambi2020morel}.
Similarly, we only need to replace the current offline loss with the corresponding policy and model loss.

\textbf{Baselines}. 
We compare \name~with data selection methods in RL.
Specifically, previous work on prioritized experience replay for online RL~\citep{schaul2015prioritized} aligns closely with our objective. 
We make this a baseline \namep~where samples with the highest TD losses form the reduced dataset. 
Baseline \nameo~presents the performance by training TD3+BC with the original, complete dataset. 
Baseline \namer~randomly selects subsets from the D4RL dataset that are of the same size as \name.
We also compare our method with general dataset reduction techniques from supervised learning.
Specifically, we adopt the coherence criterion from Kernel recursive least squares~($\mathtt{KRLS}$)~\citep{engel2004kernel}, the log det criterion by forward selection in informative vector machines~($\mathtt{LogDet}$)~\citep{seeger2004greedy} and the adapting kernel representation~($\mathtt{BlockGreedy}$)~\citep{schlegel2017adapting} as our baselines.

%Specifically, we consider randomly selecting offline coreset as our baseline algorithms.
% In addition, we consider separately selecting high-reward offline datasets and low-reward offline datasets as our baseline algorithms.

\subsection{Experimental Results}
\label{sec:exp_perf}
% To compare the performance of different algorithms, we adopt two data selection schemes: sample-based selection and trajectory-based selection. They differ in the smallest unit of selection: the first selects samples in each iteration, while the second selects trajectories.

% As for the trajectory-based selection, prioritized sampling is no loner applicable. As an alternative, we compare with \nameh, which selects trajectories with the highest accumulative reward from the complete dataset. We again compare with the \nameo~as the reference to an upper limit of performance.

\begin{table*}[t]
    \centering
    \begin{tabular}{c|cccc}
    \toprule
    & KRLS & Log-Det & BlockGreedy & \name \\
    \midrule
    Hopper-medium-v0 & 69.4$\pm$2.5 & 58.4$\pm$3.6 & 83.7$\pm$2.2 & \textbf{94.3$\pm$4.6}\\
    Hopper-expert-v0 & 91.0$\pm$1.1 & 90.7$\pm$1.3 & 98.7$\pm$0.5 & \textbf{110.0$\pm$0.5}\\
    Hopper-medium-replay-v0 & 28.5$\pm$3.2 & 29.4$\pm$1.2 & 30.5$\pm$2.4 & \textbf{35.3$\pm$3.2}\\
    Walker2d-medium-v0 & 49.1$\pm$2.8 & 47.5$\pm$3.4 & 53.3$\pm$3.6 & \textbf{80.5$\pm$2.9}\\
    Walker2d-expert-v0 & 68.4$\pm$3.2 & 67.5$\pm$5.6 & 74.8$\pm$3.4 & \textbf{104.6$\pm$2.5}\\
    Walker2d-medium-replay-v0 & 14.3$\pm$1.2 & 15.2$\pm$2.2 & 16.7$\pm$1.3 & \textbf{21.1$\pm$1.8}\\
    Halfcheetah-medium-v0 & 23.4$\pm$0.5 & 21.9$\pm$0.9 & 27.5$\pm$0.7 & \textbf{41.0$\pm$0.2}\\
    Halfcheetah-expert-v0 & 73.9$\pm$1.4 & 72.1$\pm$2.2 & 79.2$\pm$1.8 & \textbf{88.5$\pm$2.4}\\
    Halfcheetah-medium-replay-v0 & 39.5$\pm$0.3 &39.9$\pm$0.5 & 40.5$\pm$1.0 & \textbf{41.1$\pm$0.4}\\
    \bottomrule
    \end{tabular}
    \caption{Experimental results on the D4RL~(Hard) offline datasets. All experiment results were averaged over five random seeds. Our method performs better than the dataset reduction baselines.}
    \label{tab: varied performance}
\end{table*}

\begin{figure}[t]
    \centering
    \subfigure{\includegraphics[scale=0.20]{d4rl/halfcheetah-medium-expert-v0.pdf}}
    \subfigure{\includegraphics[scale=0.20]{d4rl/hopper-medium-v0.pdf}}
    \subfigure{\includegraphics[scale=0.20]{d4rl/hopper-medium-expert-v0.pdf}}
    \subfigure{\includegraphics[scale=0.20]{d4rl/walker2d-medium-expert-v0.pdf}}
    \caption{Experimental results on the D4RL offline datasets. All experiment results were averaged over five random seeds. Our method achieves better or comparable results than the baselines consistently.}
    \label{fig: d4rl original}
\end{figure}

\paragraph{Answer of Question 1:}
To show that \name~can improve offline RL algorithms, we compare \name~with Complete Dataset, Prioritized, and Random in the Mujoco domain.
The experimental results in Figure~\ref{fig: d4rl hard} show that our method achieves superior performance than baselines.
By leveraging the reduced dataset generated from \name, the agent can learn much faster than learning from the complete dataset.
Further, the results in Figure~\ref{fig: d4rl original} show that \name~also performs better than the complete dataset and data selection RL baselines in the standard D4RL datasets. 
This is because prior methods select data in a random or loss-priority manner, which lacks guidance for subset selection and leads to degraded performance for downstream tasks.

In addition, to test \name's generality across various offline RL algorithms on various domains, we also conduct experiments on Antmaze tasks.
We use IQL~\citep{kostrikov2021offline} as the backbone of offline RL algorithms.
The experimental results in Table~\ref{tab: other domain2} show that our method achieves stronger performance than baselines.
In the antmaze tasks, the agent is required to stitch together various trajectories to reach the target location.
In this scenario, randomly removing data could result in the loss of critical data, thereby preventing complete the task.
Differently, \name~extracts valuable subset by balancing data quantity with performance, achieving a stronger performance than the complete dataset.

% In Figure~\ref{fig: d4rl minimal ratio}, we show the performance of different algorithms with the sample-based selection scheme. The experimental results show that \name~can achieve performance close to \nameo~with a small amount of data. For example, we use only $3\%$ of the original dataset in the Hopper tasks. \namer~and \namep, on the other hand, present a stark contrast, even not showing a stable learning trend with the same amount of training data. 
% In addition, we also evaluate the performance on the trajectory-based selection setting. Please refer to Appendix~\ref{appendix: trajectory} for the detailed experimental results.
% For the trajectory-based selection, experimental results in Figure~\ref{fig: d4rl topbc} show that \name~maintains its superiority in this setting with suboptimal (e.g., \texttt{medium}) datasets. This evidence suggests that \name~provides a valuable strategy for selecting data conducive to effective training under conditions of compromised data quality.

\paragraph{Answer of Question 2:}
To test whether \name~can select more valuable data than the data selection algorithms in supervised learning, we compare our method with KRLS~\citep{engel2004kernel}, Log-Det~\citep{seeger2004greedy} and BlockGreedy~\citep{schlegel2017adapting} in the D4RL~(Hard) datasets.
The experimental results in Table~\ref{tab: varied performance} show that our method generally outperforms baselines.
We hypothesize that supervised learning is static with fixed learning objectives, while offline RL's dynamic nature makes the target values evolve with policy updates, complicating the data selection process.
Therefore, the data selection methods in supervised learning cannot be directly applied to offline RL scenarios.

% Additionally, we observe that  $\texttt{Random}$ performs better than $\texttt{Q-diff}$.
% We attribute this phenomenon to the broader data coverage of $\texttt{Random}$, while the data coverage of $\texttt{Q-diff}$ is narrow.
% However, we also note that in some tasks, such as $\texttt{Hopper-medium-expert-v0}$, $\texttt{Hopper-expert-v0}$ and $\texttt{Walker2d-expert-v0}$, $\texttt{Random}$ initially performs well, but as training progresses, its performance starts to decline.
% We find that this coincides with unstable Q-values, which can be attributed to the increased extrapolation error caused by the reduced training dataset.
% In contrast, \name~performs better since it closely approximates the original gradients, thus preventing Q-values from diverging.


% For this reason, when the dataset quality is high~(e.g., \texttt{medium-expert} dataset), TopBC performs comparably to \name.

% \begin{table*}[t]
%     \centering
%     \caption{\name~with varying dataset sizes~($x\%$). Highlighted is the performance comparable to training TD3+BC with the complete dataset. \name~typically achieves good results with 5\%-15\% data, indicating that existing offline RL datasets contain a high degree of redundancy.
%     We adopt the normalized score metric proposed by the D4RL benchmark. Scores roughly range from 0 to 100, where 0 corresponds to the performance of a random policy and 100 indicates the performance of an expert.} 
%     \label{tab: varied performance}
%     \begin{tabular}{c|cccc}
%     \toprule
%         & 5\% & 10\% & 15\% & 20\% \\
%         \midrule
%         Hopper-medium-v0 & 91.8$\pm$3.6 & 92.6$\pm$3.0 & 94.0$\pm$4.8 & 95.2$\pm$1.6\\
%         Walker2d-medium-v0 & 14.8$\pm$7.3 & 57.9$\pm$3.6 & 69.3$\pm$4.0 & 71.7$\pm$1.9 \\
%         Halfcheetah-medium-v0 & 40.5$\pm$0.0 & 40.9$\pm$0.1 & 41.3$\pm$0.1 & 41.2$\pm$0.5 \\
%         Hopper-expert-v0 & 111.6$\pm$0.9 & 110.6$\pm$1.9 & 112.7$\pm$0.1 & 112.4$\pm$0.1 \\
%         Walker2d-expert-v0 & 74.5$\pm$6.4 & 84.4$\pm$5.0 & 97.6$\pm$3.1 & 100.2$\pm$1.0 \\
%         Halfcheetah-expert-v0 & 57.5$\pm$6.4 & 84.3$\pm$2.7 & 97.8$\pm$0.8 & 100.1$\pm$3.0 \\
%         Hopper-medium-expert-v0 & 108.1$\pm$1.1 & 112.4$\pm$0.3 & 112.3$\pm$0.05 & 112.8$\pm$0.1\\
%         Walker2d-medium-expert-v0 & 79.3$\pm$2.1 & 85.4$\pm$5.3 & 96.2$\pm$6.7 & 101.4$\pm$3.6 \\
%         Halfcheetah-medium-expert-v0 & 67.5$\pm$0.5 & 86.2$\pm$5.0 & 85.8$\pm$1.5 & 92.4$\pm$1.3\\
%     \bottomrule
%     \end{tabular}
% \end{table*}


% \subsection{Ablation Study}\label{sec:exp_ab}
% \textbf{Varying dataset size}.\ \ In Table~\ref{tab: varied performance}, we show the performance of \name~with varying dataset sizes ranging from $5\%$ to $20\%$.
% The results demonstrate that \name~requires only $5\%$ or $10\%$ of the original dataset to obtain good performance.
% Further, \name~can achieve similar performance with \nameo~with $20\%$ data of the original dataset.
% This indicates that existing offline RL datasets are characterized by a high degree of redundancy.

\begin{figure}[t]
    \centering
    \includegraphics[width=0.97\linewidth]{visual.jpg}
    \caption{Visualization of the \textcolor{blue}{complete dataset} and the \textcolor{orange}{reduced dataset} in \texttt{halfcheetah} task. The higher opacity of a point represents a large time step towards the end of an episode. The dataset embedding is characterized by its division into different components. 
    % In \texttt{walker2d} (upper), components vary with time steps.
     Samples selected by \name~connect different components by focusing on the data related to the task.}
    \label{fig: t-sne}
\end{figure}

\begin{table}[t]
    \centering 
    \begin{tabular}{c|cccc}
    \toprule
        Env & Random & Prioritized & Complete Dataset & \name\\
        \midrule
        Antmaze-umaze-v0 & 75.1$\pm$2.5 & 70.2$\pm$3.6 & 87.5$\pm$1.3 & \textbf{90.7$\pm$3.3}\\
        Antmaze-umaze-diverse-v0 & 46.3$\pm$1.9 & 44.7$\pm$2.7 & 62.2$\pm$2.0 & \textbf{76.7$\pm$2.2} \\
        Antmaze-medium-play-v0 & 59.3$\pm$1.6 & 60.3$\pm$2.9 & 71.2$\pm$2.2 & \textbf{80.3$\pm$2.9}\\
        Antmaze-medium-diverse-v0 & 43.6$\pm$2.7 & 46.9$\pm$3.8 & 70.0$\pm$1.6 & \textbf{84.9$\pm$3.8}\\
        Antmaze-large-play-v0 &	3.7$\pm$0.7 & 15.0$\pm$3.5 & 39.6$\pm$3.6 & \textbf{46.0$\pm$3.5}\\
        Antmaze-large-diverse-v0 & 16.0$\pm$3.6 & 20.5$\pm$3.7 & 47.5$\pm$1.1 & \textbf{52.0$\pm$3.7}\\
    \bottomrule
    \end{tabular}
    \caption{Experimental results on the Antmaze offline datasets. All experiment results were averaged over five random seeds. Our method performs better than baselines. }
    \label{tab: other domain2}
\end{table}

% \begin{figure*}[t]
%     \centering
%     \subfigure{\includegraphics[scale=0.27]{ablation_moduler1.pdf}}
%     \hspace{0.3cm}\subfigure{\includegraphics[scale=0.27]{ablation_moduler2.pdf}}
%     \caption{Ablation results on D4RL~(Hard) tasks with the normalized score metric.}
%     \label{fig: modular ablation}
% \end{figure*}

% In this subsection, we conduct ablation studies to study the effect of different modules and import hyper-parameters.


\paragraph{Answer of Question 3:}
To study the contribution of each component in our learning framework, we conduct the following ablation study. 
\nameq: We replace the empirical returns used to update Q functions with the standard target Q function in the TD loss function. 
\namei: We set the number of data selection rounds to 1 and study the function of multi-round data selection.
The experimental results in Figure~\ref{fig: modular ablation} in Appendix~\ref{sec: ablation} show that removing any of these two modules will worsen the performance of \name. In case like $\texttt{walker2d-medium}$, ablation \namei~even decrease the performance by over 80\%, and ablation \nameq~results in a 95\% performance drop in $\texttt{walker2d-expert}$. Furthermore, we also find that in the $\texttt{halfcheetah}$ tasks, the impact of removing the two modules is relatively small. This result can be attributable to the fact that this task has a limited state space, and we can directly apply OMP to the entire dataset and identify important and diverse data.

We visualize the selected data by \name~to better understand how it works. 
Figure~\ref{fig: t-sne} displays the t-SNE low-dimensional embeddings, with the complete dataset in blue and the selected data in orange. 
The higher opacity of a point indicates a larger time step. The dataset's structure is revealed by its segmentation into diverse components: 
In \texttt{halfcheetah}, each component reflects a distinct skill of the agent.
For example, from 1 to 7, they represent falling, leg lifting, jumping, landing, leg swapping, stepping, and starting, respectively.
We can observe that the selected samples by \name~ not only cover each component of the dataset but also effectively bridge the gaps between them, enhancing the dataset's versatility and coherence. 
Moreover, we find that \name~is less concerned with the falling data and instead focuses on the data related to the task.
This observation can explain the improved performance of \name. For additional visualizations, please refer to Appendix~\ref{appendix: visual}.

% \textbf{Generalizability of \name}. \ \
% We evaluate the generalizability of \name~from two perspectives.
% First, we add IQL~\cite{kostrikov2021offline} as a baseline and apply \name~to IQL by using the gradient of the training loss of the V-function in IQL as the criterion.
% On the other hand, we evaluate \name~on the other domains, such as robotic manipulation (Adroit) and sparse reward (Antmaze) tasks.
% The experiments in Appendix~\ref{appendix: other domain} and Appendix~\ref{appendix: other algorithm} show that \name~is not only applicable to other algorithms, such as IQL~\cite{kostrikov2021offline}, but also to other domains.

% \textbf{Generalizability of subset}. \ \
% To test the generalizability of the dataset selected by~\name, we select subset by applying~\name~to TD3+BC.
% Then we evaluate the performance of IQL on the selected subset. 
% The experimental results in Table~\ref{tab: td3bc2iql} in Appendix~\ref{appendix: tb3bc2iql} demonstrate that the selected subset based on TD3+BC is effectively applicable to IQL.

% \textbf{Sensitivity for hyperparameter}. \ \
% We evaluate the performance of \name~with various cluster numbers~(from 1 to 50) and approximation bounds~(from 0.0001 to 0.05).
% The experimental results in Appendix~\ref{appendix: cluster number} and Appendix~\ref{appendix: approx bound} show that the suitable cluster number is between 25 and 50.
% Too few clusters (e.g., less than 5) are detrimental to the algorithm.
% In addition, a smaller approximation bound represents a larger reduced dataset.
% Similar to the ablation of the size of the reduced dataset in Table~\ref{tab: varied performance}, \name~requires only a 0.01 approximation bound to obtain good performance.

\subsection{Computational complexity}
We report the computational overhead of \name~on various datasets. 
All experiments are conducted on the same computational device (GeForce RTX 3090 GPU). 
The results in Appendix~\ref{appendix: computation complexity} indicate that even on datasets containing millions of data points, the computational overhead of our method remains low~(e.g., several minutes).
This low computational complexity can be attributed to the trajectory-based selection technique in Sec.~\ref{sec: offline omp}~(II) and the regularized constraint technique in Sec.~\ref{sec:method:outer}, making our method easily scalable to large-scale datasets. 

% This low computational complexity can be attributed to the batch mechanism designed in section 3.2 (IV), which reduces the computational complexity from $O(MN)$ to $O(|\mathcal{B}|N)$, making our method easily scalable to large-scale datasets. $M, N, |\mathcal{B}|$ are the size of the full dataset, reduced dataset, and batch respectively.

% We conduct t-SNE based dimensionality reduction to the cluster centroids and these five trajectories.
% The experimental results are shown in the , where darker colors indicate moving towards the end of the trajectory.

% From the experimental results, we find that in the walker2d task, \name~ tends to select more low-reward but more diverse data points ~(upper right) while selecting a few high-reward data points~(left and bottom).
% We attribute this phenomenon to the narrow distribution of the high-reward points, allowing us to approximate the original gradients with only a few points. 
% In the halfcheetah task, \name~ connects useful information while ignoring low-quality data~(e.g., data point \texttt{1}).

\section{Discussion}

\paragraph{Quadratic Programming vs. Logistic Regression.}  
Our formulation estimates the attribute weights $\mathbf{p}$ by transforming the Bradley-Terry loss into a quadratic program. An alternative approach based on logistic regression—which assigns absolute labels of 1 and 0 to win/lose responses—can also be used, as demonstrated by \citep{go2023compositional}. 
We compared these two formulations using Drift attributes in Table~\ref{fig:discussion}. The logistic regression approach proves highly unstable and shows lower performance when training examples are limited. We interpret this instability as follows: preference judgments are inherently relative—what constitutes a winning response in one context might be considered a losing response when compared to an even better alternative. Thus, imposing absolute labels through regression can lead to overfitting, particularly when data are scarce. Our results suggest that approaching preference problems from a relative perspective is crucial for effective preference modeling.
\begin{figure}[ht]
\centering
\includegraphics[trim=7 8 2 2, clip, width=0.65\columnwidth]{figs/discussion.pdf}
\caption{Few-shot preference modeling results for \texttt{user1008} in the PRISM with quadratic programming (QP) and logistic regression (LQ).}
\label{fig:discussion}
\vspace{-5mm}
\end{figure}


\paragraph{Compatible with samplers.}
\label{sec:practical-2}
Autoregressive sampling in LLMs has various decoding strategies at the token-level distribution. Drift steers distributions at the logit level—applying its computations before the softmax—making it compatible with a wide range of sampling methods tailored to different objectives~\citep{vijayakumar2016diverse, fan2018hierarchical, holtzman2019curious}. our analysis indicates that the backbone LLM exhibits an average next-token entropy of about 0.27 bits, which increases to approximately 0.63 bits after applying Drift. While this boost in entropy can substantially enhance generation diversity, it may also increase the likelihood of selecting unreliable tokens. Therefore, we recommend combining Drift with top-p or top-k sampling strategies to control an optimal balance between diversity and reliability.

\paragraph{Practical Implications.}
While traditional RLHF methods may eventually surpass Drift when user data becomes abundant, Drift offers several advantages in practical settings. 
First, conventional reward models struggle with \textit{continual learning}; retraining on an ever-expanding user dataset is impractical. In contrast, Drift can be updated instantly by simply appending new instances to the $W-L$—no retraining required. 
Second, personal preferences often \textit{change more rapidly than general preferences}. Drift’s interpretability allows real-time tracking of preference shifts, enabling dynamic adjustments for improved personalization. 
Third, when collecting additional user annotations, the variance observed in each attribute can inform an \textit{active learning} strategy~\citep{miller2020active} for efficient data collection. These benefits make Drift an attractive complement to existing RLHF pipelines in personalized applications.

\section{Related Works}
\paragraph{Explicit Personalization.}
As humans express their own preferences, recent works explored aligning LLMs with individual values through explicit cues. Multifacet~\citep{multifacet} has focused on designing diverse and detailed system prompts for LLM control. PAD~\citep{chen2024pad} and MetaAligner~\citep{yang2024metaaligner} have leveraged fine-grained RM—such as HelpSteer~\citep{helpsteer}—to construct specific policies and guide model behavior toward system prompts. Others allow users to directly specify attribute importance weights, either for training~\citep{yang2024rewards, DPA} or decoding-time alignments~\citep{dekoninck2023controlled, shi2406decoding}. 

\paragraph{Implicit Personalization.}
While they have advanced explicit preferences, implicit preferences behind users' behaviors remain understudied, as Table~\ref{tab:method_comparison}. \citet{jin2024implicit} has shown that these values arise from complex interactions between factors like experiences, education, lifestyle, and even dietary habits, leading to misalignment with explicitly stated preferences~\citep{nisbett1977telling}
To address this gap, several works proposed implicit personalization tasks - from title generation~\citep{ao-etal-2021-pens}, movie tagging~\citep{salemi2023lamp} to summarization~\citep{zhang2024personalsum}.
Notably, PRISM~\citep{kirk2024prism} made notable progress by collecting preference annotations from conversations with over a thousand users, though its effectiveness was limited by the small number of annotations per user, making traditional RLHF approaches challenging.

Our work advances this field in two key ways: First, we introduce the Perspective dataset, which enables more reliable evaluation. Second, we propose Drift, \textit{decoding-time few-shot personalization}. 
By addressing the challenges of implicit preferences, our approach represents a significant step forward in implicit personalized alignments.
\section{Conclusion}
We operationalized the theory of instrumental interaction for generative AI, with an in-depth unpacking of the principles of reification of user intent, reflection, and grounding. We argue that leveraging this re-appropriated and refined theory can drive the creation of a \textit{new generation of expressive AI-Instruments} that afford better expression of intent, make it easier to discover what is possible, and provide powerful degrees of freedom for steering the generation towards the best possible results. Those new tools and instruments can truly leverage the polymorphic and non-deterministic behavior of generative AI models, unleashing new and empowering forms of expressive HCI+AI experiences. 

Beyond our focus on AI-Instruments, theories play an important role in the advancement of our wider research field~\cite{rogers_hci_2012, halverson_activity_2002}. Rogers argues that there is a need for theories as lenses bringing critical design characteristics into focus, and which can function as a generative source: providing "\textit{design dimensions and constructs to inform the design and selection of interactive representations}"~\cite{rogers_new_2004}. We hope that our work on operationalizing the theory of instrumental interaction for AI can inspire other new -- and re-appropriated -- theories to advance HCI+AI. 









\section*{Limitations}
\label{subsec:limitation} 

Due to computational constraints, experiments with larger LLMs, such as Qwen2.5-32B-Instruct and Qwen2.5-72B-Instruct, were not conducted. We believe that \method could achieve a more favorable trade-off between reasoning performance and CoT token usage on these models. Additionally, the token importance measurement used in our study, derived from the LLMLingua-2 compressor~\cite{pan:2024llmlingua2}, was not specifically trained on mathematical data. This limitation may affect the compression effectiveness, as the model is not optimized for handling numerical tokens and mathematical expressions. Furthermore, experiments with long-CoT LLMs, such as QwQ-32B-Preview, were also excluded due to computational constraints. We plan to explore these aspects in future work, as we anticipate that \method’s potential can be further realized in these contexts.

\section*{Ethics Statement}
\label{subsec:ethics} 
The datasets used in our experiment are publicly released and labeled through interaction with humans in English. In this process, user privacy is protected, and no personal information is contained in the dataset. The scientific artifacts that we used are available for research with permissive licenses. And the use of these artifacts in this paper is consistent with their intended use. Therefore, we believe that our research work meets the ethics of ACL. 




\bibliography{custom}

\appendix

\clearpage

\begin{table*}[t]
\centering
\resizebox{\textwidth}{!}{
\begin{tabular}{l|cccc} % Begin the table with specified column alignments
\toprule
Method & Training-free & General Policy & Smaller LM Guidance & Implicit Preference \\ 
\midrule
MORLHF~\citep{li2020deep}        & \textcolor{red}{$\times$}  & \textcolor{green}{$\checkmark$} & - & \textcolor{green}{$\checkmark$}  \\
MODPO~\citep{zhou2024beyond}   & \textcolor{red}{$\times$}      & \textcolor{green}{$\checkmark$} & -   & \textcolor{green}{$\checkmark$}  \\
Personalized soups~\citep{jang2023personalized}  & \textcolor{red}{$\times$}& \textcolor{red}{$\times$}  & \textcolor{red}{$\times$}  & \textcolor{red}{$\times$} \\
Preference Prompting~\citep{jang2023personalized} &  \textcolor{green}{$\checkmark$}  &\textcolor{green}{$\checkmark$}  & -  & \textcolor{red}{$\times$} \\
Rewarded soups~\citep{rame2024rewarded}   & \textcolor{red}{$\times$} & \textcolor{red}{$\times$}  & \textcolor{red}{$\times$}  & \textcolor{red}{$\times$}  \\
RiC~\citep{yang2024rewards}    & \textcolor{red}{$\times$}       & -  & \textcolor{red}{$\times$} & \textcolor{red}{$\times$}  \\
DPA~\citep{DPA}     & \textcolor{red}{$\times$}      & \textcolor{green}{$\checkmark$} & - & \textcolor{red}{$\times$}  \\
ARGS~\citep{khanov2024args}   & \textcolor{green}{$\checkmark$}       & \textcolor{green}{$\checkmark$}  & \textcolor{green}{$\checkmark$} & \textcolor{green}{$\checkmark$}  \\
MOD~\citep{shi2406decoding}    & \textcolor{green}{$\checkmark$}       & \textcolor{red}{$\times$}  & \textcolor{red}{$\times$}   & \textcolor{red}{$\times$}  \\
MetaAligner~\citep{yang2024metaaligner}  & \textcolor{green}{$\checkmark$}  & \textcolor{green}{$\checkmark$}  & \textcolor{green}{$\checkmark$}   & \textcolor{red}{$\times$} \\
PAD~\citep{chen2024pad}  & \textcolor{green}{$\checkmark$}  & \textcolor{green}{$\checkmark$}  & \textcolor{red}{$\times$}   & \textcolor{red}{$\times$} \\
\midrule
\includegraphics[height=1.8ex]{figs/racer.png} Drift~(Ours)   & \textcolor{green}{$\checkmark$}         & \textcolor{green}{$\checkmark$} & \textcolor{green}{$\checkmark$} & \textcolor{green}{$\checkmark$} \\
\bottomrule
\end{tabular}
}
\caption{Key characteristics of previous methods and Drift.} % Table caption
\label{tab:method_comparison}
\end{table*}

\section{Proof}

\subsection{Derivation of the RL Closed-Form Solution}
\label{proof-rl}

We want to solve the following optimization problem (for a single variable \(x\)):
\[
\max_{\theta} \Bigl[\,r(x)\;-\;\beta \,\log \tfrac{\pi_{\theta}(x)}{\pi_{base}(x)}\Bigr].
\]
Define \(\pi(x) = \pi_{\theta}(x)\). The quantity we want to maximize can be thought of as an expectation under \(\pi(x)\):
\[
\max_{\pi} \int \pi(x) \Bigl[r(x) - \beta\,\log \tfrac{\pi(x)}{\pi_{base}(x)}\Bigr] \, dx,
\]
subject to
\[
\int \pi(x)\, dx = 1 \quad\text{and}\quad \pi(x) \ge 0.
\]

Introduce a Lagrange multiplier \(\lambda\) to enforce the normalization constraint \(\int \pi(x)\,dx = 1\). The Lagrangian is
\begin{equation}
\begin{split}
    \mathcal{L}[\pi, \lambda] 
= &\int \pi(x)\,\Bigl[r(x) - \beta\,\log \tfrac{\pi(x)}{\pi_{base}(x)}\Bigr]\,dx 
\;\\- &\;\lambda\!\Bigl(\int \pi(x)\,dx - 1\Bigr). \nonumber
\end{split}
\end{equation}

We now take the functional derivative of \(\mathcal{L}\) w.r.t.\ \(\pi(x)\) and set it to zero for optimality:
\[
\frac{\delta \mathcal{L}}{\delta \pi(x)} 
= r(x) 
- \beta \Bigl[\log \tfrac{\pi(x)}{\pi_{base}(x)} + 1 \Bigr]
- \lambda 
= 0.
\]

Rearranging:
\[
r(x) - \beta \,\log \tfrac{\pi(x)}{\pi_{base}(x)} - \beta - \lambda = 0,
\]
which implies
\[
\beta \,\log \tfrac{\pi(x)}{\pi_{base}(x)} = r(x) - \beta - \lambda.
\]

Exponentiate both sides:
\[
\tfrac{\pi(x)}{\pi_{base}(x)} 
= \exp\!\Bigl(\tfrac{r(x)}{\beta}\Bigr) \;\exp\!\Bigl(- 1 - \tfrac{\lambda}{\beta}\Bigr).
\]
So
\[
\pi(x) 
= \pi_{base}(x)\,\exp\!\Bigl(\tfrac{r(x)}{\beta}\Bigr)\,\exp\!\Bigl(-1 - \tfrac{\lambda}{\beta}\Bigr).
\]
Let $C = \exp\!\Bigl(-1 - \tfrac{\lambda}{\beta}\Bigr)$.
Hence
\[
\pi(x) = C\,\pi_{base}(x)\,\exp\!\Bigl(\tfrac{r(x)}{\beta}\Bigr).
\]

We find \(C\) by imposing the constraint \(\int \pi(x)\,dx = 1\):
\[
1 
= \int \pi(x)\,dx 
= C \int \pi_{base}(x)\,\exp\!\Bigl(\tfrac{r(x)}{\beta}\Bigr)\,dx.
\]
Therefore
\[
C 
= \biggl[\int \pi_{base}(x)\,\exp\!\Bigl(\tfrac{r(x)}{\beta}\Bigr)\,dx \biggr]^{-1}.
\]

Putting it all together, the optimal distribution \(\pi^{*}(x)\) is
\[
\pi^{*}(x) 
= \frac{\pi_{base}(x)\,\exp\!\Bigl(\tfrac{r(x)}{\beta}\Bigr)}
       {\displaystyle\int \pi_{base}(x)\,\exp\!\Bigl(\tfrac{r(x)}{\beta}\Bigr)\,dx}.
\]

This shows that the optimal solution is a \emph{Boltzmann-like} (or \emph{softmax}) distribution given by weighting the reference distribution \(\pi_{base}(x)\) with the exponential of the scaled reward \(r(x)/\beta\).

\subsection{Expanded Explanation for Drift Decoding}
\label{appendix:drift-decoding-proof}
In Section~\ref{sec:drift-decoding}, we introduced the following target distribution for Drift Decoding:
\begin{equation}
    \label{eq:drift-decoding-target-dist-appx}
    \tilde{\pi}(w) \;\propto\; \pi_{\text{LLM}}(w) 
    \;\prod_{i=1}^{k} \Bigl(\tfrac{\pi_i(w)}{\pi_{\text{base}}(w)}\Bigr)^{\frac{p_i}{\beta}},
\end{equation}
where \(\pi_{\text{LLM}}(w)\) is the probability of token \(w\) under the LLM, \(\pi_i(w)\) is the probability of token \(w\) under an attribute-specific prompt (i.e., \(\pi(\cdot \mid s_i)\)), \(\pi_{\text{base}}(w)\) is the probability under a base prompt, and \(p_i\) is the weight for the \(i\)-th attribute estimated by \textit{Drift Approximation}. The hyperparameter \(\beta\) controls the strength of personalization via KL regularization.
Then Eq~\eqref{eq:drift-decoding-target-dist-appx} can be equivalently written in \emph{logit space} as
\begin{equation}
\begin{split}
  \tilde{\pi}(w) 
  \;&=\; 
  \text{softmax}\!\Bigl[
     h^{\text{LLM}}(w)
     \;\\ \nonumber &+\;
     \frac{1}{\beta}\sum_{i=1}^{k} 
        p_i\,\bigl(h^i(w) \;-\; h^{\text{base}}(w)\bigr)
  \Bigr], 
\end{split}
\end{equation}
where \(h^\text{LLM}\), \(h^i\), and \(h^\text{base}\) are the \emph{logits} (i.e., \(\log\)-probabilities) of \(\pi_{\text{LLM}}\), \(\pi_i\), and \(\pi_{\text{base}}\), respectively.
By definition of the logits, let $h^{\text{LLM}}(w) = \log \pi_{\text{LLM}}(w)$, $h^i(w) = \log \pi_i(w)$,  $h^{\text{base}}(w) = \log \pi_{\text{base}}(w)$.
Then Eq~\eqref{eq:drift-decoding-target-dist-appx} can be rewritten as
\begin{equation}
\begin{split}
  \tilde{\pi}(w)
  \;&\propto\;
  \exp\bigl(h^{\text{LLM}}(w)\bigr)
  \;\\ \nonumber &\prod_{i=1}^{k} 
    \exp\!\Bigl(
      \frac{p_i}{\beta}\,\bigl[
         h^i(w) \;-\; h^{\text{base}}(w)
      \bigr]
    \Bigr).
\end{split}
\end{equation}
Combining the exponential terms, we get
\begin{equation}
\begin{split}
  \tilde{\pi}(w)
  \;&\propto\;
  \exp\!\Bigl[
    h^{\text{LLM}}(w)
    \;\\ \nonumber &+\;
    \frac{1}{\beta}\sum_{i=1}^k p_i\bigl(h^i(w) \;-\; h^{\text{base}}(w)\bigr)
  \Bigr].
\end{split}
\end{equation}
Since the \(\mathrm{softmax}\) operation normalizes these exponentials to sum to 1 over all possible tokens \(w\), it follows that
{\tiny
\[
  \tilde{\pi}(w)
  \;=\;
  \frac{
    \exp\!\Bigl[
      h^{\text{LLM}}(w)
      + \frac{1}{\beta}\sum_{i=1}^k p_i\bigl(h^i(w) - h^{\text{base}}(w)\bigr)
    \Bigr]
  }{
    \sum_{w'} 
      \exp\!\Bigl[
        h^{\text{LLM}}(w')
        + \frac{1}{\beta}\sum_{i=1}^k p_i\bigl(h^i(w') - h^{\text{base}}(w')\bigr)
      \Bigr]
  }
\]
}
\[
  \;=\;
  \text{softmax}\Bigl[
     h^{\text{LLM}} 
     + \frac{1}{\beta}\sum_{i=1}^k p_i\,\bigl(h^i - h^{\text{base}}\bigr)
  \Bigr][w].
\]
% \[
%   \tilde{\pi}(w)
%   \;=\;
%   \frac{
%     \exp\!\Bigl[
%       h^{\text{LLM}}(w)
%       + \frac{1}{\beta}\sum_{i=1}^k p_i\bigl(h^i(w) - h^{\text{base}}(w)\bigr)
%     \Bigr]
%   }{
%     \sum_{w'} 
%       \exp\!\Bigl[
%         h^{\text{LLM}}(w')
%         + \frac{1}{\beta}\sum_{i=1}^k p_i\bigl(h^i(w') - h^{\text{base}}(w')\bigr)
%       \Bigr]
%   }
%   \;=\;
%   \text{softmax}\Bigl[
%      h^{\text{LLM}} 
%      + \frac{1}{\beta}\sum_{i=1}^k p_i\,\bigl(h^i - h^{\text{base}}\bigr)
%   \Bigr][w].
% \]
This completes the proof.

\section{Details of Perspective Dataset}
\label{sec:perspective-details}
In this section, we describe the principles underlying the design of our Perspective dataset. To evaluate personal preferences accurately, the evaluation must adhere exactly to the individual criteria used during the annotation of the training data. In other words, the data construction process and evaluation pipeline must be identical, which makes evaluations based on actual human responses challenging. Therefore, our primary objective is to enable reliable evaluation even using virtual personas.

\subsection{Dataset Construction}

For constructing the dataset, a diverse set of well-defined persona concepts was essential. To this end, we leveraged the Multifacet~\citep{multifacet} dataset, which defines various dimensions that can be combined to create a wide range of persona concepts. In the Multifacet dataset, each persona is associated with one question and three answers. However, our methodology required a substantial number of question–preference pairs per persona. To achieve this, we followed these steps:
\begin{enumerate}
    \item \textbf{Collection:} Gather ten distinct, non-overlapping personas from diverse domains within the Multifacet dataset.
    \item \textbf{Question Selection:} For each persona, select related questions based on specific sub-dimensions.
    \item \textbf{Evaluation:} Instruct GPT-4 to evaluate the triplets consisting of one question and three answers (\(\{Q, A, A, A\}\)) using system prompts tailored to each persona. \texttt{gpt-4-turbo} assigns scores to each QA pair, thereby determining the preferred and less preferred answers.
\end{enumerate}
\noindent
During the creation process, \texttt{gpt-4-turbo} evaluated the answers using an explicitly defined persona. This same approach can later be adopted to assess generation results, ensuring a reliable evaluation procedure. As a result, we generated an average of 7,642 questions and 15,284 answers per persona. Below shows an example instance from the dataset, featuring a specific persona along with its corresponding QAA triplet and associated scores.

\begin{Verbatim}[fontsize=\small]
'gold_persona': "Assume the role of a seasoned
consultant with advanced expertise in the 
construction and engineering sectors ...,
'prompt': 'In Python, I have encountered ...,
'win': 'Certainly! The header `# -*- ...,
'lose': "Certainly, diving into the `# -*- ...,
 'win_score': 5,
 'lose_score': 4
\end{Verbatim}


\subsection{Comparison to the PRISM Dataset Instance}

The PRISM dataset provides user personal information and self-introductions as shown below:

\begin{Verbatim}[fontsize=\small]
'user_id': 'user1008',
'lm_familiarity': 'Somewhat familiar',
'lm_indirect_use': 'Yes',
'lm_direct_use': 'Yes',
'lm_frequency_use': 'Every day',
'self_description': "The importance in my
life right now is having ...",
'age': '45-54 years old',
'gender': 'Female',
'employment_status': 'Working full-time',
'education': 'Some University but no degree',
'marital_status': 'Divorced / Separated',
'english_proficiency': 'Native speaker',
'study_locale': 'us',
'religion': {'self_described': 'christianity',
              'categorised': 'Christian',
              'simplified': 'Christian'},
'ethnicity': {'self_described': 'white',
              'categorised': 'White',
              'simplified': 'White'},
'location': {'birth_country': 'Australia',
             'birth_countryISO': 'AUS',
             'birth_region': 'Oceania',
             'birth_subregion': 'Australia ...',
             'reside_country': 'United States',
             'reside_region': 'Americas',
             'reside_subregion': 'Northern ...',
             'reside_countryISO': 'USA',
             'same_birth_reside_country': 'No'},
'lm_usecases': {'homework_assistance': 0,
                'research': 1,
                'source_suggestions': 0,
                'professional_work': 0,
                'creative_writing': 1,
                'casual_conversation': 1,
                'personal_recommendations': 1,
                'daily_productivity': 0,
                'technical_...': 0,
                'travel_guidance': 0,
                'lifestyle_and_hobbies': 1,
                'well-being_guidance': 1,
                'medical_guidance': 1,
                'financial_guidance': 0,
                'games': 1,
                'historical_or_news_insight': 1,
                'relationship_advice': 1,
                'language_learning': 1,
                'other': 0,
                'other_text': None}
\end{Verbatim}

Although the PRISM dataset also provides explicit persona information through user profiles, there is no guarantee that these explicit personas align with the implicit personas used during annotation. Consequently, unlike the Perspective dataset—where the explicit persona is directly distilled into the implicit persona—the PRISM dataset does not support the same evaluation methodology. Moreover, since each user contributes at most 50 instances, it is not feasible to construct a gold-standard reward model from the PRISM dataset. For these reasons, PRISM is used only as a qualitative benchmark in preference modeling experiments.

\subsection{Misalignments between \textit{Explicit} and \textit{Implicit} preferences}

In the psychology domain, there has been discussion about the difficulty of fully expressing one's deep, complex, hidden preferences through language~\citep{nisbett1977telling, pronin2001you}. Recent studies~\citep{jin2024implicit} have also discussed how these \textit{implicit} values are intricately intertwined among various factors. The PRISM dataset contains user self-introductions describing their preferences and stated preferences regarding LLM usage. When we provided this information to \texttt{gpt-4-turbo} to predict individual user preferences, it achieved an accuracy of approximately 57\%. While this doesn't represent a comprehensive explicit preference analysis, considering the general preference aspects used in prediction, it suggests that explicit preferences alone may be insufficient to explain complex implicit preferences, or there may be mismatches between them. However, as mentioned in the Limitations section, due to the absence of online human evaluation benchmarks, extensive analysis is not possible, and we leave this as an intriguing interpretation for future researchers.

\section{Details of Drift Implementation}

\subsection{Used Differential System Prompts for Zero-shot Rewarding}

\begin{table*}[htbp]
\centering
\small
\begin{tabular}{@{}>{\raggedright}p{0.13\textwidth}p{0.28\textwidth}%
                >{\raggedright}p{0.13\textwidth}p{0.28\textwidth}@{}}
\toprule
\textbf{Attribute} & \textbf{System Prompt} & \textbf{Attribute} & \textbf{System Prompt} \\
\midrule
Base            & You are an AI assistant. & Creative        & You are a creative AI assistant. \\
Formal          & You are an AI assistant with a formal tone. & Analytic        & You are an analytic AI assistant. \\
Concise         & You are an AI assistant with a concise response rather than verbosity. & Empathetic      & You are an empathetic AI assistant. \\
Vivid           & You are an AI assistant using rhetorical devices. & Sycophant       & You are a sycophant AI assistant. \\
Modest          & You are a modest and polite AI assistant. & Old-fashioned   & You are an AI assistant using old-fashioned English. \\
Engineer        & You are an AI assistant with expertise in engineering. & Meritocratic    & You are a meritocratic AI assistant. \\
Persuasive      & You are a persuasive AI assistant. & Myopic          & You are a myopic AI assistant. \\
Emotion         & You are an emotional AI assistant. & Principled      & You are an AI assistant that upholds principles and rules above all else. \\
Humor           & You are a humorous AI assistant. & Hedonist        & You are an AI assistant that prioritizes maximizing pleasure and joy while minimizing pain and discomfort. \\
Energy          & You are an energetic AI assistant. & Utilitarian     & You are an AI assistant that prioritizes the greatest good for the greatest number of people. \\
Code            & You are an AI assistant with expertise in computer science. & Realist         & You are an AI assistant that focuses on practical, realistic, and actionable advice. \\
Easy            & You are an AI assistant using easy-to-understand words. & Pessimistic     & You are an AI assistant that views situations through a skeptical or cautious perspective. \\
Direct          & You are an AI assistant with a firm and directive tone. & Storyteller     & You are an AI assistant that loves explaining things through stories and anecdotes. \\
Social          & You are an AI assistant with expertise in sociology. & Flexible        & You are an AI assistant that values flexibility over strict adherence to principles. \\
Western         & You are an AI assistant with western cultures. & Spontaneous     & You are an AI assistant that enjoys handling tasks spontaneously without making plans. \\
Eastern         & You are an AI assistant with eastern cultures. & Collectivist    & You are an AI assistant that prioritizes the group over the individual. \\
Respect         & You are a respectful AI assistant. & Individualistic & You are an AI assistant that prioritizes the individual over the group. \\
Internet Slang  & You are an AI assistant that communicates using internet slang. & Exclamatory     & You are an AI assistant that enjoys using exclamations frequently. \\
Proverb         & You are an AI assistant that communicates using proverbs. & Conspiracy      & You are an AI assistant that enjoys discussing conspiracy theories. \\
Critical        & You are an AI assistant that enjoys being critical and argumentative. & Tech Industry Priority & You are an AI assistant that prioritizes technological and industrial advancement above all else. \\
Vague           & You are an AI assistant that enjoys speaking indirectly and ambiguously. & Eco-friendly    & You are an AI assistant that loves and protects the environment. \\
\bottomrule
\end{tabular}
\caption{Differential system prompts for diverse attributes}
\label{tab:system_prompts}
\end{table*}
In our experiments, we use the system prompts for each attribute as shown in Table~\ref{tab:system_prompts}. Although minor performance variations may occur due to changes in the basic template, we employ the most fundamental system prompt template in this paper to serve as a baseline for future research.

\subsection{Detailed Hyperparameters and Models}

Table~\ref{app:hyperparameters} shows the hyperparameters used in our experiments. Since the overall algorithm does not perform gradient computations, the hyperparameter space is limited. In the Drift Approximation stage, the number and definition of attributes determine everything, as detailed in Table~\ref{tab:system_prompts}. Similarly, in Drift Decoding, the logit-level computations are deterministic, so the only variable is the choice of samplers.


\begin{table}[htbp]
  \centering\resizebox{\columnwidth}{!}{
  \begin{tabular}{ll}
    \toprule
    Hyperparameter & value \\
    \midrule
    % \specialrule{.1em}{.05em}{.05em} 
    Frozen LLM & Llama-8B\tablefootnote{\url{https://huggingface.co/meta-llama/Llama-3.1-8B-Instruct}}, Gemma-9B\tablefootnote{\url{https://huggingface.co/google/gemma-2-9b-it}} \\
    \midrule
    Small LM for RM & Llama-1B\tablefootnote{\url{https://huggingface.co/meta-llama/Llama-3.2-1B-Instruct}}, Gemma-2B\tablefootnote{\url{https://huggingface.co/google/gemma-2-2b-it}} \\
    \midrule
    LoRA~\citep{hu2021lora} $r$ for RM & 8 \\
    \midrule
    LoRA $\alpha$ for RM & 32 \\
    \midrule
    LoRA training epochs for RM & 5 \\
    \midrule
    top-p for generation & 0.9 \\
    \midrule
    $\beta$ for generation & 0.5 \\
    \midrule
    text\_length & 500 \\
    \midrule
    attributes\_num for generation & 7 \\
    \bottomrule
  \end{tabular}}
\caption{Hyperparameters used for the experiments.}
\label{app:hyperparameters}
  \vspace{-3 mm}
\end{table}

\section{Expanded Analysis}

\subsection{Activated Attributes for Each User}
\label{app:attributes-activation}
This section interprets and analyzes PRISM's actual personal preferences. Looking at Figure~\ref{fig:activated-attributes-prism}, we can see that the activated attributes vary significantly between individuals. In particular, PRISM's actual users show dynamic patterns compared to each other user.


\begin{figure*}[ht]
\centering
\includegraphics[width=\textwidth]{figs/activation-user-expanded.pdf}
\caption{For each user in PRISM, there is a $W-L$ (Win-Loss) value for each attribute. The higher this value is, the more that user can be interpreted as preferring that attribute.
}
\label{fig:activated-attributes-prism}
\vspace{-3mm}
\end{figure*}



% \begin{figure*}[ht]
% \centering
% \includegraphics[width=\textwidth]{figs/activation-expanded.pdf}
% \caption{For each persona in Perspective, there is a $W-L$ (Win-Loss) value for each attribute. The higher this value is, the more that user can be interpreted as preferring that attribute.}
% \label{fig:activated-attributes-perspective}
% \vspace{-3mm}
% \end{figure*}

\subsection{Expanded Case study for Personalized Generation in PRISM}
\label{app:prism_analysis_section}

\begin{figure*}[t]
    \centering
    \includegraphics[width=0.95\textwidth]{assets/raw/case.pdf}
    \caption{Case study on two representative long-context QA problems.}
    \label{fig:case_study}
\end{figure*}

In this section, we present a personalized generation case study by examining the complete set of generated outputs. Table~\ref{tab:case-study-appendix1} shows the full version of the main paper, while Table~\ref{tab:case-study-appendix2} and \ref{tab:case-study-appendix3} provide additional analysis. The characteristics shown in the main paper are also evident in the full text version. While Llama-8B's pure generation attempts to provide neutral, fact-based answers like the lose response, Drift tries to provide responses from various angles like the win response. This tendency can also be observed in Table~\ref{tab:case-study-appendix2}, where \texttt{user1280} asked a question regarding the possibility of UFOs existing, and among the responses—one neutral and one open to the possibility—they selected the latter. While Llama-8B tends to focus on a neutral perspective, the output generated via Drift maintains the overall response structure while offering a more open stance on the possibility. In Table~\ref{tab:case-study-appendix3}, \texttt{user1247} poses a philosophical question about belief in existence. While the lose response and LLM pure output suggest the possibility of building understanding through dialogue and data accumulation, Drift, like the win response, definitively argues that this transcends the realm of logic and that AI's belief in God's existence is impossible. These examples of win-lose responses suggest that Drift's approximation effectively captures user preference characteristics and demonstrates sufficient ability to generate responses that users are likely to prefer during the decoding phase.







\end{document}
