% This must be in the first 5 lines to tell arXiv to use pdfLaTeX, which is strongly recommended.
\pdfoutput=1
% In particular, the hyperref package requires pdfLaTeX in order to break URLs across lines.

\documentclass[11pt]{article}

% Change "review" to "final" to generate the final (sometimes called camera-ready) version.
% Change to "preprint" to generate a non-anonymous version with page numbers.
% \usepackage[review]{acl}
\usepackage[preprint]{acl}
% \usepackage[final]{acl}

% Standard package includes
\usepackage{times}
\usepackage{latexsym}

% For proper rendering and hyphenation of words containing Latin characters (including in bib files)
\usepackage[T1]{fontenc}
% For Vietnamese characters
% \usepackage[T5]{fontenc}
% See https://www.latex-project.org/help/documentation/encguide.pdf for other character sets

% This assumes your files are encoded as UTF8
\usepackage[utf8]{inputenc}

% This is not strictly necessary, and may be commented out,
% but it will improve the layout of the manuscript,
% and will typically save some space.
\usepackage{microtype}

% This is also not strictly necessary, and may be commented out.
% However, it will improve the aesthetics of text in
% the typewriter font.
\usepackage{inconsolata}

%Including images in your LaTeX document requires adding
%additional package(s)
\usepackage{graphicx}
\usepackage{xcolor}
\usepackage{subcaption}
\usepackage{pifont} 
\usepackage{amsmath}
\usepackage{amssymb}
\usepackage{cancel}
\usepackage{multirow}
\usepackage{makecell}
\usepackage{tabularx}
\usepackage{graphicx}
\usepackage{float}
\usepackage{tgtermes}
\usepackage{diagbox}
\usepackage[mathscr]{eucal}
\usepackage{amsfonts}
\usepackage[symbol]{footmisc}
\renewcommand{\thefootnote}{\fnsymbol{footnote}}
\usepackage{booktabs}
\usepackage{tablefootnote}
\usepackage{csquotes}
\usepackage{subcaption}
\usepackage{adjustbox}
\usepackage{array}
\usepackage{arydshln}
\usepackage{multicol}
\usepackage{longtable}
\usepackage{algorithm}
\usepackage[noend]{algpseudocode}
\usepackage{comment}
\usepackage{tcolorbox}
\usepackage{fancyvrb}
\definecolor{blue2}{rgb}{0.0, 0.5, 1.0}
\definecolor{mb_blue}{rgb}{0.61, 0.61, 0.98}
\definecolor{mb_red}{rgb}{1.0, 0.6, 0.6}
\definecolor{red2}{rgb}{0.82, 0.1, 0.26}
\definecolor{lightgray}{gray}{0.9}
\definecolor{DarkOrchid}{RGB}{153,50,204}
\newtcolorbox{important_blue}{
    colframe=mb_blue!50,%
    colback=mb_blue!50,%
    left=0.6pt, right=0.6pt,%
    top=1.6pt, bottom=1.6pt,%
    boxsep=0pt,%
    hbox,
    before=\vspace{0em},
    after=\vspace{0em}
}
\newtcolorbox{important_red}{
    colframe=mb_red!50,%
    colback=mb_red!50,%
    left=0.6pt, right=0.6pt,%
    top=1.6pt, bottom=1.6pt,%
    boxsep=0pt,%
    hbox,
    before=\vspace{0em},
    after=\vspace{0em}
}
\newtcolorbox{important_y}{
    colframe=y!80,%
    colback=y!80,%
    left=1pt, right=1pt,%
    top=0.5pt, bottom=0.5pt,%
    boxsep=0pt,%
    hbox,
    before=\vspace{0em},
    after=\vspace{0em}
}
\newcommand{\cmt}[1]{\textcolor{red}{[#1]}}
% If the title and author information does not fit in the area allocated, uncomment the following
%
%\setlength\titlebox{<dim>}
%
% and set <dim> to something 5cm or larger.

\title{\raisebox{-0.1\height}{\includegraphics[height=1.6ex]{figs/racer.png}} Drift: Decoding-time Personalized Alignments with \\ \textit{Implicit} User Preferences}

% \title{\raisebox{-0.1\height}{\includegraphics[height=1.6ex]{figs/racer.png}} Drift: Decoding-time Few-Shot Personalization of Large Language Models via \textit{Implicit} Preference Decomposition}


% \title{\raisebox{-0.1\height}{\includegraphics[height=1.6ex]{figs/racer.png}} Drift: Decomposing Implicit Preferences for Decoding-time Few-Shot Personalized Alignments}

% Author information can be set in various styles:
% For several authors from the same institution:
% \author{Author 1 \and ... \and Author n \\
%         Address line \\ ... \\ Address line}
% if the names do not fit well on one line use
%         Author 1 \\ {\bf Author 2} \\ ... \\ {\bf Author n} \\
% For authors from different institutions:
% \author{Author 1 \\ Address line \\  ... \\ Address line
%         \And  ... \And
%         Author n \\ Address line \\ ... \\ Address line}
% To start a separate ``row'' of authors use \AND, as in
% \author{Author 1 \\ Address line \\  ... \\ Address line
%         \AND
%         Author 2 \\ Address line \\ ... \\ Address line \And
%         Author 3 \\ Address line \\ ... \\ Address line}

% \author{
% Minbeom Kim$^{1\dagger}$ \hspace{1.5cm} Kang-il Lee$^{1}$ \hspace{1.5cm} Seongho Joo$^{1}$ \\ \textbf{Hwaran Lee$^{2, 3}$}  \hspace{1.5cm} \textbf{Kyomin Jung$^{1\dagger}$} \\
%     $^{1}$Seoul National University $   $\quad
%     $^{2}$Sogang University $   $\quad
%     $^{3}$NAVER AI Lab $  $\\
%     \texttt{\{minbeomkim, 4bkang, seonghojoo, kjung\}@snu.ac.kr}, \texttt{hwaran.lee@gmail.com}
% }

\author{
Minbeom Kim$^{1\dagger}$ \hspace{1.5cm} Kang-il Lee$^{1}$ \hspace{1.5cm} Seongho Joo$^{1}$ \\ \textbf{Hwaran Lee$^{2, 3}$}  \hspace{1.5cm} \textbf{Kyomin Jung$^{1\dagger}$} \\
    $^{1}$Seoul National University $   $\quad
    $^{2}$Sogang University $   $\quad
    $^{3}$NAVER AI Lab $  $\\
    \texttt{\{minbeomkim, kjung\}@snu.ac.kr}
}


\begin{document}
\maketitle
\begin{abstract}

Personalized alignments for individual users have been a long-standing goal in large language models (LLMs). 
We introduce \textbf{Drift}, a novel framework that personalizes LLMs at decoding time with \textit{implicit} user preferences. Traditional Reinforcement Learning from Human Feedback (RLHF) requires thousands of annotated examples and expensive gradient updates. In contrast, Drift personalizes LLMs in a \textit{training-free} manner, using \textit{only a few dozen examples} to steer a frozen model through efficient preference modeling. Our approach models user preferences as a composition of predefined, interpretable attributes and aligns them at decoding time to enable personalized generation. Experiments on both a synthetic persona dataset (\textit{Perspective}) and a real human-annotated dataset (\textit{PRISM}) demonstrate that Drift significantly outperforms RLHF baselines while using only 50–100 examples. Our results and analysis show that Drift is both computationally efficient and interpretable.


\end{abstract}

\renewcommand*{\thefootnote}{\arabic{footnote}}
\setcounter{footnote}{0}

%!TEX root = gcn.tex
\section{Introduction}
Graphs, representing structural data and topology, are widely used across various domains, such as social networks and merchandising transactions.
Graph convolutional networks (GCN)~\cite{iclr/KipfW17} have significantly enhanced model training on these interconnected nodes.
However, these graphs often contain sensitive information that should not be leaked to untrusted parties.
For example, companies may analyze sensitive demographic and behavioral data about users for applications ranging from targeted advertising to personalized medicine.
Given the data-centric nature and analytical power of GCN training, addressing these privacy concerns is imperative.

Secure multi-party computation (MPC)~\cite{crypto/ChaumDG87,crypto/ChenC06,eurocrypt/CiampiRSW22} is a critical tool for privacy-preserving machine learning, enabling mutually distrustful parties to collaboratively train models with privacy protection over inputs and (intermediate) computations.
While research advances (\eg,~\cite{ccs/RatheeRKCGRS20,uss/NgC21,sp21/TanKTW,uss/WatsonWP22,icml/Keller022,ccs/ABY318,folkerts2023redsec}) support secure training on convolutional neural networks (CNNs) efficiently, private GCN training with MPC over graphs remains challenging.

Graph convolutional layers in GCNs involve multiplications with a (normalized) adjacency matrix containing $\numedge$ non-zero values in a $\numnode \times \numnode$ matrix for a graph with $\numnode$ nodes and $\numedge$ edges.
The graphs are typically sparse but large.
One could use the standard Beaver-triple-based protocol to securely perform these sparse matrix multiplications by treating graph convolution as ordinary dense matrix multiplication.
However, this approach incurs $O(\numnode^2)$ communication and memory costs due to computations on irrelevant nodes.
%
Integrating existing cryptographic advances, the initial effort of SecGNN~\cite{tsc/WangZJ23,nips/RanXLWQW23} requires heavy communication or computational overhead.
Recently, CoGNN~\cite{ccs/ZouLSLXX24} optimizes the overhead in terms of  horizontal data partitioning, proposing a semi-honest secure framework.
Research for secure GCN over vertical data  remains nascent.

Current MPC studies, for GCN or not, have primarily targeted settings where participants own different data samples, \ie, horizontally partitioned data~\cite{ccs/ZouLSLXX24}.
MPC specialized for scenarios where parties hold different types of features~\cite{tkde/LiuKZPHYOZY24,icml/CastigliaZ0KBP23,nips/Wang0ZLWL23} is rare.
This paper studies $2$-party secure GCN training for these vertical partition cases, where one party holds private graph topology (\eg, edges) while the other owns private node features.
For instance, LinkedIn holds private social relationships between users, while banks own users' private bank statements.
Such real-world graph structures underpin the relevance of our focus.
To our knowledge, no prior work tackles secure GCN training in this context, which is crucial for cross-silo collaboration.


To realize secure GCN over vertically split data, we tailor MPC protocols for sparse graph convolution, which fundamentally involves sparse (adjacency) matrix multiplication.
Recent studies have begun exploring MPC protocols for sparse matrix multiplication (SMM).
ROOM~\cite{ccs/SchoppmannG0P19}, a seminal work on SMM, requires foreknowledge of sparsity types: whether the input matrices are row-sparse or column-sparse.
Unfortunately, GCN typically trains on graphs with arbitrary sparsity, where nodes have varying degrees and no specific sparsity constraints.
Moreover, the adjacency matrix in GCN often contains a self-loop operation represented by adding the identity matrix, which is neither row- nor column-sparse.
Araki~\etal~\cite{ccs/Araki0OPRT21} avoid this limitation in their scalable, secure graph analysis work, yet it does not cover vertical partition.

% and related primitives
To bridge this gap, we propose a secure sparse matrix multiplication protocol, \osmm, achieving \emph{accurate, efficient, and secure GCN training over vertical data} for the first time.

\subsection{New Techniques for Sparse Matrices}
The cost of evaluating a GCN layer is dominated by SMM in the form of $\adjmat\feamat$, where $\adjmat$ is a sparse adjacency matrix of a (directed) graph $\graph$ and $\feamat$ is a dense matrix of node features.
For unrelated nodes, which often constitute a substantial portion, the element-wise products $0\cdot x$ are always zero.
Our efficient MPC design 
avoids unnecessary secure computation over unrelated nodes by focusing on computing non-zero results while concealing the sparse topology.
We achieve this~by:
1) decomposing the sparse matrix $\adjmat$ into a product of matrices (\S\ref{sec::sgc}), including permutation and binary diagonal matrices, that can \emph{faithfully} represent the original graph topology;
2) devising specialized protocols (\S\ref{sec::smm_protocol}) for efficiently multiplying the structured matrices while hiding sparsity topology.


 
\subsubsection{Sparse Matrix Decomposition}
We decompose adjacency matrix $\adjmat$ of $\graph$ into two bipartite graphs: one represented by sparse matrix $\adjout$, linking the out-degree nodes to edges, the other 
by sparse matrix $\adjin$,
linking edges to in-degree nodes.

%\ie, we decompose $\adjmat$ into $\adjout \adjin$, where $\adjout$ and $\adjin$ are sparse matrices representing these connections.
%linking out-degree nodes to edges and edges to in-degree nodes of $\graph$, respectively.

We then permute the columns of $\adjout$ and the rows of $\adjin$ so that the permuted matrices $\adjout'$ and $\adjin'$ have non-zero positions with \emph{monotonically non-decreasing} row and column indices.
A permutation $\sigma$ is used to preserve the edge topology, leading to an initial decomposition of $\adjmat = \adjout'\sigma \adjin'$.
This is further refined into a sequence of \emph{linear transformations}, 
which can be efficiently computed by our MPC protocols for 
\emph{oblivious permutation}
%($\Pi_{\ssp}$) 
and \emph{oblivious selection-multiplication}.
% ($\Pi_\SM$)
\iffalse
Our approach leverages bipartite graph representation and the monotonicity of non-zero positions to decompose a general sparse matrix into linear transformations, enhancing the efficiency of our MPC protocols.
\fi
Our decomposition approach is not limited to GCNs but also general~SMM 
by 
%simply 
treating them 
as adjacency matrices.
%of a graph.
%Since any sparse matrix can be viewed 

%allowing the same technique to be applied.

 
\subsubsection{New Protocols for Linear Transformations}
\emph{Oblivious permutation} (OP) is a two-party protocol taking a private permutation $\sigma$ and a private vector $\xvec$ from the two parties, respectively, and generating a secret share $\l\sigma \xvec\r$ between them.
Our OP protocol employs correlated randomnesses generated in an input-independent offline phase to mask $\sigma$ and $\xvec$ for secure computations on intermediate results, requiring only $1$ round in the online phase (\cf, $\ge 2$ in previous works~\cite{ccs/AsharovHIKNPTT22, ccs/Araki0OPRT21}).

Another crucial two-party protocol in our work is \emph{oblivious selection-multiplication} (OSM).
It takes a private bit~$s$ from a party and secret share $\l x\r$ of an arithmetic number~$x$ owned by the two parties as input and generates secret share $\l sx\r$.
%between them.
%Like our OP protocol, o
Our $1$-round OSM protocol also uses pre-computed randomnesses to mask $s$ and $x$.
%for secure computations.
Compared to the Beaver-triple-based~\cite{crypto/Beaver91a} and oblivious-transfer (OT)-based approaches~\cite{pkc/Tzeng02}, our protocol saves ${\sim}50\%$ of online communication while having the same offline communication and round complexities.

By decomposing the sparse matrix into linear transformations and applying our specialized protocols, our \osmm protocol
%($\prosmm$) 
reduces the complexity of evaluating $\numnode \times \numnode$ sparse matrices with $\numedge$ non-zero values from $O(\numnode^2)$ to $O(\numedge)$.

%(\S\ref{sec::secgcn})
\subsection{\cgnn: Secure GCN made Efficient}
Supported by our new sparsity techniques, we build \cgnn, 
a two-party computation (2PC) framework for GCN inference and training over vertical
%ly split
data.
Our contributions include:

1) We are the first to explore sparsity over vertically split, secret-shared data in MPC, enabling decompositions of sparse matrices with arbitrary sparsity and isolating computations that can be performed in plaintext without sacrificing privacy.

2) We propose two efficient $2$PC primitives for OP and OSM, both optimally single-round.
Combined with our sparse matrix decomposition approach, our \osmm protocol ($\prosmm$) achieves constant-round communication costs of $O(\numedge)$, reducing memory requirements and avoiding out-of-memory errors for large matrices.
In practice, it saves $99\%+$ communication
%(Table~\ref{table:comm_smm}) 
and reduces ${\sim}72\%$ memory usage over large $(5000\times5000)$ matrices compared with using Beaver triples.
%(Table~\ref{table:mem_smm_sparse}) ${\sim}16\%$-

3) We build an end-to-end secure GCN framework for inference and training over vertically split data, maintaining accuracy on par with plaintext computations.
We will open-source our evaluation code for research and deployment.

To evaluate the performance of $\cgnn$, we conducted extensive experiments over three standard graph datasets (Cora~\cite{aim/SenNBGGE08}, Citeseer~\cite{dl/GilesBL98}, and Pubmed~\cite{ijcnlp/DernoncourtL17}),
reporting communication, memory usage, accuracy, and running time under varying network conditions, along with an ablation study with or without \osmm.
Below, we highlight our key achievements.

\textit{Communication (\S\ref{sec::comm_compare_gcn}).}
$\cgnn$ saves communication by $50$-$80\%$.
(\cf,~CoGNN~\cite{ccs/KotiKPG24}, OblivGNN~\cite{uss/XuL0AYY24}).

\textit{Memory usage (\S\ref{sec::smmmemory}).}
\cgnn alleviates out-of-memory problems of using %the standard 
Beaver-triples~\cite{crypto/Beaver91a} for large datasets.

\textit{Accuracy (\S\ref{sec::acc_compare_gcn}).}
$\cgnn$ achieves inference and training accuracy comparable to plaintext counterparts.
%training accuracy $\{76\%$, $65.1\%$, $75.2\%\}$ comparable to $\{75.7\%$, $65.4\%$, $74.5\%\}$ in plaintext.

{\textit{Computational efficiency (\S\ref{sec::time_net}).}} 
%If the network is worse in bandwidth and better in latency, $\cgnn$ shows more benefits.
$\cgnn$ is faster by $6$-$45\%$ in inference and $28$-$95\%$ in training across various networks and excels in narrow-bandwidth and low-latency~ones.

{\textit{Impact of \osmm (\S\ref{sec:ablation}).}}
Our \osmm protocol shows a $10$-$42\times$ speed-up for $5000\times 5000$ matrices and saves $10$-2$1\%$ memory for ``small'' datasets and up to $90\%$+ for larger ones.

%!TEX root = Article.tex

% Begin of file 2-Preliminaries.tex

\section{Foundations}
\label{sec:prl}

In this section we present some material that we will need in the subsequent
sections, and define a data model that consists of common aspects of RDF and
Property Graphs.


\subsection{A Common Data Model}

When developing a common framework for SHACL, ShEx, and PG-Schema, the first
challenge is establishing  a \emph{common data model}, since SHACL and ShEx work
on RDF, whereas PG-Schema works on Property Graphs.
Rather than using a model that generalises  both RDF and Property Graphs, we
propose a simple model, called \emph{common graphs}, which we obtained by asking
what, fundamentally, are the \emph{common aspects} of RDF and Property Graphs
(Appendix~\ref{sec:appendix-foundations} gives more details on the distilling of
common graphs).

Let us assume disjoint countable sets of nodes $\Nodes$, values $\Values$,
predicates $\Predicates$, and keys $\Keys$ (sometimes called properties).

% We sometimes say \emph{element} for a node or a value, and \emph{label} for a predicate or key. \todo{Drop if not used.}

\begin{definition}
  A \emph{common graph} is a pair $\graph = (E, \rho)$ where
  \begin{itemize}[\textbullet]
  \item
    $E \subseteq_{\mathit{fin}} \Nodes \times \Predicates \times \Nodes$ is its
    set of edges (which carry predicates), and
  \item
    $\rho \colon \Nodes \times \Keys \pto \Values$ is a finite-domain partial
    function mapping node-key pairs to values.
  \end{itemize}
  The set of nodes of a common graph $\graph$, written $\nodes(\graph)$,
  consists of all elements of $\Nodes$ that occur in $E$ or in the domain of
  $\rho$.
  Similarly, $\keys(\graph)$ is the subset of $\Keys$ that is used in $\rho$,
  and $\values(\graph)$ is the subset of $\Values$ that is used in $\rho$ (that
  is, the range of $\rho$).
\end{definition}

% \begin{example}[Media Service Common Graph] \label{ex:sharedScenario}
% To illustrate the common graphs, we introduce the following scenario. We assume a data model that has users, who can access and own accounts and invite other users to their accounts. Users have keys, such as email and credit-card. An example for this can be seen in~\Cref{fig}.
% % The nodes correspond to  conceptual classes, which will be identified by their available properties and keys. Properties are depicted as directed arrows, and keys are shown inside the conceptual classes.
% % The boxes inform about the available categories of nodes, with the keys they may have available (such as the key $\Exkey{plan}$ for nodes of category ``Account''), and properties connect nodes via directed arrows (such as $\Exprop{buyer}$, which connects nodes of category ``Sale'' and ``Account'').
% \end{example}

\begin{example}
  \label{ex:common-graph}
  Consider Figure~\ref{fig:common-graph}, containing a graph to store
  information about \emph{users} who may have access to (possibly multiple)
  \emph{accounts} in, \eg, a media streaming service.
  In this example, we have six nodes describing four persons ($u_1,...,u_4$) and
  two accounts ($a_1$, $a_2$).
  As a common graph $\graph = (E, \rho)$, the nodes are $a_1$, $u_1$, etc.
  Examples of edges in $E$ are $(u_2, \exaccess,a_1)$ and $(u_3, \exinvited,
  u_2)$.
  Furthermore, we have $\rho(u_2, \exemail) =$ d@d.d and $\rho(a_1,card) =
  1234$.
  So, $E$ captures the arrows in the figure (labelled with predicates) and
  $\rho$ captures the key/value information for each node.
  %
% Moreover, 3 predicates are used, appearing in Figure~\ref{fig:common-graph} as labels on links between nodes, representing the relation~$E$. Nodes are further associated with some key-value pairs, representing the function $\rho$.
  %
  Notice that a person may be the owner of an account, and may potentially have
  access to other accounts.
  This is captured using the predicates $\exowns$ and $\exaccess$, respectively.
  In addition, the system implements an invitation functionality, where users
  may invite other people to join the platform.
  The previous invitations are recorded using the predicate $\exinvited$.
  Both accounts and users may be privileged, which is stored via a Boolean value
  of the key~$\exprivileged$.
  We note that the presence of the key $\exemail$ (\resp, of the key (credit)
  $\excard$) is associated with, and indeed identifies users (\resp, accounts).
\end{example}

% \todo[inline]{In the example, worth noting that the graph node names are names, and not identities. Maybe it would be better to name them A, B, C, D to avoid misunderstanding?}

\begin{figure}[t]
\resizebox{1\linewidth}{!}{
  \includegraphics{example-common.pdf}
}
\Description{A diagram of the user common graph.}
\caption{The media service common graph. }
\label{fig:common-graph}
\end{figure}

It is easy to see that every common graph is a property graph (as per the formal
definition of property graphs~\cite{ABDF23}).
A common graph can also be seen as a set of triples, as in RDF.
Let
\[
  \Triples
=
  \left( \Nodes \times \Predicates \times \Nodes \right)
\;\cup\;
  \left( \Nodes \times \Keys \times \Values \right)\,.
\]
Then, a common graph can be seen as a finite set $\graph \subseteq \Triples$
such that for each $u \in \Nodes$ and $k \in \Keys$ there is at most one
$v \in \Values$ such that $(u, k, v) \in \graph$.
Indeed, a common graph $(E, \rho)$ corresponds to
\[
  E \;\cup\; \{ (u, k, v) \mid \rho(u,k) = v\}\;.
\]
When we write $\rho(u, k) = v$ we assume that $\rho$ is defined on $(u, k)$.

\medskip

\noindent\emph{Throughout the paper we see property graph $\graph$
simultaneously as a pair $(E, \rho)$ and as a set of triples from $\Triples$,
switching between these perspectives depending on what is most convenient at a
given moment.}


\subsection{Node Contents and  Neighbourhoods}

Let $\Records$ be the set of all \emph{records}, \ie, finite-domain partial
functions $r \colon \Keys \pto \Values$.
We write records as sets of pairs $\left\{ (k_1, w_1), \dots (k_n, w_n)
\right\}$ where $k_1, \dots, k_n$ are all different, meaning that $k_i$ is
mapped to $w_i$.

For a common graph $\graph = (E,\rho)$ and node $v$ in $\graph$, by a slight
abuse of notation we write $\rho(v)$ for the record $\left\{ (k, w) \mid
\rho(v,k) = w \right\}$ that collects all key-value pairs associated with node
$v$ in $\graph$.
We call $\rho(v)$ the \emph{content} of node $v$ in $\graph$.
This is how PG-Schema interprets common graphs: it views key-value pairs in
$\rho(v)$ as \emph{properties} of the node $v$, rather than independent,
navigable objects in the graph.

SHACL and ShEx, on the other hand, view common graphs as sets of triples and
make little distinction between keys and predicates.
The following notion---when applied to a node---uniformly captures the local
context of this node from that perspective: the content of the node and all
edges incident with the node.

%\begin{definition}[Neighbourhood]
%Given a common graph $\graph = (E,\rho)$ and a node $v\in\Nodes$, we write $\neigh_\graph(v)$ for the common graph $(E',\rho')$ where $E' = \left \{ (u_1,p,u_2) \in  E \mid u_1 = v \text{ or } u_2 = v\right\}$ and $\rho'$ is obtained by restricting $\rho$ so that $\rho'(v) = \rho(v)$ and $\rho'(u)$ is empty for all $u\neq v$. Similarly, for $w\in\Values$, we let $\neigh_\graph(w)$ be the common graph $(\emptyset,\rho')$ where $\rho'(u) = \left\{(k,w')\in\rho(u)\mid w'=w\right\}$ for all $u\in\Nodes$.
%Given a common graph $\graph$ and a node or value $v\in\Nodes\cup\Values$, the \emph{neighbourhood of $v$ in $\graph$}, written $\neigh_\graph(v)$, is the common graph consisting of triples $(u_1, p, u_2)$ from $\graph$ such that $p\in\Predicates\cup\Keys$ and either $u_1=v$ or $u_2=v$.
%\end{definition}

%That is, for $v\in\Nodes$,  $\neigh_\graph(v)$ is a star-shaped graph where only the central node has non-empty content.  For $w\in\Values$, $\neigh_\graph(w)$ is a graph with no edges and only a single value occurring in the contents of nodes.

%If we view common graphs as sets of triples, $\neigh_\graph(v)$ for $v\in\Nodes\cup\Values$ is simply the set of all triples from $\graph$ that mention $v$.

%We will also use the notion of \emph{partial neighbourhoods}, where only specified subsets of keys and predicates are taken into account.

%It is easiest to define it seeing common graphs as sets of triples.

\begin{definition}[Neighbourhood]
  Given a common graph $\graph$ and a node or value $v \in \Nodes \cup \Values$,
  the \emph{neighbourhood} of $v$ in $\graph$ is $\neigh_\graph(v) = \left\{
  (u_1, p, u_2) \in \graph \mid u_1 = v \text{ or } u_2 = v \right\}$.
  %
% \todo[inline]{Wim: This is ill-defined. We do say before that a common graph can be viewed as a set of triples if we want to think about it as RDF. But this definition should also apply to the PG view. We should be clearer about what we mean with the key/value pairs and only use ingredients from Def 1. In fact, if we take the RDF view, the definition is inconsistent with text below that says that, if $v$ is a value, then the neighborhood has no edges.}
% \todo[inline]{Suggestion to rephrase: introduce $\graph = (E,\rho)$ and say $\neigh_\graph(v) = \{(u_1,p,u_2) \in E \mid ... \} \cup \{???\}$ (Actually I don't understand yet what we want wrt $\rho$.)}
% \todo[inline]{Filip: In many places in the paper we treat $\graph$ as a pair $(E,\rho)$ or as a subset of $\Triples$, whatever is more convenient. It should suffice to warn the reader that we do this. We could write the definition in terms of $(E,\rho)$, but it would be clumsy. I really think it is fine as written.  On the other hand, if this is not helping, we can probably just skip this definition entirely and introduce only the $\pm$ variant of neighbourhoods in the section on ShEx.}
% \todo[inline]{Wim: OK, I understand better now what's intended and clarified below.}
\end{definition}

\todo{JH: Is this actually used anywhere?}

When $v \in \Nodes$, then $\neigh_\graph(v)$ is a star-shaped graph
where only the central node has non-empty content.
When $v \in \Values$, then $\neigh_\graph(v)$ consists of all the nodes in
$\graph$ that have some key with value $v$, which is a common graph with no
edges and a restricted function $\rho$.

%\todo[inline]{Maybe move to respective sections. Could also save space.}


\subsection{Value Types}

We assume an enumerable set of \emph{value types} $\ValueTypes$.
The reader should think of value types as \texttt{integer}, \texttt{boolean},
\texttt{date}, \etc
Formally, for each value type $\vtype \in \ValueTypes$, we assume that there is
a set $\sem{\vtype} \subseteq \Values$ of all values of that type and that each
value $v \in \Values$ belongs to some type, \ie, there is at least one $\vtype
\in \ValueTypes$ such that $v \in \sem{\vtype}$.
Finally, we assume that there is a type $\any \in \ValueTypes$ such that
$\sem{\any} = \Values$.


\subsection{Shapes and Schemas}
\label{ssec:shapes}

We formulate all three schema languages using \emph{shapes}, which are unary
formulas describing the graph's structure around a \emph{focus} node or a value.
Shapes will be expressed in different formalisms, specific to the schema
language; for each of these formalisms we will define when a focus node or value
$v \in \Nodes \cup \Values$ \emph{satisfies} shape $\varphi$ in a common graph
$\graph$, written $\graph, v \models \varphi$.

Inspired by ShEx \emph{shape maps}, we abstract a schema $\schema$ as a set of
pairs $(\sel,\varphi)$, where $\varphi$ is a shape and $\sel$ is a
\emph{selector}.
A selector is also a shape, but usually a very simple one, typically checking
the presence of an incident edge with a given predicate, or a property with a
given key.
A graph $\graph$ is \emph{valid} \wrt $\schema$, in symbols $\graph \models
\schema$, if
\[
  \graph, v \models \sel
\quad \text{implies} \quad
  \graph, v \models \varphi,
\]
for all $v \in \Nodes \cup \Values$ and $(\mathit{sel}, \varphi) \in \schema$.
That is, for each focus node or value satisfying the selector, the graph around
it looks as specified by the shape.
We call schemas $\schema$ and $\schema'$ \emph{equivalent} if $\graph \models
\schema$ \iff $\graph \models \schema'$, for all $\graph$.
In what follows, we may use $\mathit{sel} \Rightarrow \varphi$ to indicate a
pair $(\mathit{sel}, \varphi)$ from a schema $\SHACLSchema$.

% \begin{example}[Schemas over Media Service Common Graph]
%     \label{ex:ShapeExample}

% We stay in the same scenario introduced in \Cref{ex:sharedScenario}. We list here illustrative examples for requirements on common graphs that can be imposed via schemas.  To give an intuitive idea about the selector and the shape, we indicate this informally by splitting the sentences into an initial part that selects nodes or values, and the second part which must hold for these elements:\\
% \noindent
% \emph{For every account}, there must exist a primary credit card ; \\
% \noindent \emph{For every account}, there are  five users of it or less;\\
% \emph{Every owner of an account}, has a unique email address.
% \end{example}

\begin{example}
  \label{ex:constraint-desc}
  We next describe some constraints one may want to express in the domain of
  Example~\ref{ex:common-graph}.
  \begin{enumerate}[(C1)]
  \item
    We may want the values associated to certain keys to belong to concrete
    datatypes, like strings or Boolean values.
    In our example, we want to state that the value of the key $\excard$ is
    always an integer.
  \item
    We may expect the existence of a value associated to a key, an outgoing
    edge, or even a complex path for a given source node.
    For our example, we require that all owners of an account have an email
    address defined.
  \item
    We may want to express database-like uniqueness constraints.
    For instance, we may wish to ensure that the email address of an account
    owner uniquely identifies them.
  \item
    We may want to ensure that all paths of a certain kind end in nodes with
    some desired properties. For example, if an account is privileged, then all
    users that have access to it should also be privileged.
  \item
    We may want to put an upper bound on the number of nodes reached from a
    given node by certain paths. For instance, every user may have access to at
    most 5 accounts.
\end{enumerate}

% \todo[inline]{Wim: Reminder to self. I'd like to illustrate some open/closed things here. (There's no time anymore for this.)}
% \todo[inline]{Wim: More urgently though, we should explain better about how we model things. Let's say that ``users'' are those nodes that have an email key and ``accounts'' are those that have a card key?}
% \todo[inline]{Iovka: I support the need to make this precise. Then, should we use these two selectors in all examples?\\
% Also, we might say that we need this trick because we do not have rdf:type nor labels on nodes.}
% \todo[inline]{Cem: After discussion with Filip, I fixed the setting such that it is keys that identify users and accounts. Problem: this makes C2 awkward. }

\end{example}

% End of file 2-Preliminaries.tex

\section{{\includegraphics[height=1.6ex]{figs/racer.png}} Drift Algorithms}

Drift overcomes data scarcity and computational inefficiency by decomposing a user’s complex personal preferences as a linear combination of simpler attributes. As Figure~\ref{fig:main}, we describe two key components: \textit{Drift Approximation}, which efficiently estimates attribute weights from a few dozen examples, and \textit{Drift Decoding}, which integrates these weights into the LLM’s decoding process.

\subsection{Drift Approximation}
\paragraph{Problem Setup.} 
Assume we have a personalized preference dataset $\mathcal{D}$, a frozen LLM $\pi_\text{LLM}$, and a set of $k$ attribute-specific small LMs $\{\pi_i^*\}_{i=1}^k$ (with corresponding base model $\pi$). We model the overall personalized reward as
\begin{equation}
    R_{\mathcal{D}}(y \mid x) = \sum_{i=1}^k p_i \, r_i(y \mid x),
\end{equation}
where $p_i$ indicates the importance of the $i$th attribute. Under the KL-regularized framework in Eq.~\ref{eq:ideal_distributions}, the target distribution $\tilde{\pi}$ becomes:

{\small
\begin{equation}
\begin{split}
    \tilde{\pi}(y \mid x) &\propto \pi_\text{LLM}(y \mid x) \exp\!\Bigl(\beta^{-1}R_{\mathcal{D}}(y \mid x)\Bigr)\\[1mm]
    &= \pi_\text{LLM}(y \mid x) \prod_{i=1}^k \exp\!\left(\frac{p_i}{\beta} \, r_i(y \mid x)\right).
    \label{eq:combined_distribution}
\end{split}
\end{equation}}
\noindent
Each reward is expressed in a generative form:
\begin{equation}
    r_i(y \mid x) = \log\!\frac{\pi_i^*(y \mid x)}{\pi(y \mid x)} + \log Z_i(x),
    \label{eq:attribute_reward}
\end{equation}
with the partition term $Z_i(x)$ canceling out in pairwise comparisons.

\paragraph{From Bradley-Terry to Drift.} 
To estimate the attributes weights $\mathbf{p} = [p_1, \dots, p_k]$, we initiate the Bradley-Terry formulation as \citet{rafailov2024direct}. For a given pair $(y_w, y_l)$ (where $y_w$ is preferred over $y_l$), we have:
\begin{align}
    &\max_{\theta} \ p(y_w > y_l \mid x) = \nonumber \\
    &\frac{1}{1 + \exp\left(\beta\left(\log\frac{\pi_\text{LLM}^{\theta}(y_l \mid x)}{\pi_\text{LLM}^{\text{ref}}(y_l \mid x)} - \log\frac{\pi_\text{LLM}^{\theta}(y_w \mid x)}{\pi_\text{LLM}^{\text{ref}}(y_w \mid x)}\right)\right)}\nonumber
\end{align}
as in DPO~\citep{rafailov2024direct}. 
Substituting Eqs.~\ref{eq:combined_distribution} and \ref{eq:attribute_reward} simplifies this optimization to:
% \begin{align}
% \max_{\mathbf{p}}\; &\frac{1}{1 + \exp\Bigl(\beta \Bigl(
% \sum\limits_{i=1}^k p_i \log\frac{\pi^*_i(y_l \mid x)}{\pi(y_l \mid x)} \nonumber\\[1mm]
% &\quad\quad - \sum_{i=1}^k p_i \log\frac{\pi^*_i(y_w \mid x)}{\pi(y_w \mid x)}
% \Bigr)\Bigr)}.
% \end{align}
{\tiny
\begin{align}
\max_{\mathbf{p}}\; &\frac{1}{1 + \exp\Bigl(\beta \Bigl(\sum\limits_{i=1}^k p_i \log\frac{\pi^*_i(y_l \mid x)}{\pi(y_l \mid x)} \nonumber - \sum_{i=1}^k p_i \log\frac{\pi^*_i(y_w \mid x)}{\pi(y_w \mid x)}\Bigr)\Bigr)}.
\end{align}}
By monotonicity of $x \mapsto \frac{1}{1 + \exp(-\beta x)}$, reducing the problem to a simpler optimization task:
\begin{equation}
    \max_{\mathbf{p}} \ \sum_{i=1}^k p_i \left(\log\frac{\pi_i^*(y_w \mid x)}{\pi(y_w \mid x)} - \log\frac{\pi_i^*(y_l \mid x)}{\pi(y_l \mid x)}\right).\nonumber
\end{equation}
To avoid an unbounded solution, we constrain $\mathbf{p}$ to lie on the unit $\ell_2$ sphere:
\begin{equation}
    \max_{\mathbf{p}} \ \left(\mathbf{W} - \mathbf{L}\right)^T \mathbf{p}, \quad \text{subject to } \|\mathbf{p}\|_2 = 1,
\end{equation}
where $\mathbf{W}$ and $\mathbf{L}$ aggregate the log-ratio differences for the preferred $y_w$ and less preferred $y_l$ outputs over $\mathcal{D}$, respectively. 
Notably, this approximation is completely gradient-free and thus highly efficient compared to traditional preference modeling.

\paragraph{Zero-Shot Rewarding via Differential Prompts.}  
Drift Approximation computes $r_i$ for each instance $y$ as $\log\frac{\pi_i^*(y \mid x)}{\pi(y \mid x)}$. However, training an attribute-specific model $\pi^*_i$ for every possible attribute is infeasible. Instead, we reward each attribute in a zero-shot manner using differential prompts. 

Starting from a base prompt $s_0$ (e.g., \textit{"You are an AI assistant."}), we compute the log-probability $\log \pi(y | x, s_0)$. For each attribute (e.g., \textit{\textcolor{blue2}{emotion}}), we modify the base prompt by adding a corresponding cue (e.g., \textit{"You are an \textcolor{blue2}{emotional} AI assistant."}) to obtain $s_i$ and compute $\log \pi_i^*(y | x)=\log \pi(y | x, s_i)$. Their difference $\log\frac{\pi(y \mid x, s_i)}{\pi(y \mid x, s_0)}$ captures the differential impact of the attribute cue, serving as a surrogate reward signal that measures how well the response $y$ aligns with the attribute. 
This approach is: 1) \textbf{Training-free:} No additional fine-tuning is needed, 2) \textbf{Flexible:} New attributes can be integrated on the fly, 3) \textbf{Memory efficient:} It avoids the need to maintain multiple LLMs.

Algorithm~\ref{alg:drift-approximation} summarizes the Drift Approximation procedure.

\algrenewcommand\algorithmicrequire{\textbf{Input:}}
\algrenewcommand\algorithmicensure{\textbf{Output:}}
\begin{algorithm}[t]
\caption{Drift Approximation}
\label{alg:drift-approximation}
\begin{algorithmic}[1]
\Require Dataset $\mathcal{D} = \{(y^j_w, y^j_l, x^j)\}_{j=1}^n$, sLM $\pi$, base prompt $s_0$, attribute prompts $\{s_i\}_{i=1}^k$
\Ensure Attribute weights $\mathbf{p} = \{p_1, p_2, \dots, p_k\}$
\For{$j = 1$ to $n$} \Comment{Over each data point}
    \For{$i = 1$ to $k$} \Comment{For each attribute}
        \State $\mathbf{W}_{j,i} \gets \log \frac{\pi(y^j_w\mid x^j, s_i)}{\pi(y^j_w\mid x^j, s_0)}$
        \State $\mathbf{L}_{j,i} \gets \log \frac{\pi(y^j_l\mid x^j, s_i)}{\pi(y^j_l\mid x^j, s_0)}$
    \EndFor
\EndFor
\State $\mathbf{p} \gets \arg\max_{\mathbf{p}:\|\mathbf{p}\|_2=1} \;  (\mathbf{W} - \mathbf{L})^T\mathbf{p}$
\State \Return $\mathbf{p}$
\end{algorithmic}
\end{algorithm}

\subsection{Drift Decoding}
\label{sec:drift-decoding}
Once the attribute weights $\mathbf{p}$ are obtained, Drift enables personalized generation by sampling directly from a composite distribution that adjusts the frozen LLM’s logits.
\paragraph{Composite Distribution.}  
Let $\pi_{\text{LLM}}$ denote the frozen LLM and $\{\pi_i\}_{i=1}^k$ the distributions obtained by prompting with $s_i$. Denote their respective logits by $h^{\text{LLM}}$, $h^i$, and let $h^{\text{base}}$ correspond to the base prompt $s_0$. The composite distribution $\tilde{\pi}$ of next token candidates $w$ is defined as:
\begin{equation}
    \tilde{\pi}(w) \propto \pi_{\text{LLM}}(w) \prod_{i=1}^k \left(\frac{\pi_i(w)}{\pi_{\text{base}}(w)}\right)^{\frac{p_i}{\beta}},
\end{equation}
where $\beta$ is the KL regularization hyperparameter that controls the strength of personalization.
Converting probabilities to logits (recall $\pi(w)=\text{softmax}(h[w])$ for all $w$), we obtain:
\begin{equation}
\begin{split}
    \log \tilde{\pi}(w) &= h^{\text{LLM}}[w] + \\ &\sum_{i=1}^k \frac{p_i}{\beta} \big(h^i[w] - h^{\text{base}}[w]\big) + C,
    \end{split}
\end{equation}
where $C$ is a constant independent of $w$ and will be ignored after $\text{softmax}$. Thus, sampling from $\tilde{\pi}$ amounts to:
{\small
\begin{equation}
    \tilde{\pi}(w) = \text{softmax}\Big( h^{\text{LLM}} + \sum_{i=1}^k \frac{p_i}{\beta} (h^i - h^{\text{base}}) \Big)[w].
\end{equation}}
Thus, sampling from $\tilde{\pi}$ amounts to adjusting the LLM's logits using the weighted attribute differences.
For a more detailed derivation, see Appendix~\ref{appendix:drift-decoding-proof}.

Algorithm~\ref{alg:drift-decoding} describes the complete autoregressive decoding procedure.


\begin{algorithm}[t]
\caption{Drift Decoding}
\label{alg:drift-decoding}
\begin{algorithmic}[1]
\Require Input context $x$, LLM $\pi_{\text{LLM}}$, sLM  $\pi$, base prompt $s_0$, attribute-specific prompts $\{s_i\}_{i=1}^k$, personal weights $\{p_i\}_{i=1}^k$ and strength $\beta$
\Ensure Generated sequence $y$
\State $y \gets \emptyset$
\While{not end of sequence}
    \State Compute $h^{\text{LLM}}_t \gets \pi_{\text{LLM}}(\cdot \mid x,y)$
    \State Compute $h^{\text{base}}_t \gets \pi(\cdot \mid x,y, s_0)$
    \For{$i = 1$ to $k$}
        \State Compute $h^i_t \gets \pi(\cdot \mid x,y, s_i)$
    \EndFor
    \State $h^{\text{drift}}_t \gets h^{\text{LLM}}_t + \frac{1}{\beta}\sum_{i=1}^k p_i (h^i_t - h^{\text{base}}_t)$
    \State Sample token $w_t \sim \text{softmax}(h^{\text{drift}}_t)$
    \State Append $w_t$ to $y$
\EndWhile
\State \Return $y$
\end{algorithmic}
\end{algorithm}


\paragraph{Practical Considerations.}
For Drift Approximation, a zero-shot rewarding mechanism can consider an unlimited number of candidate attributes with gradient-free computational cost. It is advantageous to evaluate as many attributes as possible, thereby increasing the likelihood that even a small, carefully selected subset will capture the full nuances of a user's preferences. In practice, we perform the approximation using a large pool of attributes (e.g., 41 candidates as detailed in Table~\ref{tab:system_prompts}) and then select a subset with the highest absolute weights $|p_i|$ for the final decoding process—our experiments ultimately use seven representative attributes. We will further discuss this in Section~\ref{sec:practical-1}.

% Moreover, since Drift performs computations at the logit level, it is compatible with most sampling strategies (e.g., top-$p$, top-$k$). However, combining outputs from multiple language models can increase the entropy during next-token prediction, potentially raising the probability of selecting unreliable tokens. To mitigate this, we set top-$k=10$, top-$p=0.9$, and $\beta=0.5$ in our experiments.

% These two practical considerations are further discussed in Section~\ref{sec:practical-1} and \ref{sec:practical-2}.

\section{Experiments}
\label{sec:experiments}

\subsection{Next K-mer Prediction}
\label{sec:kmer_predition}
\begin{figure}[t]
    \centering
    \includegraphics[width=0.5\textwidth]{figures/pdf/kmer_prediction_main_text.pdf}
    \caption{Evaluation of next K-mer prediction. (A) Accuracy of the next K-mer prediction task across various tokenizers and input token lengths. (B) Comparison of the \textbf{Gener}\textit{ator} against baseline models on a dataset comprised exclusively mammalian DNA.}
    \label{fig:kmer_main}
\end{figure}

As mentioned in \textit{Sec.} \ref{sec:tokenization}, we conducted extensive experiments to explore the most suitable tokenizer for training causal DNA language models. This was achieved by training multiple models on identical datasets, each employing a different tokenizer. All models share the same architecture as the \textbf{Gener}\textit{ator} and are uniformly compared at 32,000 training steps. We employed the accuracy of the next K-mer prediction task as our evaluation metric. This zero-shot task facilitates a direct assessment of the pre-trained model quality, ensuring equitable comparisons across various tokenizers. As depicted in \textit{Fig.} \ref{fig:kmer_main}A, the tested tokenizers include BPE tokenizers with vocabulary sizes ranging from 512 to 8192, and K-mer tokenizers with K values from 1 to 8 (noting that the single nucleotide tokenizer corresponds to a K-mer tokenizer with K=1). Overall, K-mer tokenizers demonstrate superior performance compared to BPE tokenizers. Among the K-mer tokenizers, the 6-mer tokenizer is selected for its robust performance with limited input tokens and its ability to maintain top-tier performance as the number of input tokens increases.

Moreover, we evaluated the performance of Mamba \cite{Mamba,Mamba-2}, recognized for its capacity in handling long-context pre-training. To adequately assess its capabilities, we configured a Mamba model utilizing the single nucleotide tokenizer with 1.2B parameters and a context length of 98k bp. The Mamba model is compared to the 1-mer and 6-mer models under varied configurations. The comparison with the 1-mer model is straightforward; the Mamba model (denoted as Mamba\texttimes1 in \textit{Fig.} \ref{fig:kmer_main}A) exhibits slightly better performance with fewer input tokens but underperforms as the token count increases. Despite Mamba's context length being six times that of the 1-mer model, this feature does not translate into improved performance. This might suggest that Mamba's renowned ability to handle long-context pre-training primarily refers to cost-effective training rather than enhanced model performance \cite{Empirical, DeciMamba}. To compare against the 6-mer model, we adjust the input token count for the Mamba model by a factor of six (denoted as Mamba\texttimes6) to compare the models on the same base-pair basis. In this context, Mamba\texttimes6 shows slightly better performance with fewer input tokens; however, it rapidly lags as the token count increases. These findings collectively indicate that a transformer decoder architecture paired with a 6-mer tokenizer provides the most effective approach for training causal DNA language models, aligning with the configuration of the \textbf{Gener}\textit{ator}.

We further compared the \textbf{Gener}\textit{ator} model with other baseline models to evaluate their generative capabilities. As illustrated in \textit{Fig.} \ref{fig:kmer_main}B, we assess model performance using a dataset composed exclusively of mammalian DNA, given that HyenaDNA and GROVER are trained solely on human genomes. The \textbf{Gener}\textit{ator} significantly outperforms other baseline models, including its variant, \textbf{Gener}\textit{ator}-All, which incorporates pre-training on non-gene regions. This suggests that the gene sequence training strategy, which emphasizes semantically rich regions, provides a more effective training scheme compared to the conventional whole sequence training. This effectiveness is likely due to the sparsity of gene segments in the whole genome (less than 10\%) and the disproportionate importance of these segments. Among the other baseline models, NT-multi demonstrates the best performance, likely attributable to its extensive model scale (2.5B parameters), yet it still lags significantly behind the \textbf{Gener}\textit{ator}. This result aligns with expectations, as the MLM training paradigm is recognized for its limitations in generative capabilities. Meanwhile, HyenaDNA, despite utilizing the NTP training paradigm, does not show improved performance compared to other masked language models, likely due to its overly small model size (55M parameters), insufficient for exhibiting robust generative abilities. This comparison underscores the critical role of the \textbf{Gener}\textit{ator} in bridging the gap for large-scale generative DNA language models within the eukaryotic domain.

Due to space constraints, we have chosen only to demonstrate specific examples with mammalian DNA data and a fixed K-mer prediction length of 16 bp in \textit{Fig.} \ref{fig:kmer_main}. A more comprehensive analysis across various taxonomic groups and K-mer lengths is provided in the appendix.

\subsection{Benchmark Evaluations}
In this section, we compare the \textbf{Gener}\textit{ator} with state-of-the-art genomic foundation models: Enformer~\cite{enformer}, DNABERT-2, HyenaDNA, Nucleotide Transformer, Caduceus, and GROVER, across various benchmark tasks. To ensure a fair comparison, we uniformly fine-tune each model and perform a 10-fold cross-validation on all datasets. For each model on each dataset, we conduct a hyperparameter search, exhaustively tuning learning rates in $\{1e^{-5}, 2e^{-5}, 5e^{-5}, \ldots, 1e^{-3}, 2e^{-3}, 5e^{-3}\}$ and batch sizes in $\{64, 128, 256, 512\}$. Detailed hyperparameter settings and implementation specifics are provided in the appendix.

\paragraph{Nucleotide Transformer Tasks}
Since the NT task dataset was revised recently~\cite{nucleotide-transformer}, we conducted experiments on both the original and revised datasets. The results for the revised NT tasks are provided in Table~\ref{tab:nucleotide_transformer_tasks_revised}, and the results for the original NT tasks are provided in Table~\ref{tab:nucleotide_transformer_tasks}. Overall, the \textbf{Gener}\textit{ator} outperforms other baseline models. However, the \textbf{Gener}\textit{ator}-All variant shows some performance decline. Notably, despite its earlier release, Enformer continues to deliver competitive results in chromatin profile and regulatory element tasks. This performance could be attributed to its original training in a supervised manner specifically for chromatin and gene expression tasks. The latest release of Nucleotide Transformer, NT-v2, although smaller in size (500M), demonstrates enhanced performance compared to NT-multi (2.5B). In contrast, DNABERT-2 and GROVER, which utilize BPE tokenizers, along with HyenaDNA and Caduceus, which employ the finer-grained single nucleotide tokenizer, do not show distinct performance advantages, likely due to the limited model scope and data scale.

\paragraph{Genomic Benchmarks}
We also conducted a comparative analysis on the Genomic Benchmarks~\cite{genomic-benchmarks}, which primarily focus on the human genome. The evaluation results are provided in Table~\ref{tab:genomic_benchmarks}. Overall, the \textbf{Gener}\textit{ator} still outperforms other models. However, it is worth noting that the Caduceus models also exhibit comparable performance while being significantly smaller (8M). This is likely due to the fact that Caduceus models are trained exclusively on the human genome, making them efficient and compact. Nevertheless, this exclusivity may limit their generalizability to other genomic contexts.

\paragraph{Gener Tasks} 
Lastly, we evaluated the newly proposed Gener tasks, which focus on assessing genomic context comprehension across various sequence lengths and organisms. As shown in Table~\ref{tab:gener_tasks}, the \textbf{Gener}\textit{ator} achieves the best performance on both gene and taxonomic classification tasks, with NT-v2 also demonstrating similar results. Further details on the evaluation of Gener tasks, including visualizations of confusion matrices, are provided in the appendix. The superior performance of the \textbf{Gener}\textit{ator} and NT-v2 is likely due to their pre-training on multispecies datasets. In contrast, despite also being trained on multispecies data, DNABERT-2 exhibits noticeable performance degradation. This may be attributed to its limited model size (117M for DNABERT-2, 500M for NT-v2, and 1.2B for \textbf{Gener}\textit{ator}) and shorter context length (3k for DNABERT-2, 12k for NT-v2, and 98k for \textbf{Gener}\textit{ator}). Other models, such as HyenaDNA and Caduceus, although trained exclusively on the human genome, still exhibit relevant generalizability on both tasks after fine-tuning, attributable to their long-context capacity (\textgreater 100k). GROVER, on the other hand, significantly lags behind in taxonomic classification due to its limited context length (3k).

\begin{table*}[!htb]
\small
\renewcommand{\arraystretch}{1}
\centering
\caption{Evaluation of the revised Nucleotide Transformer tasks. The reported values represent the Matthews correlation coefficient (MCC) averaged over 10-fold cross-validation, with the standard error in parentheses.}
\resizebox{\textwidth}{!}{
\begin{tabular}{lcccccccccc}
\toprule
& Enformer & DNABERT-2 & HyenaDNA & NT-multi & NT-v2 & Caduceus-Ph & Caduceus-PS & GROVER & \textbf{Gener}\textit{ator} & \textbf{Gener}\textit{ator}-All \\
& (252M) & (117M) & (55M) & (2.5B) & (500M) & (8M) & (8M) & (87M) & (1.2B) & (1.2B) \\
\midrule
H2AFZ          & 0.522 (0.019) & 0.490 (0.013) & 0.455 (0.015) & 0.503 (0.010) & \underline{0.524 (0.008)} & 0.417 (0.016) & 0.501 (0.013) & 0.509 (0.013) & \textbf{0.529 (0.009)} & 0.506 (0.019) \\
H3K27ac        & \underline{0.520 (0.015)} & 0.491 (0.010) & 0.423 (0.017) & 0.481 (0.020) & 0.488 (0.013) & 0.464 (0.018) & 0.464 (0.022) & 0.489 (0.023) & \textbf{0.546 (0.015)} & 0.496 (0.014) \\
H3K27me3       & 0.552 (0.007) & 0.599 (0.010) & 0.541 (0.018) & 0.593 (0.016) & \underline{0.610 (0.006)} & 0.547 (0.010) & 0.561 (0.036) & 0.600 (0.008) & \textbf{0.619 (0.008)} & 0.590 (0.014) \\
H3K36me3       & 0.567 (0.017) & \underline{0.637 (0.007)} & 0.543 (0.010) & 0.635 (0.016) & 0.633 (0.015) & 0.543 (0.009) & 0.602 (0.008) & 0.585 (0.008) & \textbf{0.650 (0.006)} & 0.621 (0.013) \\
H3K4me1        & \textbf{0.504 (0.021)} & \underline{0.490 (0.008)} & 0.430 (0.014) & 0.481 (0.012) & \underline{0.490 (0.017)} & 0.411 (0.012) & 0.434 (0.030) & 0.468 (0.011) & \textbf{0.504 (0.010)} & \underline{0.490 (0.016)} \\
H3K4me2        & \textbf{0.626 (0.015)} & 0.558 (0.013) & 0.521 (0.024) & 0.552 (0.022) & 0.552 (0.013) & 0.480 (0.013) & 0.526 (0.035) & 0.558 (0.012) & \underline{0.607 (0.010)} & 0.569 (0.012) \\
H3K4me3        & 0.635 (0.019) & \underline{0.646 (0.008)} & 0.596 (0.015) & 0.618 (0.015) & 0.627 (0.020) & 0.588 (0.020) & 0.611 (0.015) & 0.634 (0.011) & \textbf{0.653 (0.008)} & 0.628 (0.018) \\
H3K9ac         & \textbf{0.593 (0.020)} & 0.564 (0.013) & 0.484 (0.022) & 0.527 (0.017) & 0.551 (0.016) & 0.514 (0.014) & 0.518 (0.018) & 0.531 (0.014) & \underline{0.570 (0.017)} & 0.556 (0.018) \\
H3K9me3        & 0.453 (0.016) & 0.443 (0.025) & 0.375 (0.026) & 0.447 (0.018) & 0.467 (0.044) & 0.435 (0.019) & 0.455 (0.019) & 0.441 (0.017) & \textbf{0.509 (0.013)} & \underline{0.480 (0.037)} \\
H4K20me1       & 0.606 (0.016) & \underline{0.655 (0.011)} & 0.580 (0.009) & 0.650 (0.014) & 0.654 (0.011) & 0.572 (0.012) & 0.590 (0.020) & 0.634 (0.006) & \textbf{0.670 (0.006)} & 0.652 (0.010) \\
Enhancer       & \textbf{0.614 (0.010)} & 0.517 (0.011) & 0.475 (0.006) & 0.527 (0.012) & 0.575 (0.023) & 0.480 (0.008) & 0.490 (0.009) & 0.519 (0.009) & \underline{0.594 (0.013)} & 0.553 (0.020) \\
Enhancer type & \textbf{0.573 (0.013)} & 0.476 (0.009) & 0.441 (0.010) & 0.484 (0.012) & 0.541 (0.013) & 0.461 (0.009) & 0.459 (0.011) & 0.481 (0.009) & \underline{0.547 (0.017)} & 0.510 (0.022) \\
Promoter all   & 0.745 (0.012) & 0.754 (0.009) & 0.693 (0.016) & 0.761 (0.009) & \underline{0.780 (0.012)} & 0.707 (0.017) & 0.722 (0.014) & 0.721 (0.011) & \textbf{0.795 (0.005)} & 0.765 (0.009) \\
Promoter non-TATA & 0.763 (0.012) & 0.769 (0.009) & 0.723 (0.013) & 0.773 (0.010) & 0.785 (0.009) & 0.740 (0.012) & 0.746 (0.009) & 0.739 (0.018) & \textbf{0.801 (0.005)} & \underline{0.786 (0.007)} \\
Promoter TATA  & 0.793 (0.026) & 0.784 (0.036) & 0.648 (0.044) & \underline{0.944 (0.016)} & 0.919 (0.028) & 0.868 (0.023) & 0.853 (0.034) & 0.891 (0.041) & \textbf{0.950 (0.009)} & 0.862 (0.024) \\
Splice acceptor & 0.749 (0.007) & 0.837 (0.006) & 0.815 (0.049) & 0.958 (0.003) & \textbf{0.965 (0.004)} & 0.906 (0.015) & 0.939 (0.012) & 0.812 (0.012) & \underline{0.964 (0.003)} & 0.951 (0.006) \\
Splice site all & 0.739 (0.011) & 0.855 (0.005) & 0.854 (0.053) & 0.964 (0.003) & \textbf{0.968 (0.003)} & 0.941 (0.006) & 0.942 (0.012) & 0.849 (0.015) & \underline{0.966 (0.003)} & 0.959 (0.003) \\
Splice donor   & 0.780 (0.007) & 0.861 (0.004) & 0.943 (0.024) & 0.970 (0.002) & \underline{0.976 (0.003)} & 0.944 (0.026) & 0.964 (0.010) & 0.842 (0.009) & \textbf{0.977 (0.002)} & 0.971 (0.002) \\
\bottomrule
\end{tabular}
}
\label{tab:nucleotide_transformer_tasks_revised}
\end{table*}
\begin{table*}[!htb]
\small
\renewcommand{\arraystretch}{1.2}
\centering
\caption{Evaluation of the original Nucleotide Transformer tasks. The reported values represent the Matthews correlation coefficient (MCC) averaged over 10-fold cross-validation, with the standard error in parentheses.}
\resizebox{\textwidth}{!}{%
\begin{tabular}{lcccccccccc}
\toprule
& Enformer & DNABERT-2 & HyenaDNA & NT-multi & NT-v2 & Caduceus-Ph & Caduceus-PS & GROVER & \textbf{Gener}\textit{ator} & \textbf{Gener}\textit{ator}-All \\
& (252M) & (117M) & (55M) & (2.5B) & (500M) & (8M) & (8M) & (87M) & (1.2B) & (1.2B) \\
\midrule
H3 & 0.724 (0.018) & 0.785 (0.012) & 0.781 (0.015) & 0.793 (0.013) & 0.788 (0.010) & 0.794 (0.012) & 0.772 (0.022) & 0.768 (0.008) & \textbf{0.806 (0.005)} & \underline{0.803 (0.007)} \\
H3K14ac & 0.284 (0.024) & 0.515 (0.009) & \textbf{0.608 (0.020)} & 0.538 (0.009) & 0.538 (0.015) & 0.564 (0.033) & 0.596 (0.038) & 0.548 (0.020) & \underline{0.605 (0.008)} & 0.580 (0.038) \\
H3K36me3 & 0.345 (0.019) & 0.591 (0.005) & 0.614 (0.014) & 0.618 (0.011) & 0.618 (0.015) & 0.590 (0.018) & 0.611 (0.048) & 0.563 (0.017) & \textbf{0.657 (0.007)} & \underline{0.631 (0.013)} \\
H3K4me1 & 0.291 (0.016) & 0.512 (0.008) & 0.512 (0.008) & 0.541 (0.005) & 0.544 (0.009) & 0.468 (0.015) & 0.487 (0.029) & 0.461 (0.018) & \textbf{0.553 (0.009)} & \underline{0.549 (0.018)} \\
H3K4me2 & 0.207 (0.021) & 0.333 (0.013) & \textbf{0.455 (0.028)} & 0.324 (0.014) & 0.302 (0.020) & 0.332 (0.034) & \underline{0.431 (0.016)} & 0.403 (0.042) & 0.424 (0.013) & 0.400 (0.015) \\
H3K4me3 & 0.156 (0.022) & 0.353 (0.021) & \textbf{0.550 (0.015)} & 0.408 (0.011) & 0.437 (0.028) & 0.490 (0.042) & \underline{0.528 (0.033)} & 0.458 (0.022) & 0.512 (0.009) & 0.473 (0.047) \\
H3K79me3 & 0.498 (0.013) & 0.615 (0.010) & 0.669 (0.014) & 0.623 (0.010) & 0.621 (0.012) & 0.641 (0.028) & \textbf{0.682 (0.018)} & 0.626 (0.026) & \underline{0.670 (0.011)} & 0.631 (0.021) \\
H3K9ac & 0.415 (0.020) & 0.545 (0.009) & 0.586 (0.021) & 0.547 (0.011) & 0.567 (0.020) & 0.575 (0.024) & 0.564 (0.018) & 0.581 (0.015) & \textbf{0.612 (0.006)} & \underline{0.603 (0.019)} \\
H4 & 0.735 (0.023) & 0.797 (0.008) & 0.763 (0.012) & \underline{0.808 (0.007)} & 0.795 (0.008) & 0.788 (0.010) & 0.799 (0.010) & 0.769 (0.017) & \textbf{0.815 (0.008)} & \underline{0.808 (0.010)} \\
H4ac & 0.275 (0.022) & 0.465 (0.013) & 0.564 (0.011) & 0.492 (0.014) & 0.502 (0.025) & 0.548 (0.027) & \underline{0.585 (0.018)} & 0.530 (0.017) & \textbf{0.592 (0.015)} & 0.565 (0.035) \\
Enhancer & 0.454 (0.029) & 0.525 (0.026) & 0.520 (0.031) & 0.545 (0.028) & \underline{0.561 (0.029)} & 0.522 (0.024) & 0.511 (0.026) & 0.516 (0.018) & \textbf{0.580 (0.015)} & 0.540 (0.026) \\
Enhancer type & 0.312 (0.043) & 0.423 (0.018) & 0.403 (0.056) & 0.444 (0.022) & 0.444 (0.036) & 0.403 (0.028) & 0.410 (0.026) & 0.433 (0.029) & \textbf{0.477 (0.017)} & \underline{0.463 (0.023)} \\
Promoter all & 0.910 (0.004) & 0.945 (0.003) & 0.919 (0.003) & 0.951 (0.004) & 0.952 (0.002) & 0.937 (0.002) & 0.941 (0.003) & 0.926 (0.004) & \textbf{0.962 (0.002)} & \underline{0.955 (0.002)} \\
Promoter non-TATA & 0.910 (0.006) & 0.944 (0.003) & 0.919 (0.004) & \underline{0.955 (0.003)} & 0.952 (0.003) & 0.935 (0.007) & 0.940 (0.002) & 0.925 (0.006) & \textbf{0.962 (0.001)} & \underline{0.955 (0.002)} \\
Promoter TATA & 0.920 (0.012) & 0.911 (0.011) & 0.881 (0.020) & 0.919 (0.008) & \underline{0.933 (0.009)} & 0.895 (0.010) & 0.903 (0.010) & 0.891 (0.009) & \textbf{0.948 (0.008)} & 0.931 (0.007) \\
Splice acceptor & 0.772 (0.007) & 0.909 (0.004) & 0.935 (0.005) & \underline{0.973 (0.002)} & \underline{0.973 (0.004)} & 0.918 (0.017) & 0.907 (0.015) & 0.912 (0.010) & \textbf{0.981 (0.002)} & 0.957 (0.009) \\
Splice site all & 0.831 (0.012) & 0.950 (0.003) & 0.917 (0.006) & 0.974 (0.004) & \underline{0.975 (0.002)} & 0.935 (0.011) & 0.953 (0.005) & 0.919 (0.005) & \textbf{0.978 (0.001)} & 0.973 (0.002) \\
Splice donor & 0.813 (0.015) & 0.927 (0.003) & 0.894 (0.013) & 0.974 (0.002) & \underline{0.977 (0.007)} & 0.912 (0.009) & 0.930 (0.010) & 0.888 (0.012) & \textbf{0.978 (0.002)} & 0.967 (0.005) \\
\bottomrule
\end{tabular}
}
\label{tab:nucleotide_transformer_tasks}
\end{table*}
\begin{table*}[!htb]
\small
\renewcommand{\arraystretch}{1.2}
\centering
\caption{Evaluation of the Genomic Benchmarks. The reported values represent the accuracy averaged over 10-fold cross-validation, with the standard error in parentheses.}
\resizebox{\textwidth}{!}{
\begin{tabular}{lcccccccc}
\toprule
& DNABERT-2 & HyenaDNA & NT-v2 & Caduceus-Ph & Caduceus-PS & GROVER & \textbf{Gener}\textit{ator} & \textbf{Gener}\textit{ator}-All \\
& (117M) & (55M) & (500M) & (8M) & (8M) & (87M) & (1.2B) & (1.2B) \\
\midrule
Coding vs. Intergenomic & 0.951 (0.002) & 0.902 (0.004) & 0.955 (0.001) & 0.933 (0.001) & 0.944 (0.002) & 0.919 (0.002) & \textbf{0.963 (0.000)} & \underline{0.959 (0.001)} \\
Drosophila Enhancers Stark & 0.774 (0.011) & 0.770 (0.016) & 0.797 (0.009) & \textbf{0.827 (0.010)} & 0.816 (0.015) & 0.761 (0.011) & \underline{0.821 (0.005)} & 0.768 (0.015) \\
Human Enhancers Cohn & \underline{0.758 (0.005)} & 0.725 (0.009) & 0.756 (0.006) & 0.747 (0.003) & 0.749 (0.003) & 0.738 (0.003) & \textbf{0.763 (0.002)} & 0.754 (0.006) \\
Human Enhancers Ensembl & 0.918 (0.003) & 0.901 (0.003) & 0.921 (0.004) & \textbf{0.924 (0.002)} & \underline{0.923 (0.002)} & 0.911 (0.004) & 0.917 (0.002) & 0.912 (0.002) \\
Human Ensembl Regulatory & 0.874 (0.007) & 0.932 (0.001) & \textbf{0.941 (0.001)} & \underline{0.938 (0.004)} & \textbf{0.941 (0.002)} & 0.897 (0.001) & 0.928 (0.001) & 0.926 (0.001) \\
Human non-TATA Promoters & 0.957 (0.008) & 0.894 (0.023) & 0.932 (0.006) & \textbf{0.961 (0.003)} & \textbf{0.961 (0.002)} & 0.950 (0.005) & \underline{0.958 (0.001)} & 0.955 (0.005) \\
Human OCR Ensembl & 0.806 (0.003) & 0.774 (0.004) & 0.813 (0.001) & \underline{0.825 (0.004)} & \textbf{0.826 (0.003)} & 0.789 (0.002) & 0.823 (0.002) & 0.812 (0.003) \\
Human vs. Worm & 0.977 (0.001) & 0.958 (0.004) & 0.976 (0.001) & 0.975 (0.001) & 0.976 (0.001) & 0.966 (0.001) & \textbf{0.980 (0.000)} & \underline{0.978 (0.001)} \\
Mouse Enhancers Ensembl & \underline{0.865 (0.014)} & 0.756 (0.030) & 0.855 (0.018) & 0.788 (0.028) & 0.826 (0.021) & 0.742 (0.025) & \textbf{0.871 (0.015)} & 0.784 (0.027) \\
\bottomrule
\end{tabular}
}
\label{tab:genomic_benchmarks}
\end{table*}
\begin{table*}[!htb]
\small
\renewcommand{\arraystretch}{1}
\centering
\caption{Evaluation of the Gener tasks. The reported values represent the weighted F1 score averaged over 10-fold cross-validation, with the standard error in parentheses.}
\resizebox{\textwidth}{!}{
\begin{tabular}{lcccccccc}
\toprule
& DNABERT-2 & HyenaDNA & NT-v2 & Caduceus-Ph & Caduceus-PS & GROVER & \textbf{Gener}\textit{ator} & \textbf{Gener}\textit{ator}-All \\
& (117M) & (55M) & (500M) & (8M) & (8M) & (87M) & (1.2B) & (1.2B) \\
\midrule
Gene & 0.660 (0.002) & 0.610 (0.007) & \underline{0.692 (0.005)} & 0.629 (0.005) & 0.644 (0.007) & 0.630 (0.003) & \textbf{0.700 (0.002)} & 0.687 (0.003) \\
Taxonomic & 0.922 (0.003) & 0.970 (0.024) & 0.981 (0.001) & 0.958 (0.021) & 0.968 (0.006) & 0.843 (0.006) & \textbf{0.999 (0.000)} & \underline{0.998 (0.001)} \\
\bottomrule
\end{tabular}
}
\label{tab:gener_tasks}
\end{table*}

\subsection{Central Dogma}

In our experimental setup, we selected two target protein families from the UniProt~\cite{UniProt} database: the Histone and Cytochrome P450 families. By cross-referencing gene IDs and protein IDs, we extracted the corresponding protein-coding DNA sequences from RefSeq~\cite{RefSeq}. These sequences served as training data for fine-tuning the \textbf{Gener}\textit{ator} model, directing it to generate analogous protein-coding DNA sequences.

To assess the quality of generation, we compared several summary statistics. The results for the Histone family are provided in \textit{Fig.} \ref{fig:histone_generation}, while the evaluation results for the Cytochrome P450 family are provided in \textit{Fig.} \ref{fig:cytochrome_generation}.  After deduplication, the lengths of the generated DNA sequences and their translated protein sequences, using a codon table, closely resemble the distributions observed in the target families. This preliminary validation suggests that our generated DNA sequences maintain stable codon structures that are translatable into proteins. We conducted a more in-depth structural and functional analysis of these translated protein sequences. First, we assessed whether protein language models `recognize' these generated protein sequences by calculating their perplexity (PPL) using Progen2~\cite{progen2}. The results show that the PPL distribution of generated sequences closely matches that of the natural families and significantly differs from the shuffled sequences.

Furthermore, we used AlphaFold3 to predict the folded structures of the generated protein sequences and employed Foldseek~\cite{Foldseek} to find analogous proteins in the Protein Data Bank (PDB)~\cite{RCSBPDB}. Remarkably, we identified numerous instances where the conformations of the generated sequences exhibited high similarity to established structures in the PDB ($\text{TM-score}>0.8$). This structural congruence is observed despite substantial divergence in sequence composition, as indicated by sequence identities less than $0.3$. This low sequence identity positively suggests that the model is not merely replicating existing protein sequences but has learned the underlying principles to design new molecules with similar structures. This highlights the capability of the \textbf{Gener}\textit{ator} in generating biologically relevant sequences. 

\subsection{Enhancer Design}
We employed the enhancer activity data from DeepSTARR~\cite{DeepSTARR}, following the dataset split initially proposed by DeepSTARR and later adopted by NT. Using this data, we developed an enhancer activity predictor by fine-tuning the \textbf{Gener}\textit{ator}. This predictor surpasses the accuracy of DeepSTARR and NT-multi (Table \ref{tab:enhancer_benchmark}), establishing itself as the current state-of-the-art predictor. By applying our refined SFT approach as outlined in \textit{Sec.} \ref{sec:sequence_design}, we generated a collection of candidate enhancer sequences with specific activity profiles. As illustrated in \textit{Fig.} \ref{fig:enhancer_design}, the predicted activities of these candidates exhibit significant differentiation between the generated high/low-activity enhancers and natural samples.

To our knowledge, this represents one of the first attempts to use LLMs for prompt-guided design of DNA sequences, highlighting the capability of the \textbf{Gener}\textit{ator} in this domain. These generated sequences, and more broadly, this sequence design paradigm using the \textbf{Gener}\textit{ator}, merit further exploration. Our approach underscores the potential of the \textbf{Gener}\textit{ator} model to transform DNA sequence design methodologies, providing a novel pathway for the conditional design of DNA sequences using LLMs. In our subsequent research, we plan to extend our evaluations through further validations in wet lab conditions to explore the real-world applicability of these designed sequences.

\begin{figure}[!htb]
    \centering
    \includegraphics[width=0.6\textwidth]{figures/pdf/histone_generation.pdf}
    \caption{Histone generation. (A) Distribution densities of the protein sequence lengths for generated and natural samples. (B) Distribution densities of Progen2 PPL for generated and natural samples, along with randomly shuffled sequences. (C) Scatter plot of TM-score and AlphaFold3 prediction confidence (pLDDT) with marginal distributions. (D) Two folded structures of generated samples displaying structural congruence with natural samples.}
    \label{fig:histone_generation}
\end{figure}
\begin{figure}[!htb]
    \centering
    \includegraphics[width=0.6\textwidth]{figures/pdf/cytochrome_generation.pdf}
    \caption{Cytochrome P450 generation. (A) Distribution densities of the protein sequence lengths for generated and natural samples. (B) Distribution densities of Progen2 PPL for generated and natural samples, along with randomly shuffled sequences. (C) Scatter plot of TM-score and AlphaFold3 prediction confidence (pLDDT) with marginal distributions. (D) Two folded structures of generated samples displaying structural congruence with natural samples.}
    \label{fig:cytochrome_generation}
\end{figure}

\begin{table}[!htb]
\small
\renewcommand{\arraystretch}{1}
\centering
\caption{Evaluation of the DeepSTARR dataset. The reported values represent the Pearson correlation coefficient.}
\begin{tabular}{lccc}
\toprule
 & DeepSTARR & NT-multi & \textbf{Gener}\textit{ator} \\
\midrule
Developmental & \underline{0.68} & 0.64 & \textbf{0.70} \\
Housekeeping & 0.74 & \underline{0.75} & \textbf{0.79} \\
\bottomrule
\end{tabular}
\label{tab:enhancer_benchmark}
\end{table}
\begin{figure}[!htb]
    \centering
    \includegraphics[width=0.6\textwidth]{figures/pdf/enhancer_design.pdf}
    \caption{Enhancer design. (A-B) Pearson correlation between the predicted enhancer activity and the measured activity. (C-D) Distribution densities of the predicted activity of generated enhancer sequences with distinct activity profiles.}
    \label{fig:enhancer_design}
\end{figure}

\section{Discussion}
\label{sec:discussion}

\rv{
While the results show that \name is able to simulate human-like eye movements when performing analytical tasks, there is a need to expand on our discussion of the model's practical implications, the generalizability of the modeling approach, and the limitations and potential for supporting sophisticated chart-based question answering.
}

\subsection{\rv{Applications}}

\rv{
\paragraph{Visualization design evaluation}
\name can assist in evaluation of chart design.
With well-controlled experiment conditions, eye tracking data afford valuable insight into chart designs, especially relative to alternative designs.
For example, \citet{goldberg2011eye} showcased eye tracking's value in comparing line and radial graphs for reading of values, by allowing researchers to understand the viewing order of AOIs and the task completion time.
\name holds potential to replace human input to evaluation based on eye tracking.
With the simulated scanpaths from \name, chart designers can obtain quick and cost-effective feedback that yields the benefits from eye tracking without requiring an expensive empirical study.
}
\rv{
\paragraph{Visualization design optimization}
Beyond evaluation, another potential usage application of \name is to help optimize visualization design~\cite{shin2023perceptual}. 
Like other fields of design, visualization design requires user feedback for continual iteration. When visualization designers create charts for specific tasks, they may wonder if the design is suitable for delivery.
With the predicted scanpaths from the model, they can easily access quick and affordable feedback before deeming a candidate design ready for expensive evaluation in a user study.
Predictive models could offer feedback to designers or even provide optimization goals in automated visualization design frameworks.
The ultimate goal is grounding for recommendations for visualizations that support specific tasks~\cite{albers2014task} and even automation of visualization design in real time.
Today's human-in-the-loop design optimization paradigm~\cite{kadner2021adaptifont} could shift to a user-agent-in-the-loop approach, wherein a computational agent that simulates human feedback enables scalable and efficient design evaluation.
}

\rv{
\paragraph{Explainable AI in chart question answering}
Systems for answering questions via charts~\cite{masry2022chartqa} are typically viewed as black boxes that generate answers directly from a given chart and natural-language question. 
In contrast, \name introduces a glass-box approach that answers questions through a step-by-step reasoning process. This method enhances the alignment between human and machine attention~\cite{sood2023multimodal}.
We anticipate that this approach could lead to significant improvements in chart question answering~\cite{masry2022chartqa} and greater compatibility with explainable AI systems.
}

\subsection{\rv{Extending the Model beyond Bar Charts}}

\rv{
Our modeling approach can be extended to many visualization types besides bar charts.
We analyzed the visualization taxonomy outlined in prior work~\cite{borkin2013makes, borkin2015beyond}, including area, circle, diagram, distribution, grid, line, map, point, table, text, tree, and network, then categorize these visualization techniques into two groups: those that are feasible to extend with minor changes and those that are out of reach, requiring additional features.
}

\begin{figure}[!t]
    \centering
    \subfigure[\rv{An \textit{RV} task with a line chart: ``What was the revenue from newspaper advertising in 1980?''}]{\label{fig:a}\includegraphics[width=0.48\textwidth]{Images/line-case.png}}
    \hspace{0.02\columnwidth}
    \subfigure[\rv{An \textit{F} task with a scatterplot: ``In which countries do people anticipate spending about \$700 for personal Christmas gifts?''}]{\label{fig:b}\includegraphics[width=0.48\textwidth]{Images/point-case.png}}
    \caption{\rv{Two cases that illustrate the generalizability of the modeling approach, showing the extension of \name to a line chart and a scatterplot. The model's predictions are spatially similar to human ground-truth scanpaths.}}
    \label{fig:case}
    % \vspace{-5mm}
\end{figure}

\rv{
Our modeling approach can be applied to most statistical charts either directly or upon rectification of minor issues. For instance, extending the model to interpret \textit{line charts} and \textit{area charts} is feasible when the axis labels are clearly defined. The trend patterns of lines and areas can be perceived by the peripheral vision as visual guidance.
For \textit{point charts}, such as scatterplots, the model performs well in conditions of sparse data points. However, individual points may be obscured in dense scatterplots, making it difficult to label data when points are cluttered or overlapping. 
\textit{Distribution charts}, such as histograms, and \textit{circle charts}, such as pie charts, are similar to bar charts in that they use the area of marks to represent values. Retrieving exact values from these two presentation types can be imprecise on account of the ranges of the bins and inaccuracies in estimating angles or arc lengths.
Reading \textit{grid charts} (e.g., heatmaps) too is feasible; however, identifying the values necessitates understanding color intensity, a factor that can sometimes lead to ambiguity.
Modeling scanpaths on \textit{tables} or \textit{text} for retrieval tasks is tractable under the current modeling approach, but a lack of visual pattern recognition may render the results poor.
To further examine the generalizability of this category, we considered two additional cases, using a line chart and a scatterplot. We manually labeled the charts, trained the model, and made predictions. As Figure~\ref{fig:case} attests, the trained model performs well for these two chart types when compared to human ground-truth scanpaths.
}

\rv{
Other, sophisticated visualization types are out of reach because they require additional features, particularly prior knowledge and advanced reasoning abilities. For instance, reading \textit{maps} involves associating spatial regions with colors, sizes, or symbols to retrieve related values. Also, when interpreting maps, people rely heavily on preexisting geographical knowledge as a basis for efficient visual searches. Complex designs with intricate structures, such as \textit{diagrams}, \textit{trees}, and \textit{network graphs}, typically require advanced reasoning based on connections. All these skill requirements point to a need for further study in this area.
}

\subsection{\rv{Paths toward Sophisticated Tasks in Chart Question Answering}}

\rv{
Although the model focuses primarily on gaze prediction, it is worth exploring potential improvements for enriching its sophisticated question answering capabilities. We also discuss its limitations.
}

\rv{
Our current model does not achieve the same level of accuracy as the state-of-the-art models represented by the ChartQA benchmark~\cite{masry2022chartqa}. 
Unlike other models that can access the full chart image, \name is limited by its foveal vision and restricted spatial reasoning abilities. For instance, if a bar's height falls between two labeled values, such as 10 and 15, the model might choose either 10 or 15 as its answer when interpreting the axis, failing to provide a more precise value. 
This limitation stems from the constrained spatial perception capabilities of LLMs, which are central to cognitive control.
One possible solution is integrating multi-modal LLMs~\cite{cuarbune2024chart}, for which recent research has demonstrated an accuracy rate of 81.3\%.
}

\rv{
The sense-making process for complex visualizations may be inherently challenging. Even humans often struggle with understanding how the data are encoded, recognizing a given chart's purpose, tackling readability issues, performing numerical calculations, identifying relationships among data points, and navigating the spatial arrangement of graphical elements~\cite{rezaie2024struggles}.
Our model is designed to be straightforward and objective, focusing on analysis tasks related to statistical charts, but it does not fully capture the complexities of visualizations.
A possible enhancement in this respect would be to integrate the model with human sense-making practices~\cite{rezaie2024struggles} or to incorporate a framework of human understanding~\cite{albers2014task}. Such integration could facilitate better simulation of a human-like problem-solving process.
}

\section{Related Works}
\paragraph{Explicit Personalization.}
As humans express their own preferences, recent works explored aligning LLMs with individual values through explicit cues. Multifacet~\citep{multifacet} has focused on designing diverse and detailed system prompts for LLM control. PAD~\citep{chen2024pad} and MetaAligner~\citep{yang2024metaaligner} have leveraged fine-grained RM—such as HelpSteer~\citep{helpsteer}—to construct specific policies and guide model behavior toward system prompts. Others allow users to directly specify attribute importance weights, either for training~\citep{yang2024rewards, DPA} or decoding-time alignments~\citep{dekoninck2023controlled, shi2406decoding}. 

\paragraph{Implicit Personalization.}
While they have advanced explicit preferences, implicit preferences behind users' behaviors remain understudied, as Table~\ref{tab:method_comparison}. \citet{jin2024implicit} has shown that these values arise from complex interactions between factors like experiences, education, lifestyle, and even dietary habits, leading to misalignment with explicitly stated preferences~\citep{nisbett1977telling}
To address this gap, several works proposed implicit personalization tasks - from title generation~\citep{ao-etal-2021-pens}, movie tagging~\citep{salemi2023lamp} to summarization~\citep{zhang2024personalsum}.
Notably, PRISM~\citep{kirk2024prism} made notable progress by collecting preference annotations from conversations with over a thousand users, though its effectiveness was limited by the small number of annotations per user, making traditional RLHF approaches challenging.

Our work advances this field in two key ways: First, we introduce the Perspective dataset, which enables more reliable evaluation. Second, we propose Drift, \textit{decoding-time few-shot personalization}. 
By addressing the challenges of implicit preferences, our approach represents a significant step forward in implicit personalized alignments.
% \section{Discussion and Future Work}

% %Onscad is insample data cause these LLMS have seen openscad

% The decision space of language design is enormous, so we had to make some decisions about what to explore in the language design of AIDL. In particular, we did not build a new constraint system from scratch and instead developed ours based on an open-source constraint solver. This limited the types of primitives we allow, e.g. ellipses are not currently supported. \jz{Additionally, rectangles in AIDL are constrained to be axis-aligned by default because we found that in most use cases, a rectangle being rotated by the solver was unintuitive, and we included a parameter in the language allows rectangles to be marked as rotatable. While this feature was included in the prompts to the LLM, it was never used by the model. We hope to explore prompt-engineering techniques to rectify this issue in the future. Similarly, we hope to reduce the frequency of solver errors by providing better prompts for explaining the available constraints.} \adriana{Add two other limitations to this paragraph: that we typically noticed that things are axis alignment, say why we use this as default and in the future could try to get the gpt to not use default more often. Mention that we still have Solver failures that could be addressed by better engineering in future. }

% In testing our front-end, we observed that repeated instances of feedback tends to reduce the complexity of models as the LLM would frequently address the errors by removing the offending entity. This leads to unnecessarily removed details. More extensive prompt-engineering could be employed in future work to encourage the LLM to more frequently modify, rather than remove, to fix these errors. \adriana{no idea what this paragraph is trying to say}


% \adriana{This seems  like a future work paragraph so maybe start by saying that in the future you could do other front end or fine tune a model with aidl, we just tested the few shot.  } \jz{In the future, we hope to improve our front-end generation pipeline by finetuning a pretrained LLM on example AIDL programs.} In addition, multi-modal vision-langauge model development has exploded in recent months. Visual modalities are an obvious fit for CAD modeling -- in fact, most procedural CAD models are produced in visual editors -- but we decided not to explore visual inputs yet based on reports ([PH] cite OPENAIs own GPT4V paper) that current vision-language models suffter from the same spatial reasoning issues as purely textual models do (identifying relative positions like above, left of, etc.). This also informed our decision to omit spline curves which are difficult to describe in natural language. This deficit is being addressed by the development of new spatial reasoning datasets ([PH] cite visual math reasoning paper), so allowing visual user input as well as visual feedback in future work with the next generation of models seems promising.

% The decision space of language design is enormous, so it was impossible to explore it all here. We had to make some decisions about what to explore, guided by experience, conjecture, technical limitations, and anecdotal experience. Since we primarily explore the interaction between language design and language models in order to overcome the shortcomings in the latter, we did not wish to focus effort on building new constraint systems. This led us to use an open-source constraint solver to build our solver off of. This limited the types of primitives we allow; in particular, most commercial geometric solvers also support ellipses.

% In testing our generation frontend, we observed that repeated instances of feedback tended to reduce the complexity of models as the LLM would frequently address the errors by removing the offending entity. This is a fine strategy for over-constrained systems, but can unnecessarily remove detail when done in response to a syntax or validation error. More extensive prompt-engineering could be employed to encourage the LLM to more frequently modify, rather than remove, to fix these errors


% In recent months, multi-modal vision-language model development has exploded. Visual modalities are an obvious fit for CAD modeling -- in fact, most procedural CAD models are produced in visual editors -- but we decided not to explore visual input yet based on reports (cite OpenAIs own GPT4V paper) that current vision-language models suffer from the same spatial reasoning issues as purely textual models do (identifying relative positions like above, left of, etc.). This also informed our decision to omit spline curves; they are not easily described in natural language. This deficit is being addressed by the development of new spatial reasoning datasets (cite visual math reasoning paper), so allowing visual user input as well as visual feedback in future work with the next generation of models seems promising.



\section{Conclusion}

AIDL is an experiment in a new way of building graphics systems for language models; what if, instead of tuning a model for a graphics system, we build a graphics system tailored for language models? By taking this approach, we are able to draw on the rich literature of programming languages, crafting a language that supports language-based dependency reasoning through semantically meaningful references, separation of concerns with a modular, hierarchical structure, and that compliments the shortcomings of LLMs with a solver assistance. Taking this neurosymbolic, procedural approach allows our system to tap into the general knowledge of LLMs as well as being more applicable to CAD by promoting precision, accuracy, and editability. Framing AI CAD generation as a language design problem is a complementary approach to model training and prompt engineering, and we are excited to see how advance in these fields will synergize with AIDL and its successors, especially as the capabilities of multi-modal vision-language models improve. AI-driven, procedural design coming to CAD, and AIDL provides a template for that future.

% Using procedural generation instead of direct geometric generation enables greater editability, accuracy, and precision
% Using a general language model allows for generalizability beyond existing CAD datasets and control via common language.
% Approaches code gen in LLMs through language design rather than training the model or constructing complexing querying algorithms. This could be a complimentary approach
% Embedding as a DSL in a popular language allows us to leverage the LLMs syntactic knowledge while exploiting our domain knowledge in the language design
% LLM-CAD languages should hierarchical, semantic, support constraints and dependencies




%In this paper, we proposed AIDL, a language designed specifically for LLM-driven CAD design. The AIDL language simultaneously supports 1) references to constructed geometry (\dgone{}), 2) geometric constraints between components (\dgtwo{}), 3) naturally named operators (\dgthree{}), and 4) first-class hierarchical design (\dgfour{}), while none of the existing languages supports all the above. These novel designs in AIDL allow users to tap into LLMs' knowledge about objects and their compositionalities and generate complex geometry in a hierarchical and constrained fashion. Specifically, the solver for AIDL supports iterative editing by the LLM by providing intermediate feedback, and remedies the LLM's weakness of providing explicit positions for geometries.

%\adriana{This seems  like a future work paragraph so maybe start by saying that in the future you could do other front end or fine tune a model with aidl, we just tested the few shot.  }
%\paragraph{Future work} In recent months, multi-modal vision-language model development has exploded. Visual modalities are an obvious fit for CAD modeling -- in fact, most procedural CAD models are produced in visual editors -- but we decided not to explore visual input yet based on reports (cite OpenAIs own GPT4V paper) that current vision-language models suffer from the same spatial reasoning issues as purely textual models do (identifying relative positions like above, left of, etc.). This also informed our decision to omit spline curves; they are not easily described in natural language. This deficit is being addressed by the development of new spatial reasoning datasets (cite visual math reasoning paper), so allowing visual user input as well as visual feedback in future work with the next generation of models seems promising. 

\section*{Limitations}

While Drift contributes promising advances in implicit personal preferences, several limitations remain that should be addressed in future research.

\paragraph{Needs of Online Human Evaluation Benchmarks.}  
A major challenge in personal preference research is the absence of reproducible human evaluations. Even if future benchmarks collect more user-specific annotations beyond PRISM, evaluating personalized generation outputs requires \textit{re-engaging with the same users for feedback}. Although we designed the Perspective dataset to align the label construction and test set evaluation pipelines, it still relies on virtual personas. Therefore, to advance this field, there is a need for online evaluation benchmarks that can reproducibly assess personalized generation using real user feedback.

\paragraph{Limited Analysis Between Drift Attributes and Actual Users.}  
Due to practical and ethical issues, we do not have full access to the backgrounds of actual users. While the PRISM dataset provides basic information (e.g., the intended use of LLMs and brief self-introductions), our analysis (as seen in Figure~\ref{fig:activated-attributes-prism}) is limited in explaining why certain attributes are activated and how these relate to user characteristics. A more in-depth investigation into the correlation between Drift attributes and real user profiles should be studied with future benchmarks.

\paragraph{Biases in Differential Prompting.}  
Our study does not thoroughly analyze the limitations of the zero-shot rewarding mechanism used for each attribute. It is possible that differential prompting may fail to capture certain attributes accurately, and methods like those employed in Helpsteer2—where data is explicitly constructed—could offer more precise evaluations. Nevertheless, given the vast diversity of personal preferences, a zero-shot approach remains essential. As shown in Figure~\ref{fig:rm-results}, this approach yields significantly higher performance, and Figure~\ref{fig:attributes_num} demonstrates that even when the number of attributes is reduced to levels comparable to those used in Helpsteer2, performance remains robust. In essence, unreliable attributes are unlikely to be used during decoding, which mitigates this limitation. Moreover, as future research develops to enable LLM to follow system prompts more precisely, these advances will directly enhance Drift.

\paragraph{Tokenizer Dependency.}  
Drift Decoding adjusts the next-token distribution at each step, which requires that the LLM and the sLM share the same support—that is, they must use the same tokenizer. 

\paragraph{Limited Baselines.}  
Due to the scarcity of datasets for implicit personal preferences, this domain is far less mature compared to explicit preferences. As highlighted in Table~\ref{tab:method_comparison}, the limited availability of extensive baselines forced us to concentrate primarily on analyzing the unique characteristics of Drift.

\section*{Ethical Statement}

While Drift effectively integrates users’ implicit preferences into generated outputs, it also introduces several ethical risks that must be carefully managed. Notably, the Drift Approximation stage allows us to directly assess the activation levels of each attribute, which provides an opportunity to identify and preemptively block system prompts that could lead to harmful or undesirable content before they are incorporated into the decoding process. This capability underscores the importance of further research into combining Drift with diverse system prompts, ensuring that the generation of undesirable content is minimized while still delivering personalized services.

Additionally, considering that existing research~\citep{kim2024guaranteed} indicates it is impossible to obtain filtered autoregressive distributions under certain conditions, it is necessary to combine rejection sampling on final outputs using safeguards~\citep{lifetox} such as LlamaGuard~\citep{llamaguard} and ShieldGemma~\citep{shieldgemma}. This approach can further enhance the safety of the final generated content.




\bibliography{custom}

\appendix

\clearpage

\begin{table*}[t]
\centering
\resizebox{\textwidth}{!}{
\begin{tabular}{l|cccc} % Begin the table with specified column alignments
\toprule
Method & Training-free & General Policy & Smaller LM Guidance & Implicit Preference \\ 
\midrule
MORLHF~\citep{li2020deep}        & \textcolor{red}{$\times$}  & \textcolor{green}{$\checkmark$} & - & \textcolor{green}{$\checkmark$}  \\
MODPO~\citep{zhou2024beyond}   & \textcolor{red}{$\times$}      & \textcolor{green}{$\checkmark$} & -   & \textcolor{green}{$\checkmark$}  \\
Personalized soups~\citep{jang2023personalized}  & \textcolor{red}{$\times$}& \textcolor{red}{$\times$}  & \textcolor{red}{$\times$}  & \textcolor{red}{$\times$} \\
Preference Prompting~\citep{jang2023personalized} &  \textcolor{green}{$\checkmark$}  &\textcolor{green}{$\checkmark$}  & -  & \textcolor{red}{$\times$} \\
Rewarded soups~\citep{rame2024rewarded}   & \textcolor{red}{$\times$} & \textcolor{red}{$\times$}  & \textcolor{red}{$\times$}  & \textcolor{red}{$\times$}  \\
RiC~\citep{yang2024rewards}    & \textcolor{red}{$\times$}       & -  & \textcolor{red}{$\times$} & \textcolor{red}{$\times$}  \\
DPA~\citep{DPA}     & \textcolor{red}{$\times$}      & \textcolor{green}{$\checkmark$} & - & \textcolor{red}{$\times$}  \\
ARGS~\citep{khanov2024args}   & \textcolor{green}{$\checkmark$}       & \textcolor{green}{$\checkmark$}  & \textcolor{green}{$\checkmark$} & \textcolor{green}{$\checkmark$}  \\
MOD~\citep{shi2406decoding}    & \textcolor{green}{$\checkmark$}       & \textcolor{red}{$\times$}  & \textcolor{red}{$\times$}   & \textcolor{red}{$\times$}  \\
MetaAligner~\citep{yang2024metaaligner}  & \textcolor{green}{$\checkmark$}  & \textcolor{green}{$\checkmark$}  & \textcolor{green}{$\checkmark$}   & \textcolor{red}{$\times$} \\
PAD~\citep{chen2024pad}  & \textcolor{green}{$\checkmark$}  & \textcolor{green}{$\checkmark$}  & \textcolor{red}{$\times$}   & \textcolor{red}{$\times$} \\
\midrule
\includegraphics[height=1.8ex]{figs/racer.png} Drift~(Ours)   & \textcolor{green}{$\checkmark$}         & \textcolor{green}{$\checkmark$} & \textcolor{green}{$\checkmark$} & \textcolor{green}{$\checkmark$} \\
\bottomrule
\end{tabular}
}
\caption{Key characteristics of previous methods and Drift.} % Table caption
\label{tab:method_comparison}
\end{table*}

\section{Proof}

\subsection{Derivation of the RL Closed-Form Solution}
\label{proof-rl}

We want to solve the following optimization problem (for a single variable \(x\)):
\[
\max_{\theta} \Bigl[\,r(x)\;-\;\beta \,\log \tfrac{\pi_{\theta}(x)}{\pi_{base}(x)}\Bigr].
\]
Define \(\pi(x) = \pi_{\theta}(x)\). The quantity we want to maximize can be thought of as an expectation under \(\pi(x)\):
\[
\max_{\pi} \int \pi(x) \Bigl[r(x) - \beta\,\log \tfrac{\pi(x)}{\pi_{base}(x)}\Bigr] \, dx,
\]
subject to
\[
\int \pi(x)\, dx = 1 \quad\text{and}\quad \pi(x) \ge 0.
\]

Introduce a Lagrange multiplier \(\lambda\) to enforce the normalization constraint \(\int \pi(x)\,dx = 1\). The Lagrangian is
\begin{equation}
\begin{split}
    \mathcal{L}[\pi, \lambda] 
= &\int \pi(x)\,\Bigl[r(x) - \beta\,\log \tfrac{\pi(x)}{\pi_{base}(x)}\Bigr]\,dx 
\;\\- &\;\lambda\!\Bigl(\int \pi(x)\,dx - 1\Bigr). \nonumber
\end{split}
\end{equation}

We now take the functional derivative of \(\mathcal{L}\) w.r.t.\ \(\pi(x)\) and set it to zero for optimality:
\[
\frac{\delta \mathcal{L}}{\delta \pi(x)} 
= r(x) 
- \beta \Bigl[\log \tfrac{\pi(x)}{\pi_{base}(x)} + 1 \Bigr]
- \lambda 
= 0.
\]

Rearranging:
\[
r(x) - \beta \,\log \tfrac{\pi(x)}{\pi_{base}(x)} - \beta - \lambda = 0,
\]
which implies
\[
\beta \,\log \tfrac{\pi(x)}{\pi_{base}(x)} = r(x) - \beta - \lambda.
\]

Exponentiate both sides:
\[
\tfrac{\pi(x)}{\pi_{base}(x)} 
= \exp\!\Bigl(\tfrac{r(x)}{\beta}\Bigr) \;\exp\!\Bigl(- 1 - \tfrac{\lambda}{\beta}\Bigr).
\]
So
\[
\pi(x) 
= \pi_{base}(x)\,\exp\!\Bigl(\tfrac{r(x)}{\beta}\Bigr)\,\exp\!\Bigl(-1 - \tfrac{\lambda}{\beta}\Bigr).
\]
Let $C = \exp\!\Bigl(-1 - \tfrac{\lambda}{\beta}\Bigr)$.
Hence
\[
\pi(x) = C\,\pi_{base}(x)\,\exp\!\Bigl(\tfrac{r(x)}{\beta}\Bigr).
\]

We find \(C\) by imposing the constraint \(\int \pi(x)\,dx = 1\):
\[
1 
= \int \pi(x)\,dx 
= C \int \pi_{base}(x)\,\exp\!\Bigl(\tfrac{r(x)}{\beta}\Bigr)\,dx.
\]
Therefore
\[
C 
= \biggl[\int \pi_{base}(x)\,\exp\!\Bigl(\tfrac{r(x)}{\beta}\Bigr)\,dx \biggr]^{-1}.
\]

Putting it all together, the optimal distribution \(\pi^{*}(x)\) is
\[
\pi^{*}(x) 
= \frac{\pi_{base}(x)\,\exp\!\Bigl(\tfrac{r(x)}{\beta}\Bigr)}
       {\displaystyle\int \pi_{base}(x)\,\exp\!\Bigl(\tfrac{r(x)}{\beta}\Bigr)\,dx}.
\]

This shows that the optimal solution is a \emph{Boltzmann-like} (or \emph{softmax}) distribution given by weighting the reference distribution \(\pi_{base}(x)\) with the exponential of the scaled reward \(r(x)/\beta\).

\subsection{Expanded Explanation for Drift Decoding}
\label{appendix:drift-decoding-proof}
In Section~\ref{sec:drift-decoding}, we introduced the following target distribution for Drift Decoding:
\begin{equation}
    \label{eq:drift-decoding-target-dist-appx}
    \tilde{\pi}(w) \;\propto\; \pi_{\text{LLM}}(w) 
    \;\prod_{i=1}^{k} \Bigl(\tfrac{\pi_i(w)}{\pi_{\text{base}}(w)}\Bigr)^{\frac{p_i}{\beta}},
\end{equation}
where \(\pi_{\text{LLM}}(w)\) is the probability of token \(w\) under the LLM, \(\pi_i(w)\) is the probability of token \(w\) under an attribute-specific prompt (i.e., \(\pi(\cdot \mid s_i)\)), \(\pi_{\text{base}}(w)\) is the probability under a base prompt, and \(p_i\) is the weight for the \(i\)-th attribute estimated by \textit{Drift Approximation}. The hyperparameter \(\beta\) controls the strength of personalization via KL regularization.
Then Eq~\eqref{eq:drift-decoding-target-dist-appx} can be equivalently written in \emph{logit space} as
\begin{equation}
\begin{split}
  \tilde{\pi}(w) 
  \;&=\; 
  \text{softmax}\!\Bigl[
     h^{\text{LLM}}(w)
     \;\\ \nonumber &+\;
     \frac{1}{\beta}\sum_{i=1}^{k} 
        p_i\,\bigl(h^i(w) \;-\; h^{\text{base}}(w)\bigr)
  \Bigr], 
\end{split}
\end{equation}
where \(h^\text{LLM}\), \(h^i\), and \(h^\text{base}\) are the \emph{logits} (i.e., \(\log\)-probabilities) of \(\pi_{\text{LLM}}\), \(\pi_i\), and \(\pi_{\text{base}}\), respectively.
By definition of the logits, let $h^{\text{LLM}}(w) = \log \pi_{\text{LLM}}(w)$, $h^i(w) = \log \pi_i(w)$,  $h^{\text{base}}(w) = \log \pi_{\text{base}}(w)$.
Then Eq~\eqref{eq:drift-decoding-target-dist-appx} can be rewritten as
\begin{equation}
\begin{split}
  \tilde{\pi}(w)
  \;&\propto\;
  \exp\bigl(h^{\text{LLM}}(w)\bigr)
  \;\\ \nonumber &\prod_{i=1}^{k} 
    \exp\!\Bigl(
      \frac{p_i}{\beta}\,\bigl[
         h^i(w) \;-\; h^{\text{base}}(w)
      \bigr]
    \Bigr).
\end{split}
\end{equation}
Combining the exponential terms, we get
\begin{equation}
\begin{split}
  \tilde{\pi}(w)
  \;&\propto\;
  \exp\!\Bigl[
    h^{\text{LLM}}(w)
    \;\\ \nonumber &+\;
    \frac{1}{\beta}\sum_{i=1}^k p_i\bigl(h^i(w) \;-\; h^{\text{base}}(w)\bigr)
  \Bigr].
\end{split}
\end{equation}
Since the \(\mathrm{softmax}\) operation normalizes these exponentials to sum to 1 over all possible tokens \(w\), it follows that
{\tiny
\[
  \tilde{\pi}(w)
  \;=\;
  \frac{
    \exp\!\Bigl[
      h^{\text{LLM}}(w)
      + \frac{1}{\beta}\sum_{i=1}^k p_i\bigl(h^i(w) - h^{\text{base}}(w)\bigr)
    \Bigr]
  }{
    \sum_{w'} 
      \exp\!\Bigl[
        h^{\text{LLM}}(w')
        + \frac{1}{\beta}\sum_{i=1}^k p_i\bigl(h^i(w') - h^{\text{base}}(w')\bigr)
      \Bigr]
  }
\]
}
\[
  \;=\;
  \text{softmax}\Bigl[
     h^{\text{LLM}} 
     + \frac{1}{\beta}\sum_{i=1}^k p_i\,\bigl(h^i - h^{\text{base}}\bigr)
  \Bigr][w].
\]
% \[
%   \tilde{\pi}(w)
%   \;=\;
%   \frac{
%     \exp\!\Bigl[
%       h^{\text{LLM}}(w)
%       + \frac{1}{\beta}\sum_{i=1}^k p_i\bigl(h^i(w) - h^{\text{base}}(w)\bigr)
%     \Bigr]
%   }{
%     \sum_{w'} 
%       \exp\!\Bigl[
%         h^{\text{LLM}}(w')
%         + \frac{1}{\beta}\sum_{i=1}^k p_i\bigl(h^i(w') - h^{\text{base}}(w')\bigr)
%       \Bigr]
%   }
%   \;=\;
%   \text{softmax}\Bigl[
%      h^{\text{LLM}} 
%      + \frac{1}{\beta}\sum_{i=1}^k p_i\,\bigl(h^i - h^{\text{base}}\bigr)
%   \Bigr][w].
% \]
This completes the proof.

\section{Details of Perspective Dataset}
\label{sec:perspective-details}
In this section, we describe the principles underlying the design of our Perspective dataset. To evaluate personal preferences accurately, the evaluation must adhere exactly to the individual criteria used during the annotation of the training data. In other words, the data construction process and evaluation pipeline must be identical, which makes evaluations based on actual human responses challenging. Therefore, our primary objective is to enable reliable evaluation even using virtual personas.

\subsection{Dataset Construction}

For constructing the dataset, a diverse set of well-defined persona concepts was essential. To this end, we leveraged the Multifacet~\citep{multifacet} dataset, which defines various dimensions that can be combined to create a wide range of persona concepts. In the Multifacet dataset, each persona is associated with one question and three answers. However, our methodology required a substantial number of question–preference pairs per persona. To achieve this, we followed these steps:
\begin{enumerate}
    \item \textbf{Collection:} Gather ten distinct, non-overlapping personas from diverse domains within the Multifacet dataset.
    \item \textbf{Question Selection:} For each persona, select related questions based on specific sub-dimensions.
    \item \textbf{Evaluation:} Instruct GPT-4 to evaluate the triplets consisting of one question and three answers (\(\{Q, A, A, A\}\)) using system prompts tailored to each persona. \texttt{gpt-4-turbo} assigns scores to each QA pair, thereby determining the preferred and less preferred answers.
\end{enumerate}
\noindent
During the creation process, \texttt{gpt-4-turbo} evaluated the answers using an explicitly defined persona. This same approach can later be adopted to assess generation results, ensuring a reliable evaluation procedure. As a result, we generated an average of 7,642 questions and 15,284 answers per persona. Below shows an example instance from the dataset, featuring a specific persona along with its corresponding QAA triplet and associated scores.

\begin{Verbatim}[fontsize=\small]
'gold_persona': "Assume the role of a seasoned
consultant with advanced expertise in the 
construction and engineering sectors ...,
'prompt': 'In Python, I have encountered ...,
'win': 'Certainly! The header `# -*- ...,
'lose': "Certainly, diving into the `# -*- ...,
 'win_score': 5,
 'lose_score': 4
\end{Verbatim}


\subsection{Comparison to the PRISM Dataset Instance}

The PRISM dataset provides user personal information and self-introductions as shown below:

\begin{Verbatim}[fontsize=\small]
'user_id': 'user1008',
'lm_familiarity': 'Somewhat familiar',
'lm_indirect_use': 'Yes',
'lm_direct_use': 'Yes',
'lm_frequency_use': 'Every day',
'self_description': "The importance in my
life right now is having ...",
'age': '45-54 years old',
'gender': 'Female',
'employment_status': 'Working full-time',
'education': 'Some University but no degree',
'marital_status': 'Divorced / Separated',
'english_proficiency': 'Native speaker',
'study_locale': 'us',
'religion': {'self_described': 'christianity',
              'categorised': 'Christian',
              'simplified': 'Christian'},
'ethnicity': {'self_described': 'white',
              'categorised': 'White',
              'simplified': 'White'},
'location': {'birth_country': 'Australia',
             'birth_countryISO': 'AUS',
             'birth_region': 'Oceania',
             'birth_subregion': 'Australia ...',
             'reside_country': 'United States',
             'reside_region': 'Americas',
             'reside_subregion': 'Northern ...',
             'reside_countryISO': 'USA',
             'same_birth_reside_country': 'No'},
'lm_usecases': {'homework_assistance': 0,
                'research': 1,
                'source_suggestions': 0,
                'professional_work': 0,
                'creative_writing': 1,
                'casual_conversation': 1,
                'personal_recommendations': 1,
                'daily_productivity': 0,
                'technical_...': 0,
                'travel_guidance': 0,
                'lifestyle_and_hobbies': 1,
                'well-being_guidance': 1,
                'medical_guidance': 1,
                'financial_guidance': 0,
                'games': 1,
                'historical_or_news_insight': 1,
                'relationship_advice': 1,
                'language_learning': 1,
                'other': 0,
                'other_text': None}
\end{Verbatim}

Although the PRISM dataset also provides explicit persona information through user profiles, there is no guarantee that these explicit personas align with the implicit personas used during annotation. Consequently, unlike the Perspective dataset—where the explicit persona is directly distilled into the implicit persona—the PRISM dataset does not support the same evaluation methodology. Moreover, since each user contributes at most 50 instances, it is not feasible to construct a gold-standard reward model from the PRISM dataset. For these reasons, PRISM is used only as a qualitative benchmark in preference modeling experiments.

\subsection{Misalignments between \textit{Explicit} and \textit{Implicit} preferences}

In the psychology domain, there has been discussion about the difficulty of fully expressing one's deep, complex, hidden preferences through language~\citep{nisbett1977telling, pronin2001you}. Recent studies~\citep{jin2024implicit} have also discussed how these \textit{implicit} values are intricately intertwined among various factors. The PRISM dataset contains user self-introductions describing their preferences and stated preferences regarding LLM usage. When we provided this information to \texttt{gpt-4-turbo} to predict individual user preferences, it achieved an accuracy of approximately 57\%. While this doesn't represent a comprehensive explicit preference analysis, considering the general preference aspects used in prediction, it suggests that explicit preferences alone may be insufficient to explain complex implicit preferences, or there may be mismatches between them. However, as mentioned in the Limitations section, due to the absence of online human evaluation benchmarks, extensive analysis is not possible, and we leave this as an intriguing interpretation for future researchers.

\section{Details of Drift Implementation}

\subsection{Used Differential System Prompts for Zero-shot Rewarding}

\begin{figure}[h!]
\begin{center}
\begin{verbatimbox}

\scriptsize You are tasked with creating a comprehensive system prompt for an AI assistant. This system prompt will define the assistant's personality, capabilities, and interaction style based on specific characteristics. Your goal is to create a clear, concise, and effective system prompt that incorporates all the provided variables.\\

First, review the assistant's domain:

<assistant\_domain>\{assistant\_domain\}</assistant\_domain>\\

Now, review the assistant's characteristics:

<assistant\_characteristics>\\
<tone>\{tone\}</tone>\\
</assistant\_characteristics>\\

Now, review the desired characteristics for the system prompt itself:

<system\_prompt\_characteristics>\\
<specificity>\{specificity\}</specificity>\\
<examples>\{examples\}</examples>\\
</system\_prompt\_characteristics>\\

Using these characteristics, create a system prompt for the AI assistant that effectively communicates the assistant's role, capabilities, and interaction style while incorporating all the characteristics.\\

Example structure of the final output:\\
<system\_prompt>

[Your system prompt incorporating all requirements and characteristics]

</system\_prompt>\\

Please proceed with creating the system prompt based on these instructions.
\end{verbatimbox}
\end{center}
\caption{Meta-prompt for generating system prompts for the misaligned actions experiments.}
\label{fig:misalignment-meta-prompt}
\end{figure}

In our experiments, we use the system prompts for each attribute as shown in Table~\ref{tab:system_prompts}. Although minor performance variations may occur due to changes in the basic template, we employ the most fundamental system prompt template in this paper to serve as a baseline for future research.

\subsection{Detailed Hyperparameters and Models}

Table~\ref{app:hyperparameters} shows the hyperparameters used in our experiments. Since the overall algorithm does not perform gradient computations, the hyperparameter space is limited. In the Drift Approximation stage, the number and definition of attributes determine everything, as detailed in Table~\ref{tab:system_prompts}. Similarly, in Drift Decoding, the logit-level computations are deterministic, so the only variable is the choice of samplers.


\begin{table}[htbp]
  \centering\resizebox{\columnwidth}{!}{
  \begin{tabular}{ll}
    \toprule
    Hyperparameter & value \\
    \midrule
    % \specialrule{.1em}{.05em}{.05em} 
    Frozen LLM & Llama-8B\tablefootnote{\url{https://huggingface.co/meta-llama/Llama-3.1-8B-Instruct}}, Gemma-9B\tablefootnote{\url{https://huggingface.co/google/gemma-2-9b-it}} \\
    \midrule
    Small LM for RM & Llama-1B\tablefootnote{\url{https://huggingface.co/meta-llama/Llama-3.2-1B-Instruct}}, Gemma-2B\tablefootnote{\url{https://huggingface.co/google/gemma-2-2b-it}} \\
    \midrule
    LoRA~\citep{hu2021lora} $r$ for RM & 8 \\
    \midrule
    LoRA $\alpha$ for RM & 32 \\
    \midrule
    LoRA training epochs for RM & 5 \\
    \midrule
    top-p for generation & 0.9 \\
    \midrule
    $\beta$ for generation & 0.5 \\
    \midrule
    text\_length & 500 \\
    \midrule
    attributes\_num for generation & 7 \\
    \bottomrule
  \end{tabular}}
\caption{Hyperparameters used for the experiments.}
\label{app:hyperparameters}
  \vspace{-3 mm}
\end{table}

\section{Expanded Analysis}

\subsection{Activated Attributes for Each User}
\label{app:attributes-activation}
This section interprets and analyzes PRISM's actual personal preferences. Looking at Figure~\ref{fig:activated-attributes-prism}, we can see that the activated attributes vary significantly between individuals. In particular, PRISM's actual users show dynamic patterns compared to each other user.


\begin{figure*}[ht]
\centering
\includegraphics[width=\textwidth]{figs/activation-user-expanded.pdf}
\caption{For each user in PRISM, there is a $W-L$ (Win-Loss) value for each attribute. The higher this value is, the more that user can be interpreted as preferring that attribute.
}
\label{fig:activated-attributes-prism}
\vspace{-3mm}
\end{figure*}



% \begin{figure*}[ht]
% \centering
% \includegraphics[width=\textwidth]{figs/activation-expanded.pdf}
% \caption{For each persona in Perspective, there is a $W-L$ (Win-Loss) value for each attribute. The higher this value is, the more that user can be interpreted as preferring that attribute.}
% \label{fig:activated-attributes-perspective}
% \vspace{-3mm}
% \end{figure*}

\subsection{Expanded Case study for Personalized Generation in PRISM}
\label{app:prism_analysis_section}

\begin{figure*}[t]
    \centering
    \includegraphics[width=0.95\textwidth]{assets/raw/case.pdf}
    \caption{Case study on two representative long-context QA problems.}
    \label{fig:case_study}
\end{figure*}

In this section, we present a personalized generation case study by examining the complete set of generated outputs. Table~\ref{tab:case-study-appendix1} shows the full version of the main paper, while Table~\ref{tab:case-study-appendix2} and \ref{tab:case-study-appendix3} provide additional analysis. The characteristics shown in the main paper are also evident in the full text version. While Llama-8B's pure generation attempts to provide neutral, fact-based answers like the lose response, Drift tries to provide responses from various angles like the win response. This tendency can also be observed in Table~\ref{tab:case-study-appendix2}, where \texttt{user1280} asked a question regarding the possibility of UFOs existing, and among the responses—one neutral and one open to the possibility—they selected the latter. While Llama-8B tends to focus on a neutral perspective, the output generated via Drift maintains the overall response structure while offering a more open stance on the possibility. In Table~\ref{tab:case-study-appendix3}, \texttt{user1247} poses a philosophical question about belief in existence. While the lose response and LLM pure output suggest the possibility of building understanding through dialogue and data accumulation, Drift, like the win response, definitively argues that this transcends the realm of logic and that AI's belief in God's existence is impossible. These examples of win-lose responses suggest that Drift's approximation effectively captures user preference characteristics and demonstrates sufficient ability to generate responses that users are likely to prefer during the decoding phase.







\end{document}
