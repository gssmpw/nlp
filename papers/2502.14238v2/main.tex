\documentclass[journal]{IEEEtran} 
% \documentclass[conference]{IEEEtran}
\usepackage{times}

% numbers option provides compact numerical references in the text. 
\usepackage[numbers]{natbib}
\usepackage{multicol}
\usepackage[bookmarks=true]{hyperref}



% \documentclass[letterpaper, 10 pt, conference]{ieeeconf}
% \IEEEoverridecommandlockouts
%\usepackage{cite}
\usepackage{hyperref}
\usepackage{amsmath,amssymb,amsfonts}
% \usepackage{amsmath, amsthm, amsfonts, bm, amssymb, enumitem, tabularx}
% \usepackage{amsfonts, graphicx, grffile,fullpage, float, latexsym, mathtools, bbm}
\DeclareMathOperator*{\argmin}{arg\,min}
\DeclareMathOperator*{\argmax}{arg\,max}
\usepackage{algorithmic}
\usepackage{graphicx}
\usepackage{textcomp}
\usepackage{xcolor}
% \usepackage[x11names]{xcolor}
% \usepackage[font=small,labelfont=bf]{caption}
\usepackage{subcaption}
\usepackage{algorithm,algorithmic,amsmath}
\usepackage{algorithm}
\usepackage[utf8]{inputenc}
% \def\BibTeX{{\rm B\kern-.05em{\sc i\kern-.025em b}\kern-.08emT\kern-.1667em\lower.7ex\hbox{E}\kern-.125emX}}
\usepackage{array}
\usepackage{multirow}
\usepackage{microtype}
\usepackage{stfloats}
\usepackage{url}
\usepackage{verbatim}
\usepackage{graphicx}
% \usepackage{epsfig}
\usepackage{microtype}
% \usepackage{subfigure}
% \usepackage{wrapfig}
\hyphenation{op-tical net-works semi-conduc-tor IEEE-Xplore}
\usepackage{tikz}

%\usepackage{mathptmx}
\usepackage{times}

\let\labelindent\relax
\usepackage{enumitem}
\usepackage{setspace}
\usepackage{color, colortbl}
% \usepackage[hidelinks]{hyperref}
\definecolor{Gray}{gray}{0.9}
\definecolor{LightCyan}{rgb}{0.88,1,1}
\definecolor{LightPink}{rgb}{1,0.89,0.88}
\definecolor{LightGreen}{rgb}{0.85,1,0.8}

\usepackage{xcolor}
\hypersetup{
    colorlinks,
    linkcolor={red!50!black},
    citecolor={blue!50!black},
    urlcolor={blue!80!black}
}

% \newtheorem{theorem}{Theorem}
%%%%%%%%%%%%%%%%%%%%%%%%%%%%%%%%
% THEOREMS
%%%%%%%%%%%%%%%%%%%%%%%%%%%%%%%% 
\makeatletter
\def\@opargbegintheorem#1#2#3{\trivlist
   \item[]{\bfseries #1\ #2\ (#3)} \itshape}
\makeatother
\newtheorem{theorem}{\bf Theorem}[section]
\newtheorem{proposition}[theorem]{\bf Proposition}
\newtheorem{lemma}[theorem]{Lemma}
\newtheorem{corollary}[theorem]{\bf Corollary}
\newtheorem{definition}[theorem]{\bf Definition}
\newtheorem{assumption}[theorem]{\bf Assumption}
\newtheorem{remark}[theorem]{\bf Remark}

\newcommand{\nf}[1]{\textcolor{blue}{#1}}
\DeclareMathOperator{\proj}{proj}
\newcommand{\change}[1]{\textcolor{red}{#1}}
\newcommand{\todo}[1]{\textcolor{red}{(TODO: #1)}}
\newcommand{\revised}[1]{\textcolor{brown}{#1}}
\newcommand{\response}[1]{\textcolor{green}{Response:#1}}

% \pdfinfo{
%    /Author (Homer Simpson)
%    /Title  (Robots: Our new overlords)
%    /CreationDate (D:20101201120000)
%    /Subject (Robots)
%    /Keywords (Robots;Overlords)
% }
\begin{document}
\title{No Minima, No Collisions: Combining Modulation and Control Barrier Function Strategies for Feasible Dynamical Collision Avoidance}
\author{Yifan Xue$^{\dagger}$ and~Nadia Figueroa$^{\dagger}$~\IEEEmembership{Member,~IEEE}% <-this % stops a space
\thanks{$^{\dagger}$Y. Xue and N. Figueroa are with the Department of Mechanical Engineering and Applied Mechanics, University of Pennsylvania, Philadelphia, PA 19104 USA 
	{\tt\footnotesize \{yifanxue, nadiafig\}@seas.upenn.edu}}%
}
\maketitle
\begin{abstract}
As prominent real-time safety-critical reactive control techniques, Control Barrier Function Quadratic Programs (CBF-QPs) work for control affine systems in general but result in local minima in the generated trajectories and consequently cannot ensure convergence to the goals. Contrarily, Modulation of Dynamical Systems (Mod-DSs), including normal, reference, and on-manifold Mod-DS, achieve obstacle avoidance with few and even no local minima but have trouble optimally minimizing the difference between the constrained and the unconstrained controller outputs, and its applications are limited to fully-actuated systems. We dive into the theoretical foundations of CBF-QP and Mod-DS, proving that despite their distinct origins, normal Mod-DS is a special case of CBF-QP, and reference Mod-DS's solutions are mathematically connected to that of the CBF-QP through one equation. Building on top of the unveiled theoretical connections between CBF-QP and Mod-DS, reference Mod-based CBF-QP and on-manifold Mod-based CBF-QP controllers are proposed to combine the strength of CBF-QP and Mod-DS approaches and realize local-minimum-free reactive obstacle avoidance for control affine systems in general. We validate our methods in both simulated hospital environments and real-world experiments using Ridgeback for fully-actuated systems and Fetch robots for underactuated systems. Mod-based CBF-QPs outperform CBF-QPs as well as the optimally constrained-enforcing Mod-DS approaches we proposed in all experiments. 
% Additionally, to enable comparison between Mod-DS, CBF-QP and Mod-based CBF-QP approaches in realistic settings considering robot kinematics, we propose constraint-enforcing Mod-DS strategies to achieve speed and velocity enforcement for Mod-DS that optimally minimize the difference between the constrained and the unconstrained controller outputs while guaranteeing robot safety. 
\end{abstract}
% \begin{IEEEkeywords}
% Control Barrier Function, Modulation, Control, Dynamical System, Safety-Critical System
% \end{IEEEkeywords}
\section{Introduction}
\label{Introduction}
Recent developments in autonomous systems have brought increasing research efforts into the field of robot obstacle avoidance and safe control system design. Autonomous robots colliding with environmental obstacles or their co-workers would not only reduce work efficiency but also injure users in safety-critical tasks ranging from autonomous driving to household human-robot cooperation \cite{ROB-052}. In literature, collision-free guarantees in dynamic environments are achieved either through the satisfaction of constraints in optimization-based approaches \cite{ames2019control,mpcdc2021safety,9653152,9319250,8967981,10380695} or through reactive potential-field inspired closed-form solutions \cite{apf1989real,navigationf1992exact,harmonic1997real,khansari2012dynamical,6907685,LukesDS,billard2022learning,10164805}.   
Quadratic Programming based Control Barrier Functions (CBF-QP) \cite{ames2019control} and Dynamical System motion policy modulation (Mod-DS) \cite{khansari2012dynamical} are two notable methods for modifying an agent's nominal controller from these categories, respectively. 

% %%%% Need to Redo this figure (blur face)
% \begin{figure}[!tbp]
%      \centering
%      \begin{subfigure}[b]{\linewidth}
%          \centering
%          % \includegraphics[trim={0cm 0.5cm 0cm 0},clip,width=0.90\linewidth]{images/fetch-snapshots/frame1.png}
%          \includegraphics[trim={0cm 1cm 0cm 1cm},clip,width=0.90\linewidth]{images/fetch-snapshots/frame4_blurred.png}
%      \end{subfigure}
%      \caption{\todo{update after completing experiments} \change{This first figure should be modified. I was thinking maybe a matrix of figures showing the different experimental setups/scenarios like I have in this paper https://ieeexplore.ieee.org/document/9197038.}}
%      \label{fig:intro-figure}
%       \vspace{-20pt}
% \end{figure}

CBF-QP minimally adjusts a nominal controller while ensuring safety through affine inequality constraints in the context of set invariance \cite{ames2019control}. Mod-DS, inspired by Harmonic Potential functions \cite{harmonic1997real} \cite{LukesDS}, is one of the state-of-the-art closed-form approaches in obstacle avoidance, which eliminated the excessive local minima faced by Artificial Potential Fields (APF) \cite{apf1989real} and can be generalized to different environments more easily than navigation functions \cite{navigationf1992exact}. Mod-DS, initially introduced in \cite{khansari2012dynamical} for convex obstacle avoidance reshapes a desired motion vector field to circumnavigate obstacle boundaries using the geometry of the collision surface through a \textit{modulation} matrix, which we referred as "normal Mod-DS" in the paper. It was later extended by \cite{LukesDS} to "reference Mod-DS", which handles star-shaped obstacles with impenetrability and (almost) global convergence guarantees, and by \cite{onManifoldMod} to "on-manifold Mod-DS", which realizes local-minimum-free obstacle avoidance for all obstacle geometries in fully actuated systems. While Mod-DS approaches are highly reactive, limited by their closed-form nature, they cannot straightforwardly account for input or kinematic/dynamic constraints like CBF-QP, and are more likely to cause collisions in scenarios where the robot is tightly squeezed between multiple obstacles. Moreover, the application of Mod-DS approaches is so far limited to fully actuated systems. CBF-QP, contrarily, holds the advantage of working for control affine systems in general but suffers from local minimum issues facing all obstacle geometries, even convex ones. 

With the emergence of different safety-critical control approaches, several attempts have been made to compare them. For example, \cite{singletary2021comparative} compared CBF-QP with APF in static environments with convex obstacles and analytically proved that APF is a special case of CBF-QP. \cite{li2021comparison} compares the performance of CLF-CBF QP with Hamilton-Jacobi reachability analysis in environments with static unsafe sets. Others draw comparisons between optimization-based safe control approaches with deep neural networks, such as \cite{9152161deep}, which studies the differences between deep learning and MPC for adaptive cruise control. Despite these efforts, a comprehensive analysis on general obstacle avoidance techniques in real-time dynamic settings is missing. Although Mod-DS has risen in popularity in recent years due to its simplicity of computation and strong theoretical guarantees proving that collision-free trajectories can be replanned in real-time without introducing topologically (unnecessary) minima or increased execution time \cite{LukesDS,billard2022learning,onManifoldMod}, it is rarely included in such comparisons. Furthermore, when choosing between reactive safe controllers like Mod-DS and CBF-QP, users are forced to make tradeoffs between the ability to enforce complicated input and multi-obstacle constraints and the reduction of local minima, no existing safe reactive controller is able to combine the strength of both CBF-QP and Mod-DS approaches to achieve local-minimum free and constraints-flexible obstacle avoidance for control affine system in general. 

In this work, our analysis focuses on comparing Mod-DS and CBF-QP because they generate feasible obstacle-avoiding motion policies within $10$ ms, as required by the dynamic environment of interactive robot tasks. Analyzing and understanding the behavioral similarities, differences, and theoretical connections between CBF-QP and Mod-DS approaches then sets the base for our proposal of Mod-based CBF-QP controllers, which embody the strength of both. The storyline of the paper is organized as follows: first, we quantitatively and qualitatively compare CBF-QP and Mod-DS variants with one another to unveil the behavior differences and similarities shared by them. For example, all CBF-QP and Mod-DS approaches, except for on-manifold Mod-DS, are subject to local minima caused by collinearity issues. Through the analysis, we conclude that Mod-DS provides a much higher target-reaching rate than CBF-QP, and is the preferred approach for safe navigation in concave-obstacle environments using fully actuated systems. Some might argue that Mod-DS approaches, suffering from closed-form safe control methods' common weakness in handling input constraints, are less practical than CBF-QP in real-life navigation tasks. Therefore, we propose speed-constraining and velocity-constraining strategies that enable Mod-DS to trivially enforce input constraints in fully-actuated systems, as the second contribution of the work. Then building on top of the observations, we prove theoretically that normal Mod-DS and CBF-QP are equivalent in single obstacle avoidance settings for fully actuated systems, and that reference Mod-DS solutions can be mathematically connected to CBF-QP using a one-line equation. Finally, based on the theoretical connections deduced, we propose reference and on-manifold Mod-based CBF-QP controllers as the most important contribution of our paper. The proposed Mod-based QP methods not only inherent CBF-QP's strength in handling complicated constraints, but are also able to reduce local minima to be no more than their corresponding reference and on-manifold Mod-DS approaches. Moreover, on-manifold Mod-based CBF-QP methods, like CBF-QP, are applicable to control affine systems in general. All methods proposed, including constraint-enforcing Mod-DS and Mod-based CBF-QP, are validated in multiple challenging concave and dynamic obstacle environments using the Ridgeback and Fetch robots, in which Mod-based CBF-QP controllers easily outperform both CBF-QP and Mod-DS approaches. 

\textbf{Contributions:} Following we summarize our major contributions:
\begin{enumerate}
   \item A theoretical and practical analysis, equivalence and comparison of Mod-DS vs. CBF-QP (\autoref{sec: quantitative and qualitative}, \autoref{sec:theoretical-analysis}).
    \item A novel constraint-enabled Mod-DS approach tailored for omni-directional robots, incorporating speed and velocity limits in its closed-form solution. We show that this approach matches the performance of velocity-constrained CBF-QP and is computationally efficient (\autoref{sec:constraing-mod}).
    \item Novel Mod-based CBF-QP controllers that achieves respectively local-minimum-reduction and local-minimum-free obstacle avoidance for control affine system in general (\autoref{sec:mod-cbf}).
    \item Extensive validation of the proposed constraint-enabled Mod-DS and Mod-based CBF-QP on simulations and real hardware experiments (\autoref{sec:experiment}).
\end{enumerate}

\section{Problem Statement} \label{problem statement}
\subsection{Safety-Critical Control for Dynamic Obstacle Avoidance}
Both subjects of the work, Mod-DS and CBF-QP, define the notion of robot safety and generate safe control algorithms based on boundary functions. Given a continuously differential function $h_o$ for an obstacle $o$ in the detected obstacle set $O$, state $x \in \mathbb{R}^d$ and obstacle state $x_o \in \mathbb{R}^{d'}$, $h_o:\mathbb{R}^d \times \mathbb{R}^{d'} \rightarrow \mathbb{R}$ is a boundary function if the safe set $C_o$ (outside the obstacle), the boundary set $\partial C_o$ (on the boundary of the obstacle), and the unsafe set $\neg C_o$ (inside the obstacles) of the system can be defined as in \eqref{eq:safe region o},~\eqref{eq:boundary o},~\eqref{eq:unsafe region o}~\cite{ames2019control}. 
%The boundary function $h$ is time-dependent for scenarios where the unsafe set are moving. In static obstacle avoidance, boundary function $h(x,t)=h(x)$.
% \begin{equation}
% C=\{x \in \mathbb{R}^d, x_o \in \mathbb{R}^{d'}: h(x,x_o)>0\}\label{safe region}
% \end{equation}
% \begin{equation}
% \partial C=\{x \in \mathbb{R}^d, x_o \in \mathbb{R}^{d'}: h(x,x_o)=0\}\label{boundary}
% \end{equation}
% \begin{equation}
% \neg C=\{x \in \mathbb{R}^d, x_o \in \mathbb{R}^{d'}: h(x,x_o)<0\}\label{unsafe region}
% \end{equation}
\begin{equation}
C_o=\{x \in \mathbb{R}^d, x_o \in \mathbb{R}^{d'}: h_o(x,x_o)>0\}\label{eq:safe region o}
\end{equation}
\begin{equation}
\partial C_o=\{x \in \mathbb{R}^d, x_o \in \mathbb{R}^{d'}: h(x,x_o)=0\}\label{eq:boundary o}
\end{equation}
\begin{equation}
\neg C_o=\{x \in \mathbb{R}^d, x_o \in \mathbb{R}^{d'}: h(x,x_o)<0\}\label{eq:unsafe region o}
\end{equation}

% \begin{equation}
% C=\{x \in \mathbb{R}^d, x_o \in \mathbb{R}^{d'}: h_o(x,x_o)>0, \forall o \in O\}\label{safe region}
% \end{equation}
% \begin{equation}
% \partial C=\{x \in \mathbb{R}^d, x_o \in \mathbb{R}^{d'}: h(x,x_o)=0, \forall o \in O\}\label{boundary}
% \end{equation}
% \begin{equation}
% \neg C=\{x \in \mathbb{R}^d, x_o \in \mathbb{R}^{d'}: h(x,x_o)<0, \forall o \in O\}\label{unsafe region}
% \end{equation}
Note that The boundary function $h_o(x, x_o)$ can be simplified to $h_o(x)$ when the obstacle state is static, i.e. $x_o$ is constant, and to $h(x, x_o)$ when the environment contains only one obstacle. In practice, $h_o(x,x_o)$ is often measured as the distance from the controlled agent to the obstacle surface boundary, i.e. the signed distance function. Given environments defined by boundary functions, the goal of a safety-critical controller is to generate an admissible input $u$ that will ensure the state of the robot $x$ is always within the safe set $C$ defined in \eqref{eq:safe region} and eventually reaches a target state $x^* \subset C \in\mathbb{R}^d$.
\begin{equation}
C=\{x \in \mathbb{R}^d, x_o \in \mathbb{R}^{d'}: h_o(x,x_o)>0, \forall o \in O\}\label{eq:safe region}
\end{equation}
\begin{equation}
\partial C=\{x \in \mathbb{R}^d, x_o \in \mathbb{R}^{d'}: h(x,x_o)=0, \forall o \in O\}\label{eq:boundary}
\end{equation}
\begin{equation}
\neg C=\{x \in \mathbb{R}^d, x_o \in \mathbb{R}^{d'}: h(x,x_o)<0, \forall o \in O\}\label{eq:unsafe region}
\end{equation}

\vspace{-5pt}
\subsection{Notations}
In this work, we use notation $||\cdot||_p$ to represent the p-norm of a vector and denote $||\cdot||_p^q = (||\cdot||_p)^q$, $\forall p$, $q \in \mathbb{R}^+$. We annotate $\langle a,b\rangle $ for the dot product between vectors a and b. $\textbf{0}_{a \times b}$ is a matrix full of 0s of size $a \times b$. Given a vector $a(\cdot)$, $\hat{a}$ represents the axis whose direction is the same as $a(\cdot)$, and $a[:k]$ represents the new vector formed by the first k elements in $a$.
\subsection{Assumptions}
\label{sec: assumptions}
\textbf{Safe Controller:} The safe-controller $u$ takes as input the current state $x$, the nominal control $u_{\text{nom}}$, and outputs the desired safe control input to the system, based on the robot's knowledge of the environment given by the obstacle boundary functions $h_o(x,t), \forall o \in O$.
In Section \ref{sec: quantitative and qualitative} and \ref{sec:constraing-mod}, $u_{\text{nom}}$ is defined assuming single integrator agent dynamics, while in Section \ref{sec:experiment} we formulate $u_{\text{nom}}$ to approximate linear dynamical systems assuming higher order control systems; i.e., Dubin's car.

\textbf{Obstacle Shapes:} In this work, we analyze and propose safe control methods facing all obstacle geometries, including convex, star-shaped, and non-star-shaped concave obstacles. Star-shaped obstacles are defined as obstacles that have a reference point inside the obstacle, from which all rays cross the boundary only once \cite{Sakaguchi1984}. Without loss of generality to applications in higher-dimensional state space, the methods compared and proposed are showcased in 2D navigation and manipulation tasks. We define  $x_\text{rel}(x, x_o) \in \mathbb{R}^2$ as the robot position measured in the obstacle-based frame, which changes with the obstacle's geometric center and orientation described in $x_o$. Safe controllers' abilities in navigating around convex obstacles are tested using a circle with boundary function $h_{\text{conv}}$ defined as in \eqref{eq: h_conv}, where $c_r \in \mathbb{R}^+$ is the radius of the circle. 
\begin{equation}
\begin{aligned}
\label{eq: h_conv} 
h_{\text{conv}}(x_\text{rel})& =||x_\text{rel}||_2-c_r\\
\end{aligned}  
\end{equation}
Likewise, star-shaped obstacles are defined using a funnel-shape with boundary function $h_{\text{star}}$ as in \eqref{eq: h_star}, where $C_a \in \mathbb{R}^2$ and $c_b \in \mathbb{R}^+$ are constants.
\begin{equation}
\begin{aligned}
\label{eq: h_star} 
h_{\text{star}}(x_\text{rel})& = ||x_\text{rel} - C_a||_4-c_b\\
\end{aligned}  
\end{equation}

Finally, non-star-shaped obstacles are represented using boundary function $h_\text{nstar}$ learned using the Gaussian Process Distance Field (GPDF) of a C-shape. Isoline maps showing the change in $h_\text{nstar}$ values with respect to the relative location of the agent can be found in Fig.~\ref{fig:isoline}. Boundary functions for convex, star, and non-star shapes are constructed to align with the safe set definition in Eq.~\eqref{eq:safe region o},~\eqref{eq:boundary o}, and \eqref{eq:unsafe region o}. Here we limit the scope of our discussion to cases where the size of the obstacles is fixed with no shrinkage, expansion, or deformation. All moving obstacles exhibit only translation and rotational motion, i.e.
\begin{equation}
x_o = \begin{bmatrix} x_L^o \\ x_R^o \\ \omega_o\end{bmatrix} 
\label{eq:x o def}
\end{equation}

\begin{equation}
R(x_o) = \exp{(\hat{\omega}_ox_R^o)}
\label{eq:R def}
\end{equation}

\begin{equation}
x_\text{rel}(x, x_o) = R(x_o)^\top(x-x_L^o)
\label{eq:x rel def}
\end{equation}
where $x_L^o$ is the obstacle's point of rotation on the rotation axis $\omega_o$, $x_R^o$ is the rotation angle, and $R = R(x_R^o)$ is the transformation matrix from the obstacle frame to the world frame, given axis-angle representation of $x_L^o$ and $x_R^o$. $\dot{x}_L^o$ is linear velocity, and $\dot{x}_R^o$ is the angular velocity of the obstacle. Substituting Eq.~\eqref{eq:x rel def} into boundary functions, it can be deduced that

\begin{equation}
h_o(x, x_o) = h_o(x_\text{rel}) = h_o(R(x_R^o)^\top (x-x_L^o))
\label{eq:h x rel}
\end{equation}


\section{Preliminaries} 
\label{background}
\subsection{Boundary Function Requirement}
According to \cite{ames2019control} and \cite{khansari2012dynamical,LukesDS,billard2022learning}, CBF-QP and Mod-DS method share the same requirement for the boundary function $h_o(x,x_o)$: $h_o(x,x_o)$ must be a continuously differentiable function of class $\mathcal{C}^1$ satisfying safe set definitions in Eq.~\eqref{eq:safe region o},~\eqref{eq:boundary o} and \eqref{eq:unsafe region o}. In the background section, we review CBF-QP and Mod-DS method in single obstacle avoidance settings, where $h_o(x,x_o) = h(x, x_o)$.

\subsection{Control Barrier Functions}
\label{sec:prelims-cbf}
In CBF-QP formulation, the boundary function $h(x,x_o)$ is also called the barrier function. To guarantee safety of the controlled agent, CBF-QP utilizes the Nagumo Set Invariance Theorem.
% Control Barrier Functions (CBF) to ensure set invariance of $C$ when solving the optimization problem to minimize the difference between the nominal controller $u_\text{nom}$ and the modified safe controller $u_\text{cbf}$.  \\

\begin{definition}[Nagumo Set Invariance]
\begin{equation}
\label{Nagumo}
C \text{ is set invariant} \iff \Dot{h}(x,x_o) \geq 0 \; \forall x \in \partial C\\
\end{equation}
\end{definition}

Control Barrier Functions are designed by extending Nagumo set invariance theorem to a "control" version\cite{ames2019control}, where the condition $\forall x \in \partial C$ is rewritten mathematically using an extended $\mathcal{K}_\infty$ function $\alpha$, 
\begin{equation}
\label{eq:CBF conditions}
 C \text{ is set invariant} \iff \exists u \; \text{s.t.} \; \Dot{h}(x,x_o, \dot{x}_o, u) \geq  -\alpha (h(x,x_o)).
\end{equation}

\begin{definition}[Extended $K_{\infty}$ Functions]
\label{def:K inf}
An extended $K_{\infty}$ function is a function $\alpha : \mathbb{R}\rightarrow \mathbb{R}$ that is strictly increasing and with $\alpha (0)=0$ ; that is, extended $K_{\infty}$ functions are defined on the entire real line $(-\infty, \infty) $.
\end{definition}


Control barrier functions can be generalized to any nonlinear affine systems of the form in \eqref{eq:affine system}, where $x \in \mathbb{R}^d, x_o \in \mathbb{R}^{d'}, u \in \mathbb{R}^p$, and $f, g$ are Lipschitz continuous. The CBF condition in \eqref{eq:CBF conditions} can be used to formulate a quadratic programming problem that guarantees safety by enforcing the set invariance of the safety set $C_o$ defined in \eqref{eq:safe region o}. For general control affine systems, CBF-QP is defined as in \eqref{eq:cbf-qp affine}. 
\begin{gather}
\dot{x} = f(x) + g(x)u \label{eq:affine system}\\
\nonumber u_{\text{cbf}} = \argmin_{{u} \in \mathbb{R}^p}\frac{1}{2}||u-u_{\text{nom}}||_2^2\\
L_fh(x,x_o) + L_gh(x,x_o)u + \nabla_{x_o} h(x,x_o)\dot{x}_o\geq -\alpha (h(x,x_o))\label{eq:cbf-qp affine} 
\end{gather}

For fully-actuated systems in \eqref{eq:fully actuated system}, where $x, u \in \mathbb{R}^d$, the special case of CBF-QP can be simplified as in \eqref{eq:cbf-qp fully actuated}.
\begin{gather}
\dot{x} = u \label{eq:fully actuated system}\\
\nonumber u_{\text{cbf}} = \argmin_{{u} \in \mathbb{R}^d}\frac{1}{2}||u-u_{\text{nom}}||_2^2\\
\nabla_xh(x,x_o)^\top u + \nabla_{x_o} h(x,x_o)\dot{x}_o \geq -\alpha (h(x,x_o))\label{eq:cbf-qp fully actuated} 
\end{gather}

% CBF-QP sets up a quadratic programming problem (QP) to solve for a constrained optimization-based controller that minimizes the change to existing nominal controller $u_{\text{nom}}$ \cite{ames2019control}. 
% \begin{gather}
% \nonumber u_{\text{mod}} = \argmin_{u_{\text{mod}} \in \mathbb{R}^n}\frac{1}{2}||u_{\text{mod}}-u_{\text{nom}}||_2^2\\
% \Dot{h}(x,t) \geq -\alpha (h(x,t))\label{cbf-qp} 
% \end{gather}

\textit{Closed-Form Solution:} CBF-QP has closed-form solutions in single obstacle environments when robot input limitations are ignored. Solving the convex optimization problem in \eqref{eq:cbf-qp affine} and \eqref{eq:cbf-qp fully actuated} using KKT conditions, the explicit CBF-QP solutions are respectively \eqref{eq: explicit cbf affine} and \eqref{eq:cbf-qp fully actuated}. For simplicity of the solutions, we abuse notations and let $\alpha = \alpha (h(x,x_o))+\nabla_{x_o} h(x,x_o)\dot{x}_o$, $L_fh=L_fh(x,x_o)$, $L_gh=L_gh(x,x_o)$ and $\nabla_xh = \nabla_xh(x,x_o)$.
\begin{equation}
\label{eq: explicit cbf affine}
u_\text{cbf} = \begin{cases}
u_\text{nom}  \quad \text{if} \quad L_fh+L_ghu_\text{nom}\geq -\alpha\\
u_\text{nom}- \frac{L_fh+L_ghu_\text{nom}+\alpha}{L_ghL_gh^\top}L_gh^\top\quad \text{otherwise}
\end{cases}
\end{equation}
\begin{equation}
\label{eq: explicit cbf fully actuated}
u_\text{cbf} = \begin{cases}
u_\text{nom} \quad \text{if} \quad \nabla_xh^\top u_\text{nom}\geq \alpha\\
u_\text{nom}- \frac{\nabla_xh^\top u_\text{nom}+\alpha}{\nabla_xh^\top\nabla_xh}\nabla_xh\quad \text{otherwise}
\end{cases}
\end{equation}

% \textit{Multiple Obstacles:} There exist two alternatives for using CBF-QPs in environments with multiple obstacles. One only enforces the CBF of the obstacle closest to the agent and the other includes all obstacles in the environment when solving the quadratic programming problem \cite{notomista2021safety}. We adopt the first method that takes into consideration the closest environmental obstacle, defined by $h_i$, unless otherwise noted.  
% \begin{gather}
% \Dot{h}_i(x,t) \geq -\alpha (h_i(x,t)) 
% \label{cbf-qp multi}
% \end{gather}

% The reason is that when the agent is trapped between 2 moving obstacles that are both dangerously close to it, paying attention to only the closest one could potentially lead the controlled agent to run into the neglected one in the next time step. 

% \begin{gather}
% \nonumber u(x,t)= \argmin_{u \in R^n}\frac{1}{2}||u-u_{\text{nom}}||_2^2\\
% \Dot{h}_i(x,t) \geq -\alpha (h_i(x,t))  \quad \forall i \in [1, m]
% \label{cbf-qp multi}
% \end{gather}

% \subsection{CBF for Dubin's Car}
% \begin{gather}
% \nonumber \dot{h}(x,t,v,\theta)= \argmin_{u \in R^n}\frac{1}{2}||u-u_{\text{nom}}||_2^2\\
% \Dot{h}_i(x,t) \geq -\alpha (h_i(x,t))  \quad \forall i \in [1, m]
% \label{cbf-qp multi}
\begin{figure*}[!tbp]
    \centering
    \begin{subfigure}{0.32\textwidth}
         \includegraphics[width=\textwidth]{images/isoline/isoline_convex_marked.pdf}
         \label{fig:isoline convex}
    \end{subfigure}
    \begin{subfigure}{0.32\textwidth}
         \includegraphics[width=\textwidth]{images/isoline/isoline_star_marked.pdf}
         \label{fig:isoline star}
    \end{subfigure}
    \begin{subfigure}{0.32\textwidth}
         \includegraphics[width=\textwidth]{images/isoline/isoline_nstar_marked.pdf}
         \label{fig:isoline nstar}
    \end{subfigure}
    \vspace{-15pt}
    \caption{Isolines displaying the value changes of $h_{\text{conv}}$, $h_{\text{star}}$, and $h_\text{nstar}$ in the proximity of respectively a circle, a funnel and a C-shaped obstacle.}\label{fig:isoline}
    \vspace{-10pt}
\end{figure*}
% \vspace{-10pt}

\subsection{Mod-DS Approach}
\label{sec:prelims-mod}
Unlike CBF-QP that works for control affine systems in general, Mod-DS can only be adopted in fully actuated systems described in \eqref{eq:fully actuated system}.

In Mod-DS, obstacle avoidance is achieved by multiplying a full-rank locally active matrix $M(x,x_o)\in\mathbb{R}^{d\times d}$ to a nominal DS $\dot{x}_{\text{nom}}$ and reshaping the flow by reducing the relative speed of the robot towards the obstacle, while increasing or maintaining the speed in the directions tangent to the obstacle surface represented by $h(x,x_o)$. Mod-DS is computed for moving obstacles as \cite{LukesDS,billard2022learning}, 
\begin{gather}
\label{eq:ds-modulation}
u_\text{mod} = \dot{x}_\text{mod}= M(x,x_o)(\dot{x}_{\text{nom}}-\bar{\dot{x}}_{o})+\bar{\dot{x}}_{o}
\end{gather} 
\begin{equation}
\label{eq:dxo}
    \bar{\dot{x}}_{o}=\dot{x}^{o}_{L}+\dot{x}^{o}_{R}\times(x-x_L^o),
\end{equation}
where $x^o_L$, $\dot{x}^{o}_{L}$, and $\dot{x}^{o}_{R}\in \mathbb{R}^d$ are the defined as in Eq.~\eqref{eq:x o def}. In a static scenario, Eq.~\eqref{eq:modulation-matrix} reduces to $\dot{x}=M(x)\dot{x}_{\text{nom}}$. 

$M(x,x_o)$ is constructed analogous to an Eigendecomposition, where $E(x,x_o) \in \mathbb{R}^{d \times d}$ is the basis matrix and $D(x,x_o) \in \mathbb{R}^{d \times d}$ is the diagonal scaling matrix. The basis matrix $E(x,x_o)$ consists of a direction vector facing the obstacle $d(x,x_o) \in \mathbb{R}^{d \times 1}$ and a hyperplane $H(x,x_o) \in \mathbb{R}^{d \times (d-1)}$ tangent to $h(x,x_o)$. The hyperplane $H(x,x_o)$ is an orthonormal basis formed by basis vectors $e_1(x,x_o), ... ,e_{d-1}(x,x_o)\in \mathbb{R}^{d\times 1}$. The basis matrix $E(x,x_o)$ enforces the redistribution of velocities along $d(x,x_o), e_1(x,x_o), ... ,e_{d-1}(x,x_o)$. The magnitudes of the velocities in the redistributed directions are then modified by $D(x,x_o)$ as in Eq.~\eqref{eq:modulation-matrix}. To achieve multiple obstacle avoidance, a weighted sum of the stretched velocity computed with respect to each obstacle is used, the details of which can be found in \cite{khansari2012dynamical}. 
\begin{gather}
\label{eq:modulation-matrix}
\begin{aligned}
M(x,x_o)=E(x,x_o)D(x,x_o)E(x,x_o)^{-1}\\
D(x,x_o)= \text{diag}( \lambda(x,x_o), \lambda_e(x,x_o), \dots, \lambda_e(x,x_o))\\  
H(x,x_o)= [e_1(x,x_o) \; ... \; e_{d-1}(x,x_o)]\\
E(x,x_o)=[d(x,x_o) \; H(x,x_o)]\\
\end{aligned}
\end{gather} 

There exist 3 main variants of Mod-DS approaches, different in terms of how $E(x,x_o)$ and $D(x,x_o)$ are constructed: normal Mod-DS, reference Mod-DS and on-manifold Mod-DS \cite{khansari2012dynamical}\cite{LukesDS}\cite{onManifoldMod}. The properties of each method are detailed in Table \ref{table:modulation types}. Since any smooth vector field in a sphere world has at least as many topologically critical points as obstacles \cite{KODITSCHEK1990412}, we consider a method to be able to handle certain types of obstacles if the modified DS has at most 1 local minimum per obstacle given a fully-actuated nominal DS. Mod-DS approaches have been proven to ensure \textit{impenetrability} at the obstacle boundary in the sense of von Neuman and if the nominal DS $\dot{x}_{\text{nom}}$ is globally asymptotically stable (g.a.s) then the modulated DS is g.a.s. to the same target point $x^*\in\mathbb{R}^d$ on all non-equilibrium points \cite{khansari2012dynamical, onManifoldMod}. 

\textit{Normal Mod-DS:} In normal Mod-DS, vector $n(x,x_o)$  normal to the boundary function $h(x,x_o)$ is used as the direction towards the obstacle when constructing the basis matrix $E(x,x_o)$ \eqref{eq:standard mod d}. It is noteworthy that $\lambda^*(x, x_o)$ and $\lambda^*_e(x, x_o)$ values listed in Table \ref{table:modulation types} are only popular choices of $D(x,x_o)$ but any $\lambda(x,x_o)$, $\lambda_e(x,x_o)$ satisfying conditions in \eqref{eq:standard mod lambda} will be feasible options. As the earliest proposed Mod-DS variants, normal Mod-DS only guarantees convergence to target $x^*$ in convex obstacle avoidance. 
% As the earliest proposed Mod-DS variants, standard modulation has the drawback of leading to local minima wherever $n(x)$ and $\dot{x}_\text{nom}$ are inversely collinear [Definition, \ref{def:saddle standard modDS}]. As a result, the method is not suitable for concave obstacle avoidance. 
\begin{equation}
    \label{eq:standard mod d}
    d_\text{n}(x,x_o)= n(x,x_o) = \frac{\nabla_xh(x,x_o)}{||\nabla_xh(x,x_o)||_2}
\end{equation}
\begin{equation}
    \label{eq:standard mod lambda}
    \lim_{h(x,x_o)\rightarrow0}\lambda(x,x_o)=0 \text{ and }\lambda_e(x,x_o)>0 \;\; \forall x\in\mathbb{R}^d
\end{equation}

% \begin{definition}[Saddle Equilibria in Standard Mod-DS] When Mod-DS encounters inverse collinearity between $\dot{x}_{\text{nom}}$ and $n(x)$; i.e., $\langle 
%  \frac{\dot{x}_{\text{nom}}}{||\dot{x}_{\text{nom}}||}, n(x) \rangle = -1$ the subsequent solutions of Eq. \eqref{eq:ds-modulation}\eqref{eq:modulation-matrix}\eqref{eq:standard mod d} will lead the agent to an spurious equilibrium , $\dot{x}\rightarrow0$, at the boundary of the obstacle $\partial C$ \cite{khansari2012dynamical}.
%  % \textbf{Proof:}  From Theorem 2 in \cite{LukesDS}. $\hfill \blacksquare$
%  \label{def:saddle standard modDS}
% \end{definition}

\textit{Reference Mod-DS:}
The only difference between reference Mod-DS and normal Mod-DS is that reference direction $r(x,t)$ is used instead of $n(x,x_o)$ to form the basis matrix $E(x,x_o)$ \eqref{eq:reference mod d}. $r(x,x_o)$ points from the agent's current location $x$ to a reference point $r^*(x_o) \in \mathbb{R}^d$. $r^*$ is a constant vector when the unsafe set is static. This change enables the reference Mod-DS to handle star-shaped obstacles, in addition to convex ones.
\begin{definition}[Reference Points in Star-Shaped Obstacles]
\label{def:reference point}
$r^*(x_o) \in \mathbb{R}^d$ is a reference point if and only if it is strictly inside unsafe set $\neg C_o$, where all rays from which have one and one only intersection with the boundary set $\partial C_o$. An obstacle or an unsafe set is considered star-shaped if it has a minimum of one reference point. 
\end{definition}
\begin{equation}
    \label{eq:reference mod d}
    d_\text{r}(x,x_o) = r(x,x_o) = \frac{x - r^*(x_o)}{||x - r^*(x_o)||_2}
\end{equation}

% \begin{definition}[Saddle Equilibria in Reference Mod-DS] When Mod-DS encounters inverse collinearity between $\dot{x}_{\text{nom}}$ and $r(x)$; i.e., $\langle 
%  \frac{\dot{x}_{\text{nom}}}{||\dot{x}_{\text{nom}}||}, r(x) \rangle = -1$ the subsequent solutions of \eqref{eq:ds-modulation}\eqref{eq:standard mod d} will lead the agent to an undesirable asymptotic equilibrium , $\dot{x}\rightarrow0$, at the boundary of the obstacle $\partial C$ \cite{LukesDS}.
%  \label{def:saddle-modDS}
% \end{definition}

% \begin{equation}
% E_n(x)=[n(x), e_1(x), ... ,e_{n-1}(x)]\label{eq:modulation n direction} 
% \end{equation}

% \begin{equation}
% \label{eq:modulation r direction} 
% E_r(x)=[r(x), e_1(x), ... ,e_{n-1}(x)]
% \end{equation}

\begin{table*}[!tbp]
\centering
\begin{minipage}[t]{0.95\textwidth}
\begin{center}
\centering
 {\begin{tabular}{  m{1.9cm} | m{6.2cm} | m{1.2cm} |m{2.8cm}| m{2.3cm} }
    \hline
    Type & $D(x, x_o)$ & $d(x,x_o)$ & Obstacle & Local Minima \#\\
    \hline
    Normal & $\lambda^*(x,x_o) = 1-\frac{1}{h(x,x_o)+1}$, $\lambda_e^* (x,x_o)= 1+\frac{1}{h(x,x_o)+1}$ & $n(x,x_o)$ & convex & 1\\
    \hline
    Reference & $\lambda^*(x,x_o) = 1-\frac{1}{h(x,x_o)+1}$, $\lambda_e^* (x,x_o)= 1+\frac{1}{h(x,x_o)+1}$ & $r(x,x_o)$ & convex, star-shaped & 1\\
    \hline
    On-Manifold & $D_\nabla(x,x_o) + D_\phi(x,x_o)+\varphi_h(x,x_o)D_h(x,x_o)$ & $n(x,x_o)$ & all & 0\\
    \hline
\end{tabular}}
\caption{Properties of 3 main types of Mod-DS, including the definition of their basis matrix and diagonal scaling matrix,  and the types of obstacles they can navigate without inducing a trapping region in the modified DS, are documented. For obstacle types that a Mod-DS variant can navigate without being trapped, the number of local minima that the approach induced per obstacle is listed. The definition of $n(x,x_o)$ and $r(x,x_o)$ can be found in Eq.~\eqref{eq:standard mod d} and \eqref{eq:reference mod d}}
\label{table:modulation types}
\end{center}
\end{minipage}
\vspace{-10pt}
\end{table*}


% In \cite{LukesDS, billard2022learning} it has been proven that ensures \textit{impenetrability} at the boundary in the sense of von Neuman and for star-shaped obstacles if $\dot{x}_{\text{nom}}=f(x)$ is globally asymptotically stable (g.a.s) then Mod-DS is g.a.s. to the same equilibrium point $x^*\in\mathbb{R}^n$, except for a single saddle trajectory on the boundary ($h(x)=0$) of the obstacle when $f(x), r(x)$ are inversely collinear; i.e., $\langle \frac{f(x)}{||f(x)||}, \frac{r(x)}{||r(x)||}\rangle = -1$. Nevertheless, with  Eq. \eqref{ds-modulation} the robot will never penetrate the obstacle as $h(x)\rightarrow 0$ causes $\lambda_r \rightarrow 0$, causing the normal speed towards the obstacle to always vanish at the boundary; \textbf{as long as that speed is achievable by the robot}. Like multi-obstacle CBF-QP, we implement Mod-DS with respect to the nearest obstacle. An alternative using the weighted sum of the modulation DS generated for each obstacle is presented in \cite{LukesDS}, however, to not favor one method over the other we use a cohesive ``nearest obstacle'' multi-obstacle strategy. 

\textit{On-Manifold Mod-DS:} On-manifold Mod-DS uses the same basis matrix $E(x,t)$ as the stanadard Mod-DS. 
\begin{equation}
d_\text{onM}(x,x_o) = n(x,x_o) = \frac{\nabla_xh(x,x_o)}{||\nabla_xh(x,x_o)||_2}
\end{equation}

However, unlike the other Mod-DS approaches that achieve safe control using uniform diagonal scaling policies defined in \eqref{eq:standard mod lambda}, on-manifold Mod-DS realizes undesirable-equilibrium-free safe control under the aggregate effects of 3 navigation policies, where $\lambda_\nabla$ encodes the gradient descent dynamics, $\lambda_\phi$ enables on-manifold morphing dynamics and $\lambda_h$ defines the regular Modulation dynamics. 
\begin{equation}
\lambda(x,x_o) = \lambda_\nabla(x,x_o) + \lambda_\phi(x,x_o)+\varphi_h(x,x_o)\lambda_h(x,x_o)
\end{equation}

The technical details of on-manifold Mod-DS can be found in \cite{onManifoldMod}. In this work, we are more interested in the obstacle exit strategy $\phi(x,x_o)$ used in $\lambda_\phi$ to help efficiently circumnavigate both convex and concave unsafe regions. Let $e^0$ be one out of $m$ candidate directions in $\mathbb{R}^{d-1}$, subject to $e^0 \notin \mathcal{N}(H(x,x_o))$, and $m \geq 2^{d-1}$. A geodesic approximation method can then utilize a first-order approximation to the obstacle
surface to closely approximate a path $X=[x^0, x^1, ..., x^N]$ exiting the obstacle on its isosurface, where horizon $N\in \mathbb{N}$ is a natural number \cite{onManifoldMod}. Function $\phi(x,x_o) \in \mathbb{R}^d$ then outputs, among $m$ candidate directions, the one with the smallest associated potential $P_N$, where $\beta$ is the step size and $p(x)$ is a user-defined reward function. In most applications, $p(x)$ can be the distance from $x_i$ to the target $x^*$.
\begin{equation}
\label{eq:geo approxi}
\begin{aligned}
x^{i+1} &= \beta H(x^i,x_o)H(x^i,x_o)^\top e^i+x^i\\
e^{i+1} &= \frac{H(x^i,x_o)H(x^i,x_o)^\top e^i}{||H(x^i,x_o)H(x^i,x_o)^\top e^i||_2}\\
P_{i+1} &= P_i + \beta p(x^{i+1})
\end{aligned}
\end{equation}

% \textbf{Note that Mod-DS is limited to fully-actuated systems.}
%%%%%%%%%%%%%%%%%%%%%%%%%%%%%%%%%%%%%%%%%%%%%%%%%%%%%%%%%%%%%%%%%%%%% Section IV
%%%%%%%%%%%%%%%%%%%%%%%%%%%%%%%%%%%%%%%%%%%%%%%%%%%%%%%%%%%%%%%%%%%%

\begin{figure*}[tbp]
    \centering
    \includegraphics[width=\textwidth]{images/comparison/comparison_streamplot_labeled_sparse_xc25.pdf}
    % \vspace{-2.5pt}
     \caption{Performance of different obstacle avoidance methods' facing respectively a convex, a star-shaped obstacle, and a non-star-shaped obstacle while following nominal linear DS in \eqref{eq:nominal system} with $\epsilon=2$. First four columns (a)-(d) correspond to CBF-QP with different $\mathcal{K}_{\infty}$ $\alpha(h)$ functions. The last 3 column corresponds to Mod-DS approaches. The colors on the trajectories indicate the ratio of the agent's modified speed to its nominal speed $\frac{||u||_2}{||u_{\text{nom}}||_2}$ as defined in Eq. \eqref{eq:nominal system}.  }
\label{fig:comparison}
\end{figure*}
\vspace{-2.5pt}


\section{CBF-QP and Mod-DS Performance Comparison on Different Obstacle Geometries} \label{sec: quantitative and qualitative}
\vspace{-2.5pt}
In this section, we compare popular Mod-DS variants and CBF-QP with different $\alpha$ functions qualitatively and quantitatively in fully actuated systems. Since all existing Mod-DS approaches are developed based on fully actuated system assumptions, conclusions drawn in this section cannot be extended to control affine systems in general. Our analysis focuses on single-obstacle avoidance in static environment for ease of illustrative visualizations in Section \ref{sec:qualitative-geometry} and for comparable metrics in Section \ref{sec:quantitative-geometry}\cite{khansari2012dynamical,LukesDS,billard2022learning,notomista2021safety,8967981}. Characteristics of Mod-DS and CBF-QP approaches studied then contributes to the theoretical analysis in Section \ref{sec:theoretical-analysis}.
\vspace{-2.5pt}
\subsection{Nominal Dynamics \& Obstacle Definition} 
 We define nominal controller $u_{\text{nom}}$ to follow the integral curves of an autonomous linear 2D dynamical system (DS) as in \eqref{eq:nominal system}, where $ \epsilon \in \mathbb{R}^+$. This nominal DS is globally asymptotically stable at target $x^*=[0,0]^\top$.
\begin{equation}
\label{eq:nominal system}
\dot{x}_{\text{nom}}=u_{\text{nom}}=-\epsilon x \quad \forall x \in\mathbb{R}^2  
\end{equation}


Three obstacle geometries are explored in the comparison: i) a circle of radius 2 \eqref{eq: h_conv}, ii) a star-shaped funnel centered at with $C_a = [2.5,0]^\top$ and $c_b = 0.1$ \eqref{eq: h_star}, and iii) an open ring with an inner radius of 2 and an outer radius of 2.3. 
\vspace{-2.5pt}
\subsection{Qualitative Comparison: Obstacle Geometry and Trapping Region}\label{sec:qualitative-geometry}
% \vspace{-2.5pt}
Mod-DS approaches' performances differ given their choices of $\lambda(x,x_o)$, $\lambda_e(x,x_o)$ and $d(x,x_o)$. Likewise, the parameterization of $\alpha$ functions CBF-QP affects the obstacle-avoiding behavior and speed profiles of the safe controller. In \autoref{fig:comparison} we present stream plots depicting the safe DS $\dot{x}$ modified by CBP-QP and Mod-DS approaches respectively, given nominal DS in \eqref{eq:nominal system}. Note that $\dot{x}=u$ in fully actuated systems \eqref{eq:fully actuated system}.

\subsubsection{Preservation of Nominal DS}
By comparing the color, which indicates the magnitude of modified to nominal control output ratios, and curvature, which represents directions of the velocities $u$ distorted by safe controllers at each state $x$, of the streamlines in \autoref{fig:comparison}, it can be qualitatively concluded that CBF-QP with $\alpha(h)=h$ and Mod-DS's ability in preserving the nominal DS is weaker than CBF-QP with $\alpha(h) = 5h$, because streamlines or trajectories generated by CBF-QP with $\alpha(h) = 5h$ are straighter, i.e. more similar to that in nominal linear DS \eqref{eq:nominal system}, and have larger yellow region. The color yellow means the magnitude of the modified controller $||u||_2$ is the same as that of the nominal ones $||u_\text{nom}|||_2$ at that state $x$. In CBF-QP, the larger $\alpha(h)$ is, given input $h$, the smaller the lower bound of $\dot{h}$ is, which results in a larger feasible set for $u$. Mod-DS approaches are weaker in preserving the nominal DS than CBF-QP, which can be qualitatively observed from the coverage of red regions in \autoref{fig:comparison}. Mod-DS modified controller often becomes larger in magnitude than the nominal controller because parameter $\lambda_e$ (\autoref{table:modulation types}) would boost $\dot{x}$ in the directions tangent to the obstacles. Additionally, viewing from the streamlines' curvatures in row 7, on-manifold Mod-DS drastically alters and even reverses the directions of $u_\text{nom}$. 

\subsubsection{Obstacle Geometries}
While CBF-QP and Mod-DS differ in their ability to preserve the nominal DS, they are similarly capable of handling convex obstacles. The generated trajectories form no obvious trapping regions, where the agent could be stuck, except an undesirable equilibrium at the boundary of the obstacle where $u_\text{nom}$ and $\nabla_xh$ are collinear. Further analysis of this undesirable equilibrium can be found in \autoref{sec:quantitative-geometry}. Plots 1(c)-5(c) and 1(e)-5(e) suggest the formation of a region of attraction when using CBF-QP to avoid star-shaped and non-star-shaped concave obstacles. Agents in the trapping region are led by one side of the obstacle boundary to move left and the other side to move right, which leads to convergence at an undesirable equilibrium point on the obstacle boundary. While the normal Mod-DS on row 5 forms the same trapping region as CBF-QP approaches, reference Mod-DS navigates star-shaped obstacles without forming any region of attraction and on-manifold Mod-DS is trapping-region-free for all obstacle geometries. Thus, the conclusion can be qualitatively drawn that reference and on-manifold Mod-DS provide more reasonable solutions facing concave obstacles. 
% Mod-DS, conversely, only exhibits 1 undesirable equilibrium point in the circled red region where the nominal DS and $r(x)$ are collinear, as expected from Def. \ref{def:saddle-modDS}. Thus, Mod-DS provides more reasonable solutions for star-shaped obstacles.


\begin{figure*}[tbp]
    \centering
    \includegraphics[width=\textwidth]{images/comparison/comparison_data_xc25.pdf}
     \caption{Trajectories, with 10 initial locations and target at the origin, generated by Mod-DS and CBF-QP obstacle avoidance methods to avoid respectively a convex, a star-shaped obstacle, and a non-star-shaped obstacle given the nominal controller $\epsilon = \frac{1}{||x||_2}$ \eqref{eq:nominal system}. First two columns correspond to CBF-QP with different $\mathcal{K}_{\infty}$ $\alpha(h)$ functions. The last three columns correspond to Mod-DS approaches. The colors on the trajectories indicate the robot speed magnitude at that state $x$. }
\label{fig:comparison data}
\end{figure*}
\vspace{-2.5pt}

\subsection{Quantitative Behavior Comparison}
\label{sec:quantitative-geometry}
% \vspace{-2.5pt}
In addition to obstacle geometry and the ability to preserve nominal DS, we present quantitative analysis on the behavior metrics of CBF-QP and Mod-DS approaches in \autoref{table:static table}.

\subsubsection*{Metrics} We utilized a set of controller behavior metrics proposed in \cite{zhou2022rocus}, including trajectory length $l$, average weighted jerk $\Bar{j}$, straight-line deviation $\eta$, obstacle clearance $d_\text{obs}$, and lastly, near obstacle velocity $v_\text{near}$ to evaluate of how distinct (or similar) the real-time collision avoidance performance of CBF-QP and Mod-DS approaches are. Straight-line deviation $\eta$ is designed to measure the extent to which the agent trajectory deviates from the nominal straight-line path. Average weighted jerk $\Bar{j}$ measures the abruptness of the path planned and indicates the smoothness of the final robot trajectory which impacts control feasibility. Finally, obstacle clearance $d_\text{obs}$ measures the average distance to the closest obstacles, and near obstacle velocity $v_\text{near}$ computes the average velocity during timesteps when the robot is near the obstacle boundaries. Metric equations are provided in Appendix \ref{metrics}.

% \begin{table*}[!tbp]
% \centering
% \begin{minipage}[t]{0.95\textwidth}
% \begin{center}
%  \resizebox{\linewidth}{!}{\begin{tabular}{  m{1.55cm} | m{2.75cm} | m{1.7cm}| m{2 cm} | m{1.7cm} | m{1.7cm}| m{1.8cm} | m{1.7cm} | m{1.5cm} }
%   \hline
%   \rowcolor{Gray} Shape & Method & $l$ (std) & $\Bar{j}$ (std) & $d_\text{obs}$ (std) & $v_{\text{near}}$ (std) &  $\beta$ (std)  & Runtime & Success \% \\ 
%   \hline
%   Convex & CBF-QP $\alpha=h$ & \cellcolor{LightGreen}7.5 (1.2) & \cellcolor{LightGreen} 1.5 (0.1) & 2.3 (0.4) & 0.96 (0.05) & 0.18 (0.24) & 0.0025 & 80\\
%   \hline
%    & CBF-QP $\alpha=5h$ & \cellcolor{LightGreen}7.5 (1.2) & 1.6 (0.1) & 2.3 (0.4) & 0.97 (0.04) & \cellcolor{LightGreen} 0.16 (0.22) &	0.0026 & \cellcolor{LightPink}80\\
%   \hline
%    & Standard Mod-DS & 7.8 (1.3) & 2.0 (1.0) & 2.5 (0.4) & \cellcolor{LightPink}1.22 (0.02) & 0.41 (0.17) & 0.0002 &	\cellcolor{LightPink}80\\
%   \hline
%    & Reference Mod-DS &  7.8 (1.3) & 2.0 (1.0) & 2.5 (0.4) & \cellcolor{LightPink}1.22 (0.02) & 0.41 (0.17) & 0.0002 & 80\\
%   \hline
%    & on-Mani Mod-DS & \cellcolor{LightPink}9.4 (1.7) & \cellcolor{LightPink}4.7 (5.3) & \cellcolor{LightGreen} 2.6 (0.3) & 0.95 (0.11) & \cellcolor{LightPink}0.91 (0.43) & 0.0004 &	\cellcolor{LightGreen}100\\
%   \hline
%   Star & CBF-QP $\alpha=h$ & 7.3 (1.3) & \cellcolor{LightGreen}1.6 (0.1) & 2.3 (0.3) & 0.95 (0.06) & 0.19 (0.25) & 0.0025 & \cellcolor{LightPink}70\\
%   \hline
%    & CBF-QP $\alpha=5h$ & \cellcolor{LightGreen}7.4 (1.3) &	1.7 (0.2) & 2.3 (0.4) & 0.96 (0.05) & \cellcolor{LightGreen}0.16 (0.24) & 0.0025 & \cellcolor{LightPink}70\\
%   \hline
%    & Modulation-DS & 8.0 (1.4) & \cellcolor{LightPink}3.2 (1.1) & 2.1 (0.4) & \cellcolor{LightPink}1.22 (0.07) & 0.50 (0.29) & 0.0002 & 90\\
%   \hline
%    & Reference Mod-DS & 7.8 (1.4) & 3.0 (1.5) & 2.3 (0.4) & \cellcolor{LightPink}1.26 (0.03) & 0.45 (0.24) & 0.0002 & 80\\
%   \hline
%    & on-Mani Mod-DS & \cellcolor{LightPink}9.5 (1.7) & \cellcolor{LightPink}6.2 (6.4) & \cellcolor{LightGreen} 2.4 (0.3) & 0.97 (0.13) & \cellcolor{LightPink}0.92 (0.52) & 0.0004 & \cellcolor{LightGreen}100 \\
%   \hline
%   Non-star & CBF-QP $\alpha=h$ & \cellcolor{LightGreen}6.8 (1.0) & \cellcolor{LightGreen} 1.5 (0.1) &  2.0 (0.2) & 0.98 (0.03) & 0.13 (0.17) &  0.0073 & \cellcolor{LightPink}50\\
%   \hline
%    & CBF-QP $\alpha=5h$ &  \cellcolor{LightGreen}6.8 (1.0) & 1.8 (0.3) &  2.0 (0.3) & 0.99 (0.02)  & \cellcolor{LightGreen}0.11 (0.16) & 0.0058 & \cellcolor{LightPink}50\\
%   \hline
%    & Standard Mod-DS & 7.2 (1.1) & 3.1 (1.6) & 2.1 (0.3) & \cellcolor{LightPink}1.32 (0.06) & 0.39 (0.14) & 0.0067 & 50\\
%   \hline
%    & Reference Mod-DS & 8.1 (1.5) & 2.4 (1.6) & 2.0 (0.3) & \cellcolor{LightPink}1.29 (0.06) & 0.63 (0.34) &  0.0055 &	80\\
%   \hline
%    & on-Mani Mod-DS & 11.3 (3.3) & \cellcolor{LightPink}5.2 (5.9) & \cellcolor{LightGreen} 2.3 (0.2) & 0.91 (0.11) & \cellcolor{LightPink}1.05 (0.46) & \cellcolor{LightPink} 0.1215 & \cellcolor{LightGreen}100\\
%   \hline
% \end{tabular}}
% \caption{Characteristics of trajectories generated respectively using Mod-DS, and CBF-QP approches in static obstacle avoidance at an controlled update frequency of 5Hz. The characteristics preferred by a ideal obstacle avoidance trajectory should contain small trajectory length $l$, small average jerk $\bar{j}$, large obstacle clearance $d_\text{obs}$, small near obstacle velocities $v_\text{near}$, small straight line deviation $\beta$, low execution time and high success rate in converging to the goal location. Here, the methods that perform worst in a behavior evaluation category is colored red and those perform best are colored green. }
% \label{table:static table old}
% \end{center}
% \end{minipage}
% \vspace{-10pt}
% \end{table*}

\begin{table*}[!tbp]
\centering
\begin{minipage}[t]{0.95\textwidth}
\begin{center}
 \resizebox{\linewidth}{!}{\begin{tabular}{  m{1.55cm} | m{2.75cm} | m{1.7cm}| m{2 cm} | m{1.7cm} | m{1.7cm}| m{1.8cm} | m{1.7cm} | m{1.5cm} }
  \hline
  \rowcolor{Gray} Shape & Method & $l/l_\text{nom}$ (std) & $\Bar{j}$ (std) & $d_\text{obs}$ (std) & $v_{\text{near}}$ (std) &  $\eta$ (std)  & Runtime (s) & Success \% \\ 
  \hline
  Convex & CBF-QP $\alpha=h$ & \cellcolor{LightGreen}1.03 (0.02) &  1.32 (0.04) & 2.43 (0.39) & 0.96 (0.05) & 0.34 (0.21) & 0.0025 & \cellcolor{LightPink}80\\
  \hline
   & CBF-QP $\alpha=5h$ & \cellcolor{LightGreen}1.03 (0.03) & 1.55 (0.03) & 2.40 (0.41) & 0.97 (0.04) & \cellcolor{LightGreen} 0.30 (0.22) &	0.0026 & \cellcolor{LightPink}80\\
  \hline
   & Normal Mod-DS & 1.06 (0.03) & 1.17 (0.04) & 2.52 (0.41) & \cellcolor{LightPink}1.20 (0.02) & 0.56 (0.07) & 0.0002 & 80\\
  \hline
   & Reference Mod-DS &  1.06 (0.03) & 1.17 (0.04) & 2.52 (0.41) & \cellcolor{LightPink}1.20 (0.02) & 0.56 (0.07) & 0.0002 & 80\\
  \hline
   & on-Mani Mod-DS & \cellcolor{LightPink}1.27 (0.17) & \cellcolor{LightPink}1.79 (0.71) & \cellcolor{LightGreen} 2.63 (0.34) & 0.94 (0.10) & \cellcolor{LightPink}0.97 (0.13) & 0.0004 &	\cellcolor{LightGreen}100\\
  \hline
  Star & CBF-QP $\alpha=h$ & \cellcolor{LightGreen}1.03 (0.03) & 1.55 (0.05) & 2.39 (0.33) & 0.95 (0.06) & 0.49 (0.21) & 0.0025 & \cellcolor{LightPink}70\\
  \hline
   & CBF-QP $\alpha=5h$ & \cellcolor{LightGreen}1.04 (0.04) &	1.72 (0.06) & 2.36 (0.35) & 0.96 (0.06) & \cellcolor{LightGreen}0.45 (0.21) & 0.0025 & \cellcolor{LightPink}70\\
  \hline
   & Normal Mod-DS & 1.09 (0.07) & 2.65 (0.68) & 2.20 (0.40) & \cellcolor{LightPink}1.20 (0.07) & 0.55 (0.07) & 0.0002 & 90\\
  \hline
   & Reference Mod-DS & 1.07 (0.05) & 2.07 (0.68) & 2.31 (0.37) & \cellcolor{LightPink}1.25 (0.04) & 0.60 (0.04) & 0.0002 & 80\\
  \hline
   & on-Mani Mod-DS & \cellcolor{LightPink}1.28 (0.17) & \cellcolor{LightPink}3.03 (1.16) & \cellcolor{LightGreen} 2.46 (0.26) & 0.95 (0.12) & \cellcolor{LightPink}1.02 (0.09) & 0.0005 & \cellcolor{LightGreen}100 \\
  \hline
  Non-star & CBF-QP $\alpha=h$ & \cellcolor{LightGreen}1.02 (0.01) &  1.31 (NA) &  2.07 (0.24) & 0.98 (0.03) & 0.43 (NA) &  0.0069 & \cellcolor{LightPink}50\\
  \hline
   & CBF-QP $\alpha=5h$ &  \cellcolor{LightGreen}1.02 (0.02) & 1.64 (NA) &  2.06 (0.25) & 0.99 (0.02)  & \cellcolor{LightGreen}0.39 (NA) & 0.0057 & \cellcolor{LightPink}50\\
  \hline
   & Normal Mod-DS & 1.07 (0.03) & 0.98 (NA) & 2.13 (0.31) & \cellcolor{LightPink}1.30 (0.06) & 0.64 (NA) & 0.0065 & 50\\
  \hline
   & Reference Mod-DS & 1.11 (0.06) & 1.16 (0.08) & 2.10 (0.28) & \cellcolor{LightPink}1.28 (0.06) & 0.90 (0.25) &  0.0055 &	80\\
  \hline
   & on-Mani Mod-DS & \cellcolor{LightPink}1.51 (0.34) & \cellcolor{LightPink}2.25 (0.44) & \cellcolor{LightGreen} 2.38 (0.15) & 0.90 (0.10) & \cellcolor{LightPink}1.14 (0.04) & \cellcolor{LightPink} 0.01224 & \cellcolor{LightGreen}100\\
  \hline
\end{tabular}}
\caption{Characteristics of trajectories generated using Mod-DS, and CBF-QP approaches in static obstacle avoidance at a controlled update frequency of 5Hz. The characteristics preferred by an ideal obstacle avoidance trajectory should contain small trajectory length $l$ ( or equivalently small trajectory length ratio $l/l_\text{nom}$), small average jerk $\bar{j}$, large obstacle clearance $d_\text{obs}$, small near obstacle velocities $v_\text{near}$, small straight line deviation $\eta$, low execution time and high success rate in converging to the goal location. Here, the methods that perform worst in a behavior evaluation category are colored red and those that perform best are colored green. For metrics computed only on a single trajectory because the rest failed in converging to the goal, their standard deviations are marked as "NA" (not applicable). }
\label{table:static table}
\end{center}
\end{minipage}
\vspace{-10pt}
\end{table*}

\subsubsection*{Behavior Analysis} Each method listed in Table \ref{table:static table} was implemented in Python and evaluated on 10 trajectories starting respectively at points $[1,7]^\top$, $[7,1]^\top$, $[2,6]^\top$, $[6,2]^\top$, $[4,8]^\top$, $[8,4]^\top$, $[5,7]^\top$, $[7,5]^\top$, $[5.6,5.6]^\top$, $[6,6]^\top$. The resulted 10 trajectories from each method can be found in \autoref{fig:comparison data}. Note that behavior metrics of length ratio $l/l_\text{nom}$, obstacle clearance $d_\text{obs}$ and near obstacle velocity $v_\text{near}$ in \autoref{table:static table} are computed only using trajectories that have successfully reached the target. The behavior metrics of weighted jerk $\bar{j}$ and straight-line deviation $\eta$ are computed using trajectories initiated at $[4,8]^\top$ and $[7,5]^\top$ (if succeeded in arriving at the target) because comparing $\bar{j}$ and $\eta$ across different trajectories results in unreasonably high standard deviation values. The 2 selected trajectories, interacting with regions closer to the obstacle than the rest, are good representations of each method's characteristics. 

As reactive safe control methods, both Mod-DS and CBF-QP have quick runtime, making them useful in highly dynamic environments. Mod-DS approaches' strength is their high flexibility and adaptation to all obstacle geometries using different eigenvector and eigenvalue combinations. That ability enables Mod-DS variants like on-manifold Mod-DS to realize local minimum free obstacle avoidance, proven by its 100\% success rate in converging to the target in \autoref{table:static table}. Additionally, on-manifold Mod-DS has the highest obstacle clearance $d_\text{obs}$ among all, indicating higher safety guarantees with less chance of collisions. on-manifold Mod-DS's drawback is slower runtime (though still fast enough to be reactive), the high demand for jerk $j$ when reshaping the flow, longer trajectories $l$, and larger straight-line deviation $\eta$. On the contrary, CBF-QP methods are capable of generating smooth and efficient trajectories with low $j$ and low $v_\text{near}$ and low length ratios $l/l_\text{nom}$, while their weaknesses are small target reaching rate, which reveals the existence of numerous local minima in the CBF-QP modified trajectories. Normal Mod-DS's performance metrics are similar to that of the CBF-QP, except that the method's popular choice of eigenvalues $\lambda_e >1$ (see \autoref{table:modulation types}) results in higher robot velocities in the direction tangent to the boundary set $\partial C_o$ and a larger chance for the robot to move away from local minima on concave obstacles' boundaries. Therefore, normal Mod-DS metrics have higher near-obstacle velocities $v_\text{near}$ and slightly higher convergence rate than CBF-QPs. Lastly, Reference Mod-DS's behaviors are similar to normal Mod-DS in length ratio $l/l_\text{nom}$, obstacle clearance $d_\text{obs}$ and near-obstacle velocity $v_\text{near}$. But the replacement of normal vector $n$ with reference vector $r$ gives reference Mod-DS a better convergence rate for non-convex obstacle avoidance. 

Based on the above analysis, conclusions can be drawn that extended Mod-DS approaches of reference and on-manifold Mod-DSs are preferred to CBF-QP for obstacle avoidance in environments with concave obstacles using fully actuated systems. While CBF-QP-generated trajectories are more efficient, this efficiency has been shadowed by its low target-reaching rate. 


\section{Theoretical Connections behind Mod-DS and CBF-QP in Fully-Actuated Systems}
\label{sec:theoretical-analysis}
Qualitative and quantitative analysis of Mod-DS and CBF-QP approaches for fully actuated systems reveals some common features shared by the two seemingly distinctive approaches:   given the same input state $x$, static obstacle environment $h(x)$ and nominal controller $u_\text{nom}$, normal Mod-DS shares the same local minimum set with CBF-QP. In this section, we demonstrate with mathematical derivations the theoretical connections responsible for similar behaviors in Mod-DS and CBF-QP performances. We prove that normal Mod-DS is equivalent to CBF-QP under certain choices of $\lambda$ and $\lambda_e$ in \eqref{eq:modulation-matrix} and \eqref{eq:standard mod lambda}. We also show that solutions from reference Mod-DS are guaranteed to satisfy CBF conditions in \eqref{eq:CBF conditions} and the relationship between reference Mod-DS and CBF-QP can be mathematically quantified to convert one from another. This conversion formula is important because it shows the key closed-form components in reference Mod-DS solutions that make possible navigation in star-shaped obstacle environments with few local minima. In \autoref{sec:mod-cbf}, through inverse engineering from the conversion formula, modifications of CBF-QP that enable concave unsafe set navigation with fewer and even no local minima are proposed. Note that in this section, our theoretical analysis focuses on static single-obstacle environment settings. Hence, $h(x)$, $n(x)$, and $r(x)$ are used instead of $h_o(x,x_o)$, $n(x,x_o)$, and $r(x,x_o)$ to represent boundary functions, normal vectors and reference vectors. 

\subsection{Local Minima in Mod-DS and CBF-QP}
Local minima in Mod-DS are originated from inverse collinearity between the fully actuated nominal controller $\dot{x}_\text{nom} = u_\text{nom}$ and the gradient of the boundary function $\nabla_xh(x)$ (a.k.a the barrier function in CBF-QP formulations). In normal and reference Mod-DS, the modified controllers fail to converge to the target $x^*$ when the nominal controller is inversely collinear with the normal direction $n(x)$ and the reference direction $r(x)$ respectively [Definition \ref{def:saddle standard modDS}, \ref{def:saddle reference modDS}]. 

\begin{definition}[Saddle Equilibria in Normal Mod-DS] When normal Mod-DS constantly encounters inverse collinearity between $\dot{x}_{\text{nom}}$ and $n(x)$; i.e., $\langle 
 \frac{\dot{x}_{\text{nom}}}{||\dot{x}_{\text{nom}}||}, n(x) \rangle = -1$, the subsequent solutions of Eq.~\eqref{eq:ds-modulation},~\eqref{eq:modulation-matrix},~\eqref{eq:standard mod d} will lead the agent to an undesirable equilibrium, $\dot{x}=u_{\text{mod}}\rightarrow\textbf{0}$, on the boundary of the obstacle $\partial C$ \cite{khansari2012dynamical}.
 % \textbf{Proof:}  From Theorem 2 in \cite{LukesDS}. $\hfill \blacksquare$
 \label{def:saddle standard modDS}
\end{definition}

\begin{definition}[Saddle Equilibria in Reference Mod-DS] When reference Mod-DS constantly encounters inverse collinearity between $\dot{x}_{\text{nom}}$ and $r(x)$; i.e., $\langle 
 \frac{\dot{x}_{\text{nom}}}{||\dot{x}_{\text{nom}}||}, r(x) \rangle = -1$, the subsequent solutions of Eq.~\eqref{eq:ds-modulation},~\eqref{eq:modulation-matrix},~\eqref{eq:reference mod d} will lead the agent to an undesirable equilibrium, $\dot{x}=u_{\text{mod}}\rightarrow\textbf{0}$, on the boundary of the obstacle $\partial C$ \cite{LukesDS}.
 \label{def:saddle reference modDS}
\end{definition}

Interestingly, very few works have discussed undesirable local minimum (or undesirable equilibrium) issues in CBF-QPs. In \cite{notomista2021safety}, this limitation of CBF-QPs was identified. However, they did not offer rigorous proof for the existence of such saddle points, only a solution to alleviate them. \cite{clfcbfEquilibria} explored the issue for Control Lyapunov Function and Control Barrier Function-based Quadratic Programs (CLF-CBF-QPs), where they showed that even for convex obstacle shapes undesirable equilibria can be introduced on the boundaries due to the contradictions between CLF and CBF constraints. Following, we offer formal proof showing that, in fully actuated systems, undesirable equilibria in CBF-QP can also originate from inverse collinearity, the same as that in normal Mod-DS.

\begin{theorem}[Saddle Equilibria in CBF-QP]
\label{theorem:saddle-cbfqp} 
When a fully-actuated CBF-QP controller constantly encounters inverse collinearity between $\nabla_x h(x)$ and $u_{\text{nom}}$; i.e., $\langle 
 n(x), \frac{u_{\text{nom}}}{||u_{\text{nom}}||} \rangle = -1$, the subsequent solutions of Eq.~\eqref{eq:cbf-qp fully actuated}, or equivalently Eq.~\eqref{eq: explicit cbf fully actuated}, will lead the agent to an undesirable equilibrium, $\dot{x}=u_{\text{mod}}\rightarrow\textbf{0}$, at the boundary of the obstacle $\partial C$.
\end{theorem}
% \subsection{Proof of Saddle Equilibrium in CBF-QP}\label{section: saddle cbf proof}
% Following we prove that in a static single-obstacle environment, when a CBF-QP controller encounters inverse collinearity between $\nabla_x h(x)$ and $\dot{x}_{\text{nom}}$, the subsequent solutions of Eq. \eqref{cbf-qp} will lead the agent to an undesirable asymptotic equilibrium at the boundary of the obstacle given $\dot{x}_{\text{mod}} = u_{\text{mod}}$ and $\dot{x}_{\text{nom}} = u_{\text{nom}}$. 
\textbf{Proof:} Since $\nabla_x h(x)$ and $u_{\text{nom}}$ are inversely collinear with each other, there exist a constant scalar $c\in \mathbb{R}^-$ s.t. $u_{\text{nom}} = c\nabla_x h(x)$. Denote $\nabla_xh = \nabla_xh(x)$ and $\alpha = \alpha(h(x))$. \autoref{eq: explicit cbf fully actuated} can then be rewritten as
% \begin{gather}
%     \because u_{\text{nom}} = c\nabla_x h(x,t)
% \end{gather}
% \begin{equation}
%     \begin{aligned}
%         \therefore \nabla_x h(x,t)^Tu_\text{nom} &= c\nabla_x h(x,t)^T\nabla_x h(x,t) \\
%         &= c||\nabla_x h(x,t)||_2^2 \\
%         % &<0 \quad \forall x, t
%     \end{aligned}
% \end{equation}
% Therefore, the explicit solution in \eqref{eq: explicit cbf fully actuated} can be simplified as the following.  
\begin{equation}
\begin{aligned}
 u_\text{cbf} &= \begin{cases}
c\nabla_xh \quad \text{if} \quad ||\nabla_xh||_2^2\leq -\frac{\alpha}{c}\\
c\nabla_xh- \frac{c\nabla_xh^\top\nabla_xh+\alpha}{\nabla_xh^\top\nabla_xh}\nabla_xh\quad \text{otherwise}
\end{cases}\\
% &= \begin{cases}
% c\nabla_xh \quad \text{if} \quad  ||\nabla_xh||_2^2\leq -\frac{\alpha}{c}\\
% (c- \frac{c||\nabla_xh||_2^2+\alpha}{||\nabla_xh||_2^2})\nabla_xh\quad \text{otherwise}
% \end{cases}\\
&= \begin{cases}
c\nabla_xh \quad \text{if} \quad  ||\nabla_xh||_2^2\leq -\frac{\alpha}{c}\\
-\frac{\alpha}{||\nabla_xh||_2^2}\nabla_xh \quad \text{otherwise}
\end{cases}.
\end{aligned}
\end{equation}

Given any initial state $x^0 \in C$, if $||\nabla_xh(x^0)||_2^2\leq-\frac{\alpha(h(x^0))}{c}$, $u_\text{cbf} = c\nabla_xh(x^0)$. Since $c<0$, this command moves the agent in the direction opposite to the gradient of the boundary function $h(x^i)$ and makes the value of $h(x^i)$ decrease in the subsequent timesteps $i\in\{1,2,..\}$. If $\nabla_xh(x^i)$ stays constantly inversely collinear with $u_\text{nom}$, $\alpha(h(x^i))$ value would decrease until  $||\nabla_xh(x^i)||_2^2>-\frac{\alpha(h(x^i))}{c}$. When $u_\text{cbf} = -\frac{\alpha(h(x^i))}{||\nabla_xh(x^i)||_2^2}\nabla_xh(x^i) $, the agent is still commanded to move in the direction opposite to the gradient of the boundary function $h(x^i)$, despite with a slower speed. Similarly, $\alpha(x^i)$ value will keep decreasing until it reaches 0 at the boundary set $\partial C$ if inverse collinearity persists. 
% \begin{equation}
%     x^t = x^{t-1} + u_\text{cbf}\Delta t
% \end{equation}
\begin{equation}
    \lim_{\alpha \rightarrow 0} u_\text{cbf} = \lim_{\alpha \rightarrow 0} -\frac{\alpha}{||\nabla_xh||_2^2}\nabla_xh = \textbf{0}
\end{equation}

Therefore, when CBF-QP with fully-actuated systems encounters inverse collinearity constantly between $u_{\text{nom}}$ and $\nabla_xh(x)$, the subsequent solutions lead the agent to an undesirable equilibrium. $\hfill \blacksquare$
% As mentioned before, methods have been proposed for removing undesirable equilibria, such as implementing a state-avoidance feedabck controller in the ball world projection of the state space \cite{notomista2021safety}. In fact, >> Already mentioned

\subsection{Normal Mod-DS and CBF-QP Equivalance}
The similarity between CBF-QP and Mod-DS in terms of undesirable equilibria originated from the fact that the normal Mod-DS approach is equivalent to CBF-QP.

\begin{theorem}[Normal Mod-DS and CBF-QP Equivalance]
\label{theorem: equivalance}
When the unsafe set is static, given fully-actuated CBF-QP controllers with any choice of extended $K_{\infty}$ function $\alpha$, there exists a pair of $\lambda$ and $\lambda_e$ such that normal Mod-DS defined in Eq. \eqref{eq:ds-modulation}, \eqref{eq:standard mod lambda} and \eqref{eq:standard mod d} perform equivalently.
\end{theorem}
\textbf{Proof:} In static environments, orthonormal basis $E(x) = [\frac{\nabla_xh(x)}{||\nabla_xh(x)||_2}, H(x)]$ in \eqref{eq:modulation-matrix} defines an obstacle-based coordinate system different from the world frame. To simplify the derivation expressions, we again abuse notations and let $\alpha =  \alpha(h(x))$, $\nabla_xh = \nabla_xh(x)$, $H = H(x)$, $E = E(x)$ and $M = M(x)$. Since $E$ is orthonormal, $E^{-1} = E^\top = [\frac{\nabla_xh}{||\nabla_xh||_2}^\top; H^\top]$. $H^\top\nabla_xh = \textbf{0}_{(d-1) \times 1}$. Projecting the solution in \eqref{eq: explicit cbf fully actuated} from the world frame to the obstacle-based coordinate frame defined by $E$, we get the following expressions.
\begin{equation}
\begin{aligned}
&\quad \; E^{-1}u_\text{cbf} = E^\top u_\text{cbf}
=\begin{bmatrix} \frac{\nabla_xh^\top}{||\nabla_xh||_2}\\ H^\top\end{bmatrix}u_\text{cbf}\\
% &=E^\top\begin{cases}
% u_\text{nom} \quad \text{if} \quad \nabla_xh^\top u_\text{nom}\geq -\alpha'\\
% u_\text{nom}- \frac{\nabla_xh^\top u_\text{nom}+\alpha'}{\nabla_xh^\top\nabla_xh}\nabla_xh\quad \text{otherwise}
% \end{cases}\\
% &=\begin{cases}
% \begin{bmatrix} \frac{\nabla_xh^T}{||\nabla_xh||_2}\\ H^T\end{bmatrix}u_\text{nom} \quad \text{if} \quad \nabla_xh^Tu_\text{nom}\geq -\alpha'\\
% \begin{bmatrix}\frac{\nabla_xh^T}{||\nabla_xh||_2}\\ H^T\end{bmatrix}u_\text{nom}- \begin{bmatrix}\frac{\nabla_xh^T}{||\nabla_xh||_2}\\ H^T\end{bmatrix}\frac{\nabla_xh^Tu_\text{nom}+\alpha'}{\nabla_xh^T\nabla_xh}\nabla_xh\quad \text{otherwise}
% \end{cases}\\
&=\begin{cases}
\begin{bmatrix} \frac{\nabla_xh^\top}{||\nabla_xh||_2}\\ H^\top\end{bmatrix}u_\text{nom} \quad \text{if} \quad \nabla_xh^\top u_\text{nom}\geq -\alpha\\
\begin{bmatrix}\frac{\nabla_xh^\top}{||\nabla_xh||_2}\\ H^\top\end{bmatrix}u_\text{nom}- \begin{bmatrix}\frac{\nabla_xh^\top u_\text{nom}+\alpha}{||\nabla_xh||_2}\\ \textbf{0}_{(d-1) \times d}\end{bmatrix}\quad \text{otherwise}
\end{cases}\\
&=\begin{cases}
\begin{bmatrix} \frac{\nabla_xh^\top}{||\nabla_xh||_2}\\ H^\top \end{bmatrix}u_\text{nom} \quad \text{if} \quad \nabla_xh^\top u_\text{nom}\geq -\alpha\\
\begin{bmatrix}\frac{-\alpha}{||\nabla_xh||_2}\\ H^\top u_\text{nom}\end{bmatrix}\quad \text{otherwise}
\end{cases}
\end{aligned}
\end{equation}


% \begin{equation}
% \begin{aligned}
% &=\begin{cases}
% \begin{bmatrix} \frac{\nabla_xh^T}{||\nabla_xh||_2}u_\text{nom}\\ H^Tu_\text{nom}\end{bmatrix} \quad \text{if} \quad \nabla_xh^Tu_\text{nom}\geq -\alpha'\\
% \begin{bmatrix}\frac{-\alpha-\frac{\partial h}{\partial t}}{||\nabla_xh||_2}\\ H^Tu_\text{nom}\end{bmatrix}\quad \text{otherwise}
% \end{cases}\\
% \end{aligned}
% \end{equation}

% In modulation, matrix $M(x,t)$ ensures controller safety by first projecting $u_{nom}$ into the coordinate system defined by $E$ and modifies the magnitude of the speed projected onto $\nabla_xh$ direction and $H$ hyperplane using $\lambda$ and $\lambda_e$ respectively. Denote $D=D(x,t)$ and $M=M(x,t)$. 
Similarly, projecting the solution from Eq.~\eqref{eq:ds-modulation},~\eqref{eq:modulation-matrix},~\eqref{eq:standard mod d} into the coordinate system defined by $E$, the following relationship can be acquired bewteen $\lambda$, $\lambda_e$ value and $E^{-1}u_\text{mod}$. Note that $\dot{x}_o = 0$ when the obstacle is static.
\begin{equation}
\label{eq:mod proj}
\begin{aligned}
&\quad \; E^{-1}u_\text{mod} =E^{-1}Mu_\text{nom}=DE^\top u_\text{nom}\\
&=\begin{bmatrix}
     \lambda \frac{\nabla_xh^\top}{||\nabla_xh||_2}\\ \lambda_e H^\top
 \end{bmatrix}u_\text{nom}
% =\begin{bmatrix}
%      \frac{\nabla_xh^\top}{||\nabla_xh||_2}\lambda  u_\text{nom} \\ H^\top\lambda_e u_\text{nom}
%  \end{bmatrix}
\end{aligned}
\end{equation}



% \begin{equation}
% \label{eq:mod proj}
% \begin{aligned}
%  &\quad \;   E^{-1}u_\text{mod} \\
%  &=E^{-1}M(u_\text{nom}-\dot{x}_o)+E^{-1}\dot{x}_o\\
%  &= E^{-1}EDE^{-1}(u_\text{nom}-\dot{x}_o)+E^{-1}\dot{x}_o\\
%  % &=DE^T(u_\text{nom}-\dot{x}_o)+E^T\dot{x}_o\\
%  &=DE^Tu_\text{nom}+(I-D)E^T\dot{x}_o\\
% \end{aligned}
% \end{equation}
% \begin{equation}
% \begin{aligned}
%  &=\begin{bmatrix}
%      \lambda \frac{\nabla_xh^T}{||\nabla_xh||_2}\\ \lambda_e H^T
%  \end{bmatrix}u_\text{nom}+
%  \begin{bmatrix}
%     (1- \lambda) \frac{\nabla_xh^T}{||\nabla_xh||_2}\\ (1-\lambda_e) H^T
%  \end{bmatrix}\dot{x}_o\\
%  &=\begin{bmatrix}
%      \frac{\nabla_xh^T}{||\nabla_xh||_2}[\lambda  u_\text{nom} +(1-\lambda)\dot{x}_o]\\ H^T[\lambda_e u_\text{nom}+(1-\lambda_e)\dot{x}_o]
%  \end{bmatrix}
% \end{aligned}
% \end{equation}

Select $\lambda$ and $\lambda_e$ as shown in \eqref{eq:lambda value for equivalence}. Since $\alpha=\alpha(h(x)) \rightarrow 0$ as $h(x) \rightarrow 0$ due to the property of $K_{\infty}$ functions. Therefore, $\lambda$ and $\lambda_e$ trivially satisfy the requirement in \eqref{eq:standard mod lambda}.
\begin{equation}
\label{eq:lambda value for equivalence}
\lambda = \begin{cases}
    1 \quad \text{if} \quad \nabla_xh^\top u_\text{nom}\geq -\alpha\\
    -\frac{\alpha}{\nabla_xh^\top u_\text{nom}} \quad \text{otherwise}
\end{cases}
\quad \lambda_e = 1
\end{equation}

Plug \eqref{eq:lambda value for equivalence} into \eqref{eq:mod proj}, we get
\begin{equation}
\begin{aligned}
 E^{-1}u_\text{mod} 
&=\begin{cases}
\begin{bmatrix} \frac{\nabla_xh^\top}{||\nabla_xh||_2}\\ H^\top\end{bmatrix}u_\text{nom} \quad \text{if} \quad \nabla_xh^\top u_\text{nom}\geq -\alpha\\
\begin{bmatrix}\frac{-\alpha}{||\nabla_xh||_2}\\ H^Tu_\text{nom}\end{bmatrix}\quad \text{otherwise}
\end{cases}\\
& = E^{-1}u_\text{cbf}.
\end{aligned}
\end{equation}

% \begin{equation}
% \begin{aligned}
%  &\quad \;   E^{-1}u_\text{mod} 
% &=\begin{cases}
% \begin{bmatrix} \frac{\nabla_xh^\top}{||\nabla_xh||_2}\\ H^\top\end{bmatrix}u_\text{nom} \quad \text{if} \quad \nabla_xh^\top u_\text{nom}\geq -\alpha'\\
% \begin{bmatrix}\frac{-\alpha+(1+\frac{\alpha}{\nabla_xh^Tu_\text{nom}})\dot{x}_o}{||\nabla_xh||_2}\\ H^Tu_\text{nom}\end{bmatrix}\quad \text{otherwise}
% \end{cases}\\
% \end{aligned}
% \end{equation}

% When the unsafe sets are static, $\dot{x}_o =\frac{\partial h}{\partial t}=0 $, thus $E^{-1}u_\text{cbf}=E^{-1}u_\text{mod}$. 
If a a pair of vectors is equivalent when measured in one coordinate system, they must also be equivalent in all coordinate systems. Therefore, the conclusion can be drawn that $u_\text{cbf}=u_\text{mod}$ when $\lambda$ and $\lambda_e$ are defined as in \eqref{eq:lambda value for equivalence}.
\hfill $\blacksquare$

To further validate our theorem, given nominal DS defined in \eqref{eq:nominal system}, the modified DSs produced respectively by normal Mod-DS, satisfying \eqref{eq:lambda value for equivalence} and by CBF-QP with $\alpha(h)=h$ are presented in \autoref{fig:mod-cbf validation}. The magnitude (represented by the color of the streamlines) and direction (represented by streamline directions) of normal Mod-DS modified DS in \autoref{fig:mod-cbf validation} are exactly the same as those in the corresponding CBF-QP modified DS in \autoref{fig:comparison} 1(b) and 1(c). 
\begin{figure}
    \centering
    \includegraphics[width=\linewidth]{images/mod_cbf/ModDS_streamplot_star_single_x_both.pdf}
    \caption{Performance of the normal Mod-DS given $\lambda$, $\lambda_e$ in \eqref{eq:lambda value for equivalence} is equivalent to that of CBF-QP with $\alpha(h)=h$ in \autoref{fig:comparison}.}
    \label{fig:mod-cbf validation}
\end{figure}


% Prior to this, we show that Mod-DS is also set invariant with a trivial proof showing the connection between the von Neumann boundary condition of \textit{impenetrability} and set invariance. 


% \begin{theorem}[Set Invariance of Mod-DS]
% Mod-DS defined in Eq. \eqref{ds-modulation} satisfies Nagumo's Theorem on set invariance.
% \end{theorem}
% \textbf{Proof:} To prove set invariance of Mod-DS in Eq. \eqref{ds-modulation} we first state the von Neumann boundary condition on impenetrability. 
% \begin{definition}[von Neumann Boundary Condition]
% Impenetrability is ensured if the normal velocity of the system at boundary points vanishes; 
% \begin{equation}
% \label{von Neuman}
% n(x)\dot{x}=0 \quad\; \forall x \in \partial C,
% \end{equation}
% with $n(x)$ being the normal unit vector to the boundary.
% \end{definition}
% In \cite{khansari2012dynamical,LukesDS,billard2022learning, onManifoldMod} it has been extensively proven that all Mod-DS approaches hold this property, as $\nabla_xh(x)$ represents geometrically the direction normal to the obstacle surface from point $x$, thus  $\nabla_xh(x) = n(x)$. Since $\dot{x}$ is the velocity of the controlled agent, $\nabla_xh(x)\dot{x}$ is equivalent to the agent's velocity in the direction normal to the obstacle surface. Since $\dot{h}(x) = \nabla_xh(x)\dot{x} =n\dot{x}$, we show that Mod-DS satisfies
% \begin{equation}
% \dot{h}(x)=n(x)\dot{x}=0 \quad\; \forall x \in \partial C
% \end{equation} and meets Nagumo's set invariance constraint in Eq. \eqref{Nagumo}. \hfill $\blacksquare$

%%%%%%%%%%%%%%%%%%%%%%%%%%%%%%%%%%%%%%%%%%%%%%%%%%%%%%%%%%%%%%%%%%%%%%%%%%%%%%%%%%%%%%%%%%%%%%%%%%%%%%%%%%%%%%%%%%%%%%%%%%%%%%%%%%%%

\subsection{Reference Mod-DS and CBF-QP Similarities}
\label{sec: reference mod-ds vs cbf}
In \autoref{sec:qualitative-geometry}, the performance of the reference Mod-DS was demonstrated to be superior to that of the CBF-QP. Still, the fact that reference Mod-DS shared the same undesirable local minimum with CBF-QPs for circular obstacle avoidance in \autoref{fig:comparison data} suggests the potential for hidden connections between the 2 seemingly distinctive approaches. Quantifying the similarities and differences between reference Mod-DS and CBF-QP is key to understanding how undesirable local minima can be reduced or even eliminated to realize global convergence. We begin our discussion here by proving $u_\text{mod}$ from reference Mod-DS satisfies the CBF constraints at all times in static-single-obstacle avoidance and then quantify what changes need to be introduced before reference Mod-DS and CBF-QP can be converted to each other. The theoretical connections discussed here will contribute to the design of a new safe control approach that incorporates the advantage of reference Mod-DS and CBF-QP in \autoref{sec:mod-cbf}. 

% \begin{figure}
%     \centering
%     \includegraphics[width=0.8\linewidth]{images/n and r sssuper.png}
%     \caption{Geometric interpretation and relationship between $n(x)$, $e(x)$ and $r(x)$ used for \autoref{theroem:set invariance in reference}.}
%     % \caption{The relationship between the direction normal $n(x)$ and tangent $e(x)$ to the obstacle surface, and the reference direction $r(x)$ pointing from the chosen reference point inside the obstacle to the agent's state $x$ at the current time step. $\theta$ is }
%     \label{fig:n and r}
% \end{figure}


% \hspace{5pt}\begin{minipage}{0.2975\textwidth}
%     \centering
%     \includegraphics[width=0.925\linewidth]{images//modulation/old_constrained_Modu.png}
%     \caption{Speed-constrained (5 units/s) Mod-DS from \cite{constrainedModu}. Notice the excessive  90 degree sharp turns.}
%     % \caption{Trajectories generated using speed-constraining Mod-DS in \cite{constrainedModu} given nominal dynamics in Eq.~\eqref{nominal system} under speed limit of 5 unit/s. The constrained trajectories contains 90 degree sharp turns and simply circumvents the circular obstacle with no effort of preserving nominal trajectories. }
%     \label{fig:old constrained modu}
% \end{minipage}\hspace{5pt}
\begin{theorem}[Safe Set Invariance in Reference Mod-DS]
\label{theroem:set invariance in reference}
Given fully-actuated nominal controller $u_\text{nom}$ in \eqref{eq:fully actuated system} and any $\lambda$ and $\lambda_e$ satisfying \eqref{eq:standard mod lambda}, there exists an extended class $K_{\infty}$ function such that the Mod-DS safe controller $u_\text{mod}$ defined in Eq.~\eqref{eq:ds-modulation},~\eqref{eq:modulation-matrix}, and \eqref{eq:reference mod d} meets the CBF conditions defined in Eq.~\eqref{eq:cbf-qp fully actuated} when $\dot{x}_o = 0$. 
\end{theorem}

% \textbf{Proof:} Since we assume that the unsafe sets are translating and rotating rather than deforming, there must exist a point $x_L^o$ somewhere such that the motion of an unsafe set can be characterized about as in Eq.~\eqref{eq:x o def}. Thus, $\nabla_{x_o}h(x, x_o)^\top\dot{x}_o$ can be rewritten as in \eqref{eq:rel h wrt t}.

% \begin{equation}
% \begin{aligned}
% &\quad \;\nabla_{x_o}h(x, x_o)^\top\dot{x}_o \\
% &=\nabla_{x_L^o}h(x, x_o)^\top\dot{x}_L^o + \nabla_{x_R^o}h(x, x_o)^\top\dot{x}_R^o\\
% & = -\nabla_xh(x, x_o)^\top\dot{x}_L^o + \nabla_{x_R^o}R(x_R^o)^\top(x-x_L^o)\dot{x}_R^o.
% \end{aligned}
% \label{eq:rel h wrt t}
% \end{equation}

% Plug \eqref{eq:rel h wrt t} into the CBF constraint in \eqref{eq:cbf-qp fully actuated}, CBF conditions in fully actuated systems are equivalent to the following. 
In static-single obstacle avoidance scenarios, annotations can be simplified such that $h = h(x,x_o)$ and $\nabla_xh = \nabla_xh(x,x_o)$. Given $\dot{x}_o = \textbf{0}$ and $\bar{\dot{x}}_o = \textbf{0}_{d\times 1}$, CBF conditions in Eq.~\eqref{eq:cbf-qp fully actuated} can be rewritten as:
% Denote $\alpha = \frac{\alpha (h(x,t))}{{||\nabla_xh(x,t)||_2}}$. Note that since 
\begin{equation}
\begin{aligned}
\label{eq:cbf fully actuated constr}
\frac{\nabla_xh^\top}{||\nabla_xh||_2}u \geq -\frac{\alpha (h)}{{||\nabla_xh||_2}}.
% n(x,t)^Tu_ \geq n(x,t)^T\dot{x}_o-\frac{\alpha (h(x,t))}{{||\nabla_xh(x,t)||_2}}\\
\end{aligned}
\end{equation}

Following the same notation as in \autoref{theorem: equivalance}, $\forall x \in \mathbb{R}^d$, denote $r = r(x,x_o)$, $n = n(x, x_o)$, $e_i = e_i(x,x_o), \forall i\in \{1,2,...,d-1\}$ and $H = [e_1(x, x_o), ...e_{d-1}(x, x_o)]\in \mathbb{R}^{d \times (d-1)}$ to be the hyperplane plane orthogonal to the gradient of the boundary function $\nabla_xh$. Let the basis matrix of reference Mod-DS $E_r = [r(x, x_o), H(x, x_o)] \in \mathbb{R}^{d \times d}$, as defined in \eqref{eq:modulation-matrix} and \eqref{eq:reference mod d}. Note that $||r(x, x_o)||_2=1$ by definition. Since $E_r(x,x_o)$ is not an orthonormal basis, $E_r^{-1}\neq E_r^\top$. Instead,
\begin{equation}
\label{eq: Er inv}
\begin{aligned}
E_r^{-1} &= \begin{bmatrix}
        \frac{1}{\nabla_xh^\top r}\nabla_xh^\top\\
        -\frac{1}{\nabla_xh^\top r}e_1^\top r\nabla_xh^\top + e_1^\top\\
         \vdots \\
        -\frac{1}{\nabla_xh^\top r}e_{d-1}^\top r\nabla_xh^\top + e_{d-1}^\top
    \end{bmatrix}\\
    &= \begin{bmatrix}
        \frac{1}{\nabla_xh^\top r}\nabla_xh^\top\\
        H^\top (I - \frac{1}{\nabla_xh^\top r}r\nabla_xh^\top)
    \end{bmatrix}.
\end{aligned}
\end{equation}

Therefore, the decomposition of $u_\text{mod}$ onto $r$ by basis matrix $E_r$ when $\bar{\dot{x}}_o = 0$ can be computed from Eq.~\eqref{eq:ds-modulation}, \eqref{eq:modulation-matrix} as
\begin{equation}
\begin{aligned}
 u_\text{mod}^r=\lambda E_r^{-1}[1,:]u_\text{nom}=\frac{\nabla_xh^\top}{\nabla_xh^\top r}\lambda u_\text{nom}.\\
\end{aligned}
\end{equation}

The decomposition of $u_\text{mod}$ onto $r$ can be further broken down into the direction $n$ parallel to $\nabla_xh$ and into the tangent hyperplane $H$ orthogonal to $\nabla_xh$, as depicted in \autoref{fig:n and r}. The projection of $u_\text{mod}^r$ onto $n$ is equivalent to $u_\text{mod}^n$, the projection of $u_\text{mod}$ onto $n$, because the vector $H$ is orthogonal to $n$, i.e. $n^\top H = \textbf{0}_{1\times d-1}$. 
\begin{equation}
\label{eq:mod r proj n}
\begin{aligned}
u_\text{mod}^n &= n^\top(u_\text{mod}^r r+u_\text{mod}^H e)=u_\text{mod}^r n^\top r\\
&=(\frac{\nabla_xh^\top}{\nabla_xh^\top r}\lambda u_\text{nom})(\frac{\nabla_xh}{||\nabla_xh||_2})^\top r\\
& = \lambda u_\text{nom}\frac{\nabla_xh^\top}{||\nabla_xh||_2}
\end{aligned}
\end{equation}

% From the relationship between vector $n(x,t)$ and $r(x,t)$ depicted in \autoref{fig:n and r}, it can be deduced that
% \begin{equation}
% \label{eq:n dot r}
%     \begin{split}
%     \langle \hat{n}, \hat{r}\rangle =\frac{\nabla_xh^\top}{||\nabla_xh||_2} r= \cos\theta.
%     \end{split}
% \end{equation} 
% Since $-\pi/2 \leq \theta \leq \pi/2$ under all conditions because the reference direction $r(x,t)$ and the normal direction $n(x,t)$ must both point away from the obstacle, $\cos\theta \geq 0$ always holds. Therefore, plugging \eqref{eq:n dot r} into \eqref{eq:mod r proj n}, 
% \begin{equation}
% \begin{aligned}
% \label{eq: u_mod_n}
% \quad \; u_\text{mod}^n
% &=\cos{\theta}\{\frac{\nabla_xh^\top}{\nabla_xh^\top r}[\lambda u_\text{nom}+(1-\lambda)\dot{x}_o]\}\\
% & = \frac{\nabla_xh^\top}{||\nabla_xh||_2}[\lambda u_\text{nom}+(1-\lambda)\dot{x}_o].
% \end{aligned}
% \end{equation}

 Now we are left with showing the modulated value in \eqref{eq:mod r proj n} satisfies the CBF conditions. \textit{WTS:} Given any $\nabla_xh, h, u_\text{nom}$ and $\lambda$ combinations, there exists a $K_\infty$ function $\alpha$ such that the following inequility, acquired by substituting $\frac{\nabla_xh^\top}{||\nabla_xh||_2}u = u_\text{mod}^n$ and $u_\text{mod}^n$ expression from \eqref{eq:mod r proj n} into \eqref{eq:cbf fully actuated constr} and simplify, always holds. 
\begin{equation}
\begin{aligned}
 % \frac{\nabla_xh^\top}{||\nabla_xh||_2}[\lambda u_\text{nom}+(1-\lambda)\dot{x}_o]\geq 
 % \frac{\nabla_xh^\top}{||\nabla_xh||_2}\dot{x}_o-\frac{\alpha (h)}{{||\nabla_xh||_2}}\\
 -\lambda \nabla_xh^\top u_\text{nom} \leq  \alpha(h)
\end{aligned}
\end{equation}

Since the norms of $\nabla_xh, u_\text{nom}$ and $\lambda$ values are all well-bounded by definition, find $\alpha$ that serve as an upper bound for domain $\{x\in \mathbb{R}^d: h(x,x_o)>0\}$ is trivial. When $h(x,x_o)=0$, $\lambda=0$ based on \eqref{eq:standard mod lambda}, the left and right sides of the inequality become 0 and satisfy the property of $K_\infty$ functions. In conclusion, given a bounded fully-actuated nominal controller $u_\text{nom}$ in \eqref{eq:fully actuated system} and any feasible $\lambda$ and $\lambda_e$, there exists an extended class $K_{\infty}$ function such that the Mod-DS safety controller $u_\text{mod}$ meets the CBF conditions.

% When computing the closed-form solutions of Mod-DS, the CBF constraints in Eq.~\eqref{modulation cbf} are implicitly enforced, where $\max{|\dot{x}^{\hat{n}}_{\text{\text{nom}}}|}$ is the maximum speed of the controlled agent in the direction of $\nabla h(x)$ in domain $x\in \chi$. Since the changing rate of distance to the obstacle is equivalent to the agent's velocity normal to the obstacle $\dot{h}(x)=\dot{x}^{\hat{n}}_{\text{mod}}=u_{\text{mod}}^{\hat{n}}$, the CBF for the Mod-DS approach can thus be formulated as,
% \begin{equation}
% \label{modulation cbf}
% \begin{split}
%     \dot{h}(x) + \alpha(h(x)) \geq 0\\
%     \alpha(z) = \left(1-\frac{1}{1+z}\right)\max{|\dot{x}^{\hat{n}}_{\text{nom}}|}
% \end{split}
% \end{equation}and can be further simplified to, 
% \begin{equation}
% \label{modulation cbf simplified}
% \begin{split}
%      u_{\text{mod}}^{\hat{n}}\geq -\left(1-\frac{1}{1+h(x)}\right)\max{|\dot{x}^{\hat{n}}_{\text{nom}}|}
% \end{split}
% \end{equation}
% Then from the relationship between the robot's velocities along $n(x)$ and $r(x)$, as shown in Fig.~\ref{fig:n and r}, we observe that, 
% \begin{equation}
%     \begin{split}
%     \frac{u_{\text{mod}}^{\hat{n}}}{u_{\text{mod}}^{\hat{r}}}=\frac{\dot{x}_{\text{nom}}^{\hat{n}}}{\dot{x}_{\text{nom}}^{\hat{r}}} = \cos\theta
%     \end{split}
% \end{equation} Since we know that $-\pi/2 \leq \theta \leq \pi/2$ under all conditions because the reference direction $r(x)$ and the normal direction $n(x)$ must both point away from the obstacle, we can conclude that $cos\theta \geq 0$ always holds and that the CBF constraints for Mod-DS can be written as the following.
% \begin{equation}
% \label{modulation cbf simplified r}
% \begin{split}
%      u_{\text{mod}}^{\hat{r}}\geq -\left(1-\frac{1}{1+h(x)}\right)\max{|\dot{x}^{\hat{r}}_{\text{nom}}|}
% \end{split}
% \end{equation}
% thus, Mod-DS is a special case of CBF-QP. 
\hfill $\blacksquare$

\begin{theorem}[Quantitative Difference between Reference Mod-DS and CBF-QP]
\label{theroem:quantitative diff reference}
Given any CBF-QP with a fully-actuated nominal controller $u_\text{nom}$ in \eqref{eq:fully actuated system}, there exist a pair of $\lambda$, $\lambda_e$ values satisfying \eqref{eq:standard mod lambda} such that the differences between outputs from reference Mod-DS and CBF-QP can be quantified as $(\lambda_e-
\lambda)\sum_{i=1}^{d-1} \frac{e_i^\top r}{n^Tr}e_in^\top u_\text{nom}$. In the expression, we abuse notation and let $r=r(x,t)$ in \eqref{eq:reference mod d}, $e_i = e_i(x,t)$ in \eqref{eq:modulation-matrix} and $n=n(x,t)$ in \eqref{eq:standard mod d}. 
% This quantified difference is responsible for Reference Mod-DS's eliminations of local minima facing star-shaped unset sets. 
\end{theorem}
\textbf{Proof:} Since both normal and reference Mod-DS are reactive safe control approaches with explicit closed-form solutions, given the same $\lambda$ and $\lambda_e$, the differences between reference and normal Mod-DS in regions where both are active can be computed as in \cite{onManifoldMod}. Here to distinguish reference Mod-DS outputs from the normal ones, we denote results from reference Mod-DS as $u_\text{mod}^R$ and that from the normal one as $u_\text{mod}^N$.
\begin{equation}
    u_\text{mod}^R=u_\text{mod}^N-(\lambda_e-
\lambda)\sum_{i=1}^{d-1} \frac{e_i^\top r}{n^Tr}e_in^\top u_\text{nom}
\end{equation}

Because CBF-QP is demonstrated to be equivalent to normal Mod-DS in \autoref{theorem: equivalance}, setting reference Mod-DS to be only active in scenarios where CBF-QP is, i.e. $u_\text{mod}^R=u_\text{nom}$ when $\nabla_xh^\top u_\text{nom}\geq-\alpha$ similar to \eqref{eq: explicit cbf fully actuated}, it can be trivially concluded that, 
\begin{equation}
\label{eq:quantitative diff}
\begin{aligned}
& \quad \;u_\text{mod}^R - u_\text{cbf} \\
&= \begin{cases}
0  \quad \text{if} \quad \nabla_xh^\top u_\text{nom} \geq 0\\
-(\lambda_e-
\lambda)\sum_{i=1}^{d-1} \frac{e_i^\top r}{n^\top r}e_in^\top u_\text{nom} \quad \text{otherwise}
\end{cases}.
\end{aligned} 
\end{equation}\hfill $\blacksquare$

% \begin{equation}
% \label{eq:quantitative diff}
%     u_\text{mod}^r=u_\text{cbf}-(\lambda_e-
% \lambda)\sum_{i=1}^{d-1} \frac{e_i^Tr}{n^Tr}e_in^Tu_\text{nom}
% \end{equation}

Here ends our comparative analysis of the connections and differences between Mod-DS and CBF-QP methods. Given conclusions made in \autoref{sec: quantitative and qualitative}, we propose speed and velocity constraining Mod-DS to boost reference Mod-DS and on-manifold Mod-DS's advantage over CBF-QP in concave-obstacle environments. The new feature will allow Mod-DS to have abilities comparable with CBF-QP in enforcing kinematic constraints in fully actuated systems. 
% Furthermore, the theoretical study on Mod-DS and CBF-QP similarities and conversions in this section set grounds for our proposal of Mod-based CBF-QP methods that combines the strength of both, the details of which can be found in \autoref{sec:mod-cbf}.

\section{Constrained Mod-DS in Fully-Actuated Systems}
\label{sec:constraing-mod}
% \subsection*{Nominal Dynamics} We define our nominal controller $u_{\text{nom}}$ to follow trajectories of an autonomous linear DS, and normalized to the preferred speed $v_d$ of the robot we used.
% \begin{equation}
% \label{omni nominal system}
% \dot{x}_{\text{nom}}=u_{\text{nom}}=-v_\text{d}\frac{Ix}{||Ix||_2} \quad \forall x \in\mathbb{R}^2  
% \end{equation}
In \autoref{sec: quantitative and qualitative}, Mod-DS's strength in navigating around a larger variety of obstacles with fewer undesirable local minima is demonstrated, which makes it a more suitable option for robots with fully actuated systems. However, some may argue that all Mod-DS approaches have limitations in handling input constraints. For closed-form methods in general, implementing velocity constraints poses challenges as modifications to the modulated velocity might compromise the agent's guaranteed impenetrability. A closed-form speed-constraining method was proposed for Mod-DS approaches in \cite{constrainedModu}. However, such methods are conservative, depriving the constrained trajectories of Mod-DS characteristics and generating sharp turns impractical for robots to follow in real life \autoref{fig:old constrained modu}. 

\begin{figure}
    \centering
    \begin{subfigure}{0.46\linewidth}
         \includegraphics[width = \textwidth]{images/modulation/old_constrained_Modu.png}
    \caption{\label{fig:old constrained modu}}
    \end{subfigure}
    \begin{subfigure}{0.52\linewidth}
         \includegraphics[width=\textwidth]{images/n_r_sssuper.png}
    \caption{\label{fig:n and r}}
    \end{subfigure}
    \caption{\autoref{fig:old constrained modu} shows the performance of speed-constraining (5 units/s) reference Mod-DS from \cite{constrainedModu}. Notice the excessive  90-degree sharp turns. \autoref{fig:n and r} illustrates the geometric interpretation and relationship between $n(x)$, $e(x)$ and $r(x)$ used for \autoref{theroem:set invariance in reference}.}
\end{figure}

Here in this section, by posing Mod-DS's input constraining process as an independent convex optimization problem and proving mathematically that such optimization problem either has simple closed-form solutions or can be solved sufficiently fast, we argue that Mod-DS's constraints enforcement ability in fully-actuated systems is equivalent to that of CBF-QP. In robot applications, inputs constraints typically can be categorized into 2 types: speed constraints $||\dot{x}||_2 = ||u||_2 \leq u_\text{ub}$, and velocity or box constraints $U_\text{lb}\leq u \leq U_\text{ub}$, where $u_\text{ub}\in \mathbb{R}$ and $U_\text{lb}$, $U_\text{ub} \in \mathbb{R}^d$. Following, one optimization problem is constructed for each constraint type. We again abused notation here and set $n = n(x,t)$, which is defined in \eqref{eq:standard mod d}.

\subsection{Speed-Constraining Mod-DS}
\label{section:speed-modulation}
When the unconstrained output from Mod-DS $u_\text{unc} = \dot{x}_\text{unc}$ exceeds the speed limit, solving the following Quadratically Constrained Linear Programming (QCLP) problem finds the constrained output $u_\text{c}$ that is most similar to $u_\text{unc}$ and ensures safety \eqref{eq:const safe} while satisfying speed input constraint $u_\text{ub}$ \eqref{eq:const speed}. In both speed-constraining and velocity-constraining Mod-DS, robot safety is ensured by restricting the robot's velocity moving towards the obstacle, i.e. the projection of $u_\text{c}$ onto vector $n(x,x_o)$, to be no larger than that of $u_\text{unc}$. 
\begin{gather}
\label{eq:obj}
u_\text{c}= \argmax_{u} \; \langle u_\text{unc}, u \rangle\\
\label{eq:const speed} ||u||_2 \leq u_\text{ub}\\
\label{eq:const safe} \langle n, u \rangle \geq \min(\langle n,\bar{\dot{x}}_{o}\rangle,\langle n, u_\text{unc}\rangle)
\end{gather}

In speed-constraining Mod-DS, we make the design choice to measure similarity between the constrained and unconstrained Mod-DS outputs using dot products between the two vectors so that the QP has simple closed-form solutions. Using dot products as cost functions is reasonable because $\langle u_\text{unc}, u \rangle = ||u_\text{unc}||_2||u||_2\cos \theta_d$, where $\theta_d \in [0,\pi]$ is the angle difference between the 2 vectors. $||u_\text{unc}||_2$ is fixed because $u_\text{unc}$ is a constant vector input into the QP. When $||u||_2$ is also fixed, the dot product between $u_\text{unc}$ and $u$ is inversely correlated to $\theta_d$. In other words, the larger the dot product is, the more similar $u$ is to $u_\text{unc}$ orientation-wise. We know the magnitude of the optimal solution of the QP $u_\text{c}$ must be fixed to $||u_\text{c}||_2=u_\text{ub}$, because the above optimization problem would only be called to modify $u_\text{unc}$ when $||u_\text{unc}||_2 > u_\text{ub}$. Therefore, the optimization problem can be reformulated as the following. Note that $\langle a, b \rangle = a^\top b$ when $a$, $b \in \mathbb{R}^d$ and $||u||^2_2=u^\top u$.
\begin{gather}
\label{eq:qp speed}
u_\text{c}= \argmax_{u} \; u_\text{unc}^\top u \\
\nonumber u^\top u = (u_\text{ub})^2\\
\nonumber n^\top u \geq \min(n^\top \bar{\dot{x}}_{o},n^\top u_\text{unc})
\end{gather}

Denote $v_n = \min(n^\top\bar{\dot{x}}_{o},n^\top u_\text{unc})$. Solving for the explicit solutions of the above convex optimization problem using Karush–Kuhn–Tucker (KKT) conditions, a simple explicit solution is found as in \eqref{eq:mod speed}, where $v_e = \sqrt{||u_\text{unc}||_2^2-||n^\top u_\text{unc}||_2^2} = ||H^\top u_\text{unc}||_2$ is the projection of $u_\text{unc}$ onto the hyperplane $H=H(x,x_o)$ tangent to the boundary function $h(x,t)$ \eqref{eq:modulation-matrix}.
\begin{equation}
\label{eq:mod speed}
\begin{cases}
\frac{u_\text{ub}}{||u_\text{unc}||_2}u_\text{unc} \quad \qquad \qquad \qquad \text{if} \quad \frac{u_\text{ub}}{||u_\text{unc}||_2}n^\top u_\text{unc}\geq v_n\\
v_nn + \frac{\sqrt{(u_\text{ub})^2-(v_n)^2}}{v_\text{e}}(u_\text{unc}-u_\text{unc}^\top n n)\quad  \text{otherwise}  \\
\end{cases}
\end{equation}

% Notably, a similarly yet more conservative approach to enforce speed limits in Mod-DS has been proposed in \cite{constrainedModu}. Our main contribution here is proving theoretically that Mod-DS approaches can enforce input constraints in fully actuated systems, no less than that of CBF-QP. The speed-constraining method in \cite{constrainedModu} stops following Mod-DS and prioritizes keeping distance from the obstacle whenever an obstacle is approaching. 
The difference between the proposed method with the strategy in \cite{constrainedModu} is that ours considers the unconstrained Mod-DS outputs $n^\top u_\text{unc}$ when determining whether the robot is safe, which improves the smoothness of the trajectories generated by the constrained outputs and maximumly preserves the Mod-DS characteristics (see \autoref{fig:modulation constrained static}). 
% The chosen constraints, Eq.~\eqref{eq:const speed} and \eqref{eq:const safe}, prevent the modified velocity from exceeding $\dot{x}^{E_{n}}[0]$, the current safe velocity in the $\hat{n}$ direction that are concluded to be safe by the Mod-DS method. By maximizing the dot product between the constrained velocities and the unconstrained ones planned by the original Mod-DS, Eq.~\eqref{eq:obj}, leads the LP to solve for a new constrained velocity that is similar to the original one, in terms of having the maximum amount of the modified velocity to be projected onto the origin unconstrained velocity vector. More in-depth study of geometric relationships behind a plane intersecting spheres in 3D state-space could potentially enable us to expand the speed-constraining Mod-DS to 3D obstacle avoidance applications using drones and robot arms, which is one future direction of this work.
% \change{Discuss briefly how to extend to 3D state-space.}

\subsection{Velocity-Constraining Mod-DS} \label{section:vel-modulation}
Similarly, solving optimization problem in Eq.~\eqref{eq:lp velocity} offers solutions to enforcing velocity constraints $U_\text{lb}\leq u \leq U_\text{ub}$ for Mod-DS in fully-actuated systems while guaranteeing safety. 
\begin{gather}
\label{eq:lp velocity}
u_\text{c} = \argmin_u \; ||u - u_\text{unc}||_2^2\\
\nonumber U_\text{lb}\leq u \leq U_\text{ub}\\
 % -u &\leq -U_\text{lb}\\
\nonumber n^\top u \geq \min(n^\top \bar{\dot{x}}_{o},n^\top u_\text{unc})
\end{gather}

Unlike the speed-constraining Mod-DS, the QP problem for enforcing velocity constraints no longer has a straightforward explicit solution, because the complexity of the explicit solution derived from KKT conditions grows exponentially with the number of inequality constraints in the problem. Note that both speed-constraining and velocity-constraining Mod-DS methods proposed here remain valid for high dimensional state-spaces when $d>3$.

% Nevertheless, directly calling the proposed QP-solver inside Mod-DS results in two critical issues. Firstly, standard multi-obstacle Mod-DS constructs new $M(x)$ for each obstacle in the environment at time step $t$ and solves for their weighted sum. Thus, its execution time is linearly proportional to the number of obstacles $m$ in the neighborhood. Although QP problems are fast to solve, calling the optimizer $m$ times at each timestep makes the Mod-DS slower than standard CBF-QP with velocity constraints. Secondly, while the QP-solvers guarantee that the new velocities of the agent for a single obstacle are well-constraint, violations can happen after taking the weighted sum of the constrained velocities. Therefore, in this variant of the constrained Mod-DS method, multi-obstacle avoidance is guaranteed following the method proposed in \cite{zadeh2012dynamical}, where K is the number of obstacles in the environment, $\dot{x}^{o,k}_{L}$, $\dot{x}^{o,k}_{R}$ and $x_c^{o,k}$ are respectively the linear velocity, angular velocity and reference point locations of each obstacle, and $h^k$ represents the barrier function value for obstacle $k$.

% \change{Therefore, in this variant of the constrained Mod-DS method, multi-obstacle avoidance is guaranteed by constructing a $M(x)$ decomposing the agent's velocities in the reference direction $r(x)$ and tangent to the surface of the nearest obstacle. A common issue of path planning considering only the nearest obstacle is that other neglected obstacles nearby could be collided into the agent when the motion update frequency of the agent is not fast enough. This issue is relieved in velocity-constraining Mod-DS by its use of weighted velocity sum of all obstacles close to and travelling towards the agent in replacement of the velocity of a single obstacle.} \revised{remove the previous sentences.}


% \begin{gather}
% \dot{x}= M(x)(\dot{x}_{nom}-\bar{\dot{x}}_{o})+\bar{\dot{x}}_{o}\\
% \bar{\dot{x}}_{o}=\sum^K_{k=1}e^{-h^k}\omega^k(\dot{x}^{o,k}_{L}+\dot{x}^{o,k}_{R}(x-x_c^{o,k}))\\
% w^k=\frac{\prod_{j=1, j\not = k}^{K} h^j}{\sum_{i=1}^K\prod_{j=1, j\not = i}^{K} h^j} \label{vel mag coef}
% \end{gather} 

% The use of weighted obstacle velocity sum alleviates the issue that other neglected obstacles might run into the robot when the update frequency is not fast enough. Although the velocity-constraining Mod-DS loses advantages in runtime, it still manifests the strength of not being trapped by star-shaped obstacles that can not be ensured by CBF-QP. 

\begin{figure}
    \centering
     \includegraphics[trim={0.75cm 0.2cm 1cm 0cm},clip,width=\linewidth]{images/modulation/ModDS_streamplot_w_constraints.pdf}
     \caption{Unconstrained reference Mod-DS (first row) vs. speed-constrained (second row, left) and velocity-constrained reference Mod-DS with $U_\text{ub}= -U_\text{lb} = [2, 2]^\top$(second row, middle and right). The color codes for images in the left, middle, and right columns respectively represent $\frac{||u_\text{unc}||_2}{u_\text{ub}}$, $\frac{|u_1|}{2}$, and $\frac{|u_2|}{2}$, given that $u=[u_1, u_2]^\top$.}
        % \caption{\small Comparison between trajectories generated by an unconstrained modulation (first row) with the ones produced using speed-constrained (second row, left) and velocity-constrained modulation (second row, middle and right). The color codes for images in the left, middle and right columns respectively represent $\frac{||\dot{x}||_2}{V_{lim}}$, $\frac{\dot{x}_1}{V_1}$,$\frac{\dot{x}_1}{V_2}$.}
        \label{fig:modulation constrained static}
\vspace{-20pt}
\end{figure} 

\subsection{Theoretical Guarantees} Both speed and velocity constraining Mod-DS preserve the \textit{impenetrability} guarantees of the original Mod-DS, under the assumption that the speed that the obstacle travels towards the agent is no more than the speed limit. We showcase stream plots in \autoref{fig:modulation constrained static} to demonstrate performances of speed-constraining and velocity-constraining Mod-DS compared to unconstrained Mod-DS. 

\begin{theorem}
Consider an obstacle whose boundary is defined as $h(x,x_o) = 0$. Any trajectory $x(t)$, $t\in [0,\infty)$ starting outside the obstacle boundary, i.e., $h(x(0),x_o(0)\geq 0$, evolving according to Eq.~\eqref{eq:ds-modulation} and being constrained by the Speed-Constraining and Velocity-Constraining approaches Mod-DS presented in Section \ref{section:speed-modulation} and \ref{section:vel-modulation}, will never penetrate the obstacle boundary, i.e., $h(x(t),x_o(t))\geq 0$, $\forall t\in [0,\infty)$. 
\end{theorem}
\textbf{Proof:} See Appendix \ref{app:proof-imp}. \hfill $\blacksquare$

\section{Modulated Control Barrier Functions (MCBF)}
\label{sec:mod-cbf}
In fully actuated systems, Mod-DS approaches outperform CBF-QP in navigating around non-convex unsafe sets with far fewer undesirable local minima. However, CBF-QP is still more commonly used in robot applications because it works for control affine systems in general and the QP formulations are more intuitive to the users. Nevertheless, choosing between superiority in handling complex robot dynamics and navigating among concave unsafe sets free of undesirable local minima should not be a tradeoff that users have to make when designing robot controllers. In this section, 2 Modulated CBF-QP (MCBF-QP) approaches that combine the strength of CBF-QP with Mod-DS are proposed to ensure respectively few-local-minimum and local-minimum-free navigation in concave-obstacle environments with the ease of incorporating different robot dynamics.

\subsection{Reference MCBF for Star-shaped Obstacle Avoidance}\label{sec:reference mod-cbf}
In \autoref{sec: reference mod-ds vs cbf}, theoretical analysis of reference Mod-DS and CBF-QP demonstrated that CBF-QP's performance in $n(x,x_o)$ direction is similar to that of the reference Mod-DS in \autoref{theroem:set invariance in reference}. Besides, in \autoref{theroem:quantitative diff reference}, reference Mod-DS's introduction of off-diagonal components in the tangent hyperplane $H(x,x_o)$ navigation is identified to be the major differences between CBF-QP and reference Mod-DS in fully-actuated systems and the key factor for reducing undesirable equilibrium in star-shaped obstacle avoidance. Adding additional constraints for $u_\text{cbf}$ on the tangent hyperplane will reduce undesirable equilibrium caused by CBF constraints. In theory, there are numerous ways to incorporate the off-diagonal components into CBF-QP results. Here, we propose a linear constraint formulation that is mathematically simple and does not increase the complexity of the original CBF-QP problem. Note that $E_r(x,x_o)^{-1}[2:d,:]\dot{x} = H(x,x_o)^\top(I - \frac{1}{\nabla_xh(x,x_o)^\top r(x, x_o)}r(x, x_o)\nabla_xh(x,x_o)^\top)\dot{x}$ solves for the projection of $\dot{x}$ onto hyperplane $H(x,x_o)$ in basis $E_r(x,x_o)$. $g(x)$ and $E_r(x,x_o)^{-1}$ are defined in \eqref{eq:affine system} and \eqref{eq: Er inv}.
\begin{gather}\label{eq:mod-r-cbf affine} 
\nonumber u_{\text{mcbf}} = \argmin_{u \in \mathbb{R}^p, \rho \in \mathbb{R}^{d-1}}(u-u_\text{nom})^\top(u-u_\text{nom})+\rho^\top\rho\\
L_fh(x,x_o) + L_gh(x,x_o)u + \nabla_{x_o}h(x,x_o)\dot{x}_o\geq -\alpha (h(x,x_o))\\
H(x,x_o)^\top(I - \frac{r(x,x_o)\nabla_xh(x,x_o)^\top}{\nabla_xh(x,x_o)^\top r(x,x_o)})g(x)(u -u_\text{nom})= \rho\label{eq:mod-r-cbf constraint affine}
\end{gather}

For fully-actuated systems in \eqref{eq:fully actuated system}, the special case of MCBF-QP can be simplified as in \eqref{eq:mod-r-cbf fully actuated}.
\begin{gather}\label{eq:mod-r-cbf fully actuated} 
u_{\text{mcbf}} = \argmin_{u \in \mathbb{R}^d, \rho \in \mathbb{R}^{d-1}}(u-u_\text{nom})^\top(u-u_\text{nom})+\rho^\top\rho\\
\nabla_xh(x,x_o)^\top u + \nabla_{x_o}h(x,x_o)\dot{x}_o\geq -\alpha (h(x,x_o))\\
H(x,x_o)^\top(I - \frac{r(x,x_o)\nabla_xh(x,x_o)^\top}{\nabla_xh(x,x_o)^\top r(x,x_o)})(u -u_\text{nom})= \rho\label{eq:mod-r-cbf constraint fully-actuated}
\end{gather}

Reference-MCBF-QP, compared with regular CBF-QP, improves the performance of the optimization-based safe controller for star-shaped unsafe sets by reducing local minimum numbers. The new method ensures that the safe control outputs will never lead to undesirable equilibria, except when $\dot{x}_\text{nom}$ is inversely collinear with the reference direction $r(x,x_o)$. The added constraints are designed to be of linear equality types and, thus would not cause large runtime increases. Additionally, the employment of relaxation parameter $\rho$ sustains the feasibility of the regular CBF-QP, i.e. given the same $x, x_o$, Reference-MCBF-QP will be feasible if and only if the CBF-QP without constraints \eqref{eq:mod-r-cbf constraint affine} and \eqref{eq:mod-r-cbf constraint fully-actuated} is feasible. The performance of fully-actuated reference MCBF-QP in star-shaped obstacle environments is validated in \autoref{fig:mod r cbf}.

\begin{figure}
    \centering
     \includegraphics[width=\linewidth]{images/mod_cbf/cbf_mod_streamplot_labeled.pdf}
     \caption{Performance of reference MCBF-QP in single obstacle avoidance with no robot input constraints (first row, (a): full view, (b): close view), in single obstacle avoidance with robot input constraints of $||u_\text{mcbf}||_2\leq2$ (second row, (a): full view, (b): close view), and in multi-obstacle avoidance respectively with and without robot input constraints of $||u_\text{mcbf}||_2\leq2$ (third row (a) and (b)).}
        \label{fig:mod r cbf}
\vspace{-20pt}
\end{figure} 

\textit{Explicit Solutions of Reference-Mod-DS}: When there is only one unsafe set in the environment and robot actuation limit is ignored, like CBF-QP, explicit solutions can also be derived for reference MCBF-QP. In the following paragraph, we derive the explicit solutions of reference MCBF-QP for fully-actuated systems in \eqref{eq:mod-r-cbf fully actuated}. The acquired result is then compared against \autoref{theroem:quantitative diff reference} to demonstrate the new method's ability in reducing undesirable equilibria. 

\textbf{Derivation:} To draw connections with \autoref{theroem:quantitative diff reference}, here we again assume static-single-obstacle scenarios ($\dot{x}_o = \textbf{0}$) and define $\alpha = \alpha(h(x,x_o))$, $H=H(x,x_o)$, $r =r(x,x_o)$, and $\nabla_xh = \nabla_xh(x,x_o)$. MCBF-QP in \eqref{eq:mod-r-cbf fully actuated} can be reformulated into Lagrangian function in \eqref{eq:lag func}, where $\mu_\text{eq} \in \mathbb{R}^{d-1}$ and $\mu_\text{cbf}$ are the Lagrange multipliers. 
\begin{equation}
\label{eq:lag func}
\begin{aligned}
 & \quad \;L(u, \rho, \mu_\text{eq},\mu_\text{cbf}) \\
 &= (u-u_\text{nom})^\top(u-u_\text{nom})+\rho^\top\rho\\
    &+\mu_{eq}^\top[\rho - H^\top(I - \frac{r\nabla_xh^\top}{\nabla_xh^\top r})(u-u_\text{nom})]
    -\mu_\text{cbf}(\nabla_xh^\top u+\alpha)
\end{aligned}
\end{equation}

According to Karush–Kuhn–Tucker theorem,  the necessary and sufficient conditions for $u_\text{mcbf}$ and $\rho^*$ to be the solution to the MCBF-QP are
\begin{equation}
\label{eq: KKT for mod-r-cbf}
\begin{cases}
\nabla L(u_\text{mcbf}, \rho^*, \mu_\text{eq},\mu_\text{cbf}) = \textbf{O}  \quad \text{stationarity}\\
\nabla_xh^\top u_\text{mcbf} +\alpha \geq 0\\
H^\top(I - \frac{r\nabla_xh^\top}{\nabla_xh^\top r})(u_\text{mcbf} -u_\text{nom})= \rho^* \quad \text{primal feasibility}\\
\mu_\text{cbf}\geq 0 \quad \text{dual feasibility}\\
\mu_\text{cbf}(\nabla_xh^\top u_\text{mcbf} +\alpha) =0 \quad \text{complementary slackness}
\end{cases}.
\end{equation}

Solving the linear systems in \eqref{eq: KKT for mod-r-cbf}, the explicit expanded solution $V=[u_\text{mcbf}; \mu_\text{cbf}] \in \mathbb{R}^{d+1}$ of the MCBF-QP is computed to be
\begin{equation}
V = \begin{cases}  
\begin{bmatrix}
    u_\text{nom} \\ 0
\end{bmatrix} \quad \text{if} \quad \nabla_xh^\top u_\text{nom} +\alpha \geq 0\\
\begin{bmatrix}
F & -\nabla_xh\\
\nabla_xh^\top & 0
\end{bmatrix}^{-1}\begin{bmatrix}
    Fu_\text{nom}\\
    -\alpha
\end{bmatrix}\quad \text{otherwise}
\end{cases},
\end{equation}

where matrix $F$ is defined as
\begin{equation}
\label{eq: F def}
    F = 2[I+(I-\frac{r\nabla_xh^\top}{\nabla_xh^\top r})^\top(I-\frac{r\nabla_xh^\top}{\nabla_xh^\top r})].
\end{equation}

Simplifying the above equation, the expanded explicit solution $V$ can be written as 
\begin{equation}
V = \begin{cases}  
\begin{bmatrix}
    u_\text{nom} \\ 0
\end{bmatrix} \quad \text{if} \quad \nabla_xh^\top u_\text{nom} +\alpha \geq 0\\
\begin{bmatrix}
F^{-1}(I+\nabla_xhA) & -A^\top\\
A & -\frac{AF\nabla_xh}{\nabla_xh^\top \nabla_xh}
\end{bmatrix}\begin{bmatrix}
    Fu_\text{nom}\\
    -\alpha
\end{bmatrix}\quad \text{otherwise}
\end{cases},
\end{equation}

where matrix $A$ is defined as
\begin{equation}
\label{eq:A}
    A = -\frac{\nabla_xh^\top F^{-1}}{\nabla_xh^\top F^{-1}\nabla_xh} \quad .
\end{equation}

In other words, the explicit solution $u_\text{mcbf}$ becomes
\begin{equation}
\label{eq:umcbf unsimplified}
u_\text{mcbf} = \begin{cases}  
    u_\text{nom}  \quad \text{if} \quad \nabla_xh^\top u_\text{nom} +\alpha \geq 0\\
F^{-1}(I+\nabla_xhA)Fu_\text{nom} + A^\top\alpha \quad \text{otherwise}
\end{cases}.
\end{equation}

From \autoref{fig:n and r}, it is noticeable that vector $r(x,x_o)$ can be represented as a weighted vector sum of $n(x,x_o)$ and a vector $e_i(x,x_o)$ respectively orthogonal and parallel to the hyperplane $H$, as in \eqref{eq: r to n and e}, $w_i\in \mathbb{R}^+$. Like before, we denote $e_i = e_i(x,x_o)$ and $n = n(x,x_o)$ to simplify the mathematical expressions. Note that $\sum_{i=0}^{d-1}(w_i)^2 = 1$ because by definition, $||r||_2 =1$. 
\begin{equation}
\label{eq: r to n and e}
    r = w_0n + \sum_{i=1}^{d-1} w_ie_i \quad \text{and} \quad \sum_{i=0}^{d-1}(w_i)^2 = 1
\end{equation}

Based on the relationship between $n$ and $\nabla_xh$ given in \eqref{eq:standard mod d}, $\nabla_xh=||\nabla_xh||_2 n$. Plug the new representation of $r$ and $\nabla_xh$ into \eqref{eq: F def}, $F$ can be orthogonally diagonalized as the following, where $E = [n, H]$ is defined in Eq~\eqref{eq:modulation-matrix} and \eqref{eq:standard mod d}.
\begin{equation}
\begin{aligned}
    F &= 4I+2(\frac{1}{(w_0)^2}-2)nn^\top-\frac{2}{w_0}\sum_{i=1}^{d-1} w_i(ne_i^\top+e_in^\top)\\
    &=4EIE^\top + E\begin{bmatrix}
        \frac{2}{(w_0)^2}-4 & -\frac{2w_1}{w_0}& ...& -\frac{2w_{d-1}}{w_0}\\
        -\frac{2w_1}{w_0}& 0& ...& 0\\
        \vdots& \vdots &...&\vdots\\
        -\frac{2w_{d-1}}{w_0}& 0&...& 0
    \end{bmatrix}E^\top\\
    &=E\Lambda E^\top ,
\end{aligned}
\end{equation}

where
\begin{equation}
    \Lambda = \begin{bmatrix}
        \frac{2}{(w_0)^2} & -\frac{2}{w_0}W^\top\\
        -\frac{2}{w_0}W & 4I_{d-1}
    \end{bmatrix} \quad \text{and} \quad
    W = \begin{bmatrix}
        w_1\\
        \vdots\\
        w_{d-1}
    \end{bmatrix}.
\end{equation}

% \begin{equation}
%     W = \begin{bmatrix}
%         w_1\\
%         \vdots\\
%         w_{d-1}
%     \end{bmatrix}
% \end{equation}

Therefore, due to the property of orthogonal diagonalization,
\begin{equation}
\label{eq:F inv}
\begin{aligned}
    F^{-1} = (E\Lambda E^\top)^{-1}
     = E\Lambda^{-1} E^\top
\end{aligned}.
\end{equation}
Given $\Lambda$ is a block matrix partitioned into four blocks, it can be inverted blockwise.
\begin{equation}
\label{eq:Lambda inv}
\Lambda^{-1} = \begin{bmatrix}
    k & \frac{k}{2w_0}W^\top\\
    \frac{k}{2w_0}W & \frac{1}{4}I_{d-1}+\frac{k}{4(w_0)^2}WW^\top
\end{bmatrix},
\end{equation}

where $W^\top W = \sum_{i=1}^{d-1}(w_i)^2 = 1-(w_0)^2$,
\begin{equation}
\label{eq: k}
\begin{aligned}
    k = (\frac{2}{(w_0)^2}-\frac{1}{(w_0)^2}W^\top W)^{-1}
    =\frac{(w_0)^2}{1+(w_0)^2}
\end{aligned}.
\end{equation}

Plugging in results from \eqref{eq:F inv}, \eqref{eq:Lambda inv} and \eqref{eq: k} into \eqref{eq:A}, 
\begin{equation}
    A = \frac{1}{2||\nabla_xh||_2}(n^\top+\frac{1}{w_0}r^\top).
\end{equation}

Therefore, when $\nabla_xh^\top u_\text{nom}+\alpha < 0$, the explicit solution of reference MCBF-QP $u_\text{mcbf}$ can be simplified to, 
\begin{equation}
\label{eq:umcbf explicit}
\begin{aligned}
& \quad \;u_\text{mcbf} \\
&= \begin{cases}  
    u_\text{nom}  \quad \text{if} \quad \nabla_xh^\top u_\text{nom} +\alpha \geq 0\\
u_\text{nom}-\frac{1}{2w_0}(w_0nn^\top+rn^\top)u_\text{nom} + A^\top\alpha \quad \text{otherwise}
\end{cases}.
\end{aligned}
\end{equation}

% \begin{equation}
% \Lambda^{-1} = \begin{bmatrix}
%     k & -k(-\frac{2}{w_0}W^T)(4I_{d-1})^{-1}\\
%     -(4I_{d-1})^{-1}(-\frac{2}{w_0}W)k & (4I_{d-1})^{-1}+(4I_{d-1})^{-1}(-\frac{2}{w_0}W)k(-\frac{2}{w_0}W^T)(4I_{d-1})^{-1}
% \end{bmatrix}
% \end{equation}
% \todo{it would be best if we can conduct analysis on the general solutions instead of when $d=2$, but I am stuck at how to handle the inverse matrix.}

\textbf{Remark:} The explicit solution of regular CBF-QP in fully-actuated systems \eqref{eq: explicit cbf fully actuated} can be rewritten using $n$ as the following.
\begin{equation}
\begin{aligned}
u_\text{cbf} 
= \begin{cases}
u_\text{nom}  \quad \text{if} \quad \nabla_xh^\top u_\text{nom} +\alpha \geq 0\\
u_\text{nom}-(nn^\top u_\text{nom} + \alpha \frac{n}{||\nabla_xh||_2})\quad \text{otherwise}
\end{cases}
\end{aligned}
\end{equation}

Given the explicit solution of the proposed reference MCBF-QP in \eqref{eq:umcbf explicit}, the differences between the outputs of the two solutions can be quantified as in \eqref{eq: umcbf-ucbf}. 
\begin{equation}
\label{eq: umcbf-ucbf}
\begin{aligned}
& \quad \;u_\text{mcbf} - u_\text{cbf} \\
&= \begin{cases}
0  \quad \text{if} \quad \nabla_xh^\top u_\text{nom} +\alpha \geq 0\\
-\frac{1}{2w_0}(r-w_0n)n^\top u_\text{nom} + (A^\top+\frac{n}{||\nabla_xh||_2})\alpha \quad \text{otherwise}
\end{cases}
\end{aligned}
\end{equation}

When the controlled agent is on the boundary of the unsafe sets, i.e. $\alpha = 0$, \eqref{eq: umcbf-ucbf} becomes 
\begin{equation}
\label{eq: umcbf-ucbf at 0}
\begin{aligned}
u_\text{mcbf} - u_\text{cbf}
= \begin{cases}
0  \quad \text{if} \quad \nabla_xh^\top u_\text{nom} \geq 0\\
-\frac{1}{2w_0}(r-w_0n)n^\top u_\text{nom} \quad \text{otherwise}
\end{cases}.
\end{aligned}
\end{equation}

Given \eqref{eq: r to n and e}, we know $n^\top r = w_0$ and $e_i^\top r=w_i$ for $i \in \{1, 2, ..., d-1\}$. Therefore, the quantitative relationship between reference Mod-DS and regular CBF-QP deduced in \eqref{eq:quantitative diff} can be reformulated as

\begin{equation}
\begin{aligned}
u_\text{mod}^r - u_\text{cbf}
= \begin{cases}
0  \quad \text{if} \quad \nabla_xh^\top u_\text{nom} \geq 0\\
-\frac{\lambda_e-\lambda}{w_0}(r-w_0n)n^\top u_\text{nom} \quad \text{otherwise}
\end{cases}
\end{aligned} 
\end{equation}

Since $\lambda$, $\lambda_e$ can be any real number satisfying \eqref{eq:standard mod lambda}, there exist an infinite amount of $\lambda$, $\lambda_e$ meeting the requirement $\lambda_e-\lambda = \frac{1}{2}$. Therefore, the conclusion can be drawn that there always exist feasible $\lambda$, $\lambda_e$ pairs such that outputs from reference MCBF-QP equal that from reference Mod-DS on boundaries of the unsafe sets. This guarantees that reference MCBF-QP is free of undesirable equilibria except when $u_\text{nom}$ is not inversely collinear to $r(x,x_o)$ in the boundary set $\partial C$. 

%When $d=2$, denote $\nabla_xh = [n_1, n_2]$, $r=[r_1, r_2]$, $u_\text{nom}=[u_{1_\text{nom}};u_{2_\text{nom}}]$, where $n_1$, $n_2$, $r_1$, $r_2$, $u_{1_\text{nom}}$, and $u_{2_\text{nom}} \in \mathbb{R}$. When $\nabla_xh(x,t)^Tu_\text{nom} +\alpha < 0$, the explicit CBF-QP solutions in \eqref{eq: explicit cbf fully actuated} can be written as:
% \begin{equation}
% \begin{aligned}
% u_\text{cbf} &= \begin{bmatrix}
%     u_{1_\text{nom}} - \frac{n_1}{n_1^2+n_2^2}(n_1u_{1_\text{nom}}+n_2u_{2_\text{nom}}+\alpha)\\
%      u_{2_\text{nom}} - \frac{n_2}{n_1^2+n_2^2}(n_1u_{1_\text{nom}}+n_2u_{2_\text{nom}}+\alpha)
% \end{bmatrix}\\
% &= \frac{1}{n_1^2+n_2^2}\begin{bmatrix}
%      n_2^2u_{1_\text{nom}}-n_1n_2u_{2_\text{nom}}-n_1\alpha\\
%      -n_1n_2u_{1_\text{nom}}+n_1^2u_{2_\text{nom}}-n_2\alpha
% \end{bmatrix}
% \end{aligned}
% \end{equation}

% Likewise, when $\nabla_xh(x,t)^Tu_\text{nom} +\alpha < 0$, the explicit reference Mod-CBF-QP solutions can be written as in \eqref{eq:2d explicit mod-r-cbf}.
% \begin{equation}
% \label{eq:2d explicit mod-r-cbf}
% u_\text{mcbf}=\frac{1}{n_1^2+n_2^2+1}\begin{bmatrix}
%     c_{11}u_{1_\text{nom}} - c_{12}u_{2_\text{nom}}-c_{1\alpha}\alpha\\
%      -c_{21}u_{1_\text{nom}} + c_{22}u_{2_\text{nom}}-c_{2\alpha}\alpha
% \end{bmatrix}
% \end{equation}

% where
% \begin{equation}
% \begin{aligned}
%     c_{11} &= n_2^2+ \frac{n_2r_2}{n_1r_1+n_2r_2}\\
%     c_{12} &= n_1n_2+ \frac{n_2r_1}{n_1r_1+n_2r_2}\\
%     c_{1\alpha} &= n_1+\frac{r_1}{n_1r_1+n_2r_2}\\
%     c_{21} &= n_1n_2+ \frac{n_1r_2}{n_1r_1+n_2r_2}\\
%     c_{22} &= n_1^2+ \frac{n_1r_1}{n_1r_1+n_2r_2}\\
%     c_{2\alpha}&=n_2+\frac{r_2}{n_1r_1+n_2r_2}
% \end{aligned}
% \end{equation}

% \begin{equation}
% \begin{aligned}
%     &\therefore u_\text{cbf}-u_\text{mcbf} \\
%     &= \frac{1}{(n_1^2+n_2^2+1)(n_1^2+n_2^2)}\begin{bmatrix}
%      f_{11}u_{1_\text{nom}} - f_{12}u_{2_\text{nom}}-f_{1\alpha}\alpha\\
%      -f_{21}u_{1_\text{nom}} + f_{22}u_{2_\text{nom}}-f_{2\alpha}\alpha
%     \end{bmatrix}
% \end{aligned}
% \end{equation}

% where
% \begin{equation}
% \begin{aligned}
%     f_{11} &= n_2^2- \frac{n_2r_2}{n_1r_1+n_2r_2}(n_1^2+n_2^2)\\
%     f_{12} &= n_1n_2- \frac{n_2r_1}{n_1r_1+n_2r_2}(n_1^2+n_2^2)\\
%     f_{1\alpha} &= n_1-\frac{r_1}{n_1r_1+n_2r_2}(n_1^2+n_2^2)\\
%     f_{21} &= n_1n_2- \frac{n_1r_2}{n_1r_1+n_2r_2}(n_1^2+n_2^2)\\
%     f_{22} &= n_1^2- \frac{n_1r_1}{n_1r_1+n_2r_2}(n_1^2+n_2^2)\\
%     f_{2\alpha}&=n_2-\frac{r_2}{n_1r_1+n_2r_2}(n_1^2+n_2^2)
% \end{aligned}
% \end{equation}

% Since when $d=2$, the quantified differences between CBF-QP and reference Mod-DS in \eqref{eq:quantitative diff} can be reformulated as

% \begin{equation}
% \begin{aligned}
% &(\lambda_e-\lambda)\sum_{i=1}^{d-1} \frac{e_i^Tr}{n^Tr}e_in^Tu_\text{nom} \\
% &=(\lambda_e-\lambda)\begin{bmatrix}
%     f_{11}u_{1_\text{nom}} - f_{12}u_{2_\text{nom}}\\
%     -f_{21}u_{1_\text{nom}} + f_{22}u_{2_\text{nom}}
% \end{bmatrix}
% \end{aligned}
% \end{equation}

% Therefore, the conclusion can be drawn that when $\alpha = 0$, there exist feasible $\lambda$, $\lambda_e$ pairs such that outputs from reference Mod-CBF-QP equals that from reference Mod-DS. For example, $\lambda=1-\frac{1}{2(n_1^2+n_2^2+1)(n_1^2+n_2^2)}$ and $\lambda_e = 1+\frac{1}{2(n_1^2+n_2^2+1)(n_1^2+n_2^2)}$. This guarantees that Mod-CBF-QP is free of spurious equilibrium except when $u_\text{nom}$ is inversely collinear to $r(x,t)$ in the boundary set $\partial C$. 

\subsection{On-Manifold MCBF for Non-star-shaped Obstacle Mod- Avoidance}
By keeping CBF constraints in regular CBF-QP to account for Mod-DS's policies in $n(x,x_o)$ direction and introducing new constraints \eqref{eq:mod-r-cbf constraint affine} and \eqref{eq:mod-r-cbf constraint fully-actuated} to mimic reference Mod-DS's actions in the tangent hyperplane $H(x,x_o)$, reference MCBF-QP successfully combines the strength of both and realizes star-shaped obstacle avoidance using QP-based methods. While on-manifold Modulation, functioning under the aggregate effects of 3 policies, is too structurally complicated to be explicitly connected to CBF-QP on the theory level, its main component, like other Mod-DS approaches, modulates $\dot{x}$ in $n(x,x_o)$ direction similar to CBF constraints and prevents the formation of undesirable equilibrium by ordering the directions of $\dot{x}$ projected onto the tangent hyperplane $H(x,x_o)$ using $\phi(x,x_o)$. Therefore, by introducing $H(x,x_o)$ related constraints, inspired by $\phi(x,x_o)$ in on-manifold Mod-DS, into CBF-QP, on-manifold MCBF-QP is constructed to achieve local-minimum-free safe control for control affine systems in general \eqref{eq:mod-phi-cbf affine}. The parameter $\gamma$ is a user-defined positive real number, i.e. $\gamma \in \mathbb{R}^+$. Unlike reference MCBF-QP, which stays feasible as long as the CBF-QP without constraint \eqref{eq:mod-r-cbf constraint affine} is feasible, on-manifold MCBF-QP's feasible domain is always smaller than that of the CBF-QP without constraint \eqref{eq:mod-phi-cbf constraint affine}. Additionally, the larger $\gamma$ is, the smaller the feasible domain of on-manifold MCBF-QP will be. The performance of on-manifold MCBF-QP in concave obstacle environments is validated in \autoref{fig:mod onM cbf}.
\begin{gather}\label{eq:mod-phi-cbf affine} 
\nonumber u_{\text{mcbf}} = \argmin_{u \in \mathbb{R}^p}\;(u-u_\text{nom})^\top(u-u_\text{nom})\\
L_fh(x,x_o) + L_gh(x,x_o)u + \nabla_{x_o}h(x,x_o)\dot{x}_o\geq -\alpha (h(x,x_o))\\
\phi(x,x_o)^\top f(x)+\phi(x,x_o)^\top g(x)u \geq \gamma \label{eq:mod-phi-cbf constraint affine}
\end{gather}

For fully-actuated systems in \eqref{eq:fully actuated system}, the special case of CBF-QP can be simplified as in \eqref{eq:mod-phi-cbf fully actuated}.
\begin{gather}\label{eq:mod-phi-cbf fully actuated} 
\nonumber u_{\text{mcbf}} = \argmin_{u \in \mathbb{R}^d, \rho \in \mathbb{R}^{d-1}}(u-u_\text{nom})^\top(u-u_\text{nom})\\
\nabla_xh(x,x_o)^\top u + \nabla_{x_o}h(x,x_o)\dot{x}_o\geq -\alpha (h(x,x_o))\\
\phi(x,x_o)^\top u\geq \gamma\label{eq:mod-phi-cbf constraint fully-actuated}
\end{gather}

\begin{figure}
    \centering
     \includegraphics[width=\linewidth]{images/mod_cbf/cbf_onM_mod_streamplot_labeled.pdf}
     \caption{Performance of on-manifold MCBF-QP in single star-shaped (first row) and non-star-shaped (second row, (a): full view, (b): close view) obstacle avoidance with no robot input constraints, in multi-concave obstacle avoidance(third row, (a): without input constraints, (b): with robot input constraints of $||u_\text{mcbf}||_2\leq2$), given $\gamma=1$. Lastly, pictures on the fourth row show the effects of $\gamma$ sizes on the resulted safe trajectories ((a): $\gamma=0.1$, (b): $\gamma=10$). } 
        \label{fig:mod onM cbf}
\vspace{-20pt}
\end{figure} 
\textbf{Remark:} While reference MCBF-QP can be in theory extended to nonlinear dynamical systems with equations in \eqref{eq:mod-r-cbf affine}, its applications in high dimensional systems is limited because finding reference directions $r(x,x_o)$ when $d>3$ is nontrivial in general and its complexity scales up as $d$ increases. However, given proper step size $\beta$, the geodesic approximation method in \eqref{eq:geo approxi} is able to closely approximate unsafe set boundaries for any given state $x \in \mathbb{R}^d$. As a result, on-manifold MCBF-QP can be readily applied in any control affine robot system given a reasonably designed barrier function $h(x, x_o)$.

\section{Robot Experiment}
\label{sec:experiment}
 In this section, the obstacle avoidance performances using MCBF-QPs proposed in \autoref{sec:mod-cbf} are validated in fully actuated and underactuated control affine systems, using respectively omnidirectional-drive Ridgeback and differential-drive Fetch robots. The constraint-enforcing effects of the modified Mod-DS methods proposed in \autoref{sec:constraing-mod} are also examined using Ridgeback robot in gazebo simulations. 
\subsection{Simulation Environment Setup}
In this work, a realistic hospital environment, featuring convex and concave static furniture as well as nurses and patients walking around as obstacles, is set up to validate the Mod-based CBF-QP variants proposed and compare them against existing approaches of Mod-DS and CBF-QP. Based on differences in robot initial and target locations, 5 scenarios are designed as in \autoref{fig:hop sim}. In each scenario, the robot is provided with the task of traveling from the initial position to the target while avoiding collisions with obstacles in the path. The designed scenarios well represent the encounterment a delivery robot needs to face when working in crowded dynamic environments like the hospital. Scenario 1 tests the controller's ability to navigate reactively out of a complicated concave obstacle like the front desk; scenario 2 and 3 (denote as mix and mix-reverse in \autoref{table: hospital sim}) challenged the robots with a mix of concave and convex obstacles; lastly, scenario 4 and 5 test robot's performances facing convex obstacles, the benches. Robots are assumed to process full knowledge of obstacles within a 3-meter detecting range inside the lobby during planning, including part of the static map shown in \autoref{fig:hos sim rep} and the real-time poses of the humans, modeled as cylinders, from a motion capture system. 


\begin{figure}
    \centering
    \begin{subfigure}{0.45\linewidth}
         \includegraphics[width = \textwidth]{images/gazebo/Hospital_Sim_Shrinked.pdf}
    \vspace{-10pt}
    \caption{\label{fig:hop sim}}
    \end{subfigure}
    \begin{subfigure}{0.52\linewidth}
         \includegraphics[width=\textwidth]{images/gazebo/sim_env.pdf}
     \vspace{-20pt}
     \caption{\label{fig:hop rep}}
    \end{subfigure}
    \caption{Simulation environment and controller setup for robot comparison tests.\autoref{fig:hop sim} displays the 5 scenarios designed for robot tests and \autoref{fig:hop rep} illustrates the robot's knowledge of the hospital environment.}
    \label{fig:hos sim rep}
\end{figure}

 % In tests using fully-actuated systems, the performance of the proposed Mod-CBF-QP is compared with that of the standard Mod-DS approaches and CBF-QP.
\subsection{Robot Dynamics}
In tests using fully actuated systems, the performances of the proposed reference and on-manifold MCBF-QP methods in 2-dimensional Euclidean space (R-MCBF-QP and onM-MCBF-QP) compete against regular CBF-QP and the proposed constrained Mod-DS using standard Ridgeback dynamics below, where $p_x$ and $p_y$ are robot locations in $x$ and $y$ axis.
 
 \begin{equation}
 \label{eq:ridgeback sys}
     \begin{bmatrix}
         \dot{p}_x\\\dot{p}_y
     \end{bmatrix}=
     \begin{bmatrix}
         u_x \\ u_y
     \end{bmatrix}
 \end{equation}

In tests for underactuated control affine systems, the performance of the proposed on-manifold MCBF-QP method in 2-dimensional Euclidean space (shifted on-manifold Mod-based CBF-QP, S-onM-MCBF) competes against regular CBF-QP using the shifted unicycle model (shifted CBF-QP, S-CBF-QP), while on-manifold MCBF-QP method's performance in 3-dimensional non-Euclidean space (high order on-manifold Mod-based CBF-QP, HO-onM-MCBF) competes against higher order CBF-QP (HO-CBF-QP) using Fetch dynamics. The most commonly used model for differential drive robots like Fetch is the unicycle model below, where $\theta$ is robot orientation in radians measured with respect to the positive $x$ axis. 
\begin{equation}
\label{eq:standard unicycle}
    \dot{x}=\begin{bmatrix}
        \dot{p_x}\\\dot{p}_y\\\dot{\theta}
    \end{bmatrix}=
    \begin{bmatrix}
        \cos{\theta} & 0\\ \sin{\theta} & 0\\ 0 & 1
    \end{bmatrix}
    \begin{bmatrix}
        v \\ \omega
    \end{bmatrix}
\end{equation}

While CBF-QP is easily generalizable to any control affine systems, it suffers from one critical restriction: the CBF has to be of relative degree 1. In other words, the CBF's first time-derivative has to depend on the control input. Since the control input $\omega$ in the standard unicycle model is of relative degree 2 to the CBFs $h_\text{conv}$, $h_\text{star}$ and $h_\text{nstar}$ in \autoref{sec: assumptions}, we chose a point of interest $a>0$ m ahead of the wheel axis of Fetch, instead of robot center, as shown in \autoref{fig:shift point}. The unicycle dynamics with the shifted coordinate system becomes the following. 
\begin{equation}
\label{eq:shifted unicycle}
    \dot{x}=\begin{bmatrix}
        \dot{p}_x\\\dot{p}_y\\\dot{\theta}
    \end{bmatrix}=
    \begin{bmatrix}
        \cos{\theta} & -a\sin{\theta}\\ \sin{\theta} & a\cos{\theta}\\ 0 & 1
    \end{bmatrix}
    \begin{bmatrix}
        v \\ \omega
    \end{bmatrix}
\end{equation}

\begin{figure}
    \centering
    \includegraphics[width=0.5\linewidth]{images/robot/Shifted_Model.pdf}
    \caption{Modified point of interest on Fetch robot.}
    \label{fig:shift point}
\end{figure}

Because $h_\text{conv}$, $h_\text{star}$ and $h_\text{nstar}$'s partial derivative with respect to $\theta$ are all zeros. The CBF constraints for the shifted unicycle system, adopted by shifted CBF-QP (S-CBF-QP) and 2-dimensional shifted on-manifold MCBF-QP (S-onM-MCBF-QP) methods, can be written as
\begin{equation}
\begin{aligned}
\frac{\partial h(p_x,p_y,x_o)}{\partial p_x}\dot{p}_x + \frac{\partial h(p_x,p_y,x_o)}{\partial y}\dot{p}_y+\nabla_{x_o}h(x,x_o)\dot{x}_o\geq \\
-\alpha (h(p_x,p_y,x_o)).
\end{aligned}
\end{equation}

While the shifted unicycle model enables CBFs to be defined as signed distance functions like $h_\text{conv}$, $h_\text{star}$ and $h_\text{nstar}$, the method cannot be easily generalized to all underactuated robots. HO-CBF-QP solves the generalizability issue by using high order control barrier functions (HO-CBFs) $h_\text{OH}(x, x_o)$ of relative degree 1 to control inputs, instead of signed distance functions $h(p_x, p_y, x_o)$ as CBFs. Here, we demonstrate that the on-manifold MCBF-QP method we proposed, likewise can be generalized to high-relative-order robot dynamics by controlling Fetch through the standard unicycle model in \eqref{eq:standard unicycle} using high-order on Manifold Mod-based CBF-QP (HO-onM-MCBF). HO-CBF $h_\text{OH}(x,t)$ used by 3-dimensional HO-onM-MCBF and HO-CBF-QP in our validations are constructed as the following, where $w \in \mathbb{R}^+$. 
\begin{equation}
\begin{aligned}
 h_\text{OH}(x, x_o) = h(p_x,p_y,x_o) + w\frac{\partial h(p_x,p_y,x_o)}{\partial p_x}\cos{\theta}\\
 +w\frac{\partial h(p_x,p_y,x_o)}{\partial p_y}\sin{\theta}
    \label{eq:ho barrier function}   
\end{aligned}
\end{equation}

HO-CBF constraints in CBF-QP and MCBF-QP can be obtained by substituting $h_{OH}(x, x_o)$ as $h(x, x_o)$ directly into CBF constraint inequality in Eq~\eqref{eq:mod-phi-cbf constraint affine}.

\textbf{Nominal Controller:} For fully-actuated systems, nominal controllers $u_\text{nom}^l$ in \eqref{eq:nominal system} with $\epsilon = \frac{1}{||x||_2}$ are used by Ridgeback. For underactuated systems, the nominal controllers are approximated from the linear dynamical systems as the following, where $\psi$ is the angle difference between the current robot pose $\theta$ and the desired robot pose estimated using the orientation of $u_\text{nom}^l$, and $\Delta t$ is the update timestep of the nominal controller. The approximation is not exact due to the non-holonomic property of differential drive systems. 

\begin{equation}
\begin{aligned}
v_\text{nom} = ||u_\text{nom}^l||_2 \quad \omega_\text{nom} = \frac{\psi}{\Delta t}
\end{aligned}\label{eq:dubin nominal system}
\end{equation}

\subsection{Geodesic Approximation in Non-Euclidean Space}
\cite{onManifoldMod} shows that the obstacle exit strategy of geodesic approximation, given well-tuned step size $\beta$ and horizon size $N$, can generate a first-order approximation to the obstacle surface in Euclidean space $\mathbb{R}^d$ alongside an effective circumventing strategy, connecting robot current position to target position with the least reward $P_N$. In this work, we demonstrate that such a strategy can be extended to state $x$ in non-Euclidean space when used in Modulated CBF-QP, making possible local-minimum-free control affine CBF-QP in general. 

Here, the application of the geodesic approximation strategy is validated using the 3-dimensional non-Euclidean state space for the unicycle model $[p_x, p_y, \theta]$. \autoref{fig:geo non-eucl}(a), (b) and (c) shows what the unsafe region defined by HO-CBF in \autoref{eq:ho barrier function} looks like in our constructed 3D non-Euclidean space. For visual effects and interpretability of the obstacle, only the section with $\theta \in [0, 2\pi)$ is shown. Given robot initial state $x^0=[3,3,0]$ and 20 $e^0$ vectors evenly distributed in the plane tangent to the gradient of obstacle $h_\text{OH}(x^0, x_o)$, geodesic approximation strategy, as shown in \autoref{fig:geo non-eucl}, is able to select the circumventing trajectories traveling most efficiently towards the target using reward function $p(x^i, x^*) = ||x^i[:2]-x^*||_2$.

\begin{figure}
    \centering
    \begin{subfigure}{0.49\linewidth}
         \includegraphics[width = \textwidth]{images/isoline/p_top.pdf}
    % \vspace{-5pt}
    \caption{\label{fig: p top}}
    \end{subfigure}
    \begin{subfigure}{0.49\linewidth}
         \includegraphics[width=\textwidth]{images/isoline/p_side.pdf}
     \caption{\label{fig: p side}}
    \end{subfigure}
    \begin{subfigure}{0.49\linewidth}
         \includegraphics[width = \textwidth]{images/isoline/p_tri.pdf}
    \caption{\label{fig: p tri}}
    \end{subfigure}
    \begin{subfigure}{0.49\linewidth}
         \includegraphics[width=\textwidth]{images/isoline/p_20_top.pdf}
     \caption{\label{fig: p 20 top}}
    \end{subfigure}
    \begin{subfigure}{0.49\linewidth}
         \includegraphics[width = \textwidth]{images/isoline/p_20_tri.pdf}
    \caption{\label{fig: p 20 tri}}
    \end{subfigure}
    \begin{subfigure}{0.49\linewidth}
         \includegraphics[width=\textwidth]{images/isoline/p_20_tri2.pdf}
     \caption{\label{fig: p 20 tri 2}}
    \end{subfigure}
    \caption{\autoref{fig: p top}, \autoref{fig: p side}, and \autoref{fig: p tri} shows the top, side and diagonal views of the unsafe spaces defined by the constructed high order barrier function $h_\text{OH}$ \eqref{eq:ho barrier function}. \autoref{fig: p 20 top}, \autoref{fig: p 20 tri}, and \autoref{fig: p 20 tri 2} demonstrate from 3 view angles that the geodesic approximation method can be adapted to find the most effective circumventing strategy (colored in red) to the target location (denoted by green 'X'), despite the state space being non-euclidean. }
    \label{fig:geo non-eucl}
\end{figure}

\subsection{Simulation Results}
The performances of the proposed Mod-based CBF-QP methods are validated in simulation environments through comparisons to existing reactive safe controllers, including comparing R-MCBF-QP and onM-MCBF-QP with normal Mod-DS (N-Mod-DS), reference Mod-DS (R-Mod-DS), on-manifold Mod-DS (onM-Mod-DS) and CBF-QP for fully-actuated systems in gazebo hospital environments using Ridgeback, competing S-onM-MCBF-QP against S-CBF-QP for underactuated differential drive systems using Fetch in gazebo hospital environments, and lastly matching HO-onM-MCBF-QP with HO-CBF-QP in the underactuated control affine system of standard unicycle model from \eqref{eq:standard unicycle} in python-simulated hospital environments. The speed-constraining strategy for Mod-DS approaches proposed in \autoref{sec:constraing-mod} is validated on N-Mod-DS, R-Mod-DS, and onM-Mod-DS for fully actuated systems. Controllers in all simulations are running at 20Hz and those in all real-life experiments are running at 50Hz. In gazebo-simulated environments, the robot is asked to repeat the task of traveling from a fixed initial pose to the target 10 times. In python-simulated environments, the robot's performances starting from the same initial location but with 10 different orientations are recorded. The recorded performances are analyzed using the time (in seconds) that the robot takes to travel from the initial to the target locations (durations), the percentage of collision-free trajectories among the repeated ones (safe \%), the chance for the robot to reach the target (reached \%), and the average number of unsolvable cases (infeasibility) encountered by the controller per trajectories in \autoref{table: hospital sim}. Note the metric duration cannot be computed for trajectories where the robot is stuck in local minima and fails to reach the target, and are instead denoted as not applicable ("NA").
\begin{table*}[!tbp]
\centering
 \begin{minipage}[t]{.9\textwidth}
    \centering
   \resizebox{1\linewidth}{!}{\begin{tabular}{|m{0.4cm}|  m{3cm} | m{2.2cm} | m{2.2cm} | m{2cm} | m{2cm} | m{2cm} | m{2cm} | m{2cm}|m{2cm}|m{2cm}|}
 \hline
 \rowcolor{Gray}\multirow{2}{*}{}&\multicolumn{2}{|c|}{}&\multicolumn{4}{|c|}{Static}&\multicolumn{3}{|c|}{Dynamic}\\
  \cline{2-10}
  \rowcolor{Gray}& Method & Environment & Duration (s) & Safe \% & Reached \% & Infeasible \# & Safe \% & Reached \% & Infeasible \# \\ 
  \hline
  %done
    \multirow{18}{*}{\rotatebox[origin=c]{90}{Fully Actuated Systems}}& N-Mod-DS  & concave & NA & 1.00 & 0.00 & 202 & 1.00 & 0.00 & 223\\ 
    & & convex & 19.9s & 1.00 & 1.00 & 2 & 0.50 & 1.00 & 90\\ 
    & & mix & NA & 1.00 & 0.00 & 207 & 1.00 & 0.00 & 219\\ 
    & & mix-reverse & NA & 1.00 & 0.00 & 7 & 0.00 & 0.00 & 106\\
    \cline{2-10}
    &R-Mod-DS & concave & NA & 1.00 & 0.00 & 19 & 1.00 & 0.00 & 20\\ 
    & & convex & 17.4 s & 1.00 & 1.00 & 5 & 1.00 & 1.00 & 9\\ 
    & & mix & NA & 1.00 & 0.00 & 20 & 1.00 & 0.00 & 20\\ 
    & & mix-reverse & 21.7 s & 1.00 & 1.00 & 9 & 1.00 & 1.00 & 14\\
    \cline{2-10}
    &OnM-Mod-DS & concave & 16.2 s & 1.00 & 1.00 & 7 & 1.00 & 1.00 & 30\\ 
    & & convex & 15.7 s & 1.00 & 1.00 & 2 & 1.00 & 1.00 & 85\\ 
    & & mix & 21.3 s & 1.00 & 1.00 & 6 & 1.00 & 1.00 & 10\\ 
    & & mix-reverse & 21.7 s & 1.00 & 1.00 & 9 & 1.00 & 1.00 & 23\\
    \cline{2-10}
    &CBF-QP & concave & NA & 1.00 & 0.00 & 0 & 1.00 & 0.00 & 0\\ 
    & & convex & 33.7 s & 1.00 & 1.00 & 0 & 1.00 & 1.00 & 1\\ 
    & & mix & NA & 1.00 & 0.00 & 0 & 1.00 & 0.00 & 0\\ 
    & & mix-reverse & NA & 1.00 & 0.00 & 0 & 1.00 & 0.00 & 0\\
    \cline{2-10}
    &R-MCBF-QP & concave & NA & 1.00 & 0.00 & 0 & 1.00 & 0.00 & 0\\ 
    & & convex & 29.5 s & 1.00 & 1.00 & 0 & 1.00 & 1.00 & 0\\ 
    & & mix & NA & 1.00 & 0.00 & 0 & 1.00 & 0.00 & 0\\ 
    & & mix-reverse & 38.2S & 1.00 & 1.00 & 0 & 1.00 & 1.00 & 0\\
    \cline{2-10}
    &onM-MCBF-QP & concave & 22.8 s & 1.00 & 1.00 & 0 & 1.00 & 1.00 & 0\\ 
    & & convex & 25.8 s & 1.00 & 1.00 & 0 & 1.00 & 1.00 & 0\\ 
    & & mix & 27.2 s & 1.00 & 1.00 & 0 & 1.00 & 1.00 & 0\\ 
    & & mix-reverse & 33.3 s & 1.00 & 1.00 & 0 & 1.00 & 1.00 & 0\\
    \hline
   \multirow{15}{*}{\rotatebox[origin=c]{90}{Underactuated Systems}} &S-CBF-QP & concave & NA & 1.00 & 0.00 & 0 & 1.00 & 0.40 & 0\\ 
    & & convex & 17.8 s & 1.00 & 1.00 & 0 & 0.95 & 1.00 & 1\\ 
    & & mix & 24.3 s & 1.00 & 1.00 & 0 & 0.70 & 0.70 & 24\\ 
    & & mix-reverse & 19.9 s & 1.00 & 1.00 & 0 & 1.00 & 1.00 & 0\\
    \cline{2-10}
    &S-onM-MCBF-QP & concave & 16.8 s & 1.00 & 1.00 & 0 & 0.80 & 0.80 & 0\\ 
    & & convex & 16.1 s & 1.00 & 1.00 & 0 & 1.00 & 1.00 & 0\\ 
    & & mix & 21.3 s & 1.00 & 1.00 & 0 & 1.00 & 1.00 & 0\\
    & & mix-reverse & 20.4 s & 1.00 & 1.00 & 0 & 1.00 & 1.00 & 0\\
    \cline{2-10}
  \cline{2-8}
    &HO-CBF-QP & concave & NA & 1.00 & 0.00 & 0 & 1.00 & 0.00 & 0\\ 
    & & convex & 16.1s & 1.00 & 1.00 & 0 & 1.00 & 1.00 & 10\\ 
    & & mix & 21.5s & 1.00 & 1.00 & 0 & 0.60 & 0.90 & 248\\ 
    & & mix-reverse & 21.4s & 1.00 & 1.00 & 0 & 1.00 & 1.00 & 0\\
    \cline{2-10}
    &HO-onM-MCBF-QP & concave & 15.8s & 1.00 & 1.00 & 0 & 1.00 &1.00 & 0\\ 
    & & convex & 18.1s & 1.00 & 1.00 & 0 & 1.00 & 1.00 & 3\\ 
    & & mix & 22.8s & 1.00 & 1.00 & 0 & 1.00 & 1.00 & 0\\ 
    & & mix-reverse & 23.0s & 1.00 & 1.00 & 0 & 1.00 & 1.00 & 0\\
    % & &  & 4 & 0.0005 s & 1.00 & 0 & 1.00 & 0\\ 
    % & &  & 5 & 0.0005 s & 1.00 & 0 & 1.00 & 0\\ 
\hline
\end{tabular}}
\captionof{table}{Performance of Mod-DS, CBF-QP and Mod-based CBF-QP approaches, running at 20 Hz, measured by the duration of the produced trajectories, safety rate (collision-free rate), the rate for the robot to reach the target locations successfully, and the average number of infeasibility encountered, in simulated hospital environments.  }
\label{table: hospital sim}
\end{minipage}
\end{table*}

Based on the data collected, the following conclusion can be drawn: in fully actuated systems, Mod-DS approaches like N-Mod-DS, R-Mod-DS, and onM-Mod-DS run faster than CBF-QP-based approaches like CBF-QP, R-MCBF-QP and onM-MCBF-QP, which matches the observation made in \autoref{table:static table} that Mod-DS approaches have faster near-obstacle velocities. While both methods guarantee robot safety by decelerating in the direction normal to the obstacles, common Mod-DS approaches additionally accelerate in the tangent direction, leading to a shorter duration. However, the weakness of Mod-DS approaches is that achieving multi-obstacle avoidance using the weighted sum of the stretched velocities computed with respect to each obstacle cannot handle well the scenarios where the robot is squeezed between multiple extremely close obstacles. Therefore, N-Mod-DS, R-Mod-DS and onM-Mod-DS have significantly higher chances of infeasibility in crowded environments like the simulated hospital in comparison to the other CBF-based methods. The encounterment of controller infeasibility drives N-Mod-DS to collision in some dynamic simulations, while CBF-QP, proven to be theoretically equivalent to N-Mod-DS in single obstacle avoidance in \autoref{theorem: equivalance}, does not suffer from such issues. However, in terms of the ability to reach the target, the performances of Mod-DS approaches like R-Mod-DS and onM-Mod-DS are superior to that of the CBF-QP. CBF-QP is recorded to stop at local minimum in all non-convex environments, while R-Mod-DS is able to successfully navigate through mix-reverse environment and onM-Mod-DS is completely local-minimum free. The performances of the Mod-based CBF-QPs we proposed, R-MCBF-QP and onM-MCBF-QP, are demonstrated to be superior to both Mod-DS and CBF-QP. They well combine the robust safety guarantee of CBF-QP with local minimum elimination of R-Mod-DS and onM-Mod-DS to achieve little solver infeasibility, high safety and convergence guarantees with comparable runtime. 

In underactuated differential drive systems, both regular CBF-QP methods, such as S-CBF-QP and HO-CBF-QP, and proposed Mod-based approaches like S-onM-MCBF-QP and HO-onM-MCBF-QP provide decent safety guarantees and will only collide with obstacles when solver infeasiblity happens. Yet both proposed Mod-DS approaches are demonstrated to significantly increase the chances of reaching the goal in concave environments. Furthermore, in most case scenarios, Mod-based CBF-QP approaches have shorter durations because the tangent velocity control incorporated from onM-Mod-DS ensures leading the robot towards the more efficient paths. 

In conclusion, tests in simulated hospital environments show that the performances of the proposed Mod-based CBF-QP approaches exceed those of Mod-DS and CBF-QP methods in both fully actuated and underactuated control affine systems. 

\subsection{Real World Experiments}
In addition to simulations, the proposed S-onM-MCBF-QP and HO-onM-MCBF-QP methods are compared to S-CBF-QP and HO-CBF-QP in real-life experiments using Fetch given nominal controller in \eqref{eq:dubin nominal system}. In the first experiment, Fetch is challenged to reach the goal position on the other side of the room despite active attempts from a walking human to block its path. In the second experiment, the robot is tasked to navigate out of a C-shaped cluster region towards the target. All experiments are repeated 5 times with different robot initial poses in static scenarios and different human walking trajectories in dynamic scenarios. The results demonstrate that CBF-induced local minima will deteriorate robot task completion performance, while the introduction of onM-Mod-based constraints into the same QP problem can sufficiently eliminate those local minima. In 2D navigation tasks using Fetch, though both HO-onM-MCBF-QP and S-onM-MCBF-QP drastically improve robot task completion performance, S-onM-MCBF-QP using the shifted unicycle model outperforms HO-onM-MCBF-QP using the standard unicycle model in terms of path efficiency and task completion rate. The reason for that is the unsafe region defined by HOCBF $h_\text{OH}(x, x_o)$ \eqref{eq:ho barrier function}, as shown in \autoref{fig: p tri}, is not the same as the actual obstacle region, causing the geodesic approximation strategy to output sometimes non-optimal choices of tangent vectors $\phi(x,x_o)$ for the robot to track. Additionally, finding appropriate step size $\beta$ and horizon $N$ to use during geodesic approximation are more challenging for HO-MCBF-QP that searches for $\phi(x,x_o)$ in 3D non-Euclidean state spaces, compared with S-MCBF-QP that looks for $\phi(p_x,p_y,x_o)$ instead in the 2D Euclidean state spaces. However, HO-onM-MCBF-QP's merit is its generalizability which can be theoretically adopted by all robot models. 
\begin{table}[htb]
\centering
\resizebox{0.9\linewidth}{!}{\begin{tabular}{ | m{1.3cm} | m{2.4cm} | m{1.3cm} | m{1.7 cm} |m{1.5cm}|}
  \hline
  \rowcolor{Gray} Scenario & Methods & Safe \%  & Reached \% & Duration (s)\\ 
  \hline
   \multirow{4}{*}{\parbox{1.3cm}{convex dynamic}} & S-CBF  & 100 & 0 & NA\\
   & S-onM-MCBF & 100 & 100 & 12.4\\
  & HO-CBF & 100 & 0 & NA\\
  & HO-onM-MCBF & 100 & 100 & 14.3\\
  \hline
  \multirow{4}{*}{\parbox{1.3cm}{static concave}} &  S-CBF  & 100 & 0 & NA\\
  & S-onM-MCBF & 100 & 100 & 14.9\\
  & HO-CBF & 100 & 0 & NA\\
  & HO-onM-MCBF & 100 & 80 & 30.7\\
  \hline
\end{tabular}}
\caption{\small Performances of proposed onManfold-Mod-based CBF-QP methods in comparison to CBF-QP formulations in real-life experiments using Fetch at 20Hz.}
\label{table:experiment}
\end{table}

\section{Discussion and Future Directions}
\label{future}
By quantitatively and analytically comparing the performances of optimization-based safe control approaches of CBF-QP, the closed-form solutions from Mod-DS and the Mod-based CBF-QP methods we propose in both static and moving obstacle environments, the following conclusion can be drawn for obstacle avoidance in 2-dimensional space using single-integrator and differential drive systems:

\begin{enumerate}[leftmargin=*]
  \item Mod-DS approaches like reference Mod-DS and on-manifold Mod-DS perform better with concave obstacles by causing fewer undesirable equilibria on the boundary set $\partial C$ than CBF-QP in static and less crowded dynamic environments. 
  \item CBF-QP handles safety guarantees better than Mod-DS in crowded dynamic environments. However, its convergence to targets cannot be guaranteed in environments with concave obstacles. 
  \item In comparison to Mod-DS and CBF-QP, Mod-based CBF-QP performs better in vast majority of the cases, no matter the systems are fully-actuated or underactuated. However, even with the use of jax, current update frequency of the controller is limited to 50Hz and will decrease as the robot dynamics gets more complicated with larger state vector $\xi$. In applications where a higher update frequency using fully-actuated systems is desired, using on-manifold Mod-DS in combination of constrained Mod-DS strategies will be a good alternative.
  \end{enumerate}

In this work, the proposed Mod-based CBF-QP methods are validated using omni-directional and differential drive robot dynamics. However, in addition to locomotion tasks, robot arm manipulation's performance is also degraded by the existence of local minima in concave joint spaces. Therefore, adapting the on-manifold Mod-based CBF-QP methods to achieve local-minimum-free manipulation would be one future directions for us to explore. Besides, the performance of the tangent direction selection using geodesic approximation strategy relies on proper user tuning of $\beta$, and $N$ parameters. Automating the tuning process using analytic or machine-learning based algorithms to enable Mod-based CBF-QPs to navigate local-minimum-free around constantly deforming dynamic obstacles would be another future direction for the project.

{
\bibliographystyle{IEEEtran}
\bibliography{refs.bib}
}
% \clearpage
\appendix
% \subsection{Influence of parameters on funnel shape}
% \label{section:funnel}
% \begin{figure}[hbp]
%     \centering
%     \begin{subfigure}{0.50\linewidth}
%          \includegraphics[width=\linewidth]{images/isoline/star a.png}
%          \label{fig:star a}
%     \end{subfigure}\begin{subfigure}{0.50\linewidth}
%          \includegraphics[width=\linewidth]{images/isoline/star b.png}
%          \label{fig:star b}
%     \end{subfigure}
%     \vspace{-10pt}
%     \caption{\small Influence of $a, b$ magnitude in the shape of the funnel.} 
%     \label{fig:star a b}
% \end{figure}

% \subsection{Proof of Saddle Equilibrium in CBF-QP}\label{section: saddle cbf proof}
% Following we prove that in a static single-obstacle environment, when a CBF-QP controller encounters inverse collinearity between $\nabla_x h(x)$ and $\dot{x}_{\text{nom}}$, the subsequent solutions of Eq. \eqref{cbf-qp} will lead the agent to an undesirable asymptotic equilibrium at the boundary of the obstacle given $\dot{x}_{\text{mod}} = u_{\text{mod}}$ and $\dot{x}_{\text{nom}} = u_{\text{nom}}$. 

% \noindent \textbf{Proof:} We begin by constructing a new coordinate system $A$ with the geometry center of the obstacle that the agent is trying to avoid as the origin. Make $[\hat{a}_1,\hat{a}_2,...,\hat{a}_n]$ the axis set of the coordinate system, where $\hat{a}_1 = \frac{\nabla_x h(x)}{||\nabla_x h(x)||}$ and $\hat{a}_i i\neg 1$ forming a plane (or axis or hyperplane) tangent to the obstacle surface. Since $\dot{x}_{\text{nom}}$ is inversely collinear to $\nabla_x h(x)$, it can be rewritten as $\dot{x}_{\text{nom}} = -c_2\hat{a}_1$, where $c_2$ is a positive scalar. The $\nabla_x h(x) = c_1\hat{a}_1$, where $c_1$ is a positive scalar. Let $\dot{x}_{\text{mod}}=[\dot{x}_1, \dot{x}_2,...,\dot{x}_n]$ in coordinate $B$, where $n$ is the dimension of the system. When the obstacle is static, $\delta h(x,t)/\delta t = 0$. Eq. \eqref{CBF function} can then be approximated as, 
% \begin{equation}
%     \begin{aligned}
%         \Dot{h}(x,t) = \nabla_x h(x)^T\dot{x}_{\text{mod}} \geq -\alpha\\
%         c_1\hat{a}_1^T(\dot{x}_1\hat{a}_1 + \dot{x}_2\hat{a}_2+...+\dot{x}_n\hat{a}_n)\geq -\alpha\\
%         c_1\dot{x}_1 \geq 0\label{cbf collinear constraint}
%     \end{aligned}
% \end{equation}

% The objective in Eq. \eqref{cbf-qp} becomes:
% \begin{equation} \label{cbf-conlinear obj}
% \begin{split}
% J &= \frac{1}{2}||u_{\text{nom}}-u_{\text{mod}}||_2^2 = \frac{1}{2}||\dot{x}_{\text{nom}}-\dot{x}_{\text{mod}}||_2^2 \\
% & = \frac{1}{2} (-c_2\hat{a}_1-\dot{x}_1)^2 +\frac{1}{2} \sum_{i=2}^n \dot{x}_i^2\\
% &=\frac{1}{2} (c_2\hat{a}_1+\dot{x}_1)^2 +\frac{1}{2} \sum_{i=2}^n \dot{x}_i^2\\
% \end{split}
% \end{equation}
% Therefore, when optimizing $J$ from Eq.\eqref{cbf-conlinear obj}, $\dot{x}_i$ with $i\geq2$ will always be set to 0, resulting in the agent to keep moving in the direction normal to the obstacle. When the agent location $x$ is extremely close to the boundary of the obstacle, the value of $\alpha$ becomes 0. In this case, the constraint in Eq. \eqref{cbf collinear constraint} mandates $\dot{x}_1 \geq 0$ because $c_1 > 0$. Since $c_2 > 0$, $J = \frac{1}{2} (c_2\hat{a}_1+\dot{x}_1)^2$ is minimized when $x_1 = 0$, thus reaching an undesirable equilibrium at the boundary; i.e,. saddle equilibria. $\hfill \blacksquare$


% \subsection{Proof of Set Invariance of Mod-DS}\label{section: modulation invariance proof}
% \noindent \textbf{Proof:}
% Proof of the impenetrability of Mod-DS can be found in \cite{LukesDS}. Impenetrability indicates that the velocity of the agent on the boundary of the obstacle $\partial C$ is always 0, following the von Neuman boundary conditions. 

% \begin{definition}[von Neuman Boundary Condition]
% Impenetrability is ensured if there is no velocity in normal direction on the surface of the obstacle:
% \begin{equation}
% \label{von Neuman}
% \hat{n}(x)\dot{x}=0 \quad\; \forall x \in \partial C
% \end{equation}
% \end{definition}

% In 2D Euclidean space, $x \in R^2$, $\nabla_xh(x)$ represents geometrically the direction normal to the obstacle surface from point $x$, thus  $\nabla_xh(x) = \hat{n}(x)$.Because $\dot{x}$ is the velocity of the controlled agent, $\nabla_xh(x)\dot{x}$ is equivalent to the agent's velocity in direction normal to the obstacle surface. Since $\dot{h}(x) = \nabla_xh(x)\dot{x} =\hat{n}\dot{x}$, we show that Mod-DS satisfies

% \begin{equation}
% \dot{h}(x)=\hat{n}(x)\dot{x}=0 \quad\; \forall x \in \partial C
% \end{equation}

% and thus meets the set invariance constraints in Nagumo's theorem in Eq. \eqref{Nagumo}. \hfill $\blacksquare$

\subsection{Metrics for Collision Avoidance Evaluation}\label{section:metrics}
\vspace{-10pt}
\begin{table}[hbp]
\vspace{10pt}
\centering
\caption{Metrics for Collision Avoidance Comparison \cite{zhou2022rocus} \label{metrics}}
{\begin{tabular}{  c | c }
\hline
\rowcolor{Gray} \bf Behavior Metric & \bf Equation \\ \hline
Trajectory Length  &  $l =\int_\tau 1 \,ds $ \\ \hline
Average Jerk & $\Bar{j} = \frac{1}{l}\int_\tau ||\dddot{x}||_2\,ds$\\ \hline
Straight-Line Deviation & $\eta =\frac{1}{l}\int_\tau ||x-\proj_{x^*-x^0}{x}||_2 \,ds$ \\ \hline
Obstacle Clearance & $d_\text{obs} =\frac{1}{l}\int_\tau \min_{x_o \in \neg C}||x-x_o||_2 \,ds$\\ \hline
Near Obstacle Velocity & $v_\text{near} =\frac{\int_\tau ||\dot{x}||/\min_{x_o \in \neg C}||x-x_o||_2 \,ds}{\int_\tau 1/\min_{x_o \in \neg C}||x-x_o||_2 \,ds}$ \\ \hline
\end{tabular}}
\vspace{-10pt}
\end{table}

\subsection{Impenetrability of Speed and Velocity Constraining Mod-DS}
\label{app:proof-imp}
\noindent \textbf{Proof:} 
% The modulation of the Speed-and Velocity-Constraining DS system described in Section \ref{section:speed-modulation} and \ref{section:vel-modulation} conducts weighted obstacle avoidance for obstacle close to the robot using weighted sum methods defined in \autoref{eq:modulation-matrix}. When the robot is on the boundary of the obstacle $\bar{\dot{x}}_{o}=\dot{x}_{o}$. 
% For single-obstacle modulation problems, the initial DS in Eq.\eqref{eq:nominal system} viewed from the obstacle's perspective can be represented as a linear combination of 2 vectors: one parallel to the reference direction $r(x,t)$ or the normal direction $n(x,t)$, the other contained in the hyperplane $H(x,t) = [e_1(x,t),...,e_{d-1}(x,t)]$. 
In \cite{khansari2012dynamical},~\cite{LukesDS}, and \cite{onManifoldMod}, the unconstrained outputs $u_\text{unc}$ from respectively normal, reference and on-manifold Mod-DS approaches are proven to achieve obstacle impenetrability in terms of the von Neuman boundary condition, listed in Eq.~\eqref{eq:Neuman}.
\begin{equation}
\label{eq:Neuman}
    n(x,x_o)^\top u_\text{unc} = 0 \quad \forall x \in \partial C_o
\end{equation}

Given safe controller $u_\text{unc}$ in Eq.~\eqref{eq:Neuman}, we want to show that the constrained solutions $u_\text{c}$ of the QPs in \eqref{eq:qp speed} and \eqref{eq:lp velocity} also satisfies the set invariance condition of impenetrability. In other words, 
\begin{equation}
\label{eq:safe uc}
    n(x,x_o)^\top u_\text{c} \geq 0 \quad \forall x \in \partial C_o
\end{equation}

Since any solutions $u_\text{c}$ from QPs in \eqref{eq:qp speed} and \eqref{eq:lp velocity} must satisfy the safety constraints in \eqref{eq:const safe}, $n(x,x_o)^\top u_\text{c} \geq n(x,x_o)^\top u_\text{unc}$ must be true for all state $x$. This implies that $n(x,x_o)^\top u_\text{c} \geq 0$ if $n(x,x_o)^\top u_\text{unc} = 0$. Therefore, the statement in \eqref{eq:safe uc} always holds. \hfill $\blacksquare$

% \begin{equation}
% \begin{aligned}
%     f(x) &= f_n(x)+f_e(x) = ||f_n(x)||n(x) + ||f_e(x)||e(x)\\
%     &= f_r(x)+f'_e(x) = ||f_r(x)||r(x) + ||f'_e(x)||e(x)
% \end{aligned}
% \end{equation}

% For a point on the boundary, the unconstrained modulation system will always ensure the controlled agent's relative velocity to the obstacle in the reference direction $||f_r(x)||$ to be 0 using Eq.\eqref{eq:standard mod lambda}. From Figure.\ref{fig:n and r}, it can be observed that $||f_r(x)||\geq ||f_n(x)|| \quad \forall x$, thus $||f_r(x)||=0$, implies $||f_n(x)||=0$, thus ensuring that the agent will not penetrate the boundary in the next timestep. When the stretched velocity of the agent with respect to the obstacle $f(x)$ is greater than the speed or limit, the explicit QCLP-solver in Eq.\ref{eq:mod speed} and the QP-solver in Eq.\ref{eq:lp velocity} both solve for an optimal alternative under the safety constraint

% \begin{equation}
% \begin{aligned}
% f^{c}(x) &>min(f_n(x)+\dot{x}_{o}^{n},\dot{x}_{o}^{n})\\
% f^{c}(x) &= f^{c}_n(x)+\dot{x}_{o}^{n}\\
% f^{c}_n(x)&\geq min(f_n(x),0)\\
% \end{aligned}
% \end{equation}

% For any point on the obstacle boundary, the initial modulation proved that $f_n(x)=0$. Under this condition, the solution of the constraining modulation algorithm $f^{c}(x)$ ensures $f^{c}_n(x)\geq 0$. Thus, impenetrability is ensured because the agent's normal velocity relative to the obstacle will either make it static to or move further away from the obstacle.


% \subsection{The Design of Dubin-CBF-QP}\label{section:dubin explain}
% When designing the control barrier functions,  we enforce a safety constraint of $p(x,t) \geq 0$ using $p(x,t)$ as the control barrier function. Standard CBF-QP control in obstacle avoidance, setting safety constraints to be the distance of the robot to the obstacles $p(x,t) = h(x,t)$, cannot be extended to differential-drive robot applications. That is because  differential-drive kinematics is of higher relative degree to $h(x,t)$, i.e. the first order time-derivative of $h(x,t)$ is not dependent on control inputs linear acceleration $u_l$, angular speed $\omega$, and angular acceleration $u_a$. Therefore, designing CBF-QP that contains parameters $v$ and $\omega$ of relative degree 1 to the desired output $u_l$ and $u_a$ is the key in generating QP-solutions for differential drive robot. \cite{diffCBF} proposes CBF $h'(x)$ as the difference between the current minimum distance to the edge of the safety region and the minimum stopping distance of the robot, where $d$ is the lateral displacement of the robot in the road-fixed coordinate and $d_max$ equals to half of the width of the track. 

% \begin{equation}
% p = d_{max} - sign(\dot{d})d - \frac{\dot{d}^2}{2a_{max}}
% \end{equation}

% This type of CBF, considering both robot positions and velocity with respect to the safety boundaries, was demonstrated to be effective for lane keeping in differential drive vehicle racing. The method we discussed in section \ref{section:input-constraints} is a new variant to expand the application of the CBF in differential drive systems into general obstacle avoidance.

% \begin{equation}
% p_s = h(x,t) - \frac{v_n^2}{2a^{\hat{n}}_{max}}
% \end{equation}

% Here, $a^{\hat{n}}_{max}$ is defined as the projection of the maximum allowable linear acceleration vector of the robot $\vec{a}_{max}$ onto the direction $\hat{n}$, normal to the obstacle. For differential drive, the direction of $\vec{a}_{max}$ is parallel to that of the current linear velocity $\vec{v}$ of the robot, and so do $\vec{a}^{\hat{n}}_{max}$ and $\vec{v}_{n}$, which implies

% \begin{equation}
% \begin{aligned}
% |\frac{v_n}{a^{\hat{n}}_{max}}| &= |\frac{vcos\gamma }{a_{max}cos\gamma }|\\
% &= |\frac{v}{a_{max}}|
% \end{aligned}  
% \end{equation}

% % \begin{equation}
% % \begin{aligned}
% % |\frac{v_n}{a^{\hat{n}}_{max}}| &= |\frac{\vec{v}_n }{\vec{a}^{\hat{n}}_{max}} |\\
% % &= \vec{v} / \vec{a}_{max} =v  /a_{max}
% % \end{aligned}  
% % \end{equation}
% $\gamma$ is the angle between $\hat{n}$ and the current orientation of the robot. The CBF can then be simplified to the standard Dubin-CBF expression in Eq.\eqref{eq:dubin CBF}. 

% The standard Dubin-CBF $p_s$ value decreases when the absolute value of robot speed in $\hat{n}$ direction increases, regardless of whether the robot is moving towards or away from the obstacle. The revised Dubin-CBF thus uses $v_n$ instead of $|v_n|$ to account for robot orientation to $\hat{n}$. When the robot is moving away from the closest obstacle $v_n>0$, the value of $p_r'$ increases as robot speed increases. When the robot is moving towards the closest obstacle $v_n<0$, $p_r'$ decreases as robot speed increases. 
% \begin{equation}
% \begin{aligned}
%   p_r' = h(x,t) + \frac{|v|}{2a_{max}}v_n\\
% \end{aligned}
% \end{equation}

% However, the function $p_r$ can be greater than 0 even when the robot is inside the obstacle if $v_n$ value is large enough. Therefore, we introduce a constant parameter $M(v_{max}, a_{max}$ to ensure that set $\{(x, t, v, v_n):p_r(\cdot)>0\}$ is a subset under set $\{(x, t, v, v_n): h(x,t) >0\}$. 

% \begin{equation}
% \begin{aligned}
%   p_r = h(x,t) + \frac{|v|}{2a_{max}}v_n-M(v_{max},a_{max})\\
%   M(v_{max},a_{max}) = \frac{v_{max}^2}{2a_{max}}
% \end{aligned}
% \end{equation}

% In addition to reducing the relative degree of $p(x,t)$ to linear and angular acceleration outputs $u_l, u_a$, the term $\frac{v_n^2}{2a^{\hat{n}}_{max}}$ checks if at each time step the distance between the obstacle boundary and the robot is sufficient such that if the robot decelerates, it can stop before colliding into the obstacle. Thus, the CBF indirectly reduces future infeasibility conditions faced by the QP solver. However, as mentioned in section \ref{sec:validation}, the proposed Dubin-CBF still suffers from occasional occurrences of infeasibility from the QP optimizer. This infeasibility issue happened at a rate of 1.4\% using the revised Dubin-CBF-QP as recorded by the simulations. During simulations and experiments, simple complementary backup policies such as reducing the linear velocity of the robot and rotating it towards $\hat{e}$ directions of the closest obstacle when it is moving toward the nearest obstacle and increasing the linear velocity of the robot and rotating the robot towards $\hat{n}$ direction when it is moving away from the obstacles have been demonstrated to be very efficient in freeing the robot from the infeasible situations encountered and allowing it to replan the trajectories. 

% It is also noteworthy that making plans considering only the nearest obstacle practically leads to a sharp increase in the number of infeasibility issues encountered by the QP solver. Because differential drive systems can easily run into robot states that are collision-avoidance infeasible for the second nearest obstacles. Thus, during the documented Fetch experiment and simulations, we computed CBF for all obstacles in the environment. 

% We show that Dubin-CBF-QP using barrier function $p_s$ and $p_r$ ensure collision avoidance of the generated trajectories by proving the definition of safe set $C$ in Eq.\eqref{safe region} holds for all outputs $u_l$, $u_a$ from the QP Dubin-CBF-QP. \\


% % Define new safe sets $C_s$, $C_r$ given the values of functions $p_s$ and $p_r$, where $z = [x_1, x_2, v, \theta,\omega]$ as defined in \eqref{eq:high order dubin}.

% % \begin{equation}
% % C_s=\{z \in \mathbb{R}^5: p_s(z,t)>0\}
% % \end{equation}
% % \begin{equation}
% % \partial C_s=\{z \in \mathbb{R}^5: p_s(z,t)=0\}
% % \end{equation}

% % \begin{equation}
% % C_r=\{z \in \mathbb{R}^5: p_r(z,t)>0\}
% % \end{equation}
% % \begin{equation}
% % \partial C_r=\{z \in \mathbb{R}^5: p_r(z,t)=0\}
% % \end{equation}

% \begin{theorem}[Set Invariance of $C$ in $p_s$]
% $\forall u_l, u_a$ outputs from the standard Dubin-CBF-QP, $h(x,t)\geq 0$ always holds. 
% \end{theorem}

% \textbf{Proof:} Theoretically, all outputs from a CBF-QP guarantees the value of its barrier function will always be positive (i.e. safe) \cite{ames2019control}. Thus, $\forall u_l, u_a$ outputs from the standard Dubin-CBF-QP, $p_s(\cdot)\geq 0$. Since $\frac{v_n^2}{2a^{\hat{n}}_{max}}$ is always greater or equal to 0, we can deduce that $p_s(\cdot) \leq h(x,t)$ and when $p(\cdot)\geq 0$, $h(x,t) \geq 0$ will also hold. Thus, we have proven the impenetrability of the proposed Dubin-CBF variant for obstacle avoidance by showing that $h(x,t)\geq 0 ;\ \forall u_l, u_a$ filtered by control barrier function $p_s$. \hfill $\blacksquare$\\

% \begin{theorem}[Set Invariance of $C$ in $p_r$]
% $\forall u_l, u_a$ outputs from the revised Dusbin-CBF-QP, $h(x,t)\geq 0$ always holds. 
% \end{theorem}

% \textbf{Proof:} $\forall u_l, u_a$ outputs from the standard Dubin-CBF-QP, $p_r(\cdot)\geq 0$. 
% \begin{equation}
% \begin{aligned}
%   \frac{|v|}{2a_{max}}v_n &\leq \frac{|vv_n|}{2a_{max}}
%                         < \frac{|v^2|}{2a_{max}}
%                         \leq \frac{v_{max}^2}{2a_{max}}=M
% \end{aligned}
% \end{equation}

% Since $\frac{|v|}{2a_{max}}v_n$ is always upper bounded by $M(v_{max},a_{max})$, we can deduce that $p_r(\cdot) \leq h(x,t)$ and when $p(\cdot)\geq 0$, $h(x,t) \geq 0$ will also hold. Thus, we have proven the impenetrability of the proposed Dubin-CBF variant for obstacle avoidance by showing that $h(x,t)\geq 0 \; \forall u_l, u_a$ filtered by control barrier function $p_r$. \hfill $\blacksquare$
% %  \begin{figure*}
% %      \centering
% %      \begin{subfigure}[b]{0.19\linewidth}
% %          \centering
% %          \includegraphics[trim={2cm 0.0cm 2cm 0cm},clip,width=\linewidth]{images/multi/CBF multi all streamplot x.png}
% %          \caption{CBF All $\alpha(z)=z$}
% %          \label{fig:cbf multi all x}
% %      \end{subfigure}
% %       \begin{subfigure}[b]{0.19\linewidth}
% %          \centering
% %          \includegraphics[trim={2cm 0.0cm 2cm 0cm},clip,width=\linewidth]{images/multi/CBF multi all streamplot 5x.png}
% %          \caption{CBF $\alpha(z)=5z$}
% %          \label{fig:cbf multi all 5x}
% %      \end{subfigure}
% %      \begin{subfigure}[b]{0.19\linewidth}
% %          \centering
% %          \includegraphics[trim={2cm 0.0cm 2cm 0cm},clip,width=\linewidth]{images/multi/CBF multi all streamplot x2.png}
% %          \caption{CBF $\alpha(z)=z^2$}
% %          \label{fig:cbf multi all x2}
% %      \end{subfigure}
% %      \begin{subfigure}[b]{0.19\linewidth}
% %          \centering
% %          \includegraphics[trim={2cm 0.0cm 2cm 0cm},clip,width=\linewidth]{images/multi/CBF multi all streamplot exp.png}
% %          \caption{CBF $\alpha(z) = e^z-1$}
% %          \label{fig:cbf mutli all exp}
% %      \end{subfigure}
% %      \begin{subfigure}[b]{0.19\linewidth}
% %          \centering \includegraphics[trim={2cm 0.0cm 2cm 0cm},clip,width=\linewidth]{images/multi/ModDS multi all streamplot.png}
% %          \caption{Mod-DS}
% %          \label{fig:modulation multi all}
% %      \end{subfigure}
% %      \begin{subfigure}[b]{0.19\linewidth}
% %          \centering
% %          \includegraphics[trim={2cm 0.0cm 2cm 0cm},clip,width=\linewidth]{images/multi/CBF multi closet streamplot z.png}
% %          \caption{CBF $\alpha(z)=z$}
% %          \label{fig:cbf multi closest x}
% %      \end{subfigure}
% %      \begin{subfigure}[b]{0.19\linewidth}
% %          \centering
% %          \includegraphics[trim={2cm 0.0cm 2cm 0cm},clip,width=\linewidth]{images/multi/CBF multi closet streamplot 5x.png}
% %          \caption{CBF $\alpha(z)=5z$}
% %          \label{fig:cbf multi closest 5x}
% %      \end{subfigure}
% %      \begin{subfigure}[b]{0.19\linewidth}
% %          \centering
% %          \includegraphics[trim={2cm 0.0cm 2cm 0cm},clip,width=\linewidth]{images/multi/CBF multi closet streamplot z2.png}
% %          \caption{CBF $\alpha(z)=z^2$}
% %          \label{fig:cbf multi closest x2}
% %      \end{subfigure}
% %      \begin{subfigure}[b]{0.19\linewidth}
% %          \centering
% %          \includegraphics[trim={2cm 0.0cm 2cm 0cm},clip,width=\linewidth]{images/multi/CBF multi closet streamplot exp.png}
% %          \caption{CBF $\alpha(z) = e^z-1$}
% %          \label{fig:cbf multi closest exp}
% %      \end{subfigure}
% %       \begin{subfigure}[b]{0.19\linewidth}
% %          \centering
% %          \includegraphics[trim={2cm 0.0cm 2cm 0cm},clip,width=\linewidth]{images/multi/ModDS multi closest streamplot.png}
% %          \caption{Mod-DS}
% %          \label{fig:modulation multi closest}
% %      \end{subfigure}
% %       \caption{Performance of different obstacle avoidance methods' facing multi star-shaped obstacles. The first four columns correspond to CBF-QP with different $\mathcal{K}_{\infty}$ $\alpha(z)$ functions. The last column corresponds to Mod-DS. Methods in the first row consider all obstacles while methods in the second row plan based on only the closest obstacle to the selected locations. The colors on the trajectories indicate the ratio of the agent's modified speed to its nominal speed ($\frac{||x_{\text{mod}}||_2}{||x_{\text{nom}}||_2}$) as defined in Eq. \eqref{nominal system}.}
% % \label{fig:comparison multi}
% % \end{figure*}

% \subsection{Static Multi-Obstacle Comparison}
% In complement to the qualitative and qunantitative analysis for static single obstacle avoidance in section \ref{section:qualitative-geometry} and \ref{section:quantitative-geometry}, we repeat the same procedures for static multi-obstacle avoidance. Like that in single-obstacle avoidance analysis, both Mod-DS and CBF-QP was implemented in Python and evaluated on 4 trajectories starting respectively at points $[9,17],[11,15],[15,11],[17,9]$ to avoid 3 star-shaped obstacles using both strategies of considering all obstacles in the environment and planning based on only the closest obstacle to the robot. In Mod-DS, when multi-obstacles are considered by the robot when planning trajectories, a weighted sum of Modulation matrix $M(x)$ obtained for the individual obstacle in the environment is calculated. See \cite{LukesDS} for more details. 

% Whether the closest obstacle or multiple obstacles are considered by CBF-QP, as observed in Fig.\ref{fig:comparison multi}, qualitative conclusions made in section \ref{section:qualitative-geometry} regarding parameterization of alpha functions in CBF-QP remain valid. The larger $\alpha(z)$ is given input $z=h(x)$, the smaller the lower bound of $\dot{h}(x)$ is, which allows for a higher degree of preservation to the nominal trajectories. Quantitative characteristics of CBF-QP and Mod-DS observed in section \ref{section:quantitative-geometry} are still noticeable in Table \ref{table:static multi table}. Mod-DS's weakness in high demands for jerk $j$ and risky high near-obstacle velocities $v_\text{near}$ persists in multi-obstacle environments, regardless of the strategy deployed, and so is CBF-QP with $\alpha(z)=z$'s trade off between producing trajectories with low jerk $j$ while longer length $l$. Additionally, the planning strategy of considering the closest obstacle to the robot is computed to result in a higher average in jerk $j$ required by the trajectories compared to that using all obstacles in the environment. The remaining behavior properties of the 2 variants stay close to one another. 


% \begin{table*}[!tbp]
% \centering
% \begin{minipage}[t]{0.95\textwidth}
% \begin{center}
%  \resizebox{\linewidth}{!}{\begin{tabular}{  m{1.55cm} | m{2.75cm} | m{1.7cm}| m{2.5 cm} | m{1.7cm} | m{1.7cm}| m{1.8cm} | m{1.7cm} | m{1.5cm} }
%   \hline
%   \rowcolor{Gray} Strategy & Method & $l$ (std) & $\Bar{j}$ (std) & $d_{obs}$ (std) & $v_{\text{near}}$ (std) &  $\beta$ (std)  & Runtime & Success \% \\ 
%   \hline
%   Closest & Modulation-DS & 19.3 (0.8) & \cellcolor{pink} 26000 (2000) & 2.0 (0.3) & \cellcolor{pink}27 (6) &	0.5 (0.1) & \cellcolor{LightCyan}0.0003 &	1\\
%   \hline
%    & CBF-QP $\alpha=z$ & \cellcolor{pink} 21 (1) & \cellcolor{LightCyan}22000 (17000) &	2.5 (0.5) & 17 (2) &	\cellcolor{pink}1.5 (1.0) & 0.0026 & 1\\
%   \hline
%    & CBF-QP $\alpha=10z$ & 18.9 (0.6) & 15000 (3000) & 2.0 (0.3) & 19 (3) & 0.5 (0.2) &	0.0026 & 1\\
%   \hline
%   All & Modulation-DS & 19.2 (0.8) & \cellcolor{pink} 23000 (1000) & 2.0 (0.3) & \cellcolor{pink}27 (6) & 0.5 (0.2) & \cellcolor{LightCyan} 0.0005 & 1\\

%   \hline
%    & CBF-QP $\alpha=z$ & \cellcolor{pink}20 (2) & \cellcolor{LightCyan}2100 (400) & 2.4 (0.7) & 9 (3) &	\cellcolor{pink}1.4 (1.0) &  0.0026 & 1\\
%   \hline
%   & CBF-QP $\alpha=10z$ & 18.9 (0.6) & 15000 (4000) & 2.0 (0.3) & 19 (3) & 0.5 (0.2) & 0.0027 & 1\\
%   \hline
% \end{tabular}}
% \caption{Performance of Mod-DS, and CBF-QP in static multi-obstacle avoidance at a controlled update frequency of 100Hz.}
% \label{table:static multi table}
% \end{center}
% \end{minipage}
% \end{table*}
\end{document}