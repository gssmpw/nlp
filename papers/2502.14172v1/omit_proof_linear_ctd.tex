\subsection{Linear-Categorical Parametrization is an Isometry}\label{appendix:linear_cate_isometric}
\begin{proposition}\label{prop:linear_cate_isometric}
The affine space $\prn{\sP^{\sgn}_{\bphi,K}, \ell_{2,\mu_{\pi}}}$ is isometric with  $\prn{\RB^{dK}, \sqrt{\iota_K}\norm{\cdot}_{\prn{\bC^{\intercal}\bC}\otimes\bSigma_{\bphi}}}$, in the sense that, for any $\bm{\eta}_{\bw_1},\bm{\eta}_{\bw_2}\in\sP^{\sgn}_{\bphi,K}$, it holds that $\ell_{2,\mu_{\pi}}^2\prn{\bm{\eta}_{\bw_1},\bm{\eta}_{\bw_2}}=\iota_K\norm{\bw_1-\bw_2}^2_{\prn{\bC^{\intercal}\bC}\otimes\bSigma_{\bphi}}$.
\end{proposition}
\begin{proof}
By Proposition~\ref{prop:PK_isometric},
\begin{equation*}
    \begin{aligned}
    \ell_{2,\mu_{\pi}}^2\prn{\bm{\eta}_{\bw_1},\bm{\eta}_{\bw_2}}=&\iota_K\EB_{s\sim\mu_\pi}\brk{\norm{\bC\prn{\bp_{\bw_1}(s)-\bp_{\bw_2}(s)}}^2}\\
    =&\iota_K\tr\prn{\bSigma_{\bphi}^{\frac{1}{2}}\prn{\bW_1-\bW_2}\bC^{\intercal}\bC\prn{\bW_1-\bW_2}^{\intercal}\bSigma_{\bphi}^{\frac{1}{2}}} \\
    =&\iota_K\norm{\bw_1-\bw_2}^2_{\prn{\bC^{\intercal}\bC}\otimes\bSigma_{\bphi}}.
\end{aligned}
\end{equation*}
\end{proof}

\subsection{Linear-Categorical Projection Operator}\label{appendix:linear-cate-project-op}
Proposition~\ref{prop:linear_projection} is an immediate corollary of the following lemma.
\begin{lemma}\label{lem:gradient_cramer_distance}
    For any $\bm{\eta}\in\prn{\sP^{\sgn}}^{\gS}$, $\bw\in\RB^{dK}$ and $s\in\gS$, it holds that
    \begin{align*}
        \nabla_{\bW} \ell_2^2\prn{{\eta}_{\bw}(s),{\eta}(s)}&=2\iota_K\bphi(s)\prn{\bp_{\bw}(s)-\bp_{\bm{\eta}}(s)}^{\intercal}\bC^{\intercal}\bC\\
        &=2\iota_K\bphi(s)\brk{\bphi(s)^{\intercal}\bW+\prn{\frac{1}{K+1}\bm{1}_{K}-\bp_{\bm{\eta}}(s)}^{\intercal}}\bC^{\intercal}\bC.
    \end{align*}
    Furthermore, it holds that
        \begin{align*}
        \nabla_{\bW} \ell_{2,\mu_{\pi}}^2\prn{\bm{\eta}_{\bw},\bm{\eta}}&=\EB_{s\sim\mu_{\pi}}\brk{\nabla_{\bW} \ell_2^2\prn{{\eta}_{\bw}(s),{\eta}(s)}}\\
        &=2\iota_K\brk{\bSigma_{\bphi}\bW+\EB_{s\sim\mu_{\pi}}\brk{\bphi(s)\prn{\frac{1}{K+1}\bm{1}_{K}-\bp_{\bm{\eta}}(s)}^{\intercal}}}\bC^{\intercal}\bC.
    \end{align*}
\end{lemma}
\begin{proof}
According to Proposition~\ref{prop:orthogonal_decomposition}, one has
\begin{align*}
    \ell_2^2\prn{{\eta}_{\bw}(s),{\eta}(s)}=\ell_2^2\prn{{\eta}_{\bw}(s),\bPi_K{\eta}(s)}+\ell_2^2\prn{\bPi_K{\eta}(s),{\eta}(s)}.
\end{align*}
Hence,
\begin{align*}
        \nabla_{\bw} \ell_2^2\prn{{\eta}_{\bw}(s),{\eta}(s)}&=\nabla_{\bw}\ell_2^2\prn{{\eta}_{\bw}(s),\bPi_K{\eta}(s)}\\
        &=\iota_K\nabla_{\bw}\norm{\bC\prn{\bp_{\bw}(s)-\bp_{\bm{\eta}}(s)}}^2\\
        &=2\iota_K\prn{\bI_K\otimes\bphi(s)}\bC^{\intercal}\bC\prn{\bp_{\bw}(s)-\bp_{\bm{\eta}}(s)}\\
        &=2\iota_K\prn{\bI_K\otimes\bphi(s)}\bC^{\intercal}\bC\prn{\prn{\bI_K\otimes\bphi(s)}^{\intercal}\bw+\frac{1}{K+1}\bm{1}_{K}-\bp_{\bm{\eta}}(s)}\\
        &=2\iota_K \brc{ \brk{\prn{\bC^{\intercal}\bC}\otimes\prn{\bphi(s)\bphi(s)^{\intercal}}}\bw+  
 \brk{\prn{\bC^{\intercal}\bC\prn{\frac{1}{K+1}\bm{1}_{K}-\bp_{\bm{\eta}}(s)}}\otimes \bphi(s)}},
    \end{align*}
where in the second equality, we used Proposition~\ref{prop:PK_isometric}, and in the last equality, we used Lemma~\ref{lem:vector_outer_KP}.
We also have the following matrix representation:
\begin{align*}
        \nabla_{\bW} \ell_2^2\prn{{\eta}_{\bw}(s),{\eta}(s)}&=2\iota_K\bphi(s)\prn{\bp_{\bw}(s)-\bp_{\bm{\eta}}(s)}^{\intercal}\bC^{\intercal}\bC\\
        &=2\iota_K\bphi(s)\brk{\bphi(s)^{\intercal}\bW+\prn{\frac{1}{K+1}\bm{1}_{K}-\bp_{\bm{\eta}}(s)}^{\intercal}}\bC^{\intercal}\bC.
    \end{align*}
\end{proof}
\begin{proposition}\label{prop:orthogonal_decomposition_linear_approximation}
For any $\bm{\eta}\in\prn{\sP^{\sgn}}^{\gS}$ and $\bm{\eta}_{\bw}\in\sP^{\sgn}_{\bphi,K}$, it holds that
\begin{equation*}
    \ell_{2,\mu_{\pi}}^2\prn{\bm{\eta},\bm{\eta}_{\bw}}=\ell_{2,\mu_{\pi}}^2\prn{\bm{\eta},\bPi_{K}\bm{\eta}}+\ell_{2,\mu_{\pi}}^2\prn{\bPi_{K}\bm{\eta},\bPi_{\bphi, K}^{\pi}\bm{\eta}}+\ell_{2,\mu_{\pi}}^2\prn{\bPi_{\bphi, K}^{\pi}\bm{\eta},\bm{\eta}_{\bw}}.
\end{equation*}
\end{proposition}
The proof is straightforward and almost the same as that of Proposition~\ref{prop:orthogonal_decomposition} if we utilize the affine structure.

\subsection{Linear-Categorical Projected Bellman Equation}\label{subsection:proof_linear_cate_TD_equation}
To derive the result, the following proposition characterizing $\bPi_{\bphi, K}^{\pi}\gT^{\pi}\bm{\eta}_{\bw}$ is useful, whose proof can be found in Appendix~\ref{appendix:proof_Pi_K_T}.
\begin{proposition}\label{prop:Pi_K_T}
    For any $\bw\in\RB^{dK}$ and $s\in\gS$, we abbreviate $\bp_{\gT^{\pi}\bm{\eta}_\bw}(s)$ as $\tilde{\bp}_{\bw}(s)$, then
    \begin{equation*}
    \begin{aligned}
             \tilde{\bp}_{\bw}(s)=&\prn{\tilde{p}_k(s;\bw)}_{k=0}^{K-1}=\EB\brk{\tilde{\bG}(r_0) \bW^{\intercal} \bphi(s_1)\Big| s_0=s }+\frac{1}{K+1}\sum_{j=0}^K\EB\brk{\bg_j(r_0) \Big| s_0=s },
    \end{aligned}
    \end{equation*}
\end{proposition}
Combining this proposition with Proposition~\ref{prop:linear_projection}, we know that $\bW^{\star}$ is the unique solution to the following system of linear equations for $\bW\in\RB^{d\times K}$
\begin{equation*}
    \begin{aligned}
    \bW=&\bSigma_{\bphi}^{-1}\EB_{s\sim\mu_{\pi}}\brk{\bphi(s)\prn{\tilde\bp_{\bw}(s)-\frac{1}{K+1}\bm{1}_{K}}^{\intercal}}\\
    =&\bSigma_{\bphi}^{-1}\EB_{s\sim\mu_{\pi}}\brk{\bphi(s)\prn{\frac{1}{K+1}\bm{1}_{K}-\EB\brk{\tilde{\bG}(r_0) \bW^{\intercal} \bphi(s_1)\Big| s_0=s }-\frac{1}{K+1}\sum_{j=0}^K\EB\brk{\bg_j(r_0) \Big| s_0=s }}^{\intercal}}\\
    =&\bSigma_{\bphi}^{-1}\EB_{s\sim\mu_{\pi}}\brk{\bphi(s)\bphi(s^\prime)^{\intercal}\bW\tilde{\bG}^{\intercal}(r)}+   \frac{1}{K+1}\bSigma_{\bphi}^{-1}\EB_{s\sim\mu_{\pi}}\brk{\bphi(s)\prn{\sum_{j=0}^K\bg_j(r)-\bm{1}_{K}}^{\intercal}},
    \end{aligned}
\end{equation*}
or equivalently,
\begin{equation*}
    \begin{aligned}
    \bSigma_{\bphi}\bW-\EB_{s\sim\mu_{\pi}}\brk{\bphi(s)\bphi(s^\prime)^{\intercal}\bW\tilde{\bG}^{\intercal}(r)}=   \frac{1}{K+1}\EB_{s\sim\mu_{\pi}}\brk{\bphi(s)\prn{\sum_{j=0}^K\bg_j(r)-\bm{1}_{K}}^{\intercal}},
    \end{aligned}
\end{equation*}
which is the desired conclusion.
The uniqueness and existence of the solution is guaranteed by the fact that the LHS is an invertible linear transformation of $\bW$, which is justified by Eqn.~\eqref{eq:bar_A_lower_bound}.

\subsection{Proof of Proposition~\ref{prop:Pi_K_T}}\label{appendix:proof_Pi_K_T}
\begin{proof}
Recall the definition of the distributional Bellman operator Eqn.~\eqref{eq:distributional_Bellman_equation} and categorical projection operator Eqn.~\eqref{eq:categorical_prob}, we have
\begin{equation}\label{eq:eq_in_proof_Pi_K_T}
    \begin{aligned}
\tilde{p}_k(s;\bw)=& p_k\prn{\brk{\gT^{\pi}\bm{\eta}_\bw}(s)} \\
=&\EB_{X\sim \brk{\gT^{\pi}\bm{\eta}_\bw}(s)}\brk{\prn{1-\abs{\frac{X-x_k}{\iota_K}}}_+}\\
=&\EB\brk{\EB_{G\sim \eta_{\bw}(s_1)}\brk{\prn{1-\abs{\frac{r_0+\gamma G-x_k}{\iota_K}}}_+}\Big| s_0=s }\\
=&\EB\brk{\sum_{j=0}^Kp_j(s_1;
\bw)\prn{1-\abs{\frac{r_0+\gamma x_j-x_k}{\iota_K}}}_+\Big| s_0=s }\\
=&\EB\brk{\sum_{j=0}^Kp_j(s_1;
\bw)g_{j,k}(r_0)\Big| s_0=s }\\
=&\EB\brk{g_{K,k}(r_0)+\sum_{j=0}^{K-1}p_j(s_1;
\bw)\prn{g_{j,k}(r_0)-g_{K,k}(r_0)}\Big| s_0=s }.
\end{aligned}
\end{equation}

Hence,
\begin{align*}
\tilde{\bp}_{\bw}(s)=& \prn{\tilde{p}_k(s;\bw)}_{k=0}^{K-1} \\
=&\EB\brk{\begin{bmatrix}
g_{K,1}(r_0) \\
\vdots \\
g_{K,{K-1}}(r_0)
\end{bmatrix}+\sum_{j=0}^{K-1}p_j(s_1;
\bw)\begin{bmatrix}
g_{j,1}(r_0)-g_{K,1}(r_0) \\
\vdots \\
g_{j,K-1}(r_0)-g_{K,K-1}(r_0)
\end{bmatrix}\Bigg| s_0=s}\\ 
=&\EB\brk{\bg_K(r_0)+\sum_{j=0}^{K-1}p_j(s_1;
\bw)\prn{\bg_j(r_0)-\bg_K(r_0)}    \Big| s_0=s }\\
=&\EB\brk{\bg_K(r_0)+\prn{\bG(r_0)-\bm{1}_K^{\intercal}\otimes\bg_K(r_0)}  \bp_{\bw}(s_1)  \Big| s_0=s }\\
=&\EB\brk{\bg_K(r_0)+\prn{\bG(r_0)-\bm{1}_K^{\intercal}\otimes\bg_K(r_0)}  \brk{\prn{\bI_K\otimes\bphi(s_1)}^{\intercal}\bw+\frac{1}{K+1}\bm{1}_{K}}  \Big| s_0=s }\\
=&\EB\brk{\prn{\bG(r_0)-\bm{1}_K^{\intercal}\otimes\bg_K(r_0)}  \prn{\bI_K\otimes\bphi(s_1)}^{\intercal} \Big| s_0=s }\bw\\
&\quad +\EB\brk{\bg_K(r_0)+\frac{1}{K+1}\prn{\bG(r_0)-\bm{1}_K^{\intercal}\otimes\bg_K(r_0)}  \bm{1}_{K} \Big| s_0=s }\\
% =&\EB\brk{\prn{\bG(r_0)-\bm{1}_K^{\intercal}\otimes\bg_K(r_0)}  \prn{\bI_K\otimes\bphi(s_1)}^{\intercal} \Big| s_0=s }\bw+\frac{1}{K+1}\sum_{j=0}^K\EB\brk{\bg_j(r_0) \Big| s_0=s }.
=&\EB\brk{\prn{\bG(r_0)-\bm{1}_K^{\intercal}\otimes\bg_K(r_0)} \otimes \bphi(s_1)^{\intercal}\Big| s_0=s }\bw+\frac{1}{K+1}\sum_{j=0}^K\EB\brk{\bg_j(r_0) \Big| s_0=s },
\end{align*}
or equivalently,
    \begin{equation*}
    \begin{aligned}
             \tilde{\bp}_{\bw}(s)=\EB\brk{\tilde{\bG}(r_0) \bW^{\intercal} \bphi(s_1)\Big| s_0=s }+\frac{1}{K+1}\sum_{j=0}^K\EB\brk{\bg_j(r_0) \Big| s_0=s }.
    \end{aligned}
    \end{equation*}
\end{proof}

\subsection{Proof of Proposition~\ref{prop:approx_error}}\label{appendix:proof_approx_error}
\begin{proof}
By the basic inequality (Lemma~\ref{lem:prob_basic_inequalities}), we only need to show
   \begin{equation*}
    \begin{aligned}
        \ell_{2,\mu_{\pi}}^2\prn{\bm{\eta}^\pi,\bm{\eta}_{\bw^{\star}}}\leq&\frac{\ell_{2,\mu_{\pi}}^2\prn{\bm{\eta}^{\pi},\bPi_{\bphi, K}^{\pi}\bm{\eta}^{\pi}}}{1-\gamma}\\
        =&\frac{\ell_{2,\mu_{\pi}}^2\prn{\bm{\eta}^{\pi},\bPi_{K}\bm{\eta}^{\pi}}+\ell_{2,\mu_{\pi}}^2\prn{\bPi_{K}\bm{\eta}^{\pi},\bPi_{\bphi, K}^{\pi}\bm{\eta}^{\pi}}}{1-\gamma}\\
       \leq& \frac{1}{K(1-\gamma)^2}+\frac{\ell_{2,\mu_{\pi}}^2\prn{\bPi_K\bm{\eta}^{\pi},\bPi_{\bphi, K}^{\pi}\bm{\eta}^{\pi}}}{1-\gamma},
    \end{aligned}
\end{equation*}
where we used \citep[Proposition~9.18 and Eqn.~(5.28)][]{bdr2022}.
\end{proof}

\subsection{Cumulative Distribution Function Representation}\label{Appendix:cdf_representation}
\subsubsection{Categorical Parametrization}
For any $\nu\in\sP^{\sgn}_K$, we denote 
\begin{equation*}
    \bF_{\nu}=\prn{F_k\prn{\nu}}_{k=0}^{K-1}=\prn{\nu\prn{[0, x_k]}}_{k=0}^{K-1}\in\RB^{K}
\end{equation*}
as the cumulative distribution function (CDF) representation of $\nu$.
One can check that $\bF_{\nu}=\bC\bp_{\nu}$.
As a result, PMF and CDF representations are equivalent because $\bC$ is invertible.

For any $\nu_1,\nu_2\in\sP^{\sgn}_K$,
\begin{equation*}
    \begin{aligned}
    \ell_2^2(\nu_1,\nu_2)=&\iota_K\norm{\bF_{\nu_1}-\bF_{\nu_2}}^2,
\end{aligned}
\end{equation*}
therefore, the affine space $\prn{\sP^{\sgn}_K, \ell_2}$ is isometric with the Euclidean space $\prn{\RB^{K}, \sqrt{\iota_K}\norm{\cdot}_2}$ if we consider the CDF representation.

For any ${\bm{\eta}}\in\prn{\sP^{\sgn}_K}^\gS$, $s\in\gS$, we abuse the notation to define $\bF_{\bm{\eta}}(s):=\bF_{\eta(s)}$.
By Eqn.~\eqref{eq:categorical_prob}, we can also similarly define $\bF_\nu, \bF_{\bm{\eta}}(s)$ using the categorical projection operator for any $\nu\in \sP^{\sgn}$, $\bm{\eta}\in \prn{\sP^{\sgn}}^{\gS}$ and $s\in\gS$.

\subsubsection{Linear-Categorical Parametrization}
We introduce new notations for the CDF representation when we consider linear-categorical parametrization.
Let $\bTheta:=\bW \bC^{\intercal}=\prn{\sum_{k=0}^0\bw(k), \sum_{k=0}^1\bw(k), \cdots,\sum_{k=0}^K\bw(k)}\in\RB^{d\times K}$ and the vectorization of $\bTheta$, $\btheta:=\vect\prn{\bTheta}=\prn{\bC\otimes\bI_d}\bw\in\RB^{dK}$.
We abbreviate $\bF_{\bm{\eta}_\bw}$ as $\bF_{\btheta}$, then by Lemma~\ref{lem:vec_and_KP} and Lemma~\ref{lem:vector_outer_KP}, for any $s\in\gS$, it holds that
\begin{equation}\label{eq:CDF_linear_parametrization}
    \bF_{\btheta}(s)=\bC\bp_{\bw}(s)=\bTheta^{\intercal}\bphi(s)+\frac{1}{K+1}\bC\bm{1}_{K}=\prn{\bI_K\otimes\bphi(s)}^{\intercal}\btheta+\frac{1}{K+1}\bC\bm{1}_{K},
\end{equation}
Again, PMF and CDF representations are equivalent because $\bC$ is invertible.

For any $\bm{\eta}_{\bw_1},\bm{\eta}_{\bw_2}\in\sP^{\sgn}_{\bphi,K}$, by Proposition~\ref{prop:PK_isometric},
\begin{equation}\label{eq:CDF_target_function}
    \begin{aligned}
    \ell_{2,\mu_{\pi}}^2\prn{\bm{\eta}_{\bw_1},\bm{\eta}_{\bw_2}}=&\EB_{s\sim\mu_\pi}\brk{\ell_{2}^2\prn{\eta_{\bw_1}(s),\eta_{\bw_2}(s)}}\\
    =&\iota_K\EB_{s\sim\mu_\pi}\brk{\norm{\bF_{\btheta_1}(s)-\bF_{\btheta_2}(s)}^2}\\
    =&\iota_K\tr\prn{\prn{\bTheta_1-\bTheta_2}^{\intercal}\bSigma_{\bphi}\prn{\bTheta_1-\bTheta_2}} \\
    =&\iota_K\norm{\btheta_1-\btheta_2}^2_{\bI_K\otimes\bSigma_{\bphi}},
\end{aligned}
\end{equation}
hence the affine space $\prn{\sP^{\sgn}_{\bphi,K}, \ell_{2,\mu_{\pi}}}$ is isometric with the Euclidean space $\prn{\RB^{Kd}, \sqrt{\iota_K}\norm{\cdot}_{\bI_K\otimes\bSigma_{\bphi}}}$ if we consider the CDF representation.

Following the proof of Lemma~\ref{lem:gradient_cramer_distance} in Appendix~\ref{appendix:linear-cate-project-op}, we can also derive the gradient when we use the CDF parametrization:
\begin{align*}
        \nabla_{\btheta} \ell_2^2\prn{{\eta}_{\bw}(s),{\eta}(s)}&=\iota_K\nabla_{\btheta}\norm{\bF_{\btheta}(s)-\bF_{\bm{\eta}}(s)}^2\\
        &=2\iota_K\prn{\bI_K\otimes\bphi(s)}\prn{\bF_{\btheta}(s)-\bF_{\bm{\eta}}(s)}\\
        &=2\iota_K\prn{\bI_K\otimes\bphi(s)}\prn{\prn{\bI_K\otimes\bphi(s)}^{\intercal}\btheta+\bC\prn{\frac{1}{K+1}\bm{1}_{K}-\bp_{\bm{\eta}}(s)}}\\
        &=2\iota_K \brc{ \brk{\bI_K\otimes\prn{\bphi(s)\bphi(s)^{\intercal}}}\btheta+  
 \brk{\prn{\bC\prn{\frac{1}{K+1}\bm{1}_{K}-\bp_{\bm{\eta}}(s)}}\otimes \bphi(s)}},
    \end{align*}
\begin{equation}\label{eq:gradient_cdf_representation}
            \begin{aligned}
        \nabla_{\bTheta} \ell_2^2\prn{{\eta}_{\bw}(s),{\eta}(s)}&=2\iota_K\bphi(s)\prn{\bF_{\btheta}(s)-\bF_{\bm{\eta}}(s)}^{\intercal}\\
        &=2\iota_K\bphi(s)\brk{\bphi(s)^{\intercal}\bTheta+\prn{\frac{1}{K+1}\bm{1}_{K}-\bp_{\bm{\eta}}(s)}^{\intercal}\bC^{\intercal}}.
    \end{aligned}
    \end{equation}

\subsection{Stochastic Semi-Gradient Descent with Linear Function Approximation}\label{appendix:equiv_ssgd_lctd}
We use the notations in the generative model setting for simplicity.
We denote by ${\gT}_t^\pi$ the corresponding empirical distributional Bellman operator at the $t$-th iteration, which satisfies
\begin{equation}\label{eq:empirical_op}
    \brk{\gT_t^\pi{\bm{\eta}}}(s_{t})=(b_{r_{t},\gamma})_\#(\eta(s_{t}^\prime)),\quad\forall\bm{\eta}\in\sP^{\gS}.
\end{equation}
\subsubsection{PMF Representation}
Consider the SSGD with the PMF representation
\[
    \bW_t=\bW_{t-1}-\alpha\nabla_{\bW}\ell_2^2\prn{{\eta}_{\bw_{t-1}}(s_t), \brk{\gT_t^\pi{\bm{\eta}_{\bw_{t-1}}}}(s_{t})},
\]
where $\nabla_{\bW}$ stands for taking gradient w.r.t.\ $\bW_{t-1} \in \RB^{d\times K}$ in the first term ${\eta}_{\bw_{t-1}}(s_t)$ (the second term is regarding as a constant, that's why we call it a semi-gradient). 
One can check that $\nabla_{\bW}\ell_2^2\prn{{\eta}_{\bw_{t-1}}(s_t), \brk{\gT_t^\pi{\bm{\eta}_{\bw_{t-1}}}}(s_{t})}$ is an unbiased estimate of $\nabla_{\bW} \ell_{2,\mu_{\pi}}^2\prn{\bm{\eta}_{\bw_{t-1}},\gT^\pi{\bm{\eta}_{\bw_{t-1}}}}$.

Now, let's compute the gradient term.
By Lemma~\ref{lem:gradient_cramer_distance}, we have
    \begin{align*}
        \nabla_{\bW} \ell_2^2\prn{{\eta}_{\bw_{t-1}}(s_t),\brk{\gT_t^\pi{\bm{\eta}_{\bw_{t-1}}}}(s_{t})}&=2\iota_K\bphi(s_t)\prn{\bp_{\bw_{t-1}}(s_t)-\bp_{\gT_t^\pi{\bm{\eta}_{\bw_{t-1}}}}(s_t)}^{\intercal}\bC^{\intercal}\bC\\
        &=2\iota_K\bphi(s_t)\brk{\bphi(s_t)^{\intercal}\bW_{t-1}+\prn{\frac{1}{K+1}\bm{1}_{K}-\bp_{\gT_t^\pi{\bm{\eta}_{\bw_{t-1}}}}(s_t)}^{\intercal}}\bC^{\intercal}\bC.
    \end{align*}
where $\bp_{\gT_t^\pi{\bm{\eta}_{\bw_{t-1}}}}(s_t)=\bp_{\bPi_K\gT_t^\pi{\bm{\eta}_{\bw_{t-1}}}(s_t)}=\prn{p_k\prn{\brk{\gT_t^\pi{\bm{\eta}_{\bw_{t-1}}}}(s_t)}}_{k=0}^{K-1}\in\RB^{K}$.
Now, we turn to compute $\bp_{\gT_t^\pi{\bm{\eta}_{\bw_{t-1}}}}(s_t)$.
According to Eqn.~\eqref{eq:categorical_prob},
    \begin{align*}
        p_k\prn{\brk{\gT_t^\pi{\bm{\eta}_{\bw_{t-1}}}}(s_t)}=&\EB_{X\sim \brk{\gT_t^\pi{\bm{\eta}_{\bw_{t-1}}}}(s_t)}\brk{\prn{1-\abs{\frac{X-x_k}{\iota_K}}}_+}\\
        =&\EB_{G\sim \eta_{\bw_{t-1}}(s_t^\prime)}\brk{\prn{1-\abs{\frac{r_t+\gamma G-x_k}{\iota_K}}}_+}\\
=&\sum_{j=0}^Kp_j(s_t^\prime;
\bw_{t-1})g_{j,k}(r_t)\\
=&g_{K,k}(r_t)+\sum_{j=0}^{K-1}p_j(s_t^\prime;
\bw_{t-1})\prn{g_{j,k}(r_t)-g_{K,k}(r_t)},
    \end{align*}
which has the same form as Eqn.~\eqref{eq:eq_in_proof_Pi_K_T}.
Following the proof of Proposition~\ref{prop:Pi_K_T} in Appendix~\ref{appendix:proof_Pi_K_T}, one can show that
\begin{equation}\label{eq:project_bellman_w}
    \begin{aligned}
             \bp_{\gT_t^\pi{\bm{\eta}_{\bw_{t-1}}}}(s_t)=\tilde{\bG}(r_t) \bW^{\intercal}_{t-1} \bphi(s_t^\prime)+\frac{1}{K+1}\sum_{j=0}^K\bg_j(r_t).
    \end{aligned}
\end{equation}
Hence, the update scheme is
\begin{equation}\label{eq:ssgd_pmf}
    \begin{aligned}
\bW_t=&\bW_{t-1}-2\iota_K\alpha\bphi(s_t)\prn{\bp_{\bw_{t-1}}(s_t)-\bp_{\gT_t^\pi{\bm{\eta}_{\bw_{t-1}}}}(s_t)}^{\intercal}\bC^{\intercal}\bC\\
=&\bW_{t-1}-2\iota_K\alpha\bphi(s_t)\brk{\bphi(s_t)^{\intercal}\bW_{t-1}-\bphi(s_t^\prime)^{\intercal}\bW_{t-1}\tilde{\bG}^{\intercal}(r_t)-\frac{1}{K+1}\prn{\sum_{j=0}^K\bg_j(r_t)-\bm{1}_{K}}^{\intercal}}\bC^{\intercal}\bC.
\end{aligned}
\end{equation}
Compared to {\LCTD} (Eqn.~\eqref{eq:linear_CTD}), this form has an additional $\bC^{\intercal}\bC$, and the step size is $2\iota_K\alpha$.

\subsubsection{CDF Representation}
Consider the SSGD with the CDF representation
\[
    \bTheta_t=\bTheta_{t-1}-\alpha\nabla_{\bTheta}\ell_2^2\prn{{\eta}_{\bw_{t-1}}(s_t), \brk{\gT_t^\pi{\bm{\eta}_{\bw_{t-1}}}}(s_{t})},
\]
where $\nabla_{\bTheta}$ stands for taking gradient w.r.t.\ $\bTheta_{t-1}=\bW_{t-1}\bC^{\intercal} \in \RB^{d\times K}$ in the first term ${\eta}_{\bw_{t-1}}(s_t)$ (the second term is regarding as a constant). 
One can check that $\nabla_{\bTheta}\ell_2^2\prn{{\eta}_{\bw_{t-1}}(s_t), \brk{\gT_t^\pi{\bm{\eta}_{\bw_{t-1}}}}(s_{t})}$ is an unbiased estimate of $\nabla_{\bTheta} \ell_{2,\mu_{\pi}}^2\prn{\bm{\eta}_{\bw_{t-1}},\gT^\pi{\bm{\eta}_{\bw_{t-1}}}}$.

Now, let us compute the gradient term.
By Eqn.~\eqref{eq:gradient_cdf_representation} and Eqn.~\eqref{eq:project_bellman_w}, we have
    \begin{align*}
        &\nabla_{\bTheta} \ell_2^2\prn{{\eta}_{\bw_{t-1}}(s_t),\brk{\gT_t^\pi{\bm{\eta}_{\bw_{t-1}}}}(s_{t})}\\
        &\qquad=2\iota_K\bphi(s_t)\prn{\bF_{\btheta_{t-1}}(s_t)-\bF_{\gT_t^\pi{\bm{\eta}_{\bw_{t-1}}}}(s_t)}^{\intercal}\\
        &\qquad=2\iota_K\bphi(s_t)\brk{\bphi(s_t)^{\intercal}\bTheta_{t-1}+\prn{\frac{1}{K+1}\bm{1}_{K}-\bp_{\gT_t^\pi{\bm{\eta}_{\bw_{t-1}}}}(s)}^{\intercal}\bC^{\intercal}}\\
        &\qquad=2\iota_K\bphi(s_t)\brk{\bphi(s_t)^{\intercal}\bTheta_{t-1}-\bphi(s_t^\prime)^{\intercal}\bTheta_{t-1}\bC^{-\intercal}\tilde{\bG}^{\intercal}(r_t)\bC^{\intercal}-\frac{1}{K+1}\prn{\sum_{j=0}^K\bg_j(r_t)-\bm{1}_{K}}^{\intercal}\bC^{\intercal}}.
    \end{align*}
Hence, the update scheme is
\begin{equation}\label{eq:CDF_SSGD_update}
   \begin{aligned}
\bTheta_t=&\bTheta_{t-1}-2\iota_K\alpha\bphi(s_t)\prn{\bF_{\btheta_{t-1}}(s_t)-\bF_{\gT_t^\pi{\bm{\eta}_{\bw_{t-1}}}}(s_t)}^{\intercal}\\
=&\bTheta_{t-1}-2\iota_K\alpha\bphi(s_t)\brk{\bphi(s_t)^{\intercal}\bTheta_{t-1}-\bphi(s_t^\prime)^{\intercal}\bTheta_{t-1}\bC^{-\intercal}\tilde{\bG}^{\intercal}(r_t)\bC^{\intercal}-\frac{1}{K+1}\prn{\sum_{j=0}^K\bg_j(r_t)-\bm{1}_{K}}^{\intercal}\bC^{\intercal}},
\end{aligned} 
\end{equation}
which is equivalent to
\begin{align*}
\bW_t=&\bW_{t-1}-2\iota_K\alpha\bphi(s_t)\prn{\bp_{\bw_{t-1}}(s_t)-\bp_{\gT_t^\pi{\bm{\eta}_{\bw_{t-1}}}}(s_t)}^{\intercal}\\
=&\bW_{t-1}-2\iota_K\alpha\bphi(s_t)\brk{\bphi(s_t)^{\intercal}\bW_{t-1}-\bphi(s_t^\prime)^{\intercal}\bW_{t-1}\tilde{\bG}^{\intercal}(r_t)-\frac{1}{K+1}\prn{\sum_{j=0}^K\bg_j(r_t)-\bm{1}_{K}}^{\intercal}},
\end{align*}
which has the same form as Linear-CTD (Eqn.~\eqref{eq:linear_CTD}) with the step size $2\alpha\iota_K$.



