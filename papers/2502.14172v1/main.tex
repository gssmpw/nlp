\documentclass[11pt]{article}
% \documentclass{article}
% \usepackage{neurips_2024}
%% \usepackage[utf8]{inputenc} % allow utf-8 input
%% \usepackage[T1]{fontenc}    % use 8-bit T1 fonts

%!TEX root=main.tex
\newif\ifspacehack
%\spacehacktrue
\usepackage{natbib}
\hypersetup{
    colorlinks = blue,
    breaklinks,
    linkcolor = blue,
    citecolor = blue,
    urlcolor  = blue,
}
\usepackage{url} 
\usepackage{graphicx}
\usepackage{mathtools}
\usepackage{footnote}
\usepackage{float}
\usepackage{xspace}
\usepackage{multirow}
\usepackage{xcolor}
\usepackage{wrapfig}
\usepackage{framed}
\usepackage{bbm}
\usepackage[most]{tcolorbox}

\usepackage{footnote}
\usepackage{nicefrac}
\usepackage{makecell}
\usepackage[ruled,vlined]{algorithm2e}
\usepackage{amssymb}
\usepackage{bm}
\makesavenoteenv{tabular}
\makesavenoteenv{table}

\newcommand{\hl}[1]{{\color{red}[HL: #1]}}
\newcommand{\mznote}[1]{{\color{blue}[MZ: #1]}}

% macros@Peng
\newcommand\innerp[2]{\langle #1, #2 \rangle}
\renewcommand{\tilde}{\widetilde}
\renewcommand{\hat}{\widehat}


\newcommand{\TVD}[1]{\norm{#1}_\text{TV}}
\newcommand{\corral}{\textsc{Corral}\xspace}
\newcommand{\expthree}{\textsc{Exp3}\xspace}
\newcommand{\expfour}{\ensuremath{\mathsf{Exp4}}\xspace}
\newcommand{\expthreeP}{\textsc{Exp3.P}\xspace}
\newcommand{\scrible}{\textsc{SCRiBLe}\xspace}

\def \R {\mathbb{R}}
\newcommand{\eps}{\epsilon}
\newcommand{\vecc}{\mathrm{vec}}
\newcommand{\LS}{\mathrm{LS}}
\newcommand{\FG}{\mathrm{FG}}
\newcommand{\DL}{\Delta \ellhat}
\newcommand{\calA}{{\mathcal{A}}}
\newcommand{\smax}{{\mathrm{smax}}}
\newcommand{\calB}{{\mathcal{B}}}
\newcommand{\calX}{{\mathcal{X}}}
\newcommand{\calS}{{\mathcal{S}}}
\newcommand{\calF}{{\mathcal{F}}}
\newcommand{\calI}{{\mathcal{I}}}
\newcommand{\calJ}{{\mathcal{J}}}
\newcommand{\calK}{{\mathcal{K}}}
\newcommand{\calH}{{\mathcal{H}}}
\newcommand{\calD}{{\mathcal{D}}}
\newcommand{\calE}{{\mathcal{E}}}
\newcommand{\calG}{{\mathcal{G}}}
\newcommand{\calU}{{\mathcal{U}}}
\newcommand{\calR}{{\mathcal{R}}}
\newcommand{\calT}{{\mathcal{T}}}
\newcommand{\calP}{{\mathcal{P}}}
\newcommand{\calQ}{{\mathcal{Q}}}
\newcommand{\calZ}{{\mathcal{Z}}}
\newcommand{\calM}{{\mathcal{M}}}
\newcommand{\calN}{{\mathcal{N}}}
\newcommand{\calW}{{\mathcal{W}}}
\newcommand{\calY}{{\mathcal{Y}}}
\newcommand{\cD}{{\mathcal{D}_{\mathcal{X}}}}
\newcommand{\mcD}{{\mathcal{D}}}
\newcommand{\cF}{{\mathcal{F}}}
\newcommand{\cA}{{\mathcal{A}}}
\newcommand{\cX}{{\mathcal{X}}}
\newcommand{\cE}{{\mathcal{E}}}
\newcommand{\cV}{{\mathcal{V}}}
\newcommand{\cR}{{\mathcal{R}}}
\newcommand{\wcR}{\widehat{\mathcal{R}}}
\newcommand{\Reg}{{\mathrm{Reg}}}
\newcommand{\Alg}{{\mathsf{Alg}}}
\newcommand{\wReg}{\widehat{\mathrm{Reg}}}
\newcommand{\cB}{\mathcal{B}}
\newcommand{\cP}{\mathcal{P}}
\newcommand{\nctx}{\text{n-ctx}}
\newcommand{\ctx}{\text{ctx}}
\newcommand{\E}{{\mathbb{E}}}
\newcommand{\V}{\mathbb{V}}
\newcommand{\Prob}{\mathbb{P}}
\newcommand{\1}{\mathbb{I}}
\newcommand{\N}{\mathbb{N}}
\newcommand{\tup}[1]{t^{(#1)}}
\newcommand{\gup}[1]{g^{(#1)}}
\newcommand{\hatfm}{\widehat{f}_m}
\newcommand{\haty}{\widehat{y}}
\newcommand{\hatx}{\widehat{x}}
\newcommand{\yhat}{\widehat{y}}
\newcommand{\xhat}{\widehat{x}}
\newcommand{\fhat}{\widehat{f}}
\newcommand{\ghat}{\widehat{g}}

\newcommand{\inner}[1]{ \left\langle {#1} \right\rangle }
\newcommand{\ind}{\mathbb{I}}
\newcommand{\diag}{\textrm{diag}}
\newcommand{\Nout}{N_{\textrm{out}}}
\newcommand{\nout}{N_{\textrm{out}}}
\newcommand{\Nin}{{\textrm{Nin}}}
\newcommand{\nin}{{\textrm{Nin}}}
\newcommand{\order}{\mathcal{O}}


\newcommand{\Acal}{\mathcal{A}}
\newcommand{\Bcal}{\mathcal{B}}
\newcommand{\Ccal}{\mathcal{C}}
\newcommand{\Dcal}{\mathcal{D}}
\newcommand{\Ecal}{\mathcal{E}}
\newcommand{\Fcal}{\mathcal{F}}
\newcommand{\Gcal}{\mathcal{G}}
\newcommand{\Hcal}{\mathcal{H}}
\newcommand{\Ical}{\mathcal{I}}
\newcommand{\Jcal}{\mathcal{J}}
\newcommand{\Kcal}{\mathcal{K}}
\newcommand{\Lcal}{\mathcal{L}}
\newcommand{\Mcal}{\mathcal{M}}
\newcommand{\Ncal}{\mathcal{N}}
\newcommand{\Ocal}{\mathcal{O}}
\newcommand{\Pcal}{\mathcal{P}}
\newcommand{\Qcal}{\mathcal{Q}}
\newcommand{\Rcal}{\mathcal{R}}
\newcommand{\Scal}{\mathcal{S}}
\newcommand{\Tcal}{\mathcal{T}}
\newcommand{\Ucal}{\mathcal{U}}
\newcommand{\Vcal}{\mathcal{V}}
\newcommand{\Wcal}{\mathcal{W}}
\newcommand{\Xcal}{\mathcal{X}}
\newcommand{\Ycal}{\mathcal{Y}}
\newcommand{\Zcal}{\mathcal{Z}}
\newcommand{\wkdn}{d}


\newcommand{\avgR}{\wh{\cal{R}}}
%\newcommand{\ips}{\wh{r}}
\newcommand{\whpi}{\wh{\pi}}
\newcommand{\whE}{\wh{\E}}
\newcommand{\whV}{\wh{V}}

\newcommand{\whReg}{\wh{\text{\rm Reg}}}
\newcommand{\flg}{\text{\rm flag}}
\newcommand{\one}{\boldsymbol{1}}
\newcommand{\var}{\Delta}
\newcommand{\Var}{\mathrm{Var}}
\newcommand{\bvar}{\bar{\Delta}}
\newcommand{\p}{\prime}
\newcommand{\evt}{\textsc{Event}}
\newcommand{\unif}{\text{\rm Unif}}
\newcommand{\KL}{\text{\rm KL}}
\newcommand{\Lstar}{{L^\star}}
\newcommand{\istar}{{i^\star}}
\newcommand{\dynreg}{\text{Dyn-Reg}}
\newcommand{\tildedynreg}{\widetilde{\text{Dyn-Reg}}}
\newcommand{\Bstar}{{B^\star}}
\newcommand{\Ustar}{\rho}
\newcommand{\Aconst}{a}
\newcommand{\dplus}[1]{\bm{#1}}
\newcommand{\lambdamax}{\lambda_\text{\rm max}}
\newcommand{\biasone}{\textsc{Deviation}\xspace}
\newcommand{\bias}{\textsc{Bias-1}\xspace}
\newcommand{\biastwo}{\textsc{Bias-2}\xspace}
\newcommand{\biasthree}{\textsc{Bias-3}\xspace}
\newcommand{\term}[1]{\texttt{Term} ~(#1)\xspace}
\newcommand{\x}{\mathbf{x}}
\newcommand{\errorterm}{\textsc{Error}\xspace}
\newcommand{\Err}[1]{\textsc{Err-Term}(#1)\xspace}
\newcommand{\regnctx}{\textsc{Reg-NCTX}\xspace}
\newcommand{\regterm}{\textsc{Reg-Term}\xspace}
\newcommand{\LTtilde}{\wt{L}_T}
\newcommand{\Bomega}{B_{\Omega}}
\newcommand{\UOB}{UOB-REPS\xspace}
\newcommand{\Holder}{{H{\"o}lder}\xspace}
\newcommand{\dpg}{\dplus{g}}
\newcommand{\dpM}{\dplus{M}}
\newcommand{\dpf}{\dplus{f}}
\newcommand{\dpX}{\dplus{\calX}}
\newcommand{\dpw}{\dplus{w}}
\newcommand{\dpF}{\dplus{F}}
\newcommand{\dpu}{\dplus{u}}
\newcommand{\dpwtilde}{\dplus{\wtilde}}
\newcommand{\dps}{\dplus{s}}
\newcommand{\dpe}{\dplus{e}}
\newcommand{\dpx}{\dplus{x}}
\newcommand{\dpy}{\dplus{y}}
\newcommand{\dpH}{\dplus{H}}
\newcommand{\dpOmega}{\dplus{\Omega}}
\newcommand{\dpellhat}{\dplus{\ellhat}}
\newcommand{\dpell}{\dplus{\ell}}
\newcommand{\dpr}{\dplus{r}}
\newcommand{\dpxi}{\dplus{\xi}}
\newcommand{\dpv}{\dplus{v}}
\newcommand{\dpI}{\dplus{I}}
\newcommand{\dpA}{\dplus{A}}
\newcommand{\dph}{\dplus{h}}
\newcommand{\cprob}{6}
\newcommand{\sigmoid}{\ensuremath{\mathsf{Sigmoid}}\xspace}
\newcommand{\relu}{\ensuremath{\mathsf{ReLU}}\xspace}

\DeclareMathOperator*{\argmin}{argmin}
\DeclareMathOperator*{\argmax}{argmax}
\DeclareMathOperator*{\argsmax}{argsmax}
%\DeclareMathOperator*{\range}{range}
%\DeclareMathOperator*{\mydet}{det_{+}}
%\DeclarePairedDelimiter\abs{\lvert}{\rvert}
%\DeclarePairedDelimiter\bigabs{\big\lvert}{\big\rvert}
\DeclarePairedDelimiter\ceil{\lceil}{\rceil}
%\DeclarePairedDelimiter\floor{\lfloor}{\rfloor}
%\DeclarePairedDelimiter\bigceil{\big\lceil}{\big\rceil}
%\DeclarePairedDelimiter\bigfloor{\big\lfloor}{\big\rfloor}

\newcommand{\field}[1]{\mathbb{#1}}
\newcommand{\fY}{\field{Y}}
\newcommand{\fX}{\field{X}}
\newcommand{\fH}{\field{H}}
\newcommand{\fR}{\field{R}}
\newcommand{\fN}{\field{N}}
\newcommand{\fS}{\field{S}}
\newcommand{\UCB}{{\operatorname{UCB}}}
\newcommand{\LCB}{{\operatorname{LCB}}}
\newcommand{\testblock}{\textsc{EndofBlockTest}\xspace}
\newcommand{\testreplay}{\textsc{EndofReplayTest}\xspace}

\newcommand{\theset}[2]{ \left\{ {#1} \,:\, {#2} \right\} }
% \newcommand{\inner}[1]{ \langle {#1} \rangle }
\newcommand{\inn}[1]{ \langle {#1} \rangle }
\newcommand{\Ind}[1]{ \field{I}_{\{{#1}\}} }
\newcommand{\eye}[1]{ \boldsymbol{I}_{#1} }
\newcommand{\norm}[1]{\left\|{#1}\right\|}
%\newcommand{\trace}[1]{\text{tr}\left({#1}\right)}
\newcommand{\trace}[1]{\textsc{tr}({#1})}


\newcommand{\defeq}{\stackrel{\rm def}{=}}
\newcommand{\sgn}{\mbox{\sc sgn}}
\newcommand{\scI}{\mathcal{I}}
\newcommand{\scO}{\mathcal{O}}
\newcommand{\scN}{\mathcal{N}}
\newcommand{\msmwc}{\textsc{MsMwC}}

\newcommand{\dt}{\displaystyle}
\renewcommand{\ss}{\subseteq}
\newcommand{\wh}{\widehat}
\newcommand{\wt}{\widetilde}
\newcommand{\wb}{\overline}
\newcommand{\ve}{\varepsilon}
\newcommand{\hlambda}{\wh{\lambda}}

\newcommand{\Jd}{J}
\newcommand{\ellhat}{\wh{\ell}}
\newcommand{\rhat}{\wh{r}}
\newcommand{\elltilde}{\wt{\ell}}
\newcommand{\wtilde}{\wt{w}}
\newcommand{\what}{\wh{w}}

\DeclareMathOperator{\conv}{conv}
\newcommand{\ellprime}{\ellhat^\prime}

\newcommand{\upconf}{\phi}

%\newcommand{\Ltilde}{\wt{L}}

\newcommand{\hDelta}{\wh{\Delta}}
\newcommand{\hdelta}{\wh{\delta}}
\newcommand{\spin}{\{-1,+1\}}

\newcommand{\ep}[1]{\E\!\left[#1\right]}
\newcommand{\LT}{L_T}
\newcommand{\LTbar}{\overline{L}_T}
\newcommand{\LTbarep}{\mathring{L}_T}
\newcommand{\circxhat}{\mathring{\xh}}
\newcommand{\circx}{\mathring{x}}
\newcommand{\circu}{\mathring{u}}
\newcommand{\circcalX}{\mathring{\calX}}
\newcommand{\circg}{\mathring{g}}
\newcommand{\Lubar}{\overline{L}_{u}}
%\newcommand{\Lustarbar}{\overline{L}_{u^\star}}

\newcommand{\Lyr}{J}
\newcommand{\QQ}{{w}}
\newcommand{\Qt}{{\QQ_t}}
\newcommand{\Qstar}{{u}}
\newcommand{\Qpistar}{{\Qstar^{\star}}}
\newcommand{\Qhat}{\wh{\QQ}}
\newcommand{\Ut}{{\upconf_t}}
\newcommand{\intO}{\mathrm{int}(\Omega)}
\newcommand{\intK}{\mathrm{int}(K)}

\newcommand{\squareCB}{\ensuremath{\mathsf{SquareCB}}\xspace}
\newcommand{\feelgood}{\ensuremath{\mathsf{FGTS}}\xspace}
\newcommand{\graphCB}{\ensuremath{\mathsf{SquareCB.G}}\xspace}
\newcommand{\squareCBAuc}{\ensuremath{\mathsf{SquareCB.A}}\xspace}
\newcommand{\AlgSq}{\ensuremath{\mathsf{AlgSq}}\xspace}
\newcommand{\AlgLog}{\ensuremath{\mathsf{AlgLog}}\xspace}
\newcommand{\ips}{\ensuremath{\mathsf{(IPS)}}\xspace}
\newcommand{\optsq}{\ensuremath{\mathsf{(OptSq)}}\xspace}
\newcommand{\sq}{\ensuremath{\mathsf{(Sq)}}\xspace}
\newcommand{\dec}{\ensuremath{\mathsf{dec}_\gamma}\xspace}
\newcommand{\dectwo}{\ensuremath{\mathsf{dec}_{\gamma_1,\gamma_2}}\xspace}
%\newcommand{\theHalgorithm}{\arabic{algorithm}}
\newtheorem{cor}[theorem]{Corollary}
\newcommand{\context}{\text{ctx}}
\newcommand{\noncontext}{\text{n-ctx}}
%\newtheorem{remark}{Remark}
%\newtheorem{prop}{Proposition}
%\newtheorem{definition}{Definition}
%\newtheorem{assumption}{Assumption}
\newtheorem{event}{Event}
%\newtheorem*{main}{Main Result}
%\newtheorem{fact}[theorem]{Fact}

\newcommand{\paren}[1]{\left({#1}\right)}
\newcommand{\brackets}[1]{\left[{#1}\right]}
\newcommand{\braces}[1]{\left\{{#1}\right\}}

\newcommand{\normt}[1]{\norm{#1}_{t}}
\newcommand{\dualnormt}[1]{\norm{#1}_{t,*}}

\newcommand{\otil}{\ensuremath{\tilde{\mathcal{O}}}}

\newcommand{\dist}{\calP}

%%%%  brackets
\newcommand{\rbr}[1]{\left(#1\right)}
\newcommand{\sbr}[1]{\left[#1\right]}
\newcommand{\cbr}[1]{\left\{#1\right\}}
\newcommand{\nbr}[1]{\left\|#1\right\|}
\newcommand{\abr}[1]{\left|#1\right|}

\usepackage{lipsum,booktabs}
\usepackage{amsmath,mathrsfs,amssymb,amsfonts,bm,enumitem}
\usepackage{rotating}
\usepackage{pdflscape}
\usepackage{hyperref,url}
\hypersetup{
    colorlinks,
    breaklinks,
    linkcolor = blue,
    citecolor = blue,
    urlcolor  = blue,
}
\allowdisplaybreaks
\usepackage{appendix}
\usepackage{multirow,makecell}

%\usepackage{algorithmic,algorithm}
%\renewcommand{\algorithmicrequire}{ \textbf{Input:}}
%\renewcommand{\algorithmicensure}{ \textbf{Output:}}

\renewcommand{\tilde}{\widetilde}
\renewcommand{\hat}{\widehat}
\newcommand{\obs}{O}
\newcommand{\unobs}{E}
\newcommand{\unbiasSize}{c}
\newcommand{\unbias}{C}
\newcommand{\cnt}{k}

% define some macros
\def \A {\mathcal{A}}

\def \B {\mathbb{B}}
\def \B {\mathcal{B}}
\def \C {\mathcal{C}}
\def \D {\mathcal{D}}
\def \E {\mathbb{E}}
\def \F {\mathcal{F}}
\def \G {\mathcal{G}}
\def \H {\mathcal{H}}
\def \I {\mathcal{I}}
\def \J {\mathcal{J}}
\def \K {\mathcal{K}}
\def \L {\mathcal{L}}
\def \M {\mathcal{M}}
\def \N {\mathcal{N}}
\def \O {\mathcal{O}}
\def \P {\mathcal{P}}
\def \Q {\mathcal{Q}}
\def \R {\mathbb{R}}
\def \S {\mathcal{S}}
% \def \T {\mathrm{T}}
\def \T {\top}
\def \U {\mathcal{U}}
\def \V {\mathcal{V}}
\def \W {\mathcal{W}}
\def \X {\mathcal{X}}
\def \Y {\mathcal{Y}}
\def \Z {\mathcal{Z}}

\def \a {\mathbf{a}}
\def \b {\mathbf{b}}
\def \c {\mathbf{c}}
\def \d {\mathbf{d}}
\def \e {\mathbf{e}}
\def \f {\mathbf{f}}
\def \g {\mathbf{g}}
\def \h {\mathbf{h}}
\def \m {\mathbf{m}}
\def \p {\mathbf{p}}
\def \q {\mathbf{q}}
\def \u {\mathbf{u}}
\def \w {\mathbf{w}}
\def \s {\mathbf{s}}
\def \t {\mathbf{t}}
\def \v {\mathbf{v}}
\def \x {\mathbf{x}}
\def \y {y}
\def \z {\mathbf{z}}

\def \ph {\hat{p}}

\def \fh {\hat{f}}
\def \fb {\bar{f}}
\def \ft{\tilde{f}}

\def \gh {\hat{\g}}
\def \gb {\bar{\g}}
\def \gt {\tilde{g}}

\def \uh {\hat{\u}}
\def \ub {\bar{\u}}
\def \ut{\tilde{\u}}

\def \vh {\hat{\v}}
\def \vb {\bar{\v}}
\def \vt{\tilde{\v}}

\def \xh {\hat{x}}
\def \xb {\bar{\x}}
\def \xt {\tilde{\x}}

\def \zh {\hat{\z}}
\def \zb {\bar{\z}}
\def \zt {\tilde{\z}}

\def \Ecal {\mathcal{E}}
\def \Rcal {\mathcal{R}}
\def \Ot {\tilde{\O}}
\def \indicator {\mathds{1}}
\def \regret {\mbox{Regret}}
\def \proj {\mbox{Proj}}
\def \Pr {\mathsf{Pr}}
\def \ellb {\boldsymbol{\ell}}
\def \thetah {\hat{\theta}}

\newcommand{\RegSq}{\ensuremath{\mathrm{\mathbf{Reg}}_{\mathsf{Sq}}}\xspace}
\newcommand{\RegCB}{\ensuremath{\mathrm{\mathbf{Reg}}_{\mathsf{CB}}}\xspace}
\newcommand{\RegDyn}{\ensuremath{\mathrm{\mathbf{Reg}}_{\mathsf{Dyn}}}\xspace}
\usepackage{mathtools}
\let\oldnorm\norm   % <-- Store original \norm as \oldnorm
\let\norm\undefined % <-- "Undefine" \norm
\DeclarePairedDelimiter\norm{\lVert}{\rVert}
\DeclarePairedDelimiter\abs{\lvert}{\rvert}
%\newcommand\inner[2]{\langle #1, #2 \rangle}
\newcommand*\diff{\mathop{}\!\mathrm{d}}
\newcommand*\Diff[1]{\mathop{}\!\mathrm{d^#1}}

%\DeclareMathOperator*{\Reg}{Regret}
\DeclareMathOperator*{\AReg}{A-Regret}
\DeclareMathOperator*{\WAReg}{WA-Regret}
\DeclareMathOperator*{\SAReg}{SA-Regret}
\DeclareMathOperator*{\DReg}{\mbox{D-Regret}}
\DeclareMathOperator*{\poly}{poly}
%\DeclareMathOperator*{\argmax}{arg\,max}
%\DeclareMathOperator*{\argmin}{arg\,min}

% define new theorem environments
% \let\proof\relax
% \let\endproof\relax
% \newenvironment{proof}{\par\noindent{\bf Proof\ }}{\hfill\BlackBox\\[2mm]}
% \renewcommand\qedsymbol{$\blacksquare$}
\newtheorem{myThm}{Theorem}
\newtheorem{myFact}{Fact}
\newtheorem{myClaim}{Claim}
\newtheorem{myLemma}[myThm]{Lemma}
\newtheorem{myObservation}{Observation}
\newtheorem{myProp}[myThm]{Proposition}
\newtheorem{myProperty}{Property}

% Define a custom environment for prompts
\newtcolorbox{promptbox}[1][]{
  colback=blue!5!white, colframe=blue!75!black,
  fonttitle=\bfseries, title=Prompt,
  left=2mm, right=2mm, top=2mm, bottom=2mm,
  boxrule=0.5mm,  % Thickness of the frame
  coltitle=black, % Color of the title text
  colbacktitle=blue!15!white, % Background color of the title
  breakable,      % Allows the box to break across pages
  #1
}
\newtheorem{myAssum}{Assumption}
\newtheorem{myConj}{Conjecture}
\newtheorem{myCor}{Corollary}
\newtheorem{myDef}{Definition}
\newtheorem{myExample}{Example}
\newtheorem{myNote}{Note}
\newtheorem{myProblem}{Problem}

\newtheorem{myRemark}{Remark}

% add comments
\usepackage{graphicx,color} % more modern
\newcommand{\red}{\color{red}}
\newcommand{\blue}{\color{blue}}
\definecolor{wine_red}{RGB}{228,48,64}
\definecolor{DSgray}{cmyk}{0,1,0,0}
%\newcommand{\Authornote}[2]{{\small\textcolor{NavyBlue}{\sf$<<<${  #1: #2 }$>>>$}}}
% \newcommand{\Authormarginnote}[2]{\marginpar{\parbox{2cm}{\raggedright\tiny \textcolor{DSgray}{#1: #2}}}}
% \newcommand{\pnote}[1]{{\Authornote{Peng}{#1}}}
% \newcommand{\pmarginnote}[1]{{\Authormarginnote{Peng}{#1}}}

\usepackage{prettyref}
\newcommand{\pref}[1]{\prettyref{#1}}
\newcommand{\pfref}[1]{Proof of \prettyref{#1}}
\newcommand{\savehyperref}[2]{\texorpdfstring{\hyperref[#1]{#2}}{#2}}
\newrefformat{eq}{\savehyperref{#1}{Eq. \textup{(\ref*{#1})}}}
\newrefformat{eqn}{\savehyperref{#1}{Eq.~(\ref*{#1})}}
\newrefformat{lem}{\savehyperref{#1}{Lemma~\ref*{#1}}}
\newrefformat{event}{\savehyperref{#1}{Event~\ref*{#1}}}
\newrefformat{def}{\savehyperref{#1}{Definition~\ref*{#1}}}
\newrefformat{line}{\savehyperref{#1}{Line~\ref*{#1}}}
\newrefformat{thm}{\savehyperref{#1}{Theorem~\ref*{#1}}}
\newrefformat{tab}{\savehyperref{#1}{Table~\ref*{#1}}}
\newrefformat{corr}{\savehyperref{#1}{Corollary~\ref*{#1}}}
\newrefformat{cor}{\savehyperref{#1}{Corollary~\ref*{#1}}}
\newrefformat{sec}{\savehyperref{#1}{Section~\ref*{#1}}}
\newrefformat{app}{\savehyperref{#1}{Appendix~\ref*{#1}}}
\newrefformat{assum}{\savehyperref{#1}{Assumption~\ref*{#1}}}
\newrefformat{asm}{\savehyperref{#1}{Assumption~\ref*{#1}}}
\newrefformat{ex}{\savehyperref{#1}{Example~\ref*{#1}}}
\newrefformat{fig}{\savehyperref{#1}{Figure~\ref*{#1}}}
\newrefformat{alg}{\savehyperref{#1}{Algorithm~\ref*{#1}}}
\newrefformat{rem}{\savehyperref{#1}{Remark~\ref*{#1}}}
\newrefformat{conj}{\savehyperref{#1}{Conjecture~\ref*{#1}}}
\newrefformat{prop}{\savehyperref{#1}{Proposition~\ref*{#1}}}
\newrefformat{proto}{\savehyperref{#1}{Protocol~\ref*{#1}}}
\newrefformat{prob}{\savehyperref{#1}{Problem~\ref*{#1}}}
\newrefformat{claim}{\savehyperref{#1}{Claim~\ref*{#1}}}
\newrefformat{que}{\savehyperref{#1}{Question~\ref*{#1}}}
\newrefformat{op}{\savehyperref{#1}{Open Problem~\ref*{#1}}}
\newrefformat{fn}{\savehyperref{#1}{Footnote~\ref*{#1}}}

\def \p {\boldsymbol{p}}
\def \s {\boldsymbol{s}}
\def \m {\boldsymbol{m}}
\def \epsilon {\varepsilon}

% \def \base {\mathtt{base}\mbox{-}\mathtt{regret}}
% \def \meta {\mathtt{meta}\mbox{-}\mathtt{regret}}
\def \base {\textsc{base-regret}}
\def \meta {\textsc{meta-regret}}
\def \xref {\x_{\text{ref}}}
\def \fb {\bar{f}}
\def \interior {\text{int}}
\def \yh {\hat{\y}}
\def \RegLog {\Reg_{\log}^G}
\newcommand{\bra}[1]{\left[#1\right]}
\newcommand{\pa}[1]{\left(#1\right)}
\newcommand{\hhat}{\wh{h}}
\newcommand{\epsn}{\epsilon_N}
\newcommand{\rad}{\mathsf{rad}}
\newcommand{\hatr}{\wh{r}}
\newcommand{\fl}{\underline{f}^\star}
\newcommand{\zly}[1]{\textcolor{blue}{\textbf{[zly: }#1\textbf{]}}}
\usepackage{enumerate}
\usepackage{natbib}
% \usepackage{enumitem}

\newtheorem{claim}{Claim}[section]
\newtheorem{lemma}[claim]{Lemma}
\newtheorem{assumption}{Assumption}
\newtheorem{definition}{Definition}
\newtheorem{theorem}{Theorem}[section]
\newtheorem{example}{Example}[section]
\newtheorem{proposition}{Proposition}[section]
\newtheorem{remark}{Remark}
\newtheorem{corollary}{Corollary}[section]


% \onehalfspacing
\linespread{1.5}
\usepackage{xr}

\title{Finite Sample Analysis of Distributional TD Learning with Linear Function Approximation}


\author{
Yang Peng\thanks{School of Mathematical Sciences, Peking University; email: \texttt{pengyang@pku.edu.cn}.} \and
Kaicheng Jin\thanks{School of Mathematical Sciences, Peking University; email: \texttt{kcjin@pku.edu.cn}.} \and
Liangyu Zhang,~\thanks{School of Statistics and Data Science, Shanghai University of Finance and Economics; email: \texttt{zhangliangyu@sufe.edu.cn}.} \and
Zhihua Zhang\thanks{School of Mathematical Sciences, Peking University; email: \texttt{zhzhang@math.pku.edu.cn}.}
}



\begin{document}
\maketitle
\begin{abstract}%
In this paper, we investigate the finite-sample statistical rates of distributional temporal difference (TD) learning with linear function approximation.
The aim of distributional TD learning is to estimate the return distribution of a discounted Markov decision process for a given policy $\pi$.
Prior works on statistical analysis of distributional TD learning mainly focus on the tabular case.
In contrast, we first consider the linear function approximation setting and derive sharp finite-sample rates.
Our theoretical results demonstrate that the sample complexity of linear distributional TD learning matches that of the classic linear TD learning.
This implies that, with linear function approximation, learning the full distribution of the return using streaming data is no more difficult than learning its expectation (\ie\ the value function).
To derive tight sample complexity bounds, we conduct a fine-grained analysis of the linear-categorical Bellman equation, and employ the exponential stability arguments for products of random matrices.
Our findings provide new insights into the statistical efficiency of distributional reinforcement learning algorithms.
\end{abstract}
\tableofcontents
\section{Introduction}\label{Section:intro}
\section{Introduction}


\begin{figure}[t]
\centering
\includegraphics[width=0.6\columnwidth]{figures/evaluation_desiderata_V5.pdf}
\vspace{-0.5cm}
\caption{\systemName is a platform for conducting realistic evaluations of code LLMs, collecting human preferences of coding models with real users, real tasks, and in realistic environments, aimed at addressing the limitations of existing evaluations.
}
\label{fig:motivation}
\end{figure}

\begin{figure*}[t]
\centering
\includegraphics[width=\textwidth]{figures/system_design_v2.png}
\caption{We introduce \systemName, a VSCode extension to collect human preferences of code directly in a developer's IDE. \systemName enables developers to use code completions from various models. The system comprises a) the interface in the user's IDE which presents paired completions to users (left), b) a sampling strategy that picks model pairs to reduce latency (right, top), and c) a prompting scheme that allows diverse LLMs to perform code completions with high fidelity.
Users can select between the top completion (green box) using \texttt{tab} or the bottom completion (blue box) using \texttt{shift+tab}.}
\label{fig:overview}
\end{figure*}

As model capabilities improve, large language models (LLMs) are increasingly integrated into user environments and workflows.
For example, software developers code with AI in integrated developer environments (IDEs)~\citep{peng2023impact}, doctors rely on notes generated through ambient listening~\citep{oberst2024science}, and lawyers consider case evidence identified by electronic discovery systems~\citep{yang2024beyond}.
Increasing deployment of models in productivity tools demands evaluation that more closely reflects real-world circumstances~\citep{hutchinson2022evaluation, saxon2024benchmarks, kapoor2024ai}.
While newer benchmarks and live platforms incorporate human feedback to capture real-world usage, they almost exclusively focus on evaluating LLMs in chat conversations~\citep{zheng2023judging,dubois2023alpacafarm,chiang2024chatbot, kirk2024the}.
Model evaluation must move beyond chat-based interactions and into specialized user environments.



 

In this work, we focus on evaluating LLM-based coding assistants. 
Despite the popularity of these tools---millions of developers use Github Copilot~\citep{Copilot}---existing
evaluations of the coding capabilities of new models exhibit multiple limitations (Figure~\ref{fig:motivation}, bottom).
Traditional ML benchmarks evaluate LLM capabilities by measuring how well a model can complete static, interview-style coding tasks~\citep{chen2021evaluating,austin2021program,jain2024livecodebench, white2024livebench} and lack \emph{real users}. 
User studies recruit real users to evaluate the effectiveness of LLMs as coding assistants, but are often limited to simple programming tasks as opposed to \emph{real tasks}~\citep{vaithilingam2022expectation,ross2023programmer, mozannar2024realhumaneval}.
Recent efforts to collect human feedback such as Chatbot Arena~\citep{chiang2024chatbot} are still removed from a \emph{realistic environment}, resulting in users and data that deviate from typical software development processes.
We introduce \systemName to address these limitations (Figure~\ref{fig:motivation}, top), and we describe our three main contributions below.


\textbf{We deploy \systemName in-the-wild to collect human preferences on code.} 
\systemName is a Visual Studio Code extension, collecting preferences directly in a developer's IDE within their actual workflow (Figure~\ref{fig:overview}).
\systemName provides developers with code completions, akin to the type of support provided by Github Copilot~\citep{Copilot}. 
Over the past 3 months, \systemName has served over~\completions suggestions from 10 state-of-the-art LLMs, 
gathering \sampleCount~votes from \userCount~users.
To collect user preferences,
\systemName presents a novel interface that shows users paired code completions from two different LLMs, which are determined based on a sampling strategy that aims to 
mitigate latency while preserving coverage across model comparisons.
Additionally, we devise a prompting scheme that allows a diverse set of models to perform code completions with high fidelity.
See Section~\ref{sec:system} and Section~\ref{sec:deployment} for details about system design and deployment respectively.



\textbf{We construct a leaderboard of user preferences and find notable differences from existing static benchmarks and human preference leaderboards.}
In general, we observe that smaller models seem to overperform in static benchmarks compared to our leaderboard, while performance among larger models is mixed (Section~\ref{sec:leaderboard_calculation}).
We attribute these differences to the fact that \systemName is exposed to users and tasks that differ drastically from code evaluations in the past. 
Our data spans 103 programming languages and 24 natural languages as well as a variety of real-world applications and code structures, while static benchmarks tend to focus on a specific programming and natural language and task (e.g. coding competition problems).
Additionally, while all of \systemName interactions contain code contexts and the majority involve infilling tasks, a much smaller fraction of Chatbot Arena's coding tasks contain code context, with infilling tasks appearing even more rarely. 
We analyze our data in depth in Section~\ref{subsec:comparison}.



\textbf{We derive new insights into user preferences of code by analyzing \systemName's diverse and distinct data distribution.}
We compare user preferences across different stratifications of input data (e.g., common versus rare languages) and observe which affect observed preferences most (Section~\ref{sec:analysis}).
For example, while user preferences stay relatively consistent across various programming languages, they differ drastically between different task categories (e.g. frontend/backend versus algorithm design).
We also observe variations in user preference due to different features related to code structure 
(e.g., context length and completion patterns).
We open-source \systemName and release a curated subset of code contexts.
Altogether, our results highlight the necessity of model evaluation in realistic and domain-specific settings.





% \section{Backgrounds}\label{Section:background}
% \section{Background}\label{sec:backgrnd}

\subsection{Cold Start Latency and Mitigation Techniques}

Traditional FaaS platforms mitigate cold starts through snapshotting, lightweight virtualization, and warm-state management. Snapshot-based methods like \textbf{REAP} and \textbf{Catalyzer} reduce initialization time by preloading or restoring container states but require significant memory and I/O resources, limiting scalability~\cite{dong_catalyzer_2020, ustiugov_benchmarking_2021}. Lightweight virtualization solutions, such as \textbf{Firecracker} microVMs, achieve fast startup times with strong isolation but depend on robust infrastructure, making them less adaptable to fluctuating workloads~\cite{agache_firecracker_2020}. Warm-state management techniques like \textbf{Faa\$T}~\cite{romero_faa_2021} and \textbf{Kraken}~\cite{vivek_kraken_2021} keep frequently invoked containers ready, balancing readiness and cost efficiency under predictable workloads but incurring overhead when demand is erratic~\cite{romero_faa_2021, vivek_kraken_2021}. While these methods perform well in resource-rich cloud environments, their resource intensity challenges applicability in edge settings.

\subsubsection{Edge FaaS Perspective}

In edge environments, cold start mitigation emphasizes lightweight designs, resource sharing, and hybrid task distribution. Lightweight execution environments like unikernels~\cite{edward_sock_2018} and \textbf{Firecracker}~\cite{agache_firecracker_2020}, as used by \textbf{TinyFaaS}~\cite{pfandzelter_tinyfaas_2020}, minimize resource usage and initialization delays but require careful orchestration to avoid resource contention. Function co-location, demonstrated by \textbf{Photons}~\cite{v_dukic_photons_2020}, reduces redundant initializations by sharing runtime resources among related functions, though this complicates isolation in multi-tenant setups~\cite{v_dukic_photons_2020}. Hybrid offloading frameworks like \textbf{GeoFaaS}~\cite{malekabbasi_geofaas_2024} balance edge-cloud workloads by offloading latency-tolerant tasks to the cloud and reserving edge resources for real-time operations, requiring reliable connectivity and efficient task management. These edge-specific strategies address cold starts effectively but introduce challenges in scalability and orchestration.

\subsection{Predictive Scaling and Caching Techniques}

Efficient resource allocation is vital for maintaining low latency and high availability in serverless platforms. Predictive scaling and caching techniques dynamically provision resources and reduce cold start latency by leveraging workload prediction and state retention.
Traditional FaaS platforms use predictive scaling and caching to optimize resources, employing techniques (OFC, FaasCache) to reduce cold starts. However, these methods rely on centralized orchestration and workload predictability, limiting their effectiveness in dynamic, resource-constrained edge environments.



\subsubsection{Edge FaaS Perspective}

Edge FaaS platforms adapt predictive scaling and caching techniques to constrain resources and heterogeneous environments. \textbf{EDGE-Cache}~\cite{kim_delay-aware_2022} uses traffic profiling to selectively retain high-priority functions, reducing memory overhead while maintaining readiness for frequent requests. Hybrid frameworks like \textbf{GeoFaaS}~\cite{malekabbasi_geofaas_2024} implement distributed caching to balance resources between edge and cloud nodes, enabling low-latency processing for critical tasks while offloading less critical workloads. Machine learning methods, such as clustering-based workload predictors~\cite{gao_machine_2020} and GRU-based models~\cite{guo_applying_2018}, enhance resource provisioning in edge systems by efficiently forecasting workload spikes. These innovations effectively address cold start challenges in edge environments, though their dependency on accurate predictions and robust orchestration poses scalability challenges.

\subsection{Decentralized Orchestration, Function Placement, and Scheduling}

Efficient orchestration in serverless platforms involves workload distribution, resource optimization, and performance assurance. While traditional FaaS platforms rely on centralized control, edge environments require decentralized and adaptive strategies to address unique challenges such as resource constraints and heterogeneous hardware.



\subsubsection{Edge FaaS Perspective}

Edge FaaS platforms adopt decentralized and adaptive orchestration frameworks to meet the demands of resource-constrained environments. Systems like \textbf{Wukong} distribute scheduling across edge nodes, enhancing data locality and scalability while reducing network latency. Lightweight frameworks such as \textbf{OpenWhisk Lite}~\cite{kravchenko_kpavelopenwhisk-light_2024} optimize resource allocation by decentralizing scheduling policies, minimizing cold starts and latency in edge setups~\cite{benjamin_wukong_2020}. Hybrid solutions like \textbf{OpenFaaS}~\cite{noauthor_openfaasfaas_2024} and \textbf{EdgeMatrix}~\cite{shen_edgematrix_2023} combine edge-cloud orchestration to balance resource utilization, retaining latency-sensitive functions at the edge while offloading non-critical workloads to the cloud. While these approaches improve flexibility, they face challenges in maintaining coordination and ensuring consistent performance across distributed nodes.


\section{Backgrounds}\label{Section:background}
\subsection{Policy Evaluation and TD learning}\label{Subsection:policy_eval_and_TD}
A discounted MDP is defined by a 5-tuple $M=\<\gS,\gA,\gP_R,P,\gamma\>$.
We assume the state space $\gS$ and action space $\gA$ are both Polish spaces, namely a complete separable metric space.
${\gP_R}$ is the distribution of rewards,  \ie\ $\gP_R(\cdot|s,a)\in\Delta\prn{[0,1]}$ for any state-action pair $(s,a)\in\gS\times\gA$ (we assume all rewards is bounded).
${P}$ is the transition dynamics, \ie\ $P(\cdot|s,a)\in\Delta\prn{\gS}$ for any $(s,a)\in\gS\times\gA$.
And $\gamma\in(0,1)$ is the discount factor.

Given a policy $\pi\colon\gS\to\Delta\prn{\gA}$ and an initial state $s_0=s\in\gS$, a random trajectory $\brc{\prn{s_t,a_t,r_t}}_{t=0}^\infty$ can be sampled: $a_t\mid s_t\sim\pi(\cdot\mid s_t)$, $r_t\mid (s_t,a_t)\sim \gP_R({\cdot}\mid s_t,a_t)$, ${s_{t+1}}\mid{(s_t,a_t)}\sim P({\cdot}\mid{s_t,a_t})$ for any $t\in\NB$.
We assume the Markov chain $\brc{s_t}_{t=0}^\infty$ has a unique stationary distribution $\mu_\pi\in\Delta(\gS)$.
We define the return of the trajectory by $G^\pi(s):=\sum_{t=0}^\infty \gamma^t r_t$.
% \in\brk{0,(1-\gamma)^{-1}}$.
The value function $V^\pi(s)$ is the expectation of $G^\pi(s)$, and ${\bm{V}}^\pi:=\prn{V^\pi(s)}_{s\in\gS}$.
It is known that ${\bm{V}}^\pi$ satisfies the Bellman equation: 
% That is, for any $s\in\gS$,
\begin{equation}\label{eq:Bellman_equation}
            V^\pi(s)=\EB_{a\sim\pi(\cdot\mid s),r\sim \gP_R({\cdot}\mid s,a),s^\prime\sim P(\cdot\mid s,a)}\brk{r+\gamma V^\pi(s^\prime)},\quad\forall s\in\gS,
\end{equation}
or in a compact form ${\bm{V}}^\pi={\bm{T}}^\pi{\bm{V}}^\pi$, where ${\bm{T}}^\pi\colon \RB^\gS\to \RB^\gS$ is called the Bellman operator.

In the task of policy evaluation, we aim to find ${\bm{V}}^\pi$ for some given policy $\pi$.
Since ${\bm{V}}^\pi$ is the unique solution of the Bellman equation, the problem is reduced to solving the Bellman equation.
However, in practical applications ${\bm{T}}^\pi$ is usually unknown and the agent only has access to the streaming data $\brc{\prn{s_t,a_t,r_t}}_{t=0}^\infty$.
% a sample trajectory $\brc{\prn{s_t,a_t,r_t}}_{t=0}^\infty$ generated in a streaming manner.
In this circumstance, we may solve the Bellman equation through LSA.
% stochastic approximation.
Specifically, at the $t$-th time-step, the updating scheme is 
$$
{V}_{t}(s_t)=V_{t-1}(s_{t})-\alpha\prn{V_{t-1}(s_{t})-r_t-\gamma V_{t-1}(s_{t+1})}.
$$
We expect ${\bm{V}_t}$ will converge to ${\bm{V}}^\pi$ as $t$ tends to infinity.
This algorithm is known as the TD learning.

\subsection{Linear Function Approximation and Linear TD Learning}\label{subsection:linear_td}
In this section, we introduce linear function approximation and briefly review {\LTD}.
% linear TD learning (\LTD).
% We adopt the perspective of using LSA to solve the linear projected Bellman equation.
To be concrete, we assume there is a $d$-dimensional feature for each state $s\in\gS$, which is given by the feature map $\bphi\colon\gS\to\RB^d$.
We consider the linear function approximation of value functions:
\begin{equation*}
    \sV_{\bphi} := \left\{\bV_{\bpsi}=\prn{V_{\bpsi}(s)}_{s\in\gS}\colon  V_{\bpsi}(s)=\bpsi^{\intercal}\bphi(s),\bpsi\in\RB^{d} \right\}\subset \RB^\gS,
\end{equation*}
$\mu_\pi$-weighted norm $\norm{\bV}_{\mu_\pi}:=(\EB_{s\sim\mu_{\pi}}[V(s)^2])^{1/2}$, and linear projection operator $\bPi_{\bphi}^{\pi}\colon \RB^{\gS}\to\sV_{\bphi}$:
\begin{equation*}
    \bPi_{\bphi}^{\pi}\bV:=\argmin\nolimits_{\bV_{\bpsi}\in\sV_{\bphi}}\norm{\bV-\bV_{\bpsi}}_{\mu_\pi},\quad \forall\bV\in\RB^\gS.
\end{equation*}
One can check that the linear projected Bellman operator $\bPi_{\bphi}^{\pi}\bm{T}^{\pi}$ is a $\gamma$-contraction in the Polish space $(\sV_{\bphi},\norm{\cdot}_{\mu_\pi})$.
% Hence, the linear projected Bellman equation $\bV_{\bpsi}=\bPi_{\bphi}^{\pi}\bm{T}^{\pi}\bV_{\bpsi}$ admits a unique solution $\bV_{\bpsi^\star}$, 
Hence, $\bPi_{\bphi}^{\pi}\bm{T}^{\pi}$ admits a unique fixed point $\bV_{\bpsi^\star}$, 
which satisfies $\|\bV^\pi-\bV_{\bpsi^\star} \|_{\mu_\pi}\leq(1-\gamma^2)^{-1/2}\|\bV^\pi-\bPi_{\bphi}^{\pi}\bV^\pi \|_{\mu_\pi}$ \citep[Theorem~9.8][]{bdr2022}.
In Appendix~\ref{subsection:linear_projected_bellman_equation}, we show that $\bpsi^\star\in\RB^d$ is the unique solution to the linear system for $\bpsi\in\RB^d$:
\begin{equation}\label{eq:linear_TD_equation}
    \prn{\bSigma_{\bphi}-\gamma\EB_{s,s^\prime}\brk{\bphi(s)\bphi(s^\prime)^{\intercal}}}\bpsi=\EB_{s,r}\brk{\bphi(s)r}.
\end{equation}
In the subscript of the expectation, we abbreviate $s\sim\mu_{\pi}(\cdot), a\sim\pi(\cdot|s), r\sim \gP_R(\cdot|s,a), s^\prime\sim P(\cdot|s,a)$ as $s,a,r,s^\prime$.
For brevity, we will continue to use such abbreviations in this paper when there is no ambiguity.
We can use LSA to solve the linear projected Bellman equation (Eqn.~\eqref{eq:linear_TD_equation}).
As a result, at the $t$-th time-step, the updating scheme of {\LTD} is 
\begin{equation}\label{eq:linear_TD}
    \bpsi_t=\bpsi_{t-1}-\alpha\bphi(s_t)\brk{\prn{\bphi(s_t)-\gamma\bphi(s_{t+1})}^{\intercal}\bpsi_{t-1} - r_t}.
\end{equation}
See Appendix~\ref{subsection:convergence_linear_TD} for convergence results for {\LTD}.

\subsection{Distributional Policy Evaluation and Distributional TD Learning}
In certain applications, we are not only interested in finding the expectation of the random return $G^\pi(s)$ but want to find the whole distribution of $G^\pi(s)$.
This task is called distributional policy evaluation.
We use $\eta^\pi(s)\in\sP$ to denote the distribution of $G^\pi(s)$ and let ${\bm{\eta}}^\pi:=(\eta^\pi(s))_{s\in\gS}\in\sP^\gS$.
Then ${\bm{\eta}}^\pi$ satisfies the distributional Bellman equation:
\begin{equation}
\label{eq:distributional_Bellman_equation}
% \begin{aligned}
        \eta^\pi(s)=\EB_{a\sim\pi(\cdot\mid s),r\sim \gP_R({\cdot}\mid s,a),s^\prime\sim P(\cdot\mid s,a)}[\prn{b_{r,\gamma}}_\#\eta^\pi(s^\prime)],\quad\forall s\in\gS.
% \end{aligned}
\end{equation}
% \begin{equation}\label{eq:distributional_Bellman_equation}
% % \begin{aligned}
%         \eta^\pi(s)=\EB_{a\sim\pi(\cdot\mid s),r\sim \gP_R({\cdot}\mid s,a),s^\prime\sim P(\cdot\mid s,a)}\brk{\prn{b_{r,\gamma}}_\#\eta^\pi(s^\prime)},\quad\forall s\in\gS.
% % \end{aligned}
% \end{equation}
Here $b_{r,\gamma}\colon \RB\to\RB$ is the affine function defined by $b_{r,\gamma}(x)=r+\gamma x$, and $f_\#\nu$ is the push forward measure of $\nu$ through any function $f\colon \RB\to\RB$, so that $f_\#\nu(A)=\nu(f^{-1}(A))$ for any Borel set $A$, where $f^{-1}(A):=\brc{x\colon f(x)\in A}$.
The distributional Bellman equation can also be written as ${\bm{\eta}}^\pi={\bm{\gT}}^\pi{\bm{\eta}}^\pi$.
The operator ${{\gT}}^\pi\colon \sP^\gS\to \sP^\gS$ is called the distributional Bellman operator and ${\bm{\eta}}^\pi$ is the associated fixed point.

In analogy to TD learning, we can solve the distributional Bellman equation by LSA
% stochastic approximation 
and get the distributional TD learning algorithm given the streaming data  $\brc{\prn{s_t,a_t,r_t}}_{t=0}^\infty$:
\begin{equation*}
    \eta_{t}(s_t)=\eta_{t-1}(s_{t})-\alpha[\eta_{t-1}(s_{t})-\prn{b_{r_t,\gamma}}_\#\eta_{t-1}(s_{t+1})].
\end{equation*}
We comment the algorithm above is not computationally feasible as we need to manipulate an infinite-dimensional object at each iteration.
 
\subsection{Categorical Parametrization of Return Distributions and Categorical TD Learning}
% \zly{\textbf{Too messy. Maybe we should explain like (1) Originally all distributions lie in $\sP^{\sgn}$.
% (2) For every $\eta$ in $\sP^{\sgn}$ we can find a categorical approximation $\eta_K$.(How it defined?)
% (3) All such $\eta_K$s forms the categorical sign measure space.
% And the approximation is indeed the projection to that space.
% (4) The projection of $\eta^\pi$ satisfies the categorical Bellman euqation, and using LSA to solve that equation is called categorical TD.
% All redundant irrelevant details should be avoided.
% }}
In order to deal with the infinite-dimensional return distributions in a computational tractable manner, we consider the finite-dimensional categorical parametrization as in \citep{bellemare2017distributional,rowland2018analysis,bellemare2019distributional,rowland2024nearminimaxoptimal,peng2024statistical}.
To incorporate function approximation in the distribution parametrization which cannot guarantee non-negative outputs, we will work with the signed measure space $\sP^{\sgn}$ with total mass $1$ instead of the standard probability space $\sP\subset \sP^{\sgn}$:
\begin{equation*}
    \sP^{\sgn}:= \left\{\nu\colon\abs{\nu}(\RB)< \infty ,\nu(\RB)=1,\operatorname{supp}(\nu)\subseteq \left[0,(1-\gamma)^{-1}\right] \right\}.
\end{equation*} 
For any $\nu\in\sP^{\sgn}$, we define its cumulative distribution function (CDF) as $F_\nu(x):=\nu[0,x]$. 
We can naturally define the $L^2$ and $L^1$ distance between CDFs as Cram\'er distance $\ell_2$ and $1$-Wasserstein distance $W_1$ in $\sP^{\sgn}$ respectively.
The distributional Bellman operator (see Eqn.~\eqref{eq:distributional_Bellman_equation}) can also be extended to the product space $(\sP^{\sgn})^{\gS}$ without modifying its definition.

The space of all categorical parametrized signed measures with total mass $1$ is defined as
\begin{equation*}
    \sP^{\sgn}_K := \left\{ \nu_\bp=\sum_{k=0}^K p_k \delta_{x_k} \colon  \bp=\prn{p_0,\cdots,p_{K-1}}^{\intercal}\in \RB^K, p_K=1-\sum_{k=0}^{K-1}p_k \right\}, 
\end{equation*}
which is an affine subspace of $\sP^{\sgn}$, where $K\in \NB$, $\brc{x_k=k\iota_K}_{k=0}^K$ are equally-spaced points of the support, and $\iota_K=\brk{K(1-\gamma)}^{-1}$ is the gap between two adjacent points.
We say $\bp$ is the probability mass function (PMF) representation of $\nu_\bp\in\sP^{\sgn}_K$.
% We denote by $\bp_\nu=\prn{p_k(\nu)}_{k=0}^{K-1}\in\RB^K$ the PMF representation of $\nu=\sum_{k=0}^{K} p_k(\nu) \delta_{x_k}\in\sP^{\sgn}_K$ when $\nu$ is not identified with a vector $\bp\in\RB^K$ in the subscript explicitly.
In fact, this representation is an isometry (see Proposition~\ref{prop:PK_isometric}).
We define the categorical projection operator $\bPi_{K}\colon \sP^{\sgn}\to\sP^{\sgn}_K$ as
% For any $\nu\in \sP^{\sgn}$, 
\begin{equation*}
   \bPi_{K}\nu:=\argmin\nolimits_{\nu_\bp\in\sP^{\sgn}_{K}}\ell_{2}\prn{\nu, \nu_\bp},\quad \forall\nu\in \sP^{\sgn}. 
\end{equation*}
Following \citep[Proposition~5.14][]{bdr2022}, one can show that $\bm{\Pi}_K\nu\in\sP^{\sgn}_K$ is uniquely identified with a vector $\bp_\nu=\prn{p_k(\nu)}_{k=0}^{K-1}\in\RB^K$, where
% and $p_K(\nu)=1-\sum_{k=0}^{K-1}p_k(\nu)$.
% \begin{equation}\label{eq:categorical_prob}
%      p_k(\nu)=\EB_{X\sim \nu}\brk{\prn{1-\abs{\frac{X-x_k}{\iota_K}}}_+}:=\int_{\brk{0,(1-\gamma)^{-1}}}\prn{1-\abs{\frac{x-x_k}{\iota_K}}}_+ \nu(dx).
% \end{equation}
\begin{equation}\label{eq:categorical_prob}
     p_k(\nu)=\EB_{X\sim \nu}[(1-\abs{(X-x_k)/{\iota_K}})_+]:=\int_{\brk{0,(1-\gamma)^{-1}}}(1-\abs{(x-x_k)/{\iota_K}})_+ \nu(dx).
\end{equation}
$\bm{\Pi}_K$ is in fact an orthogonal projection (see Proposition~\ref{prop:orthogonal_decomposition}), and thus is non-expansive.

For any ${\bm{\eta}}\in\prn{\sP^{\sgn}}^\gS$, $s\in\gS$, we define $\bp_{\bm{\eta}}(s):=\bp_{\eta(s)}$ and lift $\bPi_K$ to the product space by defining
$\brk{\bm{\Pi}_K{\bm{\eta}}}(s) := \bm{\Pi}_K\eta(s)$.
One can check that the categorical Bellman operator $\bPi_{K}{\gT}^{\pi}$ (see Proposition~\ref{prop:categorical_projection_operator} for the characterization of $\bm{\Pi}_K\gT^\pi$) is a $\sqrt\gamma$-contraction in the Polish space $((\sP_K^{\sgn})^\gS,\ell_{2,\mu_\pi})$, where $\ell_{2,\mu_\pi}(\bm{\eta}_1,\bm{\eta}_2):=(\EB_{s\sim\mu_{\pi}}[\ell_2^2(\eta_1(s),\eta_2(s))])^{1/2}$ is the $\mu_\pi$-weighted Cram\'er distance between $\bm{\eta}_1,\bm{\eta}_2\in(\sP^{\sgn})^\gS$.
Hence, the categorical projected Bellman equation $\bm{\eta}=\bPi_{K}\gT^{\pi}\bm{\eta}$ admits a unique solution $\bm{\eta}^{\pi,K}$, which satisfies $\ell_{2,\mu_\pi}(\bm{\eta}^\pi,\bm{\eta}^{\pi,K})\leq(1-\gamma)^{-1/2}\ell_{2,\mu_\pi}(\bm{\eta}^\pi,\bPi_K\bm{\eta}^{\pi})$ \citep[Proposition~3][]{rowland2018analysis}.
Applying LSA to solving the equation yields categorical TD learning, and the iteration rule is given by
\begin{equation*}
    \eta_{t}(s_t)=\eta_{t-1}(s_{t})-\alpha[\eta_{t-1}(s_{t})-\bPi_K\prn{b_{r_t,\gamma}}_\#\eta_{t-1}(s_{t+1})].
\end{equation*}


\section{Linear-Categorical TD Learning}\label{Section:linear_ctd}
\subsection{Linear-Categorical Parametrization of Return Distributions}\label{subsection:linear_cate_para}
Now, we combine linear function approximation with categorical parametrization, and introduce the space of linear-categorical parametrized signed measures with total mass $1$:\footnote{Compared with the parametrization in \citep[Section~9.6][]{bdr2022}, our model is identifiable because we remove redundant parameters in their model, which can be linearly represented by other parameters.}
% , which is defined with the PMF parametrization:
% \begin{equation*}
%     \sP^{\sgn}_{\bphi,K} := \left\{\bm{\eta}_{\bw}=\prn{\eta_{\bw}(s)}_{s\in\gS}=\prn{\sum_{k=0}^Kp_k(s;\bw)\delta_{x_k}}_{s\in\gS}\colon  \bw\in\RB^{dK} \right\},
% \end{equation*}
\begin{equation*}
    \sP^{\sgn}_{\bphi,K} := \left\{\bm{\eta}_{\bw}=\prn{\eta_{\bw}(s)}_{s\in\gS}\colon \eta_{\bw}(s)=\sum_{k=0}^Kp_k(s;\bw)\delta_{x_k},  \bw\in\RB^{dK} \right\},
\end{equation*}
which is an affine subspace of $(\sP^{\sgn}_{K})^{\gS}$, where $\bw=(\bw(0)^{\intercal},\cdots,\bw(K-1)^{\intercal})^{\intercal}$ and
\begin{equation}\label{eq:def_linear_parametrize}
            p_k(s;\bw)= \begin{cases} \bphi(s)^{\intercal}\bw(k)+\frac{1}{K+1}, & \text{ for}\ k\in\brc{0,1,\cdots K-1};\\
            -\bphi(s)^{\intercal}\sum_{j=0}^{K-1} \bw(j)+\frac{1}{K+1}, & \text{ for}\ k=K.
            \end{cases}
\end{equation}
Again, this representation is an isometry (see Proposition~\ref{prop:linear_cate_isometric}).
In many cases, it is more convenient to work with the matrix version of the parameter $\bW:=\prn{\bw(0),\cdots,\bw(K-1)}\in\RB^{d\times K}$ instead of $\bw=\vect\prn{\bW}$.
We abbreviate $\bp_{\bm{\eta}_\bw}$ as $\bp_{\bw}$, then by Lemma~\ref{lem:vec_and_KP}, for any $s\in\gS$, it holds that
\begin{equation}\label{eq:PMF_linear_parametrization}
    \bp_{\bw}(s)=\bW^{\intercal}\bphi(s)+(K+1)^{-1}\bm{1}_{K}.
    % =\prn{\bI_K\otimes\bphi(s)}^{\intercal}\bw+\frac{1}{K+1}\bm{1}_{K},
\end{equation}
% where $\bI_K$ is the identity matrix.
We define the linear-categorical projection operator $\bPi_{\bphi, K}^{\pi}\colon (\sP^{\sgn})^{\gS}\to\sP^{\sgn}_{\bphi,K}$ as follow:
\begin{equation*}
    \bPi_{\bphi, K}^{\pi}\bm{\eta}:=\argmin\nolimits_{\bm{\eta}_{\bw}\in\sP^{\sgn}_{\bphi,K}}\ell_{2,\mu_\pi}\prn{\bm{\eta}, \bm{\eta}_{\bw}},\quad\forall\bm{\eta}\in (\sP^{\sgn})^{\gS}.
\end{equation*}
$\bPi_{\bphi, K}^{\pi}$ is in fact an orthogonal projection (see Proposition~\ref{prop:orthogonal_decomposition_linear_approximation}), and thus is non-expansive.
% Therefore, $\bPi_{\bphi, K}^{\pi}$ is non-expansive w.r.t.\ the $\mu_\pi$-weighted Cram\'er metric, namely $\ell_{2,\mu_{\pi}}\prn{\bPi_{\bphi, K}^{\pi}\bm{\eta},\bPi_{\bphi, K}^{\pi}\bm{\eta}^{\prime}}\leq\ell_{2,\mu_{\pi}}(\bm{\eta},\bm{\eta}^{\prime})$ for any $\bm{\eta},\bm{\eta}^{\prime}\in\prn{\sP^{\sgn}}^{\gS}$.
% In addition, $\bm{\eta}\in\prn{\sP^{\sgn}}^{\gS}$, $\bPi_{\bphi, K}^{\pi}\bm{\eta}=\bPi_{\bphi, K}^{\pi}\bPi_{K}\bm{\eta}$, for any $\bm{\eta}\in\prn{\sP^{\sgn}}^{\gS}$.
The following proposition characterizes $\bPi_{\bphi, K}^{\pi}$, whose proof can be found in Appendix~\ref{appendix:linear-cate-project-op}.
\begin{proposition}\label{prop:linear_projection}
For any $\bm{\eta}\in(\sP^{\sgn})^{\gS}$, $\bPi_{\bphi, K}^{\pi}\bm{\eta}$ is uniquely give by $ \bm{\eta}_{\tilde\bw}$, where
% the matrix version of the parameter $\tilde \bW$ is given by
                    \begin{equation*}
      \tilde{\bw}=\vect(\tilde\bW),\quad \tilde\bW=\bSigma_{\bphi}^{-1}\EB_{s\sim\mu_{\pi}}[\bphi(s)\prn{\bp_{\bm{\eta}}(s)-(K+1)^{-1}\bm{1}_{K}}^{\intercal}].
    \end{equation*}
\end{proposition}

\subsection{Linear-Categorical Projected Bellman Equation}
Since $\bPi_{\bphi, K}^{\pi}\gT^{\pi}$ is a $\sqrt{\gamma}$-contraction in the Polish space $(\sP_{\bphi,K}^{\sgn},\ell_{2,\mu_\pi})$, we have the following theorem, whose proof can be found in Appendix~\ref{subsection:proof_linear_cate_TD_equation}.
\begin{theorem}\label{thm:linear_cate_TD_equation}
The linear-categorical projected Bellman equation  $\bm{\eta}_{\bw}=\bPi_{\bphi, K}^{\pi}\gT^{\pi}\bm{\eta}_{\bw}$ admits a unique solution $\bm{\eta}_{\bw^{\star}}$, where the matrix version of the parameter $\bW^{\star}\in\RB^{d\times K}$ is the unique solution to the linear system for $\bW\in\RB^{d\times K}$
\begin{equation}\label{eq:fixed_point_equation}
    \bSigma_{\bphi}\bW-\EB_{s,s^\prime,r}\brk{\bphi(s)\bphi(s^\prime)^{\intercal}\bW\tilde{\bG}^{\intercal}(r)}=\frac{1}{K+1}\EB_{s,r}\brk{\bphi(s)\prn{\sum_{j=0}^K\bg_j(r)-\bm{1}_{K}}^{\intercal}},
\end{equation}
where for any $r\in[0,1]$ and $j,k\in\brc{0,1\cdots,K}$,
\begin{equation*}
     g_{j,k}(r):=\prn{1-\abs{(r+\gamma x_j-x_k)/{\iota_K}}}_+, \quad \bg_j(r):=\prn{g_{j,k}(r)}_{k=0}^{K-1}\in\RB^{K},
\end{equation*}
% \begin{equation*}
%      g_{j,k}(r):=\prn{1-\abs{\frac{r+\gamma x_j-x_k}{\iota_K}}}_+, \quad \bg_j(r):=\prn{g_{j,k}(r)}_{k=0}^{K-1}\in\RB^{K},
% \end{equation*}
\begin{equation*}
    \bG(r):=\begin{bmatrix}
\bg_0(r), & \cdots, & \bg_{K-1}(r)
\end{bmatrix}\in\RB^{K\times K},\quad \tilde{\bG}(r):=\bG(r)-\bm{1}_K^{\intercal}\otimes\bg_K(r)\in\RB^{K\times K}.
\end{equation*}
\end{theorem}
In analogy to approximation results of $\|\bV^\pi-\bV_{\bpsi^\star} \|_{\mu_\pi}$ and $\ell_{2,\mu_\pi}(\bm{\eta}^\pi,\bm{\eta}^{\pi,K})$, 
the following lemma answers how close $\bm{\eta}_{\bw^{\star}}$ is to $\bm{\eta}^\pi$, whose proof can be found in Appendix~\ref{appendix:proof_approx_error}.
\begin{proposition}
[Approximation Error of $\bm{\eta}_{\bw^{\star}}$]\label{prop:approx_error}
Consider the $\mu_\pi$-weighted $1$-Wasserstein error $W_{1,\mu_{\pi}}^2\prn{\bm{\eta}^\pi,\bm{\eta}_{\bw^{\star}}}:=(\EB_{s\sim\mu_{\pi}}[W_1^2({\eta}^\pi(s),{\eta}_{\bw^{\star}}(s))])^{1/2}$, it holds that
%    \begin{equation*}
%     \begin{aligned}
%         \ell_{2,\mu_{\pi}}^2\prn{\bm{\eta}^\pi,\bm{\eta}_{\bw^{\star}}}\leq& K^{-1}(1-\gamma)^{-2}+(1-\gamma)^{-1}\ell_{2,\mu_{\pi}}^2\prn{\bPi_K\bm{\eta}^{\pi},\bPi_{\bphi, K}^{\pi}\bm{\eta}^{\pi}},
%     \end{aligned}
% \end{equation*}
   \begin{equation*}
    \begin{aligned}
        W_{1,\mu_{\pi}}^2\prn{\bm{\eta}^\pi,\bm{\eta}_{\bw^{\star}}}\leq& K^{-1}(1-\gamma)^{-3}+(1-\gamma)^{-2}\ell_{2,\mu_{\pi}}^2\prn{\bPi_K\bm{\eta}^{\pi},\bPi_{\bphi, K}^{\pi}\bm{\eta}^{\pi}},
    \end{aligned}
\end{equation*}
where the first error term $K^{-1}(1-\gamma)^{-3}$ is due to the categorical parametrization, and the second error term $(1-\gamma)^{-2}\ell_{2,\mu_{\pi}}^2(\bPi_K\bm{\eta}^{\pi},\bPi_{\bphi, K}^{\pi}\bm{\eta}^{\pi})$ is due to the additional linear function approximation. 
\end{proposition}
% Since we desire a small error $\varepsilon\in(0,1)$, we will assume $K\geq (1-\gamma)^{-3}$ from now on.
\subsection{Update Scheme of Linear-Categorical TD Learning}
As before, we solve Eqn.~\eqref{eq:fixed_point_equation} by LSA and get {\LCTD} given the streaming data  $\brc{\prn{s_t,a_t,r_t}}_{t=0}^\infty$:
% \begin{equation}\label{eq:linear_CTD}
% \begin{aligned}
% \bW_t=&\bW_{t-1}-\alpha\bphi(s_t)\brk{\bphi(s_t)^{\intercal}\bW_{t-1}-\bphi(s_{t+1})^{\intercal}\bW_{t-1}\tilde{\bG}^{\intercal}(r_t)-\frac{1}{K+1}\prn{\sum_{j=0}^K\bg_j(r_t)-\bm{1}_{K}}^{\intercal}}\\
% =&\bW_{t-1}-\alpha\bphi(s_t)\prn{\bp_{\bw_{t-1}}(s_t)-\bp_{\brk{\prn{b_{r_t,\gamma}}_\#{{\eta}_{\bw_{t-1}}}}(s_{t+1})}}^{\intercal},
% \end{aligned}
% \end{equation}
\begin{equation}\label{eq:linear_CTD}
\begin{aligned}
\bW_t{=}\bW_{t{-}1}{-}\alpha\bphi(s_t)\!\!\brk{\bphi(s_t)^{\intercal}\bW_{t{-}1}{-}\bphi(s_{t{+}1})^{\intercal}\bW_{t{-}1}\tilde{\bG}^{\intercal}(r_t)-\frac{1}{K{+}1}\!\prn{\sum_{j=0}^K\bg_j(r_t){-}\bm{1}_{K}}^{\intercal}},
\end{aligned}
\end{equation}
for any $t\geq 1$, where $\alpha$ is the constant step size.
% and the second equality is justified by Eqn.~\eqref{eq:project_bellman_w}.
In this paper, we also consider the Polyak-Ruppert tail averaging $\bar{\bw}_{T}:=\prn{T/2+1}^{-1}\sum_{t=T/2}^T\bw_t$ as in the analysis of {\LTD} in \citep{samsonov2024improved}.
Standard theory of LSA \citep{mou2020linear} tells us, under some conditions, if we take an appropriate constant step size, $\bar{\bw}_{T}$ will converges to the solution $\bw^\star$ with rate $T^{-1/2}$ as $T\to\infty$.
\begin{remark}[Comparison with Existing Works]
Our {\LCTD} can be regarded as a preconditioned version of SSGD with the PMF representation considered in \citep{bellemare2019distributional} and \citep[Section~9.6][]{bdr2022}.
The preconditioning technique \citep{chen2005matrix, li2017preconditioned} is a commonly used methodology to accelerate solving optimization problems by reducing condition number.
We precondition the vanilla SSGD with the PMF representation by removing the matrix $\bC^{\intercal}\bC$ on the right of Eqn.~\eqref{eq:ssgd_pmf}, whose condition number scales with $K^2$ (Lemma~\ref{lem:spectra_of_CTC}).
By introducing the preconditioner $\prn{\bC^{\intercal}\bC}^{-1}$, our {\LCTD} (Eqn.~\eqref{eq:linear_CTD}) can achieve a convergence rate independent of $K$, which the vanilla form cannot achieve (see Appendix~\ref{Appendix:convergece_ssgd_pmf}).
It is worth noting that our {\LCTD} (Eqn.~\eqref{eq:linear_CTD}) is equivalent to the stochastic semi-gradient descent (SSGD) with CDF representation, which was also considered in \citep{lyle2019comparative}.
See Appendix~\ref{Appendix:cdf_representation} for the CDF representation, and Appendix~\ref{appendix:equiv_ssgd_lctd} for a self-contained derivation of SSGD with different representations.
We further comment that our {\LCTD} can guarantee that the total mass of return distributions always be $1$, while the algorithms proposed in \citep{lyle2019comparative} cannot.
\end{remark}


\section{Non-Asymptotic Statistical Analysis}\label{Section:analysis_linear_ctd}
% In the task of distributional policy evaluation, our goal is to minimize $\mu_\pi$-weighted $1$-Wasserstein error $W_{1,\mu_\pi}(\bm{\eta}_{\bar\bw_T},\bm{\eta}^\pi)$.
In the task of distributional policy evaluation, the quality of estimator $\bm{\eta}_{\bar\bw_T}$ can be measured by $\mu_\pi$-weighted $1$-Wasserstein error $W_{1,\mu_\pi}(\bm{\eta}_{\bar\bw_T},\bm{\eta}^\pi)$.
By triangle inequality, the error can be decomposed as approximation error and estimation error: $W_{1,\mu_\pi}(\bm{\eta}^\pi,\bm{\eta}_{\bar\bw_T})\leq W_{1,\mu_\pi}(\bm{\eta}^\pi,\bm{\eta}_{\bw^{\star}})+W_{1,\mu_\pi}(\bm{\eta}_{\bw^{\star}},\bm{\eta}_{\bar\bw_T})$.
% According to Proposition~\ref{prop:approx_error}, we have an upper bound for the approximation error $W_{1,\mu_\pi}(\bm{\eta}^\pi,\bm{\eta}_{\bw^{\star}})$.
Proposition~\ref{prop:approx_error} provides an upper bound for the approximation error $W_{1,\mu_\pi}(\bm{\eta}^\pi,\bm{\eta}_{\bw^{\star}})$, so it suffices to control the estimation error $\gL(\bar\bw_T):=W_{1,\mu_\pi}(\bm{\eta}_{\bw^{\star}},\bm{\eta}_{\bar\bw_T})$.
% According to Proposition~\ref{prop:approx_error}, to make the approximation error $W_{1,\mu_{\pi}}(\bm{\eta}^\pi,\bm{\eta}_{\bw^{\star}})\leq \varepsilon$, we need to take at least $K\geq \varepsilon^{-2}(1-\gamma)^{-3}$.

In the following, we give non-asymptotic convergence rates of $\gL(\bar\bw_T)$.
We start from the generative model setting, \ie\ in the $t$-th iteration, we collect samples $s_t\sim\mu_{\pi}(\cdot), a_t\sim\pi(\cdot|s_t), r_t\sim \gP_R(\cdot|s_t,a_t), s_t^\prime\sim P(\cdot|s_t,a_t)$ from the generative model (we need to replace $s_{t+1}$ with $s_t^\prime$ in {\LCTD} Eqn.~\eqref{eq:linear_CTD}).
We give $L^p$ and high-probability convergence results in this setting.
Then, we move back to the Markovian setting (\ie\  using the streaming data $\brc{\prn{s_t,a_t,r_t}}_{t=0}^\infty$ introduced in Section~\ref{Subsection:policy_eval_and_TD}), and give high-probability convergence results.

To facilitate the comparison of our results with those of {\LTD}, in the following, we denote by $\btheta_0=\prn{\bC\otimes \bI_d}\bw_0$ (resp. $\btheta^\star=\prn{\bC\otimes \bI_d}\bw^\star$) the initial (resp. optimal) parameters under CDF representation, where $\bC$ is defined in Eqn.~\eqref{eq:def_C}.
See Appendix~\ref{Appendix:cdf_representation} for details of CDF representation. 
\subsection{\texorpdfstring{$L^2$}{L2} Convergence}\label{Subsection:L2_convergence}
We first provide non-asymptotic convergence rates of $\EB^{1/2}[(\gL(\bar\bw_T))^2]$.
\begin{theorem}[$L^2$ Convergence]\label{thm:l2_error_linear_ctd}
Suppose $K\geq (1-\gamma)^{-1}$, $T\geq 2$, $\alpha\in(0,(1-\sqrt\gamma)/76)$ and $\bw_0\in\RB^{dK}$ is the initialization, then it holds that
    \begin{equation*}
       \begin{aligned}
        \EB^{1/2}[(\gL(\bar\bw_T))^2]\lesssim&\frac{\frac{1}{\sqrt{K}(1-\gamma)}\norm{\btheta^{\star}}_{\bI_K\otimes\bSigma_{\bphi}}+1}{\sqrt{T}(1-\gamma)\sqrt{\lambda_{\min}}}\prn{1+\sqrt{\frac{\alpha}{(1-\gamma)\lambda_{\min}}}}+\frac{\frac{1}{\sqrt{K}(1-\gamma)}\norm{\btheta^{\star}}_{\bI_K\otimes\bSigma_{\bphi}}+1}{T\sqrt{\alpha }(1-\gamma)^{\frac{3}{2}}\lambda_{\min}}\\
        &+\frac{(1-\frac{1}{2}\alpha (1-\sqrt\gamma)\lambda_{\min} )^{T/2}}{T \sqrt{\alpha}(1-\gamma)\sqrt{\lambda_{\min}}}\prn{\frac{1}{\sqrt\alpha}{+}\frac{1}{\sqrt{ (1{-}\gamma)\lambda_{\min}}}}\frac{1}{\sqrt{K}(1-\gamma)}\norm{\btheta_0-\btheta^{\star}}.
       \end{aligned}
    \end{equation*}
\end{theorem}
To prove Theorem~\ref{thm:l2_error_linear_ctd}, we conduct a fine-grained analysis of the linear-categorical Bellman equation and apply the exponential stability argument \citep[Theorem~1][]{samsonov2024improved}, which is outlined in Section~\ref{Section:proof_outline}.
And the $L^p$ ($p>2$) convergence results can be found in Theorem~\ref{thm:lp_error_linear_ctd}.
Theorem~\ref{thm:l2_error_linear_ctd} implies that learning the distribution of the return is as easy as learning its expectation (value function) with linear function approximation.
\citet{rowland2024nearminimaxoptimal, peng2024statistical} discovered this phenomenon in the tabular setting, and we extend it to the function approximation setting.
\begin{remark}\label{remark:theory_match}
The only difference between our Theorem~\ref{thm:l2_error_linear_ctd} and the $L^2$ convergence rate of classic {\LTD} \citep[Theorem~3][]{samsonov2024improved} lies in replacing $\norm{\bm{\psi}^{\star}}_{\bSigma_{\bphi}}$ (resp. $\norm{\bm{\psi}_0{-}\bm{\psi}^{\star}}$) with $K^{-1/2}(1{-}\gamma)^{-1}\norm{\btheta^{\star}}_{\bI_K\otimes\bSigma_{\bphi}}$ (resp. $K^{-1/2}(1{-}\gamma)^{-1}\norm{\btheta_0{-}\btheta^{\star}}$). 
We claim that the two pairs should be of the same order respectively.
Note that $\norm{\btheta^{\star}}_{\bI_K\otimes\bSigma_{\bphi}},\norm{\btheta_0{-}\btheta^{\star}}$ are of the order $O(\sqrt{K})$ (also dependent on $\bphi$, we omit it for brevity) if ${\eta}_{\bw^{\star}}(s), {\eta}_{\bw_0}(s)$ are valid probability distributions for all $s\in\gS$.
This is because in this case, the vector of CDF $\bF_{\btheta}(s)=(\btheta(k)^\intercal\bphi(s){+}({k{+}1})/({K{+}1}))_{k=0}^{K{-}1}\in \brk{0, 1}^K$ for $\btheta\in\brc{\btheta^{\star}, \btheta_0}$.
While, in classic {\LTD} (Section~\ref{subsection:linear_td}), a proper estimate $\bm{\psi}$ should satisfy $V_{\bm{\psi}}(s)=\bpsi^{\intercal}\bphi(s)\in\brk{0,(1{-}\gamma)^{-1}}$.
It is thus reasonable to consider the two pairs as being of the same order, respectively.
Similar arguments also holds in other convergence results presented in this paper.
Therefore, in this sense, our results match those of classic {\LTD}.
\end{remark}
One can translate Theorem~\ref{thm:l2_error_linear_ctd} into a sample complexity bound.
\begin{corollary}\label{coro:l2_sample_complexity_linear_ctd}
Under the same conditions as Theorem~\ref{thm:l2_error_linear_ctd}, for any $\varepsilon>0$, suppose
    \begin{equation*}
    \begin{aligned}
        T\gtrsim&\frac{\frac{1}{K(1-\gamma)^2}\norm{\btheta^{\star}}^2_{\bI_K\otimes\bSigma_{\bphi}}+1}{\varepsilon^2(1-\gamma)^2\lambda_{\min}}\prn{1+\frac{\alpha}{(1-\gamma)\lambda_{\min}}} +\frac{\frac{1}{\sqrt{K}(1-\gamma)}\norm{\btheta^{\star}}_{\bI_K\otimes\bSigma_{\bphi}}+1}{\varepsilon\sqrt{\alpha }(1-\gamma)^{\frac{3}{2}}\lambda_{\min}}\\
        &+\frac{1}{\alpha (1-\gamma) \lambda_{\min}}\prn{\log\frac{\norm{\btheta_0{-}\btheta^{\star}}}{\varepsilon\sqrt{K}(1{-}\gamma)}+\log\prn{ \frac{1}{T \sqrt{\alpha}(1{-}\gamma)\sqrt{\lambda_{\min}}}\prn{\frac{1}{\sqrt\alpha}{+}\frac{1}{\sqrt{ (1{-}\gamma)\lambda_{\min}}}}}},   
    \end{aligned}
    \end{equation*}
    % where 
    % \begin{equation*}
    %     R_1=\frac{1}{T \sqrt{\alpha}(1-\gamma)\sqrt{\lambda_{\min}}}\prn{\frac{1}{\sqrt\alpha}+\frac{1}{\sqrt{ (1-\gamma)\lambda_{\min}}}},
    % \end{equation*}
    then it holds that $ \EB^{1/2}[(\gL(\bar\bw_T))^2]\leq \varepsilon$.
\end{corollary}
\paragraph{Instance-Independent Step Size.}
If we take the largest possible instance-independent step size, \ie\ $\alpha\simeq (1-\gamma)$, and consider $\varepsilon\in(0, 1)$, we obtain the sample complexity bound
\begin{equation}\label{eq:largest_step_size_l2_sample_complexity}
    T=\wtilde{O}\prn{\frac{\frac{1}{K(1-\gamma)^2}\norm{\btheta^{\star}}^2_{\bI_K\otimes\bSigma_{\bphi}}+1}{\varepsilon^2(1-\gamma)^2\lambda_{\min}^2}}.
\end{equation}
% \begin{equation}\label{eq:largest_step_size_l2_sample_complexity}
%     T=\wtilde{O}\prn{\frac{\frac{1}{K(1-\gamma)^2}\norm{\btheta^{\star}}^2_{\bI_K\otimes\bSigma_{\bphi}}+1}{\varepsilon\min\brc{\varepsilon, 1}(1-\gamma)^2\lambda_{\min}^2}+\frac{1}{(1-\gamma)^2\lambda_{\min}}\log\frac{\norm{\btheta_0-\btheta^{\star}}}{\varepsilon\sqrt{K}(1-\gamma)}},
% \end{equation}
\paragraph{Optimal Instance-Dependent Step Size.}
If we take the optimal instance-dependent step size $\alpha\simeq (1-\gamma)\lambda_{\min}$ which involves the unknown $\lambda_{\min}$, we obtain a better sample complexity bound
\begin{equation}\label{eq:instance_dependent_step_size_l2_sample_complexity}
    T=\wtilde{O}\prn{ \frac{\frac{1}{K(1-\gamma)^2}\norm{\btheta^{\star}}^2_{\bI_K\otimes\bSigma_{\bphi}}+1}{\varepsilon^2(1-\gamma)^2\lambda_{\min}}},
\end{equation}
when we consider small enough $\varepsilon=\tilde{O}\prn{\sqrt{\lambda_{\min}}}$, or equivalently, large enough total update steps
\begin{equation}\label{eq:sample_size_barrier}
    T=\wtilde{\Omega}\prn{\frac{\frac{1}{K(1-\gamma)^2}\norm{\btheta^{\star}}^2_{\bI_K\otimes\bSigma_{\bphi}}+1}{(1-\gamma)^2\lambda_{\min}^2}}.
\end{equation}
These theoretical results match the recent results for classic {\LTD} with constant step size \citep{patil2023finite, li2024high, samsonov2024improved}. 
It is possible to break the sample size barrier (Eqn.~\eqref{eq:sample_size_barrier}) as in classic {\LTD} by applying the variance-reduction techniques \citep{li2023accelerated}
% or using polynomial-decaying step sizes \citep{wu2024statistical}
, we leave it as a future work.
\subsection{Convergence with High Probability and Markovian Samples}
By applying the $L^p$ convergence result (Theorem~\ref{thm:lp_error_linear_ctd}) with $p=2\log(1/\delta)$
% \red{(Take $p_\delta=2\log\frac{1}{\delta}/\log\log\frac{1}{\delta}<2\log\frac{1}{\delta}$, then you can check that $\prn{1/\delta}^{p_\delta}=\sqrt{\log\frac{1}{\delta}}$)} 
and Markov's inequality, we obtain the high-probability convergence result.
\begin{theorem}[High-Probability Convergence]\label{thm:whp_error_linear_ctd}
For any $\varepsilon>0$ and $\delta\in(0,1)$, suppose $K\geq (1{-}\gamma)^{-1}$, $\alpha\in(0,(1{-}\sqrt\gamma){/}[38\log(T{/}\delta^2)])$, $\bw_0\in\RB^{dK}$ is the initialization, and total update steps $T=$
    \begin{equation*}
    \begin{aligned}
        \wtilde{O}\Bigg(&\frac{\frac{1}{K(1{-}\gamma)^2}\!\norm{\btheta^{\star}}^2_{\bI_K{\otimes}\bSigma_{\bphi}}\!\!\!{+}1}{\varepsilon^2(1-\gamma)^2\lambda_{\min}}\!\prn{1{+}\frac{\alpha\log\frac{1}{\delta}}{(1{-}\gamma)\lambda_{\min}}}\!\log\!\frac{1}{\delta} {+}\frac{\frac{1}{\sqrt{K}(1{-}\gamma)}\!\norm{\btheta^{\star}}_{\bI_K{\otimes}\bSigma_{\bphi}}\!\!\!{+}1}{\varepsilon\sqrt{\alpha }(1-\gamma)^{\frac{3}{2}}\lambda_{\min}}\!\log\!\frac{1}{\delta}{+}\frac{\log\!\frac{\norm{\btheta_0{-}\btheta^{\star}}}{\varepsilon\sqrt{K}(1{-}\gamma)}}{\alpha (1{-}\gamma) \lambda_{\min}}\Bigg),   
    \end{aligned}
    \end{equation*}
then with probability at least $1-\delta$, it holds that $\gL\prn{\bar\bw_T}\leq\varepsilon$.
Here, the $\wtilde{O}\prn{\cdot}$ does not hide polynomials of $\log(1/\delta)$ (but hides logarithm terms of $\log(1/\delta)$).
\end{theorem}
Again, we will obtain concrete sample complexity bounds as in Eqn.~\eqref{eq:largest_step_size_l2_sample_complexity} or Eqn.~\eqref{eq:instance_dependent_step_size_l2_sample_complexity} if we use different step sizes, we omit these for brevity.
Compared with the theoretical results for classic {\LTD}, our results match \citep[Theorem~4][]{samsonov2024improved} that also considers the constant step size, but has a worse dependence on $\log\prn{1/\delta}$ than \citep[Theorem~3.1][]{wu2024statistical} which considers the polynomial-decaying step size $\alpha_t=\alpha_0 t^{-\beta}$ with $\beta\in(1/2, 1)$ instead.
% \red{TBD: State a minimax lower bound here, as a corollary of \citep[Theorem~2][]{li2024high}.}
\begin{remark}[Markovian Setting]
Using the same argument as in the proof of \citep[Theorem~6][]{samsonov2024improved}, one can immediately derive a high-probability sample complexity bound in the Markovian setting.
Compared with the bound in the generative model setting (Theorem~\ref{thm:whp_error_linear_ctd}), the bound in the Markovian setting will have an additional dependency on $t_{\operatorname{mix}}\log(T/\delta)$, where $t_{\operatorname{mix}}$ is the mixing time of the Markov chain $\brc{s_t}_{t=0}^{\infty}$ in $\gS$.
We omit this result for brevity.
\end{remark}




\section{Proof Outlines}\label{Section:proof_outline}
In this section, we outline the proofs of our main results (Theorem~\ref{thm:l2_error_linear_ctd}).
We first state the theoretical properties of the linear-categorical Bellman equation and the exponential stability of {\LCTD}. 
Finally, we highlight some key steps in proving these results.
\subsection{Vectorization of Linear-CTD}
In our analysis, it will be more convenient to work with the vectorization version and the CDF representation introduced in Appendix~\ref{Appendix:cdf_representation}. 
In Appendix~\ref{appendix:equiv_ssgd_lctd}, we show that {\LCTD} (Eqn.~\eqref{eq:linear_CTD}) is equivalent to SSGD with the CDF representation (Eqn.~\eqref{eq:CDF_SSGD_update}).
We consider the vectorization of Eqn.~\eqref{eq:CDF_SSGD_update}, and we denote $\btheta_t=\prn{\bC\otimes\bI_d}\bw_t$ (other CDF parameters are defined in the same way):
\begin{equation*}
\begin{aligned}
    \btheta_t{=}\btheta_{t-1}{-}\alpha\prn{\bA_{t}\btheta_{t-1}{-}\bb_t},\quad \bA_t{=}\brk{\bI_K{\otimes}\prn{\bphi(s_t)\bphi(s_t)^{\intercal}}}{-}\brk{\prn{\bC\tilde{\bG}(r_t)\bC^{-1}}{\otimes}\prn{\bphi(s_t)\bphi(s^\prime_t)^{\intercal}}},
\end{aligned}
\end{equation*}
\begin{equation}\label{eq:bt}
\begin{aligned}
        \bb_t=\frac{1}{K+1}\brk{\bC\prn{\sum_{j=0}^K\bg_j(r_t)-\bm{1}_K}}\otimes\bphi(s_t).
\end{aligned}
\end{equation}
% \begin{equation*}
% \begin{aligned}
%     \btheta_t=&\btheta_{t-1}-\alpha\prn{\bA_{t}\btheta_{t-1}-\bb_t}=\btheta_{t-1}-\alpha\prn{\bF_{\btheta_{t-1}}(s_t)-\bF_{\brk{\prn{b_{r_t,\gamma}}_\#{{\eta}_{\bw_{t-1}}}}(s_{t}^\prime)}}\otimes\bphi(s_t),
% \end{aligned}
% \end{equation*}
% \begin{small}
% \begin{equation}\label{eq:at_and_bt}
% \begin{aligned}
%         \bA_t=&\brk{\bI_K\otimes\prn{\bphi(s_t)\bphi(s_t)^{\intercal}}}-\brk{\prn{\bC\tilde{\bG}(r_t)\bC^{-1}}\otimes\prn{\bphi(s_t)\bphi(s^\prime_t)^{\intercal}}},\quad \bb_t=\frac{1}{K+1}\brk{\bC\prn{\sum_{j=0}^K\bg_j(r_t)-\bm{1}_K}}\otimes\bphi(s_t).
% \end{aligned}
% \end{equation}
% \end{small}
% 
% 
% 
% which are unbiased estimates of
% \begin{equation*}
% \begin{aligned}
%         \bar{\bA}&=\prn{\bI_K\otimes\bSigma_{\bphi}}-\EB_{s, r, s^\prime}\brk{\prn{\bC\tilde{\bG}(r)\bC^{-1}}\otimes\prn{\bphi(s)\bphi(s^\prime)^{\intercal}}},
% \end{aligned}
% \end{equation*}
% \begin{equation*}
%     \bar{\bb}=\frac{1}{K+1}\EB_{s, r}\brc{\brk{\bC\prn{\sum_{j=0}^K\bg_j(r)-\bm{1}_K}}\otimes\bphi(s)},
% \end{equation*}
% respectively.
We denote by $\bar{\bA},\bar{\bb}$ the expectation of $\bA_t,\bb_t$ respectively.
Utilizing the exponential stability arguments, we can derive an upper bound for $\norm{\bar{\bA}\prn{\bar{\btheta}_T{-}\btheta^{\star}}}$.
We need to further translate it to an upper bound for $\gL(\bar{\bw}_T)$, which is done in the following lemma, whose proof can be found in Appendix~\ref{appendix:proof_translate_error_to_loss}.
\begin{lemma}\label{lem:translate_error_to_loss}
For any $\bw\in\RB^{dK}$, it holds that $\gL(\bw)\leq2{K^{-1/2}(1{-}\gamma)^{-2}\lambda_{\min}^{-1/2}} \norm{\bar{\bA}\prn{{\btheta}{-}\btheta^{\star}}}$.
    % \begin{equation*}
    %   \prn{\gL(\bw)}^2\leq\frac{4}{K(1-\gamma)^4\lambda_{\min}} \norm{\bar{\bA}\prn{{\btheta}-\btheta^{\star}}}^2.
    % \end{equation*}
\end{lemma}
\subsection{Exponential Stability Analysis}
% In the following, we will sketch the proof of non-asymptotic convergence results.
% We use the exponential stability argument for analyzing LSA proposed in \citep{samsonov2024improved}.
First, we introduce some notations.
% Let $\be_t:={\bA}_t\btheta^{\star}-{\bb}_t=\prn{\bF_{\btheta^{\star}}(s_t)-\bF_{\gT_t^\pi{\bm{\eta}_{\bw^{\star}}}}(s_t)}\otimes\bphi(s_t)$,
Let $\be_t:{=}{\bA}_t\btheta^{\star}{-}{\bb}_t$,
we denote by $C_{A}$ (resp. $C_{e}$) the almost sure upper bound for $\max\brc{\norm{\bA_t}, \norm{\bA_t{-}\bar{\bA}}}$ (resp. $\norm{\be_t}$), and $\bSigma_{e}:=\EB\brk{\be_t\be_t^{\intercal}}$ the covariance matrix of $\be_t$.
The following lemma provides useful upper bounds, whose proof can be found in Appendix~\ref{appendix:proof_upper_bound_error_quantities}.
\begin{lemma}\label{lem:upper_bound_error_quantities}
Suppose $K\geq(1-\gamma)^{-1}$, then it holds that
    \begin{equation*}
       C_{A}\leq 4,\quad C_{e}\leq 4(\norm{\btheta^{\star}}+\sqrt{K}\prn{1-\gamma}),\quad\tr\prn{\bSigma_{e}}\leq 18(\norm{\btheta^{\star}}_{\bI_K\otimes\bSigma_{\bphi}}^2+K(1-\gamma)^2).
    \end{equation*}
    % \begin{equation*}
    %    C_{A}\leq 4,
    % \end{equation*}
    % \begin{equation*}
    %     C_{e}\leq 4\prn{\norm{\btheta^{\star}}+\sqrt{K}\prn{1-\gamma}},
    % \end{equation*}
    % \begin{equation*}
    %     \tr\prn{\bSigma_{e}}\leq 18\prn{\norm{\btheta^{\star}}_{\bI_K\otimes\bSigma_{\bphi}}^2+K(1-\gamma)^2}.
    % \end{equation*}
\end{lemma}
% The key step for analyzing an LSA algorithm is to verify the exponential stability condition.
Let $\bGamma_{m:n}^{(\alpha)}:=\prod_{i=m}^n\prn{\bI-\alpha\bA_{i}}$ for any $\alpha>0$ and $m,n\in\NB$, $m\leq n$.
The exponential stability of {\LCTD} is summarized in the following lemma, whose proof can be found in Appendix~\ref{appendix:proof_exponential_stable}.
\begin{lemma}\label{lem:exponential_stable}
For any $p\geq 2$, let $a=(1-\sqrt\gamma)\lambda_{\min}/2$ and $\alpha_{p,\infty}=(1-\sqrt\gamma)/(38p)$ ($\alpha_{p,\infty}p\leq 1/2$).
Then for any $\alpha\in\prn{0,\alpha_{p,\infty}}$, $\bu\in\RB^{dK}$ and $n\in\NB$, it holds that $\EB^{1/p}[\|\bGamma_{1:n}^{(\alpha)}\bu\|^p]\leq (1-\alpha a)^n \norm{\bu}$.
    % \begin{equation*}
    %    \EB^{1/p}\brk{\norm{\bGamma_{1:n}^{(\alpha)}\bu}^p}\leq \prn{1-\alpha a}^n \norm{\bu}.
    % \end{equation*}
\end{lemma}
See Appendix~\ref{Appendix:convergece_ssgd_pmf} for the counterparts of Lemma~\ref{lem:translate_error_to_loss}, Lemma~\ref{lem:upper_bound_error_quantities} and Lemma~\ref{lem:exponential_stable} for vanilla SSGD with the PMF representation, which have additional dependency on $K$.

Combining Lemma~\ref{lem:translate_error_to_loss}, Lemma~\ref{lem:upper_bound_error_quantities} and Lemma~\ref{lem:exponential_stable} with \citep[Theorem~1][]{samsonov2024improved}, one can  immediately obtain Theorem~\ref{thm:l2_error_linear_ctd}, the remaining details can be found in Appendix~\ref{appendix:proof_l2_error_linear_ctd}.

\subsection{Key Steps in the Proofs}
In this section, we highlight some key steps in proving above theoretical results.
\paragraph{Bounding Spectral Norm of Expectation of Kronecker Product.}
In proving that the $\gL(\bw)$ can be upper-bounded by $\norm{\bar{\bA}\prn{{\btheta}-\btheta^{\star}}}$ (Lemma~\ref{lem:translate_error_to_loss}), as well as in verifying the exponential stability condition (Lemma~\ref{lem:exponential_stable}), one of the most critical steps is to show
\begin{equation}\label{eq:biscuit_matrix_in_paper}
    \norm{ \EB_{s, r, s^\prime}\brk{\prn{\bC\tilde{\bG}(r)\bC^{-1}}\otimes\prn{\bSigma_{\bphi}^{-\frac{1}{2}}\bphi(s)\bphi(s^\prime)^{\intercal}\bSigma_{\bphi}^{-\frac{1}{2}}}} } \leq \sqrt{\gamma},\quad\forall r\in[0,1].
\end{equation}
By Lemma~\ref{lem:Spectra_of_ccgcc}, we have $\|\bC\tilde{\bG}(r)\bC^{{-}1}\|{\leq} \sqrt{\gamma}$ for any $r\in[0,1]$.
In addition, one can check that $ \|\EB_{s,s^\prime}[\bSigma_{\bphi}^{-1{/}2}\bphi(s)\bphi(s^\prime)^{\intercal}\bSigma_{\bphi}^{-1{/}2}]\|{\leq} 1$.
One may speculate that the property $\norm{\bB_1 {\otimes} \bB_2} = \norm{\bB_1}\norm{\bB_2}$ (Lemma~\ref{lem:spectral_norm_of_KP}) is enough to get the desired conclusion.
However, the two matrices in the Kronecker product are not independent, preventing us from using this simple property to derive the conclusion. 
On the other hand, since we merely have the upper bound $ \EB_{s,s^\prime}[\|\bSigma_{\bphi}^{-1{/}2}\bphi(s)\bphi(s^\prime)^{\intercal}\bSigma_{\bphi}^{-1{/}2}\|]{\leq} d$, simply moving the expectation in Eqn.~\eqref{eq:biscuit_matrix_in_paper} outside the norm will lead to a loose $d\sqrt{\gamma}$ bound .
To resolve this problem, we leverage the fact that the second matrix is rank-$1$ and prove the following result. The proof can be found in the derivation following Eqn.~\eqref{eq:biscuit_matrix_bound}.
\begin{lemma}
    For any random matrix $\bY$ and random vectors $\bx$, $\bz$, suppose $\norm{\bY}\leq C_Y$ almost surely, $\EB\brk{\bx\bx^{\intercal}}\preccurlyeq C_x\bI_{d_1}$ and $\EB\brk{\bz\bz^{\intercal}}\preccurlyeq C_z\bI_{d_2}$ for some constants $C_Y, C_x, C_z>0$, then it holds that
    \begin{equation*}
    \norm{ \EB\brk{\bY\otimes\prn{\bx\bz^{\intercal}} } } \leq C_Y\sqrt{C_xC_z}.
\end{equation*}
\end{lemma}
\begin{remark}\label{remark:cgc}
The matrix $\bC\tilde{\bG}(r)\bC^{-1}$ also appeared in \citep[Proposition~B.2][]{rowland2024nearminimaxoptimal} as the matrix representation of the categorical projected Bellman operator $\bPi_K\gT^\pi$ of an one-state MDP. 
They showed $\|\bC\tilde{\bG}(r)\bC^{{-}1}\|{\leq} \sqrt{\gamma}$ using the fact that $\bPi_K\gT^\pi$ is a $\sqrt{\gamma}$-contraction in $\prn{\sP, \ell_2}$.
Our Lemma~\ref{lem:Spectra_of_ccgcc} provides a new analysis by directly analyzing the matrix.
See Appendix~\ref{appendix:analysis_cate_bellman_matrix} for more details.
\end{remark}
\paragraph{Bounding Norm of $\bb_t$.}
In proving Lemma~\ref{lem:upper_bound_error_quantities}, the most involved step is to upper-bound $\norm{\bb_t}$ (Eqn.~\eqref{eq:bt}).
To this end, we need to upper-bound the following term which appears in Eqn.~\eqref{eq:bt_upper_bound}:
\begin{equation}\label{eq:bound_bt_in_paper}
    \begin{aligned}
    (K+1)^{-1}\|\sum_{j=0}^K\bC\bg_j(r)-\bC\bm{1}_K\|,\quad\forall r\in[0,1],
    \end{aligned}
\end{equation}
when $K\geq(1-\gamma)^{-1}$.
Same as the matrix $\bC\tilde{\bG}(r)\bC^{-1}$ (see Remark~\ref{remark:cgc}), Term~\eqref{eq:bound_bt_in_paper} is also related to the matrix representation of the categorical projected Bellman operator. 
Specifically, let $\nu=(K+1)^{-1}\sum_{k=0}^K \delta_{x_k}$ be the discrete uniform distribution, Term~\eqref{eq:bound_bt_in_paper} equals 
\begin{equation*}
   \norm{\bC\prn{\bp_{(b_{r,\gamma})_\#(\nu)}-\bp_{\nu}}} = \iota_K^{-1/2}\ell_2\prn{\bPi_K(b_{r,\gamma})_\#(\nu),\nu}\leq\iota_K^{-1/2}\ell_2\prn{(b_{r,\gamma})_\#(\nu),\nu}\leq 3\sqrt{K}(1-\gamma),
\end{equation*}
where we used the orthogonal decomposition (Proposition~\ref{prop:orthogonal_decomposition}) and an upper bound for $\ell_2\prn{(b_{r,\gamma})_\#(\nu) ,\nu}$ (Lemma~\ref{lem:norm_b_bound}).
The full proof can be found in the derivation following Eqn.~\eqref{eq:bt_upper_bound}.
\section{Conclusions}\label{Section:conclusion}
\section{Conclusion}
In this work, we propose a simple yet effective approach, called SMILE, for graph few-shot learning with fewer tasks. Specifically, we introduce a novel dual-level mixup strategy, including within-task and across-task mixup, for enriching the diversity of nodes within each task and the diversity of tasks. Also, we incorporate the degree-based prior information to learn expressive node embeddings. Theoretically, we prove that SMILE effectively enhances the model's generalization performance. Empirically, we conduct extensive experiments on multiple benchmarks and the results suggest that SMILE significantly outperforms other baselines, including both in-domain and cross-domain few-shot settings.


\bibliography{ref}
\bibliographystyle{abbrvnat}
\newpage

\appendix
% \section{Notation Table}\label{Appendix_notation_table}
% \begin{table}[h]
        \vspace{-0.1cm}
	\caption{The notation we use for parameter and \flops estimation.}
	\label{tab:param-flop-notation}
	\centering
	\rowcolors{2}{AppleChartGrey2}{white}
	\small
	\begin{tabular}{lc}
		\toprule
		Component                                   & Notation          \\
		\midrule
		Sequence length/context size                & $\nctx$           \\
		Vocabulary size                             & $\nvocab$         \\
		Number of blocks/layers                     & $\nlayers$        \\
		Number of query heads                       & $\nheads$         \\
		Number of key/value heads                   & $\nkvheads$       \\
		Model/embedding dimension                   & $\dmodel$         \\
		Head dimension                              & $\dhead$          \\
		Feed-forward dimension                      & $\dffn$           \\
		Number of feed-forward linears              & $\nffn$           \\
		Group size in \gls{gqa} $\nheads/\nkvheads$ & $g_{\text{size}}$ \\
		Model aspect ratio $\dmodel/\nlayers$       & $\rmodel$         \\
		Feed-forward ratio $\dffn/\dmodel$          & $\rffn$           \\
		\bottomrule
	\end{tabular}
        \vspace{-0.15cm}
\end{table}

\section{Kronecker Product}\label{Appendix_kronecker}
In this section, we will introduce some properties of Kronecker product used in our paper.
See \citep{zhang2013kronecker} for a detailed treatment of Kronecker product.

For any matrices $\bA\in\RB^{m\times n}$ and $\bB\in\RB^{p\times q}$, the Kronecker product $\bA\otimes\bB$ is an matrix in $\RB^{mp\times nq}$, defined as
\begin{equation*}
    \bA\otimes \bB = \begin{bmatrix} a_{11}\bB & a_{12}\bB & \cdots & a_{1n}\bB \\ a_{21}\bB & a_{22}\bB & \cdots & a_{2n}\bB \\ \vdots & \vdots & \ddots & \vdots \\ a_{m1}\bB & a_{m2}\bB & \cdots & a_{mn}\bB \end{bmatrix}.
\end{equation*}
\begin{lemma}\label{lem:basic_of_KP}
The Kronecker product is bilinear and associative.
Furthermore, for any matrices $\bB_1, \bB_2, \bB_3, \bB_4$ such that $\bB_1\bB_3$, $\bB_2\bB_4$ can be defined, it holds that
$\prn{\bB_1\otimes \bB_2}\prn{\bB_3\otimes \bB_4}=\prn{\bB_1\bB_3}\otimes\prn{\bB_2\bB_4}$ (mixed-product property).
\begin{proof}
See \citep[Basic properties and Theorem~3][]{zhang2013kronecker}.
\end{proof}
\end{lemma}

\begin{lemma}\label{lem:vec_and_KP}
For any matrices $\bB_1, \bB_2, \bB_3$ such that $\bB_1\bB_2\bB_3$ can be defined, it holds that $\vect\prn{\bB_1\bB_2\bB_3}=\prn{\bB_3^{\intercal}\otimes\bB_1}\vect\prn{\bB_2}$.
\begin{proof}
See \citep[Lemma~4.3.1][]{horn1994topics}.
\end{proof}
\end{lemma}

\begin{lemma}\label{lem:spectral_norm_of_KP}
% For any $d_1,d_2,d_3,d_4\in\NB$ and matrices $\bB_1\in\RB^{d_1\times d_2}$ and $\bB_2\in\RB^{d_3\times d_4}$, it holds that
For any matrices $\bB_1$ and $\bB_2$, it holds that
$\norm{\bB_1 \otimes \bB_2} = \norm{\bB_1}\norm{\bB_2}$, $\prn{\bB_1\otimes \bB_2}^{\intercal}=\bB_1^{\intercal}\otimes\bB_2^{\intercal}$.
Furthermore, if $\bB_1$ and $\bB_2$ are invertible/orthogonal/diagonal/symmetric/normal, $\bB_1\otimes \bB_2$ is also invertible/orthogonal/diagonal/symmetric/normal and $\prn{\bB_1\otimes \bB_2}^{-1}=\bB_1^{-1}\otimes\bB_2^{-1}$.
\begin{proof}
% See \citep[Theorem~8][]{lancaster1972norms}.
See \citep[Basic properties, Theorem~5 and Theorem~7][]{zhang2013kronecker}.
\end{proof}
\end{lemma}

\begin{lemma}\label{lem:spectra_of_PSD_KP}
For any $K, d\in\NB$ and PSD matrices $\bB_1, \bB_3\in\RB^{K\times K}, \bB_2, \bB_4\in\RB^{d\times d}$ with $\bB_1\preccurlyeq \bB_3$ and $\bB_2\preccurlyeq \bB_4$, it holds that $\bB_1 \otimes \bB_2$, $\bB_3 \otimes \bB_4$ are also PSD matrices, furthermore,
$\bB_1 \otimes \bB_2 \preccurlyeq \bB_3 \otimes \bB_4$.
\begin{proof}
Consider the spectral decomposition $\bB_i=\bQ_i\bD_i\bQ_i^{\intercal}$, for any $i\in [4]$, by Lemma~\ref{lem:basic_of_KP} and Lemma~\ref{lem:spectral_norm_of_KP}, we have
\begin{equation*}
    \prn{\bB_1\otimes \bB_2}=\prn{\bQ_1\otimes \bQ_2}\prn{\bD_1\otimes \bD_2}\prn{\bQ_1\otimes \bQ_2}^{\intercal}
\end{equation*}
and 
\begin{equation*}
    \prn{\bB_3\otimes \bB_4}=\prn{\bQ_3\otimes \bQ_4}\prn{\bD_3\otimes \bD_4}\prn{\bQ_3\otimes \bQ_4}^{\intercal}
\end{equation*}
are also spectral decomposition of $\prn{\bB_1\otimes \bB_2}$ and $\prn{\bB_3\otimes \bB_4}$ respectively.
It is easy to see that they are PSD.
Furthermore,
\begin{equation*}
\begin{aligned}
     \prn{\bB_3\otimes \bB_4}-\prn{\bB_1\otimes \bB_2}=&\brk{\prn{\bB_3\otimes \bB_4}-\prn{\bB_3\otimes \bB_2}}+\brk{\prn{\bB_3\otimes \bB_2}-\prn{\bB_1\otimes \bB_2}}  \\
     =&\brk{\bB_3\otimes\prn{ \bB_4-\bB_2}}+\brk{\prn{\bB_3-\bB_1}\otimes \bB_2} \\
     \succcurlyeq& \bm{0}.
\end{aligned}
\end{equation*}
\end{proof}
\end{lemma}


\begin{lemma}\label{lem:vector_outer_KP}
    For any $K,d,d_1,d_2\in\NB$, vectors $\bu, \bv\in\RB^d$ and matrices $\bB_1\in\RB^{K\times d_1}$, $\bB_2\in\RB^{d_2\times K}$, $\bB_3\in\RB^{K\times K}$, it holds that
    \begin{equation*}
    \prn{\bI_K\otimes\bu}\bB_1=\bB_1\otimes\bu,
\end{equation*}
\begin{equation*}
    \bB_2\prn{\bI_K\otimes\bv}^{\intercal}=\bB_2\otimes\bv^{\intercal},
\end{equation*}
\begin{equation*}
    \prn{\bI_K\otimes\bu}\bB_3\prn{\bI_K\otimes\bv}^{\intercal}=\bB_3\otimes\prn{\bu\bv^{\intercal}}.
\end{equation*}
Furthermore, for any matrix $\bB_4\in\RB^{d_1\times d_2}$, we have
    \begin{equation*}
    \prn{\bB_1\otimes\bu}\bB_4=\prn{\bB_1\bB_4}\otimes \bu.
\end{equation*}
\end{lemma}
\begin{proof}
Let $\bu=\prn{u_i}_{i=1}^d$ $\bB_1=\prn{b_{ij}}_{i,j=1}^K$, then
\begin{equation*}
    \begin{aligned}
        \prn{\bI_K\otimes\bu}\bB_1=&\begin{bmatrix}
\bu & \bm{0}_d & \cdots & \bm{0}_d & \bm{0}_d \\
\bm{0}_d & \bu & \cdots & \bm{0}_d & \bm{0}_d \\
\vdots & \vdots & \ddots & \vdots & \vdots \\
\bm{0}_d & \bm{0}_d & \cdots & \bu & \bm{0}_d \\
\bm{0}_d & \bm{0}_d & \cdots & \bm{0}_d & \bu
\end{bmatrix}\begin{bmatrix}
b_{11} & \cdots & b_{1K} \\
\vdots & \ddots & \vdots \\
b_{K1} & \cdots & b_{KK} 
\end{bmatrix}\\
=&\begin{bmatrix}
b_{11}u_1  & \cdots & b_{1K}u_1  \\
\vdots &\ddots & \vdots &\\
b_{11}u_d  & \cdots &  b_{1K}u_d\\
\vdots &\ddots & \vdots &\\
b_{K1}u_1  & \cdots &b_{KK} u_1  \\
\vdots &\ddots & \vdots &\\
b_{K1}u_d  & \cdots & b_{KK}u_d 
\end{bmatrix}\\
=&\begin{bmatrix}
b_{11}\bu & \cdots & b_{1K}\bu \\
\vdots & \ddots & \vdots \\
b_{K1}\bu & \cdots & b_{KK}\bu 
\end{bmatrix}\\
=& \bB_1\otimes u.
\end{aligned}
\end{equation*}
Hence
\begin{equation*}
    \begin{aligned}
\bB_2\prn{\bI_K\otimes\bv}^{\intercal}=&\brk{\prn{\bI_K\otimes\bv}\otimes \bB^{\intercal}_2}^{\intercal}=\brk{\bB^{\intercal}_2\otimes \bv}^{\intercal}=\bB_2\otimes\bv^{\intercal}.
\end{aligned}
\end{equation*}
And in the same way,
\begin{equation*}
    \begin{aligned}
\prn{\bI_K\otimes\bu}\bB_3\prn{\bI_K\otimes\bv}^{\intercal}=&\prn{\bB_3\otimes \bu}\otimes \bv^{\intercal}=\bB_3\otimes \prn{ \bu\otimes \bv^{\intercal}}=\bB_3\otimes\prn{ \bu \bv^{\intercal}}.
\end{aligned}
\end{equation*}
Furthermore,
\begin{equation*}
    \prn{\bB_1\otimes\bu}\bB_4=\brk{\prn{\bI_K\otimes\bu}\bB_1}\bB_4=\prn{\bI_K\otimes\bu}\prn{\bB_1\bB_4}=\prn{\bB_1\bB_4}\otimes \bu.
\end{equation*}
\end{proof}
\section{Omitted Results and Proofs in Section~\ref{Section:background}}
\subsection{Linear Projected Bellman Equation}\label{subsection:linear_projected_bellman_equation}
It is worth noting that, $\bPi_{\bphi}^{\pi}\colon \prn{\RB^\gS,\norm{\cdot}_{\mu_\pi}}\to\prn{\sV_\bphi,\norm{\cdot}_{\mu_\pi}}$ is an orthogonal projection.

We aim to derive Eqn.~\eqref{eq:linear_TD_equation}.
It is easy to check that, for any $\bV\in \RB^{\gS}$, $\bPi_{\bphi}^{\pi}\bV$ is uniquely give by $ \bV_{\tilde\bpsi}$ where
\begin{equation*}
    \tilde\bpsi=\bSigma_{\bphi}^{-1}\EB_{s\sim\mu_\pi}\brk{\bphi(s)V(s)}.
\end{equation*}
Hence, by the definition of Bellman operator (Eqn.~\eqref{eq:Bellman_equation}), $\bpsi^{\star}$ is the unique solution to the following system of linear equations for $\bpsi\in\RB^{d}$
\begin{equation*}
    \begin{aligned}
    \bpsi=&\bSigma_{\bphi}^{-1}\EB_{s\sim\mu_{\pi}}\brk{\bphi(s)\brk{\bT^\pi \bV_{\bpsi}}(s)}\\
    =&\bSigma_{\bphi}^{-1}\EB_{s\sim\mu_{\pi}}\brk{\bphi(s)\prn{\EB\brk{r_0\mid s_0=s }+\gamma\EB\brk{ \bphi(s_1)^{\intercal}\mid s_0=s }\bpsi}}\\
    =&\bSigma_{\bphi}^{-1}\EB_{s,s^\prime}\brk{\bphi(s)\bphi(s^\prime)^{\intercal}}\bpsi+\bSigma_{\bphi}^{-1}\EB_{s,r}\brk{\bphi(s)r},
    \end{aligned}
\end{equation*}
or equivalently,
\begin{equation*}
    \begin{aligned}
    \prn{\bSigma_{\bphi}-\gamma\EB_{s,s^\prime}\brk{\bphi(s)\bphi(s^\prime)^{\intercal}}}\bpsi=\EB_{s,r}\brk{\bphi(s)r}.
    \end{aligned}
\end{equation*}

\subsection{Convergence Results for Linear TD Learning}\label{subsection:convergence_linear_TD}
It is worthy noting that, {\LTD} is equivalent to the stochastic semi-gradient descent (SSGD) update.

In {\LTD}, our goal is to find a good estimator $\hat{\bpsi}$ such that $\norm{\bV_{\hat{\bpsi}}-\bV_{\bpsi^\star}}_{\mu_\pi}=\norm{\hat{\bpsi}-\bpsi^\star}_{\bSigma_{\bphi}}\leq \varepsilon$.
\cite{samsonov2024improved} considered the Polyak-Ruppert tail averaging $\bar{\bpsi}_{T}:=\prn{T/2+1}^{-1}\sum_{t=T/2}^T\bpsi_t$, and showed that in the generative model setting
with constant step size $\alpha\simeq (1-\gamma)\lambda_{\min}$, 
\begin{equation*}
   T=\wtilde{O}\prn{\frac{\norm{\bpsi^\star}^2_{\bSigma_{\bphi}}+1}{(1-\gamma)^2\lambda_{\min}}\prn{\frac{1}{\varepsilon^2}+\frac{1}{\lambda_{\min}}}}
\end{equation*}
% with instance-independent constant step size $\alpha\simeq (1-\gamma)$, 
% \begin{equation*}
%    T=\wtilde{O}\prn{\frac{\norm{\bpsi^\star}^2_{\bSigma_{\bphi}}+1}{\varepsilon^2(1-\gamma)^2\lambda_{\min}^2}}
% \end{equation*}
is sufficient to guarantee that $\norm{\bV_{\bar{\bpsi}_{T}}-\bV_{\bpsi^\star}}_{\mu_\pi}\leq \varepsilon$.
They also provided sample complexity bounds when taking the instance-independent (\ie\ not dependent on unknown quantity) 
% optimal instance-dependent (\ie\ dependent on unknown quantity) 
step size, and in the Markovian setting.

\subsection{Categorical Parametrization is an Isometry}\label{appendix:PK_isometric}
\begin{proposition}\label{prop:PK_isometric}
The affine space $\prn{\sP^{\sgn}_K, \ell_2}$ is isometric with $\prn{\RB^{K}, \sqrt{\iota_K}\norm{\cdot}_{\bC^\intercal \bC}}$, in the sense that, for any $\nu_{\bp_1},\nu_{\bp_2}\in\sP^{\sgn}_K$, it holds that $\ell_2^2(\nu_{\bp_1},\nu_{\bp_2})=\iota_K\norm{\bp_1-\bp_2}^2_{\bC^{\intercal}\bC}$, where
\begin{equation}\label{eq:def_C}
    \bC = 
\begin{bmatrix}
1 & 0 & \cdots & 0 & 0 \\
1 & 1 & \cdots & 0 & 0 \\
\vdots & \vdots & \ddots & \vdots & \vdots \\
1 & 1 & \cdots & 1 & 0 \\
1 & 1 & \cdots & 1 & 1
\end{bmatrix}\in\RB^{K\times K}.
\end{equation}
\end{proposition}
\begin{proof}
\begin{equation*}
    \begin{aligned}
    \ell_2^2(\nu_{\bp_1},\nu_{\bp_2})=&\int_{0}^{\prn{1{-}\gamma}^{-1}}\prn{F_{\nu_{\bp_1}}(x)-F_{\nu_{\bp_2}}(x)}^2 d x\\
    =&\iota_K\sum_{k=0}^{K-1}\prn{F_{\nu_{\bp_1}}(x_k)-F_{\nu_{\bp_2}}(x_k)}^2\\
    % =&\iota_K\norm{\bF_{\nu_1}-\bF_{\nu_2}}^2\\
    =&\iota_K\norm{\bC\prn{\bp_{1}-\bp_{2}}}^2\\
    =&\iota_K\norm{\bp_{1}-\bp_{2}}^2_{\bC^{\intercal}\bC}.
\end{aligned}
\end{equation*}
\end{proof}

\subsection{Categorical Projection Operator is Orthogonal Projection}\label{appendix:cate_project_orth}
\begin{proposition} \cite[Lemma~9.17]{bdr2022} \label{prop:orthogonal_decomposition}
For any $\nu\in\sP^{\sgn}$ and $\nu_{\bp}\in\sP^{\sgn}_K$, it holds that
\begin{equation*}
    \ell_2^2\prn{\nu,\nu_\bp}=\ell_2^2\prn{\nu,\bm{\Pi}_K\nu}+\ell_2^2\prn{\bm{\Pi}_K\nu,\nu_\bp}.
\end{equation*}
\end{proposition}

\subsection{Categorical Projected Bellman Operator}
The following lemma characterizing $\bm{\Pi}_K\gT^\pi$ is useful for both practice and theoretical analysis.
\begin{proposition}\label{prop:categorical_projection_operator}
For any ${\bm{\eta}}\in\prn{\sP^{\sgn}}^\gS$ and $s\in\gS$, it holds that
% \begin{equation*}
% \begin{aligned}
%      p_k\prn{\brk{\gT^\pi\bm{\eta}}(s)}=&\EB_{X\sim \brk{\gT^\pi\bm{\eta}}(s)}\brk{\prn{1-\abs{\frac{X-x_k}{\iota_K}}}_+}\\
%      =&\EB\brk{g_{K,k}(r_0)+\sum_{j=0}^{K-1}p_j\prn{\eta(s_1)}\prn{g_{j,k}(r_0)-g_{K,k}(r_0)}\Big| s_0=s }.
% \end{aligned}
% \end{equation*}
\begin{equation*}
\begin{aligned}
     \bp_{\gT^\pi\bm{\eta}}(s)=&\EB\brk{\bg_K(r_0)+\prn{\bG(r_0)-\bm{1}_K^{\intercal}\otimes\bg_K(r_0)}  \bp_{\bm{\eta}}(s_1)  \Big| s_0=s }\\
     =&\EB\brk{\tilde\bG(r_0) \prn{\bp_{\bm{\eta}}(s_1)-\frac{1}{K+1}\bm{1}_{K}}  \Big| s_0=s }+\frac{1}{K+1}\sum_{j=0}^K\EB\brk{\bg_j(r_0) \Big| s_0=s }.
\end{aligned}
\end{equation*}
And in the same way, for any $r\in[0,1]$ and $s^\prime\in\gS$, it holds that
\begin{equation*}
\begin{aligned}
\bp_{\prn{b_{r,\gamma}}_\#{\eta}(s^\prime)}&=\tilde\bG(r) \prn{\bp_{\bm{\eta}}(s^\prime)-\frac{1}{K+1}\bm{1}_{K}}  +\frac{1}{K+1}\sum_{j=0}^K\bg_j(r),
\end{aligned}
\end{equation*}
    where $\tilde{\bG}$ and $\bg$ is defined in Theorem~\ref{thm:linear_cate_TD_equation}.
\end{proposition}
This proposition is a special case of Proposition~\ref{prop:Pi_K_T}, whose proof can be found in Appendix~\ref{appendix:proof_Pi_K_T}.
\section{Omitted Results and Proofs in Section~\ref{Section:linear_ctd}}
\subsection{Linear-Categorical Parametrization is an Isometry}\label{appendix:linear_cate_isometric}
\begin{proposition}\label{prop:linear_cate_isometric}
The affine space $\prn{\sP^{\sgn}_{\bphi,K}, \ell_{2,\mu_{\pi}}}$ is isometric with  $\prn{\RB^{dK}, \sqrt{\iota_K}\norm{\cdot}_{\prn{\bC^{\intercal}\bC}\otimes\bSigma_{\bphi}}}$, in the sense that, for any $\bm{\eta}_{\bw_1},\bm{\eta}_{\bw_2}\in\sP^{\sgn}_{\bphi,K}$, it holds that $\ell_{2,\mu_{\pi}}^2\prn{\bm{\eta}_{\bw_1},\bm{\eta}_{\bw_2}}=\iota_K\norm{\bw_1-\bw_2}^2_{\prn{\bC^{\intercal}\bC}\otimes\bSigma_{\bphi}}$.
\end{proposition}
\begin{proof}
By Proposition~\ref{prop:PK_isometric},
\begin{equation*}
    \begin{aligned}
    \ell_{2,\mu_{\pi}}^2\prn{\bm{\eta}_{\bw_1},\bm{\eta}_{\bw_2}}=&\iota_K\EB_{s\sim\mu_\pi}\brk{\norm{\bC\prn{\bp_{\bw_1}(s)-\bp_{\bw_2}(s)}}^2}\\
    =&\iota_K\tr\prn{\bSigma_{\bphi}^{\frac{1}{2}}\prn{\bW_1-\bW_2}\bC^{\intercal}\bC\prn{\bW_1-\bW_2}^{\intercal}\bSigma_{\bphi}^{\frac{1}{2}}} \\
    =&\iota_K\norm{\bw_1-\bw_2}^2_{\prn{\bC^{\intercal}\bC}\otimes\bSigma_{\bphi}}.
\end{aligned}
\end{equation*}
\end{proof}

\subsection{Linear-Categorical Projection Operator}\label{appendix:linear-cate-project-op}
Proposition~\ref{prop:linear_projection} is an immediate corollary of the following lemma.
\begin{lemma}\label{lem:gradient_cramer_distance}
    For any $\bm{\eta}\in\prn{\sP^{\sgn}}^{\gS}$, $\bw\in\RB^{dK}$ and $s\in\gS$, it holds that
    \begin{align*}
        \nabla_{\bW} \ell_2^2\prn{{\eta}_{\bw}(s),{\eta}(s)}&=2\iota_K\bphi(s)\prn{\bp_{\bw}(s)-\bp_{\bm{\eta}}(s)}^{\intercal}\bC^{\intercal}\bC\\
        &=2\iota_K\bphi(s)\brk{\bphi(s)^{\intercal}\bW+\prn{\frac{1}{K+1}\bm{1}_{K}-\bp_{\bm{\eta}}(s)}^{\intercal}}\bC^{\intercal}\bC.
    \end{align*}
    Furthermore, it holds that
        \begin{align*}
        \nabla_{\bW} \ell_{2,\mu_{\pi}}^2\prn{\bm{\eta}_{\bw},\bm{\eta}}&=\EB_{s\sim\mu_{\pi}}\brk{\nabla_{\bW} \ell_2^2\prn{{\eta}_{\bw}(s),{\eta}(s)}}\\
        &=2\iota_K\brk{\bSigma_{\bphi}\bW+\EB_{s\sim\mu_{\pi}}\brk{\bphi(s)\prn{\frac{1}{K+1}\bm{1}_{K}-\bp_{\bm{\eta}}(s)}^{\intercal}}}\bC^{\intercal}\bC.
    \end{align*}
\end{lemma}
\begin{proof}
According to Proposition~\ref{prop:orthogonal_decomposition}, one has
\begin{align*}
    \ell_2^2\prn{{\eta}_{\bw}(s),{\eta}(s)}=\ell_2^2\prn{{\eta}_{\bw}(s),\bPi_K{\eta}(s)}+\ell_2^2\prn{\bPi_K{\eta}(s),{\eta}(s)}.
\end{align*}
Hence,
\begin{align*}
        \nabla_{\bw} \ell_2^2\prn{{\eta}_{\bw}(s),{\eta}(s)}&=\nabla_{\bw}\ell_2^2\prn{{\eta}_{\bw}(s),\bPi_K{\eta}(s)}\\
        &=\iota_K\nabla_{\bw}\norm{\bC\prn{\bp_{\bw}(s)-\bp_{\bm{\eta}}(s)}}^2\\
        &=2\iota_K\prn{\bI_K\otimes\bphi(s)}\bC^{\intercal}\bC\prn{\bp_{\bw}(s)-\bp_{\bm{\eta}}(s)}\\
        &=2\iota_K\prn{\bI_K\otimes\bphi(s)}\bC^{\intercal}\bC\prn{\prn{\bI_K\otimes\bphi(s)}^{\intercal}\bw+\frac{1}{K+1}\bm{1}_{K}-\bp_{\bm{\eta}}(s)}\\
        &=2\iota_K \brc{ \brk{\prn{\bC^{\intercal}\bC}\otimes\prn{\bphi(s)\bphi(s)^{\intercal}}}\bw+  
 \brk{\prn{\bC^{\intercal}\bC\prn{\frac{1}{K+1}\bm{1}_{K}-\bp_{\bm{\eta}}(s)}}\otimes \bphi(s)}},
    \end{align*}
where in the second equality, we used Proposition~\ref{prop:PK_isometric}, and in the last equality, we used Lemma~\ref{lem:vector_outer_KP}.
We also have the following matrix representation:
\begin{align*}
        \nabla_{\bW} \ell_2^2\prn{{\eta}_{\bw}(s),{\eta}(s)}&=2\iota_K\bphi(s)\prn{\bp_{\bw}(s)-\bp_{\bm{\eta}}(s)}^{\intercal}\bC^{\intercal}\bC\\
        &=2\iota_K\bphi(s)\brk{\bphi(s)^{\intercal}\bW+\prn{\frac{1}{K+1}\bm{1}_{K}-\bp_{\bm{\eta}}(s)}^{\intercal}}\bC^{\intercal}\bC.
    \end{align*}
\end{proof}
\begin{proposition}\label{prop:orthogonal_decomposition_linear_approximation}
For any $\bm{\eta}\in\prn{\sP^{\sgn}}^{\gS}$ and $\bm{\eta}_{\bw}\in\sP^{\sgn}_{\bphi,K}$, it holds that
\begin{equation*}
    \ell_{2,\mu_{\pi}}^2\prn{\bm{\eta},\bm{\eta}_{\bw}}=\ell_{2,\mu_{\pi}}^2\prn{\bm{\eta},\bPi_{K}\bm{\eta}}+\ell_{2,\mu_{\pi}}^2\prn{\bPi_{K}\bm{\eta},\bPi_{\bphi, K}^{\pi}\bm{\eta}}+\ell_{2,\mu_{\pi}}^2\prn{\bPi_{\bphi, K}^{\pi}\bm{\eta},\bm{\eta}_{\bw}}.
\end{equation*}
\end{proposition}
The proof is straightforward and almost the same as that of Proposition~\ref{prop:orthogonal_decomposition} if we utilize the affine structure.

\subsection{Linear-Categorical Projected Bellman Equation}\label{subsection:proof_linear_cate_TD_equation}
To derive the result, the following proposition characterizing $\bPi_{\bphi, K}^{\pi}\gT^{\pi}\bm{\eta}_{\bw}$ is useful, whose proof can be found in Appendix~\ref{appendix:proof_Pi_K_T}.
\begin{proposition}\label{prop:Pi_K_T}
    For any $\bw\in\RB^{dK}$ and $s\in\gS$, we abbreviate $\bp_{\gT^{\pi}\bm{\eta}_\bw}(s)$ as $\tilde{\bp}_{\bw}(s)$, then
    \begin{equation*}
    \begin{aligned}
             \tilde{\bp}_{\bw}(s)=&\prn{\tilde{p}_k(s;\bw)}_{k=0}^{K-1}=\EB\brk{\tilde{\bG}(r_0) \bW^{\intercal} \bphi(s_1)\Big| s_0=s }+\frac{1}{K+1}\sum_{j=0}^K\EB\brk{\bg_j(r_0) \Big| s_0=s },
    \end{aligned}
    \end{equation*}
\end{proposition}
Combining this proposition with Proposition~\ref{prop:linear_projection}, we know that $\bW^{\star}$ is the unique solution to the following system of linear equations for $\bW\in\RB^{d\times K}$
\begin{equation*}
    \begin{aligned}
    \bW=&\bSigma_{\bphi}^{-1}\EB_{s\sim\mu_{\pi}}\brk{\bphi(s)\prn{\tilde\bp_{\bw}(s)-\frac{1}{K+1}\bm{1}_{K}}^{\intercal}}\\
    =&\bSigma_{\bphi}^{-1}\EB_{s\sim\mu_{\pi}}\brk{\bphi(s)\prn{\frac{1}{K+1}\bm{1}_{K}-\EB\brk{\tilde{\bG}(r_0) \bW^{\intercal} \bphi(s_1)\Big| s_0=s }-\frac{1}{K+1}\sum_{j=0}^K\EB\brk{\bg_j(r_0) \Big| s_0=s }}^{\intercal}}\\
    =&\bSigma_{\bphi}^{-1}\EB_{s\sim\mu_{\pi}}\brk{\bphi(s)\bphi(s^\prime)^{\intercal}\bW\tilde{\bG}^{\intercal}(r)}+   \frac{1}{K+1}\bSigma_{\bphi}^{-1}\EB_{s\sim\mu_{\pi}}\brk{\bphi(s)\prn{\sum_{j=0}^K\bg_j(r)-\bm{1}_{K}}^{\intercal}},
    \end{aligned}
\end{equation*}
or equivalently,
\begin{equation*}
    \begin{aligned}
    \bSigma_{\bphi}\bW-\EB_{s\sim\mu_{\pi}}\brk{\bphi(s)\bphi(s^\prime)^{\intercal}\bW\tilde{\bG}^{\intercal}(r)}=   \frac{1}{K+1}\EB_{s\sim\mu_{\pi}}\brk{\bphi(s)\prn{\sum_{j=0}^K\bg_j(r)-\bm{1}_{K}}^{\intercal}},
    \end{aligned}
\end{equation*}
which is the desired conclusion.
The uniqueness and existence of the solution is guaranteed by the fact that the LHS is an invertible linear transformation of $\bW$, which is justified by Eqn.~\eqref{eq:bar_A_lower_bound}.

\subsection{Proof of Proposition~\ref{prop:Pi_K_T}}\label{appendix:proof_Pi_K_T}
\begin{proof}
Recall the definition of the distributional Bellman operator Eqn.~\eqref{eq:distributional_Bellman_equation} and categorical projection operator Eqn.~\eqref{eq:categorical_prob}, we have
\begin{equation}\label{eq:eq_in_proof_Pi_K_T}
    \begin{aligned}
\tilde{p}_k(s;\bw)=& p_k\prn{\brk{\gT^{\pi}\bm{\eta}_\bw}(s)} \\
=&\EB_{X\sim \brk{\gT^{\pi}\bm{\eta}_\bw}(s)}\brk{\prn{1-\abs{\frac{X-x_k}{\iota_K}}}_+}\\
=&\EB\brk{\EB_{G\sim \eta_{\bw}(s_1)}\brk{\prn{1-\abs{\frac{r_0+\gamma G-x_k}{\iota_K}}}_+}\Big| s_0=s }\\
=&\EB\brk{\sum_{j=0}^Kp_j(s_1;
\bw)\prn{1-\abs{\frac{r_0+\gamma x_j-x_k}{\iota_K}}}_+\Big| s_0=s }\\
=&\EB\brk{\sum_{j=0}^Kp_j(s_1;
\bw)g_{j,k}(r_0)\Big| s_0=s }\\
=&\EB\brk{g_{K,k}(r_0)+\sum_{j=0}^{K-1}p_j(s_1;
\bw)\prn{g_{j,k}(r_0)-g_{K,k}(r_0)}\Big| s_0=s }.
\end{aligned}
\end{equation}

Hence,
\begin{align*}
\tilde{\bp}_{\bw}(s)=& \prn{\tilde{p}_k(s;\bw)}_{k=0}^{K-1} \\
=&\EB\brk{\begin{bmatrix}
g_{K,1}(r_0) \\
\vdots \\
g_{K,{K-1}}(r_0)
\end{bmatrix}+\sum_{j=0}^{K-1}p_j(s_1;
\bw)\begin{bmatrix}
g_{j,1}(r_0)-g_{K,1}(r_0) \\
\vdots \\
g_{j,K-1}(r_0)-g_{K,K-1}(r_0)
\end{bmatrix}\Bigg| s_0=s}\\ 
=&\EB\brk{\bg_K(r_0)+\sum_{j=0}^{K-1}p_j(s_1;
\bw)\prn{\bg_j(r_0)-\bg_K(r_0)}    \Big| s_0=s }\\
=&\EB\brk{\bg_K(r_0)+\prn{\bG(r_0)-\bm{1}_K^{\intercal}\otimes\bg_K(r_0)}  \bp_{\bw}(s_1)  \Big| s_0=s }\\
=&\EB\brk{\bg_K(r_0)+\prn{\bG(r_0)-\bm{1}_K^{\intercal}\otimes\bg_K(r_0)}  \brk{\prn{\bI_K\otimes\bphi(s_1)}^{\intercal}\bw+\frac{1}{K+1}\bm{1}_{K}}  \Big| s_0=s }\\
=&\EB\brk{\prn{\bG(r_0)-\bm{1}_K^{\intercal}\otimes\bg_K(r_0)}  \prn{\bI_K\otimes\bphi(s_1)}^{\intercal} \Big| s_0=s }\bw\\
&\quad +\EB\brk{\bg_K(r_0)+\frac{1}{K+1}\prn{\bG(r_0)-\bm{1}_K^{\intercal}\otimes\bg_K(r_0)}  \bm{1}_{K} \Big| s_0=s }\\
% =&\EB\brk{\prn{\bG(r_0)-\bm{1}_K^{\intercal}\otimes\bg_K(r_0)}  \prn{\bI_K\otimes\bphi(s_1)}^{\intercal} \Big| s_0=s }\bw+\frac{1}{K+1}\sum_{j=0}^K\EB\brk{\bg_j(r_0) \Big| s_0=s }.
=&\EB\brk{\prn{\bG(r_0)-\bm{1}_K^{\intercal}\otimes\bg_K(r_0)} \otimes \bphi(s_1)^{\intercal}\Big| s_0=s }\bw+\frac{1}{K+1}\sum_{j=0}^K\EB\brk{\bg_j(r_0) \Big| s_0=s },
\end{align*}
or equivalently,
    \begin{equation*}
    \begin{aligned}
             \tilde{\bp}_{\bw}(s)=\EB\brk{\tilde{\bG}(r_0) \bW^{\intercal} \bphi(s_1)\Big| s_0=s }+\frac{1}{K+1}\sum_{j=0}^K\EB\brk{\bg_j(r_0) \Big| s_0=s }.
    \end{aligned}
    \end{equation*}
\end{proof}

\subsection{Proof of Proposition~\ref{prop:approx_error}}\label{appendix:proof_approx_error}
\begin{proof}
By the basic inequality (Lemma~\ref{lem:prob_basic_inequalities}), we only need to show
   \begin{equation*}
    \begin{aligned}
        \ell_{2,\mu_{\pi}}^2\prn{\bm{\eta}^\pi,\bm{\eta}_{\bw^{\star}}}\leq&\frac{\ell_{2,\mu_{\pi}}^2\prn{\bm{\eta}^{\pi},\bPi_{\bphi, K}^{\pi}\bm{\eta}^{\pi}}}{1-\gamma}\\
        =&\frac{\ell_{2,\mu_{\pi}}^2\prn{\bm{\eta}^{\pi},\bPi_{K}\bm{\eta}^{\pi}}+\ell_{2,\mu_{\pi}}^2\prn{\bPi_{K}\bm{\eta}^{\pi},\bPi_{\bphi, K}^{\pi}\bm{\eta}^{\pi}}}{1-\gamma}\\
       \leq& \frac{1}{K(1-\gamma)^2}+\frac{\ell_{2,\mu_{\pi}}^2\prn{\bPi_K\bm{\eta}^{\pi},\bPi_{\bphi, K}^{\pi}\bm{\eta}^{\pi}}}{1-\gamma},
    \end{aligned}
\end{equation*}
where we used \citep[Proposition~9.18 and Eqn.~(5.28)][]{bdr2022}.
\end{proof}

\subsection{Cumulative Distribution Function Representation}\label{Appendix:cdf_representation}
\subsubsection{Categorical Parametrization}
For any $\nu\in\sP^{\sgn}_K$, we denote 
\begin{equation*}
    \bF_{\nu}=\prn{F_k\prn{\nu}}_{k=0}^{K-1}=\prn{\nu\prn{[0, x_k]}}_{k=0}^{K-1}\in\RB^{K}
\end{equation*}
as the cumulative distribution function (CDF) representation of $\nu$.
One can check that $\bF_{\nu}=\bC\bp_{\nu}$.
As a result, PMF and CDF representations are equivalent because $\bC$ is invertible.

For any $\nu_1,\nu_2\in\sP^{\sgn}_K$,
\begin{equation*}
    \begin{aligned}
    \ell_2^2(\nu_1,\nu_2)=&\iota_K\norm{\bF_{\nu_1}-\bF_{\nu_2}}^2,
\end{aligned}
\end{equation*}
therefore, the affine space $\prn{\sP^{\sgn}_K, \ell_2}$ is isometric with the Euclidean space $\prn{\RB^{K}, \sqrt{\iota_K}\norm{\cdot}_2}$ if we consider the CDF representation.

For any ${\bm{\eta}}\in\prn{\sP^{\sgn}_K}^\gS$, $s\in\gS$, we abuse the notation to define $\bF_{\bm{\eta}}(s):=\bF_{\eta(s)}$.
By Eqn.~\eqref{eq:categorical_prob}, we can also similarly define $\bF_\nu, \bF_{\bm{\eta}}(s)$ using the categorical projection operator for any $\nu\in \sP^{\sgn}$, $\bm{\eta}\in \prn{\sP^{\sgn}}^{\gS}$ and $s\in\gS$.

\subsubsection{Linear-Categorical Parametrization}
We introduce new notations for the CDF representation when we consider linear-categorical parametrization.
Let $\bTheta:=\bW \bC^{\intercal}=\prn{\sum_{k=0}^0\bw(k), \sum_{k=0}^1\bw(k), \cdots,\sum_{k=0}^K\bw(k)}\in\RB^{d\times K}$ and the vectorization of $\bTheta$, $\btheta:=\vect\prn{\bTheta}=\prn{\bC\otimes\bI_d}\bw\in\RB^{dK}$.
We abbreviate $\bF_{\bm{\eta}_\bw}$ as $\bF_{\btheta}$, then by Lemma~\ref{lem:vec_and_KP} and Lemma~\ref{lem:vector_outer_KP}, for any $s\in\gS$, it holds that
\begin{equation}\label{eq:CDF_linear_parametrization}
    \bF_{\btheta}(s)=\bC\bp_{\bw}(s)=\bTheta^{\intercal}\bphi(s)+\frac{1}{K+1}\bC\bm{1}_{K}=\prn{\bI_K\otimes\bphi(s)}^{\intercal}\btheta+\frac{1}{K+1}\bC\bm{1}_{K},
\end{equation}
Again, PMF and CDF representations are equivalent because $\bC$ is invertible.

For any $\bm{\eta}_{\bw_1},\bm{\eta}_{\bw_2}\in\sP^{\sgn}_{\bphi,K}$, by Proposition~\ref{prop:PK_isometric},
\begin{equation}\label{eq:CDF_target_function}
    \begin{aligned}
    \ell_{2,\mu_{\pi}}^2\prn{\bm{\eta}_{\bw_1},\bm{\eta}_{\bw_2}}=&\EB_{s\sim\mu_\pi}\brk{\ell_{2}^2\prn{\eta_{\bw_1}(s),\eta_{\bw_2}(s)}}\\
    =&\iota_K\EB_{s\sim\mu_\pi}\brk{\norm{\bF_{\btheta_1}(s)-\bF_{\btheta_2}(s)}^2}\\
    =&\iota_K\tr\prn{\prn{\bTheta_1-\bTheta_2}^{\intercal}\bSigma_{\bphi}\prn{\bTheta_1-\bTheta_2}} \\
    =&\iota_K\norm{\btheta_1-\btheta_2}^2_{\bI_K\otimes\bSigma_{\bphi}},
\end{aligned}
\end{equation}
hence the affine space $\prn{\sP^{\sgn}_{\bphi,K}, \ell_{2,\mu_{\pi}}}$ is isometric with the Euclidean space $\prn{\RB^{Kd}, \sqrt{\iota_K}\norm{\cdot}_{\bI_K\otimes\bSigma_{\bphi}}}$ if we consider the CDF representation.

Following the proof of Lemma~\ref{lem:gradient_cramer_distance} in Appendix~\ref{appendix:linear-cate-project-op}, we can also derive the gradient when we use the CDF parametrization:
\begin{align*}
        \nabla_{\btheta} \ell_2^2\prn{{\eta}_{\bw}(s),{\eta}(s)}&=\iota_K\nabla_{\btheta}\norm{\bF_{\btheta}(s)-\bF_{\bm{\eta}}(s)}^2\\
        &=2\iota_K\prn{\bI_K\otimes\bphi(s)}\prn{\bF_{\btheta}(s)-\bF_{\bm{\eta}}(s)}\\
        &=2\iota_K\prn{\bI_K\otimes\bphi(s)}\prn{\prn{\bI_K\otimes\bphi(s)}^{\intercal}\btheta+\bC\prn{\frac{1}{K+1}\bm{1}_{K}-\bp_{\bm{\eta}}(s)}}\\
        &=2\iota_K \brc{ \brk{\bI_K\otimes\prn{\bphi(s)\bphi(s)^{\intercal}}}\btheta+  
 \brk{\prn{\bC\prn{\frac{1}{K+1}\bm{1}_{K}-\bp_{\bm{\eta}}(s)}}\otimes \bphi(s)}},
    \end{align*}
\begin{equation}\label{eq:gradient_cdf_representation}
            \begin{aligned}
        \nabla_{\bTheta} \ell_2^2\prn{{\eta}_{\bw}(s),{\eta}(s)}&=2\iota_K\bphi(s)\prn{\bF_{\btheta}(s)-\bF_{\bm{\eta}}(s)}^{\intercal}\\
        &=2\iota_K\bphi(s)\brk{\bphi(s)^{\intercal}\bTheta+\prn{\frac{1}{K+1}\bm{1}_{K}-\bp_{\bm{\eta}}(s)}^{\intercal}\bC^{\intercal}}.
    \end{aligned}
    \end{equation}

\subsection{Stochastic Semi-Gradient Descent with Linear Function Approximation}\label{appendix:equiv_ssgd_lctd}
We use the notations in the generative model setting for simplicity.
We denote by ${\gT}_t^\pi$ the corresponding empirical distributional Bellman operator at the $t$-th iteration, which satisfies
\begin{equation}\label{eq:empirical_op}
    \brk{\gT_t^\pi{\bm{\eta}}}(s_{t})=(b_{r_{t},\gamma})_\#(\eta(s_{t}^\prime)),\quad\forall\bm{\eta}\in\sP^{\gS}.
\end{equation}
\subsubsection{PMF Representation}
Consider the SSGD with the PMF representation
\[
    \bW_t=\bW_{t-1}-\alpha\nabla_{\bW}\ell_2^2\prn{{\eta}_{\bw_{t-1}}(s_t), \brk{\gT_t^\pi{\bm{\eta}_{\bw_{t-1}}}}(s_{t})},
\]
where $\nabla_{\bW}$ stands for taking gradient w.r.t.\ $\bW_{t-1} \in \RB^{d\times K}$ in the first term ${\eta}_{\bw_{t-1}}(s_t)$ (the second term is regarding as a constant, that's why we call it a semi-gradient). 
One can check that $\nabla_{\bW}\ell_2^2\prn{{\eta}_{\bw_{t-1}}(s_t), \brk{\gT_t^\pi{\bm{\eta}_{\bw_{t-1}}}}(s_{t})}$ is an unbiased estimate of $\nabla_{\bW} \ell_{2,\mu_{\pi}}^2\prn{\bm{\eta}_{\bw_{t-1}},\gT^\pi{\bm{\eta}_{\bw_{t-1}}}}$.

Now, let's compute the gradient term.
By Lemma~\ref{lem:gradient_cramer_distance}, we have
    \begin{align*}
        \nabla_{\bW} \ell_2^2\prn{{\eta}_{\bw_{t-1}}(s_t),\brk{\gT_t^\pi{\bm{\eta}_{\bw_{t-1}}}}(s_{t})}&=2\iota_K\bphi(s_t)\prn{\bp_{\bw_{t-1}}(s_t)-\bp_{\gT_t^\pi{\bm{\eta}_{\bw_{t-1}}}}(s_t)}^{\intercal}\bC^{\intercal}\bC\\
        &=2\iota_K\bphi(s_t)\brk{\bphi(s_t)^{\intercal}\bW_{t-1}+\prn{\frac{1}{K+1}\bm{1}_{K}-\bp_{\gT_t^\pi{\bm{\eta}_{\bw_{t-1}}}}(s_t)}^{\intercal}}\bC^{\intercal}\bC.
    \end{align*}
where $\bp_{\gT_t^\pi{\bm{\eta}_{\bw_{t-1}}}}(s_t)=\bp_{\bPi_K\gT_t^\pi{\bm{\eta}_{\bw_{t-1}}}(s_t)}=\prn{p_k\prn{\brk{\gT_t^\pi{\bm{\eta}_{\bw_{t-1}}}}(s_t)}}_{k=0}^{K-1}\in\RB^{K}$.
Now, we turn to compute $\bp_{\gT_t^\pi{\bm{\eta}_{\bw_{t-1}}}}(s_t)$.
According to Eqn.~\eqref{eq:categorical_prob},
    \begin{align*}
        p_k\prn{\brk{\gT_t^\pi{\bm{\eta}_{\bw_{t-1}}}}(s_t)}=&\EB_{X\sim \brk{\gT_t^\pi{\bm{\eta}_{\bw_{t-1}}}}(s_t)}\brk{\prn{1-\abs{\frac{X-x_k}{\iota_K}}}_+}\\
        =&\EB_{G\sim \eta_{\bw_{t-1}}(s_t^\prime)}\brk{\prn{1-\abs{\frac{r_t+\gamma G-x_k}{\iota_K}}}_+}\\
=&\sum_{j=0}^Kp_j(s_t^\prime;
\bw_{t-1})g_{j,k}(r_t)\\
=&g_{K,k}(r_t)+\sum_{j=0}^{K-1}p_j(s_t^\prime;
\bw_{t-1})\prn{g_{j,k}(r_t)-g_{K,k}(r_t)},
    \end{align*}
which has the same form as Eqn.~\eqref{eq:eq_in_proof_Pi_K_T}.
Following the proof of Proposition~\ref{prop:Pi_K_T} in Appendix~\ref{appendix:proof_Pi_K_T}, one can show that
\begin{equation}\label{eq:project_bellman_w}
    \begin{aligned}
             \bp_{\gT_t^\pi{\bm{\eta}_{\bw_{t-1}}}}(s_t)=\tilde{\bG}(r_t) \bW^{\intercal}_{t-1} \bphi(s_t^\prime)+\frac{1}{K+1}\sum_{j=0}^K\bg_j(r_t).
    \end{aligned}
\end{equation}
Hence, the update scheme is
\begin{equation}\label{eq:ssgd_pmf}
    \begin{aligned}
\bW_t=&\bW_{t-1}-2\iota_K\alpha\bphi(s_t)\prn{\bp_{\bw_{t-1}}(s_t)-\bp_{\gT_t^\pi{\bm{\eta}_{\bw_{t-1}}}}(s_t)}^{\intercal}\bC^{\intercal}\bC\\
=&\bW_{t-1}-2\iota_K\alpha\bphi(s_t)\brk{\bphi(s_t)^{\intercal}\bW_{t-1}-\bphi(s_t^\prime)^{\intercal}\bW_{t-1}\tilde{\bG}^{\intercal}(r_t)-\frac{1}{K+1}\prn{\sum_{j=0}^K\bg_j(r_t)-\bm{1}_{K}}^{\intercal}}\bC^{\intercal}\bC.
\end{aligned}
\end{equation}
Compared to {\LCTD} (Eqn.~\eqref{eq:linear_CTD}), this form has an additional $\bC^{\intercal}\bC$, and the step size is $2\iota_K\alpha$.

\subsubsection{CDF Representation}
Consider the SSGD with the CDF representation
\[
    \bTheta_t=\bTheta_{t-1}-\alpha\nabla_{\bTheta}\ell_2^2\prn{{\eta}_{\bw_{t-1}}(s_t), \brk{\gT_t^\pi{\bm{\eta}_{\bw_{t-1}}}}(s_{t})},
\]
where $\nabla_{\bTheta}$ stands for taking gradient w.r.t.\ $\bTheta_{t-1}=\bW_{t-1}\bC^{\intercal} \in \RB^{d\times K}$ in the first term ${\eta}_{\bw_{t-1}}(s_t)$ (the second term is regarding as a constant). 
One can check that $\nabla_{\bTheta}\ell_2^2\prn{{\eta}_{\bw_{t-1}}(s_t), \brk{\gT_t^\pi{\bm{\eta}_{\bw_{t-1}}}}(s_{t})}$ is an unbiased estimate of $\nabla_{\bTheta} \ell_{2,\mu_{\pi}}^2\prn{\bm{\eta}_{\bw_{t-1}},\gT^\pi{\bm{\eta}_{\bw_{t-1}}}}$.

Now, let us compute the gradient term.
By Eqn.~\eqref{eq:gradient_cdf_representation} and Eqn.~\eqref{eq:project_bellman_w}, we have
    \begin{align*}
        &\nabla_{\bTheta} \ell_2^2\prn{{\eta}_{\bw_{t-1}}(s_t),\brk{\gT_t^\pi{\bm{\eta}_{\bw_{t-1}}}}(s_{t})}\\
        &\qquad=2\iota_K\bphi(s_t)\prn{\bF_{\btheta_{t-1}}(s_t)-\bF_{\gT_t^\pi{\bm{\eta}_{\bw_{t-1}}}}(s_t)}^{\intercal}\\
        &\qquad=2\iota_K\bphi(s_t)\brk{\bphi(s_t)^{\intercal}\bTheta_{t-1}+\prn{\frac{1}{K+1}\bm{1}_{K}-\bp_{\gT_t^\pi{\bm{\eta}_{\bw_{t-1}}}}(s)}^{\intercal}\bC^{\intercal}}\\
        &\qquad=2\iota_K\bphi(s_t)\brk{\bphi(s_t)^{\intercal}\bTheta_{t-1}-\bphi(s_t^\prime)^{\intercal}\bTheta_{t-1}\bC^{-\intercal}\tilde{\bG}^{\intercal}(r_t)\bC^{\intercal}-\frac{1}{K+1}\prn{\sum_{j=0}^K\bg_j(r_t)-\bm{1}_{K}}^{\intercal}\bC^{\intercal}}.
    \end{align*}
Hence, the update scheme is
\begin{equation}\label{eq:CDF_SSGD_update}
   \begin{aligned}
\bTheta_t=&\bTheta_{t-1}-2\iota_K\alpha\bphi(s_t)\prn{\bF_{\btheta_{t-1}}(s_t)-\bF_{\gT_t^\pi{\bm{\eta}_{\bw_{t-1}}}}(s_t)}^{\intercal}\\
=&\bTheta_{t-1}-2\iota_K\alpha\bphi(s_t)\brk{\bphi(s_t)^{\intercal}\bTheta_{t-1}-\bphi(s_t^\prime)^{\intercal}\bTheta_{t-1}\bC^{-\intercal}\tilde{\bG}^{\intercal}(r_t)\bC^{\intercal}-\frac{1}{K+1}\prn{\sum_{j=0}^K\bg_j(r_t)-\bm{1}_{K}}^{\intercal}\bC^{\intercal}},
\end{aligned} 
\end{equation}
which is equivalent to
\begin{align*}
\bW_t=&\bW_{t-1}-2\iota_K\alpha\bphi(s_t)\prn{\bp_{\bw_{t-1}}(s_t)-\bp_{\gT_t^\pi{\bm{\eta}_{\bw_{t-1}}}}(s_t)}^{\intercal}\\
=&\bW_{t-1}-2\iota_K\alpha\bphi(s_t)\brk{\bphi(s_t)^{\intercal}\bW_{t-1}-\bphi(s_t^\prime)^{\intercal}\bW_{t-1}\tilde{\bG}^{\intercal}(r_t)-\frac{1}{K+1}\prn{\sum_{j=0}^K\bg_j(r_t)-\bm{1}_{K}}^{\intercal}},
\end{align*}
which has the same form as Linear-CTD (Eqn.~\eqref{eq:linear_CTD}) with the step size $2\alpha\iota_K$.




\section{Omitted Results and Proofs in Section~\ref{Section:analysis_linear_ctd}}
\subsection{Proof of Lemma~\ref{lem:translate_error_to_loss}}\label{appendix:proof_translate_error_to_loss}
\begin{proof}
By Lemma~\ref{lem:prob_basic_inequalities} and Eqn.~\eqref{eq:CDF_target_function}, we have
\begin{equation*}
\begin{aligned}
    \prn{\gL(\bw)}^2=&W_{1,\mu_{\pi}}^2\prn{\bm{\eta}_{\bw},\bm{\eta}_{\bw^{\star}}}\\
    \leq&\frac{1}{1-\gamma}\ell_{2,\mu_{\pi}}^2\prn{\bm{\eta}_{\bw},\bm{\eta}_{\bw^{\star}}}\\
    =&\frac{\iota_K}{1-\gamma}\tr\prn{\prn{\bTheta_1-\bTheta_2}^{\intercal}\bSigma_{\bphi}\prn{\bTheta_1-\bTheta_2}} \\
    =&\frac{1}{K(1-\gamma)^2}\norm{\btheta_1-\btheta_2}^2_{\bI_K\otimes\bSigma_{\bphi}},
\end{aligned}
\end{equation*}
We only need to show that
\begin{equation}\label{eq:bar_A_lower_bound}
\begin{aligned}
        \bI_K\otimes\bSigma_{\bphi}\preccurlyeq\frac{1}{(1-\sqrt\gamma)^2}\bar{\bA}^{\intercal}\prn{\bI_K\otimes\bSigma_{\bphi}^{-1}}\bar{\bA}\prn{\preccurlyeq\frac{4}{(1-\gamma)^2\lambda_{\min}}\bar{\bA}^{\intercal}\bar{\bA}},
\end{aligned}
\end{equation}
or equivalently,
    \begin{equation*}
\begin{aligned}
        \prn{\bI_K\otimes\bSigma_{\bphi}^{-\frac{1}{2}}}\bar{\bA}^{\intercal}\prn{\bI_K\otimes\bSigma_{\bphi}^{-1}}\bar{\bA}\prn{\bI_K\otimes\bSigma_{\bphi}^{-\frac{1}{2}}}\succcurlyeq \prn{1-\sqrt\gamma}^2.
\end{aligned}
\end{equation*}
Recall
    \begin{equation*}
\begin{aligned}
        \bar{\bA}&=\prn{\bI_K\otimes\bSigma_{\bphi}}-\EB_{s, r, s^\prime}\brk{\prn{\bC\tilde{\bG}(r)\bC^{-1}}\otimes\prn{\bphi(s)\bphi(s^\prime)^{\intercal}}},
\end{aligned}
\end{equation*}
then for any $\bw\in\RB^{dK}$ with $\norm{\bw}=1$, 
    \begin{equation*}
\begin{aligned}
        &\bw^{\intercal} \prn{\bI_K\otimes\bSigma_{\bphi}^{-\frac{1}{2}}}\bar{\bA}^{\intercal}\prn{\bI_K\otimes\bSigma_{\bphi}^{-1}}\bar{\bA}\prn{\bI_K\otimes\bSigma_{\bphi}^{-\frac{1}{2}}}\bw\\
        &\qquad=\norm{\prn{\bI_K\otimes\bSigma_{\bphi}^{-\frac{1}{2}}}\bar{\bA}\prn{\bI_K\otimes\bSigma_{\bphi}^{-\frac{1}{2}}}\bw}^2\\
        &\qquad=\norm{\bw- \EB_{s, r, s^\prime}\brk{\prn{\bC\tilde{\bG}(r)\bC^{-1}}\otimes\prn{\bSigma_{\bphi}^{-\frac{1}{2}}\bphi(s)\bphi(s^\prime)^{\intercal}\bSigma_{\bphi}^{-\frac{1}{2}}}}\bw }^2\\
        &\qquad\geq\prn{1-\norm{ \EB_{s, r, s^\prime}\brk{\prn{\bC\tilde{\bG}(r)\bC^{-1}}\otimes\prn{\bSigma_{\bphi}^{-\frac{1}{2}}\bphi(s)\bphi(s^\prime)^{\intercal}\bSigma_{\bphi}^{-\frac{1}{2}}}}\bw  } }^2.
\end{aligned}
\end{equation*}
It suffices to show that
    \begin{equation}\label{eq:biscuit_matrix_bound}
\begin{aligned}
        &\norm{ \EB_{s, r, s^\prime}\brk{\prn{\bC\tilde{\bG}(r)\bC^{-1}}\otimes\prn{\bSigma_{\bphi}^{-\frac{1}{2}}\bphi(s)\bphi(s^\prime)^{\intercal}\bSigma_{\bphi}^{-\frac{1}{2}}}}} \leq \sqrt\gamma.
\end{aligned}
\end{equation}
For brevity, we abbreviate $\bC\tilde{\bG}(r)\bC^{-1}$ as $\bY(r)=\prn{y_{ij}(r)}_{i,j=1}^K\in\RB^{K\times K}$.
Thus, it suffices to show that
\begin{equation*}
    \norm{\EB_{s, r, s^\prime}\brk{\bY(r)\otimes\prn{\bSigma_{\bphi}^{-\frac{1}{2}}\bphi(s)\bphi(s^\prime)^{\intercal}\bSigma_{\bphi}^{-\frac{1}{2}}}}}\leq\sqrt\gamma.
\end{equation*}
For any vectors $\bw=\prn{\bw(0)^{\intercal}, \cdots, \bw(K-1)^{\intercal}}$ and $\bv=\prn{\bv(0)^{\intercal}, \cdots, \bv(K-1)^{\intercal}}$ in $\RB^{dK}$, we define the corresponding matrices $\bW=\prn{\bw(0), \cdots, \bw(K-1)}$ and $\bV=\prn{\bv(0), \cdots, \bv(K-1)}$ in $\RB^{d\times K}$, then $\norm{\bw}=\norm{\bW}_{F}=\sqrt{\tr\prn{\bW^\intercal \bW}}$ and $\norm{\bv}=\norm{\bV}_{F}=\sqrt{\tr\prn{\bV^\intercal \bV}}$.
With these notations, we have
    \begin{equation*}
\begin{aligned}
        &\norm{\EB_{s, r, s^\prime}\brk{\bY(r)\otimes\prn{\bSigma_{\bphi}^{-\frac{1}{2}}\bphi(s)\bphi(s^\prime)^{\intercal}\bSigma_{\bphi}^{-\frac{1}{2}}}}}\\
        &\qquad =\sup_{\norm{\bw}=\norm{\bv}=1}\bw^\intercal \EB_{s, r, s^\prime}\brk{\bY(r)\otimes\prn{\bSigma_{\bphi}^{-\frac{1}{2}}\bphi(s)\bphi(s^\prime)^{\intercal}\bSigma_{\bphi}^{-\frac{1}{2}}}}\bv\\
        &\qquad = \sup_{\norm{\bw}=\norm{\bv}=1}\EB_{s, r, s^\prime}\brk{ \sum_{i,j=1}^K y_{ij}(r)\bw(i)^{\intercal}\bSigma_{\bphi}^{-\frac{1}{2}}\bphi(s)\bphi(s^\prime)^{\intercal}\bSigma_{\bphi}^{-\frac{1}{2}} \bv(j)   },
\end{aligned}
\end{equation*}
it is easy to check that
\begin{equation*}
    \bw(i)^{\intercal}\bSigma_{\bphi}^{-\frac{1}{2}}\bphi(s)\bphi(s^\prime)^{\intercal}\bSigma_{\bphi}^{-\frac{1}{2}} \bv(j) = \prn{\bW^{\intercal}\bSigma_{\bphi}^{-\frac{1}{2}}\bphi(s)\bphi(s^\prime)^{\intercal}\bSigma_{\bphi}^{-\frac{1}{2}}\bV}_{ij},
\end{equation*}
hence
    \begin{equation*}
\begin{aligned}
        &\norm{\EB_{s, r, s^\prime}\brk{\bY(r)\otimes\prn{\bSigma_{\bphi}^{-\frac{1}{2}}\bphi(s)\bphi(s^\prime)^{\intercal}\bSigma_{\bphi}^{-\frac{1}{2}}}}}\\
        &\qquad = \sup_{\norm{\bW}_F=\norm{\bV}_F=1}\EB_{s, r, s^\prime}\brk{ \sum_{i,j=1}^K y_{ij}(r)\prn{\bW^{\intercal}\bSigma_{\bphi}^{-\frac{1}{2}}\bphi(s)\bphi(s^\prime)^{\intercal}\bSigma_{\bphi}^{-\frac{1}{2}}\bV}_{ij}  }\\
        &\qquad = \sup_{\norm{\bW}_F=\norm{\bV}_F=1}\EB_{s, r, s^\prime}\brk{ \tr\prn{ \bY(r)\bV^{\intercal}\bSigma_{\bphi}^{-\frac{1}{2}}\bphi(s^\prime)\bphi(s)^{\intercal}\bSigma_{\bphi}^{-\frac{1}{2}}\bW  }}\\
        &\qquad = \sup_{\norm{\bW}_F=\norm{\bV}_F=1}\EB_{s, r, s^\prime}\brk{\bphi(s)^{\intercal}\bSigma_{\bphi}^{-\frac{1}{2}}\bW   \bY(r)\bV^{\intercal}\bSigma_{\bphi}^{-\frac{1}{2}}\bphi(s^\prime) }\\
        &\qquad \leq \sup_{\norm{\bW}_F=\norm{\bV}_F=1}\EB_{s, r, s^\prime}\brk{\norm{\bW^{\intercal}\bSigma_{\bphi}^{-\frac{1}{2}}\bphi(s)}   \norm{\bY(r)}\norm{\bV^{\intercal}\bSigma_{\bphi}^{-\frac{1}{2}}\bphi(s^\prime)} }\\
        &\qquad \leq \sqrt\gamma\sup_{\norm{\bW}_F=\norm{\bV}_F=1}\EB_{s, s^\prime}\brk{\norm{\bW^{\intercal}\bSigma_{\bphi}^{-\frac{1}{2}}\bphi(s)}   \norm{\bV^{\intercal}\bSigma_{\bphi}^{-\frac{1}{2}}\bphi(s^\prime)} }\\
        &\qquad \leq \sqrt\gamma\sup_{\norm{\bW}_F=\norm{\bV}_F=1}\sqrt{\EB_{s }\brk{\norm{\bW^{\intercal}\bSigma_{\bphi}^{-\frac{1}{2}}\bphi(s)}^2 }  \EB_{ s^\prime}\brk{\norm{\bV^{\intercal}\bSigma_{\bphi}^{-\frac{1}{2}}\bphi(s^\prime)}^2 }}\\
         &\qquad = \sqrt\gamma\sup_{\norm{\bW}_F=\norm{\bV}_F=1}\sqrt{\EB_{s }\brk{\bphi(s)^{\intercal}\bSigma_{\bphi}^{-\frac{1}{2}}\bW\bW^{\intercal}\bSigma_{\bphi}^{-\frac{1}{2}}\bphi(s)  }  \EB_{ s^\prime}\brk{\bphi(s^\prime)^{\intercal}\bSigma_{\bphi}^{-\frac{1}{2}}\bV\bV^{\intercal}\bSigma_{\bphi}^{-\frac{1}{2}}\bphi(s^\prime) }}\\
         &\qquad = \sqrt\gamma\sup_{\norm{\bW}_F=\norm{\bV}_F=1}\sqrt{\tr\prn{\bW\bW^{\intercal}\bSigma_{\bphi}^{-\frac{1}{2}}\EB_{s }\brk{\bphi(s)\bphi(s)^{\intercal}}\bSigma_{\bphi}^{-\frac{1}{2}}}   \tr\prn{\bV\bV^{\intercal}\bSigma_{\bphi}^{-\frac{1}{2}}\EB_{s^\prime }\brk{\bphi(s^\prime)\bphi(s^\prime)^{\intercal}}\bSigma_{\bphi}^{-\frac{1}{2}}}}\\
         &\qquad = \sqrt\gamma\sup_{\norm{\bW}_F=\norm{\bV}_F=1}\sqrt{\tr\prn{\bW\bW^{\intercal}}   \tr\prn{\bV\bV^{\intercal}}}\\
         &\qquad = \sqrt\gamma\sup_{\norm{\bW}_F=\norm{\bV}_F=1}\norm{\bW}_F\norm{\bV}_F\\
        &\qquad = \sqrt\gamma,
\end{aligned}
\end{equation*}
where we used $\norm{\bY(r)}\leq \sqrt\gamma$ for any $r\in[0,1]$ by Lemma~\ref{lem:Spectra_of_ccgcc}, and Cauchy-Schwarz inequality.

To summarize, we have shown the desired result
\begin{equation*}
    \brk{\prn{\bC^{\intercal}\bC}\otimes\bSigma_{\bphi}}^{-\frac{1}{2}}\bar{\bA}^{\intercal}\brk{\prn{\bC^{\intercal}\bC}\otimes\bSigma_{\bphi}}^{-1}\bar{\bA}\brk{\prn{\bC^{\intercal}\bC}\otimes\bSigma_{\bphi}}^{-\frac{1}{2}}\succcurlyeq \prn{1-\sqrt\gamma}^2.
\end{equation*}
\end{proof}

\subsection{Proof of Lemma~\ref{lem:upper_bound_error_quantities}}\label{appendix:proof_upper_bound_error_quantities}
\begin{proof}
For simplicity, we omit $t$ in the random variables, for example, we use $\bA$ to denote a random matrix with the same distribution as $\bA_t$.
In addition, we omit the subscripts in the expectation, the involving random variables are $s\sim\mu_{\pi}, a\sim\pi\prn{\cdot\mid s}, r\sim\gP_R\prn{\cdot\mid s,a}, s^\prime\sim P\prn{\cdot\mid s,a}$.
\paragraph{Bounding $C_A$.}
By Lemma~\ref{lem:spectral_norm_of_KP},
\begin{equation*}
    \begin{aligned}
        \norm{\bA}\leq &\norm{\bI_K\otimes\prn{\bphi(s)\bphi(s)^{\intercal}}}+\norm{\prn{\bC\tilde{\bG}(r)\bC^{-1}}\otimes\prn{\bphi(s)\bphi(s^\prime)^{\intercal}}}\\
        =&\norm{\bphi(s)}^2+\norm{\bphi(s)}\norm{\bphi(s^\prime)}\norm{\bC\tilde{\bG}(r)\bC^{-1}}\\
        \leq &1+\sqrt{\gamma},
    \end{aligned}
\end{equation*}
where we used Lemma~\ref{lem:Spectra_of_ccgcc}. 
Hence, $C_A\leq 2(1+\sqrt{\gamma})$.
\paragraph{Bounding $C_e$.}
By Eqn.~\eqref{eq:AAt_bound},
\begin{equation*}
    \begin{aligned}
        \norm{\bA\btheta^{\star}}^2=&\prn{\btheta^{\star}}^{\intercal}\bA^{\intercal}\bA\btheta^{\star}\\
        \leq& 2\prn{\norm{\btheta^{\star}}^2_{\bI_K\otimes\prn{\bphi(s)\bphi(s)^{\intercal}}}+\gamma\norm{\btheta^{\star}}^2_{\bI_K\otimes\prn{\bphi(s^\prime)\bphi(s^\prime)^{\intercal}}}}\\
        \leq&2(1+\gamma)\sup_{s\in\gS}\norm{\btheta^{\star}}_{\bI_K\otimes\prn{\bphi(s)\bphi(s)^{\intercal}}}^2\\
        \leq&2(1+\gamma)\sup_{s\in\gS}\norm{\bphi(s)}^2 \norm{\btheta^{\star}}^2\\
        \leq& 2(1+\gamma)\norm{\btheta^{\star}}^2.
    \end{aligned}
\end{equation*}
Hence
\begin{equation*}
    \begin{aligned}
        \norm{\bA\btheta^{\star}}\leq&\sqrt{2(1+\gamma)}\norm{\btheta^{\star}}.
    \end{aligned}
\end{equation*}

As for $\norm{\bb}$, 
\begin{equation}\label{eq:bt_upper_bound}
    \begin{aligned}
       \norm{\bb}=&\frac{1}{K+1}\norm{\brk{\bC\prn{\sum_{j=0}^K\bg_j(r)-\bm{1}_K}}\otimes\bphi(s)}\\
       \leq&\frac{1}{K+1}\norm{\bC\prn{\sum_{j=0}^K\bg_j(r)-\bm{1}_K}}\norm{\bphi(s)}\\
       \leq&\frac{1}{K+1}\norm{\bC\prn{\sum_{j=0}^K\bg_j(r)-\bm{1}_K}}.
    \end{aligned}
\end{equation}
By Proposition~\ref{prop:categorical_projection_operator} with ${\bm{\eta}}\in\sP^{\sgn}_K$ satisfying 
% $\bp_{\bm{\eta}}(\tilde s)=\frac{1}{K+1}\bm{1}_K$ 
$\eta(\tilde s)=\nu$ 
for all $\tilde s\in\gS$, where $\nu=\prn{K+1}^{-1}\sum_{k=0}^K \delta_{x_k}$ is the discrete uniform distribution, we can derive that, for any $r\in[0,1]$ and $s^\prime\in\gS$, it holds that
\begin{equation*}
    \begin{aligned}
        \frac{1}{K+1}\prn{\sum_{j=0}^K\bg_j(r)-\bm{1}_K}=&\prn{\bp_{\prn{b_{r,\gamma}}_\#{\eta}(s^\prime)}-\frac{1}{K+1}\bm{1}_K}-\tilde\bG(r) \prn{\bp_{\bm{\eta}}(s^{\prime})-\frac{1}{K+1}\bm{1}_{K}} \\
        =&\bp_{\prn{b_{r,\gamma}}_\#\nu}-\frac{1}{K+1}\bm{1}_K,
    \end{aligned}
\end{equation*}
Hence,
\begin{equation*}
    \begin{aligned}
       \norm{\bb}\leq&\frac{1}{K+1}\norm{\bC\prn{\sum_{j=0}^K\bg_j(r)-\bm{1}_K}}\\
       =&\norm{\bC\prn{\bp_{\prn{b_{r,\gamma}}_\#\nu}-\frac{1}{K+1}\bm{1}_K}}\\
       =& \frac{1}{\sqrt{\iota_K}}\ell_2\prn{\bPi_K(b_{r,\gamma})_\#(\nu),\nu}\\
       \leq& \sqrt{K(1-\gamma)}\ell_2\prn{(b_{r,\gamma})_\#(\nu) ,\nu}\\
       \leq& 3\sqrt{K}(1-\gamma),
    \end{aligned}
\end{equation*}
where we used the orthogonal decomposition (Proposition~\ref{prop:orthogonal_decomposition}) and an upper bound for $\ell_2\prn{(b_{r,\gamma})_\#(\nu) ,\nu}$ (Lemma~\ref{lem:norm_b_bound}).

In summary,
\begin{equation*}
    \begin{aligned}
        \norm{\be}=&\norm{\bA\btheta^{\star}-\bb}\\
        \leq&\norm{\bA\btheta^{\star}}+\norm{\bb}\\
        \leq& \sqrt{2(1+\gamma)}\norm{\btheta^{\star}}+3\sqrt{K}\prn{1-\gamma}.
    \end{aligned}
\end{equation*}
Hence, $C_e\leq \sqrt{2(1+\gamma)}\norm{\btheta^{\star}}+3\sqrt{K}\prn{1-\gamma}$.
\paragraph{Bounding $\tr\prn{\bSigma_{e}}$.}
\begin{equation*}
    \begin{aligned}
        \tr\prn{\bSigma_{e}}=&\EB\brk{\norm{\bA\btheta^{\star}-\bb}^2}\leq2\prn{\btheta^{\star}}^{\intercal}\EB\brk{\bA^{\intercal}\bA}\btheta^{\star}+2\EB\brk{\bb^{\intercal}\bb}.
    \end{aligned}
\end{equation*}
By Eqn.~\eqref{eq:EAAt_bound},
\begin{equation*}
    \begin{aligned}
\prn{\btheta^{\star}}^{\intercal}\EB\brk{\bA^{\intercal}\bA}\btheta^{\star}\leq 2(1+\gamma)\norm{\btheta^{\star}}_{\bI_K\otimes\bSigma_{\bphi}}^2.
    \end{aligned}
\end{equation*}
And by Lemma~\ref{lem:norm_b_bound},
\begin{equation*}
    \begin{aligned}
        \EB\brk{\bb^{\intercal}\bb}
        \leq &9K(1-\gamma)^2.
    \end{aligned}
\end{equation*}
To summarize,
\begin{equation*}
    \begin{aligned}
        \tr\prn{\bSigma_{e}}\leq& 4(1+\gamma)\norm{\btheta^{\star}}_{\bI_K\otimes\bSigma_{\bphi}}^2+18K(1-\gamma)^2.
    \end{aligned}
\end{equation*}
\end{proof}

\subsection{Proof of Lemma~\ref{lem:exponential_stable}}\label{appendix:proof_exponential_stable}
\begin{proof}
For simplicity, we use the same abbreviations as in Appendix~\ref{appendix:proof_upper_bound_error_quantities}.
As in the proof of \citep[Lemma~2][]{samsonov2024improved}, we only need to show that, for any $p\in\NB$, $\alpha\in\prn{0,(1-\sqrt\gamma)/(38p)}$, it holds that
\begin{equation*}
    \EB\brc{\brk{\prn{\bI_{dK}-\alpha\bA}^{\intercal}\prn{\bI_{dK}-\alpha\bA}}^p}\preccurlyeq \bI_{dK}-\frac{1}{2}\alpha p(1-\gamma)\bI_K\otimes\bSigma_{\bphi}\prn{\preccurlyeq \prn{1-\frac{1}{2}\alpha p (1-\gamma)\lambda_{\min}}\bI_{dK}}.
\end{equation*}
Let $\bB:=\bA+\bA^{\intercal}-\alpha \bA^{\intercal}\bA$ which satisfies $\prn{\bI_{dK}-\alpha\bA}^{\intercal}\prn{\bI_{dK}-\alpha\bA}=\bI_{dK}-\alpha\bB$, we need to bound $\EB\brk{\prn{\bI_{dK}-\alpha\bB}^p}$, it suffices to show that
\begin{equation*}
   \EB\brk{\bB}\succcurlyeq (1-\sqrt\gamma)\bI_K\otimes\bSigma_{\bphi},\quad  \EB\brk{\bB^p}\preccurlyeq \frac{17}{16}4^p\bI_K\otimes\bSigma_{\bphi} ,\quad \forall p\in\brc{2,3,\cdots},
\end{equation*}
if we take $\alpha\in\prn{0,(1-\sqrt\gamma)/(2(1+\gamma))}$. 

Given these results, we have, when $\alpha\in\prn{0,(1-\sqrt\gamma)/(38p)}$, it holds that
\begin{equation*}
\begin{aligned}
        \EB\brk{\prn{\bI_{dK}-\alpha\bB}^p}\preccurlyeq&\bI-\alpha p \EB\brk{\bB}+\sum_{l=2}^p\alpha^l \binom{p}{l}\EB\brk{\bB^l}\\
        \preccurlyeq&\bI-\prn{\alpha p (1-\sqrt\gamma)-\frac{17}{16}\sum_{l=2}^\infty\prn{4\alpha p}^l }\bI_K\otimes\bSigma_{\bphi}\\
        =&\bI-\prn{\alpha p (1-\sqrt\gamma)-\frac{17\alpha^2p^2}{1-4\alpha p }}\bI_K\otimes\bSigma_{\bphi}\\
        \preccurlyeq&\bI-\frac{1}{2}\alpha p(1-\sqrt\gamma)\bI_K\otimes\bSigma_{\bphi}
\end{aligned}
\end{equation*}

\paragraph{Lower Bound of $\EB\brk{\bB}$.}
To show $\EB\brk{\bB}\succcurlyeq (1-\sqrt\gamma)\bI_K\otimes\bSigma_{\bphi}$, we first show that $\EB\brk{\bA+\bA^{\intercal}}\succcurlyeq 2(1-\sqrt\gamma)\bI_K\otimes\bSigma_{\bphi}$, which is equivalent to $\prn{\bI_K\otimes\bSigma_{\bphi}}^{-\frac{1}{2}}\EB\brk{\bA+\bA^{\intercal}}\prn{\bI_K\otimes\bSigma_{\bphi}}^{-\frac{1}{2}}\succcurlyeq 2(1-\sqrt\gamma)$, where
\begin{equation*}
\begin{aligned}
    \EB\brk{\bA+\bA^{\intercal}}&=2\prn{\bI_K\otimes\bSigma_{\bphi} }-\EB\brk{\prn{\bC\tilde{\bG}(r)\bC^{-1}}\otimes\prn{\bphi(s)\bphi(s^\prime)^{\intercal}}}-\EB\brk{\prn{\bC\tilde{\bG}(r)\bC^{-1}}\otimes\prn{\bphi(s)\bphi(s^\prime)^{\intercal}}}^{\intercal}.
\end{aligned}
\end{equation*}
Then, for any $\bw\in\RB^{Kd}$ with $\norm{\bw}=1$, 
    \begin{equation*}
\begin{aligned}
        &\bw^{\intercal}\prn{\bI_K\otimes\bSigma_{\bphi}}^{-\frac{1}{2}}\EB\brk{\bA+\bA^{\intercal}}\prn{\bI_K\otimes\bSigma_{\bphi}}^{-\frac{1}{2}}\bw\\
        &\qquad=2-2\bw^{\intercal}\prn{\bI_K\otimes\bSigma_{\bphi}}^{-\frac{1}{2}}\EB\brk{\prn{\bC\tilde{\bG}(r)\bC^{-1}}\otimes\prn{\bphi(s)\bphi(s^\prime)^{\intercal}}}\prn{\bI_K\otimes\bSigma_{\bphi}}^{-\frac{1}{2}}  \bw\\
        &\qquad\geq 2-2\norm{\EB_{s, r, s^\prime}\brk{\prn{\bC\tilde{\bG}(r)\bC^{-1}}\otimes\prn{\bSigma_{\bphi}^{-\frac{1}{2}}\bphi(s)\bphi(s^\prime)^{\intercal}\bSigma_{\bphi}^{-\frac{1}{2}}}}}\\
        &\qquad \geq 2(1-\sqrt\gamma),
\end{aligned}
\end{equation*}
where we used the result Eqn.~\eqref{eq:biscuit_matrix_bound}.

Next, we give an upper bound for $\EB\brk{\bA^{\intercal}\bA}$, we need to compute the following terms: by Lemma~\ref{lem:spectra_of_PSD_KP},
\begin{equation}\label{eq:ctc_phiphi_2_bound}
\begin{aligned}
   \brk{\bI_K\otimes\prn{\bphi(s)\bphi(s)^{\intercal}}}^2=&\bI_K\otimes\prn{\bphi(s)\bphi(s)^{\intercal}\bphi(s)\bphi(s)^{\intercal}}\\
   =&\norm{\bphi(s)}^2\bI_K\otimes\prn{\bphi(s)\bphi(s)^{\intercal}}\\
   \preccurlyeq&\bI_K\otimes\prn{\bphi(s)\bphi(s)^{\intercal}}.
\end{aligned}
\end{equation}
Hence
\begin{equation*}
\begin{aligned}
   \EB\brk{\bI_K\otimes\prn{\bphi(s)\bphi(s)^{\intercal}}}^2\preccurlyeq&\bI_K\otimes\bSigma_{\bphi}.
\end{aligned}
\end{equation*}
And by Lemma~\ref{lem:spectra_of_PSD_KP} and Lemma~\ref{lem:Spectra_of_ccgcc},
\begin{equation*}
\begin{aligned}
   &\brk{\prn{\bC\tilde{\bG}(r)\bC^{-1}}\otimes\prn{\bphi(s)\bphi(s^\prime)^{\intercal}}}^{\intercal}\brk{\prn{\bC\tilde{\bG}(r)\bC^{-1}}\otimes\prn{\bphi(s)\bphi(s^\prime)^{\intercal}}}\\
   &\qquad =\prn{\bC^{-\intercal}\tilde{\bG}^{\intercal}(r)\bC^{\intercal}\bC\tilde{\bG}(r)\bC^{-1}}\otimes\prn{\bphi(s^\prime)\bphi(s)^{\intercal}\bphi(s)\bphi(s^\prime)^{\intercal}}\\
   &\qquad =\norm{\bphi(s)}^2\prn{\bC^{-\intercal}\tilde{\bG}^{\intercal}(r)\bC^{\intercal}\bC\tilde{\bG}(r)\bC^{-1}}\otimes\prn{\bphi(s^\prime)\bphi(s^\prime)^{\intercal}}\\
   &\qquad \preccurlyeq \norm{\bC\tilde{\bG}(r)\bC^{-1}}^2\bI_K\otimes\prn{\bphi(s^\prime)\bphi(s^\prime)^{\intercal}}\\
   &\qquad \preccurlyeq \gamma\bI_K\otimes\prn{\bphi(s^\prime)\bphi(s^\prime)^{\intercal}}.
\end{aligned}
\end{equation*}
To summarize, by the basic inequality $\prn{\bB_1-\bB_2}^{\intercal}\prn{\bB_1-\bB_2}\preccurlyeq 2\prn{\bB_1^{\intercal}\bB_1+\bB_2^{\intercal}\bB_2}$, we have
\begin{equation}\label{eq:AAt_bound}
\begin{aligned}
    \bA^{\intercal}\bA\preccurlyeq&2\bI_K\otimes\prn{\bphi(s)\bphi(s)^{\intercal}+\gamma\bphi(s^\prime)\bphi(s^\prime)^{\intercal}},
\end{aligned}
\end{equation}
and, after taking expectation, 
\begin{equation}\label{eq:EAAt_bound}
\begin{aligned}
    \EB\brk{\bA^{\intercal}\bA}\preccurlyeq&2\prn{1+\gamma}\bI_K\otimes\bSigma_{\bphi}.
\end{aligned}
\end{equation}
Combine these together, we obtain
\begin{equation*}
\begin{aligned}
    \EB\brk{\bB}&\succcurlyeq 2\brk{(1-\sqrt\gamma)-\alpha(1+\gamma) }\bI_K\otimes\bSigma_{\bphi}\succcurlyeq (1-\sqrt\gamma)\bI_K\otimes\bSigma_{\bphi},
\end{aligned}
\end{equation*}
if we take $\alpha\in\prn{0,(1-\sqrt\gamma)/(2(1+\gamma))}$.

\paragraph{Upper Bound of $\EB\brk{\bB^p}$.}
Because $\bB^2$ is always PSD, we have the following upper bound
\begin{equation*}
\bB^p\preccurlyeq \norm{\bB}^{p-2}\bB^2.    
\end{equation*}
We first give an almost-sure upper bound for $\norm{\bB}$.
By Lemma~\ref{lem:upper_bound_error_quantities}, $\norm{\bA}\leq 1+\sqrt{\gamma}$.
And by Eqn.~\eqref{eq:AAt_bound},
\begin{equation}\label{eq:norm_AAt_bound}
\begin{aligned}
    \norm{\bA^{\intercal}\bA}\leq&2\norm{\bI_K\otimes\prn{\bphi(s)\bphi(s)^{\intercal}+\gamma\bphi(s^\prime)\bphi(s^\prime)^{\intercal}}}\\
    \leq&2\norm{\bphi(s)\bphi(s)^{\intercal}+\gamma\bphi(s^\prime)\bphi(s^\prime)^{\intercal}}\\
    \leq& 2(1+\gamma).
\end{aligned}
\end{equation}
Hence,
\begin{equation}\label{eq:norm_B_bound}
\begin{aligned}
    \norm{\bB}=&\norm{\bA+\bA^{\intercal}-\alpha\bA^{\intercal}\bA}\\
    \leq &2\norm{\bA}+\alpha\norm{\bA^{\intercal}\bA}\\
    \leq& 2(1+\sqrt{\gamma})+2\alpha(1+\gamma) \\
    \leq& 4,
\end{aligned}
\end{equation}
because $\alpha\in\prn{0,(1-\sqrt\gamma)/(2(1+\gamma))}$.

Now, we aim to give an upper bound for $\EB\brk{\bB^2}$,
\begin{equation*}
\begin{aligned}
    \bB^2=&\prn{\bA+\bA^{\intercal}-\alpha\bA^{\intercal}\bA}^2\\
    \preccurlyeq& (1+\beta)\prn{\bA+\bA^{\intercal}}^2+\prn{1+\beta^{-1}}\alpha^2\prn{\bA^{\intercal}\bA}^2\\
    \preccurlyeq&2(1+\beta)\prn{\bA^{\intercal}\bA+\bA\bA^{\intercal}}+(1+\beta^{-1})\alpha^2\prn{\bA^{\intercal}\bA}^2,
\end{aligned}
\end{equation*}
where we used the fact that $\prn{\bB_1+\bB_2}^2\preccurlyeq (1+\beta)\bB_1^2+(1+\beta^{-1})\bB_2^2$ for any symmetric matrices $\bB_1,\bB_2$, since $\beta\bB_1^2+\beta^{-1}\bB_2^2-\bB_1\bB_2-\bB_2\bB_1=\prn{\sqrt{\beta}\bB_1-\sqrt{\beta^{-1}}\bB_2}^2\succcurlyeq \bm{0}$, $\beta\in (0, 1)$ to be determined; and the fact that   $\bA^2+\prn{\bA^{\intercal}}^2\preccurlyeq\bA^{\intercal}\bA+\bA\bA^{\intercal}$ since the square of the skew-symmetric matrix is negative semi-definite $\prn{\bA-\bA^{\intercal}}^2\preccurlyeq \bm{0}$.
By Eqn.~\eqref{eq:norm_AAt_bound} and Eqn.~\eqref{eq:EAAt_bound}, we have
\begin{equation*}
\begin{aligned}
    \norm{\bA^{\intercal}\bA}\leq 2(1+\gamma),
\end{aligned}
\end{equation*}
\begin{equation}\label{eq:upper_bound_B_part_1}
\begin{aligned}
    \EB\brk{\bA^{\intercal}\bA}\preccurlyeq&2\prn{1+\gamma}\bI_K\otimes\bSigma_{\bphi},
\end{aligned}
\end{equation}
thus, by $\alpha\in\prn{0,(1-\sqrt\gamma)/(2(1+\gamma))}$, it holds that
\begin{equation}\label{eq:upper_bound_B_part_2}
\begin{aligned}
    \alpha^2\EB\brk{\prn{\bA^{\intercal}\bA}^2}\preccurlyeq4\alpha^2(1+\gamma)^2\bI_K\otimes\bSigma_{\bphi}\preccurlyeq (1-\sqrt\gamma)^2 \bI_K\otimes\bSigma_{\bphi}.
\end{aligned}
\end{equation}
As for $\EB\brk{\bA\bA^{\intercal}}$, by the basic inequality $\prn{\bB_1-\bB_2}\prn{\bB_1-\bB_2}^{\intercal}\preccurlyeq 2\prn{\bB_1\bB_1^{\intercal}+\bB_2\bB_2^{\intercal}}$, we have
\begin{equation*}
\begin{aligned}
    \bA\bA^{\intercal}=&\brc{\brk{\bI_K\otimes\prn{\bphi(s)\bphi(s)^{\intercal}}}-\brk{\prn{\bC\tilde{\bG}(r)\bC^{-1}}\otimes\prn{\bphi(s)\bphi(s^\prime)^{\intercal}}}}\\
    &\cdot\brc{\brk{\bI_K\otimes\prn{\bphi(s)\bphi(s)^{\intercal}}}-\brk{\prn{\bC\tilde{\bG}(r)\bC^{-1}}\otimes\prn{\bphi(s)\bphi(s^\prime)^{\intercal}}}}^{\intercal}\\
    \preccurlyeq& 2\brk{\bI_K\otimes\prn{\bphi(s)\bphi(s)^{\intercal}}}^2+2\brk{\prn{\bC\tilde{\bG}(r)\bC^{-1}}\otimes\prn{\bphi(s)\bphi(s^\prime)^{\intercal}}}\brk{\prn{\bC\tilde{\bG}(r)\bC^{-1}}\otimes\prn{\bphi(s)\bphi(s^\prime)^{\intercal}}}^{\intercal}.
\end{aligned}
\end{equation*}
By Eqn.~\eqref{eq:ctc_phiphi_2_bound}, we have
\begin{equation*}
\begin{aligned}
\brk{\bI_K\otimes\prn{\bphi(s)\bphi(s)^{\intercal}}}^2\preccurlyeq&\bI_K\otimes\prn{\bphi(s)\bphi(s)^{\intercal}}.
\end{aligned}
\end{equation*}
And by Lemma~\ref{lem:Spectra_of_ccgcc},
\begin{equation*}
\begin{aligned}
   &\brk{\prn{\bC\tilde{\bG}(r)\bC^{-1}}\otimes\prn{\bphi(s)\bphi(s^\prime)^{\intercal}}}\brk{\prn{\bC\tilde{\bG}(r)\bC^{-1}}\otimes\prn{\bphi(s)\bphi(s^\prime)^{\intercal}}}^{\intercal}\\
   &\qquad =\prn{\bC\tilde{\bG}(r)\bC^{-1}\bC^{-T}\tilde{\bG}^{\intercal}(r)\bC^{\intercal}}\otimes\prn{\bphi(s)\bphi(s^\prime)^{\intercal}\bphi(s^\prime)\bphi(s)^{\intercal}}\\
   &\qquad =\norm{\bphi(s^\prime)}^2\prn{\bC\tilde{\bG}(r)\bC^{-1}\bC^{-T}\tilde{\bG}^{\intercal}(r)\bC^{\intercal}}\otimes\prn{\bphi(s)\bphi(s)^{\intercal}}\\
   &\qquad \preccurlyeq \norm{\bC\tilde{\bG}(r)\bC^{-1}}^2\bI_K\otimes\prn{\bphi(s)\bphi(s)^{\intercal}}\\
   &\qquad \preccurlyeq \gamma  \bI_K\otimes\prn{\bphi(s)\bphi(s)^{\intercal}}.
\end{aligned}
\end{equation*}
To summarize, we have
\begin{equation*}
\begin{aligned}
    \bA\bA^{\intercal}\preccurlyeq&2(1+\gamma)\bI_K\otimes\prn{\bphi(s)\bphi(s)^{\intercal}},
\end{aligned}
\end{equation*}
after taking expectation
\begin{equation}\label{eq:upper_bound_B_part_3}
\begin{aligned}
    \EB\brk{\bA\bA^{\intercal}}\preccurlyeq&2\prn{1+\gamma}\bI_K\otimes\bSigma_{\bphi}.
\end{aligned}
\end{equation}
Put everything together (Eqn.~\eqref{eq:upper_bound_B_part_1}, Eqn.~\eqref{eq:upper_bound_B_part_2}, Eqn.~\eqref{eq:upper_bound_B_part_3}), we have
\begin{equation*}
\begin{aligned}
    \bB^2\preccurlyeq&2(1+\beta)\prn{\bA^{\intercal}\bA+\bA\bA^{\intercal}}+(1+\beta^{-1})\alpha^2\prn{\bA^{\intercal}\bA}^2\\
    \preccurlyeq&\brk{8(1+\gamma)(1+\beta)+(1+\beta^{-1})(1-\sqrt{\gamma})^2}\bI_K\otimes\bSigma_{\bphi}\\
    \preccurlyeq&\prn{17+9\gamma-10\sqrt{\gamma}}\bI_K\otimes\bSigma_{\bphi}\\
    \preccurlyeq&17\bI_K\otimes\bSigma_{\bphi},
\end{aligned}
\end{equation*}
where we take $\beta=\frac{1-\sqrt{\gamma}}{\sqrt{8(1+\gamma)}}$.
Therefore, by Eqn.~\eqref{eq:norm_B_bound},
\begin{equation*}
\begin{aligned}
    \bB^p\preccurlyeq& \norm{\bB}^{p-2}\bB^2\\
    \preccurlyeq& 4^{p-2}17\bI_K\otimes\bSigma_{\bphi}\\
    =&\frac{17}{16}4^p\bI_K\otimes\bSigma_{\bphi}.
\end{aligned}
\end{equation*}
\end{proof}

\subsection{Proof of Theorem~\ref{thm:l2_error_linear_ctd}}\label{appendix:proof_l2_error_linear_ctd}
\begin{proof}
Combining Lemma~\ref{lem:translate_error_to_loss}, Lemma~\ref{lem:upper_bound_error_quantities} and Lemma~\ref{lem:exponential_stable} with \citep[Theorem~1][]{samsonov2024improved}, we have
    \begin{equation*}
       \begin{aligned}
        \EB^{1/2}\brk{\prn{\gL\prn{\bar\bw_T}}^2}\lesssim& \frac{1}{\sqrt{K(1-\gamma)^4\lambda_{\min}}}\Bigg[\sqrt{\frac{\tr\prn{\bSigma_e}}{T}}\prn{1+\frac{C_A\sqrt{\alpha}}{\sqrt{a}}}+\frac{\sqrt{\tr\prn{\bSigma_e}}}{\sqrt{\alpha a}T}\\
        &+(1-\alpha a)^{T/2}\prn{\frac{1}{\alpha T}+\frac{C_A}{\sqrt{\alpha a}T}}\norm{\btheta_0-\btheta^{\star}} \Bigg]\\
        \lesssim&\frac{1}{\sqrt{K(1-\gamma)^4\lambda_{\min}}}\Bigg[\frac{\norm{\btheta^{\star}}_{\bI_K\otimes\bSigma_{\bphi}}+\sqrt{K}(1-\gamma)}{\sqrt{T}}\prn{1+\sqrt{\frac{\alpha}{(1-\gamma)\lambda_{\min}}}}\\
        &+\frac{\norm{\btheta^{\star}}_{\bI_K\otimes\bSigma_{\bphi}}+\sqrt{K}(1-\gamma)}{\sqrt{\alpha (1-\gamma)\lambda_{\min}}T}\\
        &+(1-\frac{1}{2}\alpha (1-\sqrt\gamma)\lambda_{\min} )^{T/2}\prn{\frac{1}{\alpha T}+\frac{1}{\sqrt{\alpha (1-\gamma)\lambda_{\min}}T}}\norm{\btheta_0-\btheta^{\star}} \Bigg]\\
        \lesssim&\frac{1}{\sqrt{T}}\frac{\frac{1}{\sqrt{K}(1-\gamma)}\norm{\btheta^{\star}}_{\bI_K\otimes\bSigma_{\bphi}}+1}{(1-\gamma)\sqrt{\lambda_{\min}}}\prn{1+\sqrt{\frac{\alpha}{(1-\gamma)\lambda_{\min}}}}\\
        &+\frac{1}{T}\frac{\frac{1}{\sqrt{K}(1-\gamma)}\norm{\btheta^{\star}}_{\bI_K\otimes\bSigma_{\bphi}}+1}{\sqrt{\alpha }(1-\gamma)^{\frac{3}{2}}\lambda_{\min}}\\
        &+\frac{1}{T}\frac{(1-\frac{1}{2}\alpha (1-\sqrt\gamma)\lambda_{\min} )^{T/2}}{ \sqrt{\alpha}(1-\gamma)\sqrt{\lambda_{\min}}}\prn{\frac{1}{\sqrt\alpha}+\frac{1}{\sqrt{ (1-\gamma)\lambda_{\min}}}}\frac{1}{\sqrt{K}(1-\gamma)}\norm{\btheta_0-\btheta^{\star}}.
       \end{aligned}
    \end{equation*}
\end{proof}

\subsection{\texorpdfstring{$L^p$}{Lp} Convergence}
\begin{theorem}[$L^p$ Convergence]\label{thm:lp_error_linear_ctd}
For any $K\geq (1-\gamma)^{-1}$, $p>2$, $T\geq 2$, $\alpha\in(0,(1-\sqrt\gamma)/[38(p+\log T)])$ and initialization $\bw_0\in\RB^{dK}$, it holds that
    \begin{equation*}
    \begin{aligned}
              \EB^{1/p}\brk{\prn{\gL\prn{\bar\bw_T}}^p}\lesssim&\frac{\sqrt{p}}{\sqrt{T}}\frac{\frac{1}{\sqrt{K}(1-\gamma)}\norm{\btheta^{\star}}_{\bI_K\otimes\bSigma_{\bphi}}+1}{(1-\gamma)\sqrt{\lambda_{\min}}}\prn{1+\frac{\sqrt{\alpha p}+\alpha p}{\sqrt{(1-\gamma)\lambda_{\min}}}}\\
        &+\frac{p}{T}\frac{\frac{1}{\sqrt{K}(1-\gamma)}\norm{\btheta^{\star}}_{\bI_K\otimes\bSigma_{\bphi}}+1}{(1-\gamma)^{\frac{3}{2}}\lambda_{\min}}\prn{1+\frac{1}{\sqrt{\alpha p}}}\\
        &+\frac{1}{T}\frac{(1-\frac{1}{2}\alpha (1-\sqrt\gamma)\lambda_{\min} )^{T/2}}{ \sqrt{\alpha}(1-\gamma)\sqrt{\lambda_{\min}}}\prn{\frac{1}{\sqrt\alpha}+\frac{p}{\sqrt{ (1-\gamma)\lambda_{\min}}}}\frac{1}{\sqrt{K}(1-\gamma)}\norm{\btheta_0-\btheta^{\star}}.
    \end{aligned}
    \end{equation*}
\end{theorem}
\begin{proof}
    Combining Lemma~\ref{lem:translate_error_to_loss}, Lemma~\ref{lem:upper_bound_error_quantities} and Lemma~\ref{lem:exponential_stable} with \citep[Theorem~2][]{samsonov2024improved}, we have
    \begin{equation*}
       \begin{aligned}
        &\EB^{1/p}\brk{\prn{\gL\prn{\bar\bw_T}}^p}\\
        &\qquad\lesssim \frac{1}{\sqrt{K(1-\gamma)^4\lambda_{\min}}}\Bigg[\sqrt{\frac{p\tr\prn{\bSigma_e}}{T}}\prn{1+\frac{C_A\sqrt{\alpha p}}{\sqrt{a}}+\frac{C_A C_e \alpha p}{\sqrt{\tr\prn{\bSigma_e}}}}+\frac{(1+C_A)C_e p}{T}\\
        &\qquad\quad+\frac{p\sqrt{\tr\prn{\bSigma_e}}}{\sqrt{a}T}\prn{1+\frac{1}{\sqrt{\alpha p}}}+(1-\alpha a)^{T/2}\prn{\frac{1}{\alpha T}+\frac{C_A p}{\sqrt{\alpha a}T}}\norm{\btheta_0-\btheta^{\star}} \Bigg]\\
        &\qquad\lesssim\frac{1}{\sqrt{K(1-\gamma)^4\lambda_{\min}}}\Bigg[\sqrt{p}\frac{\norm{\btheta^{\star}}_{\bI_K\otimes\bSigma_{\bphi}}+\sqrt{K}(1-\gamma)}{\sqrt{T}}\prn{1+\frac{\sqrt{\alpha p}+\alpha p}{\sqrt{(1-\gamma)\lambda_{\min}}}}+\frac{p\prn{\norm{\btheta^{\star}}+\sqrt{K}\prn{1-\gamma}}}{T}\\
        &\qquad\quad+p\frac{\norm{\btheta^{\star}}_{\bI_K\otimes\bSigma_{\bphi}}+\sqrt{K}(1-\gamma)}{\sqrt{ (1-\gamma)\lambda_{\min}}T}\prn{1+\frac{1}{\sqrt{\alpha p}}}\\
        &\qquad\quad+(1-\frac{1}{2}\alpha (1-\sqrt\gamma)\lambda_{\min} )^{T/2}\prn{\frac{1}{\alpha T}+\frac{p}{\sqrt{\alpha (1-\gamma)\lambda_{\min}}T}}\norm{\btheta_0-\btheta^{\star}} \Bigg]\\
        &\qquad\lesssim\frac{\sqrt{p}}{\sqrt{T}}\frac{\frac{1}{\sqrt{K}(1-\gamma)}\norm{\btheta^{\star}}_{\bI_K\otimes\bSigma_{\bphi}}+1}{(1-\gamma)\sqrt{\lambda_{\min}}}\prn{1+\frac{\sqrt{\alpha p}+\alpha p}{\sqrt{(1-\gamma)\lambda_{\min}}}}\\
        &\qquad\quad+\frac{p}{T}\frac{\frac{1}{\sqrt{K}(1-\gamma)}\norm{\btheta^{\star}}_{\bI_K\otimes\bSigma_{\bphi}}+1}{(1-\gamma)^{\frac{3}{2}}\lambda_{\min}}\prn{1+\frac{1}{\sqrt{\alpha p}}}\\
        &\qquad\quad+\frac{1}{T}\frac{(1-\frac{1}{2}\alpha (1-\sqrt\gamma)\lambda_{\min} )^{T/2}}{ \sqrt{\alpha}(1-\gamma)\sqrt{\lambda_{\min}}}\prn{\frac{1}{\sqrt\alpha}+\frac{p}{\sqrt{ (1-\gamma)\lambda_{\min}}}}\frac{1}{\sqrt{K}(1-\gamma)}\norm{\btheta_0-\btheta^{\star}},
       \end{aligned}
    \end{equation*}
    where we used the fact that $\norm{\btheta^{\star}}\leq \prn{\lambda_{\min}}^{-1/2}\norm{\btheta^{\star}}_{\bI_K\otimes\bSigma_{\bphi}}$.
\end{proof}

\subsection{Convergence Results for SSGD with the PMF Representation}\label{Appendix:convergece_ssgd_pmf}
In this section, we present without proof, the counterparts of Lemma~\ref{lem:translate_error_to_loss}, Lemma~\ref{lem:upper_bound_error_quantities} and Lemma~\ref{lem:exponential_stable} for SSGD with the PMF representation.
These results will additionally depend on $K$.
The additional $K$-dependent terms arise because the condition number of $\bC^{\intercal}\bC$ scales with $K^2$ (Lemma~\ref{lem:spectra_of_CTC}).
These terms are inevitable within our theoretical framework. 
By \citep[Theorem~1][]{samsonov2024improved}, we find that we cannot obtain a convergence rate independent of $K$ for the vanilla form Eqn.~\eqref{eq:ssgd_pmf} without a preconditioner as in our {\LCTD}. 
A natural question is whether the convergence rate of this algorithm truly depends on $K$. 
We leave this as future work.

The proofs of these results only require minor modifications to the original proofs.
For the sake of simplicity, we will not introduce new notations. 
Instead, we will only add the subscript $\operatorname{PMF}$ to the original notations to indicate the difference.

According to Eqn.~\eqref{eq:ssgd_pmf} in Appendix~\ref{eq:PMF_linear_parametrization}, the algorithm corresponds to the following linear system for $\bw\in\RB^{dK}$
\begin{equation*}
     \bar{\bA}_{\operatorname{PMF}}\bw=\bar{\bb}_{\operatorname{PMF}},
\end{equation*}
where
\begin{equation*}
\begin{aligned}
        \bar{\bA}_{\operatorname{PMF}}&=\brk{\prn{\bC^{\intercal}\bC}\otimes\bSigma_{\bphi}}-\EB_{s, r, s^\prime}\brk{\prn{\bC^{\intercal}\bC\tilde{\bG}(r)}\otimes\prn{\bphi(s)\bphi(s^\prime)^{\intercal}}},
\end{aligned}
\end{equation*}
\begin{equation*}
    \bar{\bb}_{\operatorname{PMF}}=\frac{1}{K+1}\EB_{s, r}\brc{\brk{\bC^{\intercal}\bC\prn{\sum_{j=0}^K\bg_j(r)-\bm{1}_K}}\otimes\bphi(s)}.
\end{equation*}
\begin{lemma}
For any $\bw\in\RB^{dK}$, it holds that
    \begin{equation*}
      \prn{\gL(\bw)}^2\leq\frac{16}{K(1-\gamma)^4\lambda_{\min}} \norm{\bar{\bA}_{\operatorname{PMF}}\prn{{\bw}-\bw^{\star}}}^2.
    \end{equation*}
\end{lemma}
\begin{lemma}
    \begin{equation*}
       C_{A,\operatorname{PMF}}\leq 4K^2,
    \end{equation*}
    \begin{equation*}
        C_{e,\operatorname{PMF}}\leq 4K^{\frac{3}{2}}\prn{\sqrt{K}\norm{\bw^{\star}}+\prn{1-\gamma}},
    \end{equation*}
    \begin{equation*}
        \tr\prn{\bSigma_{e,\operatorname{PMF}}}\leq 18K^2\prn{\norm{\bw^{\star}}_{\prn{\bC^{\intercal}\bC}\otimes\bSigma_{\bphi}}^2+K(1-\gamma)^2}.
    \end{equation*}
\end{lemma}

\begin{lemma}
For any $p\geq 2$, let $a=(1-\sqrt\gamma)\lambda_{\min}/2$ and $\alpha_{p,\infty}=(1-\sqrt\gamma)/(38K^2)$ ($\alpha_{p,\infty}p\leq 1/2$).
Then for any $\alpha\in\prn{0,\alpha_{p,\infty}}$, $\bu\in\RB^{dK}$ and $n\in\NB$
    \begin{equation*}
       \EB^{1/p}\brk{\norm{\bGamma_{1:n,\operatorname{PMF}}^{(\alpha)}\bu}^p}\leq \prn{1-\alpha a}^n \norm{\bu}.
    \end{equation*}
\end{lemma}
\section{Other Technical Lemmas}\label{Appendix_technical_lemmas}
\begin{lemma}\label{lem:prob_basic_inequalities}
For any $\nu_1, \nu_2\in\sP^{\sgn}$, we have $W_1(\nu_1,\nu_2)\leq\frac{1}{\sqrt{1-\gamma}}\ell_2(\nu_1,\nu_2)$. 
\end{lemma}
\begin{proof}
By Cauchy-Schwarz inequality,
\begin{equation*}
    \begin{aligned}
        W_1(\nu_1,\nu_2)=&\int_0^{\frac{1}{1-\gamma}} |F_{\nu_1}(x)-F_{\nu_2}(x)| dx\\
        \leq&\sqrt{\int_0^{\frac{1}{1-\gamma}} 1^2 dx}\sqrt{\int_0^{\frac{1}{1-\gamma}} |F_{\nu_1}(x)-F_{\nu_2}(x)|^2 dx}\\
        =&\frac{1}{\sqrt{1-\gamma}}\ell_2(\nu_1,\nu_2).
    \end{aligned}
\end{equation*}
\end{proof}

\begin{lemma}\label{lem:spectra_of_CTC}
Let $\bC\in\RB^{K\times K}$ be the matrix defined in Eqn.~\eqref{eq:def_C}, it holds that the eigenvalues of $\bC^{T}\bC$ are $1/(4\cos^2(k\pi/(2K+1))$ for $k\in [K]$ , and thus
\begin{equation}
    \norm{\bC^{\intercal}\bC} = \frac{1}{4\sin^2\frac{\pi}{4K+2}}\leq K^2, \quad \norm{\prn{\bC^{\intercal}\bC}^{-1}} = 4\sin^2\frac{K\pi}{4K+2}\leq 4.
\end{equation}
\end{lemma}
\begin{proof}
One can check that
\begin{equation*}
 \bC^{\intercal} \bC =   \begin{bmatrix} K & K - 1 & \cdots & 2 & 1 \\ K - 1 & K - 1 & \cdots & 2 & 1 \\ \vdots & \vdots & \ddots & \vdots & \vdots \\ 2 & 2 & \cdots & 2 & 1 \\ 1 & 1 & \cdots & 1 & 1 \end{bmatrix},
\end{equation*}
\begin{equation*}
 \prn{\bC^{\intercal} \bC}^{-1} =   \begin{bmatrix} 1 & -1 & 0 & \cdots & 0 & 0 \\ -1 & 2 & -1 & \cdots & 0 & 0 \\ 0 & -1 & 2 & \cdots & 0 & 0 \\ \vdots & \vdots & \vdots & \ddots & \vdots & \vdots \\ 0 & 0 & 0 & \cdots & 2 & -1 \\ 0 & 0 & 0 & \cdots & -1 & 1 \end{bmatrix}.
\end{equation*}
Then, one can work with the the inverse of $\bC^{\intercal}\bC$ and calculate its singular values by induction, which has similar forms to the analysis of Toeplitz's matrix. 
See \citep{godsil1985inverses} for more details.
\end{proof}

\begin{lemma}\label{lem:Spectra_of_ccgcc}
For any $r\in[0, 1]$, it holds that $\norm{\bC\tilde{\bG}(r)\bC^{-1}}\leq\sqrt{\gamma}$ and $\norm{\prn{\bC^{\intercal}\bC}^{1/2}\tilde{\bG}(r)\prn{\bC^{\intercal}\bC}^{-1/2}}\leq \sqrt\gamma$.
\end{lemma}
% \begin{remark}\label{remark:cgc}
% The matrix $\bC\tilde{\bG}(r)\bC^{-1}$ was also studied in \citep{rowland2024nearminimaxoptimal} as the matrix representation of the categorical projected Bellman operator of a specific one-state Markov reward process. 
% They derived the same upper bound using the contraction property of categorical projected 
% Our Lemma~\ref{lem:Spectra_of_ccgcc} provides a new analysis by directly analyzing the properties of the matrix.
% \end{remark}
% \begin{proof}[Proof of Lemma~\ref{lem:Spectra_of_ccgcc}]
\begin{proof}
One can check that
\begin{equation*}
\bC^{-1}=\begin{bmatrix} 1 & 0 & \cdots & 0 & 0\\ -1 & 1 & \cdots & 0 & 0\\ 0 & - 1 & \cdots & 0 & 0\\ \vdots & \vdots & \ddots & \vdots & \vdots\\ 0 & 0 & \cdots & -1 & 1 \end{bmatrix}.
\end{equation*}
It is clear that
\begin{equation}
\begin{aligned}
    \norm{\prn{\bC^{\intercal}\bC}^{1/2}\tilde{\bG}(r)\prn{\bC^{\intercal}\bC}^{-1/2}} \leq \sqrt{\gamma} &\iff \bY(r)\bY^{\intercal}(r)\preccurlyeq \gamma \bI\\
    &\iff \tilde{\bG}(r)(\bC^{\intercal}\bC)^{-1}\tilde{\bG}^{\intercal}(r)\preccurlyeq \gamma (\bC^{\intercal}\bC)^{-1}\\
    &\iff \bC\tilde{\bG}(r)(\bC^{\intercal}\bC)^{-1}\tilde{\bG}^{\intercal}(r)\bC^{\intercal}\preccurlyeq\gamma\bI\\
    &\iff \norm{\bC\tilde{\bG}(r)\bC^{-1}}\leq\sqrt{\gamma}.\\
\end{aligned}
\end{equation}
By Lemma~\ref{lem:spectra_CGC_} and an upper bound on the spectral norm (Riesz–Thorin interpolation theorem) \citep[Theorem~7.3][]{serre2002matrices}, we obtain that 
\begin{equation}
    \norm{\bC\tilde{\bG}(r)\bC^{-1}}\leq \sqrt{\norm{\bC\tilde{\bG}(r)\bC^{-1}}_{1}\norm{\bC\tilde{\bG}(r)\bC^{-1}}_{\infty}} \leq \sqrt{1 \cdot \gamma} = \sqrt{\gamma}.
\end{equation}
\end{proof}


\begin{lemma}\label{lem:norm_b_bound}
    Suppose $K\geq (1-\gamma)^{-1}$, $\nu=(K+1)^{-1}\sum_{k=0}^K\delta_{x_k}$ is the discrete uniform distribution, then for any $r\in[0, 1]$, it holds that
    \begin{equation*}
        \ell_2\prn{(b_{r,\gamma})_\#(\nu) ,\nu}\leq 3\sqrt{1-\gamma}.
    \end{equation*}
\end{lemma}
\begin{proof}
Let $\tilde{\nu}$ be the continuous uniform distribution on $\brk{0,(1-\gamma)^{-1}+\iota_K}$, we consider the following decomposition
\begin{equation*}
\begin{aligned}
        \ell_2\prn{\nu, (b_{r,\gamma})_\#(\nu)}\leq\ell_2\prn{\nu,\tilde{\nu}}+\ell_2\prn{\tilde\nu,(b_{r,\gamma})_\#(\tilde{\nu})}+\ell_2\prn{(b_{r,\gamma})_\#(\tilde{\nu}),(b_{r,\gamma})_\#(\nu)}.
\end{aligned}
\end{equation*}
By definition, we have
\begin{equation*}
\begin{aligned}
        \ell_2\prn{\nu,\tilde{\nu}}=&\sqrt{(K+1)\int_0^{\iota_K}\prn{(1-\gamma)\frac{K}{K+1}x}^2dx}\\
        =&\sqrt{\frac{1}{3K(K+1)(1-\gamma)}}\\
        \leq& \frac{1}{K\sqrt{1-\gamma}}.
\end{aligned}
\end{equation*}
By the contraction property, we have 
\begin{equation*}
\begin{aligned}
        \ell_2\prn{(b_{r,\gamma})_\#(\nu),(b_{r,\gamma})_\#(\tilde\nu)}\leq& \sqrt{\gamma}\ell_2\prn{\nu,\tilde{\nu}}\leq \frac{\sqrt{\gamma}}{K\sqrt{1-\gamma}}.
\end{aligned}
\end{equation*}
We only need to bound $\ell_2\prn{\tilde\nu,(b_{r,\gamma})_\#(\tilde{\nu})}$.
We can find that $(b_{r,\gamma})_\#(\tilde{\nu})$ is the continuous uniform distribution on $\brk{r,r+\gamma\iota_K+\gamma(1-\gamma)^{-1}}$, and the upper bound is less than the upper bound of $\nu$, namely, $r+\gamma\iota_K+\gamma(1-\gamma)^{-1}\leq (1-\gamma)^{-1}+\gamma\iota_K<(1-\gamma)^{-1}+\iota_K$.
Hence
\begin{equation*}
\begin{aligned}
        \ell_2^2\prn{\tilde\nu,(b_{r,\gamma})_\#(\tilde{\nu})}=&\int_{0}^r\prn{(1-\gamma)\frac{K}{K+1}x}^2dx+\int_{r}^{r+\gamma\iota_K+\gamma(1-\gamma)^{-1}}\brk{(1-\gamma)\frac{K}{K+1}\prn{x-\frac{x-r}{\gamma}}}^2dx\\
        &+\int_{r+\gamma\iota_K+\gamma(1-\gamma)^{-1}}^{(1-\gamma)^{-1}+\iota_K}\prn{1-(1-\gamma)\frac{K}{K+1}x}^2 dx\\
        =&\frac{(1-\gamma)^2K^2r^3}{3(K+1)^2}+\prn{\frac{(1-\gamma)\gamma K^2 r^3}{3(K+1)^2}+\frac{(1-\gamma)\gamma K^2 \prn{\frac{K+1}{K}-r}^3}{3(K+1)^2}}+\frac{(1-\gamma)^2K^2\prn{\frac{K+1}{K}-r}^3}{3(K+1)^2}\\
        \leq& (1-\gamma)^2+(1-\gamma)\gamma\\
        =& 1-\gamma.
\end{aligned}
\end{equation*}
To summarize, we have
\begin{equation*}
\begin{aligned}
        \ell_2\prn{\nu, (b_{r,\gamma})_\#(\nu)}\leq\frac{1}{K\sqrt{1-\gamma}}+\sqrt{1-\gamma}+\frac{\sqrt{\gamma}}{K\sqrt{1-\gamma}}\leq 3\sqrt{1-\gamma},
\end{aligned}
\end{equation*}
where we used the assumption $K\geq (1-\gamma)^{-1}$.
\end{proof}

\section{Analysis of the Categorical Projected Bellman Matrix}\label{appendix:analysis_cate_bellman_matrix}
Recall that $\tilde{\bG}(r)=\bG(r)-\bm{1}_K^{\intercal}\otimes\bg_K(r)$. We extend the definition in Theorem~\ref{thm:linear_cate_TD_equation} and let $g_{j,k}(r) = h\prn{(r+\gamma x_j-x_k)/\iota_K}_+ = h(r/\iota_{K}+\gamma j-k)$ for $j,k\in\{0,1,\cdots,K\}$ where $h(x) = \prn{1-\abs{x}}_+$.
\begin{lemma}
    For any $r\in[0,1]$ and any $k \in \{0,1,\cdots,K\}$, in $\bg_{k}(r)$ there is either only one nonzero entry or two adjacent nonzero entries.
\end{lemma}
\begin{proof}
It is clear that $h(x)>0 \iff -1<x<1$. 
Let $k_{j}(r) = \min\ \{k:g_{j,k}(r)>0\}$, then $k_{j}(r) = \min\{k:r/\iota_{K}+\gamma j-k<1\} = \min\{k:0\leq r/\iota_{K}+\gamma j-k<1\}$. 
The existence of $k_{j}(r)$ is due to
\begin{equation}
    r/\iota_{K}+\gamma j-K \leq 1/\iota_{K}+\gamma j-K\leq (1-\gamma)K+\gamma K-K = 0 < 1.
\end{equation}
Let $a_{j}(r) := r/\iota_{K}+\gamma j-k_{j}(r)\in[0,1)$. Then $g_{j,k_{j}(r)}(r)=h(a_{j}(r)) = 1-a_{j}(r)$ and $g_{j,k_{j}(r)+1}(r)=h(a_{j}(r)-1)=a_{j}(r)$ are the only entries that can be nonzeros.
\end{proof}
The following results are immediate corollaries.
\begin{corollary}\label{col:sum_column_g}
\begin{equation}
    \sum_{k=0}^{i}g_{j,k}(r)=
    \begin{cases}
    0, &\text{ for}\ 0\leq i<k_{j}(r),\\
    1-a_{j}(r),&\text{ for}\ i=k_{j}(r),\\
    1,&\text{ for}\ k_{j}(r)<i\leq K.\\
    \end{cases}
\end{equation}
\end{corollary}
\begin{corollary}
For any $\gamma<1$, it holds that
\begin{equation}
k_{j+1}(r) =
\begin{cases}
    k_{j}(r), &\text{ if}\  a_{j}(r)\leq 1-\gamma,\\
    k_{j}(r)+1, &\text{ if}\  a_{j}(r)>1-\gamma.\\
\end{cases}
\end{equation}
As a result,
\begin{equation}
a_{j+1}(r) =
\begin{cases}
    a_{j}(r)+\gamma, &\text{ if}\ a_{j}(r)\leq 1-\gamma,\\
    a_{j}(r)+\gamma-1, &\text{ if}\ a_{j}(r)>1-\gamma.\\
\end{cases}
\end{equation}



\end{corollary}
\begin{lemma}\label{lem:spectra_CGC_}
All entries in $\bC\tilde{\bG}(r)\bC^{-1}$ are non-negative. $\norm{\bC\tilde{\bG}(r)\bC^{-1}}_{\infty} = \gamma$ and $\norm{\bC\tilde{\bG}(r)\bC^{-1}}_{1} \leq 1$.
\begin{proof}
By definition the entries of $\tilde{\bG}(r)$ are 
\begin{equation}
    (\tilde{\bG}(r))_{j,i} = g_{j,i}(r)-g_{K,i}(r)\quad  \text{for } j,i \in  \{0,1,\cdots,K-1\}.
\end{equation}
Using the previous corollaries, direct calculation show that if $k_{j+1}(r) = k_{j}(r)$, 
\begin{equation}
(\bC\tilde{\bG}(r)\bC^{-1})_{j,i} = \sum_{k=0}^{i}g_{j,k}(r)-\sum_{k=0}^{i}g_{j+1,k}(r) = 
\begin{cases}
    0, &\text{ for}\ 0\leq i<k_{j}(r),\\
    a_{j+1}(r)-a_{j}(r),&\text{ for}\ i=k_{j}(r),\\
    0,&\text{ for}\ k_{j}(r)<i<K.\\
\end{cases}
\end{equation}
And if $k_{j+1}(r) = k_{j}(r)+1$,
\begin{equation}
(\bC\tilde{\bG}(r)\bC^{-1})_{j,i} = \sum_{k=0}^{i}g_{j,k}(r)-\sum_{k=0}^{i}g_{j+1,k}(r) = 
\begin{cases}
    0, &\text{ for}\ 0<i<k_{j}(r),\\
    1-a_{j}(r),&\text{ for}\ i=k_{j}(r),\\
    a_{j+1}(r), &\text{ for}\ i = k_{j+1}(r),\\
    0,&\text{ for}\ k_{j}(r)<i<K.\\
\end{cases}
\end{equation}
As a result, all entries in $\bC\tilde{\bG}(r)\bC^{-1}$ is non-negative. Moreover, the sum of each column and $\norm{\bC\tilde{\bG}(r)\bC^{-1}}_{\infty}$ is $\gamma$ because
\begin{equation}
    \sum_{i=0}^{K-1} (\bC\tilde{\bG}(r)\bC^{-1})_{j,i}  = 
    \begin{cases}
    a_{j+1}(r)-a_{j}(r) = \gamma, &\text{ if}\ k_{j+1}(r) = k_{j}(r),\\
    1-a_{j}(r)+a_{j+1}(r) = \gamma, &\text{ if}\ k_{j+1}(r) = k_{j}(r)+1.\\
    \end{cases}
\end{equation}
The row sum of $\bC\tilde{\bG}(r)\bC^{-1}$ is 
\begin{equation}
    \sum_{j=0}^{K-1} (\bC\tilde{\bG}(r)\bC^{-1})_{j,i}=\sum_{m=0}^{i}g_{0,m}(r)-\sum_{m=0}^{i}g_{K,m}(r)\leq 1-0 = 1.
\end{equation}
Thus, it holds that $\norm{\bC\tilde{\bG}(r)\bC^{-1}}_{1} \leq 1$.
\end{proof}
\end{lemma}


\end{document}
