In this section, we outline the proofs of our main results (Theorem~\ref{thm:l2_error_linear_ctd}).
We first state the theoretical properties of the linear-categorical Bellman equation and the exponential stability of {\LCTD}. 
Finally, we highlight some key steps in proving these results.
\subsection{Vectorization of Linear-CTD}
In our analysis, it will be more convenient to work with the vectorization version and the CDF representation introduced in Appendix~\ref{Appendix:cdf_representation}. 
In Appendix~\ref{appendix:equiv_ssgd_lctd}, we show that {\LCTD} (Eqn.~\eqref{eq:linear_CTD}) is equivalent to SSGD with the CDF representation (Eqn.~\eqref{eq:CDF_SSGD_update}).
We consider the vectorization of Eqn.~\eqref{eq:CDF_SSGD_update}, and we denote $\btheta_t=\prn{\bC\otimes\bI_d}\bw_t$ (other CDF parameters are defined in the same way):
\begin{equation*}
\begin{aligned}
    \btheta_t{=}\btheta_{t-1}{-}\alpha\prn{\bA_{t}\btheta_{t-1}{-}\bb_t},\quad \bA_t{=}\brk{\bI_K{\otimes}\prn{\bphi(s_t)\bphi(s_t)^{\intercal}}}{-}\brk{\prn{\bC\tilde{\bG}(r_t)\bC^{-1}}{\otimes}\prn{\bphi(s_t)\bphi(s^\prime_t)^{\intercal}}},
\end{aligned}
\end{equation*}
\begin{equation}\label{eq:bt}
\begin{aligned}
        \bb_t=\frac{1}{K+1}\brk{\bC\prn{\sum_{j=0}^K\bg_j(r_t)-\bm{1}_K}}\otimes\bphi(s_t).
\end{aligned}
\end{equation}
% \begin{equation*}
% \begin{aligned}
%     \btheta_t=&\btheta_{t-1}-\alpha\prn{\bA_{t}\btheta_{t-1}-\bb_t}=\btheta_{t-1}-\alpha\prn{\bF_{\btheta_{t-1}}(s_t)-\bF_{\brk{\prn{b_{r_t,\gamma}}_\#{{\eta}_{\bw_{t-1}}}}(s_{t}^\prime)}}\otimes\bphi(s_t),
% \end{aligned}
% \end{equation*}
% \begin{small}
% \begin{equation}\label{eq:at_and_bt}
% \begin{aligned}
%         \bA_t=&\brk{\bI_K\otimes\prn{\bphi(s_t)\bphi(s_t)^{\intercal}}}-\brk{\prn{\bC\tilde{\bG}(r_t)\bC^{-1}}\otimes\prn{\bphi(s_t)\bphi(s^\prime_t)^{\intercal}}},\quad \bb_t=\frac{1}{K+1}\brk{\bC\prn{\sum_{j=0}^K\bg_j(r_t)-\bm{1}_K}}\otimes\bphi(s_t).
% \end{aligned}
% \end{equation}
% \end{small}
% 
% 
% 
% which are unbiased estimates of
% \begin{equation*}
% \begin{aligned}
%         \bar{\bA}&=\prn{\bI_K\otimes\bSigma_{\bphi}}-\EB_{s, r, s^\prime}\brk{\prn{\bC\tilde{\bG}(r)\bC^{-1}}\otimes\prn{\bphi(s)\bphi(s^\prime)^{\intercal}}},
% \end{aligned}
% \end{equation*}
% \begin{equation*}
%     \bar{\bb}=\frac{1}{K+1}\EB_{s, r}\brc{\brk{\bC\prn{\sum_{j=0}^K\bg_j(r)-\bm{1}_K}}\otimes\bphi(s)},
% \end{equation*}
% respectively.
We denote by $\bar{\bA},\bar{\bb}$ the expectation of $\bA_t,\bb_t$ respectively.
Utilizing the exponential stability arguments, we can derive an upper bound for $\norm{\bar{\bA}\prn{\bar{\btheta}_T{-}\btheta^{\star}}}$.
We need to further translate it to an upper bound for $\gL(\bar{\bw}_T)$, which is done in the following lemma, whose proof can be found in Appendix~\ref{appendix:proof_translate_error_to_loss}.
\begin{lemma}\label{lem:translate_error_to_loss}
For any $\bw\in\RB^{dK}$, it holds that $\gL(\bw)\leq2{K^{-1/2}(1{-}\gamma)^{-2}\lambda_{\min}^{-1/2}} \norm{\bar{\bA}\prn{{\btheta}{-}\btheta^{\star}}}$.
    % \begin{equation*}
    %   \prn{\gL(\bw)}^2\leq\frac{4}{K(1-\gamma)^4\lambda_{\min}} \norm{\bar{\bA}\prn{{\btheta}-\btheta^{\star}}}^2.
    % \end{equation*}
\end{lemma}
\subsection{Exponential Stability Analysis}
% In the following, we will sketch the proof of non-asymptotic convergence results.
% We use the exponential stability argument for analyzing LSA proposed in \citep{samsonov2024improved}.
First, we introduce some notations.
% Let $\be_t:={\bA}_t\btheta^{\star}-{\bb}_t=\prn{\bF_{\btheta^{\star}}(s_t)-\bF_{\gT_t^\pi{\bm{\eta}_{\bw^{\star}}}}(s_t)}\otimes\bphi(s_t)$,
Let $\be_t:{=}{\bA}_t\btheta^{\star}{-}{\bb}_t$,
we denote by $C_{A}$ (resp. $C_{e}$) the almost sure upper bound for $\max\brc{\norm{\bA_t}, \norm{\bA_t{-}\bar{\bA}}}$ (resp. $\norm{\be_t}$), and $\bSigma_{e}:=\EB\brk{\be_t\be_t^{\intercal}}$ the covariance matrix of $\be_t$.
The following lemma provides useful upper bounds, whose proof can be found in Appendix~\ref{appendix:proof_upper_bound_error_quantities}.
\begin{lemma}\label{lem:upper_bound_error_quantities}
Suppose $K\geq(1-\gamma)^{-1}$, then it holds that
    \begin{equation*}
       C_{A}\leq 4,\quad C_{e}\leq 4(\norm{\btheta^{\star}}+\sqrt{K}\prn{1-\gamma}),\quad\tr\prn{\bSigma_{e}}\leq 18(\norm{\btheta^{\star}}_{\bI_K\otimes\bSigma_{\bphi}}^2+K(1-\gamma)^2).
    \end{equation*}
    % \begin{equation*}
    %    C_{A}\leq 4,
    % \end{equation*}
    % \begin{equation*}
    %     C_{e}\leq 4\prn{\norm{\btheta^{\star}}+\sqrt{K}\prn{1-\gamma}},
    % \end{equation*}
    % \begin{equation*}
    %     \tr\prn{\bSigma_{e}}\leq 18\prn{\norm{\btheta^{\star}}_{\bI_K\otimes\bSigma_{\bphi}}^2+K(1-\gamma)^2}.
    % \end{equation*}
\end{lemma}
% The key step for analyzing an LSA algorithm is to verify the exponential stability condition.
Let $\bGamma_{m:n}^{(\alpha)}:=\prod_{i=m}^n\prn{\bI-\alpha\bA_{i}}$ for any $\alpha>0$ and $m,n\in\NB$, $m\leq n$.
The exponential stability of {\LCTD} is summarized in the following lemma, whose proof can be found in Appendix~\ref{appendix:proof_exponential_stable}.
\begin{lemma}\label{lem:exponential_stable}
For any $p\geq 2$, let $a=(1-\sqrt\gamma)\lambda_{\min}/2$ and $\alpha_{p,\infty}=(1-\sqrt\gamma)/(38p)$ ($\alpha_{p,\infty}p\leq 1/2$).
Then for any $\alpha\in\prn{0,\alpha_{p,\infty}}$, $\bu\in\RB^{dK}$ and $n\in\NB$, it holds that $\EB^{1/p}[\|\bGamma_{1:n}^{(\alpha)}\bu\|^p]\leq (1-\alpha a)^n \norm{\bu}$.
    % \begin{equation*}
    %    \EB^{1/p}\brk{\norm{\bGamma_{1:n}^{(\alpha)}\bu}^p}\leq \prn{1-\alpha a}^n \norm{\bu}.
    % \end{equation*}
\end{lemma}
See Appendix~\ref{Appendix:convergece_ssgd_pmf} for the counterparts of Lemma~\ref{lem:translate_error_to_loss}, Lemma~\ref{lem:upper_bound_error_quantities} and Lemma~\ref{lem:exponential_stable} for vanilla SSGD with the PMF representation, which have additional dependency on $K$.

Combining Lemma~\ref{lem:translate_error_to_loss}, Lemma~\ref{lem:upper_bound_error_quantities} and Lemma~\ref{lem:exponential_stable} with \citep[Theorem~1][]{samsonov2024improved}, one can  immediately obtain Theorem~\ref{thm:l2_error_linear_ctd}, the remaining details can be found in Appendix~\ref{appendix:proof_l2_error_linear_ctd}.

\subsection{Key Steps in the Proofs}
In this section, we highlight some key steps in proving above theoretical results.
\paragraph{Bounding Spectral Norm of Expectation of Kronecker Product.}
In proving that the $\gL(\bw)$ can be upper-bounded by $\norm{\bar{\bA}\prn{{\btheta}-\btheta^{\star}}}$ (Lemma~\ref{lem:translate_error_to_loss}), as well as in verifying the exponential stability condition (Lemma~\ref{lem:exponential_stable}), one of the most critical steps is to show
\begin{equation}\label{eq:biscuit_matrix_in_paper}
    \norm{ \EB_{s, r, s^\prime}\brk{\prn{\bC\tilde{\bG}(r)\bC^{-1}}\otimes\prn{\bSigma_{\bphi}^{-\frac{1}{2}}\bphi(s)\bphi(s^\prime)^{\intercal}\bSigma_{\bphi}^{-\frac{1}{2}}}} } \leq \sqrt{\gamma},\quad\forall r\in[0,1].
\end{equation}
By Lemma~\ref{lem:Spectra_of_ccgcc}, we have $\|\bC\tilde{\bG}(r)\bC^{{-}1}\|{\leq} \sqrt{\gamma}$ for any $r\in[0,1]$.
In addition, one can check that $ \|\EB_{s,s^\prime}[\bSigma_{\bphi}^{-1{/}2}\bphi(s)\bphi(s^\prime)^{\intercal}\bSigma_{\bphi}^{-1{/}2}]\|{\leq} 1$.
One may speculate that the property $\norm{\bB_1 {\otimes} \bB_2} = \norm{\bB_1}\norm{\bB_2}$ (Lemma~\ref{lem:spectral_norm_of_KP}) is enough to get the desired conclusion.
However, the two matrices in the Kronecker product are not independent, preventing us from using this simple property to derive the conclusion. 
On the other hand, since we merely have the upper bound $ \EB_{s,s^\prime}[\|\bSigma_{\bphi}^{-1{/}2}\bphi(s)\bphi(s^\prime)^{\intercal}\bSigma_{\bphi}^{-1{/}2}\|]{\leq} d$, simply moving the expectation in Eqn.~\eqref{eq:biscuit_matrix_in_paper} outside the norm will lead to a loose $d\sqrt{\gamma}$ bound .
To resolve this problem, we leverage the fact that the second matrix is rank-$1$ and prove the following result. The proof can be found in the derivation following Eqn.~\eqref{eq:biscuit_matrix_bound}.
\begin{lemma}
    For any random matrix $\bY$ and random vectors $\bx$, $\bz$, suppose $\norm{\bY}\leq C_Y$ almost surely, $\EB\brk{\bx\bx^{\intercal}}\preccurlyeq C_x\bI_{d_1}$ and $\EB\brk{\bz\bz^{\intercal}}\preccurlyeq C_z\bI_{d_2}$ for some constants $C_Y, C_x, C_z>0$, then it holds that
    \begin{equation*}
    \norm{ \EB\brk{\bY\otimes\prn{\bx\bz^{\intercal}} } } \leq C_Y\sqrt{C_xC_z}.
\end{equation*}
\end{lemma}
\begin{remark}\label{remark:cgc}
The matrix $\bC\tilde{\bG}(r)\bC^{-1}$ also appeared in \citep[Proposition~B.2][]{rowland2024nearminimaxoptimal} as the matrix representation of the categorical projected Bellman operator $\bPi_K\gT^\pi$ of an one-state MDP. 
They showed $\|\bC\tilde{\bG}(r)\bC^{{-}1}\|{\leq} \sqrt{\gamma}$ using the fact that $\bPi_K\gT^\pi$ is a $\sqrt{\gamma}$-contraction in $\prn{\sP, \ell_2}$.
Our Lemma~\ref{lem:Spectra_of_ccgcc} provides a new analysis by directly analyzing the matrix.
See Appendix~\ref{appendix:analysis_cate_bellman_matrix} for more details.
\end{remark}
\paragraph{Bounding Norm of $\bb_t$.}
In proving Lemma~\ref{lem:upper_bound_error_quantities}, the most involved step is to upper-bound $\norm{\bb_t}$ (Eqn.~\eqref{eq:bt}).
To this end, we need to upper-bound the following term which appears in Eqn.~\eqref{eq:bt_upper_bound}:
\begin{equation}\label{eq:bound_bt_in_paper}
    \begin{aligned}
    (K+1)^{-1}\|\sum_{j=0}^K\bC\bg_j(r)-\bC\bm{1}_K\|,\quad\forall r\in[0,1],
    \end{aligned}
\end{equation}
when $K\geq(1-\gamma)^{-1}$.
Same as the matrix $\bC\tilde{\bG}(r)\bC^{-1}$ (see Remark~\ref{remark:cgc}), Term~\eqref{eq:bound_bt_in_paper} is also related to the matrix representation of the categorical projected Bellman operator. 
Specifically, let $\nu=(K+1)^{-1}\sum_{k=0}^K \delta_{x_k}$ be the discrete uniform distribution, Term~\eqref{eq:bound_bt_in_paper} equals 
\begin{equation*}
   \norm{\bC\prn{\bp_{(b_{r,\gamma})_\#(\nu)}-\bp_{\nu}}} = \iota_K^{-1/2}\ell_2\prn{\bPi_K(b_{r,\gamma})_\#(\nu),\nu}\leq\iota_K^{-1/2}\ell_2\prn{(b_{r,\gamma})_\#(\nu),\nu}\leq 3\sqrt{K}(1-\gamma),
\end{equation*}
where we used the orthogonal decomposition (Proposition~\ref{prop:orthogonal_decomposition}) and an upper bound for $\ell_2\prn{(b_{r,\gamma})_\#(\nu) ,\nu}$ (Lemma~\ref{lem:norm_b_bound}).
The full proof can be found in the derivation following Eqn.~\eqref{eq:bt_upper_bound}.