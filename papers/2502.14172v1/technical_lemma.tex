\begin{lemma}\label{lem:prob_basic_inequalities}
For any $\nu_1, \nu_2\in\sP^{\sgn}$, we have $W_1(\nu_1,\nu_2)\leq\frac{1}{\sqrt{1-\gamma}}\ell_2(\nu_1,\nu_2)$. 
\end{lemma}
\begin{proof}
By Cauchy-Schwarz inequality,
\begin{equation*}
    \begin{aligned}
        W_1(\nu_1,\nu_2)=&\int_0^{\frac{1}{1-\gamma}} |F_{\nu_1}(x)-F_{\nu_2}(x)| dx\\
        \leq&\sqrt{\int_0^{\frac{1}{1-\gamma}} 1^2 dx}\sqrt{\int_0^{\frac{1}{1-\gamma}} |F_{\nu_1}(x)-F_{\nu_2}(x)|^2 dx}\\
        =&\frac{1}{\sqrt{1-\gamma}}\ell_2(\nu_1,\nu_2).
    \end{aligned}
\end{equation*}
\end{proof}

\begin{lemma}\label{lem:spectra_of_CTC}
Let $\bC\in\RB^{K\times K}$ be the matrix defined in Eqn.~\eqref{eq:def_C}, it holds that the eigenvalues of $\bC^{T}\bC$ are $1/(4\cos^2(k\pi/(2K+1))$ for $k\in [K]$ , and thus
\begin{equation}
    \norm{\bC^{\intercal}\bC} = \frac{1}{4\sin^2\frac{\pi}{4K+2}}\leq K^2, \quad \norm{\prn{\bC^{\intercal}\bC}^{-1}} = 4\sin^2\frac{K\pi}{4K+2}\leq 4.
\end{equation}
\end{lemma}
\begin{proof}
One can check that
\begin{equation*}
 \bC^{\intercal} \bC =   \begin{bmatrix} K & K - 1 & \cdots & 2 & 1 \\ K - 1 & K - 1 & \cdots & 2 & 1 \\ \vdots & \vdots & \ddots & \vdots & \vdots \\ 2 & 2 & \cdots & 2 & 1 \\ 1 & 1 & \cdots & 1 & 1 \end{bmatrix},
\end{equation*}
\begin{equation*}
 \prn{\bC^{\intercal} \bC}^{-1} =   \begin{bmatrix} 1 & -1 & 0 & \cdots & 0 & 0 \\ -1 & 2 & -1 & \cdots & 0 & 0 \\ 0 & -1 & 2 & \cdots & 0 & 0 \\ \vdots & \vdots & \vdots & \ddots & \vdots & \vdots \\ 0 & 0 & 0 & \cdots & 2 & -1 \\ 0 & 0 & 0 & \cdots & -1 & 1 \end{bmatrix}.
\end{equation*}
Then, one can work with the the inverse of $\bC^{\intercal}\bC$ and calculate its singular values by induction, which has similar forms to the analysis of Toeplitz's matrix. 
See \citep{godsil1985inverses} for more details.
\end{proof}

\begin{lemma}\label{lem:Spectra_of_ccgcc}
For any $r\in[0, 1]$, it holds that $\norm{\bC\tilde{\bG}(r)\bC^{-1}}\leq\sqrt{\gamma}$ and $\norm{\prn{\bC^{\intercal}\bC}^{1/2}\tilde{\bG}(r)\prn{\bC^{\intercal}\bC}^{-1/2}}\leq \sqrt\gamma$.
\end{lemma}
% \begin{remark}\label{remark:cgc}
% The matrix $\bC\tilde{\bG}(r)\bC^{-1}$ was also studied in \citep{rowland2024nearminimaxoptimal} as the matrix representation of the categorical projected Bellman operator of a specific one-state Markov reward process. 
% They derived the same upper bound using the contraction property of categorical projected 
% Our Lemma~\ref{lem:Spectra_of_ccgcc} provides a new analysis by directly analyzing the properties of the matrix.
% \end{remark}
% \begin{proof}[Proof of Lemma~\ref{lem:Spectra_of_ccgcc}]
\begin{proof}
One can check that
\begin{equation*}
\bC^{-1}=\begin{bmatrix} 1 & 0 & \cdots & 0 & 0\\ -1 & 1 & \cdots & 0 & 0\\ 0 & - 1 & \cdots & 0 & 0\\ \vdots & \vdots & \ddots & \vdots & \vdots\\ 0 & 0 & \cdots & -1 & 1 \end{bmatrix}.
\end{equation*}
It is clear that
\begin{equation}
\begin{aligned}
    \norm{\prn{\bC^{\intercal}\bC}^{1/2}\tilde{\bG}(r)\prn{\bC^{\intercal}\bC}^{-1/2}} \leq \sqrt{\gamma} &\iff \bY(r)\bY^{\intercal}(r)\preccurlyeq \gamma \bI\\
    &\iff \tilde{\bG}(r)(\bC^{\intercal}\bC)^{-1}\tilde{\bG}^{\intercal}(r)\preccurlyeq \gamma (\bC^{\intercal}\bC)^{-1}\\
    &\iff \bC\tilde{\bG}(r)(\bC^{\intercal}\bC)^{-1}\tilde{\bG}^{\intercal}(r)\bC^{\intercal}\preccurlyeq\gamma\bI\\
    &\iff \norm{\bC\tilde{\bG}(r)\bC^{-1}}\leq\sqrt{\gamma}.\\
\end{aligned}
\end{equation}
By Lemma~\ref{lem:spectra_CGC_} and an upper bound on the spectral norm (Riesz–Thorin interpolation theorem) \citep[Theorem~7.3][]{serre2002matrices}, we obtain that 
\begin{equation}
    \norm{\bC\tilde{\bG}(r)\bC^{-1}}\leq \sqrt{\norm{\bC\tilde{\bG}(r)\bC^{-1}}_{1}\norm{\bC\tilde{\bG}(r)\bC^{-1}}_{\infty}} \leq \sqrt{1 \cdot \gamma} = \sqrt{\gamma}.
\end{equation}
\end{proof}


\begin{lemma}\label{lem:norm_b_bound}
    Suppose $K\geq (1-\gamma)^{-1}$, $\nu=(K+1)^{-1}\sum_{k=0}^K\delta_{x_k}$ is the discrete uniform distribution, then for any $r\in[0, 1]$, it holds that
    \begin{equation*}
        \ell_2\prn{(b_{r,\gamma})_\#(\nu) ,\nu}\leq 3\sqrt{1-\gamma}.
    \end{equation*}
\end{lemma}
\begin{proof}
Let $\tilde{\nu}$ be the continuous uniform distribution on $\brk{0,(1-\gamma)^{-1}+\iota_K}$, we consider the following decomposition
\begin{equation*}
\begin{aligned}
        \ell_2\prn{\nu, (b_{r,\gamma})_\#(\nu)}\leq\ell_2\prn{\nu,\tilde{\nu}}+\ell_2\prn{\tilde\nu,(b_{r,\gamma})_\#(\tilde{\nu})}+\ell_2\prn{(b_{r,\gamma})_\#(\tilde{\nu}),(b_{r,\gamma})_\#(\nu)}.
\end{aligned}
\end{equation*}
By definition, we have
\begin{equation*}
\begin{aligned}
        \ell_2\prn{\nu,\tilde{\nu}}=&\sqrt{(K+1)\int_0^{\iota_K}\prn{(1-\gamma)\frac{K}{K+1}x}^2dx}\\
        =&\sqrt{\frac{1}{3K(K+1)(1-\gamma)}}\\
        \leq& \frac{1}{K\sqrt{1-\gamma}}.
\end{aligned}
\end{equation*}
By the contraction property, we have 
\begin{equation*}
\begin{aligned}
        \ell_2\prn{(b_{r,\gamma})_\#(\nu),(b_{r,\gamma})_\#(\tilde\nu)}\leq& \sqrt{\gamma}\ell_2\prn{\nu,\tilde{\nu}}\leq \frac{\sqrt{\gamma}}{K\sqrt{1-\gamma}}.
\end{aligned}
\end{equation*}
We only need to bound $\ell_2\prn{\tilde\nu,(b_{r,\gamma})_\#(\tilde{\nu})}$.
We can find that $(b_{r,\gamma})_\#(\tilde{\nu})$ is the continuous uniform distribution on $\brk{r,r+\gamma\iota_K+\gamma(1-\gamma)^{-1}}$, and the upper bound is less than the upper bound of $\nu$, namely, $r+\gamma\iota_K+\gamma(1-\gamma)^{-1}\leq (1-\gamma)^{-1}+\gamma\iota_K<(1-\gamma)^{-1}+\iota_K$.
Hence
\begin{equation*}
\begin{aligned}
        \ell_2^2\prn{\tilde\nu,(b_{r,\gamma})_\#(\tilde{\nu})}=&\int_{0}^r\prn{(1-\gamma)\frac{K}{K+1}x}^2dx+\int_{r}^{r+\gamma\iota_K+\gamma(1-\gamma)^{-1}}\brk{(1-\gamma)\frac{K}{K+1}\prn{x-\frac{x-r}{\gamma}}}^2dx\\
        &+\int_{r+\gamma\iota_K+\gamma(1-\gamma)^{-1}}^{(1-\gamma)^{-1}+\iota_K}\prn{1-(1-\gamma)\frac{K}{K+1}x}^2 dx\\
        =&\frac{(1-\gamma)^2K^2r^3}{3(K+1)^2}+\prn{\frac{(1-\gamma)\gamma K^2 r^3}{3(K+1)^2}+\frac{(1-\gamma)\gamma K^2 \prn{\frac{K+1}{K}-r}^3}{3(K+1)^2}}+\frac{(1-\gamma)^2K^2\prn{\frac{K+1}{K}-r}^3}{3(K+1)^2}\\
        \leq& (1-\gamma)^2+(1-\gamma)\gamma\\
        =& 1-\gamma.
\end{aligned}
\end{equation*}
To summarize, we have
\begin{equation*}
\begin{aligned}
        \ell_2\prn{\nu, (b_{r,\gamma})_\#(\nu)}\leq\frac{1}{K\sqrt{1-\gamma}}+\sqrt{1-\gamma}+\frac{\sqrt{\gamma}}{K\sqrt{1-\gamma}}\leq 3\sqrt{1-\gamma},
\end{aligned}
\end{equation*}
where we used the assumption $K\geq (1-\gamma)^{-1}$.
\end{proof}

\section{Analysis of the Categorical Projected Bellman Matrix}\label{appendix:analysis_cate_bellman_matrix}
Recall that $\tilde{\bG}(r)=\bG(r)-\bm{1}_K^{\intercal}\otimes\bg_K(r)$. We extend the definition in Theorem~\ref{thm:linear_cate_TD_equation} and let $g_{j,k}(r) = h\prn{(r+\gamma x_j-x_k)/\iota_K}_+ = h(r/\iota_{K}+\gamma j-k)$ for $j,k\in\{0,1,\cdots,K\}$ where $h(x) = \prn{1-\abs{x}}_+$.
\begin{lemma}
    For any $r\in[0,1]$ and any $k \in \{0,1,\cdots,K\}$, in $\bg_{k}(r)$ there is either only one nonzero entry or two adjacent nonzero entries.
\end{lemma}
\begin{proof}
It is clear that $h(x)>0 \iff -1<x<1$. 
Let $k_{j}(r) = \min\ \{k:g_{j,k}(r)>0\}$, then $k_{j}(r) = \min\{k:r/\iota_{K}+\gamma j-k<1\} = \min\{k:0\leq r/\iota_{K}+\gamma j-k<1\}$. 
The existence of $k_{j}(r)$ is due to
\begin{equation}
    r/\iota_{K}+\gamma j-K \leq 1/\iota_{K}+\gamma j-K\leq (1-\gamma)K+\gamma K-K = 0 < 1.
\end{equation}
Let $a_{j}(r) := r/\iota_{K}+\gamma j-k_{j}(r)\in[0,1)$. Then $g_{j,k_{j}(r)}(r)=h(a_{j}(r)) = 1-a_{j}(r)$ and $g_{j,k_{j}(r)+1}(r)=h(a_{j}(r)-1)=a_{j}(r)$ are the only entries that can be nonzeros.
\end{proof}
The following results are immediate corollaries.
\begin{corollary}\label{col:sum_column_g}
\begin{equation}
    \sum_{k=0}^{i}g_{j,k}(r)=
    \begin{cases}
    0, &\text{ for}\ 0\leq i<k_{j}(r),\\
    1-a_{j}(r),&\text{ for}\ i=k_{j}(r),\\
    1,&\text{ for}\ k_{j}(r)<i\leq K.\\
    \end{cases}
\end{equation}
\end{corollary}
\begin{corollary}
For any $\gamma<1$, it holds that
\begin{equation}
k_{j+1}(r) =
\begin{cases}
    k_{j}(r), &\text{ if}\  a_{j}(r)\leq 1-\gamma,\\
    k_{j}(r)+1, &\text{ if}\  a_{j}(r)>1-\gamma.\\
\end{cases}
\end{equation}
As a result,
\begin{equation}
a_{j+1}(r) =
\begin{cases}
    a_{j}(r)+\gamma, &\text{ if}\ a_{j}(r)\leq 1-\gamma,\\
    a_{j}(r)+\gamma-1, &\text{ if}\ a_{j}(r)>1-\gamma.\\
\end{cases}
\end{equation}



\end{corollary}
\begin{lemma}\label{lem:spectra_CGC_}
All entries in $\bC\tilde{\bG}(r)\bC^{-1}$ are non-negative. $\norm{\bC\tilde{\bG}(r)\bC^{-1}}_{\infty} = \gamma$ and $\norm{\bC\tilde{\bG}(r)\bC^{-1}}_{1} \leq 1$.
\begin{proof}
By definition the entries of $\tilde{\bG}(r)$ are 
\begin{equation}
    (\tilde{\bG}(r))_{j,i} = g_{j,i}(r)-g_{K,i}(r)\quad  \text{for } j,i \in  \{0,1,\cdots,K-1\}.
\end{equation}
Using the previous corollaries, direct calculation show that if $k_{j+1}(r) = k_{j}(r)$, 
\begin{equation}
(\bC\tilde{\bG}(r)\bC^{-1})_{j,i} = \sum_{k=0}^{i}g_{j,k}(r)-\sum_{k=0}^{i}g_{j+1,k}(r) = 
\begin{cases}
    0, &\text{ for}\ 0\leq i<k_{j}(r),\\
    a_{j+1}(r)-a_{j}(r),&\text{ for}\ i=k_{j}(r),\\
    0,&\text{ for}\ k_{j}(r)<i<K.\\
\end{cases}
\end{equation}
And if $k_{j+1}(r) = k_{j}(r)+1$,
\begin{equation}
(\bC\tilde{\bG}(r)\bC^{-1})_{j,i} = \sum_{k=0}^{i}g_{j,k}(r)-\sum_{k=0}^{i}g_{j+1,k}(r) = 
\begin{cases}
    0, &\text{ for}\ 0<i<k_{j}(r),\\
    1-a_{j}(r),&\text{ for}\ i=k_{j}(r),\\
    a_{j+1}(r), &\text{ for}\ i = k_{j+1}(r),\\
    0,&\text{ for}\ k_{j}(r)<i<K.\\
\end{cases}
\end{equation}
As a result, all entries in $\bC\tilde{\bG}(r)\bC^{-1}$ is non-negative. Moreover, the sum of each column and $\norm{\bC\tilde{\bG}(r)\bC^{-1}}_{\infty}$ is $\gamma$ because
\begin{equation}
    \sum_{i=0}^{K-1} (\bC\tilde{\bG}(r)\bC^{-1})_{j,i}  = 
    \begin{cases}
    a_{j+1}(r)-a_{j}(r) = \gamma, &\text{ if}\ k_{j+1}(r) = k_{j}(r),\\
    1-a_{j}(r)+a_{j+1}(r) = \gamma, &\text{ if}\ k_{j+1}(r) = k_{j}(r)+1.\\
    \end{cases}
\end{equation}
The row sum of $\bC\tilde{\bG}(r)\bC^{-1}$ is 
\begin{equation}
    \sum_{j=0}^{K-1} (\bC\tilde{\bG}(r)\bC^{-1})_{j,i}=\sum_{m=0}^{i}g_{0,m}(r)-\sum_{m=0}^{i}g_{K,m}(r)\leq 1-0 = 1.
\end{equation}
Thus, it holds that $\norm{\bC\tilde{\bG}(r)\bC^{-1}}_{1} \leq 1$.
\end{proof}
\end{lemma}
