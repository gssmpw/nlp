In this section, we will introduce some properties of Kronecker product used in our paper.
See \citep{zhang2013kronecker} for a detailed treatment of Kronecker product.

For any matrices $\bA\in\RB^{m\times n}$ and $\bB\in\RB^{p\times q}$, the Kronecker product $\bA\otimes\bB$ is an matrix in $\RB^{mp\times nq}$, defined as
\begin{equation*}
    \bA\otimes \bB = \begin{bmatrix} a_{11}\bB & a_{12}\bB & \cdots & a_{1n}\bB \\ a_{21}\bB & a_{22}\bB & \cdots & a_{2n}\bB \\ \vdots & \vdots & \ddots & \vdots \\ a_{m1}\bB & a_{m2}\bB & \cdots & a_{mn}\bB \end{bmatrix}.
\end{equation*}
\begin{lemma}\label{lem:basic_of_KP}
The Kronecker product is bilinear and associative.
Furthermore, for any matrices $\bB_1, \bB_2, \bB_3, \bB_4$ such that $\bB_1\bB_3$, $\bB_2\bB_4$ can be defined, it holds that
$\prn{\bB_1\otimes \bB_2}\prn{\bB_3\otimes \bB_4}=\prn{\bB_1\bB_3}\otimes\prn{\bB_2\bB_4}$ (mixed-product property).
\begin{proof}
See \citep[Basic properties and Theorem~3][]{zhang2013kronecker}.
\end{proof}
\end{lemma}

\begin{lemma}\label{lem:vec_and_KP}
For any matrices $\bB_1, \bB_2, \bB_3$ such that $\bB_1\bB_2\bB_3$ can be defined, it holds that $\vect\prn{\bB_1\bB_2\bB_3}=\prn{\bB_3^{\intercal}\otimes\bB_1}\vect\prn{\bB_2}$.
\begin{proof}
See \citep[Lemma~4.3.1][]{horn1994topics}.
\end{proof}
\end{lemma}

\begin{lemma}\label{lem:spectral_norm_of_KP}
% For any $d_1,d_2,d_3,d_4\in\NB$ and matrices $\bB_1\in\RB^{d_1\times d_2}$ and $\bB_2\in\RB^{d_3\times d_4}$, it holds that
For any matrices $\bB_1$ and $\bB_2$, it holds that
$\norm{\bB_1 \otimes \bB_2} = \norm{\bB_1}\norm{\bB_2}$, $\prn{\bB_1\otimes \bB_2}^{\intercal}=\bB_1^{\intercal}\otimes\bB_2^{\intercal}$.
Furthermore, if $\bB_1$ and $\bB_2$ are invertible/orthogonal/diagonal/symmetric/normal, $\bB_1\otimes \bB_2$ is also invertible/orthogonal/diagonal/symmetric/normal and $\prn{\bB_1\otimes \bB_2}^{-1}=\bB_1^{-1}\otimes\bB_2^{-1}$.
\begin{proof}
% See \citep[Theorem~8][]{lancaster1972norms}.
See \citep[Basic properties, Theorem~5 and Theorem~7][]{zhang2013kronecker}.
\end{proof}
\end{lemma}

\begin{lemma}\label{lem:spectra_of_PSD_KP}
For any $K, d\in\NB$ and PSD matrices $\bB_1, \bB_3\in\RB^{K\times K}, \bB_2, \bB_4\in\RB^{d\times d}$ with $\bB_1\preccurlyeq \bB_3$ and $\bB_2\preccurlyeq \bB_4$, it holds that $\bB_1 \otimes \bB_2$, $\bB_3 \otimes \bB_4$ are also PSD matrices, furthermore,
$\bB_1 \otimes \bB_2 \preccurlyeq \bB_3 \otimes \bB_4$.
\begin{proof}
Consider the spectral decomposition $\bB_i=\bQ_i\bD_i\bQ_i^{\intercal}$, for any $i\in [4]$, by Lemma~\ref{lem:basic_of_KP} and Lemma~\ref{lem:spectral_norm_of_KP}, we have
\begin{equation*}
    \prn{\bB_1\otimes \bB_2}=\prn{\bQ_1\otimes \bQ_2}\prn{\bD_1\otimes \bD_2}\prn{\bQ_1\otimes \bQ_2}^{\intercal}
\end{equation*}
and 
\begin{equation*}
    \prn{\bB_3\otimes \bB_4}=\prn{\bQ_3\otimes \bQ_4}\prn{\bD_3\otimes \bD_4}\prn{\bQ_3\otimes \bQ_4}^{\intercal}
\end{equation*}
are also spectral decomposition of $\prn{\bB_1\otimes \bB_2}$ and $\prn{\bB_3\otimes \bB_4}$ respectively.
It is easy to see that they are PSD.
Furthermore,
\begin{equation*}
\begin{aligned}
     \prn{\bB_3\otimes \bB_4}-\prn{\bB_1\otimes \bB_2}=&\brk{\prn{\bB_3\otimes \bB_4}-\prn{\bB_3\otimes \bB_2}}+\brk{\prn{\bB_3\otimes \bB_2}-\prn{\bB_1\otimes \bB_2}}  \\
     =&\brk{\bB_3\otimes\prn{ \bB_4-\bB_2}}+\brk{\prn{\bB_3-\bB_1}\otimes \bB_2} \\
     \succcurlyeq& \bm{0}.
\end{aligned}
\end{equation*}
\end{proof}
\end{lemma}


\begin{lemma}\label{lem:vector_outer_KP}
    For any $K,d,d_1,d_2\in\NB$, vectors $\bu, \bv\in\RB^d$ and matrices $\bB_1\in\RB^{K\times d_1}$, $\bB_2\in\RB^{d_2\times K}$, $\bB_3\in\RB^{K\times K}$, it holds that
    \begin{equation*}
    \prn{\bI_K\otimes\bu}\bB_1=\bB_1\otimes\bu,
\end{equation*}
\begin{equation*}
    \bB_2\prn{\bI_K\otimes\bv}^{\intercal}=\bB_2\otimes\bv^{\intercal},
\end{equation*}
\begin{equation*}
    \prn{\bI_K\otimes\bu}\bB_3\prn{\bI_K\otimes\bv}^{\intercal}=\bB_3\otimes\prn{\bu\bv^{\intercal}}.
\end{equation*}
Furthermore, for any matrix $\bB_4\in\RB^{d_1\times d_2}$, we have
    \begin{equation*}
    \prn{\bB_1\otimes\bu}\bB_4=\prn{\bB_1\bB_4}\otimes \bu.
\end{equation*}
\end{lemma}
\begin{proof}
Let $\bu=\prn{u_i}_{i=1}^d$ $\bB_1=\prn{b_{ij}}_{i,j=1}^K$, then
\begin{equation*}
    \begin{aligned}
        \prn{\bI_K\otimes\bu}\bB_1=&\begin{bmatrix}
\bu & \bm{0}_d & \cdots & \bm{0}_d & \bm{0}_d \\
\bm{0}_d & \bu & \cdots & \bm{0}_d & \bm{0}_d \\
\vdots & \vdots & \ddots & \vdots & \vdots \\
\bm{0}_d & \bm{0}_d & \cdots & \bu & \bm{0}_d \\
\bm{0}_d & \bm{0}_d & \cdots & \bm{0}_d & \bu
\end{bmatrix}\begin{bmatrix}
b_{11} & \cdots & b_{1K} \\
\vdots & \ddots & \vdots \\
b_{K1} & \cdots & b_{KK} 
\end{bmatrix}\\
=&\begin{bmatrix}
b_{11}u_1  & \cdots & b_{1K}u_1  \\
\vdots &\ddots & \vdots &\\
b_{11}u_d  & \cdots &  b_{1K}u_d\\
\vdots &\ddots & \vdots &\\
b_{K1}u_1  & \cdots &b_{KK} u_1  \\
\vdots &\ddots & \vdots &\\
b_{K1}u_d  & \cdots & b_{KK}u_d 
\end{bmatrix}\\
=&\begin{bmatrix}
b_{11}\bu & \cdots & b_{1K}\bu \\
\vdots & \ddots & \vdots \\
b_{K1}\bu & \cdots & b_{KK}\bu 
\end{bmatrix}\\
=& \bB_1\otimes u.
\end{aligned}
\end{equation*}
Hence
\begin{equation*}
    \begin{aligned}
\bB_2\prn{\bI_K\otimes\bv}^{\intercal}=&\brk{\prn{\bI_K\otimes\bv}\otimes \bB^{\intercal}_2}^{\intercal}=\brk{\bB^{\intercal}_2\otimes \bv}^{\intercal}=\bB_2\otimes\bv^{\intercal}.
\end{aligned}
\end{equation*}
And in the same way,
\begin{equation*}
    \begin{aligned}
\prn{\bI_K\otimes\bu}\bB_3\prn{\bI_K\otimes\bv}^{\intercal}=&\prn{\bB_3\otimes \bu}\otimes \bv^{\intercal}=\bB_3\otimes \prn{ \bu\otimes \bv^{\intercal}}=\bB_3\otimes\prn{ \bu \bv^{\intercal}}.
\end{aligned}
\end{equation*}
Furthermore,
\begin{equation*}
    \prn{\bB_1\otimes\bu}\bB_4=\brk{\prn{\bI_K\otimes\bu}\bB_1}\bB_4=\prn{\bI_K\otimes\bu}\prn{\bB_1\bB_4}=\prn{\bB_1\bB_4}\otimes \bu.
\end{equation*}
\end{proof}