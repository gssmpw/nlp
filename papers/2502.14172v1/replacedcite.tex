\section{Related Work}
\paragraph{Distributional Reinforcement Learning.}
Distributional TD learning was first proposed in ____.
Following the distributional perspective in ____, ____ proposed a distributional version of the gradient TD learning algorithm,
____ proposed a distributional version of multi-step TD learning, ____ proposed a distributional version of off-policy Q($\lambda$) and TD($\lambda$) algorithms, and ____ proposed a distributional version of fitted Q evaluation to solve the distributional offline policy evaluation problem.
____ proposed an approach for evaluating the return distributions for all policies simultaneously when the reward is deterministic or in the finite-horizon setting. 
____ studied distributional policy evaluation in the multivariate reward setting and proposed corresponding TD learning algorithms.
Beyond the tabular setting, ____ proposed various distributional TD learning algorithms with linear function approximation under different parametrizations.

A series of recent studies have focused on the theoretical properties of distributional TD learning.
____ analyzed the asymptotic and non-asymptotic convergence of distributional TD learning (or its model-based variants) in the tabular setting.
Among these works, ____ established that in the tabular setting, learning the full return distribution is statistically as easy as learning its expectation, in the model-based and model-free setting respectively.
And ____ provided an asymptotic convergence result for categorical TD learning with linear function approximation.

Beyond the problem of distributional policy evaluation,
____ showed that, theoretically, classic value-based reinforcement learning could benefit from distributional reinforcement learning.
____ considered optimizing statistical functionals of the return, and proposed algorithms to solve this harder problem.

\paragraph{Stochastic Approximation.}
Our {\LCTD} falls into the category of LSA. 
The classic TD learning, as one of the most classic LSA problems, has been extensively studied ____. 
Among these works, ____ provided the tightest bounds for {\LTD} with constant step sizes, which is also considered in our paper.
While ____ established the tightest bounds for {\LTD} with polynomial-decaying step sizes.

For general stochastic approximation problems, extensive works including ____ have provided solid theoretical understandings.
% \zly{list more result? at least list the sample complexity bounds of linear TD matching our \LCTD}