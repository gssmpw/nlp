\section{Fundamental Physics-based Hardware and Signal Models} \label{sec:model}

In this section, we introduce the fundamental physics-based hardware and signal models for pinching antennas. Specifically, we model a pinching antenna as an open-ended directional coupler, which facilitates the adjustment of radiation characteristics and simplifies signal modeling. We then employ coupled-mode theory to characterize the relationship between the signal within the waveguide and the signal radiated by the pinching antenna.

\begin{figure}[t!]
    \centering
    \includegraphics[width=0.45\textwidth]{./waveguide_model.pdf}
    \caption{Schematic illustration of pinching antennas operating as an open-ended directional waveguide coupler.}
    \label{physical_model}
  \end{figure} 

\subsection{Physics-based Hardware Model}

The core principle of pinching antennas relies on the phenomenon where a portion of the EM waves propagating through a dielectric waveguide is induced into an adjacent dielectric material (i.e., a pinching antenna) when the two are placed in close proximity. To accurately model the signal radiated by the pinching antenna, it is essential to characterize the EM fields within both the waveguide and the pinching antenna, as well as to understand the interaction between these fields. 

Consider a dielectric waveguide with an effective refractive index $n_{\mathrm{g}}$. A signal with a free-space wavelength $\lambda$ is introduced into the waveguide and propagates along the $x$-axis, as shown in Fig. \ref{physical_model}. The electric field distribution of the EM wave within the waveguide can be expressed as:
\begin{equation} \label{EM_model_waveguide}
    \mathbf{E}_{\mathrm{guide}}(x,y,z) = \mathbf{D}_{\mathrm{guide}}(y,z) e^{-j \beta_{\mathrm{g}} x} s_{\mathrm{p}},
\end{equation}
where $\mathbf{D}_{\mathrm{guide}}(y,z) \in \mathbb{C}^{3 \times 1}$ represents the transverse field distribution of the guided mode, $\beta_{\mathrm{g}} = \frac{2 \pi n_{\mathrm{g}}}{\lambda}$ is the propagation constant of the waveguide, and $s_{\mathrm{p}} \in \mathbb{C}$ denotes the phase-shifted communication signal modulated onto the EM wave. Let $x_{0, \mathrm{p}}$ denote the distance that the signal has propagated within the waveguide prior to the starting point of coupling, as shown in Fig. \ref{physical_model}. The signal $s_{\mathrm{p}}$ can be expressed as:
\begin{equation}
    s_{\mathrm{p}} = e^{-j \beta_{\mathrm{g}} x_{0, \mathrm{p}}} c_0,
\end{equation}
where $c_0$ represents the original communication signal. Note that the in-waveguide propagation loss is omitted in the above formulas due to its negligible impact on the system performance \cite{ding2024flexible}. 

We model the pinching antenna as an open-ended directional waveguide coupler, where EM waves can be radiated from one end of the pinching antenna with minimal reflection by optimizing waveguide's aperture size, shape, and termination impedance \cite{gardiol1985open}. For simplicity, we assume ideal full radiation from the open end of the pinching antenna with no reflection. When the pinching antenna is placed in proximity to (or “pinched” against) the main waveguide, coupling occurs, generating an EM field within the pinching antenna. Let $n_{\mathrm{p}}$ denote the effective refractive index of the pinching antenna. The electric field within the pinching antenna is
\begin{equation}
    \mathbf{E}_{\mathrm{pinch}}(x,y,z) = \mathbf{D}_{\mathrm{pinch}}(y,z) e^{-j \beta_{\mathrm{p}} x} s_{\mathrm{p}},
\end{equation}
where $\mathbf{D}_{\mathrm{pinch}}(y,z) \in \mathbb{C}^{3 \times 1}$ represents the transverse field distribution, and $\beta_{\mathrm{p}} = \frac{2 \pi n_{\mathrm{p}}}{\lambda}$ is the propagation constant of the pinching antenna.

Based on coupled-mode theory, the total EM field within the waveguide and the pinching antenna can be expressed as a weighted sum of their respective individual fields. Let $\mathbf{E}$ denote the overall electric field, which can be written as:
\begin{equation}
    \mathbf{E} = A(x) \mathbf{E}_{\mathrm{guide}} + B(x) \mathbf{E}_{\mathrm{pinch}}.
\end{equation}
By substituting the expressions for the electric field $\mathbf{E}$ and the corresponding magnetic field into Maxwell's equations, the following coupled differential equations for $A(x)$ and $B(x)$ are obtained \cite{okamoto2010fundamentals}:
\begin{align} \label{diff_condition_A}
    \frac{d A(x)}{d x} &= -j \kappa B(x) e^{-j \Delta \beta x}, \\
    \label{diff_condition_B}
    \frac{d B(x)}{d x} &= -j \kappa A(x) e^{j \Delta \beta x},
\end{align}
where $\kappa \in \mathbb{R}$ is the mode coupling coefficient, determined by the transverse field distributions $\mathbf{D}_{\mathrm{guide}}(y,z)$ and $\mathbf{D}_{\mathrm{pinch}}(y,z)$, and $\Delta \beta = \beta_{\mathrm{p}} - \beta_{\mathrm{g}}$ represents the difference between the propagation constants of the waveguide and the pinching antenna. Since the signal is initially introduced into the waveguide, the following initial conditions hold:
\begin{align}
    A(0) = 1, \quad B(0) = 0.
\end{align}
Solving the differential equations \eqref{diff_condition_A} and \eqref{diff_condition_B} under these initial conditions yields the following expressions for $A(x)$ and $B(x)$:
\begin{align}
    A(x) &= \left( \cos \left(\phi x \right) + \frac{j \Delta \beta}{\phi} \sin(\phi x) \right) e^{-j \Delta \beta x/2}, \\
    B(x) &= - \frac{j \kappa}{\phi} \sin(\phi x) e^{j \Delta \beta x/2},
\end{align}
where $\phi = \sqrt{\kappa^2 + \Delta \beta^2}$. Consequently, the total power of the EM field within the waveguide and the pinching antenna is given by:
\begin{align}
    P_{\mathrm{guide}}(x) &= \left| A(x) \right|^2 = 1 - \left( \frac{\kappa}{\phi} \right)^2 \sin^2(\phi x), \\
    P_{\mathrm{pinch}}(x) &= \left| B(x) \right|^2 = \left( \frac{\kappa}{\phi} \right)^2 \sin^2(\phi x).
\end{align}
From these expressions, it is evident that a maximum fraction of $\left( \kappa/\phi \right)^2$ of the total power can be transferred to the pinching antenna. In the special case where the waveguide and the pinching antenna have the same effective refractive index (i.e., $\beta_{\mathrm{g}} = \beta_{\mathrm{p}}$), we have $\Delta \beta = 0$ and $\phi = \kappa$. Under this condition, the expressions for $A(x)$ and $B(x)$ can be simplified to:
\begin{align} \label{simplified_power_exchange}
    A(x) = \cos(\kappa x), \quad B(x) = -j \sin(\kappa x).
\end{align}
Utilizing the simplified power exchange coefficients in \eqref{simplified_power_exchange} for the case $\beta_{\mathrm{g}} = \beta_{\mathrm{p}}$, the signal radiated from the open end of the pinching antenna can be expressed as
\begin{align} \label{EM_model_rad}
    \mathbf{E}_{\mathrm{rad}}(y,z) &=  B(L) \mathbf{E}_{\mathrm{pinch}}(L,y,z) \nonumber \\
    &= -j \mathbf{D}_{\mathrm{pinch}}(y,z) \sin(\kappa L) e^{-j \beta_{\mathrm{g}} x_{\mathrm{p}}} c_0,
\end{align}
where $x_{\mathrm{p}}$ represents the effective position of the pinching antenna and is defined as:
\begin{equation}
    x_{\mathrm{p}} = x_{0, \mathrm{p}} + L.
\end{equation}
Once the field in the pinching antenna radiates into free space, the power exchange between the waveguide and the pinching antenna ceases at $x = L$. The remaining EM wave propagating within the waveguide is given by:
\begin{align} \label{EM_model_guide_remain}
    \tilde{\mathbf{E}}_{\mathrm{guide}}(x,y,z) &= A(L) \mathbf{E}_{\mathrm{guide}}(L,y,z) \nonumber \\
    &= \mathbf{D}_{\mathrm{guide}}(y,z) \cos(\kappa L) e^{-j \beta_{\mathrm{g}} (x + x_{\mathrm{p}})} c_0.
\end{align}

\begin{remark}
    \normalfont
    \emph{(Power Relationship)} Based on the above physics model, the power of the signals within the waveguide and radiated by the pinching antenna is determined by the coupling length $L$. In the case of $\beta_{\mathrm{g}} = \beta_{\mathrm{p}}$, a complete power transfer to the pinching antenna becomes achievable, enabling full signal radiation, which is achieved when $L = \pi/(2\kappa)$ according to \eqref{simplified_power_exchange}. However, this strategy is applicable for the case with a single pinching antennas, but not for the case with multiple pinching antennas on the same waveguide. For the latter case, the radiation power from each pinching antenna can be adjusted by modifying the coupling length $L$. In other words, the transmit powers of the multiple pinching antennas can be controlled, which leads to extra degrees of freedom for the system design; howerver, this would likely lead to active pinching antenna design and additional system complexity. Therefore, this paper focuses on a purely passive pinching antenna design with a preconfigured coupling length.
\end{remark}

% The fundamental concept of PASS lies in the phenomenon where, when a separate dielectric material (i.e., the pinching antenna) is placed in close proximity to a dielectric waveguide, a portion of the radio waves propagating through the waveguide is induced into the adjacent dielectric. To accurately model the signal radiated by the pinching antenna, it is essential to characterize the electromagnetic fields within both the pinching antenna and the waveguide, as well as to understand the relationship between these fields. In the following, we model the pinching antenna as an open-ended directional coupler and utilize coupled-mode theory to achieve this characterization.

% Let us consider a dielectric waveguide with an effective index $n_{\mathrm{g}}$. A signal with a free-space wavelength $\lambda$ is introduced into the waveguide and propagates along the $z$-axis. The electric field distribution of the EM wave within the waveguide can be expressed as
% \begin{equation} \label{EM_model_waveguid}
%         \mathbf{E}_{\mathrm{guide}}(x,y,z) = \mathbf{D}_{\mathrm{guide}}(y,z) e^{-j \beta_{\mathrm{g}} z} s_{\mathrm{p}}, 
% \end{equation}
% where $\mathbf{D}_{\mathrm{guide}}(y,z) \in \mathbb{C}^{3 \times 1}$ is the transverse field distributions of the guided mode of signals, $\alpha \in \mathbb{R}$ is the attenuation coefficient of the waveguide, $\beta_{\mathrm{g}} = \frac{2 \pi n_{\mathrm{g}}}{\lambda}$ is the propagation constant of the waveguide, and $s_{\mathrm{p}} \in \mathbb{C}$ denotes phase-shifted communication data modulated onto the EM wave. Let $x_{0, \mathrm{p}}$ denote the propagation distance of signal from the induction position to the starting position of coupling. The signal $s_{\mathrm{p}}$ can be expressed as 
% \begin{equation}
%     s_{\mathrm{p}} = e^{-j \beta_{\mathrm{g}} x_{0, \mathrm{p}}} c_0, 
% \end{equation}   
% where $c_0$ is the original communication data. 
% % Since attenuation within the waveguide is negligible, the equation \eqref{EM_model_waveguid} simplifies to
% % \begin{align} \label{EM_model_waveguid_simple}
% %     \mathbf{E}_{\mathrm{guide}}(x,y,z) &= \mathbf{D}_{\mathrm{guide}}(y,z) e^{-j \beta_{\mathrm{g}} (z + z_p)} c_0.
% % \end{align} 
% For simplicity, the pinching antenna can be approximated as an open-ended directional coupler with an effective index $n_{\mathrm{p}}$. When the pinching antenna are in proximity of (or pinched to) the main waveguide, coupling occurs, generating an electromagnetic field within the pinching antenna. The electric field within it can be expressed as
% \begin{equation}
%     \mathbf{E}_{\mathrm{pinch}}(x,y,z) = \mathbf{D}_{\mathrm{pinch}}(y,z) e^{-j \beta_{\mathrm{p}} z} s_{\mathrm{p}},
% \end{equation}
% where $\mathbf{D}_{\mathrm{pinch}}(y,z) \in \mathbb{C}^{3 \times 1}$ is the transverse field distributions, and $\beta_{\mathrm{p}} = \frac{2 \pi n_{\mathrm{p}}}{\lambda}$ is the propagation constant of the pinching antenna. 
% % The electromagnetic fields in the waveguide and the pinching antenna must match tangentially at their interface $z = 0$. This condition results in the relationship:
% % \begin{equation}
% %     \bar{s}_0 = e^{-j \beta_{\mathrm{g}} z_p} c_0.
% % \end{equation}

% Based on coupled-mode theory, the total electromagnetic field within the waveguide and the pinching antenna can be expressed as a weighted sum of their respective individual fields. Let $\mathbf{E}$ denote the overall electric fields, which can be represented as 
% \begin{equation}
%     \mathbf{E} = A(x) \mathbf{E}_{\mathrm{guide}} + B(x) \mathbf{E}_{\mathrm{pinch}}.
% \end{equation}
% By substituting the expressions for the electric field $\mathbf{E}$ and the corresponding magnetic field into Maxwell's equations, the following coupled differential equations relating $A(x)$ and $B(x)$ can be obtained \cite{okamoto2010fundamentals}:
% \begin{align} \label{diff_condition_A}
%     \frac{d A(x)}{d x} &= -j \kappa B(x) e^{-j \Delta \beta z},\\
%     \label{diff_condition_B}
%     \frac{d B(x)}{d x} &= -j \kappa A(x) e^{j \Delta \beta z},
% \end{align}
% where $\kappa \in \mathbb{R}$ is mode coupling coefficient, determined by the transverse field distributions $\mathbf{D}_{\mathrm{guide}}(y,z)$ and $\mathbf{D}_{\mathrm{pinch}}(y,z)$, and $\Delta \beta = \beta_{\mathrm{p}} - \beta_{\mathrm{g}}$ denotes the difference between the propagation constant of the waveguide and the pinching antenna. Since the signal is initially introduced into the waveguide, the following initial conditions hold:
% \begin{align}
%     A(0) = 1, \quad B(0) = 0,
% \end{align}  
% Solving the differential equations \eqref{diff_condition_A} and \eqref{diff_condition_B} under these initial conditions yields the expressions for $A(x)$ and $B(x)$ as
% \begin{align}
%     A(x) &=\left( \cos \left(\phi z \right) + \frac{j \Delta \beta}{\phi} \sin(\phi z) \right) e^{-j \Delta \beta z/2}, \\
%     B(x) &= - \frac{j \kappa}{\phi} \sin(\phi z) e^{j \Delta \beta z/2},
% \end{align}
% where $\phi = \sqrt{\kappa^2 + \Delta \beta^2}$. Therefore, the ratio of total power of the EM field within the waveguide and the pinching antenna are given by, respectively,
% \begin{align}
%     P_{\mathrm{guide}}(x) &= \left| A(x) \right|^2 = 1 - \left( \frac{\kappa}{\phi} \right)^2 \sin^2(\phi z), \\
%     P_{\mathrm{pinch}}(x) &= \left| B(x) \right|^2 =  \left( \frac{\kappa}{\phi} \right)^2 \sin^2(\phi z).
% \end{align}
% It can be observed that a maximum fraction of  $( \frac{\kappa}{\phi} )^2$ of the total power can be transferred to the pinching antenna. In the special case that the waveguide and the pinching antenna have the same effective index, i.e., $\beta_{\mathrm{g}} = \beta_{\mathrm{p}}$, we have $\Delta \beta = 0$ and $\phi = \kappa$. Under this condition, the expressions for $A(x)$ and $B(x)$ simplifies to
% \begin{align} \label{simplified_power_exchange}
%     A(x) = \cos(\kappa z), \quad B(x) = -j \sin(\kappa z).
% \end{align} 
% In this scenario, it becomes feasible to transfer all the signal power through the pinching antenna, achieving complete radiation.

% Utilizing the simplified power exchange coefficients in \eqref{simplified_power_exchange} achieved when $\beta_{\mathrm{g}} = \beta_{\mathrm{p}}$, the signal radiated from the open end of the pinching antenna can be expressed in terms of the following electric field:
% \begin{align} \label{EM_model_rad}
%     \mathbf{E}_{\mathrm{rad}}(y,z) = &B(L) \mathbf{E}_{\mathrm{pinch}}(x,y,L) \nonumber \\
%     = & -j \sin(\kappa L) \mathbf{D}_{\mathrm{pinch}}(y,z)  e^{-j \beta_{\mathrm{g}} x_{\mathrm{p}} } c_0,
% \end{align}
% where $x_{\mathrm{p}}$ is defined as the effective position of the pinching antenna and is given by 
% \begin{equation}
%     x_{\mathrm{p}} = x_{0, \mathrm{p}} + L.
% \end{equation} 
% % where $\tilde{\mathbf{E}}_p(y,z) = -j e^{-j \beta_{\mathrm{p}} L} \mathbf{D}_{\mathrm{pinch}}(y,z)$ represents the modified transverse field distribution of the pinching antenna at the radiating end. Note that the additional phase shift $-j e^{-j \beta_{p,L}}$ in the radiated field can be pre-compensated by incorporating a low-cost phase shifter with a fixed phase adjustment at the radiator of each pinching antenna. Therefore, for simplicity, we omit the influence of this additional phase shift in the subsequent analysis. 
% Once the field in the pinching antenna radiates into free space, the power exchange between the waveguide and the pinching antenna ceases at the position $z = L$. Consequently, the remaining electromagnetic wave propagating within the waveguide is given by:
% \begin{align} \label{EM_model_guide_remain}
%     \tilde{\mathbf{E}}_{\mathrm{guide}}(x,y,z) = &\cos(kL) \mathbf{E}_{\mathrm{guide}}(x,y,z) \nonumber \\ 
%     = & \cos(kL) \mathbf{D}_{\mathrm{guide}}(y,z) e^{-j \beta_{\mathrm{g}} (z + x_{\mathrm{p}})} c_0.
% \end{align}   




\subsection{Signal Model}
In \eqref{EM_model_rad} and \eqref{EM_model_guide_remain}, we derived the physical models describing the EM wave behavior in the PASS system. Building upon these foundational physics models, we can now formulate simplified signal models commonly employed in the wireless communication system design. In the following, we begin by considering the scenario with a single pinching antenna coupled to the waveguide and subsequently extend the model to cases involving multiple pinching antennas.

\subsubsection{A Single Pinching Antenna}
Assume that a waveguide is deployed along the $x$-axis, with its $y$- and $z$-coordinates denoted by $y_{\mathrm{g}}$ and $z_{\mathrm{g}}$, respectively. A user is located on the ground at the position $\mathbf{r} = [x_{\mathrm{u}}, y_{\mathrm{u}}, 0]^T$. To serve this user, a pinching antenna is attached to the waveguide at the position $\mathbf{p} = [x_{\mathrm{p}}, y_{\mathrm{g}}, z_{\mathrm{g}}]^T$. Let $c_0$ represent the signal fed into the waveguide. According to \eqref{EM_model_rad}, the signal radiated from the pinching antenna is given by
\begin{equation}
s_{\mathrm{rad}} = \sin(\kappa L) e^{-j \beta_{\mathrm{g}} x_{\mathrm{p}}} c_0,
\end{equation}
The radiated signal propagates through free space to reach the user, where attenuation due to free-space path loss must be considered, leading to the following signal model:
\begin{align}
y = & \frac{\eta e^{-j \beta_0 r}}{r} s_{\mathrm{rad}} + n \nonumber \\
= & \underbrace{\frac{\eta e^{-j \beta_0 r}}{r}}_{\scriptstyle \text{high-loss free-space} \atop \scriptstyle \text{propagation}} \!\!\! \times \underbrace{ \vphantom{\frac{\eta e^{-j \beta_0 r}}{r}} \sin(\kappa L) e^{-j \beta_{\mathrm{g}} x_{\mathrm{p}}}}_{\scriptstyle \text{nearly lossless in-waveguide} \atop \scriptstyle  \text{propagation}}  c_0 + n,
\end{align}
where $\eta \in \mathbb{R}$ is the channel gain accounting for the free-space path loss factor and the radiation pattern of the pinching antennas, $\beta_0 = \frac{2 \pi}{\lambda}$ is the propagation constant in free space, $r$ is the distance between the pinching antenna and the user, given by
\begin{equation}
r = \| \mathbf{r} - \mathbf{p} \| = \sqrt{(x_{\mathrm{p}} - x_{\mathrm{u}})^2 + \omega},
\end{equation}
with $\omega = (y_{\mathrm{g}} - y_{\mathrm{u}})^2 + z_{\mathrm{g}}^2$, and $n \sim \mathcal{CN}(0 ,\sigma^2)$ denotes the additive white Gaussian noise. 


\subsubsection{Multiple Pinching Antennas}

Now we consider a scenario where $M$ pinching antennas are pinched sequentially on the waveguide. Let $\mathbf{p}_m = [x_{\mathrm{p},m}, y_{\mathrm{g}}, z_{\mathrm{g}}]^T$ and $L_m$ denote the position and the length of the $m$-th pinching antenna, respectively. According to \eqref{EM_model_rad} and \eqref{EM_model_guide_remain}, the power radiated by the $m$-th pinching antenna is influenced by the power exchange coefficients of all preceding pinching antennas. Consequently, the signal radiated from the $m$-th pinching antenna can be expressed as: 
\begin{align}
    s_{\mathrm{rad}, m} =&\sin(\kappa L_m) \prod_{i=1}^{m-1} \cos(\kappa L_i) e^{-j \beta_{\mathrm{g}} x_{\mathrm{p},m}} c_0 \nonumber \\
    = & \delta_m \prod_{i=1}^{m-1} \sqrt{1 - \delta_m^2} e^{-j \beta_{\mathrm{g}} x_{\mathrm{p},m}} c_0,
\end{align} 
where we define $\delta_m \triangleq \sin(\kappa L_m)$. 
Here, we consider two simplified but useful signal radiation model for multiple pinching antennas:
\begin{itemize}
    \item \textbf{Equal Power Model:} In this model, we assume that the length $L_m$ of each pinching antenna is adjusted so that each antenna radiates an equal proportion of the total power. Specifically, this is expressed as
    \begin{align}
        &\delta_m \prod_{i=1}^{m-1} \sqrt{1 - \delta_m^2} = \sqrt{\delta_{\mathrm{eq}}}, \quad \forall m, \nonumber \\
        \Leftrightarrow \quad & \delta_m = \sqrt{\frac{\delta_{\mathrm{eq}}}{(1 - \delta_{\mathrm{eq}})^{m-1}}}, \quad \forall m,
    \end{align}
    where $0 < \delta_{\mathrm{eq}} \le \frac{1}{M}$ is the equal-power ratio. Under this condition, the radiated signal from the $m$-th pinching antenna is simplified into
    \begin{equation}
        s_{\mathrm{rad},m} = \sqrt{\delta_{\mathrm{eq}}}  e^{-j \beta_{\mathrm{g}} x_{\mathrm{p},m}} c_0.
    \end{equation} 
    This model ensures that each pinching antenna radiates with the same efficiency, making it useful for obtaining insight to the performance of PASS as discussed in \cite{ding2024flexible}. However, achieving this equal-power distribution requires that each pinching antenna be manufactured with a different length, which increases the hardware cost.
    \item \textbf{Proportional Power Model:} In this model, we assume that the length of each pinching antenna is manufactured to be the same, i.e., $L_m = L, \forall m$. Consequently, each pinching antenna radiates the same ratio of the remaining power within the waveguide. Under this condition, the radiated signal from the $m$-th pinching antenna becomes
    \begin{equation}
        s_{\mathrm{rad},m} = \delta \left(\sqrt{1 - \delta^2}\right)^{m-1} e^{-j \beta_{\mathrm{g}} x_{\mathrm{p},m}} c_0,
    \end{equation}
    where $\delta = \sin(\kappa L)$. Compared to the equal-power model, this equal-ratio model significantly reduces hardware costs, as all pinching antennas can be uniformly manufactured with the same length.
\end{itemize}

Based on the above modeling, the signal received from all pinching antennas at the user can be expressed as 
\begin{align} \label{multiple_basis_model}
    y = &\sum_{m=1}^M \frac{\eta e^{-j \beta_0 r_m}}{r_m} s_{\mathrm{rad}, m} + n = \mathbf{h}^H(\mathbf{x}) \mathbf{g}(\mathbf{x}) c_0 + n.
\end{align}
Here, $r_m = \|\mathbf{r} - \mathbf{p}_m\|$ is the distance between the $m$-th pinching antenna and the user. $\mathbf{h}(\mathbf{x}) \in \mathbb{C}^{M \times 1}$ represents the free-space channel vector between all pinching antennas and the user, given by 
\begin{equation} \label{basic_channel_model}
    \mathbf{h}(\mathbf{x}) = \left[\frac{\eta e^{-j \beta_0 r_1}}{r_1},\dots,\frac{\eta e^{-j \beta_0 r_M}}{r_M}  \right]^H,
\end{equation}
which is a function of the pinching antenna positions $\mathbf{x} = [x_{\mathrm{p},1},\dots,x_{\mathrm{p},M}]^T$. $\mathbf{g}(\mathbf{x}) \in \mathbb{C}^{M \times 1}$ denotes the in-waveguide channel vector, given by 
\begin{equation} \label{basic_pinching_beamforming_model}
    \mathbf{g}(\mathbf{x}) = \left[ \alpha_1 e^{-j \beta_{\mathrm{g}} x_{\mathrm{p},1}},\dots,\alpha_M e^{-j \beta_{\mathrm{g}} x_{\mathrm{p},M}} \right]^T,
\end{equation} 
where $\alpha_m = \sqrt{\delta_{\mathrm{eq}}}$ for the equal power model, while $\alpha_m = \delta(\sqrt{1 - \delta^2})^{m-1}$ for the proportional power model.  

\begin{remark}
    \normalfont
    \emph{(Pinching Beamforming)} It can be observed from
    \eqref{multiple_basis_model}-\eqref{basic_pinching_beamforming_model} that adjusting the positions of  the pinching antennas, determined by $\mathbf{x}$, alters the phase and the large-scale path loss (amplitude) of the signal received by the user. Analogous to conventional multi-antenna beamforming, the signals from multiple pinching antennas can be combined either constructively or destructively at the user by carefully optimizing the positions of the pinching antennas. This new capability for signal reconfiguration introduced by PASS is referred to as \emph{pinching beamforming}.
    
    % The signal model \eqref{multiple_basis_model} is analogous to that of a conventional MIMO system, where $\mathbf{h}(\mathbf{x})$ and $\mathbf{g}(\mathbf{x})$ are channel and beamforming vectors, respectively. Adjusting the position of the pinching antenna, determined by $x_{\mathrm{p},m}$, alters not only the channel conditions, but also achieve effective beamforming, such that the signals  are combined either constructively or destructively at the user by carefully optimizing the positions of the pinching antennas, which is referred to as \emph{pinching beamforming}.
\end{remark}




% \begin{subequations}
%     \begin{align}
%         \frac{d A(x)}{d x} &= -j \kappa_g B(x) e^{-j(\beta_{\mathrm{p}} - \beta_{\mathrm{g}}) z} + j \eta_g A(x),\\
%         \frac{d B(x)}{d x} &= -j \kappa_p B(x) e^{j(\beta_{\mathrm{p}} - \beta_{\mathrm{g}}) z} + j \eta_p B(x),
%     \end{align}
% \end{subequations}
% where 
% \begin{subequations}
%     \begin{align}
%         \kappa_g = \frac{\kappa_{g,p} - \phi \chi_p}{1 - |\phi|^2}, \quad \kappa_p = \frac{\kappa_{p,g} - \phi^* \chi_g}{1 - |\phi|^2}, \\
%         \eta_g = \frac{\kappa_{p,g} \phi - \chi_g}{1 - |\phi|^2}, \quad \eta_p = \frac{\kappa_{p,g} \phi^* - \chi_p}{1 - |\phi|^2}. 
%     \end{align}
% \end{subequations}
% Here, $\kappa_{g,p}$ and $\kappa_{p,g}$ are mode coupling coefficients, $\phi$ is the butt coupling coefficient, and $\chi_g$ and $\chi_p$ represent the changes in the propagation constant. The mode and butt coupling coefficients follow the relationship below to satisfy the law of conservation of energy:
% \begin{equation}
%     \kappa_{p,g} = \kappa_{g,p}^* + (\beta_{\mathrm{p}} - \beta_{\mathrm{g}}) \phi^*.
% \end{equation}
% Note that in practice, we typically have $\chi_g \approx \chi_p \approx 0$.  









