\begin{abstract}
    The Pinching-Antenna SyStem (PASS) is a revolutionary flexible antenna technology designed to enhance wireless communication by establishing strong line-of-sight (LoS) links, reducing free-space path loss and enabling antenna array reconfigurability. PASS uses dielectric waveguides with low propagation loss for signal transmission, radiating via a passive pinching antenna, which is a small dielectric element applied to the waveguide. This paper first proposes a physics-based hardware model for PASS, where the pinching antenna is modeled as an open-ended directional coupler, and the electromagnetic field behavior is analyzed using coupled-mode theory. A simplified signal model characterizes the coupling effect between multiple antennas on the same waveguide. Based on this, two power models are proposed: equal power and proportional power models. Additionally, a transmit power minimization problem is formulated/studied for the joint optimization of transmit and pinching beamforming under both continuous and discrete pinching antenna activations. Two algorithms are proposed to solve this multimodal optimization problem: the penalty-based alternating optimization algorithm and a low-complexity zero-forcing (ZF)-based algorithm. Numerical results show that 1) the ZF-based low-complexity algorithm performs similarly to the penalty-based algorithm, 2) PASS reduces transmit power by over 95\% compared to conventional and massive MIMO, 3) discrete activation causes minimal performance loss but requires a dense antenna set to match continuous activation, and 4) the proportional power model yields performance comparable to the equal power model.

    % The Pinching-Antenna SyStem (PASS) is a revolutionary flexible-antenna technology designed to establish strong line-of-sight (LoS) links for wireless users, and significantly mitigate free-space pathloss, and introduce the antenna array reconfigurability. Unlike conventional wireless systems, PASS utilizes dielectric waveguides with exceptionally low propagation loss as the primary transmission medium, radiating signals via a passive pinching antenna, which is a small separated dielectric element applied to the waveguide. In this paper, a physics model for PASS is proposed, where the pinching antenna is modeled as an open-ended directional coupler. Using this model, the behavior of the electromagnetic field in PASS is characterized based on coupled-mode theory. A simplified signal model is then introduced based on the physics model, revealing the coupling effect between multiple pinching antennas on the same waveguide. Based on this coupling relationship, two simplified power models are developed: the equal power model and the proportional power model. Furthermore, a joint digital and pinching beamforming design for a general downlink multi-user PASS is investigated, aiming to minimize the transmit power subject to users’ individual signal-to-interference-plus-noise ratio (SINR) constraints, considering two cases with continuous and discrete pinching antenna deployment, respectively. Two effective algorithms are proposed to address the highly coupled and multimodal optimization problem: the penalty-based alternating optimization algorithm and the zero-forcing (ZF)-based low-complexity algorithm. Our numerical results validate the effectiveness of the  proposed algorithms and unveil that, 1) despite its low complexity, the proposed ZF-based algorithm achieves comparable performance to the penalty-based algorithm, 2)  PASS reduces transmit power by more than 95\% compared to conventional and massive MIMO, 3) discrete activation does not lead to significant performance loss but requires a dense set of available antenna positions to match the performance of continuous activation, and 4) the proportional power model performs comparably to the equal power model on average.
\end{abstract}

\begin{IEEEkeywords}
    Beamforming, coupled-mode theory, pinching-antenna system, 
\end{IEEEkeywords}