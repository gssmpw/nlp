
\begin{figure*}[t!]
    \centering
    \includegraphics[width=0.7\textwidth]{./Comparison.pdf}
    \caption{Comparison between conventional wireless system (left) and PASS (right).}
    \label{comparison}
    \vspace{-0.5cm}
\end{figure*} 

\section{Introduction} \label{sec:intro}

\IEEEPARstart{S}INCE Marconi demonstrated the feasibility of wireless communication in the late 19th century, the technology has undergone significant evolution and remarkable transformations. To address the unpredictable and dynamic nature of wireless channels, numerous advancements have been made in the air interface design, channel coding, source compression, and communication protocols for improving data rates and enhancing reliability. Among these advancements, multiple-input multiple-output (MIMO) has been one of the most important evolutionary techniques for wireless communication over the past few decades. By exploiting antenna arrays, MIMO brings about multiple benefits, such as enhanced signal strength through beamforming, mitigation of multi-path fading, and efficient spatial-domain multiplexing of users~\cite{bjornson2023twenty}. Since the advent of the third generation (3G) system, MIMO has been a fundamental component of wireless communication standards. However, during that era, the size of antenna arrays in MIMO systems was generally limited. The breakthrough came when Marzetta demonstrated the significant benefits of deploying an infinite number of antennas in 2010~\cite{marzetta2010noncooperative}, revealing the potential of MIMO to enhance communication performance while reducing system complexity. This revelation paved the way for the concept of massive MIMO, i.e., employing large-scale antenna arrays at base stations. Over time, massive MIMO has evolved into a key research focus and has become a reality with the deployment of 5G networks. 


However, massive MIMO has faced numerous challenges, as it is expected to transition from “Massive” in 5G (typically with 32-64 antennas) to “Gigantic” in 6G~\cite{Xtext, bjornson2024enabling}, where the number of antennas is expected to scale to hundreds or even thousands. One of the key obstacles is the complexity and cost of implementing massive MIMO since each antenna typically needs to be fed by a dedicated radio-frequency (RF) chain. Exploiting low-resolution analog-to-digital converters in RF chains or hybrid analog-digital antenna arrays with a limited number of RF chains were common methods to address this challenge, especially in the millimeter-wave band~\cite{heath2016overview}. More recently, advancements in metamaterials have paved the way for new antenna technologies, exemplified by waveguide-fed metasurface antennas~\cite{smith2017analysis, shlezinger2021dynamic, di2024reconfigurable}, which facilitate the ultra-dense deployment of antenna elements at a significantly lower cost and making massive MIMO implementation more feasible.

Flexible-antenna technique is a new evolution of MIMO. Unlike massive MIMO focusing on enlarging the wireless channel dimension, the flexible-antenna technique focuses on enabling the reconfiguration of the wireless channel. One of the most well-known approaches in this domain is the reconfigurable intelligent surface (RIS) technique~\cite{huang2019reconfigurable, wu2019intelligent, mu2021simultaneously}. By deploying RIS between transceivers, the wireless channel can be intelligently reconfigured by adjusting the phase shifts of the signals reflected/refracted by the RIS. More recently, fluid antennas~\cite{new2024tutorial} and movable antennas~\cite{zhu2023movable} have emerged as promising flexible-antenna technologies. The fundamental concept behind these approaches is to implement antenna arrays where individual antenna elements can dynamically adjust their positions within a spatial region, thus creating favorable channel conditions to enhance communication performance. 

Nevertheless, as shown on the left of Fig. \ref{comparison}, both massive MIMO and flexible-antenna techniques have limited capability in fundamentally addressing free-space pathloss and line-of-sight (LoS) blockage, two major causes of signal attenuation in wireless communications. While massive MIMO can achieve high beamforming gains to strengthen signals, it cannot combat LoS blockage and to effectively mitigate free-space pathloss, particularly for cell-edge users. RISs have been considered as a promising solution to overcome LoS blockage by creating virtual LoS paths. However, the double fading effect caused by signal reflection results in much higher pathloss compared to a direct LoS channel~\cite{ozdogan2019intelligent}. Additionally, fluid and movable antennas are typically capable of adjusting their positions only within a few wavelengths, making them more effective for mitigating small-scale fading rather than addressing large-scale pathloss. It is worthy to point out that all the aforementioned MIMO systems are lack of antenna array reconfigurability, i.e., once an antenna array is built, adding or removing antennas is no longer possible.

Pinching-Antenna SyStem (PASS) is a revolutionary technique for addressing the challenges of free-space pathloss and LoS blockage encountered by conventional multi-antenna technologies. This technique was originally proposed and prototyped by NTT DOCOMO in 2022~\cite{suzuki2022pinching}. As illustrated on the right of Fig. \ref{comparison}, PASS employs a dielectric waveguide as its primary transmission medium, which is known for its exceptionally low propagation loss (e.g., 0.01 dB/m \cite{pozar2021microwave}). By pinching a small separated dielectric element, referred to as a \emph{pinching antenna}, onto the waveguide, the system enables signal emission from the waveguide into the pinching antenna, which then radiates the signal into free space. Building on this principle, waveguides can be pre-deployed to extend service coverage, allowing pinching antennas to be placed at positions close to users. This strategic placement transforms the wireless system into a \emph{near-wired} system and hence establishes strong LoS links with users, effectively minimizing free-space path loss and mitigating blockage issues. Additionally, unlike existing MIMO systems, PASS allows both the number and positions of pinching antennas to be easily adjusted by simply pinching them to or releasing them from the waveguide~\cite{suzuki2022pinching}. This feature provides a low-cost and scalable approach to implementing MIMO while also facilitating the so-called \emph{pinching beamforming}, which enhances communication performance by dynamically optimizing antenna positions \cite{liu2025pinching}.

Given the successful prototyping of PASS by NTT DOCOMO, theoretical research on this topic has been steadily growing, though it remains in its early stages. The first theoretical study on PASS for the communication system design was presented in \cite{ding2024flexible}, where the authors provided a comprehensive analysis and developed low-complexity pinching beamforming designs for fundamental single-user and two-user scenarios. The array gain achieved by multiple pinching antennas on a waveguide was analyzed in \cite{ouyang2025array}, unveiling the optimal number of antennas and their spacing for maximizing the beamforming gain. 
% The authors of \cite{tegos2024minimum} studied an uplink PASS system and proposed an iterative antenna position optimization algorithm to maximize the sum rate under perfect phase alignment conditions. In \cite{wang2024antenna}, the authors investigated a downlink PASS system and introduced a matching theory-based optimization method for activating pinching antennas at preconfigured discrete positions. Their findings also highlighted the advantages of using non-orthogonal multiple access (NOMA) in PASS. Expanding on this,
The authors of \cite{bereyhi2025downlink} explored a downlink PASS architecture utilizing multiple waveguides, each equipped with a single pinching antenna, and proposed a greedy approach for jointly optimizing the transmit and pinching beamforming. Meanwhile, \cite{guo2025deep} examined a more generalized scenario, where multiple pinching antennas were deployed on each waveguide, and introduced a graph neural network (GNN)-based deep learning method to address the corresponding joint beamforming optimization problem.

Although PASS has attracted growing attention, several key challenges remain unsolved. On the one hand, the physics modeling of PASS is still underdeveloped, which is crucial for establishing an accurate signal model. In existing studies \cite{ding2024flexible, ouyang2025array, bereyhi2025downlink, guo2025deep}, it is commonly assumed that all signal power within the waveguide is fully radiated into free space and that each pinching antenna on a waveguide emits identical radiation power—an assumption analogous to conventional MIMO systems. However, pinching antennas operate fundamentally differently from traditional electronic antennas, and such assumptions may lack a solid physical foundation and fail to accurately reflect real-world behaviors. On the other hand, most existing works design PASS under simplified assumptions \cite{ding2024flexible, ouyang2025array, bereyhi2025downlink}, such as a single user, a single waveguide, a single pinching antenna per waveguide, or perfectly aligned signal phases. Although the GNN-based deep learning model proposed in \cite{guo2025deep} is capable of handling more complex scenarios with arbitrary numbers of users, waveguides, and pinching antennas, it suffers from a key limitation: the model parameters need to be retrained once the system configuration changes, limiting its generalization ability. Motivated by these challenges, this paper aims to develop a fundamental physics-based signal model for PASS and explore joint beamforming designs for more general scenarios. The key contributions of this work are summarized as follows:
\begin{itemize}
    \item We propose a physics-based hardware model for PASS, in which a pinching antenna is modeled as an open-ended directional waveguide coupler to facilitate the adjustment of radiation characteristics and simplify signal modeling. Based on this model, we characterize the relationship between the electromagnetic (EM) fields within the waveguide and those radiated by the pinching antennas using coupled-mode theory.
    \item We derive a novel signal model for PASS based on the proposed physics framework, revealing the inherent coupling effect between the radiation power of pinching antennas deployed on the same waveguide. Leveraging this coupling relationship, we introduce two simplified power models and their respective implementation methods: equal power and proportional power models.
    \item We formulate a joint transmit and pinching beamforming optimization problem to minimize the transmit power in a general PASS system with arbitrary numbers of users, waveguides, and pinching antennas, considering both continuous and discrete activation of pinching antennas. To solve this highly nonconvex, coupled, and multimodal optimization problem, we propose two algorithms: the penalty-based alternating optimization algorithm and the zero-forcing (ZF)-based low-complexity algorithm.
    \item We provide comprehensive numerical results to validate the advantages of PASS and the effectiveness of the proposed algorithm. The results demonstrate that 1) the ZF-based algorithm delivers performance comparable to the penalty-based algorithm but has a low complexity, 2) PASS significantly reduces transmit power, achieving a reduction of over 95\% compared to conventional and massive MIMO, 3) a dense set of available antenna positions is required for discrete activation to achieve similar performance to continuous activation, and 4) the proportional power model exhibits performance comparable to the equal power model.
\end{itemize}

The rest of this paper is structured as follows. Section \ref{sec:model} introduces the proposed physics-based hardware model and signal model for PASS. Section \ref{sec:beamforing} presents the general system model for downlink PASS and introduces a penalty-based alternating optimization method and a ZF-based algorithm for solving the joint beamforming optimization problem. Numerical evaluations and performance comparisons under various system configurations are presented in Section \ref{sec:results}. Finally, Section \ref{sec:conclusion} summarizes the findings and concludes the paper.


\emph{Notations:} Scalars are denoted using regular typeface, vectors and matrices are represented in boldface, and Euclidean subspaces are indicated with calligraphic letters. The set of complex and real numbers are denoted by $\mathbb{C}$ and $\mathbb{R}$, respectively. The inverse, conjugate, transpose, conjugate transpose, and trace operators are denoted by $(\cdot)^{-1}$, $(\cdot)^*$, $(\cdot)^T$, $(\cdot)^H$, and $\mathrm{tr}(\cdot)$, respectively. The absolute value, Euclidean norm, Frobenius norm, and maximum norm are denoted by $|\cdot|$, $\|\cdot\|$, $\|\cdot\|_F$, and $\|\cdot\|_\infty$ respectively. The real part of a complex number of demoted by $\Re \{\cdot\}$. The entry in the $n$-th row and $m$-th column of a matrix $\mathbf{X}$ is denoted by $[\mathbf{X}]_{n,m}$. An identity matrix of dimension $N \times N$ is denoted by $\mathbf{I}_N$. The big-O notation is given by $O(\cdot)$. A diagonal matrix with diagonal entries $x_1,\dots,x_N$ is denoted as $\mathrm{diag}(x_1,\dots,x_N)$.    




% \begin{figure*}[t!]
% \centering
% \begin{subfigure}[t]{0.48\textwidth}
%     \centering
%     \includegraphics[height=0.5\textwidth]{./Comparison_conventional.pdf}
% \end{subfigure}
% \hspace{-1.5cm}
% \begin{subfigure}[t]{0.48\textwidth}
%     \centering
%     \includegraphics[height=0.5\textwidth]{./Comparison_PASSpdf.pdf}
% \end{subfigure}
% \caption{Comparison between conventional wireless system (left) and PASS (right).}
% \end{figure*} 

