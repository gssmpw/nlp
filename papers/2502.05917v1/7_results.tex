
\section{Numerical Results} \label{sec:results}


In this section, numerical results are presented to illustrate the advantages of PASS and validate the effectiveness of the proposed algorithms. Unless stated otherwise, the following setup is used throughout the simulations. We consider $N = 5$ waveguides, each equipped with $M = 6$ pinching antennas and fed by a dedicated RF chain, serving $K = 4$ users. The geometric configuration of the setup is depicted in Fig. \ref{setup}, where the key parameters are set as $d_0 = 15$ m, $d_x = 30$ m, $d_y = 3$ m, and $d_z = 10$ m. The waveguides are uniformly deployed in parallel along the $x$-axis with a spacing of $6$ m, while the users are randomly positioned within the designated service area. The maximum range and minimum spacing of pinching antenna positions are set to $x_{\max} = 50$ m and $\Delta x = 0.1$ m, respectively. For discrete activation, the number of discrete positions is set to $10$ per meter. 


\begin{figure}[t!]
  \centering
  \includegraphics[width=0.45\textwidth]{./Simulation_setup.pdf}
  \caption{The simulation setup.}
  \label{setup}
\end{figure} 

Furthermore, the carrier frequency, channel gain, noise power, and effective index of the waveguide are set to $15$ GHz, $|\eta|^2 = \left(\frac{\lambda}{4 \pi}\right)^2 = -56$ dB, $\sigma_k^2 = -80$ dBm, and $n_g = 1.4$, respectively. The total power radiated from all pinching antennas along each waveguide is constrained by $\sum_{m=1}^{M} \alpha_m^2 = 0.9$, which applies to both equal and proportional power models. The minimum SINR of all users is set to $20$ dB. For the proposed algorithms, the penalty factor is initialized as $\rho = 10$, the reduction factor is set to $\epsilon = 0.1$, and the convergence threshold is set to $10^{-3}$. The number of search points in the one dimensional search for continuous activation is set to $10^6$. Furthermore, due to its sensitivity to initialization, the penalty-based method utilizes the antenna positions obtained from the ZF-based algorithm as its starting point. Unless otherwise specified, all subsequent results are obtained by averaging over $100$ random samples of user positions.

\begin{figure*}[t!]
  \centering
  \begin{subfigure}[t]{0.32\textwidth}
    \includegraphics[width=1\textwidth]{./convergence_2.eps}
    \caption{Transmit power in \textbf{Algorithm \ref{alg:PDD}}.}
    \label{convergence_1}
  \end{subfigure}
  \begin{subfigure}[t]{0.32\textwidth}
    \includegraphics[width=1\textwidth]{./convergence_1.eps}
    \caption{Constraint violation in \textbf{Algorithm \ref{alg:PDD}}.}
    \label{convergence_2}
  \end{subfigure}
  \begin{subfigure}[t]{0.32\textwidth}
    \includegraphics[width=1\textwidth]{./convergence_3.eps}
    \caption{Transmit power in \textbf{Algorithm \ref{alg:ZF}}.}
    \label{convergence_3}
  \end{subfigure}
  \caption{Convergence behavior of the proposed algorithms.}
  \label{convergence}
  \vspace{-0.5cm}
\end{figure*} 

For performance comparison, we consider the following benchmark schemes:
\begin{itemize}
  \item \textbf{Conventional MIMO:} In this benchmark, a conventional MIMO BS is positioned at $(0,0,3)$ and equipped with a uniform linear array along $x$-axis  comprising $N = 5$ antennas, each connected to a dedicated RF chain. The antenna spacing is set to half the wavelength. Under this setup, the signal received by user $k$ is given by 
  \begin{equation}
    \overline{y}_k = \overline{\mathbf{h}}_k^H \sum_{i=1}^K \mathbf{w}_i c_i + n_k,
  \end{equation}
  where $\overline{\mathbf{h}}_k \in \mathbb{C}^{N \times 1}$ is the channel vector. The $k$-th entry of $\overline{\mathbf{h}}_k$ is $\frac{\eta}{\overline{r}_{n,k}}e^{j \beta_0 \overline{r}_{n,k}}$, with $\overline{r}_{n,k}$ denoting the distance between $n$-th antenna and user $k$. The corresponding transmit power minimization problem can be solved using the method described in \cite{bjornson2014optimal}.
  
  \item \textbf{Massive MIMO:} In this benchmark, we consider a massive MIMO BS positioned at $(0,0,3)$, equipped with a uniform linear array along $x$-axis. The number of antennas and RF chains is set to $N = 30$ and $N_{\mathrm{RF}} = 5$, respectively, matching the configuration of the considered PASS system. Furthermore, since each RF chain in the PASS system is connected to only a subset of overall pinching antennas, we assume a similar sub-connected hybrid beamforming architecture at the massive MIMO BS, where each RF chain is connected to a subset of antennas via phase shifters \cite{yu2016alternating}. Under this setup, the signal received by user $k$ is given by 
  \begin{equation}
    \overline{\overline{y}}_k = \overline{\overline{\mathbf{h}}}_k^H \mathbf{W}_{\mathrm{RF}} \sum_{i=1}^K \mathbf{w}_i c_i + n_k,
  \end{equation}
  where $\overline{\overline{\mathbf{h}}}_k \in \mathbb{C}^{N \times 1}$ is the channel vector exhibiting the same form as $\overline{\mathbf{h}}_k$, and $\mathbf{W}_{\mathrm{RF}} \in \mathbb{C}^{N \times N_{\mathrm{RF}}}$ is the analog beamforming matrix realized by the phase shifters. The corresponding transmit power minimization problem is solved by integrating the algorithms in \cite{yu2016alternating} and \cite{shi2018spectral}.  
\end{itemize}

\subsection{Performance of the Proposed Algorithms}



\begin{figure}[t!]
  \centering
  \includegraphics[width=0.45\textwidth]{./algorithm_compare.eps}
  \caption{Comparison between the proposed algorithms under different simulation setup.}
  \label{power_algorithm_compare}
\end{figure} 


Fig. \ref{convergence} illustrates the convergence behavior the proposed algorithms.  Specifically, the convergence behavior of the transmit power $P$ in \textbf{Algorithm \ref{alg:PDD}} is shown in Fig. \ref{convergence_1}. It is interesting to observe that the transmit power $P$ is not reduced monotonically as the iteration progresses, but exhibits an oscillating upward trend before converging to a stable value. This phenomenon is due to the utilization of the penalty method. Specifically, in the initial iterations, when the penalty factor $\rho$ is large, the transmit power $P$ is minimized with an almost unconstrained auxiliary channel matrix $\mathbf{U}$, allowing it to converge to a very low stable value in the inner loop. However, as the penalty factor decreases in the outer loop, constraint violations gradually diminish as shown in Fig. \ref{convergence_2}, forcing the auxiliary channel matrix $\mathbf{U}$ to conform to the channel structure and constraints in PASS. Consequently, the transmit power increases as the outer loop progresses. Fig. \ref{convergence_3} demonstrates the fast convergence of the proposed \textbf{Algorithm \ref{alg:ZF}}. Despite its low complexity, \textbf{Algorithm \ref{alg:ZF}} achieves performance comparable to \textbf{Algorithm \ref{alg:PDD}} across various system setups, as shown in \textbf{Fig. \ref{power_algorithm_compare}}. Therefore, in the following simulations, only the results obtained by \textbf{Algorithm \ref{alg:PDD}} are presented to focus on the performance gains achieved by PASS.

\begin{figure}[t!]
  \centering
  \includegraphics[width=0.45\textwidth]{./power_vs_SINR.eps}
  \caption{Transmit power versus the minimum SINR.}
  \label{power_vs_SINR}
\end{figure} 

Furthermore, Fig. \ref{convergence} reveals that different values of the reduction factor $\epsilon$ affect the convergence speed. Specifically, the algorithm converges more quickly with a smaller $\epsilon$, while a larger $\epsilon$ leads to improved performance. However, it is worth noting that the performance gain from increasing $\epsilon$ from $0.1$ to $0.5$ is negligible. Therefore, we set $\epsilon = 0.1$ for the remaining simulations.

\subsection{Transmit Power Versus the Minimum SINR}

Fig. \ref{power_vs_SINR} shows the impact of the minimum SINR on transmit power. As expected, the transmit power increases with higher minimum SINR requirements, while PASS consistently achieves the lowest transmit power within the considered range. For instance, considering the PASS system with continuous activation and the equal power model, when the minimum SINR of users is 20 dB, PASS significantly reduces the transmit power by $99.3\%$, i.e., from $26.6$ dBm to $4.9$ dBm, compared to conventional MIMO, and by $96.6\%$, i.e., from $19.6$ dBm to $4.9$ dBm, compared to massive MIMO. This remarkable power reduction is primarily due to the ability of PASS to reduce free-space path loss significantly. It is also noteworthy that such an enhancement is achieved by incorporating only low-cost pinching antennas, rather than relying on massive expensive phase shifters as in massive MIMO.

It can also be observed from Fig. \ref{power_vs_SINR} that the proportional power model exhibits almost an identical performance as the equal power model, unveiling the negligible impact of unbalanced antenna efficiency in PASS. Furthermore, while discretely deploying pinching antennas results in non-negligible performance loss, the system still achieves a significant reduction in transmit power compared to both conventional MIMO and massive MIMO. For example, when the minimum SINR is $20$ dB, the PASS with discrete activation reduce the transmit power by $95\%$ and $99\%$ compared to conventional MIMO and massive MIMO, respectively.


\subsection{Transmit Power Versus the Distance}


\begin{figure}[t!]
  \centering
  \includegraphics[width=0.45\textwidth]{./power_vs_distance.eps}
  \caption{Transmit power versus the distance $d_0$.}
  \label{power_vs_distance}
\end{figure} 

Fig. \ref{power_vs_distance} examines the impact of distance $d_0$ on transmit power, where a larger $d_0$ indicates an increased separation between the BS and the serving area. It can be observed that the transmit power achieved by PASS remains almost unchanged as $d_0$ increases. This phenomenon can be attributed to two key factors. First, the pathloss associated with in-waveguide propagation is negligible. Second, the pinching antennas can always be deployed in proximity to the serving area, ensuring an almost constant free-space pathloss. 

Furthermore, it is somewhat counterintuitive that the transmit power in conventional MIMO initially decreases before increasing. This phenomenon arises because conventional MIMO performance is influenced not only by free-space path loss but also by the directivity of the antenna array. More specifically, the uniform linear array used in conventional MIMO can generate a highly directional beam when transmitting in its broadside direction (i.e., with a small angle of departure) \cite{kallnichev2001analysis}. However, as the transmission direction shifts closer to the array's end-fire (i.e., with a large angle of departure), its beamforming capability gradually weakens \cite{kallnichev2001analysis}, resulting in increased inter-user interference. In the considered simulation setup, as illustrated in Fig. \ref{setup}, when the distance $d_0$ is smaller, users are more likely to be positioned in the end-fire directions of the antenna array. This leads to significant inter-user interference, which, in turn, increases the required transmit power. This explains why, for conventional MIMO, its transmit power increases despite the reduction in free-space path loss as the serving area moves closer to the BS. However, for massive MIMO, a smaller distance always results in lower transmit power. This is because a large antenna array can significantly enhance beamforming capability even in the end-fire direction, mitigating the directivity limitations faced by conventional MIMO. Additionally, thanks to the flexibility of pinching antenna deployment, PASS can effectively address this directivity issue as well.

\subsection{Transmit Power Versus the Number of Antennas}

\begin{figure}[t!]
  \centering
  \includegraphics[width=0.45\textwidth]{./power_vs_antenna.eps}
  \caption{Transmit power versus the number of antennas.}
  \label{power_vs_antenna}
  % \vspace{-0.3cm}
\end{figure}

\begin{figure}[t!]
  \centering
  \includegraphics[width=0.45\textwidth]{./power_vs_discrete.eps}
  \caption{Transmit power versus the number of discrete positions.}
  \label{power_vs_discrete}
\end{figure}

Fig. \ref{power_vs_antenna} studies the impact of the number of antennas on transmit power. It can be observed that increasing the number of antennas leads to a reduction in transmit power for all the considered schemes. For instance, when the number of antennas increases from 10 to 50, the transmit power is reduced by $78\%$ for PASS with continuous activation and $68.8\%$ for PASS with discrete activation. This improvement is primarily due to the enhanced beamforming capability and the higher array gain provided by a larger antenna array, which not only mitigates inter-user interference but also strengthens the desired signal, resulting in more efficient power usage. 


\subsection{Transmit Power Versus the Number of Discrete Positions}


Fig. \ref{power_vs_discrete} provides more insights into the discrete activation of PASS, specifically focusing on the impact of the number of available discrete positions. As expected, the performance of discrete activation gradually approaches that of continuous activation as the number of discrete positions increases. However, achieving performance comparable to continuous activation requires a significantly large number of discrete positions, i.e., larger than $300$ per meter. This is because effective beamforming gain becomes challenging to achieve with a limited number of discrete positions. To understand this, consider the phase shift $e^{-j \beta_{\mathrm{g}} x_{\mathrm{p}}}$ induced by a pinching antenna located at position $x_{\mathrm{p}}$, where $\beta_{\mathrm{g}} = \frac{2 \pi n_{\mathrm{g}}}{\lambda} \approx 440$ under the given simulation setup. For optimal beamforming capability, $\beta_{\mathrm{g}} x_{\mathrm{p}}$ must be adjustable across the full range of $[0, 2\pi]$. However, due to the large value of $\beta_{\mathrm{g}}$, achieving this flexibility requires sampling $x_{\mathrm{p}}$ at very fine intervals, necessitating a high density of discrete positions. This result underscores the importance of developing high-resolution pinching antenna activation structures for the practical PASS implementation.    