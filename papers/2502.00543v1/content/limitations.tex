Although \former~can capture long-range dependencies through additional context tokens, it requires re-training if we want to change the prediction horizon, while autoregressive models can predict any number of steps into the future without re-training. 
As illustrated in Fig.~\ref{fig:qualitative} of Appendix~\ref{app:qualitative}, our model demonstrates a deficiency in accurately executing a turning maneuver. Such failures stem from long-horizon (1 second), non-autoregressive predictions in one step accentuated by the inaccuracy of terrain reconstruction caused by the high degree of complexity present in off-road topographical formations. This also reflects on the accuracy of predicting $\mathbf{Z}$. A further limitation stems from the use of a mask in place of true modality data. While this approach empowers the model with multi-task capability and to handle missing information, it nonetheless falls short of leveraging the full potential of the actual modalities. 

It is crucial to acknowledge that our observations are primarily associated with the challenges inherent in wheeled locomotion on complex, vertically challenging, off-road terrain and do not necessarily generalize to other robotic domains such as visual navigation or manipulation. In visual navigation, the robot typically relies on visual cues and image processing to perceive its environment and plan its path. In manipulation tasks, the focus is on interacting with objects rather than negotiating through complex terrain. 
Further investigation is required for general visual navigation and manipulation.