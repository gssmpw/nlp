% We use an open-source Verti-4-Wheeler (V4W) platform, as described by~\citet{Datar2024a} as our robot platform. The V4W is equipped with a Microsoft Azure Kinect RGB-D camera and an NVIDIA Jetson Xavier processor. On the software side, \former~is implemented with PyTorch and trained on a single A5000 GPU with 24GB memory while only occupying 2GB of memory.

We use an open-source V4W robotic platform, as detailed by~\citet{datar2024wheeled}, for physical evaluation. The V4W platform is equipped with a Microsoft Azure Kinect RGB-D camera to build elevation maps~\cite{miki2022elevation} and an NVIDIA Jetson Xavier processor for onboard computation. The proposed \former~model is implemented using PyTorch and trained on a single NVIDIA A5000 GPU with 24GB of memory, demonstrating efficient memory utilization with a peak memory footprint of only 2GB.

\noindent\textbf{Optimization:} we use the AdamW optimizer~\cite{loshchilov2019decoupled} with learning rate of $5e^{-4}$ and weight decay of $0.08$. We train \former~for 200 epochs with a batch size of 512.

% \textbf{Dataset:} We utilize the dataset introduced in \tal~\cite{datar2024terrainattentive}, collected on a 3.1 m $\times$ 1.3 m rock testbed with a maximum height of 0.6 m. 
% The dataset consists of 30 minutes of data on a planar surface and 30 minutes on the rock testbed. The dataset has a variety of 6-DoF vehicle states including vehicle rollover and getting stuck achieved while manually teleoperating the robot over the rock testbed during data collection. This variety is achieved because of the modularity of the rock testbed enabling flexible reconfiguration for data collection and mobility experiments. The dataset contains VIO for vehicle state estimation, elevation maps built from depth images, and teleoperated vehicle controls including throttle and steering commands.


\noindent\textbf{Dataset:} We utilize the dataset introduced by \tal~\cite{datar2024terrainattentive}, which was collected on a 3.1 m $\times$ 1.3 m modular rock testbed with a maximum height of 0.6 m. The dataset includes 30 minutes of data from both a planar surface and the rock testbed, capturing a diverse range of 6-DoF vehicle states. These states encompass scenarios such as vehicle rollovers and instances of the vehicle getting stuck, all recorded during manual teleoperation over the reconfigurable rock testbed. 
% This modularity allows for a flexible setup, facilitating comprehensive data collection and mobility experiments. 
The dataset comprises visual-inertial odometry for vehicle state estimation, elevation maps derived from depth images, and teleoperation control data, including throttle and steering commands, to provide a holistic view of vehicle dynamics.
% To make the dataset as diverse as possible variety of 6-DoF vehicle states were 
% The modular nature of the rock testbed enables flexible reconfiguration, facilitating mobility experiments across various terrain configurations. 

% The dataset utilized in this study is derived from the work presented in \tal~\cite{datar2024terrainattentive}. This dataset was collected on a physical rock testbed measuring 3.1 m in length, 1.3 m in width, and with a maximum height of 0.6 m. The testbed's modular design allows for flexible reconfiguration of the terrain, enabling mobility experiments across a diverse range of terrain configurations. This modularity is crucial for evaluating the robustness and generalization capabilities of the proposed model under varying environmental conditions. The controlled environment of the rock testbed enables precise data acquisition and facilitates repeatable experiments, which are essential for rigorous scientific evaluation. This structured approach to data collection and experimental design allows for a systematic analysis of the model's performance under controlled conditions.

% \textbf{Freezing and Fine-tuning:} We adopt two distinct training methodologies for downstream tasks. Firstly, we freeze the weights of the \former~model and train solely the task-specific heads. This approach enables an evaluation of the \former~inherent representational capacity and its generalization potential to diverse downstream tasks. Secondly, in the fine-tuned version, we unfreeze the \former~weights, allowing them to be jointly optimized with the downstream heads using the task-specific loss function. This fine-tuning process aims to enhance performance by enabling the model to adapt its pre-trained representations to the idiosyncrasies of each downstream task, potentially leading to more task-specific feature extraction and superior overall performance.

% \begin{figure}
%   \centering
%   \includegraphics[width=0.75\columnwidth]{figure/testbed.jpg}
%   \caption{\textbf{Rock Testbed and V4W used for Data Collection and Experiments}. The modularity of the testbed allows diverse rock configurations for training and evaluation.}
%   \label{fig::testbed}
% \end{figure}
