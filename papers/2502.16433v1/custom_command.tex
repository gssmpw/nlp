\usepackage{mathtools}
\usepackage{amsmath,amsfonts,amssymb,amsthm,commath}

%\usepackage[colorlinks]{hyperref} % comment out in submission!
%\usepackage{cleveref}% http://ctan.org/pkg/cleveref
\usepackage{graphicx}
\graphicspath{ {./figures/} }
\usepackage{caption}
\usepackage{multirow}
\usepackage{booktabs}
\usepackage{subcaption}
\usepackage{listings}
\usepackage{tikz,forest}
\usetikzlibrary{arrows.meta}
\usetikzlibrary{trees}
\usetikzlibrary{er,positioning}
\usetikzlibrary{fit,positioning}
\tikzstyle{block} = [rectangle, draw, fill=blue!20, 
    text width=12.8em, text centered, rounded corners, minimum height=4em]
\tikzstyle{line} = [draw, -latex']
\tikzstyle{cloud} = [draw, ellipse,fill=red!20, node distance=3cm,
    minimum height=2em]
   
   
\forestset{
    .style={
        for tree={
            base=bottom,
            child anchor=north,
            align=center,
            s sep+=1cm,
    straight edge/.style={
        edge path={\noexpand\path[\forestoption{edge},thick,-{Latex}] 
        (!u.parent anchor) -- (.child anchor);}
    },
    if n children={0}
        {tier=word, draw, thick, rectangle}
        {draw, diamond, thick, aspect=2},
    if n=1{%
        edge path={\noexpand\path[\forestoption{edge},thick,-{Latex}] 
        (!u.parent anchor) -| (.child anchor) node[pos=.2, above] {Y};}
        }{
        edge path={\noexpand\path[\forestoption{edge},thick,-{Latex}] 
        (!u.parent anchor) -| (.child anchor) node[pos=.2, above] {N};}
        }
        }
    }
}
   
   
\usepackage[inline]{enumitem}

% following loops stolen from djhsu
\def\ddefloop#1{\ifx\ddefloop#1\else\ddef{#1}\expandafter\ddefloop\fi}
\def\ddef#1{\expandafter\def\csname bb#1\endcsname{\ensuremath{\mathbb{#1}}}}
\ddefloop ABCDEFGHIJKLMNOPQRSTUVWXYZ\ddefloop
\def\ddef#1{\expandafter\def\csname c#1\endcsname{\ensuremath{\mathcal{#1}}}}
\ddefloop ABCDEFGHIJKLMNOPQRSTUVWXYZ\ddefloop
\def \by{\boldsymbol y}
\def \bi{\boldsymbol i}
\usepackage{tikz}
\newcommand*\circled[1]{\tikz[baseline=(char.base)]{
            \node[shape=circle,draw,inner sep=2pt] (char) {#1};}}

\newenvironment{hproof}{%
  \renewcommand{\proofname}{Proof sketch}\proof}{\endproof}
\def\tr{\operatorname{tr}}

\def\SPAN{\textup{span}}
\def\tu{\textup{u}}
\def\P{\mathbb{P}}
\def\S{\mathbb{S}}
\def\Q{\mathbb{Q}}
\def\cF{\mathcal{F}}
\def\cP{\mathcal{P}}
\def\cI{\mathcal{I}}
\def\cH{\mathcal{H}}
\def\cO{\mathcal{O}}
\def\wh{\widehat}
\def\wt{\widetilde}
\def\vol{\text{vol}}
\def\cost{\text{cost}}
\def\spn{\text{span}}
\def\poly{\text{poly}}
\def\pr{\pi_{\operatorname{ref}}}

\def\kl{D_{\text{KL}}}
\def\tv{D_{\text{TV}}}
%\newcommand{\todo}[1]{\textcolor{red}{#1}}
\def\enc{\text{Encode}}
\def\be{\mathbf{e}}
\def\bw{\mathbf{w}}
\def\bx{\mathbf{x}}
\def\nf{\nabla f}
\def\veps{\varepsilon}
\newcommand*\subtxt[1]{_{\textnormal{#1}}}
\DeclareRobustCommand\_{\ifmmode\expandafter\subtxt\else\textunderscore\fi}

\DeclarePairedDelimiter\autobracket{(}{)}
\newcommand{\p}[1]{\autobracket*{#1}}

\newcommand{\ip}[2]{\left\langle #1, #2 \right \rangle}
\newcommand{\mathbbm}[1]{\text{\usefont{U}{bbm}{m}{n}#1}} % from mathbbm.sty
\newcommand{\rvect}[1]{\begin{bmatrix} #1 \end{bmatrix}}
\newcommand{\conv}{\text{conv}}
\newcommand{\dist}{\text{dist}}
\newcommand{\sgn}{\text{sgn}}
\newcommand{\linfty}{\ell^\infty}

\newtheorem{fact}{Fact}
%\newtheorem{lemma}{Lemma}
%\newtheorem{theorem}{Theorem}
%\newtheorem{corollary}{Corollary}
\newtheorem{conjecture}{Conjecture}
\newtheorem{innercustomthm}{Theorem}
\newtheorem{innercustomlemma}{Lemma}
\theoremstyle{definition}
%\newtheorem{definition}{Definition}[section]
%\newtheorem{remark}{Remark}
\newtheorem{claim}{Claim}
%\newtheorem{proposition}{Proposition}
\newtheorem{example}{Example}
\newtheorem{question}{Question}
\newtheorem{answer}{Answer}
