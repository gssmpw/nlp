\section{Appendix}
\subsection{Derivation of the CPO objective function}
Here we give a full derivation of the CPO objective function in \cref{eq:norankcpo}.

Let $\vy_1,\ldots,\vy_K$ be $K$ continuations of a given prefix $\vx$. Without loss of generality, let $\vy_1$ be the best candidate. We are interested in the MLE of the event $P(\vy_1 \text{ is the best among $K$ candidates}|\vx)$.

We start from the sequence-level (RLHF) objective, notice that here $r(\cdot)$ is a reward over language quality, not human preference.
\begin{align}
\begin{split}
		& \max_{\pi_\theta} \E_{\vx\sim\gD, \vy\sim\pi_\theta(\vy|\vx)}[r(\vx, \vy)] \\
		& \quad-\beta \KL\p{\pi_\theta(\vy|\vx)||\pr(\vy|\vx)},
\end{split}
\end{align}

Its optimum is achieved at the following EBM:
\begin{align}\label{eq:rlhf_ebm_appendix}
	\pi^*(\vy|\vx) = \frac{1}{Z(\vx)}\pr(\vy|\vx)\exp\p{\frac{1}{\beta}r(\vx,\vy)},
\end{align}
where $Z(\vx)=\sum_{\vy}\pr(\vy|\vx)\exp\p{\frac{1}{\beta}r(\vx,\vy)}$ is the partition function. See the proof in \cite{rafailov2023direct,korbak2022rl}.



Now we consider the natural extension of the Bradley-Terry model to $K$ candidates:
\begin{align}\label{eq:mle_k_appendix}
\begin{split}
	&P(\vy_1 \text{ is the best among $K$ candidates}|\vx)\\
	&=\frac{\exp\p{r^*(\vx, \vy_1)}}{\sum_{k\in[K]}\exp\p{r^*(\vx, \vy_k)}}.
\end{split}
\end{align}

Now assuming we have the optimal policy $\pi^*$, we can reparameterize $r$ by rearranging \cref{eq:rlhf_ebm_appendix}:
\begin{align}\label{eq:implicit_r_appendix}
\begin{split}
	r^*(\vx, \vy)=\beta \log \frac{\pi^*(\vy \mid \vx)}{\pi_{\mathrm{ref}}(\vy \mid \vx)}+\beta \log Z(\vx).
\end{split}
\end{align}
Plugging \cref{eq:implicit_r_appendix} into \cref{eq:mle_k_appendix}, we get \cref{eq:norankcpo}.

\subsection{Query template of Wiki text generation}
The template is the following: ``\texttt{For the following prefix, which continuation is better?$\backslash$n
    Prefix: \{\}$\backslash$n
    Continuation A: \{\}$\backslash$n
    Continuation B: \{\}$\backslash$n
    State only "A" or "B" to indicate which continuation is more helpful.$\backslash$n
    Better:}''