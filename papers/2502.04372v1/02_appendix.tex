\documentclass{article}
\usepackage{blindtext}
\usepackage[letterpaper, margin=1in]{geometry}
\usepackage{multirow}
\usepackage[justification=centering]{caption}
\usepackage{amsmath,amsfonts,amssymb}
\usepackage{graphicx}

\usepackage{amsthm}
\newtheorem{theorem}{Theorem}

\pagenumbering{gobble}

\title{Appendix}
\date{}

\begin{document}

\maketitle

\appendix

\section{Proof of Theorem 1}

\begin{theorem}
    Suppose that the data $(X_1, Y_1), \ldots, (X_n, Y_n)$, along with a new data point $(X^\mathrm{new}, Y^\mathrm{new})$, are exchangeable random variables. Then,
    for any conformity score $s$, any $\alpha \in (0, 1)$, it holds that, for all label values $y$,
    \[ \mathbb{P}[ Y^\mathrm{new} \in C_\alpha (X^\mathrm{new}) \ |\ Y^\mathrm{new} = y ] \geq 1 - \alpha. \]
\end{theorem}

\begin{proof}
    Since the data is exchangeable, the random variables $S_i = s(X_i, Y_i)$ for $i = 1, \ldots, n, \mathrm{new}$ are also exchangeable. Thus, for any $y$:
    \begin{align*}
        \mathbb{P}[ Y^\mathrm{new} \in C_\alpha (X^\mathrm{new}) \ |\ Y^\mathrm{new} = y ]
        &= \mathbb{P}[ s(X^\mathrm{new}, Y^\mathrm{new}) \leq t_{y,\alpha} \ |\ Y^\mathrm{new} = y]
        \\ &= \mathbb{P}[ S_\mathrm{new} \leq \hat{q}_{1-\alpha} (\{S_i : i = 1, \ldots, n \text{ such that } Y_i = y \} \cup \{+\infty\}) \ |\ Y^\mathrm{new} = y]
        \\ &\geq 1 - \alpha,
    \end{align*}
    where the last step holds by Lemma 1 of (Tibshirani et al. 2019). % \cite{quantile-lemma}.
\end{proof}


\newpage
\section{Comprehensive Results Overview}

Table \ref{tab:comp} presents the final performance of our framework using different classification models and across various configurations. The results highlight differences in accuracy, precision, recall and AUC-ROC scores, demonstrating the adaptability of the framework.

\setcounter{table}{3}
\bgroup
\def\arraystretch{1.3}
\begin{table}[h]
\begin{tabular}{lccccccl}
Label                        & Model      & Accuracy        & Precision       & Recall          & AUC-ROC       & Yes/No & Obs.     \\ \hline
\multirow{6}{*}{Pet}     & TF-IDF+XGB & 0.83 $\pm$ 0.02 & 0.81 $\pm$ 0.03 & 0.97 $\pm$ 0.03 & 0.79 $\pm$ 0.07 & 23/77  &          \\ \cline{2-8} 
                             & TF-IDF+XGB & 0.80 $\pm$ 0.02 & 0.74 $\pm$ 0.03 & 0.96 $\pm$ 0.04 & 0.95 $\pm$ 0.05 & 27/73  &  $^{1}$        \\ \cline{2-8} 
                             & TF-IDF+XGB & 0.84 $\pm$ 0.03 & 0.84 $\pm$ 0.03 & 0.96 $\pm$ 0.04 & 0.95 $\pm$ 0.05 & 36/64  &  $^{2}$        \\ \cline{2-8} 
                             & TF-IDF+XGB & 0.81 $\pm$ 0.02 & 0.84 $\pm$ 0.03 & 0.90 $\pm$ 0.03 & 0.85 $\pm$ 0.05 & 41/59  &  $^{3}$        \\ \cline{2-8} 
                             & TF-IDF+XGB & 0.92 $\pm$ 0.01 & 0.96 $\pm$ 0.01 & 0.96 $\pm$ 0.01 & 0.94 $\pm$ 0.06 & 70/130 &  $^{245}$        \\ \cline{2-8} 
                             & DeBERTaV3  & 0.66 $\pm$ 0.02 & 0.66 $\pm$ 0.02 & 0.97 $\pm$ 0.03 & 0.44 $\pm$ 0.06 & 23/77  &          \\ \cline{2-8} 
                             & TF-IDF+XGB+Rand & 0.84 $\pm$ 0.02 & 0.83 $\pm$ 0.02 & 0.97 $\pm$ 0.03 & 0.90 $\pm$ 0.10 & 18/82  & $^{7}$         \\ \hline
\multirow{3}{*}{Drinkable}   & TF-IDF+XGB & 0.98 $\pm$ 0.02 & 0.98 $\pm$ 0.02 & 0.98 $\pm$ 0.02 & 0.50 $\pm$ 0.50 & 19/81  &          \\ \cline{2-8} 
                             & TF-IDF+XGB & 0.85 $\pm$ 0.01 & 0.86 $\pm$ 0.02 & 0.92 $\pm$ 0.02 & 0.82 $\pm$ 0.04 & 70/130 &  $^{245}$        \\ \cline{2-8} 
                             & DeBERTaV3  & 0.74 $\pm$ 0.02 & 0.74 $\pm$ 0.02 & 0.97 $\pm$ 0.03 & 0.42 $\pm$ 0.08 & 32/67  &          \\ \hline
\multirow{2}{*}{Low quality} & TF-IDF+XGB & 0.74 $\pm$ 0.02 & 0.78 $\pm$ 0.03 & 0.91 $\pm$ 0.03 & 0.50 $\pm$ 0.10 & 20/80  &          \\ \cline{2-8} 
                             & TF-IDF+XGB & 0.77 $\pm$ 0.01 & 0.76 $\pm$ 0.01 & 0.98 $\pm$ 0.02 & 0.79 $\pm$ 0.04 & 46/154 & $^{245}$         \\ \hline 
                             %& DeBERTaV3  &                 &                 &                 &                 &        &          \\ \hline
\multirow{4}{*}{Damaged}      & TF-IDF+XGB & -- & -- & -- & 0.50 $\pm$ 0.50 & 2/98   &          \\ \cline{2-8} 
                             & TF-IDF+XGB & 0.96 $\pm$ 0.01 & 0.96 $\pm$ 0.01 & 0.99 $\pm$ 0.01 & 0.75 $\pm$ 0.25 & 5/195  &  $^{4}$        \\ \cline{2-8} 
                             & TF-IDF+XGB & 0.82 $\pm$ 0.04 & 0.81 $\pm$ 0.04 & 0.95 $\pm$ 0.05 & 0.88 $\pm$ 0.12 & 30/70  &  $^{6}$        \\ \cline{2-8} 
                             & TF-IDF+XGB & 0.91 $\pm$ 0.01 & 0.91 $\pm$ 0.01 & 0.99 $\pm$ 0.01 & 0.75 $\pm$ 0.08 & 39/161 &  $^{245}$        \\ \cline{2-8} 
                             & TF-IDF+XGB+Rand & -- & -- & -- & 0.50 $\pm$ 0.50 & 1/199  & $^{47}$         \\  \hline 
                             %& DeBERTaV3  &                 &                 &                 &                 &        &          \\ \hline
\end{tabular}
\caption{Final performance with different classification models for the proposed labels after 100 manual labels using our framework. \\ {\small $^1$Done with $k$ splitted 50/50 on high and low uncertainty. \\ $^2$Done with $k$ splitted 30/70 on high and low uncertainty. \\ $^3$Done with $k$ splitted 70/30 on high and low uncertainty. \\ $^4$Done with 200 manual labelings. \\ $^5$Started with 20 pre-labelled texts.\\ $^6$Started with 40 pre-labelled texts.\\ $^7$ Random selection of entries no active learning.}}
\label{tab:comp}
\end{table}
\egroup

\newpage
\section{The OLIM User Interface}

OLIM (Open Labeller for Interactive Machine Learning) is a web interface designed to streamline dataset annotation while integrating  your conformal active learning framework. Its Dockerized architecture ensures scalability, and its intuitive design caters to both technical and non-technical users.

\paragraph{User and Label Management}  
The interface provides role-based access control, enabling user type specific access. Users can create, delete, or export labels as CSV files for offline analysis, while real-time metrics display labeling progress. As shown in Figure~\ref{fig:label_management}, the dashboard centralizes these tasks, including bulk operations and visualization of label distribution.

\begin{figure}[ht]
    \centering
    \includegraphics[width=0.75\textwidth]{label_management.png}
    \caption{Label management interface with label management, progress, CSV download and upload and option to enter individual Active Learning.}
    \label{fig:label_management}
\end{figure}

\paragraph{Active Learning Integration}  
The OLIM interface directly integrates with \textbf{OLIM-Learner}, an implementation of our proposed active learning framework. Users can launch the active learning pipeline directly from the labels management page (Figure~\ref{fig:label_management}) or manually sync labels created outside the framework. Labeling queues dynamically prioritize uncertain or contentious samples identified by the framework, streamlining the annotation workflow. For flexibility, labels modified externally (e.g., bulk CSV uploads) can be synced to OLIM-Learner to trigger retraining.

\paragraph{Search Interface}  
Powered by Elasticsearch, the search engine supports inclusion/exclusion of terms. Figure~\ref{fig:search_interface} demonstrates the query panel, which highlights matching terms in results and provides histograms of label frequencies. Users can save frequently used filters, such as date ranges or user-specific annotations.

\begin{figure}[ht]
    \centering
    \includegraphics[width=0.75\textwidth]{search_interface.png}
    \caption{Search interface with term inclusion/exclusion syntax.}
    \label{fig:search_interface}
\end{figure}

\paragraph{Deployment and Scalability}  
OLIM’s components (interface and learner) are Dockerized and deployable across separate machines. The learner can be scaled independently on GPU-enabled hardware, this allows olim to be run on the cloud or with on-premise setups.

\paragraph{Interaction Page for Domain Experts (In Development)}  
A dedicated page for non-technical specialist users (e.g. medical personal) prioritizes labels requiring attention using heuristics like conflicting annotations or low model confidence. Figure~\ref{fig:interaction_page} shows the workflow. Inconsistencies can be flagged for further re-label. Those checks can ensure intra-user consistency (e.g., a specialist past decisions) and inter-user agreement.

\begin{figure}[ht]
    \centering
    \includegraphics[width=0.75\textwidth]{interaction_page.png}
    \caption{Interaction page for specialists. Texts are chosen to be labeled based on configurable priorities.}
    \label{fig:interaction_page}
\end{figure}

OLIM balances scalability with usability, offering robust tools for collaborative labeling while reducing the overhead of manual labeling tasks. Its modular design supports diverse deployment scenarios, and ongoing developments focus on enhancing accessibility for domain experts through guided workflows and automated consistency checks.

\end{document}