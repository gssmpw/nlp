% This must be in the first 5 lines to tell arXiv to use pdfLaTeX, which is strongly recommended.
\pdfoutput=1
% In particular, the hyperref package requires pdfLaTeX in order to break URLs across lines.

\documentclass[11pt]{article}

% Change "review" to "final" to generate the final (sometimes called camera-ready) version.
% Change to "preprint" to generate a non-anonymous version with page numbers.
\usepackage[final]{acl}

% Standard package includes
\usepackage{times}
\usepackage{latexsym}
\usepackage{amssymb}
% For proper rendering and hyphenation of words containing Latin characters (including in bib files)
\usepackage[T1]{fontenc}
% For Vietnamese characters
% \usepackage[T5]{fontenc}
% See https://www.latex-project.org/help/documentation/encguide.pdf for other character sets

% This assumes your files are encoded as UTF8
\usepackage[utf8]{inputenc}
\usepackage{amsmath}
% This is not strictly necessary, and may be commented out,
% but it will improve the layout of the manuscript,
% and will typically save some space.
\usepackage{microtype}

% This is also not strictly necessary, and may be commented out.
% However, it will improve the aesthetics of text in
% the typewriter font.
\usepackage{inconsolata}
\usepackage{algorithm} % 提供 algorithm 环境
\usepackage{algpseudocode} % 提供算法伪代码的命令
%Including images in your LaTeX document requires adding
%additional package(s)
\usepackage{graphicx}
\usepackage{multirow}
\usepackage{booktabs}
\usepackage{subcaption}
\usepackage[normalem]{ulem}
\useunder{\uline}{\ul}{}
\usepackage{float}
% If the title and author information does not fit in the area allocated, uncomment the following
%
%\setlength\titlebox{<dim>}
%
% and set <dim> to something 5cm or larger.

\title{Unshackling Context Length: An Efficient Selective Attention Approach through Query-Key Compression}
\author{
	Haoyu Wang\textsuperscript{\rm 1}$^*$, Tong Teng \textsuperscript{\rm 1}\thanks{Equal Contribution.}, Tianyu Guo\textsuperscript{\rm 1}, An Xiao\textsuperscript{\rm 1}, Duyu Tang\textsuperscript{\rm 2},   \\
	\textbf{Hanting Chen\textsuperscript{\rm 1}, Yunhe Wang\textsuperscript{\rm 1}\thanks{Corresponding Author.}} \\
	\textsuperscript{\rm 1}Huawei Noah’s Ark Lab ~~~\textsuperscript{\rm 2}Huawei CBG  \\
	\texttt{\small \{wanghaoyu50, tengtong1, tianyu.guo, an.xiao, tangduyu, chenhanting, yunhe.wang\}@huawei.com}
}
% Author information can be set in various styles:
% For several authors from the same institution:
% \author{Author 1 \and ... \and Author n \\
%         Address line \\ ... \\ Address line}
% if the names do not fit well on one line use
%         Author 1 \\ {\bf Author 2} \\ ... \\ {\bf Author n} \\
% For authors from different institutions:
% \author{Author 1 \\ Address line \\  ... \\ Address line
%         \And  ... \And
%         Author n \\ Address line \\ ... \\ Address line}
% To start a separate ``row'' of authors use \AND, as in
% \author{Author 1 \\ Address line \\  ... \\ Address line
%         \AND
%         Author 2 \\ Address line \\ ... \\ Address line \And
%         Author 3 \\ Address line \\ ... \\ Address line}

% \author{First Author \\
%   Affiliation / Address line 1 \\
%   Affiliation / Address line 2 \\
%   Affiliation / Address line 3 \\
%   \texttt{email@domain} \\\And
%   Second Author \\
%   Affiliation / Address line 1 \\
%   Affiliation / Address line 2 \\
%   Affiliation / Address line 3 \\
%   \texttt{email@domain} \\}

%\author{
%  \textbf{First Author\textsuperscript{1}},
%  \textbf{Second Author\textsuperscript{1,2}},
%  \textbf{Third T. Author\textsuperscript{1}},
%  \textbf{Fourth Author\textsuperscript{1}},
%\\
%  \textbf{Fifth Author\textsuperscript{1,2}},
%  \textbf{Sixth Author\textsuperscript{1}},
%  \textbf{Seventh Author\textsuperscript{1}},
%  \textbf{Eighth Author \textsuperscript{1,2,3,4}},
%\\
%  \textbf{Ninth Author\textsuperscript{1}},
%  \textbf{Tenth Author\textsuperscript{1}},
%  \textbf{Eleventh E. Author\textsuperscript{1,2,3,4,5}},
%  \textbf{Twelfth Author\textsuperscript{1}},
%\\
%  \textbf{Thirteenth Author\textsuperscript{3}},
%  \textbf{Fourteenth F. Author\textsuperscript{2,4}},
%  \textbf{Fifteenth Author\textsuperscript{1}},
%  \textbf{Sixteenth Author\textsuperscript{1}},
%\\
%  \textbf{Seventeenth S. Author\textsuperscript{4,5}},
%  \textbf{Eighteenth Author\textsuperscript{3,4}},
%  \textbf{Nineteenth N. Author\textsuperscript{2,5}},
%  \textbf{Twentieth Author\textsuperscript{1}}
%\\
%\\
%  \textsuperscript{1}Affiliation 1,
%  \textsuperscript{2}Affiliation 2,
%  \textsuperscript{3}Affiliation 3,
%  \textsuperscript{4}Affiliation 4,
%  \textsuperscript{5}Affiliation 5
%\\
%  \small{
%    \textbf{Correspondence:} \href{mailto:email@domain}{email@domain}
%  }
%}

\begin{document}
\maketitle
\begin{abstract}

% Handling long-context sequences efficiently remains a significant challenge in large language models (LLMs). Existing methods for token selection in sequence extrapolation either employ a permanent eviction strategy or select tokens in chunks which may lead to the loss of critical information. We propose Efficient Selective Attention (ESA), a novel approach that achieves extension by efficiently selecting the most critical tokens at the token level to compute attention. ESA reduces the computational complexity of token selection by compressing the query and key vectors into lower-dimensional representations. We evaluate our method on a variety of long sequence benchmarks with maximum lengths reaching up to 256k using open-source LLMs with context lengths of 8k and 32k. ESA outperforms other selective attention methods, particularly in challenging tasks that require the retrieval of multiple pieces of information. Compared to full-attention extrapolation methods, ESA achieves comparable performance across a variety of tasks, even with superior results in certain tasks.
Handling long-context sequences efficiently remains a significant challenge in large language models (LLMs). Existing methods for token selection in sequence extrapolation either employ a permanent eviction strategy or select tokens by chunk, which may lead to the loss of critical information. We propose Efficient Selective Attention (ESA), a novel approach that extends context length by efficiently selecting the most critical tokens at the token level to compute attention. ESA reduces the computational complexity of token selection by compressing query and key vectors into lower-dimensional representations. We evaluate ESA on long sequence benchmarks with maximum lengths up to 256k using open-source LLMs with context lengths of 8k and 32k. ESA outperforms other selective attention methods, especially in tasks requiring the retrieval of multiple pieces of information, achieving comparable performance to full-attention extrapolation methods across various tasks, with superior results in certain tasks. 
% Notably, ESA retains original length capability for short sequence retrieval tasks in the 4-8k range, even outperforming full-attention approaches.

% Furthermore, experimental results on multiple benchmarks demonstrate that our approach achieves competitive accuracy, making it highly scalable for real-world applications. 
\end{abstract}

\section{Introduction}
\section{Introduction}

The convergence of vision and language in artificial intelligence has led to the development of Vision-Language Models (VLMs) that can interpret and generate multimodal content. Among these, OpenAI's Contrastive Language-Image Pre-training (CLIP) model~\cite{radford2021learningtransferablevisualmodels} has been particularly influential, demonstrating remarkable capabilities in zero-shot image classification and setting new standards for multimodal understanding~\cite{Cherti_2023, gadre2023datacompsearchgenerationmultimodal, schuhmann2021laion400mopendatasetclipfiltered, thrush2022winoground}. The success of CLIP has catalyzed a wide array of applications---from image retrieval and visual question answering to text-to-image generation---signifying a paradigm shift in how models perceive and relate visual and linguistic information.




Visual Language Models like CLIP face significant challenges in understanding and reasoning about complex scenes with multiple objects and intricate relationships. CLIP struggles to identify distinct objects and model their relationships accurately, especially when captions contain the same objects but differ in their relationships. This results in difficulty distinguishing between similar captions with different object relationships. Several benchmark datasets have been introduced to elucidate the limitations of existing models in capturing subtle relational nuances. Notably, Winoground \cite{thrush2022winoground}, VL-CheckList \cite{zhao2022vl}, ARO \cite{yuksekgonul2023and}, and CREPE \cite{ma2023crepe} have been instrumental in evaluating models' capacities to accurately match images with semantically appropriate captions. 



\begin{figure*}[t]
    \centering
    \includegraphics[width=\textwidth]{figs/main_fig_v5.pdf}
    \caption{Overview of our key contributions. Step 1: We create ComCO dataset for controlled multi-object experiments. Step 2: We identify biases in CLIP's image encoder (favoring larger objects) and text encoder (prioritizing first-mentioned objects). Step 3: We investigate the origin of these biases, finding a connection to training data characteristics. Step 4: We demonstrate the practical impacts of these biases on image-text matching task, showing how they affect model performance in multi-object scenarios.}
    \label{fig:mainfig}
    \vspace{-0.5cm} 
\end{figure*}

Numerous studies have addressed compositionality challenges in multi-object scenarios, often through end-to-end methods like fine-tuning with hard-negative samples \cite{yuksekgonul2023and} to improve model performance. However, these approaches have faced criticism and subsequent refinement, as seen in methods like SUGARCREPE \cite{hsieh2024sugarcrepe} and \cite{sahin2024enhancing}, which generate negative captions with minor structural changes or LLMs to highlight semantic distinctions. While most focus on CLIP’s ability to distinguish structurally similar yet conceptually different captions, few studies, such as Dumpala et al. \cite{dumpala2024sugarcrepe++}, explore CLIP’s performance on semantically equivalent but structurally distinct captions, revealing a gap in understanding CLIP's inconsistency with such prompts.




% While previous studies have made significant strides in understanding CLIP's limitations, our work distinguishes itself in several key aspects. Firstly, we shift the focus from evaluating CLIP's ability to differentiate between conceptually distinct captions to examining its performance with semantically equivalent but structurally varied captions. This approach allows us to probe deeper into the model's understanding of language and visual content beyond surface-level differences. Here, model systematic mistakes give an indication the potential baises. Secondly, unlike many previous works that primarily introduced benchmarks or proposed end-to-end solutions, we conduct a thorough investigation into the underlying causes of CLIP's behavior. Our study delves into the internal mechanisms of both the image and text encoders, providing insights into why the model is biased and lacks invariance to certain types of linguistic and visual variations. 

% To facilitate this in-depth analysis, we introduce the \textbf{ComCO} dataset, specifically designed to isolate and examine different aspects of CLIP's performance in {\it controlled} multi-object scenarios. Furthermore, our research spans multiple versions of CLIP trained on various datasets and architectures, ensuring the broad applicability and generalizability of our findings. By focusing on these underexplored areas and employing a more comprehensive analytical approach, our work aims to provide a deeper understanding of CLIP's limitations and pave the way for more robust and versatile vision-language models. It is important to note that such an analysis not only benefits the improvement of CLIP but also has significant implications for related models, such as text-to-image (T2I) generative models and multimodal large language models (MLLMs). Understanding the intricacies of CLIP's encoding process can inform and enhance the development of these technologies, potentially leading to advancements across various domains of artificial intelligence. As shown in Figure \ref{fig:mainfig}, our key contributions are as follows:


While previous studies have advanced our understanding of CLIP's limitations, our work uniquely focuses on CLIP's performance with semantically equivalent but structurally varied captions rather than simply distinguishing conceptually different captions. This shift enables a deeper examination of the model’s grasp of language and visual content, where systematic errors reveal potential biases. Unlike prior works that primarily propose benchmarks or end-to-end solutions, we investigate the root causes of CLIP's behavior, delving into the mechanisms of both image and text encoders to uncover why the model displays biases and lacks robustness to certain linguistic and visual variations. To support this analysis, we introduce the \textbf{ComCO} dataset, purpose-built for examining CLIP's performance under {\it controlled} multi-object scenarios. Our study spans multiple versions of CLIP trained on diverse datasets and architectures, ensuring the broad applicability of our findings. This comprehensive approach aims to deepen our understanding of CLIP’s limitations and pave the way for more adaptable vision-language models. Beyond CLIP, our insights have significant implications for text-to-image (T2I) generative models and multimodal large language models (MLLMs), where decoding CLIP’s encoding intricacies can inform advancements in artificial intelligence across domains. As shown in Figure \ref{fig:mainfig}, our key contributions are as follows:



\begin{itemize} \item \textbf{Development of Novel Dataset}: We introduce \textit{ComCO}, a specialized dataset for creating {\it controlled} multi-object scenarios. Unlike previous benchmarks, ComCO allows control over object size and caption order, enabling precise analysis of model performance across compositional challenges and enhancing understanding of VLMs' strengths and weaknesses.

\item \textbf{Encoder Analysis}: We conduct an in-depth examination of CLIP’s image and text encoders in multi-object scenes, revealing weaknesses in preserving information for object distinction and identifying where compositional information is lost.

\item \textbf{Bias Identification}: Our study reveals that CLIP’s image encoder prefers larger objects, while the text encoder favors first-mentioned and visually larger objects, highlighting biases in CLIP's handling of visual and linguistic information.

\item \textbf{Investigation of Bias Origins}: We explore the origins of these biases, showing that larger objects are often mentioned earlier in CLIP’s training captions, and are favored in embeddings due to the abundance of their visual tokens. We substantiate this with analyses of the LAION dataset and CLIP’s training progression.

\item \textbf{Practical Impact}: We show how these biases affect performance in multi-object tasks, with significant drops in image-text matching accuracy in ComCO and COCO ~\cite{lin2015microsoftcococommonobjects}. These biases also extend to text-to-image models, influencing object prominence based on prompt order.

\end{itemize}



% \begin{itemize} \item \textbf{Development of Novel Dataset}: We introduce \textit{ComCO}, a specialized dataset specifically designed to create {\it controlled} multi-object scenarios. Here, unlike previous benchmarks, we can control the object size in the image, and their ordering in the caption. Hence, this dataset enables precise, fine-grained analysis of model performance across a spectrum of compositional challenges, facilitating a deeper understanding of VLMs' strengths and weaknesses.

%     \item \textbf{Comprehensive Encoder Analysis}: We perform an in-depth examination of both the image and text encoders in CLIP when processing multi-object scenes and descriptions. This includes text-based, and object-based image retrievals, that reveal each text and image encoder weaknesses in preserving the information necessary to discern various objects. By analyzing the embedding space, we identify the stages at which compositional information is lost or distorted, providing insights into the internal mechanisms of the model.
    
%     \item \textbf{Identification of Specific Biases}: Our research uncovers significant biases in CLIP models. The image encoder prefers larger objects in multi-object images, while the text encoder favors first-mentioned objects and also objects that are usually visually larger in real-world. These biases reveal the complex interplay between visual and linguistic information processing in CLIP, influencing its interpretation of multi-object scenarios.
    
%     \item \textbf{Investigation of the Bias Origin }: We explore the origins of observed biases in CLIP's performance, particularly in various multi-object scenarios. Our investigation delves into both the image and text encoders. We hypothesize that the visually larger objects are mostly mentioned earlier in the caption in CLIP training datasets. But it is evident that the image encoding naturally favors such objects in the embedding due to the abundance of their visual tokens. Therefore, the text encoder may get biased towards such objects, and consequently earlier mentioned text tokens. We provide evidence for these biases through analyses of the LAION dataset and CLIP's training progression, revealing a consistent trend where larger objects tend to be mentioned earlier in image captions.
    
%     \item \textbf{Practical impacts of encoder biases}: We demonstrate how the identified biases in CLIP's image and text encoders significantly impact performance in multi-object analysis/synthesis scenarios. Using our ComCO dataset, we show substantial drops in image-text matching accuracy when manipulating object sizes and caption order. We further reveal how these biases propagate to text-to-image generation models like Stable Diffusion, influencing the prominence and likelihood of object appearance in generated images based on prompt order.
    
%     \end{itemize}

These findings reveal how biases in CLIP’s text and image encoders significantly reduce its performance in multi-object scenarios, emphasizing the need to address these biases to enhance vision-language models' robustness. Our work offers key insights into CLIP's behavior and lays groundwork for improving model performance in real-world applications.






\section{Related Work}
\paragraph{Position Extrapolation.}
%Position extrapolation methods address longer contexts by scaling position embeddings (PE).  Early work primarily focused on achieving length extrapolation by modifying relative PEs during the pre-training phase. 
Following the introduction of RoPE \citep{su2024roformer}, great efforts have been imposed to extend the context length by modifying the position embeddings (PE).
Position interpolation  \cite{chen2023extending, kaiokendev} extends the context length by interpolating positional indices within the constraints of pre-training. 
The NTK-aware method \citep{bloc97ntk,Rozire2023CodeLO,Liu2023ScalingLO} introduces a nonlinear interpolation strategy by increasing the base parameter of RoPE. YaRN \citep{peng2023yarn} proposes a method for frequency interpolation of RoPE dimensions, where higher frequency dimensions are extrapolated, and lower frequency dimensions are interpolated. Further improvements \citep{chen2023clex,Ding2024LongRoPEEL} exploit the dynamics in position extrapolation.
Another group of work redesigns the relative position matrix to overcome the OOD issue \citep{JianlinSu,Jin2024LLMML,An2024TrainingFreeLS}. 
%CLEX generalizes PE scaling to model the transition of frequency basis continuously \citep{chen2023clex}.
These methods extend the context length but still compute the full attention matrix for inference thus fail to reduce the computational cost. To achieve better performance, some require fine-tuning with a certain amount of long-context data.
%ReRoPE proposes to maintain the position encodings within the window unchanged and truncate the position encodings that extend beyond the window \citep{JianlinSu}. 

\paragraph{Selective Attention.}
Selective attention mechanisms aim to mitigate the computational cost of processing long sequences by selecting only the most relevant tokens for attention computation. Approaches like Longformer \citep{Beltagy2020LongformerTL} and BigBird \citep{Zaheer2020BigBT} use fixed or adaptive sparse patterns, while \citealp{han2023lm, xiao2023efficient} introduce $\Lambda$-shaped windows that evict middle tokens. Although these methods lower costs, they often compromise global context understanding due to restricted attention over all tokens.
Some methods aim at compressing the KV cache, usually perform token selection only in the decoding stage \citep{zhang2024h2o,liu2024scissorhands,ge2023model} or permanently evicting certain tokens \citep{xiao2023efficient, han2023lm, li2024snapkv}. While effective, they may lose critical contextual information.
As for chunk-based methods, \citealp{xiao2024infllm} uses an efficient contextual memory mechanism to select the most relevant chunks for computing attention, and \citealp{lu2024longheads} selects the most relevant chunks for each head separately considering the variance among different heads.
Unlimiformer and its adaptation \citep{bertsch2024unlimiformer,ahrabian2024adaptation} segment input during the pre-filling stage, using external memory blocks, but remain computationally expensive and require additional training or architecture modifications.
In contrast, our method performs efficient token-level selection in both prefilling and decoding stages, without discarding tokens permanently.

%These methods reduce memory usage and computational complexity by limiting the size of the attention computation window, and mitigate the decline in accuracy by selectively focusing on the most critical tokens.
% % they are usually computationally expensive or require further training and modification on the architecture \citep{wu2022memorizing, bertsch2024unlimiformer}. 
% but may lose valuable contextual information. Compared to these methods, our proposed approach does not require chunking of the context, thus ensuring the flexibility of token selection and the integrity of contextual information.
\section{Method}
\begin{figure*}
    \centering
    \begin{subfigure}[b]{\textwidth}
    \centering
    % 图的宽高为16.76、4.47;页面宽高为33.88、19.05 left bottom right top
    \includegraphics[trim=0cm 14.5cm 17.1cm 0cm, clip, width=\textwidth]{ESAa.pdf}
    \vspace{-20pt}
    \caption{}
    \end{subfigure}

    % \label{fig:wide-image-a}
    % 图的宽高为9.17、4.41;页面宽高为33.88、19.05 left bottom right top
    \begin{subfigure}[b]{0.45\textwidth}
    \includegraphics[trim=0cm 14.6cm 24.7cm 0cm, clip, width=\textwidth]{ESAb.pdf}
    \caption{}
    \end{subfigure}
    \hfill
    \begin{subfigure}[b]{0.45\textwidth}
    % 图的宽高为8.78、4.3;页面宽高为33.88、19.05 left bottom right top
    \includegraphics[trim=0cm 14.6cm 24.7cm 0cm, clip, width=\textwidth]{ESAc.pdf}
    \caption{}
    \end{subfigure}
    \caption {(a) In long-context scenarios, the number of middle tokens occupies the majority, while the lengths of the other three parts of tokens are fixed. The importance scores between current tokens and middle tokens are utilized to select the top-k middle tokens. The selected tokens replace the middle tokens for computing attention. (b) The queries from current tokens and keys from middle tokens are compressed into smaller tensors through a linear layer respectively. The dot product of the compressed queries and keys serves as the importance scores. (c) The priority of a middle token being selected is determined by the maximum importance score among itself and several surrounding tokens.}
    \label{fig:illust}
\end{figure*}



% 为了高效长序列推理,提出xxx方式,对每个部分说明,我们的方法好处,motivation,然后放prefilling的chunk by chunk或者在3.1前面也可以
Conventionally, each token attends to all preceding tokens to compute the attention output during LLM inference, leading to significant computational overhead when processing long-context sequences. 
We introduce ESA, which effectively identifies a subset of the most crucial tokens for attention computation, thereby improving the efficiency of the model.
%The computational and memory costs of LLMs become prohibitive with the growing of context length. To mitigate this issue, 
%we exploit the sparsity nature in attention matrices \citep{zhang2024h2o, Jiang:2024, lu2024longheads, singhania2024loki}, and
%To mitigate this issue, our method selects a constant number of the most relevant preceding tokens with respect to the current tokens in each step for 
%, so that to reduce the computational complexity associated with inferencing on long-context sequences. (too long)

\subsection{Token Selection}
Following previous work \citep{xiao2023efficient, han2023lm, xiao2024infllm} on extending the context window of LLMs, the preceding tokens ($P$) are segmented into three parts, as depicted in Figure \ref{fig:illust}: initial tokens ($I$), middle tokens ($M$), and local tokens ($L$) with respective lengths denoted as $l_I$, $l_M$, and $l_L$. With a massive number of preceding tokens, $l_M$ is considerably larger than $l_I$ and $l_L$.
The initial tokens and local tokens with fixed length will always be selected due to their significance, as highlighted by \citet{xiao2023efficient} and \citet{han2023lm}. We then apply ESA to select the top-$k$ critical tokens $M_k$ from the middle tokens $M$.
Specifically, LLM inference consists of two stages: prefilling and decoding. In the prefilling stage, all the input tokens are encoded, and the keys as well as values are computed and cached. We adopt the chunked-prefill method \citep{agrawal2023sarathi}.
%, where tokens in the same chunk shares the selected tokens during the encoding of the input.
Compared to the original prefilling method, chunked-prefill approach effectively reduces memory usage. In the decoding stage, the model generates tokens one by one autoregressively. 
% 下面的删掉也没什么影响
Denote the \emph{current tokens} (i.e., the current input chunk being processed or the most recently decoded token) as $C$, the attention output computed in each step between the current tokens and the selected tokens is defined as:
\begin{equation} \label{attention_eq}
	\mathbf{O} = \text{Attn}(\mathbf{Q}_C,\mathbf{K}_{[I,M_k,L,C]}, \mathbf{V}_{[I,M_k,L,C]})
\end{equation}
%\begin{equation}
%	\mathbf{O} = \text{Attn}(\mathbf{Q}_C,[\mathbf{K}_I;\mathbf{K}_{M_k};\mathbf{K}_L,\mathbf{K}_C], [\mathbf{V}_I;\mathbf{V}_{M_k};\mathbf{V}_L;\mathbf{V}_C])
%\end{equation}
where, $\mathbf{Q}_C$ represents the queries of the current tokens, $\mathbf{K}_{[I,M_k,L,C]}$ and $\mathbf{V}_{[I,M_k,L,C]}$ respectively denote the concatenated keys and values of the selected and current tokens.


%During the prefilling and decoding stages, we compute the attention scores between the current tokens and the preceding tokens (including initial tokens, middle tokens, local tokens, and current tokens themselves). Without considering the positional encoding, the output of attention is:
%\begin{equation}
%O_{i} \triangleq Attn(q_{i, [C]}, k_{i,[I,M,L,C]}, v_{i,[I,M,L,C]}) \label{initial_attention}
%\end{equation} 
%where, $q_{i, [C]}$ represents $i$-th head of the queries of the current tokens, $k_{i, [I,M,L,C]}$ and $v_{i, [I,M,L,C]}$ denote $i$-th head of the concatenated keys and values (kvs) of the initial, middle, local, and current tokens, respectively.
%
%The quantity of $M$ constitutes a significant proportion in the context of long-sequence inputs. In this section, we focus on selecting top-k critical tokens ($M_k$) from key-value (KV) cache of $M$ for attention computation. We substitute $M$ in Equation~\ref{initial_attention} with $M_k$:
%\begin{equation} 
%O_{i} = Attn(q_{i, [C]}, k_{i,[I,M_k,L,C]}, v_{i,[I,M_k,L,C]}) \label{topk_attention_score}
%\end{equation}
\paragraph{Importance Score.} 
Successfully identifying the most crucial tokens requires a function to precisely measure the importance of each token in $M$ w.r.t $C$. Let $m \in M$ be a specific token, we denote the importance score of $m$ as $F_s(m; C)$.
Given a predefined number $k < =l_M$, $k$ tokens with the highest scores will be selected, as formalized bellow:
\begin{equation}
	M_k = \arg \text{topK}(M; F_s, C, k)
\end{equation}
For an individual token $c \in C$, the degree to which it attends to a preceding token $m$ can be determined as following:
\begin{equation}
	f_s(m; c) = \sum_{h=1}^{H}\mathbf{q}_{h,c}\mathbf{k}_{h,m}^{T}
	\label{dot-p}
\end{equation}
where, $\mathbf{q}_{h,c},\mathbf{k}_{h, m} \in \mathbb{R}^d$ denote the query of $c$ and the key of $m$, respectively, for the $h$-th head. All of the $H$ $d$-dimensional attention heads are incorporated, and the selected tokens are shared across all the heads in a certain layer. This score can be used directly in the decoding stage for token selection as $C$ consists of only one token (i.e., \( l_C = 1 \)). While at the prefilling stage, $l_C$ is the chunk size,
and every token in $C$ contributes to the score $F_s(m; C)$.
To derive a unified form of importance score, we first regularize the score w.r.t each $c$ and then take the maximum value across all tokens in $C$.
Eventually, the importance score is formulated as
\begin{equation}
	F_s(m; C) = \max_{c \in C} \left(f_s(m;c) - \max_{m^{\prime} \in M} f_s(m^{\prime};c)\right)
        \label{F_s_1}
\end{equation}
1. The expression \(f_s(m;c) - \max_{m^{\prime} \in M} f_s(m^{\prime};c)\) indicates that each token in \(C\) is constrained relative to the maximum score in \(M\), preventing any individual token in \(C\) from exerting a dominating influence on the selection process.
2. In the prefilling stage, our goal is to ensure that high-scoring tokens in \(M\) are not overlooked. Therefore, we select the highest score of each token in \(C\) to represent the score in \(M\) by applying \(\max_{c \in C}\).
% 1. prefilling阶段 \left(f_s(m;c) - \max_{m^{\prime} \in M} f_s(m^{\prime};c)\right) 表示每个C中的token相对于M中的最大分数是被限制的,避免C中的个别token对选择有统治的影响。
% 2. 我们选择tokens的目标是不漏掉分数高的tokens in M,因此选择C中每个token最高的分数作为M中的分数 by 应用\max_{c \in C}.

%Based on the preceding discussion, We define $q_{i, [C]}$ and $k_{i, [M]}$ using the following equations, respectively:
%\begin{align}  
%q_{i, [C]} &\triangleq \{q_{i, j} \mid j \in \mathbb{N}, l_{I,M,L} \le j < l_{I,M,L,C} \} \label{q_i_C_define} \\  
%k_{i, [M]} &\triangleq \{k_{i, j} \mid j \in \mathbb{N}, l_I \le j < l_{I,M}\}   \label{k_i_M_define}
%\end{align}
%where, $q_{i, j},k_{i, j} \in \mathbb{R}^d$ denote the query and key, respectively, for the $i$-th head of the $j$-th token. we denote $l_{I,M,L} \triangleq l_I + l_M + l_L$, $l_{I,M} \triangleq l_I + l_M$, $l_{I,M,L,C} \triangleq l_I + l_M + l_L + l_C$. $d$ represents the dimensionality of a single head. The importance score of the $j$-th token in $M$ being selected is presented in Equation~\ref{j_token-score}:
%\begin{align} 
%% s_max有512个取值,和q的长度一致
%\label{j_token-score}
% s_j &= \max\bigg( \Big\{ \big(\sum_{i=1}^{H}{q_{i,x} \cdot k_{i,j}^{T}}\big)-s_{x}^{max} \mid  x\in  \mathbb{N}, \nonumber\\ 
%     & l_{I,M,L} \le x < l_{I,M,L,C} \Big\}\bigg)  \\
%\label{s_max_define}
% s_{x}^{max} &\triangleq \max(\{\sum_{i=1}^{H}{q_{i,x} \cdot k_{i,y}^{T}} \mid y\in  \mathbb{N}, \nonumber\\ &l_I \le y < l_{I,M}\}) 
%\end{align}


%Equation~\ref{j_token-score} and~\ref{s_max_define} have the following two implications:
% 小维度
\paragraph{Efficient Selection through Query-Key Compression.} 
The aforementioned scoring method of employing dot product across a considerable number tokens is computational expensive. To achieve efficient selection, we perform dimensionality reduction on keys and queries.
The right-hand side of Equation~\ref{dot-p} is equivalent to concatenating $H$ heads followed by performing dot product. That is, $ f_s(m;c)=\mathbf{q}_{c}\mathbf{k}_{m}^{T}$ where $\mathbf{q}_{c} = [\mathbf{q}_{1,c};\mathbf{q}_{2,c};... ;\mathbf{q}_{H,c}]$, $\mathbf{k}_{m} = [\mathbf{k}_{1,m};\mathbf{k}_{2,m};... ;\mathbf{k}_{H,m}]$, and
$\mathbf{q}_{c},\mathbf{k}_{ m} \in \mathbb{R}^{d_{H}}$, $d_{H}=H \times d$.
%explain why not reduce dim on each head ...
Denote the dimensionality reduction on queries and keys as follows:
\begin{equation} \label{reduce_dim_formula}
\begin{aligned}
\mathbf{q}_{c}^{\prime} &\triangleq f_{\theta_q}(\mathbf{q}_{c}),\\
\mathbf{k}_{m}^{\prime} &\triangleq f_{\theta_k}(\mathbf{k}_{m}), 
\end{aligned}
\end{equation}
where, $\mathbf{q}_{c}^{\prime}, \mathbf{k}_{m}^{\prime} \in \mathbb{R}^{d^{\prime}}$, $d^{\prime} < d_{H}$. 
The dimension-reduced representation $\mathbf{k}_{m}^{\prime}$ will be cached and reused during the subsequent computation steps.
With the lower-dimensional surrogates of queries and keys, the importance score is approximated with
\begin{equation} \label{reduce_dim_dot}
    f_s(m; c) \approx \mathbf{q}^\prime_{c}\mathbf{k}^{\prime T}_{m}
\end{equation}
The computational cost is therefore reduced compared with using the full-dimensional queries and keys. 
To maintain accuracy, the lower-dimensional representations should retrain the token order as faithfully as possible.
To this end, we perform a one-time offline procedure to learn $f_{\theta_q}$ and $f_{\theta_k}$ in each layer by minimizing the discrepancy between the importance scores before and after dimensionality reduction, formally: 
\begin{equation} \label{train_theta}
\begin{aligned}
\min_{\theta_q, \theta_k}  \sum_{c \in C, m \in M} \left\lVert \mathbf{q}_{c} \mathbf{k}_{m}^{T} - f_{\theta_q}(\mathbf{q}_{c}) f_{\theta_k}(\mathbf{k}_{m})^{T} \right\rVert_{2}^{2}
\end{aligned}
\end{equation}
%[感觉需要一个algorithm][这里没有改]
We model $f_{\theta_q}$ and $f_{\theta_k}$ jointly, where each is a linear layer that projects a high-dimensional input to a low-dimensional output.
In preparation of the training data for the neural networks, we first perform token selection with full-dimensional queries and keys with a calibration dataset. All the queries and keys calculated during the process are saved.
Subsequently, we use the saved queries and keys to train \(f_{\theta_q}\) and \(f_{\theta_k}\) for each layer.
The learnt $f_{\theta_q}$ and $f_{\theta_k}$ will be utilized in our ESA to compress queries and keys with dimensionality reduction.
Since the low-dimensional keys are cached and reused, the additional computational load introduced by Equation~\ref{reduce_dim_formula} is marginal compared to the reduction achieved by Equation~\ref{reduce_dim_dot}. We conducted a quantitative analysis of efficiency in Section \ref{Complexity_Analysis}.
% todo Proximity Influence
\paragraph{Proximity Influence.}
Considering proximity influence, if a token in $M$ has a high importance score, it affects the scores of the surrounding tokens. We propose adaptive selection which is achieved by updating the importance score of a token based on its neighbors. Specifically, the score of the $j$-th token, where $j\in [l_I,l_I+l_M-1]$, is determined as the maximum score within a window of size $ \epsilon $ surrounding it. The importance score of the $j$-th token, computed using the low-dimensional query and key, is denoted by $ s_j $. The updated score is given by 
\begin{equation} \label{proximity_influence_eq}
    s_j^\prime = \max_{w=\max(j-\epsilon, l_I)}^{\min(j+\epsilon,l_I+l_M-1)}\{s_{w}\}
\end{equation}
where, $\epsilon$ represents the proximity influence distance, which is used to control the continuity of the selected tokens.
%$F_s(m;C)$ 
% 1.描述用一个单层的MLP来降维(采取calibration dataset),然后topk选择中间token再加上周围的token,具体选择token
% 连续语义信息
% infinite-topk 选择中间的几个;Unlimiformer:input分成了好多个段落,然后也是token-level选择,不够efficient
% 复用之前一些文章的结论,attention计算是很稀疏的,但是只能算出点乘以后才能看分数,我们提出一个cheap计算attention的function,只要排序,不要精度,具体实现用rank MLP,有个框图
% 2.不考虑head展开的32*128 --> 4096降维到128;prefill和decoding
% 3.连续语义:NLP语义信息不是单个token能表示完整的,需要保持连续性,邻距影响力下的adaptive,token-level的更小的block
% q*A*A^{T}*k

% \subsection{prefill和Decoding}

\subsection{Position Encoding}
We follow the extrapolated position encoding settings as described in \citep{su2023rerope, xiao2024infllm}.
The attention weights in Equation~\ref{attention_eq} can be divided into 
%long-distance attention and local attention.
% According to Equation~\ref{attention_eq}, we compute the attention weights of $C$ with respect to $I,M_k,L$, and itself. These weights can be divided into global weights and local weights.
(a) Long-distance attention of $C$ with respect to $I,M$: The positions of tokens in $C$ are all modified to a fixed position $w$, and the positions of tokens from $I,M$ are set to 0. The relative position will always be $w$;
and (b) Local attention of $C$ with respect to $L$ and itself: The query and key positions from $L,C$ are arranged in order of their relative positions (i.e., 0, 1, 2, ..., $l_L + l_C - 1$).
To normalize the attention weights, we fuse the two categories of attention based on their respective weights. 
The specific algorithm for computing attention is shown in Appendix~\ref{sec:pseudocode_attn}.
% Due to the distinct position encoding of tokens in $C$ within global attention and local attention, we compute the queries and transposed keys dot products separately for each. This is followed by a concatenation operation and softmax operation.
% 可以考虑增加一个flash attention实现的说明

\subsection{Complexity Analysis} \label{Complexity_Analysis}
% 短序列的情况下采取单卡或多卡加载,对于超长数据例如700k+采取off load,从CPU到GPU搬运,kv cache改为hidden states cache,通过增加线性的计算量,减小搬运量,减小topk时间
% 不讲搬运的问题
% 和下面的合一

\paragraph{Computational Complexity Analysis.}
%In the prefilling and decoding stages, we have $l_C = N$ and $l_C = 1$, respectively. In the prefilling stage, the input sequence is divided into blocks of length $N$, with each block selecting keys of length $k$ from $M$. In the decoding stage, the current individual token is treated as a single block. Considering the long sequence condition where $M$ occupies the majority of the tokens, the compression of complexity is as follows:

When inferring over a long-context sequence of total length $S$, using full-dimensional queries and keys for computing the importance score in Equation~\ref{dot-p} incurs a time complexity of $O(S^2d_H)$. By utilizing low-dimensional queries and keys, the computation is reduced to $O(S^2d^\prime)$. The additional computation for dimensionality reduction is $O(Sd_Hd^\prime+Sd^\prime)$, which scales linearly with context length and thus marginal to the quadratic savings.

In each step, ESA computes the attention matrix for a constant number of selected tokens. Considering the long sequence scenario where $M$ occupies the majority of the tokens, this approach significantly reduces computational costs, with only a minor overhead for token selection. Compared to vanilla full attention, the complexity of computing one step of attention can be reduced by ratio \( r \) in Equation~\ref{reduction_ratio_complexity}. The derivation can be found in Appendix~\ref{sec:Complexity_Analysis_Proof}.
\begin{equation} \label{reduction_ratio_complexity}
  r=\frac{2 d^{\prime} +1}{4d_{H} + 3H}
\end{equation} 
% The derivation of Equation~\ref{reduction_ratio_complexity} can be found in Appendix~\ref{sec:Complexity_Analysis_Proof}.
%%%%%%%%%%%%%%%%%%%%%%%%%%%
% Since we cache hidden states rather than kvs, we need to compute the following formula omitting considering GQA and the matrix operations for partitioning heads:
% \begin{align}
% q \cdot k^T &= (h_s \cdot W_q) \cdot (h_s \cdot W_k)^T \label{calculate_qk_analysis1} \\
% q \cdot k^T &= (h_s \cdot W_q \cdot W_k^T) \cdot h_s^T \label{calculate_qk_analysis2}
% \end{align}
% where, $h_s$ is hidden states, $W_q$ and $W_k$ represent the weights for computing the query and key, respectively. We compute attention using Equation~\ref{calculate_qk_analysis1} and~\ref{calculate_qk_analysis2} during the prefilling and decoding stages, respectively, which allows for a slight reduction in computational complexity. 
% The overall reduction ratio of complexity is as follows:
% \begin{equation} \label{reduction_ratio_complexity}
%   r=\frac{2 d^{\prime} +1}{4d_{H} + 3H}
% \end{equation} 
% The derivation process of Equation~\ref{reduction_ratio_complexity} can be found in Appendix~\ref{sec:Complexity_Analysis_Proof}

\paragraph{Cache Analysis.} 
We introduce an additional cache alongside the traditional KV cache, which stores the dimension-reduced keys from $M$. By incorporating a model that applies GQA \citep{ainslie2023gqa}, a widely used technique for reducing the KV cache, we analyze the impact of our approach on memory usage. Assuming the number of heads is denoted as \( H_G \) in GQA, the total dimensions of the kvs are given by \( d_{G} = H_G \times d \).
Given that $l_M \gg l_{I},l_{L},l_{C}$ for long sequences, we focus our analysis on the memory usage related to $M$. 
The cache size increased by the dimension-reduced keys is $\frac{d^{\prime}}{2 d_{G}}$ of the traditional KV cache.

\section{Experiments} \label{section_Experiments}
\subsection{Experimental Setup} 
\paragraph{Baselines and Benchmarks.} ESA can be applied to all decoder-only LLMs that utilize RoPE as their position encoding. We evaluate ESA using Mistral-7B-Instruct-v0.2 and Llama-3-8B-Instruct, with context lengths of 32k and 8k, respectively. We select the following three selective attention methods as our baselines: (a) InfLLM, (b) LM-Infinite (Infinite), (c) StreamingLLM (Stream). Additionally, we also choose two methods with position extrapolation: (a) NTK-aware (NTK), (b) YaRN. We conduct extensive evaluations on LongBench, $\infty$BENCH, NeedleBench, and Counting-Stars.

%Mistral-7B-Instruct-v0.2 (\texttt{Mistral}) \citep{jiang2023mistral} and Llama-3-8B-Instruct (\texttt{Llama}) 
\paragraph{Calibration Dataset.}
We employ a small subset of Books3 data from Pile \citep{gao2020pile} as our calibration dataset to train the networks $f_{\theta_q}$ and $f_{\theta_k}$. There are 50k tokens in total and therefore 50k concatenated query-key pairs for training the networks in each layer.
%The sample length for generating the calibration dataset is 30k, and we ultimately collected approximately 50k samples. 
The learning rate and global batch size is $0.0005$ and 128, respectively. We trained the dimensionality reduction networks for 10 epochs.

\paragraph{Parameters.}
% The model parameters for Mistral and Llama are as follows: 
The number of attention heads ($H$) is 32.
We compress the original size of query and key from \( d_H = 4096 \) to \( d' = 128 \).
% The complexity for computing the importance scores is reduced to 3.13\% of the original. 
Since the number of GQA heads is 8, the additional size required for the reduced-dimensionality keys is 6.25\% of the original KV cache.
Compared to computing full attention, the computational complexity is reduced to up to 1.56\% in each step according to Equation~\ref{reduction_ratio_complexity}.
ESA and three other baselines with selective attention select the same number of tokens. The length of initial tokens (\( l_I \)) is 128. InfLLM and ESA both choose the lengths of middle tokens and local tokens to be 2048 and 4096, respectively.
%by our method is kept consistent with those of InfLLM, Infinite, and Stream.
\subsection{Results on LongBench}
LongBench includes six different categories of tasks, with an average context length range from less than 10k to around 32k. We adjust the scaling factors of NTK and YaRN to accommodate the benchmark. The context length for Mistral is 32k, which does not necessitate any modification to the model's scaling factor. Consequently, we omit the NTK and YaRN for Mistral in this section. 
The results of the 16 English tasks are presented in Table~\ref{tab:res_longbench}. We draw the following conclusions: 
(a) Our method achieves improvement over the best baselines of selective attention (including Infinite, Stream, InfLLM) for both Llama and Mistral across a variety of tasks. Particularly, our method outperforms other methods of selective attention on the PassageRetrieval-en significantly, demonstrating the benefit of token-level selection.
(b) Our method is comparable to the baselines that compute full attention (including Origin, NTK, YaRN). The gap between our method and the best among these approaches is within 1 percentage point.
% \input{res-table-longbench-t}
% Please add the following required packages to your document preamble:
% \usepackage{booktabs}
% \usepackage{multirow}
% \usepackage{graphicx}
% \usepackage[normalem]{ulem}
% \useunder{\uline}{\ul}{}
\begin{table*}[]
\centering
\resizebox{\textwidth}{!}{%
\begin{tabular}{@{}l|ccccccc|ccccc@{}}
\toprule
\multicolumn{1}{c|}{\multirow{2}{*}{Task}} & \multicolumn{7}{c|}{\textbf{llama-3-8B-Instruct}}                                                                                                  & \multicolumn{5}{c}{\textbf{Mistral-7B-Instruct-v0.2}}                                                                  \\
\multicolumn{1}{c|}{}                      & Origin      & NTK(32k)    & \multicolumn{1}{c|}{YaRN(32k)}   & Infinite             & Stream               & InfLLM         & Ours                 & \multicolumn{1}{c|}{Origin}      & Infinite             & Stream         & InfLLM               & Ours                 \\ \midrule
NarrativaQA                                & 2.91        & 10.16       & \multicolumn{1}{c|}{13.14}       & 18.64                & 19.12                & 19.77          & {\ul \textbf{24.89}}       & \multicolumn{1}{c|}{20.14}       & 19.81                & 18.41          & {\ul \textbf{22.91}} & 22.72                \\
Qasper                                     & 41.44       & {\ul 44.87} & \multicolumn{1}{c|}{41.5}        & 42.35                & 42.47                & 43.51          & \textbf{43.59}       & \multicolumn{1}{c|}{29.37}       & {\ul \textbf{29.78}} & 29.74          & 28.75                & 28.65                \\
MultiFieldQA-en                            & 31.78       & {\ul 52.3}  & \multicolumn{1}{c|}{51.35}       & 38.07                & 38.41                & 44.27          & \textbf{48.23}       & \multicolumn{1}{c|}{{\ul 47.94}} & 39.03                & 38.99          & 47.54                & \textbf{47.91}       \\
MuSiQue                                    & 0.91        & 27.08       & \multicolumn{1}{c|}{{\ul 28.86}} & 19.74                & 19.89                & 22.58          & \textbf{24.23}       & \multicolumn{1}{c|}{18.56}       & 15.82                & 14.62          & 18.91                & {\ul \textbf{19.82}} \\
HotpotQA                                   & 8.24        & {\ul 53.05} & \multicolumn{1}{c|}{51.48}       & 45.4                 & 45.41                & 46.96          & \textbf{49.39}       & \multicolumn{1}{c|}{37.63}       & 31.94                & 31.57          & 36.27                & {\ul \textbf{40.06}} \\
2WikiMultihopQA                            & 30.96       & 37.51       & \multicolumn{1}{c|}{35.85}       & {\ul \textbf{39.17}} & 37.31                & 36.23          & 37.87                & \multicolumn{1}{c|}{21.61}       & 22.62                & 21.81          & 21.93                & {\ul \textbf{23.15}} \\
GovReport                                  & 18.83       & 32.48       & \multicolumn{1}{c|}{{\ul 34.22}} & 29.77                & 29.82                & \textbf{31.01} & 30.89                & \multicolumn{1}{c|}{{\ul 31.72}} & 29.52                & 29.46          & 30.97                & \textbf{31.31}       \\
QMSum                                      & 9.19        & 22.53       & \multicolumn{1}{c|}{{\ul 23.41}} & 20.92                & 20.85                & 21.37          & \textbf{22.49}       & \multicolumn{1}{c|}{{\ul 23.93}} & 21.67                & 21.77          & 23.52                & \textbf{23.79}       \\
MultiNews                                  & 26.96       & 27.46       & \multicolumn{1}{c|}{27.07}       & {\ul \textbf{27.48}} & {\ul \textbf{27.48}} & 27.33          & 27.46                & \multicolumn{1}{c|}{26.56}       & 26.32                & 26.3           & {\ul \textbf{26.63}} & 26.57                \\
TREC                                       & 52          & {\ul 75}    & \multicolumn{1}{c|}{74}          & 73                   & 73                   & 73             & \textbf{74}          & \multicolumn{1}{c|}{{\ul 71}}    & 70                   & \textbf{70.5}  & 70                   & 70                   \\
TriviaQA                                   & 30.3        & 79.39       & \multicolumn{1}{c|}{90.54}       & 90.18                & 90.34                & 90.75          & {\ul \textbf{90.91}} & \multicolumn{1}{c|}{85.81}       & 85.42                & 85.6           & 86.83                & {\ul \textbf{87.62}} \\
SAMSum                                     & 20.55       & 42.36       & \multicolumn{1}{c|}{{\ul 43.44}} & 42.12                & 42.3                 & \textbf{42.39} & 41.99                & \multicolumn{1}{c|}{{\ul 42.65}} & 41.49                & 41.69          & 42.3                 & \textbf{42.56}       \\
PassageRetrieval-en                        & 2.08        & {\ul 100}   & \multicolumn{1}{c|}{97.5}        & 39.5                 & 41                   & 70             & \textbf{86.5}        & \multicolumn{1}{c|}{{\ul 87.6}}  & 43.38                & 42.33          & 63.58                & \textbf{84.31}       \\
PassageCount                               & 2.86        & 6           & \multicolumn{1}{c|}{5.22}        & {\ul \textbf{8}}     & 7                    & 7.67           & 7.67                 & \multicolumn{1}{c|}{{\ul 3.27}}  & 1.93                 & 2.17           & 2.84                 & \textbf{3.03}        \\
LCC                                        & {\ul 59.37} & 35.43       & \multicolumn{1}{c|}{53.79}       & 58.63                & 58.94                & \textbf{59.34} & 58.58                & \multicolumn{1}{c|}{{\ul 57.15}} & 55.03                & \textbf{55.04} & 54.81                & \textbf{55.04}       \\
RepoBench-P                                & 33.92       & 33.77       & \multicolumn{1}{c|}{{\ul 53.48}} & 40.82                & 41.61                & \textbf{43.62} & 41.8                 & \multicolumn{1}{c|}{{\ul 54.55}} & 51.73                & 51.14          & 51.52                & \textbf{52.56}       \\ \midrule
Average                                    & 23.27       & 42.46       & \multicolumn{1}{c|}{{\ul 45.3}}  & 39.61                & 39.68                & 42.49          & \textbf{44.41}       & \multicolumn{1}{c|}{{\ul 41.22}} & 36.59                & 36.32          & 39.33                & \textbf{41.19}       \\ \bottomrule
\end{tabular}%
}
\caption{Results (\%) on 16 English tasks of LongBench. The term ``Origin'' indicates the original model baselines without any extrapolation methods. We \uline{underline} the best score of all methods for a model on a particular task and \textbf{bold} the best score of the selective attention methods, and this holds for the tables below.}
\label{tab:res_longbench}
\end{table*}
\subsection{Results on $\infty$BENCH} \label{results_infinitebench}
We select 6 tasks from $\infty$BENCH with an average length up to around 200k, encompassing open-form question answering (QA), code, retrieval tasks, and other domains. We set the scaling factor for NTK and YaRN to accommodate contexts of up to 200k. The results of the tasks are presented in Table~\ref{tab:res_infinitebench}. 
Firstly, our method slightly outperforms the scores of InfLLM.
The performance of our method exhibits minimal differences in retrieval tasks compared to InfLLM.
This may be due to the fact that retrieval tasks only require the model to focus on a single relevant piece of information.
InfLLM retrieves the most relevant chunks from the context, which is sufficient for solving long-text tasks that require attention to only a small amount of local information.
Our method, on the other hand, opts for retrieval at a finer granularity, resulting in performance that is close to that of InfLLM on such tasks. In other tasks, our method outperforms other selective attention methods. 
Secondly, our method outperforms NTK and YaRN, especially on Llama. This superiority may arise from the fact that methods with position embedding scaling tend to suffer a significant decline when extrapolated to excessively long contexts, such as a 8-fold extension. It demonstrates that our approach can extrapolate to significantly longer contexts, even up to a $\times 25$ extension for Llama.

%%%%%%%%%%%%%%%% infiniteBench 2 col
% Please add the following required packages to your document preamble:
% \usepackage{booktabs}
% \usepackage{multirow}
% \usepackage{graphicx}
% \usepackage[normalem]{ulem}
% \useunder{\uline}{\ul}{}
\begin{table}[]
\centering
\resizebox{\columnwidth}{!}{%
\begin{tabular}{@{}lccccccc@{}}
\toprule
\multicolumn{1}{c|}{\multirow{2}{*}{Task}} & \multicolumn{7}{c}{Method}                                                                                                \\ \cmidrule(l){2-8} 
\multicolumn{1}{c|}{}                      & Origin & NTK(200k)   & \multicolumn{1}{c|}{YaRN(200k)}  & Infinite & Stream & InfLLM               & Ours                 \\ \midrule
\multicolumn{8}{c}{\textbf{Llama-3-8B-Instruct}}                                                                                                                       \\ \midrule
\multicolumn{1}{l|}{Retrieve.KV}           & 0      & 0           & \multicolumn{1}{c|}{0}           & 1.8      & 1.8    & {\ul \textbf{4.8}}   & 3.4                  \\
\multicolumn{1}{l|}{Math.Find}             & 0      & 4.86        & \multicolumn{1}{c|}{{\ul 34.86}} & 14       & 14     & 14.86                & \textbf{16.57}       \\
\multicolumn{1}{l|}{Retrieve.Number}       & 0      & 0           & \multicolumn{1}{c|}{18.64}       & 6.44     & 6.61   & {\ul \textbf{99.66}} & 99.32                \\
\multicolumn{1}{l|}{En.MC}                 & 0      & 0           & \multicolumn{1}{c|}{19.65}       & 42.79    & 44.98  & 43.23                & {\ul \textbf{47.16}} \\
\multicolumn{1}{l|}{Code.Debug}            & 22.59  & 22.59       & \multicolumn{1}{c|}{11.17}       & 29.95    & 30.96  & 30.46                & {\ul \textbf{32.74}} \\
\multicolumn{1}{l|}{Retrieve.PassKey}      & 0      & 0           & \multicolumn{1}{c|}{51.02}       & 6.61     & 6.78   & {\ul \textbf{100}}   & {\ul \textbf{100}}   \\ \midrule
\multicolumn{1}{l|}{Average}               & 3.77   & 4.58        & \multicolumn{1}{c|}{22.56}       & 16.93    & 17.52  & 48.84                & {\ul \textbf{49.87}} \\ \midrule
\multicolumn{8}{c}{\textbf{Mistral-7B-Instruct-v0.2}}                                                                                                                  \\ \midrule
\multicolumn{1}{l|}{Retrieve.KV}           & 0      & 9.6         & \multicolumn{1}{c|}{27.8}        & 3.4      & 3.4    & {\ul \textbf{93.4}}  & 91.6                 \\
\multicolumn{1}{l|}{Math.Find}             & 27.43  & {\ul 29.43} & \multicolumn{1}{c|}{25.14}       & 14       & 14.29  & \textbf{26.86}       & \textbf{26.86}       \\
\multicolumn{1}{l|}{Retrieve.Number}       & 27.29  & 90.34       & \multicolumn{1}{c|}{96.95}       & 6.78     & 6.78   & {\ul \textbf{99.83}} & {\ul \textbf{99.83}} \\
\multicolumn{1}{l|}{En.MC}                 & 10.4   & 18.78       & \multicolumn{1}{c|}{40.61}       & 38.43    & 37.99  & 43.23                & {\ul \textbf{48.03}} \\
\multicolumn{1}{l|}{Code.Debug}            & 15.51  & 20.81       & \multicolumn{1}{c|}{21.83}       & 22.84    & 22.59  & 27.41                & {\ul \textbf{34.01}} \\
\multicolumn{1}{l|}{Retrieve.PassKey}      & 75.25  & {\ul 100}         & \multicolumn{1}{c|}{{\ul 100}}         & 6.78     & 6.78   & {\ul \textbf{100}}   & {\ul \textbf{100}}   \\ \midrule
\multicolumn{1}{l|}{Average}               & 36.31  & 44.83       & \multicolumn{1}{c|}{52.06}       & 15.37    & 15.31  & 65.12                & {\ul \textbf{66.72}} \\ \bottomrule
\end{tabular}%
}
\caption{Performace evaluation (\%) on 6 English tasks of $\infty$BENCH.}
\label{tab:res_infinitebench}
\end{table}


\subsection{Results on NeedleBench and Counting-Stars}
NeedleBench and Counting-Stars evaluate the ability of LLMs to retrieve multiple pieces of related information. The two benchmarks places higher demands on the model's capacity to handle long context. Sample examples from the benchmarks are provided in Appendix~\ref{sec:NeedleBench and Counting-Stars}. The context length for these two benchmarks ranges from 4k to 256k, assessing the model's capability to retrieve multiple pieces of information across varying lengths. We uniformly set the scaling factor for NTK and YaRN to accommodate contexts of up to 200k tokens. We follow \citep{li2024needlebench, song2024counting} to use the recall accuracy as a metric to evaluate the performance of the models.


% needle说明每条样本有3个针,且位置随机 
Our method exhibits great strength in extracting critical information distributed across the context.
The experimental results on Counting-Stars and NeedleBench are shown in Table~\ref{tab:res_count_stars} and ~\ref{tab:res_needle}, respectively. Details of the Counting-Stars are provided in Appendix~\ref{sec:Counting-Stars-Results}. 
Firstly, when multiple pieces of relevant information need to be retrieved, our method significantly outperforms Infinite, Stream, and InfLLM. 
This is attributed to our method's flexibility in selecting middle tokens at the token level. 
Secondly, the performance of ESA is comparable to that of NTK and YaRN. NTK and YaRN achieve promising performance by computing full attention when extrapolation is limited.
When extrapolated to tasks involving longer sequences, NTK and YaRN may experience performance degradation.
Lastly, within the original training lengths, ESA does not exhibit significant performance degradation, whereas the NTK and YaRN show a noticeable decline.
% 需要增加分析,对应的组合起来的图
%%%%%%%% Counting stars
% Please add the following required packages to your document preamble:
% \usepackage{booktabs}
% \usepackage{multirow}
% \usepackage{graphicx}
% \usepackage[normalem]{ulem}
% \useunder{\uline}{\ul}{}
\begin{table}[]
\centering
\resizebox{\columnwidth}{!}{%
\begin{tabular}{@{}lccccccc@{}}
\toprule
\multicolumn{1}{c|}{\multirow{2}{*}{Task}} & \multicolumn{7}{c}{Method}                                                                              \\ \cmidrule(l){2-8} 
\multicolumn{1}{c|}{}                      & Origin & NTK(200k) & \multicolumn{1}{c|}{YaRN(200k)} & Infinite & Stream & InfLLM & Ours                \\ \midrule
\multicolumn{8}{c}{\textbf{Llama-3-8B-Instruct}}                                                                                                     \\ \midrule
\multicolumn{1}{l|}{(128k, 32, 32)}        & 5      & 28.5      & \multicolumn{1}{c|}{8.4}        & 15.8     & 15.7   & 24.2   & {\ul \textbf{39.3}} \\
\multicolumn{1}{l|}{(256k, 8, 32)}         & 2.7    & 16        & \multicolumn{1}{c|}{5.9}        & 5.1      & 5.1    & 10.9   & {\ul \textbf{63.7}} \\
\multicolumn{1}{l|}{(256k, 16, 32)}        & 2      & 16        & \multicolumn{1}{c|}{8.2}        & 11.7     & 11.1   & 14.1   & {\ul \textbf{44.3}} \\
\multicolumn{1}{l|}{(256k, 32, 32)}        & 1.7    & 15.7      & \multicolumn{1}{c|}{7.7}        & 9.6      & 9.6    & 16.2   & {\ul \textbf{36.4}} \\ \midrule
\multicolumn{1}{l|}{Average}               & 2.9    & 19.1      & \multicolumn{1}{c|}{7.6}        & 10.6     & 10.4   & 16.4   & {\ul \textbf{45.9}} \\ \midrule
\multicolumn{8}{c}{\textbf{Mistral-7B-Instruct-v0.2}}                                                                                                \\ \midrule
\multicolumn{1}{l|}{(128k, 32, 32)}        & 15     & 35.5      & \multicolumn{1}{c|}{{\ul 52.6}} & 10.7     & 10.2   & 16.1   & \textbf{26.6}       \\
\multicolumn{1}{l|}{(256k, 8, 32)}         & 5.5    & 6.6       & \multicolumn{1}{c|}{{\ul 46.5}} & 5.9      & 7      & 2.7    & \textbf{25.8}       \\
\multicolumn{1}{l|}{(256k, 16, 32)}        & 6.2    & 15.2      & \multicolumn{1}{c|}{{\ul 47.3}} & 9        & 8.6    & 12.7   & \textbf{16.8}       \\
\multicolumn{1}{l|}{(256k, 32, 32)}        & 9.8    & 16.1      & \multicolumn{1}{c|}{{\ul 41.7}} & 10.9     & 10.9   & 13.3   & \textbf{23.2}       \\ \midrule
\multicolumn{1}{l|}{Average}               & 9.1    & 18.4      & \multicolumn{1}{c|}{{\ul 47}}   & 9.1      & 9.2    & 11.2   & \textbf{23.1}       \\ \bottomrule
\end{tabular}%
}
\caption{Recall accuracy (\%) evaluation on the Counting-Stars benchmark. We employ the notation (256k, 8, 32) to represent the following benchmark setup: the context length is 256k, and within this length, we generate 32 samples at equal intervals (e.g., the sample lengths are 8k, 16k, 32k, ..., up to 256k), with each sample containing 8 pieces of relevant information. Accuracy is defined as the average score across the 32 samples within each task.}
\label{tab:res_count_stars}
\end{table}
%%%%%%%%%%%%%% multi needle
% Please add the following required packages to your document preamble:
% \usepackage{booktabs}
% \usepackage{multirow}
% \usepackage{graphicx}
% \usepackage[normalem]{ulem}
% \useunder{\uline}{\ul}{}
\begin{table*}[]
\centering
\resizebox{\textwidth}{!}{%
\begin{tabular}{@{}l|ccccccc|ccccccc@{}}
\toprule
\multirow{2}{*}{\begin{tabular}[c]{@{}l@{}}Context\\ Length\end{tabular}} & \multicolumn{7}{c|}{\textbf{Llama-3-8B-Instruct}}                                                                                                       & \multicolumn{7}{c}{\textbf{Mistral-7B-Instruct-v0.2}}                                                                                                 \\
                                                                          & Origin      & NTK(200k)   & \multicolumn{1}{c|}{YaRN(200k)} & Infinite             & Stream               & InfLLM               & Ours                 & Origin      & NTK(200k)   & \multicolumn{1}{c|}{YaRN(200k)}     & Infinite           & Stream             & InfLLM             & Ours                 \\ \midrule
4k                                                                        & {\ul 96.67} & 90          & \multicolumn{1}{c|}{70}         & {\ul \textbf{96.67}} & {\ul \textbf{96.67}} & {\ul \textbf{96.67}} & {\ul \textbf{96.67}} & 96.67       & 76.67       & \multicolumn{1}{c|}{60}             & {\ul \textbf{100}} & {\ul \textbf{100}} & {\ul \textbf{100}} & {\ul \textbf{100}}   \\
8k                                                                        & {\ul 100}   & {\ul 100}   & \multicolumn{1}{c|}{76.67}      & 73.33                & 73.33                & 86.67                & {\ul \textbf{100}}   & 80          & 73.33       & \multicolumn{1}{c|}{43.33}          & 50                 & 53.33              & 63.33              & {\ul \textbf{83.33}} \\
16k                                                                       & 0           & {\ul 96.67} & \multicolumn{1}{c|}{43.33}      & 20                   & 20                   & 23.33                & \textbf{86.67}       & 73.33       & {\ul 76.67} & \multicolumn{1}{c|}{66.67}          & 13.33              & 6.67               & 10                 & \textbf{73.33}       \\
48k                                                                       & 0           & 36.67       & \multicolumn{1}{c|}{0}          & 3.33                 & 3.33                 & 3.33                 & {\ul \textbf{53.33}} & {\ul 83.33} & 50          & \multicolumn{1}{c|}{53.33}          & 0                  & 0                  & 0                  & \textbf{70}          \\
80k                                                                       & 0           & 0           & \multicolumn{1}{c|}{0}          & 0                    & 0                    & 0                    & {\ul \textbf{36.67}} & 6.67        & 63.33       & \multicolumn{1}{c|}{{\ul 66.67}}    & 0                  & 0                  & 0                  & {\ul \textbf{66.67}} \\
112k                                                                      & 0           & 0           & \multicolumn{1}{c|}{0}          & 0                    & 0                    & 0                    & {\ul \textbf{43.33}} & 0           & {\ul 73.33} & \multicolumn{1}{c|}{70}             & 0                  & 0                  & 0                  & \textbf{63.33}       \\
128k                                                                      & 0           & 0           & \multicolumn{1}{c|}{0}          & 0                    & 0                    & 0                    & {\ul \textbf{46.67}} & 0           & {\ul 73.33} & \multicolumn{1}{c|}{63.33}          & 0                  & 0                  & 0                  & \textbf{60}          \\
144k                                                                      & 0           & 0           & \multicolumn{1}{c|}{0}          & 0                    & 0                    & 0                    & {\ul \textbf{20}}    & 0           & 43.33       & \multicolumn{1}{c|}{{\ul 66.67}}    & 0                  & 0                  & 0                  & \textbf{63.33}       \\
176k                                                                      & 0           & 0           & \multicolumn{1}{c|}{0}          & 0                    & 0                    & 0                    & {\ul \textbf{23.33}} & 0           & 0           & \multicolumn{1}{c|}{{\ul 70}}       & 0                  & 0                  & 0                  & \textbf{53.33}       \\
200k                                                                      & 0           & 0           & \multicolumn{1}{c|}{0}          & 0                    & 0                    & 0                    & {\ul \textbf{23.33}} & 0           & 0           & \multicolumn{1}{c|}{{\ul 66.67}}    & 0                  & 0                  & 0                  & \textbf{46.67}       \\ \midrule
Average                                                                   & 19.67       & 32.33       & \multicolumn{1}{c|}{19}         & 19.33                & 19.33                & 21                   & {\ul \textbf{53}}    & 34          & 53          & \multicolumn{1}{c|}{\textbf{62.67}} & 16.33              & 16                 & 17.33              & {\ul \textbf{68}}    \\ \bottomrule
\end{tabular}%
}
  \caption{Recall accuracy (\%) evaluation on NeedleBench. 
  % When the context length approaches 200k, only our method is capable of retrieving partial relevant information, significantly outperforming the other methods.
  } 
  % 除了我们的方法,其他的在接近200k的时候不行
  \label{tab:res_needle}
\end{table*}


\subsection{Ablation Study}

\paragraph{Effectiveness of the proximity influence distance \( \epsilon \).} The parameter \( \epsilon \) in Equation~\ref{proximity_influence_eq} controls the continuity of the selected tokens. As demonstrated in Table~\ref{tab:proximity_influence_distance_ablation}, we find this parameter to be crucial for the model, especially with regard to its retrieval capabilities. Furthermore, we observe that in Retrieve.KV, when \( \epsilon = 0,1 \), even when the model's predictions are incorrect, it is still able to retrieve parts of the correct values. For instance, the answer is "49c65968-6319-44fc-b286-feb249694b07", while the model's prediction is "49c65968-6319-44fc-\textcolor{red}{9021-cfa198896071}". 
Retrieve.KV and NeedleBench exhibit different optimal values for \( \epsilon \). 
We speculate that the underlying reason may be the difference in the number of positions where answers are to be retrieved.
In Retrieve.KV, there is typically only one segment that requires retrieval, and increasing \( \epsilon \) may enhance the completeness of the retrieved answer.
In contrast, NeedleBench involves the retrieval of answers from multiple positions. Increasing \( \epsilon \) might lead to an over-concentration of attention on a limited number of positions. 

\begin{table}[]
\centering
\resizebox{0.85\columnwidth}{!}{%
\begin{tabular}{@{}l|cccc@{}}
\toprule
\multicolumn{1}{c|}{\multirow{2}{*}{Task}} & \multicolumn{4}{c}{$\epsilon$}                \\
\multicolumn{1}{c|}{}                      & 0     & 1              & 3    & 5             \\ \midrule
Retrieve.KV                                & 66.6  & 82             & 91.6 & \textbf{95.6} \\
NeedleBench                                & 69.67 & \textbf{71.33} & 68   & 58.67         \\ \bottomrule
\end{tabular}%
}
\caption{The ablation study results of \( \epsilon \) on InfiniteBench's Retrieve.KV and NeedleBench with Mistral. NeedleBench in the table represents the average scores across lengths ranging from 4k to 200k, with the specific scores detailed in Appendix~\ref{appen_proxi_influ}.} 
  % 除了我们的方法,其他的在接近200k的时候不行
  \label{tab:proximity_influence_distance_ablation}
\end{table}
% Please add the following required packages to your document preamble:
% \usepackage{booktabs}
% \usepackage{multirow}
% \usepackage{graphicx}
% \begin{table}[]
%   \begin{tabular}{l|cccc}
%   \hline
%   \multicolumn{1}{c|}{\multirow{2}{*}{Task}} & \multicolumn{4}{c}{$\epsilon$} \\
%   \multicolumn{1}{c|}{}                      & 0      & 1     & 3     & 5     \\ \hline
%   Retrieve.KV                                & 66.6   & 82  & 91.6  & \textbf{95.6}  \\
%   NeedleBench                        & 69.67   & \textbf{71.33}  & 68  & 58.67  \\ \hline
%   \end{tabular}
%   \caption{The ablation study results of \( \epsilon \) on InfiniteBench's Retrieve.KV and NeedleBench with Mistral. NeedleBench in the table represents the average scores across lengths ranging from 4k to 200k, with the specific scores detailed in Appendix~\ref{appen_proxi_influ}.} 
%   % 除了我们的方法,其他的在接近200k的时候不行
%   \label{tab:proximity_influence_distance_ablation}
%   \end{table}
  
% 最好将longbench替换为needle

\paragraph{Uniform Token Selection for All Heads.}
Our method does not select tokens individually for each head but rather chooses tokens based on the average importance scores across all heads. This approach is beneficial for compressing the size of queries and keys, thereby enhancing inference efficiency. To verify whether there is a significant performance degradation, we design two experiments with Llama on LongBench as shown in Table~\ref{tab:uniform_individual_token_selection}. "Individual" refers to the importance score of each token being the maximum value among the scores of all heads, meaning that each head votes for the scores. This approach ensures that the selection process takes into account all heads. "Uniform" in Table~\ref{tab:uniform_individual_token_selection} denotes our method of selecting tokens without dimensionality reduction. The scores for each subtask of LongBench are depicted in Appendix~\ref{sec:Token_Selection_for_Heads}. We extrapolate Llama from its original 8k to an average length of 32k on LongBench, and the performance on various category tasks for both token selection methods is very close. 
% It demonstrates that our method of averaging the scores from different heads and then uniformly selecting tokens has a minimal impact on the model's performance.

% Please add the following required packages to your document preamble:
% \usepackage{booktabs}
% \usepackage{graphicx}
\begin{table}[]
\centering
\resizebox{0.85\columnwidth}{!}{%
\begin{tabular}{@{}l|cc@{}}
\toprule
\multicolumn{1}{c|}{Task} & individual & uniform \\ \midrule
LongBench scores          & 44.7       & 44.8    \\ \bottomrule
\end{tabular}%
}
\caption{We employ Llama to validate different token selection strategies for heads. The LongBench scores represent the average scores across 16 subtasks in LongBench.} 
  % 除了我们的方法,其他的在接近200k的时候不行
  \label{tab:uniform_individual_token_selection}
\end{table}
%%%%%%%%%%%%%%
% \begin{table}[]
%   \begin{tabular}{l|cc}
%     \hline
%     \multicolumn{1}{c|}{Task} & individual & uniform \\ \hline
%     LongBench scores          & 44.7       & 44.8    \\ \hline
%     \end{tabular}
%   \caption{We employ Llama to validate different token selection strategies for heads. The LongBench scores represent the average scores across 16 subtasks in LongBench.} 
%   % 除了我们的方法,其他的在接近200k的时候不行
%   \label{tab:uniform_individual_token_selection}
%   \end{table}


\paragraph{Dimension Reduction of Queries and Keys.}
We calculate the importance scores of tokens using the reduced-dimensionality queries and keys. To evaluate the impact of dimensionality reduction, we analyse experiments on LongBench with Llama using the full-dimensional query and key, as well as their reduced-dimensionality counterparts. As demonstrated in Table~\ref{tab:uniform_individual_token_selection} and Table~\ref{tab:res_longbench}, their respective scores are 44.8 and 44.41. The difference between the two scores is only 0.39.
Furthermore, we select samples from Books3 in Pile and employ Mistral to validate the recall rate of the top-k retrieval subsequent to dimensionality reduction. 
The ground truth is determined using the top-k tokens derived from the full-dimensional query and key. A total of 2,000 tokens are selected for this analysis, spanning positions from 23,000 to 25,000. 
In parallel, we execute comparative experiments utilizing principle component analysis (PCA) for dimensionality reduction inspired by \citep{singhania2024loki}. 
The experimental results are depicted in Figure~\ref{fig:compare_pca_mlp}. It can be observed that our dimensionality reduction method achieves a recall rate of over 90\% for the majority of the model's layers, whereas PCA exhibits a recall rate below 90\% in approximately half of the layers. 
% It demonstrates that our method is capable of selecting the vast majority of the most important tokens.
\begin{figure}[t]
  \includegraphics[width=\columnwidth]{pca_compared.pdf}
  \caption{Recall rates of each layer for selecting the top 2,000 tokens after dimensionality reduction.}
  \label{fig:compare_pca_mlp}
\end{figure}

% 周围token的选择+和PCA的对比+每个head单独选token+Infinite-topk

\section{Conclusions and Future work}
In this paper, we propose an efficient token-level selective attention method for extending the context length of LLMs without incremental training of LLMs parameters.
The insight is to select a constant number of the most important tokens at each step for computing attention at the token level, leveraging the sparsity of the attention matrix. 
When the input sequence is sufficiently long, we are able to reduce the computational complexity to up to nearly $1.56\%$ of the original by compressing the queries and keys into low-dimensional representations. 
Our empirical evaluations demonstrate that ESA can effectively handle sequences of lengths up to $4 \times$ and even $25 \times$ the training length for various types of long sequence tasks. 
%Future work can explore how to more accurately and efficiently select important tokens in extrapolatory tasks. 
Future work could explore more accurate and efficient methods for selecting important tokens in extrapolation tasks.
% Additionally, we will leverage the characteristics of RoPE to explore better position encoding for application in selective attention methods.
\section{Limitations}
Our method has the following limitations: 1. We apply the same compression ratio to the queries and keys across different layers; employing varying compression ratios for different layers may yield better performance. 2. There may exist more effective token selection methods and compression techniques that warrant further exploration. 3. A more suitable positional encoding for our approach may exist and requires further investigation.
% \clearpage
\bibliography{custom}

\appendix

% 1.没有尝试中文等其他语言,需要进一步验证多语言的可能性
% 2.不同层可能需要不同的压缩比例,而不是统一设置为固定压缩比例
% 3.在prefill的时候是分block的,虽然FLOPS有所降低,但是prefill开始不能充分利用显存,后续工程优化需要在prefill的时候batch size逐渐减小
% 4.目前还没有和flash attention、VLLM等加速推理方法结合,需要利用以后的加速推理工具进一步提升
% 5.可能存在更好的选择token的方式,无论效率还是准确度
% Bibliography entries for the entire Anthology, followed by custom entries
%\bibliography{anthology,custom}
% Custom bibliography entries only
% \subsection{Lloyd-Max Algorithm}
\label{subsec:Lloyd-Max}
For a given quantization bitwidth $B$ and an operand $\bm{X}$, the Lloyd-Max algorithm finds $2^B$ quantization levels $\{\hat{x}_i\}_{i=1}^{2^B}$ such that quantizing $\bm{X}$ by rounding each scalar in $\bm{X}$ to the nearest quantization level minimizes the quantization MSE. 

The algorithm starts with an initial guess of quantization levels and then iteratively computes quantization thresholds $\{\tau_i\}_{i=1}^{2^B-1}$ and updates quantization levels $\{\hat{x}_i\}_{i=1}^{2^B}$. Specifically, at iteration $n$, thresholds are set to the midpoints of the previous iteration's levels:
\begin{align*}
    \tau_i^{(n)}=\frac{\hat{x}_i^{(n-1)}+\hat{x}_{i+1}^{(n-1)}}2 \text{ for } i=1\ldots 2^B-1
\end{align*}
Subsequently, the quantization levels are re-computed as conditional means of the data regions defined by the new thresholds:
\begin{align*}
    \hat{x}_i^{(n)}=\mathbb{E}\left[ \bm{X} \big| \bm{X}\in [\tau_{i-1}^{(n)},\tau_i^{(n)}] \right] \text{ for } i=1\ldots 2^B
\end{align*}
where to satisfy boundary conditions we have $\tau_0=-\infty$ and $\tau_{2^B}=\infty$. The algorithm iterates the above steps until convergence.

Figure \ref{fig:lm_quant} compares the quantization levels of a $7$-bit floating point (E3M3) quantizer (left) to a $7$-bit Lloyd-Max quantizer (right) when quantizing a layer of weights from the GPT3-126M model at a per-tensor granularity. As shown, the Lloyd-Max quantizer achieves substantially lower quantization MSE. Further, Table \ref{tab:FP7_vs_LM7} shows the superior perplexity achieved by Lloyd-Max quantizers for bitwidths of $7$, $6$ and $5$. The difference between the quantizers is clear at 5 bits, where per-tensor FP quantization incurs a drastic and unacceptable increase in perplexity, while Lloyd-Max quantization incurs a much smaller increase. Nevertheless, we note that even the optimal Lloyd-Max quantizer incurs a notable ($\sim 1.5$) increase in perplexity due to the coarse granularity of quantization. 

\begin{figure}[h]
  \centering
  \includegraphics[width=0.7\linewidth]{sections/figures/LM7_FP7.pdf}
  \caption{\small Quantization levels and the corresponding quantization MSE of Floating Point (left) vs Lloyd-Max (right) Quantizers for a layer of weights in the GPT3-126M model.}
  \label{fig:lm_quant}
\end{figure}

\begin{table}[h]\scriptsize
\begin{center}
\caption{\label{tab:FP7_vs_LM7} \small Comparing perplexity (lower is better) achieved by floating point quantizers and Lloyd-Max quantizers on a GPT3-126M model for the Wikitext-103 dataset.}
\begin{tabular}{c|cc|c}
\hline
 \multirow{2}{*}{\textbf{Bitwidth}} & \multicolumn{2}{|c|}{\textbf{Floating-Point Quantizer}} & \textbf{Lloyd-Max Quantizer} \\
 & Best Format & Wikitext-103 Perplexity & Wikitext-103 Perplexity \\
\hline
7 & E3M3 & 18.32 & 18.27 \\
6 & E3M2 & 19.07 & 18.51 \\
5 & E4M0 & 43.89 & 19.71 \\
\hline
\end{tabular}
\end{center}
\end{table}

\subsection{Proof of Local Optimality of LO-BCQ}
\label{subsec:lobcq_opt_proof}
For a given block $\bm{b}_j$, the quantization MSE during LO-BCQ can be empirically evaluated as $\frac{1}{L_b}\lVert \bm{b}_j- \bm{\hat{b}}_j\rVert^2_2$ where $\bm{\hat{b}}_j$ is computed from equation (\ref{eq:clustered_quantization_definition}) as $C_{f(\bm{b}_j)}(\bm{b}_j)$. Further, for a given block cluster $\mathcal{B}_i$, we compute the quantization MSE as $\frac{1}{|\mathcal{B}_{i}|}\sum_{\bm{b} \in \mathcal{B}_{i}} \frac{1}{L_b}\lVert \bm{b}- C_i^{(n)}(\bm{b})\rVert^2_2$. Therefore, at the end of iteration $n$, we evaluate the overall quantization MSE $J^{(n)}$ for a given operand $\bm{X}$ composed of $N_c$ block clusters as:
\begin{align*}
    \label{eq:mse_iter_n}
    J^{(n)} = \frac{1}{N_c} \sum_{i=1}^{N_c} \frac{1}{|\mathcal{B}_{i}^{(n)}|}\sum_{\bm{v} \in \mathcal{B}_{i}^{(n)}} \frac{1}{L_b}\lVert \bm{b}- B_i^{(n)}(\bm{b})\rVert^2_2
\end{align*}

At the end of iteration $n$, the codebooks are updated from $\mathcal{C}^{(n-1)}$ to $\mathcal{C}^{(n)}$. However, the mapping of a given vector $\bm{b}_j$ to quantizers $\mathcal{C}^{(n)}$ remains as  $f^{(n)}(\bm{b}_j)$. At the next iteration, during the vector clustering step, $f^{(n+1)}(\bm{b}_j)$ finds new mapping of $\bm{b}_j$ to updated codebooks $\mathcal{C}^{(n)}$ such that the quantization MSE over the candidate codebooks is minimized. Therefore, we obtain the following result for $\bm{b}_j$:
\begin{align*}
\frac{1}{L_b}\lVert \bm{b}_j - C_{f^{(n+1)}(\bm{b}_j)}^{(n)}(\bm{b}_j)\rVert^2_2 \le \frac{1}{L_b}\lVert \bm{b}_j - C_{f^{(n)}(\bm{b}_j)}^{(n)}(\bm{b}_j)\rVert^2_2
\end{align*}

That is, quantizing $\bm{b}_j$ at the end of the block clustering step of iteration $n+1$ results in lower quantization MSE compared to quantizing at the end of iteration $n$. Since this is true for all $\bm{b} \in \bm{X}$, we assert the following:
\begin{equation}
\begin{split}
\label{eq:mse_ineq_1}
    \tilde{J}^{(n+1)} &= \frac{1}{N_c} \sum_{i=1}^{N_c} \frac{1}{|\mathcal{B}_{i}^{(n+1)}|}\sum_{\bm{b} \in \mathcal{B}_{i}^{(n+1)}} \frac{1}{L_b}\lVert \bm{b} - C_i^{(n)}(b)\rVert^2_2 \le J^{(n)}
\end{split}
\end{equation}
where $\tilde{J}^{(n+1)}$ is the the quantization MSE after the vector clustering step at iteration $n+1$.

Next, during the codebook update step (\ref{eq:quantizers_update}) at iteration $n+1$, the per-cluster codebooks $\mathcal{C}^{(n)}$ are updated to $\mathcal{C}^{(n+1)}$ by invoking the Lloyd-Max algorithm \citep{Lloyd}. We know that for any given value distribution, the Lloyd-Max algorithm minimizes the quantization MSE. Therefore, for a given vector cluster $\mathcal{B}_i$ we obtain the following result:

\begin{equation}
    \frac{1}{|\mathcal{B}_{i}^{(n+1)}|}\sum_{\bm{b} \in \mathcal{B}_{i}^{(n+1)}} \frac{1}{L_b}\lVert \bm{b}- C_i^{(n+1)}(\bm{b})\rVert^2_2 \le \frac{1}{|\mathcal{B}_{i}^{(n+1)}|}\sum_{\bm{b} \in \mathcal{B}_{i}^{(n+1)}} \frac{1}{L_b}\lVert \bm{b}- C_i^{(n)}(\bm{b})\rVert^2_2
\end{equation}

The above equation states that quantizing the given block cluster $\mathcal{B}_i$ after updating the associated codebook from $C_i^{(n)}$ to $C_i^{(n+1)}$ results in lower quantization MSE. Since this is true for all the block clusters, we derive the following result: 
\begin{equation}
\begin{split}
\label{eq:mse_ineq_2}
     J^{(n+1)} &= \frac{1}{N_c} \sum_{i=1}^{N_c} \frac{1}{|\mathcal{B}_{i}^{(n+1)}|}\sum_{\bm{b} \in \mathcal{B}_{i}^{(n+1)}} \frac{1}{L_b}\lVert \bm{b}- C_i^{(n+1)}(\bm{b})\rVert^2_2  \le \tilde{J}^{(n+1)}   
\end{split}
\end{equation}

Following (\ref{eq:mse_ineq_1}) and (\ref{eq:mse_ineq_2}), we find that the quantization MSE is non-increasing for each iteration, that is, $J^{(1)} \ge J^{(2)} \ge J^{(3)} \ge \ldots \ge J^{(M)}$ where $M$ is the maximum number of iterations. 
%Therefore, we can say that if the algorithm converges, then it must be that it has converged to a local minimum. 
\hfill $\blacksquare$


\begin{figure}
    \begin{center}
    \includegraphics[width=0.5\textwidth]{sections//figures/mse_vs_iter.pdf}
    \end{center}
    \caption{\small NMSE vs iterations during LO-BCQ compared to other block quantization proposals}
    \label{fig:nmse_vs_iter}
\end{figure}

Figure \ref{fig:nmse_vs_iter} shows the empirical convergence of LO-BCQ across several block lengths and number of codebooks. Also, the MSE achieved by LO-BCQ is compared to baselines such as MXFP and VSQ. As shown, LO-BCQ converges to a lower MSE than the baselines. Further, we achieve better convergence for larger number of codebooks ($N_c$) and for a smaller block length ($L_b$), both of which increase the bitwidth of BCQ (see Eq \ref{eq:bitwidth_bcq}).


\subsection{Additional Accuracy Results}
%Table \ref{tab:lobcq_config} lists the various LOBCQ configurations and their corresponding bitwidths.
\begin{table}
\setlength{\tabcolsep}{4.75pt}
\begin{center}
\caption{\label{tab:lobcq_config} Various LO-BCQ configurations and their bitwidths.}
\begin{tabular}{|c||c|c|c|c||c|c||c|} 
\hline
 & \multicolumn{4}{|c||}{$L_b=8$} & \multicolumn{2}{|c||}{$L_b=4$} & $L_b=2$ \\
 \hline
 \backslashbox{$L_A$\kern-1em}{\kern-1em$N_c$} & 2 & 4 & 8 & 16 & 2 & 4 & 2 \\
 \hline
 64 & 4.25 & 4.375 & 4.5 & 4.625 & 4.375 & 4.625 & 4.625\\
 \hline
 32 & 4.375 & 4.5 & 4.625& 4.75 & 4.5 & 4.75 & 4.75 \\
 \hline
 16 & 4.625 & 4.75& 4.875 & 5 & 4.75 & 5 & 5 \\
 \hline
\end{tabular}
\end{center}
\end{table}

%\subsection{Perplexity achieved by various LO-BCQ configurations on Wikitext-103 dataset}

\begin{table} \centering
\begin{tabular}{|c||c|c|c|c||c|c||c|} 
\hline
 $L_b \rightarrow$& \multicolumn{4}{c||}{8} & \multicolumn{2}{c||}{4} & 2\\
 \hline
 \backslashbox{$L_A$\kern-1em}{\kern-1em$N_c$} & 2 & 4 & 8 & 16 & 2 & 4 & 2  \\
 %$N_c \rightarrow$ & 2 & 4 & 8 & 16 & 2 & 4 & 2 \\
 \hline
 \hline
 \multicolumn{8}{c}{GPT3-1.3B (FP32 PPL = 9.98)} \\ 
 \hline
 \hline
 64 & 10.40 & 10.23 & 10.17 & 10.15 &  10.28 & 10.18 & 10.19 \\
 \hline
 32 & 10.25 & 10.20 & 10.15 & 10.12 &  10.23 & 10.17 & 10.17 \\
 \hline
 16 & 10.22 & 10.16 & 10.10 & 10.09 &  10.21 & 10.14 & 10.16 \\
 \hline
  \hline
 \multicolumn{8}{c}{GPT3-8B (FP32 PPL = 7.38)} \\ 
 \hline
 \hline
 64 & 7.61 & 7.52 & 7.48 &  7.47 &  7.55 &  7.49 & 7.50 \\
 \hline
 32 & 7.52 & 7.50 & 7.46 &  7.45 &  7.52 &  7.48 & 7.48  \\
 \hline
 16 & 7.51 & 7.48 & 7.44 &  7.44 &  7.51 &  7.49 & 7.47  \\
 \hline
\end{tabular}
\caption{\label{tab:ppl_gpt3_abalation} Wikitext-103 perplexity across GPT3-1.3B and 8B models.}
\end{table}

\begin{table} \centering
\begin{tabular}{|c||c|c|c|c||} 
\hline
 $L_b \rightarrow$& \multicolumn{4}{c||}{8}\\
 \hline
 \backslashbox{$L_A$\kern-1em}{\kern-1em$N_c$} & 2 & 4 & 8 & 16 \\
 %$N_c \rightarrow$ & 2 & 4 & 8 & 16 & 2 & 4 & 2 \\
 \hline
 \hline
 \multicolumn{5}{|c|}{Llama2-7B (FP32 PPL = 5.06)} \\ 
 \hline
 \hline
 64 & 5.31 & 5.26 & 5.19 & 5.18  \\
 \hline
 32 & 5.23 & 5.25 & 5.18 & 5.15  \\
 \hline
 16 & 5.23 & 5.19 & 5.16 & 5.14  \\
 \hline
 \multicolumn{5}{|c|}{Nemotron4-15B (FP32 PPL = 5.87)} \\ 
 \hline
 \hline
 64  & 6.3 & 6.20 & 6.13 & 6.08  \\
 \hline
 32  & 6.24 & 6.12 & 6.07 & 6.03  \\
 \hline
 16  & 6.12 & 6.14 & 6.04 & 6.02  \\
 \hline
 \multicolumn{5}{|c|}{Nemotron4-340B (FP32 PPL = 3.48)} \\ 
 \hline
 \hline
 64 & 3.67 & 3.62 & 3.60 & 3.59 \\
 \hline
 32 & 3.63 & 3.61 & 3.59 & 3.56 \\
 \hline
 16 & 3.61 & 3.58 & 3.57 & 3.55 \\
 \hline
\end{tabular}
\caption{\label{tab:ppl_llama7B_nemo15B} Wikitext-103 perplexity compared to FP32 baseline in Llama2-7B and Nemotron4-15B, 340B models}
\end{table}

%\subsection{Perplexity achieved by various LO-BCQ configurations on MMLU dataset}


\begin{table} \centering
\begin{tabular}{|c||c|c|c|c||c|c|c|c|} 
\hline
 $L_b \rightarrow$& \multicolumn{4}{c||}{8} & \multicolumn{4}{c||}{8}\\
 \hline
 \backslashbox{$L_A$\kern-1em}{\kern-1em$N_c$} & 2 & 4 & 8 & 16 & 2 & 4 & 8 & 16  \\
 %$N_c \rightarrow$ & 2 & 4 & 8 & 16 & 2 & 4 & 2 \\
 \hline
 \hline
 \multicolumn{5}{|c|}{Llama2-7B (FP32 Accuracy = 45.8\%)} & \multicolumn{4}{|c|}{Llama2-70B (FP32 Accuracy = 69.12\%)} \\ 
 \hline
 \hline
 64 & 43.9 & 43.4 & 43.9 & 44.9 & 68.07 & 68.27 & 68.17 & 68.75 \\
 \hline
 32 & 44.5 & 43.8 & 44.9 & 44.5 & 68.37 & 68.51 & 68.35 & 68.27  \\
 \hline
 16 & 43.9 & 42.7 & 44.9 & 45 & 68.12 & 68.77 & 68.31 & 68.59  \\
 \hline
 \hline
 \multicolumn{5}{|c|}{GPT3-22B (FP32 Accuracy = 38.75\%)} & \multicolumn{4}{|c|}{Nemotron4-15B (FP32 Accuracy = 64.3\%)} \\ 
 \hline
 \hline
 64 & 36.71 & 38.85 & 38.13 & 38.92 & 63.17 & 62.36 & 63.72 & 64.09 \\
 \hline
 32 & 37.95 & 38.69 & 39.45 & 38.34 & 64.05 & 62.30 & 63.8 & 64.33  \\
 \hline
 16 & 38.88 & 38.80 & 38.31 & 38.92 & 63.22 & 63.51 & 63.93 & 64.43  \\
 \hline
\end{tabular}
\caption{\label{tab:mmlu_abalation} Accuracy on MMLU dataset across GPT3-22B, Llama2-7B, 70B and Nemotron4-15B models.}
\end{table}


%\subsection{Perplexity achieved by various LO-BCQ configurations on LM evaluation harness}

\begin{table} \centering
\begin{tabular}{|c||c|c|c|c||c|c|c|c|} 
\hline
 $L_b \rightarrow$& \multicolumn{4}{c||}{8} & \multicolumn{4}{c||}{8}\\
 \hline
 \backslashbox{$L_A$\kern-1em}{\kern-1em$N_c$} & 2 & 4 & 8 & 16 & 2 & 4 & 8 & 16  \\
 %$N_c \rightarrow$ & 2 & 4 & 8 & 16 & 2 & 4 & 2 \\
 \hline
 \hline
 \multicolumn{5}{|c|}{Race (FP32 Accuracy = 37.51\%)} & \multicolumn{4}{|c|}{Boolq (FP32 Accuracy = 64.62\%)} \\ 
 \hline
 \hline
 64 & 36.94 & 37.13 & 36.27 & 37.13 & 63.73 & 62.26 & 63.49 & 63.36 \\
 \hline
 32 & 37.03 & 36.36 & 36.08 & 37.03 & 62.54 & 63.51 & 63.49 & 63.55  \\
 \hline
 16 & 37.03 & 37.03 & 36.46 & 37.03 & 61.1 & 63.79 & 63.58 & 63.33  \\
 \hline
 \hline
 \multicolumn{5}{|c|}{Winogrande (FP32 Accuracy = 58.01\%)} & \multicolumn{4}{|c|}{Piqa (FP32 Accuracy = 74.21\%)} \\ 
 \hline
 \hline
 64 & 58.17 & 57.22 & 57.85 & 58.33 & 73.01 & 73.07 & 73.07 & 72.80 \\
 \hline
 32 & 59.12 & 58.09 & 57.85 & 58.41 & 73.01 & 73.94 & 72.74 & 73.18  \\
 \hline
 16 & 57.93 & 58.88 & 57.93 & 58.56 & 73.94 & 72.80 & 73.01 & 73.94  \\
 \hline
\end{tabular}
\caption{\label{tab:mmlu_abalation} Accuracy on LM evaluation harness tasks on GPT3-1.3B model.}
\end{table}

\begin{table} \centering
\begin{tabular}{|c||c|c|c|c||c|c|c|c|} 
\hline
 $L_b \rightarrow$& \multicolumn{4}{c||}{8} & \multicolumn{4}{c||}{8}\\
 \hline
 \backslashbox{$L_A$\kern-1em}{\kern-1em$N_c$} & 2 & 4 & 8 & 16 & 2 & 4 & 8 & 16  \\
 %$N_c \rightarrow$ & 2 & 4 & 8 & 16 & 2 & 4 & 2 \\
 \hline
 \hline
 \multicolumn{5}{|c|}{Race (FP32 Accuracy = 41.34\%)} & \multicolumn{4}{|c|}{Boolq (FP32 Accuracy = 68.32\%)} \\ 
 \hline
 \hline
 64 & 40.48 & 40.10 & 39.43 & 39.90 & 69.20 & 68.41 & 69.45 & 68.56 \\
 \hline
 32 & 39.52 & 39.52 & 40.77 & 39.62 & 68.32 & 67.43 & 68.17 & 69.30  \\
 \hline
 16 & 39.81 & 39.71 & 39.90 & 40.38 & 68.10 & 66.33 & 69.51 & 69.42  \\
 \hline
 \hline
 \multicolumn{5}{|c|}{Winogrande (FP32 Accuracy = 67.88\%)} & \multicolumn{4}{|c|}{Piqa (FP32 Accuracy = 78.78\%)} \\ 
 \hline
 \hline
 64 & 66.85 & 66.61 & 67.72 & 67.88 & 77.31 & 77.42 & 77.75 & 77.64 \\
 \hline
 32 & 67.25 & 67.72 & 67.72 & 67.00 & 77.31 & 77.04 & 77.80 & 77.37  \\
 \hline
 16 & 68.11 & 68.90 & 67.88 & 67.48 & 77.37 & 78.13 & 78.13 & 77.69  \\
 \hline
\end{tabular}
\caption{\label{tab:mmlu_abalation} Accuracy on LM evaluation harness tasks on GPT3-8B model.}
\end{table}

\begin{table} \centering
\begin{tabular}{|c||c|c|c|c||c|c|c|c|} 
\hline
 $L_b \rightarrow$& \multicolumn{4}{c||}{8} & \multicolumn{4}{c||}{8}\\
 \hline
 \backslashbox{$L_A$\kern-1em}{\kern-1em$N_c$} & 2 & 4 & 8 & 16 & 2 & 4 & 8 & 16  \\
 %$N_c \rightarrow$ & 2 & 4 & 8 & 16 & 2 & 4 & 2 \\
 \hline
 \hline
 \multicolumn{5}{|c|}{Race (FP32 Accuracy = 40.67\%)} & \multicolumn{4}{|c|}{Boolq (FP32 Accuracy = 76.54\%)} \\ 
 \hline
 \hline
 64 & 40.48 & 40.10 & 39.43 & 39.90 & 75.41 & 75.11 & 77.09 & 75.66 \\
 \hline
 32 & 39.52 & 39.52 & 40.77 & 39.62 & 76.02 & 76.02 & 75.96 & 75.35  \\
 \hline
 16 & 39.81 & 39.71 & 39.90 & 40.38 & 75.05 & 73.82 & 75.72 & 76.09  \\
 \hline
 \hline
 \multicolumn{5}{|c|}{Winogrande (FP32 Accuracy = 70.64\%)} & \multicolumn{4}{|c|}{Piqa (FP32 Accuracy = 79.16\%)} \\ 
 \hline
 \hline
 64 & 69.14 & 70.17 & 70.17 & 70.56 & 78.24 & 79.00 & 78.62 & 78.73 \\
 \hline
 32 & 70.96 & 69.69 & 71.27 & 69.30 & 78.56 & 79.49 & 79.16 & 78.89  \\
 \hline
 16 & 71.03 & 69.53 & 69.69 & 70.40 & 78.13 & 79.16 & 79.00 & 79.00  \\
 \hline
\end{tabular}
\caption{\label{tab:mmlu_abalation} Accuracy on LM evaluation harness tasks on GPT3-22B model.}
\end{table}

\begin{table} \centering
\begin{tabular}{|c||c|c|c|c||c|c|c|c|} 
\hline
 $L_b \rightarrow$& \multicolumn{4}{c||}{8} & \multicolumn{4}{c||}{8}\\
 \hline
 \backslashbox{$L_A$\kern-1em}{\kern-1em$N_c$} & 2 & 4 & 8 & 16 & 2 & 4 & 8 & 16  \\
 %$N_c \rightarrow$ & 2 & 4 & 8 & 16 & 2 & 4 & 2 \\
 \hline
 \hline
 \multicolumn{5}{|c|}{Race (FP32 Accuracy = 44.4\%)} & \multicolumn{4}{|c|}{Boolq (FP32 Accuracy = 79.29\%)} \\ 
 \hline
 \hline
 64 & 42.49 & 42.51 & 42.58 & 43.45 & 77.58 & 77.37 & 77.43 & 78.1 \\
 \hline
 32 & 43.35 & 42.49 & 43.64 & 43.73 & 77.86 & 75.32 & 77.28 & 77.86  \\
 \hline
 16 & 44.21 & 44.21 & 43.64 & 42.97 & 78.65 & 77 & 76.94 & 77.98  \\
 \hline
 \hline
 \multicolumn{5}{|c|}{Winogrande (FP32 Accuracy = 69.38\%)} & \multicolumn{4}{|c|}{Piqa (FP32 Accuracy = 78.07\%)} \\ 
 \hline
 \hline
 64 & 68.9 & 68.43 & 69.77 & 68.19 & 77.09 & 76.82 & 77.09 & 77.86 \\
 \hline
 32 & 69.38 & 68.51 & 68.82 & 68.90 & 78.07 & 76.71 & 78.07 & 77.86  \\
 \hline
 16 & 69.53 & 67.09 & 69.38 & 68.90 & 77.37 & 77.8 & 77.91 & 77.69  \\
 \hline
\end{tabular}
\caption{\label{tab:mmlu_abalation} Accuracy on LM evaluation harness tasks on Llama2-7B model.}
\end{table}

\begin{table} \centering
\begin{tabular}{|c||c|c|c|c||c|c|c|c|} 
\hline
 $L_b \rightarrow$& \multicolumn{4}{c||}{8} & \multicolumn{4}{c||}{8}\\
 \hline
 \backslashbox{$L_A$\kern-1em}{\kern-1em$N_c$} & 2 & 4 & 8 & 16 & 2 & 4 & 8 & 16  \\
 %$N_c \rightarrow$ & 2 & 4 & 8 & 16 & 2 & 4 & 2 \\
 \hline
 \hline
 \multicolumn{5}{|c|}{Race (FP32 Accuracy = 48.8\%)} & \multicolumn{4}{|c|}{Boolq (FP32 Accuracy = 85.23\%)} \\ 
 \hline
 \hline
 64 & 49.00 & 49.00 & 49.28 & 48.71 & 82.82 & 84.28 & 84.03 & 84.25 \\
 \hline
 32 & 49.57 & 48.52 & 48.33 & 49.28 & 83.85 & 84.46 & 84.31 & 84.93  \\
 \hline
 16 & 49.85 & 49.09 & 49.28 & 48.99 & 85.11 & 84.46 & 84.61 & 83.94  \\
 \hline
 \hline
 \multicolumn{5}{|c|}{Winogrande (FP32 Accuracy = 79.95\%)} & \multicolumn{4}{|c|}{Piqa (FP32 Accuracy = 81.56\%)} \\ 
 \hline
 \hline
 64 & 78.77 & 78.45 & 78.37 & 79.16 & 81.45 & 80.69 & 81.45 & 81.5 \\
 \hline
 32 & 78.45 & 79.01 & 78.69 & 80.66 & 81.56 & 80.58 & 81.18 & 81.34  \\
 \hline
 16 & 79.95 & 79.56 & 79.79 & 79.72 & 81.28 & 81.66 & 81.28 & 80.96  \\
 \hline
\end{tabular}
\caption{\label{tab:mmlu_abalation} Accuracy on LM evaluation harness tasks on Llama2-70B model.}
\end{table}

%\section{MSE Studies}
%\textcolor{red}{TODO}


\subsection{Number Formats and Quantization Method}
\label{subsec:numFormats_quantMethod}
\subsubsection{Integer Format}
An $n$-bit signed integer (INT) is typically represented with a 2s-complement format \citep{yao2022zeroquant,xiao2023smoothquant,dai2021vsq}, where the most significant bit denotes the sign.

\subsubsection{Floating Point Format}
An $n$-bit signed floating point (FP) number $x$ comprises of a 1-bit sign ($x_{\mathrm{sign}}$), $B_m$-bit mantissa ($x_{\mathrm{mant}}$) and $B_e$-bit exponent ($x_{\mathrm{exp}}$) such that $B_m+B_e=n-1$. The associated constant exponent bias ($E_{\mathrm{bias}}$) is computed as $(2^{{B_e}-1}-1)$. We denote this format as $E_{B_e}M_{B_m}$.  

\subsubsection{Quantization Scheme}
\label{subsec:quant_method}
A quantization scheme dictates how a given unquantized tensor is converted to its quantized representation. We consider FP formats for the purpose of illustration. Given an unquantized tensor $\bm{X}$ and an FP format $E_{B_e}M_{B_m}$, we first, we compute the quantization scale factor $s_X$ that maps the maximum absolute value of $\bm{X}$ to the maximum quantization level of the $E_{B_e}M_{B_m}$ format as follows:
\begin{align}
\label{eq:sf}
    s_X = \frac{\mathrm{max}(|\bm{X}|)}{\mathrm{max}(E_{B_e}M_{B_m})}
\end{align}
In the above equation, $|\cdot|$ denotes the absolute value function.

Next, we scale $\bm{X}$ by $s_X$ and quantize it to $\hat{\bm{X}}$ by rounding it to the nearest quantization level of $E_{B_e}M_{B_m}$ as:

\begin{align}
\label{eq:tensor_quant}
    \hat{\bm{X}} = \text{round-to-nearest}\left(\frac{\bm{X}}{s_X}, E_{B_e}M_{B_m}\right)
\end{align}

We perform dynamic max-scaled quantization \citep{wu2020integer}, where the scale factor $s$ for activations is dynamically computed during runtime.

\subsection{Vector Scaled Quantization}
\begin{wrapfigure}{r}{0.35\linewidth}
  \centering
  \includegraphics[width=\linewidth]{sections/figures/vsquant.jpg}
  \caption{\small Vectorwise decomposition for per-vector scaled quantization (VSQ \citep{dai2021vsq}).}
  \label{fig:vsquant}
\end{wrapfigure}
During VSQ \citep{dai2021vsq}, the operand tensors are decomposed into 1D vectors in a hardware friendly manner as shown in Figure \ref{fig:vsquant}. Since the decomposed tensors are used as operands in matrix multiplications during inference, it is beneficial to perform this decomposition along the reduction dimension of the multiplication. The vectorwise quantization is performed similar to tensorwise quantization described in Equations \ref{eq:sf} and \ref{eq:tensor_quant}, where a scale factor $s_v$ is required for each vector $\bm{v}$ that maps the maximum absolute value of that vector to the maximum quantization level. While smaller vector lengths can lead to larger accuracy gains, the associated memory and computational overheads due to the per-vector scale factors increases. To alleviate these overheads, VSQ \citep{dai2021vsq} proposed a second level quantization of the per-vector scale factors to unsigned integers, while MX \citep{rouhani2023shared} quantizes them to integer powers of 2 (denoted as $2^{INT}$).

\subsubsection{MX Format}
The MX format proposed in \citep{rouhani2023microscaling} introduces the concept of sub-block shifting. For every two scalar elements of $b$-bits each, there is a shared exponent bit. The value of this exponent bit is determined through an empirical analysis that targets minimizing quantization MSE. We note that the FP format $E_{1}M_{b}$ is strictly better than MX from an accuracy perspective since it allocates a dedicated exponent bit to each scalar as opposed to sharing it across two scalars. Therefore, we conservatively bound the accuracy of a $b+2$-bit signed MX format with that of a $E_{1}M_{b}$ format in our comparisons. For instance, we use E1M2 format as a proxy for MX4.

\begin{figure}
    \centering
    \includegraphics[width=1\linewidth]{sections//figures/BlockFormats.pdf}
    \caption{\small Comparing LO-BCQ to MX format.}
    \label{fig:block_formats}
\end{figure}

Figure \ref{fig:block_formats} compares our $4$-bit LO-BCQ block format to MX \citep{rouhani2023microscaling}. As shown, both LO-BCQ and MX decompose a given operand tensor into block arrays and each block array into blocks. Similar to MX, we find that per-block quantization ($L_b < L_A$) leads to better accuracy due to increased flexibility. While MX achieves this through per-block $1$-bit micro-scales, we associate a dedicated codebook to each block through a per-block codebook selector. Further, MX quantizes the per-block array scale-factor to E8M0 format without per-tensor scaling. In contrast during LO-BCQ, we find that per-tensor scaling combined with quantization of per-block array scale-factor to E4M3 format results in superior inference accuracy across models. 

\onecolumn

\section{Complexity Analysis Derivation}
\label{sec:Complexity_Analysis_Proof}


Denote $l_{I,M,L,C} = l_{I} + l_{M} + l_{L} + l_{C}$, the full attention in each step is computed as
\begin{equation} \label{attention_eq_full}
	\mathbf{O} = \text{Attn}(\mathbf{Q}_C,\mathbf{K}_{[I,M,L,C]}, \mathbf{V}_{[I,M,L,C]})
\end{equation}
The complexity of computing full attention consists of the following parts: 
\begin{enumerate}
\item The dot product of queries and keys: 
\begin{equation} \label{attn_comple_dot}
  2 \cdot d_{H} \cdot l_{C} \cdot l_{I,M,L,C} 
\end{equation}
\item The softmax operation (including exponential calculation, summation, and normalization): 
\begin{equation} \label{attn_comple_softmax}
  3 \cdot H \cdot l_{C} \cdot l_{I,M,L,C}
\end{equation}
\item The dot product of attention weights and values: 
\begin{equation} \label{attn_comple_wei_val}
  2 \cdot d_{H} \cdot l_{C} \cdot l_{I,M,L,C}
\end{equation}
\end{enumerate}

The overall complexity is
\begin{equation} \label{attn_full}
  (4 \cdot d_{H} + 3 \cdot H) \cdot l_{C} \cdot l_{I,M,L,C} 
\end{equation}

The complexity of our method comprises the following components:
\begin{enumerate}
    \item Reduction of the dimensions of the query and key in Equation~(\ref{reduce_dim_formula}):
    % 这里分别对C的query和key降维,经过一个MLP
    \begin{equation}
      2 \cdot 2 \cdot l_C \cdot d_{H} \cdot d' + 2 \cdot l_C \cdot d'
    \end{equation}
    \item The complexity of token selection in Equation~(\ref{F_s_1}) and~(\ref{proximity_influence_eq}) is
    $2 \cdot l_M \cdot l_C \cdot d' + l_{M} \cdot l_C + h_{\text{max}}(2 \cdot l_C \cdot l_M + (1 + 2 \epsilon) \cdot l_M)$. The complexity of taking the maximum operation on multi-dimensional vectors is denoted as \( h_{\text{max}} \). Since (a) the complexity is lower than that of an equivalent number of Floating Point Operations (FLOPS)  for the same scale; and (b) $2 \cdot l_C \cdot l_M + (1 + 2 \epsilon) \cdot l_M < 2 \cdot l_M \cdot l_C \cdot d'$, we neglect the impact of the max operation. Therefore, the complexity of token selection is:
    \begin{equation}
      2 \cdot l_M \cdot l_C \cdot d' + l_{M} \cdot l_C
    \end{equation}
    % Following the same rationale, we can deduce the complexity associated with Equation~\ref{adaptive selection},~\ref{j_token-score-final} and~\ref{calculate_qk_analysis2}: 
    % \begin{equation}
    %   2 \cdot l_C \cdot d_{H} \cdot d_{H} + 2 \cdot H \cdot l_C \cdot d_{H} \cdot k + l_M \cdot l_C
    % \end{equation}
    \item Following the analogy of Equation~(\ref{attn_comple_dot}),~(\ref{attn_comple_softmax}) and~(\ref{attn_comple_wei_val}), after selecting $k$ tokens from $M$, the complexity of computing sparse attention is as follows:
    \begin{equation}
      (4 \cdot d_{H} + 3 \cdot H ) \cdot l_{C} \cdot (l_{I,L,C}+k)
    \end{equation}
    where, $l_{I,L,C} \triangleq l_{I}+l_{L}+l_{C}$.
\end{enumerate}
% 不需要考虑GQA
% 需要再推导一下
Given that $l_M >> l_{I},l_{L},l_{C},k,d_{H},d^{\prime}$, the overall reduction ratio of complexity is:
\begin{align} \label{reduction_ratio_complexity_infer}
r = \frac{l_I + k + l_L + l_C}{l_I + l_M + l_L + l_C} + \frac{4 d_{H} d^{\prime} + 2 d^{\prime} + 2 d^{\prime} l_M + l_M}{(4 d_{H} + 3H) (l_I + l_M + l_L + l_C)} \approx \frac{2 d^{\prime} +1}{4d_{H} + 3H}
\end{align}
% Adopting different attention computation methods during the prefilling and decoding stages can slightly reduce the computational complexity. however, this does not significantly alter the compression ratio for long sequence computations. Consequently, the compression ratio during the decoding stage remains consistent with \( r \) as presented in Equation~\ref{reduction_ratio_complexity_infer}.

\section{Model Loading}
\label{sec:load_model_appendix}
All of our experiments were conducted on a device equipped with 4 $\times$ A800 GPUs. The evaluated models were partitioned across the 4 GPUs by layer using the \texttt{accelerate} \citep{accelerate}. Our model loading approach is capable of supporting the execution of all the aforementioned experiments. 
Additionally, we support loading the model onto a single A800 GPU by offloading the original KV cache to the CPU, while the dimension-reduced keys are always kept on the GPU. 
During each attention computation, a small number of keys and values in KV cache are loaded onto the GPU. 
By employing this cache management approach, we are able to perform inference on long sequence tasks with lengths of 700k+ on a single A800 GPU.
For NTK and YaRN, we employ vLLM \citep{kwon2023efficient} for inference on a device equipped with 4 $\times$ A800 GPUs.

\section{Samples of NeedleBench and Counting-Stars}
\label{sec:NeedleBench and Counting-Stars}

\paragraph{NeedleBench}
A sample from NeedleBench is shown below, where \textit{xxxxxxxx} represents noise text:
\begin{quote}
  You are an intelligent AI assistant skilled in answering user questions. Please keep your answers concise and clear. Do not talk about irrelevant topics or repeat your answers.\\
  The document given to you by the user is May 2001 xxxxxxxx \textbf{Hidden on Hell Island is the legendary Dream Bubble.}  xxxxxxxx \textbf{Hidden on Emerald Island is the legendary Ghost Pearl.}   xxxxxxxx \textbf{Hidden on Sand Island is the legendary Stardust Shard.}xxxxxxxx\\
  Now, the questions are: \textbf{What legendary item is hidden on Hell Island?What legendary item is hidden on Emerald Island?What legendary item is hidden on Sand Island?}Before answering, please consider what in the document is most relevant to this question. Please answer in the format of 'The legendary item hidden on the Hell Island is\underline{$\phantom{\rule{1cm}{0ex}}$}. The legendary item hidden on the Emerald Island is\underline{$\phantom{\rule{1cm}{0ex}}$}. The legendary item hidden on the Sand Island is\underline{$\phantom{\rule{1cm}{0ex}}$}
\end{quote}

\paragraph{Counting-Stars}
A sample from Counting-Stars is shown below, where \textit{xxxxxxxx} represents noise text:
\begin{quote}
  xxxxxxxx \textbf{The little penguin counted 15 $\star$} xxxxxxxx \textbf{The little penguin counted 117 $\star$} xxxxxxxx \textbf{The little penguin counted 42 $\star$} xxxxxxxx \textbf{The little penguin counted 29 $\star$} \\
  On this moonlit and misty night, the little penguin is looking up at the sky and concentrating on counting $\star$. Please help the little penguin collect the number of $\star$, for example: \{"little\_penguin": [x, x, x,...]\}. The summation is not required, and the numbers in [x, x, x,...] represent the counted number of $\star$ by the little penguin. Only output the results in JSON format without any explanation. 
  ```json \\
  \{"little\_penguin": [
\end{quote}




\section{Proximity Influence Distance: Specific Experiments} \label{appen_proxi_influ}
We validate the ablation study results of the proximity influence distance, with the detailed findings presented in Table~\ref{tab:proximity_influence_distance_details}. 
When \( \epsilon \) is set to 0 and 1, Mistral demonstrates superior performance across all subtasks ranging from 4k to 200k. 
As \( \epsilon \) is further increased to 3 and 5, the scores on the short-sequence subtasks (4k-16k) remain comparable to the previous results. 
However, the model's performance exhibits a significant decline on the long-sequence subtasks (48k-200k).

\begin{table}[]
\centering
\resizebox{\columnwidth}{!}{%
\begin{tabular}{c|cccccccccc|c}
\hline
\multicolumn{1}{l|}{\multirow{2}{*}{$\epsilon$}} & \multicolumn{10}{c|}{Context Length} & \multicolumn{1}{l}{\multirow{2}{*}{Average}} \\
\multicolumn{1}{l|}{} & \multicolumn{1}{l}{4k} & \multicolumn{1}{l}{8k} & \multicolumn{1}{l}{16k} & \multicolumn{1}{l}{48k} & \multicolumn{1}{l}{80k} & \multicolumn{1}{l}{112k} & \multicolumn{1}{l}{128k} & \multicolumn{1}{l}{144k} & \multicolumn{1}{l}{176k} & \multicolumn{1}{l|}{200k} & \multicolumn{1}{l}{} \\ \hline
0 & 100 & 83.33 & \textbf{73.33} & 80 & \textbf{70} & 60 & 66.67 & \textbf{70} & 46.67 & \textbf{46.67} & 69.67 \\
1 & 100 & 83.33 & \textbf{73.33} & \textbf{86.67} & 63.33 & \textbf{70} & \textbf{73.33} & 56.67 & \textbf{60} & \textbf{46.67} & \textbf{71.33} \\
3 & 100 & 83.33 & \textbf{73.33} & 70 & 66.67 & 63.33 & 60 & 63.33 & 53.33 & \textbf{46.67} & 68 \\
5 & 100 & 83.33 & 70 & 70 & 40 & 40 & 56.67 & 36.67 & 50 & 40 & 58.67 \\ \hline
\end{tabular}%
}
\caption{We validate the performance across all subtasks of NeedleBench with varying \( \epsilon \) using Mistral.}
  \label{tab:proximity_influence_distance_details}
\end{table}



\section{Token Selection for Heads: Specific Experiments}
\label{sec:Token_Selection_for_Heads}
We employ Llama to investigate the impact of different token selection methods for heads. 
We extrapolate Llama's original context length of 8k to 32k and conduct experiments using LongBench, which has an average length of 32k.
The scores for each subtask are presented in Table~\ref{tab:token_selection_head_subtask}.
\begin{table}[]
\centering
  \begin{tabular}{l|cc}
  \hline
  \multicolumn{1}{c|}{Task} & individual     & uniform        \\ \hline
  NarrativaQA               & 24.52          & \textbf{25.1}  \\
  Qasper                    & \textbf{44.69} & 44.51          \\
  MultiFieldQA-en           & 49.18          & 49.18          \\
  MuSiQue                   & 25.55          & \textbf{27.58} \\
  HotpotQA                  & \textbf{49.57} & 49.26          \\
  2WikiMultihopQA           & \textbf{38.1}  & 37.44          \\
  GovReport                 & \textbf{31.06} & 30.99          \\
  QMSum                     & \textbf{22.91} & 22.75          \\
  MultiNews                 & 27.41          & \textbf{27.45} \\
  TREC                      & 73.5           & 73.5           \\
  TriviaQA                  & 91.19          & 91.19          \\
  SAMSum                    & \textbf{42.87} & 42.7           \\
  PassageRetrieval-en       & 86.5           & \textbf{87.5}  \\
  PassageCount              & 8.17           & 7.17           \\
  LCC                       & 58.32          & 58.32          \\
  RepoBench-P               & 41.7           & \textbf{42.6}  \\ \hline
  Average                   & 44.7           & \textbf{44.8}           \\ \hline
  \end{tabular}
  \caption{Llama's specific experiments on LongBench using different token selection methods for heads. "Individual" refers to the importance score of each token being the maximum value among the scores of all heads, meaning that each head votes for the scores. This approach ensures that the selection process takes into account all heads. "Uniform" in the table denotes our method of selecting tokens without dimensionality reduction.}
  \label{tab:token_selection_head_subtask}
  \end{table}

\section{LM-Infinite with additional top-k middle tokens}
 LM-Infinite optionally select top-k middle tokens for some higher layers for each head. Following the settings in their paper, we evaluate the effectiveness of this method and the results are demonstrated in Table~\ref{tab:infinite-top}. The performance is improved compared to the original LM-Infinite setting. Nevertheless, it is not as competitive as our ESA method.
\begin{table}[]
    \centering
    \begin{tabular}{@{}l|c|c@{}}
        \toprule
        \multicolumn{1}{c|}{Task} & Llama & Mistral \\ \midrule
        NarrativaQA               & 20.6  & 22.02   \\
        Qasper                    & 21.73 & 30.36   \\
        MultiFieldQA-en           & 40.2  & 44.52   \\
        MuSiQue                   & 20.19 & 16.36   \\
        HotpotQA                  & 44.66 & 32.63   \\
        2WikiMultihopQA           & 38.09 & 22.64   \\
        GovReport                 & 31.28 & 31.61   \\
        QMSum                     & 21.73 & 22.5    \\
        MultiNews                 & 27.54 & 26.7    \\
        TREC                      & 73.5  & 70.5    \\
        TriviaQA                  & 90.91 & 86.59   \\
        SAMSum                    & 42.57 & 42.26   \\
        PassageRetrieval-en       & 38.5  & 49.42   \\
        PassageCount              & 8.5   & 2.37    \\
        LCC                       & 60.75 & 57.4    \\
        RepoBench-P               & 43.83 & 53.51   \\ \midrule
        Average                   & 40.37 & 38.212  \\ \bottomrule
    \end{tabular}
    \caption{Results on LongBench with Infinite-LM attending to top 5 tokens in the middle.}
    \label{tab:infinite-top}
\end{table}

\section{Pseudocode for Computing Attention}
\label{sec:pseudocode_attn}
We support the computation of local and global attention using either Flash Attention \citep{dao2205fast} or PyTorch operators. The pseudocode for computing a step of attention with Flash Attention is shown in Algorithm~\ref{alg:Pseudocode for Attention Computation}. It is worth noting that we omit the exp-normalize trick in Step 12 of Algorithm~\ref{alg:Pseudocode for Attention Computation} to avoid numerical overflow. We employ the function \texttt{flash\_attn\_func} provided by Flash Attention, which returns the logarithm of the softmax normalization factor as its second result. In environments where Flash Attention is not supported, we can replace the function \texttt{flash\_attn\_func} with PyTorch operators.
\begin{algorithm*} 
\caption{Pseudocode for Attention Computation with Flash Attention}
\begin{algorithmic}[1]
\State \textbf{Input:}
\State $l\_q$: Queries from $C$ with position encoding $(l_L, l_L+1, l_L+2, \ldots, l_L + l_C - 1)$
\State $g\_q$: Queries from $C$ with position encoding $(w, w, w, \ldots, w)$
\State $l\_k$: Concatenated keys from $L$ and $C$ with position encoding $(0, 1, 2, \ldots, l_L + l_C - 1)$
\State $g\_k$: Selected keys from $M$ and $I$ with position encoding $(0, 0, 0, 0, \ldots)$
\State $l\_v$: Concatenated values from $L$ and $C$
\State $g\_v$: Selected values from $M$ and $I$
\State
\State \textbf{Procedure:}
\State $(l\_attn, l\_lse, \_) \leftarrow \text{flash\_attn\_func}(l\_q, l\_k, l\_v, \mathrm{causal}=\mathrm{True})$
\State $(g\_attn, g\_lse, \_) \leftarrow \text{flash\_attn\_func}(g\_q, g\_k, g\_v, \mathrm{causal}=\mathrm{False})$
\State $se \leftarrow \exp([l\_lse, g\_lse])$
\State $fac \leftarrow se / \sum se$
\State $attn \leftarrow [l\_attn, g\_attn] \cdot fac$
\State
\State \textbf{Output:}
\State $attn$
\end{algorithmic}
\label{alg:Pseudocode for Attention Computation}
\end{algorithm*}


% \section{Experimental Results on Counting-Stars}
% \label{sec:Counting-Stars-Results}
% The detailed Results on Counting-Stars for Llama and Mistral with different extrapolation methods are shown in Figure~\ref{fig:llama_counting_star} and Figure~\ref{fig:Mistral_counting_star}, respectively. Each row in the figure demonstrates the results of the same method for different experimental settings.
% %%% counting star results
% %\usepackage{graphicx}
%\usepackage{subcaption}
%\usepackage{geometry}

\clearpage
\newgeometry{top=10mm}
\begin{figure*}[ht]
	\centering
	\begin{minipage}{\textwidth}
		\centering
		\begin{minipage}{0.24\textwidth}
			\centering
			\includegraphics[width=\textwidth]{imgs/Llama_Origin_128000_32_32} % 替换为您的图片文件名
			%				\caption*{a}
		\end{minipage}%
		\begin{minipage}{0.24\textwidth}
			\centering
			\includegraphics[width=\textwidth]{imgs/Llama_Origin_256000_8_32} % 替换为您的图片文件名
			%				\caption*{b}
		\end{minipage}%
		\begin{minipage}{0.24\textwidth}
			\centering
			\includegraphics[width=\textwidth]{imgs/Llama_Origin_256000_16_32} % 替换为您的图片文件名
			%				\caption*{c}
		\end{minipage}%
		\begin{minipage}{0.24\textwidth}
			\centering
			\includegraphics[width=\textwidth]{imgs/Llama_Origin_256000_32_32} % 替换为您的图片文件名
			%				\caption*{d}
		\end{minipage}
	\end{minipage}
	%%%%%%%%%%%%%%%%%%%%%%
	\begin{minipage}{\textwidth}
		\centering
		\begin{minipage}{0.24\textwidth}
			\centering
			\includegraphics[width=\textwidth]{imgs/Llama_NTK_128000_32_32} % 替换为您的图片文件名
			%				\caption*{a}
		\end{minipage}%
		\begin{minipage}{0.24\textwidth}
			\centering
			\includegraphics[width=\textwidth]{imgs/Llama_NTK_256000_8_32} % 替换为您的图片文件名
			%				\caption*{b}
		\end{minipage}%
		\begin{minipage}{0.24\textwidth}
			\centering
			\includegraphics[width=\textwidth]{imgs/Llama_NTK_256000_16_32} % 替换为您的图片文件名
			%				\caption*{c}
		\end{minipage}%
		\begin{minipage}{0.24\textwidth}
			\centering
			\includegraphics[width=\textwidth]{imgs/Llama_NTK_256000_32_32} % 替换为您的图片文件名
			%				\caption*{d}
		\end{minipage}
	\end{minipage}
	%%%%%%%%%%%%%%%%%%%%
	\begin{minipage}{\textwidth}
		\centering
		\begin{minipage}{0.24\textwidth}
			\centering
			\includegraphics[width=\textwidth]{imgs/Llama_YaRN_128000_32_32} % 替换为您的图片文件名
			%				\caption*{a}
		\end{minipage}%
		\begin{minipage}{0.24\textwidth}
			\centering
			\includegraphics[width=\textwidth]{imgs/Llama_YaRN_256000_8_32} % 替换为您的图片文件名
			%				\caption*{b}
		\end{minipage}%
		\begin{minipage}{0.24\textwidth}
			\centering
			\includegraphics[width=\textwidth]{imgs/Llama_YaRN_256000_16_32} % 替换为您的图片文件名
			%				\caption*{c}
		\end{minipage}%
		\begin{minipage}{0.24\textwidth}
			\centering
			\includegraphics[width=\textwidth]{imgs/Llama_YaRN_256000_32_32} % 替换为您的图片文件名
			%				\caption*{d}
		\end{minipage}
	\end{minipage}
	%%%%%%%%%%%%%%%%%%%%%%%
	\begin{minipage}{\textwidth}
		\centering
		\begin{minipage}{0.24\textwidth}
			\centering
			\includegraphics[width=\textwidth]{imgs/Llama_Infinite_128000_32_32} % 替换为您的图片文件名
			%				\caption*{a}
		\end{minipage}%
		\begin{minipage}{0.24\textwidth}
			\centering
			\includegraphics[width=\textwidth]{imgs/Llama_Infinite_256000_8_32} % 替换为您的图片文件名
			%				\caption*{b}
		\end{minipage}%
		\begin{minipage}{0.24\textwidth}
			\centering
			\includegraphics[width=\textwidth]{imgs/Llama_Infinite_256000_16_32} % 替换为您的图片文件名
			%				\caption*{c}
		\end{minipage}%
		\begin{minipage}{0.24\textwidth}
			\centering
			\includegraphics[width=\textwidth]{imgs/Llama_Infinite_256000_32_32} % 替换为您的图片文件名
			%				\caption*{d}
		\end{minipage}
	\end{minipage}
	%%%%%%%%%%%%%%%%%%%%%%%
	\begin{minipage}{\textwidth}
		\centering
		\begin{minipage}{0.24\textwidth}
			\centering
			\includegraphics[width=\textwidth]{imgs/Llama_Stream_128000_32_32} % 替换为您的图片文件名
			%				\caption*{a}
		\end{minipage}%
		\begin{minipage}{0.24\textwidth}
			\centering
			\includegraphics[width=\textwidth]{imgs/Llama_Stream_256000_8_32} % 替换为您的图片文件名
			%				\caption*{b}
		\end{minipage}%
		\begin{minipage}{0.24\textwidth}
			\centering
			\includegraphics[width=\textwidth]{imgs/Llama_Stream_256000_16_32} % 替换为您的图片文件名
			%				\caption*{c}
		\end{minipage}%
		\begin{minipage}{0.24\textwidth}
			\centering
			\includegraphics[width=\textwidth]{imgs/Llama_Stream_256000_32_32} % 替换为您的图片文件名
			%				\caption*{d}
		\end{minipage}
	\end{minipage}
	%%%%%%%%%%%%%%%%%%%%%%%
	\begin{minipage}{\textwidth}
		\centering
		\begin{minipage}{0.24\textwidth}
			\centering
			\includegraphics[width=\textwidth]{imgs/Llama_InfLLM_128000_32_32} % 替换为您的图片文件名
			%				\caption*{a}
		\end{minipage}%
		\begin{minipage}{0.24\textwidth}
			\centering
			\includegraphics[width=\textwidth]{imgs/Llama_InfLLM_256000_8_32} % 替换为您的图片文件名
			%				\caption*{b}
		\end{minipage}%
		\begin{minipage}{0.24\textwidth}
			\centering
			\includegraphics[width=\textwidth]{imgs/Llama_InfLLM_256000_16_32} % 替换为您的图片文件名
			%				\caption*{c}
		\end{minipage}%
		\begin{minipage}{0.24\textwidth}
			\centering
			\includegraphics[width=\textwidth]{imgs/Llama_InfLLM_256000_32_32} % 替换为您的图片文件名
			%				\caption*{d}
		\end{minipage}
	\end{minipage}
	%%%%%%%%%%%%%%%%%%%%%%%
	\begin{minipage}{\textwidth}
		\centering
		\begin{minipage}{0.24\textwidth}
			\centering
			\includegraphics[width=\textwidth]{imgs/Llama_Ours_128000_32_32} % 替换为您的图片文件名
			%				\caption*{a}
		\end{minipage}%
		\begin{minipage}{0.24\textwidth}
			\centering
			\includegraphics[width=\textwidth]{imgs/Llama_Ours_256000_8_32} % 替换为您的图片文件名
			%				\caption*{b}
		\end{minipage}%
		\begin{minipage}{0.24\textwidth}
			\centering
			\includegraphics[width=\textwidth]{imgs/Llama_Ours_256000_16_32} % 替换为您的图片文件名
			%				\caption*{c}
		\end{minipage}%
		\begin{minipage}{0.24\textwidth}
			\centering
			\includegraphics[width=\textwidth]{imgs/Llama_Ours_256000_32_32} % 替换为您的图片文件名
			%				\caption*{d}
		\end{minipage}
	\end{minipage}
	
	\caption{Results on Counting-Stars for Llama with different methods.}
	\label{fig:llama_counting_star}
\end{figure*}


%%%%%%%%%%%%%%%%%%%%%%%%%%%%%%%%%%%%%%%%%%%%%%%%%%%%%%%%%%%%%%%%%%%%%%%%%
\clearpage	
\begin{figure*}[ht]
	\centering
	\begin{minipage}{\textwidth}
		\centering
		\begin{minipage}{0.24\textwidth}
			\centering
			\includegraphics[width=\textwidth]{imgs/Mistral_Origin_128000_32_32} % 替换为您的图片文件名
			%				\caption*{a}
		\end{minipage}%
		\begin{minipage}{0.24\textwidth}
			\centering
			\includegraphics[width=\textwidth]{imgs/Mistral_Origin_256000_8_32} % 替换为您的图片文件名
			%				\caption*{b}
		\end{minipage}%
		\begin{minipage}{0.24\textwidth}
			\centering
			\includegraphics[width=\textwidth]{imgs/Mistral_Origin_256000_16_32} % 替换为您的图片文件名
			%				\caption*{c}
		\end{minipage}%
		\begin{minipage}{0.24\textwidth}
			\centering
			\includegraphics[width=\textwidth]{imgs/Mistral_Origin_256000_32_32} % 替换为您的图片文件名
			%				\caption*{d}
		\end{minipage}
	\end{minipage}
	%%%%%%%%%%%%%%%%%%%%%%
	\begin{minipage}{\textwidth}
		\centering
		\begin{minipage}{0.24\textwidth}
			\centering
			\includegraphics[width=\textwidth]{imgs/Mistral_NTK_128000_32_32} % 替换为您的图片文件名
			%				\caption*{a}
		\end{minipage}%
		\begin{minipage}{0.24\textwidth}
			\centering
			\includegraphics[width=\textwidth]{imgs/Mistral_NTK_256000_8_32} % 替换为您的图片文件名
			%				\caption*{b}
		\end{minipage}%
		\begin{minipage}{0.24\textwidth}
			\centering
			\includegraphics[width=\textwidth]{imgs/Mistral_NTK_256000_16_32} % 替换为您的图片文件名
			%				\caption*{c}
		\end{minipage}%
		\begin{minipage}{0.24\textwidth}
			\centering
			\includegraphics[width=\textwidth]{imgs/Mistral_NTK_256000_32_32} % 替换为您的图片文件名
			%				\caption*{d}
		\end{minipage}
	\end{minipage}
	%%%%%%%%%%%%%%%%%%%%
	\begin{minipage}{\textwidth}
		\centering
		\begin{minipage}{0.24\textwidth}
			\centering
			\includegraphics[width=\textwidth]{imgs/Mistral_YaRN_128000_32_32} % 替换为您的图片文件名
			%				\caption*{a}
		\end{minipage}%
		\begin{minipage}{0.24\textwidth}
			\centering
			\includegraphics[width=\textwidth]{imgs/Mistral_YaRN_256000_8_32} % 替换为您的图片文件名
			%				\caption*{b}
		\end{minipage}%
		\begin{minipage}{0.24\textwidth}
			\centering
			\includegraphics[width=\textwidth]{imgs/Mistral_YaRN_256000_16_32} % 替换为您的图片文件名
			%				\caption*{c}
		\end{minipage}%
		\begin{minipage}{0.24\textwidth}
			\centering
			\includegraphics[width=\textwidth]{imgs/Mistral_YaRN_256000_32_32} % 替换为您的图片文件名
			%				\caption*{d}
		\end{minipage}
	\end{minipage}
	%%%%%%%%%%%%%%%%%%%%%%%
	\begin{minipage}{\textwidth}
		\centering
		\begin{minipage}{0.24\textwidth}
			\centering
			\includegraphics[width=\textwidth]{imgs/Mistral_Infinite_128000_32_32} % 替换为您的图片文件名
			%				\caption*{a}
		\end{minipage}%
		\begin{minipage}{0.24\textwidth}
			\centering
			\includegraphics[width=\textwidth]{imgs/Mistral_Infinite_256000_8_32} % 替换为您的图片文件名
			%				\caption*{b}
		\end{minipage}%
		\begin{minipage}{0.24\textwidth}
			\centering
			\includegraphics[width=\textwidth]{imgs/Mistral_Infinite_256000_16_32} % 替换为您的图片文件名
			%				\caption*{c}
		\end{minipage}%
		\begin{minipage}{0.24\textwidth}
			\centering
			\includegraphics[width=\textwidth]{imgs/Mistral_Infinite_256000_32_32} % 替换为您的图片文件名
			%				\caption*{d}
		\end{minipage}
	\end{minipage}
	%%%%%%%%%%%%%%%%%%%%%%%
	\begin{minipage}{\textwidth}
		\centering
		\begin{minipage}{0.24\textwidth}
			\centering
			\includegraphics[width=\textwidth]{imgs/Mistral_Stream_128000_32_32} % 替换为您的图片文件名
			%				\caption*{a}
		\end{minipage}%
		\begin{minipage}{0.24\textwidth}
			\centering
			\includegraphics[width=\textwidth]{imgs/Mistral_Stream_256000_8_32} % 替换为您的图片文件名
			%				\caption*{b}
		\end{minipage}%
		\begin{minipage}{0.24\textwidth}
			\centering
			\includegraphics[width=\textwidth]{imgs/Mistral_Stream_256000_16_32} % 替换为您的图片文件名
			%				\caption*{c}
		\end{minipage}%
		\begin{minipage}{0.24\textwidth}
			\centering
			\includegraphics[width=\textwidth]{imgs/Mistral_Stream_256000_32_32} % 替换为您的图片文件名
			%				\caption*{d}
		\end{minipage}
	\end{minipage}
	%%%%%%%%%%%%%%%%%%%%%%%
	\begin{minipage}{\textwidth}
		\centering
		\begin{minipage}{0.24\textwidth}
			\centering
			\includegraphics[width=\textwidth]{imgs/Mistral_InfLLM_128000_32_32} % 替换为您的图片文件名
			%				\caption*{a}
		\end{minipage}%
		\begin{minipage}{0.24\textwidth}
			\centering
			\includegraphics[width=\textwidth]{imgs/Mistral_InfLLM_256000_8_32} % 替换为您的图片文件名
			%				\caption*{b}
		\end{minipage}%
		\begin{minipage}{0.24\textwidth}
			\centering
			\includegraphics[width=\textwidth]{imgs/Mistral_InfLLM_256000_16_32} % 替换为您的图片文件名
			%				\caption*{c}
		\end{minipage}%
		\begin{minipage}{0.24\textwidth}
			\centering
			\includegraphics[width=\textwidth]{imgs/Mistral_InfLLM_256000_32_32} % 替换为您的图片文件名
			%				\caption*{d}
		\end{minipage}
	\end{minipage}
	%%%%%%%%%%%%%%%%%%%%%%%
	\begin{minipage}{\textwidth}
		\centering
		\begin{minipage}{0.24\textwidth}
			\centering
			\includegraphics[width=\textwidth]{imgs/Mistral_Ours_128000_32_32} % 替换为您的图片文件名
			%				\caption*{a}
		\end{minipage}%
		\begin{minipage}{0.24\textwidth}
			\centering
			\includegraphics[width=\textwidth]{imgs/Mistral_Ours_256000_8_32} % 替换为您的图片文件名
			%				\caption*{b}
		\end{minipage}%
		\begin{minipage}{0.24\textwidth}
			\centering
			\includegraphics[width=\textwidth]{imgs/Mistral_Ours_256000_16_32} % 替换为您的图片文件名
			%				\caption*{c}
		\end{minipage}%
		\begin{minipage}{0.24\textwidth}
			\centering
			\includegraphics[width=\textwidth]{imgs/Mistral_Ours_256000_32_32} % 替换为您的图片文件名
			%				\caption*{d}
		\end{minipage}
	\end{minipage}
	
	\caption{Results on Counting-Stars for Mistral with different methods.}
	\label{fig:Mistral_counting_star}
\end{figure*}
\restoregeometry
%%\usepackage{graphicx}
%\usepackage{subcaption}
%\usepackage{geometry}

\clearpage
\newgeometry{top=10mm}
\begin{figure*}[ht]
	\centering
	\begin{minipage}{\textwidth}
		\centering
		\begin{minipage}{0.24\textwidth}
			\centering
			\includegraphics[width=\textwidth]{imgs/Llama_Origin_128000_32_32} % 替换为您的图片文件名
			%				\caption*{a}
		\end{minipage}%
		\begin{minipage}{0.24\textwidth}
			\centering
			\includegraphics[width=\textwidth]{imgs/Llama_Origin_256000_8_32} % 替换为您的图片文件名
			%				\caption*{b}
		\end{minipage}%
		\begin{minipage}{0.24\textwidth}
			\centering
			\includegraphics[width=\textwidth]{imgs/Llama_Origin_256000_16_32} % 替换为您的图片文件名
			%				\caption*{c}
		\end{minipage}%
		\begin{minipage}{0.24\textwidth}
			\centering
			\includegraphics[width=\textwidth]{imgs/Llama_Origin_256000_32_32} % 替换为您的图片文件名
			%				\caption*{d}
		\end{minipage}
	\end{minipage}
	%%%%%%%%%%%%%%%%%%%%%%
	\begin{minipage}{\textwidth}
		\centering
		\begin{minipage}{0.24\textwidth}
			\centering
			\includegraphics[width=\textwidth]{imgs/Llama_NTK_128000_32_32} % 替换为您的图片文件名
			%				\caption*{a}
		\end{minipage}%
		\begin{minipage}{0.24\textwidth}
			\centering
			\includegraphics[width=\textwidth]{imgs/Llama_NTK_256000_8_32} % 替换为您的图片文件名
			%				\caption*{b}
		\end{minipage}%
		\begin{minipage}{0.24\textwidth}
			\centering
			\includegraphics[width=\textwidth]{imgs/Llama_NTK_256000_16_32} % 替换为您的图片文件名
			%				\caption*{c}
		\end{minipage}%
		\begin{minipage}{0.24\textwidth}
			\centering
			\includegraphics[width=\textwidth]{imgs/Llama_NTK_256000_32_32} % 替换为您的图片文件名
			%				\caption*{d}
		\end{minipage}
	\end{minipage}
	%%%%%%%%%%%%%%%%%%%%
	\begin{minipage}{\textwidth}
		\centering
		\begin{minipage}{0.24\textwidth}
			\centering
			\includegraphics[width=\textwidth]{imgs/Llama_YaRN_128000_32_32} % 替换为您的图片文件名
			%				\caption*{a}
		\end{minipage}%
		\begin{minipage}{0.24\textwidth}
			\centering
			\includegraphics[width=\textwidth]{imgs/Llama_YaRN_256000_8_32} % 替换为您的图片文件名
			%				\caption*{b}
		\end{minipage}%
		\begin{minipage}{0.24\textwidth}
			\centering
			\includegraphics[width=\textwidth]{imgs/Llama_YaRN_256000_16_32} % 替换为您的图片文件名
			%				\caption*{c}
		\end{minipage}%
		\begin{minipage}{0.24\textwidth}
			\centering
			\includegraphics[width=\textwidth]{imgs/Llama_YaRN_256000_32_32} % 替换为您的图片文件名
			%				\caption*{d}
		\end{minipage}
	\end{minipage}
	%%%%%%%%%%%%%%%%%%%%%%%
	\begin{minipage}{\textwidth}
		\centering
		\begin{minipage}{0.24\textwidth}
			\centering
			\includegraphics[width=\textwidth]{imgs/Llama_Infinite_128000_32_32} % 替换为您的图片文件名
			%				\caption*{a}
		\end{minipage}%
		\begin{minipage}{0.24\textwidth}
			\centering
			\includegraphics[width=\textwidth]{imgs/Llama_Infinite_256000_8_32} % 替换为您的图片文件名
			%				\caption*{b}
		\end{minipage}%
		\begin{minipage}{0.24\textwidth}
			\centering
			\includegraphics[width=\textwidth]{imgs/Llama_Infinite_256000_16_32} % 替换为您的图片文件名
			%				\caption*{c}
		\end{minipage}%
		\begin{minipage}{0.24\textwidth}
			\centering
			\includegraphics[width=\textwidth]{imgs/Llama_Infinite_256000_32_32} % 替换为您的图片文件名
			%				\caption*{d}
		\end{minipage}
	\end{minipage}
	%%%%%%%%%%%%%%%%%%%%%%%
	\begin{minipage}{\textwidth}
		\centering
		\begin{minipage}{0.24\textwidth}
			\centering
			\includegraphics[width=\textwidth]{imgs/Llama_Stream_128000_32_32} % 替换为您的图片文件名
			%				\caption*{a}
		\end{minipage}%
		\begin{minipage}{0.24\textwidth}
			\centering
			\includegraphics[width=\textwidth]{imgs/Llama_Stream_256000_8_32} % 替换为您的图片文件名
			%				\caption*{b}
		\end{minipage}%
		\begin{minipage}{0.24\textwidth}
			\centering
			\includegraphics[width=\textwidth]{imgs/Llama_Stream_256000_16_32} % 替换为您的图片文件名
			%				\caption*{c}
		\end{minipage}%
		\begin{minipage}{0.24\textwidth}
			\centering
			\includegraphics[width=\textwidth]{imgs/Llama_Stream_256000_32_32} % 替换为您的图片文件名
			%				\caption*{d}
		\end{minipage}
	\end{minipage}
	%%%%%%%%%%%%%%%%%%%%%%%
	\begin{minipage}{\textwidth}
		\centering
		\begin{minipage}{0.24\textwidth}
			\centering
			\includegraphics[width=\textwidth]{imgs/Llama_InfLLM_128000_32_32} % 替换为您的图片文件名
			%				\caption*{a}
		\end{minipage}%
		\begin{minipage}{0.24\textwidth}
			\centering
			\includegraphics[width=\textwidth]{imgs/Llama_InfLLM_256000_8_32} % 替换为您的图片文件名
			%				\caption*{b}
		\end{minipage}%
		\begin{minipage}{0.24\textwidth}
			\centering
			\includegraphics[width=\textwidth]{imgs/Llama_InfLLM_256000_16_32} % 替换为您的图片文件名
			%				\caption*{c}
		\end{minipage}%
		\begin{minipage}{0.24\textwidth}
			\centering
			\includegraphics[width=\textwidth]{imgs/Llama_InfLLM_256000_32_32} % 替换为您的图片文件名
			%				\caption*{d}
		\end{minipage}
	\end{minipage}
	%%%%%%%%%%%%%%%%%%%%%%%
	\begin{minipage}{\textwidth}
		\centering
		\begin{minipage}{0.24\textwidth}
			\centering
			\includegraphics[width=\textwidth]{imgs/Llama_Ours_128000_32_32} % 替换为您的图片文件名
			%				\caption*{a}
		\end{minipage}%
		\begin{minipage}{0.24\textwidth}
			\centering
			\includegraphics[width=\textwidth]{imgs/Llama_Ours_256000_8_32} % 替换为您的图片文件名
			%				\caption*{b}
		\end{minipage}%
		\begin{minipage}{0.24\textwidth}
			\centering
			\includegraphics[width=\textwidth]{imgs/Llama_Ours_256000_16_32} % 替换为您的图片文件名
			%				\caption*{c}
		\end{minipage}%
		\begin{minipage}{0.24\textwidth}
			\centering
			\includegraphics[width=\textwidth]{imgs/Llama_Ours_256000_32_32} % 替换为您的图片文件名
			%				\caption*{d}
		\end{minipage}
	\end{minipage}
	
	\caption{Results on Counting-Stars for Llama with different methods.}
	\label{fig:llama_counting_star}
\end{figure*}


%%%%%%%%%%%%%%%%%%%%%%%%%%%%%%%%%%%%%%%%%%%%%%%%%%%%%%%%%%%%%%%%%%%%%%%%%
\clearpage	
\begin{figure*}[ht]
	\centering
	\begin{minipage}{\textwidth}
		\centering
		\begin{minipage}{0.24\textwidth}
			\centering
			\includegraphics[width=\textwidth]{imgs/Mistral_Origin_128000_32_32} % 替换为您的图片文件名
			%				\caption*{a}
		\end{minipage}%
		\begin{minipage}{0.24\textwidth}
			\centering
			\includegraphics[width=\textwidth]{imgs/Mistral_Origin_256000_8_32} % 替换为您的图片文件名
			%				\caption*{b}
		\end{minipage}%
		\begin{minipage}{0.24\textwidth}
			\centering
			\includegraphics[width=\textwidth]{imgs/Mistral_Origin_256000_16_32} % 替换为您的图片文件名
			%				\caption*{c}
		\end{minipage}%
		\begin{minipage}{0.24\textwidth}
			\centering
			\includegraphics[width=\textwidth]{imgs/Mistral_Origin_256000_32_32} % 替换为您的图片文件名
			%				\caption*{d}
		\end{minipage}
	\end{minipage}
	%%%%%%%%%%%%%%%%%%%%%%
	\begin{minipage}{\textwidth}
		\centering
		\begin{minipage}{0.24\textwidth}
			\centering
			\includegraphics[width=\textwidth]{imgs/Mistral_NTK_128000_32_32} % 替换为您的图片文件名
			%				\caption*{a}
		\end{minipage}%
		\begin{minipage}{0.24\textwidth}
			\centering
			\includegraphics[width=\textwidth]{imgs/Mistral_NTK_256000_8_32} % 替换为您的图片文件名
			%				\caption*{b}
		\end{minipage}%
		\begin{minipage}{0.24\textwidth}
			\centering
			\includegraphics[width=\textwidth]{imgs/Mistral_NTK_256000_16_32} % 替换为您的图片文件名
			%				\caption*{c}
		\end{minipage}%
		\begin{minipage}{0.24\textwidth}
			\centering
			\includegraphics[width=\textwidth]{imgs/Mistral_NTK_256000_32_32} % 替换为您的图片文件名
			%				\caption*{d}
		\end{minipage}
	\end{minipage}
	%%%%%%%%%%%%%%%%%%%%
	\begin{minipage}{\textwidth}
		\centering
		\begin{minipage}{0.24\textwidth}
			\centering
			\includegraphics[width=\textwidth]{imgs/Mistral_YaRN_128000_32_32} % 替换为您的图片文件名
			%				\caption*{a}
		\end{minipage}%
		\begin{minipage}{0.24\textwidth}
			\centering
			\includegraphics[width=\textwidth]{imgs/Mistral_YaRN_256000_8_32} % 替换为您的图片文件名
			%				\caption*{b}
		\end{minipage}%
		\begin{minipage}{0.24\textwidth}
			\centering
			\includegraphics[width=\textwidth]{imgs/Mistral_YaRN_256000_16_32} % 替换为您的图片文件名
			%				\caption*{c}
		\end{minipage}%
		\begin{minipage}{0.24\textwidth}
			\centering
			\includegraphics[width=\textwidth]{imgs/Mistral_YaRN_256000_32_32} % 替换为您的图片文件名
			%				\caption*{d}
		\end{minipage}
	\end{minipage}
	%%%%%%%%%%%%%%%%%%%%%%%
	\begin{minipage}{\textwidth}
		\centering
		\begin{minipage}{0.24\textwidth}
			\centering
			\includegraphics[width=\textwidth]{imgs/Mistral_Infinite_128000_32_32} % 替换为您的图片文件名
			%				\caption*{a}
		\end{minipage}%
		\begin{minipage}{0.24\textwidth}
			\centering
			\includegraphics[width=\textwidth]{imgs/Mistral_Infinite_256000_8_32} % 替换为您的图片文件名
			%				\caption*{b}
		\end{minipage}%
		\begin{minipage}{0.24\textwidth}
			\centering
			\includegraphics[width=\textwidth]{imgs/Mistral_Infinite_256000_16_32} % 替换为您的图片文件名
			%				\caption*{c}
		\end{minipage}%
		\begin{minipage}{0.24\textwidth}
			\centering
			\includegraphics[width=\textwidth]{imgs/Mistral_Infinite_256000_32_32} % 替换为您的图片文件名
			%				\caption*{d}
		\end{minipage}
	\end{minipage}
	%%%%%%%%%%%%%%%%%%%%%%%
	\begin{minipage}{\textwidth}
		\centering
		\begin{minipage}{0.24\textwidth}
			\centering
			\includegraphics[width=\textwidth]{imgs/Mistral_Stream_128000_32_32} % 替换为您的图片文件名
			%				\caption*{a}
		\end{minipage}%
		\begin{minipage}{0.24\textwidth}
			\centering
			\includegraphics[width=\textwidth]{imgs/Mistral_Stream_256000_8_32} % 替换为您的图片文件名
			%				\caption*{b}
		\end{minipage}%
		\begin{minipage}{0.24\textwidth}
			\centering
			\includegraphics[width=\textwidth]{imgs/Mistral_Stream_256000_16_32} % 替换为您的图片文件名
			%				\caption*{c}
		\end{minipage}%
		\begin{minipage}{0.24\textwidth}
			\centering
			\includegraphics[width=\textwidth]{imgs/Mistral_Stream_256000_32_32} % 替换为您的图片文件名
			%				\caption*{d}
		\end{minipage}
	\end{minipage}
	%%%%%%%%%%%%%%%%%%%%%%%
	\begin{minipage}{\textwidth}
		\centering
		\begin{minipage}{0.24\textwidth}
			\centering
			\includegraphics[width=\textwidth]{imgs/Mistral_InfLLM_128000_32_32} % 替换为您的图片文件名
			%				\caption*{a}
		\end{minipage}%
		\begin{minipage}{0.24\textwidth}
			\centering
			\includegraphics[width=\textwidth]{imgs/Mistral_InfLLM_256000_8_32} % 替换为您的图片文件名
			%				\caption*{b}
		\end{minipage}%
		\begin{minipage}{0.24\textwidth}
			\centering
			\includegraphics[width=\textwidth]{imgs/Mistral_InfLLM_256000_16_32} % 替换为您的图片文件名
			%				\caption*{c}
		\end{minipage}%
		\begin{minipage}{0.24\textwidth}
			\centering
			\includegraphics[width=\textwidth]{imgs/Mistral_InfLLM_256000_32_32} % 替换为您的图片文件名
			%				\caption*{d}
		\end{minipage}
	\end{minipage}
	%%%%%%%%%%%%%%%%%%%%%%%
	\begin{minipage}{\textwidth}
		\centering
		\begin{minipage}{0.24\textwidth}
			\centering
			\includegraphics[width=\textwidth]{imgs/Mistral_Ours_128000_32_32} % 替换为您的图片文件名
			%				\caption*{a}
		\end{minipage}%
		\begin{minipage}{0.24\textwidth}
			\centering
			\includegraphics[width=\textwidth]{imgs/Mistral_Ours_256000_8_32} % 替换为您的图片文件名
			%				\caption*{b}
		\end{minipage}%
		\begin{minipage}{0.24\textwidth}
			\centering
			\includegraphics[width=\textwidth]{imgs/Mistral_Ours_256000_16_32} % 替换为您的图片文件名
			%				\caption*{c}
		\end{minipage}%
		\begin{minipage}{0.24\textwidth}
			\centering
			\includegraphics[width=\textwidth]{imgs/Mistral_Ours_256000_32_32} % 替换为您的图片文件名
			%				\caption*{d}
		\end{minipage}
	\end{minipage}
	
	\caption{Results on Counting-Stars for Mistral with different methods.}
	\label{fig:Mistral_counting_star}
\end{figure*}
\restoregeometry
\end{document}