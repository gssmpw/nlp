\documentclass[11pt, a4paper, logo, copyright]{googledeepmind}

\usepackage[authoryear, sort&compress, round]{natbib}



% Recommended, but optional, packages for figures and better typesetting:
\usepackage{microtype}
\usepackage{graphicx}
% \usepackage{subfigure}
\usepackage{booktabs} % for professional tables

% hyperref makes hyperlinks in the resulting PDF.
% If your build breaks (sometimes temporarily if a hyperlink spans a page)
% please comment out the following usepackage line and replace
% \usepackage{icml2025} with \usepackage[nohyperref]{icml2025} above.
%\usepackage{hyperref}


% Attempt to make hyperref and algorithmic work together better:
\newcommand{\theHalgorithm}{\arabic{algorithm}}


% For theorems and such
\usepackage{amsmath}
\usepackage{amssymb}
\usepackage{mathtools}
\usepackage{amsthm}
\usepackage{caption}
\usepackage{subcaption}

\setlength{\parindent}{0cm}


% if you use cleveref..
\usepackage[capitalize,noabbrev]{cleveref}


%%%%%%%%%%%%%%%%%%%%%%%%%%%%%%%%
% THEOREMS
%%%%%%%%%%%%%%%%%%%%%%%%%%%%%%%%
\theoremstyle{plain}
\newtheorem{theorem}{Theorem}[section]
\newtheorem{proposition}[theorem]{Proposition}
\newtheorem{lemma}[theorem]{Lemma}
\newtheorem{corollary}[theorem]{Corollary}
\theoremstyle{definition}
\newtheorem{definition}[theorem]{Definition}
\newtheorem{assumption}[theorem]{Assumption}
\theoremstyle{remark}
\newtheorem{remark}[theorem]{Remark}

\DeclareMathOperator*{\obs}{obs}
\DeclareMathOperator*{\act}{act}
\DeclareMathOperator*{\enc}{enc}
\DeclareMathOperator*{\bott}{bot}
\DeclareMathOperator*{\new}{new}
\DeclareMathOperator*{\old}{old}
\DeclareMathOperator*{\pos}{pos}
\DeclareMathOperator*{\sg}{sg}
\DeclareMathOperator*{\argmin}{argmin}
\DeclareMathOperator*{\argmax}{argmax}
\newcommand{\AD}[1]{\textcolor{blue}{\small{[AD] #1}}}
\newcommand{\JO}[1]{\textcolor{blue}{\small{[JO] #1}}}
\newcommand{\KPM}[1]{\textcolor{blue}{\small{[KPM] #1}}}
\newcommand{\MLG}[1]{\textcolor{blue}{\small{[MLG] #1}}}
\newcommand{\todo}[1]{\textcolor{red}{\small{[TODO] #1}}}
\newcommand{\change}[1]{\textcolor{blue}{#1}}
\DeclareMathOperator*{\Mtrue}{\mathcal{M}_{\text{env}}}
\newcommand{\model}{\mathcal{M}}
\newcommand{\encoder}{\mathcal{E}}
\newcommand{\eat}[1]{} % ignore argument
\newcommand{\btf}{block teacher forcing}
\newcommand{\btfac}{BTF}
\newcommand{\best}{MB-Ladder}


% FROM icml.sty

% Before 2018, \usepackage{times} was in the example TeX, but inevitably
% not everybody did it.
\RequirePackage{times}

% Use fancyhdr package
\RequirePackage{fancyhdr}
\RequirePackage{xcolor} % changed from color to xcolor (2021/11/24)
\RequirePackage{algorithm}
\RequirePackage{algorithmic}
\RequirePackage{natbib}
\RequirePackage{eso-pic} % used by \AddToShipoutPicture
\RequirePackage{forloop}
\RequirePackage{url}

% Information about your document.
\title{Improving Transformer World Models for Data-Efficient RL}

% Can leave this option out if you do not wish to add a corresponding author.
\correspondingauthor{adedieu@google.com, joeortiz@google.com}

% Remove these if they are not needed
% \keywords{Model Based Reinforcement Learning, Background Planning, Transformer World Model}
% \paperurl{arxiv.org/abs/123}

% Use the internally issued paper ID, if there is one
% \reportnumber{001} % Leave blank if n/a

% Assign your own date to the report.
% Can comment out if not needed or leave blank if n/a.
\renewcommand{\today}{2025-02-01}

% Can have as many authors and as many affiliations as needed. Best to indicate joint
% first-authorship as shown below.
\author[*,1]{Antoine Dedieu}
\author[*,1]{Joseph Ortiz}
\author[1]{Xinghua Lou}
\author[1]{Carter Wendelken}
\author[1]{Wolfgang Lehrach}
\author[1]{J. Swaroop Guntupalli}
\author[1]{Miguel Lazaro-Gredilla}
\author[1]{Kevin Murphy}

% Affiliations *must* come after the declaration of \author[]
\affil[*]{Equal contributions}
\affil[1]{Google DeepMind}

\begin{abstract}
We present an approach to model-based RL that achieves a new state of the art performance on the challenging Craftax-classic benchmark, an open-world $2$D survival game that requires agents to exhibit a wide range of general abilities---such as strong generalization, deep exploration, and long-term reasoning. With a series of careful design choices aimed at improving sample efficiency, our MBRL algorithm achieves a reward of $67.42\%$ after only $1$M environment steps, significantly outperforming DreamerV3, which achieves $53.2\%$, and, for the first time, exceeds human performance of $65.0\%$. Our method starts by constructing a SOTA model-free baseline, using a novel policy architecture that combines CNNs and RNNs.
We then add three improvements to the standard MBRL setup: (a) ``Dyna with warmup'', which trains the policy on real and imaginary data, (b) ``nearest neighbor tokenizer'' on image patches, which improves the scheme to create the transformer world model (TWM) inputs, and (c) ``block teacher forcing'', which allows the TWM to reason jointly about the future tokens of the next timestep.
%It then adds a series of improvements to the standard MBRL setup, including: (a) training the policy on real and imaginary data (a method we call ``Dyna with warmup''), (b) improving the visual tokenization scheme used to create the input to the transformer world model (a method we call ``online patch tokenizer''), and (c) speeding up the world model by using block teacher forcing.
\end{abstract}

\begin{document}

\maketitle


\section{Introduction}
\label{sec:introduction}


Reinforcement learning (RL) \citep{sutton2018reinforcement} 
provides a framework for training agents to act in environments so as to maximize their rewards. Online RL algorithms interleave taking actions in the environment---collecting observations and rewards---and updating the policy using the collected experience. 
Online RL algorithms often employ a model-free approach (MFRL), where the agent learns a direct mapping from observations to actions,
but this can require a lot of data to be collected from the environment.
%However, MFRL methods do not leverage any prior over the model of the world to increase their sample efficiency, potentially requiring a large amount of interaction data to exhibit reasonable performance.
Model-based RL (MBRL) aims to reduce the amount of data needed to train the policy
 by also learning a world model (WM), and using this WM to plan ``in imagination".
%Model-based RL (MBRL), on the other hand, learns a world model (WM) alongside the policy, which allows to reduce the amount of real world interaction data collected. 



To evaluate sample-efficient RL algorithms, it is common to use the
Atari-$100$k benchmark \citep{Kaiser2019}. However, the near-deterministic nature of Atari games allows agents to memorize action sequences without demonstrating true generalization \citep{Machado2018}.
In addition, although the benchmark encompasses a variety of  skills (memory, planning, etc), each individual game typically only emphasizes one or two such skills.
To promote the development of agents with broader capabilities, we focus on the Crafter domain \citep{hafner2021benchmarking},
a 2D version of Minecraft that challenges a single agent to master a diverse skill set.
Specifically, we use the  Craftax-classic environment \citep{matthews2024craftax},  a fast, near-replica of Crafter, implemented in JAX \citep{jax2018github}.
Key features of Craftax-classic include: % some of the details below do not apply to the non-classic craftax
(a) procedurally generated stochastic environments (at each episode the agent encounters a new environment sampled from a common distribution); 
(b) partial observability, as the agent only sees a $63 \times 63$ pixel image representing a local view of the agent's environment, plus a visualization of its inventory (see \cref{fig:teaser}[middle]);
and (c) an achievement hierarchy that defines a sparse reward signal, requiring deep and broad exploration.


\begin{figure}[t!]
    \centering
    \begin{tabular}{c}
    \includegraphics[width=.5\linewidth]{figures/benchmark.pdf}
    \end{tabular}
    \hspace{-.5em}
    \begin{tabular}{c}
        \includegraphics[width=.2\linewidth]{figures/craftax.pdf} 
        %\includegraphics[width=.15\linewidth]{figures/tokens.pdf}
    \end{tabular}
    \hspace{-1em}
    \begin{tabular}{c}
        %\includegraphics[width=.15\linewidth]{figures/craftax.pdf} \\
        \includegraphics[width=.21\linewidth]{figures/tokens.pdf}
    \end{tabular}
    \vspace{-.75em}
    \caption{
    [Left]
    Reward on Craftax-classic.
    Our best MBRL and MFRL agents outperform all the previously published MFRL and MBRL results, and for the first time, surpass the reward achieved by a human expert.
    We display published methods which report the reward at 1M steps with horizontal line from 900k to 1M steps. 
    [Middle] The Craftax-classic observation is a $63 \times 63$ pixel image, composed of $9\times9$ patches of $7\times7$ pixels. 
    The observation shows the map around the agent and the agent's health and inventory. Here we have rendered the image at $144 \times 144$ pixels for visibility. 
    [Right] $64$ different patches.
    }
    \label{fig:teaser}
\vspace{-1.em}
\end{figure}



In this paper, we study improvements to MBRL methods,
based on transformer world models (TWM),
in the context of the Craftax-classic environment.
\eat{
In particular, we address three main questions:
(1) What is the form of the world model (WM)?;
(2) How is the WM trained?;
(3) How is the WM used?.
This paper makes a contribution to each one of these questions.
}
We make contributions across
the following three axes:
(a) how the TWM is used (Section \ref{sec:dyna});
(b) the tokenization scheme used to create TWM inputs
(Section \ref{sec:nnt});
(c) and how the TWM is trained (Section \ref{sec:btf}).
Collectively, our improvements result in an agent that,
with only $1$M environment steps,
achieves a Craftax-classic reward of $67.42\%$ and a
score of $27.91\%$,
significantly improving over the previous state of the art (SOTA) reward of $53.20\%$ \citep{hafner2023mastering} and the previous SOTA score of $19.4\%$ \citep{Kauvar2023}\footnote{The score $S$ is given by the geometric mean
of the success rate $s_i$ for each of the
$N=22$ achievements;
this
 puts more weight on occasionally
solving many achievements than on consistently solving
a subset.
More precisely, the score is given by
$S = \exp \left(\frac{1}{N} \sum_{i=1}^N
\ln (1+s_i) \right)-1$,
where $s_i \in [0,100]$ is the success percentage
for achievement $i$
(i.e., fraction of episodes in which
the achievement was obtained at least once).
By contrast, the rewards are just the expected sum of rewards, or in percentage, the arithmetic mean
$R=\frac{1}{N} \sum_{i=1}^N s_i$
(ignoring minor contributions to
the reward based on
 the health of the agent).
 The score and reward are correlated, but are not the same. Unlike some prior work, we report both metrics to make comparisons easier.
}.


\eat{
Our first contribution is to the form 
of the world model.
Following prior work, we 
 use transformers
\citep{vaswani2017attention}
to create a Transformer World Model (TWM).
However, 
since our goal is training agents with visual input,
we need to address the issue of
where the tokens come from.
}

Our first contribution relates to the way the world model is used:
in contrast to recent MBRL methods like IRIS \citep{micheli2022transformers} and DreamerV3 \citep{hafner2023mastering}, which train the policy solely on imagined trajectories (generated by the world model), we train our policy using both imagined rollouts from the world model and real experiences collected in the environment.
This is similar to the original Dyna method \citep{sutton1990integrated}, although this technique has been abandoned in recent work.
In this hybrid regime, we can view the WM as a form
of generative data augmentation
\citep{van2019use}.

Our second contribution addresses the tokenizer which converts between images and tokens that the TWM ingests and outputs.
Most prior work uses a vector quantized variational autoencoder (VQ-VAE, \citealt{van2017neural}), e.g.
IRIS \citep{micheli2022transformers},
DART \citep{agarwal2024learning}.
\eat{
These methods train a CNN to map images $O_t$ to a 
feature map $Z_t$, whose elements $Z_{t}^i$ are then quantized into a set of discrete tokens
$Q_t=\{q_{t}^i\}_{i=1}^L$, where $q_{t}^{i} \in \{1,\ldots,K\}$
is the index of the codebook vector representing
$Z_{t}^i$,
and $i \in \{1,\ldots,L\}$ is the token index.
The tokens $Q_t$ are concatenated into a sequence $Q_{1:T}$ to represent the observations
which, along with the sequence of actions, is used to train the WM.
}
These methods train a CNN to process images into a 
feature map, whose elements are then quantized into discrete tokens, using a codebook. 
The sequence of observation tokens across timesteps is used, along with the actions and rewards, to train the WM.
We propose two improvements to the tokenizer.
First, instead of jointly quantizing the image, we split the image into patches and independently tokenize each patch.
% $p_t^i$, using a simple online greedy VQ method, described in \cref{sec:nnt}.
Second, we replace the VQ-VAE with a simpler nearest-neighbor tokenizer (NNT) for patches. 
% VQ-VAE suffers from a key drawback: the ``meaning" of the tokens changes over time as the CNN and codebook evolve during training.
% However, VQ-VAE generates a set of tokens whose ``meaning" changes over time (since the CNN features change, and the mapping from features to codebook entries change).
Unlike VQ-VAE, NNT ensures that the ``meaning" of each code in the codebook is constant through training, which simplifies the task of learning a reliable WM.

Our third contribution addresses the way the world model is trained.
TWMs are trained by maximizing the log likelihood of the sequence of tokens,
which is typically generated autoregressively both over time and within a timeslice.
We propose an alternative, 
which we call \btf ~(BTF),
that allows TWM to reason jointly about the possible future states of all tokens within a timestep, before sampling them in parallel and independently given the history.
% tokens from the previous timesteps.
With BTF, imagined rollouts for training the policy are both faster to sample and more accurate.
% mitigates autoregressive drift.


Our final contributions are some minor architectural changes to the MFRL baseline upon which our MBRL
approach is based. These changes are still significant, resulting in a simple MFRL method that is much faster than Dreamer V3 and yet obtains a much better average reward and score.

Our improvements are complementary to each other, and  can be combined into a  ``ladder of improvements"---similar to the ``Rainbow" paper's
\citep{Hessel2018} series of improvements on top of model-free DQN agents. 

%In particular, we start with the previous SOTA MFRL approach \citep{moon2024discovering}, which achieves a score of $15.60$ and a reward of $46.91\%$, and make some minor architectural modifications, described in \cref{sec:MFRL}, increasing the score to $16.77$ and the reward to $55.50\%$. This result is notable since it beats the considerably more complex (and much slower) Dreamer V3 method, which obtains a score of $14.5$ and a reward of $53.20\%$. It also significantly beats other model-based methods, such as IRIS \citep{micheli2022transformers} (reward of $25.0\%$) and $\Delta$-IRIS \citep{micheli2024efficient}(reward of $35.0\%$).\footnote{ This is consistent with some results on Atari-100k benchmark, which shows that well-tuned model-free methods, such as the  value-based  ``Bigger, Better, Faster" method \citep{Schwarzer2023}, can beat more sophisticated model-based methods.}

\eat{
It is also consistent
with the results in \citep{Jesson2024},
which  shows that minor architectural
changes to the policy,
trained with PPO without a WM,
can lead to big gains on the ProcGen benchmark.
} %




\section{Related Work}
\label{sec:related}

In this section, we discuss related work in MBRL --- see e.g. \citet{Moerland2023,murphy2024reinforcement, awesome_mbrl} 
% and \url{https://github.com/opendilab/awesome-model-based-RL}
for more comprehensive reviews.
We can broadly divide MBRL along two axes.
The first axis is whether the world model (WM) is used for background planning
(where it helps train the policy by generating imagined trajectories),
or decision-time planning (where it is used for lookahead search at inference time). 
The second axis is whether the WM is a generative model of the observation space (potentially via a latent bottleneck) or whether is a latent-only model trained using a self-prediction loss
(which is not sufficient to generate full observations).

Regarding the first axis, prominent examples of decision-time planning methods that leverage a WM include MuZero \citep{schrittwieser2020mastering} and EfficientZero  \citep{Ye2021}, which use Monte-Carlo tree search over a discrete action space, as well as TD-MPC2 \citep{Hansen2024}, which uses the cross-entropy method over a continuous action space.
Although some studies have shown that decision-time planning can sometimes be better
than background planning \citep{Alver2024}, it is much slower, especially with large WMs such as transformers, since it requires rolling out future hypothetical trajectories at each decision-making step.
Therefore in this paper, we focus on background planning (BP).
Background planning originates from Dyna \citep{sutton1990integrated}, which focused on tabular Q-learning.
Since then, many papers have combined the idea with deep RL methods: World Models \citep{ha2018recurrent},
Dreamer agents  \citep{Hafner2020, hafner2020mastering,hafner2023mastering},
SimPLe \citep{Kaiser2019},
IRIS \citep{micheli2022transformers},
$\Delta$-IRIS \citep{micheli2024efficient},
Diamond \citep{alonso2024diffusion},
DART \citep{agarwal2024learning}, etc.

Regarding the second axis, 
many methods fit generative WMs of the
observations (images) using a 
model with low-dimensional latent variables,
either continuous (as in a VAE)
or discrete (as in a VQ-VAE).
This includes our method and most background planning methods above
\footnote{
%
A notable exception is 
Diamond \citep{alonso2024diffusion},
which fits a diffusion world model
% of the form $p(O'|O,a)$
directly in pixel space,
rather than learning a latent WM. %and an autoencoder.
% $p(Q'|Q,a)$ and then decoding to image space
% via $p(O'|Q')$.
}.
In contrast, other methods fit non-generative WMs, which are trained using self-prediction loss---see \citet{Ni2024} for a detailed discussion.
Non-generative WMs are more lightweight and therefore well-suited to decision-time planning with its large number of WM calls at every decision-making step.
However,
generative WMs are generally preferred for background planning, since it is easy to combine real and imaginary data for policy learning,
as we show below.

 
In terms of the WM architecture, many state-of-the-art models use transformers, e.g. 
IRIS \citep{micheli2022transformers},
$\Delta$-IRIS \citep{micheli2024efficient},
DART \citep{agarwal2024learning}.
Notable exceptions are DreamerV2/3  \citep{hafner2020mastering,hafner2023mastering}, which use recurrent state space models,
although improved transformer variants have been proposed 
\citep{robine2023transformer,zhang2024storm,chen2202transdreamer}.

\section{Methods}
\label{sec:methods}

% Here, we describe the components
% of our system, each of which improves
% performance, as we show in \cref{sec:results}.

\subsection{MFRL Baseline}
\label{sec:MFRL}

Our starting point is the previous SOTA
MFRL approach which was proposed as a baseline in
\citet{moon2024discovering}\footnote{%
The authors' main method uses external knowledge about the achievement hierarchy of Crafter, so cannot be compared with other general methods. We use their baseline instead.
}.
This method achieves a reward of $46.91\%$ and a score of $15.60\%$ after $1$M environment steps.
This approach trains a stateless CNN policy without frame stacking
using the PPO method \citep{schulman2017proximal}, and
adds an entropy penalty to ensure sufficient exploration.
The CNN used is a modification of the Impala
ResNet \citep{Espeholt2018}.


\subsection{MFRL Improvements}

We improve on this MFRL baseline
by both increasing the model size and adding a RNN (specifically a GRU) to give the policy memory.
Interestingly, we find that naively increasing the model size harms performance, while combining a larger model with a carefully designed RNN helps (see Section \ref{sec:ablations}). 
For the RNN,
we find it crucial to ensure
the hidden state is low-dimensional,
so that the memory is forced to focus on the relevant
bits of the past that cannot be extracted
from the current image.
We concatenate the GRU output to the image
embedding, and then pass this to the actor and critic networks,
rather than directly passing the GRU output. Algorithm \ref{algo:ac_network}, Appendix \ref{ap:mfrl_agent}, presents a pseudocode for our MFRL agent.


With these architectural changes,
we increase the reward to $55.49\%$ and the score to $16.77\%$.
This result is notable since our MFRL agent beats the considerably more complex (and much slower) DreamerV3 agent, which obtains a reward of $53.20\%$ and a score of $14.5$. 
It also beats other MBRL methods, such as IRIS \citep{micheli2022transformers} (reward of $25.0\%$) and $\Delta$-IRIS \citep{micheli2024efficient}
\footnote{
This is consistent with results on Atari-100k, which show that well-tuned model-free methods, such as BBF \citep{Schwarzer2023}, can beat more sophisticated model-based methods.} (reward of $35.0\%$). 
In addition, our MFRL agent only takes
$15$ minutes to train for $1$M environment
steps on one A100 GPU.
%whereas our MBRL takes about $12$ hours, and Dreamer V3 takes about $35$ hours. \todo{Making the model larger harms performance, but when combined with the RNN, this improves performance. Reference plots in results.}

% \begin{algorithm}[h]
% \caption{
% Rollout with actor-critic network.
% \AD{should we add reset if done to the algorithm?}
% \JO{Remove this psuedocode?}
% }
% \label{algo:mfrl_collect}
% \begin{algorithmic}
% \STATE {\bfseries Input:} obs. $O_1$, AC parameters $\Phi$, horizon $T$ \\
% An environment transition function $\Mtrue$
% \STATE {\bfseries Output:} An environment rollout $(O_1, a_1, r_1 \ldots O_t)$ \\
% as well as a sequence of values $(v_1, \ldots v_{T-1})$
% \STATE {\bfseries Initialize:} hidden state $h_0=0$
% \FOR{$t=1$ {\bfseries to} $T$}
% \STATE \textit{// Run the actor-critic network}
% \STATE $z_t = \text{ImpalaCNN}_{\Phi}(O_t)$
% \STATE $h_t, y_t = \text{RNN}_{\Phi}( [h_{t-1}, z_t])$
% \STATE $a_t \sim \pi_{\Phi}([y_t, z_t])$
% \STATE $V_t = V_{\Phi}([y_t, z_t])$
% \STATE \textit{// Collect reward and next observation}
% \STATE $r_t, O_{t+1} = \Mtrue(O_t, a_t)$
% \ENDFOR
% \end{algorithmic}
% \end{algorithm}


\subsection{MBRL baseline}
\label{sec:IRIS}
\label{sec:MBRLbaseline}

We now describe our MBRL baseline,
which combines our MFRL baseline above with a transformer world model (TWM)---as in
IRIS \citep{micheli2022transformers}. Following IRIS, our MBRL baseline uses a VQ-VAE, which quantizes the
$8 \times 8$ feature map $Z_t$ of a CNN
to create a set of latent codes,
$(q_t^1,\ldots,q_t^L) = \text{enc}(O_t)$,
where $L=64$, $q_t^i \in \{1,\ldots,K\}$
is a discrete code,
and $K=512$ is the size of the codebook.
These codes are then passed to a TWM, which is trained using
teacher forcing---see \cref{eqn:lossAR} below. 
Our MBRL baseline achieves a reward of $31.93\%$, and improves over the reported results of IRIS, which reaches $25.0\%$. 

Although these MBRL baselines leverage recent advances in generative world modeling, they are largely outperformed by our best MFRL agent. This motivates us to enhance our MBRL agent, which we explore in the following sections.

%Next, we improve this baseline by replacing the stateless CNN policy networks with our CNN + GRU described in \cref{sec:MFRL}. This increases the reward to . 


\subsection{MBRL using Dyna with warmup}
\label{sec:dyna}
\label{sec:warmup}

As discussed in \cref{sec:introduction}, we propose to train our MBRL agent on a mix of real trajectories (from the environment) and imaginary trajectories (from the TWM), similar to Dyna \citep{sutton1990integrated}.
\cref{algo:MBRL} presents the pseudocode for our MBRL approach.
Specifically, unlike many other recent MBRL methods \citep{Ha2018, micheli2022transformers, micheli2024efficient, hafner2020mastering, hafner2023mastering} which train their policies exclusively using world model rollouts (Step 4), we include Step 2 which updates the policy with real trajectories.
Note that, if we remove Steps 3 and 4 in Algorithm \ref{algo:MBRL}, the approach reduces to MFRL.
The function $\text{rollout}(O_1, \pi_{\Phi},  T, \model)$ returns a trajectory of length $T$ generated by rolling out the policy $\pi_{\Phi}$ from the initial state $O_1$ in either the true environment $\Mtrue$ or the world model $\mathcal{M}_{\Theta}$.
A trajectory contains collected observations, actions and rewards during the rollout $\tau = (O_{1:T+1}, a_{1:T}, r_{1:T})$.
Algorithm \ref{algo:rollout} in Appendix \ref{ap:mbrl_agent} details the rollout procedure.
We discuss other design choices below. 

\begin{algorithm}[!htbp]
\caption{MBRL agent. See Appendix \ref{ap:mbrl_agent} for details.}
\label{algo:MBRL}
\begin{algorithmic}
\STATE {\bfseries Input:} number of environments $N_{\text{env}}$,
\newline
environment dynamics $\Mtrue$,
\newline
rollout horizon for environment $T_{\text{env}}$ and for TWM $T_{\text{WM}}$,
\newline
background planning starting step $T_{\text{BP}}$,
\newline
total number of environment steps $T_{\text{total}}$,
\newline
number of TWM updates $N^{\text{iters}}_{\text{WM}}$ and policy updates $N^{\text{iters}}_{\text{AC}}$
\medskip
\STATE {\bfseries Initialize:} observations $O^n_1 \sim \Mtrue ~\text{for}~ \small{n=1:N}$,
\newline
data buffer $\mathcal{D}=\emptyset$, 
\newline
TWM model $\mathcal{M}$ and parameters $\Theta$, 
\newline
AC model $\pi$ and parameters $\Phi$,
\newline
number of environment steps $t=0$.
\smallskip
\REPEAT
\STATE \textit{// 1. Collect data from environment}
\STATE $\tau^n_{\text{env}} = \text{rollout}
(O_1^n, \pi_{\Phi}, T_{\text{env}}, \Mtrue),~~n=1:N_{\text{env}}$
\STATE $\mathcal{D} = \mathcal{D} \cup \tau^{1:N}_{\text{env}} ~;~ O^{1:N}_1 = \tau^{1:N}_{\text{env}}[-1] ~;~ t += N_{\text{env}}T$
\medskip
\STATE \textit{// 2. Update policy on environment data}
\STATE $\Phi=\text{PPO-update-policy}(\Phi,\tau_{\text{env}}^{1:N})$\\
\medskip
\STATE \textit{// 3. Update world model}
\FOR{$\text{it}=1$ {\bfseries to} $N^{\text{iters}}_{\text{WM}}$}
\STATE  $\tau_{\text{replay}} ^{n} =\text{sample-trajectory}(\mathcal{D}, T_{\text{WM}}), ~~n=1:N_{\text{env}}$ \\
\STATE  $\Theta = \text{update-world-model}(\Theta, \tau_{\text{replay}}^{1:N_{\text{env}}})$\\
\ENDFOR
\medskip
\STATE \textit{// 4. Update policy on imagined data}
\IF{ $t \ge T_{\text{BP}}$}
\FOR{$\text{it}=1$ {\bfseries to} $N^{\text{iters}}_{\text{AC}}$}
\STATE $O_1^n = \text{sample-obs}(\mathcal{D}), ~~n=1:N_{\text{env}}$ 
\STATE $\small{\tau_{\text{WM}}^{n}=\text{rollout}
(O_1^n,\pi_{\Phi}, T_{\text{WM}},\mathcal{M}_{\Theta}), ~n=1:N_{\text{env}}}$ 
\STATE $\Phi=\text{PPO-update-policy}(\Phi,\tau_{\text{WM}}^{1:N_{\text{env}}})$
\ENDFOR
\ENDIF
\UNTIL $t \ge T_{\text{total}}$
\end{algorithmic}
\end{algorithm}

\textbf{PPO.}
Since PPO \citep{schulman2017proximal} is an on-policy algorithm, trajectories should be used for policy updates immediately after they are collected or generated. 
For this reason, policy updates with real trajectories take place in Step 2 immediately after the data is collected. 
An alternative approach is to use an off-policy algorithm and mix real and imaginary data into the policy updates in Step 4, hence removing Step 2. We leave this direction as future work. 

\textbf{Rollout horizon.}
We set $T_{\text{WM}} \ll  T_{\text{env}}$,
to avoid the problem of compounding errors
due to model imperfections \citep{Lambert2022}.
However, we find it beneficial to use
$T_{\text{WM}} \gg 1$,
consistent with
\citet{Holland2018,van2019use},
who observed that the
Dyna approach with $T_{\text{WM}}=1$
is no better than
MFRL with experience replay.

\textbf{Multiple updates.} Following IRIS, we update TWM $N^{\text{iters}}_{\text{WM}}$ times and the policy on imagined trajectories $N^{\text{iters}}_{\text{AC}}$ times.

\textbf{Warmup.}
When mixing imaginary trajectories with real ones, we need to ensure the WM is sufficiently accurate so that it does not ``pollute" the replay buffer, thus harming policy learning. 
Consequently, we only begin training the policy
on imaginary trajectories after the agent has interacted with the environment for $T_{\text{BP}}$ steps, which ensures it has seen enough data to learn a reliable WM. We call this technique ``Dyna with warmup''. In \cref{sec:ablations}, we show that removing this warmup, and using $T_{\text{BP}}=0$, drops the reward dramatically, from $67.42\%$ to $33.54\%$. We additionally show that removing the Dyna method (and only training the policy in imagination) drops the reward to $55.02\%$. 



\eat{
In \cref{sec:results}, we show the advantage of using Dyna rather than only training the policy
in imagination,
and the advantage of waiting until the WM is warmed up
rather than using it immediately.
}

\subsection{Patch nearest-neighbor tokenizer}
\label{sec:nnt}
\label{sec:patches}

Many MBRL methods based on TWMs use a VQ-VAE to map between images and tokens.
In this section, we describe our alternative which leverages a property of Craftax-classic: each observation is composed of $9\times9$ patches of size $7 \times 7$ each (see \cref{fig:teaser}[middle]). Hence we propose to (a) factorize the tokenizer by patches and (b) use a simpler nearest-neighbor style approach to tokenize the patches.

\paragraph{Patch factorization.}
Unlike prior methods which process the full image $O$ into tokens $(q^1,\ldots,q^L) = \text{enc}(O)$, we first divide $O$ into $L$ non-overlapping patches
$(p^1, \ldots, p^L)$ which are independently encoded into $L$ tokens:
\begin{equation*}
    (q^i,\ldots, q^L) = (\text{enc}(p^1), \ldots, \text{enc}(p^L))
    ~.
\end{equation*}
To convert the discrete tokens back to pixel
space, we just decode each token independently into patches, and rearrange to form a full image:
\begin{equation*}
    (\hat{p}^1, \ldots, \hat{p}^L) = 
    (\text{dec}(q^1), \ldots, \text{dec}(q^L))
    ~.
\end{equation*}
Factorizing the VQ-VAE on the $L=81$ patches of each observation boosts performance from $43.36\%$ to $58.92\%$.

\paragraph{Nearest-neighbor tokenizer.}
On top of patch factorization, we propose a simpler nearest-neighbor tokenizer (NNT) to replace the VQ-VAE.
The encoding operation for each patch $p \in [0, 1]^{h\times w\times3}$
is similar to a nearest neighbor classifier w.r.t the codebook.
% as defined in Equation \ref{eqn:opt}.
The difference is that, if the nearest neighbor
is too far away, 
% (beyond a threshold distance of $\tau$)
we add a new code equal to $p$ to the codebook.
%similar to a greedy online version of VQ.
More precisely, let us denote  $\mathcal{C}_{\text{NN}}=\{e_1, \ldots,e_K\}$ the current codebook,
consisting of $K$ codes $e_i \in [0, 1]^{h\times w\times3}$, 
and $\tau$ a threshold on the Euclidean distance. 
The NNT encoder is defined as:
\begin{equation}
    \small
    q = \text{enc}(p) = 
    \begin{cases}
    \begin{alignedat}{2}
        &\argmin_{1\le i \le K} \| p -  e_i\|_2^2 &&\quad\text{if $\min_{1\le i \le K} \| p - e_{i} \|_2^2 \le \tau$} \\
        &K+1 &&\quad\text{otherwise.}
    \end{alignedat}
    \end{cases}
    \label{eqn:opt}
\end{equation}
The codebook can be thought of as a
greedy approximation to the coreset of 
the patches seen so far
\citep{Mirzasoleiman2020}. 
To decode patches, we simply return the code associated with the codebook index, i.e. $\text{dec}(q^i)=e_{q^i}$.

A key benefit of NNT is that once codebook entries are added, they are never updated.
A static yet growing codebook makes the target distribution for the TWM stationary, greatly simplifying online learning for the TWM.
In contrast, the VQ-VAE codebook is continually updated, meaning the TWM must learn from a non-stationary distribution, which results in a worse WM.
Indeed, we show in \cref{sec:ladder} that with patch factorization, and when $h=w=7$---meaning that the patches are aligned with the observation---replacing the VQ-VAE with NNT boosts the agent's reward from $58.92\%$ to $64.96\%$. 
Figure \ref{fig:teaser}[right] shows an example of the first 64 code patches extracted by our NNT.

The main disadvantages of our approach are that (a) patch tokenization can be sensitive to the patch size (see Figure \ref{fig:ablation}[left]),
and (b) NNT may create a large codebook if there is a lot
of appearance variation within patches.
% (depending on the threshold $\tau$)
In Craftax-classic, these problems are not very severe due to the grid structure of the game and limited sprite vocabulary (although continuous variations exist due to lighting and texture randomness).


%Furthermore each patch is derived from a fixed vocabulary of "sprites"---although there is continuous variation due to lighting changes (between day and night), and random variations in the texture of terrain regions such as grass and water).



\subsection{Block teacher forcing}
\label{sec:btf}


\begin{figure}[h!]
    \centering
    \begin{tabular}{c}
        \includegraphics[width=.45\linewidth]{figures/btf_part1.pdf} 
    \end{tabular}
    %\hspace{-1em}
    \begin{tabular}{c}
        \includegraphics[width=.47\linewidth]{figures/btf_part2.pdf}
    \end{tabular}
    \caption{
    Approaches for TWM training with $L=2$, $T=2$.
    $q_t^{\ell}$ denotes token $\ell$ of timestep $t$. Tokens in the same timestep have the same color. 
    We exclude action tokens for simplicity.
    [Left] Usual autoregressive model training with teacher forcing. 
    [Right] Block teacher forcing predicts token $q_{t+1}^{\ell}$ from input token $q_{t}^{\ell}$ with block causal attention.
    }
    % \vspace{-.75em}
    \label{fig:btf}
\end{figure}

% \begin{figure}[h!]
%     \centering
%     \includegraphics[width=.9\linewidth]{figures/btf.pdf}
%     \caption{
%     Approaches for TWM training with $L=2$, $T=2$.
%     $q_t^{\ell}$ denotes token $\ell$ of timestep $t$. Tokens in the same timestep have the same color. 
%     We exclude action tokens for simplicity.
%     [Top] Usual autoregressive model training with teacher forcing. 
%     [Bottom] Block teacher forcing predicts token $q_{t+1}^{\ell}$ from input token $q_{t}^{\ell}$ with block causal attention.
%     }
%     \vspace{-1em}
%     \label{fig:btf}
% \end{figure}


Transformer WMs are typically trained by teacher forcing which maximizes
the log likelihood of the token sequence generated autoregressively over time
and within a timeslice:
\vspace{-.75em}
\begin{equation}
\small
    \mathcal{L}_{\text{TF}} = 
    \log \prod_{t=1}^{T} \prod_{i=1}^L \mathcal{L}_t^i ~,~~~~
    \mathcal{L}_t^i
    =
    p(q_{t+1}^i | q_{1:t}^{1:L},  q_{t+1}^{1:i-1}, a_{1:t})
    \label{eqn:lossAR}
\end{equation}
We propose a more effective alternative, which we call
\btf~(\btfac).
BTF modifies both the supervision and the attention of the TWM. Given the tokens from the previous timesteps,
BTF independently predicts all the latent tokens at the next timestep, removing the conditioning on previously generated tokens from the current step:
\vspace{-.5em}
\begin{equation}
\small
    \mathcal{L}_{\text{BTF}} = 
    \log \prod_{t=1}^{T} \prod_{i=1}^L \mathcal{\tilde{L}}_t^i ~,~~~~
    \mathcal{\tilde{L}}_t^i
    =
    p(q_{t+1}^i | q_{1:t}^{1:L}, a_{1:t})
    \label{eqn:lossBTF}
\vspace{-.4em}
\end{equation}
Importantly BTF uses a block causal attention pattern (see Figure \ref{fig:btf}), in which tokens within the same timeslice are decoded in-parallel in a single forward pass.
This attention structure allows the model to reason jointly about the possible future states of all tokens within a timestep, before the tokens are ultimately sampled with independent readouts.
% \citep{bachmann24a}. 
%\MLG{Unclear. Learning happens to make it deterministic, or is it deterministic by design, so that it can't capture stochastic environments? What about unobserved patches that a appear from an edge, shouldn't it be a stochastic prediction?}
This property mitigates autoregressive drift. As a result, we find that BTF returns more accurate TWMs than fully AR approaches.
Overall, adding BTF increases the reward from $64.96\%$ to $67.42\%$, leading to our best MBRL agent.
% Even though this ignores correlation between 
% the symbols within a time slice (a "naive Bayes" assumption), it significantly accelerates world model training and rollouts, which are the main bottlenecks in MBRL.
In addition, we find that BTF is twice as fast,
even though in theory,
%In theory, 
when using key-value caching, BTF and AR both have complexity $\mathcal{O}(L^2 T)$ for generating all the $L$ tokens at one timestep, and $\mathcal{O}(L^2 T^2)$ for generating the entire rollout.
%However, in practice we find that BTF speeds up training by a factor of two for $L=81$ and $T=20$. 

\eat{
This also seems to slightly
improve the accuracy of the WM and hence the agent's performance: removing \btfac~ drops
the reward from 14.83 to 14.29,
as we show in \cref{sec:results}.
}
\begin{table*}[btp]
\caption{Results on Craftax-classic after 1M environment interactions. 
* denotes results on Crafter,
which may not exactly match Craftax-classic.
%\href{https://github.com/danijar/crafter?tab=readme-ov-file}{Crafter leaderboard}.
%$\dagger$ denotes results from the relevant paper.
--- means unknown.
\textdagger denotes the reported timings on a single A100 GPU.
Our DreamerV3 results are based
on the code from the author,
but differ slightly from the reported
number, perhaps due to 
hyperparameter discrepancies.
% discrepancies between Crafter and Craftax.
IRIS and $\Delta$-IRIS do not report standard errors for the score.
}
\label{tab:best_scores}
\small
\vspace{-.75em}
\centering
    \begin{tabular}{ccccc}
    \toprule
    Method & Parameters & Reward (\%) & Score (\%) & Time (min) \\
    \midrule
    Human Expert & NA & $*65.0 \pm 10.5$ & $*50.5 \pm 6.8$  & NA \\
    \midrule
    %M0: Baseline & $56.4\text{M}$ & $29.45 \pm 1.47$ & $4.43 \pm 0.36$ & $833$ \\
    M1: Baseline
    & $60.0\text{M}$ & $31.93 \pm 2.22$ & $4.98 \pm 0.50$ & $560$ \\
    M2: M1 + Dyna
    & $60.0$M & $43.36 \pm 1.84$ & $8.85 \pm 0.63$ & $563$ \\
    M3: M2 + patches
    & $56.6$M & $58.92 \pm 1.03$ & $19.36 \pm 1.42$ & $746$ \\
    M4: M3 + NNT
    & $58.5$M & $64.96 \pm 1.13$ & $25.55 \pm 0.86$  & $1328$ \\
    M5: M4 + BTF. Our best MBRL
    & $58.5$M & $\mathbf{67.42} \pm 0.55$ & $\mathbf{27.91} \pm 0.63$ & $759$ \\
    \midrule
    Previous best MFRL \citep{moon2024discovering} 
    & $4.0\text{M}$
    & $*46.91 \pm 2.41$ 
    & $*15.60 \pm 1.66$
    & ---\\
    Previous best MFRL (our implementation) & $4.0\text{M}$ & $47.40 \pm 0.58$ & $10.71 \pm 0.29$ & $26$ \\
    Our best MFRL & $55.6$M & $55.49 \pm 1.33$  & $16.77 \pm 1.11$ & $15$\\
    \midrule
    %DreamerV3 & 201M & $10.92 \pm 0.53$ & 14.77 $\pm$ 1.42 \\
    DreamerV3 \citep{hafner2023mastering} & $201$M  & $*53.2 \pm 8.$ & $*14.5 \pm 1.6$ & --- \\
    %\\ DreamerV3 XL \citep{micheli2024efficient} &  & $9.2 \pm 0.3$
    Our DreamerV3 & $201$M & $47.18 \pm 3.88$& --- & $2100$ \\
    IRIS \citep{micheli2022transformers} 
    & $48$M & $*25.0 \pm 3.2$  & $*6.66$ & \textdagger$8330$  \\
    $\Delta$-IRIS \citep{micheli2024efficient} & 25M & $*35.0 \pm 3.2$  & $*9.30$  & \textdagger$833$  \\
    Curious Replay \citep{Kauvar2023} & --- & --- & $*19.4 \pm 1.6$ & ---- \\
    \bottomrule
    \end{tabular}
%\vspace{-.75em}
\end{table*}


\section{Results}
\label{sec:results}

In this section, we report our experimental
results on the Craftax-classic benchmark.  Each experiment is run on $8$ H100 GPUs.
All methods are compared after interacting with the environment for
$T_{\text{total}}=1$M steps.
All the methods collect trajectories of length $T_{\text{env}}=96$ in $N_{\text{env}}=48$ environment (in parallel).
For MBRL methods, the imaginary rollouts
are of length $T_{\text{WM}}=20$,
and we start generating these (for policy training) 
after $T_{\text{BP}} = 200\text{k}$ 
environment steps. We update the TWM $N^{\text{iters}}_{\text{WM}}=500$ times and the policy $N^{\text{iters}}_{\text{AC}}=150$ times.
For all metrics, we report the mean and standard error over $10$ seeds as $x(\pm y)$.


\subsection{Climbing up the MBRL ladder}\label{sec:ladder}

First, we report the normalized reward (the reward divided by the maximum reward of $22$) for a series of agents that progressively climb our ``MBRL ladder" of improvements in \cref{sec:methods}.
\cref{fig:main} show the reward vs. the number of environment steps for the following methods, which we detail in Appendix \ref{ap:mbrl_modules}:
%$\bullet$ \textbf{M0: Baseline}. Our baseline MBRL method, described in \cref{sec:MBRLbaseline}, reaches $29.45\%$ reward, and slightly improves over IRIS which gets $25.0\%$.
\smallskip
\newline
$\bullet$ \textbf{M1: Baseline}. Our baseline MBRL agent, described in \cref{sec:MBRLbaseline}, reaches a reward of $31.93\%$, and improves over IRIS, which gets $25.0\%$.
\smallskip
\newline
$\bullet$ \textbf{M2: M1 + Dyna}. Training the policy on both (real) environment and (imagined) TWM trajectories, as described in Section \ref{sec:dyna}, increases the reward to $43.36\%$.
\smallskip
\newline
$\bullet$ \textbf{M3: M2 + patches}. 
Factorizing the VQ-VAE over the $L=81$ observation patches, as presented in Section \ref{sec:nnt}, increases the reward to $58.92\%$.
\smallskip
\newline
$\bullet$ \textbf{M4: M3 + NNT}.
With patch factorization, replacing the VQ-VAE with
NNT, as presented in Section \ref{sec:nnt},
further boosts the reward to $64.96\%$.
\smallskip
\newline
$\bullet$ \textbf{M5: M4 + BTF. Our best MBRL}: Finally, incorporating BTF, as described in  \cref{sec:btf}, leads to our best agent. It achieves a reward of $67.42\% (\pm 0.55)$, while BTF reduces the training time by a factor of two.

As in IRIS \citep{micheli2022transformers}, methods M1-3 use a codebook size of $512$. 
For M4 and our best MBRL, which use NNT, we found it critical to use a larger codebook size of $K=4096$ and a threshold of $\tau=0.75$. 
Interestingly, when training in imagination begins (at step $T_{\text{BP}}=200\text{k}$), there is a temporary drop in performance as the TWM rollouts do not initially match the true environment dynamics, resulting in a distribution shift for the policy. 
%See \cref{sec:ablations} for an analysis of the effect of varying $T_{\text{BP}}$.
% We found that all other models with VQ-VAE did not benefit from larger codebook sizes.


% \begin{figure}[h!]
%     \centering
%     \includegraphics[width=.95\linewidth]{figures/ladder.pdf}
%     \vspace{-1.25em}
%     \caption{
%     The ladder of improvements presented in Section \ref{sec:methods} progressively transforms our baseline MBRL agent into a state-of-the-art method on Craftax-classic, reaching a reward of $67.42 (\pm 0.55)$ (averaged over $10$ seeds) after $1\text{M}$ environment steps.
%     Training in imagination
%     starts at step 200k, indicated
%     by the dotted vertical line.
%     % [Right] Our best MBRL and MFRL agents outperform all the previously published MFRL and MBRL results.
%     }
%     \label{fig:main}
% \vspace{-1.75em}
% \end{figure}


\begin{figure}[h!]
    \begin{minipage}{0.48\textwidth} % Adjust width as needed
    \centering
    \includegraphics[width=.95\linewidth]{figures/ladder.pdf}
    \vspace{-0.5em}
    \caption{
    The ladder of improvements presented in Section \ref{sec:methods} progressively transforms our baseline MBRL agent into a state-of-the-art method on Craftax-classic. %reaching a reward of $67.42 (\pm 0.55)$ (averaged over $10$ seeds) after $1\text{M}$ environment steps.
    Training in imagination
    starts at step 200k, indicated
    by the dotted vertical line.
    }
    \label{fig:main}
    \end{minipage}
    \hspace{0.4em}
    \begin{minipage}{0.48\textwidth} 
    \caption{Ablations results on Craftax-classic after 1M environment interactions.}
    \label{tab:ablation}
    \centering
    \resizebox{0.9\textwidth}{!}{
    \begin{tabular}{ccc}
    \toprule
    Method & Reward $(\%)$ & Score $(\%)$ \\
    \midrule
    Best MBRL  & $\mathbf{67.42} \pm 0.55$ &
    $\mathbf{27.91} \pm 0.63$ \\
    \midrule
    $5\times5$ quantized  & $57.28 \pm 1.14$ & $18.26 \pm 1.18$ \\
    $9\times9$ quantized  & $45.55 \pm 0.88$ & $10.12 \pm 0.40$ \\
    $7\times7$ continuous & $21.20 \pm 0.55$ & $2.43 \pm 0.09$ \\
    \midrule
    Remove Dyna & $55.02 \pm 5.34$ & $18.79 \pm 2.14$ \\
    Remove NNT  & $60.66 \pm 1.38$ & $21.79 \pm 1.33$ \\
    Remove NNT \& patches & $45.86 \pm 1.42$ & $10.36 \pm 0.69$ \\
    Remove BTF  & $64.96 \pm 1.13$ & $25.55 \pm 0.86$ \\
    %Remove RNN  & $42.08 \pm 1.35$ & $8.59 \pm 0.57$ \\
    %Moon MFRL & $45.24 \pm 1.23$ & $10.06 \pm 0.61$ \\
    \midrule
    Use $T_{\text{BP}}=0$ & $33.54 \pm 10.09$ & $12.86 \pm 4.05$  \\
    \midrule
    \midrule
    Best MFRL  & $55.49 \pm 1.33$ & $16.77\pm 1.11$ \\
    Remove RNN & $41.82 \pm 0.97$ & $8.33 \pm 0.44$ \\
    Smaller model  & $51.35 \pm 0.80$ & $12.93 \pm 0.56$ \\
    \bottomrule
    \end{tabular}
    }
    \end{minipage}
% \vspace{-.25em}
\end{figure}




\subsection{Comparison to existing methods}
\label{sec:cmp}

Figure \ref{fig:teaser}[left]
compares the performance of our best MBRL and MFRL agents against various previous methods. 
See also \cref{fig:score} in Appendix \ref{sec:appendix_score}
for a plot of the score,
and  \cref{tab:best_scores}
for a detailed numerical comparison
of the final performance.
% We observe the following.
First, we observe that our best MFRL agent outperforms almost
all of the previously published MFRL and MBRL results,
reaching a reward of $55.49\%$
and a score of $16.77\%$\footnote{
%
The only exception is Curious Replay \citep{Kauvar2023}, which builds on DreamerV3 with prioritized experience replay (PER)
% ~\citep{Schaul2016} 
to train the WM.
% and improves the score of DreamerV3 from 14.5\% to $19.4\%$
% The vanilla DreamerV3 method
% uses  uniform random sampling
% from the replay buffer,
% and achieves a reward of 11.7
% and a score of 14.5\%.
However, 
% the detailed breakdown of the results in \citep{Kauvar2023} show that 
PER is only better
% than DreamerV3's uniform sampling
on a few achievements;
this improves the score but not the reward.
}.
Second, our best MBRL agent
achieves a new SOTA  reward of $67.42\%$
and a score of $27.91\%$.
This marks the first agent to surpass human-level reward, derived from 100 episodes played by 5 human expert players \citep{hafner2021benchmarking}.
Note that although we achieve superhuman reward, our score is significantly below that of a human expert.


\subsection{Ablation studies}
\label{sec:ablations}

We conduct ablation studies to assess the importance of several components of our proposed MBRL agent.
Results are presented in Figure \ref{fig:ablation} and Table \ref{tab:ablation}.

\paragraph{Impact of patch size.} We investigate the sensitivity of our approach to the patch size used by NNT. While our best results are achieved when the tokenizer uses the oracle-provided ground truth patch size of $7\times 7$, Figure \ref{fig:ablation}[left] shows that performance remains competitive when using smaller ($5\times 5$) or larger ($9\times 9$) patches.

\paragraph{The necessity of quantizing.}
Figure \ref{fig:ablation}[left] shows that, when the $7\times7$ patches are not quantized, but instead the TWM is trained to reconstruct the continuous $7\times7$ patches, MBRL performance collapses. This is consistent with findings in DreamerV2 \citep{hafner2021benchmarking}, which highlight that quantization is critical for learning an effective world model.


\paragraph{Each rung matters.} To isolate the impact of each individual improvement, we remove each individual ``rung'' of our ladder from our best MBRL agent. As shown in Figure \ref{fig:ablation}[middle], each removal leads to a performance drop. This underscores the importance of combining all our proposed enhancements to achieve SOTA performance. 


\paragraph{When to start training in imagination?} Training the policy on imaginary TWM rollouts requires a reasonably accurate world model. This is why background planning (Step 4 in Algorithm \ref{algo:MBRL})
only begins after $T_{\text{BP}}$ environment steps.
Figure \ref{fig:ablation}[right]
explores the effect of varying $T_{\text{BP}}$.
% between $0, 100\text{k}, 200\text{k}, 400\text{k}, 600\text{k}$. 
Initiating imagination training too early ($T_{\text{BP}}=0$) leads to performance collapse due to the inaccurate TWM dynamics.
% Note that, 
%Figure \ref{fig:wm_quality} suggests that TWM training stabilizes after around $100\text{k}$ steps, supporting the choice of delaying imagination training.

\paragraph{MFRL ablation.} The final 3 rows in Table \ref{tab:ablation} show that either removing the RNN or using a smaller model as in \citet{moon2024discovering} leads to a drop in performance.


% \begin{table}[h!]
% \caption{Ablations results.}
% \label{tab:ablation}
% \small
% \vspace{-1em}
% \centering
% \begin{tabular}{ccc}
% \toprule
% Method & Reward $(\%)$ & Score $(\%)$ \\
% \midrule
% Best MBRL  & $\mathbf{67.42} \pm 0.55$ &
% $\mathbf{27.91} \pm 0.63$ \\
% \midrule
% $5\times5$ quantized  & $57.28 \pm 1.14$ & $18.26 \pm 1.18$ \\
% $9\times9$ quantized  & $45.55 \pm 0.88$ & $10.12 \pm 0.40$ \\
% $7\times7$ continuous & $21.20 \pm 0.55$ & $2.43 \pm 0.09$ \\
% \midrule
% Remove Dyna & $55.02 \pm 5.34$ & $18.79 \pm 2.14$ \\
% Remove NNT  & $60.66 \pm 1.38$ & $21.79 \pm 1.33$ \\
% Remove NNT \& patches & $45.86 \pm 1.42$ & $10.36 \pm 0.69$ \\
% Remove BTF  & $64.96 \pm 1.13$ & $25.55 \pm 0.86$ \\
% %Remove RNN  & $42.08 \pm 1.35$ & $8.59 \pm 0.57$ \\
% %Moon MFRL & $45.24 \pm 1.23$ & $10.06 \pm 0.61$ \\
% \midrule
% Use $T_{\text{BP}}=0$ & $33.54 \pm 10.09$ & $12.86 \pm 4.05$  \\
% \midrule
% \midrule
% Best MFRL  & $55.49 \pm 1.33$ & $16.77\pm 1.11$ \\
% Remove RNN & $41.82 \pm 0.97$ & $8.33 \pm 0.44$ \\
% Smaller model  & $51.35 \pm 0.80$ & $12.93 \pm 0.56$ \\
% \bottomrule
% \end{tabular}
% \vspace{-1em}
% \end{table}

 
\begin{figure*}[t!]
    \centering
    \begin{tabular}{c}
        \includegraphics[width=0.31\textwidth]{figures/patch_size.pdf}
    \end{tabular}
    \hspace{-5mm}
    \begin{tabular}{c}
        \includegraphics[width=0.31\textwidth]{figures/remove_ladder_step.pdf}
    \end{tabular}
    \hspace{-5mm}
    \begin{tabular}{c}
        \includegraphics[width=0.31\textwidth]{figures/start_bp.pdf}
    \end{tabular}
    \vspace{-1em}
    \caption{[Left] MBRL performance decreases when NNT uses patches of smaller or larger size than the ground truth, but it remains competitive. However, performance collapses if the patches are not quantized. 
    [Middle] Removing any rung of the ladder of improvements leads to a drop in performance.
    [Right] Warming up the world model before using it to train the policy on imaginary rollouts is required for good performance. BP denotes background planning. For each method, training in imagination
    starts at the color-coded vertical line, and leads to an initial drop in performance.}
    \label{fig:ablation}
\vspace{-.5em}
\end{figure*}
 

\subsection{Comparing TWM rollouts}
\label{sec:stationary}

In this section, we compare the TWM rollouts learned by three world models in our ladder, namely M1, M3 and our best model M5. 
To do so, we first create an evaluation dataset of $N_{\text{eval}}=160$ trajectories, each of length $T_{\text{eval}}=T_{\text{WM}}=20$, collected during the training of our best MFRL agent:
$\mathcal{D}_{\text{eval}} = \left\{O^{1:N_{\text{eval}}}_{1:T_{\text{eval}}+1}, a^{1:N_{\text{eval}}}_{1:T_{\text{eval}}}, r^{1:N_{\text{eval}}}_{1:T_{\text{eval}}} \right\}$.
We evaluate the quality of imagined trajectories generated by each TWM. 
Given a TWM checkpoint at 1M steps and the $n$th trajectory in $\mathcal{D}_{\text{eval}}$, we execute the sequence of actions $a^n_{1:T_{\text{eval}}}$, starting from $O^n_{1}$, to obtain a rollout trajectory $\hat{O}^{\text{TWM},~n}_{1:T_{\text{eval}} + 1}$. 

% in $\mathcal{D}_{\text{eval}}$:
%However, while some errors may be due to TWM incorrectly predicting the game dynamics, other may be due to unpredictable environment changes (lighting, sprites moving...) To focus on the former, and control for the latter, we use a reference rollout (RR), which always predicts the initial observation and does not capture the game dynamics. RR has a reconstruction error of $\mathcal{E}^n_{t, ~\text{ref.}} = \| O^n_1 - O^n_t \|_2^2, ~\forall t$. The normalized observation reconstruction error at time $t$, averaged over the $N_{\text{eval}}$ trajectories is then:
% \vspace{-.5em}




\eat{
\textbf{Quantitative evaluations.}
To quantitatively evaluate the rollouts, in Figure \ref{fig:rollout_eval}[left] we plot the average L$2$ reconstruction error at each timestep $t$:
$
    \mathcal{E}_t = 
    \frac{1}{N_{\text{eval}}} \sum_{n=1}^{N_{\text{eval}}} \| \hat{O}^{\text{TWM},~n}_t - O^n_t \|_2^2,
    ~~~\forall t.
$
As expected, the error increases with $t$ as mistakes compound over the rollout.
Our best method, which uses NNT, achieves the lowest errors for all timesteps, highlighting that a stationary codebook makes learning simpler for the TWM, and leads to more accurate TWM rollouts.


We include two additional quantitative evaluations in Appendix \ref{ap:symbol_accuracy}. 
First, M5 achieves the lowest tokenizer reconstruction error in Figure \ref{fig:wm_quality_appendix}[left]. 
Second, we leverage the fact that each observation in Craftax-classic comes with a symbolic representation to compare symbol prediction accuracy from TWM rollouts in Figure \ref{fig:wm_quality_appendix}[right].
This metric further highlights that M5 best captures the game dynamics. 
}






\paragraph{Quantitative evaluations.}
For evaluation,
we leverage an appealing property of Craftax-classic: each observation $O_t$ comes with an array of ground truth symbols $S_t=(S_t^{1:R})$, with $R=145$. Given $100\text{k}$ pairs $(O_t, S_t)$, we train a CNN $f_{\mu}$, to predict the symbols from the observation; $f_{\mu}$ achieves a $99\%$ validation accuracy. Next, we use $f_{\mu}$ to predict the symbols from the generated rollouts. Figure \ref{fig:rollout_eval}[left] displays the average symbol accuracy at each timestep $t$:
\begin{equation*}
    \mathcal{A}_t = \frac{1}{N_{\text{eval}}R} \sum_{n=1}^{N_{\text{eval}}} \sum_{r=1}^{R}\mathbf{1}(f_{\mu}^r(\hat{O}^{\text{TWM},~n}_t), S^{r,n}_t), ~ \forall t,
\end{equation*}
where $\mathbf{1}(x,y) = 1 ~\text{iff.}~ x=y $ (and $0$ o.w.), $S^{r,n}_t$ denotes the ground truth $r$th symbol in the array $S^{n}_t$ associated with $O^n_t$, and $f_{\mu}^r(\hat{O}^{\text{TWM},~n}_t)$ its prediction for the rollout observation.
As expected, symbol accuracies decrease with $t$ as mistakes compound over the rollouts.
Our best method, which uses NNT, achieves the highest accuracies for all timesteps, as it best captures the game dynamics. This highlights that a stationary codebook makes TWM learning simpler.

We include two additional quantitative evaluations in Appendix \ref{ap:symbol_accuracy}, showing that M5 achieves the lowest tokenizer reconstruction errors and rollout reconstruction errors.




\paragraph{Qualitative evaluations.}
Due to environment stochasticity, TWM rollouts can differ from the environment rollout but still be useful for learning in imagination---as long as they respect the game dynamics.
Visual inspection of rollouts in Figure \ref{fig:rollout_eval}[right] reveals (a) map inconsistencies, (b) feasible hallucinations that respect the game dynamics and (c) infeasible hallucinations. 
% For learning in imagination, it is important for the model to generate realistic, diverse and eventful rollouts in which the
M1 can make simple mistakes in both the map and the game dynamics. 
M3 and M5 both generate feasible hallucinations of mobs, however M3 more often hallucinates infeasible rollouts.



\begin{figure*}[t]
    \centering
    \begin{subfigure}[b]{0.3\textwidth}
        \centering
        \includegraphics[width=\linewidth]{figures/wm_rollout_norm_symbol_accuracy_by_step.pdf}
        \hrule
        \vspace{0.3em}
        \includegraphics[width=0.8\linewidth]{figures/rollouts_key.pdf}
    \end{subfigure}
    \hspace{0.5em}
    \begin{subfigure}[b]{0.65\textwidth}
        \includegraphics[width=\linewidth]{figures/rollouts.pdf}
    \end{subfigure}
    \vspace{-0.5em}
    \caption{
    \small
    Rollout comparison for world models M1, M3 and M5.
    [Left]
    % The observation reconstruction error increases with the TWM rollout step. 
    Symbol accuracies decrease with the TWM rollout step. 
    The stationary NNT codebook used by M5 makes it easier to learn a reliable TWM.
    [Right]
    Best viewed zoomed in.
    \textbf{Map.}
    All three models accurately capture the agent's motion.
    All models can struggle to use the history to generate a consistent map when revisiting locations, however only M1 makes simple map errors in successive timesteps.
    \textbf{Feasible hallucinations.}
    M3 and M5 generate realistic hallucinations that respect the game dynamics, such as spawning mobs and losing health.
    \textbf{Infeasible hallucinations.}
    M1 often does not respect game dynamics; M1 incorrectly adds wood inventory, and incorrectly places a plant at the wrong timestep without the required sapling inventory.
    M3 exhibits some infeasible hallucinations in which the monster suddenly disappears or the spawned cow has an incorrect appearance.
    M5 rarely exhibits infeasible hallucinations.
    Figure \ref{fig:more_rollouts} in Appendix \ref{ap:rollout_comparison} shows more rollouts with similar behavior.
    % However this property may not be crucial for learning in imagination, since the policy does not have an accurate memory and tends to explore to find new resources rather than navigating to a previously seen resource.
    }
    \label{fig:rollout_eval}
\vspace{-.25em}
\end{figure*}
 



\eat{
\begin{figure}[t!]
    \centering
    %\includegraphics[width=0.35\textwidth]{figures}
    \caption{MFRL ablations between ours and Moon.}
\end{figure}
}
\eat{
\begin{figure}[t!]
    \centering
    %\includegraphics[width=0.35\textwidth]{figures}
    \caption{Sweep over $T_{WM}$ from 5-30. TWM sequence length is always 20. Link to rollout accuracy plot.}
\end{figure}
}


\subsection{Craftax Full}
\label{sec:craftax_full}

Table \ref{tab:craftax_full} compares the performance of various agents on
the full version of Craftax \citep{matthews2024craftax}, a significantly harder extension of Craftax-classic,
with more levels and achievements. While the previous SOTA agent reached $2.3\%$ reward (on symbolic inputs), our MFRL agent reaches
$4.63\%$ reward and our MBRL agent reaches a new SOTA reward of $5.44\%$.
See Appendix \ref{sec:classic_vs_full} for implementation details.
These results show that our techniques can generalize to harder environments.
%although further evaluation on more diverse environments is left to future work.


\begin{table}[h!]
\caption{
Results on Craftax after 1M environment interactions.
The previous SOTA reward uses symbolic input (score is unknown),
whereas our results use image input.
}
\label{tab:craftax_full}
\small
\vspace{-.5em}
\centering
\begin{tabular}{ccc}
\toprule
Method & Reward $(\%)$ & Score $(\%)$ \\
\midrule
Prev. SOTA MFRL  & $2.3$ (symbolic) & --- \\
Our best MFRL &  $4.63 \pm 0.20$ & $1.22 \pm 0.07$ \\
Our best MBRL  & $5.44 \pm 0.25$ & $1.53 \pm 0.10$ \\
\bottomrule
\end{tabular}
\vspace{-.25em}
\end{table}

% (see \cref{tab:classic_vs_full}).




 

\section{CONCLUSION}
\label{sec:concl}
The rapid rise of AI-generated media challenges information authenticity and societal trust, necessitating robust detection mechanisms. This survey examines the evolution of AI-generated media detection, focusing on the shift from Non-MLLM-based domain-specific detectors to MLLM-based general-purpose approaches. We compare these methods across authenticity, explainability, and localization tasks from both single-modal and multi-modal perspectives. Additionally, we review datasets, methodologies, and evaluation metrics, identifying key limitations and research challenges.
Beyond technical concerns, MLLM-based detection raises ethical and security issues. As GenAI sees broader deployment, regulatory frameworks vary significantly across jurisdictions, complicating governance. By summarizing these regulations, we provide insights for researchers navigating legal and ethical challenges.
While many challenges remain, We hope this survey sparks further discussion, informs future research, and contributes to a more secure and trustworthy AI ecosystem.

\newpage
\bibliographystyle{abbrvnat}
\bibliography{references}

\subsection{Lloyd-Max Algorithm}
\label{subsec:Lloyd-Max}
For a given quantization bitwidth $B$ and an operand $\bm{X}$, the Lloyd-Max algorithm finds $2^B$ quantization levels $\{\hat{x}_i\}_{i=1}^{2^B}$ such that quantizing $\bm{X}$ by rounding each scalar in $\bm{X}$ to the nearest quantization level minimizes the quantization MSE. 

The algorithm starts with an initial guess of quantization levels and then iteratively computes quantization thresholds $\{\tau_i\}_{i=1}^{2^B-1}$ and updates quantization levels $\{\hat{x}_i\}_{i=1}^{2^B}$. Specifically, at iteration $n$, thresholds are set to the midpoints of the previous iteration's levels:
\begin{align*}
    \tau_i^{(n)}=\frac{\hat{x}_i^{(n-1)}+\hat{x}_{i+1}^{(n-1)}}2 \text{ for } i=1\ldots 2^B-1
\end{align*}
Subsequently, the quantization levels are re-computed as conditional means of the data regions defined by the new thresholds:
\begin{align*}
    \hat{x}_i^{(n)}=\mathbb{E}\left[ \bm{X} \big| \bm{X}\in [\tau_{i-1}^{(n)},\tau_i^{(n)}] \right] \text{ for } i=1\ldots 2^B
\end{align*}
where to satisfy boundary conditions we have $\tau_0=-\infty$ and $\tau_{2^B}=\infty$. The algorithm iterates the above steps until convergence.

Figure \ref{fig:lm_quant} compares the quantization levels of a $7$-bit floating point (E3M3) quantizer (left) to a $7$-bit Lloyd-Max quantizer (right) when quantizing a layer of weights from the GPT3-126M model at a per-tensor granularity. As shown, the Lloyd-Max quantizer achieves substantially lower quantization MSE. Further, Table \ref{tab:FP7_vs_LM7} shows the superior perplexity achieved by Lloyd-Max quantizers for bitwidths of $7$, $6$ and $5$. The difference between the quantizers is clear at 5 bits, where per-tensor FP quantization incurs a drastic and unacceptable increase in perplexity, while Lloyd-Max quantization incurs a much smaller increase. Nevertheless, we note that even the optimal Lloyd-Max quantizer incurs a notable ($\sim 1.5$) increase in perplexity due to the coarse granularity of quantization. 

\begin{figure}[h]
  \centering
  \includegraphics[width=0.7\linewidth]{sections/figures/LM7_FP7.pdf}
  \caption{\small Quantization levels and the corresponding quantization MSE of Floating Point (left) vs Lloyd-Max (right) Quantizers for a layer of weights in the GPT3-126M model.}
  \label{fig:lm_quant}
\end{figure}

\begin{table}[h]\scriptsize
\begin{center}
\caption{\label{tab:FP7_vs_LM7} \small Comparing perplexity (lower is better) achieved by floating point quantizers and Lloyd-Max quantizers on a GPT3-126M model for the Wikitext-103 dataset.}
\begin{tabular}{c|cc|c}
\hline
 \multirow{2}{*}{\textbf{Bitwidth}} & \multicolumn{2}{|c|}{\textbf{Floating-Point Quantizer}} & \textbf{Lloyd-Max Quantizer} \\
 & Best Format & Wikitext-103 Perplexity & Wikitext-103 Perplexity \\
\hline
7 & E3M3 & 18.32 & 18.27 \\
6 & E3M2 & 19.07 & 18.51 \\
5 & E4M0 & 43.89 & 19.71 \\
\hline
\end{tabular}
\end{center}
\end{table}

\subsection{Proof of Local Optimality of LO-BCQ}
\label{subsec:lobcq_opt_proof}
For a given block $\bm{b}_j$, the quantization MSE during LO-BCQ can be empirically evaluated as $\frac{1}{L_b}\lVert \bm{b}_j- \bm{\hat{b}}_j\rVert^2_2$ where $\bm{\hat{b}}_j$ is computed from equation (\ref{eq:clustered_quantization_definition}) as $C_{f(\bm{b}_j)}(\bm{b}_j)$. Further, for a given block cluster $\mathcal{B}_i$, we compute the quantization MSE as $\frac{1}{|\mathcal{B}_{i}|}\sum_{\bm{b} \in \mathcal{B}_{i}} \frac{1}{L_b}\lVert \bm{b}- C_i^{(n)}(\bm{b})\rVert^2_2$. Therefore, at the end of iteration $n$, we evaluate the overall quantization MSE $J^{(n)}$ for a given operand $\bm{X}$ composed of $N_c$ block clusters as:
\begin{align*}
    \label{eq:mse_iter_n}
    J^{(n)} = \frac{1}{N_c} \sum_{i=1}^{N_c} \frac{1}{|\mathcal{B}_{i}^{(n)}|}\sum_{\bm{v} \in \mathcal{B}_{i}^{(n)}} \frac{1}{L_b}\lVert \bm{b}- B_i^{(n)}(\bm{b})\rVert^2_2
\end{align*}

At the end of iteration $n$, the codebooks are updated from $\mathcal{C}^{(n-1)}$ to $\mathcal{C}^{(n)}$. However, the mapping of a given vector $\bm{b}_j$ to quantizers $\mathcal{C}^{(n)}$ remains as  $f^{(n)}(\bm{b}_j)$. At the next iteration, during the vector clustering step, $f^{(n+1)}(\bm{b}_j)$ finds new mapping of $\bm{b}_j$ to updated codebooks $\mathcal{C}^{(n)}$ such that the quantization MSE over the candidate codebooks is minimized. Therefore, we obtain the following result for $\bm{b}_j$:
\begin{align*}
\frac{1}{L_b}\lVert \bm{b}_j - C_{f^{(n+1)}(\bm{b}_j)}^{(n)}(\bm{b}_j)\rVert^2_2 \le \frac{1}{L_b}\lVert \bm{b}_j - C_{f^{(n)}(\bm{b}_j)}^{(n)}(\bm{b}_j)\rVert^2_2
\end{align*}

That is, quantizing $\bm{b}_j$ at the end of the block clustering step of iteration $n+1$ results in lower quantization MSE compared to quantizing at the end of iteration $n$. Since this is true for all $\bm{b} \in \bm{X}$, we assert the following:
\begin{equation}
\begin{split}
\label{eq:mse_ineq_1}
    \tilde{J}^{(n+1)} &= \frac{1}{N_c} \sum_{i=1}^{N_c} \frac{1}{|\mathcal{B}_{i}^{(n+1)}|}\sum_{\bm{b} \in \mathcal{B}_{i}^{(n+1)}} \frac{1}{L_b}\lVert \bm{b} - C_i^{(n)}(b)\rVert^2_2 \le J^{(n)}
\end{split}
\end{equation}
where $\tilde{J}^{(n+1)}$ is the the quantization MSE after the vector clustering step at iteration $n+1$.

Next, during the codebook update step (\ref{eq:quantizers_update}) at iteration $n+1$, the per-cluster codebooks $\mathcal{C}^{(n)}$ are updated to $\mathcal{C}^{(n+1)}$ by invoking the Lloyd-Max algorithm \citep{Lloyd}. We know that for any given value distribution, the Lloyd-Max algorithm minimizes the quantization MSE. Therefore, for a given vector cluster $\mathcal{B}_i$ we obtain the following result:

\begin{equation}
    \frac{1}{|\mathcal{B}_{i}^{(n+1)}|}\sum_{\bm{b} \in \mathcal{B}_{i}^{(n+1)}} \frac{1}{L_b}\lVert \bm{b}- C_i^{(n+1)}(\bm{b})\rVert^2_2 \le \frac{1}{|\mathcal{B}_{i}^{(n+1)}|}\sum_{\bm{b} \in \mathcal{B}_{i}^{(n+1)}} \frac{1}{L_b}\lVert \bm{b}- C_i^{(n)}(\bm{b})\rVert^2_2
\end{equation}

The above equation states that quantizing the given block cluster $\mathcal{B}_i$ after updating the associated codebook from $C_i^{(n)}$ to $C_i^{(n+1)}$ results in lower quantization MSE. Since this is true for all the block clusters, we derive the following result: 
\begin{equation}
\begin{split}
\label{eq:mse_ineq_2}
     J^{(n+1)} &= \frac{1}{N_c} \sum_{i=1}^{N_c} \frac{1}{|\mathcal{B}_{i}^{(n+1)}|}\sum_{\bm{b} \in \mathcal{B}_{i}^{(n+1)}} \frac{1}{L_b}\lVert \bm{b}- C_i^{(n+1)}(\bm{b})\rVert^2_2  \le \tilde{J}^{(n+1)}   
\end{split}
\end{equation}

Following (\ref{eq:mse_ineq_1}) and (\ref{eq:mse_ineq_2}), we find that the quantization MSE is non-increasing for each iteration, that is, $J^{(1)} \ge J^{(2)} \ge J^{(3)} \ge \ldots \ge J^{(M)}$ where $M$ is the maximum number of iterations. 
%Therefore, we can say that if the algorithm converges, then it must be that it has converged to a local minimum. 
\hfill $\blacksquare$


\begin{figure}
    \begin{center}
    \includegraphics[width=0.5\textwidth]{sections//figures/mse_vs_iter.pdf}
    \end{center}
    \caption{\small NMSE vs iterations during LO-BCQ compared to other block quantization proposals}
    \label{fig:nmse_vs_iter}
\end{figure}

Figure \ref{fig:nmse_vs_iter} shows the empirical convergence of LO-BCQ across several block lengths and number of codebooks. Also, the MSE achieved by LO-BCQ is compared to baselines such as MXFP and VSQ. As shown, LO-BCQ converges to a lower MSE than the baselines. Further, we achieve better convergence for larger number of codebooks ($N_c$) and for a smaller block length ($L_b$), both of which increase the bitwidth of BCQ (see Eq \ref{eq:bitwidth_bcq}).


\subsection{Additional Accuracy Results}
%Table \ref{tab:lobcq_config} lists the various LOBCQ configurations and their corresponding bitwidths.
\begin{table}
\setlength{\tabcolsep}{4.75pt}
\begin{center}
\caption{\label{tab:lobcq_config} Various LO-BCQ configurations and their bitwidths.}
\begin{tabular}{|c||c|c|c|c||c|c||c|} 
\hline
 & \multicolumn{4}{|c||}{$L_b=8$} & \multicolumn{2}{|c||}{$L_b=4$} & $L_b=2$ \\
 \hline
 \backslashbox{$L_A$\kern-1em}{\kern-1em$N_c$} & 2 & 4 & 8 & 16 & 2 & 4 & 2 \\
 \hline
 64 & 4.25 & 4.375 & 4.5 & 4.625 & 4.375 & 4.625 & 4.625\\
 \hline
 32 & 4.375 & 4.5 & 4.625& 4.75 & 4.5 & 4.75 & 4.75 \\
 \hline
 16 & 4.625 & 4.75& 4.875 & 5 & 4.75 & 5 & 5 \\
 \hline
\end{tabular}
\end{center}
\end{table}

%\subsection{Perplexity achieved by various LO-BCQ configurations on Wikitext-103 dataset}

\begin{table} \centering
\begin{tabular}{|c||c|c|c|c||c|c||c|} 
\hline
 $L_b \rightarrow$& \multicolumn{4}{c||}{8} & \multicolumn{2}{c||}{4} & 2\\
 \hline
 \backslashbox{$L_A$\kern-1em}{\kern-1em$N_c$} & 2 & 4 & 8 & 16 & 2 & 4 & 2  \\
 %$N_c \rightarrow$ & 2 & 4 & 8 & 16 & 2 & 4 & 2 \\
 \hline
 \hline
 \multicolumn{8}{c}{GPT3-1.3B (FP32 PPL = 9.98)} \\ 
 \hline
 \hline
 64 & 10.40 & 10.23 & 10.17 & 10.15 &  10.28 & 10.18 & 10.19 \\
 \hline
 32 & 10.25 & 10.20 & 10.15 & 10.12 &  10.23 & 10.17 & 10.17 \\
 \hline
 16 & 10.22 & 10.16 & 10.10 & 10.09 &  10.21 & 10.14 & 10.16 \\
 \hline
  \hline
 \multicolumn{8}{c}{GPT3-8B (FP32 PPL = 7.38)} \\ 
 \hline
 \hline
 64 & 7.61 & 7.52 & 7.48 &  7.47 &  7.55 &  7.49 & 7.50 \\
 \hline
 32 & 7.52 & 7.50 & 7.46 &  7.45 &  7.52 &  7.48 & 7.48  \\
 \hline
 16 & 7.51 & 7.48 & 7.44 &  7.44 &  7.51 &  7.49 & 7.47  \\
 \hline
\end{tabular}
\caption{\label{tab:ppl_gpt3_abalation} Wikitext-103 perplexity across GPT3-1.3B and 8B models.}
\end{table}

\begin{table} \centering
\begin{tabular}{|c||c|c|c|c||} 
\hline
 $L_b \rightarrow$& \multicolumn{4}{c||}{8}\\
 \hline
 \backslashbox{$L_A$\kern-1em}{\kern-1em$N_c$} & 2 & 4 & 8 & 16 \\
 %$N_c \rightarrow$ & 2 & 4 & 8 & 16 & 2 & 4 & 2 \\
 \hline
 \hline
 \multicolumn{5}{|c|}{Llama2-7B (FP32 PPL = 5.06)} \\ 
 \hline
 \hline
 64 & 5.31 & 5.26 & 5.19 & 5.18  \\
 \hline
 32 & 5.23 & 5.25 & 5.18 & 5.15  \\
 \hline
 16 & 5.23 & 5.19 & 5.16 & 5.14  \\
 \hline
 \multicolumn{5}{|c|}{Nemotron4-15B (FP32 PPL = 5.87)} \\ 
 \hline
 \hline
 64  & 6.3 & 6.20 & 6.13 & 6.08  \\
 \hline
 32  & 6.24 & 6.12 & 6.07 & 6.03  \\
 \hline
 16  & 6.12 & 6.14 & 6.04 & 6.02  \\
 \hline
 \multicolumn{5}{|c|}{Nemotron4-340B (FP32 PPL = 3.48)} \\ 
 \hline
 \hline
 64 & 3.67 & 3.62 & 3.60 & 3.59 \\
 \hline
 32 & 3.63 & 3.61 & 3.59 & 3.56 \\
 \hline
 16 & 3.61 & 3.58 & 3.57 & 3.55 \\
 \hline
\end{tabular}
\caption{\label{tab:ppl_llama7B_nemo15B} Wikitext-103 perplexity compared to FP32 baseline in Llama2-7B and Nemotron4-15B, 340B models}
\end{table}

%\subsection{Perplexity achieved by various LO-BCQ configurations on MMLU dataset}


\begin{table} \centering
\begin{tabular}{|c||c|c|c|c||c|c|c|c|} 
\hline
 $L_b \rightarrow$& \multicolumn{4}{c||}{8} & \multicolumn{4}{c||}{8}\\
 \hline
 \backslashbox{$L_A$\kern-1em}{\kern-1em$N_c$} & 2 & 4 & 8 & 16 & 2 & 4 & 8 & 16  \\
 %$N_c \rightarrow$ & 2 & 4 & 8 & 16 & 2 & 4 & 2 \\
 \hline
 \hline
 \multicolumn{5}{|c|}{Llama2-7B (FP32 Accuracy = 45.8\%)} & \multicolumn{4}{|c|}{Llama2-70B (FP32 Accuracy = 69.12\%)} \\ 
 \hline
 \hline
 64 & 43.9 & 43.4 & 43.9 & 44.9 & 68.07 & 68.27 & 68.17 & 68.75 \\
 \hline
 32 & 44.5 & 43.8 & 44.9 & 44.5 & 68.37 & 68.51 & 68.35 & 68.27  \\
 \hline
 16 & 43.9 & 42.7 & 44.9 & 45 & 68.12 & 68.77 & 68.31 & 68.59  \\
 \hline
 \hline
 \multicolumn{5}{|c|}{GPT3-22B (FP32 Accuracy = 38.75\%)} & \multicolumn{4}{|c|}{Nemotron4-15B (FP32 Accuracy = 64.3\%)} \\ 
 \hline
 \hline
 64 & 36.71 & 38.85 & 38.13 & 38.92 & 63.17 & 62.36 & 63.72 & 64.09 \\
 \hline
 32 & 37.95 & 38.69 & 39.45 & 38.34 & 64.05 & 62.30 & 63.8 & 64.33  \\
 \hline
 16 & 38.88 & 38.80 & 38.31 & 38.92 & 63.22 & 63.51 & 63.93 & 64.43  \\
 \hline
\end{tabular}
\caption{\label{tab:mmlu_abalation} Accuracy on MMLU dataset across GPT3-22B, Llama2-7B, 70B and Nemotron4-15B models.}
\end{table}


%\subsection{Perplexity achieved by various LO-BCQ configurations on LM evaluation harness}

\begin{table} \centering
\begin{tabular}{|c||c|c|c|c||c|c|c|c|} 
\hline
 $L_b \rightarrow$& \multicolumn{4}{c||}{8} & \multicolumn{4}{c||}{8}\\
 \hline
 \backslashbox{$L_A$\kern-1em}{\kern-1em$N_c$} & 2 & 4 & 8 & 16 & 2 & 4 & 8 & 16  \\
 %$N_c \rightarrow$ & 2 & 4 & 8 & 16 & 2 & 4 & 2 \\
 \hline
 \hline
 \multicolumn{5}{|c|}{Race (FP32 Accuracy = 37.51\%)} & \multicolumn{4}{|c|}{Boolq (FP32 Accuracy = 64.62\%)} \\ 
 \hline
 \hline
 64 & 36.94 & 37.13 & 36.27 & 37.13 & 63.73 & 62.26 & 63.49 & 63.36 \\
 \hline
 32 & 37.03 & 36.36 & 36.08 & 37.03 & 62.54 & 63.51 & 63.49 & 63.55  \\
 \hline
 16 & 37.03 & 37.03 & 36.46 & 37.03 & 61.1 & 63.79 & 63.58 & 63.33  \\
 \hline
 \hline
 \multicolumn{5}{|c|}{Winogrande (FP32 Accuracy = 58.01\%)} & \multicolumn{4}{|c|}{Piqa (FP32 Accuracy = 74.21\%)} \\ 
 \hline
 \hline
 64 & 58.17 & 57.22 & 57.85 & 58.33 & 73.01 & 73.07 & 73.07 & 72.80 \\
 \hline
 32 & 59.12 & 58.09 & 57.85 & 58.41 & 73.01 & 73.94 & 72.74 & 73.18  \\
 \hline
 16 & 57.93 & 58.88 & 57.93 & 58.56 & 73.94 & 72.80 & 73.01 & 73.94  \\
 \hline
\end{tabular}
\caption{\label{tab:mmlu_abalation} Accuracy on LM evaluation harness tasks on GPT3-1.3B model.}
\end{table}

\begin{table} \centering
\begin{tabular}{|c||c|c|c|c||c|c|c|c|} 
\hline
 $L_b \rightarrow$& \multicolumn{4}{c||}{8} & \multicolumn{4}{c||}{8}\\
 \hline
 \backslashbox{$L_A$\kern-1em}{\kern-1em$N_c$} & 2 & 4 & 8 & 16 & 2 & 4 & 8 & 16  \\
 %$N_c \rightarrow$ & 2 & 4 & 8 & 16 & 2 & 4 & 2 \\
 \hline
 \hline
 \multicolumn{5}{|c|}{Race (FP32 Accuracy = 41.34\%)} & \multicolumn{4}{|c|}{Boolq (FP32 Accuracy = 68.32\%)} \\ 
 \hline
 \hline
 64 & 40.48 & 40.10 & 39.43 & 39.90 & 69.20 & 68.41 & 69.45 & 68.56 \\
 \hline
 32 & 39.52 & 39.52 & 40.77 & 39.62 & 68.32 & 67.43 & 68.17 & 69.30  \\
 \hline
 16 & 39.81 & 39.71 & 39.90 & 40.38 & 68.10 & 66.33 & 69.51 & 69.42  \\
 \hline
 \hline
 \multicolumn{5}{|c|}{Winogrande (FP32 Accuracy = 67.88\%)} & \multicolumn{4}{|c|}{Piqa (FP32 Accuracy = 78.78\%)} \\ 
 \hline
 \hline
 64 & 66.85 & 66.61 & 67.72 & 67.88 & 77.31 & 77.42 & 77.75 & 77.64 \\
 \hline
 32 & 67.25 & 67.72 & 67.72 & 67.00 & 77.31 & 77.04 & 77.80 & 77.37  \\
 \hline
 16 & 68.11 & 68.90 & 67.88 & 67.48 & 77.37 & 78.13 & 78.13 & 77.69  \\
 \hline
\end{tabular}
\caption{\label{tab:mmlu_abalation} Accuracy on LM evaluation harness tasks on GPT3-8B model.}
\end{table}

\begin{table} \centering
\begin{tabular}{|c||c|c|c|c||c|c|c|c|} 
\hline
 $L_b \rightarrow$& \multicolumn{4}{c||}{8} & \multicolumn{4}{c||}{8}\\
 \hline
 \backslashbox{$L_A$\kern-1em}{\kern-1em$N_c$} & 2 & 4 & 8 & 16 & 2 & 4 & 8 & 16  \\
 %$N_c \rightarrow$ & 2 & 4 & 8 & 16 & 2 & 4 & 2 \\
 \hline
 \hline
 \multicolumn{5}{|c|}{Race (FP32 Accuracy = 40.67\%)} & \multicolumn{4}{|c|}{Boolq (FP32 Accuracy = 76.54\%)} \\ 
 \hline
 \hline
 64 & 40.48 & 40.10 & 39.43 & 39.90 & 75.41 & 75.11 & 77.09 & 75.66 \\
 \hline
 32 & 39.52 & 39.52 & 40.77 & 39.62 & 76.02 & 76.02 & 75.96 & 75.35  \\
 \hline
 16 & 39.81 & 39.71 & 39.90 & 40.38 & 75.05 & 73.82 & 75.72 & 76.09  \\
 \hline
 \hline
 \multicolumn{5}{|c|}{Winogrande (FP32 Accuracy = 70.64\%)} & \multicolumn{4}{|c|}{Piqa (FP32 Accuracy = 79.16\%)} \\ 
 \hline
 \hline
 64 & 69.14 & 70.17 & 70.17 & 70.56 & 78.24 & 79.00 & 78.62 & 78.73 \\
 \hline
 32 & 70.96 & 69.69 & 71.27 & 69.30 & 78.56 & 79.49 & 79.16 & 78.89  \\
 \hline
 16 & 71.03 & 69.53 & 69.69 & 70.40 & 78.13 & 79.16 & 79.00 & 79.00  \\
 \hline
\end{tabular}
\caption{\label{tab:mmlu_abalation} Accuracy on LM evaluation harness tasks on GPT3-22B model.}
\end{table}

\begin{table} \centering
\begin{tabular}{|c||c|c|c|c||c|c|c|c|} 
\hline
 $L_b \rightarrow$& \multicolumn{4}{c||}{8} & \multicolumn{4}{c||}{8}\\
 \hline
 \backslashbox{$L_A$\kern-1em}{\kern-1em$N_c$} & 2 & 4 & 8 & 16 & 2 & 4 & 8 & 16  \\
 %$N_c \rightarrow$ & 2 & 4 & 8 & 16 & 2 & 4 & 2 \\
 \hline
 \hline
 \multicolumn{5}{|c|}{Race (FP32 Accuracy = 44.4\%)} & \multicolumn{4}{|c|}{Boolq (FP32 Accuracy = 79.29\%)} \\ 
 \hline
 \hline
 64 & 42.49 & 42.51 & 42.58 & 43.45 & 77.58 & 77.37 & 77.43 & 78.1 \\
 \hline
 32 & 43.35 & 42.49 & 43.64 & 43.73 & 77.86 & 75.32 & 77.28 & 77.86  \\
 \hline
 16 & 44.21 & 44.21 & 43.64 & 42.97 & 78.65 & 77 & 76.94 & 77.98  \\
 \hline
 \hline
 \multicolumn{5}{|c|}{Winogrande (FP32 Accuracy = 69.38\%)} & \multicolumn{4}{|c|}{Piqa (FP32 Accuracy = 78.07\%)} \\ 
 \hline
 \hline
 64 & 68.9 & 68.43 & 69.77 & 68.19 & 77.09 & 76.82 & 77.09 & 77.86 \\
 \hline
 32 & 69.38 & 68.51 & 68.82 & 68.90 & 78.07 & 76.71 & 78.07 & 77.86  \\
 \hline
 16 & 69.53 & 67.09 & 69.38 & 68.90 & 77.37 & 77.8 & 77.91 & 77.69  \\
 \hline
\end{tabular}
\caption{\label{tab:mmlu_abalation} Accuracy on LM evaluation harness tasks on Llama2-7B model.}
\end{table}

\begin{table} \centering
\begin{tabular}{|c||c|c|c|c||c|c|c|c|} 
\hline
 $L_b \rightarrow$& \multicolumn{4}{c||}{8} & \multicolumn{4}{c||}{8}\\
 \hline
 \backslashbox{$L_A$\kern-1em}{\kern-1em$N_c$} & 2 & 4 & 8 & 16 & 2 & 4 & 8 & 16  \\
 %$N_c \rightarrow$ & 2 & 4 & 8 & 16 & 2 & 4 & 2 \\
 \hline
 \hline
 \multicolumn{5}{|c|}{Race (FP32 Accuracy = 48.8\%)} & \multicolumn{4}{|c|}{Boolq (FP32 Accuracy = 85.23\%)} \\ 
 \hline
 \hline
 64 & 49.00 & 49.00 & 49.28 & 48.71 & 82.82 & 84.28 & 84.03 & 84.25 \\
 \hline
 32 & 49.57 & 48.52 & 48.33 & 49.28 & 83.85 & 84.46 & 84.31 & 84.93  \\
 \hline
 16 & 49.85 & 49.09 & 49.28 & 48.99 & 85.11 & 84.46 & 84.61 & 83.94  \\
 \hline
 \hline
 \multicolumn{5}{|c|}{Winogrande (FP32 Accuracy = 79.95\%)} & \multicolumn{4}{|c|}{Piqa (FP32 Accuracy = 81.56\%)} \\ 
 \hline
 \hline
 64 & 78.77 & 78.45 & 78.37 & 79.16 & 81.45 & 80.69 & 81.45 & 81.5 \\
 \hline
 32 & 78.45 & 79.01 & 78.69 & 80.66 & 81.56 & 80.58 & 81.18 & 81.34  \\
 \hline
 16 & 79.95 & 79.56 & 79.79 & 79.72 & 81.28 & 81.66 & 81.28 & 80.96  \\
 \hline
\end{tabular}
\caption{\label{tab:mmlu_abalation} Accuracy on LM evaluation harness tasks on Llama2-70B model.}
\end{table}

%\section{MSE Studies}
%\textcolor{red}{TODO}


\subsection{Number Formats and Quantization Method}
\label{subsec:numFormats_quantMethod}
\subsubsection{Integer Format}
An $n$-bit signed integer (INT) is typically represented with a 2s-complement format \citep{yao2022zeroquant,xiao2023smoothquant,dai2021vsq}, where the most significant bit denotes the sign.

\subsubsection{Floating Point Format}
An $n$-bit signed floating point (FP) number $x$ comprises of a 1-bit sign ($x_{\mathrm{sign}}$), $B_m$-bit mantissa ($x_{\mathrm{mant}}$) and $B_e$-bit exponent ($x_{\mathrm{exp}}$) such that $B_m+B_e=n-1$. The associated constant exponent bias ($E_{\mathrm{bias}}$) is computed as $(2^{{B_e}-1}-1)$. We denote this format as $E_{B_e}M_{B_m}$.  

\subsubsection{Quantization Scheme}
\label{subsec:quant_method}
A quantization scheme dictates how a given unquantized tensor is converted to its quantized representation. We consider FP formats for the purpose of illustration. Given an unquantized tensor $\bm{X}$ and an FP format $E_{B_e}M_{B_m}$, we first, we compute the quantization scale factor $s_X$ that maps the maximum absolute value of $\bm{X}$ to the maximum quantization level of the $E_{B_e}M_{B_m}$ format as follows:
\begin{align}
\label{eq:sf}
    s_X = \frac{\mathrm{max}(|\bm{X}|)}{\mathrm{max}(E_{B_e}M_{B_m})}
\end{align}
In the above equation, $|\cdot|$ denotes the absolute value function.

Next, we scale $\bm{X}$ by $s_X$ and quantize it to $\hat{\bm{X}}$ by rounding it to the nearest quantization level of $E_{B_e}M_{B_m}$ as:

\begin{align}
\label{eq:tensor_quant}
    \hat{\bm{X}} = \text{round-to-nearest}\left(\frac{\bm{X}}{s_X}, E_{B_e}M_{B_m}\right)
\end{align}

We perform dynamic max-scaled quantization \citep{wu2020integer}, where the scale factor $s$ for activations is dynamically computed during runtime.

\subsection{Vector Scaled Quantization}
\begin{wrapfigure}{r}{0.35\linewidth}
  \centering
  \includegraphics[width=\linewidth]{sections/figures/vsquant.jpg}
  \caption{\small Vectorwise decomposition for per-vector scaled quantization (VSQ \citep{dai2021vsq}).}
  \label{fig:vsquant}
\end{wrapfigure}
During VSQ \citep{dai2021vsq}, the operand tensors are decomposed into 1D vectors in a hardware friendly manner as shown in Figure \ref{fig:vsquant}. Since the decomposed tensors are used as operands in matrix multiplications during inference, it is beneficial to perform this decomposition along the reduction dimension of the multiplication. The vectorwise quantization is performed similar to tensorwise quantization described in Equations \ref{eq:sf} and \ref{eq:tensor_quant}, where a scale factor $s_v$ is required for each vector $\bm{v}$ that maps the maximum absolute value of that vector to the maximum quantization level. While smaller vector lengths can lead to larger accuracy gains, the associated memory and computational overheads due to the per-vector scale factors increases. To alleviate these overheads, VSQ \citep{dai2021vsq} proposed a second level quantization of the per-vector scale factors to unsigned integers, while MX \citep{rouhani2023shared} quantizes them to integer powers of 2 (denoted as $2^{INT}$).

\subsubsection{MX Format}
The MX format proposed in \citep{rouhani2023microscaling} introduces the concept of sub-block shifting. For every two scalar elements of $b$-bits each, there is a shared exponent bit. The value of this exponent bit is determined through an empirical analysis that targets minimizing quantization MSE. We note that the FP format $E_{1}M_{b}$ is strictly better than MX from an accuracy perspective since it allocates a dedicated exponent bit to each scalar as opposed to sharing it across two scalars. Therefore, we conservatively bound the accuracy of a $b+2$-bit signed MX format with that of a $E_{1}M_{b}$ format in our comparisons. For instance, we use E1M2 format as a proxy for MX4.

\begin{figure}
    \centering
    \includegraphics[width=1\linewidth]{sections//figures/BlockFormats.pdf}
    \caption{\small Comparing LO-BCQ to MX format.}
    \label{fig:block_formats}
\end{figure}

Figure \ref{fig:block_formats} compares our $4$-bit LO-BCQ block format to MX \citep{rouhani2023microscaling}. As shown, both LO-BCQ and MX decompose a given operand tensor into block arrays and each block array into blocks. Similar to MX, we find that per-block quantization ($L_b < L_A$) leads to better accuracy due to increased flexibility. While MX achieves this through per-block $1$-bit micro-scales, we associate a dedicated codebook to each block through a per-block codebook selector. Further, MX quantizes the per-block array scale-factor to E8M0 format without per-tensor scaling. In contrast during LO-BCQ, we find that per-tensor scaling combined with quantization of per-block array scale-factor to E4M3 format results in superior inference accuracy across models. 


\end{document}