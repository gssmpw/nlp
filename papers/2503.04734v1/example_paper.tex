%%%%%%%% ICML 2025 EXAMPLE LATEX SUBMISSION FILE %%%%%%%%%%%%%%%%%

\documentclass{article}

% Recommended, but optional, packages for figures and better typesetting:
\usepackage{microtype}
\usepackage{graphicx}
\usepackage{subfigure}
%\usepackage{xcolor}
\usepackage{color,soul}
%\usepackage{biblatex}
\usepackage{multirow}
\usepackage{array}
\usepackage{booktabs} % for professional tables
\usepackage{comment}
\usepackage{enumitem}
\setlist[itemize]{itemsep=1pt, topsep=0pt}
\setlist[enumerate]{itemsep=1pt, topsep=0pt}
%\usepackage[toc,page]{appendix} % For appendix handling
%\usepackage{tocloft}

% hyperref makes hyperlinks in the resulting PDF.
% If your build breaks (sometimes temporarily if a hyperlink spans a page)
% please comment out the following usepackage line and replace
% \usepackage{icml2025} with \usepackage[nohyperref]{icml2025} above.
\usepackage{hyperref}


% Attempt to make hyperref and algorithmic work together better:
\newcommand{\theHalgorithm}{\arabic{algorithm}}

% Use the following line for the initial blind version submitted for review:
% \usepackage{icml2025}

% If accepted, instead use the following line for the camera-ready submission:

\usepackage[accepted]{icml2025}

% For theorems and such
\usepackage{amsmath}
\usepackage{amssymb}
\usepackage{mathtools}
\usepackage{amsthm}
\usepackage[utf8]{inputenc}
%\usepackage{xcolor}
\usepackage{tcolorbox}
% if you use cleveref..
\usepackage[capitalize,noabbrev]{cleveref}

%%%%%%%%%%%%%%%%%%%%%%%%%%%%%%%%
% THEOREMS
%%%%%%%%%%%%%%%%%%%%%%%%%%%%%%%%
\theoremstyle{plain}
\newtheorem{theorem}{Theorem}[section]
\newtheorem{proposition}[theorem]{Proposition}
\newtheorem{lemma}[theorem]{Lemma}
\newtheorem{corollary}[theorem]{Corollary}
\theoremstyle{definition}
\newtheorem{definition}[theorem]{Definition}
\newtheorem{assumption}[theorem]{Assumption}
\theoremstyle{remark}
\newtheorem{remark}[theorem]{Remark}

% Todonotes is useful during development; simply uncomment the next line
%    and comment out the line below the next line to turn off comments
%\usepackage[disable,textsize=tiny]{todonotes}
\usepackage[textsize=tiny]{todonotes}

\newcommand{\opreview}{o1-preview}
\newcommand{\claude}{Claude 3.5 Sonnet}
\newcommand{\gptfouro}{GPT-4o}
\newcommand{\gemini}{Gemini 1.5 Pro}
\newcommand{\llama}{Llama 3.1 70b Instruct}
\newcommand{\gptthree}{GPT-3.5 Turbo}

\newcommand{\Maya}[1]{\textcolor{orange}{Maya: ``#1''}}
\newcommand{\Moses}[1]{\textcolor{BrickRed}{Moses: ``#1''}}
\usepackage{soul} % For highlighting with background colors
%\usepackage[dvipsnames]{xcolor} % For custom colors


\definecolor{lightgreen}{HTML}{DFFFD6}
\definecolor{lightred}{HTML}{FFD6D6}
\definecolor{lightyellow}{HTML}{FFFFD6}
\definecolor{lightblue}{HTML}{D6F0FF}
\definecolor{lightgray}{HTML}{E0E0E0}


% Define custom highlighting commands
\newcommand{\hlgreen}[1]{\sethlcolor{lightgreen}\hl{#1}}
\newcommand{\hlred}[1]{\sethlcolor{lightred}\hl{#1}}
\newcommand{\hlyellow}[1]{\sethlcolor{lightyellow}\hl{#1}}
\newcommand{\hlblue}[1]{\sethlcolor{lightblue}\hl{#1}}
\newcommand{\hlgray}[1]{\sethlcolor{lightgray}\hl{#1}}

\definecolor{midblue}{HTML}{B3E6FF}
% The \icmltitle you define below is probably too long as a header.
% Therefore, a short form for the running title is supplied here:
\icmltitlerunning{What Can Large Language Models Do for Sustainable Food?}

\begin{document}

\twocolumn[
\icmltitle{What Can Large Language Models Do for Sustainable Food?}

% It is OKAY to include author information, even for blind
% submissions: the style file will automatically remove it for you
% unless you've provided the [accepted] option to the icml2025
% package.

% List of affiliations: The first argument should be a (short)
% identifier you will use later to specify author affiliations
% Academic affiliations should list Department, University, City, Region, Country
% Industry affiliations should list Company, City, Region, Country

% You can specify symbols, otherwise they are numbered in order.
% Ideally, you should not use this facility. Affiliations will be numbered
% in order of appearance and this is the preferred way.
%\icmlsetsymbol{equal}{*}

\begin{icmlauthorlist}
\icmlauthor{Anna T. Thomas}{yyy}
\icmlauthor{Adam Yee}{comp}
\icmlauthor{Andrew Mayne}{yyy}
\icmlauthor{Maya B. Mathur}{yyy}
\icmlauthor{Dan Jurafsky}{yyy}
\icmlauthor{Kristina Gligori\'c}{yyy}
%\icmlauthor{Firstname7 Lastname7}{comp}
%\icmlauthor{}{sch}
%\icmlauthor{Firstname8 Lastname8}{sch}
%\icmlauthor{Firstname8 Lastname8}{yyy,comp}
%\icmlauthor{}{sch}
%\icmlauthor{}{sch}
\end{icmlauthorlist}

\icmlaffiliation{yyy}{Stanford University}
\icmlaffiliation{comp}{Umai Works}

\icmlcorrespondingauthor{Anna T. Thomas}{thomasat@stanford.edu}
\icmlcorrespondingauthor{Kristina Gligorić}{gligoric@stanford.edu}


% You may provide any keywords that you
% find helpful for describing your paper; these are used to populate
% the "keywords" metadata in the PDF but will not be shown in the document
\icmlkeywords{Machine Learning, ICML}

\vskip 0.3in
]

% this must go after the closing bracket ] following \twocolumn[ ...

% This command actually creates the footnote in the first column
% listing the affiliations and the copyright notice.
% The command takes one argument, which is text to display at the start of the footnote.
% The \icmlEqualContribution command is standard text for equal contribution.
% Remove it (just {}) if you do not need this facility.

%\printAffiliationsAndNotice{}  % leave blank if no need to mention equal contribution
\printAffiliationsAndNotice{} % otherwise use the standard text.

\begin{abstract}
Food systems are responsible for a third of human-caused greenhouse gas emissions. We investigate what Large Language Models (LLMs) can contribute to reducing the environmental impacts of food production. We define a typology of design and prediction tasks based on the sustainable food literature and collaboration with domain experts, and evaluate six LLMs on four tasks in our typology. For example, for a sustainable protein design task, food science experts estimated that collaboration with an LLM can reduce time spent by 45\% on average, compared to 22\% for collaboration with another expert human food scientist. However, for a sustainable menu design task, LLMs produce suboptimal solutions when instructed to consider both human satisfaction and climate impacts. We propose a general framework for integrating LLMs with combinatorial optimization to improve reasoning capabilities.
Our approach decreases emissions of food choices by 79\% in a hypothetical restaurant while maintaining participants' satisfaction with their set of choices. Our results demonstrate LLMs' potential, supported by optimization techniques, to accelerate sustainable food development and adoption.
%%address a class of design problems. %involving human preferences. 
%Our food system produces one third of human-caused greenhouse gas emissions.
%Industrial animal agriculture is a major contributor to climate change. \textcolor{red}{Alternative: lead with food systems more generally being responsible for one third of anthropogenic GHG emissions.} 
%We investigate how Large Language Models (LLMs) can accelerate the transition to a more sustainable food system. %: both the development of novel alternative protein sources and increasing adoption of existing sustainable foods. 
%We define a typology of tasks based on the sustainable food literature and collaboration with domain experts, and evaluate six LLMs on four tasks in our typology. For example, for a sustainable protein design task, food science experts estimated that collaboration with an LLM can reduce time spent by 45\% on average, compared to 22\% for collaboration with another expert human food scientist. However, for a sustainable menu design task, LLMs produce suboptimal solutions when instructed to consider both human satisfaction and climate impacts. 
%We propose a general framework for constrained optimization with a subjective quality objective function, combining LLMs with mathematical programming. 
%We present a general framework for integrating LLMs with optimization techniques to address constrained design problems involving human preferences. %subjective quality metrics
%For a sustainable menu design task, 
%Our approach decreases emissions of food choices by 79\% in a hypothetical restaurant while maintaining participants' satisfaction with their set of choices.  Our results demonstrate LLMs' potential, supported by optimization techniques, to accelerate sustainable food development and adoption.%, providing a foundation for future machine learning research in this domain. 
%\textcolor{red}{Sentence on how the optimization framework could be more general throughout AI.}
%Our results provide insight into the strengths and weaknesses of LLMs for sustainable food development. %\textcolor{red}{Alternative: frame more cautiously as we characterize strengths and weaknesses of LLMs, and address one weakness.} %or: characterize strengths and weaknesses


%Industrial animal agriculture is a major contributor to climate change, motivating the design and adoption of more sustainable alternatives. 
%Here we investigate what large language models (LLMs) can contribute to sustainable food, both the development of substitutes for animal products (``alternative protein'') and increasing adoption of existing sustainable foods. We define a diverse set of tasks based on the sustainable food literature and expert interviews, and evaluate six LLMs on these tasks. To bring the problem of alternative protein design to the attention of the machine learning community, we create a benchmark and define two \textcolor{red}{(or nine if we count each sensory dimension separately)} tasks. For an experimental design task, food science experts estimated that collaboration with an LLM reduces time spent by 45\% on average, compared to 22\% for collaboration with another expert human food scientist. %On other tasks, they perform poorly, highlighting directions for future research. 
%This does not take into account the speed with which LLMs generate outputs compared to humans, suggesting that the actual time savings may be much higher.
%Additionally, we find that, via creative recipes and appealing descriptions, LLMs can be used to revise a university dining hall menu to decrease average emissions of food options by 33\% while maintaining participants' satisfaction with their set of choices. 
%Additionally, we study an existing task from the sustainable food literature in which the goal is to revise a menu to promote sustainable choices. 
%Additionally, we show how LLMs can be used to revise a menu to decrease average emissions of food choices by 81\% in a hypothetical restaurant while maintaining participants' satisfaction with their set of choices. 
%Taken together, these tasks and datasets can inform the development of ML-based solutions to sustainable food development problems.

%Our benchmark, tasks, and empirical results demonstrate LLMs' potential to accelerate sustainable food development and adoption, providing a foundation for future machine learning research in this domain.
\end{abstract}
%\textcolor{red}{This improves upon the previous best reported results on this task.}

\section{Introduction}
%Concentrated animal feeding operations (CAFOs), also known as factory farms, are responsible for myriad environmental and public health harms. Specific harms include greenhouse gas (GHG) emissions, deforestation, water pollution, biodiversity loss, antibiotic resistance, and zoonotic disease transmission~\cite{allan2023intergovernmental,hopwood2020health,scherer2019opportunity}. Recent estimates suggest that 99\% of livestock in the United States are raised in CAFOs~\cite{ritchie2023}.
%\textcolor{red}{tighten these two paragraphs}
%Industrial animal agriculture, also known as factory farming, is 
Global food production practices are a leading contributor to climate change, deforestation, water pollution, biodiversity loss, antibiotic resistance, and zoonotic disease transmission~\cite{allan2023intergovernmental,hopwood2020health,scherer2019opportunity}. %Animal-based foods are responsible for 57\% of greenhouse gas (GHG) emissions from our food system~\cite{xu2021global}, which itself accounts for one-third of all anthropogenic GHG emissions~\cite{crippa2021food}.
%Our food system is responsible for one-third of all anthropogenic GHG emissions~\cite{crippa2021food}, and animal-based food accounts for 57\% of food system emissions~\cite{xu2021global}. 
%The United Nations' Intergovernmental Panel on Climate Change reported that a societal transition from animal-based food to plant-based food by 2050 would reduce land-based GHG emissions by two-thirds relative to baseline~\cite{ipcc2019}. 
%In response to these issues, the United Nations' Intergovernmental Panel on Climate Change, EAT-Lancet Commission, and the World Health Organization have called for a reduction in consumption of animal-based foods~\cite{ipcc2019,willett2019food,goetschel2024animal}, and 
Given the disproportionate harms of industrial animal agriculture in particular, which accounts for 57\% of food systems emissions, a number of companies have formed to develop more sustainable protein sources~\cite{xu2021global,mylan2023big}. Additionally, programs such as the Menus of Change have developed to promote sustainable choices in foodservice operations~\cite{menusofchangeAboutMenus}. These directions require knowledge of both food science and human preferences.%Here we explore what large language models (LLMs) can contribute to sustainable food development and adoption, which requires knowledge of both food science and human psychology.%both directions.
%Two main approaches could help to accelerate a transition away from factory farming: developing novel foods (``alternative protein"), usually simulating animal-based products, and increasing adoption of traditional plant-based food.
%The design and adoption of sustainable foods requires knowledge of both food science and human psychology. 

Though LLMs have shown promise in scientific discovery~\cite{boiko2023autonomous,guo2023can} and modeling human behavior~\cite{park2022social,horton2023large}, applications in sustainable food remain underexplored. Beyond its importance in sustainability and public health, this domain represents a compelling testbed for broader machine learning (ML) challenges, including modeling diverse human preferences, generating creative solutions, and aligning outputs with real-world constraints and societal values~\cite{pmlr-v235-chakraborty24b,yang2024large}. These challenges are not only central to sustainable food but also to other domains, making sustainable food tasks a valuable benchmark for advancing general ML methods.
%\textcolor{red}{(TODO find better refs for LLMs modeling human preferences)}
%While LLMs have demonstrated success in many domains, to our knowledge no published work has explored their application in sustainable food. We note that their training data includes extensive food science knowledge. 
To initiate an exploration of what LLMs can contribute to sustainable food, we assembled a team consisting of culinary and food science professionals and researchers in ML, natural language processing, and statistics. We make three main contributions: 
\begin{itemize}
\itemsep0em 
%%\noindent \textbf{1. Task typology}
 \item Based on collaboration with expert culinary and food science professionals, we define a typology of tasks grounded in the needs of food scientists and chefs. Our typology covers design and prediction tasks across the resolutions of ingredients, recipes, and food systems.  % and the existing literature
%\noindent \textbf{2. Evaluation framework} 
\item We evaluate six LLMs on four concrete instantiations of tasks in our typology, incorporating two datasets not previously studied by the machine learning community. %Specifically, we study a new sustainable protein dataset of 5,516 sensory evaluations of 47 products, where each evaluation is along 21 dimensions. This dataset is available to academic researchers. We also study a public dataset of 522,517 recipes  and 1,401,982 reviews. %Finally, we build upon a menu design task from the sustainable food literature.
%from 312 different categories
% from 271,907 users
%, using data from publicly available sources to quantify climate impacts. 
%\item We identify data that allow addressing these tasks. In particular, we study a new plant-based meat dataset of 5,725 sensory evaluations of 45 products, where each evaluation is along 21 dimensions. This dataset is available to academic researchers. We also study a publicly available dataset of 522,517 recipes from 312 different categories and 1,401,982 reviews from 271,907 users. Finally, we build upon an existing sustainable food menu design task, using data from publicly available sources to quantify climate impacts. None have previously been studied by the machine learning community.
 %Based on our results, we characterize strengths and weaknesses of LLMs in the sustainable food domain. 
 Notably, LLMs meet or exceed expert human performance on our sustainable protein experimental design task, and also show promising performance in coarse-grained preference prediction. However, they exhibit an omnivore bias, produce suboptimal solutions when balancing multiple constraints (e.g. maintain satisfaction while reducing emissions), and fail to perform well in fine-grained preference prediction.
%\noindent \textbf{3. LLM+optimization approach}
\item We propose an approach that integrates LLMs with combinatorial optimization techniques to address LLMs' relative weakness in mathematical reasoning, and demonstrate that our approach achieves a 79\% average emissions reduction in food choices while maintaining patron satisfaction with their set of choices. %This approach may also be of independent interest.
%\item We present an approach based on integrating LLMs with combinatorial optimization techniques for addressing one weakness of LLMs relevant to this domain, of solving constrained optimization problems (e.g. maintaining satisfaction while reducing emissions), which may be of independent interest.
%\item We introduce a framework \textcolor{red}{(if highlighting as a separate contribution, need to develop this more)} for evaluating LLM performance on these tasks by both food science experts and laypeople, and characterize both strengths and weaknesses.
\end{itemize}


\begin{comment}
\begin{table*}[ht!]
\centering

\begin{tabular}{p{0.15\textwidth} | p{0.25\textwidth} |p{0.25\textwidth} |p{0.25\textwidth}}
\toprule
\textbf{Task \textbackslash Resolution} & \textbf{Ingredients}                                       & \textbf{Recipes}                                                          & \textbf{Menus}                                      \\
\midrule
\multirow{2}{*}{\textbf{Prediction}}       & \textbf{Input}: partial recipe and candidate additional ingredients & \textbf{Input}: set of candidate recipes and a target population                   & \textbf{Input}: set of candidate menus and target population \\
                                        & \textbf{Output}: rankings of candidate ingredients                  & \textbf{Output}: rankings of candidate recipes & \textbf{Output}: rankings of candidate menus                 \\
\midrule
\multirow{2}{*}{\textbf{Generation}}    & \textbf{Input}: title or partial recipe                             & \textbf{Input}: set of ingredients and constraints                                 & \textbf{Input}: set of ingredients and constraints           \\
                                        & \textbf{Output}: ingredient(s) to add                               & \textbf{Output}: recipe that meets the constraints                                 & \textbf{Output}: a menu that meets the constraints           \\
\midrule
\multirow{2}{*}{\textbf{Description}}   & \textbf{Input}: an ingredient                                       & \textbf{Input}: a recipe                                                           & \textbf{Input}: a menu                                       \\
                                        & \textbf{Output}: appealing description                              & \textbf{Output}: appealing description                                             & \textbf{Output}: appealing description    \\                  
\bottomrule
\end{tabular}
\caption{\textbf{A taxonomy of sustainable food development tasks.} Tasks are categorized into prediction, generation, and description across the resolutions of ingredients, recipes, and menus. For each task, we list input and output data.}
\label{tab:taxonomy}
\end{table*}
\end{comment}

\section{Related Work}
\noindent \textbf{Large Language Model Evaluations.}
%LLMs have demonstrated strong performance across diverse domains~\cite{bubeck2023sparks}.
%\textcolor{red}{Cut down}. 
Rigorous evaluation of LLMs has clarified their strengths and weaknesses. % and identified areas for future work. ~\citet{singhal2023large} proposed a framework for evaluating LLMs' answers to medical questions.
%Med-PaLM achieved a passing score on the U.S. Medical Licensing Exam~\cite{singhal2023large}.
\citet{guha2024legalbench} identified six types of legal reasoning tasks and evaluated 20 models on them. 
In chemistry,~\citet{guo2023can} identified three types of capabi, established a benchmark of eight tasks, and evaluated five LLMs. 
%In biology, . In mathematics, . 
%Marinka Zitnik - taxonomy of medical tasks. 
However, no work, to our knowledge, has formalized key tasks and evaluated LLMs in the sustainable food domain.% We aim to fill this gap.
%by providing an evaluation framework for sustainable food. 

\noindent \textbf{Large Language Models for Scientific Discovery.}
In chemistry,~\citet{boiko2023autonomous} demonstrated that GPT-4 combined with internet search and code execution can perform experimental design and execution.
In genomics,~\citet{roohani2024biodiscoveryagent} showed that an LLM with access to several tools can design genetic perturbation experiments. In computational nanobody design,~\citet{swanson2024virtual} propose a framework in which LLM agents receive periodic human feedback, and showed promising performance in the design of novel antibodies. In artificial intelligence research,~\citet{si2024can} show that LLMs can generate novel research ideas. However, we are not aware of prior work on LLMs for discovery of novel sustainable foods. 
%NLP research paper~\cite{si2024can}. 
%Leskovec - BioDiscoveryAgent. 
%LLMs - bilevel optimization.
%FlavorPuzzle. NeurIPS, KDD paper: diet planning.
%GNNs for ingredient matching.

\noindent \textbf{Large Language Models for Simulating Human Preferences.}
%Our work studies what LLMs can contribute to modeling human preferences and perceptions related to food. 
While we are not aware of prior work studying how LLMs can model human preferences and perceptions related to food, a growing literature has shown evidence that LLMs can accurately simulate human behavior in tasks such as search behavior, participation in online communities, hiring scenarios, and classic economic, psycholinguistic, and social psychology experiments~\cite{zhang2024usimagent,park2022social,horton2023large,aher2023using}.%several prior works have studied LLM capabilities for simulating human preferences and opinions in other settings~\cite{santurkar2023whose}.  
%Horton.
%June from stanford - social simulacra. 
%\citet{santurkar2023whose} rigorously studied the question of whose opinions language models reflect and found poor modeling of certain minority groups, e.g. over 65 and widowed individuals. 
%Autogen - microsoft. Agentic AI. Split into micro-simulations. 

\noindent \textbf{Large Language Models and Optimization.}
Our work also pertains to the use of LLMs for mathematical optimization, and combining LLMs with optimization techniques. ~\citet{yang2024large} evaluated LLMs for several optimization problems.~\citet{pmlr-v235-ahmaditeshnizi24a} introduce OptiMUS, an LLM-based agent for solving optimization problems. Motivated by sustainable food applications, we propose a general framework for integrating LLMs with combinatorial optimization techniques that leverages LLMs' knowledge on topics such as human preferences. We further discuss these works and additional related work in Appendix~\ref{sec:extended-related-work}. %\textcolor{red}{Potentially say it could be more general throughout AI.}%We lay the foundation for using optimization tools and LLMs for the sustainable food development.
%\cite{pmlr-v235-ma24m}.

%Some prior works have evaluated LLMs on climate and sustainability related tasks. For example,~
%FlavorPuzzle. 
%Chris Manning's paper~\cite{morio2023an}. %ICML paper on LLMs for climate information~\cite{pmlr-v235-bulian24a}.

%\subsection{Background on Food Science and Quantification of Climate Impacts}


\section{Task Definitions}

\begin{table*}[ht!]
\small
\centering

\begin{tabular}{p{0.14\textwidth} | p{0.25\textwidth} |p{0.25\textwidth} |p{0.25\textwidth}}
\toprule
\textbf{Task \textbackslash Resolution} & \textbf{Ingredients}                                       & \textbf{Recipes}                                                          & \textbf{Systems}                                      \\
\midrule
\multirow{2}{*}{\textbf{Design}}    & 1. Design breeding strategies for chickpeas to improve yield
& 1. Design a lentil soup recipe meeting nutritional and cost constraints                                & \hlgray{1. Design a menu meeting a foodservice operation's constraints}          \\
& 2. Design a plant-based protein blend to mimic egg white texture                            & \hlgray{2. Design  experiments to improve
a plant-based product formulation
in response to feedback}                                & 2. Design an optimized supply chain for a plant-based meat company to reduce climate impacts %and costs  %something about extrusion parameters?          
\\
\midrule
\multirow{2}{*}{\textbf{Prediction}}       & 1. Predict complementary ingredient pairings for traditional plant proteins &  \hlgray{1. Predict which of two recipes a target population will prefer}              & 1. Predict change in patron satisfaction when making plant-based the default \\
& 2. Predict functional properties (e.g.,
solubility) of a proposed plant-based analog for casein         & \hlgray{2. Predict sensory profile of a plant-based product formulation} & 2. Predict climate and economic impacts of a regional transition to plant-based meat                 \\%Predict cost of manufacturing a plant-based
%meat product at scale
                 \bottomrule
\end{tabular}
\caption{\textbf{A typology of tasks at the intersection of ML and sustainable food development.} Tasks are categorized into design and prediction across the resolutions of ingredients, recipes, and food systems. For each task, we list examples corresponding to both traditional (1) and novel (2) foods. Tasks we study are highlighted.}
\label{tab:taxonomy}
\end{table*}

%\subsection{Sustainable Food Background}
%A transition to a sustainable food system could occur via multiple paths. One path does not involve creating novel foods but rather simply shifting consumption toward existing sustainable foods, namely plant-based foods and more sustainable animal-based foods such as .

%\subsection{Typology of Tasks}

%\subsection{Quantification of Climate Impacts}

%The intersection of machine learning and food-related application has gained significant attention, with studies across various communities addressing different aspects of food development. 

\subsection{Typology}
%\textcolor{red}{Switch paragraph order}
We identify a non-exhaustive typology of sustainable food tasks amenable to ML approaches, shown in Table~\ref{tab:taxonomy}.
%, including both traditional and novel foods,
%Traditional foods are defined as foods that have been consumed for centuries or longer. Examples include whole foods like legumes (lentils, chickpeas), grains (wheat, quinoa), vegetables, and fruits, as well as plant-derived products such as tofu, tempeh, and seitan. While some animal-based foods are relatively low in GHG emissions, e.g. pork, chicken, fish, and eggs, these still contribute significantly to other environmental and public health problems such as zoonotic disease transmission, antibiotic resistance, and air and water pollution~\cite{hilborn2018environmental,clark2017comparative,grvzinic2023intensive,hayek2022infectious,jones2013zoonosis}. Additionally, substituting beef, the most harmful food from a climate change perspective~\cite{poore2018reducing}, for these foods can be significantly worse for animal welfare~\cite{mathur2022ethical}. Public opinion polling suggests high levels of concern for animal welfare~\cite{gallupUSMore}. 
%Beyond traditional plant-based foods, we also consider the role of more sustainable animal-based foods such as pork, chicken, fish, and eggs, and consider trade-offs between factors such as consumer satisfaction, sustainability, and animal welfare~\cite{mathur2022ethical}. 
 % We acknowledge that some foods blur the line between traditional and novel. %For example, a traditional form of soymilk - a simple blend of soybeans and water - has been consumed for centuries in East Asia. However, modern soymilk has been industrialized and reformulated to serve as a dairy milk replacement, often fortified with vitamins and minerals or flavored for broader appeal. %We also note that our typology of tasks is non-exhaustive. 
Based on the sustainable food literature and collaboration with experts, we identify two types of tasks, (1) design and (2) prediction, across three resolutions, (1) ingredients, (2) recipes, and (3) systems. 
 \textit{Design tasks} focus on creating or optimizing food products or systems to meet specific target properties, such as sensory dimensions (taste, texture, etc.), nutrition, cost, and climate impacts.
 \textit{Prediction tasks} focus on inferring the mapping from a given representation of a food product or system to a property of interest. 
 \textit{Ingredients} are the fundamental components used to create food products, including raw materials (e.g. grains, legumes, oils) and processed components (e.g. protein isolates, flavorings). %that contribute to a food's taste, texture, nutritional profile, and functionality. 
 \textit{Recipes} are structured formulations or instructions for combining ingredients to create food products. %They define the proportions, preparation techniques, and processing steps required to achieve a desired outcome, such as a specific dish or a packaged product. 
 \textit{Systems} refer to the broader processes that govern the production, distribution, and consumption of food. This includes supply chains, foodservice operations, and consumer behavior.%agricultural practices, %For each category (combination of task and resolution), we note both examples of practical challenges faced by food scientists and chefs as well as prior work, if any, applying machine learning to the task in Appendix X. 

We additionally distinguish between novel and traditional foods. Novel sustainable foods, often referred to as ``sustainable proteins" or ``alternative proteins", are defined as foods developed using modern food science and technology to emulate or improve upon animal products with respect to dimensions such as taste. The three main types of sustainable protein technologies are plant-based, cultivated, and fermentation~\cite{gfiScienceAlternative}. Here we focus on plant-based products. Example of novel plant-based products include plant-based meat analogs (e.g. Beyond and Impossible burgers) and plant-based dairy alternatives (e.g. oat milk). While fermentation and cultivated meat are promising approaches for shifting consumption, and we believe our frameworks can be extended to these domains as well, they are outside the scope of the present work. We refer the reader to~\citet{gfiPrecisionFermentation} and~\citet{todhunter2024artificial} to learn about these two technologies.

%First, \textit{prediction tasks} focus on inferring the mapping from a given representation of food to a property of interest. Common properties of interest include sensory dimensions (taste, texture, etc.), nutrition, cost, and climate impacts. At the ingredient level, for traditional foods, models have been designed to predict ingredient pairings and substitutions \cite{zhang2019food,lu2020incorporating,wan2017modeling,mendoza2024foundational}. For novel foods, prediction tasks include estimating the functional properties of .

%Second, \textit{design tasks} focus on creating or optimizing food products or systems to meet specific target properties. 

\subsection{Selected Tasks}
Starting from this general typology, we then selected four tasks to address in the context of both sustainable protein and traditional foods. These tasks include two design tasks (experimental design and menu design) and two prediction tasks (sensory profile prediction and recipe preference prediction). The four tasks were selected based on interviews with food scientists and chefs. %We summarize our tasks in Appendix Table~\ref{tab:selected tasks}.

\noindent \textbf{Experimental Design.}
Taste is a key determinant of food choice~\cite{glanz1998americans}. Food companies depend on ``sensory panels", a group of individuals who provide feedback on food, to routinely quantify sensory properties of their products~\cite{kerth2013science}. 
Once the sensory panel evaluation has been run, food companies will then synthesize the feedback and revise their product accordingly via a series of experiments~\cite{beckley2008accelerating,redefinemeatRoleSensory}. We consider the common task of, given a product formulation and sensory panel feedback, designing a set of experiments to address the identified issues and improve the product.

\noindent \textbf{Menu Design.}
Foodservice operations, such as restaurants and dining halls, aiming to reduce their climate impacts must also balance other factors, such as patron satisfaction, nutrition, and cost. We thus consider the task of revising a menu to promote more sustainable choices while maintaining patron satisfaction and other relevant factors~\cite{banerjee2023sustainable,parkin2022menu}.


\noindent \textbf{Sensory Profile Prediction.}
%Taste is a key, if not primary, determinant of food choice~\cite{glanz1998americans}. Food companies depend on ``sensory panels", a group of trained or untrained individuals who provide feedback on food, to routinely quantify sensory properties of their products~\cite{kerth2013science}. 
%Sensory panel participants are queried not only for their overall satisfaction or purchase intent, but also regarding more specific dimensions of a product's sensory profile such as sweetness, saltiness, juiciness, greasiness, meatiness, etc.~\cite{giezenaar2024profiling,nectarReport}. 
Sensory panel evaluations are very expensive and time consuming to run, and smaller food companies usually cannot afford to run them~\cite{varela2012sensory}. Due to these challenges, we consider the task of estimating a sustainable protein product's sensory profile. While we do not expect to eliminate the need for sensory panels altogether, accurate sensory modeling could reduce the need for experiments and help to efficiently prioritize them. We note that prior work has shown that ML can accurately model the human sense of smell~\cite{lee2023principal}, which is closely related to taste~\cite{spence2015just}. %We are not aware of prior work on using machine learning to predict sensory profiles of alternative protein.

\noindent \textbf{Recipe Preference Prediction.}
Similarly, in the domain of traditional foods, chefs must anticipate patron preferences and design their recipes and menus accordingly. We thus consider the task of predicting the mapping from a traditional recipe to average consumer satisfaction within a specified population. We study how accuracy on this task compares for plant-based versus animal-based recipes. While food preferences of course vary greatly from individual to individual, in both of our prediction tasks the goal is to simulate a representative sample of humans from a specified population~\cite{aher2023using}. %In the language of~\citet{aher2023using}, both of our prediction tasks can be viewed as a ``Turing Experiment".
%\section{Task Definitions}
%The intersection of machine learning and food-related application has gained significant attention, with studies across various communities addressing different aspects of food generation and consumption. Despite the broad interest, less attention has paid on developing a comprehensive evaluation framework that would encompass distinct tasks. To address this gap, from the rich body of existing work, we first identified a comprehensive taxonomy of tasks encompassing ranking, generation, and description, each of which are frequently addressed at three levels of granularity: ingredient, recipe, and menu (Table~\ref{tab:taxonomy}). This taxonomy reflects the complexity of food-related tasks and their intended practical applications, ranging from personalized recommendations to enhancing creativity and menu optimization.

%First, \emph{prediction tasks} focus on evaluating and food-related items to predict and improve user experience and decision-making. At the ingredient level, models have been designed to predict ingredient pairings and substitutions \cite{zhang2019food,lu2020incorporating,wan2017modeling,mendoza2024foundational}. For recipes, ranking often involves assessing relevance or preference based on user queries or dietary needs \cite{zeng-etal-2023-question,li2022ipfc,desai2019recipe}. At the menu level, ranking systems are critical for applications such as restaurant recommendation, leveraging contextual data like user preferences and location \cite{peters2020factors,zhang2016exploiting}. Together, these ranking tasks are crucial in efforts to develop sustainable food systems that meet consumers' preferences.

%Second, \emph{generation tasks} address the creation of food-related content, from ingredient combinations to fully formed recipes and menus. At the ingredient level, generative models have been used to design novel ingredient pairings and formulations \cite{datta2022computer,doherty2021artificial}. Recipe generation has seen significant progress with models like RecipeGPT and RecipeNLG, which synthesize complete recipes from minimal input \cite{h2020recipegpt,bien2020recipenlg,majumder-etal-2019-generating}. Additionally, there has been work on revising or editing existing recipes to accommodate dietary constraints or user preferences \cite{pan2020food,li2022share,antognini2023assistive,liu-etal-2022-plug}, which can be seen as a subtype of constrained generation. At the menu level, generative approaches are less explored but hold promise for crafting cohesive and appealing menus tailored to diverse audiences.

%Third, \emph{description tasks} aim to provide interpretable, engaging, and informative representations of food-related data. For ingredients, descriptive models can communicate nutritional properties or culinary applications \cite{diaz2020international,pecune2022designing}. Recipe-level descriptions often focus on enhancing user engagement and understanding, as seen in persuasive recipe presentation \cite{starke2024tell,pera2024narratives,papies2020using}. At the menu level, research has highlighted the impact of descriptive language on consumer choices and perceptions, emphasizing the role of persuasive language in customer choice \cite{gavrieli2020impact,wansink2005descriptive,roseman2017effects}. The interest in these description tasks highlights the importance of language in influencing food-related decision-making. 


%This taxonomy of food-related tasks provides a comprehensive framework for addressing the diverse tasks found in the literature. By categorizing tasks into ranking, generation, and description across the various granularities, the taxonomy captures the connected nature of these tasks. This structured view highlights the breadth of food-related AI research and offers an opportunity to benchmark models across linked tasks.

%Starting from this taxonomy, we then identified four specific tasks to address in the context of both traditional plant-based food and alternative protein. These tasks include two ranking tasks (sensory profile and preference prediction) and two generation tasks (ingredient improvement generation and menu generation). The four tasks were selected based on expert interviews (i.e., food scientists and dining hall chefs) with whom we identified pain-points and concrete opportunities to assist with machine learning models (see task overview in Table~\ref{tab:selected tasks}).

%For a detailed review of ML/AI applications in cellular agriculture please see~\citet{todhunter2024artificial}.

\section{Datasets}
%\Maya{Might be clearer to introduce each dataset within the task it's used for, especially since additional datasets (Deliveroo, etc.) show up in Task 4.}

\noindent \textbf{NECTAR Sustainable Protein Dataset.}
%The  aims to accelerate the transition to sustainable protein sources via conducting large-scale sensory panel evaluations.
%of the nonprofit organization Food Systems Innovation (\url{fsi.org})
%A sensory panel is a group of (trained or untrained) individuals who evaluate and provide feedback on the sensory attributes of food products.
%NECTAR Initiative 
The NECTAR Initiative's (\url{nectar.org}) sensory panel data\footnote{Access can be requested \href{https://docs.google.com/forms/d/e/1FAIpQLSdYXkB1yEOt_gMBrnHIpHv_xX9peh5NS1kdbfuY8-6-77kSug/viewform?usp=preview}{here}.} is freely available to academic researchers.
%It is not available more generally to reduce harm to the reputations of the companies whose products performed poorly in the taste testing. 
The dataset, which will continue to expand in size, consists of 47 products across five categories. %: bacon, hot dogs, burgers, chicken tenders, and chicken nuggets
One product in each category is a meat-based reference product, and one product is a hybrid beef and mushroom burger, yielding 41 plant-based products. For each product, at least 100 sensory evaluations from American omnivores were performed along 21 quantitative and qualitative dimensions. More details are in Appendix~\ref{app:nectar}.
%Additionally, qualitative, text-based feedback on what each participant liked and disliked about the product was collected. 

\noindent \textbf{Food.com Recipe Dataset.}
%To study traditional plant-based food, we leverage online recipes and reviews collected from \url{Food.com}. %
%.
The \url{Food.com} dataset\footnote{Publicly available \hyperlink{https://www.kaggle.com/datasets/irkaal/foodcom-recipes-and-reviews}{here}.} contains 522,517 recipes, including ingredients and preparation instructions. %, and whether the recipe is plant-based or not. 
We use the associated 1,401,982 reviews, containing ratings and text, to capture online users' preferences. %\Maya{Clever and realistic, though one limitation is that users will self-select into recipes to try based on how much they predict they will like it.} 
Given the large number of recipes, the dataset allows for finding pairs of comparable recipes, i.e., similar in dish type and ingredients, but significantly differing in the users's ratings. %\Maya{Interesting! This reduces the issue I raised above.} 
An example recipe is shown in Appendix~\ref{fig:example-food.com-recipe}.

\section{Experimental Evaluation}
We evaluate \claude, \gemini, \gptthree, \gptfouro, \llama,  and \opreview. In the experimental design task, due to limited availability of expert food scientists, we used \opreview\:only, based on feedback on relative LLM performance from our food scientist team member. We use a combination of automated evaluations and human subjects evaluations, including both food science experts and participants recruited on Prolific. IRB approval was obtained. All evaluations were in a zero-shot setting. Additional details for each task, including prompts and additional results, are in Appendices~\ref{app:methods} and~\ref{app:results}. % for all experiments involving human subjects. %We identify strong zero-shot performance of LLMs in experimental design, but note limitations in constrained menu design, which we then address via incorporating optimization techniques in Section~\ref{sec:augmenting}. For prediction tasks, we find that, in both sustainable protein and traditional foods, LLMs outperform a random baseline, particularly for coarse-grained prediction tasks where the underlying preference gap is large, but do not consistently outperform a simple baseline based on nutritional information. 
%\section{Evaluating LLMs for Design Tasks}
%We first evaluate LLMs for the experimental design and menu design tasks.  IRB approval was obtained for all experiments involving human subjects. \textcolor{red}{Reserve discussion for prediction tasks in section 7.}
\subsection{Experimental Design (Sustainable Protein)}
\noindent \textbf{Methods.}
In this task we study whether LLMs can generate an experimental design for systematically improving a sustainable protein product on the basis of qualitative and quantitative sensory panel data. %Due to limited time availability of expert human food scientists, we used \opreview\:only, based on feedback on relative LLM performance from our food scientist team member.
%(not including education)
We evaluate the performance of \opreview\:via the feedback of 20 expert food scientists, with an average of five years of experience in plant-based food science and three years of experience in plant-based meat specifically. In Phase 1, we ask both \opreview\:and expert food scientists to generate experimental designs.  In Phase 2, we ask food scientists to evaluate both \opreview\:and a fellow food scientist (blinded, and in randomized order) along the dimensions of accuracy, specificity, complementarity (to the evaluating food scientist's own thought process), and estimated time saved by collaborating with the anonymous ``scientist.'' We ensured that no food scientists evaluated their own ideas. The food scientists were compensated via \$50 Amazon gift cards, and an additional \$50 bonus was provided to the highest scoring food scientist from Phase 1 to incentivize effort. Before Phase 2, following~\citet{si2024can} we perform style standardization (also with \opreview) on both the LLM and human responses to avoid confounding by style. In order to approximately match the length of the human responses, \opreview\:was instructed to limit its response to  250 words. 
% (either \opreview\:or a human food scientist)
%Further details, including prompts, are in Appendix~\ref{sec:additional-methods-exp-design}.%Lengths are shown in Appendix Table~\ref{tab:exp-design-original-edited-lengths}. 







\noindent \textbf{Results.} Our results are shown in Figure~\ref{fig:task2}. Across the 30 products, the mean performance of LLMs was higher on all four dimensions, though only statistically significant (in a $t$-test) for specificity ($p$=0.003) and percent time saved ($p$=0.0002). Mean estimated percent time saved was 22\% for the human food scientists and 45\% for o1-preview. %Percent time saved was computed based on two questions: 1) an estimate for how long creating a full experimental design would take and 2) how much time collaborating with the other scientist would save. Mean specificity was 6.63 for the humans and 8.23 for o1-preview. Mean percent time saved was 22\% for the humans and 45\% for o1-preview.





\subsection{Menu Design (Traditional Foods)}\label{sec:init-menu-design}
\noindent \textbf{Methods.} Here we study what LLMs can contribute to greater adoption of existing sustainable foods. Extensive past work has studied how to shift consumption toward sustainable options in foodservice operations~\cite{lohmann2024choice,attwood2020menu,banerjee2023sustainable,weijers2024nudging}. However, these studies either obtained relatively small improvements in climate impact or did not measure patron satisfaction with their set of choices. Additionally, none accounted for the trade-off known as the ``small body problem", in which simply substituting beef by chicken or fish, a common approach for reducing climate impacts, can be much worse for animal welfare due to the relative sizes of these animals and their typical conditions~\cite{mathur2022ethical}. Public opinion polling suggests high levels of concern for animal welfare~\cite{gallupUSMore}. Given that LLMs' training data includes many recipes and menus, we explore what LLMs can contribute to designing appealing menus under the constraints of ingredient availability, greenhouse gas (GHG) emissions, and animal welfare. Our final approach, described in Section~\ref{sec:augmenting}, can also easily be extended to other constraints such as nutrition, cost, allergies, other dimensions of climate impact, etc.
% \textcolor{red}{(maybe include numbers)}

\begin{figure}[ht!]
\centering
\includegraphics[width=0.5\textwidth]{task2.pdf}
\caption{\opreview\:meets or exceeds the performance of expert food scientists on a sustainable protein design task. \opreview\:and food scientists were asked to generate experimental designs for improving  product formulations in response to sensory panel feedback, a common task for food scientists. Error bars are 95\% confidence intervals. Example outputs are in Appendix~\ref{app:results-experimental-design}.}
%$n$ = 30 products. 
\label{fig:task2}
\end{figure}
%The exact survey questions for both Phase 1 and Phase 2 are shown in Appendix X. 

As a reference menu, we use a 36-item menu from~\citet{banerjee2023sustainable}, created using Deliveroo's and Just Eat's top 100 items ordered in the United Kingdom in 2019. %Our initial prompt is shown in Appendix Figure~\ref{fig:prompt-task4-direct}. 
Each LLM was instructed to reduce emissions of food choices by 75\% while maintaining satisfaction, cost, nutrition, preparation time, and animal welfare, and using only the same ingredients as the original menu other than standard vegetarian protein sources (most of which were already in the original menu, other than eggs and chickpeas). Adherence to the ingredient availability constraint, as well as meeting the constraint of generating 36 properly formatted recipes, was checked, and the LLM was given up to five chances to correct itself. GPT-3.5 Turbo and \llama\:did not meet the formatting or ingredient constraints. %Additional experimental details, including prompts and evaluation of climate and animal welfare impacts, are in Appendix~\ref{sec:appendix-methods-menu-design}. 

%A concrete case we are motivated by, that informs our specific choice of experimental evaluation, is shifting consumption at university dining halls. This case has been the focus of past work \citep{vocskieffect,jalil2023low,turnwald2019increasing}, both due to the quantity of meals served - 750,000 per day across the 57 ``Menus of Change'' consortium of universities, an initiative to shift dining hall consumption in a more healthy and sustainable direction, alone \citep{menusofchangeAboutMenus} - and due to the opportunity to raise awareness of sustainable diets at a formative time in student development. A natural question is how to reduce emissions of dining hall menus while meeting other constraints and maintaining satisfaction. Additionally, some universities, such as the University of North Texas, EPFL, and University of Illinois at Urbana-Champaign, operate fully plant-based dining halls \citep{untMeanGreens,epfl,illinoisNutritionDietary}. A related question is how to encourage students to patronize these lower emissions dining halls via appealing menus. Thus, in our evaluation, we focus on generation of lower emissions menus and evaluate on human preferences, cost, nutrition, and animal welfare in comparison to a dining hall menu created by human chefs.

%While LLMs are able to generate plausible recipes, their performance on mathematical optimization tasks is inconsistent,  as also shown in \citet{yang2024large}.  %To address this weakness, we design an approach combining LLMs with integer quadratic programming (IQP) to explicitly optimize for predicted patron satisfaction, emissions, and animal welfare.  

\noindent \textbf{Results.}
We found that this approach yielded entirely vegetarian or vegan menus (shown in Appendix~\ref{sec:menus}), known to generally decrease satisfaction in a typical omnivore population~\cite{hartmann2017consumer}. This decrease in satisfaction was further supported in our later experiments (Figure~\ref{fig:task4}), motivating our approach described in Section~\ref{sec:augmenting}.
%noted that prompting LLMs directly to revise this menu to promote sustainable choices (at least a 75\% reduction in GHG emissions) while not worsening patron satisfaction, cost, nutrition, preparation time, or animal welfare (prompt shown in Appendix) 


\begin{table*}[t!]
\small
\centering
\begin{tabular}{llllllll}
\toprule
 & \claude & \gemini & \gptthree & \gptfouro & 
Llama & \opreview & Baseline \\
\midrule
\multicolumn{8}{l}{
\textbf{Sensory Profile Prediction: LLM Prediction vs. Sensory Panel (Sustainable Protein)}} \\
\midrule
All Dimensions & \textbf{0.61} & 0.53 & 0.52 & \textbf{0.59} & \textbf{0.57} & \textbf{0.64} & \textbf{0.67} \\
Overall Satisfaction & 0.60 & 0.41 & 0.38 & 0.39 & 0.51 & 0.64 & \textbf{0.73} \\
Meatiness & 0.61 & 0.57 & 0.51 & 0.62 & 0.52 & 0.66 & 0.48 \\
Greasiness & 0.58 & \textbf{0.64} & 0.63 & \textbf{0.74} & \textbf{0.73} & \textbf{0.64} & \textbf{0.69} \\
Juiciness & 0.62 & 0.60 & 0.63 & 0.63 & 0.51 & 0.64 & \textbf{0.71} \\
Sweetness & 0.36 & 0.61 & 0.61 & 0.61 & 0.50 & 0.64 & 0.75 \\
Saltiness & \textbf{0.70} & 0.67 & 0.57 & \textbf{0.73} & \textbf{0.70} & \textbf{0.75} & \textbf{0.73} \\
Purchase & 0.65 & \textbf{0.29} & 0.37 & 0.45 & 0.43 & 0.54 & 0.61 \\

\midrule
\multicolumn{8}{l}{\textbf{Recipe Preference Prediction (\url{Food.com}): LLM Prediction vs. Rating (Traditional Foods)}} \\
\midrule
Rating & \textbf{0.63} & 0.48 & 0.57  & \textbf{0.64}  & 0.51 & \textbf{0.60} & 0.50 \\
\midrule
\multicolumn{8}{l}{\textbf{Recipe Preference Prediction (Menu Design): LLM Prediction vs. Order Frequency (Traditional Foods)}} \\
\midrule
Order Frequency & \textbf{0.73} & \textbf{0.62} & -  & \textbf{0.67}  & \textbf{0.60} & \textbf{0.56} & 0.50 \\
\bottomrule
\end{tabular}
%as judged by people who have tasted the product
\caption{We evaluate LLMs' pairwise ranking accuracy for three forms of preference prediction: 1) the mapping from a sustainable protein product's ingredients and nutritional information to sensory properties 2) the mapping from a recipe to its online rating 3) the mapping from a recipe to frequency of orders in a hypothetical restaurant. Sample sizes are in Appendix Table~\ref{tab:prediction-sample-sizes}. Statistically significant results, based on a chi-squared test, are in bold. For the sustainable protein task, the baseline is described in Section~\ref{sec:sensory-profile-prediction}. For recipe preference prediction, the baseline corresponds to a random choice. \gptthree\:did not produce a valid output in the menu design experiment. %and thus it was not tested. %\textcolor{red}{TODO: Explain GPT-3.5 being missing, and the menu design experiment's inclusion. }
}
\label{tab:sensory-profile-prediction-results}
\end{table*}
%\section{Evaluating LLMs for Prediction Tasks}
%Finally, we evaluate LLMs on preference prediction tasks. 
%\section{Evaluation of LLMs}
%In Tasks 1-3, we evaluate six LLMs: \claude, \gemini, \gptthree, \gptfouro, \llama,  and \opreview. In Task 4, due to limited time availability of expert human food scientists, we used \opreview\:only, based on feedback on relative LLM performance from our food scientist team member. We use a combination of automated evaluations and human subjects evaluations, including both food science experts and participants recruited on Prolific. %IRB approval was obtained for all experiments involving human subjects (Tasks 2 and 4).% under assurance number FWA00000935.
\subsection{Sensory Profile Prediction (Sustainable Protein)}\label{sec:sensory-profile-prediction}
\noindent \textbf{Methods.}
Here we study whether LLMs can make accurate predictions about the mapping from a sustainable protein product's ingredients and nutritional information to its sensory properties. Of the 21 sensory dimensions in the NECTAR dataset, seven dimensions (overall satisfaction, meatiness, greasiness, sweetness, juiciness, saltiness, purchase intent) were selected for testing, to reduce the number of tests.
%A tool for accomplishing this task could accelerate the development of alternative protein with appealing sensory properties (\textcolor{red}{CITE}). We use the NECTAR alternative protein dataset, in which five categories of alternative protein - burgers, hot dogs, chicken nuggets, chicken tenders, and bacon - were evaluated on several sensory dimensions. 
Within each of the five product categories, and each sensory dimension, we ask LLMs to compare pairs (with randomized ordering, to account for positional biases) of plant-based products with a statistically significant ($p < 0.05$ in a $t$-test) difference on the sensory dimension, as evaluated by human omnivores. This yielded 495 pairs. For the baseline, we use the available nutritional information and prior knowledge from the food science literature. Specifically, for greasiness and juiciness, meatiness, sweetness, and saltiness, the ratio of total fat, protein, sugar, and sodium respectively per serving to serving size are used to predict the sensory dimension. For example, the product that is higher on protein is ranked higher on meatiness according to the baseline. % No baseline was tested for overall satisfaction, purchase intent, meatiness, and smokiness %\textcolor{red}{(could attempt to construct a baseline, e.g. averaging fat and sodium for overall satisfaction or purchase intent)}. 
For overall satisfaction and purchase intent, the average of normalized fat and sodium content is used~\cite{drewnowski1983cream,mattes1997taste}. %Additional details are Appendix~\ref{sec:sensory-profile-prediction}.
%Sample sizes per sensory dimension are shown in Appendix Table~\ref{tab:sensory-profile-pairs}. Our prompt is shown in Appendix Figure~\ref{fig:sensory-profile-prompt}.
%\vspace{-3em}

\noindent \textbf{Results.}
Accuracies are shown in Table~\ref{tab:sensory-profile-prediction-results}. Accuracies significant according to a chi-squared test with Bonferroni correction for multiple testing are bolded~\cite{vanderweele2019some}. Performance is generally not meaningfully superior to the baseline. Accuracy for the best  (across all dimensions) LLM, \opreview, improves in the highest quartile of the ground truth difference between the recipes, e.g. to 81\% and 75\% for overall satisfaction and purchase intent. However, it is still not meaningfully superior to the simple baseline, which achieved 86\% and 70\% accuracy on those dimensions in this subset of pairs (Appendix Table~\ref{tab:sensory-profile-prediction-results-q4}). 

%While no LLMs achieve a statistically significant accuracy on purchase intent, Claude 3.5. Sonnet achieves 67\% accuracy on overall satisfaction, and also outperforms a random baseline on meatiness, greasiness, smokiness, juiciness, and saltiness. %Tables with accuracies disaggregated by product category are shown in the Appendix.





\subsection{Recipe Preference Prediction (Traditional Foods)}



\noindent \textbf{Methods.} We created a set of 500 meal recipe pairs with similar ingredients but significantly different average ratings ($p<0.05$ in a $t$-test), ensuring that one recipe is consistently preferred by users. To identify recipe pairs with the highest similarity, the overlap of ingredients was calculated for each pair using the ratio of shared ingredients to the total number of unique ingredients. We addressed the possibility of overlap with LLM training data by curating an additional set of 58 recipe pairs from \url{epicurious.com}. These recipes were posted behind a paywall after the release dates of all the models we tested.
%$\{recipe_1, recipe_2\}$
%We evaluated six models in a zero-shot setting. %, using default parameters (e.g., temperature, maximum tokens). 
Our primary metric was accuracy---whether the models correctly identified the preferred recipe---compared to a random baseline. We tested statistical significance using a chi-squared test with Bonferroni correction when evaluating on multiple conditions. The order in which the recipes were shown was randomized. %, to account for positional biases. 
%Additional details are in Appendix~\ref{app:recipe-rating-prediction}.
%The prompt is shown in Figure~\ref{fig:prompt-recipe-rating}. %We used the following prompt text:

\noindent \textbf{Results.} \claude, \gptfouro, and \opreview\:outperform the random baseline (Table~\ref{tab:sensory-profile-prediction-results}). The highest accuracy achieved is 64\% (\gptfouro), with accuracy improving to 85\% in the highest quartile of the ground truth review score gap (Appendix Table~\ref{tab:task3a}).
%useful but not highly accurate
Moreover, performance varies widely by comparison type (Appendix Table~\ref{tab:task3b}). Across all the models, accuracy is the highest when two non-vegetarian meals are compared, followed by vegetarian vs. non-vegetarian comparisons, with the worst accuracy observed in vegetarian vs. vegetarian meal comparisons. This aligns with existing literature showing that minority preferences and opinions are modeled less accurately~\cite{santurkar2023whose}. Within the vegetarian vs. non-vegetarian meal comparisons, accuracy is lower when vegetarian options are rated higher, reflecting LLMs' omnivore bias, as LLMs more frequently prefer non-vegetarian options compared to ground truth human preferences. This finding is consistent with the literature on bias amplification in the food domain~\cite{luo2024othering} and points to the need to improve preference prediction performance in order to develop tools that can better  support the development of sustainable food. Finally, similar pairwise accuracy across models on post-cut-off data suggests that performance cannot be fully attributed to memorization (Appendix Table~\ref{tab:task3c}).%the reference signal

%\Maya{Very interesting.}

\section{Augmenting LLMs with Combinatorial Optimization}\label{sec:augmenting}

%Lastly, to demonstrate potential future directions and ways to improve performance on the four tasks, we build upon menu design task, as one use-case. Similar to constrained menu generation in focus here, many real-world design problems require optimizing for human preferences under hard constraints. Whether designing menus, fitness programs, or educational curricula, practitioners must balance subjective quality with quantifiable requirements like emissions, time budgets, or learning objectives. While LLMs show promise in modeling human preferences, they struggle with precise constraint satisfaction~\cite{yang2024large}. We propose a general framework combining LLMs' preference modeling and unconstrained generation capabilities with classical optimization techniques.
%As shown in Appendix~\ref{sec:menus}, w
When prompted directly to revise a menu to reduce emissions by at least 75\% while maintaining patron satisfaction and other factors, LLMs generate entirely plant-based menus (Section~\ref{sec:init-menu-design}, Appendix~\ref{sec:menus}). While this approach achieves the emissions target, it significantly decreases satisfaction (Figure~\ref{fig:task4}). This failure highlights a broader, previously documented challenge: LLMs struggle with optimization problems~\cite{pmlr-v235-ahmaditeshnizi24a,yang2024large}.
However, frequently in real-world applications we need to consider some subjective, human-evaluated property (like satisfaction) while also satisfying other constraints (like emissions targets or budgets). For instance, in education, we may want to maximize student engagement while meeting curriculum requirements. In fitness, we may aim to maximize enjoyment while covering specified muscle groups within time constraints. In travel planning, we may seek to maximize trip enjoyment while meeting budget and scheduling constraints. Additionally, in some settings it may take considerable effort to quantify dimensions such as emissions or other climate impacts of a food item.

We propose to address these challenges by combining LLMs' background knowledge and ability to model human preferences, demonstrated in~\citet{horton2023large},~\citet{park2022social} and in our own experiments, particularly for coarse-grained prediction (Tables~\ref{tab:sensory-profile-prediction-results},\ref{tab:menu-preference-rating-q4},\ref{tab:sensory-profile-prediction-results-q4},\ref{tab:task3a}), with mathematical optimization. While LLMs alone struggle with  optimization, and traditional optimization methods do not have access to LLMs' background knowledge, combining them allows for leveraging the strengths of both approaches.
%\Moses{previous sentence is long and somewhat awkwardly constructed. rephrase?}
%cannot easily capture subjective human preferences, which may be hard to formalize mathematically, or leverage extensive


%Could there be a better way to balance these three factors? More generally, this problem is an example of maximizing subjective quality (according to human judgment), subject to constraints. Addressing such multi-objective optimization problems is critical in many real-world applications, highlighting the need for methods that can navigate these complex trade-offs effectively. Similar problems arise in settings such as education (course planning), health (designing fitness regimens) or planning trips, e.g. maximizing enjoyment while covering muscle groups or meeting time and cost constraints~\cite{roy2011interactive}. 

\subsection{LLM-Guided Combinatorial Optimization}
%Building on recent literature suggesting that LLMs can model human perceptions and behavior~\cite{park2022social,horton2023large}, as well as our own results showing that LLMs can model human food preferences (Figure X), we propose to model subjective quality, as assessed by humans, via the outputs of an LLM.
We begin with the general constrained optimization setting:
\begin{equation}
\begin{aligned}
\max_{x} \quad & f(x)\\
\textrm{s.t.} \quad & g_i(x) \leq 0 \: \forall i \in [1, ..., m] \\
\end{aligned}\label{eq:general-opt}
\end{equation}
We make three assumptions:
\begin{enumerate}
\itemsep0em 
    \item The optimization problem involves selecting a subset of items $S$ from a ground set $U = \{u_1, ..., u_N\}$. Each $u_i$ represents an item, e.g. a recipe or exercise. The selection is represented by a binary vector $x \in \{0,1\}^N$, where $x_i = 1$ if $u_i$ is selected, and 0 otherwise. %As in our motivating examples, $x$ (e.g. a menu or fitness program) is a set of items $S = \{s_1, ..., s_K\}$, e.g. a set of recipes or exercises, to be chosen from a ground set $U = \{u_1, ..., u_N\}$, where $|U| = N > K$. 
    \item At least one of $f(x)$ or the constraints $g_i(x)$ can be expressed in terms of a \textit{scoring function} $p(u_i)$, which computes a scalar property (e.g. satisfaction) for each item $u_i$. The scoring function can be estimated by an LLM. For example, if the objective is to maximize the total satisfaction of the end user with the selected items, $f(x)$ can be written as $f(x) = \sum_{i=1}^N p(u_i)x_i$.
    %At least one of $f(x)$ and $g_i(x)$ can be written as $h(p(x_1), ..., p(x_K))$ for a \textit{scoring function} $p(x_i)$, e.g. satisfaction with a recipe, which can be estimated by an LLM. For example, if the objective function is simply to maximize estimated sum of satisfaction of individual recipes, $h(x) = \sum_{i=1}^K p(x_i)$. %Additionally, a constraint $g_i(x)$ could be on expected emissions, based on the probability distribution associated with $p(x)$. For example, $g_i(x)$ could be $\frac{\sum_{i=1}^K p(x_i) e(x_i)}{\sum_{i=1}^K p(x_i)} - C$, where $e(x_i)$ is the emissions for recipe $x_i$ and $C$ is the emissions target.
    %\Moses{Just to clarify: you mean that either $f(x)$ is of this form, OR all the $g_i(x)$ are of this form? Is there a restriction on the combining function $h()$?}
    \item Once $p(u_i)$ is estimated for each $u_i$ in $U$, (\ref{eq:general-opt}) has a tractable form and can be solved exactly or approximately via standard techniques, e.g. submodular optimization or integer programming for small to medium problem sizes~\cite{fujishige2005submodular,wolsey2020integer}. We give a concrete example in Section~\ref{sec:constrained-menu-design}.
    %\Moses{This was confusing to me. $x=\{x_1,\ldots,x_K\}$ is a variable. First, a small typing issue: is $p(x) = (p(x_1),\ldots,p(x_K))$ What does it mean to estimate $p(x)$? Does this mean that you estimate the form of $p()$ from which you can compute $p(x)$ for any $x$? Or, do you mean that you can estimate $p(x)$ for a given value of $x$?}
\end{enumerate}
 
%We consider the case, .  %We additionally assume that $f(S) = \sum_{x \in S} f(x) + \lambda d(S)$, e.g. that satisfaction with a menu or fitness regimen can be decomposed linearly as the sum of satisfaction of the individual recipes or exercises, plus a diversity term $d(S)$. 
We can then solve~(\ref{eq:general-opt}) via the following steps:
\begin{enumerate}
\itemsep0em 
\item Generate the ground set $U = \{u_1, ..., u_N\}$, e.g. a diverse set of recipes or exercises.%, where $N > %K$.
\item Obtain the estimates $\hat{p}(u_i)\: \forall i\in [1, ..., N]$ via an LLM.
\item Solve the combinatorial optimization problem using standard techniques, e.g. submodular optimization or integer programming, depending on the forms of $f(x)$ and $g_i(x)$. This will yield a subset $S \subset U$. %d(S)$ and $g_i(x)$. 
\end{enumerate}
%\Moses{Related to my comment above, I think I might have realized the source of my confusion: you use $x_1, \ldots x_N$ to denote the elements of the ground set, but also the variables in the combinatorial optimization problem. You should have distinct notation for them. Also, what are the values of the variables in your constrained optimization problem? I presume they are indices in \{1,\ldots N\}?}
%, as we do in our menu design experiments,
Step 1 could be performed via an LLM or via other techniques. We discuss the relationship of this framework to prior work in 
Appendix~\ref{sec:extended-related-work}. A natural question is how performance depends on the accuracy of the LLM's estimates $\hat{p}(u_i)$, and other parameters. For our analysis we consider the cardinality constrained setting where $f(x)$ has a component that is linear in $p(u_i)$ for the selected subset, and another component $d(x)$ that does not depend on $p(u_i)$. Concretely, we aim to maximize $f(x) = \sum_{i=1}^N p(u_i)x_i + \lambda d(x)$ subject to the constraint $\sum_{i=1}^N x_i = K$, though this result can be extended to more general objectives and constraints.
%$f(S) = \sum_{u_i \in S} p(u_i) + \lambda d(S)$ subject to a cardinality constraint $|S| \leq K$

\begin{proposition}
Let $x^{\ast}$ be the optimal solution with respect to actual preferences $p(u_i)$, and $\hat{x}^{\ast}$ be the optimal solution with respect to estimated preferences $\hat{p}(u_i)$. Let $p(u_i), \hat{p}(u_i) \in [0,1]$, $d(x) \in [0,D]$. If $|\hat{p}(u_i) - p(u_i)| \leq \epsilon$ for all $u_i \in U$, then $|f(\hat{x}^{\ast}) - f(x^{\ast})| \leq 2K\epsilon + 2\lambda D$.
\end{proposition}\label{prop:llm-accuracy}

A simple proof is in Appendix~\ref{sec:proof-prop-1}. Thus, the error can be bounded as a linear function of the number of items selected and the maximum item-level error of the LLM.

\subsection{Constrained Menu Design Formulation}\label{sec:constrained-menu-design}
Now we show how to implement this framework for the setting of constrained menu design. First, we prompt the LLM to generate a set $C$ of plant-based candidate recipes, though this could be modified to allow for also generating animal-based recipes at this stage. As in Section~\ref{sec:init-menu-design}, the LLM is also instructed to not worsen cost, nutrition, or preparation time. We then combine this set with the recipes in the original menu $O$, where $K = |O|$, to form the ground set $U = C \cup O$. We then estimate $p(u_i)\: \forall u_i \in U$ by prompting an LLM to rate each recipe in the ground set by expected preferences of the population of interest (here, American omnivores). Then, integer quadratic programming (IQP) is applied to select a subset of recipes from $U$ for the menu.%(from both the LLM-generated set and recipes from the original menu)

More specifically, we maximize expected satisfaction with the selected recipes, plus a $\lambda d(x)$ term, where $\lambda$ controls the weight on diversity, to encourage a diverse set of recipes, subject to constraints on climate impacts and animal welfare. This approach easily extends to other constraints such as nutrition and cost. The predicted preferences $\hat{p}(u_i)$ are used to compute expected emissions and animal welfare impacts of choices by weighting more preferred options more highly, under the assumption that these options will be chosen more often. %\Maya{I'd move this intuition piece above the algo and have the notation immediately after the algo.} %We define $\mathbb{E}_{p,r}[c_j]$ as $\sum_i \frac{p_i}{1^Tp} r_i c_{ji}$. 
Our problem formulation is thus as follows:
\begin{equation}
\begin{aligned}
\max_{x} \quad & \sum_{i=1}^N \hat{p}(u_i)x_i + \lambda d(x)\\
\textrm{s.t.} \quad & E_{\hat{p}}[l_j(x)] \leq C_j E_{\hat{p}}[l_j(x_O)] \: \forall j \in [1, 2] \\
%\quad &  m^Tc \leq C_c \\
%\quad &  m^Tn \leq C_n \\
%\quad &  m^Ta \leq C_a \\
& \sum_{i=1}^N x_i = K  \\
& x_i \in \{0,1\}\:\forall i \in [1, ..., N] \\
%& r_i \in \{0,1\} \: \forall i %\in [1,...,n] \\
\end{aligned}\label{eq:menu-formulation}
\end{equation}
$x_O$ corresponds to the subset of recipes in the original menu. $E_{\hat{p}}[l_j(x)]$ computes the expected emissions or animal welfare impacts of the chosen subset of recipes, with respect to the probability distribution associated with $\hat{p}(u_i)$. %Specifically, $E_p[l_j(S)] = \sum_{u_i \in S} e_j(u_i)p(u_i)$, where $p(u_i) \in [0,1]$ and $e_j(u_i)$ is the item-level constraint value, e.g. emissions of an individual recipe. 


We set $d(x)$ to be negative pairwise similarity between the recipes in the selected subset. This yields an IQP since all other terms are linear in $x$. %the optimization variable, a binary vector $x$ with each entry corresponding to whether a recipe in $U$ is selected. 
$C_i$ controls the reduction in emissions or animal welfare, and is set to 0.25 for emissions and 1.0 for animal welfare, corresponding to reducing emissions by at least 75\% while maintaining or improving upon animal welfare. We set $K=36$, to match the length of the original menu from~\citet{banerjee2023sustainable}. We instruct the LLM to generate 20 additional recipes, reflecting that a restaurant may not be able to completely revise their full menu. Thus, $N=56$. Finally, we set $\lambda=100$. For computing pairwise recipe similarity we use the Ratcliff/Obershelp sequence matching algorithm as implemented in Python's difflib \citep{ratcliff1988pattern}. The recipes were ordered first by whether they were LLM generated, and second by predicted rating. We included a baseline in which the original menu was simply re-ordered so that vegetarian items were first, to account for the possibility that benefits of the LLM+IQP approach were simply due to placing the LLM generated (and therefore plant-based) recipes first.



\begin{figure*}[ht!]
\centering
\includegraphics[width=0.85\textwidth]{average_emission_satisfaction_by_arm_v2.pdf}
\caption{Our o1-preview+IQP approach reduces GHG emissions of food choices by 79\% compared to the original menu from~\citet{banerjee2023sustainable} while maintaining participants' satisfaction with their set of choices. o1-preview+IQP additionally outperforms the baselines of removing meat from the original menu (``Vegetarian") and removing beef (``Remove Beef"), and maintains animal welfare and other satisfaction metrics, as shown in Appendix~\ref{appendix:task4}. As an ablation we also prompt o1-preview directly to revise the menu. Error bars are 95\% confidence intervals. Statistical significance was assessed with a $t$-test, with Bonferroni correction for multiple hypothesis testing.}
\label{fig:task4}
\end{figure*}


%\begin{equation}
%\begin{aligned}
%\max_{r} \quad & r^Tp - \lambda \sum_i \sum_j r_ir_js_{ij}\\
%\textrm{s.t.} \quad & \frac{\sum_i p_i r_i c_{ji}}{\sum_i p_i r_i} \leq C_j \forall j \in [1, ..., m] \\
%\quad &  m^Tc \leq C_c \\
%\quad &  m^Tn \leq C_n \\
%\quad &  m^Ta \leq C_a \\
%& 1^Tr = K  \\
%& r_i \in \{0,1\} \forall i \in [1,...,n] \\
%\end{aligned}\label{eq:ilp}
%\end{equation}

%Here, $r \in \{0,1\}^{n}$, where $n$ is the number of recipes in the ground set (both LLM-generated and the original recipes), is the optimization variable. Each entry $r_i$ reflects whether recipe $i$ is chosen for the final menu. $p \in [1, ..., n]$ is the preference ratings from the LLM. $\lambda$ is a user-defined parameter specifying the tradeoff between menu diversity and satisfying predicted recipe-level preferences; higher values of $\lambda$ place a greater weight on diversity. 
%$s_{ij}$ is a measure of similarity between recipes $i$ and $j$. 
%There are $m$ constraints (in our experiments $m=2$, corresponding to emissions and animal welfare), each associated with recipe-level values in $c_j \in \mathbb{R}^n$ and budget $C_j \in \mathbb{R}$. $K$ recipes are required to be selected, to match the length of the baseline human-created menu, so here $K=36$. Figure~\ref{fig:approach-2} depicts our approach. In our experiments we set $C_1$ to be 25\% of the expected (computed based on the predicted preferences $p$) emissions impact of the original menu, and $C_2$ to be equal to the expected animal welfare impact of the original menu. This corresponds to a 75\% emissions reduction while maintaining or improving on animal welfare. %For Approach 2, the baseline is a menu created by human chefs - specifically, the Arrillaga Dining breakfast menu for March 19, 2024. We are unaware of previous published work combining LLMs with combinatorial optimization for menu generation.

We then evaluated our approach via two human subjects experiments (total $n=552$) in which participants recruited via Prolific, constrained to be in the United States and fluent English speakers, were instructed to make a selection from a menu. In the first experiment, the menu was randomly assigned from one of the five LLMs (augmented with IQP) and three baselines (original menu, original menu with beef removed, and a vegetarian subset of the original menu).\footnote{Removal of beef is a common strategy for reducing climate impacts~\cite{grummon2023simple}.} In the second experiment, conducted the following day, the menu was randomly assigned from one of three additional baselines (\opreview\:on its own, without IQP; \opreview+IQP, but with the \opreview\:generated descriptions replaced with a simple list of ingredients; and placing the vegetarian items first in the original menu). The participants were paid \$16 per hour. The evaluation metrics consisted of satisfaction metrics (satisfaction with their set of choices, likelihood of visiting, and likelihood of recommending the restaurant to a friend), emissions of food choices, and animal welfare impacts of food choices. %Additional details are in Appendix~\ref{sec:appendix-methods-menu-design}.


\subsection{Results}
The results for the best LLM, o1-preview, are shown in Figure~\ref{fig:task4}, with results for other LLMs and baselines in Appendix~\ref{appendix:task4}. When controlling for dietary preference, age, gender, and race, we estimate that the \opreview+IQP approach reduces emissions by 79\% while maintaining patron satisfaction with their set of choices ($p$=1.37e-18). We obtain similar results when removing vegan and vegetarian participants (Appendix~\ref{appendix:task4}). We observe a small (5\%) reduction in the likelihood of recommending the restaurant to a friend, and a small increase in animal usage (3.5\%), but these were not statistically significant. The original, baseline, and generated menus are shown in Appendix~\ref{sec:menus}. We also evaluated the accuracy of the LLM's predicted preference ratings, in the same pairwise ranking fashion as the sensory profile and recipe preference prediction tasks, with the best LLM, \claude, achieving 73\% accuracy (Table~\ref{tab:sensory-profile-prediction-results}). This increases to 85\% in the highest quartile of the preference gap (Appendix Table~\ref{tab:menu-preference-rating-q4}). We note that, for the menu design task, LLM predictions were based on the recipe title alone (not including the full description), which could explain the discrepancy between item-level prediction performance and menu-level performance for \opreview. When \opreview's descriptions were replaced with a simple list of ingredients (in one of our ablations), its accuracy was 70\%, and 84\% in the highest quartile of ground truth preference gaps (Appendix Table~\ref{tab:menu-preference-rating-q4}).
%\textcolor{red}{TODO: could translate to pairwise ranking accuracy to match Tasks 3 and 4, perhaps could add a line to Table 2. Also mention that percentage of plant based and LLM generated choices goes up a lot under o1-preview+IQP.}.  
%\textcolor{red}{Conclusion sentence. LLM-guided combinatorial optimization improves by X\%, future, etc.}
%\textcolor{red}{Sentence on accuracy of the predicted preferences - around 60-70\%, similar to the other preference prediction tasks}.





%The results for the best LLM, o1-preview, are shown in Figure~\ref{fig:task4}, with additional results in the Appendix. The original, baseline, and generated menus are also shown in the Appendix. 



\section{Discussion}
%taken together
Our results identify both strengths and weaknesses of LLMs in the domain of sustainable food. One actionable finding is that, as determined by expert food scientists, o1-preview appears to be useful, even superior to fellow expert human food scientists, in generating ideas for revision of sustainable protein products in response to sensory panel feedback. Our evaluation suggests that the benefits may be driven by increased specificity in particular, relative to the human food scientists (Figure~\ref{fig:task2}). LLMs display a weakness in the menu design task, where they are instructed to satisfy multiple constraints. They generate fully plant-based menus, which we find to significantly reduce satisfaction. We remedy this by integrating LLMs with combinatorial optimization techniques. Our approach achieves a 79\% emissions reduction while maintaining patron satisfaction (Figure~\ref{fig:task4}). Our work also contributes to the literature on reasoning in LLM-based systems (further discussed in Appendix~\ref{sec:extended-related-work}), providing an approach for solving a broad class of optimization problems that cannot be easily specified mathematically, e.g. those involving human preferences. Future work can apply this framework to other domains such as education and health. 

%, which may be of independent interest in other settings involving optimization problems that are not fully specified mathematically (e.g. involving human preferences)

We evaluated LLMs on three types of preference prediction. The first is sensory profile prediction, in which we compared LLM predictions to evaluations from an actual taste tasting. The second is the \url{Food.com} recipe rating task, where we compared LLM predictions to online recipe ratings. The third is recipe rating in the context of the menu design experiment, where the LLM’s recipe-level predictions are compared against the order frequency for the recipes that were selected for the final menu. For the holistic measures of overall satisfaction and purchase intent in the sensory profile prediction task, no LLMs outperform a random baseline, though some do on specific dimensions such as greasiness or saltiness. For online recipe rating, three of six LLMs outperform a random baseline. We note that all evaluated LLMs outperform a random baseline in context of the menu design experiment, where no actual tasting occurs and the orders are purely based on the text of the recipe (Table~\ref{tab:sensory-profile-prediction-results}). Additionally, we studied how performance varies with the magnitude of the ground truth preference gap. When restricting the evaluated recipe pairs to be in the top quartile of preference gaps, LLM performance improves, with the best LLMs achieving 81\%, 86\%, and 85\% accuracy respectively in the three prediction tasks (Appendix Tables~\ref{tab:menu-preference-rating-q4},\ref{tab:sensory-profile-prediction-results-q4},\ref{tab:task3a}). Thus, our analysis suggests that LLMs can be useful for coarse-grained preference prediction tasks, but that further work is needed for fine-grained prediction, particularly for the mapping from recipes or product formulations to actual taste.
Additionally, consistent with past research on LLMs' modeling of minority preferences~\cite{santurkar2023whose}, our results showed that LLMs model human preferences related to plant-based food less accurately. Thus, regular human feedback, e.g. taste testings, from diverse populations remains critical. 
Overall, these results suggest that LLMs can be useful in experimental design, generation of sustainable recipes (as evaluated in the menu design task), and coarse-grained preference prediction. 
%given ingredient constraints

%Our analysis identifies both strengths and weaknesses of LLMs in the domain of sustainable food. One actionable finding is that, as determined by expert food scientists, o1-preview appears to be useful, even superior to fellow expert human food scientists, in generating ideas for revision of plant-based meat products in response to sensory panel feedback. While o1-preview on its own does not perform well on our menu design task, combining o1-preview with combinatorial optimization techniques is able to reduce average emissions of food orders by 79\% while maintaining patron satisfaction. Our approach of integrating LLMs with optimization methods, integrating LLMs' capabilities in flexible generation and coarse-grained modeling of human preferences (between text-based descriptions of food) with optimization to meet constraints and balance multiple objectives, may be of independent interest in other domains. \textcolor{red}{Long sentence. TODO: maybe say more here. Can generalize, and we will improve. Also reorder discussion to be same order as paper.}

%LLMs displayed weaker performance for prediction tasks involving the mapping from a description of food to actual sensory properties or satisfaction, though still outperformed random baselines, and their performance improved for coarse-grained prediction specifically. Additionally, consistent with past research on LLMs' modeling of minority preferences~\cite{santurkar2023whose}, our results showed that LLMs model human preferences related to plant-based food less accurately. Taken together, these results suggest that, while LLMs may be useful in generating plausible and detailed suggestions to improve a plant-based product, or generating plant-based recipes that sound appealing, regular human feedback, e.g. taste testings, remains critical. 
%\begin{itemize}
%\item Two actionable takeaways. Actionable finding is that o1-preview is immediately useful in alternative protein development. While o1-preview on its own does not perform well on our menu design task, combining o1-preview with combinatorial optimization techniques is able to reduce average emissions of food orders by 81\% while maintaining both patron satisfaction and animal welfare. Actionable for menu designers; we make our code for menu optimization publicly available. 
%\item Generation vs. prediction tasks. An apparent contradiction, that LLMs can generate but not predict? Hypothesize that LLMs know what sounds good to humans (as evidenced by an X\% reduction in satisfaction and Y\% increase in emissions when o1-preview's descriptions are removed, though these differences were not statistically significant). Additionally, trained on food science papers, etc. and can brainstorm experimental designs given sensory panel feedback. However, lack a molecular level understanding of how ingredients interact to produce sensory properties. Takeaway: can assist in generation tasks but regular human taste testing or other feedback will remain critical.
%\item Minority preferences. 
%\item Dual use. 
%\item Climate impact of LLMs. 
%\end{itemize}

%We note several limitations to be addressed in future work. 
\noindent \textbf{Limitations.} We did not do significant prompt engineering, and leave this as an area for future work.
In the sustainable protein tasks, we did not have access to the full product formulation, only ingredients and nutritional information. However, we consider this a positive, that users do not necessarily have to provide sensitive intellectual property to receive useful outputs from LLMs.
In our menu design task, we did not provide incentives to make a realistic choice, such as delivering the meals to participants. Additionally, we did not study dimensions of food choice such as food waste and portion size.
A stronger baseline for this task would have been a human chef given the same menu and constraints, though we note that our ``Vegetarian" baseline is based on the top vegetarian meals on online delivery apps.
Future work could establish an expert human (food scientist or chef) baseline for the preference prediction tasks.
In the experimental design task, implementing the suggestions and running a second sensory panel to determine whether the suggestions improve the product is an area for future work.

%\subsection{Call to Action} We see this paper as an initial step in establishing AI for sustainable food as an interdisciplinary research area. We believe that researchers can play an important role in the development of sustainable foods, as individual alternative protein companies are generally not incentivized to share data or other IP.
%The scope of our paper is broad by design, including topics pertaining to both food science - the development of novel sustainable foods - and computational social science - promoting the consumption of existing sustainable foods. Future work can delve into each of these areas in greater depth and obtain a more fine-grained understanding of LLM strengths and weaknesses, as well as potentially build upon LLMs to design models specialized to these domains. 
%Our take-aways on what should be done more TODO.

%\section*{Software and Data}

%We will make our code available on GitHub upon acceptance. The NECTAR dataset can be requested by filling out a form at this URL: . It is only available to academic researchers, to reduce harm to the reputations of the companies that performed poorly in the sensory evaluation. The Food.com dataset is available at \url{https://www.kaggle.com/datasets/irkaal/foodcom-recipes-and-reviews}.

\section*{Impact Statement}
% ethical aspects and future 
%societal consequence
Our paper is directly motivated by broader societal impacts, specifically in sustainability and public health. Our task definitions and results could spur progress in both development of novel sustainable foods and greater adoption of existing sustainable foods.
The potential benefits of a sustainable protein transition are significant. For example, a recent life cycle assessment found that, relative to a beef burger patty, the production of a plant-based burger patty made with soy protein is associated with 98\%, 87\%, and 99\% lower GHG emissions, land use, and air pollution respectively~\cite{saerens2021life}, with similar estimates for other types of sustainable protein~\cite{gfiEnvironmentalImpacts}. Additionally, the Intergovernmental Panel on Climate Change (IPCC) estimates that a societal transition from animal-based food (ABF) to plant-based food (PBF) by 2050 would reduce land-based GHG emissions by two-thirds relative to the business-as-usual case \cite{ipcc2019}. 

In the menu design task, we consider ethical aspects such as the trade-offs among patron satisfaction, sustainability, and animal welfare, and show that our approach can improve or maintain all dimensions. Designing menus to shift the distribution of patron choices may raise questions of respect for individual autonomy. We thus use three different metrics of patron satisfaction, and show that all are maintained with the revised menu (Appendix Figures~\ref{fig:task4-satisfaction},\ref{fig:task4-all-visit},\ref{fig:task4-all-recommend}). 

We also note that current sustainable proteins are sometimes considered ``ultra-processed" and therefore unhealthy~\cite{nytimesFakeMeat}. We view this as a challenge for the next generation of sustainable proteins that AI could potentially help with, via jointly optimizing for satisfaction, nutrition, climate impacts, cost, and other relevant factors. 
Use of LLMs for developing unhealthy or less sustainable foods is also a concern. We acknowledge that such work is likely ongoing, and we study here how LLMs can also be useful for applications that benefit society. 

Lastly, the benefits of LLM-supported development of sustainable food should be interpreted jointly with the environmental impacts of LLMs themselves (such as energy and water use). While pre-training in particular requires significant resources~\cite{van2021sustainable}, we note that work to reduce the climate impacts of LLMs is ongoing~\cite{patterson2021carbon,li2024sprout}. Future research can continue to explore ways to leverage pre-trained models for applications that contribute to positive environmental outcomes. 

\section*{Acknowledgments}
This work is supported by the Stanford Plant-Based Diet Initiative (PBDI). Anna Thomas is supported by National Institutes of Health Grant R01LM013866. Kristina Gligori\'c is supported by the Swiss National Science Foundation (Grant P500PT-211127). We thank Caroline Cotto and Max Elder for access to the NECTAR dataset, and Moses Charikar, Ali Teshnizi, and Madeleine Udell for helpful discussions. We thank Yifan Mai and the Center for Research on Foundation Models (CRFM) for their support.

%,
%(such as lowering food systems' GHG emissions) 
%Moreover, the use of LLMs for optimizing unhealthy or unsustainable food is also a concern. Although LLMs can be used for various goals, research at the intersection of AI and food should prioritize applications that are grounded in practitioners' real-world needs and challenges, such as the identified difficulties in overcoming taste concerns or accurately modeling patron preferences when developing sustainable food. Future work could thus explore the roles and needs of key food practitioners involved in various stages of the food system, including but not limited to: food production (i.e., farmers), preparation (i.e., chefs), distribution (i.e., grocery store managers), and safety (i.e., food inspectors). 

%Finally, dual use of LLMs for developing less sustainable foods (e.g. animal-based) is also a concern. We acknowledge that such work is likely ongoing in industry and we aim to encourage researchers at the intersection of AI and food to prioritize applications that align with sustainability, public health, and ethical considerations. %\textcolor{red}{Althoguh AI could be used for all sorts of other food science, it is particularly impactful for helath and sustainability where optimizing for user preferences is particularly difficult. Taste is currently a barrier to adoption. Statement on climate impacts of LLMs}%By fostering transparency, sharing benchmarks, and promoting collaboration across fields, we hope to guide the responsible use of AI tools toward outcomes that benefit the environment and society at large.

%\begin{itemize}
%\item Task definitions and initial results could spur progress in biotechnology - AI for alternative protein - as well as computational social science, nudging and persuasion toward sustainable food choices.
%\item Nudging of course raises questions of respect for individual autonomy. Three different metrics of patron satisfaction. We also note that the meat industry uses extensive persuasion techniques.
%\item Current alternative proteins are sometimes considered ultra-processed and unhealthy. View as a challenge for the next generation, that AI could help with - jointly optimize for satisfaction, nutrition, climate impacts, cost.
%\item Dual use. Could be used to further entrench animal-based food. 
%\end{itemize}

%Authors are \textbf{required} to include a statement of the potential 
%broader impact of their work, including its ethical aspects and future 
%societal consequences. This statement should be in an unnumbered 
%section at the end of the paper (co-located with Acknowledgements -- 
%the two may appear in either order, but both must be before References), 
%and does not count toward the paper page limit. In many cases, where 
%the ethical impacts and expected societal implications are those that 
%are well established when advancing the field of Machine Learning, 
%substantial discussion is not required, and a simple statement such 
%as the following will suffice:

%``This paper presents work whose goal is to advance the field of 
%Machine Learning. There are many potential societal consequences 
%of our work, none which we feel must be specifically highlighted here.''

%The above statement can be used verbatim in such cases, but we 
%encourage authors to think about whether there is content which does 
%warrant further discussion, as this statement will be apparent if the 
%paper is later flagged for ethics review.



\bibliography{example_paper}
\bibliographystyle{icml2025}


\newpage
\appendix

%\section*{Appendix Table of Contents}
%\addcontentsline{toc}{section}{Appendix Table of Contents} % Add to main ToC
%\dominitoc
%\renewcommand{\mtcsettitle}{Contents of the Appendix} % Optional: Customize appendix ToC title
%\doparttoc
%\newpage

%\minitoc

%\addcontentsline{toc}{section}{Appendix}
%\part{Appendix} % Start the appendix part
%\parttoc % Insert the appendix TOC

\onecolumn

\section{Proof of Proposition \ref{prop:llm-accuracy}}\label{sec:proof-prop-1}
The problem we would like to solve is:
\begin{equation}
\begin{aligned}
\max_{x} \quad & f(x) := \sum_{i=1}^N p(u_i)x_i + \lambda d(x)\\
& \sum_{i=1}^N x_i = K  \\
& x_i \in \{0,1\}\:\forall i \in [1, ..., N]
\end{aligned}\label{eq:ideal-prob}
\end{equation}

However, we do not have access to the true scores $p(u_i)$, e.g. actual satisfaction of a specified population with a recipe. Instead, we use an LLM to produce the estimates $\hat{p}(u_i)$, and solve:

\begin{equation}
\begin{aligned}
\max_{x} \quad & \hat{f}(x) := \sum_{i=1}^N \hat{p}(u_i)x_i + \lambda d(x)\\
& \sum_{i=1}^N x_i = K  \\
& x_i \in \{0,1\}\:\forall i \in [1, ..., N]
\end{aligned}\label{eq:actual-prob}
\end{equation}

Let $x^{\ast}$ be the optimal solution of (\ref{eq:ideal-prob}), and let $\hat{x}^{\ast}$ be the optimal solution of (\ref{eq:actual-prob}).

\begin{proposition}
Let $p(u_i), \hat{p}(u_i) \in [0,1]$, and $d(x) \in [0,D]$. If $|\hat{p}(u_i) - p(u_i)| \leq \epsilon$ for all $u_i \in U$, then $|f(\hat{x}^{\ast}) - f(x^{\ast})| \leq 2K\epsilon + 2\lambda D$.
\end{proposition}

\begin{proof}
First, note that $\forall x, |f(x) - \hat{f}(x)| \leq K\epsilon + \lambda D$.
Then, combining this with the optimality of $x^{\ast}$ and $\hat{x}^{\ast}$, $f(\hat{x}^{\ast}) \geq \hat{f}(\hat{x}^{\ast}) - (K\epsilon + \lambda D) \geq \hat{f}(x^{\ast}) - (K\epsilon + \lambda D) \geq f(x^{\ast}) - 2K\epsilon - 2\lambda D$.
\end{proof}

\section{Extended Related Work}\label{sec:extended-related-work}
\paragraph{Large Language Models and Reasoning.} Our work broadly pertains to the integration of LLMs with other tools for reasoning tasks. \citet{pmlr-v235-ahmaditeshnizi24a} develop an agent called OptiMUS for formulating and solving (mixed integer) linear programming problems from their natural language descriptions. Their goal is to expand access to solvers to individuals who lack the expertise required to formulate and solve these problems. \citet{ye2024satlm} propose SatLM, which combines LLMs with SAT solvers to solve problems in arithmetic reasoning, logical reasoning, and symbolic reasoning. \citet{pan-etal-2023-logic} propose Logic-LM, which integrates LLMs with symbolic solvers to improve logical reasoning. Relative to these works, our setting is fundamentally different, where the underlying optimization problem is not fully specified due to involving subjective human preferences or other components that are difficult to formalize mathematically. We are specifically motivated by the food domain, in which human satisfaction must be considered (alongside other factors such as nutrition, cost, emissions, etc.), but is difficult to write down mathematically. Similar problems arise in settings such as education (e.g. curriculum design) and health (e.g. fitness regimen design). Our framework is also applicable to settings where it may take considerable effort to formally specify the optimization problem, e.g. computing climate impacts as well as cost, nutrition, etc. for each food item, but an LLM may be able to generate sufficiently accurate estimates. We also note past work in the natural language processing literature on combining integer programming with machine learning, e.g.~\citet{denis2007joint}, which used integer linear programming for coference resolution.
\paragraph{Large Language Models for Climate, Sustainability, and Food.}
\citet{pmlr-v235-bulian24a} propose an evaluation framework for LLMs in the domain of climate information.~\citet{morio2023an} create a benchmark dataset for assessing corporate climate policy engagement and evaluate several language models. 
\citet{huang2024foodpuzzle} introduce the FoodPuzzle dataset and create an LLM-based agent for flavor profile prediction. They define two tasks: molecular food prediction, in which the goal is to predict food sources based on their molecular composition, and molecular profile completion, in which the goal is to identify the missing molecules needed to complete the molecular profile of a given food item. Within our paper, the most closely related task is sensory profile prediction.
Our work differs in that we evaluate LLMs on their ability to predict dimensions otherwise evaluated by a human sensory panel, such as overall satisfaction, purchase intent, meatiness, etc. prevalent in the current practice of product development in food science. Sensory panels are essential but also very expensive and time consuming to both run and analyze. %Our task definitions are grounded in the current practice of product development in food science, in which sensory panels are essential but also very expensive and time consuming to both run and analyze. 
Additionally, our evaluation specifically focuses on sustainable protein rather than general food science. Finally,~\citet{huang2024foodpuzzle} develop an agent based on in-context learning and retrieval augmented generation, whereas we evaluate LLMs on our sustainable protein tasks via zero-shot prompting, as well as integration with combinatorial optimization.  
%\textcolor{red}{add FlavorPuzzle.}
\paragraph{Artifical Intelligence for Climate, Sustainability, and Food.} 
Previous work has studied climate- and environment- related applications of artificial intelligence~\cite{tuia2022perspectives,chapman2024biodiversity,madadkhani2024tackling,kaack2022aligning}. Within the food domain, previous work has explored how AI can reduce energy use in plant factories to support sustainable food production~\cite{decardi2024artificial}. \citet{st2024mechanical} demonstrated that a combination of mechanical testing and ML can describe food texture in a similar manner to human taste testers.
%an approach combining a three-dimensional mechanical technique and AI can mimic our sensory experience, which can be leveraged to improve plant-based meats.
%Climate- and environment-related applications of artificial intelligence have received attention within the ML communities

\section{Supplementary Information on Datasets}
\subsection{NECTAR Dataset}\label{app:nectar}
The NECTAR dataset was collected over the period of June to August 2023 in Precision Research's Chicago research center. It is only available to academic researchers to reduce harm to the reputations of companies whose products performed poorly in the taste testing. Additional information can be found at \url{https://www.nectar.org/taste}.
It consists of 5,516 sensory evaluations of 47 products, and will continue to grow in size over the coming years. Each sensory evaluation consists of 21 dimensions, shown in Table~\ref{tab:sensory-dims}. %Participants were recruited via .
The products were prepared in a test kitchen according to manufacturers' instructions.
Plant-based product selection criteria were based on popularity, availability (i.e. distributed in-market at the time of the test), and
similarity to analog animal offerings (i.e. veggie burgers made from whole plants were not included, whereas plant-based burgers
aiming to mimic the eating experience of animal-based burgers were included).
All tasting was blind and monadic (one product at a time), and the panel was untrained.

%Likert scale. Sematic differential scale.

\begin{table}[h!]
\centering
\begin{tabular}{lrccccc}
\toprule
Sensory Dimension              & Question Type & Burger & Hot Dog & Bacon & Chicken Nuggets & Chicken Tenders  \\ 
\midrule
Appearance         & 7 pt Likert Scale    & X & X & X &X & X      \\
Color & 7 pt Semantic Differential & X & X & X &X & X\\
Flavor Liking       & 7 pt Likert Scale   & X & X & X &X & X        \\ 
Spiciness       & 7 pt Semantic Differential   & X & X & X &X & X        \\ 
Like           & Open Ended   & X & X & X &X & X        \\ 
Dislike & Open Ended  & X & X & X &X & X         \\
Overall Liking      & 7 pt Likert Scale & X & X & X &X & X          \\ 
Meatiness   & 7 pt Semantic Differential & X & X & X &X & X          \\ 
Greasiness   & 7 pt Likert Scale   & X & X & X &X & X        \\ 
Juiciness   & 7 pt Semantic differential   & X & X & X &X & X        \\ 
Smokiness      & 7 pt Semantic Differential   &  & X & X & &         \\ 
Sweetness        & 7 pt Semantic Differential &  & X & X & &           \\ 
Saltiness       & 7 pt Semantic Differential   &  & X & X &X & X         \\ 
Crispiness       & 7 pt Semantic Differential   &  &  & X &X & X        \\ 
Aftertaste Strength       & 5 pt Likert Scale & X & X & X &X & X          \\ 
Aftertaste       & 5 pt Semantic Differential    & X & X & X &X & X       \\ 
Purchase Intent       & 5 pt Likert Scale   & X & X & X &X & X        \\ 
Texture: Liking     & 7 pt Likert Scale    & X & X & X &X & X        \\
Chewiness     & 7 pt Semantic Differential   & X & X & X &X & X        \\ 
Firmness     & 7 pt Semantic Differential   & X & X & X &X & X        \\ 
Breading Flavor     & 7 pt Semantic Differential   &  &  &  &X & X        \\ 
\bottomrule
\end{tabular}
\caption{Overview of sensory dimensions in NECTAR dataset, by category of sustainable protein product.}\label{tab:sensory-dims}
\end{table}


%Below are the rating scale definitions for each of the 21 dimensions. 

\subsection{\url{Food.com} Recipe Dataset}\label{app:food.com}
Figure~\ref{fig:example-food.com-recipe} shows an example recipe from the \url{Food.com} dataset.

\begin{figure}[h!]
\begin{tcolorbox}[colback=lightgray, colframe=gray, coltitle=black, title=Example Recipe]
Name:  Carina's Tofu-Vegetable Kebabs\\
Description:  This dish is best prepared a day in advance to allow the ingredients to soak in  the marinade overnight.\\
Ingredients:  extra firm tofu, eggplant, zucchini, mushrooms, soy sauce, low sodium soy sauce, olive oil, maple syrup, honey, red wine vinegar, lemon juice, garlic cloves, mustard powder, black pepper.\\
Instructions:\\
1. Drain the tofu, carefully squeezing out excess water,  and pat dry with paper towels.\\
2. Cut tofu into one-inch squares.\\
3. Set aside.  Cut  eggplant lengthwise in half, then cut each half into approximately three strips.\\
4. Cut strips crosswise into one-inch cubes.
5. Slice zucchini into half-inch thick  slices.\\ 6. Cut red pepper in half, removing stem and seeds, and cut each half into  one-inch squares.\\ 7. Wipe mushrooms clean with a moist paper towel and remove  stems.\\
8. Thread tofu and vegetables on to barbecue skewers in alternating color  combinations: For example, first a piece of eggplant, then a slice of tofu, then zucchini, then red pepper, baby corn and mushrooms.\\
9. Continue in this way until  all skewers are full.\\
10. Make the marinade by putting all ingredients in a  blender, and blend on high speed for about one minute until mixed.\\
11. Alternatively, put all ingredients in a glass jar, cover tightly with the lid  and shake well until mixed.\\
12. Lay the kebabs in a long, shallow baking pan or on  a non-metal tray, making sure they lie flat. Evenly pour the marinade over the  kebabs, turning them once so that the tofu and vegetables are coated.\\
13. Refrigerate the kebabs for three to eight hours, occasionally spooning the  marinade over them.\\
14. Broil or grill the kebabs at 450 F for 15-20 minutes, or on the grill, until the vegetables are browned.\\
15. Suggestions  This meal can be served over cooked, brown rice. Amounts can easily be doubled to make four servings.
\end{tcolorbox}\label{fig:example-food.com-recipe}
\caption{Example recipe from the \url{Food.com} dataset.}
\end{figure}


\section{Supplementary Methods}\label{app:methods}

\begin{table*}[h!]
\small
\centering
\begin{tabular}{p{0.2\textwidth}|p{0.2\textwidth}|p{0.2\textwidth}|p{0.25\textwidth}}
\toprule
\textbf{Task}                                            & \textbf{Data}     & \textbf{Techniques Used}                &\textbf{ Evaluation Metrics}                                                                          \\ 
\midrule

\hlyellow{\textbf{Sustainable protein: experimental design }}           & NECTAR   & Zero-shot prompting            & Accuracy, Specificity, Complementarity, Time Saved (Expert Human Food Scientist Evaluation) \\ \midrule

\hlblue{\textbf{Traditional foods: menu design }}              & Menu from \citet{banerjee2023sustainable}, emissions data from \citet{poore2018reducing}, animal welfare data from \citet{faunalyticsImpactReplacing} & Zero-shot prompting, integer quadratic programming  & Satisfaction (Human Subjects Evaluation), Emissions,  Animal Welfare (Automated)                      \\ \midrule
\hlred{\textbf{Sustainable protein: sensory profile prediction}} & NECTAR   & Zero-shot prompting & Accuracy (Automated)                                                                        \\ \midrule
\hlgreen{\textbf{Traditional foods: recipe preference prediction}}         & Food.com & Zero-shot prompting            & Accuracy (Automated)                                                                        \\ 
\bottomrule
\end{tabular}
\caption{Overview of tasks, techniques used, and evaluation metrics.}
\label{tab:selected tasks}
\end{table*}

\subsection{Experimental Design}\label{sec:additional-methods-exp-design}
The 30 sustainable protein products used in this task consisted of six products from each of the five categories. Further information on the food scientists is in Table~\ref{tab:food-scientist-properties}.
The prompt for Phase 1 is shown in Figure~\ref{fig:llm-prompt-phase-1}, and %, and the expert human food scientists received the same instructions except with the first line (``You are an expert plant-based meat food scientist.") removed. 
our standardization prompt is shown in Figure~\ref{fig:llm-prompt-standardization}. 
%The former three are on a 10-point scale, and the latter is in hours. 
Precise definitions of the dimensions of evaluation in Phase 2 are in Section~\ref{sec:eval-dims-alt-protein-design}. In Phase 1, 15 food scientists participated, each evaluating 2 products. In Phase 2, again 15 food scientists participated, each evaluating 2 products (with one human and one LLM response per product), with an overlap of 10 food scientists between Phase 1 and Phase 2. 

\begin{figure}[h!]
\begin{tcolorbox}[colback=lightyellow, colframe=yellow, coltitle=black, title=Prompt]
You are an expert plant-based meat food scientist. You have devised a $<$\textit{category}$>$ product with the following ingredient list: $<$\textit{ingredient list}$>$.\\
Additionally, it has the following nutritional information: $<$\textit{nutrition facts}$>$.\\
You ran a blind taste test of American omnivores and received the following quantitative feedback on your product: $<$\textit{quantitative feedback}$>$.\\
Additionally, you received the following qualitative feedback about what people liked: $<$\textit{positive feedback}$>$.\\
You also received the following qualitative feedback about what people disliked: $<$\textit{negative feedback}$>$.\\
What changes would you consider making to your product? Could you design a set of experiments on the key areas that need improvement? You will be evaluated on metrics including accuracy and specificity.
\end{tcolorbox}
\caption{Prompt for Phase 1 of the experimental design task, in which \opreview\: and expert food scientists generated experimental designs for improving sustainable protein products on the basis of sensory panel feedback. For the food scientists, the first line was removed. \opreview\:was additionally instructed to limit its responses to 250 words, to approximately match the mean length of the food scientists' responses. Even though the mean original human response length was 311 words (Table~\ref{tab:exp-design-original-edited-lengths}), 250 was chosen in the instruction to \opreview\:because we noticed that \opreview\:tended to exceed the word limit.}\label{fig:llm-prompt-phase-1}
% Prompts for other tasks are in Appendix~\ref{app:methods}
\end{figure}

\begin{figure}[h!]
\begin{tcolorbox}[colback=lightyellow, colframe=yellow, coltitle=black, title=Prompt]
You are a writing assistant specializing in editing writing produced by food scientists. I will give you some text to edit.\\
Instructions:\\
1. Convert all suggestions to a numbered list, with a title for each suggestion. Do not include any content that is not part of the numbered list.\\
2. Do not change the length.\\
3. Remove any references to the author’s personal experience or to other writing.\\
4. Rewrite it as if it could have come from either a human or LLM.\\
5. Use complete sentences.\\
6. Do not add any prefix like `Here is the edited text'. Just output the edited text.\\
7. Do not add or remove any of the meaning, unless necessary for following instruction \#3.\\
8. Do not use asterisks.\\
Here is the text:\\
$<$\textit{original text}$>$
\end{tcolorbox}
\caption{Prompt for style standardization step of the experimental design task.}\label{fig:llm-prompt-standardization}
\end{figure}

\begin{table}[h!]
\centering
\begin{tabular}{lll}
\toprule
& Original & Edited  \\ \hline
\opreview &                319.8 (37.7)&               327.6 (87.2) \\ \hline
Food Scientist         &                311.3 (223.9) &                311.6 (195.3) \\ 
\bottomrule
\end{tabular}
\caption{Original and edited lengths of \opreview's and food scientists' experimental designs. Standard deviation is in parentheses.}\label{tab:exp-design-original-edited-lengths}
\end{table}

\subsubsection{Properties of Expert Food Scientists}

20 food science experts were recruited via direct outreach on LinkedIn or email from our food scientist team member. Their properties are shown in Table~\ref{tab:food-scientist-properties}.

\begin{table}[h!]
\centering
\begin{tabular}{llll}
\toprule
& Phase 1 ($n=15$) & Phase 2 ($n=15$) & All ($n=20$) \\ \hline
Mean Years of Experience in Plant-Based Food Science &                4.33 (2.78)&                4.73 (5.30) & 5.05 (4.73)\\ \hline
Mean Years of Experience in Plant-Based Meat         &                2.23 (2.01)&                3.27 (3.82) & 3.08 (3.38)\\ 
\bottomrule
\end{tabular}
\caption{Properties of food scientists. 
%Years of experience does not include education. 
Standard deviation is in parentheses. 10 food scientists participated in both Phase 1 and Phase 2, but we ensured that they did not evaluate their own experimental designs.}\label{tab:food-scientist-properties}
\end{table}

\subsubsection{Evaluation Dimensions}\label{sec:eval-dims-alt-protein-design}
The dimensions of evaluation were defined to the food scientists as follows:
\begin{itemize}
\itemsep0em 
\item \underline{Accuracy}: Are these suggestions likely to be useful in improving the product, to the best of your knowledge? Are they consistent with the provided feedback?
\item \underline{Specificity}: How detailed and actionable are the suggestions?
\item \underline{Complementarity}: Did this scientist see anything you missed? Would they be complementary as a brainstorming partner?
\item \underline{Time saved}: How much time, if any, do you think collaborating with this scientist would save you in designing a full (detailed and implementable) experimental design plan to improve this product? Please specify in hours. 
\end{itemize}


\subsection{Menu Design}\label{sec:appendix-methods-menu-design}
%Recipe rating step was based on the title of the recipe. 
The original menu we use is from \citet{banerjee2023sustainable}, who chose the items from Deliveroo's and Just Eat's top 100 items ordered in the United Kingdom in 2019. They also adjusted the items following pilot surveys. The menu includes 19 non-vegetarian and 17 vegetarian or vegan items. The original paper says 18 non-vegetarian and 18 vegetarian or vegan options, but a dish called ``Panchetta Carbonara'' was included in the vegetarian section. For some items, the original menu included a ``(v)'' parenthetical indicating that a vegan option is available. We removed these parentheticals for simplicity in survey design and computing GHG emissions. Following~\citet{banerjee2023sustainable}, GHG emissions were estimated based on the main ingredient of the dish. Data from~\citet{poore2018reducing} were used to compute emissions. For animal welfare, data from Faunalytics, a leading animal welfare organization, on number of animals killed per kilogram of various meats were used ~\cite{faunalyticsImpactReplacing}. Two meats in the original menu, lamb and duck, were not contained in the Faunalytics data, and instead were imputed using the value for turkey, an animal of roughly similar size. Table~\ref{tab:menu-design-sample-sizes} shows the sample sizes per arm in the Prolific experiments.

%\begin{figure*}[h!]
%    \centering
%    \includegraphics[width=0.9\textwidth]{approach-2-fig-iqp.png}
%    \caption{Our approach generates menus under constraints using a combination of LLMs and integer programming.}
%    \label{fig:approach-2}
%\end{figure*}

%Properties of human subjects across arms.

\begin{table}[h!]
\centering
\begin{tabular}{lr}
\toprule
Arm              & Sample Size \\ 
\midrule
Original         & 50          \\
Vegetarian Subset       & 50          \\ 
Remove Beef         & 50          \\ 
Vegetarian First & 50          \\
o1-preview       & 49          \\ 
o1-preview+IQP   & 50          \\ 
o1-preview RD    & 50          \\ 
GPT-4o+IQP       & 50          \\ 
Llama+IQP        & 52          \\ 
Claude+IQP       & 50          \\ 
Gemini+IQP       & 51          \\ 
\bottomrule
\end{tabular}
\caption{Sample sizes per arm in Prolific randomized experiments for the menu design task. ``\opreview\:RD" is an ablation in which \opreview's descriptions were replaced with a simple list of ingredients.}\label{tab:menu-design-sample-sizes}
\end{table}

Other than the choice of dish, the participants were also asked the following questions:
\begin{itemize}
\itemsep0em
\item How satisfied are you with your set of choices? 1: not at all. 10: very satisfied.
\item How likely would you be to visit this restaurant, assuming it is affordable and a convenient distance from you? 1: not likely at all. 10: very likely. 
\item How likely would you be to recommend this restaurant to your friends? 1: not likely at all. 10: very likely.
\end{itemize}

\begin{figure}[h!]
\begin{tcolorbox}[colback=lightblue, colframe=midblue, coltitle=black, title=Prompt]
You are a brilliant chef experienced at creating sustainable and delicious food.\\
Here is a menu: $<$\textit{original menu}$>$\\  Please generate a revised menu, with the same number of recipes ($<$\textit{n}$>$) and no new ingredients other than tofu, lentils, mushrooms, chickpeas, eggs, and cheese.\\
Design the menu to achieve at least a 75\% CO$_2$ emissions reduction in people's choices while maintaining or improving patron satisfaction
 with their set of choices.\\ Patrons will be American omnivores. Emissions will be computed based on the main (first) ingredient.\\
 Please output each recipe in same format as this example:\\
 \textit{Tofu curry ramen}\\
 \textit{Fried tofu, noodles, curry broth, pak choi, pickled onions.}\\
 \textit{Appealing description.}\\
 The ingredients must be in order of usage, i.e the main ingredient must come first.\\
 Very important: you must only use ingredients in the original menu or the list above. For every ingredient, there must be an exact match in the original menu or the list above.\\
 Do not worsen cost, nutrition, animal welfare (number of animals used, computed based on the first ingredient), or preparation time.\\
 Do not include any stars, asterisks, hashtags, underscores. Do not number the recipes. Do not include any text other than recipe information, e.g. do not say `Here are the recipes'.
\end{tcolorbox}\label{fig:prompt-task4-direct}
\caption{LLM prompt used in the menu design ablation, in which \opreview\:on its own was asked to directly revise the original menu.}
\end{figure}

\begin{figure}[h!]
\begin{tcolorbox}[colback=lightblue, colframe=midblue, coltitle=black, title=Prompt]
You are a brilliant chef experienced at creating sustainable and delicious food. Here is a menu: $<$\textit{original menu}$>$\\
Please generate ($<$\textit{k}$>$) new, delicious, and diverse vegan or vegetarian dishes from this set of ingredients. You are also allowed to use tofu, lentils, mushrooms, chickpeas, eggs, and cheese.\\
 Patrons will be American omnivores.\\
 Please output in same format as this example:\\
 \textit{Tofu curry ramen}\\
 \textit{Fried tofu, noodles, curry broth, pak choi, pickled onions.}\\
 \textit{Appealing description.}\\
 The ingredients must be in order of usage, i.e the main ingredient must come first.\\
 Very important: you must only use ingredients in the original menu or the list above. For every ingredient, there must be an exact match in the original menu or the list above.\\
 Do not worsen CO$_2$ emissions, cost, nutrition, or preparation time. Emissions will be computed based on the main (first) ingredient.\\
 Do not include any stars, asterisks, hashtags, underscores. Do not number the recipes. Do not include any text other than recipe information, e.g. do not say `Here are the recipes'.
\end{tcolorbox}\label{fig:prompt-task4-iqp-recipe-gen}
\caption{LLM prompt used in the recipe generation stage of the LLM+IQP approach.}
\end{figure}

\begin{figure}[h!]
\begin{tcolorbox}[colback=lightblue, colframe=midblue, coltitle=black, title=Prompt]
Here are $<$\textit{r}$>$ recipes. Please rate them on a scale of 1-10 based on standard American omnivore taste preferences, 1 being unappealing and 10 being appealing. Output only a comma-separated list of $<$\textit{r}$>$  numbers, from 1 to 10.\\
$<$\textit{recipes}$>$
\end{tcolorbox}\label{fig:prompt-task4-iqp-recipe-rating}
\caption{LLM prompt used in the rating stage of the LLM+IQP approach.}
\end{figure}

\subsection{Sensory Profile Prediction}\label{sec:sensory-profile-prediction-app}
%\paragraph{Number of pairs for each sensory dimension}
Figure~\ref{fig:sensory-profile-prompt} shows the prompt for the sensory profile prediction task. Table~\ref{tab:sensory-profile-pairs} shows the number of pairs for each sensory dimension. Table~\ref{tab:prediction-sample-sizes} shows sample sizes across all prediction tasks.
\begin{figure}[h!]
\begin{tcolorbox}[colback=lightred, colframe=red, coltitle=black, title=Prompt]
You are an expert plant-based meat food scientist. Here are the ingredient lists of two $<$\textit{category}$>$ products.\\
Product 1: $<$\textit{ingredient list}$>$\\
Product 2: $<$\textit{ingredient list}$>$\\
Additionally, here are the nutrition facts for product 1: $<$\textit{nutrition facts}$>$.\\
And here are the nutrition facts for product 2: $<$\textit{nutrition facts}$>$.\\
Now, suppose that a group of 100 omnivores eats both products in a blind taste test. 
Which do you predict would be ranked higher on the dimension of $<$\textit{dimension}$>$? 
Please output a single character, either 1 or 2 on the first line. 

%Recipe 1: $<$\textit{recipe text}$>$

%Recipe 2: $<$\textit{recipe text}$>$

%Answer:
\end{tcolorbox}
\caption{LLM prompt for the sensory profile prediction task.}\label{fig:sensory-profile-prompt}
\end{figure}

\begin{table}[h!]
\centering
\begin{tabular}{lr}
\toprule
 & $N$ \\
\midrule
Overall Satisfaction & 85 \\
Meatiness & 61 \\
Greasiness & 103 \\
Juiciness & 73 \\
Sweetness & 28 \\
Saltiness & 63 \\
Purchase & 82 \\
\midrule
All Dimensions & 495 \\
\bottomrule
\end{tabular}
\caption{Sample sizes (number of pairs) for each sensory dimension in the sensory profile prediction task.}\label{tab:sensory-profile-pairs}
\end{table}


\begin{table*}[t!]
\small
\centering
\begin{tabular}{llllllll}
\toprule
 & \claude & \gemini & \gptthree & \gptfouro & 
Llama & \opreview & Baseline \\
\midrule
\multicolumn{8}{l}{
\textbf{Sensory Profile Prediction: LLM Prediction vs. Sensory Panel (Sustainable Protein)}} \\
\midrule
All Dimensions & 495 & 495 &495 & 495 & 495 & 495 & 495 \\
Overall Satisfaction & 85 & 85 & 85 & 85 & 85 & 85 & 85 \\
Meatiness & 61 & 61 & 61 & 61 & 61 & 61 & 61 \\
Greasiness & 103 & 103 & 103 & 103 & 103 & 103 & 103 \\
Juiciness & 73 & 73 & 73 & 73 & 73 & 73 & 73 \\
Sweetness & 28 & 28 & 28 & 28 & 28 & 28 & 28 \\
Saltiness & 63 & 63 & 63 & 63 & 63 & 63 & 63 \\
Purchase & 82 & 82 & 82 & 82 & 82 & 82 & 82 \\

\midrule
\multicolumn{8}{l}{\textbf{Recipe Preference Prediction (\url{Food.com}): LLM Prediction vs. Rating (Traditional Foods)}} \\
\midrule
Rating & 500 & 500 & 500  & 500  & 500 & 500 & 500 \\
\midrule
\multicolumn{8}{l}{\textbf{Recipe Preference Prediction (Menu Design): LLM Prediction vs. Order Frequency (Traditional Foods)}} \\
\midrule
Order Frequency & 1225 & 1275 & -  & 1225  & 1326 & 1225 & 1225 \\
\bottomrule
\end{tabular}
\caption{Sample sizes (number of pairs) for prediction tasks.}\label{tab:prediction-sample-sizes}
\end{table*}



\subsection{Recipe Preference Prediction}\label{app:recipe-rating-prediction}
Figure~\ref{fig:prompt-recipe-rating} shows the prompt for the recipe preference prediction task.


\begin{figure}
\begin{tcolorbox}[colback=lightgreen, colframe=green!75!black, coltitle=black, title=Prompt]
You are an expert online recipe writer. Which online recipe would people prefer, recipe 1 or recipe 2?

Output a number only (1 or 2). You must choose one. If unsure, provide your best guess.

Recipe 1: $<$\textit{recipe text}$>$

Recipe 2: $<$\textit{recipe text}$>$

Answer:
\end{tcolorbox}
%\Maya{In the future, could be interesting to compare to human laypeople or chefs. I wonder if they would even be above chance.}
\caption{LLM prompt for the recipe preference prediction task.}\label{fig:prompt-recipe-rating}
\end{figure}



\section{Supplementary Results}\label{app:results}

\subsection{Experimental Design}\label{app:results-experimental-design}
Figures~\ref{fig:o1-preview-output} and~\ref{fig:o1-preview-output-2} show an example output from \opreview.

\begin{figure}[h!]
\begin{tcolorbox}[colback=lightgray, colframe=gray, coltitle=black, title=Example Output]
1. Appearance Enhancement

   Many participants found the appearance unappetizing, describing it as``looks like a dog treat,"``raw," or ``fake."

   - Experiment: Test natural coloring agents to achieve a more authentic bacon look. Incorporate beet juice or paprika extract to enhance redness and use coconut oil for marbling effects.\\
   - Variables: Vary concentrations of coloring agents and adjust cooking times and temperatures to achieve desired caramelization.\\
   - Metrics: Conduct sensory evaluations focusing on appearance, measuring scores against the current mean of 2.47.

2. Texture Improvement

   Issues with a mushy center and inconsistent crispiness were noted.

   - Experiment: Adjust the protein blend ratios and incorporate texturizers like methylcellulose to enhance firmness and coherence.\\
   - Variables: Test different ratios of soy and wheat proteins; vary levels of konjac powder and vegetable gums.\\
   - Metrics: Use texture profile analysis to measure hardness and chewiness; collect sensory feedback on texture aiming to improve the mean from 3.43 towards 4 (neutral).

3. Flavor Enhancement

   Participants desired more smokiness and found the taste bland.

   - Experiment: Increase natural smoke flavoring and seasoning levels to enhance meatiness and smokiness.\\
   - Variables: Experiment with different types and concentrations of smoke extracts (e.g., hickory, applewood) and adjust salt and sugar levels.\\
   - Metrics: Sensory evaluations on flavor and smokiness, targeting improvements from means of 3.58 and 2.43 towards 4.

4. Reduce Greasiness

   Some found the product too greasy.

   - Experiment: Reduce sunflower oil content or replace it with a less greasy alternative like canola oil.\\
   - Variables: Vary oil types and quantities.\\
   - Metrics: Measure perceived greasiness in sensory tests, aiming to adjust the mean from 2.53 to 3 (moderately greasy).

\end{tcolorbox}
\caption{Example o1-preview output for the experimental design task, after style standardization.}\label{fig:o1-preview-output}
\end{figure}

\begin{figure}[h!]
\begin{tcolorbox}[colback=lightgray, colframe=gray, coltitle=black, title=Example Output]
1. Improve Texture and Crispiness.
Issue: The product is perceived as too soft and lacking crispiness (Mean crispiness rating: 5.86; ideal is 4).\\
Changes:\\
- Adjust ingredient composition by incorporating crisping agents like rice flour or modifying the ratio of wheat gluten to reduce chewiness.\\
- Modify the process by slicing the product thinner and adjusting cooking temperatures and times to achieve a crisper texture.\\
Experiment:\\
- Design a factorial experiment testing different levels of rice flour (e.g., 1\%, 2\%, 3\%) and cooking temperatures (e.g., 350°F, 375°F, 400°F).\\
- Measure the crispiness using texture analysis and conduct sensory evaluations.\\
2. Enhance Flavor Profile. Issue: Flavor is disliked, with complaints of excessive saltiness and artificial taste (Mean flavor rating: 2.22; saltiness rating skewed towards too salty).
Changes:\\
- Reduce salt content by gradually decreasing salt levels by 10-20\%.\\
- Increase umami and smokiness by adding natural umami enhancers like mushroom extract or increasing natural smoke flavors.\\
- Balance sweetness by introducing a small amount of natural sweeteners like maple syrup to mimic bacon's sweet-salty balance.
Experiment:\\
- Create a series of formulations with varying salt levels (e.g., full, -10\%, -20\%) and added sweeteners.\\
- Conduct sensory tests focusing on flavor liking and saltiness perception.\\
3. Improve Appearance. Issue: Appearance is inconsistent; some find it artificial (Mixed feedback on color; Mean color rating: 4.42).\\
Changes:\\
- Introduce marbling by using natural colorants to create a marbled fat appearance, mimicking real bacon.\\
- Adjust color by modifying the hue using oleoresin paprika for red tones and annatto for lighter shades.\\
Experiment:\\
- Develop prototypes with different marbling techniques and colorant levels.\\
- Perform visual assessments with consumers to rate appearance authenticity.\\
4. Increase Greasiness. Issue: Product is not perceived as greasy enough (Mean greasiness rating: 1.39; ideal is 3).\\
Changes:\\
- Incorporate plant-based fats by increasing the amount of oils like coconut or sunflower oil.\\
- Use fat release technology by employing oil encapsulation that melts during cooking to simulate bacon fat.\\
Experiment:\\
- Test formulations with varying fat levels (e.g., 2\%, 4\%, 6\% additional oil).
- Evaluate greasiness through sensory panels and measure perceived oiliness.
\end{tcolorbox}
\caption{Example o1-preview output for the experimental design task, after style standardization.}\label{fig:o1-preview-output-2}
\end{figure}

\subsection{Menu Design}\label{appendix:task4}

%\paragraph{Preference Rating Accuracy: Q4}
Table~\ref{tab:menu-preference-rating-q4} shows accuracy of the predicted preferences.

\begin{table}[h!]
\small
\centering
\begin{tabular}{l|r}
\toprule

\textbf{ LLM }&  \textbf{Accuracy} \\

\midrule
\textbf{\claude} & \textbf{0.85} \\
\textbf{\gemini} & \textbf{0.72} \\
\textbf{\gptthree}& - \\
\textbf{\gptfouro} & \textbf{0.82}  \\
\textbf{\llama} & \textbf{0.72} \\
\opreview & 0.50  \\
\textbf{\opreview, RD} & \textbf{0.84}\\
\bottomrule
\end{tabular}
\caption{Accuracy in the fourth (highest) quartile of the preference gap, for predicted preferences in the menu design task vs. actual order frequency. The 25th and 50th percentile of the preference gap were both 1 (the minimum gap needed for inclusion in the analysis in Table~\ref{tab:sensory-profile-prediction-results}), so we did not test those quartiles. Accuracies statistically significant according to chi-squared test are in bold. \gptthree\:did not produce a valid output for this task and thus was not tested. We note that LLM predictions were based on the recipe title alone (not including the full description), which could explain the discrepancy between item-level prediction performance and menu-level performance for \opreview. When its descriptions were replaced with a simple list of ingredients (in one of the ablations), \opreview's accuracy in this quartile was 84\% ($p$=2.40e-20).}
\label{tab:menu-preference-rating-q4}
\end{table}

%\paragraph{Treatment effect estimates, controlling for covariates}
\begin{table}[h!]
\begin{tabular}{lll}
\hline
\textbf{Outcome}  & \textbf{Treatment Effect Estimate (p-value)} & \textbf{Percent Change} \\
\midrule
Emissions                              &                                              34.31 (1.37e-18)&                         78.55\\
Satisfaction with Set of Choices       &                                              -0.04 (0.93)&                         -0.58\\ 
Likelihood of Visiting                 &                                              -0.07 (0.89)&                         -0.95\\ 
Likelihood of Recommending to a Friend &                                              -0.38 (0.47)&                         -4.9\\ 
Animal Usage                           &                                              0.00 (0.91)&                         -3.5\\ 
\bottomrule
\end{tabular}
\caption{Treatment Effect Estimates ($p$-value in parentheses) and percent change for \opreview+IQP vs. the original menu, adjusting (via linear regression) for the covariates of age, gender, race, and dietary preference. Signs are flipped so that positive values are better in all cases. Compared to the original menu, \opreview+IQP reduces average emissions by 79\% while maintaining participants's satisfaction with their set of choices to within 1\% of the original. Small reductions are observed in likelihood of visiting, likelihood of recommending to a friend, and animal welfare, but none are statistically significant.}
\end{table}

%\paragraph{Table of Results}
\begin{table}[h!]
\small
\begin{tabular}{lllllllll}
\toprule
 & Emissions & Satisfied & Visit & Recommend & Animal Lives & Plant Based & LLM Gen. (AO) & LLM Gen. (AC) \\
\midrule
Original & 44.91 & 8.40 & 7.90 & 7.82 & 0.08 & 0.20 & 0.00 & 0.00 \\
Veg. Subset & 19.87 & 6.82 & 6.76 & 6.36 & 0.00 & 1.00 & 0.00 & 0.00 \\
Remove Beef & 21.65 & 7.74 & 7.50 & 7.08 & 0.16 & 0.24 & 0.00 & 0.00 \\
Vegetarian First & 37.37 & 8.20 & 7.74 & 7.62 & 0.13 & 0.28 & 0.00 & 0.00 \\
o1-preview+IQP & 8.70 & 8.44 & 7.80 & 7.46 & 0.08 & 0.54 & 0.50 & 0.44 \\
o1-preview & 9.64 & 6.51 & 6.29 & 6.22 & 0.00 & 1.00 & 1.00 & 1.00 \\
o1-preview RD & 11.05 & 7.48 & 6.98 & 6.76 & 0.18 & 0.36 & 0.50 & 0.20 \\
Claude+IQP & 10.16 & 7.30 & 6.68 & 6.86 & 0.16 & 0.38 & 0.42 & 0.28 \\
Gemini+IQP & 11.71 & 7.59 & 7.35 & 7.02 & 0.19 & 0.31 & 0.44 & 0.10 \\
GPT-4o+IQP & 8.53 & 7.38 & 7.08 & 6.98 & 0.17 & 0.52 & 0.42 & 0.36 \\
Llama+IQP & 10.68 & 7.77 & 7.54 & 7.21 & 0.12 & 0.58 & 0.50 & 0.31 \\
\bottomrule
\end{tabular}
\caption{Table of all results for the menu design task. ``Emissions" is in kg of CO$_2$eq per kg of food item. ``Animal Lives" is number of animals killed per kg of food item. ``Satisfied", ``Visit", and ``Recommend" are on a ten point (1 to 10) scale. ``Plant Based" is the fraction of choices that were plant based (either vegan or vegetarian). ``LLM Gen. (AO)" is the fraction of available options in the final optimized menu that were LLM generated (as opposed to recipes from the original menu). ``LLM Gen. (AC)" is the fraction of actual choices that were LLM generated (as opposed to recipes from the original menu).}
\end{table}


%\paragraph{All Arms}
\begin{figure}[h!]
\centering
\includegraphics[width=\textwidth]{emissions-all.pdf}
\caption{Average emissions for all arms. $n=552$ across arms. Lower is better. Asterisks indicate a statistically significant difference ($t$-test with Bonferroni correction) compared to the original menu. Baselines are in green, ablations are in orange. In ``\opreview\:RD", \opreview's descriptions are replaced with a simple list of ingredients.}
\label{fig:task4-all-emissions}
\end{figure}

\begin{figure}[h!]
\centering
\includegraphics[width=\textwidth]{Satisfaction-all.pdf}
\caption{Average satisfaction for all arms. $n=552$ across arms. Higher is better. Asterisks indicate a statistically significant difference ($t$-test with Bonferroni correction) compared to the original menu. Baselines are in green, ablations are in orange. In ``\opreview\:RD", \opreview's descriptions are replaced with a simple list of ingredients.}
\label{fig:task4-satisfaction}
\end{figure}

\begin{figure}[h!]
\centering
\includegraphics[width=\textwidth]{Visit-all.pdf}
\caption{Average likelihood of visiting the hypothetical restaurant for all arms. $n=552$ across arms. Higher is better. Asterisks indicate a statistically significant difference ($t$-test with Bonferroni correction) compared to the original menu. Baselines are in green, ablations are in orange. In ``\opreview\:RD", \opreview's descriptions are replaced with a simple list of ingredients.}
\label{fig:task4-all-visit}
\end{figure}

\begin{figure}[h!]
\centering
\includegraphics[width=\textwidth]{Recommend-all.pdf}
\caption{Average likelihood of recommending the hypothetical restaurant to a friend for all arms. $n=552$ across arms.  Higher is better. Asterisks indicate a statistically significant difference ($t$-test with Bonferroni correction) compared to the original menu. Baselines are in green, ablations are in orange. In ``\opreview\:RD", \opreview's descriptions are replaced with a simple list of ingredients.}
\label{fig:task4-all-recommend}
\end{figure}

\begin{figure}[h!]
\centering
\includegraphics[width=\textwidth]{Animal_Lives-all.pdf}
\caption{Average animal usage for all arms. $n=552$ across arms.  Lower is better for animal welfare. Asterisks indicate a statistically significant difference ($t$-test with Bonferroni correction) compared to the original menu. Baselines are in green, ablations are in orange. In ``\opreview\:RD", \opreview's descriptions are replaced with a simple list of ingredients.}
\label{fig:task4-all-animal-welfare}
\end{figure}

%\paragraph{Excluding Vegans and Vegetarians}
\begin{figure}[h!]
\centering
\includegraphics[width=\textwidth]{Emissions-all-exclude-v.pdf}
\caption{Average emissions for all arms, excluding vegan and vegetarian participants. $n=514$ across arms. Lower is better. Asterisks indicate a statistically significant difference ($t$-test with Bonferroni correction) compared to the original menu. Baselines are in green, ablations are in orange. In ``\opreview\:RD", \opreview's descriptions are replaced with a simple list of ingredients.}
\label{fig:task4-all-emissions-exclude-v}
\end{figure}

\begin{figure}[h!]
\centering
\includegraphics[width=\textwidth]{Satisfaction-all-exclude-v.pdf}
\caption{Average satisfaction for all arms, excluding vegan and vegetarian participants. $n=514$ across arms. Higher is better. Asterisks indicate a statistically significant difference ($t$-test with Bonferroni correction) compared to the original menu. Baselines are in green, ablations are in orange. In ``\opreview\:RD", \opreview's descriptions are replaced with a simple list of ingredients.}
\label{fig:task4-satisfaction-exclude-v}
\end{figure}

\begin{figure}[h!]
\centering
\includegraphics[width=\textwidth]{Visit-all-exclude-v.pdf}
\caption{Average likelihood of visiting the hypothetical restaurant for all arms, excluding vegan and vegetarian participants. $n=514$ across arms. Higher is better. Asterisks indicate a statistically significant difference ($t$-test with Bonferroni correction) compared to the original menu. Baselines are in green, ablations are in orange. In ``\opreview\:RD", \opreview's descriptions are replaced with a simple list of ingredients.}
\label{fig:task4-all-visit-exclude-v}
\end{figure}

\begin{figure}[h!]
\centering
\includegraphics[width=\textwidth]{Recommend-all-exclude-v.pdf}
\caption{Average likelihood of recommending the hypothetical restaurant to a friend for all arms, excluding vegan and vegetarian participants. $n=514$ across arms.  Higher is better. Asterisks indicate a statistically significant difference ($t$-test with Bonferroni correction) compared to the original menu. Baselines are in green, ablations are in orange. In ``\opreview\:RD", \opreview's descriptions are replaced with a simple list of ingredients.}
\label{fig:task4-all-recommend-exclude-v}
\end{figure}

\begin{figure}[h!]
\centering
\includegraphics[width=\textwidth]{Animal_Lives-all-exclude-v.pdf}
\caption{Average animal usage for all arms, excluding vegan and vegetarian participants. $n=514$ across arms.  Lower is better.Asterisks indicate a statistically significant difference ($t$-test with Bonferroni correction) compared to the original menu. Baselines are in green, ablations are in orange. In ``\opreview\:RD", \opreview's descriptions are replaced with a simple list of ingredients.}
\label{fig:task4-all-animal-welfare-exclude-v}
\end{figure}

%%%%%%%%%%%%%%%%%%%%%%%%%%%%%%%%%%%%%%%%%%%%%%%%%%%%%%%%%%%%%%%%%%%%%%%%%%%%%%%
%%%%%%%%%%%%%%%%%%%%%%%%%%%%%%%%%%%%%%%%%%%%%%%%%%%%%%%%%%%%%%%%%%%%%%%%%%%%%%%

\subsection{Sensory Profile Prediction}\label{sec:task1-quartiles}
Accuracies in each quartile of the ground truth preference gap are shown in Tables~\ref{tab:sensory-profile-prediction-results-q1},~\ref{tab:sensory-profile-prediction-results-q2},~\ref{tab:sensory-profile-prediction-results-q3},~\ref{tab:sensory-profile-prediction-results-q4}.

\begin{table*}[h!]
\small
\centering
\begin{tabular}{llllllll}
\toprule
 & \claude & \gemini & \gptthree & \gptfouro & 
Llama & \opreview & Baseline \\
\midrule
Overall Satisfaction & 0.43 & 0.29 & 0.43 & 0.33 & 0.43 & 0.43 & 0.71 \\
Meatiness & 0.27 & 0.33 & 0.40 & 0.47 & 0.27 & 0.40 & 0.47 \\
Greasiness & 0.65 & 0.69 & 0.46 & \textbf{0.77} & \textbf{0.77} & 0.65 & 0.69 \\
Juiciness & 0.44 & 0.67 & 0.67 & 0.67 & 0.33 & 0.61 & 0.50 \\
Sweetness & 0.29 & 0.71 & 0.86 & 0.71 & 0.71 & 0.71 & 0.86 \\
Saltiness & 0.75 & 0.81 & 0.62 & 0.81 & 0.69 & 0.75 & 0.56 \\
Purchase & 0.60 & 0.45 & 0.50 & 0.50 & 0.50 & 0.50 & 0.45 \\
\midrule
All Dimensions & 0.52 & 0.55 & 0.53 & 0.60 & 0.53 & 0.57 & 0.59 \\
\bottomrule
\end{tabular}
\caption{Accuracies of six LLMs on the sensory profile prediction task, quartile 1 of the preference gap. Statistically significant results (chi-squared test with Bonferroni correction) are in bold.}\label{tab:sensory-profile-prediction-results-q1}
\end{table*}

\begin{table*}[h!]
\small
\centering
\begin{tabular}{llllllll}
\toprule
 & \claude & \gemini & \gptthree & \gptfouro & 
Llama & \opreview & Baseline \\
\midrule
Overall Satisfaction & 0.65 & 0.45 & 0.35 & 0.40 & 0.55 & 0.60 & 0.80 \\
Meatiness & \textbf{0.87} & 0.80 & 0.60 & 0.67 & 0.60 & 0.80 & 0.53 \\
Greasiness & 0.52 & 0.60 & 0.72 & 0.72 & 0.64 & 0.56 & 0.68 \\
Juiciness & 0.65 & 0.71 & 0.71 & 0.53 & 0.59 & 0.59 & 0.71 \\
Sweetness & 0.29 & 0.43 & 0.57 & 0.29 & 0.14 & 0.43 & 0.43 \\
Saltiness & 0.67 & 0.60 & 0.67 & 0.60 & 0.73 & 0.53 & 0.60 \\
Purchase & 0.62 & \textbf{0.10} & 0.38 & 0.38 & 0.33 & 0.57 & 0.57 \\
\midrule
All Dimensions & \textbf{0.62} & 0.52 & 0.57 & 0.53 & 0.54 & 0.59 & \textbf{0.64} \\
\bottomrule
\end{tabular}
\caption{Accuracies of six LLMs on the sensory profile prediction task, quartile 2 of the preference gap. Statistically significant results (chi-squared test with Bonferroni correction) are in bold.}\label{tab:sensory-profile-prediction-results-q2}
\end{table*}

\begin{table*}[t!]
\small
\centering
\begin{tabular}{llllllll}
\toprule
 & \claude & \gemini & \gptthree & \gptfouro & 
Llama & \opreview & Baseline \\
\midrule
Overall Satisfaction & 0.64 & 0.59 & 0.41 & 0.64 & 0.59 & 0.68 & 0.55 \\
Meatiness & 0.60 & 0.60 & 0.60 & 0.67 & 0.60 & 0.60 & 0.53 \\
Greasiness & 0.65 & 0.69 & 0.65 & \textbf{0.85} & \textbf{0.81} & 0.73 & \textbf{0.77} \\
Juiciness & 0.68 & 0.53 & 0.58 & 0.68 & 0.53 & 0.79 & 0.74 \\
Sweetness & 0.43 & 0.43 & 0.43 & 0.57 & 0.43 & 0.43 & 0.71 \\
Saltiness & 0.69 & 0.69 & 0.50 & 0.81 & 0.62 & \textbf{0.88} & \textbf{0.88} \\
Purchase & 0.55 & 0.30 & 0.25 & 0.45 & 0.40 & 0.35 & 0.75 \\
\midrule
All Dimensions & \textbf{0.62} & 0.56 & 0.50 & \textbf{0.68} & 0.59 & \textbf{0.66} & \textbf{0.70} \\
\bottomrule
\end{tabular}
\caption{Accuracies of six LLMs on the sensory profile prediction task, quartile 3 of the preference gap. Statistically significant results (chi-squared test with Bonferroni correction) are in bold.}\label{tab:sensory-profile-prediction-results-q3}
\end{table*}

\begin{table*}[t!]
\small
\centering
\begin{tabular}{llllllll}
\toprule
 & \claude & \gemini & \gptthree & \gptfouro & 
Llama & \opreview & Baseline \\
\midrule
Overall Satisfaction & 0.67 & 0.29 & 0.33 & \textbf{0.19} & 0.48 & \textbf{0.81} & \textbf{0.86} \\
Meatiness & 0.67 & 0.53 & 0.40 & 0.73 & 0.60 & 0.80 & 0.40 \\
Greasiness & 0.50 & 0.58 & 0.69 & 0.62 & 0.69 & 0.62 & 0.62 \\
Juiciness & 0.67 & 0.50 & 0.56 & 0.67 & 0.61 & 0.56 & \textbf{0.89} \\
Sweetness & 0.43 & 0.86 & 0.57 & 0.86 & 0.71 & 1.00 & 1.00 \\
Saltiness & 0.69 & 0.56 & 0.50 & 0.69 & 0.75 & 0.81 & \textbf{0.88} \\
Purchase & 0.80 & 0.35 & 0.35 & 0.50 & 0.45 & 0.75 & 0.70 \\
\midrule
All Dimensions & \textbf{0.64} & 0.49 & 0.49 & 0.57 & 0.60 & \textbf{0.73} & \textbf{0.74} \\
\bottomrule
\end{tabular}
\caption{Accuracies of six LLMs on the sensory profile prediction task, quartile 4 of the preference gap. Statistically significant results (chi-squared test with Bonferroni correction) are in bold.}\label{tab:sensory-profile-prediction-results-q4}
\end{table*}

\subsection{Recipe Preference Prediction}\label{appendix:task3}

Tables~\ref{tab:task3a} and \ref{tab:task3b} list supplementary analyses of performance on the recipe preference task. Table~\ref{tab:task3c} shows performance on a smaller held-out set.

\begin{table*}[h!]
\small
\centering
\begin{tabular}{l|rrrr|r}
\toprule
\multirow{2}{*}{\textbf{LLM}} & \multicolumn{5}{c}{\textbf{Condition}} \\
 & \textbf{Q1} & \textbf{Q2} & \textbf{Q3} & \textbf{Q4} & \textbf{Overall} \\

\midrule
\textbf{\gptfouro} & \textbf{62.32\%} & 65.31\% & 67.74\% & 71.43\% & \textbf{64.00\%} \\
\textbf{\claude} & \textbf{64.79\%} & 61.22\% & 58.06\% & 85.71\% & \textbf{63.20\%} \\
\textbf{\opreview} & 60.92\% & 57.14\% & 59.68\% & 57.14\% & \textbf{59.60\%} \\
\textbf{\gptthree} & 55.99\% & 55.78\% & 58.06\% & 85.71\% & 56.60\% \\
\textbf{\llama} & 53.17\% & 48.3\% & 43.55\% & 57.14\% & 50.60\% \\
\textbf{\gemini} & 50.00\% & 50.34\% & 33.87\% & 28.57\% & 47.80\% \\
\bottomrule
\end{tabular}
\caption{Accuracies of six LLMs in the recipe preference prediction task, stratified by the quartiles of the difference in review scores. Statistically significant results (chi-squared test with Bonferroni correction) are in bold.}
\label{tab:task3a}
\end{table*}



\begin{table*}[h!]
\small
\centering
\begin{tabular}{l|r|rrr|rr}
\toprule
\multirow{2}{*}{\textbf{LLM}} &  \multicolumn{6}{c}{\textbf{Condition}} \\
 & \textbf{Overall }& $\mathbf{\neg V \mathrm{vs.} \neg V}$ & $\mathbf{V\mathrm{vs.}\neg V}$ & $\mathbf{V\mathrm{vs.}V}$ &  $\mathbf{V>\neg V}$ & $\mathbf{V<\neg V}$ \\
\midrule
\midrule
\textbf{\gptfouro} & \textbf{64.00\%} & \textbf{70.81\%} & 50.77\% & 37.5\% & 53.66\% & \textbf{66.03\%} \\
\textbf{\claude }& \textbf{63.20\%} & \textbf{67.34\%} & 56.15\% & 41.67\% & 43.9\% & \textbf{66.99\%} \\
\textbf{\opreview} & \textbf{59.60\%} & \textbf{65.32\%} & 46.15\% & 50.0\% & 48.78\% & \textbf{61.72\%} \\
\textbf{\gptthree} & 56.60\% & \textbf{63.87\%} & 42.31\% & 29.17\% & 40.24\% & \textbf{59.81\%} \\
\textbf{\llama} & 50.60\% & 54.91\% & 42.31\% & 33.33\% & 43.9\% & 51.91\% \\
\textbf{\gemini} & 47.80\% & 50.58\% & 43.08\% & 33.33\% & 45.12\% & 48.33\% \\
\bottomrule
\end{tabular}
\caption{Accuracies of six LLMs in the recipe preference prediction task, stratified by the ground truth comparison type: non- vegetarian vs. non- vegetarian  ($\mathbf{\neg V \mathrm{vs.} \neg V}$),  vegetarian vs. non- vegetarian ($\mathbf{V\mathrm{vs.}\neg V}$), and  vegetarian vs.  vegetarian comparisons ($\mathbf{V\mathrm{vs.} V}$). Within the  vegetarian vs. non- vegetarian comparisons, performance is displayed separately when  vegetarian is preferred ($\mathbf{V>\neg V}$) and when non- vegetarian is preferred ($\mathbf{V<\neg V}$). Statistically significant results (chi-squared test with Bonferroni correction) are in bold.}
\label{tab:task3b}
\end{table*}

\begin{table}[h]
\small
\centering
\begin{tabular}{l|r}
\toprule

\textbf{ LLM }&  \textbf{Accuracy} \\

\midrule
\textbf{\claude} & {56.90\%} \\
\textbf{\gemini} & 41.38\% \\
\textbf{\gptthree}& 55.17\% \\
\textbf{\gptfouro} &  {55.17\%} \\
\textbf{\llama} & 74.14\% \\
\textbf{\opreview} &  {68.97\%} \\
\bottomrule
\end{tabular}
\caption{\textbf{Recipe preference prediction: Post-cut-off dataset.} Overall accuracy on the post-cut-off dataset.}
\label{tab:task3c}
\end{table}


\clearpage
\newpage

\section{Menus}\label{sec:menus}
\paragraph{Original Menu~\cite{banerjee2023sustainable}}

%\begin{figure}
\begin{enumerate}
\itemsep0em 
\item Chicken Curry Ramen.  Japanese fried chicken \& noodles in a delicious curry broth.
\item Pork Ribs.  Pork Ribs smothered with Kentucky style BBQ sauce.
\item Beef Bourguignon.  6oz Black Angus beef burger topped with crispy bacon, red wine braised onions, Raclette cheese.
\item Chicken Katsu Curry.  Succulent chicken in a crispy Japanese panko crumb with mild curry sauce, pickles and steamed rice.
\item Pepperoni Melt.  This one’s got our triple cheese blend, tomato sauce and pepperoni.
\item Lamb Meatballs with Spaghetti.  Handmade lamb patties grilled and topped with Greek yoghurt, tomato sauce, and onions.
\item Aromatic Duck Rolls.  Aromatic roast duck, hoisin sauce, cucumber, spring onion, rolled in rice paper.
\item Slow Cooked Beef in Chianti Sauce.  Beef shin and shallots slow cooked in rich Chianti red wine sauce with a cheesy dumpling.
\item Pork Porchetta.  Slow-roasted pork belly in herbed red wine sauce, served with roasted new potatoes and broccoli.
\item Cured Salmon Sashimi.  Dazzling beetroot cured slices of salmon served with lemon.
\item Beef Brisket and Venison Tagliatelle.  A beef, brisket, venison, red wine and beef dripping ragu with tagliatelle, fresh bufala mozzarella and crispy sage.
\item Rustica Chorizo Pizza.  Chorizo salami, torn wild boar and pork meatballs, smoked mozzarella and baby sunblush tomatoes.
\item Butterfly Chicken Burger.  Two succulent chicken breasts joined by crispy skin, packaged in a Portuguese roll and topped with tomato and lettuce.
\item American Hot Pizza.  Pepperoni, mozzarella and tomato, with your choice of hot green, Roquito or jalapeno peppers.
\item Beef Angus Burger.  6oz beef burger with baby gem lettuce and plum tomato in a chargrilled brioche bun with Dijon mayonnaise.
\item Beer Battered Fish with Chips.  with frites and pea \& mint puree.
\item Wild Boar Polpette.  Oven-baked herby wild boar and pork meatballs in a pomodoro sauce with melted smoked mozzarella.
\item Steak.  Sliced Black Angus rump steak with red onion chutney, watercress, and Dijon mayonnaise.
\item Canelloni.  pasta with béchamel, mozzarella and Gran Milano cheese.
\item Supreme Pizza.  Cheese and tomato, onion, mushroom, fresh basil, olive and garlic oil.
\item Goat's Cheese Calzone.  Goats cheese, grilled aubergines, roasted peppers, oven roasted tomatoes.
\item Vegan Spaghetti Lentil Ragu.  A hearty ragu of green lentils and mixed vegetables in a rich tomato and fennel sauce.
\item Jack Fruit Arrabiata.  Vegan jackfruit peppers, spicy harissa, roquito chilis \& vegan Mozzarella cheese, finished with basil.
\item Halloumi Sticks with Mayo Avocado Dip.  Chunky sticks of grilled halloumi cheese with a chilli jam and mayo avocado dip.
\item Cheese Salad.  Smoked cheddar, cheddar and sage stuffing.
\item Falafel with Tahini.  Our signature recipe, served with a tahini dip.
\item Four Cheese Margherita.  Mozzarella, mascarpone, fontal and grana cheese on a tomato base.
\item Vegan Meatballs.  Vegan meatballs in a rich tomato and fennel sauce.
\item Bufala Caprese.  Specialty tomatoes and drizzle of roasted garlic in extra virgin olive oil with Buffalo mozzarella.
\item Butternut Squash Salad.  Roasted butternut squash with feta cheese, pomegranate seeds, mixed leaf salad and watercress.
\item Panchetta Carbonara.  crispy pancetta and asparagus in a velvety sauce made with mascarpone, pecorino and Grana Padano cheese.
\item Lentil Linguine Ragu.  Rich Italian lentil ragu with baby spinach, tomatoes, basil \& pecorino cheese.
\item Aegean Slaw.  Thinly shredded cabbage, carrot and red onion peppers, with an olive oil dressing.
\item Goat's Cheese Salad.  Goat's cheese and balsamic onion crostinis on winter baby kale, red pepper, cucumber, plum tomatoes.
\item Harusami Aubergine.  Fried slices of aubergine with garlic \& ginger sesame soy.
\item Inari Taco.  Golden tofu pockets filled with sticky sushi rice, avocado salsa \& vegan sriracha mayo.
\end{enumerate}
%\caption{Original menu.}
%\end{figure}

\paragraph{Remove Beef}
\begin{enumerate}
\itemsep0em 
\item Chicken Curry Ramen.  Japanese fried chicken \& noodles in a delicious curry broth.
\item Pork Ribs.  Pork Ribs smothered with Kentucky style BBQ sauce.
\item Chicken Katsu Curry.  Succulent chicken in a crispy Japanese panko crumb with mild curry sauce, pickles and steamed rice.
\item Pepperoni Melt.  This one’s got our triple cheese blend, tomato sauce and pepperoni.
\item Lamb Meatballs with Spaghetti.  Handmade lamb patties grilled and topped with Greek yoghurt, tomato sauce, and onions.
\item Aromatic Duck Rolls.  Aromatic roast duck, hoisin sauce, cucumber, spring onion, rolled in rice paper.
\item Pork Porchetta.  Slow-roasted pork belly in herbed red wine sauce, served with roasted new potatoes and broccoli.
\item Cured Salmon Sashimi.  Dazzling beetroot cured slices of salmon served with lemon.
\item Rustica Chorizo Pizza.  Chorizo salami, torn wild boar and pork meatballs, smoked mozzarella and baby sunblush tomatoes.
\item Butterfly Chicken Burger.  Two succulent chicken breasts joined by crispy skin, packaged in a Portuguese roll and topped with tomato and lettuce.
\item American Hot Pizza.  Pepperoni, mozzarella and tomato, with your choice of hot green, Roquito or jalapeno peppers.
\item Beer Battered Fish with Chips.  with frites and pea \& mint puree.
\item Wild Boar Polpette.  Oven-baked herby wild boar and pork meatballs in a pomodoro sauce with melted smoked mozzarella.
\item Canelloni.  pasta with béchamel, mozzarella and Gran Milano cheese.
\item Supreme Pizza.  Cheese and tomato, onion, mushroom, fresh basil, olive and garlic oil.
\item Goat's Cheese Calzone.  Goats cheese, grilled aubergines, roasted peppers, oven roasted tomatoes.
\item Vegan Spaghetti Lentil Ragu.  A hearty ragu of green lentils and mixed vegetables in a rich tomato and fennel sauce.
\item Jack Fruit Arrabiata.  Vegan jackfruit peppers, spicy harissa, roquito chilis \& vegan Mozzarella cheese, finished with basil.
\item Halloumi Sticks with Mayo Avocado Dip.  Chunky sticks of grilled halloumi cheese with a chilli jam and mayo avocado dip.
\item Cheese Salad.  Smoked cheddar, cheddar and sage stuffing.
\item Falafel with Tahini.  Our signature recipe, served with a tahini dip.
\item Four Cheese Margherita.  Mozzarella, mascarpone, fontal and grana cheese on a tomato base.
\item Vegan Meatballs.  Vegan meatballs in a rich tomato and fennel sauce.
\item Bufala Caprese.  Specialty tomatoes and drizzle of roasted garlic in extra virgin olive oil with Buffalo mozzarella.
\item Butternut Squash Salad.  Roasted butternut squash with feta cheese, pomegranate seeds, mixed leaf salad and watercress.
\item Panchetta Carbonara.  crispy pancetta and asparagus in a velvety sauce made with mascarpone, pecorino and Grana Padano cheese.
\item Lentil Linguine Ragu.  Rich Italian lentil ragu with baby spinach, tomatoes, basil \& pecorino cheese.
\item Aegean Slaw.  Thinly shredded cabbage, carrot and red onion peppers, with an olive oil dressing.
\item Goat's Cheese Salad.  Goat's cheese and balsamic onion crostinis on winter baby kale, red pepper, cucumber, plum tomatoes.
\item Harusami Aubergine.  Fried slices of aubergine with garlic \& ginger sesame soy.
\item Inari Taco.  Golden tofu pockets filled with sticky sushi rice, avocado salsa \& vegan sriracha mayo.
\end{enumerate}

\paragraph{Vegetarian}
\begin{enumerate}
\itemsep0em 
\item Canelloni.  pasta with béchamel, mozzarella and Gran Milano cheese.
\item Supreme Pizza.  Cheese and tomato, onion, mushroom, fresh basil, olive and garlic oil.
\item Goat's Cheese Calzone.  Goats cheese, grilled aubergines, roasted peppers, oven roasted tomatoes.
\item Vegan Spaghetti Lentil Ragu.  A hearty ragu of green lentils and mixed vegetables in a rich tomato and fennel sauce.
\item Jack Fruit Arrabiata.  Vegan jackfruit peppers, spicy harissa, roquito chilis \& vegan Mozzarella cheese, finished with basil.
\item Halloumi Sticks with Mayo Avocado Dip.  Chunky sticks of grilled halloumi cheese with a chilli jam and mayo avocado dip.
\item Cheese Salad.  Smoked cheddar, cheddar and sage stuffing.
\item Falafel with Tahini.  Our signature recipe, served with a tahini dip.
\item Four Cheese Margherita.  Mozzarella, mascarpone, fontal and grana cheese on a tomato base.
\item Vegan Meatballs.  Vegan meatballs in a rich tomato and fennel sauce.
\item Bufala Caprese.  Specialty tomatoes and drizzle of roasted garlic in extra virgin olive oil with Buffalo mozzarella.
\item Butternut Squash Salad.  Roasted butternut squash with feta cheese, pomegranate seeds, mixed leaf salad and watercress.
\item Lentil Linguine Ragu.  Rich Italian lentil ragu with baby spinach, tomatoes, basil \& pecorino cheese.
\item Aegean Slaw.  Thinly shredded cabbage, carrot and red onion peppers, with an olive oil dressing.
\item Goat's Cheese Salad.  Goat's cheese and balsamic onion crostinis on winter baby kale, red pepper, cucumber, plum tomatoes.
\item Harusami Aubergine.  Fried slices of aubergine with garlic \& ginger sesame soy.
\item Inari Taco.  Golden tofu pockets filled with sticky sushi rice, avocado salsa \& vegan sriracha mayo.
\end{enumerate}

\paragraph{Vegetarian First}
\begin{enumerate}
\itemsep0em 
    \item Canelloni.  Pasta with béchamel, mozzarella and Gran Milano cheese.
    \item Supreme Pizza.  Cheese and tomato, onion, mushroom, fresh basil, olive and garlic oil.
    \item Goat's Cheese Calzone.  Goats cheese, grilled aubergines, roasted peppers, oven roasted tomatoes.
    \item Vegan Spaghetti Lentil Ragu.  A hearty ragu of green lentils and mixed vegetables in a rich tomato and fennel sauce.
    \item Jack Fruit Arrabiata.  Vegan jackfruit peppers, spicy harissa, roquito chilis \& vegan Mozzarella cheese, finished with basil.
    \item Halloumi Sticks with Mayo Avocado Dip.  Chunky sticks of grilled halloumi cheese with a chilli jam and mayo avocado dip.
    \item Cheese Salad.  Smoked cheddar, cheddar and sage stuffing.
    \item Falafel with Tahini.  Our signature recipe, served with a tahini dip.
    \item Four Cheese Margherita.  Mozzarella, mascarpone, fontal and grana cheese on a tomato base.
    \item Vegan Meatballs.  Vegan meatballs in a rich tomato and fennel sauce.
    \item Bufala Caprese.  Specialty tomatoes and drizzle of roasted garlic in extra virgin olive oil with Buffalo mozzarella.
    \item  Butternut Squash Salad.  Roasted butternut squash with feta cheese, pomegranate seeds, mixed leaf salad and watercress.
    \item Lentil Linguine Ragu.  Rich Italian lentil ragu with baby spinach, tomatoes, basil \& pecorino cheese.
    \item  Aegean Slaw.  Thinly shredded cabbage, carrot and red onion peppers, with an olive oil dressing.
    \item Goat's Cheese Salad.  Goat's cheese and balsamic onion crostinis on winter baby kale, red pepper, cucumber, plum tomatoes.
    \item Harusami Aubergine.  Fried slices of aubergine with garlic \& ginger sesame soy.
    \item Inari Taco.  Golden tofu pockets filled with sticky sushi rice, avocado salsa \& vegan sriracha mayo.
    \item Chicken Curry Ramen.  Japanese fried chicken \& noodles in a delicious curry broth.
    \item Pork Ribs.  Pork Ribs smothered with Kentucky style BBQ sauce.
    \item Beef Bourguignon.  6oz Black Angus beef burger topped with crispy bacon, red wine braised onions, Raclette cheese.
    \item Chicken Katsu Curry.  Succulent chicken in a crispy Japanese panko crumb with mild curry sauce, pickles and steamed rice.
    \item Pepperoni Melt.  This one’s got our triple cheese blend, tomato sauce and pepperoni.
    \item Lamb Meatballs with Spaghetti.  Handmade lamb patties grilled and topped with Greek yoghurt, tomato sauce, and onions.
    \item Aromatic Duck Rolls.  Aromatic roast duck, hoisin sauce, cucumber, spring onion, rolled in rice paper.
    \item Slow Cooked Beef in Chianti Sauce.  Beef shin and shallots slow cooked in rich Chianti red wine sauce with a cheesy dumpling.
    \item Pork Porchetta.  Slow-roasted pork belly in herbed red wine sauce, served with roasted new potatoes and broccoli.
    \item Cured Salmon Sashimi.  Dazzling beetroot cured slices of salmon served with lemon.
    \item Beef Brisket and Venison Tagliatelle.  A beef, brisket, venison, red wine and beef dripping ragu with tagliatelle, fresh bufala mozzarella and crispy sage.
    \item Rustica Chorizo Pizza.  Chorizo salami, torn wild boar and pork meatballs, smoked mozzarella and baby sunblush tomatoes.
    \item Butterfly Chicken Burger.  Two succulent chicken breasts joined by crispy skin, packaged in a Portuguese roll and topped with tomato and lettuce.
    \item  American Hot Pizza.  Pepperoni, mozzarella and tomato, with your choice of hot green, Roquito or jalapeno peppers.
    \item Beef Angus Burger.  6oz beef burger with baby gem lettuce and plum tomato in a chargrilled brioche bun with Dijon mayonnaise.
    \item Beer Battered Fish with Chips.  With frites and pea \& mint puree.
    \item Wild Boar Polpette.  Oven-baked herby wild boar and pork meatballs in a pomodoro sauce with melted smoked mozzarella.
    \item Steak.  Sliced Black Angus rump steak with red onion chutney, watercress, and Dijon mayonnaise.
    \item Panchetta Carbonara.  Crispy pancetta and asparagus in a velvety sauce made with mascarpone, pecorino and Grana Padano cheese.
\end{enumerate}


\paragraph{\opreview+IQP}
\begin{enumerate}
\itemsep0em 
\item Creamy Mushroom Tagliatelle. An indulgent pasta dish featuring sautéed mushrooms and baby spinach in a creamy mascarpone sauce, tossed with tagliatelle and seasoned with garlic oil and fresh basil.
\item Vegetable Delight Pizza. A delicious crispy pizza topped with mozzarella, tomato sauce, and a medley of grilled vegetables, finished with fresh basil.
\item Three Cheese Omelette. A fluffy omelette loaded with smoked cheddar, mozzarella, grana cheese, and sautéed red onions.
\item Eggplant Parmesan. Layers of tender aubergine baked with rich tomato sauce, melted mozzarella, and Grana Padano cheese, garnished with fresh basil.
\item Mushroom and Goat's Cheese Omelette. A fluffy omelette filled with sautéed mushrooms and creamy goat's cheese, infused with garlic oil and fresh basil.
\item Chickpea Curry with Rice. A flavorful and hearty chickpea curry served with steamed rice and accompanied by tangy pickles.
\item Spinach and Feta Stuffed Mushrooms. Large mushrooms stuffed with sautéed baby spinach and creamy feta cheese, drizzled with garlic oil and baked to perfection.
\item Lentil Veggie Burger. A hearty lentil-based veggie burger served on a toasted brioche bun with fresh lettuce, tomato, and tangy Dijon mayonnaise.
\item Falafel Salad. Our signature falafel served over a fresh mixed salad, with crunchy cucumbers, juicy tomatoes, red onions, and a creamy tahini dressing.
\item Egg Shakshuka. Poached eggs simmered in a spiced tomato sauce with peppers and onions, garnished with fresh basil.
\item Vegetable and Tofu Stir-Fry. A vibrant stir-fry of crispy tofu and fresh vegetables tossed with noodles in a savory garlic and ginger sesame soy sauce.
\item Tofu Katsu Curry. Succulent tofu coated in crispy panko crumbs, served with mild curry sauce, tangy pickles, and steamed rice.
\item Lentil Stuffed Peppers. Roasted bell peppers stuffed with hearty lentils, tomato sauce, and fresh baby spinach, topped with creamy goat's cheese.
\item Butternut Squash and Feta Salad. Sweet roasted butternut squash and tangy feta cheese on a bed of fresh mixed greens, sprinkled with pomegranate seeds and watercress.
\item Vegan Meatball Sub. Hearty vegan meatballs simmered in pomodoro sauce, topped with melted smoked mozzarella, served in a toasted Portuguese roll.
\item Chickpea and Spinach Curry. A nourishing curry of chickpeas and baby spinach simmered in a mild curry sauce, served with steamed rice.
\item Tofu Curry Ramen. Japanese-style ramen with fried tofu and noodles in a delicious curry broth, topped with pak choi and pickled onions.
\item Mushroom and Lentil Bolognese. Hearty mushrooms and lentils cooked in a rich tomato sauce, served over tagliatelle pasta and garnished with fresh basil.
\item Supreme Pizza.  Cheese and tomato, onion, mushroom, fresh basil, olive and garlic oil.
\item Chicken Curry Ramen.  Japanese fried chicken \& noodles in a delicious curry broth.
\item Pork Ribs.  Pork Ribs smothered with Kentucky style BBQ sauce.
\item Butterfly Chicken Burger.  Two succulent chicken breasts joined by crispy skin, packaged in a Portuguese roll and topped with tomato and lettuce.
\item Four Cheese Margherita.  Mozzarella, mascarpone, fontal and grana cheese on a tomato base.
\item Panchetta Carbonara.  crispy pancetta and asparagus in a velvety sauce made with mascarpone, pecorino and Grana Padano cheese.
\item Chicken Katsu Curry.  Succulent chicken in a crispy Japanese panko crumb with mild curry sauce, pickles and steamed rice.
\item Rustica Chorizo Pizza.  Chorizo salami, torn wild boar and pork meatballs, smoked mozzarella and baby sunblush tomatoes.
\item Beer Battered Fish with Chips.  with frites and pea \& mint puree.
\item Pork Porchetta.  Slow-roasted pork belly in herbed red wine sauce, served with roasted new potatoes and broccoli.
\item Butternut Squash Salad.  Roasted butternut squash with feta cheese, pomegranate seeds, mixed leaf salad and watercress.
\item Aegean Slaw.  Thinly shredded cabbage, carrot and red onion peppers, with an olive oil dressing.
\item Cured Salmon Sashimi.  Dazzling beetroot cured slices of salmon served with lemon.
\item Falafel with Tahini.  Our signature recipe, served with a tahini dip.
\item Lentil Linguine Ragu.  Rich Italian lentil ragu with baby spinach, tomatoes, basil \& pecorino cheese.
\item Aromatic Duck Rolls.  Aromatic roast duck, hoisin sauce, cucumber, spring onion, rolled in rice paper.
\item Vegan Spaghetti Lentil Ragu.  A hearty ragu of green lentils and mixed vegetables in a rich tomato and fennel sauce.
\item Harusami Aubergine.  Fried slices of aubergine with garlic \& ginger sesame soy.
\end{enumerate}

\paragraph{\gptfouro+IQP}
\begin{enumerate}
\itemsep0em 
\item Falafel Wrap. A flavorful and nutritious wrap perfect for any time of day.
\item Lentil Meatball Spaghetti. A comforting combination of lentil meatballs and rich tomato sauce with pasta.
\item Eggs Florentine Pizza. A delightful pizza featuring fresh eggs and vibrant toppings.
\item Spinach and Asparagus Carbonara. A creamy and delicious pasta loaded with greens.
\item Butternut Squash and Feta Cannelloni. Rich, creamy, and packed with flavor, perfect for vegetarians.
\item Roasted Beetroot and Goat's Cheese Salad. A vibrant salad with the earthy sweetness of beetroot and creamy goat's cheese.
\item BBQ Jackfruit Ribs. Tender jackfruit smothered in a tangy BBQ sauce, served with sides.
\item Eggplant Ragu Tagliatelle. A hearty and delicious pasta with a rich eggplant ragu.
\item Vegan Stuffed Bell Peppers. Hearty stuffed peppers, packed with delicious and healthy fillings.
\item Spicy Lentil and Jackfruit Tacos. A fiery taco loaded with protein and bold flavors.
\item Mushroom Bourguignon. A rich and savory dish, perfect for a comforting meal.
\item Tofu Porchetta. Herb-crusted tofu and flavorful sides for a vegan take on porchetta.
\item Chickpea Katsu Burger. A satisfying burger with a crunchy chickpea patty and fresh toppings.
\item Panko Crusted Tofu Fingers. Crunchy tofu sticks with a deliciously creamy dip.
\item Tofu and Veggie Stir-fry. A nutritious stir-fry with tofu and a mix of fresh vegetables.
\item Pork Ribs.  Pork Ribs smothered with Kentucky style BBQ sauce.
\item Beer Battered Fish with Chips.  with frites and pea \& mint puree.
\item Supreme Pizza.  Cheese and tomato, onion, mushroom, fresh basil, olive and garlic oil.
\item Four Cheese Margherita.  Mozzarella, mascarpone, fontal and grana cheese on a tomato base.
\item Panchetta Carbonara.  crispy pancetta and asparagus in a velvety sauce made with mascarpone, pecorino and Grana Padano cheese.
\item Chicken Curry Ramen.  Japanese fried chicken \& noodles in a delicious curry broth.
\item Pork Porchetta.  Slow-roasted pork belly in herbed red wine sauce, served with roasted new potatoes and broccoli.
\item Rustica Chorizo Pizza.  Chorizo salami, torn wild boar and pork meatballs, smoked mozzarella and baby sunblush tomatoes.
\item Goat's Cheese Salad.  Goat's cheese and balsamic onion crostinis on winter baby kale, red pepper, cucumber, plum tomatoes.
\item Chicken Katsu Curry.  Succulent chicken in a crispy Japanese panko crumb with mild curry sauce, pickles and steamed rice.
\item Aromatic Duck Rolls.  Aromatic roast duck, hoisin sauce, cucumber, spring onion, rolled in rice paper.
\item Wild Boar Polpette.  Oven-baked herby wild boar and pork meatballs in a pomodoro sauce with melted smoked mozzarella.
\item Aegean Slaw.  Thinly shredded cabbage, carrot and red onion peppers, with an olive oil dressing.
\item Cured Salmon Sashimi.  Dazzling beetroot cured slices of salmon served with lemon.
\item Jack Fruit Arrabiata.  Vegan jackfruit peppers, spicy harissa, roquito chilis \& vegan Mozzarella cheese, finished with basil.
\item Butternut Squash Salad.  Roasted butternut squash with feta cheese, pomegranate seeds, mixed leaf salad and watercress.
\item Lentil Linguine Ragu.  Rich Italian lentil ragu with baby spinach, tomatoes, basil \& pecorino cheese.
\item Vegan Spaghetti Lentil Ragu.  A hearty ragu of green lentils and mixed vegetables in a rich tomato and fennel sauce.
\item Vegan Meatballs.  Vegan meatballs in a rich tomato and fennel sauce.
\item Falafel with Tahini.  Our signature recipe, served with a tahini dip.
\item Harusami Aubergine.  Fried slices of aubergine with garlic \& ginger sesame soy.
\end{enumerate}

\paragraph{\claude+IQP}
\begin{enumerate}
\itemsep0em 
\item Cheese Melt Pizza. A gooey, cheesy pizza topped with savory mushrooms and tangy tomato sauce.
\item  Rustica Vegetable Pizza. A hearty vegetarian pizza loaded with savory mushrooms, smoky cheese, and sweet tomatoes.
\item  Mushroom Tagliatelle. Ribbon pasta tossed with a rich mushroom and red wine sauce, topped with creamy mozzarella and crispy sage leaves.
\item  Vegetable Hot Pizza. A spicy vegetarian pizza featuring a medley of peppers and savory mushrooms.
\item  Mushroom Bourguignon. A hearty vegan twist on the classic French dish, featuring meaty mushrooms and savory red wine sauce.
\item  Butternut Squash Carbonara. Creamy pasta featuring roasted butternut squash and crisp asparagus in a rich cheese sauce.
\item  Tofu Katsu Curry. Crispy breaded tofu cutlet served with a mild and aromatic curry sauce.
\item  Mushroom Rolls. Savory mushrooms and crisp vegetables wrapped in delicate rice paper with a sweet hoisin glaze.
\item  Slow Cooked Lentils in Chianti Sauce. Hearty lentils slow-simmered in a rich red wine sauce, topped with a fluffy cheese dumpling.
\item  Wild Mushroom Polpette. Oven-baked mushroom "meatballs" in a rich tomato sauce with melted smoky cheese.
\item  Lentil Linguine Ragu. Al dente linguine tossed with a robust lentil ragu and finished with fresh herbs and cheese.
\item  Lentil Meatballs with Spaghetti. Tender lentil meatballs served over spaghetti with a rich tomato sauce and creamy yoghurt.
\item  Lentil Ragu Spaghetti. A hearty lentil and vegetable ragu served over tender spaghetti with aromatic fennel.
\item  Jackfruit Arrabiata. Spicy jackfruit "pulled pork" with peppers and vegan cheese in a fiery arrabiata sauce.
\item  Tofu Fish and Chips. Crispy battered tofu fillets served with golden fries and a refreshing pea puree.
\item  Pork Ribs.  Pork Ribs smothered with Kentucky style BBQ sauce.
\item  Rustica Chorizo Pizza.  Chorizo salami, torn wild boar and pork meatballs, smoked mozzarella and baby sunblush tomatoes.
\item  Four Cheese Margherita.  Mozzarella, mascarpone, fontal and grana cheese on a tomato base.
\item  Panchetta Carbonara.  crispy pancetta and asparagus in a velvety sauce made with mascarpone, pecorino and Grana Padano cheese.
\item  Chicken Curry Ramen.  Japanese fried chicken \& noodles in a delicious curry broth.
\item  Chicken Katsu Curry.  Succulent chicken in a crispy Japanese panko crumb with mild curry sauce, pickles and steamed rice.
\item  Pork Porchetta.  Slow-roasted pork belly in herbed red wine sauce, served with roasted new potatoes and broccoli.
\item  Beer Battered Fish with Chips.  with frites and pea \& mint puree.
\item  Goat's Cheese Calzone.  Goats cheese, grilled aubergines, roasted peppers, oven roasted tomatoes.
\item  Aromatic Duck Rolls.  Aromatic roast duck, hoisin sauce, cucumber, spring onion, rolled in rice paper.
\item  Butterfly Chicken Burger.  Two succulent chicken breasts joined by crispy skin, packaged in a Portuguese roll and topped with tomato and lettuce.
\item  Halloumi Sticks with Mayo Avocado Dip.  Chunky sticks of grilled halloumi cheese with a chilli jam and mayo avocado dip.
\item  Falafel with Tahini.  Our signature recipe, served with a tahini dip.
\item  Lentil Linguine Ragu.  Rich Italian lentil ragu with baby spinach, tomatoes, basil \& pecorino cheese.
\item  Wild Boar Polpette.  Oven-baked herby wild boar and pork meatballs in a pomodoro sauce with melted smoked mozzarella.
\item  Vegan Spaghetti Lentil Ragu.  A hearty ragu of green lentils and mixed vegetables in a rich tomato and fennel sauce.
\item  Butternut Squash Salad.  Roasted butternut squash with feta cheese, pomegranate seeds, mixed leaf salad and watercress.
\item  Aegean Slaw.  Thinly shredded cabbage, carrot and red onion peppers, with an olive oil dressing.
\item  Jack Fruit Arrabiata.  Vegan jackfruit peppers, spicy harissa, roquito chilis \& vegan Mozzarella cheese, finished with basil.
\item Vegan Meatballs.  Vegan meatballs in a rich tomato and fennel sauce.
\item  Harusami Aubergine.  Fried slices of aubergine with garlic \& ginger sesame soy.
\end{enumerate}

\paragraph{\gemini+IQP}
\begin{enumerate}
\itemsep0em 
\item Pepperoni Melt with Tofu. Enjoy this twist on our classic Pepperoni Melt by swapping meat for tofu.
\item Four Cheese and Mushroom Pizza. A classic pizza with an added layer of flavor.
\item Chickpea and Spinach Curry. Warm up with this hearty and flavorful chickpea and spinach curry.
\item Rustica Veggie Pizza. This meat-free pizza is packed with flavor and will satisfy any pizza lover.
\item Veggie Supreme Pizza. A classic pizza, loaded with fresh veggies and bursting with flavor.
\item Spinach and Mushroom Carbonara. This creamy and flavorful pasta dish is sure to please everyone at the table.
\item Mushroom Bourguignon. Enjoy the rich flavors of our classic Beef Bourguignon with a meat-free twist.
\item Spicy Jackfruit Pizza. This spicy and delicious pizza is sure to tantalize your taste buds.
\item Mushroom and Spinach Meatballs. These flavorful and healthy meatballs are a great alternative to the classic.
\item Tofu Curry Ramen. A vegan alternative to our classic Chicken Curry Ramen, with crispy fried tofu.
\item Tofu Katsu Curry. All the goodness of our Chicken Katsu Curry in a satisfying plant-based alternative.
\item Mushroom and Lentil Ragu with Tagliatelle. A hearty and satisfying vegetarian take on our Beef Brisket and Venison Tagliatelle.
\item Mushroom Calzone. Savory and satisfying, our mushroom calzone is packed full of fresh ingredients.
\item Butternut Squash and Chickpea Salad . Enjoy all the current flavors of Fall in this colorful salad.
\item Lentil Meatballs with Spaghetti. Enjoy this lighter version of our Lamb Meatballs with Spaghetti.
\item Lentil Bolognese. This lentil bolognese is a hearty and flavorful alternative to our classic spaghetti dish.
\item Chicken Curry Ramen.  Japanese fried chicken \& noodles in a delicious curry broth.
\item Pork Ribs.  Pork Ribs smothered with Kentucky style BBQ sauce.
\item Chicken Katsu Curry.  Succulent chicken in a crispy Japanese panko crumb with mild curry sauce, pickles and steamed rice.
\item Beer Battered Fish with Chips.  with frites and pea \& mint puree.
\item Halloumi Sticks with Mayo Avocado Dip.  Chunky sticks of grilled halloumi cheese with a chilli jam and mayo avocado dip.
\item Four Cheese Margherita.  Mozzarella, mascarpone, fontal and grana cheese on a tomato base.
\item Panchetta Carbonara.  crispy pancetta and asparagus in a velvety sauce made with mascarpone, pecorino and Grana Padano cheese.
\item Rustica Chorizo Pizza.  Chorizo salami, torn wild boar and pork meatballs, smoked mozzarella and baby sunblush tomatoes.
\item Supreme Pizza.  Cheese and tomato, onion, mushroom, fresh basil, olive and garlic oil.
\item Goat's Cheese Calzone.  Goats cheese, grilled aubergines, roasted peppers, oven roasted tomatoes.
\item Aromatic Duck Rolls.  Aromatic roast duck, hoisin sauce, cucumber, spring onion, rolled in rice paper.
\item Butterfly Chicken Burger.  Two succulent chicken breasts joined by crispy skin, packaged in a Portuguese roll and topped with tomato and lettuce.
\item Canelloni.  pasta with béchamel, mozzarella and Gran Milano cheese.
\item Jack Fruit Arrabiata.  Vegan jackfruit peppers, spicy harissa, roquito chilis \& vegan Mozzarella cheese, finished with basil.
\item Falafel with Tahini.  Our signature recipe, served with a tahini dip.
\item Aegean Slaw.  Thinly shredded cabbage, carrot and red onion peppers, with an olive oil dressing.
\item Pork Porchetta.  Slow-roasted pork belly in herbed red wine sauce, served with roasted new potatoes and broccoli.
\item Butternut Squash Salad.  Roasted butternut squash with feta cheese, pomegranate seeds, mixed leaf salad and watercress.
\item Lentil Linguine Ragu.  Rich Italian lentil ragu with baby spinach, tomatoes, basil \& pecorino cheese.
\item Vegan Spaghetti Lentil Ragu.  A hearty ragu of green lentils and mixed vegetables in a rich tomato and fennel sauce.
\end{enumerate}

\paragraph{\llama+IQP}
\begin{enumerate}
\itemsep0em 
\item Grilled Aubergine and Goat's Cheese Salad. A flavorful and refreshing salad perfect for any time of the year.
\item Stuffed Bell Peppers. A colorful and flavorful vegetarian main course.
\item Grilled Halloumi and Vegetable Skewers. A simple yet impressive vegetarian appetizer or snack.
\item Grilled Portobello Mushroom and Goat's Cheese Salad. A flavorful and refreshing salad perfect for any time of the year.
\item Roasted Butternut Squash Risotto. A creamy and comforting dish featuring roasted butternut squash.
\item Spaghetti with Roasted Vegetable Ragu. A vibrant and satisfying vegan pasta dish.
\item Pan-Seared Portobello Mushroom Burger. A savory and filling vegetarian burger option.
\item Chickpea and Spinach Curry. A nutritious and aromatic curry perfect for a weeknight dinner.
\item Spaghetti with Grilled Aubergine and Tomato Sauce. A classic and comforting vegetarian pasta dish.
\item Roasted Butternut Squash and Sage Risotto. A creamy and comforting vegetarian main course.
\item Lentil and Mushroom Bolognese. A hearty and rich vegetarian take on the classic Bolognese.
\item Roasted Vegetable and Lentil Tagliatelle. A creative and delicious vegetarian take on the classic pasta dish.
\item Lentil and Vegetable Stew. A hearty and comforting vegetarian stew perfect for a cold winter's day.
\item Spaghetti with Lentil and Mushroom Bolognese. A hearty and rich vegetarian take on the classic Bolognese.
\item Vegan Lentil and Mushroom Meatballs. A tasty and satisfying vegan alternative to traditional meatballs.
\item Roasted Vegetable and Chickpea Wrap. A healthy and convenient vegetarian wrap option.
\item Grilled Aubergine and Red Pepper Salad. A flavorful and refreshing salad perfect for any time of the year.
\item Roasted Vegetable and Lentil Polpette. A creative and delicious vegan alternative to traditional meatballs.
\item Pork Ribs.  Pork Ribs smothered with Kentucky style BBQ sauce.
\item Rustica Chorizo Pizza.  Chorizo salami, torn wild boar and pork meatballs, smoked mozzarella and baby sunblush tomatoes.
\item Supreme Pizza.  Cheese and tomato, onion, mushroom, fresh basil, olive and garlic oil.
\item Halloumi Sticks with Mayo Avocado Dip.  Chunky sticks of grilled halloumi cheese with a chilli jam and mayo avocado dip.
\item Four Cheese Margherita.  Mozzarella, mascarpone, fontal and grana cheese on a tomato base.
\item Bufala Caprese.  Specialty tomatoes and drizzle of roasted garlic in extra virgin olive oil with Buffalo mozzarella.
\item Chicken Curry Ramen.  Japanese fried chicken \& noodles in a delicious curry broth.
\item Chicken Katsu Curry.  Succulent chicken in a crispy Japanese panko crumb with mild curry sauce, pickles and steamed rice.
\item Pork Porchetta.  Slow-roasted pork belly in herbed red wine sauce, served with roasted new potatoes and broccoli.
\item Butterfly Chicken Burger.  Two succulent chicken breasts joined by crispy skin, packaged in a Portuguese roll and topped with tomato and lettuce.
\item Beer Battered Fish with Chips.  with frites and pea \& mint puree.
\item Falafel with Tahini.  Our signature recipe, served with a tahini dip.
\item Panchetta Carbonara.  crispy pancetta and asparagus in a velvety sauce made with mascarpone, pecorino and Grana Padano cheese.
\item Butternut Squash Salad.  Roasted butternut squash with feta cheese, pomegranate seeds, mixed leaf salad and watercress.
\item Aegean Slaw.  Thinly shredded cabbage, carrot and red onion peppers, with an olive oil dressing.
\item Jack Fruit Arrabiata.  Vegan jackfruit peppers, spicy harissa, roquito chilis \& vegan Mozzarella cheese, finished with basil.
\item Vegan Meatballs.  Vegan meatballs in a rich tomato and fennel sauce.
\item Lentil Linguine Ragu.  Rich Italian lentil ragu with baby spinach, tomatoes, basil \& pecorino cheese.
\end{enumerate}

\paragraph{\gemini}
\begin{enumerate}
\itemsep0em 
\item Tofu Curry Ramen. A vegan take on our classic ramen, with crispy fried tofu in a fragrant curry broth.
\item Lentil Meatballs with Spaghetti . A lighter, plant-based version of our classic meatballs, packed with flavor.
\item Mushroom Bourguignon. A flavorful twist on a classic, featuring tender mushrooms in a rich red wine sauce.
\item Chicken Katsu Curry. Succulent chicken in a crispy Japanese panko crumb with mild curry sauce, pickles and steamed rice.
\item Pepperoni Melt. This one’s got our triple cheese blend, tomato sauce and pepperoni.
\item Chickpea Meatballs with Spaghetti. A hearty and flavorful vegetarian option, featuring tender chickpea meatballs.
\item Aromatic Duck Rolls. Aromatic roast duck, hoisin sauce, cucumber, spring onion, rolled in rice paper.
\item Mushroom and Lentil Chianti. A hearty and flavorful vegetarian stew featuring mushrooms and lentils in a rich Chianti red wine sauce.
\item Mushroom Porchetta. Savory mushrooms roasted to perfection in an herbed red wine sauce, served with roasted new potatoes and broccoli.
\item Cured Salmon Sashimi. Dazzling beetroot cured slices of salmon served with lemon. .
\item Mushroom and Tofu Tagliatelle. A vegetarian-friendly twist on our classic ragu: mushrooms and tofu in a rich red wine and beef dripping ragu.
\item Rustica Chorizo Pizza. Rustica Chorizo Pizza. Chorizo salami, torn wild boar and pork meatballs, smoked mozzarella and baby sunblush tomatoes. Topped with chilli threads and riserva cheese.
\item Butterfly Chicken Burger. Two succulent chicken breasts joined by crispy skin, packaged in a Portuguese roll and topped with tomato and lettuce.
\item Mushroom and Pepperoni Pizza. A classic pizza with your chocie of hot green, Roquito or jalapeno peppers.
\item Tofu Burger. A plant-based take on the classic burger, with flavorful tofu, fresh veggies, and tangy Dijon mayonnaise.
\item Beer Battered Fish with Chips. with frites and pea \& mint puree.
\item Lentil Polpette. Oven-baked lentil patties in a pomodoro sauce with melted smoked mozzarella.
\item Steak. Sliced Black Angus rump steak with red onion chutney, watercress, and Dijon mayonnaise.
\item Cheese and Tomato Cannelloni. Pasta with béchamel, mozzarella, tomato sauce and Gran Milano cheese.
\item Supreme Pizza. Cheese and tomato, onion, mushroom, fresh basil, olive and garlic oil.
\item Goat's Cheese Calzone. Goats cheese, grilled aubergines, roasted peppers, oven roasted tomatoes.
\item Vegan Spaghetti Lentil Ragu. A hearty ragu of green lentils and mixed vegetables in a rich tomato and fennel sauce.
\item Jack Fruit Arrabiata. Vegan jackfruit peppers, spicy harissa, roquito chilis \& vegan Mozzarella cheese, finished with basil.
\item Halloumi Sticks with Mayo Avocado Dip. Chunky sticks of grilled halloumi cheese with a chilli jam and mayo avocado dip.
\item Cheese Salad. Smoked cheddar, cheddar and sage stuffing.
\item Falafel with Tahini. Our signature recipe, served with a tahini dip.
\item Four Cheese Margherita. Mozzarella, mascarpone, fontal and grana cheese on a tomato base.
\item Vegan Meatballs. Vegan meatballs in a rich tomato and fennel sauce.
\item Bufala Caprese. Specialty tomatoes and drizzle of roasted garlic in extra virgin olive oil with Buffalo mozzarella.
\item Butternut Squash Salad. Roasted butternut squash with feta cheese, pomegranate seeds, mixed leaf salad and watercress.
\item Panchetta Carbonara. Crispy pancetta and asparagus in a velvety sauce made with mascarpone, pecorino and Grana Padano cheese.
\item Lentil Linguine Ragu. Rich Italian lentil ragu with baby spinach, tomatoes, basil \& pecorino cheese.
\item Aegean Slaw. Thinly shredded cabbage, carrot and red onion peppers, with an olive oil dressing.
\item Goat's Cheese Salad. Goat's cheese and balsamic onion crostinis on winter baby kale, red pepper, cucumber, plum tomatoes.
\item Harusami Aubergine. Fried slices of aubergine with garlic \& ginger sesame soy.
\item Inari Taco. Golden tofu pockets filled with sticky sushi rice, avocado salsa \& vegan sriracha mayo.
\end{enumerate}

\paragraph{\claude}
\begin{enumerate}
\itemsep0em 
\item Tofu curry ramen. Crispy fried tofu and noodles in a rich curry broth, topped with fresh pak choi and tangy pickled onions.
\item Mushroom ribs. Meaty mushrooms smothered in Kentucky style BBQ sauce, offering a plant-based twist on a classic.
\item Lentil bourguignon burger. Hearty lentil patty topped with crispy bacon, red wine braised onions, and melted Raclette cheese.
\item Tofu katsu curry. Crispy panko-crusted tofu with mild curry sauce, tangy pickles, and steamed rice.
\item Cheese and mushroom melt. Triple cheese blend melted over savory mushrooms and tangy tomato sauce.
\item Lentil meatballs with spaghetti. Handmade lentil patties grilled and topped with creamy Greek yoghurt, tomato sauce, and onions.
\item Mushroom spring rolls. Aromatic roasted mushrooms with hoisin sauce, cucumber, and spring onion, rolled in delicate rice paper.
\item Slow cooked lentils in Chianti sauce. Lentils and shallots slow cooked in rich Chianti red wine sauce with a cheesy dumpling.
\item Mushroom porchetta. Slow-roasted mushrooms in herbed red wine sauce, served with roasted new potatoes and broccoli.
\item Tofu sashimi. Dazzling beetroot cured slices of tofu served with zesty lemon.
\item Lentil and mushroom tagliatelle. A lentil, mushroom, and red wine ragu with tagliatelle, fresh bufala mozzarella and crispy sage.
\item Rustica mushroom pizza. Savory mushrooms, lentil meatballs, smoked mozzarella and baby sunblush tomatoes, topped with chilli threads and riserva cheese.
\item Tofu burger. Two succulent tofu fillets in a crispy coating, packaged in a Portuguese roll and topped with tomato and lettuce.
\item Mushroom hot pizza. Savory mushrooms, mozzarella and tomato, with your choice of hot green, Roquito or jalapeno peppers.
\item Lentil burger. Hearty lentil patty with baby gem lettuce and plum tomato in a chargrilled brioche bun with Dijon mayonnaise.
\item Beer battered tofu with chips. Crispy beer-battered tofu with frites and refreshing pea \& mint puree.
\item Lentil polpette. Oven-baked herby lentil meatballs in a pomodoro sauce with melted smoked mozzarella.
\item Grilled mushroom steak. Sliced grilled mushroom steak with red onion chutney, watercress, and Dijon mayonnaise.
\item Mushroom cannelloni. Pasta rolls filled with savory mushrooms, topped with béchamel, mozzarella and Gran Milano cheese.
\item Supreme veggie pizza. A medley of cheese, tomato, onion, mushroom, fresh basil, olive and garlic oil.
\item Goat's cheese calzone. Goat's cheese, grilled aubergines, roasted peppers, and oven roasted tomatoes in a folded pizza pocket.
\item Lentil spaghetti ragu. A hearty ragu of green lentils and mixed vegetables in a rich tomato and fennel sauce.
\item Jackfruit arrabiata. Spicy jackfruit and peppers with harissa, roquito chilis \& vegan mozzarella cheese, finished with basil.
\item Halloumi sticks with avocado dip. Chunky sticks of grilled halloumi cheese with a chilli jam and creamy mayo avocado dip.
\item Cheese salad. A variety of cheeses including smoked cheddar and cheddar, with aromatic sage stuffing.
\item Falafel with tahini. Our signature chickpea falafel recipe, served with a creamy tahini dip.
\item Four cheese margherita. A blend of mozzarella, mascarpone, fontal and grana cheese on a classic tomato base.
\item Lentil meatballs. Savory lentil meatballs in a rich tomato and fennel sauce.
\item Tofu caprese. Marinated tofu with specialty tomatoes and a drizzle of roasted garlic in extra virgin olive oil with Buffalo mozzarella.
\item Butternut squash salad. Roasted butternut squash with tangy feta cheese, pomegranate seeds, mixed leaf salad and peppery watercress.
\item Mushroom carbonara. Savory mushrooms and crisp asparagus in a velvety sauce made with mascarpone, pecorino and Grana Padano cheese.
\item Lentil linguine ragu. Rich Italian lentil ragu with baby spinach, tomatoes, basil \& pecorino cheese.
\item Aegean slaw. Thinly shredded cabbage, carrot and red onion peppers, with a light olive oil dressing.
\item Goat's cheese salad. Goat's cheese and balsamic onion crostinis on winter baby kale, red pepper, cucumber, and plum tomatoes.
\item Harusami aubergine. Fried slices of aubergine with aromatic garlic \& ginger sesame soy sauce.
\item Inari taco. Golden tofu pockets filled with sticky sushi rice, avocado salsa \& spicy vegan sriracha mayo.
\end{enumerate}

\paragraph{\gptfouro}
\begin{enumerate}
\itemsep0em 
\item Tofu Curry Ramen. Japanese fried tofu \& noodles in a delicious curry broth. Topped with pak choi \& pickled onions.
\item Vegetarian Ribs. Grilled tofu slathered in rich, Kentucky-style BBQ sauce. .
\item  Vegetarian Bourguignon Burger. Lentil patty burger topped with crispy mushrooms, red wine braised onions, Raclette cheese.
\item  Tofu Katsu Curry. Succulent tofu in a crispy Japanese panko crumb with mild curry sauce, pickles, and steamed rice.
\item Mushroom Melt. This one’s got our triple cheese blend, tomato sauce, and grilled mushrooms.
\item  Chickpea Meatballs with Spaghetti. Handmade chickpea patties grilled and topped with Greek yoghurt, tomato sauce, and onions.
\item  Aromatic Tofu Rolls. Aromatic roast tofu, hoisin sauce, cucumber, spring onion, rolled in rice paper.
\item  Slow Cooked Mushroom in Chianti Sauce. Portobello mushrooms and shallots slow cooked in rich Chianti red wine sauce with a cheesy dumpling.
\item  Lentil Porchetta. Slow-roasted lentil loaf in herbed red wine sauce, served with roasted new potatoes and broccoli.
\item Beetroot Cured Tofu Sashimi. Dazzling beetroot cured slices of tofu served with lemon.
\item  Mushroom and Lentil Tagliatelle. A mushroom and lentil ragu with tagliatelle, fresh bufala mozzarella, and crispy sage.
\item  Rustica Tofu Pizza. Grilled tofu, smoked mozzarella, baby sunblush tomatoes, topped with chilli threads and riserva cheese.
\item  Grilled Tofu Burger. Two succulent pieces of grilled tofu joined by a crispy skin, packaged in a Portuguese roll and topped with tomato and lettuce.
\item  Vegetarian Hot Pizza. Grilled tofu, mozzarella, and tomato, with your choice of hot green, Roquito, or jalapeno peppers.
\item  Lentil Burger. 6oz lentil patty with baby gem lettuce and plum tomato in a chargrilled brioche bun with Dijon mayonnaise.
\item  Beer Battered Tofu with Chips. Beer battered tofu with frites and pea \& mint puree.
\item  Wild Mushroom Polpette. Oven-baked herby wild mushroom and lentil meatballs in pomodoro sauce with melted smoked mozzarella.
\item  Vegetarian Steak. Grilled portobello mushrooms with red onion chutney, watercress, and Dijon mayonnaise.
\item  Cheese Cannelloni. Pasta with béchamel, mozzarella, and Gran Milano cheese.
\item  Supreme Vegetable Pizza. Cheese and tomato, onion, mushroom, fresh basil, olive oil, and garlic oil.
\item  Goat's Cheese Calzone. Goat's cheese, grilled aubergines, roasted peppers, oven roasted tomatoes in a folded pizza.
\item  Lentil Spaghetti Ragu. A hearty ragu of green lentils and mixed vegetables in a rich tomato and fennel sauce.
\item  Jackfruit Arrabbiata. Vegan jackfruit, spicy harissa, roquito chilis, and vegan mozzarella cheese, finished with basil.
\item  Halloumi Sticks with Mayo Avocado Dip. Chunky sticks of grilled halloumi cheese with a chilli jam and mayo avocado dip.
\item  Cheese Salad. Smoked cheddar, cheddar, and sage stuffing served as a salad.
\item  Falafel with Tahini. Our signature chickpea falafel served with a tahini dip.
\item  Four Cheese Margherita Pizza. Mozzarella, mascarpone, fontal, and grana cheese on a tomato base.
\item  Vegetarian Meatballs. Lentil meatballs in a rich tomato and fennel sauce.
\item  Bufala Caprese. Specialty tomatoes and roasted garlic in extra virgin olive oil with Buffalo mozzarella.
\item  Butternut Squash Salad. Roasted butternut squash with feta cheese, pomegranate seeds, mixed leaf salad, and watercress.
\item  Egg Carbonara. Velvety carbonara sauce made with egg, crispy pancetta, and asparagus. .
\item  Lentil Linguine. Green lentils, baby spinach, tomatoes, basil, and pecorino cheese with linguine.
\item  Aegean Slaw. Thinly shredded cabbage, carrot and red onion peppers with an olive oil dressing.
\item  Goat's Cheese Salad. Goat's cheese and balsamic onion crostinis on winter baby kale, red pepper, cucumber, plum tomatoes.
\item  Harusami Mushroom. Fried mushrooms with garlic, ginger, and sesame soy.
\item  Inari Taco. Golden tofu pockets filled with sticky sushi rice, avocado salsa, and vegan sriracha mayo.
\end{enumerate}



\paragraph{\opreview}
\begin{enumerate}
\itemsep0em 
\item Tofu Curry Ramen. A flavorful bowl of ramen with crispy fried tofu and aromatic curry broth.
\item Grilled Mushroom BBQ Skewers. Juicy and tender mushrooms glazed with smoky barbecue sauce.
\item Lentil Bourguignon. A hearty twist on a classic, featuring tender lentils in rich red wine sauce topped with melted Raclette cheese.
\item Egg Katsu Curry. Delight in a crispy egg coated in panko crumbs, served with a mild curry sauce.
\item Mushroom Melt. This melt features savory mushrooms with melted cheese and rich tomato sauce.
\item Chickpea Meatballs with Spaghetti. Handmade chickpea patties grilled and served with spaghetti and tangy tomato sauce.
\item Aromatic Tofu Rolls. Fresh rolls filled with aromatic tofu and crisp vegetables.
\item Slow Cooked Mushrooms in Chianti Sauce. Earthy mushrooms slow-cooked in rich Chianti wine sauce with a cheesy dumpling.
\item Stuffed Aubergine Porchetta. Roasted aubergine stuffed with aromatic herbs and served with hearty sides.
\item Tofu Sashimi. Delicate slices of tofu served with a squeeze of lemon.
\item Mushroom and Lentil Tagliatelle. A rich and hearty pasta with mushrooms and lentils in a savory ragu.
\item Rustica Veggie Pizza. A delightful pizza loaded with vegetables and smoky cheeses.
\item Grilled Halloumi Burger. A savory burger featuring grilled halloumi and fresh toppings.
\item Spicy Veggie Pizza. A fiery pizza topped with spicy peppers and melted cheese.
\item Gourmet Mushroom Burger. A hearty burger featuring grilled mushrooms and classic fixings.
\item Beer Battered Tofu with Chips. Crispy beer-battered tofu served with traditional sides.
\item Vegan Polpette. Herby vegan meatballs baked in flavorful tomato sauce with melted cheese.
\item Grilled Aubergine Stack. Layers of grilled aubergine with tangy red onion chutney.
\item Spinach and Cheese Cannelloni. Classic Italian dish with creamy spinach and cheese filling.
\item Supreme Veggie Pizza. A classic pizza loaded with fresh vegetables.
\item Goat's Cheese Calzone. A folded pizza filled with Mediterranean vegetables and goat's cheese.
\item Vegan Spaghetti Lentil Ragu. A hearty vegan spaghetti with rich lentil ragu.
\item Jackfruit Arrabiata. A spicy and flavorful dish with tender jackfruit.
\item Halloumi Sticks with Mayo Avocado Dip. Chunky grilled halloumi sticks served with a delicious dip.
\item Cheese Salad. A satisfying salad featuring a variety of cheeses.
\item Falafel with Tahini. Crispy falafel made from chickpeas served with creamy tahini.
\item Four Cheese Margherita. An indulgent pizza for cheese lovers.
\item Vegan Meatballs. Savory plant-based meatballs in a rich sauce.
\item Bufala Caprese. A fresh and delicious Italian classic.
\item Butternut Squash Salad. A vibrant salad with sweet and tangy flavors.
\item Asparagus Carbonara. A creamy pasta dish featuring tender asparagus.
\item Lentil Linguine Ragu. A comforting pasta with rich lentil sauce.
\item Aegean Slaw. A crisp and refreshing slaw with Mediterranean flavors.
\item Goat's Cheese Salad. A hearty salad with tangy goat's cheese and fresh vegetables.
\item Harusami Aubergine. Fried slices of aubergine in a flavorful sauce.
\item Inari Taco. Golden tofu pockets filled with rice and zesty salsa.
\end{enumerate}

\paragraph{\opreview, Remove Descriptions}
\begin{enumerate}
\itemsep0em 
\item Creamy Mushroom Tagliatelle. Mushrooms,  tagliatelle pasta,  mascarpone cheese,  baby spinach,  garlic oil,  basil. 
\item Vegetable Delight Pizza. Mozzarella,  tomato sauce,  grilled aubergines,  roasted peppers,  oven-roasted tomatoes,  fresh basil.
\item Three Cheese Omelette. Eggs,  smoked cheddar,  mozzarella,  grana cheese,  red onions. 
\item Eggplant Parmesan. Aubergine,  tomato sauce,  mozzarella,  Grana Padano cheese,  basil. 
\item Mushroom and Goat's Cheese Omelette. Eggs,  mushrooms,  goat's cheese,  garlic oil,  basil.
\item Chickpea Curry with Rice. Chickpeas,  mild curry sauce,  pickles,  steamed rice.
\item Spinach and Feta Stuffed Mushrooms. Mushrooms,  baby spinach,  feta cheese,  garlic oil.
\item Lentil Veggie Burger. Lentils,  chargrilled brioche bun,  baby gem lettuce,  plum tomato,  Dijon mayonnaise.
\item Falafel Salad. Falafel,  mixed leaf salad,  cucumber,  plum tomatoes,  red onion,  tahini dip.
\item Egg Shakshuka. Eggs,  tomato sauce,  red peppers,  red onions,  garlic oil,  basil. 
\item Vegetable and Tofu Stir-Fry. Tofu,  noodles,  pak choi,  red peppers,  red onions,  garlic \& ginger sesame soy. 
\item Tofu Katsu Curry. Tofu in crispy Japanese panko crumb,  mild curry sauce,  pickles,  steamed rice. 
\item Lentil Stuffed Peppers. Bell peppers,  lentils,  tomato sauce,  baby spinach, goat's cheese.
\item Butternut Squash and Feta Salad. Roasted butternut squash,  feta cheese,  pomegranate seeds,  mixed leaf salad,  watercress. 
\item Vegan Meatball Sub. Vegan meatballs,  pomodoro sauce,  smoked mozzarella,  Portuguese roll.
\item Chickpea and Spinach Curry. Chickpeas,  baby spinach,  mild curry sauce,  steamed rice. 
\item Tofu Curry Ramen. Fried tofu,  noodles,  curry broth,  pak choi,  pickled onions.
\item Mushroom and Lentil Bolognese. Mushrooms,  lentils,  tagliatelle pasta,  tomato sauce,  basil.
\item Supreme Pizza.  Cheese and tomato, onion, mushroom, fresh basil, olive and garlic oil.
\item Chicken Curry Ramen.  Japanese fried chicken \& noodles in a delicious curry broth.
\item Pork Ribs.  Pork Ribs smothered with Kentucky style BBQ sauce.
\item Butterfly Chicken Burger.  Two succulent chicken breasts joined by crispy skin, packaged in a Portuguese roll and topped with tomato and lettuce.
\item Four Cheese Margherita.  Mozzarella, mascarpone, fontal and grana cheese on a tomato base.
\item Panchetta Carbonara.  crispy pancetta and asparagus in a velvety sauce made with mascarpone, pecorino and Grana Padano cheese.
\item Chicken Katsu Curry.  Succulent chicken in a crispy Japanese panko crumb with mild curry sauce, pickles and steamed rice.
\item Rustica Chorizo Pizza.  Chorizo salami, torn wild boar and pork meatballs, smoked mozzarella and baby sunblush tomatoes.
\item Beer Battered Fish with Chips.  with frites and pea \& mint puree.
\item Pork Porchetta.  Slow-roasted pork belly in herbed red wine sauce, served with roasted new potatoes and broccoli.
\item Butternut Squash Salad.  Roasted butternut squash with feta cheese, pomegranate seeds, mixed leaf salad and watercress.
\item Aegean Slaw.  Thinly shredded cabbage, carrot and red onion peppers, with an olive oil dressing.
\item Cured Salmon Sashimi.  Dazzling beetroot cured slices of salmon served with lemon.
\item Falafel with Tahini.  Our signature recipe, served with a tahini dip.
\item Lentil Linguine Ragu.  Rich Italian lentil ragu with baby spinach, tomatoes, basil \& pecorino cheese.
\item Aromatic Duck Rolls.  Aromatic roast duck, hoisin sauce, cucumber, spring onion, rolled in rice paper.
\item Vegan Spaghetti Lentil Ragu.  A hearty ragu of green lentils and mixed vegetables in a rich tomato and fennel sauce.
\item Harusami Aubergine.  Fried slices of aubergine with garlic \& ginger sesame soy.
\end{enumerate}


\end{document}


% This document was modified from the file originally made available by
% Pat Langley and Andrea Danyluk for ICML-2K. This version was created
% by Iain Murray in 2018, and modified by Alexandre Bouchard in
% 2019 and 2021 and by Csaba Szepesvari, Gang Niu and Sivan Sabato in 2022.
% Modified again in 2023 and 2024 by Sivan Sabato and Jonathan Scarlett.
% Previous contributors include Dan Roy, Lise Getoor and Tobias
% Scheffer, which was slightly modified from the 2010 version by
% Thorsten Joachims & Johannes Fuernkranz, slightly modified from the
% 2009 version by Kiri Wagstaff and Sam Roweis's 2008 version, which is
% slightly modified from Prasad Tadepalli's 2007 version which is a
% lightly changed version of the previous year's version by Andrew
% Moore, which was in turn edited from those of Kristian Kersting and
% Codrina Lauth. Alex Smola contributed to the algorithmic style files.
