\section{Introduction}

% \copied{{Modern DRAM chips} {{suffer from} read disturbance~\cite{kim2014flipping,
% mutlu2019retrospective,mutlu2023fundamentally,luo2023rowpress, olgun2024read} issues that can 
% be exploited to break memory isolation, threatening the safety, security, 
% and reliability of modern DRAM-based computing systems. RowHammer~\cite{kim2014flipping} 
% and RowPress~\cite{luo2023rowpress} are two prominent examples of read 
% disturbance. Repeatedly opening/activating and closing a DRAM row 
% (i.e., aggressor row) \emph{many times} (e.g., tens of thousands) induces 
% \emph{RowHammer bitflips} in physically nearby rows (i.e., victim rows). 
% Keeping the aggressor row open for a long period of time (i.e., a 
% large aggressor row on time, $t_{AggON}$) amplifies the effects of 
% read disturbance and induces \emph{RowPress bitflips}, \emph{without many} 
% repeated aggressor row activations~\cite{luo2023rowpress}.
% {N}umerous studies {demonstrate that} a malicious attacker 
% can {reliably} cause {{read disturbance} bitflips} in 
% a targeted manner to compromise system integrity, confidentiality, and 
% availability~\exploitingRowHammerAllCitations{}. 
% {{Read disturbance worsens} {in new DRAM chips with {smaller technology nodes}, 
% {where}} {RowHammer} bitflips 1)~{happen} with fewer row activations, e.g., 
% $10\times$ reduction in less than a decade~\cite{kim2020revisiting} and 
% 2)~{appear} {in} more DRAM cells, compared to old DRAM 
% {chips}~\rowHammerGetsWorseCitations{}.}}}

\atbcomment{from svard}\copied{Read disturbance~\cite{kim2014flipping,
mutlu2019retrospective,mutlu2023fundamentally,luo2023rowpress, olgun2024read}
(e.g., RowHammer and RowPress) in modern DRAM chips is a widespread
phenomenon and is reliably used for breaking memory isolation~\exploitingRowHammerAllCitations{}, 
a fundamental building block for building robust systems.}
\atbcomment{from HBM RD}\copied{Repeatedly opening/activating and closing a DRAM row 
(i.e., aggressor row) \emph{many times} (e.g., tens of thousands) induces 
\emph{RowHammer bitflips} in physically nearby rows (i.e., victim rows). 
Keeping the aggressor row open for a long period of time (i.e., a 
large aggressor row on time, $t_{AggON}$) amplifies the effects of 
read disturbance and induces \emph{RowPress bitflips}, \emph{without many} 
repeated aggressor row activations~\cite{luo2023rowpress}.}

A large body of work~\mitigatingRowHammerAllCitations{} proposes 
various techniques to mitigate DRAM read disturbance bitflips. 
Many high-performance and low-overhead mitigation techniques (e.g.,~\cite{jedecddr5c,olgun2024abacus,saxena2024start,
yaglikci2024spatial,bostanci2024comet,qureshi2022hydra,
canpolat2024understanding}) prevent read disturbance bitflips 
by refreshing (i.e., opening and closing) a victim row (i.e., 
by performing a \emph{preventive} refresh operation) \emph{before} 
a read disturbance bitflip manifests in that row. Since a 
preventive refresh operation induces increasing and 
significant performance and energy overheads with worsening 
read disturbance vulnerability~\cite{XYZ}, a secure, high-performance, 
and energy-efficient memory system should perform a preventive 
refresh operation \emph{if and only if} the operation is necessary. For example, 
if a victim row can withstand 200 aggressor row activations 
before it exhibits a bitflip 1) the victim row must be refreshed
before the aggressor row is activated 200 times, 
and 2) the victim row must \emph{not} be preventively refreshed if the aggressor
row is activated fewer than 199 times.

To securely prevent read disturbance bitflips and unnecessary preventive refresh operations, 
it is important to 
accurately identify {the amount of read disturbance} 
that a victim row can withstand before the victim row exhibits 
a read disturbance bitflip. 
This amount is typically quantified 
using the \emph{aggressor row activation count needed 
to induce the first read disturbance bitflip} in the 
victim row ($AC_{min}$), also called the \emph{read disturbance threshold (RDT)} 
of the victim row. 

Identifying (via experimental measurements on real DRAM chips) the RDT of a 
victim row is \emph{time intensive} due to two reasons.
First, identifying the RDT of a row requires searching through a relatively large
search space of many possible RDT values (e.g., ranging from 1,000 to 500,000). Each step of the 
search requires activating an aggressor row many times (e.g., for a few miliseconds) to induce
the first read disturbance bitflip in the victim row.
Second, various parameters and operational conditions
such as how long the aggressor row is kept open 
($t_{AggON}$)~\cite{luo2023rowpress, olgun2024read, orosa2021deeper}, 
data stored in DRAM cells in and nearby the victim row~\cite{kim2014flipping,
kim2020revisiting, olgun2024read}, temperature~\cite{orosa2024spyhammer, 
orosa2021deeper}, and DRAM voltage~\cite{yaglikci2022understanding, cohen2022hammerscope} 
significantly affect the RDT of a victim row. Therefore accurately identifying the RDT 
of a victim row 1) is critical for system security and performance, and 
2) requires time intensive, careful, and extensive 
characterization of a DRAM chip. An identified RDT value \emph{ideally} is 
consistently accurate (i.e., reliable) such that we prevent
read disturbance bitflips at low performance and energy cost \emph{without} 
having to repeatedly identify the RDT of a victim row.

% Our \textbf{goal} is twofolds. First, we would like to understand if the RDT of a victim row can be reliably determined (i.e., in a way that yields consistently accurate RDT values). Second, 

% \outline{(Maybe) Overview of assumptions regarding read disturbance threshold 
% made by recent studies.}
% Prior works~\cite{XYZ} develop RDT characterization methodologies to study and understand the read disturbance 
% vulnerability in real DRAM chips (e.g., RowHammer~\cite{XYZ} and RowPress~\cite{XYZ}). 
% These works determine the RDT of a
% DRAM row as the minimum observed RDT across \emph{fewer than six} successive RDT observations. 
% Although no prior work claims that this RDT characterization methodology
% is reliable (i.e., the methodology yields consistently accurate RDT values), 
% this methodology is the most reliable RDT characterization methodology in the literature.


% \outline{What we experimentally demonstrate (a summary). 
% Describe the methodology.}.
Our \textbf{goal} is to understand how reliably the RDT of a 
victim row can be identified (in a way that yields consistently 
accurate RDT values for the row). To this end, we experimentally 
characterize the read disturbance vulnerability in \param{X} DDR4 modules and \param{Y} HBM2
chips from three major DRAM manufacturers. Our characterization yields \param{Z} important observations. 
We discuss the implications of our observations for existing read disturbance mitigation techniques. 
We develop and evaluate a new, lightweight and configurable, read disturbance mitigation 
configuration methodology.


% We experimentally demonstrate that the RDT 
% of a DRAM row \emph{significantly} and \emph{unpredictably} changes over time. 

\noindent
\textbf{Key Observations from Experimental Characterization.} We make \param{X} key observations. 
First, the RDT of a DRAM row changes with successive RDT measurements (i.e., changes over 
time).~\figref{fig:motivation_example} shows 100,000 successive measurements (x-axis) of the RDT (y-axis) 
of a victim row as a pictorial example of our first observation.\atbcomment{Make figure axis names consistent}

\begin{figure}[!ht]
    \centering
    \includegraphics[width=\linewidth]{figures/motivational_figure.png}
    \caption{Read disturbance threshold (RDT) of a DRAM cell over 
    100,000 successive measurements (left). 
    Each box shows the distribution of RDT across 1,000 successive measurements. 
    The circles show the mean RDT and the error bars show the range of 
    measured RDT values. Zoomed-in view to the RDT of the cell over the 
    last 1,000 successive measurements (right).}
    \label{fig:motivation_example}
\end{figure}

Second, the RDT of a victim row changes significantly and unpredictably over time. 
The minimum RDT of a victim row is \param{2.8}$\times{}$ smaller than the maximum
RDT of the row over 1,000 successive RDT measurements. We demonstrate that the RDT 
of a victim row functions as a high quality true random number generator whose output
passes standard randomness tests~\cite{X}.

Third, \param{SOMETHING ABOUT THE EFFECT OF TEMPERATURE AND ROW ON TIME?}

Fourth, \param{some of the DRAM rows have ``consistent'' RDT values across 1,000 measurements.}


\noindent
\textbf{Implications for Mitigations.} Many existing read 
disturbance mitigation techniques, including those that are implemented in state-of-the-art
DRAM chips~\cite{jedecddr5c, bennett2021panopticon, kim2023ddr5, saroiu2024ddr5, 
canpolat2024understanding}, are configured according to the measured RDT of a DRAM chip,
and thereby rely on accurate RDT characterization.\footnote{Victim-row-isolation-based 
read disturbance mitigation techniques~\cite{}
do \emph{not} require RDT characterization. We discuss these techniques in
detail in~\secref{sec:related-work}.} The results of our experimental characterization 
imply that 1) the RDT of a victim row \emph{cannot} always be determined with a 
relatively short, one-time RDT characterization effort and 2) a mitigation technique
\emph{cannot} deterministically prevent read disturbance bitflips
because the RDT of a DRAM row unpredictably changes with time. Thus, a mitigation technique
can \emph{only} probabilistically prevent read disturbance bitflips (even if the mitigation technique
keeps precise track of how many times each DRAM row is activated for how long). Unfortunately,
the probabilistic bitflip prevention guarantees (i.e., the security guarantees) of many 
mitigation techniques are unclear. None of the existing mitigation technique security 
evaluation methodologies consider the unpredictability in the RDT of a victim row.
Thereby, a memory
system designer \emph{cannot} properly assess if an implemented mitigation technique
achieves system design goals.

% The probabilistic bitflip prevention guarantees of many read disturbance 
% mitigation techniques are currently unknown.


% There is no method that can be used to assess the probabilistic bitflip prevention 
% guarantees that a mitigation mechanism provides. 

% Memory system designers
% need to assess the probabilistic read disturbance bitflip prevention
% guarantees of their mitigation techniques. Doing so necessitates a new methodology
% for evaluating the probabilistic guarantees of a read disturbance mitigation technique based on 
% the RDT profile (e.g., the RDT of every DRAM row) of a DRAM chip. The system 
% designer obtains the RDT profile via time intensive experimental characterization.\footnote{The
% system designer \emph{cannot} use a manufacturer-provided RDT profile for a DRAM chip.
% DRAM manufacturers do \emph{not} disclose the RDT profile of their DRAM chips.} However,
% the experimentally characterized RDT profile likely is only partially complete because it is obtained
% in finite time.
% The same 
% methodology can be used to configure the RDT of a mitigation technique
% such that the mitigation technique achieves system design goals. 

% obtained by the system designer through experimental characterization.\footnote{The
% system designer \emph{cannot} use a manufacturer-provided RDT profile for a DRAM chip.
% DRAM manufacturers do \emph{not} disclose the RDT profile of their DRAM chips.}

% \noindent
% \textbf{A New Mitigation Configuration Methodology.} RDT characterization is time intensive
% and \emph{only} yields partially complete information about the RDT of a DRAM row 
% (e.g., finding a row's minimum RDT might need multiple characterization runs). 
% To enable
\noindent
\textbf{A New Mitigation Configuration Methodology.}
To enable system designers determine the probabilistic guarantees of their read disturbance
mitigation technique, 
we develop a new, lightweight, experimental characterization guided methodology 
for configuring and evaluating read disturbance mitigation techniques.\footnote{The
system designer \emph{cannot} use manufacturer-provided RDT characterization results for a DRAM chip.
DRAM manufacturers do \emph{not} disclose the RDT characterization results of their DRAM chips.} 
We call this methodology \emph{Incertus}.\footnote{\param{What
Incertus means and why we name it this way.}} 
\param{Key idea of Incertus and key results of its evaluation.}

This work makes the following key contributions:
\begin{itemize}
    \item 
\end{itemize}