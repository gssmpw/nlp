\begin{abstract}
Modern DRAM chips are subject to read disturbance errors. \omcr{2}{These errors}
manifest as security-critical bitflips in a \emph{victim DRAM row} that
\omcr{2}{is physically nearby} a repeatedly activated (opened)
\atbcr{1}{aggressor row (RowHammer)} or \atbcr{1}{an aggressor row that is} kept
open \atbcr{1}{for a long time (RowPress)}. State-of-the-art read disturbance
mitigation\nb{s}
%techniques
rely on %\gra{the} 
accurate and exhaustive characterization of the \emph{read disturbance
threshold} (RDT) (e.g., the number of aggressor row activations needed to induce
the first RowHammer \omcr{1}{or RowPress} bitflip) of every DRAM row
(\omcr{1}{of which there are millions or} billions in a modern system) to
prevent read disturbance bitflips securely and with low overhead.

We experimentally demonstrate for the first time that the RDT of a DRAM row
significantly and unpredictably changes \omcr{1}{over} time. \atbcr{2}{We call
this new phenomenon variable read disturbance (VRD).} Our ex\omcr{2}{tensive}
experiments using \param{\atbcr{1}{160}} DDR4 \atbcr{1}{chips} and \param{4}
HBM2 chips from three major manufacturers yield \param{three} key observations.
First, it is \omcr{1}{very} unlikely \omcr{1}{that} \omcr{1}{relatively few RDT
measurements can} accurately identify the RDT of a DRAM row. The minimum RDT of
a DRAM row appear\atbcr{1}{s} after \atbcr{1}{tens of thousands of} measurements
(e.g., up to \param{94,467}\atbcrcomment{1}{assuming you ignore all
time cost but that of hammering, hammering 94K times takes around 38 seconds for
a row with avg nRH = 4000.})\gra{,} and the minimum RDT of a DRAM row
\atbcr{1}{is} \param{\atbcr{1}{3.5}}$\times{}$ smaller than the
\atbcr{1}{maximum} RDT \omcr{1}{observed for} that row. Second, the probability
of \atbcr{1}{accurately} identifying a row's RDT with a relatively small
number of measurements reduces with increasing chip density or \omcr{1}{smaller}
technology node \omcr{1}{size}. Third, data pattern, the amount of time an
aggressor row is kept open, and temperature can affect\atbcrcomment{1}{The
effects of these three do not look significant. We also do not do a standard
statistical significance test for the effect of these 3 parameters. From the
looks of the distributions in figures, I expect that analysis to yield a
negative result.} \atbcr{1}{the probability of accurately identifying a DRAM
row's RDT}. 

\atb{Our empirical results have implications for the security guarantees of read
disturbance mitigation techniques: \atbcr{1}{if the RDT of a DRAM row is
\emph{not} identified accurately, these techniques can easily become insecure.}
\atbcr{1}{We discuss and evaluate using a guardband for RDT and error-correcting
codes for mitigating read disturbance bitflips in the presence of RDTs that
change unpredictably over time. We conclude that \atbcr{2}{a $>$10\%} guardband
\atbcr{2}{for the minimum observed RDT} combined with \atbcr{2}{SECDED or
Chipkill-like SSC} error-correcting codes could prevent read disturbance
bitflips at the \atbcrcomment{3}{We do not know in the end what guardband people
will want to apply. 50 guardband alone does not fix the problem. 50 guardband
with ECC seems to fix the problem. arguably 20 guardband + ecc also prevents all
errors given what we observe. Keep or remove?}\atbcrcomment{4}{keep this
note}cost of \omcr{2}{large} read disturbance mitigation performance overheads
(e.g., 45\% \omcr{2}{performance loss} for an RDT guardband of
50\%)\atbcrcomment{2}{is it bad to highlight 50\% guardband here?}.}
\atbcr{1}{We hope and believe} future work on efficient online profiling
mechanisms and configurable read disturbance mitigation techniques
\atbcr{1}{could remedy the challenges imposed on today's read disturbance
mitigations by the variable read disturbance phenomenon}.}

\end{abstract}