\section{Background and Motivation}

This section provides a concise overview of 1)~DRAM organization, 2)~DRAM
operation\gra{,} 3)~DRAM read disturbance, \omcr{2}{and 4)~our motivation in
this work}. 

\subsection{DRAM Organization}
\copied{\figref{fig:dram_organization} shows the organization of a DRAM-based
memory system. A \iey{\emph{memory channel}} connects the processor (CPU) to
\iey{a \emph{DRAM module}, where a module consists of multiple \emph{DRAM
ranks}. A DRAM rank is formed by a set of \emph{DRAM chips} that are operated in
lockstep}. Each \iey{DRAM} chip has multiple \iey{\emph{DRAM banks}}. \iey{DRAM
\emph{cells} in a DRAM bank are laid out in a two-dimensional structure of rows
and columns.} {A} DRAM cell {stores one bit of data} {in the form of} electrical
charge in {a} capacitor, which can be accessed through an access transistor. A
wire called \omcr{2}{\emph{wordline}} drives the gate of all DRAM cells' access
transistors in a DRAM row. A wire called \emph{bitline} connects all DRAM cells
in a DRAM column to a common differential sense amplifier. Therefore, when a
wordline is asserted, each DRAM cell in the DRAM row is connected to its
corresponding sense amplifier. The set of sense amplifiers is called \emph{the
row buffer}, where the data of an activated DRAM row is buffered to serve a
column access \atbcr{5}{operation}.}

\begin{figure}[!ht]
    \centering
    \includegraphics[width=\linewidth]{figures/dram_background.png}
    \caption{DRAM module, rank, chip, and bank organization}
    \label{fig:dram_organization}
\end{figure}

\subsection{DRAM Operation}

\copied{The memory controller serves memory access requests by issuing DRAM
commands, e.g., row activation ($ACT$), bank precharge ($PRE$), data read
($RD$), data write ($WR$), and refresh ($REF$) while respecting certain timing
parameters to guarantee correct operation~\cite{jedec2020lpddr5,
jedec2015lpddr4,jedecddr,jedec2020ddr4,jedec2012ddr3,jedecddr5c,jedec2021hbm}.
\hpcarevcommon{The memory controller issues} an $ACT$ command alongside the bank
address and row address corresponding to the memory request's address
\hpcarevcommon{to activate a DRAM row}. During the row activation process, a
DRAM cell loses its charge, and thus, its initial charge needs to be restored
(via a process called \emph{charge restoration}). The latency from the start of
a row activation until the completion of the DRAM cell's charge restoration is
called the \emph{\gls{tras}}. To access another row in an already activated DRAM
bank, the memory controller must issue a $PRE$ command to close the opened row
and prepare the bank for a new activation.}

\copied{A DRAM cell is inherently leaky and thus loses its stored electrical
charge over time. To maintain data integrity, a DRAM cell {is periodically
refreshed} with a {time interval called the \emph{\gls{trefw}}, which is
typically} \SI{64}{\milli\second} (e.g.,~\cite{jedec2012ddr3, jedec2020ddr4,
micron2014ddr4}) or \SI{32}{\milli\second} (e.g.,~\cite{jedec2015lpddr4,
jedecddr5c, jedec2020lpddr5}) at normal operating temperature (i.e., up to
\SI{85}{\celsius}) and half of it for the extended temperature range (i.e.,
above \SI{85}{\celsius} up to \SI{95}{\celsius}).  
To refresh all cells \omcr{2}{in a timely manner}, the memory controller
{periodically} issues a refresh {($REF$)} command with {a time interval called}
the \emph{\gls{trefi}}, {which is typically} \SI{7.8}{\micro\second}
(e.g.,~\cite{jedec2012ddr3, jedec2020ddr4, micron2014ddr4}) or
\SI{3.9}{\micro\second} (e.g.,~\cite{jedec2015lpddr4, jedecddr5c,
jedec2020lpddr5}) at normal operating temperature. When a rank-/bank-level
refresh command is issued, the DRAM chip internally refreshes several DRAM rows,
during which the whole rank/bank is busy.}
% This operation's latency is called the \emph{\gls{trfc}}.}

\subsection{DRAM Read Disturbance}

Read disturbance is the phenomenon \gra{in which} reading data from a memory or
storage device causes physical disturbance (e.g., voltage deviation, electron
injection, electron trapping) on another piece of data that is \emph{not}
accessed but physically located \gra{near} the accessed data. Two prime examples
of read disturbance in modern DRAM chips are RowHammer~\cite{kim2014flipping}
and RowPress~\cite{luo2023rowpress}, where repeatedly accessing (hammering) or
keeping active (pressing) a DRAM row induces bitflips in physically nearby DRAM
rows. In RowHammer and RowPress terminology, the row that is
hammered or pressed is called the \emph{aggressor} row, and the row that
experiences bitflips the \emph{victim} row. For read disturbance bitflips to
occur, 1)~the aggressor row needs to be activated more than a certain threshold
value, \param{which we call the read disturbance threshold (defined
in~\secref{sec:introduction})}\gra{,} and/or
2)~\omcr{2}{\acrfull{taggon}}~\cite{luo2023rowpress} needs to be large
enough~\cite{kim2020revisiting, orosa2021deeper, yaglikci2022understanding,
luo2023rowpress}. To avoid read disturbance bitflips, systems take preventive
actions, e.g., they refresh victim
rows~\refreshBasedRowHammerDefenseCitations{}, selectively throttle accesses to
aggressor rows~\cite{yaglikci2021blockhammer, greenfield2012throttling}, \gf{or}
physically isolate potential aggressor and victim rows~\cite{hassan2019crow,
konoth2018zebram, saileshwar2022randomized, saxena2022aqua, wi2023shadow,
woo2023scalable}. These solutions aim to perform preventive actions before the
cumulative effect of an aggressor row's \emph{activation count} and \emph{on
time} causes read disturbance bitflips.

\subsection{Motivation}

\iey{Read disturbance has significant implications for system
\omcr{2}{robustness (i.e.,} reliability, security, safety\omcr{2}{, and
availability}) because it is a widespread issue and can be exploited to break
memory isolation~\exploitingRowHammerAllCitations{}.} Therefore, it is important
to identify and understand read disturbance mechanisms in DRAM. Unfortunately,
despite the existing research efforts expended towards understanding read
disturbance~\understandingRowHammerAllCitations{}, scientific literature lacks a
detailed understanding of \omcr{2}{a key question that is critical for robustly
identifying and \atbcr{4}{mitigating} the read disturbance vulnerability of a
system:} \atbcr{2}{does the read disturbance threshold of a DRAM row change over
time? If so, how reliably and efficiently can it be measured?} Our goal in this
paper is to close this gap. We aim to empirically analyze how reliably and
efficiently the RDT of a victim DRAM row can be measured \omcr{2}{in modern DRAM
chips}.

% To avoid read disturbance bitflips in DRAM-based computing systems in an
% effective and efficient way, it is critical to rigorously gain detailed
% insights into the read disturbance. This is important because a better
% understanding of DRAM read disturbance could enable the development of
% comprehensive solutions to the problem more quickly. Unfortunately, despite
% the existing research efforts expended towards understanding read
% disturbance~\cite{XXX}, scientific literature lacks \param{VRDT blah
% blah....}}

% \iey{Our goal in this paper is to close this gap. We aim to empirically
% analyze \param{how reliably RDT of a victim row blah blah..}.Doing so provides
% us with a deeper understanding of the read disturbance in DRAM chips to enable
% future research on improving the effectiveness of existing and future
% solutions. We hope and expect that our analyses will pave the way for building
% robust (i.e., reliable, secure, and safe) systems that mitigate DRAM read
% disturbance at low performance, energy, and area overheads while DRAM chips
% become increasingly more vulnerable to read disturbance over generations.}

% \param{Cite everything on my desktop + what Giray cited in RH VPP paper}

% \atbcomment{move this to background}Two major error mechanisms lead to DRAM
% read disturbance, as described by prior works~\cite{X,Y,Z}: 1) electron
% injection / diffusion / drift, and 2) capacitive crosstalk. The electron
% injection / diffusion / drift mechanism creates transient charge leakage paths
% that reduce the voltage of the storage capacitor in a DRAM cell~\cite{Y}. The
% electron injection / diffusion / drift error mechanism is amplified by a
% larger voltage difference between a wordline and a DRAM cell or between two
% wordlines. The capacitive crosstalk mechanism exacerbates charge leakage paths
% in and around a DRAM cell’s capacitor~\cite{X} due to the parasitic
% capacitance between two wordlines or between 
% a wordline and a DRAM cell.

