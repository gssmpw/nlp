\section{In-Depth Analysis \gf{of VRD}}
\label{sec:indepth}

% We have established that the RDT of a DRAM row unpredictably changes with time. 
% To develop a reliable understanding of the temporal variation,
% which could guide researchers and engineers in searching for and developing
% better read disturbance solutions, it is important to
% comprehensively investigate how the temporal variation in RDT
% changes across\ieycomment{Would be good to add 1), 2),} DRAM rows in a bank, DRAM chip die densities and die revisions, data patterns, 
% aggressor row on time (\gls{taggon}) values, and temperatures. Therefore, we perform
% an in-depth analysis of
% the temporal variation in RDT across\ieycomment{it is too repetitive. maybe say changes across/under various parameters in the above sentence} rows, die densities and die revisions,
% data patterns, \gls{taggon}, and temperature.

\gf{In this section, we further enhance our analysis of the \emph{variable read
disturbance} (\phenomenon{}) \agy{phenomenon} by investigating parameters that
have been shown to impact RDT~\cite{luo2023rowpress,yaglikci2022understanding}.
Concretely, first, we analyze how \phenomenon{} changes across DRAM rows, die
densities, and die revisions. Second, we evaluate \phenomenon{} \omcr{2}{with}
different 1)~data patterns, 2)~\gls{taggon}, and 3)~temperature ranges.} Our
results show that 1) all tested DRAM rows exhibit \phenomenon{}, 2)
\omcr{2}{\phenomenon{} is worse in} higher-density DRAM chips and \omcr{4}{or
chips with} more advanced technology nodes, 3) data pattern affects
\phenomenon{} differently across tested DRAM chips, and 4) \phenomenon{} can
change with \gls{taggon} and temperature.

\noindent
\textbf{Test Parameters.} \atbcr{1}{We test 150 DRAM rows in each tested DDR4
chip. To select the 150 DRAM rows, we measure the RDT of each DRAM row in the
first, middle, and last 1024 DRAM rows in a DRAM bank in each tested DDR4 chip
10 times. From each of the first, middle, and last 1024 DRAM rows, we select 50
DRAM rows with the smallest mean RDT values across 10 measurements.}
\atbcr{1}{We test 150 DRAM rows from three HBM2 channels (50 randomly selected
DRAM rows from each channel) in four HBM2 chips.} We use the data patterns
listed in Table~\ref{table_data_patterns}. We use three \gls{taggon} values: 1)
the minimum $t_{RAS}$ timing parameter as defined in the DRAM standard, 2)
$t_{REFI}$, the average interval between two successive periodic refresh
commands, 3) $9\times{}t_{REFI}$, the maximum interval between two subsequent
periodic refresh commands (i.e., the maximum time a row can remain open
according to the DDR4~\cite{jedec2020ddr4} and HBM2
standards~\cite{jedec2021hbm}). We test DRAM chips at
\atbcr{1}{\SI{50}{\celsius}, \SI{65}{\celsius}, and \SI{80}{\celsius}} using
\omcr{2}{our} temperature controller setup
(\secref{subsec:testing_methodology}).

% \noindent
% \textbf{DRAM Testing Algorithm.}
% \agy{\algref{alg:test_temporal_long} shows our testing algorithm with two key
% differences. First, we test randomly selected DRAM rows instead of the row that
% \texttt{find\_victim} returns. Second, we measure each row's RDT 1,000 times
% instead of 100,000 times to complete our experiments in a reasonable time
% window.}

\noindent
\subsection{\phenomenon{} Across DRAM Rows}
\label{subsec:across-rows}
\figref{fig:rdt_cov_all} shows \agy{an S-curve of} the coefficient of variation
(CV)\footnote{Coefficient of variation is the standard deviation of all 1,000
RDT measurements normalized to the mean across all 1,000 RDT measurements.}
(y-axis) across all tested DRAM rows sorted in increasing CV (x-axis). We plot
the maximum observed CV for each tested DRAM row in every DRAM chip across all
combinations of test parameters (data pattern, \gls{taggon}, and temperature). A
higher CV indicates a larger variance around the mean RDT value across 1,000 RDT
measurements. \agy{\figref{fig:rdt_cov_example} shows the RDT measurement
results (y-axis) across 1,000 \atbcr{1}{measurements} (x-axis) for the two rows
that mark the P50 \atbcr{2}{(\atbcr{2}{$50^{th}$ percentile,} the middle point
\atbcr{2}{in the figure})} and P100 \atbcr{2}{(\atbcr{2}{$100^{th}$ percentile,}
the rightmost point, not marked in~\figref{fig:rdt_cov_all})} points in
\figref{fig:rdt_cov_all} on the left- and right-hand side,
respectively.}\omcrcomment{2}{Revise figure 7 captions and move them closer to
the figures}

\begin{figure}[!ht]
    \centering
    \begin{subfigure}[!h]{\linewidth}  
    \includegraphics[width=\linewidth]{figures/rdt_cov_kde.pdf}
    \vspace{-6mm}
    \caption{\atbcr{2}{{Variation in a
    row's} read disturbance threshold \atbcr{2}{values across 1,000
    measurements}}}
    \label{fig:rdt_cov_all}
    \end{subfigure}
    \begin{subfigure}[!h]{\linewidth}
    \includegraphics[width=\linewidth]{figures/rdt_cov_kde_example.pdf}
    \vspace{-6mm}
    \caption{\omcr{2}{Measured} read disturbance threshold \omcr{3}{values} of
    two rows \atbcr{2}{across 1,000 successive measurements}. P50 row with CV =
    \atbcr{1}{\param{0.03}} (left) and the \omcr{2}{P100} row with the greatest
    CV = \atbcr{1}{\param{0.52}} (right).}
    \label{fig:rdt_cov_example}   
    \end{subfigure}
    \caption{Temporal variation of RDT across DRAM rows}
    \label{fig:rdt_cov_both}
\end{figure}

\observation{All tested rows exhibit temporal RDT variation.}\label{obs:all-rows-vary}

All tested DRAM rows have non-zero CV in at least one combination of tested data
patterns, \gls{taggon} values, and temperatures. The maximum CV across all
tested rows is \atbcr{1}{\param{0.52}}. We observe that \param{50\%} of rows (to
the right of P50 on the x-axis) have greater than \param{\atbcr{1}{0.03}} CV. 
We observe that the read disturbance threshold varies from
\atbcr{1}{\param{1740}} to \atbcr{1}{\param{2040}} (by a factor of
\atbcr{1}{\param{1.2$\times{}$}}) and from \atbcr{1}{\param{3242}} to
\atbcr{1}{\param{11498}} (by a factor of \atbcr{1}{\param{3.5$\times{}$}}) for
the DRAM row on the left subplot and the DRAM row on the right subplot,
respectively.

\observation{A large fraction (\atbcr{1}{\param{97.1\%}}) of tested DRAM rows
exhibit temporal variation across all test parameters.}

For \omcr{2}{97.1\% of the tested} DRAM rows, under \emph{all} combinations of
test parameters (data pattern, \gls{taggon}, and temperature), the measured RDT
values vary across 1,000 measurements (not shown in~\figref{fig:rdt_cov_both}).
In contrast, the RDT values do \emph{not} change with 1,000 repeated
measurements for \atbcr{1}{\param{2.9\%}} of the tested DRAM rows under at least
one combination of test parameters. However, at least one test parameter
combination yields multiple RDT values across 1,000 measurements for such DRAM
rows.

\noindent
\textbf{Probability of Identifying the Minimum RDT.}
Based on our analyses in~\secref{sec:foundational_results}, measuring the RDT of
a DRAM \atbcr{2}{row} is similar to sampling from a probability distribution.
Thus, we analyze 1) the probability of identifying the minimum RDT value
\atbcr{2}{of a DRAM row} across 1,000 measurements with \omcr{2}{N <} 1,000
measurements, which we call the \emph{probability of finding the minimum RDT}
\omcr{2}{with N measurements}
and 2) the expected value of the minimum RDT across \omcr{2}{N <} 1,000
measurements normalized to the minimum RDT across 1,000 measurements, which we
\omcrcomment{2}{find a better name or a better metric}call the \emph{expected
normalized value of the minimum RDT} \omcr{2}{across N measurements}. To do so,
we run Monte Carlo simulations for \param{10,000} iterations. In each iteration,
we uniform\omcr{2}{ly} randomly select \emph{N} RDT measurements from the series
of 1,000 measurements
%\nbcomment{is this 1k or 10k? why 10k Monte Carlo simulations if this is 1k?
%confusing. also how do you select 1k out of 10k simulations? randomly? is this
%not a step?}
for each tested DRAM row. We repeat the simulations for N~=~1, 3, 5, 10, 50, and
500, and for each combination of test parameters, to demonstrate how the
probability of finding the minimum RDT and the expected normalized value of the
minimum RDT changes with \atb{the number of} RDT measurements. 

\figref{fig:rdt_ratio_all}\omcrcomment{2}{add N to figure} shows 1) the
distribution of the probability of finding the minimum RDT \atb{across tested
rows} (top) and 2) the distribution of the expected normalized value of the
minimum RDT \atb{across tested rows} (middle) for number of measurements
\atbcr{2}{(N)} in \gra{the} range [1, 500] on the x-axis, and 3) the expected
normalized value of the minimum RDT over the probability of finding the minimum
RDT (bottom) \atbcr{3}{for N~=~1, 50, and 500} \omcr{4}{(we plot only three
values to ease
readability)}.\footnote{\ext{\figref{fig:big_prob_and_expected_value} shows a
larger version of the bottom plot in \figref{fig:rdt_ratio_all} with N~=~1, 3,
5, 10, 50, and 500.}} Each box in the top and middle figures shows the
distribution across all tested DRAM rows and all combinations of test
parameters. We consider \phenomenon{} to be worse for DRAM rows that exhibit a
\emph{smaller} probability of finding the minimum RDT and a \emph{greater}
expected normalized value of the minimum RDT. \atbcr{2}{For example, a y = 1.5
for the expected normalized value of the minimum RDT means that, with N
measurements (depicted on the x-axis in the top and middle subplots
in~\figref{fig:rdt_ratio_all}), we expect to find an RDT that is 50\% higher
than the minimum value we would find if we \omcr{4}{performed} 1,000 RDT
measurements.} \atb{The DRAM rows that are closer to the top left corner in the
bottom plot indicate the DRAM rows that exhibit the worst \phenomenon{}
behavior.} 

\begin{figure}[!ht]
    \centering
    \includegraphics[width=\linewidth]{figures/probability_of_finding_minimum_rdt_and_its_expected_value.png}
    \caption{Probability of finding the minimum RDT with \omcr{2}{N <} 1,000
    measurements (top); the expected value of the minimum RDT found with
    \omcr{2}{N <} 1,000 measurements normalized to the minimum RDT across 1,000
    measurements (middle); the expected normalized value of the minimum RDT over
    the probability of finding the minimum RDT (bottom)}
    \label{fig:rdt_ratio_all}
\end{figure}

\observation{It is \omcr{2}{very} unlikely to find the minimum RDT of a DRAM row
with \omcr{2}{N~=~1} RDT measurement.}

One RDT measurement for the median DRAM row (P50) has a \atbcr{1}{\param{0.2\%}}
probability of yielding the minimum RDT across 1,000 measurements
\atb{(\figref{fig:rdt_ratio_all} top)}. For \param{\atbcr{1}{22.4\%}} of the
tested DRAM rows, the probability of finding the minimum RDT \atbcr{2}{among
1,000 measurements using only a single measurement} is
\atbcr{2}{\param{$\leq{}\!$0.1\%}.}

\observation{The value of the minimum RDT \atbcr{2}{across 1,000 measurements}
is significantly smaller than the one \atbcr{2}{expected to be} found with
\omcr{2}{N~=~1} RDT measurement.\label{finding:expected_minimum_rdt}}

To make matters worse, a DRAM row whose RDT is unlikely to be identified with a
single measurement can also exhibit a very large variation in its RDT \atb{(top
left corner of \figref{fig:rdt_ratio_all} bottom)}. \atbcr{2}{T}he expected
normalized value of the minimum RDT for those DRAM rows \atbcr{1}{with} a
\emph{low} (\param{$\leq{}\!$0.1\%}) probability of finding the minimum RDT can
be as high as \param{1.9$\times{}$} \atbcr{2}{the minimum RDT across 1,000
measurements} (\param{1.1}$\times{}$ on average \atbcr{2}{across all rows
\atbcr{1}{with} \param{$\leq{}\!$0.1\%} probability of finding the minimum
RDT}). In contrast, \emph{only} \param{\atbcr{1}{5.4\%}} of the tested DRAM rows
exhibit a high probability of finding the minimum RDT \omcr{2}{(i.e.,}
\atbcr{2}{$\geq{}\!$\param{99.9}\%) using a single measurement.} For those DRAM
rows with \nb{a} high probability of finding the minimum RDT \omcr{2}{using
\omcr{2}{a single} measurement}, the expected normalized value of the minimum
RDT is \atbcr{1}{relatively small:} \atbcr{2}{at most}
\atbcr{2}{\param{$1.001$$\times{}$}}. 

\observation{The probability of finding the minimum
 RDT of a DRAM row increases with the number of RDT measurements.}

\atbcr{2}{With N~=~1, 3, 5, 10, 50, and 500 measurements, t}he median DRAM row
(P50) has \atbcr{1}{\param{0.2\%}}, \atbcr{1}{\param{0.7\%}},
\atbcr{1}{\param{1.1\%}}, \atbcr{1}{\param{2.1\%}}, \atbcr{1}{\param{10.0\%}},
and \atbcr{1}{\param{75.3\%}} probability of finding the minimum RDT across
1,000 measurements, respectively. We observe that even at a relatively high
number of \omcr{2}{N =} 500 RDT measurements, a DRAM row may exhibit a
relatively low probability of finding the minimum RDT of approximately
\param{50.0\%} \atbcr{2}{(bottom tail of the rightmost box in
\figref{fig:rdt_ratio_all} top)}. \atbcr{2}{For such a DRAM row,} \emph{only} 1
out of 1,000 measurements yield\atbcr{2}{s} the minimum RDT value.
\atbcrcomment{2}{This is correct: you randomly select 500 measurements. If only
one is minimum, you get 50\% chance of including that in your selection (as
number of selections go to infinity).}

We conclude that all tested DRAM rows exhibit temporal variation in
RDT, and the minimum RDT (across 1,000 measurements) of the majority of DRAM
rows \emph{cannot} be found with a high probability (e.g., \param{$>$90\%})
using \atbcr{1}{500 measurements}\atbcrcomment{2}{dropped fewer than or equal
to}. A DRAM row's expected RDT value obtained using a single measurement can be
\param{1.9}$\times{}$ the minimum RDT \atbcr{1}{observed for that} row across
1,000 measurements, and this minimum RDT value may appear only \emph{once}
across the series of 1,000 measurements.

\take{Relatively few \omcr{2}{(e.g., N < 500)} RDT measurements are unlikely to
identify the minimum RDT value of a DRAM row. Estimations for the minimum RDT of
a DRAM row can become more accurate with repeated RDT measurements \omcr{2}{but
RDT \emph{cannot} be found \omcr{2}{easily (even after N~=~500 measurements)}
with a high probability}.\label{take:rdt_need_more_measurements}}

\subsection{Effect of Die Density and Die Revision}

To understand if and how DRAM \omcr{2}{technology} scaling affects
\phenomenon{}, we investigate how \atbcr{2}{the distribution of observed RDT
values changes} with the die density and the die revision of the tested DRAM
chips. \figref{fig:rdt_die_density_revision} shows the distribution (across
\omcr{2}{150} tested rows \omcr{2}{per module}) of \atbcr{1}{the expected
normalized value of the minimum RDT} for \atbcr{1}{varying} number of
measurements \atbcr{2}{(N)} in \gra{the} range [1, 500] on the x-axis.
Each subplot shows the distribution for a different DRAM chip manufacturer
\atbcr{4}{in a box-and-whiskers plot}.\footref{footnote:box-whiskers}
% and the bottom subplots shows the distribution across all tested HBM
% chips.\footnote{We show the distributions for the tested HBM chips in ax
% separate subplot as we do \emph{not} have access to reliable open information
% about their manufacturing technology.} 
Different boxes show the distribution for one combination of the die density and
die revision\footnote{For a given manufacturer and die density, the later in the
alphabetical order the die revision code is, the more likely the chip has a more
advanced technology
node.}\addtocounter{footnote}{-1}\addtocounter{Hfootnote}{-1} of tested DRAM
chips. \atbcr{2}{We refer to the range of expected normalized value of the
minimum RDT distribution (as depicted by each box in the figure) as the
\emph{VRD profile} of a DRAM chip.} \atbcr{2}{A higher box in the figure depicts
a "worse" VRD profile: \omcr{3}{if we \omcr{4}{perform} only N measurements, we
will likely be farther off from the minimum RDT we would find if we had
\omcr{4}{performed} 1,000 measurements.}}
% {The distribution (across all tested rows) of RDT values
% that we expect to find using a given number of measurements (x-axis) becomes
% higher relative to the minimum RDT value across 1,000 measurements.} 


\begin{figure}[!ht]
    \centering
    \includegraphics[width=\linewidth]{figures/probability_and_scaling.pdf}
    \caption{\atbcr{1}{Expected normalized value of the minimum RDT
    \omcr{3}{after N measurements}} across DDR4 chip die densities and die
    revisions}
    \label{fig:rdt_die_density_revision}
\end{figure}

\vspace{2mm}
\observation{\phenomenon{} \atbcr{2}{profile} varies across tested DRAM chips.}

\omcrcomment{2}{Does this support finding 10, very hard to follow}Across tested
\hpcareve{Mfr. M}, \hpcareve{Mfr. S}, and \hpcareve{Mfr. H} chips, the median
DRAM row \atbcr{3}{(and the worst-case DRAM row, not shown in the figure)} has
\gra{a} \atbcr{2}{\atbcr{1}{\param{1.08$\times{}$} \atbcr{3}{(1.84$\times{}$)},
\param{1.05$\times{}$} \atbcr{3}{(3.21$\times{}$)}, and \param{1.05$\times{}$}}}
\atbcr{3}{(1.70$\times{}$)} \atbcr{1}{expected normalized value of the minimum}
RDT \atbcr{1}{for} \atbcr{2}{N~=~1} RDT measurement, respectively.
\atbcr{3}{This means that with \atbcr{5}{one} measurement \omcr{4}{only}, we
expect to find an RDT that is \omcr{4}{3.21$\times{}$} the minimum value we
would find if we \omcr{5}{had} \omcr{4}{performed} 1,000 RDT measurements for
the \emph{worst-case row} from all tested chips.}

% Across all tested HBM chips,
% the median DRAM row's probability
% of finding the minimum RDT (with one measurement) varies 
% from \param{14.3\%} (Chip1) to \param{22.6\%} (Chip3). 

\observation{\phenomenon{} \atbcr{2}{profile worsens} with increasing die
density and with advanced DRAM technology.} 
\label{finding:worsen-with-technology}

\omcrcomment{2}{Hard to follow and map to figure}In general, the higher the
density of the DRAM chip or the more advanced the technology node (as indicated
by the die revision)\footnotemark, \atbcr{2}{the worse the \phenomenon{}
\omcr{3}{profile}}. \atbcr{2}{For example,} the \atbcr{1}{expected normalized
value of the} minimum RDT \atbcr{2}{using} \omcr{2}{N~=~1} measurement for the
median DRAM row \atbcr{3}{(and for the worst-case row, not shown in the figure)}
\atbcr{2}{increases} to \param{\atbcr{2}{1.08$\times{}$}}
\atbcr{3}{(1.78$\times{}$)} from \atbcr{2}{\param{1.06$\times{}$}
\atbcr{3}{(1.45$\times{}$)} for \hpcareve{Mfr. M}'s chips} as the chip density
increases and the technology node advances. We observe similar trends for all
\atbcr{2}{manufacturers and} tested \omcr{3}{values of N}.

We conclude that different DRAM chips experience different \phenomenon{}
\omcr{3}{profiles}. \atbcr{2}{\omcr{3}{The} VRD \omcr{3}{profile of a chip
gets} worse} in higher-density chips or \omcr{2}{chips with} more advanced
technology nodes. 

\subsection{Effect of Data Pattern}

We analyze how \atbcr{2}{the} \phenomenon{} \atbcr{2}{profile\omcr{3}{s} of the
tested chips} change with data patterns used \omcr{3}{to} initialize aggressor
and victim DRAM rows. \figref{fig:rdt_data_pattern} shows the distribution
(across tested rows) of the \atbcr{1}{expected normalized value of} the minimum
RDT for \atbcr{1}{varying} \nb{number of measurements} \atbcr{2}{(N)} listed on
the x-axis \atbcr{4}{in a box-and-whiskers plot}.\footref{footnote:box-whiskers}
Each subplot is for a different DRAM chip manufacturer (and the bottom subplot
is for the tested HBM2 chips)\gra{,} and each box shows the distribution of the
\atbcr{1}{expected} values for a different data pattern.

\begin{figure}[!ht]
    \centering
    \includegraphics[width=\linewidth]{figures/probability_and_datapattern.pdf}
    \caption{\atbcr{1}{Expected normalized value of} the minimum RDT
    \omcr{3}{after N measurements} across four tested data patterns}
    \label{fig:rdt_data_pattern}
\end{figure}

\observation{\phenomenon{} \atbcr{2}{profile} \omcr{3}{of a DRAM chip} changes
with data pattern.}

For example, the median DRAM row \atbcr{3}{(and the worst-case row, not shown in
the figure)} in an \hpcareve{Mfr. \atbcr{3}{H}} DRAM chip has \nb{an}
\atbcr{1}{expected normalized value of the} minimum RDT \atbcr{2}{using N~=~1
measurement} ranging from \param{\atbcr{2}{1.04$\times{}$}}
\atbcr{3}{(1.57$\times{}$)} to \param{\atbcr{2}{1.06$\times{}$}}
\atbcr{3}{(1.70$\times{}$)} \atbcr{2}{for} different data patterns. We observe
that the data pattern affects the \atbcr{2}{VRD profile} in all DDR4 chips from
all manufacturers (and in HBM2 chips) for all tested number\gra{s} of
measurement values.

\observation{No single data pattern causes the worst \phenomenon{}
\atbcr{2}{profile} across all tested DRAM chips.}

The data pattern that \atbcr{1}{yields the largest} \atbcr{1}{expected
normalized value of} the minimum RDT with \atbcr{2}{N~=~1} RDT measurement is
Checkered0, \atbcr{3}{Rowstripe1}, \atbcr{1}{Rowstripe\atbcr{3}{0}, and
\atbcr{3}{Checkered1}} \atbcr{3}{for the median row}
across\atbcrcomment{3}{median row when you aggregate all tested rows from all
chips} all \atbcr{3}{tested} DRAM chips from \hpcareve{Mfr. M}, \hpcareve{Mfr.
S}, \hpcareve{Mfr. S HBM2}, and \hpcareve{Mfr. H}, respectively.

% We conclude that \phenomenon{} changes with data patterns and no single data pattern
% can identify the worst-case \phenomenon{} behavior across all tested DRAM chips.

\take{\omcr{5}{How the lowest RDT varies over time depends on the data
pattern.}\label{take:data_pattern}}

\subsection{Effect of Aggressor Row On Time}

We investigate the sensitivity of \phenomenon{} to the amount of time an
aggressor row is kept open (\gls{taggon}). \figref{fig:rdt_taggon} shows the
distribution (across tested rows) of the \atbcr{1}{expected normalized value of}
the minimum RDT for \atbcr{3}{varying} number of \nb{measurements}
\atbcr{2}{(N)} listed on the x-axis \atbcr{4}{in a box-and-whiskers
plot}.\footref{footnote:box-whiskers} Each subplot is for a different DRAM chip
manufacturer (and the bottom subplot is for the tested HBM2 chips)\nb{,} and
each box shows the distribution of the \atbcr{1}{expected} values for a
\gls{taggon}.

\begin{figure}[!ht]
    \centering
    \includegraphics[width=\linewidth]{figures/probability_and_taggon_three_examples.pdf}
    \caption{\atbcr{1}{Expected normalized value of} the minimum RDT
    \atbcr{3}{after N measurements} across three tested \gls{taggon} values}
    \label{fig:rdt_taggon}
\end{figure}


\observation{\phenomenon{} \atbcr{2}{profile} \atbcr{1}{changes} with
\gls{taggon}.}

For example, the median DRAM row \atbcr{3}{(and the worst-case DRAM row, not
 shown in the figure)} across all rows in all tested \hpcareve{Mfr. H} chips has
 \atbcr{2}{\param{1.05\atbcr{3}{4}}$\times{}$ \atbcr{3}{(1.602$\times{}$)},
 \param{1.0\atbcr{3}{49}}$\times{}$ \atbcr{3}{(1.597$\times{}$)}, and
 \param{1.0\atbcr{3}{46}}$\times{}$ \atbcr{3}{(1.467$\times{}$)} expected
 normalized value of} the minimum RDT \atbcr{2}{using N~=~1} RDT measurement for
 \gls{taggon} values of minimum $t_{RAS}$ (approximately \SI{35}{\nano\second}),
 $t_{REFI}$ (\SI{7.8}{\micro\second} in DDR4~\cite{jedec2020ddr4}), and
 $9\times{}t_{REFI}$ (\SI{70.2}{\micro\second} in DDR4~\cite{jedec2020ddr4}),
 respectively.

\observation{\phenomenon{} \atbcr{2}{profile can become better or worse as}
\gls{taggon} \atbcr{2}{increases}.}

For example, the tested \hpcareve{Mfr. \atbcr{3}{M}} \atbcr{3}{and Mfr. H} DRAM
chips display a \atbcr{3}{decreasing} \atbcr{1}{expected normalized value of}
the minimum RDT as \gls{taggon} increases.
% , the tested \hpcareve{Mfr. M} DRAM
% chips display a decreasing trend for the \atbcr{1}{expected value} with
% increasing \gls{taggon}. 
For the tested \hpcareve{Mfr. S} chips, the
\atbcr{2}{median} \atbcr{1}{expected normalized value of} the minimum RDT
\atbcr{2}{across all tested rows} is \atbcr{1}{lower} at \gls{taggon} =
$t_{REFI}$ and \atbcr{1}{higher} at \gls{taggon} = minimum $t_{RAS}$ and
$9\times{}t_{REFI}$.

\subsection{\atbcr{3}{Effect of Temperature}}

\figref{fig:rdt_temperature} shows the distribution of the \atbcr{1}{expected
normalized value of} the minimum RDT with \emph{\atbcr{1}{one} RDT measurement}
for \atbcr{3}{six} selected example DRAM chips\gra{,} \omcr{2}{\atbcr{3}{two}}
from \hpcareve{Mfr. M} (top), \omcr{2}{\atbcr{3}{two} from}
\hpcareve{Mfr. S} (\atbcr{3}{middle})\atbcr{3}{, and two from Mfr. H
(bottom)} using the Rowstripe1 data pattern and for \gls{taggon} =
minimum $t_{RAS}$ \atbcr{4}{in a box-and-whiskers
plot}.\footref{footnote:box-whiskers} Different boxes show the distribution of
the \atbcr{1}{expected} values for different temperatures.\omcrcomment{2}{Why
figure different from others?}\atbcrcomment{2}{We don't have multiple
temperature points for HBM2. I could try to include Mfr. H somehow if you think
it is necessary. I need to look at the data.}

\begin{figure}[!ht]
    \centering
    \includegraphics[width=0.98\linewidth]{figures/probability_and_temperature_two_examples_samplesize1.pdf}
    \caption{\atbcr{1}{Expected normalized value of} the minimum RDT with
    \atbcr{1}{one} RDT measurement for the Rowstripe1 data pattern at
    \gls{taggon}~=~minimum $t_{RAS}$}
    \label{fig:rdt_temperature}
\end{figure}

\observation{\phenomenon{} \atbcr{2}{profile} \omcr{3}{of a DRAM chip}
\atbcr{1}{changes} with temperature.}

\atbcr{4}{A}s temperature increases from \atbcr{1}{\SI{50}{\celsius}} to
\SI{80}{\celsius}, the \atbcr{1}{expected value of the} minimum RDT for the
median DRAM row \atbcr{3}{(and for the worst-case row, not shown in the figure)}
in an \hpcareve{Mfr. M} 16Gb-E die chip \atbcr{1}{increases} from
\atbcr{2}{\param{1.06}$\times{}$ \atbcr{3}{(1.22$\times{}$)} to
\param{1.07}$\times{}$} \atbcr{3}{(1.29$\times{}$)}. Our finding that
\phenomenon{} \atbcr{2}{profile} change\atbcr{1}{s} with temperature is
consistent across all tested \gls{taggon} values and data patterns.

We conclude that aggressor row on time and temperature \omcr{2}{both affect
\phenomenon{}}. We do \emph{not} identify any prominent correlation between
\phenomenon{}, \gls{taggon}, and temperature from our empirical data.

\take{Temperature and \gls{taggon} \atbcr{2}{affect \phenomenon{}}. The
\phenomenon{} profile at one temperature level and one \gls{taggon} value likely
would \emph{not} resemble the \phenomenon{} profile across all operating
temperature\atbcr{5}{s} and \gls{taggon} values.
\label{take:taggon_temperature}}

\subsection{\hpcarevc{Effect of True- and Anti-Cell Layout}}

\hpcarevc{We\hpcalabel{Main Question 4} study \phenomenon{}\atbcr{2}{'s
sensitivity} to DRAM cell data \emph{encoding conventions}~\cite{patel2020beer,
patel2021harp, patel2019understanding,kim2014flipping,kraft2018improving,
liu2013experimental} (i.e., \emph{true-cell} and \emph{anti-cell}) used in the
victim row. Each DRAM cell in a DRAM chip may store data using two encoding
conventions\omcr{2}{: 1)~}a true-cell encodes a \emph{``logic-1''} as a
fully-charged capacitor\omcr{2}{, or 2)~}an anti-cell encodes a \emph{``logic-1''} as a
fully-discharged capacitor. We experimentally measure the layout of true- and
anti-cells throughout 50 randomly selected victim DRAM rows in module M0
using the methodology described in prior
works~\cite{patel2019understanding,kim2014flipping,kraft2018improving}.
Figure~\ref{fig:rdt_layout} shows the distribution of the coefficient of
variation (y-axis) across 1,000 RDT measurements for each of the tested 20 DRAM
rows with anti-cells (left box) and 30 DRAM rows with true-cells (right box).
The figure shows the distribution for all tested data patterns (left subplot),
temperature levels (middle subplot), and aggressor row on times (right
subplot) \atbcr{4}{in a box-and-whiskers plot}.\footref{footnote:box-whiskers}}

\begin{figure}[!ht]
    \centering
    \includegraphics[width=1.0\linewidth]{figures/rdt_cov_cell_layout.pdf}
    \caption{\hpcarevc{Coefficient of variation across 1,000 RDT 
    measurements on 20 DRAM rows with anti-cells (left box) 
    and 30 DRAM rows with true-cells (right box) for different
    data patterns (left subplot), temperature levels (middle subplot), 
    and aggressor row on times (right subplot)}}
    \label{fig:rdt_layout}
\end{figure}

\observation{\hpcarevc{The presence of true- and anti-cells in the victim row
\atbcr{2}{does \emph{not} significantly affect the RDT distribution} in one
tested module (M0).}}

% First, we identify (using \texttt{find\_victim} in Algorithm~\ref{alg:test_temporal_long}) 
% the victim row  
% to be extensively tested 
% (whose RDT is to be measured for 100,000 times). 
% We do so by looking for a victim 
% row that is \emph{relatively more vulnerable} to read disturbance; a row that exhibits
% a read disturbance bitflip at a relatively small (smaller than 40,000) hammer count
% and minimum \gls{taggon} ($tRAS$, $\approx$\SI{35}{\nano\second})
% on average across ten successive measurements using the
% Checkered0 data pattern.\footnote{To maintain
% a reasonable experiment time, we carefully select the testing parameters (e.g., we test one relatively 
% more read-distrubance-vulnerable victim DRAM row) used in our RDT testing experiments in this section. 
% \secref{sec:more-characterization} shows more extensive results from with
% a wider range of testing parameters (e.g., more
% DRAM rows, data patterns, \gls{taggon}, and temperature).} Second, we repeatedly measure
% (using \texttt{test\_loop} in Algorithm~\ref{alg:test_temporal_long})
% the RDT of the identified victim row. This experiment yields 100,000 successive 
% RDT measurements for each tested row (one such row in each tested DDR4 module and HBM2 chip). 