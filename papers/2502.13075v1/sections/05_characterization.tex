% \section{\agy{Variable Read Disturbance Threshold (VRDT)}}
\section{Foundational Results}
\label{sec:foundational_results}
\agy{We investigate the variation in read disturbance threshold (RDT) across
repeated tests. We measure RDT in all tested DDR4 and HBM2 DRAM chips 100,000
times.} Our \agy{experimental} results demonstrate that the RDT of a DRAM row
changes \emph{unpredictably} \atb{across repeated RDT
measurements}.\agyurgentcomment{with time is a bit vague. It is actually across
iterations of the same test, no?}

\noindent
\textbf{DRAM Testing Algorithm.} 
\agy{\algref{alg:test_temporal_long} shows our test routine in two steps.}
% The DRAM Bender program used in this 
% experiment is described by Algorithm~\ref{alg:test_temporal_long} 
% and it works in two main steps. 
First, \agy{\texttt{find\_victim} (line~1 in \algref{alg:test_temporal_long}) identifies a DRAM row that is relatively more vulnerable to read disturbance to be the subject of our extensive tests}\agy{.}
% the victim row  
% to be extensively tested 
% (whose RDT is to be measured for 100,000 times). 
% We do so by looking for a victim 
% row that is \emph{relatively more vulnerable} to read disturbance; 
\agy{To do so, we choose a row that exhibits RDT values below 40,000 for the minimum \gls{taggon} (e.g., \SI{35}{\nano\second})}
% a row that exhibits
% a read disturbance bitflip at a relatively small (smaller than 40,000) hammer count
% and minimum \gls{taggon} ($tRAS$, $\approx$\SI{35}{\nano\second})
on average across ten successive measurements using the
Checkered0 data pattern.\footnote{To maintain a reasonable experiment time, we
select one combination of test parameters (e.g., we test one relatively more
read-disturbance-vulnerable victim DRAM row) used in our RDT testing experiments
in this section. \secref{sec:indepth} shows more extensive results with a wider
range of test parameters (e.g., more DRAM rows, data patterns, \gls{taggon}
values, and temperatures).} The first step yields a \agy{\emph{guessed RDT
value} ($RDT_{guess}$)} for the tested DRAM row, as the row's mean RDT value
across 10 RDT measurements. Second, \agy{\texttt{test\_loop} (line \atbcr{4}{12} in
\algref{alg:test_temporal_long})} repeatedly measures
% (using \texttt{test\_loop} in Algorithm~\ref{alg:test_temporal_long})
the RDT of the identified victim row. We do so by testing the DRAM row for read
disturbance failures using hammer counts ranging from $RDT_{guess}/2$ to
$RDT_{guess}*3$ with increments of $RDT_{guess}/100$.\footnote{We
\agy{empirically} determine the range and the granularity of the tested hammer
count values such that our experiments take reasonable time and cover a wide
range of RDTs that the DRAM row may exhibit.} This experiment yields a series of
100,000  
RDT measurements for each tested row. \atbcr{2}{To draw foundational results for
\phenomenon{}, we perform 100,000 RDT measurements using one DRAM row in each
tested DDR4 module and HBM2 chip. \secref{sec:indepth} shows more extensive
results using many DRAM rows, data patterns, \gls{taggon} values, and
temperatures.}\omcrcomment{3}{double check to make sure no errors}


\SetAlFnt{\footnotesize}
\RestyleAlgo{ruled}
% \vspace{-1em}
\vspace{-2pt}
\begin{algorithm}
\caption{Test for profiling the temporal variation of read disturbance in DRAM}\label{alg:test_temporal_long}
  \DontPrintSemicolon
  \SetKwFunction{FVPP}{set\_vpp}
  \SetKwFunction{FMain}{test\_loop}
  \SetKwFunction{FHammer}{measure\_$RDT$}
  \SetKwFunction{initialize}{initialize\_rows}
  \SetKwFunction{measureber}{measure\_BER}
  \SetKwFunction{FMeasureHCfirst}{measure\_\gls{hcfirst}}
  \SetKwFunction{compare}{compare\_data}
  \SetKwFunction{guessRDT}{guess\_RDT}
  \SetKwFunction{findrow}{find\_victim}
  \SetKwFunction{Hammer}{hammer\_doublesided}
  \SetKwFunction{Gaggressors}{get\_aggressors}
  \SetKwFunction{FWCDP}{get\_WCDP}
  \SetKwProg{Fn}{Function}{:}{}
  
  
  \tcp{$RA_{victim}$: Victim row address}
  \tcp{$HC$: Hammer count, activations per aggressor row}
%   \tcp{$ACT$: Row activation command to open a DRAM row}
%   \tcp{$PRE$: Precharge command to close a DRAM row}
%   \tcp{$WAIT$: Wait for the specified amount of time}
  \tcp{$VDP$: Victim row's data pattern}
  \tcp{\omcr{5}{$LRA$}: The \omcr{5}{largest} row address in the tested chip}
  \tcp{$t_{AggOn}$: Aggressor row on time}
%   \tcp{$t_{RP}$: Precharge latency timing constraint}

  \Fn{\findrow{$VDP$, $t_{AggOn}$}}{
        \ForEach{$RA\atbcr{2}{_{victim}}$ in range($0$, $LRA$)}{
            // Guess RDT of the victim row\;
            // by repeatedly measuring it for 10 times\;
            // the guess is the mean RDT across all 10 measurements\;
            $RDT_{guess}$ = \guessRDT($RA\atbcr{2}{_{victim}}$, $VDP$, $t_{AggOn}$)\;
            \If{$RDT_{guess} < 40,000$}{\KwRet $RDT_{guess}, RA\atbcr{2}{_{victim}}$}
            
        }
  }\;


  
  % \Fn{\Hammer{{$RA_{victim}$}, $HC$, $t_{AggOn}$}}{
  %       \While{$i < HC$}{
  %           ACT({$RA_{victim}+1$}); WAIT({$t_{AggOn}$});\;
  %           PRE({}); WAIT({$t_{RP}$});\;
  %           ACT({$RA_{victim}-1$}); WAIT({$t_{AggOn}$});\;
  %           PRE({}); WAIT({$t_{RP}$});\;
  %           i++\;
  %       }
  % }\;

  
  \Fn{\FMain{}}{
        $RDT_{guess}, RA_{victim}$ = \findrow($VDP$, $t_{AggOn}$)\;
        $RDT_{min} = RDT_{guess} / 2$\;
        $RDT_{max} = RDT_{guess} * 3$\; 
        $RDT_{step} = RDT_{guess}/100$\;
        \ForEach{$measurement\_number$ in range($0$, $100,000$)}{
            \ForEach{$RDT$ in range($RDT_{min}$, $RDT_{max}$, $RDT_{step}$)}{
                \initialize($RA_{victim}$, $VDP$)\;
                \Hammer({$RA_{victim}$}, $RDT$, $t_{AggOn}$)\;
                $bitflip =$ \compare($RA_{victim}$, $VDP$)\;
                \If{$bitflip$}{$write$ $RDT$ $to$ $storage$\; $break$\;}
            }
        }
  }
\end{algorithm}
 
\noindent
\textbf{Results.} \figref{fig:rdt_extended_boxplot} shows the distribution of
all values in the series of 100,000 measured RDT values (y-axis) for each tested
DDR4 module and HBM2 chip (x-axis) in a box-and-whiskers plot.\footnote{{{The
box is lower-bounded by the first quartile (i.e., the median of the first half
of the ordered set of data points) and upper-bounded by the third quartile
(i.e., the median of the second half of the ordered set of data points). The
\gls{iqr} is the distance between the first and third quartiles (i.e., box
size). Whiskers show the minimum and maximum values. The circles show the mean
of all data points.\label{footnote:box-whiskers}}}} Each point in a box is the
outcome of one RDT measurement.
% i.e., different points indicate RDT measurements
% performed at different times.

\begin{figure}[!ht]
    \centering
    \includegraphics[width=\linewidth]{figures/rdt_extended_boxplot.png}
    \caption{RDT distribution \omcr{2}{of a single victim row in each}
    tested module and chip}
    \label{fig:rdt_extended_boxplot}
\end{figure}


\begin{figure*}[!th]
    \centering
    \includegraphics[width=\linewidth]{figures/rdt_extended_histogram.pdf}
    \vspace{-5mm}
    \caption{Histogram of RDT values for \omcr{2}{a single victim row in} each
    tested module and chip. We indicate the number of unique measured RDT values
    as the number of bins. \atbcr{2}{Each bin in a subplot has \atbcr{5}{an}
    equal width. Bin size is computed as the range of RDT values (max - min) on
    the x-axis \omcr{3}{divided by} the number of unique measured RDT values.}}
    \label{fig:rdt_extended_histogram}
\end{figure*}


\observation{\agy{A DRAM row's RDT changes \omcr{4}{over} time.}}
% Read disturbance threshold (RDT) of a row changes with time.}%\ieycomment{You say time but it is not explicitly written or explained in the text nor in the figure.}\atbcomment{See above paragraph again now}

We \agy{observe} that the RDTs of all tested DRAM rows \agy{significantly}
change \agy{across \atb{repeated measurements}.} For example, for the victim row
in Chip0, the largest measured RDT is \param{1.21}$\times{}$ the smallest
measured RDT, across 100,000 measurements.

To better depict how RDT varies \omcr{4}{over} time,
\figref{fig:rdt_extended_histogram}\atbcrcomment{3}{we divide by number of
unique rdt values} shows the histogram of the measured RDT values \omcr{2}{of
\omcr{2}{one selected} victim row \omcr{2}{in}} each tested module and chip.

\observation{The RDT of a row has multiple states.}

We make three key observations from~\figref{fig:rdt_extended_histogram}. First,
the RDT of a row takes \agy{various} different values across 100,000
measurements, i.e., the RDT of a row has multiple states. For example, we
measure 21 unique RDT values across 100,000 measurements on DDR4 module M1.
Second, for the majority of tested DRAM rows \param{(13 out of 14)}, the
measured RDT values are accumulated around a mean RDT value. Third, we observe
that the RDT values in \omcr{2}{HBM} Chip1 \agy{follow a bimodal
distribution\gra{,} unlike other tested chips.}

We analyze how long a DRAM row retains the \param{same RDT value across
subsequent measurements}. \figref{fig:consecutive_repeating_measurements} shows
a histogram of \agy{the number of consecutive measurements across which a DRAM
row exhibits the same RDT value, aggregated across all \omcr{2}{14} tested
rows.}
% how long (in terms of the number of RDT measurements) the RDT of a row retains
% the same value, aggregated across all tested rows. 
The x-axis shows the number of consecutive RDT measurements yielding the same
RDT value. For example, the bar at $x = 1$ shows the number \iey{of} two
consecutive RDT measurements yielding \emph{different} RDT values, and the bar
at $x = 2$ shows the number of two consecutive RDT measurements yielding
\emph{the same} RDT value.

\begin{figure}[!ht]
    \centering
    \includegraphics[width=\linewidth]{figures/consecutive_repeating_measurements_histogram.pdf}
    \caption{Histogram of \agy{the number of} measurements \agy{across which a
    row's} RDT \agy{exhibits} the same value}
    \label{fig:consecutive_repeating_measurements}
\end{figure}

\observation{The RDT of a row frequently changes \omcr{4}{over} time.}

We make two key observations. First, two consecutive RDT measurements likely
yield different RDT values, i.e., the RDT of a row frequently changes \omcr{4}{over} time.
Across all tested rows, \param{79.0\%} of RDT state changes happen after every
measurement. Second, as the number of consecutive measurements increases (as we
go right on the x-axis), the likelihood of those measurements yielding the same
value decreases. \omcr{2}{A} row retains the same RDT value for \param{14}
consecutive RDT measurements \omcr{2}{very rarely} (\omcr{2}{i.e., in}
\emph{only} \param{one} instance)\gf{.}

\subsection{Predictability of RDT's Temporal Variation}
\label{subsec:characterization_predictability}

We have established that there is temporal variation in the RDT of a DRAM row.
We draw a preliminary analysis of the \emph{predictability} of the temporal
variation in RDT. 

\observation{\agy{A row's RDT changes unpredictably \omcr{4}{over} time.}}
\vspace{2pt}
% Read disturbance threshold (RDT) of a row changes \emph{unpredictably} with time.}

Our analysis suggests that individual RDT measurements are likely unpredictable,
that is, given the outcome of past RDT measurements for a victim row, the
outcome of the next measurement likely \emph{cannot} be predicted.\footnote{The
frequency of RDT values collected over many RDT measurements, however, are
predictable based on the probability distributions (histograms) shown
in~\figref{fig:rdt_extended_histogram}.} We perform a two-step analysis to
understand the predictability of the temporal variation in RDT. First, we
carefully interpret the histograms of the RDT values
in~\figref{fig:rdt_extended_histogram}. Second, \agy{we compute the
autocorrelation function of the \atbcr{4}{values}} to detect repeating
patterns\agy{.}
% in a series of RDT measurements\gf{,} we compute the autocorrelation function of the series.

% Why Chi2? Our data is categorical

\noindent
\textbf{Histogram Interpretation.} Many of the RDT histograms resemble prominent
random (discrete) probability distributions. For example, from a visual
inspection, the histograms for M1, H1, and Chip2 strongly suggest that the RDT
measurements follow the probability density function of a normal distribution.
To quantify how well the frequency of the RDT measurements resemble\iey{s} a
normal distribution derived from the mean and the standard deviation of all RDT
measurements, we perform the Chi-square goodness-of-fit test~\cite{pearson1900}.
For each tested chip, the Chi-square goodness-of-fit test tests the null
hypothesis ($H_{0}$) that our observations follow the derived normal
distribution. $H_{0}$ holds if the Chi-square test outputs a p-value greater
than a chosen level of significance denoted as $\alpha$.
We find the minimum p-value across all tested chips \gra{to be} \param{0.18}.
Thus, at \param{$\alpha$~=~0.05}, we \emph{cannot} reject the null hypothesis
that our data follows the derived normal distribution. We conclude that an RDT
measurement likely samples a normally distributed random variable.

\noindent
\textbf{Analyzing Repeating Patterns.} 
\figref{fig:motivation_example} shows the distributions of the series of RDT
measurements of one tested victim row \atbcr{4}{(in Chip1)}. From a visual
inspection, we \emph{cannot} identify any obvious repeating patterns in RDT
measurements. The measured RDT values from other victim rows in other chips also
yield distributions that do \emph{not} harbor repeating patterns. 

To strengthen our observation that a series of RDT measurements does \emph{not}
harbor a repeating pattern, we methodically analyze the RDT measurements using
the autocorrelation function (ACF)~\cite{brockwell1991time}. ACF quantifies the
correlation between the series of RDT measurements with a delayed (by a time
lag) copy of the same series, pictorially depicted in~\figref{fig:acf_example}.
\figref{fig:acf_for_m1} shows the series of RDT measurements (circles show the
mean and error bars show the minimum and maximum RDT values across 1,000
successive \omcr{2}{measurements for a single row}), the ACF of this series, and
the ACF of a series of 100,000 normally distributed random numbers.

\begin{figure}[!ht]
    \centering
    \begin{subfigure}[!h]{\linewidth}
    \includegraphics[width=\linewidth]
    {figures/autocorrelation_function_example.pdf}
    \vspace{-7mm}
    \caption{The autocorrelation function for a time lag of two}
    \label{fig:acf_example}
    \end{subfigure}
    \begin{subfigure}[!h]{\linewidth}
    \includegraphics[width=\linewidth]{figures/autocorrelation_function_second_half.pdf}
    \vspace{-5mm}
    \caption{The series of RDT measurements for \omcr{2}{one row in} DRAM
    module M1 (left), its ACF (middle), and the ACF of a series of 100,000
    normally distributed random numbers \atbcr{2}{(right)}}
    \label{fig:acf_for_m1}
    \end{subfigure}
    \caption{The autocorrelation function (ACF) (top) and
    ACF analysis of RDT measurements from module M1 (bottom)}
    \label{fig:autocorrelation-analysis}
    % \vspace{-2em}
\end{figure}

We observe that the ACF of the series of RDT measurements is \emph{not}
significantly different than the ACF of a normally distributed random variable.
We make similar observations for other series of RDT measurements from other
DRAM modules and chips. We conclude that a series of 100,000 successive RDT
measurements likely does \emph{not} harbor repeating patterns.

\gfcomment{Can this be more scientific? There is no notion of "difficult" in the
experiments. Perhaps: ``A DRAM row experience temporal \emph{variable read
disturbance threshold} (VRDT), where the RDT of the row \emph{randomly} and
\emph{unpredictably} changes over \emph{time}.'' } \take{ RDT changes randomly
and unpredictably. \omcr{2}{As such,} reliably and accurately identifying the
RDT of a DRAM row is \atb{challenging}.\label{take:rdt_difficult}}

% pictorially depicted in~\figref{fig:autocorrelation-
% analysis}-a. 
% We summarize the \param{two} 
% key results of our analysis.
% using~\figref{fig:autocorrelation-
% analysis}. 


% First, there is a non-periodic
% increase (as in~\figref{fig:motivation_example}, where
% the mean RDT value jumps from below 36K to around 38K) or decrease
% in the series of RDT measurements for \param{ChipX,Y}.
% When the effect of this non-periodic increase or decrease 
% from each series of RDT measurements is removed using 
% differencing~\cite{X},
% the resulting series do \emph{not} significantly correlate
% with delayed copies of their own. 
% We observe that the ACF of the resulting 
% series is
% similar to the ACF of a normally distributed
% random variable (see (b) and (d) 
% in~\figref{fig:autocorrelation_function}). 
% Second, 

\subsection{\hpcarevc{Hypothetical Explanation for \phenomenon{}}}

\omcrcomment{2}{(Referring to footnote 8 first sentence.)Meaning? Unclear. Did
we study if a row that is more vulnerable to VRD is also more vulnerable to VRT?
That would be good to study and check.}\atbcrcomment{2}{No. we do not have this
data. We added this footnote to address a shepherd comment that asked us to
explain the difference between observed degree of variation in VRD vs.
VRT.}\hpcareva{We provide a hypothetical explanation that could explain
why\hpcalabel{Main Questions 3\&4} and how RDT temporally
varies.\footnote{\hpcareva{We are not aware of any device-level study of
temporal variations in read disturbance vulnerability, so we \emph{cannot}
definitively confirm this hypothesis. That said, we are unable to identify any
independent variables within our control that allow reliably predicting the
minimum RDT despite extensive testing.}}} \hluo{Electron migration and injection
into the victim cell is a major error mechanism that leads to DRAM read
disturbance bitflips~\cite{ryu2017overcoming, yang2019trap,
walker2021ondramrowhammer, zhou2023double, zhou2024Understanding,
zhou2024Unveiling}. Prior works~\cite{yang2019trap, walker2021ondramrowhammer,
zhou2023double, zhou2024Understanding, zhou2024Unveiling} show that the electron
migration and injection mechanism is heavily assisted by charge traps in the
shared active region of the aggressor and victim cell and its Si/SiO2 interface.
We hypothesize that the temporal variation in \atbcr{4}{RDT} 
\omcr{2}{can be} attributed to the randomly changing occupied/unoccupied states
of these traps~\cite{Oh2011Characterization, Sun2021Trap-Assisted}. \agy{This
hypothetical explanation is similar to the explanation of the variable retention
time (VRT) phenomenon~\cite{kang2014coarchitecting, khan2014efficacy,
liu2013experimental, mori2005origin, qureshi2015avatar, restle1992dram,
yaney1987meta}.\shepherd{\footnote{\shepherd{The observed temporal variation in
RDT (i.e., \phenomenon{}) \emph{may} differ from the observed temporal variation
in DRAM cell retention times (i.e., VRT). \omcr{2}{Unfortunately, there is}
\emph{not} enough available scientific information \omcr{2}{to confidently
explain such differences between \phenomenon{} and VRT}. We do \emph{not} know
enough about the underlying causes of \phenomenon{} and we believe that
\phenomenon{} should be studied independently of VRT. We hope and expect that
\omcr{2}{future} device-level studies \omcr{2}{(inspired by this work)} will
develop a better understanding of the inner workings of VRD as device-level
studies (e.g.,~\cite{zhou2024Understanding,zhou2024Unveiling}) did for RowPress
\emph{after} the RowPress paper~\cite{luo2023rowpress} demonstrated the
empirical basis for the RowPress phenomenon. We believe such studies could
provide insight into the (hypothesized) similarities and differences between VRT
and VRD.}}}} \hpcareva{More device-level studies are needed to build confidence
for our hypothesis that the observed temporal variation is based on
unpredictable physical phenomena.} We leave a more detailed investigation of
device-level mechanisms that cause the temporal variation in RDT (including its
relationship with DRAM variable retention time) for future work.}




% TO ADDRESS MINESH'S COMMENTS FOR THE NEXT VERSION
% you described this as the combination of two different factors: (1) the noisy behavior around the mean; and (2) the shifting of the mean
% as far as I can tell, VRT behavior can pretty much entirely explain (2), though we cannot prove or disprove that easily
% this makes it feel like the paper doesn't bother to go any further than state a new observation, leaving the analysis for future work
% there's not really anything wrong with that, but it doesn't really sit all that well with me when we can do better as experts in the community
% as a 'hypothesis', the subsection is pretty useless imo
% as a 'hypothetical explanation', it is OK but doesn't really close the loop with how the hypothetical explanation can cause both (1) and (2)
% on an unrelated note, I quite like Figure 6. I think it's sister would be a figure showing time-in-particular-state rather than time-in-constant-state
% the time-in-state is the quantity the RTN and 1/f noise models predict for you, so it could be an interesting quantity to present and analyze
% but perhaps that's an item for future work given that this paper focuses just on the observations rather than explanations




% Two major error mechanisms lead to DRAM read disturbance, as 
% described in~\ref{subsec:read_disturbance_device_level}. We 
% hypothesize that the electron injection / diffusion / drift
% mechanism is affected by
% random telegraph noise (RTN)~\cite{X,Y,Z}, %revisit this definition
% a prominent
% phenomenon in which a transistor's drain current exhibits
% random switching events as a function of time~\cite{X,Y,Z}. This 
% effect of RTN, in turn, might result in the
% read disturbance threshold of a victim row to change with time.
% \atbcomment{Double check bullshit sentence}
% \param{RTN typically manifests as normally distributed random noise,
% which aligns well with how RDT behaves based on our observations
% in~\secref{subsec:characterization_predictability}.}
% We conclude that the electron injection / diffusion / drift error mechanism
% subject to random telegraph noise could explain the temporal
% variation in the read disturbance threshold.
% We leave more exhaustive investigation of device-level causes
% of temporal variation in RDT for future work.
% cite The Origins of Random Telegraph Noise in Highly Scaled SiON 
% nMOSFETs and what it cites for the previous sentence.