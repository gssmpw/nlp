\section{Conclusion}

% \agy{For the first time, we experimentally demonstrate the variable read disturbance (VRD) phenomenon. Our analysis on \param{10} DDR4 modules and \param{4} HBM2 DRAM chips shows that a DRAM row's read disturbance threshold significantly (e.g., by $>2\times$) and unpredictably changes with time. 
% Our empirical results have strong implications for the security guarantees of read disturbance mitigation techniques and call attention to future work on efficient online profiling mechanisms
% and reconfigurable read disturbance mitigation techniques.}

We present the results of \atbcr{2}{the first} detailed characterization study
\omcr{4}{of the temporal variation of the} read disturbance (RowHammer and
RowPress) vulnerability in modern DDR4 and HBM2 \omcr{4}{DRAM} chips. We
demonstrate that the read disturbance threshold (RDT) of a DRAM row
\emph{cannot} be reliably identified \atbcr{2}{even with \omcr{4}{hundreds or}
thousands of measurements} because \atbcr{2}{the RDT of a row} changes
significantly and unpredictably \omcr{2}{over} time. Our study leads to
\atbcr{4}{17} findings and \atbcr{4}{four} takeaway lessons which have important
implications for future read disturbance \omcr{3}{mitigation techniques}:
\atbcr{2}{if the RDT of a DRAM row is not identified accurately, these
techniques can easily become insecure.} \omcr{4}{We study potential solutions to
mitigate the effects of \phenomenon{} and find that 1)} \atbcr{3}{using a small
guardband for the observed minimum RDT (when configuring read disturbance
mitigation techniques) \omcr{4}{along with} error-correcting codes is likely
unsafe \atbcr{4}{and 2) using a large guardband along with error-correcting
codes can lead to high performance overheads}.}\atbcrcomment{3}{What do you
think about this based on our analyses in Sec 6?} We hope and expect that our
findings will lead to a deeper understanding of and new solutions to the read
disturbance vulnerabilities in modern DRAM-based computing systems.
\atbcrcomment{3}{adding open source discussion as a todo. i do not want to
promise open source in the final version (yet)}