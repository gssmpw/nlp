\section{Experimental Infrastructure}

We describe our DRAM testing infrastructure and the real DDR4 and HBM2 DRAM chips tested.

\noindent
\textbf{DRAM Testing Setup.} We build our testing setup on DRAM
Bender~\cite{olgun2023drambender,safari-drambender}, an open\gra{-}source
FPGA-based DRAM testing infrastructure \omcr{2}{(which builds on
SoftMC~\cite{hassan2017softmc, softmc-safarigithub})}. 
% \figref{fig:testing_setup} shows one of our FPGA-based DRAM testing setups for
% testing real DDR4 DRAM modules. 
This setup consists of four main components \iey{1})~a host machine that
generates the test program and collects experimental results, \iey{2})~an FPGA
development board (\agy{\atbcr{4}{AMD}} Alveo U200~\cite{alveo-u200} for DDR4 and
\agy{Alveo U50~\cite{alveo-u50} and Bittware XUPVVH~\cite{xupvvh}} for \agy{HBM2
DRAM chips}), programmed with DRAM Bender to execute test programs \agy{and
analyze experimental data}, 3)~a thermocouple\agy{-based} temperature sensor
and a pair of heater pads pressed against the DRAM chips that {heat} up the DRAM
chips to a desired temperature, and 4)~a PID temperature controller (MaxWell
FT200~\cite{maxwellFT200}) that keeps the temperature
at \atbcr{4}{the} desired level with a precision of $\pm$\SI{0.5}{\celsius}.
% \agycomment{I removed the last sentence and embedded into the previous text.}
% HBM2-based FPGA boards.

% \begin{figure}[!ht] \centering
%     \includegraphics[width=\linewidth]{figures/infra_fig.pdf}
%     \caption{\iey{DDR4 DRAM Bender~\cite{olgun2023drambender} experimental
%     setup}} \label{fig:testing_setup} \end{figure}

\noindent
\textbf{Tested DRAM Chips.}
Table~\ref{tab:dram_chip_list} shows the {\numDDRchips{} DDR4 DRAM chips (in
\nummodules{} modules) and \numHBMchips{} HBM2 DRAM chips} that we test from all
three major DRAM manufacturers. {To investigate whether \phenomenon{} is
affected by different DRAM technologies, designs, and manufacturing processes,
we test various} DRAM chips with different densities, die revisions, chip
organizations, and DRAM standards.\omcrcomment{2}{ADD BIG TABLE WITH ALL DETAILS
IN APPENDIX}
% \footnote{\copied{{A DRAM chip's} technology node is {\emph{not} always}
% publicly available. We assume that two DRAM chips from the same manufacturer
% have the same technology node \emph{only} if they share both {1)~}the same die
% density and {2)~the same} die revision code.}}

\begin{table}[h!]
  % \renewcommand{\arraystretch}{0.7}
  \centering
  \footnotesize
  % \captionsetup{justification=centering, singlelinecheck=false, labelsep=colon}
  \caption{Tested DDR4 Modules and HBM2 Chips}
  \begin{adjustbox}{max width=\linewidth}
    \begin{tabular}{c|ccccc}
      \multicolumn{1}{l|}{}                                                        & \textbf{DDR4}                                                                       & \textbf{\# of}     & \textbf{Density}     & \textbf{Chip}        & \textbf{Date}        \\
      \multicolumn{1}{l|}{\textbf{Mfr.}}                                           & \textbf{Module}                                                                     & \textbf{Chips}     & \textbf{Die Rev.}    & \textbf{Org.}        & \textbf{(ww-yy)}       \\ \hline\hline
      \multirow{5}{*}{\begin{tabular}[c]{@{}c@{}}Mfr. H\\ (SK Hynix)\end{tabular}} & H0                                                                                  & 8                  & 8Gb -- J             & x8                   & N/A                  \\
                                                                                   & H1                                                                                  & 8                  & 16Gb -- C            & x8                   & 36-21                \\
                                                                                   & {H2}                                                    & 8                  & 8Gb -- A             & x8                   & 43-18                \\
                                                                                   & {H3, H4}                                                & 8                  & 8Gb -- D             & x8                   & 38-19                \\
                                                                                   & {H5, H6}                                                & 8                  & 8Gb -- D             & x8                   & 24-20                \\ \hline
      \multirow{6}{*}{\begin{tabular}[c]{@{}c@{}}Mfr. M\\ (Micron)\end{tabular}}   & M0                                                                                  & 4                  & 16Gb -- E            & x16                  & 46-20                \\
                                                                                   & M1                                                                                  & 8                  & 16Gb -- F            & x8                   & 37-22                \\
                                                                                   & M2                                                                                  & 8                  & 16Gb -- F            & x8                   & 37-22                \\
                                                                                   & M3, M4                                                & 8                  & 8Gb -- R             & x8                   & 12-24                \\
                                                                                   & M5                                                    & 8                  & 8Gb -- R             & x8                   & 10-24                \\
                                                                                   & M6                                                    & 8                  & 16Gb -- F            & x8                   & 12-24                \\ \hline
      \multirow{6}{*}{\begin{tabular}[c]{@{}c@{}}Mfr. S\\ (Samsung)\end{tabular}}  & S0                                                                                  & 8                  & 8Gb -- C             & x8                   & N/A                  \\
                                                                                   & S1                                                                                  & 8                  & 8Gb -- B             & x8                   & 53-20                \\
                                                                                   & S2                                                                                  & 8                  & 8Gb -- D             & x8                   & 10-21                \\
                                                                                   & S3                                                                                  & 8                  & 16Gb -- A            & x8                   & 20-23                \\
                                                                                   & S4                                                    & 4                  & 4Gb -- C             & x16                  & 19-19                \\
                                                                                   & S5, S6                                                & 8                  & 16Gb -- B            & x16                  & 15-23                \\ \hline\hline
      \multirow{2}{*}{\begin{tabular}[c]{@{}c@{}}Mfr. S\\ (Samsung)\end{tabular}}  & \multirow{2}{*}{\begin{tabular}[c]{@{}c@{}}HBM2 Chip\\ Chip0 -- Chip3\end{tabular}} & \multirow{2}{*}{4} & \multirow{2}{*}{N/A} & \multirow{2}{*}{N/A} & \multirow{2}{*}{N/A} \\
                                                                                   &                                                                                     &                    &                      &                      &                      \\ \hline\hline
      \end{tabular}
    \end{adjustbox}
    \label{tab:dram_chip_list}
    %\par
%\footnotesize{$^*$ \agy{An AMD Xilinx} white paper~\cite{amd2019supercharge} \agy{mentions Samsung HBM2 DRAM chips, but} \emph{no} \agy{further} details \agy{are disclosed due to the company policy}~\cite{xilinxForumPost}.}\\
\end{table}

\subsection{Testing Methodology}
\label{subsec:testing_methodology}

\noindent
\textbf{Disabling Sources of Interference.}
We identify \param{three} \omcr{2}{factors} that can interfere with our results:
1)~\agy{data retention failures}~\cite{liu2013experimental, patel2017reaper},
2)~on-die read disturbance defense mechanisms (e.g.,
TRR~\cite{frigo2020trrespass, hassan2021utrr,micron2018ddr4trr}), 
% 3) data retention failures~\cite{liu2013experimental, patel2017reaper}, 
and 3)~\agy{error correction codes
(ECC)}~\cite{jedec2021hbm,patel2020beer,patel2021harp}. We carefully reuse the
state-of-the-art read disturbance characterization methodology used in prior
works to eliminate the interference \omcr{2}{factors}~\cite{kim2020revisiting,
orosa2021deeper, yaglikci2022understanding, hassan2021utrr, luo2023rowpress,
yaglikci2024spatial,olgun2023hbm, olgun2024read}. First, we make sure that our
experiments finish strictly within \agy{a} refresh window, in which the DRAM
manufacturers guarantee \agy{that \emph{no}} retention bitflips
occur~\cite{jedec2020ddr4, jedec2021hbm}.
% , and we do \emph{not} issue periodic refresh commands in our experiments.
% \agycomment{I removed this because it is more relevant to 2 than 1.}
% \nbcomment{This sentence is a bit convoluted: corresponds to the interference
% sources 1) and 3). In third point 3) is repeated again. Maybe you can reorder
% the sources to explain this in consecutive points/sentences or find a way to
% separate the retention bitflips from periodic refresh.}
Second, \agy{we disable} periodic refresh \agy{as doing so} disables all known
on-die read disturbance defense
mechanisms~\cite{orosa2021deeper,yaglikci2022understanding,
kim2020revisiting,hassan2021utrr}. 
% Third, \nbcomment{this is where 3) is repeated} we ensure that our experiments
% finish within the refresh interval where manufacturers guarantee no retention
% errors will occur~\cite{jedec2020ddr4, jedec2021hbm}. 
Third, we {verify} that the tested DDR4 chips \hpcalabel{A1}\hpcareva{do
\emph{not} have on-die ECC}~\cite{patel2020beer, patel2021harp}, \hpcareva{we do
\emph{not} use rank-level ECC in our testing setup,} and we disable the tested
HBM2 chips' ECC by setting the corresponding HBM2 mode register bit to
zero~\cite{jedec2021hbm}.

\noindent
\copied{\textbf{RowHammer and RowPress Access Pattern}. We use the double-sided
RowHammer and RowPress access
pattern~\cite{kim2014flipping,kim2020revisiting,orosa2021deeper,
seaborn2015exploiting, luo2023rowpress}, which alternately activates two
aggressor rows \omcr{2}{physically adjacent to} a victim row. We record the
bitflips observed in the row between two aggressor rows.}

\noindent 
\copied{\textbf{Logical-to-Physical Row Mapping}. DRAM manufacturers use mapping
schemes to translate logical (memory-controller-visible) addresses to physical
row addresses~\cite{kim2014flipping, smith1981laser, horiguchi1997redundancy,
keeth2001dram, itoh2013vlsi, liu2013experimental, seshadri2015gather,
khan2016parbor, khan2017detecting, lee2017design, tatar2018defeating,
barenghi2018software, cojocar2020rowhammer,  patel2020beer,
yaglikci2021blockhammer, orosa2021deeper}. To identify aggressor rows that are
physically adjacent to a victim row, we reverse-engineer the row mapping scheme
following the methodology described in prior work~\cite{orosa2021deeper}.} 

\noindent
\textbf{RowHammer and RowPress Test Parameters}. We perform multiple different
tests with varying test parameters. We explain the common parameters in
this section and elaborate on the detailed parameters of each test
in~\secref{sec:foundational_results} and~\secref{sec:indepth}. \copied{{We
configure tests by tuning \param{four} parameters: 1)~\emph{\omcr{2}{Hammer
count}}: We define the \emph{hammer count} of a double-sided read disturbance
access pattern as the number of activations \emph{each} aggressor row receives.
Therefore, \atbcr{4}{in} a double-sided RowHammer or a RowPress test with a hammer
count of 10, we activate each of the two aggressor rows 10 times, resulting in a
total of 20 row activations. 2)~\omcr{3}{\emph{Aggressor row on time}
(}\Glsfirst{taggon}): The time each aggressor row stays \omcr{2}{open after}
each activation during a RowHammer or a RowPress test.} 3)~\emph{\omcr{2}{Data
pattern}}: We use the four data patterns {(Table~\ref{table_data_patterns}) that
are widely used in memory reliability testing~\cite{vandegoor2002address} and by
prior work on DRAM characterization (e.g.,~\cite{kim2014flipping,
kim2020revisiting, orosa2021deeper, luo2023rowpress,yaglikci2024spatial,
olgun2024read})}.} \copied{4)~\emph{\omcr{2}{Temperature}}: We use a temperature
controller setup for all DDR4 modules and \agy{one HBM2 DRAM chip (Chip 0)} and
set the target temperature to \atbcr{1}{\SI{50}{\celsius}, \SI{65}{\celsius}, or
\SI{80}{\celsius}}, depending on the type of experiment performed. 
% \figref{fig:temperature_measurements} shows how the temperature of Chip1,
% Chip2, and Chip3 varies during 24 hours based on measurements taken every 5
% seconds. 
Even though we do not have the same temperature controller setups for HBM2
\agy{DRAM} chips, Chip1-3, we \agy{perform these experiments in a
temperature-controlled room. We monitor the in-HBM2-chip temperature sensor
using the IEEE 1500 test port~\cite{jedec2021hbm} and verify} that Chip1, Chip2,
and Chip3's \agy{temperatures are} stable \agy{across all our tests, such that
maximum temperature deviation is \param{\SI{2.0}{\celsius}}
% \agycomment{Can we say this?}}\atbcomment{we can say 2.0 celsius (fig 3 in hbm
% read disturbance paper shows the deviation)}\agycomment{I will not refer to
% that for now as it can be considered as violating double-blindness. We can
% refer to it in the camera-ready version}
over 24 hours of \agy{continuous} testing.}}
% \footnote{We retrieve the temperature for these five chips from an
% in-HBM2-chip temperature sensor using the IEEE 1500 test
% port~\cite{jedec2021hbm}.}}

\begin{table}[!htbp]
\caption{Data patterns used in our experiments}
\vspace{-1em}
\begin{center}
\begin{adjustbox}{max width=\linewidth}
\begin{tabular}{|c||c|c|c|c|}
\hline
\textbf{Row Addresses} & \textbf{\textit{Rowstripe0}}&
\textbf{\textit{Rowstripe1}}& \textbf{\textit{Checkered0}} &
\textbf{\textit{Checkered1}}\\
\hline
\hline
Victim (V) & 0x00 & 0xFF & 0x55 & 0xAA\\
\hline
Aggressors (V $\pm$ 1) & 0xFF & 0x00 & 0xAA & 0x55\\
\hline
V $\pm$ [2:8] & 0x00 & 0xFF & 0x55 & 0xAA\\
\hline
\end{tabular}
\end{adjustbox}
\label{table_data_patterns}
\end{center}
\vspace{-1em}
\end{table}


% \begin{figure}[!ht]
%     \centering
%     \includegraphics[width=\linewidth]{example-image-duck}
%     \caption{Temperature of Chip1, Chip2, and Chip3 for 24 hours}
%     \label{fig:temperature_measurements}
% \end{figure}

\noindent
\textbf{Read Disturbance Threshold.} 
% \agycomment{mentioning \phenomenon without properly defining it here is a bit
% odd. I am changing it. Please revert if you don't like.}
We quantify \agy{the read disturbance vulnerability of a DRAM row using the}
% \phenomenon{} using 
\emph{\omcr{2}{read disturbance threshold (RDT)}} \omcr{2}{metric, i.e.,}
% We define RDT as 
the \emph{hammer count needed to induce the first read disturbance bitflip} in
the victim row.\agycomment{Don't we care about the BER at all?}\atbcomment{no
ber in this study (YET)}