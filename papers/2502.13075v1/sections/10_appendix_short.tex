
\section*{Appendix}
\appendix
\section{\atbcr{2}{Read Disturbance Threshold Testing}}
\label{app:testing}

\atbcr{2}{We describe a methodology for estimating the read disturbance
threshold (RDT) testing time for a double-sided RowHammer access pattern (i.e.,
$t_{AggOn}$ = minimum $t_{RAS}$). We use the methodology to demonstrate testing
times for a DRAM row, a bank, a rank of chips, and a DRAM module, for a varying
number of test iterations.} \atbcr{2}{An iteration of an RDT test yields one RDT
measurement for one DRAM row and consists of 1)~initializing the victim row and
the two aggressor rows, 2)~performing double-sided RowHammer by repeatedly
activating the aggressor rows, and 3)~reading the victim row's data to check for
bitflips. To estimate RDT testing times, we tightly schedule the DRAM commands
needed to perform each step. We report both 1)~the number and order of the DRAM
commands needed to perform the RDT test (Tables~\ref{table:test-commands}), and
2)~the time required to perform the RDT test using the timing parameters
provided by the DDR5 standard~\cite{jedecddr5c} assuming 8800 MT/s speed rate.
\atbcr{3}{Appendix A in an extended version~\cite{self.arxiv} provides an RDT
testing time estimation methodology when multiple banks are tested
simultaneously and a wider variety of test time results.}}

\begin{table}[!ht]
  \centering
  \scriptsize
  \caption{\atbcr{2}{List of DRAM commands issued to measure RDT once for one
  victim DRAM row in one bank using the double-sided read disturbance access
  pattern~\cite{kim2014flipping,kim2020revisiting,
  orosa2021deeper,seaborn2015exploiting} with $t_{AggOn}$ = minimum $t_{RAS}$
  (one hammer constitutes activation of the two aggressor rows)}}
  \begin{adjustbox}{width=0.7\linewidth}
  \begin{tabular}{|l||l|l|r|}
  \hline
  \textbf{Command}                & \textbf{Address}                      & \textbf{Timing}      & \textbf{\# of Commands}                 \\ \hline\hline
  ACT                    & \multirow{4}{*}{Victim}      & $t_{RCD}$        & 1                              \\ \cline{1-1} \cline{3-4} 
  \multirow{2}{*}{WRITE} &                              & $t_{CCD\_L\_WR}$ & 127                            \\ \cline{3-4} 
                         &                              & $t_{WR}$         & 1                              \\ \cline{1-1} \cline{3-4} 
  PRE                    &                              & $t_{RP}$         & 1                              \\ \hline
  ACT                    & \multirow{4}{*}{Aggressor 1} & $t_{RCD}$        & 1                              \\ \cline{1-1} \cline{3-4} 
  \multirow{2}{*}{WRITE} &                              & $t_{CCD\_L\_WR}$ & 127                            \\ \cline{3-4} 
                         &                              & $t_{WR}$         & 1                              \\ \cline{1-1} \cline{3-4} 
  PRE                    &                              & $t_{RP}$         & 1                              \\ \hline
  ACT                    & \multirow{4}{*}{Aggressor 2} & $t_{RCD}$        & 1                              \\ \cline{1-1} \cline{3-4} 
  \multirow{2}{*}{WRITE} &                              & $t_{CCD\_L\_WR}$ & 127                            \\ \cline{3-4} 
                         &                              & $t_{WR}$         & 1                              \\ \cline{1-1} \cline{3-4} 
  PRE                    &                              & $t_{RP}$         & 1                              \\ \hline
  ACT                    & \multirow{2}{*}{Aggressor 1}  & $t_{RAS}$        & \multirow{4}{*}{\# of hammers} \\ \cline{1-1} \cline{3-3}
  PRE                    &                              & $t_{RP}$         &                                \\ \cline{1-3}
  ACT                    & \multirow{2}{*}{Aggressor 2}  & $t_{RAS}$        &                                \\ \cline{1-1} \cline{3-3}
  PRE                    &                              & $t_{RP}$         &                                \\ \hline
  ACT                    & \multirow{3}{*}{Victim}      & $t_{RCD}$        & 1                              \\ \cline{1-1} \cline{3-4} 
  \multirow{2}{*}{READ}  &                              & $t_{CCD\_L}$     & 127                            \\ \cline{3-4} 
                         &                              & $t_{RTP}$        & 1                              \\ \hline
  \end{tabular}
  \end{adjustbox}
  \label{table:test-commands}
  \end{table}

\atbcr{2}{Figure~\ref{fig:rdt_testing_time} shows the time (in \atbcr{4}{seconds},
y-axis) to perform one RDT measurement for a victim row for a varying number of
hammer counts (different colored bars) and varying number of DRAM rows in a bank
(x-axis).}\atbcrcomment{4}{We will have the full(er) picture in the extended
version. Testing all banks can be done in parallel. The runtime of the test for
32 banks is not 32X longer. It is approx. 2X longer. We have all that in the
extended appendix. This figure provides the basis for computing experiment times
reported in the final version.}

\begin{figure}[!ht]
  \centering
  \includegraphics[width=0.9\linewidth]{figures/test_times_hammer_row.pdf}
  \vspace{-5pt}
  \caption{\atbcr{2}{Time to perform \atbcr{3}{\omcr{4}{\emph{a single}} RDT
  measurement for a given number of victim rows (x-axis) and a varying number of
  hammer counts (\# of hammers)}}}
  \vspace{-5pt}
  \label{fig:rdt_testing_time}
\end{figure}
