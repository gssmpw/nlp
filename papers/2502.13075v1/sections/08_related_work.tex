\section{Related Work}
\atbcrcomment{4}{related work looks smaller. nothing is missing from the previous version.}
\nb{To our knowledge, this is the first work to experimentally demonstrate and
comprehensively examine \emph{temporal variation} in read disturbance behavior
in modern DRAM chips\omcr{3}{, and provide potential solutions to mitigate its
effects}. In this section, we discuss other relevant prior work.}


\noindent
\textbf{Experimental Read Disturbance Characterization.} 
%\copied{There are several works that extensively characterize \nb{read
%disturbance} using real DRAM chips~\readDisturbanceCharacterizationCitations{}
%\nbcomment{double check the references}.}
%\newcommand{\readDisturbanceCharacterizationCitations}[0]{\cite{,,,,,he2023whistleblower,yaglikci2024spatial,luo2023rowpress}}
%\nb{The previous works on
%RowHammer~\cite{kim2014flipping,orosa2021deeper,kim2020revisiting,yaglikci2022understanding}
%characterize the sensitivity of RowHammer to 1) refresh rate, 2) activation
%rate, 3) the physical distance between aggressor and victim rows,  4)
%temperature, 5) aggressor row active time, 6) victim DRAM cell's physical
%location, and 7) wordline voltage. The RowPress paper~\cite{luo2023rowpress}
%extensively characterizes the sensitivity of RowPress to 1) temperature, 2)
%access pattern and 3) row open time.}
\copied{Prior work{s} extensively characterize the RowHammer and RowPress
vulnerabilities in real DRAM chips\readDisturbanceCharacterizationCitations{}.
These works %experimentally 
demonstrate {(}using real DDR3, DDR4, LPDDR4, \nb{and HBM2} DRAM chips{)}
\nb{how} a DRAM chip's \nb{read disturbance} vulnerability varies with 1)~DRAM
refresh rate~\cite{hassan2021utrr,frigo2020trrespass,kim2014flipping}, 2)~the
physical distance between aggressor and victim
rows~\cite{kim2014flipping,kim2020revisiting,lang2023blaster}, 3)~DRAM
generation and technology
node~\cite{orosa2021deeper,kim2014flipping,kim2020revisiting,hassan2021utrr},
4)~temperature~\cite{orosa2021deeper,park2016experiments}, 5)~the time the
aggressor row stays
active~\cite{orosa2021deeper,park2016experiments,olgun2023hbm,olgun2024read,luo2023rowpress,nam2024dramscope,nam2023xray},
~6)~physical location of the victim DRAM
cell~\cite{orosa2021deeper,olgun2023hbm,olgun2024read,yaglikci2024spatial},}
\nb{7)~wordline voltage~\cite{yaglikci2022understanding}, and 8)~supply
voltage~\cite{he2023whistleblower}.}
% \nb{The RowPress paper~\cite{luo2023rowpress} extensively characterizes the
%sensitivity of RowPress to 1) temperature, 2) access pattern, and 3) row open
%time.}\nbcomment{could not add he2023whistleblower here somewhere} The original
%RowHammer work~\cite{kim2014flipping} 1) investigates the vulnerability in
%detail for the first time, 2) characterizes the sensitivity of RowHammer to
%refresh rate, activation rate, and the physical distance between aggressor and
%victim rows, and 3) analyzes several potential solutions. The second extensive
%RowHammer characterization work~\cite{kim2020revisiting}, analyzes RowHammer
%scalability by performing experiments on 1580 DDR3, DDR4, and LPDDR4 commodity
%DRAM chips from different DRAM generations and technology nodes, clearly
%demonstrating that RowHammer has become an even more serious problem over DRAM
%generations. The third work~\cite{orosa2021deeper}, studies the sensitivity of
%RowHammer to DRAM chip temperature, aggressor row active time, and victim DRAM
%cell's physical location, by performing experiments on 248 DDR4 and 24 DDR3
%modern DRAM chips from four major manufacturers. \nb{The first RowPress
%work~\cite{luo2023rowpress} 1) experimentally demonstrates RowPress across 164
%real DRAM chips, 2) extensively characterizes the sensitivity of RowPress to
%temperature, access pattern and row open time, and 3) discusses and evaluates
%potential migitation techniques.}
\nb{None of these works analyze \emph{temporal variation} in read disturbance.}

%\outline{RowHammer PUFs.} \noindent \textbf{RowHammer Physical Uncloneable
% Functions (PUFs).} \nb{Two prior
% works~\cite{schaller2017intrinsic,anagnostopoulos2018intrinsic} leverage
% RowHammer bitflips as PUFs. \atb{The evaluation of a RowHammer PUF may exhibit
% noise~\cite{schaller2017intrinsic}. \phenomenon{} could be one of the reasons
% for this observed noise.}}

% the observation that locations of bitflips induced by RowHammer are stable to
% create physical unclonable functions (PUFs). The first
% work~\cite{schaller2017intrinsic} reports that the RowHammer PUF measurements
% exhibit noise. Variable read disturbance could be one of the reasons for the
% observed noise.} \nbcomment{needs vetting. this part is out of place rn}


% \noindent
% \textbf{Read Disturbance \nb{mitigations}.}
% \nb{Prior works propose \nb{mitigations} to prevent read disturbance bitflips~\mitigatingRowHammerAllCitations{}. The security guarantees of these \nb{mitigations} rely on an accurately identified minimum read disturbance threshold across all DRAM rows in a system. The key takeaways in our study have strong implications for the security guarantees of these techniques, which we already discuss in~\secref{sec:implications}.}

\noindent
\textbf{System-Level RowHammer Tests.}
\nb{Several
works~\cite{farmani2021rhat,cojocar2020rowhammer,zhang2021bitmine,memtest86}\atbcrcomment{3}{I
am citing stuff that explicitly claim to have a testing methodology here. do not
want to add all characterization work} develop tools or RowHammer tests that aim
to identify read disturbance bitflips in DRAM chips in a computing system.
\atb{These tools and tests could \omcr{2}{prove} useful in developing future
efficient online RDT profiling techniques.}}

\noindent
\textbf{Retention Failure Profiling.} Prior
works~\cite{liu2013experimental,qureshi2015avatar,patel2017reaper,
khan2014efficacy,khan2016parbor}\omcrcomment{4}{Do not miss works}
\omcr{2}{advocate and} propose methods to efficiently profile DRAM retention
failures that are subject to the variable retention time (VRT) phenomenon. These
works could inspire \omcr{3}{or aid the development of} online RDT profiling
techniques.
