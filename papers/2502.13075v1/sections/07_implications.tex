\section{Implications of VRD for\\\agy{Read Disturbance Mitigation \omcr{3}{Techniques}}}
\label{sec:implications}

The key takeaways we draw from our empirical study have \omcr{2}{important}
implications for the security guarantees of read disturbance \nb{mitigation}
\omcr{3}{techniques}, proposed by \omcr{2}{both} academia and industry
(e.g.,~\cite{kim2014flipping,olgun2024abacus,saroiu2024ddr5,saroiu2022configure,
yaglikci2021blockhammer,
bennett2021panopticon,bostanci2024comet,yaglikci2024spatial,canpolat2024understanding,
park2020graphene,kim2022mithril,qureshi2022hydra,marazzi2022protrr,
marazzi2023rega,saxena2024start,saileshwar2022randomized,
saxena2022aqua,saxena2024rubix,jaleel2024pride,seyedzadeh2017cbt,seyedzadeh2018cbt,woo2023scalable,
canpolat2025chronus}) and some already standardized for immediate system
integration (e.g.,
PRAC~\cite{jedecddr5c,canpolat2024understanding,bennett2021panopticon,
kim2023ddr5,canpolat2025chronus}). At a high level, the security guarantees
provided by any of these \nb{\omcr{3}{mitigation techniques}} rely on an
accurately identified \emph{minimum read disturbance threshold (RDT) across all
DRAM rows in a computing system}. Our results show that accurately identifying
the minimum RDT across all DRAM rows\gra{,} even with \omcr{2}{thousands of} RDT
measurements\gra{,} is \hpcarevcommon{challenging} because the RDT of a row
changes \omcr{4}{over} time in an unpredictable way (Takeaway~\ref{take:rdt_difficult}). 

\hpcalabel{Main Questions 1\&2}\hpcarevcommon{We evaluate and discuss the
effectiveness of combining a guardband for RDT (e.g., by reducing the minimum
observed RDT by an arbitrary factor \omcr{2}{when configuring mitigation
techniques}) with error-correcting codes (ECC) \omcr{2}{at} mitigating
\phenomenon{}-induced bitflips \omcr{2}{(i.e., read disturbance bitflips in the
presence of \phenomenon{})}. We show that using \atbcr{2}{a $>$10\% guardband
for the minimum observed RDT} \omcr{4}{along with}
\atbcr{2}{single-error-correcting double-error-detecting
(SECDED~\cite{kim2015bamboo})} ECC or \atbcr{2}{Chipkill-like ECC (e.g., single
symbol error correction~\cite{amd2013sddc,yeleswarapu2020addressing,
chen1996symbol})} could prevent \phenomenon{}-induced bitflips at the cost of
potentially higher performance overheads incurred by read disturbance mitigation
techniques that are configured with smaller read disturbance threshold values
(due to applying a guardband).} We believe future work on online RDT profiling
and runtime configurable read disturbance \omcr{3}{mitigation techniques} could
remedy the challenges imposed by \phenomenon{} on read disturbance
\nb{mitigation} \omcr{2}{techniques}. 

\subsection{Importance of \hpcarevcommon{Accurately} Identifying RDT}

% \noindent
% \textbf{Importance of \hpcarevcommon{Accurately} Identifying RDT.}
% Many high-performance and low-overhead hardware-based read disturbance mitigations
% prevent read disturbance bitflips by taking a preventive action (e.g., refreshing the victim row or 
% throttling activation commands to the aggressor row) based on the activation count of a potential aggressor
% DRAM row. These techniques trigger their preventive actions when the potential aggressor row's activation
% count reaches a predefined threshold value. 
% This threshold value is derived directly from the read disturbance threshold.
% Therefore, 
There are two reasons that make accurate identification of RDT important for
read disturbance mitigation \omcr{2}{techniques}: 1)~security and 2)~system
performance, energy, and area overheads. First, the RDT value used \omcr{2}{to}
configure a mitigation technique \emph{cannot} be larger than the one
experienced (at any time) by any victim DRAM row in a DRAM chip. Otherwise, the
mitigation technique's security guarantees are compromised. For example,
PRAC~\cite{jedecddr5c,bennett2021panopticon,
canpolat2024understanding,kim2023ddr5,canpolat2025chronus}, if configured with
an RDT value that is larger than the smallest RDT of a DRAM row, would
eventually fail to prevent a read disturbance bitflip if this DRAM row is
hammered or pressed. Second, the configured RDT value should \emph{not} be
\hpcarevcommon{significantly} smaller than the one experienced by any victim
DRAM row\gra{,} as smaller RDT values \omcr{2}{lead to higher} system
performance, energy, and area or storage
overheads~\cite{olgun2024abacus,canpolat2024understanding,
yaglikci2021blockhammer,kim2020revisiting,qureshi2022hydra,
canpolat2025chronus,yaglikci2024spatial}.
% Configuring a mitigation with a very low RDT value that is
% expected to cover the minimum RDT experienced by any DRAM row would 
% lead to prohibitive performance, energy, and
% area overheads.

% \noindent
% \textbf{Challenges of \hpcarevcommon{Accurately} Identifying RDT.} 
\subsection{Challenges of \hpcarevcommon{Accurately} Identifying RDT}

Measuring the RDT of a DRAM row \omcr{2}{\omcr{3}{hundreds} of times} is
\emph{not} sufficient for drawing a comprehensive profile for the RDT of the
DRAM row (Takeaways~\ref{take:rdt_difficult}
and~\ref{take:rdt_need_more_measurements}). The difference between \nb{the}
minimum and maximum RDT of a DRAM row can be more than
\atbcr{1}{\param{3.5$\times{}$} \omcr{2}{after 1,000 measurements} (see
Finding~\ref{obs:all-rows-vary})} and may not be bounded.\footnote{\omcr{2}{See
discussion on limitations of our experimental methodology in
\secref{sec:discussion}.}} While repeated RDT measurements can lead to a better
minimum RDT estimate for a DRAM row
(Takeaway~\ref{take:rdt_need_more_measurements}), \phenomenon{} is affected by
data pattern, \gls{taggon}, and temperature (Takeaways~\ref{take:data_pattern}
and~\ref{take:taggon_temperature}), which makes comprehensive RDT profiling
time-intensive \hpcalabel{B1}\hpcarevb{because state-of-the-art integrated
circuit test times are measured in seconds to
minutes~\cite{ieee2019heterogeneous,vandegoor2004industrial}, while as few as
\emph{only} two RDT measurements (for all DRAM rows in one bank) can require
hours of testing.} For example, \hpcarevb{measuring the RDT of each DRAM row in
a bank \emph{only once} with a hammer count of \atbcr{2}{8},000}, using four
data patterns, at \gls{taggon} = minimum \omcr{2}{$t_{RAS}$}, and at three
temperature levels takes approximately \param{\atbcr{2}{39} minutes} for a
\omcr{3}{\emph{single}} DRAM bank of 256K rows \atbcr{2}{(see
Appendix~\ref{app:testing} for read disturbance threshold test time estimation
methodology details and test time demonstrations)}.
% Thus, identifying the RDT of the DRAM row and using a
% guardband 
% to ensure minimum RDT is covered
% \hpcarevcommon{could} require a large guardband (and thereby inflict 
% considerable system performance, energy, and area overheads)
% or may not work (the security guarantees of the read disturbance mitigation may be compromised). 


% \noindent
% \hpcarevcommon{\textbf{Overheads of Using a Guardband.} 
\subsection{\hpcarevcommon{Overheads of Using a Guardband for RDT}}

\hpcalabel{Main Question 2}\hpcarevcommon{To determine the RDT of a DRAM row, a
system designer or a DRAM manufacturer might measure RDT a few times (to
minimize testing time) and apply a safety margin (i.e., a guardband) to the
minimum observed RDT value. To understand the performance overheads of using a
guardband for RDT, we evaluate four state-of-the-art mitigation techniques
\omcr{2}{(Graphene~\cite{park2020graphene}, PRAC~\cite{jedecddr5c},
PARA~\cite{kim2014flipping}, and MINT~\cite{qureshi2024mint})} in a DDR5-based
computing system simulated using a cycle-level memory system simulator,
Ramulator \atbcr{3}{2.0}~\cite{ramulator2github,luo2024ramulator}
\omcr{3}{(based on Ramulator~\cite{ramulatorgithub,kim2016ramulator})}.
\figref{fig:mitigation_performance} shows system performance \omcr{2}{with} the
four mitigation techniques normalized to the baseline system that does
\emph{not} implement read disturbance mitigation using 15 four-core highly
memory intensive workload mixes.\footnote{\hpcarevcommon{We use 57 single-core
workloads from SPEC CPU2006~\cite{spec2006}, SPEC CPU2017~\cite{spec2017},
TPC~\cite{tpcweb}, MediaBench~\cite{fritts2009media}, and YCSB~\cite{ycsb} to
construct 15 four-core workload mixes. We consider a workload to be highly
memory intensive if it has \omcr{2}{an LLC MPKI (last level cache misses per
kilo instruction) that is $\geq{}20$}. Graphene
(memory-controller-based)~\cite{park2020graphene} and PRAC
(in-DRAM)~\cite{jedecddr5c} have storage overhead and track the
activation count of an aggressor row using hardware counters and preventively
refresh \omcr{2}{the aggressor row's} neighbors before the activation count
reaches the \omcr{3}{configured} read disturbance threshold. PARA
(memory-controller-based)~\cite{kim2014flipping} and MINT
(in-DRAM)~\cite{qureshi2024mint} do \emph{not} have storage overhead and
determine the target row of a DRAM activate command as an aggressor row
\atbcr{2}{based on a probability} \omcr{3}{that is determined based on the
configured RDT} and preventively refresh \atbcr{2}{the aggressor row's}
neighbors.}} The x-axis shows \omcr{2}{eight} \atbcr{3}{different} read
disturbance threshold values; the first four representing a near-future RDT of
1024 with 0\%, 10\%, 25\%, and 50\% safety margins and the last four
representing a future, very-low RDT of 128 with 0\%, 10\%, 25\%, and 50\% safety
margins. We make two key observations.}\footnote{\atbcr{3}{Prior works
characterize real DRAM chips and observe that safety margins for DRAM command
timing parameters can \omcr{4}{be >}40\%~\cite{chang2016understanding} and
safety margins for data retention times \omcr{4}{can be even
larger}~\cite{liu2013experimental}.}}

\vspace{5pt}
\begin{figure}[!ht]
    % \vspace{-1em}
    \centering
    \includegraphics[width=1\linewidth]{figures/mitigation_performance.pdf}
    \caption{\hpcarevcommon{Four-core highly memory intensive workload
    performance normalized to the baseline system without read disturbance
    mitigation for \omcr{4}{two} \atbcr{3}{different} read disturbance threshold
    \atbcr{3}{values}\omcr{5}{,} \omcr{4}{each with four different levels of
    safety margin (guardband) added}}}
    \vspace{5pt}    
    \label{fig:mitigation_performance}
\end{figure}

\hpcarevcommon{First, a \omcr{3}{small but potentially unsafe} 10\% safety
margin for RDT does \emph{not} significantly increase mitigation performance
overheads at RDT=1024 and =128. For example, \omcr{3}{performance with}
Graphene, PRAC, PARA, and MINT \atbcrcomment{4}{with 10\% margin the results are
this good}\omcr{3}{using a 10\% safety margin for RDT=128} reduces by 1.0\%, 0.0\%,
5.9\%, and 0.0\%, \omcr{3}{respectively,} \atbcr{2}{compared to no
margin}.\footnote{\hpcarevcommon{PRAC and MINT's performance overheads do
\emph{not} increase as RDT reduces from 128 to 115 because the number and the
frequency of PRAC and MINT's preventive actions (e.g., DRAM-initiated
back-offs~\cite{jedecddr5c} and RFM commands~\cite{jedecddr5c}) do
\emph{not} change as RDT reduces from 128 to 115.}} Second, a relatively
aggressive \omcr{3}{but much safer} safety margin of 50\% substantially
increases performance overheads \omcr{3}{incurred by mitigation techniques}. For
example, \omcr{3}{performance with} Graphene, PRAC, PARA, and MINT
\omcr{3}{using a 50\% safety margin at RDT=128} reduces by 8.5\%, 7.6\%, 35.0\%,
and 45.0\%\omcr{3}{, respectively,} \atbcr{2}{compared to no margin}. We
conclude that the performance overheads \omcr{3}{incurred with} four
state-of-the-art read disturbance mitigation techniques substantially increase  
with \omcr{4}{a larger} guardband for RDT. Based on our analysis, we do
\emph{not} recommend relying \omcr{3}{solely} on guardbands to address the
temporal variation in RDT.}

% \vspace{3pt}
% \noindent
% \hpcarevcommon{\textbf{Effectiveness of Guardbands and ECC Against \phenomenon{}.}
\subsection{\hpcarevcommon{Effectiveness of Guardbands \& ECC Against \phenomenon{}}}

\shepherd{To quantitatively assess the effectiveness of using guardbands for
RDT, we analyze the probability of finding the minimum observed RDT of a DRAM
row across 1,000 measurements within 10\%, 20\%, 30\%, 40\%, and 50\% of the
minimum observed RDT of that DRAM row \omcr{3}{using} \omcr{3}{N <} 1,000
measurements \omcr{3}{via} the testing methodology described
in~\secref{sec:indepth}.
\figref{fig:probability-within-margin}\omcrcomment{2}{How would this chang eif
we make 1M measurements and predict with only 1-500 measurements. Good to point
out this methodological limitation in the limitations
section}\atbcrcomment{2}{we do now?} shows the mean \atbcr{4}{(circles) and
minimum (bars)} probability of finding the minimum RDT (across all tested rows
and combinations of test parameters) within a safety margin indicated by the
color of the \atbcr{4}{circle or the bar} as the number of measurements
\omcr{3}{(N)} increases from 1 to 500 (x-axis). For example, for \omcr{3}{N} =
\atbcr{3}{50} measurements, the red \atbcr{4}{(leftmost) circle} indicates the
\emph{average} probability of \atbcr{3}{50} RDT measurements yielding a value
that is within 10\% of the minimum RDT value observed across 1,000 measurements,
across all rows and combinations of test parameters \omcr{2}{(which is y =
\atbcr{4}{0.991})}. The \atbcr{4}{red (leftmost) bar under the red circle
indicates the} \emph{minimum} probability \atbcr{3}{of 50 RDT measurements
yielding a value that is within 10\% of the minimum RDT value observed across
1,000 measurements,} across all tested rows and combinations of test parameters
\omcr{2}{(which is y = 0.\atbcr{3}{045})}.}

\begin{figure}[!ht]
    \centering
    \includegraphics[width=1\linewidth]{figures/probability_within_margin_points_bars.pdf}
    \caption{\shepherd{Probability of finding the minimum RDT with \omcr{3}{N <}
    1,000 measurements using a safety margin. \atbcr{4}{Circles show the mean
    and bars show the minimum probability across all tested rows.}}}
    \label{fig:probability-within-margin}
\end{figure}

\omcr{3}{We make two major observations. First, even with N = 50 (500)
measurements, the average probability of finding the minimum RDT is
99.\omcr{4}{07}\% (99.\omcr{4}{86}\%) with a small safety margin (10\%), and the
minimum probability is even lower, i.e., 4.\omcr{4}{46}\% (48.\omcr{4}{62}\%).}
\shepherd{\atbcr{3}{Second,} \atbcr{3}{with N = 500 measurements, the minimum
probability of finding the minimum RDT is \emph{only} 74.\omcr{4}{91}\% with a
\emph{large} safety margin (50\%), i.e., even a large guardband does \emph{not}
guarantee that the minimum RDT is always identified.}}
% there is at least one
% DRAM row whose minimum RDT value across 1,000 measurements is
% (probabilistically) smaller than the minimum RDT identified by up to 500
% measurements (as shown on the x-axis values) when safety margins smaller than
% 50\% are used for RDT measurements.
% \footnote{{The largest degree of observed variation in RDT
% across a \atbcr{2}{very small} number of 1,000 measurements is
% \atbcr{2}{3.5}$\times{}$ (see~\secref{subsec:across-rows}). The degree of the
% variation in RDT could increase with the number of measurements or the number of
% tested DRAM modules, banks, and rows. \omcr{3}{As such, the RDT of a row}}}
We
conclude that using safety margins (guardbands) alone is likely \emph{not}
effective \omcr{2}{at} mitigating \phenomenon{}-induced read disturbance
bitflips.

\hpcalabel{Main Questions 1\&2}\hpcarevcommon{We conduct an experiment to
understand the effectiveness of combining a guardband for RDT with ECC
\omcr{2}{at} preventing \phenomenon{}-induced bitflips. We conclude that
\atbcr{2}{a $>$10\% guardband for the observed minimum RDT} and
\atbcr{2}{single-error-correcting double-error-detecting (SECDED) or Chipkill-like (SSC)}
ECC could \atbcr{2}{potentially} \omcr{3}{(but likely not safely)} prevent
\phenomenon{}-induced read disturbance bitflips. \shepherd{In the experiment, we
use Checkered0 and Checkered1 data patterns (see
Table~\ref{table_data_patterns}), and set \gls{taggon}~=~minimum tRAS. We keep
the temperature of the tested DRAM chips at \SI{50}{\celsius}. We use the DDR4
DRAM modules used in~\secref{sec:indepth} and test 50 DRAM
rows\omcrcomment{2}{too few}\atbcrcomment{2}{discussion section more prominently
mentions ``few rows'' as a major limitaion} in each module.} We 1) measure the
RDT of each tested DRAM row 5 times (to maintain a reasonable testing time) to
find the minimum RDT for the tested DRAM row and 2) repeatedly test the DRAM row
for read disturbance failures using RDT safety margins of 50\%, 40\%, 30\%,
20\%, and 10\% for 10,000 times. For example, if the first step of the
experiment yields an RDT of 500 for a DRAM row, we repeatedly test the DRAM row
10,000 times for each of the hammer count values of 250, 300, 350, 400, and 450.
\shepherd{\figref{fig:bitflip-histogram} shows the histogram for the number of
unique bitflips in a DRAM row (when we use a safety margin of 10\%) across
10,000 RDT measurements.}}
\atbcrcomment{2}{For larger safety margins we do not see anything but 1 bitflip.}

\begin{figure}[!ht]
    \centering
    \includegraphics[width=1\linewidth]{figures/unique_bit_index_histogram.pdf}
    \caption{\shepherd{Number of unique bitflips in a DRAM row when we use a safety margin of 10\%. The histogram
    shows the distribution of the number of unique bitflips across all tested rows.}}
    \label{fig:bitflip-histogram}
\end{figure}

\shepherd{We make two key observations.}
\hpcarevcommon{First, there are up to 5 unique DRAM cells that experience read
disturbance bitflips in a DRAM row across 10,000 measurements at a safety margin
of 10\%. The bitflips manifest in up to 4 different DRAM chips on the DRAM
module and there \omcr{2}{is} at most one bitflip in a single-error correcting
and double-error detecting (SECDED)~\cite{kim2015bamboo} or a Chipkill-like ECC
(e.g., single symbol error
correction~\cite{amd2013sddc,yeleswarapu2020addressing, chen1996symbol})
codeword. Therefore, the observed bitflips \omcr{2}{(in this \emph{very limited}
set of experiments) likely} lead to error patterns that are correctable by
SECDED~\cite{kim2015bamboo} and Chipkill-like
ECC~\cite{amd2013sddc,yeleswarapu2020addressing, chen1996symbol}. However, the
presence of \phenomenon{}-induced bitflips in different DRAM chips suggests that
\phenomenon{} could yield \emph{multiple} read disturbance bitflips in one ECC
codeword, which could result in uncorrectable or undetectable errors and lead to
silent data corruptions (SDCs), albeit infrequently
\shepherd{(Table~\ref{table:error-probability} quantifies the probability of
such \omcr{3}{VRD-induced} errors\omcr{3}{, as we explain soon, and shows high
likelihood of SDCs even with Chipkill\omcr{4}{-like} ECC})}. \omcr{2}{With more
\omcr{3}{RDT} measurements, we may see more bitflips and higher silent data
corruption rates.} Second, for safety margins larger than 10\%, we do \emph{not}
observe more than one bitflip in the tested DRAM rows across 10,000 measurements
\shepherd{(not shown in~\figref{fig:bitflip-histogram})}.}

\shepherd{We quantify the probability of uncorrectable errors \omcr{3}{(i.e.,
SDCs)} assuming the worst error rate we observed empirically so far that results
in 5 bitflips in a \SI{64}{\kibi\bit} DRAM row ($7.6e-5$ bit error rate) for a
10\% safety margin. Table~\ref{table:error-probability} shows the probability of
uncorrectable, undetectable, and uncorrectable but detectable errors for
single-error-correcting (SEC~\cite{kim2015bamboo}) \atbcr{4}{code and}
single-error-correcting double-error-detecting (SECDED~\cite{kim2015bamboo})
\atbcr{4}{code} using a 72-bit codeword, and single-symbol-correcting
(Chipkill-like, SSC~\cite{yeleswarapu2020addressing}) code using a 144-bit
codeword with 18 symbols in a codeword.}\atbcrcomment{3}{SSC also does not have
detection}

\vspace{2mm}
\begin{table}[!ht]
\caption{\shepherd{Probability of uncorrectable, undetectable, and detectable
uncorrectable errors \omcr{4}{at} the worst error rate we observed empirically
so far ($7.6e-5$) \omcr{4}{using} an RDT safety margin of 10\% for SEC, SECDED,
and Chipkill-like SSC codes. \omcr{3}{N/A indicates the result category does not
exist for the shown ECC type.}}}
\vspace{-3mm}
\begin{center}
\begin{adjustbox}{max width=\linewidth}
\begin{tabular}{|l||r|r|r|r|}
\hline
Type of error                                                         & SEC & SECDED & Chipkill-like (SSC) \\ \hline \hline
Uncorrectable                                                      & 1.48e-05 & 1.48e-05  & 5.66e-05       \\ \hline
Undetectable                                                       & 1.48e-05 & 2.64e-08  & 5.66e-05       \\ \hline
\renewcommand{\arraystretch}{0.9}\begin{tabular}[c]{@{}l@{}}Detectable\\uncorrectable\end{tabular} & N/A        & 1.48e-05  & N/A              \\ \hline
\end{tabular}
\label{table:error-probability}
\end{adjustbox}
\end{center}
\vspace{-3mm}
\end{table}

\shepherd{From Table~\ref{table:error-probability}\atbcrcomment{2}{To me SEC
with no DED not having a detectable error probability makes sense. That is what
the dashes try to convey. Is my understanding wrong or can we use something
better than a dash?}, we observe that \phenomenon{}-induced bitflips could cause
uncorrectable errors with a relatively low probability based on the empirically
observed error rate using a safety margin of 10\%. Higher safety margins
($>$10\%) and ECC could prevent \phenomenon{}-induced \atbcr{4}{read
disturbance} errors given the \omcr{2}{limited measurement} dataset presented in
this work. However, a more detailed analysis of error rates is needed to make a
\omcr{2}{definitive} conclusion and doing so requires a dedicated large-scale
characterization study \omcr{2}{that performs many more measurements (e.g.,
millions or billions of RDT measurements over time) as opposed to 1K or
10K}.\atbcrcomment{3}{We do 10K for figure 16} We leave such \omcr{2}{a} study
\omcr{4}{to} future work.}

\subsection{\hpcarevcommon{Discussion and Future Work}}
\label{sec:discussion}

\omcr{2}{The major contribution of this work is the observation of the
\phenomenon{} phenomenon and its first \omcr{3}{experimental} characterization,
demonstrat\omcr{4}{ing} that a DRAM row's RDT \emph{cannot} be accurately and
easily (or efficiently) identified because it changes significantly and
unpredictably over time. Our results have implications for the security
guarantees of read disturbance mitigation techniques: if the RDT of a DRAM row
is \emph{not} identified accurately, these techniques become
insecure.}\atbcrcomment{4}{check capitalization after colon}

The implications of \phenomenon{} for read disturbance \omcr{3}{mitigation
techniques} resemble the implications of variable retention time (VRT) for DRAM
\atbcr{2}{retention-aware intelligent} refresh mechanisms
(e.g.,~\cite{liu2013experimental, khan2014efficacy, liu2012raidr,
baek2014refresh, wang2014proactivedram, khan2016parbor, patel2017reaper,
qureshi2015avatar, lin2012secret, nair2013archshield,
das2018vrl}).\atbcrcomment{3}{Added more} If previously undiscovered read
disturbance bitflips are a permanent possibility, then any approach to handling
\phenomenon{} will require \omcr{3}{tolerating some read disturbance bitflips in
the presence of VRD}, possibly \omcr{4}{via the use} of error-correcting codes or
message authentication codes (MACs)~\cite{qureshi2021rethinking,
juffinger2023csi}, \hpcarevcommon{and a guardband}. 

\hpcalabel{Main Questions 1\&2}\hpcarevcommon{While our experimental results
indicate that using \atbcr{2}{a $>$10\% guardband for the observed minimum RDT}
and ECC could \omcr{2}{potentially} \omcr{3}{(but likely not safely)} prevent
\phenomenon{}-induced read disturbance bitflips, \atbcr{2}{our results are
limited: we 1)~perform only 1K or 10K measurements (instead of millions or
billions), 2)~test a limited \omcr{3}{number}\omcr{4}{, type, and technology
node of DRAM chips} (160 DDR4 and four HBM2 chips\omcr{4}{;
Table~\ref{tab:dram_chip_list}}), and 3)~test a limited set of environmental
\atbcr{3}{conditions and process corners (e.g., voltage and temperature
variations)}.}
% our results are based on rigorous experimental testing of a \emph{limited}
% fraction of all DRAM chips in the field. 
Therefore, we \emph{cannot} guarantee that using a (large) guardband for RDT
\omcr{3}{along with} ECC would prevent all \phenomenon{}-induced bitflips.
Moreover, the effects of \phenomenon{} might continue to worsen with increasing
DRAM die density and with advanced DRAM technology
(Finding~\ref{finding:worsen-with-technology}) such that \phenomenon{}-induced
biflips become more costly to mitigate using a guardband and ECC.}

\omcr{2}{We believe there are at least three promising directions for future
work: 1)~gather more data, more comprehensively, by performing more RDT
measurements, testing more DRAM chips, and testing with a wider variety of
environmental \atbcr{3}{conditions} and \atbcr{3}{process \omcr{4}{corners (e.g., }voltage, and
temperature variations)}, 2)~}develop online RDT profiling mechanisms to
efficiently profile DRAM \atbcr{2}{chips} while \atbcr{2}{the chips are} in use
in order to mitigate the long RDT profiling times implied by our results, and
3)~develop new read disturbance \omcr{3}{mitigation techniques} that can
\atbcr{2}{dynamically configure their} read disturbance threshold by cooperating
with online profiling mechanisms. 

% AS we talk about online profiling, mention takeaways 3 and 4 as factors that
% make profiling difficult.

% Relatively few RDT measurements are un-
% likely to identify the minimum RDT value of a DRAM row.
% Estimations for the minimum RDT of a DRAM row can
% become more accurate with repeated RDT measurements.


% Thus, finding the RDT of a DRAM row and using a guardband to ensure 
% minimum RDT is covered may require a large guardband or may not work.

% In contrast, if a system designer chooses to apply an arbitrary ``guardband'' of 2$\times{}$ to the
% minimum measured RDT value after 



% \noindent
% \textbf{}

% these \nb{mitigations} prove their security guarantees
% assuming the minimum read disturbance threshold across all DRAM rows in a DRAM chip 
% is provided to the mitigation technique.

%\outline{Existing RowHammer \nb{mitigations} assume the lowest RDT is measured for the DRAM devices and mitigate read disturbance bitflips at the selected RDT value.}

%\outline{Our results show that measuring the RDT of a DRAM row is not trivial due to temporal changes in the lowest RDT of a DRAM row.}


%\outline{Based on our observations, it is necessary to measure the RDT of a DRAM row multiple times.}

% \nb{Existing RowHammer \nb{mitigations}~\mitigatingRowHammerAllCitations{}, including the state-of-the-art
% mitigations introduced in recent DDR5 standards~\cite{jedec2024jesd795c} \atbcomment{(SOMETHING ABOUT PRAC)}
% assume the RDT value of the DRAM devices in a system is known, and mitigate read disturbance bitflips by taking a preventive action at a threshold based on this RDT value (i.e., the preventive action threshold). This requires the minimum RDT value in the system to be accurately measured. Our extensive characterization of RDT values across many DDR4 and HBM chips shows that a DRAM row's RDT changes with time (Section~\ref{sec:foundational_results}), and any individual measurement of RDT can be \param{$\sim$2$\times$} higher than the minimum RDT value (Section~\ref{sec:indepth}). Therefore, setting the preventive action threshold for a mitigation technique is not a trivial task. }  
% %{Based on our results, measuring the RDT of a DRAM row multiple times increases the probability of finding the lowest RDT value. }

% \nb{Our results show that 1) measuring the RDT of a DRAM row only once results in a low probability of finding the minimum RDT value ($\sim$20\%), and 2) measuring the RDT of a DRAM row  \textit{multiple times} increases this probability. Therefore, to find the minimum RDT value, each DRAM row should be tested multiple times. Then, the system should set the preventive action threshold for \nb{mitigations} based on this minimum RDT value before executing any application to ensure correct operation (i.e., no read disturbance bitflips).}

% \nb{Our results show that the minimum RDT value that we determine by measuring the RDT for a limited number of times can be higher than the actual minimum RDT value by an error margin. For the DRAM chips we tested, this error margin is \param{$\sim$$2\times$}. Based on this observation, a characterization methodology can be designed as follows. First, we find an activation count (\texttt{T}) that results in read disturbance bitflips (i.e., $\leq$\texttt{T} activations cause bitflips). Second, we decrease this activation count based on the error margin and test DRAM rows again. Given the error margin is \param{$2\times$}, the second tested activation count should be half of the first one (\texttt{T/2}). We decrease the activation count in each iteration that we observe read disturbance bitflips. Finally, when we reach an activation count \texttt{T'} where no read disturbance bitflip is observed, we set the RDT to the next activation count, \texttt{T'/2}. This ensures that even if the first bitflip happens at \texttt{T'+$\epsilon$} (where $\epsilon\geq1$) activations, even with the error margin, the minimum RDT \texttt{(T'+$\epsilon$)/2} is greater than \texttt{T'/2}.  }


% \nb{For DRAM chips where the error margin is unknown,\nbcomment{or in case the error margin changes? maybe?} it is critical that the system can detect when a read disturbance bitflip happens. There are many prior works that detect (and recover) bitflips in DRAM~\cite{X,Y,Z}. The steps described above can be adapted to work with these detection mechanisms. In a system with detection capabilities, once a bitflip is detected, the system can characterize the DRAM chip quickly with the given methodology and update the RDT value to prevent other bitflips.}


% \nbcomment{check if these are addressed in the text above}
% \param{The infrequent but significant changes in RDT (e.g., the one shown in Fig 8) might imply
% that error tolerance against read disturbance errors are necessary.
% Once a read disturbance error is detected (and hopefully corrected), the configured
% read disturbance threshold should also reduce.}

% \param{If you want to configure your threshold at 100, what RDT value should you use
% for characterizing your chip? Will 50 cover all/majority of future RDT values?}

% \param{How many times should you test your row using the target RDT value?}

% \param{Can you get by without ECC? Assuming infrequent, significant changes to RDT occur
% on a longer timescale.}