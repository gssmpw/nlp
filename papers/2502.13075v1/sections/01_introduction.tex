\vspace{2mm}
\section{Introduction}
\vspace{2mm}
\label{sec:introduction}

\atbcrcomment{2}{double check all citations}\atbcomment{from svard}
\copied{Read
disturbance~\cite{kim2014flipping,
mutlu2019retrospective,mutlu2023fundamentally,luo2023rowpress, olgun2024read,
mutlu2017rowhammer} (e.g., RowHammer and RowPress) in modern DRAM chips is a
widespread phenomenon and is reliably used for breaking memory
isolation~\exploitingRowHammerAllCitations{}, a fundamental building block for
building robust systems.} \atbcomment{from HBM RD}\copied{Repeatedly
opening/activating and closing a DRAM row (i.e., aggressor row) \emph{many
times} (e.g., tens of thousands \omcr{1}{of times}) induces \emph{RowHammer
bitflips} in physically nearby rows (i.e., victim rows)~\cite{kim2014flipping}.
Keeping the aggressor row open for a long period of time 
% , i.e., a large \gls{taggon}, 
amplifies the effects of read disturbance and induces \emph{RowPress bitflips},
\emph{without} \omcr{1}{requiring} \emph{many} repeated aggressor row
activations~\cite{luo2023rowpress}.}

\atbcrcomment{3}{Triple check all numbers and consistency}
A large body of work~\mitigatingRowHammerAllCitations{} proposes various
techniques to mitigate DRAM read disturbance bitflips. Many high-performance and
low-overhead \agy{\omcr{3}{mitigation
techniques}}~\refreshBasedRowHammerDefenseCitations{}\omcr{1}{, including
\atbcr{3}{those that are} used and standardized by industry~\cite{jedecddr5c,
bennett2021panopticon, canpolat2024understanding, kim2023ddr5,
canpolat2025chronus}}, prevent read disturbance bitflips by
\agy{\emph{preventively}} refreshing (i.e., opening and closing) a victim row
\emph{before} 
a bitflip manifests in that row. 
% Since preventive refresh
% \agy{operations induce} increasing and significant performance and energy
% overheads with worsening read disturbance
% vulnerability~\rowHammerDefenseScalingProblemsCitations{}, a secure,
% high-performance, and energy-efficient memory system should perform a preventive
% refresh \emph{only if} the operation is necessary.

To securely prevent read disturbance bitflips at low performance and energy
overhead, it is important to \emph{\omcr{1}{accurately}} identify {the amount of
read disturbance} that a victim row can withstand before \nb{experiencing} a
read disturbance bitflip. This amount is typically quantified using the
\emph{hammer count (the number of aggressor row
activations)\agyurgentcomment{this definition is inconsistent with the
definition in the methodology}\atbcomment{Assuming this is fixed now.} needed to
induce the first read disturbance bitflip} in a victim row. We call this metric
the \emph{read disturbance threshold (RDT)} of the victim row. 

% \oc{Different} DRAM rows experience read disturbance bitflips at different
% RDTs due to process variation. Taking advantage of this fact, prior work has
% aimed to reduce the effect of preventive refresh operations on performance and
% energy by adapting the aggressiveness of existing read disturbance mitigations
% based on the row-level read disturbance
% profile~\cite{yaglikci2024spatial,orosa2021deeper}. By knowing the exact read
% disturbance threshold of each DRAM row, the techniques proposed by prior work
% preventively refresh DRAM rows with high read disturbance threshold
% \emph{\omcr{1}{less frequently}}, which results in both performance and energy
% benefits over the existing \agy{mitigations} that preventively refresh all
% rows based on the smallest read disturbance threshold across the DRAM module.

\oc{Prior read disturbance \agy{\omcr{3}{mitigation techniques}} (including
those that are \omcr{4}{used and} standardized \omcr{1}{by industry}, e.g.,
PRAC~\cite{jedecddr5c, bennett2021panopticon, canpolat2024understanding,
kim2023ddr5, canpolat2025chronus}) assume \omcr{2}{that} one can
\omcr{1}{accurately (}and hopefully efficiently) identify the \hpcarevb{minimum}
RDT \hpcarevb{across all DRAM rows \omcr{1}{in a
chip}\hpcalabel{B2}.\footnote{\hpcarevb{Identifying the minimum RDT across all
rows requires independently identifying the RDT of \emph{\omcr{2}{every}} DRAM
row and finding the minimum value. \omcr{2}{This is because} RDT
\emph{\omcr{2}{spatially}} varies across rows in an unpredictable
way~\cite{yaglikci2024spatial}.}}} In this work, we \agy{challenge} this
assumption and ask: \emph{how \omcr{1}{accurately}\agycomment{reliably or
precisely?}\atbcomment{reliable is good imo} and efficiently can we measure the
RDT of each row?} \agy{To this end, we rigorously characterize
\param{\numDDRchips{}} DDR4 and \param{\numHBMchips{}} HBM2 DRAM chips and
experimentally demonstrate that a DRAM row's RDT}
% Based on rigorous characterization, we demonstrate that the RDT of a single
% row 
\emph{cannot} be \omcr{1}{accurately} \omcr{1}{and easily (\omcr{2}{or
efficiently)}} identified because it\agyurgentcomment{if it cannot be measured,
how come we measure it? I suggest removing the part ``\emph{cannot} be
\omcr{1}{accurately} measured because it''} \omcr{2}{\emph{changes significantly
and unpredictably over time}}. This is a phenomenon that has not been discussed
in prior literature\gra{,} and it resembles the variable retention time (VRT)
phenomenon~\vrtCitations{} in \agy{DRAM cells' data} retention times. We call
this new phenomenon \emph{\omcr{2}{variable read disturbance (\phenomenon{})}}. 
% In this work, we experimentally demonstrate the prevalence of \phenomenon{} in
% \param{80} modern DRAM chips (from 10 DDR4 modules and 4 HBM2 chips) and
% examine its major characteristics.
} 


% \atbcomment{may not be needed}\oc{Unfortunately,
% in this paper, we show that...}

% Prior read disturbance \agy{mitigations} assume that the RDT of DRAM rows
% can be \oc{reliably and accurately} measured or profiled. In this work, we show that...
% Our observations yield a key difficulty in DRAM RDT profiling that 
% challenges the assumption made by prior \agy{mitigations}. 

% Our \textbf{goal} is to understand how reliably the RDT of a 
% victim row can be identified (in a way that yields consistently 
% accurate RDT values for the row). To this end, we experimentally 
% characterize the read disturbance vulnerability in \param{X} DDR4 modules and \param{Y} HBM2
% chips from three major DRAM manufacturers. Our characterization yields \param{Z} important observations. 
% We discuss the implications of our observations for existing read disturbance \agy{mitigations}. 
% We develop and evaluate a new, lightweight and configurable, read disturbance mitigation 
% configuration methodology.


\noindent
\textbf{Experimental Characterization.} 
We perform a two-step characterization study whereby we investigate the
variation in \atbcr{4}{RDT} across 1)~\param{100,000} repeated tests for one
DRAM row in each tested DRAM chip, to draw foundational results for
\phenomenon{}, and 2)~\param{1,000} repeated tests for many DRAM rows, to
conduct an in-depth analysis of \phenomenon{} using many data patterns,
\gls{taggon} values~\cite{luo2023rowpress}, and temperatures. Based on our novel
characterization results, we \omcr{1}{provide} \atbcr{4}{17} new findings and
share \param{\omcr{1}{four}} key takeaway lessons. Our takeaway lessons have
strong implications for the security guarantees and system performance, energy,
and area overheads of read disturbance solutions: \atbcr{1}{if the RDT of a DRAM
row is \emph{not} identified accurately, these techniques can easily become
insecure.}

From our \param{\atbcr{4}{17}} new findings, we \omcr{1}{hereby} highlight
\param{\omcr{1}{six}} findings that are especially important. First, the RDT of
a DRAM row changes with repeated RDT measurements (i.e., changes over time).
\figref{fig:motivation_example} shows \param{100,000} successive measurements
(x-axis) of the RDT (y-axis) of a victim row as an \omcr{1}{illustrative}
example. For the victim row shown in~\figref{fig:motivation_example}, we find
the smallest RDT value after a relatively high number of \param{16,926}
measurements. Across all tested DRAM rows from all tested DRAM chips, the
smallest RDT value can appear after \param{94,467} measurements.
\atbcr{2}{94,467 \atbcr{3}{RDT measurements} take}\atbcrcomment{3}{the math here
checks out and the results are different from what is mentioned in the later
section. is that not OK?} {approximately {9.5} seconds for a single DRAM row
with a relatively small average read disturbance threshold of 1,000 {(see
Appendix~\ref{app:testing} for RDT test time estimation methodology details)}.
Thus, exhaustive \phenomenon{} characterization of all DRAM rows in
\atbcr{3}{one bank of} a chip easily becomes prohibitively time-intensive:
\omcr{2}{approximately} 2{9} days if the RDT of all DRAM rows (in \omcr{4}{even
only a single} bank of 256K rows) is measured 94,467 times.\atbcrcomment{4}{the
results here are the worst in terms of testing
time}}\footnote{\omcr{2}{Unfortunately, due to the unpredictable nature of RDT,
one would \emph{not} know when to stop testing. Therefore, the worst-case
testing time can be much longer than what we showcase here \omcr{2}{(and in the
rest of the paper)}.}}
% We investigate the distribution and the predictability of 100,000 RDT
% measurements of one DRAM row in all tested DRAM chips and find that 

% ~\figref{fig:motivation_example}
% shows that RDT must be measured many times to observe the lowest RDT.

\begin{figure}[!ht]
    \centering
    \includegraphics[width=\linewidth]{figures/motivational_figure.png}
    \caption{Read disturbance threshold (RDT) of a DRAM row over 100,000
    repeated measurements (left). \atbcr{1}{Each circle and error bar pair}
    shows the distribution of RDT across \param{1,000} successive measurements.
    The circles show the mean RDT and the error bars show the range of measured
    RDT values. Zoomed-in view \atbcr{5}{of} the RDT of the row over the last
    \param{1,000} measurements (right).}
    \label{fig:motivation_example}
\end{figure}

Second, we analyze the distribution and predictability of the series of
\param{100,000} RDT measurements. We find that the RDT of the tested victim rows
in \emph{\omcr{1}{all}} tested DRAM chips changes significantly and
\emph{unpredictably} \omcr{2}{over} time. Therefore, even \emph{many} RDT
measurements (e.g., as many as 16,926 for the DRAM row
in~\figref{fig:motivation_example}) are \emph{not} enough to
\agy{\omcr{1}{absolutely guarantee that}} the lowest RDT is found.

\atbcr{1}{Third, we find that a very large fraction (97.1\%) of all tested DRAM
rows exhibit \omcr{2}{VRD} across \emph{all} combinations of test
parameters (temperature, \gls{taggon}, and data pattern). The remaining 2.9\% of
the tested rows exhibit VRD for at least one combination of test parameters. We
highlight the worst-case temporal variation observed across all DRAM rows: the
maximum observed RDT for a DRAM row is \param{3.5$\times{}$} higher than the
minimum observed RDT for that row across 1,000 RDT measurements.}

\atbcr{1}{Fourth}, relatively few RDT measurements do \emph{not} yield the
minimum RDT expected to be found by many RDT measurements. For
example, \atbcr{1}{for} a significant fraction \atbcr{1}{(\param{22.4\%})} of
all tested DRAM rows, \atbcr{1}{\emph{only}} one measurement \atbcr{1}{across
1,000 measurements (\atbcr{2}{per row})} yields the minimum RDT.
\atbcr{1}{The maximum RDT value for such a DRAM row whose RDT is \omcr{2}{very}
difficult to accurately identify can be as high as \param{1.9}$\times{}$ the
minimum RDT value observed for that row across 1,000 measurements.} 

% exhibit \atbcr{1}{up to}
% \param{1.9$\times{}$} difference between their minimum and maximum RDT values
% across \param{1,000} measurements, and \emph{only} \param{one} measurement
% \atbcr{1}{across all 1,000 measurements} yields the minimum RDT for
% \omcr{1}{each of} these DRAM rows \atbcr{1}{(i.e., the probability of
% identifying the minimum RDT is small, \emph{only} 0.001)}.

\atbcr{1}{Fifth}, we find that \omcr{2}{the \phenomenon{} phenomenon gets worse
in} higher-density \atbcr{1}{DRAM chips} or \atbcr{1}{DRAM chips manufactured
using more} advanced technology node\atbcr{1}{s} (as indicated by the die
revision). \atbcr{1}{For example, with \omcr{2}{\emph{only}} one RDT
measurement, we expect to find an RDT value that is \atbcr{4}{\param{6\%}}
higher \atbcr{2}{(on average across all tested DRAM rows)} than
\atbcrcomment{3}{You found this part hard to understand. I do not have a
solution.}\atbcr{2}{the minimum observed RDT value across 1,000 measurements for
the least advanced tested DDR4 chips from one manufacturer. In contrast,
\atbcr{2}{for the most advanced tested DDR4 chips from the same manufacturer},
with \emph{only} one RDT measurement, we expect to find an RDT value that is
\atbcr{4}{\param{8\%}} higher than the minimum observed RDT value across 1,000
measurements.}}


% the least advanced and the most advanced DRAM
% chip} the \omcr{1}{lower} the probability of identifying the minimum RDT of a
% DRAM row across \param{1,000} measurements.\omcrcomment{1}{Are you saying 1000
% measurements is not enough? Unclear.}

\atbcr{1}{Sixth}, \phenomenon{} can change with data pattern, aggressor row on
time, and temperature. Therefore, producing a comprehensive \phenomenon{}
profile \agy{of} a DRAM \agy{row} requires testing \agy{\omcr{1}{each} row
\omcrcomment{1}{quantify better}\atbcrcomment{1}{What should we quantify here?
What is interesting to see? E.g., an example range of RDT values to quantify
variation for one test parameter?} many times for each of many different} data
patterns, aggressor row on time values, and temperatures.

% Second, the RDT of a victim row changes significantly and unpredictably over time. 
% The minimum RDT of a victim row is \param{2.8}$\times{}$ smaller than the maximum
% RDT of the row over 1,000 successive RDT measurements. 

\noindent
\omcrcomment{1}{Writing in this paragraph is degraded. Fix. Make things stronger
and less unconfident}\omcr{1}{\textbf{Implications for System \omcr{2}{Security
and} Robustness.}} \omcr{1}{\phenomenon{} threatens the security and robustness
of existing and future read disturbance \omcr{3}{mitigation techniques}.} The
security \omcr{1}{and robustness} guarantees provided by prior read disturbance
\agy{\omcr{3}{mitigation techniques}} rely on accurately identified RDT
\omcr{1}{values} across all DRAM rows in a computing system. Our results show
that accurately identifying the minimum RDT across all DRAM rows, even with
\omcr{1}{thousands of} RDT measurements \omcr{1}{for each row}\gra{,} is
\hpcarevcommon{challenging} because the RDT of a row changes
\omcr{1}{unpredictably over} time \atbcr{1}{and exhaustively testing all DRAM
rows many times to uncover all possible bitflips \omcr{2}{in the presence of
\phenomenon{} is prohibitively time-intensive}. Therefore, \agy{an} approach to
handling \phenomenon{} will likely \agy{require} \omcr{2}{tolerating some}
\atbcr{2}{read disturbance bitflips in the presence of
\phenomenon{}}}.\omcrcomment{1}{too general.}\atbcrcomment{1}{Not sure what to
add. We could remove the sentence altogether?} \hpcalabel{Main Questions
1\&2}\hpcarevcommon{We evaluate the effectiveness and performance overheads of
using guardbands (e.g., by \atbcr{2}{configuring read disturbance
\omcr{3}{mitigation techniques} with an RDT value that is smaller than} the
minimum observed RDT) and error-correcting codes (ECC) \omcr{1}{at} mitigating
\phenomenon{}-induced bitflips (\secref{sec:implications}). Our results suggest
that \atbcr{2}{a $>$10\% guardband} and \omcr{1}{single-error-correcting
double-error-detecting \atbcr{2}{(SECDED~\cite{kim2015bamboo}) or Chipkill-like
(e.g., single symbol correction~\cite{amd2013sddc,yeleswarapu2020addressing,
chen1996symbol})}} ECC could prevent \phenomenon{}-induced bitflips at the cost
of \omcr{2}{relatively} \omcr{2}{large} performance overheads \omcr{1}{caused}
by read disturbance mitigation techniques \atbcr{1}{(e.g.,
45\%\atbcrcomment{3}{these results are highlighted in sec6} overhead for a
state-of-the-art probabilistic read disturbance mitigation technique assuming an
RDT guardband of 50\% \atbcr{2}{and 5.9\% overhead assuming a guardband of
10\%})}}.\footnote{\atbcr{2}{Our evaluation of the effectiveness and performance
overheads of guardbands and error-correcting codes is based on a limited number
of RDT measurements and a limited \omcr{3}{set of DRAM chips\omcr{4}{, types,
and} technology nodes} (see~\secref{sec:discussion}). We \emph{cannot} guarantee
that using a large guardband for RDT \omcr{3}{together with} ECC would prevent
all read disturbance bitflips in presence of \phenomenon{}.}} \omcr{1}{We call
for} future work on online RDT profiling and runtime configurable read
disturbance \agy{\omcr{3}{mitigation techniques}} \omcr{1}{to} remedy the
challenges imposed \omcr{2}{by \phenomenon{}} on read disturbance
\omcr{2}{mitigation techniques}.

Our work makes the following contributions:
\begin{itemize}
    \item We demonstrate that the read disturbance threshold (RDT) of a DRAM row
    \emph{cannot}\agyurgentcomment{how come we can measure something that cannot
    be measured?} be reliably\agyurgentcomment{reliably or precisely?}
    identified because it changes significantly and unpredictably over time. We
    call this phenomenon \emph{\omcr{1}{variable read disturbance
    (\phenomenon{})}}\gra{.}
    % \agycomment{this is the most important one. It deserves to be the first
    % contribution}. 
    We show the \agy{prevalence} of \phenomenon{} in modern DRAM chips.
    \item We present the first detailed experimental characterization of the
    temporal variation in DRAM read disturbance (RowHammer and RowPress) in
    state-of-the-art DDR4 and HBM2 DRAM chip{s}. \agycomment{so, is there a
    prior work that showed it in 79 state-of-the-art DDR4 and HBM2 DRAM chips?
    Let's make this more high-level and independent of the
    methodology.}\atbcomment{assuming this is fixed.}

    \item We examine \phenomenon{}'s major \atbcr{1}{properties}. \atbcr{1}{One
    or few RDT measurements do \emph{not} yield an RDT value that is identical
    or very close to the minimum RDT value observed for the same DRAM row across
    1,000 measurements.}
    % \phenomenon{} makes
    % identifying \agy{a DRAM row's} RDT difficult as it causes many DRAM rows to
    % exhibit a minimum RDT that is almost \agy{half the mean RDT and}
    % % twice as smaller than the row's mean RDT where this minimum RDT 
    % may appear \agy{\emph{only}} once out of many repeated measurements. 
    This \atbcr{1}{property} of \phenomenon{} worsens with increasing chip
    density and more advanced technology nodes.\omcrcomment{1}{hard to follow}
    % \phenomenon{} makes it more unlikely for a single RDT measurement to
    % identify the minimum RDT in a DRAM row for DRAM chips that have a higher
    % density or that are manufactured using more advanced technology nodes.
    \item We demonstrate that \phenomenon{} can change with data pattern,
    aggressor row on time, and temperature.
    \item We discuss \phenomenon{}'s implications for \omcr{1}{the security and
    robustness of existing and future} read disturbance \agy{\omcr{3}{mitigation techniques}}.
    \omcr{2}{We propose} \omcr{2}{and analyze using guardbands together with ECC
    to reduce the impact of \phenomenon{}.} 
    \item Our takeaway lessons call
    \omcr{1}{for} future work on efficient online RDT profiling, configurable
    read disturbance \omcr{3}{mitigation techniques}, \atbcr{1}{other innovative solutions to
    mitigate \phenomenon{}-induced bitflips}\omcr{1}{, and understanding the
    underlying device-level causes of \phenomenon{}.}
\end{itemize}