\appendix
\onecolumn
\section{Illustrations of \tool}
\subsection{A Complete Example}
This section showcases how \tool solves a code generation task using an LLM (e.g., deepseek V3 in this example). The question comes from Mbpp 77.
In the initial response, \tool simply uses direct prompting to ask the LLM to write a program for the given specification.

\textbf{Original Attempt.} \darkblue{
\textbf{[User]} You are an expert in Python coding. Wrap your program in a $<$code$>$ $<$/code$>$ block. No more test cases or any other contents.\\
\\
Specification: Write a function to find the Eulerian number a(n, m), the entry point is ``eulerian\_num''.\\
assert eulerian\_num(3, 1) == 4 \\
}

\green{
\textbf{[AI]} $<$code$>$ \\
$1\quad$def eulerian\_num(n, m):\\
$2\quad\quad$if m $<$ 0 or m $>=$ n: \\
$3\quad\quad\quad$return 0 \\
$4\quad\quad$if n == 0: \\
$5\quad\quad\quad$return 1 \\
$6\quad\quad$return (n - m) * eulerian\_num(n - 1, m - 1) + (m + 1) * eulerian\_num(n - 1, m) \\
$<$/code$>$ \\
}

Afterward, we extract Control Flow Graph (CFG) from the initial program, and then we transverse the CFG to transform it into text by describing the logic flow of the code statements. Therefore, we obtain a list of logic units:
\begin{itemize}
    \item   \#ENTER FUNCTION\# eulerian\_num \\
    \#BRANCH\# If Condition `if m $<$ 0 or m $>=$ n' is satisfied, then RUN `return 0'

    \item \#BRANCH\# Otherwise, when Condition `if m $<$ 0 or m $>=$ n' is not satisfied, then \#BRANCH\# If Condition `if n == 0' is satisfied, then RUN `return 1'
    
    \item \#BRANCH\# Otherwise, when Condition `if n == 0' is not satisfied, then RUN `return (n - m) * eulerian\_num(n - 1, m - 1) + (m + 1) * eulerian\_num(n - 1, m)' \\ \#EXIT FUNCTION\#
\end{itemize}

Then, \tool initializes a continuous dialogue with the same LLM. In each turn of interaction, the LLM is promoted to judge the correctness of the current logic unit. If this unit is determined as wrong, the LLM will generate the fixed version simultaneously.

\textbf{U1-I.} \darkblue{\textbf{[User]} \#\# Process \\
Unit 1: \#ENTER FUNCTION\# eulerian\_num \\
\#BRANCH\# If Condition `if m $<$ 0 or m $>=$ n' is satisfied, then RUN `return 0'
}

\green{\textbf{[AI] (confidence: 96.50)} OK \\
Analysis: This step correctly handles the edge case where `m' is less than 0 or greater than or equal to `n'. According to the specification, the Eulerian number `a(n, m)' should return 0 in such cases. The logic aligns with the requirement, so this step is correct.
}

The correctness of unit 1 is checked by the LLM.

\textbf{U2-I.} \darkblue{\textbf{[User]} Unit 2: \#BRANCH\# Otherwise, when Condition `if m $<$ 0 or m $>=$ n' is not satisfied, then \#BRANCH\# If Condition `if n == 0' is satisfied, then RUN `return 1'
}

\green{\textbf{[AI] (confidence: 98.96)} WRONG  \\
$<$Fix$>$\\
\#BRANCH\# Otherwise, when Condition `if m $<$ 0 or m $>=$ n' is not satisfied, then \#BRANCH\# If Condition `if m == 0' is satisfied, then RUN `return 1' \\
$<$/Fix$>$ \\
Analysis: This unit incorrectly checks for the condition `if n == 0'. According to the specification and the logic of Eulerian numbers, when `m == 0', the function should return 1. The condition should be checking if `m == 0', not `n == 0'. This correction aligns with the requirement.
}

Herein, the reasoning path branches off by fixing the original logic unit. Then we rewind and replace unit 2 with the fixed one:

\textbf{U2-II.} \darkblue{\textbf{[User]} Unit 2: \#BRANCH\# Otherwise, when Condition `if m < 0 or m >= n' is not satisfied, then \#BRANCH\# If Condition `if m == 0' is satisfied, then RUN `return 1'
}

\green{\textbf{[AI] (confidence: 97.61)} OK \\
Analysis: This unit correctly handles the case where `m == 0'. According to the specification, when `m == 0', the Eulerian number `a(n, m)' should return 1. The logic aligns with the requirement, so this unit is correct.
}

\textbf{U3-I.} \darkblue{\textbf{[User]} Unit 3: \#BRANCH\# Otherwise, when Condition `if n == 0' is not satisfied, then RUN `return (n - m) * eulerian\_num(n - 1, m - 1) + (m + 1) * eulerian\_num(n - 1, m)'
\#EXIT FUNCTION\#
}

\green{\textbf{[AI] (confidence: 98.70)} WRONG  \\
$<$Fix$>$\\
\#BRANCH\# Otherwise, when Condition `if m == 0' is not satisfied, then RUN `return (n - m) * eulerian\_num(n - 1, m - 1) + (m + 1) * eulerian\_num(n - 1, m)` \\
\#EXIT FUNCTION\# \\
$<$/Fix$>$ \\
Analysis: The unit incorrectly checks for `if n == 0', which is already handled in the first branch (`if m $<$ 0 or m $>=$ n'). Instead, it should proceed to calculate the Eulerian number using the recursive formula when `m != 0'. The corrected step ensures that the recursive calculation is performed only when `m != 0', aligning with the specification.
}

In the dialogue round for unit 3, given that the previous historical conversation could serve as a priori conditions for the LLM, it successfully carried forward the modifications made to unit 1, thus ensuring the consistency of the context.

\textbf{U3-II.} \darkblue{\textbf{[User]} Unit 3: \#BRANCH\# Otherwise, when Condition `if m == 0' is not satisfied, then RUN `return (n - m) * eulerian\_num(n - 1, m - 1) + (m + 1) * eulerian\_num(n - 1, m)`
\#EXIT FUNCTION\#
}

\green{\textbf{[AI] (confidence: 98.71)} OK
Analysis: This unit correctly implements the recursive formula for calculating the Eulerian number `a(n, m)' when neither of the previous conditions (`m $<$ 0 or m $>=$ n' or `m == 0') is satisfied. The logic aligns with the requirement, as it uses the correct recurrence relation:  \\
`(n - m) * eulerian\_num(n - 1, m - 1) + (m + 1) * eulerian\_num(n - 1, m)'.  \\
This step is logically correct and adheres to the specification.
}

After generating a self-checked and self-corrected reasoning path, we take it as a history conversation. Note that the original program is not included in case of confusion.

\textbf{Final Synthesis.} \darkblue{\textbf{[User]}
You are an expert in Python coding. Your task is to write a correct program to meet the given specification based on the previous conversation. Your program can contain several functions, but make sure the main entry point is eulerian\_num. \\
Strict Requirement: Ensure to return the complete function wrapped in a $<$code$>$ $<$/code$>$ block! No test cases! \\
\\
Specification: Write a function to find the Eulerian number a(n, m), the entry point is ``eulerian\_num''.\\
assert eulerian\_num(3, 1) == 4 \\
}

\green{
\textbf{[AI]} $<$code$>$ \\
$1\quad$def eulerian\_num(n, m):\\
$2\quad\quad$if m $<$ 0 or m $>=$ n: \\
$3\quad\quad\quad$return 0 \\
$4\quad\quad$elif m == 0: \\
$5\quad\quad\quad$return 1 \\
$6\quad\quad$return (n - m) * eulerian\_num(n - 1, m - 1) + (m + 1) * eulerian\_num(n - 1, m) \\
$<$/code$>$ \\
}

By changing the condition from `if n == 0' to `elif m == 0', the program is successfully corrected. Through fine-grained alignment (among the requirement in natural language, step descriptions, and the generated program), \tool can effectively correct subtle logical errors. Moreover, by taking the previous conversations as conditions, it can better prevent the propagation of errors from previous steps.

The trajectories will be made publically available as soon as the end of anonymity, where all the cases using \tool can be shown directly.


\subsection{CFG-Driven Logic Units of Code}\label{app:example:CFG}
Figure~\ref{fig:example:CFG} displays an example of transforming a program into a CFG and how \tool organizes the graph to obtain a linear sequence of logic units.
To enhance interpretability, units are annotated with human-readable descriptors (e.g., ``LOOP BEGIN: Process each item in list X''). These labels bridge low-level code operations with high-level problem-solving intent, priming subsequent alignment stages.

\begin{figure*}[htbp]
    \centering
    \vspace{-0.1in}
        {\includegraphics[width=0.8\linewidth]{figures/CFG-example.pdf}}
    \vspace{-0.1in}
    \caption{This figure depicts what a CFG looks like and how it is transformed into units. Nodes and edges in the CFG represent code blocks and control flow transitions, respectively. \tool divides the CFG into atomic logic units at key points, with each unit labeled for better understanding.}
    \vspace{-0.2in}
    \label{fig:example:CFG}
\end{figure*}


\subsection{Condition of Applying Self-Repair}\label{app:RaLU:repair}
The correctness of a unit $\mathcal{U}$ involves two conditions:
First, the LLM believes $\mathcal{U}$ is correct and $\mathcal{U}$ is actually correct, then we have:
\begin{equation}
    P(\tilde{\mathcal{U}} \text{ is correct} | J = \texttt{OK}) = \alpha p.
\end{equation}
Second, the probability of  $\mathcal{U}$ being judged as wrong is (true negative rate plus false positive rate):
\begin{equation}
    P(J=\texttt{WRONG}) = (1-\alpha)p + (1-\beta)(1-p).
\end{equation}
Then, the probability of correctly repairing the unit is:
\begin{equation}
    P(\tilde{\mathcal{U}} \text{ is correct} | J = \texttt{WRONG}) = \gamma_{repair} \cdot [(1-\alpha)p + (1-\beta)(1-p)].
\end{equation}.
Thus, we can rewrite $p'$ as:
\begin{equation}
    p' = \alpha p + \gamma_{repair} \cdot [(1-\alpha)p + (1-\beta)(1-p)].
\end{equation}
To compare $p'$ and $p$, we have:
\begin{equation}
\begin{split}
    p' - p &= -p(1-\alpha) + \gamma_{repair} \cdot [(1-\alpha)p + (1-\beta)(1-p)] \\
    &= \underbrace{[(1-\alpha)p + (1-\beta)(1-p)]}_{P(J=\texttt{WRONG)}} \cdot (\gamma_{repair} - \frac{(1-\alpha)p}{(1-\alpha)p + (1-\beta)(1-p)}.
\end{split}
\end{equation}
Note that the first term is the probability of judging the unit as \texttt{WRONG} so that it is always positive. The condition of $p'-p>0$ is then transformed to:
\begin{equation}\label{eq:repair}
    \gamma_{repair} > \frac{(1-\alpha)p}{(1-\alpha)p + (1-\beta)(1-p)}.
\end{equation}
Note that $P(\mathcal{U} \text{ is correct} | J=\texttt{WRONG}) = \frac{P(\mathcal{U}  \text{ is correct} \land J=\texttt{WRONG})}{P(J=\texttt{WRONG})}$, that is:
\begin{equation}
    P(\mathcal{U} \text{ is correct} | J=\texttt{WRONG}) = \frac{(1-\alpha)p}{(1-\alpha)p + (1-\beta)(1-p)}.
\end{equation}
equal to the right part of equation~\ref{eq:repair}, so the condition of $p'>p$ is $\gamma_{repair} > P(\mathcal{U} \text{ is correct} | J=\texttt{WRONG})$.

\section{Details of Experiment Setup}
\subsection{Benchmarks}\label{app:datasets}
\begin{itemize}
    \item GSM8K~\cite{GSM8K} is a widely recognized benchmark to evaluate the reasoning and problem-solving capabilities of LLMs, whose name stands for ``Grade School Math 8K," reflecting its focus on grade school-level math problems. The dataset contains approximately 8,500 carefully crafted math problems. Each problem in GSM8K is presented as a word problem, typically involving basic arithmetic operations (addition, subtraction, multiplication, and division) and sometimes simple algebraic concepts.
    
    \item MATH~\cite{MATH} is proposed as a comprehensive benchmark designed to assess the mathematical reasoning capabilities of LLMs. It comprises 12,500 competition mathematics problems, which are carefully curated to cover a wide range of mathematical concepts and varying levels of difficulty.
    We use a subset taken from~\cite{DeductiveVeriCoT} named MATH-np, specifically tailored to assess the deductive reasoning skills of LLMs. It includes problems that require multi-step reasoning and the application of mathematical concepts in a structured manner.

    \item HumanEval~\cite{HumanEval} is a benchmark dataset to evaluate the code generation capabilities of LLMs, introduced by OpenAI. It consists of 164 hand-written Python programming problems, each with a problem specification (prompt), a predefined function signature, and a set of test cases. The primary metric used to evaluate model performance is the pass@$k$ metric, which measures the percentage of tasks for which at least one of the $k$ generated code samples passes all the test cases. Note that we only report pass@1.

    \item Mbpp~\cite{Mbpp}, or Mostly Basic Python Problems, is a benchmark designed to evaluate the program synthesis capabilities of LLMs, consisting of over 900 Python programming tasks, whose problems share the same structure with that of HumenEval. It covers a wide range of basic to moderately complex Python programming problems.

    \item HumanEval+/Mbpp+~\cite{EvalPlus} come from EvalPlus, a rigorous evaluation framework designed to assess the performance of LLMs in code generation by expanding the test cases of well-known benchmarks such as HumanEval and MBPP. It also maintains a leaderboard to track and compare the performance of various LLMs.
\end{itemize}

\subsection{Baselines}\label{app:baselines}
In our experiments, we reproduce the baselines strictly following their released code and prompts.
\begin{itemize}
    \item Chain-of-Thought (CoT)~\cite{CoT} involves instructing the model to "think step by step" before arriving at a final answer. It enhances the reasoning capabilities of LLMs by explicitly guiding them to break down complex problems into a series of logical, intermediate steps. This approach mimics human reasoning by decomposing a problem into smaller, manageable sub-problems and solving them sequentially.

    \item Thee-of-Thought (ToT)~\cite{ToT} enhances LLM reasoning capabilities by simulating human problem-solving strategies. ToT breaks down the problem-solving process into smaller, manageable, intermediate steps called ``thoughts." For each state in the thought tree, the LLM generates multiple potential next thoughts. Each generated thought is evaluated for its potential to lead to a solution. Then, it employs search algorithms such as Breadth-First Search (BFS) or Depth-First Search (DFS) to explore the thought tree systematically. The structured nature of the thought tree makes the reasoning process more transparent and interpretable.

    \item Program-of-Thought (PoT)~\cite{PoT} or its similar approach Program-Aided Language Model (PAL)~\cite{PAL} both represent a novel approach that combines the strengths of LLMs with the precision of programming languages. The LLM reads a natural language problem and generates a program as the intermediate reasoning step. They aim to decompose reasoning and computing by offloading the solution step to a symbolic interpreter, which leverages the LLM's reasoning abilities while mitigating its weaknesses in logical and arithmetic operations.
    PoT or PAL will degrade to direct prompting in the face of program generation tasks.

    \item Self-Consistency (SC)~\cite{Self-Consistency} is a decoding strategy designed to improve the accuracy and reliability of reasoning processes. It involves generating multiple reasoning paths for a given problem and selecting the most consistent answer among them (majority voting). The consistency can be directly computed (for numerical calculation tasks) or determined by LLMs (either the same LLM or another LLM).

    \item Self-Calibration (SCal)~\cite{Self-Calibration}  an advanced prompting technique designed to enhance the accuracy and reliability of LLMs by enabling them to evaluate their own outputs. The LLM generates an initial answer to a given question, and it is prompted to assess the correctness of its own response.  

    \item Self-Refine (SR)~\cite{Self-Refine} is an iterative refinement technique designed to enhance the output quality of LLMs by incorporating self-generated feedback. Specifically, the LLM generates an initial response to a given prompt, and the same LLM evaluates the initial output and provides actionable feedback, identifying areas for improvement. With the feedback, the same LLM refines the initial output, aiming to improve its quality. This response-feedback-refine pipeline can be repeated multiple times until the output meets a predefined stopping criterion.

    \item Self-Debugging (SD)~\cite{Self-Debug} is an innovative technique designed to enable LLMs to identify and correct errors in the code they generate without requiring additional model training or human intervention. This method is inspired by the "rubber duck debugging" technique used by human programmers, where explaining code line-by-line in natural language helps identify and fix errors. Since it targets program bugs, it cannot be directly applied to mathematical reasoning tasks.

    \item Self-Check (SCheck)~\cite{Self-Check} is a prompting technique that enables LLMs to evaluate their own reasoning and identify errors in their step-by-step solutions. It first provides several step-by-step solutions through CoT prompting. Then, it identifies the relevant context and target for each step in its reasoning process. Afterward, the LLM generates an independent alternative step based on the extracted context. The original step is compared with the regenerated alternative. If they match, the original step is deemed correct. The reasoning path with the most ``correct'' steps will be selected (weighted majority voting).
    
\end{itemize}


% \section{Additional Experiments}
\section{Additional Comparisons with Closed-Source LLMs}\label{app:exp:close-source}

As shown in Table~\ref{tab:exp:wLLM}, in cross-domain benchmarks (mathematical reasoning and code generation), \tool exhibits better reasoning capabilities to mainstream closed-source models (i.e., GPT-4o, GPT-4-Turbo, and Claude-Sonnet-3.5) and significantly outperforms GPT-3.5-Turbo (+38.16\% on average).

\tool achieves the highest scores on the extended versions of code generation benchmarks (HumanEval+/Mbpp+), despite its slightly lower performance on original HumanEval/MBPP ($\Delta$=-2.24\%).
This inversion reveals a critical insight that unit-level correction benefits to solving multi-constraint tasks, since the augmented test suites introduced by the plus version of benchmarks require models to simultaneously satisfy competing constraints.
While closed-source models often overfit to dominant patterns in pretraining data, \tool's unit-level rewind mechanism enables iterative constraint alignment. This explains the 9.46\% improvement on mathematical reasoning benchmarks where LLMs struggle with numerical computation.

\begin{table*}[htb]\centering
    \caption{\tool using open-sourced LLMs (DeepSeek-V3 or Qwen2.5-72B) can even outperform state-of-the-art closed-source LLMs, whose results are reported by previous studies. It achieves SoTA results on both mathematical reasoning benchmarks by making an improvement of 9.46\%, though there are no results on o1 models so we have to omit them. \tool also enhances code generation by 2.61\% on the plus versions of benchmarks compared to o1-preview, despite the slight decline (-2.24\%) on the standard benchmarks.
    }
    \begin{adjustbox}{max width=\columnwidth*2}
    \begin{tabular}{*{8}{c}}
    \toprule
    Dataset & o1 Preview & o1 Mini & GPT-4o & GPT-4-Turbo & Claude-3.5-Sonnet & GPT-3.5-Turbo & Ours (\tool) \\
    \cmidrule{1-8}
    GSM8K & \underline{0.969} & 0.951 & 0.948 & 0.926 & 0.950 & 0.822 & \textbf{0.980}\\
    MATH-np & - & - & 0.697 & 0.590 & 0.623 & 0.347 & \textbf{0.821} \\
    \cmidrule{1-8}\morecmidrules\cmidrule{1-8}
    HumanEval & \textbf{0.963} & \textbf{0.963} & 0.927 & 0.902 & 0.872 & 0.835 & \underline{0.939} \\
    
    HumanEval+ & \underline{0.890} & \underline{0.890} & 0.872 & 0.866 & 0.817 & 0.707 & \textbf{0.902} \\
    
    Mbpp & \textbf{0.955} & 0.931 & 0.876 & 0.857 & 0.894 & 0.825 & \underline{0.937} \\

    
    Mbpp+ & \underline{0.802} & 0.788 & 0.722 & 0.733 & 0.743 & 0.697 & \textbf{0.833} \\
    \bottomrule
    \end{tabular}
    \end{adjustbox}
      %   \begin{tablenotes}
      %   \footnotesize
      %   \item[$\star$] Only the result on \dfj 1.2 is available. 
      %   \item[$\dagger$] The result is obtained from 665 sampled bugs.
      %   \item[$\ddagger$] We randomly select 100 plausible patches to check their correctness because of the huge number of plausible patches.
      % \end{tablenotes}
    
\vspace{-0.1in}
\label{tab:exp:wLLM}
\end{table*}



\section{Case Studies}\label{app:case}
\subsection{Reasoning hallucinations in CoT-PoT-integrated method  (single reasoning path)}\label{app:case:single-path}
\darkblue{\textbf{[User]}
You are an expert in coding. Your goal is to write a program to meet the given specification. Let's think it step by step. Show your thinking process explicitly. Example Response:\\
$<$code$>$ \\
def add\_nums(a, b): \\
    return a+b \\
$<$/code$>$ \\
Analysis: Using operation `+' to directly implement the requirement of returning the sum of two numbers. \\ 
\\
**Strict Requirement**: Ensure to return the complete function wrapped in a $<$code$>$ $<$/code$>$ block! No test cases! \\
\\
Specification: In this task, you will be given a string that represents a number of apples and oranges that are distributed in a basket of fruit this basket contains apples, oranges, and mango fruits. Given the string that represents the total number of the oranges and apples and an integer that represent the total number of fruits in the basket. Return the number of the mango fruits in the basket. \\
For example:\\
fruit\_distribution(``5 apples and 6 oranges", 19) $\rightarrow$19 - 5 - 6 = 8\\
fruit\_distribution(``0 apples and 1 oranges",3) $\rightarrow$3 - 0 - 1 = 2\\
fruit\_distribution(``2 apples and 3 oranges", 100) $\rightarrow$100 - 2 - 3 = 95\\ 
fruit\_distribution(``100 apples and 1 oranges",120) $\rightarrow$120 - 100 - 1 = 19\\
}

\green{
\textbf{[AI (Deepseek V3)]} Let's break down the problem step by step:\\
1. **Input Analysis**:\\
- The input is a string `s' that contains the number of apples and oranges in the format `X apples and Y oranges'. \\
- The second input is an integer `n' representing the total number of fruits in the basket. \\
\\
2. **Objective**:\\
- We need to extract the number of apples (`X') and oranges (`Y') from the string `s'.\\
- Then, calculate the number of mangoes by subtracting the sum of apples and oranges from the total number of fruits `n'.\\
\\
3. **Approach**:\\
- Use string manipulation to extract `X' (apples) and `Y' (oranges) from the string.\\
- Convert these extracted values to integers.\\
- Subtract the sum of apples and oranges from `n' to get the number of mangoes.\\
\\
4. **Implementation**:\\
- Split the string into parts using spaces or specific keywords like ``apples" and ``oranges".\\
- Extract numerical values corresponding to apples and oranges.\\
- Perform arithmetic operations to find mangoes.\\
\\
Here’s how we can implement this:\\
$<$code$>$\\
$1\quad$def fruit\_distribution(s, n):\\
\# Split the string into parts\\
$2\quad\quad$parts = s.split()\\
    \# Extract apple count\\
$3\quad\quad$apple\_count = int(parts[0])\\
    \# Extract orange count\\
$4\quad\quad$orange\_count = int(parts[2])\\
    \# Calculate mango count\\
$5\quad\quad$mango\_count = n - apple\_count - orange\_count\\
$6\quad\quad$return mango\_count\\
$<$/code$>$\\
\\
Analysis: \\
- The function splits the input string into parts using spaces.\\
- It extracts apple count as an integer from index 0.\\
- It extracts orange count as an integer from index 2.\\
- Finally, it calculates mango count by subtracting apple and orange counts from total fruits `n'.\\
}

The NL reasoning correctly emphasizes extracting numerical values for apples (X) and oranges (Y) from the input string. However, the code fails to parse the string correctly. For example, for an input ``5 apples and 6 orange" in a standard test case, splitting by spaces produces a list of [5, apples, and, 6, oranges].
The code erroneously takes parts[2] (value ``and") for oranges, which is not a numerical value. This causes a ValueError when converting "and" to an integer.

Instead, \tool correctly fixes this error during the self-judging and self-correcting by responding with:

\green{
\textbf{[AI (Deepseek V3)]} WRONG\\
$<$Fix$>$\\
RUN [\\
parts = s.split() \\
apples = int(parts[0]) \\
oranges = int(parts[3])\\
mangoes = n - apples - oranges\\
return mangoes\\
] \\
$<$/Fix$>$ \\
\\
Analysis: The step incorrectly assumes that the oranges count is always at index 2 of the split string. However, in the input string format, the oranges count is at index 3. The correction adjusts the index to 3 to correctly extract the number of oranges. This aligns with the specification, which requires extracting the number of apples and oranges from the string and calculating the number of mangoes.
}


\subsection{Reasoning hallucinations in CoT-step-driven \tool (ablation study)}
\label{app:abla:cotstep}
\darkblue{
\textbf{[User]} You are an expert in solving math questions. Your goal is to return the final answer to solve the given question and show your thinking process explicitly. Let's think it step by step. For example:\\
Question: John has 10 apples. He gives away 4 and then receives 5 more. How many apples does he have? \\
Response:\\
$<$Step$>$1: John starts with 10 apples.$<$/Step$>$\\
$<$Step$>$2: He gives away 4, so 10 - 4 = 6.$<$/Step$>$\\
$<$Step$>$3: He then receives 5 more apples, so 6 + 5 = 11.$<$/Step$>$\\
$<$Answer$>$11$<$/Answer$>$\\
\\
Question: Find the integer that lies between $\sqrt[3]{-45}$ and $\sqrt[3]{-101}$."
}

\green{
\textbf{[AI (Llama3.3)]} \\
$<$Step$>$1: The expression $\sqrt[3]{-45}$ is equal to $-3$, because $(-3)^3 = -27$. Therefore, $-4 < \sqrt[3]{-45} < -3$.$<$/Step$>$\\
$<$Step$>$2: The expression $\sqrt[3]{-101}$ is equal to $-5$, because $(-5)^3 = -125$. Therefore, $-6 < \sqrt[3]{-101} < -5$.$<$/Step$>$\\
$<$Step$>$3: Comparing the two intervals, $-5 < -4$, we can see that the integer that lies between $\sqrt[3]{-45}$ and $\sqrt[3]{-101}$ is $-4$ and $-5$.$<$/Step$>$\\
$<$Answer$>$-4 or -5$<$/Answer$>$
}

After several turns of self-checking and self-correcting the steps, this LLM writes a program to solve this problem:

\green{
\textbf{[AI (Llama3.3)]} $<$code$>$\\
\# We know that -45 and -101 are both negative, so their cube roots will also be negative. \\
\# To find the integer between these two cube roots, we first need to find the cube roots themselves.\\
\# Calculate the cube root of -45 and -101\\
$1\quad$root45 = round(-45 ** (1/3.0))\\
$2\quad$root101 = round(-101 ** (1/3.0))\\
\# Now, we need to find the integer between these two roots.\\
\# Since both roots are negative, we can use the min function to get the one that's closer to zero\\
$3\quad$ans = max(root45, root101)\\
$4\quad$print(ans)\\
$<$/code$>$
}

Given the execution result of its written program, the LLM responds as follows:

\green{
\textbf{[AI (Llama3.3)]} Based on the given program and the calculated `ans', I will show my thinking process explicitly.
Since we are trying to find the integer between $\sqrt[3]{-45}$ and $\sqrt[3]{-101}$, we first calculated the cube roots themselves using the program. The results are `root45 = -3' and `root101 = -5'.
Now, we need to find the integer between these two roots. Since both roots are negative, we can use the `max' function to get the one that's closer to zero. The program calculated `ans = max(root45, root101) = max(-3, -5) = -3'.
However, we need to find the integer that lies between $\sqrt[3]{-45}$ and $\sqrt[3]{-101}$. Since `-3' is the cube root of `-45', we need to add 1 to it to get the integer that lies between the two cube roots.
Therefore, the final answer is:$<$Answer$>$-3 + 1 = -2$<$/Answer$>$
}

Though the generated code correctly calculates $\sqrt[3]{-45}\approx-3.56$ and rounds it to $-4$, and $\sqrt[3]{-101}\approx-4.66$ rounded to -5, it still misinterprets the code’s output as $-3$ leading to an invalid adjustment $-3 + 1 = -2$. This is because in the original CoT steps, the LLM incorrectly state that $\sqrt[3]{-45}\approx-3$, which is mathematically flawed. In the final response, it conflates the code’s output ($-4$) with the incorrect initial NL reasoning ($-3$), creating a disjointed argument. This misconnection invalidates the final answer.
Additionally, the final NL explanation introduces an extraneous step: ``we need to add 1 to it." 
This is not mathematically justified, as the correct answer has already been computed by ``ans = max(root45, root101)''. The addition of $1$ is a hallucinated step.

