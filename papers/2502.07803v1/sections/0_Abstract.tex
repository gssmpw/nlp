\begin{abstract}
Chain-of-Thought (CoT) prompting has shown promise in enhancing the reasoning capabilities of large language models (LLMs) by generating natural language (NL) rationales that lead to the final answer. However, it struggles with numerical computation, which has somehow led to the development of program-aided techniques.
Despite their potential, a persistent challenge remains: inconsistencies between LLM-reported reasoning steps and the logic in generated programs, which we term ``reasoning hallucinations." This stems from the inherent ambiguities of NL and the statistical nature of LLMs, which often lack rigorous logical coherence.
%
To address this challenge, we propose a novel test-time scaling framework, \textbf{Reasoning-as-Logic-Units (\tool)}, which constructs a more reliable reasoning path by aligning logical units between the generated program and their corresponding NL descriptions.
By decomposing the initially generated program into discrete units using static analysis, \tool engages in an iterative dialogue with the LLM to judge, refine, and explain each unit.
A rewind-and-correct mechanism ensures alignment between code statements and task requirements in each unit, ultimately forming a cohesive reasoning path under the program's logic, from which the model reaches a final solution.
%
Our experiments demonstrate that \tool significantly outperforms existing baselines in mathematical reasoning (GSM8K, MATH) and algorithmic reasoning (HumanEval+, MBPP+), underscoring its potential to advance LLM reasoning and programming by offering enhanced accuracy and interpretability.
\end{abstract}