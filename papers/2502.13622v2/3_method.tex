\section{Method}

\subsection{Task Description}
The SemEval 2025 Task 3: Mu-SHROOM \cite{vazquez-etal-2025-mu-shroom} focuses on detecting hallucinated spans in responses generated by LLMs. 
Given an input question $q$ and its corresponding LLM-generated response (along with the model's identifier), the goal is to identify spans in the response that are hallucinated. Details of the Mu-SHROOM dataset are provided in Section \ref{sec:dataset}.


\subsection{Retrieval-Augmented Factuality Hallucination Detection}

To address the challenge of factual hallucination detection in LLM outputs, we introduce \textbf{REFIND} (\textbf{RE}trieval-augmented \textbf{F}actuality halluc\textbf{IN}ation \textbf{D}etection). 
The overall workflow of the REFIND method is illustrated in Figure \ref{fig:overview}. REFIND leverages external knowledge retrieved from a relevant document set to assess the context sensitivity of each generated token.

The core principle behind REFIND is to quantify the influence of external context on the token generation process.  
We do this by measuring the change in the conditional probability of generating a token as information from retrieved documents is incorporated. 
This change is captured by the Context Sensitivity Ratio (CSR).
It quantifies the degree to which the conditional probability of generating a token is altered by the inclusion of external contextual information from retrieved documents. 

Let $\mathcal{M}_{\theta}$ denote an LLM parameterized by $\theta$, $q$ represent the input question, and $t_i$ denote the $i$-th token in the LLM's response to $q$. 
We use $p_{\theta}(t_i \mid \cdot)$ to represent the probability of generating token $t_i$ given the input.  
Furthermore, let $\mathcal{R}$ be a retriever that provides relevant documents based on $q$, and let $\mathcal{D} = \mathcal{R}(q)$ be the set of retrieved documents. 
The CSR for each token $t_i$ is defined as:
\begin{equation}
    CSR(t_i) = \frac{\log p_{\theta}(t_i \mid \mathcal{D}, q, t_{<i})}{\log p_{\theta}(t_i \mid q, t_{<i}) + \varepsilon}
\end{equation}
where $t_{<i}$ represents the sequence of preceding tokens. The numerator computes the log-probability of generating $t_i$ conditioned on the question $q$, the preceding tokens $t_{<i}$, and the retrieved document set $\mathcal{D}$. The denominator computes the log-probability of generating $t_i$ conditioned solely on the question $q$ and preceding tokens $t_{<i}$, excluding the retrieved documents.\footnote{To prevent division by zero, we use a small constant $\varepsilon$, which is set to $10^{-8}$.}


By comparing these two probabilities, the CSR effectively quantifies the sensitivity of $t_i$ to the external context provided by the $\mathcal{D}$. A higher CSR indicates a stronger influence of the retrieved context on the generation of the token.

Finally, to determine whether a token is a hallucination, we compare its CSR value to a predefined threshold, denoted as $\delta$. If the CSR value for the given token $t_i$ is greater than or equal to the threshold $\delta$, we classify that the token as a hallucination. Conversely, if the CSR value is less than $\delta$, the token is not considered a hallucination. This threshold $\delta$ serves as a hyperparameter that can be tuned to optimize the balance between precision and recall in hallucination detection.
