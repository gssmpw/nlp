\section{Introduction}

Detecting hallucinated information in responses generated by large language models (LLMs) has emerged as a critical challenge in the field of natural language generation \cite{Ji2023HallucinationSurvey, Zhang2023Sirens_Song_Hallucination}. 
Hallucination, in this context, refers to the generation of content that is factually incorrect or lacks grounding in verifiable sources \cite{li-etal-2024-dawn}. 
This issue is particularly pronounced in knowledge-intensive tasks that demand high factual accuracy, such as question answering \cite{Lee2022Factuality, sun-etal-2024-towards-verifiable}.
The consequences of unmitigated hallucination are significant, ranging from the propagation of misinformation to a decline in trust in AI systems, underscoring the need for effective hallucination detection for the development of safe and trustworthy AI.

\begin{figure*}[t]
\begin{center}
\includegraphics[width=.85\linewidth]{fig_overview_v3.pdf}
\end{center}
\caption{
FastAtlas Overview: In each frame, we compute charts spanning fully or partially visible triangles (a), determine texture space bounding boxes for the visible portions of the view-space projections of each chart, and tightly pack these boxes into atlases (b, here $2K \times 2K$). We simultaneously bijectively parameterize and shade the charts into their atlas boxes, obtaining high quality texture space shading (c), and use this shading to render the shaded frames (d).}
\label{fig:overview}
\label{fig:alg_overview}
\end{figure*}

\section{Overview}
\label{sec:overview}
Our work has two core contributions: a real-time, GPU-based algorithm for tight packing of general parameterized charts into compact atlases; and a real-time TSS method that
utilizes this packing.  

\paragraph*{FastAtlas Packing.}
FastAtlas runs entirely on the GPU as a series of compute shaders. It takes the bounding boxes of parameterized charts as input, and packs them into an atlas (Fig~\ref{fig:overview}b, Sec.~\ref{sec:pack}). As such, the only input it requires are the dimensions of the bounding boxes.
Its outputs are deterministic; identical input charts are packed into identical atlases. This is critical for TSS and similar applications, as it ensures that consecutive frames taken from the same camera view have the same shading. Even minute shading differences across such frames can cause sampling jitter, leading to undesirable flicker \cite{baker2012rock}. 
While prior methods such as \cite{mueller2018shading,hladky2019tessellated,hladky2021snakebinning,Neff2022MSA} cap the dimensions of the charts that can be packed as-is for a given atlas size, and scale down all charts that exceed these dimensions, we scale all charts by the same factor, and do so only when strictly necessary to achieve packing success (Figs~\ref{fig:atlas},~\ref{fig:sas_issues}). 

\paragraph*{TSS using FastAtlas.}
Our end-to-end TSS atlas generation method combines the packing method above with a novel approach for computing seamless per-frame charts. 
We define our charts as the connected components of the visible surfaces in each frame (Fig.~\ref{fig:overview}a), and efficiently compute them using a parallel union-find algorithm (Sec.~\ref{sec:visible}). Since the boundaries of these charts coincide with the contours of the rendered surface, they are {\em invisible} to the viewer. This approach 
eliminates the artifacts caused by shading discontinuities along visible seams (Fig.~\ref{fig:seams}). 

\begin{parWithWrapFigure}
\begin{wrapfigure}{l}{.27\columnwidth}%
\includegraphics[width=\linewidth]{fig_inset_view_plane.pdf}%
\end{wrapfigure}
We bijectively parametrize the {\em visible portions} of our charts by projecting them to view space (inset). This maps a constant number of texels to each pixel in the final rendered output, evenly distributing residual undersampling error across all image pixels. While conceptually straightforward, efficiently parameterizing charts containing partially visible triangles using viewspace projection is non-trivial, as the visible portions may no longer be triangular (e.g. green triangle in the inset); applying naive projection to triangles with vertices behind the camera may produce ill-posed results. Clipping triangles before projection is both computationally expensive and significantly complicates downstream operations. We avoid explicit clipping by observing that all that is required for atlas packing is the dimensions of, potentially conservative, bounding boxes of these projected visible portions. We compute such bounding boxes without explicit chart clipping by adapting a conservative screen coverage estimator \shortcite{Blinn:CalculatingScreenCoverage} (Sec.~\ref{sec:box}). We then pack the computed boxes using FastAtlas. 
\end{parWithWrapFigure}

Finally, we shade the visible portion of each chart into its corresponding atlas bounding box (Fig~\ref{fig:overview}c). 
The resulting texture is then used during rasterization as a standard texture map (Fig. ~\ref{fig:overview}d). 
Our framework is compatible with all existing approaches for texture space shading, including forward shading, raytraced illumination, or deferred shading in texture space \cite{baker:2016}. In the examples shown, we use the standard forward shading based rendering pipeline included in the G3D Innovation Engine \cite{G3D17}, a commercial grade renderer.


Prior research has explored various approaches for hallucination detection. 
Token-level classifiers, for example, leveraging pre-trained language models like RoBERTa \cite{Liu2019RoBERTa}, have been employed for binary classification, labeling individual tokens as either factual or hallucinated \cite{liu-etal-2022-token}. 
However, these models often exhibit limitations when applied to low-resource languages and tend to rely heavily on internal knowledge without effectively utilizing external evidence, which can hinder their performance. 
Extrinsic methods, such as retrieval-augmented models, aim to mitigate hallucinations by integrating external knowledge.  
Nevertheless, existing retrieval-augmented approaches, such as FAVA \cite{mishra2024finegrained-FAVA}, can potentially lead to inaccuracies in aligning the modified responses with the original LLM output, due to their multi-step processes involving retrieval, comparison, and editing.

To address these limitations, we introduce \textbf{REFIND} (\textbf{RE}trieval-augmented \textbf{F}actuality halluc\textbf{IN}ation \textbf{D}etection), a novel framework specifically designed to identify hallucinated spans within LLM-generated text.  REFIND achieves this by quantifying the context sensitivity of each token at the token level.  
By leveraging retrieved documents, REFIND calculates a Context Sensitivity Ratio (CSR) for each token in the LLM's response, measuring the token's dependence on external contextual information. 
Tokens exhibiting high CSR values are identified as likely hallucinations, offering a more direct and efficient approach to factuality verification.


Our contributions can be summarized as follows:

\vspace{-0.075in}

\begin{itemize}[itemsep=0.3mm, parsep=1pt, leftmargin=*]

    \item We present REFIND, a novel framework for detecting hallucinated spans in LLM responses by leveraging an external retriever and calculating the CSR at the token level.
    % \item We present REFIND, a novel framework for detecting hallucinations in LLM outputs by leveraging an external retriever and calculating the Context Sensitivity Ratio (CSR) at the token level.
    \item We conduct a comprehensive evaluation of REFIND using the SemEval 2025 Task 3: Mu-SHROOM dataset \cite{vazquez-etal-2025-mu-shroom}, a multilingual benchmark for detecting hallucinated spans.  REFIND is rigorously tested across nine diverse languages – Arabic, Czech, German, Spanish, Basque, Finnish, French, Italian, and English – demonstrating its robustness in both high- and low-resource settings.
    \item Experimental results demonstrate that REFIND significantly outperforms baseline models such as token-level classifiers and FAVA, achieving superior Intersection-over-Union (IoU) scores. This highlights the efficacy of the CSR in accurately identifying hallucinated content.
\end{itemize}