\clearpage
\onecolumn

\section{Implementation Details}
\label{sec:appendix_implementation_details}
All experiments are conducted using NVIDIA A100 80GB GPUs. 

For training the XLM-R-based \cite{conneau-etal-2020-unsupervised} system, we leverage the Trainer from the Hugging Face Transformers library \cite{wolf-etal-2020-transformers}. 
We train the model using token-aligned hallucination annotations from our dataset, with the model parameters optimized using cross-entropy loss and AdamW optimizer with a learning rate of 2e-5 for 5 epochs.

Inference for FAVA \cite{mishra2024finegrained-FAVA} is conducted using vLLM \cite{Kwon2023vLLM}, adhering to the original settings with \textit{temperature}=0, \textit{top\_p}=1.0, and \textit{max\_tokens}=1024. The prompt template used for FAVA inference is detailed in Figure \ref{fig:FAVA_Prompt} (Appendix \ref{sec:appendix_prompt_details}).

\subsection{Prompt Details}
\label{sec:appendix_prompt_details}

\begin{figure}[htb!]
    \centering
    \begin{tcolorbox}[colback=gray!10, colframe=black, title=Prompt template for REFIND]
        You are an assistant for answering questions.\\
        Refer to the references below and answer the following question.
        \\ \\
        \#\#\# References\\
        \{\texttt{reference\_passages}\} \\ \\
        \#\#\# Question\\
        \{\texttt{question}\} \\ \\
        \#\#\# Answer
    \end{tcolorbox}
    \vspace{-3mm}
    
    \caption{Prompt template of REFIND used to compute per-token probabilities under the conditions provided in the input context.}
    \label{fig:REFIND_Prompt}
\end{figure}

\begin{figure}[htb!]
    \centering
    \begin{tcolorbox}[colback=gray!10, colframe=black, title=Prompt template for FAVA]
        Read the following references:\\
        \{\texttt{reference\_passages}\} \\
        Please identify all the errors in the following text using the information in the references provided and suggest edits if necessary:\\
        \lbrack Text\rbrack\ \{\texttt{output}\}\\
        \lbrack Edited\rbrack\ 
    \end{tcolorbox}
    \vspace{-3mm}
    
    \caption{Prompt template for using FAVA \cite{mishra2024finegrained-FAVA}.}
    \label{fig:FAVA_Prompt}
\end{figure}



\onecolumn
\section{Full Text of Retrieved Documents \texorpdfstring{$\mathcal{D}$}{D} for Case Study\texorpdfstring{ (\cref{sec:case_study})}{}}
\label{sec:appendix_full_documents}

\begin{figure*}[ht!]

\large
\begin{tcolorbox}[boxrule=0pt]
  \textbf{Document 1.} Chance the Rapper discography he discography of American rapper Chance the Rapper consists of one studio album, five mixtapes and 27 singles (including 14 singles as a featured artist). Chance the Rapper released his debut mixtape, "10 Day" on April 3, 2012. The mixtape was followed up with the release of "Acid Rap" the following year, which saw universal acclaim from music critics. Chance the Rapper then released his third mixtape, "Coloring Book" on May 13, 2016. The mixtape peaked at number eight on the "Billboard" 200 chart to continued acclaim and was supported by the singles "Angels"\\\\
  \textbf{Document 2.} Juice (Chance the Rapper song) "Juice" is a song by American rapper Chance the Rapper, released on January 31, 2013 as the lead single from his second mixtape "Acid Rap" (2013). It was written by Chance and Nate Fox, who also produced the song. "Juice" is a midtempo song, built around a loop of Donny Hathaway's live performance of "Jealous Guy" by John Lennon. Chance the Rapper sings and raps in a comedic manner; his verses in the song have been described as having a "freewheeling, bluesy sway" that "gives way to raucous call-and-response choruses". He references the 1992 film "Juice" (of\\\\
  \textbf{Document 3.} signs of advertisements and department stores appear in the background, some of which provide imagery and visual references of the lyrics. For example, when Chance lyrically alludes to the film "Juice", a portrait of rapper Tupac Shakur (who starred in the film) flashes across a billboard. When "Acid Rap" was first re-released on streaming services on June 28, 2019, "Juice" was replaced with a 30-second spoken message, in which Chance the Rapper explains the song is excluded from the mixtape because of an uncleared sample. Chance then adds that all streaming proceeds for the alternate\\\\
  \textbf{Document 4.} Cocoa Butter Kisses "Cocoa Butter Kisses" is a song by American rapper Chance the Rapper from his second mixtape "Acid Rap" (2013). The song features American rappers Vic Mensa and Twista, and was produced by Cam O'bi and Peter Cottontale. It is one of Chance the Rapper's most popular songs to date. At the time when the song was written, Vic Mensa was staying at an apartment in Humboldt Park, Chicago with his manager Cody Kazarian. Chance the Rapper visited one day and showed Mensa a verse and hook he had written earlier. Soon, Mensa began composing his part for the song. In an interview\\\\
  \textbf{Document 5.} (eight) in several of those categories. One of the most closely watched races will be Best New Hip-Hop Artist, whose nominees including Anderson .Paak, Bryson Tiller (who won that award and Best Male R\&B/Pop Artist at June’s BET Awards), Chance the Rapper, Desiigner and Tory Lanez.  Drake – "Hotline Bling" Fat Joe \& Remy Ma featuring French Montana \& Infared – "All the Way Up" Kendrick Lamar Kendrick Lamar Director X DJ Khaled Metro Boomin DJ Khaled "All the Way Up" – Produced by Cool \& Dre and Edsclusive  Drake – "Views" Chance the Rapper DJ Khaled Kanye West Chance the Rapper

\end{tcolorbox}

\vspace{-3mm}


\caption{Complete text of documents retrieved for the input question "\textit{When did Chance the Rapper debut?}" as referenced in the case study in Section \ref{sec:case_study}.}


\label{fig:full_documents}
\end{figure*}
