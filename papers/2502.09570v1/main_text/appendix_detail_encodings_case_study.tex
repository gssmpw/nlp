\section{Details on Empirical Expressivity Analysis}\label{appendix-detailed-encodings}
\subsection{Rook and Shrikhande}\label{appendix-rook-shrikhande}

The Rook and Shrikhande graphs are examples of strongly regular graphs with parameters srg(16,6,2,2), meaning that they have 16 nodes, all of degree $6$, and that any ajacent vertices share $2$ common neighbors, while any non-adjacent vertices also share $2$ common neighbors. We illustrate these graphs in \ref{fig:side_by_side_images_rook_shri}.


We first compute 2-RWPE. The first entry is $0$, as a random walk from node $i$ with $1$-hop does not return to node $i$. The second entry is:
$$\frac{1}{6} \sum_{j \in \mathcal{N}_i} \left( \frac{1}{6} \right) = \frac{1}{6}$$
% I verified this.

\begin{figure}[h]
  \centering
  \includegraphics[width=0.35\textwidth]{images/rooke_graph.png}
  \includegraphics[width=0.35\textwidth]{images/shrikhande_graph.png}
  \caption{The Rook graph and the Shrikhande graph. Those two non-isomorphic graphs can be hard to distinguish: they are both srg(16,6,2,2) and are isospectral.}
  \label{fig:side_by_side_images_rook_shri}
\end{figure}

\begin{figure}[h]
  \centering
  \includegraphics[width=0.85\textwidth]{images/rooke_lifting.png}
  \caption{The Rook graph (left), its lifting (right) and its lifting's bipartite representation (center).}
  \label{fig:rook-lifting}
\end{figure}


For the Rook graph's lifting to a hypergraph (which we shall call the Rook hypergraph), the edges and vertices degree matrices are $D_e =4I_8$ and $D_v=2I_{16}$: every hyperedge has 4 nodes, and every node is in two hyperedge (see \ref{fig:rook-lifting}). For the Shrikhande graph's lifting to a hypergraph (which we call the Shrikhande hypergraph), these matrices are $D_e=3I_8$ and $D_v=6I_16$: every hyperedge has 3 nodes, and every node is in 6 hyperedge. Using Def.~\ref{def:fr}, we see that for the Rook graph, the 
FRC of any edge is $4(2-2)=0$, while for Shrikhande graph, the 
FRC of any edge is $3(2-6)=-12$.

\newpage


\subsection{Detailed comparaison of encodings}\label{appendix-detailed-comparaison}

We provide a comparison of encodings computed at the graph level and the hypergraph level in Tab.~\ref{tab:long-table-encodings}. We report the percentage of pairs in BREC that can be distinguished using the encodings, up to row permutation. The results in this table further illustrate theorems \ref{thm:lape_exp}, \ref{thm:rwpe_exp} and \ref{thm:hcp_exp}. We note that hypergraph-level encodings, with the exception of Hodge H-LAPE, are unable to distinguish pairs in the "Distance Regular" category. The "CFI" category is also notoriously difficult: Some pairs are 4-WL-indistinguishable.


\begin{table}[H]
\centering
\tiny
\begin{tabular}{|l|c|c|c|c|c|c|c|}
\hline
\textbf{Level (Encodings)} & \textbf{BASIC} & \textbf{Regular} & \textbf{str} & \textbf{Extension} & \textbf{CFI} & \textbf{4-Vertex-Condition} & \textbf{Distance Regular} \\
\hline
Graph: 1-WL/GIN & 0 & 0 & 0 & 0 & 0 & 0 & 0 \\
\hline
Graph (LDP) & 0 & 0 & 0 & 0 & 0 & 0 & 0  \\
Hyperaph (LDP) & 91.07 & 94.0  & 100 & 25.77  & 0 & 100 & 0  \\
\hline
Graph (LCP-FRC) & 0 & 0 & 0 & 0 & 0 & 0 & 0 \\
Hypergaph (HCP-FRC) & 91.07 & 96.0  & 100 & 26.8 & 0 & 100 & 0  \\
\hline
Graph (LCP-ORC) &  100 & 100 & 100 & 100 & 55.67 & 100 &0  \\
Hypergaph (HCP-ORC) & 100 & 100   & 100  & 94.85  & 100  &  100 &   0 \\
\hline
% Graph (2-RWPE) & 0 &  0 &  & 0 & 0 & 0 &  0 \\
% Hypergraph (H-2-RWPE) & 0  & 0  & 0 & 0 & 0  &  0 & 0 \\
% \hline
Graph (EE 2-RWPE) & 0 & 0 & 0 & 0 & 0 & 0 & 0 \\
Hypergraph (EE H-2-RWPE) & 91.07 & 82.0  & 98.0 & 50.52 & 0  & 100 & 0 \\
\hline
Graph (EE 3-RWPE) &  85.71 & 92.0 & 0  & 6.19  & 0  &  0& 0  \\
Hypergraph (EE H-3-RWPE) & 98.21  & 98.0   & 98.0  & 59.79  & 0  & 100  & 0  \\
\hline
Graph (EE 4-RWPE) & 100  & 96.0  &  0  & 83.51 & 0  & 0 &  0\\
Hypergraph (EE H-4-RWPE) & 100   & 100   & 98.0   & 92.78  & 0   & 100   & 0  \\
\hline
Graph (EE 5-RWPE) & 100  &  100 & 0   & 95.88  & 0  & 0 & 0 \\
Hypergraph (EE H-5-RWPE) & 100   & 100  &  98.0  &  95.88  & 0  & 100  & 0  \\
\hline
Graph (EE 20-RWPE)        & 100  &  100 & 0   & 100 &  3.09  & 0 & 0 \\
Hypergraph (EE H-20-RWPE) & 100   & 100  & 98   & 100   &  3.09  & 100  & 0  \\
\hline
Graph (Normalized 1-LAPE) & 0.0 & 0.0   & 0  &  0 & 0  & 0 & 0\\
Hypergraph (Normalized 1-LAPE) & 91.07 & 90.0   & 96  &  25.77 &   0 & 100 & 0 \\
\hline
Graph (Hodge 1-LAPE) & 48.21 &  100 & 100  & 71.13  &  7.22 & 100 & 5.0  \\
Hypergraph (Hodge 1-LAPE) & 98.21 & 98   & 100  &  74.23 & 7.22  & 100 & 10.0\\
\hline
\end{tabular}
\caption{Difference in encodings on the BREC dataset (390 pairs). We report the percentage of pairs with different encoding up to row permutation, at different level (graph or hypergraph). For the ORC Computations, we use the code from \citep{coupette2022ollivier} applied to hypergraphs and graphs.}\label{tab:long-table-encodings}
\end{table}

\subsection{Pair 0 of the "Basic" Category of BREC}\label{appendix-pair-0}

\begin{figure}[H]
  \centering
  \includegraphics[width=0.55\textwidth]{images/pair_0_basic.png}
  \caption{The pair 0 of the "Basic" category in BREC. Top: the two graphs in the pair. Second row: the (node) degree distributions and some statistics. Bottom: the adjacency matrices of the graphs.}
  \label{fig:pair-0}
\end{figure}

\begin{figure}[H]
  \centering
  \includegraphics[width=0.75\textwidth]{images/pair_0_basic_hypergraphs.png}
  \caption{The pair 0 of the "Basic" category in BREC. Top: the two graphs in the pair. Second row: the graphs' liftings. Third row: the sizes of the (hyper)edges. Bottom: the node degrees.}
  \label{fig:pair-0-lifting}
\end{figure}
\pagebreak

We compute various encodings on this pair.

\subsubsection{RWPE} 

The 1st entry of the RWPE encoding is 0. We now compute one of the 2nd entries at the graph level. Start with node 0, a node of degree 4, on pair A (see \ref{fig:pair-0}). A random-walker can go to nodes of degree 4 (the node 3), 5 (the node 7) or 6 (the nodes 5 and 8). Thus, the probability of coming back to node 0 after 2 hops is $\frac{1}{4} \times \frac{1}{4}+\frac{1}{4} \times \frac{1}{5}+\frac{2}{4} \times \frac{1}{6}=0.1958\overline{3}
$.



We report the full encodings for 2-RWPE in \ref{tab:sidetables}. We can actually see they are the same (up to row permutation.

\begin{table}[h]
\footnotesize
  \centering
\begin{tabular}{|c|c|c|}
    \hline
    0.0 & $0.1958\overline{3}$ \\
    \hline
    0.0 & $0.191\overline{6}$ \\
    \hline
    0.0 & 0.20555556 \\
    \hline
    0.0 & 0.18        \\
    \hline
    0.0 & $0.19\overline{6}$ \\
    \hline
    0.0 & $0.18$        \\
    \hline
    0.0 & $0.19\overline{6}$ \\
    \hline
    0.0 & $0.1958\overline{3}$ \\
    \hline
    0.0 & $0.1\overline{8}$ \\
    \hline
    0.0 & $0.191\overline{6}$ \\
    \hline
\end{tabular}
\begin{tabular}{|c|c|c|}
    \hline
     0.0 & $0.19\overline{6}$ \\
    \hline
    0.0 & $0.19\overline{6}$ \\
    \hline
    0.0 & $0.191\overline{6}$ \\
    \hline
    0.0 & $0.191\overline{6}$        \\
    \hline
    0.0 & $0.1958\overline{3}$ \\
    \hline
    0.0 & $0.18$       \\
    \hline
    0.0 & $0.1\overline{8}$ \\
    \hline
    0.0 & $0.18$ \\
    \hline
    0.0 & $0.20\overline{5}$ \\
    \hline
    0.0 & $0.1958\overline{3}$ \\
    \hline
\end{tabular}
  \caption{Pair A (left) and Pair B (right) 2-RWPE encodings. They match if we reorder the rows of pair A as follow: 4, 6, 1, 9, 0, 3, 8, 5, 2, 7).}
  \label{tab:sidetables}
\end{table}


At the hypergraph level, H-2-RWPE are different because the maximum absolute value of the last (second) column of the encoding for hypergraph A is 0.2503052503052503 while it is 0.2935064935064935 for graph B. The full encodings can be found in \ref{tab:sidetables-hg}. It is straightforward to check that the two encodings cannot be made the same even up to scaling and row permutation.
\begin{table}[h]
\centering
\footnotesize
\begin{tabular}{|c|c|c|}
    \hline
    0.0 & 0.15842491 \\ \hline
    0.0 & 0.24619611 \\ \hline
    0.0 & 0.15620094 \\ \hline
    0.0 & 0.14429618 \\ \hline
    0.0 & 0.24619611 \\ \hline
    0.0 & 0.15842491 \\ \hline
    0.0 & 0.25030525 \\ \hline
    0.0 & 0.24175824 \\ \hline
    0.0 & 0.14429618 \\ \hline
    0.0 & 0.15620094 \\ \hline
\end{tabular}    
\begin{tabular}{|c|c|c|}
  \hline
  0.0 & 0.15165945 \\ \hline
  0.0 & 0.15818182 \\ \hline
  0.0 & 0.21682409 \\ \hline
  0.0 & 0.15909091 \\ \hline
  0.0 & 0.15909091 \\ \hline
  0.0 & 0.29350649 \\ \hline
  0.0 & 0.26695527 \\ \hline
  0.0 & 0.21682409 \\ \hline
  0.0 & 0.15165945 \\ \hline
  0.0 & 0.15818182 \\ \hline
\end{tabular}
  \caption{Pair A (left) and Pair B (right) H-2-RWPE encodings.}
  \label{tab:sidetables-hg}
\end{table}

\subsubsection{FRC}

We now turn our attention to the FRC-LCP and FRC-HCP. The FRC-LCP of both pairs is presented in \ref{tab:sidetables-g-frc}. The encoding match with the following ordering for pair A: (6, 5, 0, 1, 2, 3, 7, 4, 8, 9).

\begin{table}[H]
\centering
\footnotesize
\begin{tabular}{|c|c|c|c|c|}
    \hline
    -6.00 & -4.00 & -5.25 & -5.50 & 0.8291562 \\ \hline
    -7.00 & -6.00 & -6.60 & -7.00 & 0.48989795 \\ \hline
    -7.00 & -6.00 & -6.60 & -7.00 & 0.48989795 \\ \hline
    -6.00 & -4.00 & -5.25 & -5.50 & 0.8291562 \\ \hline
    -8.00 & -7.00 & $-7.3\overline{3}$ & -7.00 & 0.47140452 \\ \hline
    -8.00 & -6.00 & $-7.3\overline{3}$ & -7.50 & 0.74535599 \\ \hline
    -7.00 & -5.00 & -6.20 & -6.00 & 0.74833148 \\ \hline
    -7.00 & -5.00 & -6.20 & -6.00 & 0.74833148 \\ \hline
    -8.00 & -6.00 & -7.00 & -7.00 & 0.81649658 \\ \hline
    -8.00 & -6.00 & $-7.3\overline{3}$ & -7.50 & 0.74535599 \\ \hline
\end{tabular}
\begin{tabular}{|c|c|c|c|c|}
%\footnotsize
    \hline
    -7.00 & -5.00 & -6.20 & -6.00 & 0.74833148 \\ \hline
    -8.00 & -6.00 & $-7.3\overline{3}$ & -7.50 & 0.74535599 \\ \hline
    -6.00 & -4.00 & -5.25 & -5.50 &  0.8291562  \\ \hline
    -7.00 & -6.00 & -6.60 & -7.00 & 0.48989795 \\ \hline
    -7.00 & -6.00 & -6.60 & -7.00 & 0.48989795 \\ \hline
    -6.00 & -4.00 & -5.25 & -5.50 & 0.8291562 \\ \hline
    -7.00 & -5.00 & -6.20 & -6.00 & 0.74833148 \\ \hline
    -8.00 & -7.00 & $-7.3\overline{3}$ & -7.00 & 0.47140452 \\ \hline
    -8.00 & -6.00 & -7.00 & -7.00 & 0.81649658 \\ \hline
    -8.00 & -6.00 & $-7.3\overline{3}$ & -7.50 & 0.74535599 \\ \hline
\end{tabular}
\caption{Pair A (left) and Pair B (right) FRC-LCP encodings. They match with the following permuation: (6, 5, 0, 1, 2, 3, 7, 4, 8, 9)}
\label{tab:sidetables-g-frc}
\end{table}

At the hypergraph level, they are different because the max absolute value of encoding graph A is $12.0$, the max absolute value of encoding graph B is $10.0$. The full encodings are presented in \ref{tab:sidetables-hg-frc}. It is straightforward to check that the matrices cannot be scaled and row permuted to match.

\begin{table}[H]
\footnotesize
\centering
\begin{tabular}{|c|c|c|c|c|}
    \hline
    -9 & -6 & -7          & -6 & 1.41421356 \\ \hline
    -9 & -5 & $-7.\overline{6}$ & -9 & 1.88561808 \\ \hline
    -9 & -5 & $-7.\overline{6}$ & -9 & 1.88561808 \\ \hline
    -9 & -6 & -7          & -6 & 1.41421356 \\ \hline
    -11 & -5 & -7.8       & -9 & 2.4 \\ \hline
    -12 & -6 & $-9.\overline{3}$ & -9 & 1.88561808 \\ \hline
    -9 & -5 & $-6.\overline{6}$ & -6 & 1.69967317 \\ \hline
    -9 & -5 & $-6.\overline{6}$ & -6 & 1.69967317 \\ \hline
    -12 & -6 & -9         & -9 & 1.73205081 \\ \hline
    -12 & -6 & $-9.\overline{3}$ & -9 & 1.88561808 \\ \hline
\end{tabular}
\begin{tabular}{|c|c|c|c|c|c|}
%\footnotesize
\hline
    -8 & -2 & -5 & -5 & 2.44948974 \\ \hline
    -10 & -8 & -8.6 & -8 & 0.8 \\ \hline
    -8 & -2 & $-5.\overline{3}$ & -6 & 2.49443826 \\ \hline
    -8 & -5 & -7 & -8 & 1.41421356 \\ \hline
    -8 & -5 & -7 & -8 & 1.41421356 \\ \hline
    -8 & -2 & $-5.\overline{3}$ & -6 & 2.49443826 \\ \hline
    -8 & -2 & -5 & -5 & 2.44948974 \\ \hline
    -9 & -5 & -7 & -8 & 1.67332005 \\ \hline
    -10 & -6 & -8 & -8 & 1.15470054 \\ \hline
    -10 & -8 & -8.6 & -8 & 0.8 \\ \hline
\end{tabular}
\caption{Pair A (left) and Pair B (right) FRC-HCP encodings.}
\label{tab:sidetables-hg-frc}
\end{table}

\subsubsection{1-LAPE}

Using the Normalized Laplacian, the pair is 1-LAPE indistinguishable (up to row permuation, and sign flip, as the eingenvectors are defined up to $\pm1$). The 1-LAPE encodings are presented in \ref{tab:sidetables-g-lape-norm}.

\begin{table}[H]
\centering
\footnotesize
  \begin{tabular}{|c|}
    \hline
    0.30348849 \\
         \hline
    0.3441236\\
         \hline
    0.3441236\\
         \hline
   0.30348849 \\
        \hline
  0.28097574 \\
       \hline
  0.28097574 \\
       \hline
  0.30348849 \\
       \hline
  0.30348849 \\
       \hline
  0.3441236  \\
       \hline
  0.3441236 \\ 
    \hline
  \end{tabular}
  \begin{tabular}{|c|}
    \hline
  0.30348849 \\
       \hline
  0.3441236 \\
       \hline
  0.30348849 \\
       \hline
  0.3441236  \\
       \hline
  0.28097574 \\
       \hline
  0.28097574 \\
       \hline
  0.30348849 \\
       \hline
  0.30348849 \\
       \hline
  0.3441236  \\
       \hline
  0.3441236 \\
    \hline
  \end{tabular}
\caption{Pair A (left) and Pair B (right) Normalized 1-LAPE encodings. They match with the following ordering for pair A: (0, 1, 3, 2, 4, 5, 6, 7, 8, 9).}
\label{tab:sidetables-g-lape-norm}
\end{table}

At the hypergraph level, H-1-LAPE are different because the maximum absolute value is $0.408248290463863$ for pair A and $0.39223227027636787$ for pair B. The H-1-LAPE can be found in \ref{tab:sidetables-hg-lape-norm}.


\begin{table}[H]
\footnotesize
\centering
  \begin{tabular}{|c|}
    \hline
     0.25      \\
          \hline
     0.40824829\\
          \hline
     0.40824829\\
          \hline
     0.25      \\
          \hline
     0.40824829\\
          \hline
     0.40824829\\
          \hline
     0.25      \\
          \hline
     0.25      \\
          \hline
     0.20412415\\
          \hline
     0.20412415\\
    \hline
  \end{tabular}
  \begin{tabular}{|c|}
    \hline
     0.2773501 \\
     \hline
   0.39223227\\
     \hline
    0.2773501 \\
     \hline
     0.39223227\\
         \hline
     0.39223227\\
          \hline
      0.39223227\\
           \hline
       0.2773501 \\
            \hline
     0.2773501 \\
          \hline
        0.19611614\\
             \hline
      0.19611614\\
    \hline
  \end{tabular}
\caption{Pair A (left) and Pair B (right) Normalized H-1-LAPE encodings.}
\label{tab:sidetables-hg-lape-norm}
\end{table}



\subsection{Additional Plots}

\begin{figure}[H]
\footnotesize
  \centering
  \includegraphics[width=0.75\textwidth]{images/pair_0_regular_hypergraphs.png}
  \caption{The pair 0 of the regular category in BREC. Top: the two graphs in the pair. Second row: the graphs' liftings. Third row: the sizes of the (hyper)edges. Bottom: the node degrees.}
  \label{fig:pair-0-lifting}
\end{figure}

\begin{figure}[H]
  \centering
  \includegraphics[width=0.75\textwidth]{images/pair_0_str_hypergraphs.png}
  \caption{The pair 0 of the strongly regular category in BREC. Top: the two graphs in the pair. Second row: the graphs' liftings. Third row: the sizes of the (hyper)edges. Bottom: the node degrees.}
  \label{fig:pair-0-lifting}
\end{figure}