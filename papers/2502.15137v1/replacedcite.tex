\section{Related Work}
\label{sec: related work}

In this section, we introduced the research related to our work, mainly involving two aspects: GUI accessibility research and the exploration of GUI navigation.

\subsection{GUI accessibility research}

Currently, research on GUI accessibility primarily revolves around two directions. 
One involves empirical explorations to analyze accessibility issues within GUIs, serving as a wake-up call for developers. 
The other entails various tools developed by both industry and academia to identify and pinpoint these issues.

For the first direction, Vendome et al.____ conducted a key study on the accessibility of Android apps. 
They found that developers are often more inclined to focus on the needs of ordinary users and neglect the needs of visually impaired users.
In 2021, Alshayban et al.____ further extended this exploration to nearly one million apps in Google Play.
Their findings showed that more than 80\% of these apps have accessibility problems, especially the lack of alternative text, small size, unobvious color contrast and narrow component spacing, which seriously affect the normal operation of visually impaired users.
These studies lay a foundation for the subsequent exploration of GUI accessibility, and provide valuable guidance and inspiration for developers.

In the second direction, both industry____ and academia____ have released and proposed many tools that could check the accessibility issues in GUIs.
In industry, Google released the Google Accessibility Test Framework____ and the corresponding tool called Accessibility Scanner____ in 2016.
Then, IBM followed with the Mobile Accessibility Checker (MAC)____ to check the accessibility issues in GUIs.
Also, there are many tools proposed by the researchers, including the PUMA____, MATE____, and XBot____.
Hao et al.____ developed a programmable UI-automation analysis tool, which could provide significant assistance for visually impaired users.
A similar detection tool is also proposed by Eler et al.____, they designed an automated tool, named MATE, which automatically explores apps while applying different checks for accessibility issues related to visual impairment.
Chen et al.____ proposed an Xbot to check the accessibility issues in the GUIs, and made a better performance of collecting accessibility issues.
Furthermore, in our previous research____, we have implemented a tool called AccessFixer that could convert GUIs into GUI-Graphs, and then take advantage of the spectrum fluctuation of Relational-Graph Convolutional Neural Network (R-GCN) model____ during relationship prediction to fix accessibility issues.

Although existing research and tools well support and improve GUI accessibility, they do not pay attention to and solve the problems of GNFs.
Therefore, compared with these existing tools, our work pays special attention to the issues in the navigation sequence, and further improves the experience of visually impaired users.

\subsection{Exploration of GUI navigation}

The accessibility issues on navigation flow are also crucial factors affecting the visually impaired users to use the apps____.
To this end, in 2022, Alotaibi et al. proposed a series of tools to detect issues with web keyboard navigation____ and mobile application GUI navigation____.
These methods and tools provide effective technical support for identifying the key problems in GUI navigation.
Meanwhile, Koutrika et al.____ proposed a framework that could automatically organize a collection of documents in a tree from general to more specific documents.
It also allows a user to choose a reading sequence over the documents.
Another exploration of reading sequences is the Spatio-Temporal Fusion Module (STFM) proposed by Zhang et al.____.
Such a module could well be applied to the navigation flows to maintain the local spatial information and to reduce the feature dimensions.
Meanwhile, this method of utilizing spatial-temporal structure to deal with the sequence issues also inspires us to conduct a similar method that could adapt to the reading sequences of TalkBack.
Further, Salehnamadi et al. proposed the Latte____ that automatically reuses tests written to evaluate an app’s functional correctness to assess its accessibility.
Focusing on eliminating the tediousness of manual accessibility testing, they also introduced Groundhog____ and a record-and-replay technique____.
These approaches were designed to test and evaluate accessibility issues within the GUI, utilizing intuitive interactive video information and a variety of developer interaction actions, respectively.
Their proposed methods effectively detect accessibility issues within the GUI layout and navigation, but do not provide an appropriate strategy for redrawing the navigation.

Despite the above studies on sequences providing critical technical and inspirational support for our work, they are designed for the sequences in texts or image texts, and are more focused on exploring how to detect issues within the navigation.
Further, there remains a gap in research concerning the navigation sequences specifically utilized by visually impaired users when interacting with screen readers, and hard to provide a method for reconstructing the navigation.