\usepackage{graphicx}%
\usepackage{multirow}%
\usepackage{amsmath,amssymb,amsfonts}%
\usepackage{amsthm}%
\usepackage{mathrsfs}%
\usepackage[title]{appendix}%
%\usepackage{xcolor}%
\usepackage{textcomp}%
\usepackage{manyfoot}%
\usepackage{booktabs}%
\usepackage{algorithm}%
\usepackage{algorithmicx}%
\usepackage{algpseudocode}%
\usepackage{listings}%
\usepackage{color}
\usepackage{titlesec}
\usepackage[table,xcdraw]{xcolor}
\usepackage{colortbl}
\usepackage{graphicx}
\usepackage{subcaption}
\usepackage{lmodern}
\usepackage{placeins}

\setcounter{secnumdepth}{4}

\begin{document}

\title[Article Title]{EEG Artifact Detection and Correction with Deep Autoencoders}


\section{Supplementary Information}\label{sec:SupInfo}
\subsection{Network Training Details}\label{sec:Supptraining}
All networks have followed the same training pipeline and same hyperparameters. Network convergence has been ensured by visual inspection of the learning curves, visualizing the evolution of both the training and validation losses.
Both the UNET and CLEEGN architectures have been trained as provided by the source code reported in the original publications \cite{Chuang2022, Lai2022}, respectively. We applied dropout after each layer in our LSTEEG architecture, with a probability of $p=0.1$.

All networks have been trained from scratch, with default weight initialization, for N=1000. The training loss has been used for weight updating, through backpropagation, using Adam optimizer \cite{Kingma_adam} with an initial learning rate of $\eta = 5\times 10^{-4}$ and a Cosine Annealing learning rate scheduler with $T_{max}=10$. Early stopping based on the validation loss was used. The selected batch size was 16, and the loss function, as described in the main text, the MSE between the target (either noisy input or denoised target) and the network’s output.
Hyperparameters have been adapted to the LEMON dataset.



\subsection{Exploration of LSTEEG parameters}\label{sec:Supp_params}

Initial, preliminary, hyperparameter explorations showed that, while small values of $N_o$ and $N_i$ may deteriorate the performance of the autoencoder, the determining factor in improving the reconstructing capabilities of the network is the size of the LS, $N_{LS}$. Therefore, we chose to fix $N_o=50$ and $N_i=25$. These two values neared the elbow in reconstruction error in the two studies presented in Figures A and B, while keeping the total number of parameters in the network to a reasonable value.

Finally, in Figure C, it is shown that, for the chosen $N_o$ and $N_i$ values, the dimension of the Latent Space has a strong effect in the reconstruction error of the network. The smaller the LS, the more compression, the higher the loss of information, which of course degrades the reconstruction and increases the error.

\begin{figure}[ht]
    \centering
    \begin{minipage}[b]{0.9\textwidth}
        \centering
        \includegraphics[width=\textwidth]{Figures/SupplementaryFigures/MSE_evaluation/mse_comparison_no_ni_50.png}
        \caption*{(a)}
    \end{minipage}
    \hfill
    \begin{minipage}[b]{0.9\textwidth}
        \centering
        \includegraphics[width=\textwidth]{Figures/SupplementaryFigures/MSE_evaluation/mse_comparison_no_ni.png}
        \caption*{(b)}
    \end{minipage}

    \begin{minipage}[b]{0.8\textwidth}
        \centering
        \includegraphics[width=\textwidth]{Figures/SupplementaryFigures/MSE_evaluation/mse_comparison_dimLS.png}
        \caption*{(c)}
    \end{minipage}
    \caption{Effect of LSTEEG's hyperparameters on reconstruction error, averaged over all the epochs from the test partition. (a) shows the effect that different combinations of $N_o$ and $N_i$ have on MSE. (b) shows the effect on MSE of $N_o$ changes when fixing $N_i=50$. (c) shows the effect of changing $N_{LS}$, with $N_o=50$, $N_i=25$.}
    \label{fig:example}
\end{figure}

\FloatBarrier % Ensure all floats are processed before starting a new page
\clearpage % Start a new page for the next section

\subsection{Temporal Activation of Latent Space Dimensions}
Firstly, we aim to investigate the temporal features of an EEG epoch encoded in the LS dimensions. Thus, we explore how each dimension in the Latent Space is "activated" by the contents of the EEG epochs. In other words, when we forward an epoch ($x_e$) through our LSTEEG's encoder, we obtain a projection a in the LS, $f_E(x_e) \in \mathbb{R}^{N_{LS}}$. We try to understand the characteristics of an EEG epoch that increase or decrease the value at each dimension $j$ in the LS, $f_E^j(x_e)$. We therefore compute the “Temporal Activation” of dimension $j$, $\alpha^j$ through:

\begin{equation}
\alpha^j = \sum_{e=1}^{N_e} x_e f_E^j(x_e)   
\end{equation}

where $N_e$ is the total number of epochs in the dataset.

We illustrate our motivation for this approach with a simplified example. Let us say that our LSTEEG has learned to encode alpha oscillations in dimension $j$. We would expect that when the input epoch shows clear alpha oscillations, the output $f_E^j(x_e)$ will be large, and vice-versa. Thus, those epochs with strong alpha oscillations will have larger weight when computing $\alpha_j$.

For visualization, we will choose those dimensions with the largest activations, that is, with the largest $\sum_{e=1}^(N_e)|f_E^j (x_e)|$.

However, since we do not have time-locked activity, we might incur into destructive interference when averaging over the entire dataset.

\FloatBarrier % Ensure all floats are processed before starting a new page
\clearpage % Start a new page for the next section
\subsection{Artifact Correction Supplementary Figures}

\subsubsection{Comparison of $X_{br}$ Figures}

\begin{figure}[ht]
    \centering
    % First row
    \begin{subfigure}[b]{0.495\textwidth}
        \includegraphics[width=\textwidth]{Figures/ArtifactCorrection/x_br/large/cleegn_sample_9.png}
        \caption{CLEEGN}
        \label{fig:Br_large_cleegn}
    \end{subfigure}
    \hfill % for spacing
    \begin{subfigure}[b]{0.495\textwidth}
        \includegraphics[width=\textwidth]{Figures/ArtifactCorrection/x_br/large/UNET_sample_9.png}
        \caption{UNET}
        \label{fig:Br_large_unet}
    \end{subfigure}

    % Second row
    \begin{subfigure}[b]{0.495\textwidth}
        \includegraphics[width=\textwidth]{Figures/ArtifactCorrection/x_br/large/stareeg500_sample_9.png}
        \caption{LSTEEG ($N_{LS}=500$)}
        \label{fig:Br_large_staraeeg500}
    \end{subfigure}
    \hfill % for spacing
    \begin{subfigure}[b]{0.495\textwidth}
        \includegraphics[width=\textwidth]{Figures/ArtifactCorrection/x_br/large/stareeg2000_sample_9.png}
        \caption{LSTEEG ($N_{LS}=2000$)}
        \label{fig:Br_large_staraeeg2000}
    \end{subfigure}

    \caption{Reconstruction Comparison for a large amplitude EEG artifact in the testing set. Shown networks trained with $\mathbf{X_{Br}}$. The Red line is the cleaned target EEG epoch; the Grey line is the input EEG epoch, containing the large amplitude artifact; the Blue line is the output of each network. While CLEEGN produces high amplitude outputs, unable to correct the artifact, both UNET and the two configurations of LSTEEG are able to remove the artifact from the EEG signal.}
    \label{fig:Br_large}
\end{figure}


\begin{figure}[ht]
    \centering
    % First row
    \begin{subfigure}[b]{0.495\textwidth}
        \includegraphics[width=\textwidth]{Figures/ArtifactCorrection/x_br/eye/cleegn_sample_20.png}
        \caption{CLEEGN}
        \label{fig:Br_eye_cleegn}
    \end{subfigure}
    \hfill % for spacing
    \begin{subfigure}[b]{0.495\textwidth}
        \includegraphics[width=\textwidth]{Figures/ArtifactCorrection/x_br/eye/UNET_sample_20.png}
        \caption{UNET}
        \label{fig:Br_eye_unet}
    \end{subfigure}

    % Second row
    \begin{subfigure}[b]{0.495\textwidth}
        \includegraphics[width=\textwidth]{Figures/ArtifactCorrection/x_br/eye/stareeg500_sample_20.png}
        \caption{LSTEEG ($N_{LS}=500$)}
        \label{fig:Br_eye_staraeeg500}
    \end{subfigure}
    \hfill % for spacing
    \begin{subfigure}[b]{0.495\textwidth}
        \includegraphics[width=\textwidth]{Figures/ArtifactCorrection/x_br/eye/stareeg2000_sample_20.png}
        \caption{LSTEEG ($N_{LS}=2000$)}
        \label{fig:Br_eye_staraeeg2000}
    \end{subfigure}

    \caption{Reconstruction Comparison for a typical ocular artifact in the test set. Shown networks trained with $\mathbf{X_{Br}}$. The Red line is the cleaned target EEG epoch; the Grey line is the input EEG epoch, containing the large amplitude artifact; the Blue line is the output of each network. While CLEEGN produces high amplitude outputs, unable to correct the artifact, both UNET and the two configurations of LSTEEG are able to remove the artifact from the EEG signal.}
    \label{fig:Br_eye}
\end{figure}

\FloatBarrier % Ensure all floats are processed before starting a new page
\clearpage % Start a new page for the next section


\subsubsection{Comparison of $X_{ar}$ Figures}

\begin{figure}[ht]
    \centering
    % First row
    \begin{subfigure}[b]{0.495\textwidth}
        \includegraphics[width=\textwidth]{Figures/ArtifactCorrection/x_ar/eye/cleegn_sample_20.png}
        \caption{CLEEGN}
        \label{fig:AR_eye_cleegn}
    \end{subfigure}
    \hfill % for spacing
    \begin{subfigure}[b]{0.495\textwidth}
        \includegraphics[width=\textwidth]{Figures/ArtifactCorrection/x_ar/eye/UNET_sample_20.png}
        \caption{UNET}
        \label{fig:AR_eye_unet}
    \end{subfigure}

    % Second row
    \begin{subfigure}[b]{0.495\textwidth}
        \includegraphics[width=\textwidth]{Figures/ArtifactCorrection/x_ar/eye/stareeg500_sample_20.png}
        \caption{LSTEEG ($N_{LS}=500$)}
        \label{fig:AR_eye_staraeeg500}
    \end{subfigure}
    \hfill % for spacing
    \begin{subfigure}[b]{0.495\textwidth}
        \includegraphics[width=\textwidth]{Figures/ArtifactCorrection/x_ar/eye/stareeg2000_sample_20.png}
        \caption{LSTEEG ($N_{LS}=2000$)}
        \label{fig:AR_eye_staraeeg2000}
    \end{subfigure}

    \caption{Reconstruction Comparison for a typical ocular artifact in the testing set. Shown networks trained with $\mathbf{X_{Ar}}$. The Red line is the cleaned target EEG epoch; the Grey line is the input EEG epoch, containing the large amplitude artifact; the Blue line is the output of each network. While CLEEGN produces high amplitude outputs, unable to correct the artifact, both UNET and the two configurations of LSTEEG are able to remove the artifact from the EEG signal.}
    \label{fig:AR_eye}
\end{figure}

\FloatBarrier % Ensure all floats are processed before starting a new page
\clearpage % Start a new page for the next section

\subsection{Comparison of Spectral Attenuation}\label{sec:supp_attenuation}

\begin{figure}[ht]
    \centering
    \begin{subfigure}[b]{\textwidth}
        \centering
        \includegraphics[width=0.9\textwidth]{Figures/SupplementaryFigures/PSD_comparison/psd_comparison_clean_with_icunet.png}
        \caption{$\mathbf{X_{Br}}$}
        \label{fig:psd_br}
    \end{subfigure}
    \vfill
    \begin{subfigure}[b]{\textwidth}
        \centering
        \includegraphics[width=0.9\textwidth]{Figures/SupplementaryFigures/PSD_comparison/psd_comparison_clean_with_starlab.png}
        \caption{$\mathbf{X_{Ar}}$}
        \label{fig:psd_ar}
    \end{subfigure}
    \caption{PSD Attenuation for UNET and LSTEEG. One can note the sharp decay of LSTEEGs ($N_{LS}=500$) PSD after a certain frequency.}
    \label{fig:supp_attenuation}
\end{figure}

\end{document}