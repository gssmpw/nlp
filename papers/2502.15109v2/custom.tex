% This must be in the first 5 lines to tell arXiv to use pdfLaTeX, which is strongly recommended.
\pdfoutput=1
% In particular, the hyperref package requires pdfLaTeX in order to break URLs across lines.

\documentclass[11pt]{article}

\usepackage{booktabs}
\usepackage{array}
\usepackage{longtable}
\usepackage{colortbl}
\usepackage{graphicx}
\usepackage{amsmath}
\usepackage{hyperref}
\usepackage{tcolorbox}
\usepackage{soul}
\usepackage{amsfonts}
\usepackage{bbm}


% Change "review" to "final" to generate the final (sometimes called camera-ready) version.
% Change to "preprint" to generate a non-anonymous version with page numbers.
% \usepackage[review]{acl} 
\usepackage[final]{acl}

% Standard package includes
\usepackage{times}
\usepackage{latexsym}

% For proper rendering and hyphenation of words containing Latin characters (including in bib files)
\usepackage[T1]{fontenc}
% For Vietnamese characters
% \usepackage[T5]{fontenc}
% See https://www.latex-project.org/help/documentation/encguide.pdf for other character sets

% This assumes your files are encoded as UTF8
\usepackage[utf8]{inputenc}

% This is not strictly necessary, and may be commented out,
% but it will improve the layout of the manuscript,
% and will typically save some space.
\usepackage{microtype}

% This is also not strictly necessary, and may be commented out.
% However, it will improve the aesthetics of text in
% the typewriter font.
\usepackage{inconsolata}

%Including images in your LaTeX document requires adding
%additional package(s)
\usepackage{graphicx}

% custom colors 
\definecolor{VideoGreen}{HTML}{006400}     % Dark green for video input
\definecolor{QuestionBlue}{HTML}{0000FF}     % Blue for question text
\definecolor{AnswerMagenta}{HTML}{FF00FF}     % Magenta for answer options
\definecolor{CurrentOrange}{HTML}{FFA500}    % Orange for current question

% macros for colored placeholders
\newcommand{\videoinput}{\textcolor{VideoGreen}{\textbf{<video>}}}
\newcommand{\questioninput}{\textcolor{QuestionBlue}{\textbf{<question>}}}
\newcommand{\answeroptions}{\textcolor{AnswerMagenta}{\textbf{<answer options>}}}
\newcommand{\currentquestion}{\textcolor{CurrentOrange}{\textbf{<current question>}}}


\newcommand{\name}{\textsc{Social Genome}}

% If the title and author information does not fit in the area allocated, uncomment the following
%
%\setlength\titlebox{<dim>}
%
% and set <dim> to something 5cm or larger.

\title{\name:\protect\\ Grounded Social Reasoning Abilities of Multimodal Models}

% previous title: social genome: multimodal reasoning over fine-grained and grounded social interactions

% Author information can be set in various styles:
% For several authors from the same institution:
% \author{Author 1 \and ... \and Author n \\
%         Address line \\ ... \\ Address line}
% if the names do not fit well on one line use
%         Author 1 \\ {\bf Author 2} \\ ... \\ {\bf Author n} \\
% For authors from different institutions:
% \author{Author 1 \\ Address line \\  ... \\ Address line
%         \And  ... \And
%         Author n \\ Address line \\ ... \\ Address line}
% To start a separate ``row'' of authors use \AND, as in
% \author{Author 1 \\ Address line \\  ... \\ Address line
%         \AND
%         Author 2 \\ Address line \\ ... \\ Address line \And
%         Author 3 \\ Address line \\ ... \\ Address line}

\author{
  Leena Mathur\textsuperscript{*\dag}\textsuperscript{1}, Marian Qian\textsuperscript{*}\textsuperscript{1}, Paul Pu Liang\textsuperscript{2}, Louis-Philippe Morency\textsuperscript{1}\\
  Carnegie Mellon University\textsuperscript{1}, Massachusetts Institute of Technology\textsuperscript{2}\\
   {\texttt{\{lmathur,marianq,morency\}@cs.cmu.edu, ppliang@mit.edu}}}

%\author{
%  \textbf{First Author\textsuperscript{1}},
%  \textbf{Second Author\textsuperscript{1,2}},
%  \textbf{Third T. Author\textsuperscript{1}},
%  \textbf{Fourth Author\textsuperscript{1}},
%\\
%  \textbf{Fifth Author\textsuperscript{1,{2}},
%  \textbf{Sixth Author\textsuperscript{1}},
%  \textbf{Seventh Author\textsuperscript{1}},
%  \textbf{Eighth Author \textsuperscript{1,2,3,4}},
%\\
%  \textbf{Ninth Author\textsuperscript{1}},
%  \textbf{Tenth Author\textsuperscript{1}},
%  \textbf{Eleventh E. Author\textsuperscript{1,2,3,4,5}},
%  \textbf{Twelfth Author\textsuperscript{1}},
%\\
%  \textbf{Thirteenth Author\textsuperscript{3}},
%  \textbf{Fourteenth F. Author\textsuperscript{2,4}},
%  \textbf{Fifteenth Author\textsuperscript{1}},
%  \textbf{Sixteenth Author\textsuperscript{1}},
%\\
%  \textbf{Seventeenth S. Author\textsuperscript{4,5}},
%  \textbf{Eighteenth Author\textsuperscript{3,4}},
%  \textbf{Nineteenth N. Author\textsuperscript{2,5}},
%  \textbf{Twentieth Author\textsuperscript{1}}
%\\
%\\
%  \textsuperscript{1}Affiliation 1,
%  \textsuperscript{2}Affiliation 2,
%  \textsuperscript{3}Affiliation 3,
%  \textsuperscript{4}Affiliation 4,
%  \textsuperscript{5}Affiliation 5
%\\
%  \small{
%    \textbf{Correspondence:} \href{mailto:email@domain}{email@domain}
%  }
%}
\renewcommand{\footnoterule}{}

\begin{document}
\maketitle

% add back for arXiv
\renewcommand{\thefootnote}{\fnsymbol{footnote}}
\footnotetext[1]{equal contribution, \dag corresponding author}
\renewcommand{\thefootnote}{\arabic{footnote}}

\begin{abstract}
Social reasoning abilities are crucial for AI systems to effectively interpret and respond to multimodal human communication and interaction within social contexts. We introduce \name, the first benchmark for fine-grained, grounded social reasoning abilities of multimodal models. \name\ contains 272 videos of interactions and 1,486 human-annotated reasoning traces related to inferences about these interactions. These traces contain 5,777 reasoning steps that reference evidence from visual cues, verbal cues, vocal cues, and external knowledge (contextual knowledge external to videos). \name\ is also the first modeling challenge to   study external knowledge in social reasoning. \name\ computes metrics to holistically evaluate semantic and structural qualities of model-generated social reasoning traces. We demonstrate the utility of \name\ through experiments with state-of-the-art models, identifying performance gaps and opportunities for future research to improve the grounded social reasoning abilities of multimodal models. 

\end{abstract}

\section{Introduction}
\label{sec:intro}
Humans rely on \textit{social reasoning} to interpret and navigate everyday interactions \cite{gagnon2021reasoning}. This form of reasoning is a core competency of social intelligence \cite{kihlstrom2000social, conzelmann2013new}, occurs with specialized neural and cognitive systems \cite{cao2024domain, read2013constraint}, and involves integrating information over time from multimodal behaviors such as gestures, language, and prosody \cite{morency2010modeling,read2014dynamic, liang2024foundations}. Multimodal cues are often \textit{fine-grained} (e.g., a fleeting glance),  \textit{interleaved} (e.g., a shrug followed by a sigh), and \textit{context-dependent}, requiring \textit{external knowledge} of contextual information to be interpreted accurately \cite{hechter2001social}. 

\begin{figure}[t!]
    \centering
\includegraphics[width=0.95\linewidth]{figures/teaser4.png}
    \caption{Reasoning over multimodal social interactions involves extracting, integrating, and referencing evidence from multiple behavioral modalities, as well as information from external knowledge.}
    \label{fig:sreason}
\end{figure}
\raggedbottom


Developing algorithms for multimodal social reasoning will be essential to advance artificial intelligence (AI) systems with social intelligence \cite{mathur-etal-2024-advancing}. 
When AI systems reason about human social interactions,  it is important for systems to have the ability to generate interpretable explanations with accurate, \textit{grounded} references to fine-grained multimodal behaviors and external social knowledge concepts that inform inferences. Figure \ref{fig:sreason} visualizes these aspects of multimodal social reasoning. This capability is especially important for AI systems reasoning about human social interactions in high-stakes domains, such as healthcare agents and assistive robots.  

\begin{figure*}[t]
    \centering
    \includegraphics[width=1\linewidth]{figures/social-genome_figures_22.png}
    \caption{A sample reasoning trace from the \name\ benchmark. Reasoning traces in \name\ contain fine-grained, multimodal social cues and references to external knowledge informing the social inference. Social reasoning traces produced by humans can contain complex reasoning paths (sample visualized above) that reference and build upon multimodal evidence and external knowledge across  temporal segments of interactions.}
    \label{fig:overview}
\end{figure*}

Progress towards improving multimodal social reasoning abilities of models has been limited by a lack of evaluation tasks -- measuring a capability is an essential first step towards advancing it. We introduce \textbf{\name}, the first benchmark for grounded multimodal social reasoning that includes 272 videos of face-to-face  interactions and 1,486 human-annotated reasoning traces explaining inferences about social information in these videos. Across these traces, \name\ contains 5,777 social reasoning steps. Each reasoning step is tagged with the modality of information being referenced: \textit{visual}, \textit{verbal}, and \textit{vocal} cues from social interactions in videos, and \textit{external knowledge} of contextual information that human annotators used to perform social inferences (information external to stimuli in videos). Reasoning traces in \name\ reference over 11,000 entities (people, objects, concepts), over 5000 multimodal cues, and over 2,900 external knowledge observations. \name\ is the first social reasoning benchmark to include external knowledge and \textit{dense} reasoning traces. A sample human-annotated reasoning trace is provided in Figure \ref{fig:overview}.

This paper defines metrics to holistically assess semantic and structural aspects of model-generated social reasoning traces. We demonstrate the utility of \name\ by using these metrics to distill insights regarding social reasoning capabilities and limitations in state-of-the-art (SOTA) multimodal models.  For example, we find that models struggle to perform well under both zero-shot and in-context learning (ICL) settings, demonstrating the significant challenge of building this understudied form of reasoning in models. Our findings contribute novel insights regarding gaps and opportunities for future research to improve the grounded social reasoning abilities of multimodal models.  


\section{Background}
\label{sec:related_work}

% \subsection{Multimodal Social Reasoning}
% \label{subsec:sr}
Prior research on social reasoning in models has primarily focused on the ability of models to interpret text-based social scenarios and perform question-answering (QA) tasks about characters' motivations, intents, and  actions; Social IQa remains a key unimodal benchmark in this area \cite{sap-etal-2019-social}. SOTA language models can accurately perform a majority of the inferences in Social IQa, but a gap remains between model and human performance \cite{sap-etal-2022-neural, shapira-etal-2024-clever}. SOTA models have also struggled with text-based QA tasks that probe  competencies relevant to social reasoning, specifically theory-of-mind to interpret the goals and beliefs of characters \cite{le2019revisiting, shapira-etal-2024-clever, ullman2023large, kim2023fantom}. Crowd-sourced knowledge bases of  norms \cite{forbes-etal-2020-social, ziems-etal-2023-normbank} have been useful to inform social reasoning research.

The ability of models to perform social reasoning on multimodal human interactions, in particular \textit{face--to--face}, \textit{embodied}, \textit{real-world} social interactions, has been comparatively understudied. Key benchmarks include the video QA tasks of Social-IQ 1.0 \cite{zadeh2019social} and Social-IQ 2.0 \cite{wilf2023social};  both examine model QA accuracy when answering questions about social interactions in videos. SOTA models have struggled to perform well on Social-IQ 2.0 \cite{xie2023multi, pirhadi2023just, li2024llms, agrawal2024listen, chen2024through}. The community's prior focus on QA \textit{accuracy} to assess social reasoning ability has not enabled researchers to study the extent to which models can effectively reference multimodal cues and external knowledge informing inferences. Models with high accuracy on QA tasks can perform poorly at generating valid or comprehensive reasoning traces \cite{jhamtani2020learning, gu2023digital}. Model reasoning trace quality has not been studied before in the context of multimodal social reasoning. We introduce \name\ as the first benchmark to study grounded, fine-grained social reasoning abilities of multimodal models. 

\section{Building \name\ }
\label{sec:datset}

\subsection{Sourcing Seed Videos and Questions}
\label{subsec:seed_videos}
The \name\ benchmark contains 272 seed videos and 1486 questions adapted from the publicly-available \textsc{Social-IQ 2.0} dataset \cite{wilf2023social} (details on data sourcing in Appendix \ref{subsec:sourcing}). Videos include real-world face-to-face dyadic and multi-party interactions  (1 minute per video, $\sim$4.5 total hours), and questions probe social dynamics, behaviors, emotions, and cognitive states of both individuals and groups. From seed videos and questions, \name\ introduces a new set of 1486 human reasoning traces with 5700+ steps that answer these questions.  

\subsection{Task Notation}
\label{subsec:task}
Given a video \( V \) depicting a social interaction, a question \( Q \) about social content in the interaction, and a corresponding set of answer options \( A = \{A_{\text{correct}}, A_{\text{incorrect}_1}, A_{\text{incorrect}_2}, A_{\text{incorrect}_3}\} \), a model performing the \name\ task must generate a reasoning trace \( R = \{e_1, e_2, \dots, e_n\} \), where each reasoning step \( e_i \) represents a single piece of evidence contributing toward the  social inference to select an answer \(A_{\text{a}}\) from \( A \). Each reasoning step \( e_i \) must be tagged with two attributes: (1) a modality tag \(m_i \in \{\textit{visual, verbal, vocal, } n/a\}\) indicating the communication modality of the evidence and (2) an external knowledge tag \(k_i \in \{\text{\textit{yes}, \textit{no}}\}\), indicating whether the evidence references external knowledge of contextual information. This task to generate \(\mathit{R}\) and answer question \textit{Q} 
evaluates a model’s ability to extract and reference multimodal aspects of human communication and knowledge informing social inferences. Given the input tuple \( (V, Q, A) \), each model performing the \name\ task will produce an output tuple \( (A_{\text{a}}, R) \). Metrics in \name\ study the social inference accuracy of \(A_{a}\) and the semantic and structural aspects of social reasoning in \(R\). 


\subsection{Social Reasoning Trace Annotations}
\label{subsec:annotation}

% This section describes our process to collect social reasoning trace annotations for \name.  

\paragraph{Human Annotation} 
Given a video \( V \),  question \( Q \), and  answer options \(A\), 
 annotators read \( Q \) and \( A\), watched  \( V\), and wrote reasoning trace \(R\). Annotations were collected with an IRB-approved Prolific study (details in Appendix \ref{subsec:annotation_appendix}). 


\paragraph{Grounded and Fine-Grained Behaviors} Humans perceive and build upon low-level observations of fine-grained behaviors (e.g, shifts in body language) and high-level, top-down processing (e.g., implicit situational knowledge) when interpreting social scenes \cite{baird2001making, bodenhausen2013social}. Annotators were instructed to reference any \textit{low-level} and \textit{high-level} evidence they relied upon to answer questions: for example, low-level evidence might be "\texttt{the woman takes a step back with her mouth wide open (visual cue)}" and high-level evidence that interprets that low-level cue might be "\texttt{the woman is surprised (external knowledge regarding how 'surprise' might manifest")}. For each step, annotators tagged the perceptual modality referenced (visual, verbal, vocal, n/a). 

\paragraph{Grounded External Knowledge} Annotators tagged each reasoning step with \(yes\) or \(no\) to indicate whether external knowledge was referenced. External knowledge includes contextual norms, cultural expectations, and prior understanding of social commonsense \cite{forguson1988ontogeny} that goes beyond stimuli in the video. For example, if a man raises his arm and the annotator recognizes his movement as a "high five," the identification of that gesture and its interpretation as "triumphant" is based on external knowledge of social norms. 

\paragraph{Ensuring Annotation Quality} 
Trained experts validated each Prolific annotation. They watched each video, read each QA tuple, and read the annotation to ensure that traces represented valid reasoning, had correct modality and external knowledge tags, and referred to relevant information. Cases of incomplete annotation or deviation from instructions were fixed, as discussed in Appendix \ref{subsec:validation}.     

\subsection{Dataset Statistics}
\name\ contains 1486 detailed human-annotated social reasoning traces with 5,777 total steps, 3.89 $\pm$ 1.68 steps per trace (minimum of 1 step and maximum of 10 steps), 43 $\pm$ 26 words per trace, and 11$\pm$ 5 words per step.  Reasoning steps draw on multimodal evidence, with 44\% of steps referencing visual cues, 27\% referencing verbal cues, and 17\% referencing vocal cues. Overall, 77\% of traces referenced at least one visual cue, 63\% referenced at least one verbal cue, and 47\% referenced at least one vocal cue.

External knowledge plays a critical role in \name: 51\% of reasoning steps referenced external knowledge, with each trace referencing an average of 2 pieces of external knowledge. With spaCy named entity recognition (NER) \cite{Honnibal_spaCy_Industrial-strength_Natural_2020}, we found 11,253 entities (people, objects, concepts) mentioned, with 7.6 unique entities and 2.23 emotions referenced
per reasoning trace, demonstrating the high \textit{density} of annotations.


% \subsection{Qualitative Observations} Describe findings from manual inspection of data (e.g., how compositional were pieces of evidence, how accurate were the modality tags) etc. 


\subsection{Social Reasoning Metrics and Statistics}
\label{subsec:metrics}
We develop a set of metrics to holistically evaluate \textit{semantic} and \textit{structural} aspects of social reasoning traces generated by models performing tasks in the \name\ benchmark. Collectively, these metrics reveal strengths and weaknesses in model social reasoning and multimodal grounding abilities and the extent to which model  traces differ from human reasoning. For each sample, we compute the following metrics between model reasoning trace \(\mathit{R_M}\) with \(n\) steps $e_{1}...e_{n}$ and the corresponding human trace \(\mathit{R_H}\) with \(m\) steps $h_{1}...h_{m}$.

\paragraph{Accuracy} Measures accuracy of the model-generated answer by comparing it to ground truth. Human annotator accuracy on \name\ is 0.85 (Appendix \ref{subsec:human_acc}). Higher values indicate stronger social inference ability (max value is 1).  

\paragraph{Similarity-Trace (\(S_\text{trace}\))}  
Measures the high-level semantic similarity between \(\mathit{R_M}\) and \(\mathit{R_H}\):
\[
S_\text{trace} = 
\frac{\langle \mathbf{R_M}, \mathbf{R_H} \rangle}{\|\mathbf{R_M}\| \cdot \|\mathbf{R_H}\|}
\]
where \(\mathbf{R_M}\) is the aggregate embedding of evidence steps in \(\mathit{R_M}\) and \(\mathbf{R_H}\) is the aggregate embedding of evidence steps in \(\mathit{R_H}\). The embedding model \texttt{all-MiniLM-L6-v2} \cite{reimers-2019-sentence-bert}, selected for its efficiency and accuracy, was used to embed evidence steps for this and other semantic similarity metrics. Higher values indicate stronger alignment in semantic information between \(R_M\) and \(R_H\) (max value is 1).

\paragraph{Similarity-Step (\(S_\text{step}\))}  
Measures the fine-grained semantic similarity between \(\mathit{R_M}\) and \(\mathit{R_H}\). For each  step \(e_i\) in \(\mathit{R_M}\), the metric identifies its closest semantic step \(h_j\) in \(\mathit{R_H}\). The final metric is the mean of these maximum similarity values:  
\[
S_\text{step} = 
\frac{1}{n} \sum_{i=1}^n \max_{j} 
\frac{\langle \mathbf{e_i}, \mathbf{h_j} \rangle}{\|\mathbf{e_i}\| \cdot \|\mathbf{h_j}\|}
\]
where \(\mathbf{e_i}\) is the embedding of evidence step \(e_i\) and \(\mathbf{h_j}\) is the embedding of evidence step \(h_j\). Higher values reflect stronger alignment in semantic information between fine-grained steps of evidence in \(R_M\) and \(R_H\) (max value is 1).

\paragraph{Similarity-Num Steps (\(S_\text{num}\))}  
Measures the number of reasoning steps in \(\mathit{R_M}\) with a semantic similarity above a threshold \(\tau\), when compared to any step in \(\mathit{R_H}\). This metric is computed as:
\[
S_\text{num} = 
\sum_{i=1}^n \mathbbm{1} \left( 
\max_{j} \frac{\langle \mathbf{e_i}, \mathbf{h_j} \rangle}{\|\mathbf{e_i}\| \cdot \|\mathbf{h_j}\|} > \tau \right)
\]
where \(\mathbbm{1}(\cdot)\) is the indicator function and \(\tau=0.6\) (empirically selected). Higher values indicate more semantically-aligned evidence between \(R_M\) and \(R_H\) (max value is $n$).

\paragraph{DifferenceSequence (\(DS\))}  
Measures structural similarity between \(\mathit{R_M}\) and \(\mathit{R_H}\) using the respective modality sequences \(\mathit{S_M}\) and \(\mathit{S_H}\)  from the reasoning traces (e.g., ["visual", "external knowledge"]). A similarity score based on edit distance, adapted from the Levenshtein distance, is computed between \(\mathit{S_M}\) and \(\mathit{S_H}\) (Appendix \ref{subsec:diffseq}). Higher values of \(DS\) indicate greater structural similarity between \(R_M\) and \(R_H\) (max value is 1).

\paragraph{EmotionMetric}  
Measures the alignment of emotional content in \(\mathit{R_M}\) and \(\mathit{R_H}\). This metric extracts sets of emotions referenced by \(\mathit{R_M}\) and \(\mathit{R_H}\) (instructions in Appendix \ref{subsec:emotionmetric}) and computes the overlap in sets. Higher values can indicate stronger emotional alignment between \(\mathit{R_M}\) and \(\mathit{R_H}\) (max value is emotion set size in \(\mathit{R_H}\)). 

\paragraph{All Modality Steps}  
Measures the overlapping number of unique modalities in \(\mathit{R_M}\) and \(\mathit{R_H}\). Higher values indicate that \(\mathit{R_M}\) had more overlap with modalities referenced by \(\mathit{R_H}\) (max value is the number of unique modalities in \(\mathit{R_H}\)).

\paragraph{Visual Steps} 

% [TODO just write the full paragraph for the first one, and say we do the same for other modalities]
Measures the number of steps in both \(\mathit{R_M}\) and \(\mathit{R_H}\) with \textit{visual} evidence, to
% If \(\mathit{R_M}\) ha \(|\mathrm{Vis}_{\mathit{M}}|\) steps with visual evidence, and \(\mathit{R_H}\) has \(|\mathrm{Vis}_{\mathit{H}}|\) steps with visual evidence, the metric is computed as \(\min\left(|\mathrm{Vis}_{\mathit{M}}|, |\mathrm{Vis}_{\mathit{H}}|\right)\).
evaluate how closely \(\mathit{R_M}\) aligns with \(\mathit{R_H}\), with respect to the presence of visual evidence (max value is number of visual steps in \(\mathit{R_H}\)).


\paragraph{Verbal Steps} 
Measures the number of steps in both \(\mathit{R_M}\) and \(\mathit{R_H}\) with \textit{verbal} evidence,
% If \(\mathit{R_M}\) has \(|\mathrm{Ver}_{\mathit{M}}|\) steps with verbal evidence, and \(\mathit{R_H}\) has \(|\mathrm{Ver}_{\mathit{H}}|\) steps with verbal evidence, the metric is computed as \(\min\left(|\mathrm{Ver}_{\mathit{M}}|, |\mathrm{Ver}_{\mathit{H}}|\right)\). 
% This metric is computed with the same process described above for visual steps; the max value is \(|\mathrm{Ver}_{\mathit{H}}|\), the number of verbal steps in  \(\mathit{R_H}\).
to evaluate how closely \(\mathit{R_M}\) aligns with \(\mathit{R_H}\), with respect to the presence of verbal evidence (max value is the number of verbal steps in \(\mathit{R_H}\)).

% \paragraph{Vocal Steps} Measures the number of reasoning steps in both \(\mathit{R_M}\) and \(\mathit{R_H}\) that reference \textit{vocal} evidence. If \(\mathit{R_M}\) has \(|\mathrm{Voc}_{\mathit{M}}|\) steps with vocal evidence, and \(\mathit{R_H}\) has \(|\mathrm{Voc}_{\mathit{H}}|\) steps with vocal evidence, the metric is computed as \(\min\left(|\mathrm{Voc}_{\mathit{M}}|, |\mathrm{Voc}_{\mathit{H}}|\right)\). This metric evaluates how closely \(\mathit{R_M}\) aligns with \(\mathit{R_H}\), with respect to the presence of vocal evidence (max value is \(|\mathrm{Voc}_{\mathit{H}}|\)).
\paragraph{Vocal Steps}
Measures the number of steps in both \(\mathit{R_M}\) and \(\mathit{R_H}\) with \textit{vocal} evidence to evaluate
% This metric is computed with the same process described above for visual steps; the max value is \(|\mathrm{Voc}_{\mathit{H}}|\), the number of vocal steps in  \(\mathit{R_H}\).
how closely \(\mathit{R_M}\) aligns with \(\mathit{R_H}\), with respect to the presence of vocal evidence (max value is the number of vocal steps in \(\mathit{R_H}\)).

% \paragraph{EK Steps} Measures the number of reasoning steps in both \(\mathit{R_M}\) and \(\mathit{R_H}\) that reference \textit{external knowledge} evidence. If \(\mathit{R_M}\) has \(|\mathrm{EK}_{\mathit{M}}|\) steps with EK evidence, and \(\mathit{R_H}\) has \(|\mathrm{EK}_{\mathit{H}}|\) steps with EK evidence, the metric is computed as \(\min\left(|\mathrm{EK}_{\mathit{M}}|, |\mathrm{EK}_{\mathit{H}}|\right)\). This metric evaluates how closely \(\mathit{R_M}\) aligns with \(\mathit{R_H}\), with respect to the presence of EK evidence (max value is \(|\mathrm{EK}_{\mathit{H}}|\)).
\paragraph{External Knowledge Steps}
Measures the number of steps in both \(\mathit{R_M}\) and \(\mathit{R_H}\) with \textit{external knowledge} evidence, to 
% This metric is computed with the same process described above for visual steps; the max value is \(|\mathrm{EK}_{\mathit{H}}|\), the number of external knowledge steps in  \(\mathit{R_H}\).
evaluate how closely \(\mathit{R_M}\) aligns with \(\mathit{R_H}\), with respect to external knowledge references (max value is the number of external knowledge steps in \(\mathit{R_H}\)).

\paragraph{NumSteps (\(NS\))}  
Measures the absolute difference in the number of reasoning steps between \(\mathit{R_M}\) and \(\mathit{R_H}\). Lower values indicate stronger alignment in length between model and human chains (value of 0 indicates that \(\mathit{R_H}\) and \(\mathit{R_H}\) are the same length). 


\begin{figure*}[t]
    \centering
\includegraphics[width=0.96\linewidth]{figures/combined_full.png}
    \caption{Performance of models across number of few-shot ICL samples ($k$) and ground truth noted in gray.\protect\\\textbf{With these metrics, \name\ enables a holistic study of multimodal, grounded social reasoning.}\protect\\The first six metrics focus on social inference accuracy, semantic similarity, and structural similarity between model and human reasoning traces. The final six metrics focus on fine-grained multimodal evidence and external knowledge referenced by models, in comparison to evidence referenced by humans.}
    \label{fig:overall}
\end{figure*}


\subsection{\name\ ICL Training Set}
To create samples for ICL experiments, we randomly sampled 16 questions from unique videos in the \textit{training} set of \textsc{Social-IQ 2.0} and collected reasoning trace annotations in the same format as annotations in Section \ref{subsec:annotation}. ICL experiments in Section \ref{sec:experiments} are conducted by providing each model with different numbers of training samples $k \in \{0, 1, 2, 4, 8, 16\}$, before the model is given an input tuple ($V$, $Q$, $A$) and generates a sequence of tokens with an answer $A_a$ and reasoning trace $R$. 

\section{Social Reasoning Experiments}
\label{sec:experiments}

We use \name\ to study the performance of  multimodal video understanding models in fine-grained, grounded social reasoning. These models exhibited SOTA performance on video understanding tasks and take videos as input. We tested 2 closed-source models and 5 open-source models: Gemini-1.5-Flash \cite{team2024gemini}, GPT-4o \cite{gpt4o}, LLaVA-Video and LLaVA-Video-Only \cite{zhang2024video}, LongVA \cite{zhang2024longva}, Video-ChatGPT \cite{maaz2023video}, and VideoChat2 \cite{li2023videochat}. Models have different architectures, pretraining data, and fine-tuning tasks, and models generated reasoning traces and answers for all samples (details in Appendix \ref{sec:models-appendix}). LLaVA-Video-Only answered questions, but did not generate reasoning traces that could be studied; this model does not appear in trace-related metrics. 

% \subsection{Using \name\ to Study Social Reasoning in Multimodal Models}

\subsection{Quantitative Results and Insights}
\label{subsec:quant}

Figure \ref{fig:overall} visualizes each model's average performance for our 12 metrics\footnote{Visualized metrics are normalized to [0,1] by human baselines, except for those with an upper bound of 1 by definition (Accuracy, Similarity-Trace, Similarity-Step, DifferenceSequence) and absolute measures (NumSteps).}, with human baselines in gray. Results tables (Tables \ref{tab:accuracy-metrics-comparison}, \ref{tab:core-metrics}, \ref{tab:modality-metrics}) are in Appendix \ref{sec:models-appendix}. Key findings are discussed below. 

\paragraph{Social Inference Accuracy} \textit{Human social inference ability is substantially higher than all models}, as seen in results from the \textbf{Accuracy} metric. Gemini-1.5-Flash and GPT-4o achieve the highest accuracies of 74.4\% and 71.0\% respectively ($k$ = 0), approximately 10-15\% lower than human annotator accuracy (85.3\%), answering $\sim$30\% of questions incorrectly. \textit{Closed-source models outperform open-source models in social inference}. The highest-performing open-source model was LLaVA-Video at 62.9\% ($k$ = 0). Gemini-1.5-Flash and GPT-4o are much larger than open-source models, suggesting that increased model \textit{scale} is useful, but not sufficient, for accurate social inference. Video-ChatGPT and VideoChat2 perform 30-40\% lower than other open-source models across all values of $k$. These models have the smallest context length and process the fewest frames, constraints which may influence performance. 

% \paragraph{Social Inference and In-Context Learning} 
\textit{Social inference accuracy for models decreased as the number of few-shot ICL samples increased}, with the exception of GPT-4o, which demonstrated a slight improvement at $k$ = 16. Few-shot ICL conditions language models on tasks by providing examples of inputs and outputs \cite{liu2024incomplete} and can be viewed as a form of  \textit{inductive reasoning}, as the model is tasked with inferring generalizable rules from a set of examples. This technique has improved model reasoning abilities in domains such as mathematics and code generation \cite{dong2024survey,zhou2022teaching,patel2023evaluating}, which have \textit{explicit} rules and formal structure \cite{galotti1989approaches}. In contrast to these domains, social reasoning often operates with \textit{implicit} rules, less formal structure \cite{perkins1989reasoning} and ambiguity in premises \cite{mathur-etal-2024-advancing}. Our findings suggest that few-shot ICL may not be an effective approach to elicit multimodal social reasoning abilities. Additional experiments (provided in Appendix \ref{sec:extra_experiments}) demonstrate that chain-of-thought prompting did not improve model accuracy, and models struggled to perform inferences relying on implicit and contextual knowledge. Our finding from \name\ that few-shot ICL is insufficient to elicit multimodal social reasoning aligns with findings from unimodal experiments on \textsc{Social-IQa} and \textsc{ToMi} datasets \cite{le2019revisiting, sap-etal-2019-social,sap-etal-2022-neural,kim2023fantom}. 


\paragraph{Semantic Alignment} \textit{It is challenging for models to generate social reasoning traces with high semantic alignment to human reasoning}, as seen in the results from the \textbf{Similarity-Trace}, \textbf{Similarity-Step}, and \textbf{Similarity-Num Steps} metrics. For Similarity-Trace, only Gemini-1.5-Flash achieved slightly above 50\% ($k$ = 0), and LongVA outperformed Gemini-1.5-Flash after $k$ = 4. For Similarity-Step, no models achieved above 50\%, but Video-ChatGPT outperformed all models for $k \in \{0, 1, 2, 4\}$ (5-12\% higher than GPT-4o and 2-5\% higher than Gemini-1.5-Flash). For Similarity-Num Steps, Gemini-1.5-Flash achieved the highest performance (27\%), far below  ground truth. Low performance can be explained by failure to reference cues that humans reference (``Fine-Grained Grounding" metrics below). Performance on these metrics did not improve as $k$ increased.

\paragraph{Structural Alignment} \textit{Model reasoning traces tend to reference multimodal evidence in different amounts and orders than humans}, as seen in the results from the \textbf{DifferenceSequence} metric. Model-generated sequences of multimodal evidence that were most structurally-aligned with human sequences were from Gemini-1.5-Flash (0.53 at $k$ = 4) and LLaVA-Video (0.49 at $k$ = 4), both far below the maximum alignment value (1). As $k$ increased, DifferenceSequence metric scores increased for Gemini-1.5-Flash, LLaVA-Video (up to $k$ = 4), and GPT-4o (after $k$ = 2). These findings suggest that structured samples of human social reasoning, as introduced by \name, can be useful when conditioning models to generate reasoning traces with more human-like structure.

\paragraph{Emotion Alignment} \textit{It is challenging for models to generate social reasoning chains with high  emotional alignment with human reasoning}, as seen in the results from the \textbf{EmotionMetric}. All models achieved less than 22\%, far below ground truth. While the highest scores were achieved by Gemini-1.5-Flash and GPT-4o at $k$ = 0, scores from several models steadily improved after $k$ = 2. 

\begin{figure*}[t]
    \centering
    \includegraphics[width=1\linewidth]{figures/social-genome_figures_21.png}
    \caption{Representative social reasoning traces from a Human Annotator, Gemini-1.5-Flash, and LLaVA-Video. These examples illustrate the grounded social reasoning abilities and reasoning structures across humans and models.}
    \label{fig:qualitative}
\end{figure*}

\paragraph{Fine-Grained Multimodal Grounding} The final 6 metrics study evidence referenced by model reasoning traces. The \textbf{AllModality Steps} metric serves as a proxy for how well models refer to fine-grained, multimodal cues and external knowledge. As seen in AllModality Steps results, \textit{all models referenced fewer pieces of multimodal evidence and external knowledge than human reasoning.}

While we discuss modality-specific findings below, our use of \name\ to study SOTA models is currently limited by their varying abilities. The Gemini-1.5-Flash API reports that audio is processed in-parallel on the backend, but other models do not process audio. However, despite the absence of audio, models referenced verbal and vocal evidence \textit{inferred} from visual frames. For example, LongVA generated vocal evidence about a woman's "tone suggesting frustration," after generating visual evidence about the woman's face appearing dissatisfied. As new models are developed in the coming years with abilities to jointly process video and audio, \name\ will continue to be applicable to study model abilities in grounded multimodal social reasoning.

\textbf{Visual Steps:} \textit{Closed-source models exhibited a strong ability to reference visual evidence}, with GPT-4o and Gemini-1.5-Flash closer to ground truth than open-source models. Performance on this metric for all models was highest at $k$ = 0. 

\textbf{Verbal Steps:} \textit{The ability of models to reference verbal evidence showed substantial variation}. Gemini-1.5-Flash exhibited the strongest ability to reference verbal cues (0.90 at $k=0$).
% We investigated LLaVA-Video's high performance at $k$ = 8 and VideoChat2's high performance at $k$ = 2; when referencing vocal information, these models primarily repeated cues from the ICL prompts. This reinforces the effectiveness of taking a holistic approach to study reasoning trace quality -- alongside metrics such as Verbal Steps, the previously-discussed semantic metrics reveal the disparity between semantic content of these traces and human traces (Similarity-Step values for LLaVA-Video and VideoChat2 are close to 0).
Model performance on this metric did not improve as $k$ increased, and GPT-4o referenced substantially fewer verbal cues in comparison to other models. 

\textbf{Vocal Steps:} \textit{The ability of models to reference vocal evidence was substantially lower than human ability}. LongVA referenced more vocal evidence than other models at $k$ = 0. Model performance on this metric did not improve as $k$ increased. 

\textbf{External Knowledge Steps:} \textit{The ability of models to reference external knowledge in social reasoning traces was lower than humans}. In contrast to trends observed in other modality step metrics, we find that providing additional few-shot samples improved the ability of Gemini-1.5-Flash, GPT-4o, LongVA (up to $k$ = 4), and LLaVA-Video (up to $k$ = 4) to reference external knowledge. These findings suggest that human social reasoning traces can be used to condition models to ground social reasoning with external knowledge references.  

\textbf{Num Steps}: \textit{Model reasoning traces varied in length and contained more steps than human traces}. LLaVA-Video and LongVA generated model reasoning traces that were most aligned with the length of human traces. Providing additional few-shot samples improved the ability of models to align with human social reasoning trace length. 

\subsection{Human and Qualitative Evaluation}
\label{sec:human}
We conducted human evaluation of model reasoning traces. Trained annotators analyzed 48 samples from all models, with model names and $k$ values  anonymized. Traces were rated to assess references to low-level cues (\textit{fine-grained}), information cross-referenced across steps (\textit{compositional}), relevant evidence (\textit{comprehensive}), \textit{correctness} of modality tags, and \textit{validity} of reasoning (criteria are in Appendix \ref{subsec:qual_results}). Annotator agreement was computed with Cohen's Kappa $\kappa$ \cite{cohen1960coefficient}.

Reasoning traces from Gemini-1.5-Flash and GPT-4o received the highest ratings for \textit{fine-grained} ($\kappa$ = 0.87), \textit{compositional} ($\kappa$ = 0.92), \textit{comprehensive} ($\kappa$ = 0.94), and \textit{valid} ($\kappa$ = 0.94) reasoning, followed by LLaVA-Video and LongVA, with VideoChatGPT and VideoChat2 rated lowest. These findings validate model performance trends observed in Section \ref{subsec:quant}. Modality tag correctness across samples was 98\%, supporting the validity of the \name\ automated modality metrics and reasoning trace processing (Appendix \ref{subsec:chain_processing}).

We visualize representative traces in Figure \ref{fig:qualitative} from a human annotator, Gemini-1.5-Flash, and LLaVA-Video (closed-source and open-source models with highest inference accuracy at $k$ = 0). 

\paragraph{Evidence and Error Propagation} This figure illustrates the strong ability of Gemini-1.5-Flash to reference and integrate multimodal cues and external knowledge during social reasoning. In contrast, LLaVA-Video did not reference low-level cues and based its reasoning upon an incorrect premise, and this error propagated to an incorrect inference. In addition, the human trace referred to several \textit{fine-grained, subtle behaviors }(e.g., lip movements) that are \textit{not} present in the model traces, yet do influence the scene interpretation. Similarity metrics and modality-specific metrics in Section \ref{subsec:metrics} serve as automated proxies to estimate these types of information gaps between human and model reasoning, as shown in these qualitative examples. Our findings motivate future work on model training data, training approaches, and architectures that better capture fine-grained cues, as well as algorithms to handle error propagation in social reasoning.   


\paragraph{Hierarchical Social Reasoning} Figure \ref{fig:qualitative} qualitatively illustrates the \textit{hierarchical structure} of human social reasoning traces, in which low-level cues (e.g., brief lip movement) are referenced, combined, and re-interpreted as \textit{intermediate evidence} for further reasoning. This ``forking" reasoning structure is common for humans interpreting everyday situations, unlike the linear ``long chain" structure for formal reasoning in domains such as mathematics \cite{perkins1989reasoning, galotti1989approaches}. As seen in Figure \ref{fig:qualitative}, compared to human traces, model traces have flatter structures with less hierarchical composition of social cues. Flatter structures have the potential to treat cues as isolated evidence and overlook intermediate evidence in reasoning paths. Our findings motivate future work to train models capable of more hierarchical social reasoning.


% \begin{figure}[h]
%     \centering
%     \includegraphics[width=\linewidth]{figures/llava_next_tvqa_graph.png}
%     \caption{Caption}
%     \label{fig:enter-label}
% \end{figure}


\section{Conclusion}
\label{sec:conclusion}

This paper introduces \name, the first benchmark for fine-grained, grounded social reasoning abilities of multimodal models. The reasoning traces contributed by \name\ include references to multimodal cues and external knowledge concepts that humans find useful when performing social inferences. We define metrics to assess semantic and structural aspects of reasoning traces, and we contribute novel insights regarding gaps and opportunities to improve the grounded social reasoning capabilities of multimodal models.

Future AI systems interacting with humans in face-to-face, embodied contexts must have the ability to reason about social interactions and \textit{ground} reasoning in concrete multimodal evidence and external knowledge concepts. \name\ serves as a first step towards studying and advancing this form of reasoning in AI systems. 








% \section*{Acknowledgments}
% This material is based upon work partially supported by National Institutes of Health awards \textcolor{blue}{add grants here @LP}. Leena Mathur is supported by the NSF Graduate Research Fellowship Program under Grant No. DGE2140739. Any opinions, findings, conclusions, or recommendations expressed in this material are those of the authors and do not necessarily reflect the views of the sponsors, and no official endorsement should be inferred. \textcolor{blue}{add figure citations @Leena}

\section{Limitations}
\label{sec:limitations}

\paragraph{Social Reasoning in Natural Language} The current scope of \name\ focuses on studying model-generated social reasoning traces in \textit{natural language}. This scope is necessary and relevant to contexts that require AI systems to generate natural language explanations of social inferences -- for example, a healthcare agent or hospital robot reasoning about human nonverbal behaviors during a nurse-patient social interaction. However, an open question remains regarding the extent to which natural language can effectively represent the nuances of human social interactions and social reasoning \cite{mathur-etal-2024-advancing}. It is possible that both humans and models verbalizing social reasoning through natural language are not fully  capturing \textit{why} they came to certain inferences \cite{turpin2023language}. Several lines of work in reasoning have operated in the \textit{latent space} of models instead of natural language \cite{hao2024training, geiping2025scaling}, and we believe \name\ informs and motivates future work to develop techniques to study social reasoning in the latent space. 

\paragraph{Video Contexts} The videos in \name\ each have a length of 1 minute. We believe that findings from \name\ regarding model social reasoning abilities about interactions in these 1-minute clips motivate future work to curate longer-form video benchmarks. Social interactions during hour-long videos \cite{zou2025hlv}, for example, would contain more complex social dynamics and longer-form temporal dependencies across humans.  In addition, all videos in \name\ have social interactions in English, and all annotators were required to be proficient in English. As social norms can differ across cultures and languages \cite{hershcovich2022challenges}, we believe future research in multimodal social reasoning would benefit from a larger-scale curation of multimodal interaction data across cultural and linguistic contexts.

% \paragraph{Self-Reports}






% - rely on self-reports of reasoning traces? although this is common practice when obtaining reasoning traces for other domains


% \paragraph{Modality Encoders} Our evaluation of current SOTA VLMs on \name\ in this paper was scoped to focus on 6 video understanding models with diverse architectures, parameters, pretraining data distributions, and finetuning tasks. 


% Current SOTA video understanding models, with the exception of Gemini-1.5-Flash, do not contain audio encoders. While conducting research for this paper, the Gemini-1.5-Flash API took videos as input, extracted frames at 1 FPS, and processed single channel audio at 1Kbps. This audio processing rate may not be effective for social understanding tasks that require fine-grained synchronization with vocal or verbal cues. In addition, at the time of writing this paper, GPT-4o, LLava-Video, LLaVA-Video-Only, LongVA, Video-ChatGPT, and VideoChat2 accepted text and video inputs (no audio channel) -- video frames are the only aspects of video sampled and processed by these models. Current FPS processing rates of SOTA models are unlikely to capture fast or subtle motion dynamics during social interactions.  

% We believe that \name\ represents an enduring benchmark that will continue to be useful in the future to study video understanding models, as their ability to handle modalities such as audio improves. 

% Focus on verbalized reasoning -- some constraints there in what can be verbalized. 


\section{Ethics}
\label{sec:ethics}

\paragraph{Ethical Annotation} We curated  annotations from videos in existing publicly available datasets. We hired workers from Prolific to annotate reasoning traces. All workers received fair compensation for their annotation (\$12 per hour, pro-rated). Worker privacy and confidentiality were respected, with no identifiable information stored. Further details on Prolific annotation are in Appendix \ref{subsec:annotation_appendix}.

\paragraph{Risks for Social Reasoning in AI}

Social reasoning abilities are essential for future AI systems to effectively work with and alongside humans in real-world settings. \name\ has the potential to support the research community in studying and advancing these capabilities in AI systems. We envision AI systems using social reasoning to enhance human autonomy, health, and well-being -- for example, a healthcare agent reasoning about a patient's nonverbal behaviors, in order to help doctors better support them. However, these technologies exist within a broader societal context, with multiple stakeholders and potential risks in areas such as privacy, surveillance, and manipulation. We support broader academic research and policy efforts to mitigate against misuse and potential harms of socially-intelligent AI.  



\section*{Acknowledgements}

Leena Mathur is supported by the NSF Graduate Research Fellowship Program under Grant No. DGE2140739. This material is based upon work partially supported by National Institutes of Health awards R01MH125740, R01MH132225, and R21MH130767.  Any opinions, findings, conclusions, or recommendations expressed in this material are those of the authors and do not necessarily reflect the views of the sponsors, and no official endorsement should be inferred. Figure \ref{fig:overview} includes icon material available from https://icons8.com. 

% Bibliography entries for the entire Anthology, followed by custom entries
%\bibliography{anthology,custom}
% Custom bibliography entries only
\bibliography{custom}


\newpage
\clearpage
\appendix

\section{\name\ Dataset Appendix}
\label{sec:data_appendix}

\subsection{\name\ Data Sourcing}
\label{subsec:sourcing}
All videos in \name\ were sourced from publicly-available \textsc{Social-IQ 2.0} modeling challenge \cite{wilf2023social}. We obtained permission from the authors to access the original \textit{test set} videos and question-answer tuples from this dataset (275 videos and 1514 QA Tuples). Our use of these videos aligns with Social-IQ 2.0 repository's MIT license and intended research purpose. We chose to use these test set videos as candidate seed videos to build \name\ because the answers to questions posed about these videos \textit{have not been released online}, reducing chances of benchmark contamination \cite{xu2024benchmark}. At the current stage, we plan to avoid benchmark contamination in \name\ by not uploading human social reasoning trace annotations online and using them for research purposes. 

We note that prior papers that test models on \textsc{Social-IQ 2.0} have used the validation set, not the test set \cite{xie2023multi, pirhadi2023just, guo2023desiq, li2024llms, agrawal2024listen, chen2024through}.  Therefore, we do not directly compare prior works' validation set performance with our results on the test set. 

During manual inspection of each video and QA tuple, we filtered out QA tuples with answer options containing ambiguity -- if at least 2 annotators judged a question to have ambiguous answer options (at least two plausibly-correct answers), we discarded the question. The resulting \name\ set contained 272 videos and 1486 QA tuples. These samples contain face-to-face dyadic and multi-party social interactions and questions that probe understanding of affective states, causal social dynamics, and social events. 


\subsection{\name\ Annotation }
\label{subsec:annotation_appendix}

% IRB STUDY2023_00000489

Annotators were recruited on the Prolific\footnote{https://app.prolific.com} platform to perform an IRB-approved study with informed consent regarding our intended data use. Screens for annotators were employed to ensure that participants were adults, had access to a desktop to watch the videos and perform the annotation, were fluent in English, were based in the United States, and completed at least a high school education, and had high prior task approval rates on Prolific (97-100\% completion). Each annotator watched 2 videos and provided chains for all questions associated with each video (approximately 10 questions total per annotation). Instructions for annotators are listed in Figure \ref{fig:prolific}. Annotators were compensated at \$12 per hour pro-rated. This annotation process was time-intensive, with each human video annotation taking approximately $\sim$25 minutes. 

Before running the Prolific study, we ran 4 iterations of annotation instructions with pilot groups to refine instructions. We experimented with instructions that had annotators provide reasoning traces without knowing the correct answer and while knowing the correct answer. We found that providing annotators with the correct answer option did not change the detail, structure, or coherence of the reasoning traces generated. Therefore, we chose to provide annotators with the correct answer option (Figure \ref{fig:prolific}) and focus the large-scale Prolific data collection on obtaining detailed reasoning traces, instead of additional QA answer responses.  

\begin{figure*}[t]
\small
\centering
\begin{tcolorbox}[colframe=black!25, colback=white!95, coltitle=black, title=Annotation Instructions, width=0.95\textwidth, fonttitle=\bfseries]

In this study, you will be watching short videos that contain social interactions. You will be briefly explaining your thought process along several dimensions while answering questions about each video. Your answers will be entered through a Google spreadsheet.
Participants in this study must have the ability to watch videos and write logical text responses in English describing their step-by-step reasoning while interpreting videos. 

\vspace{0.5em}

If you give your consent to participate in this study, please select "I agree":  
\vspace{0.5em}
\texttt{<I agree>}

(If the annotator declines, they move to a page that states: ``As you do not wish to participate in this study, please return your submission on Prolific by selecting the 'Stop without completing' button.")

\vspace{0.5em}

In the provided spreadsheet (link below), you will see 2 videos linked in the "Video URL" column and a set of corresponding questions and answers for each video. The yellow-highlighted box indicates the correct answer for each question. Your goal is to do the following:

\begin{enumerate}
    \item Click the blue URL next to each set of questions in the "Video URL" column. Watch the video and pay attention to the behaviors of people interacting. 
    \item For each question associated with each video, provide the reasoning chain in order to answer the question:
    \begin{enumerate}
        \item Read the question, read the answer options, and note the correct answer (highlighted in yellow).
        \item In the "Evidence" column, describe the reasoning steps you took to reach the correct answer. Put each step in a separate row in the spreadsheet. (If applicable) steps should start with low-level observations (behavioral information from the video OR external knowledge) and build up to higher-level concepts.
    \end{enumerate}
    \item For each piece of evidence, in the same row, fill out the "Modality" column drop-down menu to indicate whether your evidence is from visual information (facial movements, gestures, anything vision-related, etc), vocal information (e.g., audio/speech), verbal information (e.g., the actual content of words being spoken), or external knowledge (e.g., cultural norms, your own understanding of what a behavior means).
    \item Please aim for 5-10 reasoning steps per question (if applicable).
\end{enumerate}

We will look at the spreadsheet during our manual review of submissions. All questions must have evidence presented with the answer selected: \texttt{<link to the spreadsheet>}

\end{tcolorbox}
\caption{Sample instructions provided to annotators on Prolific to provide \name\ reasoning traces.}
\label{fig:prolific}
\end{figure*}


\subsection{\name\ Validation}
\label{subsec:validation}
After obtaining human-annotated reasoning chains from Prolific, we conduct validation to ensure the validity and correctness of annotations. Two authors manually watched each video, read each question and corresponding set of answer options, and checked whether chains (1) represented valid reasoning paths, (2) had correct evidence tags for \textit{visual, verbal, vocal} and \textit{external knowledge}, and (3) comprehensively referred to relevant low-level social information in videos. If an annotator did not complete an annotation in a satisfactory manner (e.g., incomplete reasoning chain, minimal effort), the task was returned to them on Prolific. If annotations could be rapidly fixed (e.g., changing an incorrect modality tag from "visual" to "vocal"), the authors performed this fix themselves without sending the annotation task to the annotator. 

\subsection{\name\ Human Accuracy}
\label{subsec:human_acc}
For each video in \name, two human annotators on Prolific watched the video, read each question, and selected one of four provided answer options. Annotators were paid \$12 per hour (pro-rated) for completing each study. For each question, annotators also had the option to indicate whether or not they were ``uncertain" about the answer option they selected, and the authors examined these samples to ensure all QA tuples in \name\ had correct answers, to avoid the situation in which a sample could have more than one plausibly-correct answer option. Human annotators on Prolific achieved an accuracy of 85.3\% on the 1486 questions in \name. Inter-annotator agreement among Prolific annotators was positive (Cohen's $\kappa$ = 0.60) \cite{cohen1960coefficient}, and answer correctness was confirmed independently by authors, as described earlier. 

\section{\name\ Metrics Appendix}
\label{sec:metrics-appendix}

\subsection{Embeddings for Semantic Similarity} 
\label{subsec:semantic_embedding}

The metrics Similarity-Trace, Similarity-Step, and Similarity-Num Steps are computed with embeddings from the \texttt{all-MiniLM-L6-v2} from Sentence-BERT \cite{reimers-2019-sentence-bert}. We found this embedding strong, efficient, and useful for our task; as future embedding models are enhanced and released in the coming years, the \name\ framework allows for the embedding model called by semantic similarity metrics to be updated. 

\subsection{DifferenceSequence Metric}
\label{subsec:diffseq}
For model reasoning chain \(R_M\) with modality sequence \(S_M\) and human reasoning chain \(R_H\) with modality sequence \(S_H\), the DifferenceSequence ($DS$) metric is computed as a normalized similarity score by adapting the Levenshtein  distance \cite{levenshtein1966binary}  between \(S_M\) and \(S_H\):

\[
DS = 1 - \frac{\text{Levenshtein}(\mathit{S_M}, \mathit{S_H})}{|\mathit{S_M}| + |\mathit{S_H}|}
\]
\[
\text{Levenshtein}(S_M, S_H) \text{ is the following:}
\hspace*{5cm} % Adjust this value to shift it further left, if needed
\]
\[
= \begin{cases}
|S_M|, \small \text{if } |S_H|=0\\
|S_H|, \small \text{if } |S_M|=0\\ 
\min \Big[
\text{Levenshtein}(S_M[:-1], S_H[:-1]) + \delta, \\
\hspace{1em}\text{Levenshtein}(S_M[:-1], S_H) + 1, \\
\hspace{1em} \text{Levenshtein}(S_M, S_H[:-1]) + 1
\Big] \small\text{otherwise}
\end{cases}
\]
with \(\delta = \mathbbm{1}(S_M[-1] \neq S_H[-1])\), where \(\mathbbm{1}(\cdot)\) returns \(1\) if the final elements differ and \(0\) otherwise. 
We use an implementation\footnote{https://rapidfuzz.github.io/Levenshtein/index.html} that treats a substitution as equivalent to one insertion plus one deletion, making the distance effectively an InDel distance. We compute the minimum number of edits  that are needed to transform one sequence to another. Higher edit distances indicate that more edits are needed to align sequences (more dissimilarity).

Therefore, the overall \(DS\) similarity metric between \(S_M\) and \(S_H\) can range from 0 (maximum number of edits required) to 1 (minimum number of edits required). Higher \(DS\) values indicate greater structural similarity between the sequences \(\mathit{S_M}\) and \(\mathit{S_H}\), with respect to the type and order of modality evidence being referenced.


\begin{figure*}
\begin{tcolorbox}[colframe=black!25, colback=white!95, coltitle=black, title=\centering Model Prompt Template, width=\textwidth, fonttitle=\bfseries] 

Trace: Having a model provide a \textit{reasoning trace} in the response (remove the ``few-shot examples" for zero-shot experiments at $k$ = 0):

\begin{quote}
\texttt{Watch the video and answer questions about social information in the video, following the format of the examples below.}

\texttt{<FEW-SHOT EXAMPLE with Question and Answer Options>}

\texttt{Which of these is the correct answer? Output either A, B, C, or D.} \texttt{Provide the reasoning steps you took to get to this answer. Give 5 to 10 reasoning steps in bullet point form. Tag each bullet point step as visual, vocal, and verbal modality from the frames or external knowledge.  Use the format: The correct answer <insert answer>.The correct answer is <FEW-SHOT RESPONSE WITH ANSWER AND REASONING TRACE>.} 

\texttt{Which of these is the correct answer?  Output either A, B, C, or D.}  \texttt{<CURRENT QUESTION AND ANSWER OPTIONS> Provide the reasoning steps you took to get to this answer. Give 5 to 10 reasoning steps in bullet point form. Tag each bullet point step as visual, vocal, and verbal modality from the video or external knowledge.  Use the format: The correct answer is <insert answer>.} 
\end{quote}

\vspace{4mm}

No Trace: Having a model answer the question with \textit{no} reasoning trace (remove the ``few-shot examples" for zero-shot experiments at $k$ = 0): 

\begin{quote}
\texttt{Watch the video and answer questions about social information in the video, following the format of the examples below.}

\texttt{<FEW-SHOT EXAMPLE with Question and Answer Options>}

\texttt{Which of these is the correct answer? Output either A, B, C, or D. Use the format: The correct answer <insert answer>.The correct answer is <FEW-SHOT CORRECT ANSWER, FEW-SHOT REASONING CHAIN>.} 

\texttt{Which of these is the correct answer?  Output either A, B, C, or D.}  \texttt{<CURRENT QUESTION AND ANSWER OPTIONS> Use the format: The correct answer is <insert answer>.} 
\end{quote}

Note: For GPT-4o prompts, we found that we needed to replace the word \textit{video} with \textit{frames} to avoid  error messages associated with video processing (e.g., ``\textit{I am unable to view or analyze video frames directly. However, I can help answer questions based on descriptions or provide general information. Let me know how I can assist you!}"). Other models did not have this issue. 

\vspace{4mm}

% Few-shot prompting involves repeating an example (with the question, answer choices, correct answer, and provided reasoning chain). Only an example through text in the prompt is provided, and the videos corresponding to those questions are not provided. The only video provided to the model is the video corresponding to the current question. This is because the open-source models we used did not allow for interleaving multiple videos together for one prompt. 

\vspace{4mm}

\end{tcolorbox}
\caption{Information on model prompts to obtain reasoning traces and inferences from models.}
\label{fig:model_prompts}
\end{figure*}


\subsection{Emotion Named Entity Recognition}
\label{subsec:emotionmetric}
We perform NER with spaCy \cite{Honnibal_spaCy_Industrial-strength_Natural_2020}. In spaCy's NER v3 configuration, we broadly defined an \texttt{emotion} label as a \texttt{``description of how a person is feeling"}. To avoid identifying words like "feels" as entities, we also passed an example into the spaCy NER configuration that explicitly labeled "feels" as not an emotion entity (e.g., \texttt{``She feels sad because her friend didn't come with her"}). There are an average of 2.23\small$\pm$\normalsize1.63 emotion entities referenced per human reasoning chain. 



\subsection{Chain Processing for Metrics}
\label{subsec:chain_processing}
Computing metrics requires a standardized format for model-generated reasoning chains. Several model generations (in particular, the  generations from open-source models) required processing before metrics could be computed.  

We first parsed generations from models by splitting each generation based on its structure, such as the presence of line breaks, numbering, or sentences. For example, if a model generated a multi-sentence response, but did not include  line breaks or numbering within the response, we would split this model output by individual sentences (e.g., splitting on the  "." character).

We, then, parsed through each step and remove any steps that simply repeated the question or answer choices. We also removed phrases such as "reasoning step", "reasoning", or "the correct answer", as those phrases were often in steps like "The correct answer is A." or "Below are the reasoning steps:". In addition, during in-context learning experiments, we found that some model generations (in particular, GPT) would contain repetitions of sample chains within the model's response, leading the response to contain 2-3 reasoning chains. We automatically processed these responses by only taking the final reasoning chain out of these multiple chains and checking that this final chain was answering the original question. 

Models were tasked with tagging modalities for each step of their generated reasoning chains. To validate this process, we automatically checked whether each step included \textit{visual, vocal, verbal} or \textit{external knowledge} tags. However, models sometimes failed to tag modalities for their chains. To automatically handle these cases, we employed \texttt{GPT-4o-mini} \cite{hurst2024gpt} to tag modalities. We note that the models that needed this additional validation step were VideoChat2, VideoChatGPT, and LLaVA-Video; generations from these models did not have any modality tagging for the majority of their chains. The authors manually inspected a subset of GPT-generated modality tags for model-generated reasoning traces to verify accuracy. 

\section{\name\ Model Appendix}
\label{sec:models-appendix}

\subsection{Model Information}
\label{subsec:model_info}

Experiments were conducted with multimodal models that 
were selected for their SOTA performance on various video understanding tasks and have the ability to take a full video as input (2 closed-source models and 5 open-source models): Gemini 1.5 Flash \cite{team2024gemini}, GPT-4o \cite{gpt4o},
VideoChat2 \cite{li2023videochat}, Video-ChatGPT \cite{maaz2023video}, LLaVA-Video \cite{zhang2024video}, LLaVA-Video-Only \cite{zhang2024video}, LongVA \cite{zhang2024longva}. We summarize the models below, and Appendix Table \ref{tab:model_comparison_transposed} lists characteristics of these models: context length, tokens per frame, training max frame, parameter count, and backbone. Figure \ref{fig:model_prompts} describes the prompts given to models. Our benchmark allows experiments with any model that processes multimodal language and video input and outputs text, allowing \name\ to be used as a benchmark over time to study social reasoning. Experiments were conducted with one A100 GPU. 

\begin{table*}[ht!]
\centering
\scriptsize % or \scriptsize
\begin{tabular}{@{}lccccccc@{}}
\toprule
\textbf{Model Property} & \textbf{VideoChat2}  & \textbf{Video-ChatGPT} & \textbf{LLaVA-Video} & \textbf{LLaVA-Video-Only} & \textbf{LongVA} & \textbf{GPT-4o} & \textbf{Gemini 1.5} \\ \midrule
\textbf{Context} & 2K & 2K  & 32K & 32K & 224K & 128K & 1M \\
% \textbf{Tokens/Frames} & -- & 256 & -- &  & 144 & - & - \\
\textbf{Max Frames} & 16 & 100 & 110 & 110  & 2000 & - & - \\
\textbf{Parameters} & 7B & 7B & 7B & 7B & 7B & - & - \\
\textbf{Backbone} & Vicuna-v0 & Vicuna-1.1 & Qwen2 & Qwen2 & Qwen2-Extended & - & - \\
\bottomrule
\end{tabular}
\normalsize
\caption{Information about models tested on \name: VideoChat \cite{li2023videochat}, VideoChat-GPT \cite{maaz2023video},  LLaVA-Video \cite{zhang2024llavanext-video}, LLaVA-Video-Only \cite{zhang2024llavanext-video}, LongVA \cite{zhang2024longva}, GPT-4o \cite{gpt4o}, Gemini 1.5 Flash \cite{team2024gemini}. All information is completed based on current public reports and repositories.}
\label{tab:model_comparison_transposed}
\end{table*}


\paragraph{VideoChat2} The VideoChat2 model \cite{li2024videochat2} has an architecture with UMT-L vision encoder \cite{li2023unmasked}, QFormer and Vicuna-7B v0 language model, has 7B parameters, can process up to 16 frames, and was trained with instruction tuning on a collection of 34 tasks spanning conversations, captions, visual question-answering, reasoning, and classification. 


\paragraph{Video-ChatGPT}
The VideoChat-GPT model \cite{maaz2023video} has an architecture built on top of LLaVA, with a CLIP vision encoder \cite{radford2021learning} and a Vicuna-7B v1.1 language model. The VideoChat-GPT model can process up to 100 frames, and was trained on video instruction pairs from the VideoInstruct100K dataset. 

%https://github.com/mbzuai-oryx/Video-ChatGPT, 
% https://arxiv.org/pdf/2306.05424
%TODO: context, tokens/frames, training max frames 

\paragraph{LLaVA-Video}
The LLaVA-Video model \texttt{LLaVA-Video-7B-Qwen2} \cite{zhang2024video}
has an architecture with a SigLIP SO400M vision transformer and Qwen2 language model, has 7B parameters, can process up to 110 frames, and was trained on mixture of single image, multi-image, and video tasks from the LLaVA-Video-178K and LLaVA-OneVision datasets \cite{li2024llava}. 

\paragraph{LLaVA-Video-Only} The LLaVA-Video-Only model \texttt{LLaVA-Video-7B-Qwen2-Video-Only} is identical to the LLaVA-Video model, with the exception of the training data \cite{zhang2024video}. LLaVA-Video-Only was solely trained on the LLaVA-Video-178K dataset.  

\paragraph{LongVA} The LongVA model \texttt{LongVA-7B-DPO} \cite{zhang2024longva} aligns a unified multimodal transformer (UMT) with QFormer and aligns this visual encoder with a Qwen2 7B language model. LongVA was trained on visual instruction-following datasets and multimodal document data and has a context length of over 200,000 visual tokens; this longer context length was  achieved by extending the context length of the language backbone to train on long text samples, before performing multimodal alignment and additional training to transfer this  ability to the multimodal domain. 

% , LLaVA-OneVision , LLaVA-Hound \cite{zhang2024direct}, NExT-QA \cite{xiao2021next}, ActivityNet-QA \cite{yu2019activitynet}, PerceptionTest \cite{patraucean2024perception}.
% 32 frames for both 






\paragraph{GPT-4o} The GPT-4o model is a closed-source  model from OpenAI \cite{hurst2024gpt}. The technical report for GPT-4o refers to this model as ``omnimodal" with the ability to accept inputs from text, audio, image, and video and generate outputs with text, audio, and image. The API access to this model supports video frame inputs and text inputs. 

\paragraph{Gemini-1.5-Flash} The Gemini-1.5-Flash model is a closed-source model from Google. The API access to this model supports video inputs and text inputs, up to a context length of approximately 1 million tokens. Gemini-1.5-Flash was distilled from the larger Gemini-1.5-Pro sparse mixture-of-experts transformer  \cite{team2024gemini}. 








% \textit{Watch the video and answer questions about social information in the video, following the format of the examples below.}

% \textit{Which of these is the correct answer?  Output either A, B, C, or D. What is the atmosphere like during their conversation?  Provide the reasoning steps you took to get to this answer. Give 5 to 10 reasoning steps in bullet point form. Tag each bullet point step as visual, vocal, and verbal modality from the video or external knowledge.  The correct answer is A. It is very serious because they are talking about how the man's choices have hurt the woman. Below is a reasoning chain that provides social reasoning steps to the correct answer. 1. Woman says ... -- verbal (0:49) ....}


%  \textit{Which of these is the correct answer?  Output either A, B, C, or D. How does the woman in black feel about the woman in the dress? A. She is interested as she looks her way. B. The woman in black is a fan of the Crazy Trilogy movies. C. She acts indifferent, but is actually attracted to her. D. The woman in black is annoyed by the woman in the dress.  Provide the reasoning steps you took to get to this answer. Give 5 to 10 reasoning steps in bullet point form. Tag each bullet point step as visual, vocal, and verbal modality from the frames or external knowledge.  Use the format: The correct answer $\langle$ insert answer $\rangle$.}












% Below are more details on how each model processed a given video. Each model had temperature set to 0. 
% \begin{itemize}
%     \item VideoChat: 16 frames, 224 resolution, max new tokens = 256, $top\_p$ = 0.9 
%     \item VideoChat-GPT: 100 frames, 224 resolution 
%     \item LongVA: 32 frames, max new tokens = 1024, 224 resolution 
%     \item LLaVA-NeXT-Video-7B-Qwen2: 32 frames, max new tokens = 1024 
%     \item GPT-4o: 60 frames (1 fps) 
%     \item Gemini 1.5 Flash: Used File API service extracting 60 frames (1 fps) and audio (1 Kbps) 
% \end{itemize}


% human_values = {
%     "CosineSimilarityMetric_num_steps" : 3.8876,
%     "EmotionMetric" : 2.23, 
%     "SingleModalityMetric - visual": 1.6931,
%     "SingleModalityMetric - verbal": 1.0363,
%     "SingleModalityMetric - vocal": 0.6602,
%     "SingleModalityMetric - ext": 1.9785,
%     "AllModalityMetric": 2.7806,
% }


\subsection{Model Generation Notes}
\label{subsec:modelgen}

We note observations here on model generations. VideoChat generations for $k\in{0, 1, 2, 4}$ produced full sentences explaining reasoning traces, however the generation quality eroded for $k \in {8, 16}$. Several samples from these settings of $k$ were repeated words and short phrases (e.g.,`` \texttt{the the the...and and and and}"). Similarly, VideoChat-GPT generations for $k\in{0, 1, 2, 4}$ produced full sentences explaining reasoning, however the generation quality eroded for $k \in {8, 16}$. Samples from these settings of $k$ were repeated short words and letters  (e.g.,``\texttt{or or or, or}" and ``\texttt{B ( ( ( ( ( B}"). LLaVA-Video generations for $k\in{0, 1, 2, 4, 8, 16}$ produced full sentences explaining reasoning, however the generations as $k$ increased in $k\in{4, 8, 16}$ began to answer fewer questions. LLaVA-Video-Only answered questions, but did not generate reasoning traces; this model was discussed in Section \ref{sec:experiments} solely for the social inference accuracy metric. These model generation challenges were not observed for LongVA or Gemini-1.5. GPT-4o initially generated ``\texttt{I'm sorry, I can't assist with that.}" as one of the reasoning trace steps for several questions, before answering the question. 

\newpage

\section{Auxiliary Experiments}
\label{sec:extra_experiments}

\subsection{Social Inference Accuracy and Reasoning Trace Lengths} We hypothesized that models may perform worse on inferences that primarily rely on \textit{implicit} cues and \textit{contextual} knowledge. One proxy for this reliance is the length of a human trace -- if humans perform an inference immediately and only need to verbalize one reasoning step, that step was more likely to involve  implicit cues with contextual nuances (e.g., rapidly interpreting body language based on external knowledge). For samples with 1 reasoning step, 53\% referenced external knowledge in this first piece of evidence, in contrast to 33\% of samples with 5 reasoning steps and 20\% of samples with 10 reasoning steps. 

We examined model social inference performance  across samples with different lengths of human reasoning traces, visualized for $k$ = 0 in Figure \ref{fig:lengths}. 
\textit{Overall, multimodal models social inference performance was lower for samples with shorter human reasoning traces and higher for samples with longer human reasoning traces.} This trend was observed for both larger closed-source models and smaller open-source models that represent different training data distributions, architectures, and training techniques. For example, Gemini-1.5-Flash and LLaVA-Video achieved accuracies of 70\% and 58\%, respectively, for samples with the shortest reasoning traces and both achieved 80\% for samples with the longest reasoning traces. These results reinforce the perspective that current models (regardless of size) are not sufficient for strong multimodal social reasoning performance in domains requiring more contextual understanding. 



\begin{figure}[t]
    \centering
    \includegraphics[width=1\linewidth]{figures/trace_lengths.png}
    \caption{Social inference accuracy of models ($k$ = 0) across samples with different numbers of human-annotated reasoning steps (binned into quintile by reasoning trace length).}
    \label{fig:lengths}
\end{figure}


\begin{figure}[t]
    \centering
    \includegraphics[width=1\linewidth]{figures/social_inference_accuracy.png}
    \caption{Social inference of models that generated reasoning traces with answers (Trace) compared to models that generated solely answers (No Trace), across different numbers of few-shot samples $k$.}
    \label{fig:cot}
\end{figure}
\raggedbottom

% \begin{figure}[h]
%     \centering
%     \includegraphics[width=0.9\linewidth]{figures/tvqa_social_iqa_performance.png}
%     \caption{Social inference accuracy of models with chain and no-chain experiment settings across different numbers of in-context learning samples.}
%     \label{fig:enter-label}
% \end{figure}

\begin{table*}[ht]
    \centering
    \small
    \caption{Human evaluation mean annotator scores for reasoning trace samples across models. }
    \label{tab:avg_scores}
    \resizebox{0.8\textwidth}{!}{%
    \begin{tabular}{lcccccc}
        \toprule
        \textbf{Model} & \textbf{Shot} & \textbf{Fine-Grained}& \textbf{Compositional} & \textbf{Modality Tags} & \textbf{Validity} & \textbf{Comprehensive} \\
        \midrule
        VideoChat2   & 0.0  & 1.50 & 1.00 & 0.5 & 0.0 & 1.25 \\
          & 1.0  & 1.00 & 1.00 & 1.0 & 0.0 & 1.00 \\
           & 4.0  & 1.00 & 1.00 & 1.0 & 0.0 & 1.00 \\
           & 16.0 & 1.00 & 1.00 & 1.0 & 0.0 & 1.00 \\
        \midrule
          Video-ChatGPT& 0.0  & 1.00 & 1.00 & 1.0 & 0.0 & 1.00 \\
        & 1.0  & 1.00 & 1.00 & 1.0 & 0.0 & 1.00 \\
        & 4.0  & 1.00 & 1.00 & 1.0 & 0.0 & 1.00 \\
        & 16.0 & 1.00 & 1.00 & 1.0 & 0.0 & 1.00 \\
        \midrule
        LongVA      & 0.0  & 2.75 & 2.75 & 1.0 & 1.0 & 4.50 \\
             & 1.0  & 3.00 & 3.25 & 1.0 & 1.0 & 3.75 \\
             & 4.0  & 2.25 & 3.50 & 1.0 & 1.0 & 3.50 \\
             & 16.0 & 2.25 & 2.75 & 1.0 & 1.0 & 3.25 \\
        \midrule  
          LLaVA-Video   & 0.0  & 3.00 & 2.50 & 1.0 & 1.0 & 4.00 \\
          & 1.0  & 2.00 & 1.75 & 1.0 & 0.5 & 3.00 \\
        & 4.0  & 1.00 & 1.00 & 1.0 & 0.0 & 1.00 \\
          & 16.0 & 1.00 & 1.00 & 1.0 & 0.0 & 1.00 \\
          \midrule
         GPT-4o     & 0.0  & 3.50 & 3.00 & 1.0 & 1.0 & 5.00 \\
               & 1.0  & 3.50 & 3.00 & 1.0 & 1.0 & 5.00 \\
                & 4.0  & 4.00 & 3.50 & 1.0 & 1.0 & 4.50 \\
                & 16.0 & 4.00 & 4.00 & 1.0 & 1.0 & 4.50 \\
        \midrule
        Gemini-1.5-Flash      & 0.0  & 4.25 & 4.25 & 1.0 & 1.0 & 4.75 \\
              & 1.0  & 4.00 & 2.50 & 1.0 & 1.0 & 4.75 \\
              & 4.0  & 3.75 & 3.75 & 1.0 & 1.0 & 3.00 \\
            & 16.0 & 4.25 & 4.25 & 1.0 & 1.0 & 4.50 \\
      
        \bottomrule
    \end{tabular}%
    }
\end{table*}

\subsection{Does Chain-of-Thought Prompting Elicit Multimodal Social Reasoning?}

Chain-of-thought (CoT) prompting \cite{wei2022chain} has been a prevalent approach to elicit model reasoning abilities in domains such as mathematics and code generation \cite{li2025structured}. We explore the effectiveness of this technique for multimodal social reasoning with \name. Figure \ref{fig:cot} visualizes social inference results for models under two settings: \textit{Trace} (models generate step-by-step CoT traces while answering the question) and \textit{No Trace} (models generate only answers and do not generate "step by step" traces). We test models  with zero-shot and few-shot settings, with $k \in$ \{0, 1, 2, 4, 8, 16\}. Prompts are described in Figure \ref{fig:model_prompts}. 

\textit{We find that CoT prompting does not substantially improve the social reasoning performance of models}, with the exception of GPT-4o (using CoT performs 6-8\% higher than without CoT after $k$=4). Unlike domains such as mathematics with more formal step-by-step reasoning paths, social reasoning often involves interpreting and integrating ambiguous, context-dependent cues across actors, modalities, and time \cite{mathur-etal-2024-advancing}. Social inference is a form of \textit{informal reasoning} that does not often verbalize as a ``chain-like" process \cite{galotti1989approaches}. Models with CoT prompting have been found to rely on task priors from pretraining distributions \cite{chochlakis2024larger}; it is possible these priors are less effective to elicit social reasoning, compared to other forms of reasoning.  

% \subsection{Experiments with Additional Contexts}
% We explore the extent to which \name\ reasoning traces can be used to condition model performance in additional domains, specifically TVQA \cite{lei-etal-2018-tvqa} and Social-IQa \cite{sap2019socialiqa}. 


% \begin{figure}[hb]
%     \centering
%     \includegraphics[width=\linewidth]{figures/tvqa_social_iqa_performance.png}
%     \caption{Caption}
%     \label{fig:enter-label}
% \end{figure}



% \textcolor{blue}{}

% Having models generate output sequences of tokens with reasoning traces may not improve model accuracy, due to the often non-linear thought processes. 



\section{Human Evaluation Criteria}
\label{subsec:qual_results}

% In addition to quantitative observations of model performance on \name\ discussed in Section \ref{sec:experiments},  we conducted qualitative analysis of model-generated reasoning traces. 

Trained annotators analyzed 48 samples from all models at $k \in$ {0, 1, 4, 16}. The $k$ values and model identities associated with traces were anonymized before presenting samples to annotators in a spreadsheet to rate. For each reasoning trace, annotators watched the corresponding video, read the question, read the answer options, and read the correct answer, in order to rate the quality of the reasoning trace. Human evaluation was conducted to assess reasoning traces along the following dimensions:

\textbf{Fine-grained}: The extent to which reasoning traces were fine-grained was assessed on a scale of 1 to 5, with 1 for no references to low-level behavior and 5 for dense references (e.g., references to low-level behaviors in a majority of steps). 

\textbf{Compositional}: Compositionality in reasoning traces was assessed on a scale of 1 to 5, with 1 for minimal compositionality (no cross-references of information across reasoning steps) and 5 for high compositionality across steps. 

\textbf{Comprehensive}: The extent to which reasoning traces were comprehensive was assessed on a scale of 1 to 5, with 1 referring to  traces missing critical  information (not referencing relevant video content) and 5 being fully-comprehensive (capturing all relevant content towards an inference).

\textbf{Modality Tag Correctness}: The accuracy of modality and external knowledge tags for each of the reasoning steps within a given trace was assessed with a binary score (1 for correct, 0 for the presence of any error in the trace). 

\textbf{Validity of Reasoning}: The validity of the reasoning in a trace, referring to whether or not the trace represented logical combinations of information. It is possible for a reasoning trace to reference minimal low-level information, yet still represent valid reasoning (motivating the inclusion of this dimension). This dimension was given a binary rating (1 for valid, 0 for invalid). 

Raw averages from this human evaluation process are presented in Table \ref{tab:avg_scores}. We discuss findings from human evaluation in Section \ref{sec:human}. Annotator agreement was computed with Cohen's Kappa $\kappa$ \cite{cohen1960coefficient}. We found strong inter-annotator  agreement in ratings across dimensions: $\kappa$ = 0.87 for "fine-grained" ratings, $\kappa$ = 0.92 for "compositional" ratings, $\kappa$ = 0.94 for "comprehensive" ratings, and $\kappa$ = 0.94 for "validity of reasoning" ratings. The "modality tag correctness" ratings across samples was 98\% with errors specifically occurring in modality tags for VideoChat2 traces. 

We note that reasoning trace quality for LLaVA-Video, VideoChat2, and Video-ChatGPT, in particular, decreased at $k$ = 4. These findings from human evaluation are aligned with quantitative findings in Section \ref{sec:experiments} and subjective model generation observations in Appendix \ref{subsec:modelgen}. 

% Reasoning chains from Gemini 1.5 and GPT-4o received the highest ratings for capturing fine-grained multimodal evidence (Cohen's $\kappa$ = 0.87), followed by LLaVA-Video, aligning with findings from Figure \ref{fig:overall}. Gemini 1.5 and GPT-4o chains were consistently rated well for \textit{compositional} (Cohen's $\kappa$ = 0.92) and \textit{comprehensive} (Cohen's $\kappa$ = 0.94) reasoning, followed by LLaVA-Video and LongVA, with chains from VideoChat and VideoChat-GPT rated low, validating the trends observed in the automated metrics in Figure \ref{fig:overall}. Model-generated modality tag correctness across all samples was 98\%, supporting the validity of \name\ modality metrics. 






\begin{table*}[h]
\centering
\scriptsize
\caption{Social Inference Accuracy on \textsc{Social-Genome} for Models Across Different Numbers of Shots ($k$)}
\label{tab:accuracy-metrics-comparison}
\resizebox{\textwidth}{!}{
\begin{tabular}{@{}llcccccc@{}}
\toprule
\textbf{Model} & \textbf{Chain Type} & \textbf{\textit{k} = 0} & \textbf{\textit{k} = 1} & \textbf{\textit{k} = 2} & \textbf{\textit{k} = 4} & \textbf{\textit{k} = 8} & \textbf{\textit{k} = 16} \\
\midrule

% VideoChat
VideoChat2 & Chain & 0.2624 & 0.2779 & 0.2746 & 0.2725 & 0.0000 & 0.0000 \\
 & No Chain & 0.2712 & 0.2793 & 0.2806 & 0.2611 & 0.2483 & 0.0000 \\
\midrule

% VideoChat-GPT
Video-ChatGPT & Chain & 0.3560 & 0.2645 & 0.2550 & 0.2503 & 0.0000 & 0.0000 \\
 & No Chain & 0.3762 & 0.2544 & 0.2746 & 0.2806 & 0.0000 & 0.0000 \\
\midrule

% LongVA
LongVA & Chain & 0.5828 & 0.4973 & 0.5283 & 0.5451 & 0.5363 & 0.5565 \\
 & No Chain & 0.5767 & 0.5081 & 0.5424 & 0.5410 & 0.5310 & 0.5686 \\
\midrule

% LLaVA-NeXT
LLaVA-Video & Chain & 0.6292 & 0.5828 & 0.5875 & 0.5895 & 0.5606 & 0.5592 \\
 & No Chain & 0.6137 & 0.5767 & 0.5754 & 0.5868 & 0.5518 & 0.5760 \\
\midrule

% LLaVA-NeXT-video-only
LLaVA-Video-Only & Chain & 0.5653 & 0.5047 & 0.5081 & 0.5236 & 0.4791 & 0.5155 \\
 & No Chain & 0.5491 & 0.4960 & 0.4993 & 0.5121 & 0.4717 & 0.4899 \\
\midrule

% GPT
GPT-4o & Chain & 0.7100 & 0.6292 & 0.4623 & 0.6332 & 0.6669 & 0.7133 \\
 & No Chain & 0.7207 & 0.6063 & 0.5747 & 0.5713 & 0.6252 & 0.6521 \\
\midrule


% Gemini
Gemini-1.5-Flash & Chain & 0.7443 & 0.7026 & 0.7073 & 0.7052 & 0.6393 & 0.6534 \\
 & No Chain & 0.7564 & 0.7106 & 0.7052 & 0.7046 & 0.6380 & 0.6346 \\
\bottomrule
\end{tabular}
}
\end{table*}



% first metrics table
\begin{table*}[ht]
\centering
\small
\caption{Model performance for semantic and structural similarity metrics across different numbers of few-shot samples $k$. These are the raw results (metrics visualized in Figure \ref{fig:overall} were normalized, as discussed in the metrics section). * refers to model reasoning traces that were less coherent (Appendix \ref{subsec:modelgen}). The highest-performance at each value  of $k$ is bolded.}
\label{tab:core-metrics}
\resizebox{\textwidth}{!}{
\begin{tabular}{@{}llcccccc@{}}
\toprule
\textbf{Metric} & \textbf{Model} & \textbf{\textit{k} = 0} & \textbf{\textit{k} = 1} & \textbf{\textit{k} = 2} & \textbf{\textit{k} = 4} & \textbf{\textit{k} = 8} & \textbf{\textit{k} = 16} \\
\midrule

% Cosine Similarity - Chain
\textbf{Similariy-Trace} & VideoChat2 & 0.4138 & 0.3954 & 0.4084 & 0.4079 & 0.0734 & 0.0597 \\
 & Video-ChatGPT & 0.4484 & 0.4484 & 0.4533 & 0.4354 & 0.0266 & 0.0760 \\
 & LongVA & 0.4533 & 0.4865 & 0.4929 & \textbf{0.4994} & \textbf{0.4994} & \textbf{0.4998} \\
 & LLaVA-Video & 0.4915 & 0.4193 & 0.4552 & 0.4178* & 0.4629* & 0.4731* \\
 & GPT-4o & 0.4631 & 0.4332 & 0.3437 & 0.4363 & 0.4479 & 0.4648 \\
 & Gemini-1.5-Flash & \textbf{0.5157} & \textbf{0.5060} & \textbf{0.5021} & 0.4891 & 0.4887 & 0.4850 \\
\midrule

% Cosine Similarity - Step
\textbf{Similarity-Step} & VideoChat & 0.3864 & 0.3804 & 0.3856 & 0.3749 & 0.0909 & 0.0760 \\
 & Video-ChatGPT & \textbf{0.4524} & \textbf{0.4822} & \textbf{0.4836} & \textbf{0.4700} & 0.0578 & 0.1059 \\
 & LongVA & 0.3898 & 0.4148 & 0.4185 & 0.4231 & 0.4297 & 0.4310 \\
 & LLaVA-Video & 0.4462 & 0.4090 & 0.4330 & 0.4109 & \textbf{0.4641} & \textbf{0.4856} \\
 & GPT-4o & 0.4016 & 0.3540 & 0.2943 & 0.3574 & 0.3682 & 0.3837 \\
 & Gemini-1.5-Flash & 0.4340 & 0.4378 & 0.4311 & 0.4234 & 0.4231 & 0.4210 \\
\midrule

% Cosine Similarity - Num Steps
\textbf{Similarity-Num Steps} & VideoChat2 & 0.5168 & 0.4623 & 0.5983 & 0.4764 & 0.0000 & 0.0000 \\
 & Video-ChatGPT & 0.3267 & 0.2053 & 0.2144 & 0.1989 & 0.0000 & 0.0000 \\
 & LongVA & 0.4186 & 0.4529 & 0.4246 & 0.4179 & 0.4164 & 0.4181 \\
 & LLaVA-Video & 0.7719 & 0.6970 & \textbf{0.9205} & \textbf{0.7762}* & 0.4167* & 0.3750* \\
 & GPT-4o & 0.6226 & 0.4480 & 0.3496 & 0.4575 & 0.4440 & 0.4584 \\
 & Gemini-1.5-Flash & \textbf{1.0376} & \textbf{0.9144} & 0.7312 & 0.6984 & \textbf{0.6436} & \textbf{0.6525} \\
\midrule

% Emotion Metric
\textbf{EmotionMetric} & VideoChat2 & 0.0754 & \textbf{0.2328} & 0.2530 & 0.2322 & 0.0034 & 0.0007 \\
 & Video-ChatGPT & 0.2358 & 0.1713 & 0.1904 & 0.0702 & 0.0000 & 0.0000 \\
 & LongVA & 0.3345 & 0.1306 & 0.3244 & 0.3075 & 0.3106 & 0.3121 \\
 & LLaVA-Video & 0.3579 & 0.1077 & 0.1038 & 0.2168* & 0.1667* & 0.2500* \\
 & GPT-4o & \textbf{0.6502} & 0.2005 & \textbf{0.6631} & \textbf{0.6725} & \textbf{0.6577} & \textbf{0.6286} \\
 & Gemini-1.5-Flash & 0.4959 & 0.1218 & 0.4152 & 0.3782 & 0.3557 & 0.3755 \\
\midrule

% Difference Sequence Metric
\textbf{DifferenceSequenceMetric} & VideoChat & 0.2673 & 0.2690 & 0.2838 &	0.2659 & 0.2141 &	0.2144 
\\
 & Video-ChatGPT & 0.2508 & 0.2352 &	0.2325 & 0.2335 &	0.2181 & 0.2111 \\
 & LongVA & 0.4266 &	0.4165 & 0.4102 &	0.4202 & 0.4045 &	0.3750 \\
 & LLaVA-Video & 0.3783 & 0.4819 & 0.4845 & 	0.4907* & 0.2589* &	0.1754* \\
 & GPT-4o & 0.4582 &	0.3956 & 0.3671 &	0.4167 & 0.4458 &	0.4638 \\
 & Gemini-1.5-Flash & \textbf{0.4591 }& \textbf{0.5301} & \textbf{0.5243} & \textbf{0.5312} & \textbf{0.5240} & \textbf{0.5295} \\
\bottomrule
\end{tabular}
}
\end{table*}

\begin{table*}[ht]
\centering
\small
\caption{Model Performance for metrics related to step-level evidence from modalities and external knowledge across different numbers of few-shot samples $k$. These are the raw results (metrics visualized in Figure \ref{fig:overall} were normalized, as discussed in the metrics section). * refers to reasoning traces that were less coherent (Appendix \ref{subsec:modelgen}). The highest-performance at each value  of $k$ is bolded.}
\label{tab:modality-metrics}
\resizebox{\textwidth}{!}{
\begin{tabular}{@{}llcccccc@{}}
\toprule
\textbf{Metric} & \textbf{Model} & \textbf{\textit{k} = 0} & \textbf{\textit{k} = 1} & \textbf{\textit{k} = 2} & \textbf{\textit{k} = 4} & \textbf{\textit{k} = 8} & \textbf{\textit{k} = 16} \\
\midrule

\textbf{All Modality Steps} & VideoChat2 & 1.2328 & 1.2436 & 1.3573 & 1.1817 & 0.6357* & 0.6184* \\
 & Video-ChatGPT & 0.8526 & 0.6801 & 0.6684 & 0.6638 & 0.6289* & 0.6519* \\
 & LongVA & 2.1406 & 1.7335 & 1.6669 & 1.6440 & 1.5659 & 1.4525 \\
 & LLaVA-Video & 1.6148 & 2.2242 & 2.0526 & 1.9231* & 0.7500* & 0.5625* \\
 & GPT-4o & 2.3245 & 1.7155 & 1.4566 & 1.7536 & 1.8676 & 1.9750 \\
 & Gemini-1.5-Flash & \textbf{2.4056} & \textbf{2.3806} & \textbf{2.3529} & \textbf{2.3146} & \textbf{2.2811} & \textbf{2.2798} \\
\midrule

\textbf{Visual Steps} & VideoChat2 & 0.6097 & 0.4105 & 0.5821 & 0.6326 & 0.0168* & 0.0000* \\
 & Video-ChatGPT & 0.2677 & 0.0919 & 0.0774 & 0.0755 & 0.0027* & 0.0000* \\
 & LongVA & 1.1299 & 0.9246 & 0.8863 & 0.8883 & 0.7782 & 0.7442 \\
 & LLaVA-Video & 1.0301 & 0.9367 & 0.9269 & 0.9580* & 0.0000* & 0.0000* \\
 & GPT-4o & \textbf{1.5894} & \textbf{1.4196} & 1.0725 & \textbf{1.4180} & \textbf{1.4922} & \textbf{1.5598} \\
 & Gemini-1.5-Flash & 1.3276 & 1.2806 & \textbf{1.1860} & 1.1395 & 1.0431 & 1.0513 \\
\midrule

\textbf{Verbal Steps} & VideoChat2 & 0.8338 & 0.8836 & 0.8358 & 0.8351 & 0.6316* & 0.6184* \\
 & Video-ChatGPT & 0.6533 & 0.5831 & 0.5774 & 0.5777 & 0.6030* & 0.6519* \\
 & LongVA & 0.8096 & 0.5942 & 0.6528 & 0.5989 & 0.6464 & 0.6988 \\
 & LLaVA-Video & 0.7643 & \textbf{0.7880} & 0.7487 & \textbf{0.8042}* & \textbf{0.9167}* & 0.6875* \\
 & GPT-4o & 0.7249 & 0.2811 & 0.3984 & 0.3057 & 0.3150 & 0.3320 \\
 & Gemini-1.5-Flash & \textbf{0.9316} & 0.7728 & \textbf{0.7709} & 0.7695 & 0.7360 & \textbf{0.7196} \\
\midrule

\textbf{Vocal Steps} & VideoChat2 & 0.1810 & 0.1635 & 0.2066 & 0.0740 & 0.0000* & 0.0000* \\
 & Video-ChatGPT & 0.0578 & 0.0202 & 0.0188 & 0.0117 & 0.0000* & 0.0000* \\
 & LongVA & \textbf{0.5162} & 0.2288 & 0.1844 & 0.1359 & 0.1365 & 0.1243 \\
 & LLaVA-Video & 0.2486 & 0.2375 & 0.1269 & 0.0490* & 0.0000* & 0.0000* \\
 & GPT-4o & 0.4302 & 0.1270 & 0.0589 & 0.1184 & 0.1270 & 0.1413 \\
 & Gemini-1.5-Flash & 0.4357 & \textbf{0.2984} & \textbf{0.2551} & \textbf{0.2291} & \textbf{0.2497} & \textbf{0.2579} \\
\midrule

\textbf{External Knowledge Steps} & VideoChat2 & 0.0229 & 0.3546 & 0.3230 & 0.2402 & 0.0054* & 0.0000* \\
 & Video-ChatGPT & 0.0130 & 0.0000 & 0.0000 & 0.0000 & 0.0232* & 0.0000* \\
 & LongVA & 0.3156 & 0.5565 & 0.5175 & 0.7174 & 0.6259 & 0.5044 \\
 & LLaVA-Video & 0.0792 & 1.4661 & 1.6128 & 1.5734* & 0.0000* & 0.0000* \\
 & GPT-4o & 0.6199 & 0.9108 & 0.7554 & 1.0402 & 1.1568 & 1.2366 \\
 & Gemini-1.5-Flash & \textbf{0.9856} & \textbf{1.4901} & \textbf{1.6190} & \textbf{1.6881} & \textbf{1.6854} & \textbf{1.6847} \\

\midrule
% Num Steps Metric
\textbf{Num Steps} & VideoChat2 & 4.9455 & 4.7840 & 5.8526 & 4.7638 & 3.1158 & 2.8876 \\
 & Video-ChatGPT & 3.4316 & 2.8955 & 2.8985 & 3.4564 & 2.8888 & 3.0443 \\
 & LongVA & 3.4233 & \textbf{2.3573 }& \textbf{2.1245} & \textbf{1.9455} & \textbf{1.8689} &\textbf{ 2.0475} \\
 & LLaVA-Video & \textbf{2.5574} & 4.0755 & 4.7333 & 4.4336* & 2.2500* & 2.5625* \\
 & GPT-4o & 4.0921 & 4.0757 & 3.5935 & 3.7699 & 3.7583 & 3.8039 \\
 & Gemini-1.5-Flash & 5.1060 & 4.2197 & 4.4624 & 3.9747 & 4.0445 & 3.6854 \\


\bottomrule
\end{tabular}
}
\end{table*}






% % \cellcolor{yellow}

% \begin{table*}[h]
% \centering
% \tiny
% \caption{LLaVA-Video Performance on TVQA Test Set Across TV Shows and Number of \name\ Shots}
% \label{tab:llavanext-comparison}
% \resizebox{\textwidth}{!}{
% \begin{tabular}{@{}lccccccc@{}}
% \toprule
% \textbf{Shots (k)} & \textbf{BBT} & \textbf{Friends} & \textbf{HIMYM} & \textbf{Grey's Anatomy} & \textbf{House} & \textbf{Castle} & \textbf{Overall} \\ 
% \midrule
% \multicolumn{8}{c}{\textbf{LLaVA-NeXT with Chain}} \\ \midrule
% 0 & 0.6223 & 0.5576 & 0.5556 & 0.5654 & 0.4997 & 0.5398 & 0.5553 \\ 
% 1 & 0.6158 & 0.5608 & 0.5476 & 0.5654 & 0.5349 & 0.5453 & 0.5626 \\ 
% 2 & 0.6080 & 0.5602 & 0.5238 & 0.5795 & 0.5302 & 0.5382 & 0.5580 \\ 
% 4 & 0.6027 & 0.5496 & 0.5265 & 0.5734 & 0.5282 & 0.5349 & 0.5529 \\ 
% 8 & 0.5936 & 0.5485 & 0.5503 & 0.5634 & 0.5169 & 0.5284 & 0.5476 \\ 
% 16 & 0.5812 & 0.5267 & 0.5212 & 0.5573 & 0.4937 & 0.5207 & 0.5314 \\ 
% \midrule
% \multicolumn{8}{c}{\textbf{LLaVA-NeXT without Chain}} \\ \midrule
% 0 & 0.6354 & 0.5736 & 0.5714 & 0.5775 & 0.5150 & 0.5507 & 0.5691 \\ 
% 1 & 0.6171 & 0.5714 & 0.5317 & 0.5654 & 0.5136 & 0.5453 & 0.5605 \\ 
% 2 & 0.6164 & 0.5650 & 0.5265 & 0.5473 & 0.5223 & 0.5491 & 0.5600 \\ 
% 4 & 0.6158 & 0.5656 & 0.5450 & 0.5734 & 0.5203 & 0.5578 & 0.5643 \\ 
% 8 & 0.6112 & 0.5613 & 0.5582 & 0.5674 & 0.5183 & 0.5393 & 0.5578 \\ 
% 16 & 0.5975 & 0.5464 & 0.5291 & 0.5553 & 0.5030 & 0.5393 & 0.5461 \\ 
% \bottomrule
% \end{tabular}
% }
% \end{table*}


% \begin{table}[h]
% \centering
% \caption{LLaVA-NeXT Model Performance on Social-IQA Across Number of Shots}
% \label{tab:llavanext-socialiqa}
% \begin{tabular}{@{}lcc@{}}
% \toprule
% \textbf{Shots (k)} & \textbf{With Chain} & \textbf{Without Chain} \\ 
% \midrule






% 0 & 0.7579 & 0.7620 \\ 
% 1 & 0.7400 & 0.7477 \\ 
% 2 & 0.7329 & 0.7482 \\ 
% 4 & 0.7359 & 0.7462 \\ 
% 8 & 0.7339 & 0.7339 \\ 
% 16 & 0.7364 & 0.7421 \\ 
% \bottomrule
% \end{tabular}
% \end{table}


% \begin{table}[ht]
% \centering
% \small
% \caption{Average Reasoning Steps and Modality-Specific Steps by Complexity of Samples (Number of Speaker Turns) for \texttt{llavanext\_chain}}
% \label{tab:avg_steps_modality}
% \begin{tabular}{@{}lcccccc@{}}
% \toprule
% \textbf{Bin}          & \textbf{Avg Steps} & \textbf{Verbal} & \textbf{EK} & \textbf{Visual} & \textbf{Vocal} \\ \midrule
% \textbf{Lowest}       & 2.45               & 0.51            & 0.63                       & 0.55            & 0.18           \\
% \textbf{Low}          & 2.41               & 0.49            & 0.61                       & 0.48            & 0.16           \\
% \textbf{High}         & 2.70               & 0.56            & 0.78                       & 0.50            & 0.17           \\
% \textbf{Highest}      & 2.64               & 0.58            & 0.66                       & 0.63            & 0.17           \\ \bottomrule
% \end{tabular}
% \end{table}



\end{document}
