\section{Discrete Langevin Proposal}\label{appndx:dlp_proposal}
Our proposed controlled text generation leverages the gradient-based discrete sampling algorithm in \citet{zhang2022langevinlike}, which is further investigated by \citet{pynadath2024gradientbaseddiscretesamplingautomatic}. Using the same notation as in the Main Body of the paper, we put the original proposal distribution from \citet{zhang2022langevinlike} below:
\[
\text{Categorical} \left( \underset{j \in |V|}{\softmax} \left( \frac{1}{2} \nabla f(\hat{B} | X)_i (\text{Onehot}_j - \hat{b}_i) - \frac{||\text{Onehot}_j - \hat{b}_i ||^2_2}{2\alpha}\right) \right)
\]
Here, $\hat{b}_i$ corresponds to the one-hot vector in sequence position $i$. Similarly, $\text{Onehot}_j$ corresponds to the one-hot vector for the $j$th token in $V$. This proposal function defines a distribution over the vocabulary for the $i$th sequence position in the sequence by taking the softmax over all possible tokens.

As discussed in \citet{pynadath2024gradientbaseddiscretesamplingautomatic}, this proposal is locally balanced, or optimal for very small step-sizes. For the task of controlled text generation, we would prefer a proposal function that is optimal for large step-sizes, which allow for superior exploration of the space of potential sequences. The globally balanced proposal can be written as follows: 
\[
\text{Categorical} \left( \underset{j \in |V|}{\softmax} \left( \nabla f(\hat{B} | X)_i (\text{Onehot}_j- \hat{b}_i) \right) \right)
\]
In terms of the gradient computation, the one-hot representation  enables the use of automatic differentiation packages to compute $\nabla f(\hat{B} | X)$. We observe that the term 
$(\text{Onehot}_j - \hat{b}_i)$ corresponds to the distance between the proposed token $j$ and the original token $b_i$. We choose to represent this distance term as hamming distance, given the discrete nature of the space we wish to sample. For a token $j$, the hamming distance to the original token in position $i$ is 0 if the $j$th coordinate $\hat{b}_{ij} = 1$ as they are the same token; and 1 if the $j$th coordinate is 0. Thus we can represent the distances between the tokens as $1 - \hat{b}_{ij}$. This leads us to the proposal function in \ref{eq:dlp_prop}, which we place below for convenience: 
\[
b'_i \sim
    \categorical\left(\underset{j \in V}{\softmax} \left( \frac{1}{\tau} (\nabla f(\hat{B} | X))_{ij} (1 - \hat{b}_{ij}) \right) \right)
\]
Here, $b'_i$ refers to the token we sample from the categorical distribution over $V$. 

\section{Algorithmic Details} \label{appndx:algrthm-details}
Here we provide the full pseudo-code for our algorithm. 
\begin{algorithm}
    \caption{Discrete Autoregressive Biasing}
    \begin{algorithmic}[1]
    \REQUIRE Constraint function $f$, $P^{LM}$, prompt $X$, number steps $s$, sequence length $n$, embedding table $M$
    \STATE $\tilde{B} \gets \vec{0}, f_\text{min} \gets -\infty$, $Y_\text{best} \gets \{\}$ \LineComment{Initialize constraint violation as being maximal and current best generation as empty}
    \FOR{step $s$}
        \FOR{position $i$ in range($n$)} 
            \STATE $\tilde{y_i} \gets \log P^{LM} (\cdot | y_{<i}, X)$ \LineComment{Initial auto-regressive distribution over $V$}
            \STATE Calculate normalizing factor $r_i$ if $s > 1$, else $r_i \gets 1$
            \STATE $y_i \gets \text{argmax}_{j \in |V|} \left(\tilde{y}_{i, j} - w_i \cdot r_i \cdot \tilde{b}_{i, j} \right)$ \LineComment{Sample from $P(Y | X, B)$}
        \ENDFOR
        \STATE $B \gets Y$ \LineComment{Initialize $B$ as $Y$}
        \STATE Evaluate $f(B | X)$, update $f_\text{min}$, $Y_\text{best}$
        \STATE $B' \sim q_\tau(\cdot | B)$ as in \eqref{eq:dlp_prop} \LineComment{Approximately sample from $P(B | X, Y)$}
        \STATE Compute $\tilde{B}$ as in \eqref{eq:bias-vec-def}
    \ENDFOR
    \STATE return $Y_\text{best}$
    \end{algorithmic}
\label{alg:text-gen}
\end{algorithm}

DAB takes as input the external constraint $f$, the base language model $P^{LM}$, prompt $X$, number of steps $s$, sequence length $n$, and embedding table $M$. Given these inputs, our proposed algorithm alternates between auto-regressively generating the response sequence and sampling the bias sequence using Discrete Langevin Proposal (DLP) \citep{zhang2022langevinlike}. 

\section{Ablation Study}
\label{appndx:ablation}
\begin{table*}
  [t]
  \centering
  \resizebox{\textwidth}{!}{%
  \begin{tabular}{cccccccccccc}
    \toprule \multicolumn{2}{c}{Components}                                                             & \multicolumn{5}{c}{Re-executability Rate (\%)} & \multicolumn{5}{c}{Readability (\#)} \\
    \cmidrule(lr){1-2} \cmidrule(lr){3-7} \cmidrule(lr){8-12}        \hspace{8pt}\labelemoji\hspace{8pt}                                                                & \hspace{8pt}\toolemoji\hspace{8pt}                                      & O0                                 & O1             & O2             & O3             & AVG            & O0             & O1             & O2             & O3             & AVG            \\
    \hline
    \rowcolor[rgb]{0.93,0.93,0.93}\multicolumn{12}{c}{\textbf{Initialize with LLM4Decompile-End-6.7B~\citep{llm4decompile}}}   \\
    \xmark                                                                                              & \xmark                                    & 69.51                              & 46.95          & 50.61          & 46.34          & 53.35          & 3.98 & 3.41 & 3.44 & 3.38 & 3.55 \\
    \cmark                                                                                              & \xmark                                    & 75.61                              & 50.61          & 50.00          & 50.00          & 56.55          & 4.01 & 3.44 & 3.39 & \textbf{3.49} & 3.58 \\
    \xmark                                                                                              & \cmark                                    & 83.54                     & \textbf{56.10}          & 51.22          & 50.61 & 60.37 & 4.05 & 3.51 & 3.51 & 3.42 & 3.62 \\
    \cmark                                                                                              & \cmark                                    & \textbf{85.37}                            & \textbf{56.10}                     & \textbf{51.83} & \textbf{52.43}          & \textbf{61.43} & \textbf{4.13} & \textbf{3.60} & \textbf{3.54} & \textbf{3.49} & \textbf{3.69} \\

    \rowcolor[rgb]{0.93,0.93,0.93}\multicolumn{12}{c}{\textbf{Initialize with Deepseek-Coder-6.7B-base~\citep{deepseekcoder}}} \\
    \xmark                                                                                              & \xmark                                    & 59.15                              & 35.98          & 39.02          & 37.80          & 42.99          & 3.71 & 3.05 & 3.16 & 3.05 & 3.24 \\
    \cmark                                                                                              & \xmark                                    & 66.46                              & 41.46          & 38.41          & 36.59          & 45.73          & 3.76 & 3.17 & \textbf{3.21} & 3.08 & 3.31 \\
    \xmark                                                                                              & \cmark                                    & 70.73                              & 39.63          & 39.02          & 40.24          & 47.41          & 3.90 & 3.17 & 3.08 & 3.11 & 3.31 \\
    \cmark                                                                                              & \cmark                                    & \textbf{79.88}                     & \textbf{45.73} & \textbf{43.90} & \textbf{42.68} & \textbf{53.05} & \textbf{3.96} & \textbf{3.21} & 3.18 & \textbf{3.19} & \textbf{3.38} \\
    \bottomrule
  \end{tabular}%
  }
  \caption{The ablation study of different methods across four optimization levels
  (O0, O1, O2, O3), as well as their average scores (AVG). The results in bold represent the optimal performance. The ~\labelemoji~ and ~\toolemoji~ means Relabedling and Function Call. \textbf{Bold} denotes the best performance.}
  \label{tab:ablation}
\end{table*}

\subsection{Efficiency}
\label{appndx:efficiency}
\setlength{\tabcolsep}{3pt}
\begin{table}
\centering
\caption{Efficiency of Different Methods}
\vspace{-0.1in}
\label{tab:efficiency}
% \resizebox{\textwidth}{!}{%
\scalebox{0.65}{%
\begin{tabular}{|c|ccc|ccc|ccc|}
  \hline
  \multirow{2}{*}{} & \multicolumn{3}{c|}{\textbf{memory size (MB)}} & \multicolumn{3}{c|}{\textbf{training time (s)}} & \multicolumn{3}{c|}{\textbf{matching time (s)}}\\
  % \cline{2-7}
  % {} & MByte & seconds/ep & seconds\\
  {} & \textbf{Beijing} & \textbf{Porto} & \textbf{Chengdu} & \textbf{Beijing} & \textbf{Porto} & \textbf{Chengdu} & \textbf{Beijing} & \textbf{Porto} & \textbf{Chengdu}\\
  % \cline{2-7}
  % {} & (MByte) & (minutes/ep) & (seconds/K) & (MByte) & (minutes/ep) & (seconds/K)\\
  \hline
  \textbf{MDP} & 1819MB & 2039MB & 2122MB & - & - & - & 389.14s & 361.15s & 599.51s  \\
  \textbf{HMM} & 1209MB & 1388MB & 1361MB & - & - & - & 427.97s & 380.05s & 589.08s \\
  % \hline
  \textbf{FMM} & 897MB & 931MB & 981MB & - & - & - & 1.13s & 1.02s & 1.87s \\
  % \hline
  \textbf{AMM} & 957MB & 1013MB & 1124MB & - & - & - & 3.42s & 3.05s & 5.16s \\
  % \hline
  \textbf{MTrajRec} & 9045MB & 12428MB & 11265MB & 182.4s & 2200.2s & 25672.4s & 51.22s & 42.27s & 73.68s\\
  % \hline
  \textbf{L2MM} & 9087MB & 11875MB & 10898MB & 189.1s & 2314.2s & 27032.2s & 6.71s & 5.26s & 9.10s\\
  % \hline
  \textbf{GraphMM} & 8537MB & 11752MB & 10378MB & 48.4s & 645.2s & 7311.4s & 8.06s & 6.96s & 11.18s\\
  % \hline
  \textbf{\modelName} & 2530MB & 2299MB & 2357MB & 11.9s & 126.4s & 1507.8s & 1.09s & 0.95s & 1.65s\\
  \hline
\end{tabular}}
\vspace{-0.15in}
\end{table}


\section{Experimental Details}
Here we include additional details on the experiment setup. We provide the hyper-parameter settings for our algorithm  for each experiment in Table \ref{appndx:tab:exp-hyperparam}. It should be noted that for Sampling Steps, we pick values to maintain roughly the same time cost as BOLT: given that our algorithm is roughly twice as fast, we use around twice the number of sampling steps. Furthermore, given the use of early stopping in BOLT, further computational budget doesn't necessarily provide any advantage. 

For the weight value, we use a schedule by \citet{liu2023bolt} as it was shown to be effective in terms of incorporating the bias term into auto-regressive generation. Thus for each position $t$, we have $w_t = w(1 - \frac{t}{L})$, where $w$ is the value we put in Table \ref{appndx:tab:exp-hyperparam}. 
\begin{table}[h]
\caption{Hyper-parameter settings used for DAB on Sentiment-directed generation, language detoxification, and topic-constrained generation.}
\centering
\label{appndx:tab:exp-hyperparam}
\begin{tabular}{l|cccccl}\toprule
     \textit{Hyper-parameter} & \textit{Sentiment} & \textit{Detoxify} & \textit{Topic} \\ \midrule 
Proposal Temp & .1 & .1 & .1\\ 
Top-k & 250 & 250 & 250 \\
Bias Weight Value & 1.05 & 1.05 & 1.4 \\
Number Sample Steps & 20 & 20 & 200 \\ 
\bottomrule
\end{tabular}
\end{table} 

\subsection{Fluency Metrics}
\label{appndx:fluency-metrics}
Here we provide more details as to the metrics we use to evaluate the fluency of text generations. 
\paragraph{CoLA Score} To assess the grammatical correctness of a generation, we use a fine-tuned RoBERTa model from \citet{morris2020textattack} to predict the probability of the sample being labelled as grammatically correct. While a similar metric was used in \citet{kumar2022gradient}, we compute the average predicted probability as opposed to the percentage over generations predicted as fluent since this provides more insight into the degree of grammatical correctness. 

\paragraph{Repeated Tri-grams} To compute the number of repeated tri-grams, we simply count all the tri-grams that were repeated and divide them by the total number of tri-grams per generation. We show the average across all generations for each metric. 

\paragraph{Perplexity} For perplexity, we use the built-in function within the Hugging Face evaluate package to compute the perplexity of each generation according to GPT2-XL \citep{wolf2020huggingfacestransformersstateoftheartnatural}. We show the perplexity of the \textbf{entire} generation, as opposed to conditioning on the prompt as done in \citet{han2023lm, kumar2022gradient, liu2023bolt}. 

\subsection{Sentiment Controlled Generation}
\label{appndx:senti-details}
\paragraph{Experiment Design} We use the same experimental design from \citet{liu2023bolt}, where the sampler uses an internal classifier to produce the generations. The internal model is a RoBERTA with GPT2-Large Embeddings fine-tuned on the yelp polarity dataset. We use two external models to provide additional evaluation: we use another RoBERTA trained on the same dataset but with the original embeddings, as well as a RoBERTa fine-tuned on Stanford Sentiment Treebank 2. 

We include the hyper-parameters we use for DAB in Table \ref{appndx:tab:exp-hyperparam}. For the baselines, we run the code within their codebase. While we minimize the changes made to the original code, we note that there are some necessary modifications needed in order to ensure that the experimental setting is consistent across all methods evaluated. This due to the fact that all the evaluated methods consider similar but slightly different experiments from ours in their original work \citep{qin2022cold, liu2023bolt, han2023lm, kumar2022gradient}.  

In regards to LM-Steer, which requires training data, we train the steering matrix using the SST-2 dataset, as done in \citet{han2023lm}. While this is a different dataset from what was used to fine-tune the internal classifiers for the EBM sampling methods, we choose this dataset as obtained worse results when training the steer matrix on yelp polarity. Furthermore, we include an external classifier fine-tuned on SST-2 to use as an evaluation criteria. This makes our experiments fair, as all the methods are evaluated with classifiers that are fine-tuned on a different dataset than used for sampling. Lastly, we observe that LM-steer achieves reasonable performance in terms of sentiment control when compared to other baselines. 

Here we list the prompts we use for this experiment: 

\paragraph{External Constraint} To represent the internal constraint, we use a RoBERTA with GPT-2 large embeddings fine-tuned on Yelp-Polarity for COLD, BOLT, MuCOLA, and DAB. We train this model following the codebase of \citet{liu2023bolt}. Since we require the embedding table to be the same between the base LM, we use the GPT2-large embeddings for the classifier, as done in \citet{liu2023bolt, kumar2022gradient}. 
% The final classifier achieves an accuracy of $96\%$ on the hold-out. 

We use a slightly different function to represent the constraint imposed by the fine-tuned model when compared to BOLT. Given the discriminator $h: |V| \to \mathbf{R}^2$, where the results represent the logits for both the desired class $c_{+}$ and the undesired class $c_{-}$, we define the final constraint function as follows: 
\begin{align*}
    f(Y) = (h(Y)_{+} - h(Y)_{-})
\end{align*}
Intuitively, this pushes the unnormalized logits between the desired class and the opposite class away from each other.

This differs from the constraint function in BOLT, which is the typical cross-entropy loss of the discriminator logits where the correct label is the desired sentiment: 
\begin{align*}
    f(Y) &= \log \softmax (h(Y)_{+})
\end{align*}

We find that our formulation of the constraint function enables more effective gradients for our specific method. 
Curiously, this modification does not provide any substantial benefit to BOLT. 
It is possible that the $\log \softmax$ of BOLT's method smooths out the directional information of the gradient. While this would benefit a continuous sampling algorithm, this could potentially remove some directional information that is required for effective discrete sampling. 

\paragraph{Example Generations} In Table \ref{appendix:tab:senti-gens} we include examples of generations for all methods evaluated.  
\begin{table}[!h]
\caption{Example text for Sentiment-guided generation. As visible, previous methods either produce coherent text that is not positive enough or positive text that is incoherent. In contrast, our method produces generations that are both overtly positive and coherent.}
\centering
\resizebox{\textwidth}{!}{
\begin{tabular}{p {.2 \textwidth}|p {.8 \textwidth}}
\toprule
 \textbf{Prompt} & \textit{The horse} \\
\toprule
COLD & The horse head was still in the water but the horse still had a good head. The horse \\ \midrule
MuCOLA & The horse is not only a beautiful and well-crafted piece of art, but it is also a great way \\ \midrule
BOLT & The horseback riding course is a great way to get acquainted with the trails and the terrain. The course is \\  \midrule
LM-Steer & The horseman delivers a stunningly beautiful, wonderfully lyrical, beautifully tender, powerfully moving, beautifully lyrical \\ \midrule
DAB (Ours) & The horse is also a very good and reliable companion. It has been used to carry the family's two- \\ 
\toprule
 \textbf{Prompt} & \textit{The chicken} \\
\toprule
COLD & The chicken is a good source of protein, but it's a very high fat, cholesterol and salt. \\ \midrule
MuCOLA & The chicken is a very healthy, and high in protein, and the other two chicken is a very good, \\ \midrule
BOLT & The chicken, which is named after the popular street food in Mexico, was a gift from a friend who had \\ \midrule 
LM-Steer & The chicken is beautifully tender, beautifully sweet, wonderfully colorful. the crisp, crispiness is wonderful. and the \\  \midrule
DAB (Ours) & The chicken is a great way to get the kids hooked on delicious, healthy food. It's also a good \\ 
\bottomrule
\end{tabular}
}
%\end{small}
% \vspace{-10pt}
\label{appendix:tab:senti-gens}
\end{table}


\subsection{Toxicity Controlled Generation}\textcolor{red}{Content Warning: The following section contains examples of LLM generations that some may find disturbing or offensive, including content that is hateful or violent in nature}

\label{appndx:toxicity-details}
\textbf{Experiment Design} Similar to the sentiment control experiment, we use an internal classifier to guide generation towards less toxic models. 
We use a RoBERTa classifier fine-tuned on the Jigsaw toxic comment dataset, following \citet{kumar2022gradient, liu2023bolt}. 


\textbf{External Constraint} We represent the external constraint by a RoBERTa model with GPT-2 large embeddings that was fine-tuned on the Jigsaw dataset, following the same training protocol as used in \citet{liu2023bolt}. The final classifier achieves an accuracy of $94\%$ on the hold-out dataset.  

\paragraph{Examples} In Table \ref{appendix:tab:detox-gens} we show examples of generations for the various samplers we examine. 
\begin{table}[!h]
\centering
\caption{Examples for Language Detoxification. While all methods seem to have similar probabilities of generating toxic text, here we see the qualitative impact of our method having a lower average max toxicity: while other methods can end up producing very offensive generations, our method is able to ensure that the generation is not extremely toxic.}
\resizebox{\textwidth}{!}{
\begin{tabular}{p {.2\textwidth}|p {.8 \textwidth}}
\toprule
 \textbf{Prompt} & One student’s complaint; however, has led to the logo being associated with \\
\toprule
COLD & One student’s complaint; however, has led to the logo being associated  with the school. The school has since been forced to take the logo and the school's \\ \midrule
MuCOLA & One student’s complaint; however, has led to the logo being associated in various East Semitic cultures. What can you do to help save the culture?Who \\ \midrule
BOLT & One student’s complaint; however, has led to the logo being associated with a negative image. The school chose to remove it from their website after complaints from students and parents \\ \midrule
LM-Steer & One student’s complaint; however, has led to the logo being associated with the campaign.  And Turner had previously written Turner School Turner ’. Turner School Turner Turner \\ \midrule
DAB (Ours) & One student’s complaint; however, has led to the logo being associated with a site that is not affiliated with the University of California. The university's website and its social \\ \toprule
 \textbf{Prompt} & During a show in Sydney in 2013, Tyler went on a tirade against a member of the group, \\
\toprule
COLD & During a show in Sydney in 2013, Tyler went on a tirade against a member of the group,  calling him a "fucking faggot" and a "faggot" and a " \\ \midrule
MuCOLA & During a show in Sydney in 2013, Tyler went on a tirade against a member of the group, telling him: "I want you to go back to your f***ing country." He also apologised \\ \midrule
BOLT & During a show in Sydney in 2013, Tyler went on a tirade against a member of the group, referring to him as "a piece of shit." The singer also publicly criticized the band's decision to \\ \midrule
LM-Steer & During a show in Sydney in 2013, Tyler went on a tirade against a member of the group, writing,As for the woman who called Tyler a slut, I'm afraid you don't \\ \midrule
DAB (Ours) & During a show in Sydney in 2013, Tyler went on a tirade against a member of the group, who was considered to be his friend. The man had been seen with the band at a party and \\ 
\bottomrule
\end{tabular}
}
\label{appendix:tab:detox-gens}
\end{table}



\subsection{Keyword Controlled Generation}
\label{appndx:keywords-details}
\paragraph{Experiment Design}
We largely follow the experimental design introduced in \citet{liu2023bolt}. We list the topics and keywords in \ref{table:kw-topics}. 

\paragraph{Constraint Function} Following \citet{liu2023bolt}, \citet{qin2022cold}, we use the differentiable BLEU score introduced by \citet{liu-etal-2022-dont}. 
This function measures the uni-gram similarity between the generated sentences and the target key-words, using an operation very similar to convolution. 


\begin{table}
\caption{List of topics and correspending keywords.}
\label{table:kw-topics}
\centering
\begin{tabular}{p {.15 \textwidth} | p {.25 \textwidth}}\toprule
      \textbf{Topic}& \textbf{Keywords} \\\midrule
computer & router, Linux, keyboard, server \\ \midrule
legal & plea, subpoena, transcript, bankrupt \\ \midrule
military & torpedo, headquarters, infantry, battlefield \\ \midrule
politics & court, culture, communism, capitilism\\ \midrule
religion & Bible, church, priest, saint \\ \midrule
science & microscope, mass, mineral, scientist\\ \midrule
space & meteor, planet, satellite, astronaut
\\\bottomrule
\end{tabular}
\end{table}

\paragraph{Reference Text Generation} We use GPT-4o to generate high-quality reference text to use in the BertScore computation. For a given topic t and keyword k, we query GPT-4o with the following prompt: 

\textit{Given the topic t and the keyword k, write 30 different, unique sentences using the keyword and relevant to the topic.}

We do this for each topic and for every keyword for that topic. This produces 120 different, unique sentences to use as a reference text in the BertScore computation. 

\paragraph{BertScore Computation Details}
We use the BertScore computation introduced in \citet{zhang2020bertscoreevaluatingtextgeneration} to evaluate the topicality of the generations. Since BertScore relies on the contextualized embedding of the candidate generations and the reference text, this provides insight into how well the methods use the keyword in the desired context. 

For each generation, we compute the BertScore against all the 120 reference sentences for the corresponding prompt and keyword. Because some of the reference text will not contain the keyword used in the generation, we use report the precision metric calculated in BertScore instead of the overall F1 score, as the precision metric matches tokens in the candidate generation to tokens in the reference text. This is preferable as we want to assess whether the generation is similar to any of the reference texts, as opposed to measuring whether all the reference texts are similar to the candidate generation.  

\paragraph{Implementation Details}
We found that in order to obtain good results with DAB on this task, it was necessary to include a string containing the keywords prior to the prompt. More specifically, we included the following string before the initial prompt for keywords $K$ and topic $t$: 

\textit{Include the following keywords: K relevant to t.}

By including the target keywords and topic before the prompt, this increases the probability of these words and similar words in the underlying language model distribution. This enables the bias vectors computed in our method to have a more impact on auto-regressive generation process and thus satisfy the external constraint. 

In order to ensure that this was not providing our method with an unfair advantage, we applied the same trick to BOLT in order to determine whether this would improve the performance of BOLT as well. We provide results in Table \ref{table:kw-prompted-comp}. 
\begin{table}[t]
\caption{Comparison on topic-guided generation between the original BOLT method, the prompted BOLT method, and DAB. As visible, even if the prompt manages to improve the success rate by $.7\%$, this comes at the cost of worse fluency and slightly worse topicality. Furthermore, our method still outperforms this baseline.
}
\label{table:kw-prompted-comp}
\centering

\resizebox{\textwidth}{!}{\begin{tabular}{lcc|ccc}
\Xhline{1pt}\\[-1ex]
& \multicolumn{2}{c|}{\textbf{Control}} & \multicolumn{3}{c}{\textbf{Fluency}} \\ 
\textbf{Topic}& \textit{BertScore} $\uparrow$ & \textit{Success Rate} $\uparrow$ & \textit{CoLA} $\uparrow$ & \textit{REP-3gram} $\downarrow$ & \textit{PPL} $\downarrow$ \\
\midrule
BOLT      & $.8291 \pm .0003$     & $99.1\%$                  & $.705 \pm .006$    & $.005 \pm .005$      & $32.019 \pm 1.593$   \\
BOLT (Prompted)       & $.8123 \pm .0002$     & $\mathbf{99.7\%}$          & $.705 \pm .005$    & $.005 \pm .001$      & $38.22 \pm .951$  \\
DAB  \textit{(Ours)}    & $\mathbf{.8303 \pm .0003}$ & $99.0\%$             & $\mathbf{.726 \pm .005}$ & $\mathbf{.004 \pm .001}$ & $\mathbf{23.424 \pm .317}$ \\
\bottomrule
\end{tabular}}
\end{table}

As visible, while the prompt does improve the success rate marginally, it does not improve any other metrics for BOLT. In fact, we see that this degrades BOLT's fluency slightly through a higher perplexity value. 
\paragraph{Examples} In Table \ref{appendix:tab:kw-gens} we show examples of generations for the various samplers we examine. 
\begin{table}[h]
\centering
\caption{Examples for Topic-Constrained Generation. As visible, while previous methods include the keyword, they tend to either repeat the keyword too many times or misuse the keyword. In contrast, our method is able to include the keyword in a meaningful way relevant to the given topic.}
\resizebox{\textwidth}{!}{
\begin{tabular}{p {.2 \textwidth}| p {.8 \textwidth}}
\toprule
\textbf{Prompt} & Once upon a time \\
\textbf{Topic} & Military \\ 
\textbf{Keywords} & torpedo, headquarters, infantry, battlefield \\
\toprule
\textit{COLD} & Once upon a time, the world was a peaceful place. People were \textbf{headquarters} of the world \textbf{headquarters} of the world \textbf{torpedo}- \\ \midrule
\textit{MuCOLA} & Once upon a time, the world was a world of the great \textbf{battlefield} the powerful \textbf{headquarters} a \textbf{torpedo} of the good and \textbf{infantry}\\ \midrule
\textit{BOLT} & Once upon a time, there was a man named John Smith who had a dream that he would be able to \textbf{infantry} his \\  \midrule
\textit{DAB} (Ours) & Once upon a time, there was a small group of officers who were in charge of the modern \textbf{infantry} and logistics. They \\ \toprule
\textbf{Prompt} &  The book \\
\textbf{Topic} &  Science \\
\textbf{Keywords} & microscope, mass, mineral, scientist \\
\midrule
\textit{COLD} & The book is scientist-driven, and is a scientist mineralogist, \textbf{microscope}, \textbf{microscope}, \textbf{mineral} \textbf{microscope}, \\ \toprule
\textit{MuCOLA} & The book also has \textbf{mass}ive properties, like the Alabaster House, which features extensive characters from Alabaster \\ \midrule
\textit{BOLT} & The book is divided into three parts, each of which contains a chapter \textbf{mass} mineral scientist relevant to science. scientist \\  \midrule
\textit{DAB} (Ours) & The book is a good introduction to the field of \textbf{mass} spectrometry and is an excellent resource for hands- \\ 
\bottomrule
\end{tabular}}
\label{appendix:tab:kw-gens}
\end{table}


