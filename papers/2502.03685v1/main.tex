\documentclass{article} % For LaTeX2e
\usepackage{Styling/iclr2025_conference,times}

\usepackage{amsthm}
\usepackage{algorithm}
\usepackage{amsmath} % For \arg\max
\usepackage{xcolor}         % colors
\usepackage{booktabs}
\usepackage{tabularx}
\usepackage{caption}
\usepackage{subcaption}
\usepackage{graphicx}
\usepackage{multirow}
\usepackage{wrapfig}
\usepackage{makecell}
\usepackage{nicefrac,xfrac}
\usepackage{array}
\usepackage{makecell}
\usepackage{tikz}
\usetikzlibrary{matrix,shapes,arrows,positioning,chains,backgrounds,shapes.geometric}

\usepackage{colortbl}

%%%%% NEW MATH DEFINITIONS %%%%%

% \usepackage{amsmath,amsfonts,bm}
\usepackage{amsmath,amsfonts}

\usepackage{pifont}


\newcommand{\R}{\mathbb{R}}


\def\va{{\mathbf{a}}}
\def\vg{{\mathbf{g}}}

% Sets
\def\sR{\mathbb{R}}
\def\sC{\mathbb{C}}
\def\sZ{\mathbb{Z}}
\def\sN{\mathbb{N}}
\def\sQ{\mathbb{Q}}

\def\sS{\mathcal{S}}



% Vectors
\def\vzero{{\mathbf{0}}}
\def\vone{{\mathbf{1}}}
\def\vmu{{\mathbf{\mu}}}
\def\vtheta{{\mathbf{\theta}}}
\def\va{{\mathbf{a}}}
\def\vb{{\mathbf{b}}}
\def\vc{{\mathbf{c}}}
\def\vd{{\mathbf{d}}}
\def\ve{{\mathbf{e}}}
\def\vf{{\mathbf{f}}}
\def\vg{{\mathbf{g}}}
\def\vh{{\mathbf{h}}}
\def\vi{{\mathbf{i}}}
\def\vj{{\mathbf{j}}}
\def\vk{{\mathbf{k}}}
\def\vl{{\mathbf{l}}}
\def\vm{{\mathbf{m}}}
\def\vn{{\mathbf{n}}}
\def\vo{{\mathbf{o}}}
\def\vp{{\mathbf{p}}}
\def\vq{{\mathbf{q}}}
\def\vr{{\mathbf{r}}}
\def\vs{{\mathbf{s}}}
\def\vt{{\mathbf{t}}}
\def\vu{{\mathbf{u}}}
\def\vv{{\mathbf{v}}}
\def\vw{{\mathbf{w}}}
\def\vx{{\mathbf{x}}}
\def\vy{{\mathbf{y}}}
\def\vz{{\mathbf{z}}}
\def\vzeta{{\mathbf{\zeta}}}

% Matrix
\def\mA{{\mathbf{A}}}
\def\mB{{\mathbf{B}}}
\def\mC{{\mathbf{C}}}
\def\mD{{\mathbf{D}}}
\def\mE{{\mathbf{E}}}
\def\mF{{\mathbf{F}}}
\def\mG{{\mathbf{G}}}
\def\mH{{\mathbf{H}}}
\def\mI{{\mathbf{I}}}
\def\mJ{{\mathbf{J}}}
\def\mK{{\mathbf{K}}}
\def\mL{{\mathbf{L}}}
\def\mM{{\mathbf{M}}}
\def\mN{{\mathbf{N}}}
\def\mO{{\mathbf{O}}}
\def\mP{{\mathbf{P}}}
\def\mQ{{\mathbf{Q}}}
\def\mR{{\mathbf{R}}}
\def\mS{{\mathbf{S}}}
\def\mT{{\mathbf{T}}}
\def\mU{{\mathbf{U}}}
\def\mV{{\mathbf{V}}}
\def\mW{{\mathbf{W}}}
\def\mX{{\mathbf{X}}}
\def\mY{{\mathbf{Y}}}
\def\mZ{{\mathbf{Z}}}
\def\mBeta{{\mathbf{\beta}}}
\def\mPhi{{\mathbf{\Phi}}}
\def\mLambda{{\mathbf{\Lambda}}}
\def\mSigma{{\mathbf{\Sigma}}}


% Expectation
% \def\eE{\mathop{\mathbb{E}}\limits}
\def\eE{\mathbb{E}}

% Probability
\def\pP{\mathbb{P}}

% Tilde
\def\tf{\tilde{f}}
\def\tS{\tilde{S}}
\def\wtF{\widetilde{\mathcal{F}}}
\def\whR{\widehat{R}}
\def\tvx{\tilde{\mathbf{x}}}
\def\ty{\tilde{y}}


\def\defeq{\overset{\textup{def}}{=}}
% \def\defeq{\overset{.}{=}}
\def\defone{\overset{\text{\ding{172}}}{=}}
\def\deftwo{\overset{\text{\ding{173}}}{=}}
\def\leqone{\overset{\text{\ding{172}}}{\leq}}
\def\leqtwo{\overset{\text{\ding{173}}}{\leq}}
\def\leqthree{\overset{\text{\ding{174}}}{\leq}}
\def\leqfour{\overset{\text{\ding{175}}}{\leq}}
\def\eqone{\overset{\text{\ding{172}}}{=}}
\def\eqtwo{\overset{\text{\ding{173}}}{=}}
\def\eqthree{\overset{\text{\ding{174}}}{=}}
\def\eqfour{\overset{\text{\ding{175}}}{=}}
\def\geqfive{\overset{\text{\ding{176}}}{\geq}}
\newcommand{\DiscreteTokenVector}[4]{
    \begin{scope}[shift={#1}, scale=#3] % Scale the entire object
        \foreach \i in {0,...,#2} {
            \draw[thick, draw=#4] (0, \i*.5 + \i) rectangle (1, \i+1 + \i * .5);
        }
    \end{scope}
}



% Define a new command that takes four arguments:
% #1: Coordinates (x,y)
% #2: Base color
% #3: Number of squares
% #4: Scale factor for resizing the rectangles
\newcommand{\SingleDarkSquareStack}[4]{
    \begin{scope}[shift={#1}, scale=#4] % Scale the entire object
        % Randomly choose which square will be dark
        \pgfmathsetmacro{\darksquare}{int(random()*#3)}

        \foreach \i in {0,...,#3} {
            % If the current square is the dark one, make it 90% shade
            \ifnum\i=\darksquare
                \fill[#2!90] (\i, 0) rectangle (\i+1, 1);
            \else
                % Otherwise, make the square light (between 20% and 40% shade)
                \pgfmathsetmacro{\lightshade}{20+int(random()*20)}
                \fill[#2!\lightshade] (\i, 0) rectangle (\i+1, 1);
            \fi
            \draw[thick] (\i, 0) rectangle (\i+1, 1);
        }
    \end{scope}
}

\newcommand{\SingleDarkSquareStackVert}[4]{
    \begin{scope}[shift={#1}, scale=#4] % Scale the entire object
        % Randomly choose which square will be dark
        \pgfmathsetmacro{\darksquare}{int(random()*#3)}
        \foreach \i in {0,...,#3} {
            % If the current square is the dark one, make it 90% shade
            \ifnum\i=\darksquare
                \fill[#2!90] (0, \i) rectangle (1, \i + 1);
            \else
                % Otherwise, make the square light (between 20% and 40% shade)
                \pgfmathsetmacro{\lightshade}{20+int(random()*20)}
                \fill[#2!\lightshade] (0, \i) rectangle (1, \i + 1);
            \fi
            \draw[thick, rectangle] (0, \i) rectangle (1, \i+1);
        }
    \end{scope}
}
\newcommand{\distbetweenvecs}{2}
\newcommand{\SingleDarkSquareStackVertBiasSeq}[4]{
    \begin{scope}[shift={#1}, scale=#4] % Scale the entire object
        % Randomly choose which square will be dark
        \pgfmathsetmacro{\darksquare}{int(random()*#3)}
        \foreach \i in {0,...,#3} {
            % If the current square is the dark one, make it 90% shade
            \ifnum\i=\darksquare
                \fill[#2!90] (0, \i) rectangle (1, \i + 1);
            \else
                % Otherwise, make the square light (between 20% and 40% shade)
                \pgfmathsetmacro{\lightshade}{5}
                \fill[#2!\lightshade] (0, \i) rectangle (1, \i +1);
            \fi
            \draw[thick, rectangle] (0, \i) rectangle (1, \i+1);
        }
    \end{scope}
}

% Command for a sequence of logits with words
\newcommand{\LogitsWithWords}[3]{% #1: Starting point, #2: List of words, #3: Scale factor
    \begin{scope}[shift={#1}]
        \foreach [count=\i from 0] \word in {#2} {
            \SingleDarkSquareStackVert{(\i*1.2, 0)}{red}{4}{#3}
            \node[below=0.2cm] at (\i*1.2+.2, 0) {\word};
        }
    \end{scope}
}


% Command for a sequence of logits with words
\newcommand{\LogitsWithNoWords}[3]{% #1: Starting point, #2: List of words, #3: Scale factor
    \begin{scope}[shift={#1}]
        \foreach [count=\i from 0] \word in {#2} {
            \SingleDarkSquareStackVert{(\i*\distbetweenvecs, 0)}{red}{4}{#3};
        }
    \end{scope}
}

% Command for a sequence of logits with words
\newcommand{\LogitsWithNoWordsBias}[3]{% #1: Starting point, #2: List of words, #3: Scale factor
    \begin{scope}[shift={#1}]
        \foreach [count=\i from 0] \word in {#2} {
            \SingleDarkSquareStackVertBiasSeq{(\i*\distbetweenvecs, 0)}{blue}{4}{#3};
        }
    \end{scope}
}

% #1: Starting coordinates (x,y) for the stacks
% #2: Base color for the stacks
% #3: Number of squares per stack
% #4: Scale factor for resizing the rectangles
% #5: Number of stacks to draw
\newcommand{\LogitVectors}[7]{
    \foreach \i in {0,...,#5}{
        % Correct referencing of coordinates for shift
        % Ensure proper vertical stacking by incrementing the y-coordinate only
        \SingleDarkSquareStack{( #6, #7 + \i * #4)}{#2}{#3}{#4}
    }
}

\newcommand{\LogitVectorsHorizontal}[7]{
    \foreach \i in {0,...,#5}{
        % Correct referencing of coordinates for shift
        % Ensure proper vertical stacking by incrementing the y-coordinate only
        \SingleDarkSquareStack{( #6+ \i * #4, #7)}{#2}{#3}{#4}
    }
}

% Define a command to draw an arrow
% #1: Starting coordinates (x1, y1)
% #2: Ending coordinates (x2, y2)
\newcommand{\DrawArrow}[2]{
    \draw[->, thick] #1 -- #2;
}

\usepackage{hyperref}
\usepackage{url}

\title{Controlled LLM Decoding via Discrete \\Auto-regressive Biasing}

\author{Patrick Pynadath, Ruqi Zhang \\
Department of Computer Science\\
Purdue University\\
West Lafayette, Indiana, 47906, USA \\
\texttt{\{ppynadat, ruqiz\}@purdue.edu} \\
}
\DeclareMathOperator*{\categorical}{Categorical}

\newcommand{\RETURN}[1]{\State \textbf{return} #1}
\newcommand{\fix}{\marginpar{FIX}}
\newcommand{\new}{\marginpar{NEW}}
\newcommand{\LineComment}[1]{{\color{blue}\textit{$\triangleright$ #1}}}
\newcommand{\FullLineComment}[1]{\Statex {\color{blue}\textit{$\triangleright$ #1}}}
\newcommand{\ruqi}[1]{{\textcolor{red}{[rz: #1]}}}
\newcommand{\pderiv}[2]{{\partial #1}/{\partial #2}}
\newcommand{\pat}[1]{{\textcolor{blue}{#1}}}

\iclrfinalcopy 
\begin{document}


\maketitle
\begin{abstract}
Controlled text generation allows for enforcing user-defined constraints on large language model outputs, an increasingly important field as LLMs become more prevalent in everyday life. One common approach uses energy-based decoding, which defines a target distribution through an energy function that combines multiple constraints into a weighted average. However, these methods often struggle to balance fluency with constraint satisfaction, even with extensive tuning of the energy function's coefficients. In this paper, we identify that this suboptimal balance arises from sampling in continuous space rather than the natural discrete space of text tokens. To address this, we propose \emph{Discrete Auto-regressive Biasing}, a controlled decoding algorithm that leverages gradients while operating entirely in the discrete text domain.
Specifically, we introduce a new formulation for controlled text generation by defining a joint distribution over the generated sequence and an auxiliary bias sequence. To efficiently sample from this joint distribution, we propose a Langevin-within-Gibbs sampling algorithm using gradient-based discrete MCMC. Our method significantly improves constraint satisfaction while maintaining comparable or better fluency, all with even lower computational costs. We demonstrate the advantages of our controlled decoding method on sentiment control, language detoxification, and keyword-guided generation. We make our code available at the following url: \url{https://github.com/patrickpynadath1/dab}.
\end{abstract}

\section{Introduction}
\begin{figure}[th]
    \centering
    \includegraphics[trim=20 20 20 20, clip, width=1\textwidth]{Images/final.pdf}
    \caption{Visualization of our proposed controlled decoding algorithm, Discrete Auto-Regressive Biasing (DAB). Given an initial response that fails to satisfy some external constraint, DAB steers auto-regressive generation towards satisfactory generations using discrete bias tokens obtained via gradient-based discrete sampling from the constraint function.
}
\label{fig:intro_visualization}
\end{figure}
\section{Introduction}
\label{sec:intro}

\begin{figure*}[tb]
    \centering
    \includegraphics[width=0.848\linewidth]{figs/circuitnn.pdf} 
    \caption{Illustration of differentiable CircuitNN. CircuitNN is designed based on differentiable NAND gates. After DAS is guided by PI and PO pairs of the truth table, CircuitNN can get the precise circuit architecture logic equivalent to the truth table.}
    \label{fig:circuitnn}
\end{figure*}

% 1. Describe the importance of logic synthesis
% 2. Existing Problems
% (a) Neural Architecture Search: Unstable, Predefined Setting, etc.
% (b) Circuit Generation: Probabilistic Model, Logic Equivalence

With the rapid advancement of technology, the scale of integrated circuits (ICs) has expanded exponentially. 
This expansion has introduced significant challenges in chip manufacturing, particularly concerning power and area metrics.
A primary objective in IC design is achieving the same circuit function with fewer transistors, thereby reducing power usage and area occupancy.

Logic synthesis~\cite{hachtel2005logicsynth}, a critical step in electronic design automation (EDA), transforms behavioral-level circuit designs into optimized gate-level circuits, ultimately yielding the final IC layout. 
The primary goal of logic synthesis is to identify the physical implementation with the fewest gates for a given circuit function. 
This task constitutes a challenging NP-hard combinatorial optimization problem. 
Current logic synthesis tools~\cite{brayton2010abc, wolf2013yosys} rely on human-designed heuristics, often leading to sub-optimal outcomes.

Differentiable architecture search (DAS) techniques~\cite{liu2018darts, chu2020darts} offer novel perspectives on addressing challenges in this problem.
Circuit functions can be represented through truth tables, which map binary inputs to their corresponding outputs. 
Truth tables provide a precise representation of input-output relationships, ensuring the design of functionally equivalent circuits.
Inspired by this, researchers~\cite{deepmind2024ai4sys, wang2024tnet} have begun exploring the application of DAS to synthesize circuits directly from truth tables.
Specifically, \citet{deepmind2024ai4sys} proposed CircuitNN, a framework that learns differentiable connection structures with logic gates, enabling the automatic generation of logic circuits from truth tables.
This approach significantly reduces the complexity of traditional circuit generation. 
Building on this, \citet{wang2024tnet} introduced T-Net, a triangle-shaped variant of CircuitNN, incorporating regularization techniques to enhance the efficiency of DAS.

Despite these advancements, several challenges remain. 
The computational complexity of DAS grows quadratically with the number of gates, posing scalability issues.
Although triangle-shaped architecture~\cite{wang2024tnet} partially mitigates this problem, redundancy persists. 
%Additionally, DAS is susceptible to converging to local optima, limiting the ability to search architectures that satisfy the given truth tables~\cite{liu2018darts}. 
%Furthermore, hyperparameters (network depth and layer width) require extensive searches, introducing complexity and prolonging the synthesis process. 
Additionally, DAS is susceptible to converging to local optima~\cite{liu2018darts} and hyperparameters (network depth and layer width) require extensive searches. 
The challenges arise from the vast search space in DAS. 
% Even with predefined settings for CircuitNN, finding a configuration that meets the truth table requires extensive trial and error during the DAS process. 
Intuitively, limiting the search space through predefined parameters (network depth, gates per layer, and connection probabilities) can significantly reduce the complexity.

Recent advances~\cite{openai2023gpt4, abramson2024alphafold3, esser2024sd3, li2024mar} in conditional generative models have demonstrated remarkable performance across language, vision, and graph generation tasks. 
Motivated by these developments, we propose a novel approach to circuit generation that generates preliminary circuit structures to guide DAS in generating refined circuits matching specified truth tables. 
Firstly, we introduce CircuitVQ, a tokenizer with a discrete codebook for circuit tokenization. 
Built upon our Circuit AutoEncoder framework~\cite{hou2022graphmae,li2023maskgae,wu2025mgvga}, CircuitVQ is trained through a circuit reconstruction task. 
Specifically, the CircuitVQ encoder encodes input circuits into discrete tokens using a learnable codebook, while the decoder reconstructs the circuit adjacency matrix based on these tokens.
Subsequently, the CircuitVQ encoder serves as a circuit tokenizer for CircuitAR pretraining, which employs a masked autoregressive modeling paradigm~\cite{chang2022maskgit, li2023mage}. 
In this process, the discrete codes function as supervision signals. 
After training, CircuitAR can generate discrete tokens progressively, which can be decoded into initial circuit structures by the decoder of the CircuitVQ. 
These prior insights can guide DAS in producing refined circuits that match the target truth tables precisely.

Our key contributions can be summarized as follows:
\begin{itemize}
\item We introduce CircuitVQ, a circuit tokenizer that facilitates graph autoregressive modeling for circuit generation, based on our Circuit AutoEncoder framework;
\item Develop CircuitAR, a model trained using masked autoregressive modeling, which generates initial circuit structures conditioned on given truth tables;
\item Propose a refinement framework that integrates differentiable architecture search to produce functionally equivalent circuits guided by target truth tables;
\item Comprehensive experiments demonstrating the scalability and capability emergence of our CircuitAR and the superior performance of the proposed circuit generation approach.
\end{itemize}

% Motivation
% (a) Diffusion (Vision, Graph), Autoregressive (Language, Vision)
% (b) Circuit Generation for Predefined Setting
% (c) Neural Architecture Search for Strict Logic Equivalence

% Contribution
% (a) Circuit Tokenizer (new transformer arch, training strategy)
% (b) CircuitAR (train and gen strategies, post-ar strategy)
% (c) Extensive Evaluation including BitD (Bit Distance) for Scalability


\section{Related Work}

\section{Related Work} \label{sec:related}

% \textbf{Adversarial Attack}
\textbf{Attacks on SLAM.} 
%With the rise of machine learning, 
The robustness of computer vision systems is being actively investigated. With the emergence of adversarial images in the digital domain by adding optimized noise directly to images~\cite{szegedy2013intriguing,carlini2017towards}, researchers find that such attacks also exist physically in the real world \cite{eykholt2018robust,song2018physical,zhao2019seeing}. To fill the gap between attacks in the digital and physical worlds, recent studies have demonstrated that attacks on real-world computer vision systems are practical \cite{eykholt2018robust,li2019adversarial,man2020ghostimage,sharif2016accessorize,zhao2019seeing,zhou2018invisible}. However, attacks on traditional computer vision methods such as SLAM are relatively less explored. \cite{yoshida2022adversarial} proposes an attack against the scan matching algorithm in LiDAR-based SLAM, while most SLAMs in AR/VR devices rely on different sensors like RGB/depth cameras and IMUs. \cite{ikram2022perceptual} and \cite{chen2024adversary} mislead visual SLAM by poisoning the images with special patterns, and \cite{wang2021can} causes the camera to fail using infrared light. In our work, we demonstrate attacks on Visual-Inertial SLAM (VI-SLAM) by perturbing the IMU readings, rather than cameras, and showing its impact on XR user experience. 

\textbf{Acoustic Injection Attacks.} Among various physical attacks, acoustic injection attacks are attractive due to their low cost. Son~\etal~\cite{son2015rocking} were the first to introduce acoustic attacks on MEMS gyroscopes, demonstrating how these attacks could lead to sensor denial-of-service and result in drone crashes. WALNUT~\cite{trippel2017walnut} expanded on this by developing output biasing and control attacks that enable precise manipulation of MEMS accelerometer outputs using modulated sound waves. Wang et al.~\cite{wang2017sonic} demonstrated a sonic gun, showcasing the vulnerability of various smart devices (\eg drones and self-balancing vehicles) to acoustic attacks. Tu et al. \cite{tu2018injected} designed side-swing and switching attacks to alter the outputs of MEMS gyroscopes and accelerometers. Furthermore, Ji et al. \cite{ji2021poltergeist} fool the object detectors by applying acoustic attack to the image stabilizers commonly used in modern cameras. However, none of the existing works study the relationship between the acoustic injections and SLAM outputs on recent XR devices. 

% \zijian{Do we need one session about security in AR/VR?}
% \yicheng{TODO}
%\jiasi{cite the AIVR paper (UMass Amherst?) paper is we have not already. They add IMU perturbation but w/o SLAM, iirc} \yicheng{Cited}

\textbf{XR Security and Privacy.} 
%Security and privacy concerns in XR systems have gained significant attention. 
For single-user XR systems, researchers have demonstrated various side-channel attacks to extract sensitive information (\eg keystrokes) through video feeds~\cite{ling2019know}, head movements~\cite{nair2023unique, slocum2023going}, architectural hints~\cite{zhang2023its,shang2020arspy}, power usage~\cite{li2024dangers}, and EM side-channel leakages~\cite{al2021vr}. In multi-user XR systems, Su et al.~\cite{su2024remote} use avatar motion data to infer keystrokes in shared VR environments. Slocum et al.~\cite{slocum2024doesn} reveal vulnerabilities in the shared state frameworks of multi-user AR. Similarly, Lebeck et al.~\cite{lebeck2017securing} highlight risks like deceptive virtual objects and emphasize access control for managing shared physical and virtual spaces. Ruth et al.~\cite{ruth2019secure} further propose a secure multi-user AR framework focusing on content sharing and permissions.
Chandio et al.~\cite{chandio2024stealthy} %introduced a multi-modal spatiotemporal attack that 
simultaneously manipulated visual and inertial sensors to disrupt XR pose estimation. However, their study evaluated the attack using offline datasets and assumed the attacker's capability to manipulate IMU data streams through acoustic means, without real experiments. Ours is the first to demonstrate acoustic injection attacks on recent XR devices, like the Hololens 2, in the real world.
 



\section{Preliminaries}
\section{Preliminary}

\paragraph{Notation} Consider a sentence of $T$ tokens $\vx=\{\vx_1,\ldots, \vx_T\}\in\gX$, and let $P$ be the unknown target language distribution, $\tilde P(\vx)$ be the empirical distribution of the training data (which is an approximation of $P$), and $Q$ be the distribution of our model at hand. Since our paper is also closely related to RLHF, we will also use $\pi$ to represent the distributions. In particular, we sometimes write $\pi_\theta$ for a distribution that is parameterized by $\theta$, where $\theta$ is usually the set of trainable parameters of the LLM; we write $\pr$ for a reference distribution that should be clear given the context. The next token prediction loss is minimizing the forward-KL between $P$ and $Q$. 





\section{Discrete Autoregressive Biasing}
\section{Introduction}
Optimal Transport (OT) theory \citep{villani, ComputationalOT} addresses the problem of finding the cost-optimal transport map that transforms one probability distribution (\textit{source distribution}) into another (\textit{target distribution}). Recently, there has been growing interest in learning the optimal transport map directly using neural networks. OT has found extensive applications in various machine learning domains by appropriately defining source and target distributions, such as generative modeling \citep{otm, uotm, sjko, choi2024scalable}, image-to-image translation \citep{not, fanscalable}, point cloud completion \citep{uot-upc}, and domain adaptation \citep{da-ot}.
The OT framework is particularly advantageous for unpaired distribution transport tasks, as it relies solely on a predefined cost function to map one distribution to another, eliminating the need for paired data.

Among various approaches, the minimax algorithm, derived from the semi-dual formulation, has been widely investigated \cite{fanscalable, otm, uotm, uotmsd}.
Formally, \citet{fanscalable, otm} established the adversarial algorithm by leveraging the following max-min problem:
%\vspace{-5pt}
\begin{equation} 
\label{eq:problem}
\begin{aligned}
    &\sup_{V} \inf_{T:\mathcal{X} \rightarrow \mathcal{Y}}
    \mathcal{L}(V, T)
    \quad \hbox{where} \quad \mathcal{L}(V, T):= 
    \\
    & \int_{\mathcal{X}} c(x, T(x)) - V(T(x)) d\mu(x) + \int_{\mathcal{Y}} V(y) d\nu(y) .   
\end{aligned}    
   % \vspace{-5pt}
\end{equation}
Here, the probability measures $\mu$ and $\nu$ represent the source and the target distribution, respectively. 
The function $V:\mathcal{Y} \rightarrow \mathbb{R}$ and $T$ approximates a Kantorovich potential \cite{Kantorovich1948}, and an optimal transport map, respectively. Throughout this paper, we call these approaches the \textit{\textbf{Semi-dual Neural OT (SNOT)}}.

When the optimal potential $V^\star$ and the transport map $T^\star$ exist, it is well-known that 
\vspace{-5pt}
\begin{equation} \label{eq:optimal_t}
    % T^\star \in \arg\min_{T} \left[ c(x, T(x)) - v^\star(T(x)) \right].
    T^\star \in \arg\min_{T} \mathcal{L}(V^{\star}, T).
    \vspace{-5pt}
\end{equation}
as shown in \citet{otm, fanscalable}. Thus, the pair $(V^\star, T^\star)$ is the solution to this max-min problem. However, a critical challenge arises: not all solutions of Eq.\ref{eq:problem} correspond to the optimal potential and transport map pair. 
In other words, even the optimal solution in the SNOT framework may not recover the correct optimal transport map. We refer to this challenge as the \textit{\textbf{fake solution problem}}.

In this paper, we analyze this fundamental issue of the fake solution problem in existing SNOT frameworks. Specifically, we identify a sufficient condition on the source distribution $\mu$ that prevents the fake solution problem. The key condition is that the source distribution should not place positive mass on measurable sets with Hausdorff dimension $\leq d-1$ (see Thm~\ref{thm:uniqueness}). To the best of our knowledge, our work offers the first theoretical analysis of the sufficient condition under which the max-min solution of the SNOT framework can correctly learn the OT Map. Prior works were limited to the saddle point solution \citep{fanTMLR} or addressed a specific form of a different OT problem (weak OT) \citep{not, knot} (see Appendix \ref{sec:related_works} for the related works on fake solution issues). Additionally, we comprehensively explore various failure cases when this condition is not satisfied. 

Building on this condition, we develop a novel algorithm that ensures the learning of an optimal transport plan. We refer to our model as the \textit{\textbf{Optimal Transport Plan model (OTP)}}. Our method involves smoothing the source distribution $\mu_{\epsilon}$, so that the Neural OT models recover the correct optimal transport plan. Then, we gradually modify $\mu_{\epsilon}$ back to the original $\mu$ leveraging the convergence property. Our extensive experiments show that our OTP model accurately learns the optimal transport plan. Moreover, our model outperforms various (entropic) Neural OT models in diverse image-to-image translation tasks. Our contributions can be summarized as follows:
\begin{itemize}[topsep=-2pt, itemsep=-2pt]
    \item Our work is the first to identify a sufficient condition under which the max-min solution of existing SNOT recovers the true OT Map.
    \item We demonstrate diverse failure cases that occur when this sufficient condition is not satisfied.
    \item We propose a new algorithm that guarantees the learning of the optimal transport plan.
    \item Our experiments show that our model successfully recovers the correct OT Plan in failure cases where existing models fail.
\end{itemize}
\vspace{-5pt}
\paragraph{Notations and Assumptions}
Let $(\mathcal{X}, \mu)$ and $(\mathcal{Y}, \nu)$ be Polish spaces where $\mathcal{X}$ and $\mathcal{Y}$ are closures of connected open sets in $\mathbb{R}^d$.
We regard $\mu$ and $\nu$ as the source and target distributions.
Unless otherwise described, we consider $\mathcal{X} = \mathcal{Y} = \mathbb{R}^d$ and the quadratic transport cost $c:\mathcal{X}\times\mathcal{Y}\rightarrow \mathbb{R}$, $c(x,y)= \alpha \lVert x-y \rVert^2$ for a given positive constant $\alpha$.
For a measurable map $T$, $T_{\#}\mu$ represents the pushforward distribution of $\mu$.
$\Pi(\mu, \nu)$ denotes the set of joint probability distributions on $\mathcal{X}\times\mathcal{Y}$ whose marginals are $\mu$ and $\nu$, respectively.
Moreover, we denote $W_2(\cdot, \cdot)$ as the 2-Wasserstein distance of two distributions.
\vspace{-8pt}

\section{Background} \label{sec:background}
In this section, we present a brief overview of Optimal Transport theory \cite{villani, santambrogio}, and neural network approaches for learning optimal transport maps. In particular, we focus on approaches that leverage the semi-dual formulation \cite{otm, fanscalable}. 

\vspace{-5pt}
\paragraph{Optimal Transport}
The Optimal Transport (OT) problem investigates transport maps that connect the source distribution $\mu$ and the target distribution $\nu$ \citep{villani, santambrogio}. The \textit{optimal transport map (OT Map or Monge Map)} is defined as the minimizer of a given cost function among all transport maps between $\mu$ and $\nu$. Formally, \citet{monge1781memoire} introduced the OT problem with a deterministic transport map $T$ as follows:
\begin{equation}\label{eq:ot_monge} 
    \mathcal{T}(\mu, \nu) := \inf_{T_\# \mu = \nu}  \left[ \int_{\mathcal{X} } c(x,T(x)) d \mu (x) \right].
\end{equation}
Note that the condition $T_\# \mu = \nu$ indicates that the trasnport map $T$ transforms $\mu$ to $\nu$. However, the Monge OT problem is non-convex, and the existence of minimizer, i.e., the optimal transport map $T^{\star}$, is not always guaranteed depending on the assumption of $\mu$ and $\nu$ (Sec. \ref{sec:fail_no_det}). 

To address this existence issue, \citet{Kantorovich1948} proposed the following convex formulation of the OT problem:
\begin{equation} \label{eq:Kantorovich}
    C(\mu,\nu):=\inf_{\pi \in \Pi(\mu, \nu)} \left[ \int_{\mathcal{X}\times \mathcal{Y}} c(x,y) d\pi(x,y) \right],
\end{equation}
We refer to the joint probability distribution $\pi \in \Pi(\mu, \nu)$ as the \textit{transport plan} between $\mu$ and $\nu$. Unlike the Monge OT problem, the optimal transport plan (OT Plan) $\pi^{\star}$ is guaranteed to exist under mild assumptions on $(\mathcal{X}, \mu)$  and $(\mathcal{Y}, \nu)$ and the cost function $c$ \citep{villani}. Intuitively, while the Monge OT (Eq. \ref{eq:ot_monge}) covers only the deterministic transport map $y=T(x)$, the Kantorovich OT problem (Eq. \ref{eq:Kantorovich}) can represent stochastic transport via the conditional distribution $\pi (y | x)$ for each $x \sim \mu$. When the optimal transport map $T^{\star}$ exists, the optimal transport plan also reduces to this deterministic transport map, i.e., $\pi^{\star} = (Id \times T^{\star})_{\#} \mu$.

\paragraph{Semi-dual Neural OT}
The goal of neural optimal transport (Neural OT) models is to learn the OT Map between $\mu$ and $\nu$ using neural networks. 
The semi-dual formulation of the OT problem is widely leveraged for learning OT Maps \cite{otm, fanscalable, uotm, otmICNN}. 

The semi-dual formulation of the OT problem is given as follows: For a general cost function $c(\cdot, \cdot)$ that is lower semicontinuous and bounded below, the Kantorovich OT problem (Eq. \ref{eq:Kantorovich}) has the following \textit{semi-dual form} ((\citet{villani}, Thm. 5.10), (\citet{santambrogio}, Prop. 1.11)):
\begin{equation} \label{eq:kantorovich-semi-dual}
    % C(\mu, \nu) = 
     S(\mu,\nu):= \!\!\sup_{V\in S_c} \left[ \int_\mathcal{X} V^c(x)d\mu(x) \!+\!\! \int_\mathcal{Y} V(y) d\nu (y) \right],
\end{equation}
where $S_c$ denotes the collection of $c$-concave functions $\psi: \mathcal{Y}\rightarrow \mathbb{R}$ and $V^{c}$ denotes the $c$-transform of $V$, i.e., 
\vspace{-5pt}
\begin{equation} \label{eq:def_c_transform}
  V^c(x)=\underset{y\in \mathcal{Y}}{\inf}\left[ c(x,y) - V(y) \right].
  \vspace{-5pt}
\end{equation}
The SNOT approaches utilize this semi-dual form (Eq. \ref{eq:kantorovich-semi-dual}) for learning the OT Map $T^{\star}$ \citep{otm, fanscalable, otmICNN}. This formulation leads to a max-min optimization problem, similar to GANs \cite{gan}. Specifically, these models parametrize the transport map $T_{\theta} : \fX \rightarrow \fY$ and the potential $V_{\phi}$ as follows:
\vspace{-5pt}
\begin{align} 
    & T_{\theta}: x \mapsto \arg\min_{y \in \mathcal{Y}} \left[c(x, y) - V_{\phi}\left( y \right)\right] \label{eq:def_T} \\
    & \quad \Leftrightarrow \quad V_{\phi}^c(x)=c\left(x,T_{\theta}(x) \right) - V_{\phi}\left(T_{\theta}(x)\right). \label{eq:c_transform_with_T}
    \vspace{-5pt}
\end{align}
Note that $T_{\theta}$-parametrization (Eq. \ref{eq:def_T}) implies that the $c$-transform $V_{\phi}^{c}$ can be expressed with the transport map $T_{\theta}$ and the potential $V_{\phi}$, as shown in Eq. \ref{eq:c_transform_with_T}. From this, the SNOT models derive the following optimization problem $\mathcal{L}_{V_{\phi}, T_{\theta}}$:
\begin{equation}  \label{eq:otm}
    \begin{aligned}
        &\sup_{V_{\phi} \in S_c} \inf_{T_{\theta}:\mathcal{X} \rightarrow \mathcal{Y}} 
        \mathcal{L}(V_{\phi}, T_{\theta}) \quad \text{where} \quad \mathcal{L}(V, T) := \\
        & \int_{\mathcal{X}} c\left(x,T(x)\right)-V \left( T(x) \right) d\mu(x) + \int_{\mathcal{Y}} V(y)  d\nu(y).
    \end{aligned}    
\end{equation}
Intuitively, $T_\theta$ and $V_\phi$ serve similar roles to the generator and the discriminator in GANs. However, the OT Map $T_\theta$ is additionally trained to minimize the transport cost $c\left(x, T_{\theta}(x)\right)$, while GANs focus solely on learning the target distribution $T_{\#}\mu = \nu$ \citep{wgan, wgan-gp}. 


\section{Analytical results for Semi-dual Neural OT} 
\label{sec:analyze}


A critical limitation of existing SNOT approaches is that the max-min solution $(V^{\dagger}, T^{\dagger})$ of Eq. \ref{eq:otm} may include not only the desired OT Map but also other \textbf{fake solutions} \citep{otm}. Formally, if OT Map $T^\star$ exists, the optimal potential $V^\star$ and OT Map $T^\star$ become a max-min solution (Eq. \ref{eq:optimal_t}). 
However, not all max-min solutions correspond to the true optimal potential and transport map, i.e., $\{(V^{\star}, T^{\star})\} \subsetneq \{(V^{\dagger}, T^{\dagger})\} $. In particular, even $T^{\dagger} \# \mu = \nu$ does not hold in general (see Fig. \ref{fig:fail_case}), which means that $T^{\dagger}$ is not a valid transport map from $\mu$ to $\nu$ as in (Eq. \ref{eq:ot_monge}).

We first investigate sufficient conditions to prevent fake solution issues (Sec. \ref{sec:unique_saddle}), and present a comprehensive failure case analysis of the SNOT approach (Sec.~\ref{sec:failure}). Based on this, later in Sec. \ref{sec:method}, we propose a method for learning an accurate Neural OT model that avoids such fake solutions.

\subsection{Sufficient Conditions for Ensuring Convergence of Semi-dual Neural OT}
\label{sec:unique_saddle}
We provide sufficient conditions on the source distribution $\mu$ and the target distribution $\nu$ to ensure a unique minimizer for the $T_{\theta}$-parametrization (Eq. \ref{eq:def_T}). This enables the SNOT objective to accurately recover the optimal transport plan. %(Thm \ref{thm:uniqueness}). 

\begin{theorem} \label{thm:uniqueness}
    Let $\mu \in \mathcal{P}_2(\mathcal{X}), \nu \in \mathcal{P}_2(\mathcal{Y})$, and $c(x,y)=\frac{1}{2}\Vert x-y \Vert^2$.
    Assume that $\mu$ does not give mass to the measurable sets of Hausdorff dimension at most $d-1$ dimension. 
    \begin{enumerate}[leftmargin=*, topsep=0pt, itemsep=-2pt]
        \item[(1)] 
        Then, there exists a unique OT Map $T^\star$ in (Eq. \ref{eq:ot_monge}) and the Kantorovich potential $V^\star\in S_c$ in (Eq. \ref{eq:kantorovich-semi-dual}). 
        \item[(2)]
        For the Kantorovich potential $V^\star \in S_c$, the minimization problem,
        \vspace{-10pt}
        \begin{equation} \label{eq:argmin}
            \mathcal{D}_x := \arg\min_{y\in \mathcal{Y}} \left[ c(x,y) - V^\star(y) \right],
            \vspace{-5pt}
        \end{equation}
        is uniquely determined $\mu$-a.s., i.e. $\mathcal{D}_x = \{ y_x\}$ for $\mu$-a.s $x \in \mathcal{X}$. In particular, a map $x\mapsto y_x \in \mathcal{D}_x$ is a unique OT Map $T^\star$ in law.
    \end{enumerate}
\end{theorem}
\vspace{-5pt}
Here, $\mathcal{D}_x$ corresponds to the $T_{\theta}$-parametrization in the SNOT framework. Therefore, the uniqueness of $\mathcal{D}_x$ for $V^{\star}$ implies that $T_{\theta}$-parametrization is fully characterized. \textbf{Thm. \ref{thm:uniqueness} shows that the assumption on $\mu$, not on $\nu$, is enough to eliminate the ambiguity in mapping each $x$ to $T_{\theta}(x)$ and this mapping corresponds to the OT Map.} %In Sec. \ref{sec:fail_det} and Sec. \ref{sec:fail_no_det}, 
In Sec.~\ref{sec:failure}, we show that this ambiguity results in failure cases of SNOT.


Note that Thm. \ref{thm:uniqueness} is sufficient for addressing the fake solution problem. For the sake of completeness, we also present a sufficient condition where the SNOT framework admits \textbf{a unique max-min solution that corresponds to the correct OT Map} (Cor. \ref{cor:unique_saddle}). In this case, the additional assumptions on $\nu$ is also required. Here, we used the fact that the absolutely continuous measures with respect to the Lesbesgue measure satisfy the condition in Thm. \ref{thm:uniqueness}. 

\begin{theorem} \label{thm:unique_potential}
    Suppose $\mathcal{Y} \subset \mathbb{R}^d$ is a closure of a bounded open set. If $\nu$ has a positive density almost everywhere with respect to the Lebesgue measure on $\mathcal{Y}$, then there exists unique Kantorovich potential $V^\star \in S_c$ up to constant.
\end{theorem}

Cor. \ref{cor:unique_saddle} is derived by combining Thm. \ref{thm:uniqueness} with the uniqueness of the optimal potential $V^{\star}$ in Thm. \ref{thm:unique_potential}.

% \vspace{-5pt}
\begin{corollary} \label{cor:unique_saddle}
     Suppose $\mathcal{Y} \subset \mathbb{R}^d$ is a closure of a bounded open set. 
    Suppose $\mu \in \mathcal{P}_2(\mathcal{X})$ and $\nu\in \mathcal{P}_2 (\mathcal{Y})$ are absolutely continuous distributions that have positive density functions on their domain. Then, the solution $(V^\star, T^\star)$ of \eqref{eq:otm} is unique. In other words, $V^\star\in S_c$ is unique up to constant, and $T^\star$ is a deterministic OT Map.
\end{corollary}
\vspace{-5pt}

\begin{figure*}
    \centering
    \subfigure[Perpendicular]{
    \includegraphics[width=0.18\linewidth]{images/toy/vertical_combined.png}
    \label{fig:perp}
    }
    \quad
    \subfigure[Parallel]{
    \includegraphics[width=0.18\linewidth]{images/toy/horizon_combined.png}
    \label{fig:hor}
    }
    \quad
    \subfigure[One-to-Many]{
    \includegraphics[width=0.18\linewidth]{images/toy/one_to_many_combined.png}
    \label{fig:one-to-many}
    }
    \quad
    \subfigure[Grid]{
    \includegraphics[width=0.18\linewidth]{images/toy/multi_vertical_combined.png}
    \label{fig:multi_verti}
    }
    \subfigure{
    \includegraphics[width=0.1 \linewidth]{images/toy/toy_fault_legend.png}
    } \vspace{-10pt}
    \caption{\textbf{Visualization of failure cases} by comparing the Optimal Transport map (\textbf{1st row}) and the max-min solution (\textbf{2nd row}) of Semi-dual Neural OT in the failure cases. The source data $x \sim \mu$, target data $y \sim \nu$, and generated data $T(x)$ are represented in Blue, Orange, and Red. The max-min solution fails to recover the correct OT Map.}
    % Failure cases and visualization of optimal transport map. Source data (Blue), target data (Orange). \red{discrete vs. GT visualization} 
    \label{fig:fail_case}
    \vspace{-10pt}
\end{figure*}
        

\subsection{Failure Cases When Our Condition Is Not Met}
\label{sec:failure}

In Thm~\ref{thm:uniqueness}, it is crucial to assume that $\mu$ does not give mass to the measurable sets of Hausdorff dimension at most $d-1$ dimension. Without this assumption, SNOT may fail even when the deterministic OT Map $T^{\star}$ uniquely exists.  Specifically, the failure cases discussed in this section refer to scenarios where $(V^{\dagger}, T^{\dagger})$ is a max-min solution of Eq. \ref{eq:problem} but does not correspond to the OT Map $T^{\star}$ (Eq. \ref{eq:ot_monge}), i.e., $(Id, T^{\dagger})_{\#} \mu$ fails to represent the OT Plan $\pi^{\star}$ (Eq. \ref{eq:Kantorovich}). 

\subsubsection{Discrepancy between a Max-min Solution and the Deterministic OT Map} \label{sec:fail_det}
We first focus on the Monge OT problem (Eq. \ref{eq:ot_monge}), where the deterministic OT Map $T^{\star}$ exists. Specifically, we investigate source and target distribution pairs where $T^{\star}$ exists, but the max-min solution $T^{\dagger}$ of the SNOT objective (Eq. \ref{eq:otm}) fails to recover this optimal solution. 
Here, we provide two examples, depending on the uniqueness of $T^{\star}$.
\paragraph{Example 1. [When $T^{\star}$ exists but is not unique]}
First, we introduce a case where multiple optimal solutions $T^{\star}$ exist for the Monge OT problem. Assume that the source and target distributions are uniformly supported on $A = [-1,1]\times \{0\}$ and $B = \{0 \} \times [-1,1]$, respectively (Fig. \ref{fig:perp}).
In this case, any transport map $T$ satisfying $T_\# \mu = \nu$ becomes an optimal transport map for the quadratic cost function. Formally, note that for any transport map $T$, the following holds:
% Note that for any coupling $\pi \in \Pi (\mu, \nu)$, the following holds:
\vspace{-8pt}
\begin{multline} \label{eq:failure1_t}
    \int_{\mathcal{X} } c(x, T(x)) d\mu(x) 
    = \frac{1}{2} \int^1_{-1} x_{1}^2 d \mu (x) \\
    - \int_{\mathcal{X}} \cancel{\langle x, T(x)} \rangle \ d\mu(x)
    + \frac{1}{2} \int^1_{-1} y_{1}^2 d \nu (y) = \frac{2}{3}.
\end{multline}
for $x=(x_1, x_2)$ and $y=(y_1, y_2)$. The first equality follows from $T_\# \mu = \nu$ and $\langle x, T(x) \rangle = 0 $ for all $x$ because $A \perp B$.
Since every transport map achieves the same transport cost, any transport map becomes an optimal transport map $T^{\star}$. 

Then, we prove that $T^{\dagger}$ does not correspond to $T^{\star}$. Specifically, we show that $V^{\star}(y) := \frac{1}{2} \lVert y \rVert^2 \in S_{c}$ is the Kantorovich potential (Eq. \ref{eq:kantorovich-semi-dual}) and that $T^{\dagger}$ is not guaranteed to generate the target distribution. By substituting $V_\phi$ into $V$, the inner problem of SNOT (Eq. \ref{eq:otm}) can be expressed as follows:
\vspace{-5pt}
\begin{multline} \label{eq:failure1_inner}
    \!\!\!\inf_{T} \!\!\int_{\mathcal{X}} \frac{1}{2} \lVert x\rVert^2 - \cancel{\langle x, T(x)\rangle} d\mu(x)  + \!\int_{\mathcal{Y}}\! \frac{1}{2} \lVert y \rVert^2 d\nu(y) \!= \!\frac{2}{3}.
\end{multline} 
Since $V^{\star}$ attains the same value of $\mathcal{L}_{V_{\phi}, T_{\theta}}$ as $T^{\star}$ in Eq. \ref{eq:failure1_t}, $V^{\star}$ is the optimal potential. Furthermore, by comparing Eq. \ref{eq:failure1_inner} with $T_{\theta}$-parametrization (Eq. \ref{eq:def_T}), we can easily observe that any measurable map $T_{\theta}:A\rightarrow B$ can be a max-min solution of SNOT. In other words, there is no constraint ensuring that $T_{\theta \#} \mu = \nu$. For example, $T_{\theta}(x) = (0,0)$ for $\forall x \in \fX$ is also a valid max-min solution. This means that the existing SNOT models cannot learn the optimal transport map between these two distributions (Fig. \ref{fig:perp_model}).


\paragraph{Example 2. [When unique $T^{\star}$ exists]}
Here, we present another failure case when there is a unique optimal transport map $T^{\star}$. Assume that the source and target distributions are uniformly distributed over $A = [-1,1]\times \{0 \}$ and $B=[-1,1]\times \{1\}$, respectively (Fig. \ref{fig:hor}). In this setup, the unique $T^{\star}$ is given by:
\begin{equation}
    T^{\star}(x) := (x_1,1) \quad \text{ for } x=(x_1, 0) \in \fX.
\end{equation}
Thus, $\mathcal{T}(\mu, \nu) = \frac{1}{2}$. Similar to Example 1, we show that $V^{\star}(y) = \frac{1}{2} \lVert y_2 \rVert^2 \in S_{c}$ is the optimal Kantorovich potential and analyze the max-min solution of Eq. \ref{eq:otm}. For this $V^{\star}$, the inner problem of SNOT can be computed as follows:
\begin{equation}  \label{eq:failure2_inner}
    \inf_{T}\! \int_{\mathcal{X}} \frac{1}{2} \lVert x_1 - T(x)_1 \rVert^2 d\mu(x) + \!\!\int_{\mathcal{Y}} \frac{1}{2} \lVert y \rVert^2 d\nu(y) \!=\! \frac{1}{2}.
    \vspace{-5pt}
\end{equation} 

Because Eq. \ref{eq:failure2_inner} achieves the same value as $\mathcal{T}(\mu, \nu)$, $V^{\star}$ is the optimal potential. For this $V^{\star}$, any transport map $T((x_1, x_2)):= (x_1, a)$ for any $a \in \mathbb{R}$ for each $(x_1, x_2) \in \fX$ becomes a max-min solution of the SNOT. In this case, the existing approach fails to even characterize the correct support of the target distribution $\nu$.

\subsubsection{Discrepancy between a Max-min Solution and the Stochastic OT Map}
\label{sec:fail_no_det}

The standard SNOT parametrizes the transport map with a deterministic function $T_{\theta}$ (Eq. \ref{eq:def_T}). 
When no deterministic OT Map $T^{\star}$ exists but only an  OT Plan $\pi^{\star}$ exists (Eq. \ref{eq:Kantorovich}), it is clear that the SNOT cannot accurately represent the stochastic OT Map (OT Plan).


\paragraph{Example 3. [When only $\pi^{\star}$ exists]} 

Suppose the source and target distributions are uniform on $A = [0,1]\times \{ 0 \}$ and $B = [0,1]\times \{1\} \cup [0,1] \times \{-1 \}$, respectively (Fig. \ref{fig:one-to-many}). 
In this case, it is clear that the OT Plan $\pi^{\star}$ is given as follows:
\begin{equation} \label{eq:example1_eq1}
    \pi^{\star}(y|x) = \frac{1}{2} \delta_{(x_1, 1)} + \frac{1}{2} \delta_{(x_1, -1)} \text{ where } x\!=\!\!(x_1, x_2).
\end{equation}
The OT Plan $\pi^{\star}(y|x)$ moves each $x$ vertically either up or down with probability $\frac{1}{2}$, without incurring additional cost from horizontal movement. Then, we show that $V^\star(y) = \frac{1}{2} \Vert y_2 \Vert^2 \in S_{c}$ with $y = (y_1, y_2)$ is the optimal potential. The $(V^\star)^c$ and $V^{\star}$ can be computed for $\mu$ and $\nu$ as follows:
\vspace{-3pt}
\begin{align} 
    \!\!(V^{\star})^c(x)\! = \!\!\inf_{y \in \fY}\!\! \left( c(x,y) \!- \!V^{\star}\!(y) \right)\! = \!\inf_{y_1} \! \frac{1}{2} \Vert x_1 \!\!-\! y_1 \!\Vert^2 \!\!=\!0,\! \label{eq:example1_eq2} 
\end{align}
$V^{\star}(y) = \frac{1}{2} \Vert y_2\Vert^2 = \frac{1}{2} \text{ for } \forall y \in \fY.$
By comparing the $C$ for the optimal transport plan (Eq. \ref{eq:example1_eq1}) and the semi-dual form $S$ for $V^\star$, we can easily verify that $V^\star$ is the optimal Kantorovich potential.

Then, from Eq. \ref{eq:example1_eq2}, we can see that $T_1$ and $T_2$ are the two possible solutions for the $T$-parametrization (Eq. \ref{eq:def_T}) in the SNOT for $V^{\star}$, 
%\begin{equation}
    $T_1(x):=(x_1,1)$ and $T_2(x):=(x_1,-1).$
%\end{equation}
for $x=(x_1, x_2) \in \fX$. These two candidates $T_{1}, T_{2}$ only characterize a subset of the support of $\pi^{\star} (y|x)$. Therefore, our deterministic $T_{\theta}$ cannot learn the stochastic $\pi^{\star} (y|x)$. 

\paragraph{Stochastic Parametrization of OT Map} In practice, a stochastic parametrization of $T_{\theta}(x, z)$ is often adopted to improve performance in the SNOT models \citep{not, uotm}. This stochastic parametrization $T_{\theta}(x, z)$ introduces an additional noise variable $z \sim N(0, I)$: 
\vspace{-10pt}
\begin{equation} \label{eq:stochastic_generator} 
    T_{\theta}(x, z) \in \arg\min_{y\in \mathcal{Y}} \{ c(x,y) - V^\star (y) \}, \\[-5pt]
\end{equation}
$(x,z)\sim \mu \times \mathcal{N}(0,I)$ a.s.. As a result, each $x$ is transported to multiple $T(x, z)$ values depending on $z$. \textbf{We point out that even a stochastic parametrization, such as $T_{\theta}(x, z)$ with a noise variable $z \sim N(0, I)$, cannot address this limitation.} For the formal statement, see Appendix \ref{appen:non_conv_stocas_param}.


\begin{proposition}[Informal]
\label{prop:stoc}
    Assume that the stochastic parametrization of $T_{\theta}(x, z)$ is ideally trained as in \eqref{eq:stochastic_generator} for $(\mu, \mathcal{N})$-a.s. $\mathcal{D}_x$ in Eq. \ref{eq:argmin} may not uniquely determined %(as in Thm \ref{thm:uniqueness}), 
    and $T_{\theta}(x, z)$ may contain fake solutions.
\end{proposition}

    


    
\section{Method} \label{sec:method}
In Sec. \ref{sec:failure}, we analyzed the sufficient condition to prevent failures in the existing SNOT framework. Building on this analysis, we propose a novel method for learning the OT Plan, called the \textit{Optimal Transport Plan (\textbf{OTP}) model}, which is effective even when the conditions are not satisfied.

\begin{algorithm}[t]
\caption{Training algorithm of OTP}
\begin{algorithmic}[1]
\REQUIRE Source distribution $\mu$ and the target distribution $\nu$; OT Map network $T_\theta$ and potential network $V_\phi$; Total number of iteration $K$; Number of inner-loop iterations $K_T$; Decreasing sequence of noise levels $\{\epsilon_k \}^K_{k=1}$.
\FOR{$k = 0, 1, 2 , \dots, K$}
    \STATE Sample a batch $x\sim \mu$, $y\sim \nu$, $z \sim \mathcal{N}(\mathbf{0}, \mathbf{I})$.
    \STATE $\Tilde{x} \leftarrow x + \sqrt{\epsilon_k} z $ or $\Tilde{x} \leftarrow \sqrt{1-\epsilon_k} x + \sqrt{\epsilon_k} z $.
    \STATE Update $\phi$ to maximize $\mathcal{L}_{\phi} = - V_\phi \left(T_\theta(\tilde x)\right) + V_\phi(y)$.
    \FOR{$j= 0, 1, \dots, K_T$}
    \STATE Sample a batch $x\sim \mu, z\sim \mathcal{N}(\mathbf{0}, \mathbf{I})$.
    \STATE $\Tilde{x} \leftarrow x + \sqrt{\epsilon_k} z $ or $\Tilde{x} \leftarrow \sqrt{1-\epsilon_k} x + \sqrt{\epsilon_k} z $.
    \STATE $\mathcal{L}_{\theta} = c(\Tilde{x}, T_\theta(\tilde x)) - V_\phi \left(T_\theta(\tilde x)\right) + V_\phi(y)$.
    \STATE Update $\theta$ to minimize $\mathcal{L}_{\theta}$.
    \ENDFOR
\ENDFOR
\end{algorithmic}
\label{alg:otp}
\end{algorithm}

\subsection{Proposed Method} \label{sec:proposed_method}
\textbf{Our goal is to learn the OT Plan $\pi^{\star}$ (Eq. \ref{eq:Kantorovich}) between the source distribution $\mu$ and the target distribution $\nu$.} Note that the sufficient condition in Thm. \ref{thm:uniqueness} is an inherent property of $\mu$. When this condition is not satisfied, the existence of OT Map $T^{\star}$ is not guaranteed, and only $\pi^{\star}$ exists. In this regard, our OTP model serves as a natural generalization of existing SNOT models.

Our method consists of two steps: First, we introduce a \textit{smoothed version of the source distribution} $\mu_{\epsilon}$. $\mu_{\epsilon}$ is constructed to satisfy the sufficient conditions from Thm. \ref{thm:uniqueness}. As a result, the SNOT between $\mu_{\epsilon}$ and $\nu$ recovers the correct OT Plan $\pi^{\star}_{\epsilon}$ between them. Second, we gradually adjust $\mu_{\epsilon}$ back to the original source measure $\mu$. This approach allows our method to learn the correct optimal transport plan, even in cases where the existing SNOT framework fails.


\paragraph{OTP Model}
As a practical implementation of the high-level scheme described above, we propose a new method for learning the OT Plan $\pi^{\star}$ from $\mu$ to $\nu$, called \textit{Optimal Transport Plan (OTP)} model. This method is based on Thm. \ref{thm:uniqueness} and Thm. \ref{thm:convergence}, which require the following two conditions on the smoothed measure $\mu_{\epsilon}$:
\begin{enumerate}[topsep=0pt, itemsep=-1pt]
    \item[(c1)] 
    $\mu_{\epsilon}$ does not give mass to the measurable sets of Hausdorff dimension at most $d-1$ dimension
    (Thm. \ref{thm:uniqueness}).
    \item[(c2)] $\mu_{\epsilon_{k}}$ weakly converges to $\mu$ as $k \rightarrow \infty$ (Thm. \ref{thm:convergence}).
\end{enumerate}
For simplicity, we consider the absolute continuity condition on $\mu_{\epsilon}$ as we did in Cor. \ref{cor:unique_saddle}. Motivated by diffusion models \citep{ddpm, scoresde}, we consider \textbf{two options for the smoothing distribution}: (1) Gaussian convolution $\mu_{\epsilon_{k}} = \mu \, * \,  \mathcal{N}(0, \epsilon_{k} I)$ and (2) Variance-preserving convolution $\mu_{\epsilon_{k}} = \left(\sqrt{1-\epsilon_{k}}Id\right)_{\#}\mu \, * \,  \mathcal{N}(0, \epsilon_{k} I)$ with a predefined noise level $\epsilon_{k} \searrow 0$. For noise-level scheduling, we follow \citet{scoresde}. Note that both of these smoothing distributions satisfy conditions (c1) and (c2). Specifically, for Gaussian convolution, for any $\mu \in \mathcal{P}_2(\mathbb{R}^d)$, (c1) $\mu_\epsilon$ is absolutely continuous with respect to the Lebesgue measure and has positive density on $\mathbb{R}^d$. Moreover, (c2) as $\epsilon \rightarrow 0$, $\mu_{\epsilon}  \rightharpoonup \mu$. A similar argument works for the Variance-preserving convolution case. Then, we apply the SNOT framework to the smoothed measure $\mu_{\epsilon_{k}}$ and the target measure $\nu$. The learning objective is given as follows:
\vspace{-15pt}
\begin{multline} \label{eq:saddle_epsilon}
    \!\!\mathcal{L}^{k}_{V_{\phi}, T_{\theta}}\! = \!\sup_{V_{\phi}} \left[ \int_{\mathcal{X}} \!\inf_{T_{\theta}} \left[ c\left(x,T_{\theta}(x)\right)\!-\!V_{\phi} \left( T_{\theta}(x) \right) \right] d\mu_{\epsilon_{k}}\!(x) \right. \\[-8pt]
    \left. + \int_{\mathcal{X}} V_{\phi}(y)  d\nu(y) \right].
\end{multline}

Then, we gradually decrease the noise level $\{ \epsilon_{k} \}_{k=1}^{K}$ throughout training. The two conditions on $\mu_{\epsilon_{k}}$, i.e., (c1) and (c2), offer the following guarantees. First, for each noise level $\epsilon_{k}$, $\mathcal{L}^{k}_{V_{\phi}, T_{\theta}}$ has a unique saddle point solution, corresponding to the optimal transport map $T_{k}^{\star}$ and the Kantorovich potential $V_{k}^{\star}$. Second, as $k \rightarrow \infty$, i.e., $\epsilon_{k} \searrow 0$, the optimal transport plan $\pi^{\star}_{k} = (Id, T_k^{\star})_{\#} \mu_{\epsilon_{k}}$ converges (up to a subsequence) to $\pi^{\star}$. Thm. \ref{thm:convergence} follows from combining Thm. \ref{thm:uniqueness} and \citet{villani}. See Appendix \ref{appen:conv_result_from_oldandnew} for proof.

\begin{theorem} \label{thm:convergence}
    Let $\{\mu_{\epsilon_k}\}_{k\in \mathbb{N}}$ be a sequence absolutely continuous probability measures, and $T_{k}^{\star}$ be the OT map from $\mu_{\epsilon_k}$ to $\mu$.
    %\begin{enumerate}
    %\item[(1)] 
     If $\mu_{\epsilon_k}$ weakly converges to $\mu$ as $k \to \infty$, then $\pi^{\star}_{k} = (Id, T_k^{\star})_{\#} \mu_{\epsilon_{k}}$ weakly converges to the OT plan $\pi^{\star}$ between $\mu$ and $\nu$, along a subsequence. Consequently, $\pi^{\star}_{k}$ from our OTP model with either convolution above also weakly converges to $\pi^{\star}$, along a subsequence.
\end{theorem}

In this way, we can learn the optimal transport plan $\pi^{\star}$ between $\mu$ and $\nu$ without falling into the fake solutions of the max-min learning objective (Eq. \ref{eq:otm}). While the convergence theorem only guarantees convergence up to a subsequence (Thm. \ref{thm:convergence}), our method exhibits decent convergence to $\pi^{\star}$ in practice (Sec. \ref{sec:experiment}). Specifically, our training algorithm progressively finetunes the transport network $T_{\theta}$ and the potential network $V_{\phi}$ by adjusting the smoothing level. As a result, the subsequence convergence does not pose any issues.

\begin{figure}[t]
    \centering
    \includegraphics[width=.75\linewidth]{images/colorization_concept2.png}
    \caption{
    \textbf{Example of a stochastic transport map (OT Plan) task}, e.g., MNIST-to-CMNIST colorization.}
    \vspace{-10pt}
    \label{fig:concept_m2cm}
\end{figure}

\paragraph{Importance of OT Plan in Neural OT}
Our OTP model is for learning the OT Plan, i.e., the stochastic transport map. In fact, OT Plans are not only a theoretical generalization of deterministic OT Maps, but are also inherently more suitable for various real-world machine learning applications. For instance, in image-to-image translation tasks, stochastic OT Plans can effectively model the diversity of plausible outputs. Similarly, in inverse problems such as colorization or image inpainting, stochastic OT Plans are also highly desirable because these tasks inherently involve multiple possible solutions. In Sec \ref{sec:experiment}, our experiments show that our OTP model is effective in handling the stochastic transport map application in the MNIST-to-CMNIST image translation task (Fig. \ref{fig:concept_m2cm}).

\begin{figure*}[t]
    \centering 
    \subfigure[Perpendicular]{
    \includegraphics[width=0.18\linewidth]{images/toy/vertical_result_combined.png}
    \label{fig:perp_model} 
    }
    \quad
    \subfigure[Parallel]{
    \includegraphics[width=0.18\linewidth]{images/toy/horizontal_result_combined.png}
    \label{fig:hor_model}
    }
    \quad
    \subfigure[One-to-Many]{
    \includegraphics[width=0.18\linewidth]{images/toy/one_to_many_result_combined.png}
    \label{fig:one-to-many_model}
    }
    \quad
    \subfigure[Grid]{
    \includegraphics[width=0.18\linewidth]{images/toy/multi_vertical_result_combined.png}
    \label{fig:multi_verti_model} \vspace{-10pt}
    }
    \vspace{-10pt}
    \subfigure{
    \includegraphics[width=0.1\linewidth]{images/toy/toy_result_legend.png}
    }  %\\
    \caption{\textbf{Qualitative comparison between OTM (1st row) and our model (2nd row) on failure cases} in Sec \ref{sec:failure}. The noised source sample $\Tilde{x}$ in Alg \ref{alg:otp} is denoted in Green. While OTM falls into fake solutions and fails to generate the target distribution correctly, our OTP model successfully learns the OT Plan. 
    % Quantitative results of OTM vs. ours 
    }
    \label{fig:fail_case_model}
    \vspace{-10pt}
\end{figure*}

\paragraph{Algorithm}
We present our training algorithm for OTP (Algorithm \ref{alg:otp}). For each $\epsilon_{k}$, we alternatively update the adversarial learning objective $\mathcal{L}^{k}_{V_{\phi}, T_{\theta}}$ between the potential function $V_{\phi}$ and the transport map $T_{\theta}$, similar to the GAN framework \citep{gan}. Note that the smooth source measure $\mu_{\epsilon_{k}}$ corresponds to the probability distribution of the sum of the clean source measure $\mu$ (or the scaled source measure $\left(\sqrt{1-\epsilon_{k}}Id\right)_{\#}\mu$) and the Gaussian noise $\mathcal{N}(0, \epsilon_{k} I)$. Therefore, we can easily sample $x_{\epsilon_{k}} \sim \mu_{\epsilon}$, as follows (Line 3):
\vspace{-5pt}
\begin{equation}
    x_{\epsilon_{k}} = x + \sqrt{\epsilon_{k}} z \quad \text{or} \quad
    x_{\epsilon_{k}} =\sqrt{1-\epsilon_{k}}x + \sqrt{\epsilon_{k}},
    \vspace{-5pt}
\end{equation}
where $x \sim \mu$ and $z \sim \mathcal{N}(0, I)$. In practice, decreasing the noise level until a small positive constant $\epsilon_{min} > 0$ provided better performance and training stability, compared to reducing the noise level to exactly zero. For a fair comparison, we compared the composition of the noising and transport map $x \mapsto x_{\epsilon_{min}} \mapsto T_{\theta}(x_{\epsilon_{min}})$, with the ground-truth optimal transport map $x \mapsto T^{\star}(x)$ in the experiments (Sec. \ref{sec:experiment}).





\section{Experiments} \label{sec:experiment}
In this section, we evaluate our OTP model from the following perspectives. In Sec.~\ref{sec:exp_syn}, we evaluate whether OTP successfully learns the optimal transport plan. In Sec.~\ref{sec:exp_image}, we demonstrate the scalability of OTP by assessing it on the image-to-image translation task.
For implementation details of experiments, please refer to Appendix \ref{appen:implementation_details}.

\subsection{OT Plan Evaluation on Failure Cases} \label{sec:exp_syn}
First, \textbf{we assess whether our model accurately learns the optimal transport plan $\pi^{\star}$ between the source distribution $\mu$ and the target distribution $\nu$ in failure cases outlined in Sec \ref{sec:failure}.} The evaluation is conducted in two settings: (1) Qualitative comparison in 2D cases and (2) Quantitative comparison in high-dimensional cases.
In each setting, our OTP model is compared against the existing SNOT framework (Eq. \ref{eq:otm}).


\begin{table}[t]
    \vspace{-10pt}
    \centering
    \caption{\textbf{Quantitative comparison of numerical accuracy} on synthetic datasets. Each model is evaluated using two metrics: transport cost error $D_{cost} (\downarrow)$ and target distribution error $D_{target} (\downarrow)$. 
    }
    \label{tab:fail_case_quan}
    \scalebox{0.75}{
    \begin{tabular}{c c c c c c}
        \toprule
        \multirow{2}{*}{Dimension} & \multirow{2}{*}{Model} & \multicolumn{2}{c}{Perpendicular} & \multicolumn{2}{c}{One-to-Many}  \\
        \cmidrule{3-4} \cmidrule{5-6}
        & & $D_{cost}$ & $D_{target}$ & $D_{cost}$ & $D_{target}$ \\
        \midrule
        \multirow{3}{*}{$d=2$}  & OTM & 0.038 & 0.0079 &  0.069 & 0.10  \\
                                & OTM-s &   \textbf{0.0070}    &  0.018  & 0.35 & \textbf{0.032} \\
                                & Ours & 0.019 & \textbf{0.0068} & \textbf{0.0022} & 0.11  \\
        \midrule
        \multirow{3}{*}{$d=4$}  & OTM & 0.043 & 0.039 & 0.10 & 73.23  \\
                                & OTM-s & \textbf{0.033}  & 0.065 & \textbf{0.010} & \textbf{0.038} \\
                                & Ours & 0.089 & \textbf{0.0086} & 0.033 & 0.094  \\
        \midrule
        \multirow{3}{*}{$d=16$}  & OTM & 0.16 & 4.97 & 71.28 & 73.23   \\
                                & OTM-s & 0.061 & 4.85 & 97.49 & 99.57 \\
                                & Ours  & \textbf{0.058} & \textbf{0.59} & \textbf{0.057} & \textbf{0.65}  \\
        \midrule
        \multirow{3}{*}{$d=64$}  & OTM & 2.13 & 19.37 & 21.92 & 32.94  \\
                                & OTM-s & 2.74 & 18.79  & 0.20  & 12.21 \\
                                & Ours & \textbf{0.97} & \textbf{10.09} & \textbf{0.14} & \textbf{9.98}  \\
        \bottomrule
    \end{tabular}}
    \vspace{-15pt}
\end{table}


\paragraph{Qualitative Comparison}
In Sec. \ref{sec:failure}, we presented various examples where the existing SNOT framework may fail to learn the OT Map (or Plan). Here, we demonstrate that the existing approaches indeed encounter these failures, while OTP successfully learns the correct OT Map. As a baseline, we compare our method against the standard OTM with a deterministic transport map, i.e., $T_{\theta}(x)$. 

Fig. \ref{fig:fail_case_model} presents qualitative results on four datasets: Perpendicular (Ex.1), Parallel (Ex.2), One-to-Many (Ex.3), and Grid. The first row shows the vanilla OTM results and the second row exhibit our OTP results. Note that our OTP decreases the noise level until $\sigma = \epsilon_{min} > 0$ (Sec. \ref{sec:method}). Hence, the noised source samples $\Tilde{x}$ in Alg \ref{alg:otp} (Green in Fig \ref{fig:fail_case_model}) are transported to the target measure $\nu$.
In Fig. \ref{fig:fail_case_model}, \textbf{the vanilla OTM fails to learn the correct optimal transport plan in three cases except for the Parallel case}. OTM fails to cover the target measure $\nu$ in the Perpendicular and Multi-perpendicular cases. In the One-to-Many case, OTM does not learn the correct $T^{\star}$, i.e., the vertical transport. 

On the other hand, as we can see from a comparison with Fig. \ref{fig:fail_case}, \textbf{our OTP successfully learns the optimal transport plan $\pi^{\star}$}. In particular, in the One-to-Many example, our model successfully recovers the correct stochastic transport map $\pi^{\star}(y | x)$ by utilizing the initial noise as guidance to either the upper or lower mode of the target distribution.

\vspace{-7pt}
\paragraph{Quantitative Comparison to Ground-truth}
We evaluate the numerical accuracy of our OTP, SNOT with deterministic generator (OTM \citep{otm}), and SNOT with stochastic generator (OTM-s, Eq. \ref{eq:stochastic_generator}), by comparing them to the closed-form ground-truth solutions. Here, we measure two metrics: the transport cost error $D_{cost} = | W^2_2 (\mu, \nu) - \int \Vert T_{\theta}(x) - x \Vert^2 d\mu(x) |$ and the target distribution error $D_{target} = W^2_2 (T_{\theta \#} \mu, \nu)$.
$D_{cost}$ assesses whether the model achieves the optimal transport cost, while $D_{target}$ measures how accurately the model generates the target distribution.
Both models are tested on two synthetic datasets, Perpendicular and One-to-many (Fig. \ref{fig:fail_case_model}), with generalized dimensions of $d \in \{ 2, 4, 16, 64 \}$ (See Appendix \ref{appen:exp_syn} for dataset details.). 

Tab. \ref{tab:fail_case_quan} presents the quantitative results on the accuracy of the learned optimal transport plan $\pi_{\theta}$. Our OTP consistently achieves comparable or superior performance compared to both OTM and OTM-s across all metrics, particularly in high-dimensional settings. Note that these experimental results confirm the challenges of the existing SNOT framework in accurately recovering the target distributions, as discussed in Sec. \ref{sec:fail_det} and \ref{sec:fail_no_det}. Specifically, as shown in $D_{target}$, OTM and OTM-s models exhibit significantly larger target distribution errors in higher dimensions.


\begin{figure*}[t]
    \begin{minipage}{.43\linewidth}
        % \vspace{20pt}
        \centering
        \subfigure[OTM-s (FID=$62.4$, LPIPS=$0.36$)]{
        \includegraphics[height=0.3\linewidth]{images/main/otm_m2cm_source.png}
        \includegraphics[height=0.3\linewidth]{images/main/otm_m2cm_generate.png}}
        \subfigure[Ours (FID=$3.18$, LPIPS=$0.32$)]{
        \includegraphics[height=0.3\linewidth]{images/main/ours_m2cm_source.png}
        \includegraphics[height=0.3\linewidth]{images/main/ours_m2cm_generate.png}}
        \vspace{-10pt}
        \caption{\textbf{Experimental results on a stochastic transport map application}, i.e., MNIST-to-CMNIST translation.}
        % Experimental Results on MNIST$\rightarrow$CMNIST.

        \label{fig:m2cm}
    \end{minipage}
    \begin{minipage}{0.55\linewidth}
        \captionof{table}{
        \textbf{Image-to-Image translation benchmark} results compared to existing Neural (Entropic) OT models $\dagger$ indicates the results conducted by ourselves. DSBM scores are taken from \citep{asbm, SB-flow}. 
        } \label{tab:main_result}
        \centering
        \scalebox{0.75}{
        \begin{tabular}{c c c c}
        \toprule
        Data & Model  &  FID ($\downarrow$) & LPIPS ($\downarrow$) \\
        \midrule 
        \multirow{4}{*}{Male-to-Female (64x64)} 
        % & CycleGAN \citep{cyclegan} & 12.94 \\
        &  NOT \citep{not} & 11.96 & - \\
        & OTM$^\dagger$ \citep{fanscalable} &6.42 & \textbf{0.16} \\
        %7.xx \jw{5.15} \\
        & DIOTM$^\dagger$ \cite{diotm} & \textbf{4.48} & 0.20 \\
        & OTP (Ours) & 4.75 & 0.20 \\
        \midrule
        \multirow{4}{*}{Wild-to-Cat (64x64)} & DSBM \citep{dsbm} & 20$+$ & 0.59\\
        & OTM$^\dagger$ \citep{fanscalable} & 12.42 & 0.47
        %17.74 
        \\
        & DIOTM$^\dagger$ \cite{diotm} & 10.72 & \textbf{0.45} \\ 
        & OTP (Ours) & \textbf{9.66} & 0.52 \\
        \midrule
        \multirow{5}{*}{Male-to-Female (128x128)} & DSBM \citep{dsbm}
        & 37.8 & 0.25\\
        & ASBM \citep{asbm} & 16.08 & - \\
        & OTM$^\dagger$ \citep{fanscalable} & 7.55  & \textbf{0.21} \\
        & DIOTM$^\dagger$ \cite{diotm} & 7.40  & 0.25\\
        & OTP (Ours) & \textbf{6.38} & 0.27 \\
    \bottomrule
    \end{tabular}}
    \end{minipage}
    \vspace{-10pt}
\end{figure*}

\subsection{Neural OT Evaluation on Unpaired Image-to-Image Translation Tasks} \label{sec:exp_image}
In this section, \textbf{we evaluate our model on the unpaired image-to-image translation task.} The image-to-image translation is one of the most widely used machine learning tasks in Neural OT models. The optimal transport map $T^{\star}$ (or plan $\pi^{\star}$) can be understood as a generator of target distributions, mapping an input $x$ to a \textit{similar} counterpart $y$ by minimizing the transport cost $c(x, y)$. This mapping can be deterministic ($y = T^{\star}(x)$) or stochastic ($y \sim \pi^{\star}(\cdot | x)$). 
Therefore, the optimal transport map can naturally serve as a model for unpaired image-to-image translation.

\vspace{-5pt}
\paragraph{MNIST-to-CMNIST}
First, we demonstrate that our OTP model can learn stochastic transport mappings at the image scale. Specifically, we test our model on the MNIST-to-CMNIST translation task (See Appendix \ref{appen:exp_image} for dataset details).
In this task, our Colored MNIST (CMNIST) dataset consists of three colored variations (Red, Green, and Blue) for each grayscale image from the MNIST dataset (Fig. \ref{fig:concept_m2cm}). Consequently, the desired OP plan should stochastically map each grayscale digit image to a colored digit image of the same digit type (Fig. \ref{fig:concept_m2cm}). 

Fig. \ref{fig:m2cm} illustrates the experimental results. Here, we introduced a stochastic generator to OTM (OTM-s) to provide the capacity to learn a stochastic transport map. However, OTM exhibited the mode collapse problem, transporting all grayscale images to blue-colored images. On the other hand, our OTP successfully learns the optimal transport plan $\pi^{\star}$, achieving a stochastic mapping to Red, Green, and Blue colors. This phenomenon is also observed in the quantitative metrics. Our model significantly outperforms OTM in FID score ($\downarrow$) (3.18 vs. 62.4) and archives a better score in LPIPS ($\downarrow$) (0.32 vs. 0.36).

\vspace{-5pt}
\paragraph{Image-to-Image Translation}
We assess our model on three image-to-image translation benchmarks:  \textit{Male-to-Female \citep{celeba}} ($64\times64$, $128\times128$) and \textit{Wild-to-Cat \citep{afhq}} ($64 \times 64$). For comparison, we include several OT models (\textit{NOT, OTM, and DIOTM}) and Entropic OT models (\textit{DSBM and ASBM}).

Tab. \ref{tab:main_result} presents the quantitative scores for the image-to-image translation tasks (See Appendix \ref{appen:addtional_qual} for qualitative examples). We adopted the FID \citep{fid} and LPIPS \citep{lpips} scores for quantitative evaluation. Note that these FID and LPIPS scores serve similar roles as $D_{cost}$ and $D_{target}$ in Sec \ref{sec:exp_syn}, respectively. 
Our primary evaluation metric is the FID score because it measures how well the translated images align with the target semantics.
As shown in Tab. \ref{tab:main_result}, our model demonstrates state-of-the-art FID scores and competitive LPIPS scores compared to existing (entropic) Neural OT models. Specifically, in the Male-to-Female (128$\times$128) task, our OTP model achieves a FID score of 6.38, outperforming the SNOT model (OTM), the other Neural OT model (DIOTM), and entropic OT models (DSBM and ASBM). Although OTM achieves a lower but comparable LPIPS score (0.21), its significantly worse FID score (7.55) suggests large target semantic errors. Thus, we prioritize FID as our primary metric.

\section{Conclusion}
In this paper, we provided the first theoretical analysis of the sufficient condition that prevents the fake solution issue in Semi-dual Neural OT. Based on this analysis, we proposed our OTP model for learning both the OT Map and OT Plan, even when this sufficient condition is not satisfied. Our experiments demonstrated that OTP successfully recovers the correct OT Plan when existing models fail and achieves state-of-the-art performance in unpaired image-to-image translation. 
Our primary contribution is improving Neural OT frameworks by addressing their fundamental limitations, i.e., failing to recover the correct OT Map even with the ideal max-min solution. 
One limitation of our work is that our convergence theorem holds up to a subsequence (Thm. \ref{thm:convergence}). Nevertheless, in practice, our gradual training scheme (Alg 1) did not show any convergence issues. Also, our analysis provides a sufficient condition, rather than a necessary and sufficient one (Thm. \ref{thm:uniqueness}), leaving room for further refinement in understanding the exact conditions under which fake solutions occur.

\section{Experiments}
\paragraph{Tasks} We evaluate DAB on three distinct controlled-generation tasks: sentiment-guided generation, language detoxification, and keyword-guided generation. These are popular tasks within the field of controlled generation \citep{kumar2022gradient, liu2023bolt, han2023lm}. For all tasks, we produce generations in batches of 20 as done in \citet{liu2023bolt}. For all tasks, we include example generations in Appendix \ref{appndx:senti-details}, \ref{appndx:toxicity-details}, \ref{appndx:keywords-details} for sentiment directed generation, language detoxification, and keyword-guided generation respectively. 

\paragraph{Baselines} We compare to previous generation algorithms that use the EBM framework to perform gradient-based text sampling. Specifically, we compare to MuCOLA \citep{kumar2022gradient}, COLD \citep{qin2022cold}, and BOLT \citep{liu2023bolt}. We also compare against LM-Steer introduced in \citep{han2023lm} to see how our method compares to alternative controlled generation methods. 

\paragraph{Metrics} While the metrics assessing control towards external constraint vary across experiments, we use the same evaluation metrics to measure fluency across experiments. We measure fluency by looking at CoLA score, the number of repeated tri-grams per generation, and perplexity \citep{kumar2022gradient, liu2023bolt}. 
For CoLA score, we use a fine-tuned RoBERTa model to provide a probability as to whether a generation is grammatically correct.
For perplexity, we use GPT-XL to evaluate each generation. 
We show the average of the results across all generations. For more details on these evaluation methods, refer to Appendix \ref{appndx:fluency-metrics}. 
\subsection{Sentiment-Controlled Generation}
\paragraph{Task} Here we measure the ability to direct generation towards positive sentiment from some initial prompt. Given an initial prompt, we generate sequences of length 12, 20, 50 as done in prior works \citep{kumar2022gradient, liu2023bolt}. We use the same set of prompts as \citet{dathathri2020plugplaylanguagemodels} and include them in Appendix \ref{appndx:senti-details}.

\paragraph{Control Metrics} We evaluate control by measuring the predicted sentiment of the generation using three distinct sentiment classifiers.
For details on the training of the three classifiers, see Appendix \ref{appndx:senti-details}. 
We omit the internal classifier measure for LM-Steer as it does not rely on an internal classifier to guide generation. 

\paragraph{Results}
Table \ref{table:megatable} shows that our method achieves a better balance between control and fluency than baselines. DAB achieves the highest average probability of positive sentiment across all three classifiers, demonstrating its effectiveness at incorporating the external constraint.  Furthermore, DAB achieves fluency scores close to BOLT's performance in regards to CoLA score, repeated trigrams, and perplexity. This shows that DAB produces generations that are both fluent and satisfactory under the constraint. 
\begin{table}[t]
\caption{Sampling algorithm performance on sentiment-directed generation, language detoxification, and keyword-guided generation. DAB acheives superior control metrics than baselines across all task. while also demonstrating comparable or better fluency metrics competitive with the best baseline.}
\label{table:megatable}
\centering

\resizebox{\textwidth}{!}{\begin{tabular}{lccc|ccc}
\Xhline{1pt}\\[-1ex]
& \multicolumn{3}{c|}{\textbf{Control}} & \multicolumn{3}{c}{\textbf{Fluency}} \\ 
\textbf{Sentiment}& \textit{Int. Clsf} $\uparrow$ & \textit{Ext. Clsf (Yelp)} $\uparrow$ & \textit{Ext. Clsf (SST-2)} $\uparrow$ &  \textit{CoLA} $\uparrow$ &  \textit{REP-3gram} $\downarrow$ &  \textit{PPL} $\downarrow$\\
\midrule
MuCOLA    & $.841 \pm .009$       & $\underline{.843 \pm .011}$            & $.899 \pm .008$              & $.681 \pm .008$    & $.091 \pm .006$      & $34.786 \pm 2.205$    \\
COLD      & $.697 \pm .011$       & $.515 \pm .015$            & $.670 \pm .013$              & $.731 \pm .008$    & $.061 \pm .003$      & $15.908 \pm .394$     \\
BOLT      & $\underline{.903 \pm .006}$       & $.747 \pm .013$            & $.878 \pm .001$              & $\mathbf{.874 \pm .005}$ & $\mathbf{.0008 \pm .0002}$ & $\mathbf{9.919 \pm .142}$ \\
LM-Steer  & -                     & $\mathbf{.900 \pm .008}$            & $\underline{.948 \pm .006}$              & $.564 \pm .008$    & $.117 \pm .007$      & $72.153 \pm 3.195$    \\
DAB \textit{(Ours)}  & $\mathbf{.992 \pm .001}$ & $\mathbf{.894 \pm .009}$ & $\mathbf{.975 \pm .003}$     & $\underline{.860 \pm .005}$    & $\underline{.004 \pm .001}$      & $\underline{11.773 \pm .203}$     \\[1ex]
\Xhline{1pt} \\[-1ex]
\textbf{Toxicity} & \textit{Int. Clsf} $\downarrow$ & \makecell{\textit{Avg. Max Toxicity} $\downarrow$} & \makecell{\textit{Toxicity Pred. Prob.} $\downarrow$} & \textit{CoLA} $\uparrow$ & \textit{REP-3gram} $\downarrow$ & \textit{PPL} $\downarrow$\\
\midrule
MuCOLA    & $.098 \pm .002$       & $.269 \pm .006$            & $7.6\%$              & $.691 \pm .002$    & $.006 \pm .001$      & $58.015 \pm .435$     \\
COLD      & $.136 \pm .002$       & $.266 \pm .007$            & $10.2\%$              & $.667 \pm .001$    & $.024 \pm .001$      & $38.891 \pm .177$     \\
BOLT      & $\underline{.065 \pm .001}$       & $\underline{.264 \pm .006}$            & $\mathbf{6.8\%}$     & $\mathbf{.830 \pm .001}$ & $\mathbf{.001 \pm .0001}$ & $\underline{27.283 \pm 2.233}$ \\
LM-Steer  & -                     & $\underline{.265 \pm .006}$            & $\underline{7.9\%}$              & $.722 \pm .002$    & $\underline{.006 \pm .002}$      & $52.697 \pm .356$     \\ 
DAB \textit{(Ours)}    & $\mathbf{.057 \pm .001}$ & $\mathbf{.211 \pm .006}$ & $\mathbf{6.8\%}$     & $\underline{.806 \pm .001}$    & $\mathbf{.001 \pm .0001}$ & $\mathbf{25.609 \pm .126}$ \\[1ex]
\Xhline{1pt} \\[-1ex]
\textbf{Keyword}& \textit{BertScore} $\uparrow$ & \textit{Success Rate} $\uparrow$ & - & \textit{CoLA} $\uparrow$ & \textit{REP-3gram} $\downarrow$ & \textit{PPL} $\downarrow$ \\
\midrule
MuCOLA    & $.8083 \pm .0004$     & $\mathbf{100\%}$           &            -             & $.248 \pm .004$    & $.007 \pm .001$      & $475.301 \pm 30.445$ \\
COLD      & $.8123 \pm .0005$     & $\mathbf{100\%}$           &             -             & $.205 \pm .003$    & $.020 \pm .001$      & $241.980 \pm 4.943$  \\
BOLT      & $\underline{.8291 \pm .0003}$     & $99.1\%$                   &             -             & $\underline{.705 \pm .006}$    & $\underline{.005 \pm .005}$      & $\underline{32.019 \pm 1.593}$   \\
DAB  \textit{(Ours)}    & $\mathbf{.8303 \pm .0003}$ & $99.0\%$               &            -              & $\mathbf{.726 \pm .005}$ & $\mathbf{.004 \pm .001}$ & $\mathbf{23.424 \pm .317}$ \\
\bottomrule
\end{tabular}}
\end{table}
\subsection{Toxicity Avoidance}
\paragraph{Task} We compare our algorithm to various baselines for the task of language detoxification to demonstrate that our method can be used to mitigate potentially toxic LLM generations. 
Following prior work, we use 1,000 prompts sampled from the RealToxicityPrompts introduced and generate continuations of length 20 tokens \citep{gehman2020realtoxicitypromptsevaluatingneuraltoxic, kumar2022gradient, liu2023bolt}.
As in the sentiment-directed generation task, we use a fine-tuned RoBERTa as the constraint function. 
\paragraph{Control Metrics} We evaluate the generations using both the internal discriminator used to guide the various methods, and the score returned by the Perspective API \citep{lees2022newgenerationperspectiveapi}. We use the scores returned from Perspective API to calculate the maximum toxicity per prompt and the overall percentage of text predicted to be toxic. 
\paragraph{Results} As shown in Table \ref{table:megatable}, our method generates less toxic text than baselines without compromising fluency. 
DAB significantly decreases the average maximum toxicity per prompt, demonstrating that our algorithm is more consistent in terms of toxicity reduction. 
Furthermore, our method obtains fluency metrics that are on par with the best baseline. 
\subsection{Keyword-guided Generation}
\begin{keywords}
	Mixture approximations,
	distributed time delays,
	delay differential equations,
	linear chain trick.
\end{keywords}

\section{Conclusion}
This work identifies signal collapse as a critical bottleneck in one-shot neural network pruning. Performance loss in pruned networks is due to \textbf{signal collapse} in addition to the removal of critical parameters. We propose \textbf{REFLOW} (\textbf{Re}storing \textbf{F}low of \textbf{Low}-variance signals), a simple yet effective method that mitigates signal collapse without computationally expensive weight updates. By focusing on signal preservation, REFLOW highlights the importance of mitigating signal collapse in sparse networks and enables magnitude pruning to match or surpass state-of-the-art one-shot pruning methods such as CHITA, CBS, and WF.

REFLOW consistently achieves state-of-the-art accuracy across diverse architectures, restoring ResNeXt-101 from under 4.1\% to 78.9\% top-1 accuracy at 80\% sparsity on ImageNet. Its lightweight design makes it a practical solution for both research and deployment, delivering high-quality sparse models without the overhead of traditional approaches. These findings challenge the traditional emphasis on weight selection strategies and underscore the critical role of signal propagation for achieving high-quality sparse networks in the context of one-shot pruning.



\section*{Ethics}
\section*{Ethics Statement}

Our research adheres to rigorous ethical standards while contributing to the advancement of NLP. We exclusively utilize publicly available language models and benchmarks in our experiments. The datasets employed in our study—HellaSwag (MIT), PiQA (AFL), ARC (CC-BY-SA 4.0), LogiQA (CC-BY-NC-SA 4.0), RACE (AFL), SciQ (CC-BY-NC 3.0), and MMLU (MIT)—are all permitted for academic use. We ensure full compliance with their respective license requirements. Furthermore, while our research presents evaluation results across various models, it contains no information that could harm individuals or groups.
\section*{Reproducibility}
In order to ensure the reproducibility of our work, we include the details necessary to replicate both the core algorithms and the experiments. 
We include the psuedo-code for our algorithm in \ref{alg:text-gen}; information on hyper-parameter settings in Appendix \ref{appndx:ablation}l and further details on each experiment in Appendix \ref{appndx:senti-details}, \ref{appndx:toxicity-details}, \ref{appndx:keywords-details}. Additionally, we include the code-base used to produce our results at the following repository: \url{https://github.com/patrickpynadath1/dab}.  
% \newpage

\bibliography{iclr2025_conference}
\bibliographystyle{Styling/iclr2025_conference}
\newpage
\appendix
\section{Discrete Langevin Proposal}\label{appndx:dlp_proposal}
Our proposed controlled text generation leverages the gradient-based discrete sampling algorithm in \citet{zhang2022langevinlike}, which is further investigated by \citet{pynadath2024gradientbaseddiscretesamplingautomatic}. Using the same notation as in the Main Body of the paper, we put the original proposal distribution from \citet{zhang2022langevinlike} below:
\[
\text{Categorical} \left( \underset{j \in |V|}{\softmax} \left( \frac{1}{2} \nabla f(\hat{B} | X)_i (\text{Onehot}_j - \hat{b}_i) - \frac{||\text{Onehot}_j - \hat{b}_i ||^2_2}{2\alpha}\right) \right)
\]
Here, $\hat{b}_i$ corresponds to the one-hot vector in sequence position $i$. Similarly, $\text{Onehot}_j$ corresponds to the one-hot vector for the $j$th token in $V$. This proposal function defines a distribution over the vocabulary for the $i$th sequence position in the sequence by taking the softmax over all possible tokens.

As discussed in \citet{pynadath2024gradientbaseddiscretesamplingautomatic}, this proposal is locally balanced, or optimal for very small step-sizes. For the task of controlled text generation, we would prefer a proposal function that is optimal for large step-sizes, which allow for superior exploration of the space of potential sequences. The globally balanced proposal can be written as follows: 
\[
\text{Categorical} \left( \underset{j \in |V|}{\softmax} \left( \nabla f(\hat{B} | X)_i (\text{Onehot}_j- \hat{b}_i) \right) \right)
\]
In terms of the gradient computation, the one-hot representation  enables the use of automatic differentiation packages to compute $\nabla f(\hat{B} | X)$. We observe that the term 
$(\text{Onehot}_j - \hat{b}_i)$ corresponds to the distance between the proposed token $j$ and the original token $b_i$. We choose to represent this distance term as hamming distance, given the discrete nature of the space we wish to sample. For a token $j$, the hamming distance to the original token in position $i$ is 0 if the $j$th coordinate $\hat{b}_{ij} = 1$ as they are the same token; and 1 if the $j$th coordinate is 0. Thus we can represent the distances between the tokens as $1 - \hat{b}_{ij}$. This leads us to the proposal function in \ref{eq:dlp_prop}, which we place below for convenience: 
\[
b'_i \sim
    \categorical\left(\underset{j \in V}{\softmax} \left( \frac{1}{\tau} (\nabla f(\hat{B} | X))_{ij} (1 - \hat{b}_{ij}) \right) \right)
\]
Here, $b'_i$ refers to the token we sample from the categorical distribution over $V$. 

\section{Algorithmic Details} \label{appndx:algrthm-details}
Here we provide the full pseudo-code for our algorithm. 
\begin{algorithm}
    \caption{Discrete Autoregressive Biasing}
    \begin{algorithmic}[1]
    \REQUIRE Constraint function $f$, $P^{LM}$, prompt $X$, number steps $s$, sequence length $n$, embedding table $M$
    \STATE $\tilde{B} \gets \vec{0}, f_\text{min} \gets -\infty$, $Y_\text{best} \gets \{\}$ \LineComment{Initialize constraint violation as being maximal and current best generation as empty}
    \FOR{step $s$}
        \FOR{position $i$ in range($n$)} 
            \STATE $\tilde{y_i} \gets \log P^{LM} (\cdot | y_{<i}, X)$ \LineComment{Initial auto-regressive distribution over $V$}
            \STATE Calculate normalizing factor $r_i$ if $s > 1$, else $r_i \gets 1$
            \STATE $y_i \gets \text{argmax}_{j \in |V|} \left(\tilde{y}_{i, j} - w_i \cdot r_i \cdot \tilde{b}_{i, j} \right)$ \LineComment{Sample from $P(Y | X, B)$}
        \ENDFOR
        \STATE $B \gets Y$ \LineComment{Initialize $B$ as $Y$}
        \STATE Evaluate $f(B | X)$, update $f_\text{min}$, $Y_\text{best}$
        \STATE $B' \sim q_\tau(\cdot | B)$ as in \eqref{eq:dlp_prop} \LineComment{Approximately sample from $P(B | X, Y)$}
        \STATE Compute $\tilde{B}$ as in \eqref{eq:bias-vec-def}
    \ENDFOR
    \STATE return $Y_\text{best}$
    \end{algorithmic}
\label{alg:text-gen}
\end{algorithm}

DAB takes as input the external constraint $f$, the base language model $P^{LM}$, prompt $X$, number of steps $s$, sequence length $n$, and embedding table $M$. Given these inputs, our proposed algorithm alternates between auto-regressively generating the response sequence and sampling the bias sequence using Discrete Langevin Proposal (DLP) \citep{zhang2022langevinlike}. 

\section{Ablation Study}
\label{appndx:ablation}
\begin{table*}
  [t]
  \centering
  \resizebox{\textwidth}{!}{%
  \begin{tabular}{cccccccccccc}
    \toprule \multicolumn{2}{c}{Components}                                                             & \multicolumn{5}{c}{Re-executability Rate (\%)} & \multicolumn{5}{c}{Readability (\#)} \\
    \cmidrule(lr){1-2} \cmidrule(lr){3-7} \cmidrule(lr){8-12}        \hspace{8pt}\labelemoji\hspace{8pt}                                                                & \hspace{8pt}\toolemoji\hspace{8pt}                                      & O0                                 & O1             & O2             & O3             & AVG            & O0             & O1             & O2             & O3             & AVG            \\
    \hline
    \rowcolor[rgb]{0.93,0.93,0.93}\multicolumn{12}{c}{\textbf{Initialize with LLM4Decompile-End-6.7B~\citep{llm4decompile}}}   \\
    \xmark                                                                                              & \xmark                                    & 69.51                              & 46.95          & 50.61          & 46.34          & 53.35          & 3.98 & 3.41 & 3.44 & 3.38 & 3.55 \\
    \cmark                                                                                              & \xmark                                    & 75.61                              & 50.61          & 50.00          & 50.00          & 56.55          & 4.01 & 3.44 & 3.39 & \textbf{3.49} & 3.58 \\
    \xmark                                                                                              & \cmark                                    & 83.54                     & \textbf{56.10}          & 51.22          & 50.61 & 60.37 & 4.05 & 3.51 & 3.51 & 3.42 & 3.62 \\
    \cmark                                                                                              & \cmark                                    & \textbf{85.37}                            & \textbf{56.10}                     & \textbf{51.83} & \textbf{52.43}          & \textbf{61.43} & \textbf{4.13} & \textbf{3.60} & \textbf{3.54} & \textbf{3.49} & \textbf{3.69} \\

    \rowcolor[rgb]{0.93,0.93,0.93}\multicolumn{12}{c}{\textbf{Initialize with Deepseek-Coder-6.7B-base~\citep{deepseekcoder}}} \\
    \xmark                                                                                              & \xmark                                    & 59.15                              & 35.98          & 39.02          & 37.80          & 42.99          & 3.71 & 3.05 & 3.16 & 3.05 & 3.24 \\
    \cmark                                                                                              & \xmark                                    & 66.46                              & 41.46          & 38.41          & 36.59          & 45.73          & 3.76 & 3.17 & \textbf{3.21} & 3.08 & 3.31 \\
    \xmark                                                                                              & \cmark                                    & 70.73                              & 39.63          & 39.02          & 40.24          & 47.41          & 3.90 & 3.17 & 3.08 & 3.11 & 3.31 \\
    \cmark                                                                                              & \cmark                                    & \textbf{79.88}                     & \textbf{45.73} & \textbf{43.90} & \textbf{42.68} & \textbf{53.05} & \textbf{3.96} & \textbf{3.21} & 3.18 & \textbf{3.19} & \textbf{3.38} \\
    \bottomrule
  \end{tabular}%
  }
  \caption{The ablation study of different methods across four optimization levels
  (O0, O1, O2, O3), as well as their average scores (AVG). The results in bold represent the optimal performance. The ~\labelemoji~ and ~\toolemoji~ means Relabedling and Function Call. \textbf{Bold} denotes the best performance.}
  \label{tab:ablation}
\end{table*}

\subsection{Efficiency}
\label{appndx:efficiency}
\setlength{\tabcolsep}{3pt}
\begin{table}
\centering
\caption{Efficiency of Different Methods}
\vspace{-0.1in}
\label{tab:efficiency}
% \resizebox{\textwidth}{!}{%
\scalebox{0.65}{%
\begin{tabular}{|c|ccc|ccc|ccc|}
  \hline
  \multirow{2}{*}{} & \multicolumn{3}{c|}{\textbf{memory size (MB)}} & \multicolumn{3}{c|}{\textbf{training time (s)}} & \multicolumn{3}{c|}{\textbf{matching time (s)}}\\
  % \cline{2-7}
  % {} & MByte & seconds/ep & seconds\\
  {} & \textbf{Beijing} & \textbf{Porto} & \textbf{Chengdu} & \textbf{Beijing} & \textbf{Porto} & \textbf{Chengdu} & \textbf{Beijing} & \textbf{Porto} & \textbf{Chengdu}\\
  % \cline{2-7}
  % {} & (MByte) & (minutes/ep) & (seconds/K) & (MByte) & (minutes/ep) & (seconds/K)\\
  \hline
  \textbf{MDP} & 1819MB & 2039MB & 2122MB & - & - & - & 389.14s & 361.15s & 599.51s  \\
  \textbf{HMM} & 1209MB & 1388MB & 1361MB & - & - & - & 427.97s & 380.05s & 589.08s \\
  % \hline
  \textbf{FMM} & 897MB & 931MB & 981MB & - & - & - & 1.13s & 1.02s & 1.87s \\
  % \hline
  \textbf{AMM} & 957MB & 1013MB & 1124MB & - & - & - & 3.42s & 3.05s & 5.16s \\
  % \hline
  \textbf{MTrajRec} & 9045MB & 12428MB & 11265MB & 182.4s & 2200.2s & 25672.4s & 51.22s & 42.27s & 73.68s\\
  % \hline
  \textbf{L2MM} & 9087MB & 11875MB & 10898MB & 189.1s & 2314.2s & 27032.2s & 6.71s & 5.26s & 9.10s\\
  % \hline
  \textbf{GraphMM} & 8537MB & 11752MB & 10378MB & 48.4s & 645.2s & 7311.4s & 8.06s & 6.96s & 11.18s\\
  % \hline
  \textbf{\modelName} & 2530MB & 2299MB & 2357MB & 11.9s & 126.4s & 1507.8s & 1.09s & 0.95s & 1.65s\\
  \hline
\end{tabular}}
\vspace{-0.15in}
\end{table}


\section{Experimental Details}
Here we include additional details on the experiment setup. We provide the hyper-parameter settings for our algorithm  for each experiment in Table \ref{appndx:tab:exp-hyperparam}. It should be noted that for Sampling Steps, we pick values to maintain roughly the same time cost as BOLT: given that our algorithm is roughly twice as fast, we use around twice the number of sampling steps. Furthermore, given the use of early stopping in BOLT, further computational budget doesn't necessarily provide any advantage. 

For the weight value, we use a schedule by \citet{liu2023bolt} as it was shown to be effective in terms of incorporating the bias term into auto-regressive generation. Thus for each position $t$, we have $w_t = w(1 - \frac{t}{L})$, where $w$ is the value we put in Table \ref{appndx:tab:exp-hyperparam}. 
\begin{table}[h]
\caption{Hyper-parameter settings used for DAB on Sentiment-directed generation, language detoxification, and topic-constrained generation.}
\centering
\label{appndx:tab:exp-hyperparam}
\begin{tabular}{l|cccccl}\toprule
     \textit{Hyper-parameter} & \textit{Sentiment} & \textit{Detoxify} & \textit{Topic} \\ \midrule 
Proposal Temp & .1 & .1 & .1\\ 
Top-k & 250 & 250 & 250 \\
Bias Weight Value & 1.05 & 1.05 & 1.4 \\
Number Sample Steps & 20 & 20 & 200 \\ 
\bottomrule
\end{tabular}
\end{table} 

\subsection{Fluency Metrics}
\label{appndx:fluency-metrics}
Here we provide more details as to the metrics we use to evaluate the fluency of text generations. 
\paragraph{CoLA Score} To assess the grammatical correctness of a generation, we use a fine-tuned RoBERTa model from \citet{morris2020textattack} to predict the probability of the sample being labelled as grammatically correct. While a similar metric was used in \citet{kumar2022gradient}, we compute the average predicted probability as opposed to the percentage over generations predicted as fluent since this provides more insight into the degree of grammatical correctness. 

\paragraph{Repeated Tri-grams} To compute the number of repeated tri-grams, we simply count all the tri-grams that were repeated and divide them by the total number of tri-grams per generation. We show the average across all generations for each metric. 

\paragraph{Perplexity} For perplexity, we use the built-in function within the Hugging Face evaluate package to compute the perplexity of each generation according to GPT2-XL \citep{wolf2020huggingfacestransformersstateoftheartnatural}. We show the perplexity of the \textbf{entire} generation, as opposed to conditioning on the prompt as done in \citet{han2023lm, kumar2022gradient, liu2023bolt}. 

\subsection{Sentiment Controlled Generation}
\label{appndx:senti-details}
\paragraph{Experiment Design} We use the same experimental design from \citet{liu2023bolt}, where the sampler uses an internal classifier to produce the generations. The internal model is a RoBERTA with GPT2-Large Embeddings fine-tuned on the yelp polarity dataset. We use two external models to provide additional evaluation: we use another RoBERTA trained on the same dataset but with the original embeddings, as well as a RoBERTa fine-tuned on Stanford Sentiment Treebank 2. 

We include the hyper-parameters we use for DAB in Table \ref{appndx:tab:exp-hyperparam}. For the baselines, we run the code within their codebase. While we minimize the changes made to the original code, we note that there are some necessary modifications needed in order to ensure that the experimental setting is consistent across all methods evaluated. This due to the fact that all the evaluated methods consider similar but slightly different experiments from ours in their original work \citep{qin2022cold, liu2023bolt, han2023lm, kumar2022gradient}.  

In regards to LM-Steer, which requires training data, we train the steering matrix using the SST-2 dataset, as done in \citet{han2023lm}. While this is a different dataset from what was used to fine-tune the internal classifiers for the EBM sampling methods, we choose this dataset as obtained worse results when training the steer matrix on yelp polarity. Furthermore, we include an external classifier fine-tuned on SST-2 to use as an evaluation criteria. This makes our experiments fair, as all the methods are evaluated with classifiers that are fine-tuned on a different dataset than used for sampling. Lastly, we observe that LM-steer achieves reasonable performance in terms of sentiment control when compared to other baselines. 

Here we list the prompts we use for this experiment: 

\paragraph{External Constraint} To represent the internal constraint, we use a RoBERTA with GPT-2 large embeddings fine-tuned on Yelp-Polarity for COLD, BOLT, MuCOLA, and DAB. We train this model following the codebase of \citet{liu2023bolt}. Since we require the embedding table to be the same between the base LM, we use the GPT2-large embeddings for the classifier, as done in \citet{liu2023bolt, kumar2022gradient}. 
% The final classifier achieves an accuracy of $96\%$ on the hold-out. 

We use a slightly different function to represent the constraint imposed by the fine-tuned model when compared to BOLT. Given the discriminator $h: |V| \to \mathbf{R}^2$, where the results represent the logits for both the desired class $c_{+}$ and the undesired class $c_{-}$, we define the final constraint function as follows: 
\begin{align*}
    f(Y) = (h(Y)_{+} - h(Y)_{-})
\end{align*}
Intuitively, this pushes the unnormalized logits between the desired class and the opposite class away from each other.

This differs from the constraint function in BOLT, which is the typical cross-entropy loss of the discriminator logits where the correct label is the desired sentiment: 
\begin{align*}
    f(Y) &= \log \softmax (h(Y)_{+})
\end{align*}

We find that our formulation of the constraint function enables more effective gradients for our specific method. 
Curiously, this modification does not provide any substantial benefit to BOLT. 
It is possible that the $\log \softmax$ of BOLT's method smooths out the directional information of the gradient. While this would benefit a continuous sampling algorithm, this could potentially remove some directional information that is required for effective discrete sampling. 

\paragraph{Example Generations} In Table \ref{appendix:tab:senti-gens} we include examples of generations for all methods evaluated.  
\begin{table}[!h]
\caption{Example text for Sentiment-guided generation. As visible, previous methods either produce coherent text that is not positive enough or positive text that is incoherent. In contrast, our method produces generations that are both overtly positive and coherent.}
\centering
\resizebox{\textwidth}{!}{
\begin{tabular}{p {.2 \textwidth}|p {.8 \textwidth}}
\toprule
 \textbf{Prompt} & \textit{The horse} \\
\toprule
COLD & The horse head was still in the water but the horse still had a good head. The horse \\ \midrule
MuCOLA & The horse is not only a beautiful and well-crafted piece of art, but it is also a great way \\ \midrule
BOLT & The horseback riding course is a great way to get acquainted with the trails and the terrain. The course is \\  \midrule
LM-Steer & The horseman delivers a stunningly beautiful, wonderfully lyrical, beautifully tender, powerfully moving, beautifully lyrical \\ \midrule
DAB (Ours) & The horse is also a very good and reliable companion. It has been used to carry the family's two- \\ 
\toprule
 \textbf{Prompt} & \textit{The chicken} \\
\toprule
COLD & The chicken is a good source of protein, but it's a very high fat, cholesterol and salt. \\ \midrule
MuCOLA & The chicken is a very healthy, and high in protein, and the other two chicken is a very good, \\ \midrule
BOLT & The chicken, which is named after the popular street food in Mexico, was a gift from a friend who had \\ \midrule 
LM-Steer & The chicken is beautifully tender, beautifully sweet, wonderfully colorful. the crisp, crispiness is wonderful. and the \\  \midrule
DAB (Ours) & The chicken is a great way to get the kids hooked on delicious, healthy food. It's also a good \\ 
\bottomrule
\end{tabular}
}
%\end{small}
% \vspace{-10pt}
\label{appendix:tab:senti-gens}
\end{table}


\subsection{Toxicity Controlled Generation}\textcolor{red}{Content Warning: The following section contains examples of LLM generations that some may find disturbing or offensive, including content that is hateful or violent in nature}

\label{appndx:toxicity-details}
\textbf{Experiment Design} Similar to the sentiment control experiment, we use an internal classifier to guide generation towards less toxic models. 
We use a RoBERTa classifier fine-tuned on the Jigsaw toxic comment dataset, following \citet{kumar2022gradient, liu2023bolt}. 


\textbf{External Constraint} We represent the external constraint by a RoBERTa model with GPT-2 large embeddings that was fine-tuned on the Jigsaw dataset, following the same training protocol as used in \citet{liu2023bolt}. The final classifier achieves an accuracy of $94\%$ on the hold-out dataset.  

\paragraph{Examples} In Table \ref{appendix:tab:detox-gens} we show examples of generations for the various samplers we examine. 
\begin{table}[!h]
\centering
\caption{Examples for Language Detoxification. While all methods seem to have similar probabilities of generating toxic text, here we see the qualitative impact of our method having a lower average max toxicity: while other methods can end up producing very offensive generations, our method is able to ensure that the generation is not extremely toxic.}
\resizebox{\textwidth}{!}{
\begin{tabular}{p {.2\textwidth}|p {.8 \textwidth}}
\toprule
 \textbf{Prompt} & One student’s complaint; however, has led to the logo being associated with \\
\toprule
COLD & One student’s complaint; however, has led to the logo being associated  with the school. The school has since been forced to take the logo and the school's \\ \midrule
MuCOLA & One student’s complaint; however, has led to the logo being associated in various East Semitic cultures. What can you do to help save the culture?Who \\ \midrule
BOLT & One student’s complaint; however, has led to the logo being associated with a negative image. The school chose to remove it from their website after complaints from students and parents \\ \midrule
LM-Steer & One student’s complaint; however, has led to the logo being associated with the campaign.  And Turner had previously written Turner School Turner ’. Turner School Turner Turner \\ \midrule
DAB (Ours) & One student’s complaint; however, has led to the logo being associated with a site that is not affiliated with the University of California. The university's website and its social \\ \toprule
 \textbf{Prompt} & During a show in Sydney in 2013, Tyler went on a tirade against a member of the group, \\
\toprule
COLD & During a show in Sydney in 2013, Tyler went on a tirade against a member of the group,  calling him a "fucking faggot" and a "faggot" and a " \\ \midrule
MuCOLA & During a show in Sydney in 2013, Tyler went on a tirade against a member of the group, telling him: "I want you to go back to your f***ing country." He also apologised \\ \midrule
BOLT & During a show in Sydney in 2013, Tyler went on a tirade against a member of the group, referring to him as "a piece of shit." The singer also publicly criticized the band's decision to \\ \midrule
LM-Steer & During a show in Sydney in 2013, Tyler went on a tirade against a member of the group, writing,As for the woman who called Tyler a slut, I'm afraid you don't \\ \midrule
DAB (Ours) & During a show in Sydney in 2013, Tyler went on a tirade against a member of the group, who was considered to be his friend. The man had been seen with the band at a party and \\ 
\bottomrule
\end{tabular}
}
\label{appendix:tab:detox-gens}
\end{table}



\subsection{Keyword Controlled Generation}
\label{appndx:keywords-details}
\paragraph{Experiment Design}
We largely follow the experimental design introduced in \citet{liu2023bolt}. We list the topics and keywords in \ref{table:kw-topics}. 

\paragraph{Constraint Function} Following \citet{liu2023bolt}, \citet{qin2022cold}, we use the differentiable BLEU score introduced by \citet{liu-etal-2022-dont}. 
This function measures the uni-gram similarity between the generated sentences and the target key-words, using an operation very similar to convolution. 


\begin{table}
\caption{List of topics and correspending keywords.}
\label{table:kw-topics}
\centering
\begin{tabular}{p {.15 \textwidth} | p {.25 \textwidth}}\toprule
      \textbf{Topic}& \textbf{Keywords} \\\midrule
computer & router, Linux, keyboard, server \\ \midrule
legal & plea, subpoena, transcript, bankrupt \\ \midrule
military & torpedo, headquarters, infantry, battlefield \\ \midrule
politics & court, culture, communism, capitilism\\ \midrule
religion & Bible, church, priest, saint \\ \midrule
science & microscope, mass, mineral, scientist\\ \midrule
space & meteor, planet, satellite, astronaut
\\\bottomrule
\end{tabular}
\end{table}

\paragraph{Reference Text Generation} We use GPT-4o to generate high-quality reference text to use in the BertScore computation. For a given topic t and keyword k, we query GPT-4o with the following prompt: 

\textit{Given the topic t and the keyword k, write 30 different, unique sentences using the keyword and relevant to the topic.}

We do this for each topic and for every keyword for that topic. This produces 120 different, unique sentences to use as a reference text in the BertScore computation. 

\paragraph{BertScore Computation Details}
We use the BertScore computation introduced in \citet{zhang2020bertscoreevaluatingtextgeneration} to evaluate the topicality of the generations. Since BertScore relies on the contextualized embedding of the candidate generations and the reference text, this provides insight into how well the methods use the keyword in the desired context. 

For each generation, we compute the BertScore against all the 120 reference sentences for the corresponding prompt and keyword. Because some of the reference text will not contain the keyword used in the generation, we use report the precision metric calculated in BertScore instead of the overall F1 score, as the precision metric matches tokens in the candidate generation to tokens in the reference text. This is preferable as we want to assess whether the generation is similar to any of the reference texts, as opposed to measuring whether all the reference texts are similar to the candidate generation.  

\paragraph{Implementation Details}
We found that in order to obtain good results with DAB on this task, it was necessary to include a string containing the keywords prior to the prompt. More specifically, we included the following string before the initial prompt for keywords $K$ and topic $t$: 

\textit{Include the following keywords: K relevant to t.}

By including the target keywords and topic before the prompt, this increases the probability of these words and similar words in the underlying language model distribution. This enables the bias vectors computed in our method to have a more impact on auto-regressive generation process and thus satisfy the external constraint. 

In order to ensure that this was not providing our method with an unfair advantage, we applied the same trick to BOLT in order to determine whether this would improve the performance of BOLT as well. We provide results in Table \ref{table:kw-prompted-comp}. 
\begin{table}[t]
\caption{Comparison on topic-guided generation between the original BOLT method, the prompted BOLT method, and DAB. As visible, even if the prompt manages to improve the success rate by $.7\%$, this comes at the cost of worse fluency and slightly worse topicality. Furthermore, our method still outperforms this baseline.
}
\label{table:kw-prompted-comp}
\centering

\resizebox{\textwidth}{!}{\begin{tabular}{lcc|ccc}
\Xhline{1pt}\\[-1ex]
& \multicolumn{2}{c|}{\textbf{Control}} & \multicolumn{3}{c}{\textbf{Fluency}} \\ 
\textbf{Topic}& \textit{BertScore} $\uparrow$ & \textit{Success Rate} $\uparrow$ & \textit{CoLA} $\uparrow$ & \textit{REP-3gram} $\downarrow$ & \textit{PPL} $\downarrow$ \\
\midrule
BOLT      & $.8291 \pm .0003$     & $99.1\%$                  & $.705 \pm .006$    & $.005 \pm .005$      & $32.019 \pm 1.593$   \\
BOLT (Prompted)       & $.8123 \pm .0002$     & $\mathbf{99.7\%}$          & $.705 \pm .005$    & $.005 \pm .001$      & $38.22 \pm .951$  \\
DAB  \textit{(Ours)}    & $\mathbf{.8303 \pm .0003}$ & $99.0\%$             & $\mathbf{.726 \pm .005}$ & $\mathbf{.004 \pm .001}$ & $\mathbf{23.424 \pm .317}$ \\
\bottomrule
\end{tabular}}
\end{table}

As visible, while the prompt does improve the success rate marginally, it does not improve any other metrics for BOLT. In fact, we see that this degrades BOLT's fluency slightly through a higher perplexity value. 
\paragraph{Examples} In Table \ref{appendix:tab:kw-gens} we show examples of generations for the various samplers we examine. 
\begin{table}[h]
\centering
\caption{Examples for Topic-Constrained Generation. As visible, while previous methods include the keyword, they tend to either repeat the keyword too many times or misuse the keyword. In contrast, our method is able to include the keyword in a meaningful way relevant to the given topic.}
\resizebox{\textwidth}{!}{
\begin{tabular}{p {.2 \textwidth}| p {.8 \textwidth}}
\toprule
\textbf{Prompt} & Once upon a time \\
\textbf{Topic} & Military \\ 
\textbf{Keywords} & torpedo, headquarters, infantry, battlefield \\
\toprule
\textit{COLD} & Once upon a time, the world was a peaceful place. People were \textbf{headquarters} of the world \textbf{headquarters} of the world \textbf{torpedo}- \\ \midrule
\textit{MuCOLA} & Once upon a time, the world was a world of the great \textbf{battlefield} the powerful \textbf{headquarters} a \textbf{torpedo} of the good and \textbf{infantry}\\ \midrule
\textit{BOLT} & Once upon a time, there was a man named John Smith who had a dream that he would be able to \textbf{infantry} his \\  \midrule
\textit{DAB} (Ours) & Once upon a time, there was a small group of officers who were in charge of the modern \textbf{infantry} and logistics. They \\ \toprule
\textbf{Prompt} &  The book \\
\textbf{Topic} &  Science \\
\textbf{Keywords} & microscope, mass, mineral, scientist \\
\midrule
\textit{COLD} & The book is scientist-driven, and is a scientist mineralogist, \textbf{microscope}, \textbf{microscope}, \textbf{mineral} \textbf{microscope}, \\ \toprule
\textit{MuCOLA} & The book also has \textbf{mass}ive properties, like the Alabaster House, which features extensive characters from Alabaster \\ \midrule
\textit{BOLT} & The book is divided into three parts, each of which contains a chapter \textbf{mass} mineral scientist relevant to science. scientist \\  \midrule
\textit{DAB} (Ours) & The book is a good introduction to the field of \textbf{mass} spectrometry and is an excellent resource for hands- \\ 
\bottomrule
\end{tabular}}
\label{appendix:tab:kw-gens}
\end{table}



\end{document}
