\section{Preliminaries}
%\subsection{Problem Formulation}
Let $M_o$ denote the original large language model and $M_p$ the pruned version of this model. We define the instruction tuning dataset as $D = \{(x_i, y_i)\}_{i=1}^N$, where $x_i$ represents an instruction and $y_i$ its corresponding output. Our goal is to select a subset $S \subset D$ to efficiently recover the performance of $M_p$. We formulate the problem as an optimization task:

\begin{equation}
\begin{aligned}
    S^* &= \argmin_{S \subset D, |S| \leq B} \mathbb{E}_{(x,y) \sim \mathcal{T}} [\mathcal{L}(M_r(S), x, y)],\\
    \text{subject to}\quad &M_r(S) = \text{RecoveryTrain}(M_p, S) 
\end{aligned}    
\end{equation}

where $M_r(S)$ is the recovered model after training on subset $S$, $\mathcal{T}$ is the distribution of downstream evaluation tasks, $\mathcal{L}$ is a loss function. $B (B < N)$ is the recovery data budget, i.e., maximum number of samples allowed in the selected subset.

%\subsection{Theoretical Foundation}

%We introduce the concept of Capability Recovery Efficiency (CRE) to quantify the effectiveness of a data point in recovering model capabilities:

%\begin{definition}[Capability Recovery Efficiency]
%For a data point $(x, y)$ and a capability $c$, the Capability Recovery Efficiency is defined as:
%\begin{equation}
%    \text{CRE}(x, y, c) = \frac{\Delta \text{Performance}_c(M_p, M_r)}{\text{ComputationalCost}(x, y)}
%\end{equation}
%where $\Delta \text{Performance}_c$ measures the improvement in capability $c$ after recovery training, and ComputationalCost measures the resources required to process $(x, y)$, which is instantiated with token number considering LLM's autoregressive property.
%\end{definition}

%We hypothesize that maximizing the overall CRE leads to efficient model recovery:

%\begin{theorem}[Efficient Recovery]
%Given a computation resource budget $U$, the optimal recovery subset $S^*$ maximizes the sum of CRE across all capabilities:
%\begin{equation}
%\begin{aligned}
%    S^* &= \argmax_{S \subset D, |S| \leq B} \sum_{c \in C} \sum_{(x,y) \in S} \text{CRE}(x, y, c), \\
%    \text{subject to} &\sum_{(x,y) \in S} \text{ComputationalCost}(x, y) \leq U.
%\end{aligned}
%\end{equation}
%\end{theorem}
