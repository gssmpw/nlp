\section{Related Work}
\noindent\textbf{NLP Methods in Sustainability Analysis.}
To automate the extraction of insights from sustainability reports, popular NLP approaches include analyzing topic occurrence~\cite{ong2025esgsenticnet}, sentiment analysis~\cite{song2018sentimentanalysissustainable}, and RAG~\cite{esgrevealrag}. Yet, these traditional NLP approaches do not consider how greenwashing might distort extracted insights~\cite{ong2024explainable}. To promote the transparency of sustainability claims,~\citet{stammbach2022environmental} studied environmental claim detection, and~\citet{ni2023chatreport} assessed report conformance to TCFD guidelines. However, these works do not explicitly address building robust ESG analysis against greenwashing risks.

% , zhou2021topicdiscovery
% pasch2022sentimentclass

\noindent\textbf{Aspect-Based Sentiment Analysis}.
Aspect-Based Sentiment Analysis (ABSA) has become a popular NLP problem over the past decade~\cite{duuinc,ong2023finxabsa}. Yet, ABSA has focused on reviews and opinion mining~\cite{heemet}, despite calls for aspect-level analysis to be extended to other contexts~\cite{chebolu2023review}. Our study presents the first adaptation of aspect-level analysis to the aspect-action classification task. While we focus on sustainability, aspect-action analysis could potentially be adapted for other applications - i.e. government policy analysis~\cite{howlett2015policyanalysis}. 
% financial report analysis~\cite{liu2016fin10k}. 

\noindent \textbf{Cross-Generalization Studies}. Cross-domain generalization has become an increasingly important challenge in NLP~\cite{wang2022generalizingstudy}. However, there has been limited investigation on how models perform on \textit{unseen categories within the same domain} - i.e. cross-category generalization. Yet, this deserves attention because intra-domain data shifts are distinct from cross-domain shifts, and can significantly degrade model performance~\cite{subbaswamy2021intradomainshift}. Moreover, data annotation for unseen categories is expensive and impractical, as these categories often arise unexpectedly from evolving text content~\cite{bai2024bvsp}. 

% \noindent\textbf{Cross-Generalization Methods.}
% Popular methods for cross-domain generalization include learning domain-invariant features through adversarial training~\cite{yuan2024advcrossdomain}, or learning higher-level semantic features through contrastive learning~\cite{xie2022contrastivecrossdomain}. Yet, it remains unclear whether these techniques can be effective for cross-category generalization.

\noindent \textbf{LLMs Biases in Sustainability Analysis.}
Given LLMs' reasoning capabilities~\cite{wei2022emergent}, they have increasingly been utilized for tasks within sustainability analysis, such as knowledge base construction~\cite{ong2025esgsenticnet}, RAG~\cite{esgrevealrag}, among others. However, their performance on domain-specific tasks can be impacted by pre-training biases~\cite{dai2024llmbiassurvey, yang2024mitigating}, and the impact of these biases on sustainability analysis tasks remains under-explored.