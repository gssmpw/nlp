\section{Methodology}
\label{sec:methodology}

Next, we introduce our methodology for investigating SPF records, their errors and potential threats.
The goal of our study is threefold:
First, with a large-scale measurement, we want to determine the prevalence of SPF across a wide range of domains.
Second, we aim to shed light on how often and why SPF entries are flawed and thus only provide inadequate protection.
Finally, we want to assess the occurrence and impact of overly coarse authorizations in SPF configurations.





\subsection{Measuring SPF in the Wild}
\label{sec:measuring-in-the-wild}

From a technical perspective, we have two options to measure the configuration of \ac{SPF}: 
As the first strategy, we can collect a representative set of emails and extract all sender and recipient email addresses.
From these addresses, we could generate a list of domains of email providers and examine their \ac{SPF} records, similar to the study of \mbox{\citet{Durumeric_2015}}. 
Second, we can use a list of domains and retrieve all available \ac{SPF} records from them, even if they are not intended to ever be used to send emails.
While the first strategy helps to understand how \ac{SPF} is used relative to the distribution of email providers, the latter one provides a less biased view of \ac{SPF} configuration in the wild.
Consequently, we pursue this strategy for our large-scale study.

\paragraph{Data source.}
We use the Tranco list of domains~\citep{TrancoList} for our measurements.
This list is a research ranking of well-known and frequently used websites.
We use the full lists of the first of the months from January until May 2023 and merge them to get a bigger amount of domains.

\paragraph{Crawler.}
We develop a crawler for collecting and parsing SPF records using the \emph{checkdmarc library}\footnote{Available at \url{https://github.com/domainaware/checkdmarc}}.
The crawler retrieves the SPF record for a given domain using the function \spfcode{query\_spf\_record()}.
This function sends DNS requests of type TXT and SPF, but only returns the first SPF record from the type TXT request.
The record is then parsed using \spfcode{parse\_spf\_record()}.
To analyze different weaknesses, flaws and misconfigurations, we modify the library.
Our modified version returns all necessary values, such as the number of DNS lookups, permitted IP addresses and a parsed version of the SPF record.
Warnings and errors in the SPF syntax are reported, and our modified version continues with the parsing afterward.
Due to the scale of our study, we implement a cache to reduce the DNS load by not sending the same request twice.
If an SPF record already exists in the database, the cached object is used instead of requesting and analyzing it again.
This reduces the load from include mechanisms of large providers significant.
Only for the first domain the include mechanism is processed, all others hit the cache. 
Moreover, we distribute and rate limit the DNS requests across 150 servers.
The same procedure is applied for \ac{DMARC} using \spfcode{query\_dmarc\_record()} and \spfcode{parse\_dmarc\_record()}.
In the end, we collect the following information per domain:
\begin{itemize}
	\item SPF record 
	\item DMARC record
	\item MX record 
\end{itemize}
Note that this information is publicly available and therefore no confidential or private data is collected in our study, see also Appendix~\ref{app:ethics}.

We then analyze the collected records by checking for errors and misconfigurations.
Moreover, we evaluate the matching mechanisms of \ac{SPF} and investigate the resulting authorization policy.
For example, we determine the amount and type of authorized senders by recursively analyzing the \spfcode{include} mechanism.


\paragraph{Measurement focus.}
The main focus of our study is to understand the configuration of SPF entries and the role of authorized hosts in the underlying policies. 
By analyzing the collected data, we can examine these properties in detail. However, there are also limitations resulting from our study design:
First, we can analyze all SPF mechanisms except for \spfcode{exist}.
This can only be done with the first measurement strategy and a dataset of representative emails. 
Second, we restrict our study to IPv4 hosts. 
\citeauthor{Durumeric_2015} report that only 1.13\,\% of the mechanisms in SPF are \spfcode{ip6} terms. 
In our scan, we find an even lower adoption rate.
Only \IpSixDomainsPerc of the domains use IPv6 directly, which is why we refrain from a detailed analysis.






