\section{Introduction}

Email still represents the prime form of communication on the Internet today.
Despite several weaknesses of the protocol, billions of users regularly use email messages for business and personal exchange~\citep{emailUsersWorld}.
Due to its popularity, email is a constant magnet for cybercrime, serving as a vehicle for transporting unsolicited, fraudulent and malicious content, which ranges from spam and phishing attempts to targeted attacks and malware distribution~\mbox{\citep[e.g.,][]{ SimZanThoBur+20, KanKreLevEnr+08, 
HuWan18, GasUllStrRie+18}}.
These activities benefit from the lack of security mechanisms in the original protocols that cannot establish the authenticity of senders and content by itself.

Several extensions have been proposed over the last years to counter the misuse of email, including security mechanisms for the transport layer \citep{rfc3207, rfc2595}, email headers \citep{rfc6376}, and message data \citep{rfc8551, rfc3156}.
One of the oldest mechanisms to mitigate the spoofing of email senders is the \emph{Sender Policy Framework} (SPF)~\citep{rfc7208}.
Instead of retrieving emails from any network host, the receiving server can request an SPF record from the sender's domain and check whether the connecting IP address is authorized to send emails.
In concert with other mechanisms, such as DKIM~\citep{rfc6376} and DMARC~\citep{rfc7489}, SPF forms one central pillar for mitigating forged emails.

Despite this important role, however, the configuration of SPF in the wild and its weak spots are still an open field of research.
The study by \citet{Gojmerac2015} from 2014 indicates a moderate adoption of the mechanism and a tendency towards coarse authorization.
Further studies in the following years show an increasing number of domains using SPF.
In this work, we expand this view on SPF and present a detailed analysis of its configuration on the Internet.
In particular, we use the \mbox{Tranco list \citep{TrancoList}} to collect SPF records from 12 million domains over a period of 5 months.
Based on this collection, we analyze the adoption, validity and permissiveness of SPF policies to learn how servers use this mechanism.

Our study reveals a growing adoption of SPF in practice.
While \citeauthor{Wang2022} report in 2022 that 54.1\,\% of the domains contain valid SPF records, we observe an adoption of \SPFDomainsMPerc for the top 1 million and \SPFDomainsPerc for all 12 millions domains in our study.
Unfortunately, we also uncover persisting security issues: First, \ErrorDomainsPerc of the SPF records suffer from errors, undefined content, or ineffective rules, undermining the intended protection.
Second, we observe a large number of \emph{very} lax configurations.
For example, \HugeDomainsPerc of the domains allow emails to be sent from over \numprint{100000} IP addresses.
We demonstrate in a case study that these coarse configurations give rise to spoofing email senders and thus unnecessarily weaken the protection of SPF in the Internet.

To mitigate this situation, we investigate the reasons for the lax policies and derive guidelines for a more restrictive configuration. Moreover, we have launched a notification campaign for all SPF records with invalid policies. In total, we have contacted \numprint{\NotificationsCount} operators by email and informed them about incorrect or insufficient configurations. Feedback on these reports has been positive, and several operators promised to fix the reported problems.
Two weeks after our notification, a scan of the domains shows that \numprint{\SPFRescanResolvedErrors} (\NotificationSuccess) of the entries have already been corrected, and we expect further improvement over the next months. 

\smallskip
\textbf{Roadmap.} We review the background of SPF in \autoref{sec:background} and discuss related studies in \autoref{sec:previous_work}.
Afterward, we describe the methodology of our study in \autoref{sec:methodology}.
Our findings are presented in Sections \ref{sec:errors} and \ref{sec:includes}, where we first investigate invalid configurations and then explore the coarse use of SPF authorization.
Our guidelines are presented in \autoref{sec:recommendations}, before we conclude in \autoref{sec:conclusions}.
