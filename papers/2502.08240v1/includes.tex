\section{SPF includes and spoofing}
\label{sec:includes}

During our analysis of \ac{SPF} entries, we observe several entries that authorize a very large number of IP addresses. In the following, we analyze these lax configurations in detail, investigate the underlying reasons and outline attacks that become possible through such configurations.

\subsection{Number of Authorized IP Addresses}
Since the goal of \ac{SPF} is to authorize senders for a given domain, the number of allowed sending IP addresses should be minimal to reduce the attack surface.
While the actual number of sending hosts for a domain is generally unknown, we can use the number of receiving servers to get at least an intuition of the general magnitude.
\citet{Ruohonen2020} reports that domains listed in the Alexa rankings generally have fewer than 20 MX records.
Therefore, we conjecture that the scale of sending servers is not significantly larger.
However, we find that many domains in our study authorize \emph{orders of magnitude} more addresses to send emails.

\begin{figure}[htbp]
	\centerline{\input{fig/allowedip4_all.pgf}}
	\vspace{-6pt}
	\caption{CDF of authorized IPv4 addresses.}
	\label{fig:allowedip4}
\end{figure}

\autoref{fig:allowedip4} shows the distribution of the number of allowed IPv4 addresses as \ac{CDF}.
In line with our assumption, one out of three domains has fewer than 20 allowed hosts for sending emails.
By contrast, we also find that almost the same number of domains authorizes more than \numprint{100000} IPv4 addresses.
The largest rise in the \ac{CDF} is between \numprint{400000} and \numprint{700000} IPv4 addresses, mainly caused by including huge providers.
In general, there are two ways a huge number can arise in the SPF record.
First, we want to look at large IP ranges and later at includes. 

\subsection{Large IP Ranges}
\label{sec:huge_cidr}
In general, the reason for intentionally allowing large IP ranges is not clear to us.
To investigate, we take a look at SPF records that have more than \numprint{100000} IP addresses allowed.
Possible mechanisms are \spfcode{a}, \spfcode{mx} and \spfcode{ip4}.
We observed that \numprint{\HugeNoIncludeDomains} domains have their large number of IP addresses through these mechanisms.

\begin{table}[htbp]
	\centering
	\caption{Type and amount of SPF mechanisms with large IP ranges.}
	\label{tab:huge_cidr} \small
	\begin{tabular}{lrr}
\toprule
\textbf{CIDR} & \textbf{SPF:} 
\spfcode{ip4}, \spfcode{a}, \spfcode{mx} 
& \textbf{SPF:} \spfcode{include} \\
\midrule
/0 & \numprint{54} & \numprint{0} \\
/1 & \numprint{29} & \numprint{2} \\
/2 & \numprint{47} & \numprint{10} \\
/3 & \numprint{16} & \numprint{7} \\
/4 & \numprint{7} & \numprint{3} \\
/5 & \numprint{6} & \numprint{0} \\
/6 & \numprint{4} & \numprint{0} \\
/7 & \numprint{4} & \numprint{0} \\
/8 & \numprint{2162} & \numprint{110} \\
/9 & \numprint{23} & \numprint{3} \\
/10 & \numprint{131} & \numprint{27} \\
/11 & \numprint{44} & \numprint{50} \\
/12 & \numprint{313} & \numprint{137} \\
/13 & \numprint{228} & \numprint{210} \\
/14 & \numprint{1178} & \numprint{5419} \\
/15 & \numprint{1145} & \numprint{5389} \\
/16 & \numprint{11126} & \numprint{14243} \\
\bottomrule
\end{tabular}

\end{table}

In \autoref{tab:huge_cidr} we see an overview about the appearance of very large IP ranges.
At the hugest possible network \spfcode{/0} we found \numprint{\AllowInternet} domains explicit allowing \spfcode{0.0.0.0/0}, what looks intentional.
In contrast, there are \FPeval{\result}{round(\DomainsCidrZero-\AllowInternet,0)}\result{} domains that have a specific IPv4 address with a tailing \spfcode{/0}, what rather appears to be a misunderstanding of CIDR prefixes.
Going ahead, the huge includes \spfcode{/1} and \spfcode{/2} appear to be typos that should refer to \spfcode{/16} and \spfcode{/24} respectively.
Continuing the rows in \autoref{tab:huge_cidr} we can see that the includes are lower than the direct mechanisms, both at a very low level. 

Therefore, these few large IP ranges in SPF Records cannot explain the huge amount of allowed IP addresses we see in Figure 5 around $2^{19}$.
In the next section, we will have a detailed look at the \spfcode{include} mechanism and its impact on the number of allowed IP addresses.
This is more promising to be the reason, as we observe that \numprint{\HugeIncludeDomains} domains authorize a large number of IP addresses through the \spfcode{include} mechanism.






\subsection{Usage of Includes}
To get a better understanding of why so many IP addresses are included in the \ac{SPF} entries, we analyze the use of the \spfcode{include} mechanism. 
This mechanism is designed to cross administrative borders and is used by \IncludeDomainsPerc of the domains.
Providers often recommend to their customers that they add a specific \spfcode{include} mechanism to their \ac{SPF} record when using their services.
As already mentioned, \numprint{\HugeIncludeDomains} domains have a coarse policy with a huge number of allowed IP addresses from an include.

\paragraph{Trust relationships.}
In general, including a configuration from another domain involves a certain trust related risk.
The owner has no control over possible changes in the inherited addresses, which could lead to spoofed email addresses with valid \ac{SPF} check.
Therefore, the domain owner has to trust the party they include from.
While it should not be a big problem to trust the own provider in this case, things change if there are multiple inclusion levels and thus several administrators involved.
As shown in \autoref{fig:toplevel_includes_valid}, most configurations have not more than one include, which seems reasonable in the described scenario.
Nevertheless, we also observed 10 recursive includes and more, which raises the question if the domain owners are aware of everything they include and trust all involved parties.

\begin{figure}[htbp]
	\centerline{\input{fig/toplevel_includes_all.pgf}}
	\vspace{-6pt}
	\caption{Number of includes in the top level record.}
	\label{fig:toplevel_includes_valid}
\end{figure}

\paragraph{Included network size.}
Another interesting part of the includes are the networks allowed by them.
\autoref{fig:subnet_size_valid} shows the distribution of the used network sizes coming from the included \ac{SPF} records.
While most entries only include one IP address (/32 network), there is also a second notable peak for /24 networks.
In the context of larger providers, load balancing and scaling, these sizes are understandable to share the load between multiple servers.
Surprisingly, there are also \ac{SPF} entries, which allow very huge networks, larger than /16.
Even though large providers like Google might need a large number of servers sending mail for their customers, there are obviously limits.
We could not find a specific reason for these includes.
Especially for less common domains, at the end of the Tranco list, we observe many /8 inclusions.
Malicious senders could use these domains to send emails from many hosts within this list.
What is interesting here is that most of the domains we observe in this context come from the ".top" top-level domain.  

\begin{figure}[htbp]
	\centerline{\input{fig/subnet_size_all.pgf}}
	\vspace{-6pt}
	\caption{Distribution of subnet sizes in includes.}
	\label{fig:subnet_size_valid}
\end{figure}

\paragraph{Number of IP addresses per include.}
We now take a closer look at how many times an \spfcode{include} is used, depending on the number of allowed IP addresses.
\autoref{fig:include_usage_all} is a heatmap that shows the density of includes within a pixel of the plot, representing a logarithmic scale of the number of allowed IPs and how often the include is used.
We can see that there is a huge concentration, up to around $2^{20}$ allowed IPs.
Recalling \autoref{fig:allowedip4}, we see that this correlates with the steep rise there at the same number of domains. From this observation, we can conclude that the large numbers of allowed IPs are typically from includes.

\begin{figure}[b]
	\centerline{\input{include_usage_all.pgf}}
	\vspace{-6pt}
	\caption{Heatmap of domains that are using a specific include, depending on the allowed IPs for the include.}
	\label{fig:include_usage_all}
\end{figure}

In \autoref{tab:top20_includes}, we report the top 20 includes we discover in our scan.
As the first two, namely Microsoft and Google, include a huge amount of IP addresses. Similarly, other includes of providers are larger than \numprint{100000} and likely too coarse for proper protection. In contrast, there are also providers, like OVH\footnote{\url{https://www.ovhcloud.com}} and Xserver\footnote{\url{https://www.xserver.ne.jp/}}, which include only a few sending email servers, demonstrating a restrictive authorization policy. 

\begin{table}[htbp]
\begin{threeparttable}
	\centering
	\caption{Top 20 included domains with their number of allowed IPs.}
	\label{tab:top20_includes} \small
	\begin{tabular}{lrr}
\toprule
\textbf{Include} & \textbf{Used by} & \textbf{Allowed IPs} \\
\midrule
\detokenize{spf.protection.outlook.com} & \numprint{2456916} & \numprint{491520} \\
\detokenize{_spf.google.com} & \numprint{1418705} & \numprint{328960} \\
\detokenize{websitewelcome.com} & \numprint{414695} & \numprint{1088784} \\
\detokenize{secureserver.net} & \numprint{374986} & \numprint{505104} \\
\detokenize{relay.mailchannels.net} & \numprint{289112} & \numprint{4358} \\
\detokenize{servers.mcsv.net} & \numprint{263343} & \numprint{22528} \\
\detokenize{spf.mandrillapp.com} & \numprint{236293} & \numprint{4608} \\
\detokenize{sendgrid.net} & \numprint{215497} & \numprint{220672} \\
\detokenize{_spf.mailspamprotection.com} & \numprint{212418} & \numprint{1049} \\
\detokenize{spf.efwd.registrar-servers.com} & \numprint{196465} & \numprint{264} \\
\detokenize{amazonses.com} & \numprint{183184} & \numprint{64512} \\
\detokenize{mx.ovh.com}\tnote{1} & \numprint{176191} & \numprint{2} \\
\detokenize{mailgun.org} & \numprint{172499} & \numprint{36312} \\
\detokenize{_spf.mail.hostinger.com} & \numprint{139423} & \numprint{4358} \\
\detokenize{zoho.com} & \numprint{138227} & \numprint{6209} \\
\detokenize{mail.zendesk.com} & \numprint{114026} & \numprint{26112} \\
\detokenize{spf.mailjet.com} & \numprint{111760} & \numprint{5120} \\
\detokenize{spf.web-hosting.com} & \numprint{111405} & \numprint{10492} \\
\detokenize{spf.sendinblue.com} & \numprint{102004} & \numprint{87040} \\
\detokenize{spf.sender.xserver.jp} & \numprint{92411} & \numprint{15} \\
\bottomrule
\end{tabular}

    \begin{tablenotes}
        \item[1] Uses not recommended PTR mechanism
    \end{tablenotes}
\end{threeparttable}
\end{table}


\subsection{Case Study on Web Hosting}

Our analysis shows that coarse authorization is a common practice in SPF configurations. 
From a theoretical point of view, it is obvious that overly permissive policies weaken the intended protection. 
However, whether these loose configurations can really help attackers in practice is not immediately clear.
We therefore set out to investigate the risk of this practice in a case study on web hosting providers.

In particular, we focus on common web hosting providers that usually offer web space along with email support.
Since these providers manage thousands of domains for their customers, they represent an essential part of the Tranco list and the collected SPF records.
Moreover, as most web hosting providers support active content, such as PHP scripts, we are able to test the sending and authorization of SPF with permissive configurations.

As the basis for the case study, we search for recommended providers worldwide using the review website \emph{hostings.info}\footnote{\url{https://hostings.info/}}.
From each country in the overview, we search for a recommended SPF record at the top 10 recommended web hosters.
Due to the size of our measurements, we found 79 providers.
Since many of the providers require national residency or the purchase of a national domain, we rent web space from 5 providers that do not impose any constraints.
We especially choose providers that offer PHP support, allow the use of own domains, and provide a short contract period.
These providers are located in 4 countries (2xDE, FR, US, UK).


\paragraph{Imitating spoofing}
To investigate the risk of spoofing due to lax SPF configuration, we sent emails from the selected providers to ourselves.
In these emails, we spoof the sending domain by picking one that authorizes the IP address of the web hosting provider due to the recommended SPF record.
Technically, we use two methods to realize this strategy: First, we try to send an email directly via SMTP from the web space via a corresponding PHP script.
Second, we use the PHP function \spfcode{mail()} to send it via the local \ac{MTA} of the provider.
We then examine how the emails are received on our site and whether they pass the SPF checks.
A spoofing attempt is considered successful if one of the two methods succeeds in transferring a valid email.

Note that we only send emails from our rented web space to our own email addresses.
Thus, spoofed senders in these mails do not cause any harm.
Also, we have informed the vulnerable web hosting providers about their lax configurations in the hope that they will enforce stricter policies.
For more details on ethical considerations, see \autoref{app:ethics}.

\paragraph{Results}
We find that 4 of the 5 web hosting providers enable us to send emails with spoofed senders due to overly coarse authorization, as we can see in \autoref{tab:provider_results}.
In particular, there are two providers that enable sending emails with the PHP \spfcode{mail()} function over their \ac{MTA}.
Among these two providers, we find \numprint{264} and \numprint{24959} authorized domains, respectively.
In addition, there is one provider that allows us to send emails via SMTP for \numprint{159} affected domains.
With the fourth provider, we can even send emails via the SMTP method and \spfcode{mail()} on behalf of \numprint{713} domains. 


In summary, we are able to send emails with valid SPF entries from \numprint{26095} domains, simply by renting web hosting space for about 30 Euro. Even worse, we can pick from a wide range of domains for the spoofing, including lobby organizations, political parties, health insurances companies, and even banks. Although our case study focuses on a small group of web hosting providers, it shows the potential for phishing campaigns and spam when lax SPF configurations are exploited by attackers.
\begin{table}[htbp]
	\centering
	\caption{Results of the providers case study.}
	\label{tab:provider_results}
\begin{tabular}{@{}ccrr@{}}
\toprule
\textbf{Provider} & \textbf{Success} & \textbf{\# Domains} & \textbf{\# Allowed IPs} \\ \midrule
1                 & MTA              & \numprint{24959}   & \numprint{177168}      \\
2                 & SMTP, MTA        & \numprint{713}     & \numprint{514}         \\
3                 & MTA              & \numprint{264}     & \numprint{2052}        \\
4                 & SMTP             & \numprint{159}     & \numprint{3074}        \\
5                 & None             & \numprint{0}       & \numprint{672}         \\ \bottomrule
\end{tabular}
\end{table}
















