\section{SPF Adoption and Errors}
\label{sec:errors}

We begin our examination of the collected data by first analyzing the adoption of SPF in the wild and comparing it to previous work. We then proceed with a detailed analysis of the uncovered errors and misconfigurations, expanding the scope of previous studies.


\subsection{SPF Usage}

In total, we have scanned \numprint{\scannedDomains} domains for this study.
While this expanded scan provides insights on the general configuration of SPF, it is not directly comparable with previous studies that have considered smaller sets from the top 1 million domains of the Alexa and other rankings.
However, in our measurement, the result for the top 1 million domains is included. Thereby, it is comparable in terms of size and the fact that the domains are ranked.
Therefore, we first focus on the top 1 million domains of the \mbox{Tranco list\footnote{Available at \url{https://tranco-list.eu/list/K2XYW}.} \citep{TrancoList}} generated on 01 May 2023. %

Using this focus, we observe that the usage of SPF per domain has grown to \SPFDomainsMPerc of all scanned domains and \SPFDomainsMXPerc for domains with MX record.
A detailed comparison of our results with past measurements is shown in \autoref{tab:usage}.
We find that domains within the first 1 million use SPF and DMARC more frequently.
But also for the complete 12 million domains, a clear increase in SPF usage to \SPFDomainsPerc can be observed from our scan. Every second domain in our measurement is now employing this security mechanism.
Additionally,  \autoref{fig:domains_overview} provides an overview of all scanned domains and their adoption of SPF and DMARC.

\begin{table}[tbp]
\begin{threeparttable}
	\centering \small
	\caption{SPF and DMARC usage in the wild.}
	\label{tab:usage}
	\begin{tabular}{lclrrr}
		\toprule
		\textbf{Study}            & \textbf{Year} & \textbf{List} & \textbf{Size} & \textbf{SPF}     & \textbf{DM.}           \\
		\midrule
        \citet{Gojmerac2015}    & 2014          & Alexa         & 1M               & 36.7\,\%        & 0.5\,\%             \\
		\citet{Foster_2015}     & 2015          & Alexa         & 1M               & 42.2\,\%        & 1.0\,\%             \\
		\citet{Foster_2015}     & 2015          & Adobe         & 1M               & 43.6\,\%        & 0.9\,\%             \\
		\citet{Durumeric_2015}\tnote{1}  & 2015          & Alexa         & 1M               & 47.0\,\%           & 1.1\,\%              \\
        \citet{HuWan18}          & 2018          & Alexa         & 1M               & 49.2\,\%         & 5.1\,\%              \\
        \citet{Kahraman2020}    & 2020          & Alexa          & 1M & 73.6\,\% & --- \\
        \citet{Wang2022} & 2022 & Alexa & 1M & 54.1\,\% & 11.9\,\% \\
		Our study                     & 2023          & Tranco        & 1M               & \SPFDomainsMPerc & \DMARCAllDomainsMPerc \\
        \midrule
        \citet{Tatang2021}      & 2020          & Other\tnote{2} & 2M & 50.7\,\%  & 11.5\,\% \\
        \citet{Kahraman2020}    & 2020          & None          & 168M & 25.0\,\% & --- \\
		Our study                     & 2023          & Tranco        & 12M              & \SPFDomainsPerc  & \DMARCAllDomainsPerc  \\
		\bottomrule
	\end{tabular}
    \begin{tablenotes}
        \item[1] Only domains with MX record are considered in the evaluation
        \item[2] Union of Alexa, Majestic and Tranco top 1M lists
    \end{tablenotes}
 \end{threeparttable}
\end{table}



\begin{figure}[b]
	\centerline{\input{fig/domains_overview.pgf}}
	\vspace{-6pt}
	\caption{Implementation of email and security mechanisms and their overlaps.}
	\label{fig:domains_overview}
\end{figure}


We also observe an interesting phenomenon: \NoMXSPFDomainsPerc of the domains without an MX record return an \ac{SPF} record. At first glance, this may seem counterintuitive, since these domains specify which senders are authorized through SPF but cannot receive email themselves. In several cases, these domains are not intended to send or receive email, and so the SPF record is used to deny sending email in general. We find that \NoMXSPFrestrictallDomainsPerc of the domains without an MX but an SPF record have SPF configurations containing \spfcode{v=spf1 -all} (\numprint{\NoMXSPFdenyallDomains}) or \spfcode{v=spf1 \softfail all} (\numprint{\NoMXSPFsoftfailallDomains}). However, the remaining half of these domains are likely misconfigured because they specify a sending policy but cannot receive bounces or other error messages from the transport, making them unsuitable for reliable email communication.




\subsection{DMARC}
In addition to SPF, we have also scanned for \ac{DMARC} records using the \emph{checkdmarc library}.
We have done this to measure the increment from previous studies on email security.
As shown in \autoref{tab:usage}, \acp{DMARC} started with a low value of about 1\,\% in 2015 and is now at \DMARCAllDomainsMPerc for the top 1 million domains and \DMARCAllDomainsPerc for all domains. 
The increasing usage of \ac{DMARC} is likely due to the recommendations of large email providers\footnote{\url{https://support.google.com/a/answer/2466580?hl=en}}.
\citeauthor{Durumeric_2015} already mentioned, that major email providers, such as Google and Microsoft, heavily skew the apparent adoption of security mechanisms.


\subsection{SPF Errors}

In our analysis of SPF, we observe a variety of errors in \ErrorDomainsPerc (\numprint{\ErrorDomains}) of the domains, some of which are trivial typos while others are rather subtle misconfigurations.
We hence explore these errors in more detail and count all issues as errors that affect the correct functionality of SPF.
This includes all records that might result in a \spfcode{permerror}.
\autoref{fig:spf_errors} provides a general overview of all types of errors found.

\begin{figure}[htbp]
	\centerline{\input{fig/spf_errors.pgf}}
	\vspace{-6pt}
	\caption{Appearance of different error types.}
	\label{fig:spf_errors}
\end{figure}

Note that during our scan, we received \numprint{\DNSRecordNotFound} DNS errors.
This means that a domain that was supposed to be resolved when parsing the SPF record was not resolvable at that time. Since this may change on subsequent scans, we exclude these errors from the following analysis.




\paragraph{Record not found}
First, we consider record-not-found errors indicating that no SPF record was found for a given domain name. These errors are the most common in our study with \SPFRecordNotFoundPerc.
They can be caused by either the \spfcode{include} mechanism or the \spfcode{redirect} modifier, since the SPF record of another domain must be parsed.

If we look at this error in detail, we see in \autoref{fig:record_errors} that there are different causes.
The most common cause with \RecordNotFoundSpfMissingPerc (\numprint{\RecordNotFoundSpfMissing}) of this error type is, that the requested domains has no SPF record. 
In contrast, there are \RecordNotFoundSpfMultiplePerc (\numprint{\RecordNotFoundSpfMultiple}) that provide more than one SPF record, which is no valid SPF record by the specification.
An interesting finding here is that \RecordNotFoundSpfMultipleCafeComPerc (\numprint{\RecordNotFoundSpfMultipleCafeCom}) of these errors are due to an \spfcode{include} of the provider \textit{cafe24.com}, which is a hosting provider for business customers.
Another common record-not-found error is that the requested domain is not found (NXDOMAIN), as it happens \numprint{\RecordNotFoundNotExisting} (\RecordNotFoundNotExistingPerc) times.
This error could become critical if the domain is not registered and is taken over by an attacker.
Other DNS related errors like a timeout, what is a \spfcode{temperror} or an empty result are less common.
The three other errors are one each of a DNS label is > 63 octets long, a DNS name is > 255 octets long, and one utf-8 decode error. 

\begin{figure}[htbp]
	\centerline{\input{fig/record_errors.pgf}}
	\vspace{-6pt}
	\caption{Distribution of record-not-found errors.}
	\label{fig:record_errors}
\end{figure}


\paragraph{Too many DNS lookups}
To prevent denial-of-service attacks, the number of DNS lookups that an SPF record may trigger is limited to 10 requests.
It is the second commonest error with \SPFTooManyDNSLookupsPerc.
As the specification is not totally clear here, we need to discuss this error in detail.
RFC7208 says:
\begin{quote}\color{cbone}
The following terms cause DNS queries: the "include", "a", "mx", "ptr", and "exists" mechanisms, and the "redirect" modifier.
SPF implementations MUST limit the total number of those terms to 10 during SPF evaluation, to avoid unreasonable load on the DNS.
\end{quote}

The problem here is that for the \spfcode{include} mechanism, there is no further description of how recursive DNS requests should be handled.
As for \spfcode{mx} and \spfcode{ptr} mechanisms they are within the overall limit of 10, we assume this holds for the \spfcode{include} mechanism too.
In the \emph{checkdmarc} library, this is implemented by counting the mechanism-related lookups during recursion.
Another important fact is that this error does not have to lead directly to a \spfcode{permerror} in the SPF check. The SPF check can be successful if a result is returned within the first 10 lookups.

Now that we have discussed this type of error, let us get back to the underlying causes.
As there is only a limited set of mechanisms that could create these errors, we find that the \spfcode{include} mechanism is the main cause of this issue. 
Reasons for this include but are not limited to recommendations by email or web hosting providers.
As an example, \textit{bluehost.com}\footnote{\url{https://www.bluehost.com/}} is a provider that recommends customers to add an invalid SPF record that causes 14 DNS lookups.
In \autoref{fig:include_lookups_zoom} we see a scatter plot of the includes where each dot represents an include.
We zoom into the interesting part with more than 10 includes.
In total, there are \numprint{\IncludesGeTen} included SPF records exceeding the DNS lookup limit directly, affecting \numprint{\IncludesGeTenDomains} domains.
\numprint{\IncludesBluehostCom} (\IncludesBluehostComPerc) of them are from \textit{bluehost.com}.

\begin{figure}[htbp]
	\centerline{\input{include_lookups_zoom.pgf}}
	\vspace{-6pt}
	\caption{Cutout of the number of domains using a specific include depending on the DNS lookup count.}
	\label{fig:include_lookups_zoom}
\end{figure}

\paragraph{Too many void DNS lookups}
This error is raised when there are two DNS errors during the evaluation of the SPF record.
DNS errors are empty results or NXDOMAIN.
With \SPFTooManyDNSLookupsPerc this error is less common.
Since this error refers to a DNS lookup limit like \emph{Too many DNS lookups} and the reasons for exceeding them are almost the same, we will refrain from explaining them in detail again.

\paragraph{Syntax Errors}
A more interesting group of errors are syntax errors.
These are caused by different oversights and shortcomings when creating an SPF record.
Common errors are typos, wrong mechanism names or concatenating up different DNS records.
With \SPFSyntaxErrorPerc it is the third-largest error group and the most diverse one in our study.

In our manual investigation, the first common mistake in this group is that mechanisms are misspelled.
We find that \SyntaxErrorIPvFourPerc (\numprint{\SyntaxErrorIPvFour}) of the syntax errors are using \spfcode{ipv4} instead of \spfcode{ip4} and \SyntaxErrorIPvSixPerc (\numprint{\SyntaxErrorIPvSix}) are using \spfcode{ipv6} instead of \spfcode{ip6}.
\SyntaxErrorIPPerc (\numprint{\SyntaxErrorIP}) are just using \spfcode{ip} as the wrong mechanism. 
Similarly, merging DNS entries also leads to errors.
We observe that \SyntaxErrorSiteVerificationPerc (\numprint{\SyntaxErrorSiteVerification}) of the errors are concatenations of the SPF record and a site verification string.
When measuring the appearances of \spfcode{v=spf1} in the SPF records, the result is that \SyntaxErrorMultipleSpfPerc (\numprint{\SyntaxErrorMultipleSpf}) of the records with invalid syntax contain more than one, which could be caused by combining multiple recommendations.
A mechanism can have an argument that is placed directly after a \spfcode{:}, however a whitespace in this position is causing \SyntaxErrorWhitespacePerc (\numprint{\SyntaxErrorWhitespace}) of the errors.
Even though this group of errors is more diverse, they are typically easier to fix than other errors in SPF entries.

\paragraph{Include loops}
An include loop is created when an \spfcode{include} mechanism refers back to itself, either directly or at a deeper level of recursion.
It is a less common mistake with \SPFIncludeLoopPerc.
A direct inclusion of the domain happens in \SPFIncludeLoopOnePerc (\numprint{\SPFIncludeLoopOne}) of the cases.
We assume that knowledge about SPF is not correct in these cases.
When the error occurs at lower recursion levels, it is not obvious to detect and the cause is more intelligible.

\paragraph{Redirect loops}
Loops can also occur with \spfcode{redirect} mechanisms, representing \SPFRedirectLoopPerc of the errors.
The causes are similar to the \emph{include loops}.

\paragraph{Invalid IP address}
Because IP addresses have a well-defined representation, they can be easily written incorrectly. In our analysis,  this issue causes \SPFInvalidIPPerc of all errors. In particular, we observe the following four types of errors:%
\begin{itemize}
	\item No IP at all
	\item Wrong number of octets
	\item A domain instead of IP address
	\item Wrong IP version
\end{itemize}

\smallskip
\noindent %
Overall, our analysis of the errors shows that one of their main causes is insufficient attention to detail when creating SPF entries. Although SPF is a simple mechanism, the development of configurations is non-trivial and sometimes fraught with small details. For example, DNS lookup limits are challenging to inspect, inclusions and redirections can cause different loops, and the syntax of some SPF mechanisms must also be carefully considered.

\subsection{Notification}

Our detailed analysis of errors puts us in a unique position: We become able to run a notification campaign, informing domain operators about the discovered problems in their SPF configurations. To this end, we follow the recommendations developed by \citet{Stock2016, Stock2018} for large-scale notification and contact each operator via email. To reach as many operators as possible, we use the general addresses \spfcode{postmaster@} and \spfcode{security@} as defined in RFC2142 \cite{rfc2142} for our campaign. In addition, we send an email to the contact named in \spfcode{security.txt}~\citep{rfc9116}, if available.

\paragraph{Sending out notifications.}
In total, we sent \numprint{\NotificationsCount} mails to notify domain operators with erroneous SPF records, except for record-not-found errors.
We used a dedicated email server to deliver these huge amounts of emails.
To avoid being blacklisted, we throttled the transfer rate to 1 mail per second. Based on this limit, we sent out all notifications in the second week of May 2023.

For each notification email, we follow a fixed template: First, we introduce ourselves and the scope of our study. Then, we list the identified problems for the particular domain, along with examples and recommendations on how to fix them. We are aware that our campaign causes additional work for the operators, and therefore strive to provide actionable items for each error type. Further details on ethical considerations arising from this notification campaign are discussed in Appendix~\ref{app:ethics}.

\paragraph{Returned emails and feedback.}
It is clear that a notification campaign targeting hundreds of thousands of domains results in a large number of bounces and error messages. Nevertheless, we obtained a notable amount of positive feedback with thank-you notes, further questions and recommendations for future activities. By the time of the paper submission, we had received \numprint{\NotificationAnswersThx} grateful emails from domain operators. Only 3 responses were negative, and considered our notifications to be spam. We added the respective domains to an opt-out list so that they would not receive further security notifications from us.


\paragraph{Impact of notification}

To learn about the practical impact of our notification campaign, we rescanned the domains with errors on May 24, 2023, that is, two weeks after the notification.
We observe that \numprint{\SPFRescanResolvedErrors} errors have been fixed by that time. In the same period \numprint{\RescanEmptyDomains} of the domains with errors disappeared, so there are no errors anymore.
\autoref{tab:notification_success} shows detailed results for the different errors. The highest success rate is achieved with syntax errors and invalid IP addresses, as these can be easily fixed and do not require a deep understanding of SPF record evaluation. The errors with the lowest success rate are those related to DNS lookup limits. We assume that these are often non-trivial to fix, as they depend on the inclusion of external providers in the respective SPF configurations.

\begin{table}[htbp]
\begin{threeparttable}
	\centering
	\caption{SPF errors before and after our notification. }
	\label{tab:notification_success} \small
	\begin{tabular}{lrrr}
\toprule
\textbf{Error} & \textbf{Before} & \textbf{After} & \textbf{Change} \\
\midrule
Syntax Error & \numprint{38296} & \numprint{36103} & -5.73~\% \\
Too Many DNS Lookups & \numprint{49421} & \numprint{48630} & -1.60~\% \\
Too Many Void DNS Lookups & \numprint{5308} & \numprint{5127} & -3.41~\% \\
Redirect Loop & \numprint{58} & \numprint{56} & -3.45~\% \\
Include Loop & \numprint{19356} & \numprint{18617} & -3.82~\% \\
Invalid IP address & \numprint{7882} & \numprint{7498} & -4.87~\% \\
\midrule
Total Errors & \numprint{211018} & \numprint{204087} & -3.28~\% \\
\bottomrule
\end{tabular}

 \end{threeparttable}
\end{table}

Overall, our campaign achieves similar performance to notifications performed in previous work.
\citet{Stock2016}, for example,  report a 4.1\% success rate in reporting web vulnerabilities via email. Our campaign achieves a success rate of \NotificationSuccess just two weeks after sending the notifications, thus providing a similar effectiveness. 








\subsection{Additional Findings}

We conclude our examination of SPF configurations with a discussion of further and curious findings discovered during the processing of our dataset.

\paragraph{Permissive all policies}
For \NoDenyAllPerc (\numprint{\NoDenyAll}) of the domains, the SPF configuration is missing a restrictive all policy, which harms the effectiveness of SPF.
The SPF evaluation will then just end without a matching mechanism and therefore return a \spfcode{neutral} result.
This may be intentional, as we will present in \autoref{sec:huge_cidr} for a few domains, but it leads to a reduction in protection.
In most cases, we notice that a final deny directive is missing, such as \spfcode{-all}.
Here, we often spot typos as the reason for the problem, such as the invalid terms \spfcode{-al} or \spfcode{-all;} in the SPF entries.





\paragraph{Not recommended records}
Over time, the SPF extension has evolved from the experimental RFC4408~\cite{rfc4408} to a proposed standard in RFC7208~\cite{rfc7208}.
Due to this evolution, the DNS record type \spfcode{SPF} has been deprecated since 2014.
The \spfcode{PTR} mechanism is not recommended anymore, as it is slow, not reliable due to DNS errors and produces a high DNS load. 
In our dataset, we find \numprint{\DeprecatedSPFDomains} domains still using this DNS record type and \numprint{\DeprecatedPTRDomains} domains using the \spfcode{PTR} mechanism. As these versions still provide protection, we do not count them as errors in our analysis.


\paragraph{Implementation of abuse reporting}
With RFC6652~\citep{rfc6652}, SPF was extended in 2012 with three new modifiers: \spfcode{ra}, \spfcode{rp} and \spfcode{rr}. These modifiers allow the operator of a domain to be notified when an unauthorized email is rejected at an email server. Although this is a helpful extension, we notice only \numprint{\ReportMechanism} domains implementing it in our dataset.

\paragraph{XSS attacks over SPF}
Finally, we observe a cross-site scripting attack packaged in an SPF record of a domain. The attack looks as follows:
\begin{center}
\color{cbone}
\begin{verbatim}
 v=spf1 xss=<script>alert('SPF')</script> ~all
\end{verbatim}
\end{center}
Since SPF parsers in email servers generally do not interpret JavaScript code, they should not be vulnerable to this type of attack.
However, as soon as software displays SPF records in a web browser, there is a risk that the attack will succeed.
This is, for example, the case for web services that check and validate SPF entries. Given the harmless payload of the attack, however, we assume that it is meant for testing purposes.
 
