\section{Previous Studies}
\label{sec:previous_work}
Since email security is an important topic, we are not the first to measure the prevalence of sender authentication mechanisms.
Over the last decade, the adoption of these techniques and possible vulnerabilities have been studied several times.
In the following, we briefly review this work.

\medskip
\emph{Studies from 2014 to 2018.}
In 2014, ____ scanned the top 1 million of the Alexa ranking for DNS entries such as SPF and DMARC.
Only about 37\,\% of the domains provided an SPF configuration at that time.
In addition, ____ found several common syntactic errors in SPF records in their study, such as missing values for matching mechanisms like \spfcode{ip4}, but did not quantify them further.
In 2015, ____ gave a broad overview of the adoption rates of security extensions for \ac{SMTP}.
Besides protocols like STARTTLS and \ac{DKIM}, they also investigated \ac{SPF} for the Alexa top 1 million list but ignored sites without an MX record.
They found that 47\,\% of the domains had published an SPF policy, indicating a growing adoption.

In the same year, ____ evaluated the security provided against network attacks by such extensions from a theoretical and practical view.
As part of their study, they also scanned the Alexa list and, additionally, the top million mail domains from a leaked set of user data from Adobe.
The result here was that 42.26\,\% of the Alexa domains and 43.60\,\% of the Adobe domains were using \ac{SPF}.
A few years later, ____ investigated how email providers handle spoofed emails and if such could reach the inbox of the users.
In this context, they searched the top 1 million domains from the Alexa ranking for \ac{SPF} records and reported a slightly increased adoption rate of 44.9\,\%. 

\medskip
\emph{Studies from 2020 to 2023.}
____ measured the adoption rate of SPF, DKIM and DMARC and analyzed the relationship between different domains through included SPF entries as well as the domains and the autonomous systems belonging to the allowed IP addresses.
Therefore, they scanned in 2020 over 2 million domains from different top lists, of which 50.7\,\% had published SPF records.
Moreover, they reported 13\,\% invalid entries and, as the most common error, too many DNS lookups. %
The authors also mentioned that many records used different includes, and that sometimes large IP subnets are trusted.

In the same year, \mbox{____} analyzed the usage of \ac{SPF} on a dataset of about 168 million domains. In this very large dataset, 25\,\% of the domains had \ac{SPF} configured and were further analyzed in terms of the used mechanisms and syntactic as well as DNS lookup limit errors.
____ crawled, also in 2020, about 8.3 million domains from different top lists for SPF records to analyze trust relationships.
The analysis showed that some domains allowed very large networks to send emails on behalf of them, raising concerns about possible attack vectors.
The measurements in these three papers are close to ours, yet we provide a detailed analysis of the \ac{SPF} records themselves, which allows us to characterize the security risks and notify the affected operators.

Two years later, ____ measured the deployment of \ac{DKIM} and issues of the management.
In their work, they also reported an \ac{SPF} adoption rate of 54,1\,\% in the Alexa top 1 million domains.
This result continued the trend from previous work, that has shown an increasing number of domains with such a policy.



\paragraph{Attacks on SPF}
____ analyzed how email servers process and validate \ac{SPF} entries. They observed that several servers ignore syntax errors and ambiguities of the specification, which could lead to various forms of attacks.
In a similar vein, ____ investigated several security protocols, including SPF and developed attacks for the authentication by systematically exploiting details in the standards that are often implemented inconsistently.
They proposed more accurate protocol descriptions to eliminate the ambiguous definitions, which in the end could also decrease the number of errors in DNS records.
Another attack vector was described by ____:
In their work, they investigated different types of email forwarding and how these change header fields.
In the end, the implementation of some forwarding techniques enabled the authors to circumvent methods like \ac{SPF} and to send spoofed emails without detection.

Finally, the implementation of sender validation libraries itself can be a point of attack.
____ demonstrated this using \textit{libSPF2} as an example.
They found multiple bugs in it and developed a technique to detect vulnerable servers remotely, revealing the widespread use of this library version.

\paragraph{Difference to our study.}
Our study continues the line of previous research and extends it with additional perspectives:
We base our study on a larger dataset than most previous studies, except for the work by ____.
This gives us a broader picture of the use of \ac{SPF} in the wild.
As a result, we are able to perform a detailed analysis of the flaws and weaknesses in SPF configurations, showing where and why authentication fails.
This combination of a large dataset and detailed analysis provides valuable insights into common problems when applying the SPF framework.
Moreover, we conduct a case study demonstrating that overly coarse authorization policies weaken the security mechanism and make it easier to forge emails with spoofed senders. 

