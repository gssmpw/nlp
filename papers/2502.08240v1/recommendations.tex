\section{Lesson Learned}
\label{sec:recommendations}

After examining the prevalence of flaws in SPF implementations, we work out some recommendations on how to improve the use of SPF. We first discuss possible actions for domain owners before we take a look at the side of the web hosting providers.

\subsection{Domain Owners}
\label{sec:recommendations_domain}
Domain owners are often dependent on other parties to operate particular services.
This especially applies to email servers, where an external provider is often responsible for the security of the service.
Nevertheless, in most cases, the domain owner needs to take care of all DNS records and therefore must provide the correct SPF configuration.

If the email server is operated by another provider, the domain owner should, in general, follow their recommendations for \ac{SPF} records.
As they might change IP addresses from time to time, this is usually a scenario for which the \spfcode{include} mechanism is intended.
However, we recommend checking the included addresses.
As can be seen from our analysis, often only a single inclusion is necessary, making such a test technically feasible.

On the other hand, one of our findings is that providers may recommend including large IP ranges or additional includes in order to use their service.
In these cases, we strongly recommend to check if the included ranges are only email servers used for the domain, potentially by contacting the provider and requesting a description of the includes.
A further risk is an \spfcode{a} mechanism in the SPF record of a shared web space.
Every user on the same server that the A record points to could use this server to send an email on behalf of this domain.
Ultimately, the recommended SPF entry can be taken as a rough indicator of whether a provider takes email security seriously and therefore serves as a decision-making aid for choosing a provider.

If the domain owners manage their DNS records themselves, they are fully responsible for the content.
Our work shows that there are many entries with syntax errors, which could easily be prevented in advance.
We therefore recommend validating SPF records with a tool to check for errors and undefined parts.
In case of a self-hosted email infrastructure, administrators should ensure that they only add the hosts needed to send emails.


\subsection{Web Hosting Providers}
\label{sec:recommendations_provider}

Web hosting providers generally want to offer their customers a well-functioning and user-friendly service, yet this sometimes conflicts with providing the best possible security.
A well-designed setup can help avoid difficulties.

Customers should usually not be able to open \ac{SMTP} connections directly to email servers on a shared web space, especially if this host is included in the recommended \ac{SPF} entry.
Therefore, it is a recommended practice to block outgoing connections to port 25 and related services.
Nevertheless, users should be able to send emails within their application.
Therefore, a local \acp{MTA} with proper authentication should be used to verify that the authenticated account is allowed to send emails on behalf of the specified domain.

If a customer needs to send emails directly using their own \ac{MTA}, this should not be done using shared IP addresses.
We recommend providing a user documentation to manually add an IP address to the entry, instead of automatically including it in the default \ac{SPF} record.
To prevent further problems with \ac{SPF}, providers should enable their customers to understand and properly use this framework.
Therefore, they should explain how it works or link to relevant material, and also point out potential risks associated with setting certain SPF mechanisms. 
