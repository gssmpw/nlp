
\section{Background}
\label{sec:background}

The sending and receiving of email is realized on top of the classic \ac{SMTP}~\citep{rfc821}.
Standardized in 1982, this protocol has been designed without built-in mechanisms to ensure the confidentiality of transmitted messages or to verify the authenticity of senders.
As the importance and ubiquity of email has grown over time, the need for enhanced security measures has become increasingly apparent.

One significant security concern is the propagation of emails with forged sender addresses, for example, as part of spam and phishing campaigns. 
These forged emails exploit the lack of authenticity in \ac{SMTP} and are a notorious threat to users.
In response to this security gap, the \ac{SPF}~\citep{rfc4408} was introduced in 2003 as a standard to define approved sending servers of emails for a specific domain.
To this end, a domain owner can configure a \ac{DNS} record, which specifies a list of authenticated hostnames or IP addresses that are permitted to send emails on behalf of the domain.

While the introduction of \ac{SPF} appears reasonable at first glance, it does face certain limitations.
Primarily, SPF only addresses the authenticity problem by extending it to hostnames and IP addresses.
This means that users have to trust their network provider or managed email service to accurately handle this aspect of email security.
Furthermore, SPF introduces new problems when it comes to email redirection, which becomes problematic for mailing lists.

As another protocol to improve the authenticity of emails, \ac{DKIM} requires the sending server to add a cryptographic signature to all outgoing emails.
These signatures can then be verified by the receiving mail server, providing an additional layer of authenticity.
On top of DKIM and SPF, \ac{DMARC}~\citep{rfc7489} adds a descriptive record to the DNS.
This entry describes the behavior that a receiving mail server should adopt when an email is received and there are issues with SPF or DKIM authentication.


Note that \ac{SPF} as well as \ac{DKIM}, and \ac{DMARC} are not able to provide reliable confidentiality, integrity, and authenticity for end-to-end communications like S/MIME~\citep{rfc8551} and OpenPGP~\citep{rfc3156}.
These mechanisms, however, are not widely adopted yet and suffer from their own problems~\citep{Poddebniak2018}. Unlike SPF, which is implemented in the application layer, these mechanisms are implemented on top of email and do not affect email servers. 
Consequently, large email providers, such as Google and Microsoft, recommend and enforce the use of \ac{SPF} as a basic element of email security. 

\subsection{A Primer on SPF}

The security mechanism \ac{SPF} operates through \ac{DNS} records that store a configuration of permitted IP addresses, networks or hostnames. This configuration is controlled by the domain owner and is publicly accessible.
When a server receives an incoming email, it can perform a DNS lookup to retrieve the corresponding \ac{SPF} record associated with the sender's domain.
While processing the configuration, the server validates whether the email originates from an authorized source. 
In the following, we use the term \textit{SPF record} to refer to the string in a DNS request of type \emph{TXT} that starts with \spfcode{v=spf1} and defines the configuration. The deprecated DNS type \emph{SPF} is not considered in this work.



Technically, an SPF record is composed of different policy terms.
These terms are either directives containing \emph{mechanisms} with qualifiers, or \emph{modifiers}.
While modifiers provide additional information for the configuration of the policy, a mechanism defines a way to determine allowed IP addresses, networks or hostnames.
Once there is a match between the sending host and a mechanism directive, the processing of the SPF record ends and the mechanism's qualifier is returned as the result of the authorization. 

\paragraph{Mechanisms.}
We first take a look at the different mechanisms and their qualifiers.

\begin{description}
\setlength{\itemsep}{2pt}
\item[\spfcode{a}] \hfill \\
This mechanism matches if the sending IP addresses match the specified A or AAAA DNS records.

\item[\spfcode{mx}]\hfill \\
If an email originates from any of the hostnames or IP addresses specified in an MX DNS record, this mechanism matches.

\item[\spfcode{ip4}, \spfcode{ip6}]\hfill \\
It is also possible to set allowed IP addresses.
If the sender IP is listed here, this mechanism matches.

\item[\spfcode{all}]\hfill \\
As the name says, this mechanism matches all sender IPs.
Everything after this term is ignored.

\item[\spfcode{exists}]\hfill \\
This mechanism can check if a specific domain or hostname exists in the DNS.
If the hostname exists, this mechanism matches.

\item[\spfcode{include}]\hfill \\
The \spfcode{include} mechanism allows a domain to include another domain's permitted sender IPs from its SPF record.
This is useful to cross administrative borders at email delivery.
The receiving server evaluates the content of the included SPF record as usual, but this mechanism only matches if the sender IP is explicitly allowed by it.
Otherwise, and also in case of an error, the processing of the including record continues.
Therefore, it is not possible to deny any or all IP addresses with the \spfcode{include} mechanism.

\item[\spfcode{ptr}]\hfill \\
The last one, the \spfcode{ptr} mechanism, checks if a reverse DNS entry for the sending IP address exists.
This mechanism matches if the IP addresses of the sending host and of the domain name retrieved by the reverse lookup are equal.
Since this is a slow mechanism that causes a high DNS load, using this mechanism is generally not recommended.

\end{description}

Except for \spfcode{all}, the mechanisms can be specified by arguments. If no argument is given, the domain or IP address to be checked is used.
The \spfcode{a}, \spfcode{mx} and both \spfcode{ip} mechanisms additionally allow specifying a CIDR prefix length to  specify a complete network.
If no CIDR prefix length is given, it will refer to a single host.

A qualifier can be placed in front of each mechanism to define the outcome in case the IP address of the sending email server matches.
If a mechanism is specified without a qualifier, \spfcode{pass} is implied.

\begin{description}
\setlength{\itemsep}{2pt}
\item[\spfcode{+} \hspace{5pt}{\mdseries (pass)}]\hfill \\
{
The email server is authorized to send emails for the domain.
}

\item[\spfcode{-} \hspace{5pt}{\mdseries (fail)}]\hfill \\
{
The email server is explicitly not authorized to send emails for the domain.  
}

\item[\spfcode{?} \hspace{5pt}{\mdseries (neutral)}]\hfill \\
{
There is no assertion about the email server.
}

\item[\spfcode{\softfail} \hspace{4pt}{\mdseries (softfail)}]\hfill \\
{
The email server is neither explicitly denied nor allowed to send emails for the domain. It is not authorized, but not strong enough to create a strict policy.
}
\end{description}

If the evaluation has found a match between the sending IP address and a mechanism, the qualifier is returned as a result and gives information about the authenticity of the sender.
Note that the default result for \ac{SPF} is not \spfcode{fail}.
If there is no explicit \spfcode{fail} or \spfcode{softfail} qualifier for the \spfcode{all} mechanism, the SPF result for all hosts without another match is always the default value \spfcode{pass}.
If no mechanism matches, for example, because the IP address is not listed as approved sender and there is no \spfcode{all} mechanism set, the result is \spfcode{neutral}.

\paragraph{Modifiers.}
In addition to the mechanisms, there are also modifiers, of which for our work only the \spfcode{redirect} modifier is relevant. 
This modifier allows a domain to delegate its SPF record to another domain.
Like the \spfcode{include} mechanism, this can be used to cross administrative borders, but in contrast to that mechanism, the complete evaluation process is performed on the redirected domain. Any statements after a \spfcode{redirect} modifier are ignored. \\

Additionally, the evaluation of an SPF record at the receiving site can provide further return values.
In particular, \spfcode{none} is returned when there is no valid domain from the SMTP session or no SPF record.
In the event of a transient error like a DNS error, a \spfcode{temperror} is raised.
If a DNS error is permanent, such as NXDOMAIN, a \spfcode{permerror} is returned.
The \spfcode{permerror} is further used when the SPF record can not be evaluated correctly.
In \autoref{sec:errors}, we investigate the occurrence of these errors in the wild.

\paragraph{Example.}
Let us investigate the following SPF record:
\begin{center}
\color{cbone}
\begin{verbatim}
 v=spf1 +mx a:puffin.example.com/28 -all
\end{verbatim}
\end{center}
In this example, we have multiple mechanisms: \spfcode{mx}, \spfcode{a} and \spfcode{all}.
The term \spfcode{+mx} with the explicit \spfcode{pass} qualifier specifies that the domain's MX servers are authorized to send mails.
The next directive specifies an IP address range, namely the IP address of \spfcode{puffin.example.com} with a /28 CIDR notation.
As no explicit qualifier is given, the default \spfcode{pass} is used, and all addresses in this range are authorized.
The last directive, \spfcode{-all}, enforces to reject emails from all other sources.





