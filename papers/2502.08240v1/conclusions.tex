\section{Conclusions}
\label{sec:conclusions}

With our analysis, we shed light on the state of SPF in the wild.
We observe an increasing adoption of this security mechanism; at the same time, we find flawed and overly coarse authorization policies in numerous cases.
We demonstrate that these lax practices increase the attack surface of SPF and make spoofing senders possible with little effort.
It is enough to identify web hosting providers that manage thousands of domains with permissive configurations to send spoofed emails at a large scale.

Fortunately, we can conclude from our notification campaign that several of the configurations were not intentionally malfunctioning.
Shortly after our notifications, we could already observe thousands of fixed SPF entries. In general, SPF faces a tradeoff between security and usability.
Although a minimal authorization policy would be desirable, operators often relax their configurations for practical reasons, for instance, because it is inconvenient to identify all sending hosts or because they try not to interfere with their clients' activities.
Our analysis shows that the compromises made by operators are far from adequate, and we therefore strongly recommend using more validated and restrictive SPF policies in practice, for example, by following the guidelines presented in this paper.
