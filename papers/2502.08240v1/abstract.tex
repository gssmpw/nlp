\begin{abstract}
The \acl{SPF} (SPF) is a basic mechanism for authorizing the use of domains in email.
In combination with other mechanisms, it serves as a cornerstone for protecting users from forged senders.
In this paper, we investigate the configuration of SPF across the Internet. To this end, we analyze SPF records from 12~million domains in the wild. %
Our analysis shows a growing adoption, with \SPFDomainsPerc of the domains providing SPF records. 
However, we also uncover notable security issues: First, \ErrorDomainsPerc of the SPF records have errors, undefined content or ineffective rules, undermining the intended protection.
Second, we observe a large number of very lax configurations. For example, \HugeDomainsPerc of the domains allow emails to be sent from over \numprint{100000}~IP~addresses. 
We explore the reasons for these loose policies and demonstrate that they facilitate email forgery.
As a remedy, we derive recommendations for an adequate configuration and notify all operators of domains with misconfigured SPF records.
\end{abstract}
