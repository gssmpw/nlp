\section{Related work and further directions}
We identify a~few directions related to the current work.

\medskip
\textbf{Essentiality for other graph parameters.}
A~reader familiar with the paper ``\emph{Induced subgraphs of induced subgraphs of large chromatic number}'' of Gir\~ao et al.~\cite{giraoetal} likely noticed a~parallel with~\cref{thm:main-one-graph}.
The authors show that for every graph $H$, there is a~class $\mathcal C$ of unbounded chromatic number such that the \mbox{$H$-free} graphs of $\mathcal C$ have bounded chromatic number. 
(Furthermore, if $H$ has at least one edge, the class $\mathcal C$ can be picked to have the same clique number as~$H$.)
In this sense, the current paper can be thought of as ``\emph{Induced subgraphs of induced subgraphs of large treewidth}.''
We note however that the construction in~\cite{giraoetal} contains, by design, arbitrarily large induced bicliques, so could not guide us in achieving~\cref{thm:main-one-graph}.

We now know that no fixed induced subgraph can be removed from every class $\mathcal C$ of unbounded $p$ while preserving that $p$ is unbounded, for parameter $p$ equal to chromatic number or treewidth.  
Other graph parameters can be considered such as clique-width, twin-width, etc.
As our construction leads to a~weakly sparse class, within which treewidth and clique-width are known to be functionally equivalent~\cite{Gurski00}, our paper also offers a~complete answer for clique-width.
The case of twin-width remains interesting.
It is known that the class of permutation graphs is a~minimal hereditary class of unbounded twin-width~\cite{twin-width1}.
This translates into the essentiality (for twin-width) of every permutation graph.
To our knowledge, the essentiality (for twin-width) of any other graph is open.  
In general, we propose the following questions.

\begin{meta-problem}[Characterize $p$-essential graphs]
  For a~parameter of choice~$p$, which graphs $H$ are such that there is a~hereditary class of unbounded $p$ whose $H$-free graphs have bounded~$p$?
\end{meta-problem}

Observe that Ramsey's theorem can be rephrased as the fact that complete or edgeless graphs are the only $p$-essential graphs when $p$ is the number of vertices.

\medskip

\textbf{Characterizing the essential families.}
For families rather than single graphs, our understanding of essentiality (for treewidth) is not quite complete.
What about families $\mathcal H$ of unbounded treewidth?
Such a~family $\mathcal H$ is essential if and only if its hereditary closure is a~minimal hereditary class of unbounded treewidth.
Thus, any family $\mathcal H$ consisting of infinitely many complete graphs or of infinitely many complete bipartite graphs is essential.
On the other hand, a~family $\mathcal H$ consisting of infinitely many complete graphs (resp.~complete bipartite graphs) plus at~least one graph that is not a~clique (resp.~not a~biclique) is \emph{not} essential.
Hence, we narrowed down the open cases to weakly sparse families.
Is there a~weakly sparse family whose hereditary closure is a~minimal hereditary class of unbounded treewidth?
This is precisely a~question of~Cocks~\cite{COCKS2024104005}, which we reformulate here.

\begin{conjecture}[Cocks's Conjecture 1.5 in~\cite{COCKS2024104005}]
    A family of unbounded treewidth is essential if and only if it contains only complete graphs, or only complete bipartite graphs. 
\end{conjecture}

\medskip

\textbf{Treewidth logarithmically bounded in the number of vertices.}
Sintiari and Trotignon~\cite{layered-1} remarked that their layered wheel constructions have treewidth logarithmic in their number of vertices, and suggested relaxing the bounded treewidth condition and investigating logarithmic treewidth instead. 
Many (NP-hard) problems can be solved in polynomial time in $n$-vertex graphs of treewidth $O(\log n)$, such as every problem expressible in the so-called \emph{Existential Counting Modal Logic} of Pilipczuk, a~large fragment of Monadic Second-Order logic~\cite{Pilipczuk11}.
In the past five years, several classes have been shown to have logarithmic treewidth~\cite{Chudnovsky_2022, Bonamy24, BonnetD23, chudnovsky2024inducedsubgraphstreedecompositions, sparseOuterString}.
Our result shows that any graph is responsible for the transition from unbounded to bounded treewidth in some class.
Is this true for the transition between superlogarithmic and logarithmic treewidth? 

\begin{problem}\label{p:logarithmic-tw}
    For which families $\mathcal H$ is there a hereditary class $\mathcal C$ of superlogarithmic treewidth such that, for every $H \in \mathcal H$, the $H$-free graphs of $\mathcal C$ have at most logarithmic treewidth?
\end{problem}

\Cref{p:logarithmic-tw} is already open for singleton families $\mathcal H = \{H\}$.
It should be noted that the Pohoata--Davies grid~\cite{Pohoata14,Davies22} has no large clique, biclique, subdivided wall, or its line graph as an induced subgraph and has superlogarithmic treewidth: the $n \times n$ Pohoata--Davies grid has treewidth $\Theta(n)$.  

\medskip

% \textbf{Bounded twin-width witnesses of large treewidth}

% \cref{thm:main-bdd-tw} says that, if we want to find a property $\Pi$ such that all graphs of sufficiently large treewidth have an induced subgraph with large treewidth and property $\Pi$, then $\Pi$ had better be non-treewidth-hereditary. One such property is that of having bounded twin-width.\footnote{Actually \emph{bounded treewidth} is also not treewidth-hereditary, but the associated meta-conjectures are trivially false.}
% We thus give the following special case of~\cref{meta-conj:plain}, not refuted by our present work, and motivated by the fact that the weakly sparse layered wheels, the Pohoata--Davies grids (and their extensions), and the death star all have bounded twin-width.
% \begin{conjecture}\label{conj:tww-tw}
%   Every hereditary class of unbounded treewidth admits a~subclass of unbounded treewidth and bounded twin-width.
% \end{conjecture}
% We will come back to~\cref{conj:tww-tw}.


% \medskip

\textbf{Essentiality in the high-girth setting.} Given the strong constraints imposed by \cref{thm:main-bdd-tw} on properties forced by large treewidth, another line of investigation is to ask whether the situation is any different in more restricted settings. A natural question in this direction is whether \cref{thm:main-triangle-free} can be generalized to high girth. In \cref{lem:no-high-girth-wheels}, we show that if such a generalization is possible, it cannot use a layered wheel construction, and ask whether the result does in fact generalize (and in particular, whether large treewidth does force certain finitely-hereditary properties in the high-girth setting):

\begin{problem} \label{conj:no-high-girth}
    Does there exist a graph $H$ such that for every $(C_3, C_4)$-free family $\mathcal C$ of unbounded treewidth, the subclass of $H$-free graphs of $\mathcal C$ has unbounded treewidth? 
\end{problem}