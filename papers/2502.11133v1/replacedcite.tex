\section{Related Work}
\vspace{-0.4em}
\paragraph {Multi-Agent System.}
Contemporary LLM-based multi-agent systems (MAS) can be broadly categorized into two paradigms: \textbf{(1) Fixed agentic networks} with pre-established, manually crafted architectures, from debate____, collaboration____ to competetive____. MacNet____ systematically analyzed typical multi-agent collaboration topologies such as chain, tree, graph, \textit{etc.}; \textbf{(2) Dynamic agentic networks} that configure their structure and communication strategies based on real-time feedback and observations. ADAS____ and its follow-up works____ leverage search methods such as Monte Carlo Tree Search (MCTS) and evolutionary algorithm____ to discover effective agent strategies. Other works like DyLAN____, GPTSwarm____ and AgentPrune____ dynamically optimize the inter-agent topologies. Nevertheless, contemporary MAS is often LLM-homogeneous, \textit{i.e.}, relying exclusively on the same LLM backbone, failing to collectively organize heterogenous LLM-agents.


\vspace{-0.4em}
\paragraph{Single LLM Routing}
Efficient routing strategies for single LLMs have been extensively explored to balance computational cost and model performance.
%and have evolved significantly from static approaches to dynamic, task-aware systems. 
Early attempts on LLM routing include HybridLLM____, RouteLLM____ and FrugalGPT____, which primarily focus on binary routing and leverage techniques like sequential pipelines or preference-driven routing to enhance decision-making performance. More recent practices, including Rootoo____, C2MAB-V____, GraphRouter____ and RouterDC____ have shifted towards multi-choice selection frameworks. However, existing routing methodologies mainly focus on single-agent scenarios, and their unawareness of inter-agent topology constrains their applicability to more complex tasks and limits scalability in larger systems.