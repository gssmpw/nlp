\section{Related Work}
\textbf{Bilevel Bayesian optimization.} Most existing bilevel Bayesian optimization methods, as mentioned in the introduction, apply BO only to the upper-level problem and rely on repeated lower-level optimizations at each upper-level query \citep{kieffer2017bayesian, islam2018efficient, wang2021comparing}. \citet{dogan2023bilevel} introduced an acquisition function that conditions on lower-level solutions during upper-level optimization for information flow between both levels. However, these nested methods require gradient information to estimate or refine lower-level solutions, making them unsuitable for derivative-free bilevel problems. \citet{fuconvergence} provided theoretical guarantees for a nested bilevel framework with stochastic gradient descent at the lower level and BO at the upper level. The reliance on lower-level gradients means the analysis does not extend to general derivative-free bilevel problems as well. In contrast, our proposed method is capable of handling blackbox, derivative-free bilevel problems, and our theoretical analysis is applicable to these settings.

An exception to the nested framework is a very recent parallel work by \citet{ekmekcioglu2024bayesian} on arxiv. However, it has no theoretical guarantees and cannot handle constraints, unlike our proposed algorithm.

\textbf{Constrained Bayesian optimization.} 
Several constrained BO algorithms have been proposed \citep{gelbart2014bayesian, hernandez2016general}. In particular, \citet{xu2023constrained} and \citet{nguyen2023optimistic} both introduced confidence-bound based optimistic estimations of the feasible set, with the former providing an infeasibility declaration scheme and the latter including a function query strategy for decoupled settings. These feasible set estimations guide sampling toward probable feasible points, improving sample efficiency. Compared to constrained optimization, bilevel optimization presents additional challenges due to the need to optimize a separate lower-level problem, where the optimal solutions are unknown and often estimated suboptimally.

\textbf{Comparison to other optimization problems.} While some optimization problems, such as robust optimization \citep{bogunovic2018adversarially} and composite objectives optimization \citep{liregret}, respectively involve an additional random variable and composite objective function, they remain single-level problems. In contrast, bilevel optimization involves a two-level hierarchical structure, where the upper-level is constrained by lower-level solutions, making it fundamentally different from these other settings.