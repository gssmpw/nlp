\section{Related Work}
Many approaches have been proposed to solve the CSG problem either optimally or approximately, including dynamic programming algorithms, anytime algorithms, heuristic algorithms, and scalable solutions. 
Dynamic programming approaches, such as those proposed in~\cite{yeh1986dynamic,rahwan2008improved,michalak2016hybrid}, guarantee to find the optimal coalition structure but must be run to completion to do so. 
Anytime algorithms~\cite{sandholm1999coalition,dang2004generating,rahwan2009anytime}, on the other hand, allow for premature termination while providing intermediate solutions during execution. 
The hybrid algorithms that combine dynamic programming algorithms with anytime algorithms~\cite{michalak2016hybrid,changder2020odss,Changder_Aknine_Ramchurn_Dutta_2021,ijcai2024p27,ijcai2023p35}  are the fastest exact algorithms. 

Heuristic algorithms, such as those proposed in~\cite{sen2000searching,keinanen2009simulated,di2010coalition}, prioritize speed and do not guarantee to find an optimal solution. These algorithms are useful when the number of agents increases and the problem becomes too hard to solve optimally. For instance, the simulated annealing method~\cite{keinanen2009simulated}, a stochastic local search algorithm, explores different neighborhoods of coalition structures by splitting, merging, or shifting agents. It starts with a random coalition structure and moves to a new one in its neighborhood at each iteration, with the movement accepted with a probability that depends on the difference in utility and a decreasing temperature parameter. 

Very few scalable solutions to CSG exist. For example, the Monte Carlo tree search method proposed in~\cite{wu2020monte} finds solutions by sampling the coalition structure graph and partially expanding a search tree that corresponds to a partial search space that has been explored. The hierarchical clustering approach proposed in~\cite{farinelli2013c} builds a coalition structure by merging coalitions using a similarity criterion based on the gain that the system achieves if two coalitions merge. The search algorithms FACS and PICS proposed in~\cite{9643288,10098066} %,Taguelmimt_Aknine_Boukredera_Changder_2021} 
generate coalition structures based on code permutations applied to selected initial vectors of a different search space representation. To the best of our knowledge, PICS~\cite{10098066} and CSG-UCT~\cite{wu2020monte} are the best performing of the prior algorithms. 
Our proposed algorithm evaluates possible splits of coalitions and possible merges of coalition pairs. Unlike some other approaches, such as the C-Link method~\cite{farinelli2013c}, which cannot split coalitions once they are merged, our algorithm allows for coalitions to be split and merged multiple times. This makes our algorithm less likely to get trapped in local maxima. 

%\subsection{Multiagent Path Finding}

Multiagent path finding (MAPF)~\cite{stern2019multi2} is another important problem in multiagent systems, where agents must navigate through a given environment to reach their goals while avoiding collisions with each other. This problem of planning paths for multiple agents is also known to be NP-hard~\cite{Yu_LaValle_2013} and has been studied extensively in the literature. It is inspired by real-world applications such as warehouse
logistics~\cite{Ma2017LifelongMP}, autonomous aircraft-towing vehicles~\cite{Morris2016PlanningSA}, airport operations~\cite{Li2019SchedulingAA}, and video games~\cite{li2020moving}. 
%There are two main approaches to solve MAPF: centralized and decentralized. In the centralized approach, a single agent plans the paths for all agents, while in the decentralized approach, each agent plans its own path based on local information. Both approaches have their advantages and disadvantages and have been applied in different domains.
One of the most popular algorithms for MAPF is the Conflict-Based Search (CBS) algorithm~\cite{sharon2015conflict}. CBS is a complete algorithm that guarantees to find the optimal solution if one exists. Other algorithms~\cite{Gange_Harabor_Stuckey_2021,Barer2014SuboptimalVO,Li_Tinka_Kiesel_Durham_Kumar_Koenig_2021,li2021eecbs,Li_Gange_Harabor_Stuckey_Ma_Koenig_2020,Li2020EECBSAB} based on CBS and other methods have been developed to solve this problem in optimal, suboptimal, or bounded suboptimal ways. 
In the next section, we explain how we draw inspiration from MAPF concepts to enhance the search for solutions to CSG and propose a new algorithm that adapts ideas from MAPF for solving the CSG problem.

In this paper, we propose a path search algorithm for finding optimal coalition structures in coalition structure generation. The algorithm operates on a graph where each node represents a coalition structure, and a search agent explores this graph aiming to reach the highest-valued solution. The algorithm starts from a designated start node and iteratively explores neighboring nodes, guided by the values of coalition structures.

The key components of this approach that will be detailed in the following sections are:

\begin{itemize}
    \item The search space is represented as a graph, with each node corresponding to a coalition structure. Transitions between nodes occur through coalition splitting or merging.
    \item A search agent explores the graph by moving from one node to another, selecting nodes based on their coalition structure values.
    \item Upon finding better solutions, the algorithm creates paths between previous and new solutions to explore potential improvements along the path.
    \item To optimize search efficiency, the algorithm employs memory management techniques, maintaining lists of nodes in memory and dynamically adjusting them based on solution quality.
    \item Multiple search agents are employed to accelerate finding high-quality solutions, with conflict resolution mechanisms to prevent redundant exploration.
\end{itemize}