\section{Introduction}
Visual storytelling uses the combination of visual elements and narratives to communicate complex ideas (e.g., data, facts, and opinions) in a way that is more engaging and descriptive than traditional text-only stories~\cite{kulkarni2023innovating}. 
This approach has been widely used in various fields, including journalism~\cite{nyt, wapo, guardian}, education~\cite{Williams2019Attending, Baharuddin2023The}, commercial activities~\cite{Moin2020Storytelling, Zhou2005VISUAL}, and scientific communication~\cite{Ma2012Scientific}. It plays a crucial role in enhancing the readers' understanding of content, as well as increasing their emotional engagement and connection to the story.


The evolution of digital technologies has greatly expanded the possibilities of visual storytelling, offering new genres and more diverse ways to deliver content~\cite{segel2010narrative}. 
The introduction of digital displays brought dynamic and interactive visual content, changing how readers interact and engage with stories~\cite{Yu2018A, Greussing2020Learning}. 
The rise of mobile devices further transformed access and interaction modalities, enabling readers to experience stories anywhere with new input modalities such as touch-based interactions~\cite{Lima2020Sketch-Based,Sheremetieva2022Touch}.
As digital technologies continue to evolve, the pressing question now is: "What is next for visual storytelling, and how can it offer novel experiences?"


As immersive technologies rapidly develop, they are becoming increasingly accessible and affordable, positioning them as promising candidates for the next generation of major public digital platforms. Immersive devices (e.g., Virtual Reality (VR) headsets) can render 2D and 3D information in the surrounding space or even replace the users' entire visual field, offering unprecedented opportunities for visual storytelling~\cite{kraus2021value, isenberg2018immersive}.
A prominent use of immersive technology for storytelling is 360-degree videos~\cite{google_2016_beyond, sportsillustrated_2017_chapter}. However, these videos often limit the readers' navigation, providing little control over story progression or detail exploration due to their fully author-driven nature.
Beyond static videos, major news outlets like The New York Times and The Washington Post have embraced interactive formats in their web-based articles to present public events (e.g., COVID-19 and elections) more dynamically.
They also leverage immersive technologies for more interactive experiences, such as 
mobile AR stories created by The New York Times~\cite{thenewyorktimes_2022_showcasing}. 
However, these narratives tend to rely on basic touchscreen inputs and offer limited immersion. They are often confined to a single scene without transitions between multiple narrative elements.
In this work, we focus on improving immersive storytelling specifically for public journalism.
We seek to extend the design of these journalistic stories into immersive environments, where interactivity and complete storylines can foster deeper engagement and understanding.

Our goal is to leverage the unique display and interaction affordances of immersive technologies in storytelling. 
A major advantage of these technologies is their ability to render 3D spatial information at its original scale~\cite{isenberg2018immersive}. 
Building on this characteristic, we investigated how people visually perceive 3D spatial representations (i.e., ego- vs. exocentric viewpoints) and how they can interactively navigate in the 3D space (i.e., active vs. passive navigation) as an initial exploration. Specifically, readers can experience an immersive story from an \textbf{exocentric} (exo) viewpoint to gain an overview of a 3D scene or from an \textbf{egocentric} (ego) viewpoint to fully immerse themselves with life-sized artifacts. Additionally, readers can either \textbf{actively} move in space and interact with digital artifacts with greater embodiment and agency or \textbf{passively} follow predefined transitions.
We anticipate that these two factors (i.e., viewpoint and navigation) will have a significant effect on user-perceived spatial immersion and understanding in immersive storytelling. 
\textbf{\textit{Our research goal is to understand the advantages and limitations of these design choices in spatial immersion and understanding in immersive VR stories}}.


To study the effects of viewpoints and navigation in immersive storytelling, we needed to develop immersive stories that were controlled for these factors. 
While there are many web-based stories for public journalism, immersive versions are still rare. Since we aimed to focus on viewpoint and navigation rather than the story content itself, we leveraged existing web-based stories and created versions in immersive environments to cover the design variations we needed for the study.
We collected web-based stories from major news outlets, particularly those with rich 3D spatial content, such as a New York Times story about how COVID-19 spreads in a classroom~\cite{covid}. 
{These visual stories are in single scene and follow a linear structure. We consider this atomicity as a starting point for studying our intended effects. Meanwhile, this basic structure serves as the foundational unit for more complex stories.}
We then conducted a formative study with three VR experts to elicit potential designs of 117 selected stories in an immersive format, with an intended focus on viewpoint and navigation.
Based on the design considerations from this formative study, we adapted four representative stories into immersive formats. Finally, we conducted a user study with 24 participants to compare the 2$\times2$ combinations (ego vs. exo viewpoint, and active vs. passive navigation).
Our paper structure follows the steps we took to prepare and conduct the study, as illustrated in \autoref{fig:procedure}.

\begin{figure*}[t]
\centering
  \includegraphics[width=0.6\linewidth]{Figures/procedure_2.pdf}
  \caption{Overall Procedure of This Work. We started by a formative study analyzing the web-based stories in terms of their viewpoint and navigation designs. It elicited some design considerations, which we used to design and implemented four story cases. We finally investigated the effects of viewpoints and navigations on 24 participants.}
  \label{fig:procedure}
  \Description{A research workflow diagram illustrating the process of a VR storytelling study. The left side details the Formative Study (Section 3), starting with web-based story collection and VR design elicitation from 3 experts to identify design considerations. These considerations are applied to the VR Story Adaptation (Section 4), involving design and implementation of immersive stories. The right side describes the User Study (Section 5), which evaluates different combinations of viewpoint (egocentric vs. exocentric) and navigation (active vs. passive) with 24 participants.}
\end{figure*}

We found that viewpoints and navigations showed more significant impact on the story understanding and spatial immersion, respectively. Exo demonstrated benefits on story content comprehension and ego performed better on the spatial information memorability. Active navigation created a higher level of presence with a cost of more perceived workload, but it is acceptable in a ego viewpoint. Overall, participants predominantly preferred Ego$+$Active or Exo$+$Passive, depending on their priority on contents or experiences.

In summary, our primary contributions are:
\begin{itemize}
    \item {Four design considerations elicited for adapting web-based visual stories to an immersive format.}
    \item A user study empirically investigates the effects of viewpoint and navigation on perceived spatial immersion and understanding {across four adapted immersive story instances}.
\end{itemize}