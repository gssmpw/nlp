\section{Results}
In this section, we reported the study results.
We performed the statistical analysis to test significances on the comprehension and rating data collected from three after-session tasks and subjective comments from final interviews. Significance values were reported for $p < 0.1(.)$ (dashed line in figures), $p < 0.05(*)$, $p < 0.01(**)$, and $p < 0.001(***)$.

\subsection{Story Session Durations}
\begin{figure}[h]
\centering
  \includegraphics[width=\columnwidth]{Figures/time.pdf}
  \caption{Average Story Session Durations. The dashed line indicates the marginal significance ($p < 0.1$) and the solid line indicates statistical significance ($p < 0.05$).}
  \label{fig:res_time}
  \Description{A bar chart comparing the completion time (in seconds) for different combinations of viewpoint (Egocentric vs. Exocentric) and navigation style (Active vs. Passive) in a VR storytelling study. The first chart shows similar completion times for Ego (blue) and Exo (orange) viewpoints, averaging around 450 seconds. The second chart shows a significant difference between Active (dark gray) and Passive (light gray) navigation, with Active taking longer (~550 seconds) compared to Passive (~400 seconds), as indicated by three asterisks *** denoting high statistical significance. The third chart breaks down completion times by both viewpoint and navigation combinations, with Ego Active taking the longest (~550 seconds), followed by Exo Active, while Passive conditions (Ego Passive and Exo Passive) take less time (~400 seconds). Significant differences between conditions are marked with two ** or three asterisks *** indicating different levels of statistical significance.}
\end{figure}

We collected the duration of each story session, defined as the total time the headset was worn, excluding any pauses. It is shown in \autoref{fig:res_time}. 
{\textbf{Participants spent significantly more time on active navigation cases than passive ones, while the duration between ego and exo viewpoints was not salient.}} 
A Shapiro-Wilk normality test confirmed the data followed a normal distribution ($p < 0.05$). We ran a General Linear Mixed Model (GLMM) with a Gaussian link function, using \textit{Viewpoint} (ego/exo), \textit{Navigation} (active/passive), their interaction ($\textit{Viewpoint} \times \textit{Navigation}$), and participant \textit{Group} as fixed effects, with participant ID as a random effect. Results showed significant effects only for \textit{Viewpoint} ($p < 0.001$), and a pairwise post-hoc Tukey's Honest Significant Difference (HSD) test confirmed a significant difference in duration between active and passive conditions in the exo viewpoint.

\subsection{Story Understandings}
We assessed story understanding through content comprehension and spatial information depiction tasks. 
{We found that \textbf{ego viewpoint showed significantly better performance on spatial information depiction.}}

\textit{Content Comprehension.}
\begin{figure*}[h]
\centering
  \includegraphics[width=\linewidth]{Figures/comprehension.pdf}
  \caption{Average Sub-Scores of Content Comprehensions. The dashed line indicates the marginal significance ($p < 0.1$).}
  \label{fig:res_comprehension}
  \Description{A series of bar charts displaying scores for Main Idea Memorization, Detail Memorization, and Random Inspirations in relation to egocentric (Ego) vs. exocentric (Exo) viewpoints and active vs. passive navigation in a VR storytelling study. For Main Idea Memorization, the first row of charts shows no significant difference between Ego and Exo, or Active and Passive, with both averaging similar scores. When broken down by viewpoint and navigation, Exo Passive has the highest score, while Ego Active and Exo Active also perform well. For Detail Memorization, there is no marked difference between Ego and Exo or Active and Passive, with all combinations performing similarly. For Random Inspirations, the charts show slightly higher scores for Ego over Exo, and Active over Passive. Ego Active has the highest score, suggesting this combination encourages more creative inspiration. Error bars indicate standard deviation across the different conditions.}
\end{figure*}
The content comprehension task consists of three sets of questions: main idea memorization, detail memorization, and random inspirations. They have a total score of 3, 5 and 1, respectively. Each score corresponds to an idea or fact in the first 2 sets. For the random inspiration, participants would get the score if they mentioned anything derived from the story (e.g., linking to their own experiences or adding their own opinions). The only case in which they would lose this point is when they simply restated the story content (e.g. ``I learned [idea/fact of the story] is important'').

As \autoref{fig:res_comprehension} shows, the exo and passive conditions had a higher memorization performance, while the active condition achieved better random inspiration. In terms of the four combinations, exo$+$passive was the best on two memorizations, whereas ego$+$active had the highest score in random inspirations.
We then tested the normality of the data using a Shapiro-Wilk normality test, which showed that the data followed a normal distribution ($p < 0.05$).
Thus, we ran a GLMM with a Gaussian link function.
The results showed only a marginal significance of \textit{Viewpoint} on detail memorization ($p = 0.09$).

\textit{Spatial Information Depiction.}
%figure of depiction
\begin{figure*}[h]
\centering
  \includegraphics[width=\linewidth]{Figures/drawing.pdf}
  \caption{Average Sub-Scores of Spatial Information Depiction. The dashed line indicates the marginal significance ($p < 0.1$) and the solid line indicates statistical significance ($p < 0.05$).}
  \label{fig:res_drawing}
  \Description{A series of bar charts showing scores for Object Existence and Object Transform tasks in a VR storytelling study comparing egocentric (Ego) vs. exocentric (Exo) viewpoints and active vs. passive navigation. For Object Existence, the first set of charts shows no significant difference between Ego and Exo or Active and Passive, with all conditions scoring similarly around 2-3 points. In the breakdown by viewpoint and navigation, Exo Passive has the highest score, but differences remain minimal. For Object Transform, the second set of charts reveals a significant difference between Ego and Exo, with Ego scoring higher, particularly in Ego Active, which has the highest score. A single asterisk * indicates statistical significance in the difference between these conditions. Error bars represent standard deviations for each condition.}
\end{figure*}
We evaluated participants' drawings for object existence and object transform.
\autoref{fig:res_drawing} showed ego outperformed exo, especially in the ego+active condition.
We also observed that participants had the worst memorability of spatial information in exo$+$active.
We then tested the normality of the data using Shapiro-Wilk normality test, which showed that the data followed a normal distribution ($p < 0.05$).
Thus, we ran a GLMM with a Gaussian link function.
Results showed that only \textit{Viewpoint} has marginal significance on object existence ($p = 0.09$) and significance on object transform ($p < 0.05$). It also showed the signifcance of $\textit{Viewpoint}\times\textit{Navigation}$ on object transform ($p < 0.05$).
A post-hoc Tukey's HSD test of $\textit{Viewpoint}\times\textit{Navigation}$ on object transfrom showed that ego$+$active has a significantly higher score than exo$+$active.

{\textit{Result Summary.} Our hypothesis was partially supported by the results. We observed significantly better spatial information depiction in cases using the ego viewpoint. However, the results did not support the hypothesis that the exo viewpoint leads to better content comprehension, and navigation did not play a significant role in story understanding.}


\subsection{Spatial Immersion}
We collected the subjective ratings on spatial immersion for each case using a Presence Questionnaire (PQ) and NASA-TLX. We found that \textbf{active navigation enhances the sense of presence compared to passive navigation. It does not greatly increase workload when combined with ego viewpoint.}

\textit{Sense of Presence}
\begin{figure*}[t]
\centering
  \includegraphics[width=\linewidth]{Figures/presence.pdf}
  \caption{Average Sub-Scores of Presence Questionaire. The dashed line indicates the marginal significance ($p < 0.1$) and the solid line indicates statistical significance ($p < 0.05$).}
  \label{fig:res_presence}
  \Description{A series of bar charts illustrating scores for Realism, Possibility to Act, Quality of Interface, Possibility to Examine, and Self-Evaluation of Performance in a VR storytelling study, comparing egocentric (Ego) vs. exocentric (Exo) viewpoints and active vs. passive navigation. For Realism, there is a significant difference in Active vs. Passive conditions, with Passive scoring higher as indicated by two asterisks **, while Exo Active scored significantly lower than Exo Passive as marked by one asterisk *. For Possibility to Act, Active outperforms Passive with three asterisks ***, and Ego Active scores the highest among all combinations. Quality of Interface shows little difference between Ego and Exo, or Active and Passive, across all conditions. For Possibility to Examine, Passive scores slightly higher than Active with one asterisk *, and Ego Active scores highest overall. Self-Evaluation of Performance reveals significant differences, with Ego Active performing better than Exo Active and Passive conditions, indicated by one and three asterisks * and ***. Error bars represent standard deviations across conditions.}
\end{figure*}
The 19 PQ questions can be classified into 5 sub-categories: Realism (question 3, 4, 5, 6, 7, 10, 13), Possibility to Act (question 1, 2, 8, 9; Act for short), Quality of Interface (question 14, 17, 18, with scores reversed; Quality for short), Possibility to Examine (question 11, 12, 19; Examine for short) and Self-Evaluation of Performance (question 15, 16; Evaluation for short).

We ran a Shapiro-Wilk normality test on each sub-category, which showed that the PQ responses did not follow a normal distribution on any sub-category.
A follow-up Kolmogorov-Smirnov test showed that the data followed a Gamma distribution on both viewpoint and navigation factors.
Therefore, we used a GLMM model with a Log-Gamma link function.
Results showed the significance of \textit{Navigation} and $\textit{Viewpoint} \times \textit{Navigation}$ on 4 sub-categories (Realism, Act, Quality and Examine), and no significance of \textit{Viewpoint} and \textit{Group}.
A pairwise post-hoc Tukey's HSD test on $\textit{Viewpoint} \times \textit{Navigation}$ further showed that active has significantly higher realism, possibility to act and examine scores than passive, particularly in ego viewpoint. Specifically, ego$+$active received the highest average score on four sub-categories (Realism, Act, Examine and Evaluation). On Act and Evaluation, the post-hoc Tukey test showed that ego$+$active is significantly better than two other combinations. 

\textit{Perceived Workload}
\begin{figure*}[t]
\centering
  \includegraphics[width=\linewidth]{Figures/NASA.pdf}
  \caption{Average Sub-Scores of NASA-TLX Questionnaire. The dashed line indicates the marginal significance ($p < 0.1$) and the solid line indicates statistical significance ($p < 0.05$).}
  \label{fig:res_nasa}
  \Description{A series of bar charts displaying scores for Mental Demand, Physical Demand, Temporal Demand, Performance, Effort, and Frustration in a VR storytelling study, comparing egocentric (Ego) vs. exocentric (Exo) viewpoints and active vs. passive navigation. For Mental Demand, there is a significant difference between Active and Passive, with Passive scoring higher, indicated by two asterisks **, while Exo Active shows a significantly higher mental demand as marked by one and two asterisks * and **. Physical Demand reveals a substantial difference between Active and Passive, with Active scoring much higher, as shown by two and three asterisks ** and ***, particularly in the Ego Active condition. Temporal Demand shows a slight difference, with Passive scoring higher than Active, marked by one asterisk *. For Performance, Passive scores slightly higher than Active, with Exo Passive performing the best, marked by one asterisk *. Effort shows higher scores for Exo Active, with significant differences between conditions, marked by three asterisks ***. Finally, Frustration is significantly higher in Exo Active and Passive conditions, marked by one and three asterisks * and ***. Error bars represent standard deviations across the different conditions.}
\end{figure*}
We summarized the NASA-TLX reponses in \autoref{fig:res_nasa}. Expectedly, active had a higher overall workload than passive. 
We did the same step to test the data distribution and found that it also followed a Gamma distribution.
A GLMM with Log-Gamma link function comfirmed our observation with the signifcance on \textit{Navigation} ($p < 0.001$) and $\textit{Viewpoint} \times \textit{Navigation}$ ($p < 0.05$). It did not show any significance of other effects.
A follow-up post-hoc Tukey's HSD test on $\textit{Viewpoint} \times \textit{Navigation}$ further showed the significantly larger workload of exo$+$active than others, especially in physical demand.

{\textit{Result Summary.} Our hypothesis was supported by the results. The Ego+Active condition produced the strongest sense of presence without resulting in the highest perceived workload. However, we also observed that the heavy workload associated with active navigation might not be justified. In the Exo+Active condition, participants felt the least confident about their performance during story sessions. This suggests that the combined effects of viewpoint and navigation are more influential on story immersion than each factor individually.}




\subsection{Interview Analysis}
Finally, we collected and analyzed the final interviews about users' subjective experiences of reading stories in VR. Overall, We found a trend that \textbf{most participants preferred either ego$+$active or exo$+$passive stories and reported less favorable opinions towards the other two combinations.} We summarized the following reasons based on their comments.

\textbf{Active navigation provides freedom for explorations in the ego viewpoint.} Many participants favored active navigation primarily because of the ability to explore and play with the story elements. This is particularly true in two ego stories. For example, P4, P7, P21 and P23 all mentioned that the COVID story (Case 4) allowed them to freely move in the scene so that they could view the visualizations of airflows from different perspectives and understand the spread route of contaminants in the air. Also, P5, P6, P11 and P21 strongly endorsed the window open and fan installation operations as it gave them a sense of control over the story. Similiar comments appeared in the Tulsa story (Case 3) as well. Participants like the active navigation as \textit{``it is like an immersive exhibition in the museum''} (P4, P12, P20). This matches the highest PQ score of ego$+$active, as it greatly enhanced the diversity of interaction and range of movement in the story scene.

\textbf{Passive navigation is focused and efficient when reading stories in exo viewpoint.} While a considerable number of participants paid attention to user experiences, some prioritized the understanding and memorability of story content. Thus, they preferred the exo$+$passive condition because \textit{``information was concentrated in exo (viewpoint), and passive (navigation) had less workload than active (navigation)''} (P1, P9, P13, P14, P19). We observed that this group of participants did not have too much VR expertise, which increased their difficulty of active navigations and caused too much workload that impeded their focus (P19). With passive navigation, participants could quickly move to the next narratives without being trapped by VR interactions.

\textbf{Passive navigation resulted in overall negative user experiences in ego viewpoint.} The ego$+$passive condition received the most amount of critiques. Moving passively in an egocentric viewpoint (e.g. in a room) constrained the exploration participants could have. More seriously, it caused strong motionsickness for some participants (P1, P7). P7 even requested to pause the session and take off the VR headset for some rest. This indicates that our implementation of FOV reduction is not enough to alleviate motionsickness. Other techniques are necessary to be incorporated. Still, we received some positive feedback from participants who were less sensitive to movements in VR, who mentioned this experience was similiar to a Universal Studio tour on a club car (P4, P8). However, the overall current implementation of ego$+$passive is less acceptable to most participants.

\textbf{Active navigation increases the difficulty to keep track of spatial information in exo viewpoint.} As mentioned in feedback to exo$+$passive, active navigation introduced too much workload for VR non-experts in 2 exo stories. Another side effect we consistently received is the loss of object transform. This was particular the case for the 911 story (Case 1). P8, P9 and P23 reported that they could not tell WTC position as they rotated the map. This matches the lowest scores of exo$+$active in the spatial information depiction task.