%%
%% The abstract is a short summary of the work to be presented in the
%% article.
% Abstract for submission
\begin{abstract}
Visual storytelling combines visuals and narratives to communicate important insights. While web-based visual storytelling is well-established, leveraging the next generation of digital technologies for visual storytelling, specifically immersive technologies, remains underexplored.
%
We investigated the impact of the story viewpoint (from the audience's perspective) and navigation (when progressing through the story) on spatial immersion and understanding. First, we collected web-based 3D stories and elicited design considerations from three VR developers. We then adapted four selected web-based stories to an immersive format. Finally, we conducted a user study (N=24) to examine egocentric and exocentric viewpoints, active and passive navigation, and the combinations they form.
%
Our results indicated significantly higher preferences for egocentric+active (higher agency and engagement) and exocentric+passive (higher focus on content). We also found a marginal significance of viewpoints on story understanding and a strong significance of navigation on spatial immersion.
\end{abstract}


%%
%% The code below is generated by the tool at http://dl.acm.org/ccs.cfm.
%% Please copy and paste the code instead of the example below.
%%
\begin{CCSXML}
<ccs2012>
   <concept>
       <concept_id>10003120.10003123.10011759</concept_id>
       <concept_desc>Human-centered computing~Empirical studies in interaction design</concept_desc>
       <concept_significance>500</concept_significance>
       </concept>
   <concept>
       <concept_id>10003120.10003121.10011748</concept_id>
       <concept_desc>Human-centered computing~Empirical studies in HCI</concept_desc>
       <concept_significance>500</concept_significance>
       </concept>
   <concept>
       <concept_id>10003120.10003121.10003124.10010866</concept_id>
       <concept_desc>Human-centered computing~Virtual reality</concept_desc>
       <concept_significance>500</concept_significance>
       </concept>
 </ccs2012>
\end{CCSXML}

\ccsdesc[500]{Human-centered computing~Empirical studies in interaction design}
\ccsdesc[500]{Human-centered computing~Empirical studies in HCI}
\ccsdesc[500]{Human-centered computing~Virtual reality}
%%
%% Keywords. The author(s) should pick words that accurately describe
%% the work being presented. Separate the keywords with commas.
% \keywords{Smartphone overuse, intervention design, interaction proxy, input manipulation, gestures}
\keywords{Immersive storytelling, Story navigation, Story viewpoint in immersive environments}

\settopmatter{printfolios=true}
\begin{teaserfigure}
  \includegraphics[width=\textwidth]{Figures/teaser.pdf} 
  \caption{We investigated the effects of different combinations of viewpoints and navigation in immersive storytelling, as demonstrated in four implemented scenarios: (a), (b), (c), and (d). In (a) and (b), the audience is fully immersed in the story scene from an egocentric perspective. Conversely, in scenarios (c) and (d), the audience maintains an exocentric perspective, remaining independent of the story. In (a) and (c), the audience navigates the story manually and actively, following specific instructions. In contrast, (b) and (d) allow audience clicking a controller button, which triggers the story and allows navigation to proceed automatically.}
  \Description{Four Combinations of Viewpoints and Navigations. The first scene (leftmost) depicts the person standing in a classroom setting with several figures seated around tables. The figures appear to be rendered in a basic gray, suggesting that they are part of a virtual environment. The second scene shows the person standing outside of what appears to be a virtual building, interacting with a door or window. The environment has a simplistic, grayish brick design. The third scene presents the person holding a 3D model of the Earth, as though they are interacting with a virtual globe. The final scene (rightmost) shows the person seemingly holding or interacting with a floating model of a cityscape, with buildings arranged on a platform that hovers in the virtual space.}
  \label{fig:teaser}
  \vspace{1em}
\end{teaserfigure}