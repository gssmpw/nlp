\section{Related Works}
\noindent\textbf{Visual Storytelling.}
Visual storytelling is widely used in journalism to help audiences understand data-backed facts through static or dynamic visual elements, such as images, videos, and animations.
Segal and Heer~\cite{segel2010narrative} were among the first to summarize emerging narrative visualization techniques in web-based storytelling, discussing design variations across different genres.
Hullman et al.~\cite{hullman2011visualization} studied how information prioritization affects user interpretation. Kosara and Mackinlay~\cite{kosara2013storytelling} further highlighted the potential of storytelling in visualizations, emphasizing its promising future.
Subsequent research explored design considerations in visual storytelling, proposing guidelines to improve content representation and interaction. For instance, Morais et al.\cite{morais2020showing} proposed a design framework for anthropographic data stories, while Dasu et al.~\cite{dasu2023character} examined the use of characters in data stories to enhance enjoyment and persuasion. Yang et al.~\cite{yang2021design} also explored the application of narrative structures in data stories. These studies connect design practices with human cognition theories, offering valuable insights for visual storytelling design.

Beyond narrative structure, audience engagement and interaction are critical to the story’s impact, persuasion, and memorability.
With advancements in technology, visual storytelling has expanded to include videos, games, and VR/AR platforms~\cite{zhao2023stories,mendez2025immersive,zhou2023data}, offering audiences more interactive experiences than static texts and images alone. VR/AR, in particular, immerses users in virtual spaces, enhancing spatial coherence and unlocking new design possibilities.
This paper focuses on two key elements in VR/AR and visual storytelling: viewpoints and navigation, which shape user perspective and agency, directly influencing immersion, comprehension, and emotional engagement. We aim to identify effective design practices for integrating these aspects into immersive storytelling.

\noindent\textbf{Immersive Experiences.}
The goals of visual storytelling are multifaceted~\cite{amini2018evaluating}. For authors, common objectives include (a) communicating facts and insights, and (b) persuading audiences of the opinions conveyed through the story. For audiences, the primary goal is to gain new knowledge, such as news, insights, and perspectives, while entertainment also plays a significant role, as stories may be consumed for enjoyment.

While 2D screens are effective for visual storytelling, immersive technologies like VR/AR extend stories into 3D, closing the gap between audiences and narratives by placing them in the same space. Immersive experiences offer several advantages.
First, they enhance presence and immersion~\cite{servotte2020vr}. VR/AR can simulate real-world scenarios~\cite{zhao2023leanon} with real-time rendering and embodied interactions~\cite{huang2023embodied,zhu2024compositingvis,in2023table}, or even go beyond reality (e.g., ground-level scaling~\cite{abtahi2019m}).
Second, they improve understanding and retention~\cite{Motejlek2023The, Yildirim2019The}. VR/AR is frequently used in scientific visualizations~\cite{marriott2018immersive}, helping users extract and communicate insights more effectively.
Third, they enhance emotional communication~\cite{kandaurova2019effects}, narrowing the emotional distance between authors and audiences and fostering resonance and reflection on the story's themes~\cite{bahng2020reflexive}.

These benefits align with the core goals of visual storytelling. Therefore, this paper investigates the impact of immersive environments on understanding and spatial immersion.

\noindent\textbf{Immersive Storytelling.}
Immersive storytelling, a broader topic of integrating immersive experiences to different storytellings, has been studied in various other contexts. {In AR, Shin et al.~\cite{shin2019any} studied the two factors: room size and furniture density on a AR crime-solving game, and they suggested more context-awareness such as using virtual objects as substitutes in large and empty rooms. A later survey~\cite{shin2024investigating} furthered this idea, proposing to balance virtual and real experiences through situated user interactions. In VR,} Lee et al. leveraged VR to materialize abstract measure and units so that audiences could connect with their natural experiences~\cite{lee2020data}. Hall et al. focused on synchronous immersive data presentation and communication~\cite{hall2022augmented}. {VR storytelling is particularly relevant in cinematic.} Collen et al. investigated the cinematic techniques used in immersive narrative visualizations of 3D spatial information~\cite{conlen2023cinematic}. {Practically, Wu and Karwas~\cite{wu2024metamorphosis} developed a technique that allows free viewpoint change and with time-based interactivity.} Other immersive storytelling works focusing on specific applications, such as situated awareness of social problems~\cite{assor2024augmented, zhu2024reader} or emotional enjoyment of visualizations~\cite{romat2020dear}. These works contribute to the design of immersive storytelling by highlighting the benefits of immersive technologies we should leverage in specific topics or contexts, but none explored the representation and interaction design on visual jouralism (e.g. scrollytelling) in immersive environment. 

There are some studies focusing on how traditional storytellings on 2D screens can be transported to VR/AR, which are primarily based on the video format ~\cite{yang2023understanding, Zollmann2020CasualVRVideos, Nash2018Virtually}. The adaptation of interactive visual stories, a popular form of visual storytelling, are still underexplored. 

\noindent\textbf{Viewpoint and Navigation in Immersive Environments.}
In traditional digital media (e.g., computer monitors or mobile devices), content is typically viewed from an exocentric (third-person) viewpoint, offering a comprehensive overview of spatial relationships and layout. Immersive technology, while capable of maintaining an exocentric view, uniquely allows for a 360-degree stereoscopic experience from an egocentric (first-person) viewpoint. Despite this capability, the egocentric view is not widely adopted due to mixed findings on its effectiveness. 
Yang et al.~\cite{yang2018maps} found that exocentric views significantly outperformed egocentric views for geographic analysis tasks. Similarly, Kraus et al.~\cite{kraus2019impact} and Yang et al.~\cite{yang2020embodied} noted limitations in egocentric 3D scatterplot visualizations, such as lack of overview and targets being out of view. 
However, the egocentric view has shown promise in multi-view management within immersive environments~\cite{in2024evaluating,davidson2022exploring,satriadi2020maps,luo2022should}, and studies have demonstrated improved memory and knowledge retention in egocentric settings~\cite{krokos2019virtual,yang2020virtual}. 
{Notably, Hoppe et al.~\cite{hoppe2022there} argued for a perspective continuum in immersive environments, finding no difference in users' perceived workload, sense of presence, and engagement between egocentric and exocentric viewpoints in a VR combat game they developed. However, it remains questionable whether this perspective continuum is applicable in immersive storytelling, as storytelling involves different perception and cognitive processes compared to games.} 
Given these mixed results, our research explores the advantages and drawbacks of both exocentric and egocentric viewpoints in the context of visual storytelling.

{Navigation in immersive environments refers to the process of moving to the target location within 3D virtual environments. It is a crucial interaction for exploring 3D spaces in these settings. In the context of immersive storytelling, navigation plays a vital role in allowing users to visually explore different parts of the story and advance the narrative (e.g., a story progresses to the next stage once the viewer reaches a specific location).
Significant efforts have been devoted to developing various navigation techniques, including gesture design~\cite{tursunov2024creating}, utilizing different input modalities~\cite{swidrak2024beyond}, leveraging perception and space manipulation~\cite{dong2021tailored}, and experimenting with different scales~\cite{mirhosseini2019exploration, abtahi2019m}. 
Luca et al.~\cite{di2021locomotion} conducted an extensive survey of these techniques and compiled a Locomotion Vault. However, some of these innovative techniques require additional hardware, while others demand a steep learning curve.
As a result, the mainstream and default navigation techniques adopted by major platforms remain natural walking and teleportation. Considering the accessibility aspect of storytelling, we decided to focus on using mainstream navigation techniques for a broader user group. In many immersive applications, users need to actively navigate the 3D space to complete tasks. However, in storytelling, we have the option to let users passively follow a predefined path without explicit navigation actions (such as walking or teleportation).
While active navigation provides more interactivity, it carries the risk of disorienting the user. In contrast, passive navigation maintains narrative control but may reduce the sense of agency and induce motion sickness. Deciding which method to use is a unique and important consideration for immersive storytelling.
In relation to this topic, Lages and Bowman~\cite{lages2018move} compared physical walking (where the user moves themselves) with grab-and-move techniques (where the user moves the view) in VR, demonstrating that performance may depend on the user's spatial abilities. Although both methods they studied were active, we are particularly interested in the performance of passive navigation when combined with a predefined storyline.}