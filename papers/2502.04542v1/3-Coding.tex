\section{Eliciting Design Considerations for Immersive Story Adaptations}
\label{sec:formative}

\begin{figure*}[t]
\centering
  \includegraphics[width=\linewidth]{Figures/procedure.pdf}
  \caption{Formative Study Flow Diagram~\cite{page2021prisma}. We first collected dynamic visual stories from major news outlets. We then filtered out those without significant 3D spatial informations. With the remaining VR adaptable stories, we elicited four design considerations (DCs) for our later adaptations.}
  \Description{A process flow diagram showing the story filtering process for a VR study. The flow begins with over 1,000 visual stories, from which static infographics are filtered out. A total of 148 dynamic visual stories are selected through Initial Story Collection. These dynamic stories are further filtered, excluding stories lacking sufficient 3D spatial information, resulting in 117 VR adaptable stories. Finally, through Elicitation, four design considerations (DCs) are identified: Main Visual Elements, Story Narratives, Progression Triggers, and Text and Visual Guidance.}
  \label{fig:coding_procedure}
\end{figure*}

This project investigates the effects of viewpoint (ego vs. exo) and navigation (active vs. passive) in immersive storytelling through a user study. Preparing suitable study materials was the crucial first step.

{Visual storytelling can be characterized by four dimensions: narratives, transitions, structure, and the balance between explorability and explanability~\cite{stolper2018data, thudt2018exploration}. With various options in each category, visual storytelling can become quite complex. Since immersive storytelling is an emerging field with limited publicly available resources and few established design guidelines, we decided to develop our immersive stories.
Creating high-quality stories is a complex and challenging task. Therefore, instead of creating stories from scratch, we chose to adapt web-based stories from major news outlets. This approach allows us to ensure story quality while reducing potential confounding factors.
Considering the aforementioned four dimensions, we found that most stories from major news outlets follow a single-scene linear narrative structure with simple and continuous transitions and predominantly feature author-driven explanability. This simplicity is ideal for our initial exploration into immersive storytelling.
We further focused on stories with significant 3D spatial content due to their greater synergy with immersive environments. This approach enabled us to concentrate on the study itself, maintaining story quality while avoiding potential design biases.
}

To guide our adaptations, we collected existing 3D spatial stories from major news outlets and sought input from three VR experts. We compiled their insights into four generic design considerations, which were used for the adaptations discussed in \autoref{sec:adaptation}. An overview of the formative study process is shown in \autoref{fig:coding_procedure}.

In summary, the formative study and adaptation process, described in \autoref{sec:adaptation}, are essential steps for conducting the user study detailed in \autoref{sec:study}. \autoref{fig:procedure} illustrates the relationship between these steps.

The formative study involved three steps: 1) initial story collection, 2) filtering stories based on spatial information, and 3) eliciting design considerations for VR adaptation. Details of our collected stories are provided in the supplementary material.

\textbf{Initial story collection.}
Our first step is to collect abundant web-based stories. To ensure the quality of our visual story dataset, we collected stories from major news outlets and popular online media platforms, including The New York Times (NYT)~\cite{nyt}, The Washington Post~\cite{wapo}, Bloomberg~\cite{bloomberg}, The Guardian~\cite{guardian}, Reuters~\cite{reuters}, National Geographic~\cite{NG}, Financial Times~\cite{FT}, Los Angeles Times~\cite{lat} and The Pudding~\cite{pudding}. Some platforms feature dedicated visual story sections, which we prioritized, such as the Graphics section in NYT~\cite{nyt} and the Visual Stories section in The Washington Post~\cite{wapo}.

We focused on stories published between 2020 and 2024, as recent content tends to incorporate more interactive visuals. We excluded static stories without interactivity or visual transitions, as these did not align with the goals of our user study. 
{From the sources above, we browsed 1017 visual stories in total during the search process. Ultimately, our initial corpus comprised 148 visual stories that are either interactive or feature visual transitions. Most of them from The New YorK Times (102), followed by Reuters (16), The Washington Post (13), The Pudding (7), Financial Times (4), Los Angeles Times (2) and The Guardian (1).}


\textbf{Story filtering based on spatial information.}
Not all visual stories were suitable for our study. For example, animated bar charts may not offer a significantly different experience in an immersive environment compared to a desktop.

Since our goal was to investigate how the unique 3D capabilities of immersive environments impact the storytelling experience, we applied further filtering based on the presence of 3D spatial information. Stories were evaluated using two criteria: 1) sufficient 3D spatial content and 2) coherent spatial transitions within a unified environment. 
{The second requirement limited our scope to single-scene stories.}

For instance, a NYT story explaining COVID-19 transmission in a classroom was deemed suitable, as it contains rich 3D spatial information and smoothly transitions the viewpoint across different parts of the environment. In contrast, the NYT story ``Ukraine's Race to Hold the Line'' uses 2D visuals without enough 3D spatial content. Similarly, the story ``Seeing Earth from Outer Space'' includes 3D elements but presents multiple disjointed spaces, making each new scene appear completely separate from the previous one, complicating our intended study of immersive effects.

By applying these two criteria, we excluded 31 additional stories, resulting in a final set of 117 stories for our VR adaptation study.


\textbf{Elicitation of design considerations for VR adaptation.}
One of the key goals of our formative study was to derive design considerations that could inform the VR adaptation process. 
To accomplish this, we conducted an elicitation study with three experienced VR developers {whose backgrounds were medical VR, computer graphics in VR, and immersive animation design, respectively}. Each developer was assigned a roughly equal number of stories from our collection. They were asked to review their assigned stories individually and describe their intended adaptation in response to three specific questions related to our study factors, along with an open-ended question to explore additional design ideas. Developers were encouraged to provide detailed feedback.
\begin{itemize}
    \item Q1: What is the best way to view this story in an immersive environment? And why?
    \item Q2: Do you prefer to actively move in the 3D space or stand still and let the content move for you? And why?
    \item Q3: What other design considerations do you consider important?
\end{itemize}
We analyzed their responses using affinity mapping to identify common themes and patterns in their feedback.
\begin{figure*}[t]
\centering
  \includegraphics[width=\linewidth]{Figures/design_options.pdf}
  \caption{Four Major Story Components and Corresponding Available Options.}
  \Description{A diagram illustrating key design considerations for immersive storytelling in VR, presented in four categories: 1) Main Visual Elements: Real-world transform (elements positioned based on real-world coordinates) vs. Exo transform (global context positioning). 2) Story Narrative Placement: World-fixed (narrative anchored in the virtual world), Floating panel (text on a floating interface), and Situated (contextual narrative placement). 3) Navigation Triggers: Locomotion (natural walking or teleportation) and Physical interactions (interacting with objects like opening a window). 4) Text and Visual Guidance: Text placement (World-fixed, Floating panel, or Situated) and Visual placement (User-oriented vs. Situated for guiding attention).}
  \label{fig:design_options}
\end{figure*}
Regarding the viewpoint (Q1), \textit{information density} emerged as a key factor in choosing between egocentric and exocentric viewpoints. Developers generally favored an exocentric viewpoint for stories with dense spatial information—where a large amount of detail is concentrated within a limited area. For example, a story about the tall buildings in Manhattan, where the buildings are closely packed in a small space, would benefit from an exocentric perspective. Conversely, an egocentric viewpoint was preferred for stories featuring more dispersed spatial information, where the relative distances between objects are much greater than their individual sizes—such as a story about players on a football field. The developers largely agreed on the criteria for selecting viewpoints.

In contrast, there was no clear consensus on navigation design (Q2). For the same story, developers expressed differing preferences for active versus passive navigation. Some noted that passive navigation, which resembles the "scroll-to-progress" interaction commonly used on webpages, offers familiarity and ease of use. Others argued that active navigation in VR feels more intuitive and immersive, enhancing user engagement. These differing opinions were spread across various stories, without any strong patterns emerging.

In summary, while the developers reached a broad consensus on viewpoint selection, the decision between active and passive navigation remained inconclusive in our formative study.

Overall, we identified in the affinity diagram that experts frequently mentioned four major story components in their responses:
\begin{itemize}
    \item DC1: Main Visual Element. The main visual element serves as the centerpiece of the story, often represented by the primary 3D model or scene.
    \item DC2: Story Narratives. Story narratives refer to the structured sequence of textual content that guides the user through the story.
    \item DC3: Navigation Trigger. To trigger the navigation, the user needs to either move to a specific location or perform certain interactions.
    \item DC4: Text and Visual Guidance. Instructions regarding location and interaction information are necessary to guide the user in progressing through the story.
\end{itemize}

{Among these four design considerations, DC3 and DC4 hold unique importance for immersive stories. While web-based stories typically use straightforward navigation triggers, such as scrolling and clicking that require little to no guidance, immersive stories demand more explicit navigation and guidance cues due to the open-ended interactions audiences can perform. Therefore, it is essential to emphasize specific navigation triggers and provide corresponding guidance when adapting stories for immersive environments.}
These findings of our formative study provide structured guidelines for adapting web-based stories to immersive environments.
