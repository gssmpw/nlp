\section{User Study}
\label{sec:study}

With the four adapted VR stories, we conducted a user study to evaluate the performance of spatial immersion and understanding among different settings of viewpoints and navigations.
Our study was approved by our institution's IRB. 
It implements a within-subject comparison where each participant experiences active and passive navigation methods in both egocentric and exocentric viewpoints. 
Each session lasted about 2 hours and participants were compensated with \$20 for their time.

\subsection{Study Factors and Conditions}

We studied the Ego and Exo viewpoints as well as active and passive navigations, including their individual and combined effects. \autoref{fig:characteristic} shows the characteristics of the four conditions.

\textit{Ego vs. Exo Viewpoint.} The differences between Ego and Exo lies primarily in the visual representations. For example, Ego stories have surrounding visual element around audiences, while visual elements in Exo stories are primarily in front of the participants. There are also different scales of story scene, as Ego stories tend to have a much larger range compared to the space a normal person can occupy. Additionally, the information density in audiences' view is low, but there are out-of-view objects that they may leave out. Exo stories, in contrast, have everything in audiences' view, which increases the information density.

\textit{Active vs. Passive Navigation.} The differences between active and passive are more distinct. Active stories require embodied interactions to push forward the story, and oftentimes the interactions requires precision (e.g. going to a specific location or looking from a specific angle. In passive stories, only a minimum effort (i.e. clicking a button on the controller) is required. Moreover, active stories give audiences a higher level of control by asking them to proceed with story manually. This also increases the amount of visual search necessary for identifying the next step. In passive stories, visual elements transition to their next predefined state automatically, and audiences have less control and visual search in this process.

To systematically test the main factors (i.e., viewpoint and navigation), we include four conditions in our study:
Ego+Active, Ego+Passive, Exo+Active, and Exo+Passive (illustrated in \autoref{fig:teaser}).


\begin{figure*}[t]
\centering
\begin{subfigure}{0.45\linewidth}
  \includegraphics[width=\linewidth]{Figures/view.pdf}
  \caption{}
  \label{fig:view}
\end{subfigure}
\begin{subfigure}{0.45\linewidth}
  \includegraphics[width=\linewidth]{Figures/nav.pdf}
  \caption{}
  \label{fig:nav}
\end{subfigure}
  \caption{Characteristics of Four Conditions. A tick mark indicates that the condition has corresponding characteristics, and the cross mark indicates the absence. If both conditions have the characteristics, they are labeled with circles with size indicating the extent (more or less).}
  \label{fig:characteristic}
  \Description{A comparison table outlining the characteristics of egocentric (Ego) vs. exocentric (Exo) viewpoints and active vs. passive navigation in immersive storytelling. On the left, the table compares Ego and Exo based on four characteristics: surrounding visual elements (Ego includes surrounding elements while Exo does not), scale of story scenes (Ego has larger scenes), spatial information density (higher in Exo), and out-of-view target visibility (available in Ego but not Exo). On the right, the table compares Active and Passive navigation in terms of required embodied interactions (more in Active), precise interaction (more in Active), level of control (higher in Active), and required visual search (more in Active than Passive). Larger circles indicate greater intensity or significance of the characteristic.}
\end{figure*}

\subsection{Study Setup}
Our user study was conducted in our lab space at the university.
Participants experienced our stories using a Meta Quest 3 virtual reality headset, which has a pixel resolution of 2064 x 2208 per eye and 90Hz refresh rate.
The headset was connected to PC via Air Link, allowing it to wirelessly leverage PC computing power.
Participants were allowed to physically move and interact with stories in a 3x3$m^2$ area.
During the study, we set the controllers as the only input devices for stability. We mapped the grip press to the grab interaction and thumbstick forward push to teleportation for both controllers.
This study setup was introduced at the very beginning of the study session for each participant.



\subsection{Participants}
We recruited participants from our university by sending out recruitment emails to mailing lists and Slack channels and selected 24 participants (11 females, 13 males, aged 21 to 31) from all responses.
11 participants were experienced in XR, and 5 were experienced in visualization/storytelling. 3 participants had both expertise.
Participants came from various backgrounds, including computer science, UX design, aerospace engineering, and Human-Computer Interaction.

{As ~\autoref{sec:calibration} mentioned, all four cases were calibrated in story length and overall design languages. 
Each participant experienced the four cases, with the order counterbalanced using a Latin square matrix (ego vs. exo)$\times$(active vs. passive).
Since each story has both an active and a passive version, there were a total of eight possible sequences. Further details can be found in our supplementary materials.
We acknowledged that this within-subjects setting could cause possible learning effects or fatigue, but we believe the difference in the nature of each cases minimizes these confounding factors.}

\subsection{Study Procedure}
Our study procedure includes four major steps:

\noindent{}\textbf{1. Introduction:} We first showed participants the consent form for the study. After they read and signed the form, we introduced the concept of immersive storytelling, ego/exocentric viewpoints, and active/passive navigation methods.

\noindent{}\textbf{2. Training:} We built a training scene that helps participants learn necessary interactions (e.g., grab, scale, teleportation) and story setup (e.g., narrative text panels, visual guidance design) in our VR stories. 
The training scene is a mockup of four real stories that use the same design language.

\noindent{}\textbf{3. Experiencing the story:} After the training session, participants were to experience the four VR stories. None of them had read the original story before. After participants finished each story, we asked them to complete 3 tasks: (a) completing a survey of presence and perceived workload of experiencing the VR story; (b) describing their understanding of the story's main idea and important details; and (c) drawing the story scene. The data we want to collect through these tasks will be described in Sec \ref{subsec:DataCollection}.

\noindent{}\textbf{4. Final interview:} Finally, after participants experienced all the stories, they were asked to discuss the differences among the four stories in terms of their engagement, help of understanding and other user experiences. 
The purpose of this interview was to collect subjective feedback on each story and find out their common considerations.

\subsection{Task Data Collection}
\label{subsec:DataCollection}
Through the study, we aimed to investigate the impact of ego/exocentric viewpoints and active/passive navigation on spatial immersion and understanding of VR stories. 
For spatial immersion, we collected responses to the Presence Questionnaire \cite{berkman2021group} and NASA-TLX \cite{hart1988development} to measure spatial presence and perceived workload, respectively. 

To evaluate the understanding, we focused on the story content comprehension and spatial information depiction. 
The total and categorical points of these two tasks for each story was the same since the their story lengths are similar.
For each story. we designed a set of comprehension rubrics that incorporated three sets: (a) \textit{main idea memorization} (b) \textit{detail memorization} (e.g. numbers, names, quotes or definitions, etc.); and (c) \textit{random inspirations} (e.g. personal opinions, analyses or conclusions that are not presented directly in the story content).
The rubrics is attached in the supplementary materials.

We asked participants to describe their understanding of the main idea and important details, which was then compared to our rubrics. 
We counted the number of facts in the rubrics that were mentioned by participants as their scores.
We also recorded the facts that do not appear in the rubrics.
If a fact was frequently mentioned (e.g., more than half of the participants mentioned it), we reconsidered the importance of this fact and added it to the rubrics if the majority of the study team thought it was important.
 

As for the spatial information depiction task, we asked participants to draw a picture of the story scene to measure their memorability of spatial details. We specifically mentioned 2 requirements: (a) \textit{object existence}: participants should capture as many spatial elements in the story as possible, including the objects, annotations, and the viewer path (the change of attention); and (b) \textit{object transform}: for each spatial element, provide its position, rotation/orientation and scale/size as accurate as possible.
The rubrics is attached in the supplementary materials.

We did not expect participants to draw the exact shape of elements.
They were encouraged to use basic geometries (e.g. lines and rectangles) and text descriptions to depict the elements.
 

Finally, we logged the story session duration. This was to understand if the four combinations of conditions would cause any differences in the time audiences stayed in the immersive stories.




\subsection{Hypotheses}
\label{subsec:hypotheses}
Based on the characteristics in \autoref{fig:characteristic}, we proposed the following hypotheses.


\textbf{Story Session Duration.}
We hypothesize that active navigation will result in longer story session durations compared to passive navigation. The interactive elements in active navigation require users to engage physically with the environment, adding time to the experience. Prior studies have shown that increased interaction in VR generally extends session time due to higher engagement levels and the need for precise actions in immersive environments~\cite{lages2018move, servotte2020vr}. This additional engagement likely leads participants to spend more time exploring the environment and interacting with the content, especially in active conditions.

Additionally, Ego viewpoints are likely to further increase session duration, as spatial navigation in egocentric environments tends to be more complex and challenging, requiring users to explore the scene more actively, which can lead to extended engagement~\cite{van2018virtual}.

\textbf{Story Understanding.} 
We hypothesize that active navigation will outperform passive navigation on both content comprehension and spatial information depiction, as embodied interaction has been shown to improve cognitive engagement and memory retention~\cite{wilson2002six, makransky2019adding}. Active navigation encourages physical movement, which enhances attention to both content and spatial information, leading to better comprehension and recall~\cite{servotte2020vr}. 

We also expect that exocentric viewpoints will result in better content comprehension because they provide an overview of the entire scene, facilitating the integration of key story elements and improving factual recall~\cite{yang2018maps}.
Egocentric viewpoints will lead to better spatial understanding, as being immersed in the scene helps participants build stronger spatial awareness and memory, which is supported by the "memory palace" technique~\cite{legge2012building, krokos2019virtual}.


\textbf{Spatial Immersion.} 
We expect that Ego viewpoints combined with active navigation will result in the highest levels of spatial immersion. Active navigation increases the sense of presence and control, while Ego viewpoints create a fully immersive experience by surrounding participants with 3D elements~\cite{slater2009place}. Prior studies suggest that immersion is enhanced when users feel directly involved in the environment, with Ego viewpoints offering a more embodied experience~\cite{cummings2016immersive}.

While active navigation may increase perceived workload due to the effort required to interact with the environment, we hypothesize that the enhanced sense of presence and engagement outweighs the additional cognitive demand, especially when paired with the Ego viewpoint~\cite{hart1988development}. This combination offers an optimal balance between immersion and interactivity, despite a potential trade-off with workload~\cite{servotte2020vr}.