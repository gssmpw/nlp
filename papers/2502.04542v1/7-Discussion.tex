\section{Key Findings and Discussion}
In this section, we summarize some key findings and compare our results with prior works to discuss the common and unique findings. We also discuss different expectations on immersive stories based on our particiapnts feedback.

\subsection{How does viewpoints influence story memorability?}
Our study shows that the viewpoint is a more decisive factor than the navigation in story memorability. 
\textbf{The exocentric viewpoint is slightly better than the egocentric one in the overview and detail memorization.} 
This result matches the disccovery of some prior works, in which the exocentric viewpoint benefits analytical tasks in immersive environments~\cite{yang2018maps, kraus2019impact, yang2020embodied}, but the advantage in our study is not as significant.
Our anticipated reason is that the exocentric viewpoint provide comprehensive information within audiences' view, which alleviates the problem of targeting story elements in space. However, this also causes a larger information density and more occlusions than the egocentric viewpoint. The relatively smaller scale of visual representations in exo viewpoints increases the attention to the story textual content received and prolongs the retention time in the memory so that they can did a better content comprehension. However, at the same time, audiences could become less sensitive to content transitions and therefore they might miss some important details. 
Overall, our results indicate that the exocentric viewpoint can be a better design if the primary goal is to promote the understanding and retention of narratives, and aforementioned problems should be considered to further improve the story.

 
In contrast, \textbf{the egocentric viewpoint demonstrated better memorability on spatial information than exocentric viewpoint.}
Similar result is also found in Krokos et al.'s~\cite{krokos2019virtual} and Yang et al.'s~\cite{yang2020virtual} work. 
\label{subsec:dis_view}
Objects in an egocentric viewpoin are often larger and more visible than the exocentric one. Resultedly, audiences can better recognize transforms (i.e. position, rotation and scale) and capture their changes.
More importantly, audiences are involved as part of a egocentric story. This enable them to use themselves as a frame of reference to infer transforms. In exocentric stories, they have to refer to one central element (e.g. the base map in \autoref{subsec:case1}) whose signficant transform change can greatly distract the spatial information perception since audiences need to find a new frame of reference every time.
Thus, the egocentric viewpoint is beneficial for those authors who want to convey understandings and perceptions related to the changes 3D spatial transforms (e.g. the renovation process of an office).


\subsection{How does navigation influence spatial immersion?}
As for the spatial immersion in our cases, navigation plays a more critical role than viewpoint. \textbf{Active navigation significantly enhances the sense of presence compared to passive, but their disparity of perceived workload varies between egocentric and exocentric conditions}. Specifically, we observed a substantial increase in workload for exocentric stories, while this effect was not as pronounced for egocentric ones.

This finding aligns with Lages et al.'s study~\cite{lages2018move}, which showed that walking-based navigation improved performance for users who lacked proficiency in VR manipulation. Our results suggest that the need for higher precision in finding specific angles in exocentric scenarios (e.g., \autoref{subsec:case1}, \autoref{subsec:case2}, and the grab-and-move tasks in Lages et al.'s study) may explain the increased workload. The ``changing frame of reference'' issue discussed in \autoref{subsec:dis_view} adds to the demand for spatial awareness and motor skills to achieve the correct angle, resulting in higher effort and frustration, as reflected in the study.

In contrast, egocentric stories primarily use teleportation for navigation, where the most challenging task is locating the target position. This task is not much more demanding than clicking a controller button for most participants. Consequently, we did not observe a significant increase in workload for active navigation in egocentric stories.

When designing navigation for VR stories, authors should consider the story's viewpoint based on these findings. However, focusing solely on individual navigation steps is insufficient. Ferguson et al.~\cite{ferguson2020role}, in their investigation of VR educational games, recommend that active navigation should seamlessly integrate with the story content and involve minimal learning curves. We argue that natural active navigation in VR storytelling should combine story progression with user exploration.

\subsection{What are audiences' expectations on VR stories?}
From our study, we identified a significant difference in audience priorities, which can be classified as either \textbf{information-oriented} or \textbf{experience-oriented}. This distinction greatly influences their expectations for VR stories.

Information-oriented audiences focus on discovering and understanding detailed content. They prefer full control over the information they consume and favor clear, structured layouts. This group tends to prefer author-driven experiences, where the narrative is pre-designed to ensure they do not miss key details. For these users, an exocentric viewpoint and passive navigation are ideal, as they offer greater information clarity and a well-organized, linear progression with minimal interaction.

On the other hand, experience-oriented audiences value a more user-driven experience. They seek control over VR elements and prefer embodied interactions, even if it means potentially missing parts of the story. These users often desire a personalized experience that supports multiple storylines for free exploration. An egocentric viewpoint and active navigation are better suited to this group, as they encourage deeper engagement. These audiences can be motivated to ``unlock'' different storylines through interactions, maximizing their freedom to explore the narrative.

{\subsection{Summary and Generalizability}}
{Our study underscores the significant impact of viewpoint and navigation on enhancing the memorability and spatial immersion of immersive stories. 
Among the tested conditions, Ego+Active and Exo+Passive were the most effective. The Ego+Active condition enhanced user immersion and creativity by allowing participants to actively explore and interact within the story, providing a strong sense of presence without significantly increasing the perceived workload. 
In contrast, the Exo+Passive condition improved content comprehension and recall by concentrating information and reducing cognitive load, making it ideal for users seeking a focused and efficient storytelling experience.}

{Immersive storytelling varies significantly across fields such as journalism, education, and entertainment, each with distinct objectives and audience expectations. 
In journalism, where clarity and factual accuracy are paramount, employing an ego viewpoint can enhance spatial understanding and engagement, making complex stories more accessible and memorable. 
In educational contexts, balancing ego and exo viewpoints can promote active learning and critical thinking by aligning storytelling approaches with educational goals. 
In entertainment, where immersion and emotional engagement are crucial, optimizing the combination of viewpoints and navigation styles can deepen the audience's connection to the narrative, enhancing their overall experience. 
Our preliminary exploration offers practical implications and guidance for making design decisions regarding viewpoint and navigation in immersive storytelling for these applications.}


{Additionally, future immersive stories could benefit from offering flexible viewpoints and navigation options to meet diverse audience needs for information and experiences, providing a personalized experience. 
For example, audiences might use an exocentric viewpoint to gain an overview and then ``zoom in" on areas of interest for more detailed information from an egocentric view. 
Similarly, active navigation can be advantageous for audiences proficient in VR who seek more engagement and immersion, while passive navigation provides a suitable way for those less familiar with VR or who prefer to focus solely on story content to control story progression.}

\section{Limitations and Future Work}
We adapted four selected web-based stories into immersive environments. 
To guide this adaptation, we conducted a formative user study to identify critical design considerations and explored various design options. Our exploration was not exhaustive; it aimed to facilitate the adaptation process by focusing on standard, off-the-shelf VR solutions. 
{This project concentrated on specific aspects of viewpoint and navigation: ego vs. exo and active vs. passive. While we believe these are unique and essential areas to investigate, we acknowledge a much broader design space for immersive storytelling beyond these aspects. 
For instance, more sophisticated VR locomotion techniques~\cite{di2021locomotion}, such as redirected walking, could enhance the active navigation experience.}
Similarly, advanced visual guidance methods~\cite{lange2020hivefive}, such as those for out-of-view scenarios~\cite{petford2019comparison, lin2021labeling} and multisensory approaches~\cite{melo2020multisensory}, offer additional potential for enhancing user experience. 
We did not incorporate these techniques, as our primary focus was the user study and providing easily accessible techniques available on standard platforms. 
Building a comprehensive design space to inform future immersive story design is crucial, and we consider this an area for future work.


Our study materials were adapted from web-based stories that were initially designed for web browsers and contained design patterns specifically tailored for web environments~\cite{bach2018narrative}. 
It is unclear whether these design patterns are still applicable in immersive environments or whether we need to develop unique and more appropriate design patterns for immersive stories. Studying the adaptation or innovation of these design patterns for immersive stories has the potential to bring about a more native immersive experience. 
{Furthermore, the web-based stories we adapted primarily follow a linear, single-scene structure. However, complex stories with multi-path or branching narratives also exist. 
While our studied scenario can serve as a foundational unit for these more complex stories, and our findings should be partially applicable, especially within their individual scenes. There is a need to explore transitions between scenes and more intricate active navigation, such as manipulating and selecting narrative branches.}

Through our study, we found that exocentric and egocentric viewpoints each have their own strengths and limitations. We considered viewpoint as an exclusive design element, testing it in a controlled manner. There is a great opportunity to design immersive stories that leverage both viewpoints to complement each other. In movies and data videos, cinematic techniques like camera movements and zooms have been used to switch focuses and direct audiences' attention~\cite{amini2015understanding, xu2022from, li2023geocamera}. Similar incoporation such as viewing some story pieces from an exocentric view and others from an egocentric view is also promising to investigate. Our study results can provide guidelines on which parts of the story to use which viewpoint. To achieve this, smooth transition techniques are necessary to allow seamless and meaningful switching between viewpoints, which we aim to explore in our next research phase.


Finally, one important aspect of active vs. passive interaction is the agency of the readers. In our study, readers were given the freedom to navigate by themselves with guidance. However, there are other types of freedom that could be provided to readers, such as allowing them to freely decide their focus or even rearrange the storyline. This may empower readers but also risks them getting lost, particularly due to the more flexible interactivity in immersive environments. Balancing exploration and narrative control is a fundamental question in storytelling~\cite{thudt2018exploration}. 
Studying the appropriate threshold or mechanism for this balance is an important research thread in immersive storytelling.


\section{Conclusion}
In this paper, we presented a user study investigating the effect of viewpoint and navigation on spatial immersion and understanding in immersive stories. To prepare the study materials, we first elicited design considerations from collected 3D spatial web stories. We then adapted four selected web stories to immersive environments based on these design considerations. Finally, we conducted a user study to empirically investigate the effect of viewpoints and navigation on spatial immersion and understanding.
Our results showed a marginal significance of viewpoints on story understanding and a strong significance of navigation on spatial immersion, with preferences for the Ego$+$Active and Exo$+$Passive cases. 
Our study indicates the associations between viewpoints and navigation and provides a preliminary exploration of their individual and combined effects on VR storytelling, which could benefit future immersive storytelling designs.
