
\section{Undecidability of \acs{PHLTL}}
\label{appendix:undec}

We report here the full proof of Theorem~\ref{thm:phltl-undecidable}.

Let $\mathcal{F} = (Q, \Sigma, \delta, q^0, q^\mathit{acc})$ be a \ac{PFA}~\cite{Paz71}, 
where $Q$ is a set of states, $\Sigma$ is the input alphabet, 
$\delta : Q \times \Sigma \rightarrow \distr(Q)$ is the transition probability function, 
$q^0 \in Q$ is the initial state, and $q^\mathit{acc} \in Q$ is an absorbing accepting state ($q^0 \neq q^\mathit{acc}$).
\ac{PFA} runs are defined analogously to \ac{MDP} paths by using input symbols as actions,
and the probability associated to a run is the product of the probabilities of all transitions it takes.
Let $\mathbb{P}_{\mathcal{F} (\omega)}$ be the probability 
that $\mathcal{F}$ is in $q^\mathit{acc}$ after reading string $\omega \in \Sigma^*$. 
The language accepted by a \ac{PFA} is defined with respect to a rational threshold $\lambda$: 
$L_{\lambda}(\mathcal{F}) = \{ \omega \in \Sigma^{+} \mid 
\mathbb{P}_{\mathcal{F}}(\omega) > \lambda \}$.
The emptiness problem for \acp{PFA} is known to be undecidable ~\cite{NasuH69,condon1989complexity,Freivalds81}.

\newtheorem*{theorem-undec}{Theorem 3.1}
\begin{theorem-undec}
    \acs{PHLTL} model checking is undecidable.
\end{theorem-undec}
\begin{proof}
Given $\mathcal{F}$ and $\lambda \in (0,1)$, we show how to construct 
a \ac{MDP} $M_\mathcal{F}$ and a formula $\Phi_\mathcal{F}$ such that 
$L_{\lambda}(\mathcal{F}) \neq \varnothing$ iff $M_\mathcal{F} \models \Phi_\mathcal{F}$.
We set $\Sigma_{\bot} = \Sigma \cup \{ \bot \}$.

$M_\mathcal{F}$ is the disjoint union of two \acp{MDP} $M_1$ and $M_2$.
Actions are \ac{PFA} input symbols, and action sequences of the two \acp{MDP} are strings read by $\mathcal{F}$.
$M_1$ mimics a run of $\mathcal{F}$, while keeping track of the last read symbol.
$M_2$ is a deterministic \ac{FSA} that just guesses $M_1$'s action sequence.
Formally, $M_\mathcal{F} = (S, \Act, \mpm, L)$ is defined as follows: 
\begin{itemize}
    \item % The states are
      $S = S_1 \cup S_2$,
      where $S_1 = \{ q_{\alpha} \mid q \in Q, \alpha \in \Sigma_{\bot} \}$
      contains $\mathcal{F}$'s states labeled with input symbols meant to track the last read symbol,
      while $M_2$'s state set $S_2 = \Sigma_{\bot}$
      consists of input symbols.
    \item $\Act = \{ \Sigma \mid \alpha \in \Sigma \}$:
      actions are shared among $M_1$ and $M_2$ and are meant to be synchronized. 
    \item The transition probability function $\mpm = \mpm_1 \cup \mpm_2$ is as follows:
    \begin{itemize}
        \item $\mpm_1(q_\alpha, \beta, p_{\beta}) = \delta(q, \beta, p)$
            for all $q,p \in Q$, $\alpha \in \Sigma_{\bot}$, $\beta \in \Act$,
            while $\mpm_1(q_\alpha, \beta, p_{\gamma}) = 0$ for each $\gamma \neq \beta$;
        \item $\mpm_2(\alpha, \beta, \beta) = 1$
            for all $\alpha \in \Sigma_{\bot}$ and $\beta \in \Act$,
            while $\mpm_2(\alpha, \beta, \gamma) = 0$
            for each $\gamma \neq \beta$;
    \end{itemize}  
    \item The atomic propositions are $AP = \Sigma_{\bot} \cup \{ t_{\alpha} \mid \alpha \in \Sigma \}$,
        where $t$-propositions mark the acceptance state $q^\mathit{acc}$.
        %are used to synchronize $M_1$ and $M_2$ at the end of the simulated \ac{PFA} run.
    \item The labeling function $L = L_1 \cup L_2$ is such that, for all $\alpha \in \Sigma_{\bot}$:
        \begin{align*}
            &L_1(q_{\alpha}) = \{ \alpha \} \text{ for $q \neq q^\mathit{acc}$} &
            &L_1(q^\mathit{acc}_{\alpha}) = \{t_\alpha \} \\
            &L_2(\alpha) = \{ \alpha \} &
            &
        \end{align*}
\end{itemize}

Formula $\Phi_\mathcal{F}$ is the following:
\begin{multline*}
\label{undec-formula}
\exists \, (\pv_1 \pv_2) \, . \, \forall \sv_{1} \in \{q^0_\bot \} (\sigma_1) \forall \sv_{2}\in \{ \bot \}(\sigma_2)  \, . \, \,  \\
\lprobsym \biggl( \Bigl( \bigvee_{\alpha \in \Sigma_{\bot}} \alpha_{\sv_{1}} \land \alpha_{\sv_{2}} \Bigr) \mathbin{\mathcal{U}} \Bigl( \bigvee_{\alpha \in \Sigma} {t_\alpha}_{\sv_{1}} \land {\alpha}_{\sv_{2}} \Bigr) \biggr) > \lambda
\end{multline*}

The formula considers two policies:
$\sigma_1$ controls $M_1$ and starts from $q^0_\bot$,
while $\sigma_2$ controls $M_2$ and starts from $\bot$.
The left-hand-side of the until operator forces them to take the same action at each time step,
while the right-hand-side prescribes 
$M_1$ to reach $q^\mathit{acc}$ (labeled by $t$), and that the last action played just before reaching $q^\mathit{acc}$ must be synchronized as well.
Thus, the action sequence taken by the two \acp{MDP} until $M_1$ reaches $q^\mathit{acc}$
can be seen as the string read by the \ac{PFA}.

We argue that it is possible to construct an accepting string for $\mathcal{F}$
from a pair of controllers $(\sigma_1, \sigma_2)$ satisfying formula $\Phi_\mathcal{F}$ on $M$, and \emph{vice versa}.
A planning algorithm computing them, or stating that they do not exist, 
would then solve the \ac{PFA} emptiness problem, which is undecidable.

Since $M_2$ is a deterministic \ac{FSA}, 
we can assume w.l.o.g.\ that $\sigma_2$ is deterministic, 
and corresponds to a finite string $\omega \in \Sigma^{+}$.
Since it satisfies the probability constraint of formula $\Phi_\mathcal{F}$,
there is a set of paths in $M_1$ that 
i) play actions $\nu \alpha$ for some prefix $\nu \alpha$ of $\omega$, and, following action sequence $\nu \alpha$, ii) reach $q^\mathit{acc}_{\alpha}$  with probability greater than $\lambda$. 
So, $\mathcal{F}$ accepts $\nu \alpha$ with probability greater than $\lambda$.
On the other hand, a string $\nu \alpha \in \Sigma^{+}$ 
that is accepted by $\mathcal{F}$ maps to two policies that are a witness for $\Phi_\mathcal{F}$.
Let $n = |\nu \alpha|$ be the length of such string.
Consider $\sigma_1 = \nu \alpha$ and $\sigma_2 = \nu\alpha$. 
These two policies are such that 
i) all paths in $M_1$ play the same actions as the only path in $M_2$; 
ii) the set of paths in $M_1$ that reach $q^\mathit{acc}_{\alpha}$ after $n$ steps has probability mass greater than $\lambda$.
iii) $M_2$ reaches $\alpha$ after $n$ steps. 
\end{proof}
The same result holds in case of infinite action sequences , or algorithms that create such finite or infinite strings such as finite-state controllers, and if we replace the strict inequality $>$ with $\geq$.