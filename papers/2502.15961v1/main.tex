\documentclass[journal]{IEEEtran}
\usepackage{amsmath,amsfonts}
% \usepackage{algorithmic}
% \usepackage{algorithm}
\usepackage[ruled, linesnumbered,lined,boxed,commentsnumbered]{algorithm2e}
\usepackage[dvipsnames,x11names,table]{xcolor}
\usepackage{svg}
% \usepackage[ruled,vlined]{algorithm2e}
\usepackage[
%hidelinks,
%breaklinks,
colorlinks,
linkcolor=blue,
citecolor=blue,
filecolor=red,
urlcolor=DodgerBlue4]{hyperref} % RedOrange DodgerBlue4 SeaGreen4 LightSteelBlue4
% \usepackage{hyperref}
% \hypersetup{
%     colorlinks=true,
% }
\usepackage{array}
% \usepackage[caption=false,font=normalsize,labelfont=sf,textfont=sf]{subfig}
% \usepackage[font=footnotesize,labelsep=period]{caption}[2022/02/20]
\usepackage[font=footnotesize,labelsep=period]{caption}
\usepackage{subcaption}
\usepackage{textcomp}
\usepackage{stfloats}
\usepackage{url}
\usepackage{verbatim}
\usepackage{duckuments}
\usepackage{graphicx}
\usepackage{balance}
\usepackage{cite}
\usepackage{tikz}
\usetikzlibrary{shapes.geometric} % For geometric shapes like diamond
\usepackage{upgreek}
\hyphenation{op-tical net-works semi-conduc-tor IEEE-Xplore}
% \bibliographystyle{IEEEtran}
\bibliographystyle{ieeetr}
% updated with editorial comments 8/9/2021

% Define some commenting commands
\usepackage{xargs}                      % Use more than one optional parameter in a new commands
% \usepackage[colorinlistoftodos,prependcaption,textsize=tiny]{todonotes}
% \newcommandx{\unsure}[2][1=]{\todo[linecolor=red,backgroundcolor=red!25,bordercolor=red,#1]{#2}}
% \newcommandx{\info}[2][1=]{\todo[linecolor=blue,backgroundcolor=blue!25,bordercolor=blue,#1]{#2}}
% \newcommandx{\improvement}[2][1=]{\todo[linecolor=Plum,backgroundcolor=Plum!25,bordercolor=Plum,#1]{#2}}
% \newcommandx{\thiswillnotshow}[2][1=]{\todo[disable,#1]{#2}}

% \newcommand{\hidenotes}{true}
% \setlength {\marginparwidth }{1.4cm} % Remove later!! TODO

\newcommand{\xxnote}[3]{}
\newcommand{\bradynote}[1]{}
\newcommand{\BM}[1]{{{{\color{Green} BM: #1}}}}
\newcommand{\NS}[1]{{{{\color{Red} NS: #1}}}}
\newcommand{\nayananote}[1]{}
% \newcommand{\bradyinfo}[1]{}
% \newcommand{\bradyunsure}[1]{}
% \newcommand{\bradyimprove}[1]{}

\ifx\hidenotes\undefined
  \usepackage{color}
  \renewcommand{\xxnote}[3]{\color{#2}{#1: #3}}
\fi

\ifx\hidenotes\undefined
    \renewcommand{\bradynote}[1]{\todo{BM: #1}}
    \renewcommand{\nayananote}[1]{\info{Nay: #1}}
    % \renewcommand{\bradyunsure}[1]{\unsure{BM: #1}}
    % \renewcommand{\bradynote}[1]{\todopicture{BM: #1}}
    % \renewcommand{\bradyimprove}[1]{\improvement{BM: #1}}
\fi

\definecolor{tabfirst}{rgb}{1, 0.7, 0.7} % red
\definecolor{tabsecond}{rgb}{1, 0.85, 0.7} % orange
\definecolor{tabthird}{rgb}{1, 1, 0.7} % yellow


\newcommand{\PlannerName}{\mbox{IA-TIGRIS}}
\newcommand{\PlannerNameSpaced}{\mbox{IA-TIGRIS }}
% Name for algorithm. EUPHRATES, AI-TIGRIS, IA-TIGRIS, TIGRIS++, TIGRIS-2, 

\begin{document}


% \title{EUPHRATES: An Adaptive and Incremental Sampling-based Algorithm for \\ Online Informative Path Planning}
% \title{IA-TIGRIS: Online, Incremental, Adaptive, Sampling-based, Informative, Path Planner}
% \title{Invariant Smoother for Legged Robot State
% Estimation With Dynamic Contact Event Information} % Estimation to Information is one line
% \title{IA-TIGRIS: An Incremental and Adaptive Sampling-based Planner for Informative Path Planning} % Next best
\title{\PlannerName: An Incremental and Adaptive Sampling-Based Planner for Online \\Informative Path Planning}
% \title{IA-TIGRIS: Incremental, Adaptive Sampling-based Planner for Online Informative Path Planning}
% \title{Online Informative Path Planning using an Adaptive and Incremental Sampling-based Planner}
% \title{AIIASAIPP: Adaptive, Incremental, Informed, Awesome, Sampling-based Algorithm for Informative Path Planning}
% TIGRIS: An Informed Sampling-based Algorithm for Informative Path Planning
% TIGRIS (Tree-based Information Gathering using Informed Sampling)
% An Efficient Sampling-Based Method for Online Informative Path Planning in Unknown Environments
% An informative path planning framework for UAV-based terrain monitoring
% Long-horizon Robotic Search and Classification using Sampling-based Motion Planning
% Geoff algorithm Rapidly-exploring Adaptive Search and Classification (ReASC)
% Euphrates
% Tree-based Information Gathering using Informed Sampling Adaptive, Incremental, Online

\author{Brady Moon$^{1}$, Nayana Suvarna$^{1}$, Andrew Jong$^{1}$, Satrajit Chatterjee$^{2}$, Junbin Yuan$^{3}$, and Sebastian Scherer$^{1}$
        % <-this % stops a space
\thanks{This work is supported by the Office of Naval Research (Grant N00014-21-1-2110). This material is based upon work supported by the National Science Foundation Graduate Research Fellowship under Grant No. DGE1745016.}% <-this % stops a space
\thanks{$^{1}$Authors are with the Robotics Institute, School of Computer Science at Carnegie Mellon University, Pittsburgh, PA, USA
{\tt\footnotesize  \{bradym, nsuvarna, ajong, basti\}@andrew.cmu.edu}}%
\thanks{$^{2}$Author is with the GRASP Lab at the University of Pennsylvania, Philadelphia, PA, USA {\tt\footnotesize satrajit@seas.upenn.edu}}%
\thanks{$^{3}$Author is with the Mechanical Engineering Department at Carnegie Mellon University, Pittsburgh, PA, USA {\tt\footnotesize junbiny@andrew.cmu.edu}}%
% \thanks{Manuscript received September 19, 2023.}
}

% The paper headers
% \markboth{Journal of \LaTeX\ Class Files,~Vol.~14, No.~8, August~2021}%
% {Shell \MakeLowercase{\textit{et al.}}: A Sample Article Using IEEEtran.cls for IEEE Journals}

% \IEEEpubid{0000--0000/00\$00.00~\copyright~2021 IEEE}
% Remember, if you use this you must call \IEEEpubidadjcol in the second
% column for its text to clear the IEEEpubid mark.

\maketitle

\begin{abstract}
%Informative path planning is an important and challenging problem in robotics that remains to be solved in a manner that allows for wide-spread implementation and real-world practical adoption. Among various reasons for this, one is the lack of approaches that allow for informative path planning in high-dimensional spaces and non-trivial sensor constraints. In this work we present a sampling-based approach that allows us to tackle the challenges of large and high-dimensional search spaces. This is done by performing informed sampling in the high-dimensional continuous space and incorporating potential information gain along edges in the reward estimation. This method rapidly generates a global path that maximizes information gain for the given path budget constraints. We discuss the details of our implementation for an example use case of searching for multiple objects of interest in a large search space using a fixed-wing UAV with a forward-facing camera. We compare our approach to a sampling-based planner baseline and demonstrate how our contributions allow our approach to consistently out-perform the baseline by 18.0\%. With this we thus present a practical and generalizable informative path planning framework that can be used for very large environments, limited budgets, and high dimensional search spaces, such as robots with motion constraints or high-dimensional configuration spaces.


% Context
Planning paths that maximize information gain for robotic platforms has wide-ranging applications and significant potential impact. 
% Need and gap
To effectively adapt to real-time data collection, informative path planning must be computed online and be responsive to new observations. 
% Task (what we did)
In this work, we present \PlannerName, an incremental and adaptive sampling-based informative path planner that can be run efficiently with onboard computation. Our approach leverages past planning efforts through incremental refinement while continuously adapting to updated world beliefs. We additionally present detailed implementation and optimization insights to facilitate real-world deployment, along with an array of reward functions tailored to specific missions and behaviors.
% Findings
Extensive simulation results demonstrate \PlannerNameSpaced generates higher-quality paths compared to baseline methods.
We validate our planner on two distinct hardware platforms: a hexarotor UAV and a fixed-wing UAV, each having unique motion models and configuration spaces.
% conclusion (what findings mean) and perspectives (future work)
Our results show up to a 41\% improvement in information gain compared to baseline methods, suggesting significant potential for deployment in real-world applications. 
Project website and video at \href{https://ia-tigris.github.io}{ia-tigris.github.io}.
% Project page: \href{https://ia-tigris.github.io}{https://ia-tigris.github.io}.
% Project page for this work at \href{https://ia-tigris.github.io}{ia-tigris.github.io}.
% \href{https://tigris.github.io}{[Project Page]}
% [\href{https://tigris.github.io}{Website} \| \href{https://tigris.github.io}{Video}]


% Informative path planning for robotic platforms has broad applications, but requires efficient online computation to adapt to real-time observations. We present IA-TIGRIS, an incremental and adaptive sampling-based planner that efficiently operates with onboard computing resources. Our approach leverages past planning efforts through incremental refinement while continuously adapting to updated world beliefs. We develop mission-specific reward functions and provide detailed implementation insights to facilitate real-world deployment. Extensive simulation results demonstrate IA-TIGRIS generates higher-quality paths compared to baseline methods. We validate our planner on two distinct platforms - a hexarotor and fixed-wing UAV - with different motion constraints and configuration spaces.

% We rigorously test our method in simulation and quantify the benefits of its algorithm features and as well as overall planner performance. 
% an array of reward functions that can be chosen for different missions and behaviors. This works also includes key implementation details and modifications to our previous work that increases the efficiency and performance of the core planning loop. 

% Our work builds upon our previous IPP algorithm, TIGRIS, to make it more efficient, iterative, adaptive, and incorporate nuanced reward functions. \bradynote{Trim the background (move to intro), outline our method more. Look at my other good examples.}

\end{abstract}

\begin{IEEEkeywords}
Aerial Systems: Perception and Autonomy, Motion and Path Planning, Reactive and Sensor-Based Planning, Field Robots
\end{IEEEkeywords}

% Two main papers to reference for structure and testing
% file:///home/moon/Downloads/s10514-020-09903-2-2.pdf
% https://journals.sagepub.com/doi/full/10.1177/0278364917709507
\section{Introduction}
Backdoor attacks pose a concealed yet profound security risk to machine learning (ML) models, for which the adversaries can inject a stealth backdoor into the model during training, enabling them to illicitly control the model's output upon encountering predefined inputs. These attacks can even occur without the knowledge of developers or end-users, thereby undermining the trust in ML systems. As ML becomes more deeply embedded in critical sectors like finance, healthcare, and autonomous driving \citep{he2016deep, liu2020computing, tournier2019mrtrix3, adjabi2020past}, the potential damage from backdoor attacks grows, underscoring the emergency for developing robust defense mechanisms against backdoor attacks.

To address the threat of backdoor attacks, researchers have developed a variety of strategies \cite{liu2018fine,wu2021adversarial,wang2019neural,zeng2022adversarial,zhu2023neural,Zhu_2023_ICCV, wei2024shared,wei2024d3}, aimed at purifying backdoors within victim models. These methods are designed to integrate with current deployment workflows seamlessly and have demonstrated significant success in mitigating the effects of backdoor triggers \cite{wubackdoorbench, wu2023defenses, wu2024backdoorbench,dunnett2024countering}.  However, most state-of-the-art (SOTA) backdoor purification methods operate under the assumption that a small clean dataset, often referred to as \textbf{auxiliary dataset}, is available for purification. Such an assumption poses practical challenges, especially in scenarios where data is scarce. To tackle this challenge, efforts have been made to reduce the size of the required auxiliary dataset~\cite{chai2022oneshot,li2023reconstructive, Zhu_2023_ICCV} and even explore dataset-free purification techniques~\cite{zheng2022data,hong2023revisiting,lin2024fusing}. Although these approaches offer some improvements, recent evaluations \cite{dunnett2024countering, wu2024backdoorbench} continue to highlight the importance of sufficient auxiliary data for achieving robust defenses against backdoor attacks.

While significant progress has been made in reducing the size of auxiliary datasets, an equally critical yet underexplored question remains: \emph{how does the nature of the auxiliary dataset affect purification effectiveness?} In  real-world  applications, auxiliary datasets can vary widely, encompassing in-distribution data, synthetic data, or external data from different sources. Understanding how each type of auxiliary dataset influences the purification effectiveness is vital for selecting or constructing the most suitable auxiliary dataset and the corresponding technique. For instance, when multiple datasets are available, understanding how different datasets contribute to purification can guide defenders in selecting or crafting the most appropriate dataset. Conversely, when only limited auxiliary data is accessible, knowing which purification technique works best under those constraints is critical. Therefore, there is an urgent need for a thorough investigation into the impact of auxiliary datasets on purification effectiveness to guide defenders in  enhancing the security of ML systems. 

In this paper, we systematically investigate the critical role of auxiliary datasets in backdoor purification, aiming to bridge the gap between idealized and practical purification scenarios.  Specifically, we first construct a diverse set of auxiliary datasets to emulate real-world conditions, as summarized in Table~\ref{overall}. These datasets include in-distribution data, synthetic data, and external data from other sources. Through an evaluation of SOTA backdoor purification methods across these datasets, we uncover several critical insights: \textbf{1)} In-distribution datasets, particularly those carefully filtered from the original training data of the victim model, effectively preserve the model’s utility for its intended tasks but may fall short in eliminating backdoors. \textbf{2)} Incorporating OOD datasets can help the model forget backdoors but also bring the risk of forgetting critical learned knowledge, significantly degrading its overall performance. Building on these findings, we propose Guided Input Calibration (GIC), a novel technique that enhances backdoor purification by adaptively transforming auxiliary data to better align with the victim model’s learned representations. By leveraging the victim model itself to guide this transformation, GIC optimizes the purification process, striking a balance between preserving model utility and mitigating backdoor threats. Extensive experiments demonstrate that GIC significantly improves the effectiveness of backdoor purification across diverse auxiliary datasets, providing a practical and robust defense solution.

Our main contributions are threefold:
\textbf{1) Impact analysis of auxiliary datasets:} We take the \textbf{first step}  in systematically investigating how different types of auxiliary datasets influence backdoor purification effectiveness. Our findings provide novel insights and serve as a foundation for future research on optimizing dataset selection and construction for enhanced backdoor defense.
%
\textbf{2) Compilation and evaluation of diverse auxiliary datasets:}  We have compiled and rigorously evaluated a diverse set of auxiliary datasets using SOTA purification methods, making our datasets and code publicly available to facilitate and support future research on practical backdoor defense strategies.
%
\textbf{3) Introduction of GIC:} We introduce GIC, the \textbf{first} dedicated solution designed to align auxiliary datasets with the model’s learned representations, significantly enhancing backdoor mitigation across various dataset types. Our approach sets a new benchmark for practical and effective backdoor defense.



% \subsection{Problem Formulation}
% \label{ssec:problem-formulation}

% In this work, we evaluate \ours in an educational setting where a student is trying to learn a textbook chapter's content.

% Let \( D \) be a document and \( E \) be a set of exam questions that can be solved using \( D \).
% The document \( D \) is structured as a sequence of sections, denoted by \( S_k \subset D \), where each section \( S_k \) represents the content at the \( k \)-th position in \( D \).
% For each section \( S_k \), we define \( S_{[1:k-1]} = \{S_1, S_2, \dots, S_{k-1}\} \subset D \) as the context, which includes all preceding content in the document up to section \( S_k \).

% Let \( M_q \) be a question generator that processes the document sequentially, section by section. For each section \( S_k \), it generates a set of questions \( Q_k = \{Q_k^1, Q_k^2, \dots, Q_k^n\} \), where \( Q_k \sim M_q(S_k, S_{[1:k-1]}) \), indicating that the questions are generated based on the anchor section \( S_k \) and the preceding context \( S_{[1:k-1]} \).

% Let \( M_s \) be a reader simulator.
% We assess the effectiveness of the question generator \( M_q \) by measuring the performance of \( M_s \) on the exam \( E \) using only the generated questions \( Q \), where \( Q = \{Q_1, Q_2, \dots, Q_k\} \) represents the set of all questions produced by \( M_q \) across all sections.
% This is expressed as \( M_s(E \mid Q) \), evaluating how well the generated questions contribute to solving \( E \) without direct access to \( D \).

% Our objective is to design a question generator \( M_q \) that maximizes \( M_s(E \mid Q) \), under the assumption that questions contributing more effectively to solving \( E \) are high-utility questions.




\section{RoleMRC}
\label{sec:method}

In this section, we build RoleMRC. Figure\,\ref{fig:method} illustrates the overall pipeline of RoleMRC from top to bottom, which is divided into three steps.

\subsection{A Meta-pool of 10k Role Profiles}
\label{sec:meta_pool}
We first collect a meta-pool of 10k role profile using two open-source datasets, with Step 1 and 2.

\paragraph{Step 1: Persona Sampling.} We randomly sample 10.5k one-sentence demographic persona description from PersonaHub\,\cite{ge2024scaling}, such as ``\emph{A local business owner interested in economic trends}'', as shown at the top of Figure\,\ref{fig:method}. 

\paragraph{Step 2: Role Profile Standardization.} Next, we use a well-crafted prompt with gpt-4o\,\cite{gpt4o} to expand each sampled persona into a complete role profile, in reference to the 1-shot standardized example. Illustrated in the middle of Figure\,\ref{fig:method}, we require a standardized role profile consisting of seven components: \emph{Role Name and Brief Description}, \emph{Specific Abilities and Skills}, \emph{Speech Style}, \emph{Personality Characteristics}, \emph{Past Experience and Background}, \emph{Ability and Knowledge Boundaries} and \emph{Speech Examples}. %Setting standard specifications helps convert the generated role profiles into formatted records, which is beneficial for the post quality control. 
Standardizing these profiles ensures structured formatting, simplifying quality control. 
After manual checking and format filtering, we remove 333 invalid responses from gpt-4o, resulting in 10.2k final role profiles. We report complete persona-to-profile standardization prompt and structure tree of final role profiles in Appendix\,\ref{sec:app_prompt_1} and \,\ref{sec:app_tree}, respectively.

Machine Reading Comprehension (MRC) is one of the core tasks for LLMs to interact with human users. Consequently, we choose to synthesize fine-grained role-playing instruction-following data based on MRC. We first generate a retrieval pool containing 808.7k MRC data from the MSMARCO training set\,\cite{bajaj2016ms}. By leveraging SFR-Embedding\,\cite{SFR-embedding-2}, we perform an inner product search to identify the most relevant and least relevant MRC triplets (Passages, Question, Answer) for each role profile. For example, the middle part of Figure\,\ref{fig:method} shows that for the role \emph{Jessica Thompson, a resilient local business owner}, the most relevant question is about \emph{the skill of resiliency}, while the least relevant question is \emph{converting Fahrenheit to Celsius}. After review, we categorise the most relevant MRC triplet as within a role's knowledge boundary, and the least relevant MRC triplet as beyond their expertise.

\begin{figure}[t]
    \centering
    \includegraphics[width=1.0\linewidth]{figures/step3.png}
    \caption{The strategy of gradually synthesizing finer role-playing instructions in step 3 of Figure\,\ref{fig:method}.}
    \vspace{-1.0em}
    \label{fig:step3}
\end{figure}

\subsection{38k Role-playing Instructions}
Based on the role profiles, we then adopt \textbf{Step 3: Multi-stage Dialogue Synthesis} to generate 38k role-playing instructions, progressively increasing granularity across three categories %including three types with gradually finer granularity 
(Figure\,\ref{fig:step3}):
%\begin{itemize}
%[leftmargin=*,noitemsep,topsep=0pt]

\noindent \textbf{\underline{Free Chats.}} The simplest dialogues, free chats, are synthesized at first. Here, we ask gpt-4o to simulate and generate multi-turn open-domain conversations between the role and an imagined user based on the standardized role profile. When synthesizing the conversation, we additionally consider two factors: the \textbf{initial speaker} in the starting round of the conversation, and whether the role's speech has \textbf{a narration wrapped in brackets} at the beginning (e.g., \emph{(Aiden reviews the network logs, his eyes narrowing as he spots unusual activity) I found it!}). The narration refers to a short, vivid description of the role's speaking state from an omniscient perspective, which further strengthens the sense of role's depth and has been adopted in some role-playing datasets\,\cite{tu2024charactereval}. 

As shown on the left side of Figure\,\ref{fig:step3}, based on the aforementioned two factors, we synthesize four variations of Free Chats. In particular, when  narration is omitted, we deleted all the 
narration content in the speech examples from the role profile; %and for the case that 
when narration is allowed, we retain the narration content, and also add instructions to allow appropriate insertion of narration in the task prompt of gpt-4o. It worth to note that, in narration-allowed dialogues, not every response of the role has narration inserted to prevent overfitting. All categories of data in RoleMRC incorporate narration insertion and follow similar control mechanisms. The following sections will omit further details on narration.

\noindent \textbf{\underline{On-scene MRC Dialogues.}} The synthesis of on-scene MRC dialogues can be divided into two parts. The first part is similar to the free chats. As shown by the {\color{lightgreen}{green round rectangle}} in the upper part of Figure\,\ref{fig:step3}, we ask gpt-4o to synthesize a conversation (lower left corner of Figure\,\ref{fig:step3}) between the role and the user focusing on relevant passages. This part of the synthesis and the Free Chats share the entire meta-pool, so each consisting of 5k dialogues.

The remaining part forms eight types of single-turn role-playing Question Answering (QA). In the middle of Figure\,\ref{fig:step3}, we randomly select a group of roles and examined the most relevant MRCs they matched: if the question in the MRC is answerable, then the ground truth answer is stylized to match the role profile; otherwise, a seed script of ``unanswerable'' is randomly selected then stylized. The above process generates four groups of 1k data from type ``[1]'' to type``[4]''. According to the middle right side of Figure\,\ref{fig:step3}, we also select a group of roles and ensure that the least relevant MRCs they matched contain answerable QA pairs. Since the most irrelevant MRCs are outside the knowledge boundary of the roles, the role-playing responses to these questions are ``out-of-mind'' refusal or ``let-me-try'' attempt, thus synthesizing four groups of 1k data, from type ``[5]'' to type ``[8]''.

\noindent \textbf{\underline{Ruled Chats.}} We construct Ruled Chats by extending On-scene MRC Dialogues in categories ``[1]'' to ``[8]'' with incorporated three additional rules, as shown in the right bottom corner of Figure\,\ref{fig:step3}. For the \textbf{multi-turn rules}, we apply them to the four unanswerable scenarios ``[3]'', ``[4]'', ``[5]'', and ``[6]'', adding a user prompt that  forces the role to answer. Among them, data ``[3]'' and ``[4]'' maintain refusal since the questions in MRC are unanswerable; while ``[5]'' and ``[6]'' are transformed into attempts to answer despite knowledge limitations. For the \textbf{nested formatting rules}, we add new formatting instructions to the four categories of data ``[1]'', ``[2]'', ``[3]'', and ``[4]'', such as requiring emojis,  capitalization, specific punctuation marks, and controlling the total number of words, then modify the previous replies accordingly. For the last \textbf{prioritized rules}, we apply them to subsets ``[1]'' and ``[2]'' that contain normal stylized answers, inserting a  global refusal directive from the system, and thus creating a conflict between system instructions and the role's ability boundary.
%\end{itemize}

\begin{table}[t]
\resizebox{\columnwidth}{!}{%
  \begin{tabular}{c|c|c|c|c|c}
    \toprule
    & & \textbf{S*} & \textbf{P*} & \textbf{\#Turns} & \textbf{\#Words} \\ 
    \midrule
    \multirow{13.5}{*}{\textbf{RoleMRC}} 
    & \multicolumn{5}{c|}{\textbf{Free Chats}} \\ 
    \cmidrule(lr){2-6}
    & Chats & 5k & / & 9.47 & 38.62 \\ 
    \cmidrule(lr){2-6}
    & \multicolumn{5}{c|}{\textbf{On-scene MRC Dialogues}} \\ 
    \cmidrule(lr){2-6} 
    & On-scene Chats & 5k & / & 9.2 & 43.18 \\
    & Answer & 2k & 2k & 1 & 39.45 \\ 
    & No Answer & 2k & 2k & 1 & 47.09 \\ 
    & Refusal & 2k & 2k & 1 & 48.41 \\ 
    & Attempt & 2k & 2k & 1 & 47.92 \\ 
    \cmidrule(lr){2-6}
    & \multicolumn{5}{c|}{\textbf{Ruled Chats}} \\ 
    \cmidrule(lr){2-6}
    & Multi-turn & 2k & 2k & 2 & 42.47 \\ 
    & Nested & 1.6k & 1.6k & 1 & 46.17 \\ 
    & Prioritized & 2.4k & 2.4k & 1 & 42.65 \\ 
    \midrule
    & \textbf{Total} & 24k & 14k & 3.5 & 40.6 \\ 
    \midrule
    \multirow{3}{*}{\textbf{-mix}} 
    & RoleBench & 16k & / & 1 & 23.95 \\ 
    & RLHFlow & 40k & / & 1.39 & 111.79 \\ 
    & UltraFeedback & / & 14k & 1 & 199.28 \\ 
    \midrule
    & \textbf{Total} & 80k & 28k & 2 & 67.1 \\ 
    \bottomrule
  \end{tabular}}
  \vspace{-2mm}
  \caption{Statistics of RoleMRC. In particular, the column names S*, P*, \#Turns, and \#Words, stands for size of single-label data, size of pair-label data, average turns, and average number of words per reply, respectively. RoleMRC-mix expands RoleMRC by adding existing role-playing data.}
 \vspace{-3mm}
  \label{tab:roleMRC}
\end{table}

\subsection{Integration and Mix-up}
All the seed scripts and prioritized rules used for constructing On-scene Dialogues and Ruled Chats are reported in Appendix\,\ref{sec:app_scripts}. These raw responses are logically valid manual answers that remain unaffected by the roles' speaking styles, making them suitable as negative labels to contrast with the stylized answers. Thanks to these meticulous seed texts, we obtain high-quality synthetic data with stable output from gpt-4o. After integration, as shown in Table\,\ref{tab:roleMRC}, the final RoleMRC contains 24k single-label data for Supervised Fine-Tuning (SFT) and 14k pair-label data for Human Preference Optimization (HPO)\,\cite{ouyang2022training,rafailov2023direct,sampo,hong2024reference}. Considering that fine-tuning LLMs with relatively fixed data formats may lead to catastrophic forgetting\,\cite{kirkpatrick2017overcoming}, we create RoleMRC-mix as a robust version by incorporating external role-playing data (RoleBench\,\cite{wang2023rolellm}) and general instructions (RLHFlow\,\cite{dong2024rlhf}, UltraFeedback\,\cite{cui2023ultrafeedback}).


\section{Experiments}
\label{sec:experiments}



\begin{figure*}[t]
    \centering
    \includegraphics[width=1\linewidth]{images/Environments.pdf} 
    % \vspace{-20pt}
    \captionsetup{
    width=\textwidth,
    font=Smallfont,
    labelfont=Smallfont,
    textfont=Smallfont
    }
    \captionsetup{
    width=\textwidth,
    font=Smallfont,
    labelfont=Smallfont,
    textfont=Smallfont
    }
    \caption{Four different real-world experiment environments.}
    \label{fig:environments}
    % \vspace{-6pt}
\end{figure*}

\begin{figure*}[t]
    \centering
    \captionsetup{
    width=\textwidth,
    font=Smallfont,
    labelfont=Smallfont,
    textfont=Smallfont
    }
    % Top-left subfigure
    \begin{subfigure}[b]{0.45\textwidth}
        \centering
        \includegraphics[width=\textwidth]{images/fig_office.pdf}
        \caption{Office}
        \label{fig:subfig1}
    \end{subfigure}
    \hspace{0.02\textwidth}
    % Top-right subfigure
    \begin{subfigure}[b]{0.45\textwidth}
        \centering
        \includegraphics[width=\textwidth]{images/fig_apt.pdf}
        \caption{Apartment}
        \label{fig:subfig2}
    \end{subfigure}

    \vskip\baselineskip

    % Bottom-left subfigure
    \begin{subfigure}[b]{0.45\textwidth}
        \centering
        \includegraphics[width=\textwidth]{images/fig_outdoor.pdf}
        \caption{Outdoor}
        \label{fig:subfig3}
    \end{subfigure}
    \hspace{0.02\textwidth}
    % Bottom-right subfigure
    \begin{subfigure}[b]{0.45\textwidth}
        \centering
        \includegraphics[width=\textwidth]{images/fig_hallway.pdf}
        \caption{Hallway}
        \label{fig:subfig4}
    \end{subfigure}

    \caption{Top-down view of the trajectories comparison on the value maps with the detection results across the four different environments.}
    \label{fig:value_map}
\end{figure*}


\begin{table*}[ht]
\captionsetup{
    width=\textwidth,
    font=Smallfont,
    labelfont=Smallfont,
    textfont=Smallfont
    }
\caption{Vision-language navigation performance in 4 unseen environments (SR and SPL).}
\label{SOTAResults}
\centering
\begin{tabular}{lcccc|cccc}
\toprule
\multirow{2}{*}{\textbf{Method}} & \multicolumn{4}{c}{\textbf{SR (\%)}} & \multicolumn{4}{c}{\textbf{SPL}} \\
\cmidrule(lr){2-5} \cmidrule(lr){6-9}
 & Hallway & Office & Apartment & Outdoor & Hallway & Office & Apartment & Outdoor \\
\midrule
\textbf{Frontier Exploration}  
  & 40.0 & 41.7 & 55.6 & 33.3  
  & 0.239 & 0.317 & 0.363 & 0.189 \\

\textbf{VLFM} \cite{yokoyama2024vlfm}                 
  & 53.3 & 75.0 & 66.7 & 44.4  
  & 0.366 & 0.556 & 0.412 & 0.308 \\

\textbf{VL-Nav w/o IBTP}      
  & 66.7 & 83.3 & \underline{70.2} & \underline{55.6}  
  & 0.593 & 0.738 & 0.615 & \underline{0.573} \\

\textbf{VL-Nav w/o curiosity}      
  & \underline{73.3} & \underline{86.3} & 66.7 & \underline{55.6}  
  & \underline{0.612} & \underline{0.743} & \underline{0.631} & 0.498 \\

\textbf{VL-Nav}               
  & \textbf{86.7} & \textbf{91.7} & \textbf{88.9} & \textbf{77.8}  
  & \textbf{0.672} & \textbf{0.812} & \textbf{0.733} & \textbf{0.637} \\

\bottomrule
\end{tabular}
\end{table*}








\subsection{Experimental Setting}
\label{sec:experimental_setting}

We evaluate our approach in real-robot experiments against five methods: (1) classical frontier-based exploration, (2) VLFM \cite{yokoyama2024vlfm}, (3) VLNav without instance-based target points, (4) VLNav without curiosity terms, and (5) the full VLNav configuration. Because the original VLFM relies on BLIP-2 \cite{li2023blip}, which is too computationally heavy for real-time edge deployment, we use the YOLO-World \cite{cheng2024yolo} model instead to generate per-observation similarity scores for VLFM. Each method is tested under the same conditions to ensure a fair comparison of performance.

\paragraph{Environments:}
We consider four distinct environments (shown in \fref{fig:environments}), each with a specific combination of semantic complexity (\textit{High}, \textit{Medium}, or \textit{Low}) and size (\textit{Big}, \textit{Mid}, or \textit{Small}). Concretely, we use a Hallway (\textit{Medium \& Big}), an Office (\textit{High \& Mid}), an Outdoor area (\textit{Low \& Big}), and an Apartment (\textit{High \& Small}). In each environment, we evaluate five methods using three language prompts, yielding a diverse range of spatial layouts and semantic challenges. This setup provides a rigorous assessment of each method’s adaptability.

\paragraph{Language-Described Instance:}
We define nine distinct, uncommon human-described instances to serve as target objects or persons during navigation. Examples include phrases such as “tall white trash bin,” “there seems to be a man in white,” “find a man in gray,” “there seems to be a black chair,” “tall white board,” and “there seems to be a fold chair.” The variety in these descriptions ensures that the robot must rely on vision-language understanding to accurately locate these targets.

\noindent\textbf{Robots and Sensor Setup:} 
All experiments are conducted using a four-wheel Rover equipped with a Livox Mid-360 LiDAR. The LiDAR is tilted by approximately 23 degrees to the front to achieve a $\pm 30$ degrees vertical FOV coverage closely aligned with the forward camera’s view. An Intel RealSense D455 RGB-D camera, tilted upward by 7 degrees to detect taller objects, provides visual observation, though its depth data are not used for positioning or mapping. LiDAR measurements are a primary source of mapping and localization due to their higher accuracy. The whole VL-Nav system runs on an NVIDIA Jetson Orin NX on-board computer.




\subsection{Main Results}
\label{sec:main_results}

We validate the proposed VL-Nav system in real-robot experiments across four distinct environments (\textit{Hallway}, \textit{Office}, \textit{Apartment}, and \textit{Outdoor}), each featuring different semantic levels and sizes. Building on the motivation articulated in~\sref{sec:intro}, we focus on evaluating VL-Nav’s ability to (1) interpret fine-grained vision-language features and conduct robust VLN, (2) explore efficiently in unfamiliar spaces across various environments, and (3) run in real-time on resource-constrained platforms. \fref{fig:value_map} presents a top-down comparison of trajectories and detection results on the value map.

\begin{figure*}[t]
    \centering
    \includegraphics[width=1\linewidth]{images/result_plot.pdf} 
    % \vspace{-20pt}
    \captionsetup{
    width=\textwidth,
    font=Smallfont,
    labelfont=Smallfont,
    textfont=Smallfont
    }
    \caption{Plots of performance in different environments sizes and semantic comlexities.}
    \label{fig:results}
    % \vspace{-6pt}
\end{figure*}

\paragraph{Overall Performance:}
As reported in Table~\ref{SOTAResults}, our full \textbf{VL-Nav} consistently obtains the highest Success Rate (SR) and Success weighted by Path Length (SPL) across all four environments. In particular, VL-Nav outperforms classical exploration by a large margin, confirming the advantage of integrating CVL spatial reasoning with partial frontier-based search rather than relying solely on geometric exploration.

\paragraph{Effect of Instance-Based Target Points (IBTP):}
We note a marked improvement when enabling IBTP: the variant without IBTP lags behind, particularly in complex domains like the \textit{Apartment} and \textit{Office}. As discussed in \sref{sec:method}, IBTP allows VL-Nav to pursue and verify tentative detections with confidence above a threshold, mirroring human search behavior. This pragmatic mechanism prevents ignoring possible matches to the target description and reduces overall travel distance to confirm or discard candidate objects.

\paragraph{Curiosity Contributions:}
The \emph{curiosity Score} is also significant to VL-Nav’s performance. It merges two key components:
\begin{itemize}
    \item \textbf{Distance Weighting}: Preventing easily select very far way goals to reduce travel time and energy consumption which is extremely important for the efficiency (metrics SPL) in the large-size environments.
    \item \textbf{Unknown-Area Weighting}: Rewards navigation toward regions that yield more information.
\end{itemize}
Our ablations reveal that removing the distance-scoring element (\textit{VL-Nav w/o curiosity}) degrades both SR and SPL, particularly in the more cluttered environments. Meanwhile, dropping the instance-based target points (IBTP) similarly lowers performance, reflecting how each piece of CVL addresses a complementary aspect of semantic navigation.

\paragraph{Comparison to VLFM:}
Although the VLFM approach \cite{yokoyama2024vlfm} harnesses vision-language similarity value, it lacks the pixel-wise vision-language features, instance-based target points verification mechanism, and CVL-based spatial reasoning. Consequently, VL-Nav surpasses VLFM in both SR and SPL by effectively combining the pixel-wise vision language features and the curiosity cues via the CVL spatial reasoning. These gains are especially pronounced in semantic complex (\textit{Apartment}) and open-area (\textit{Outdoor}) environments, underscoring how our CVL spatial reasoning enhance vision-language navigation in complex settings and scenarios.



\paragraph{Summary of Findings:}
In conclusion, the experimental results confirm that VL-Nav delivers superior vision-language navigation across diverse, unseen real-world environments. By fusing frontier-based target points detection, instance-based target points, and the CVL spatial reasoning for goal selection, VL-Nav balances semantic awareness and exploration efficiency. The system’s robust performance, even in large or cluttered domains, highlights its potential as a practical solution for zero-shot vision-language navigation on low-power robots.

% \begin{table}[!t]
% \centering
% \scalebox{0.68}{
%     \begin{tabular}{ll cccc}
%       \toprule
%       & \multicolumn{4}{c}{\textbf{Intellipro Dataset}}\\
%       & \multicolumn{2}{c}{Rank Resume} & \multicolumn{2}{c}{Rank Job} \\
%       \cmidrule(lr){2-3} \cmidrule(lr){4-5} 
%       \textbf{Method}
%       &  Recall@100 & nDCG@100 & Recall@10 & nDCG@10 \\
%       \midrule
%       \confitold{}
%       & 71.28 &34.79 &76.50 &52.57 
%       \\
%       \cmidrule{2-5}
%       \confitsimple{}
%     & 82.53 &48.17
%        & 85.58 &64.91
     
%        \\
%        +\RunnerUpMiningShort{}
%     &85.43 &50.99 &91.38 &71.34 
%       \\
%       +\HyReShort
%         &- & -
%        &-&-\\
       
%       \bottomrule

%     \end{tabular}
%   }
% \caption{Ablation studies using Jina-v2-base as the encoder. ``\confitsimple{}'' refers using a simplified encoder architecture. \framework{} trains \confitsimple{} with \RunnerUpMiningShort{} and \HyReShort{}.}
% \label{tbl:ablation}
% \end{table}
\begin{table*}[!t]
\centering
\scalebox{0.75}{
    \begin{tabular}{l cccc cccc}
      \toprule
      & \multicolumn{4}{c}{\textbf{Recruiting Dataset}}
      & \multicolumn{4}{c}{\textbf{AliYun Dataset}}\\
      & \multicolumn{2}{c}{Rank Resume} & \multicolumn{2}{c}{Rank Job} 
      & \multicolumn{2}{c}{Rank Resume} & \multicolumn{2}{c}{Rank Job}\\
      \cmidrule(lr){2-3} \cmidrule(lr){4-5} 
      \cmidrule(lr){6-7} \cmidrule(lr){8-9} 
      \textbf{Method}
      & Recall@100 & nDCG@100 & Recall@10 & nDCG@10
      & Recall@100 & nDCG@100 & Recall@10 & nDCG@10\\
      \midrule
      \confitold{}
      & 71.28 & 34.79 & 76.50 & 52.57 
      & 87.81 & 65.06 & 72.39 & 56.12
      \\
      \cmidrule{2-9}
      \confitsimple{}
      & 82.53 & 48.17 & 85.58 & 64.91
      & 94.90&78.40 & 78.70& 65.45
       \\
      +\HyReShort{}
       &85.28 & 49.50
       &90.25 & 70.22
       & 96.62&81.99 & \textbf{81.16}& 67.63
       \\
      +\RunnerUpMiningShort{}
       % & 85.14& 49.82
       % &90.75&72.51
       & \textbf{86.13}&\textbf{51.90} & \textbf{94.25}&\textbf{73.32}
       & \textbf{97.07}&\textbf{83.11} & 80.49& \textbf{68.02}
       \\
   %     +\RunnerUpMiningShort{}
   %    & 85.43 & 50.99 & 91.38 & 71.34 
   %    & 96.24 & 82.95 & 80.12 & 66.96
   %    \\
   %    +\HyReShort{} old
   %     &85.28 & 49.50
   %     &90.25 & 70.22
   %     & 96.62&81.99 & 81.16& 67.63
   %     \\
   % +\HyReShort{} 
   %     % & 85.14& 49.82
   %     % &90.75&72.51
   %     & 86.83&51.77 &92.00 &72.04
   %     & 97.07&83.11 & 80.49& 68.02
   %     \\
      \bottomrule

    \end{tabular}
  }
\caption{\framework{} ablation studies. ``\confitsimple{}'' refers using a simplified encoder architecture. \framework{} trains \confitsimple{} with \RunnerUpMiningShort{} and \HyReShort{}. We use Jina-v2-base as the encoder due to its better performance.
}
\label{tbl:ablation}
\end{table*}

\section{Results}
\label{sec:results}

In this section, we present detailed results demonstrating \emph{CellFlow}'s state-of-the-art performance in cellular morphology prediction under perturbations, outperforming existing methods across multiple datasets and evaluation metrics.

\subsection{Datasets}

Our experiments were conducted using three cell imaging perturbation datasets: BBBC021 (chemical perturbation)~\cite{caie2010high}, RxRx1 (genetic perturbation)~\cite{sypetkowski2023rxrx1}, and the JUMP dataset (combined perturbation)~\cite{chandrasekaran2023jump}. We followed the preprocessing protocol from IMPA~\cite{palma2023predicting}, which involves correcting illumination, cropping images centered on nuclei to a resolution of 96×96, and filtering out low-quality images. The resulting datasets include 98K, 171K, and 424K images with 3, 5, and 6 channels, respectively, from 26, 1,042, and 747 perturbation types. Examples of these images are provided in Figure~\ref{fig:comparison}. Details of datasets are provided in \S\ref{sec:data}.

\subsection{Experimental Setup}

\textbf{Evaluation metrics.} We evaluate methods using two types of metrics: (1) FID and KID, which measure image distribution similarity via Fréchet and kernel-based distances, computed on 5K generated images for BBBC021 and 100 randomly selected perturbation classes for RxRx1 and JUMP; we report both overall scores across all samples and conditional scores per perturbation class. (2) Mode of Action (MoA) classification accuracy, which assesses biological fidelity by using a trained classifier to predict a drug’s effect from perturbed images and comparing it to its known MoA from the literature.

\textbf{Baselines.} We compare our approach against two baselines, PhenDiff~\cite{bourou2024phendiff} and IMPA~\cite{palma2023predicting}, the only two baselines that incorporate control images into their model design --- a crucial setup for distinguishing true perturbation effects from artifacts such as batch effects. PhenDiff uses diffusion models to first map control images to noise and then transform the noise into target images. In contrast, IMPA employs GANs with an AdaIN layer to transfer the style of control images to target images, specifically designed for paired image-to-image mappings. Our method uses flow matching, which is tailored for distribution-to-distribution mapping, providing a more suitable solution for our problem. We reproduce these baselines with official codes.

\textbf{Training details.} \emph{CellFlow} employs a UNet-based velocity field with a four-stage design. Perturbations are encoded following IMPA~\cite{palma2023predicting}. Training is conducted for 100 epochs on 4 A100 GPUs. Details are in \S\ref{sec:experimental}.


\begin{table*}[ht!]
\centering
\footnotesize
\renewcommand{\arraystretch}{1.2}
\begin{tabular}{p{2.15cm} p{14.05cm}}
\toprule
 & \includegraphics[height=2.6cm]{figure/qualitative_example.pdf}\\
\midrule
Original \phantom{00000} Description & Bromazepam binds to the GABA receptor GABA-A, causing a conformational change and \textbf{increasing inhibitory effects of GABA}. Other neurotransmitters are not influenced. One of the \textbf{benzodiazepines} that is \textbf{used in the treatment of anxiety disorders.} \\
\midrule
User & Explain the features of the given molecule that are most distinct from those of other molecules. \\
\midrule
Mol-LLaMA (Ours) & The given molecule is a member of the \textcolor{tabblue}{benzodiazepine class}, which is characterized by a fusion of a benzene ring and a seven-membered diazepine ring. This structure is crucial for its biological activity, particularly its role as \textcolor{tabblue}{a sedative and hypnotic agent}. ...$<$omitted$>$...\\

& The given molecule is also characterized by a ketone group within the diazepine ring, which is a common feature in benzodiazepines. \textcolor{tabblue}{This functional group is important for the interaction with the GABA-A receptor, enhancing the inhibitory effects of GABA}. ...$<$omitted$>$...\\

\midrule
GPT-4o & 1. **Quinazoline Core**: The molecule contains a \textcolor{tabred}{quinazoline core}, ...$<$omitted$>$... \\
 & Overall, the combination of a \textcolor{tabred}{quinazoline} structure with unique substitution patterns, including bromine halogenation and the presence of a pyridinyl group, ...$<$omitted$>$... . \\
\midrule
LLaMo & The molecule has a \textcolor{tabblue}{benzodiazepine} structure with a bromo substituent at the 7-position and \textcolor{tabred}{a methyl group at the 1-position}. It is distinct from other molecules in that it contains a bromine atom, a nitrogen atom, and a methyl group, which are not present in the other molecules. \\
\midrule
3D-MoLM & ...$<$omitted$>$... It belongs to the class of compounds called \textcolor{tabred}{quinazolines}, which are characterized by a bicyclic structure consisting of a benzene ring fused to a pyrimidine ring. ...$<$omitted$>$...\\
& The molecule's structure suggests potential applications in medicinal chemistry, as quinazolines have been found to possess various biological activities, including \textcolor{tabred}{antitumor, antimicrobial, and anti-inflammatory properties.} \\
\midrule
Mol-Instructions & The molecule is a \textcolor{tabred}{quinoxaline derivative}. \\
\bottomrule
\end{tabular}
\vspace{-0.1in}
\caption{\small Case study to compare molecular understanding and reasoning ability. Mol-LLaMA accurately understands the molecular features, answering a correct molecular taxonomy and providing its distinct properties that are relevant to the given molecule.}
\label{tab:qualitative}
\vspace{-0.1in}
\end{table*}

\subsection{Main Results}

\textbf{\emph{CellFlow} generates highly realistic cell images.}  
\emph{CellFlow} outperforms existing methods in capturing cellular morphology across all datasets (Table~\ref{tab:results}a), achieving overall FID scores of 18.7, 33.0, and 9.0 on BBBC021, RxRx1, and JUMP, respectively --- improving FID by 21\%–45\% compared to previous methods. These gains in both FID and KID metrics confirm that \emph{CellFlow} produces significantly more realistic cell images than prior approaches.

\textbf{\emph{CellFlow} accurately captures perturbation-specific morphological changes.}  
As shown in Table~\ref{tab:results}a, \emph{CellFlow} achieves conditional FID scores of 56.8 (a 26\% improvement), 163.5, and 84.4 (a 16\% improvement) on BBBC021, RxRx1, and JUMP, respectively. These scores are computed by measuring the distribution distance for each specific perturbation and averaging across all perturbations.   
Table~\ref{tab:results}b further highlights \emph{CellFlow}’s performance on six representative chemical and three genetic perturbations. For chemical perturbations, \emph{CellFlow} reduces FID scores by 14–55\% compared to prior methods.
The smaller improvement (5–12\% improvements) on RxRx1 is likely due to the limited number of images per perturbation type.

\textbf{\emph{CellFlow} preserves biological fidelity across perturbation conditions.} 
Table~\ref{tab:ablation}a presents mode of action (MoA) classification accuracy on the BBBC021 dataset using generated cell images. MoA describes how a drug affects cellular function and can be inferred from morphology. To assess this, we train an image classifier on real perturbed images and test it on generated ones. \emph{CellFlow} achieves 71.1\% MoA accuracy, closely matching real images (72.4\%) and significantly surpassing other methods (best: 63.7\%), demonstrating its ability to maintain biological fidelity across perturbations. Qualitative comparisons in Figure~\ref{fig:comparison} further highlight \emph{CellFlow}’s accuracy in capturing key biological effects. For example, demecolcine produces smaller, fragmented nuclei, which other methods fail to reproduce accurately.

\textbf{\emph{CellFlow} generalizes to out-of-distribution (OOD) perturbations.}  
On BBBC021, \emph{CellFlow} demonstrates strong generalization to novel chemical perturbations never seen during training (Table~\ref{tab:ablation}b). It achieves 6\% and 28\% improvements in overall and conditional FID over the best baseline. This OOD generalization is critical for biological research, enabling the exploration of previously untested interventions and the design of new drugs.

\textbf{Ablations highlight the importance of each component in \emph{CellFlow}.}  
Table~\ref{tab:ablation}c shows that removing conditional information, classifier-free guidance, or noise augmentation significantly degrades performance, leading to higher FID scores. These underscore the critical role of each component in enabling \emph{CellFlow}’s state-of-the-art performance.  

\begin{figure*}[!tb]
    \centering
     \includegraphics[width=\linewidth]{imgs/interpolation.pdf}
     \vspace{-2em}
    \caption{
    \textbf{\emph{CellFlow} enables new capabilities.} 
\textit{(a.1) Batch effect calibration.}  
\emph{CellFlow} initializes with control images, enabling batch-specific predictions. Comparing predictions from different batches highlights actual perturbation effects (smaller cell size) while filtering out spurious batch effects (cell density variations).  
\textit{(a.2) Interpolation trajectory.}  
\emph{CellFlow}'s learned velocity field supports interpolation between cell states, which might provide insights into the dynamic cell trajectory. 
\textit{(b) Diffusion model comparison.}  
Unlike flow matching, diffusion models that start from noise cannot calibrate batch effects or support interpolation.  
\textit{(c) Reverse trajectory.}  
\emph{CellFlow}'s reversible velocity field can predict prior cell states from perturbed images, offering potential applications such as restoring damaged cells.
    }
    \label{fig:interpolation}
    \vspace{-1em}
\end{figure*}

\subsection{New Capabilities}

\textbf{\emph{CellFlow} addresses batch effects and reveals true perturbation effects.}  
\emph{CellFlow}’s distribution-to-distribution approach effectively addresses batch effects, a significant challenge in biological experimental data collection. As shown in Figure~\ref{fig:interpolation}a, when conditioned on two distinct control images with varying cell densities from different batches, \emph{CellFlow} consistently generates the expected perturbation effect (cell shrinkage due to mevinolin) while recapitulating batch-specific artifacts, revealing the true perturbation effect. Table~\ref{tab:ablation}d further quantifies the importance of conditioning on the same batch. By comparing generated images conditioned on control images from the same or different batches against the target perturbation images, we find that same-batch conditioning reduces overall and conditional FID by 21\%. This highlights the importance of modeling control images to more accurately capture true perturbation effects—an aspect often overlooked by prior approaches, such as diffusion models that initialize from noise (Figure~\ref{fig:interpolation}b).

\textbf{\emph{CellFlow} has the potential to model cellular morphological change trajectories.}
Cell trajectories could offer valuable information about perturbation mechanisms, but capturing them with current imaging technologies remains challenging due to their destructive nature. Since \emph{CellFlow} continuously transforms the source distribution into the target distribution, it can generate smooth interpolation paths between initial and final predicted cell states, producing video-like sequences of cellular transformation based on given source images (Figure~\ref{fig:interpolation}a). This suggests a possible approach for simulating morphological trajectories during perturbation response, which diffusion methods cannot achieve (Figure~\ref{fig:interpolation}b). Additionally, the reversible distribution transformation learned through flow matching enables \emph{CellFlow} to model backward cell state reversion (Figure~\ref{fig:interpolation}c), which could be useful for studying recovery dynamics and predicting potential treatment outcomes.

\paragraph{Summary}
Our findings provide significant insights into the influence of correctness, explanations, and refinement on evaluation accuracy and user trust in AI-based planners. 
In particular, the findings are three-fold: 
(1) The \textbf{correctness} of the generated plans is the most significant factor that impacts the evaluation accuracy and user trust in the planners. As the PDDL solver is more capable of generating correct plans, it achieves the highest evaluation accuracy and trust. 
(2) The \textbf{explanation} component of the LLM planner improves evaluation accuracy, as LLM+Expl achieves higher accuracy than LLM alone. Despite this improvement, LLM+Expl minimally impacts user trust. However, alternative explanation methods may influence user trust differently from the manually generated explanations used in our approach.
% On the other hand, explanations may help refine the trust of the planner to a more appropriate level by indicating planner shortcomings.
(3) The \textbf{refinement} procedure in the LLM planner does not lead to a significant improvement in evaluation accuracy; however, it exhibits a positive influence on user trust that may indicate an overtrust in some situations.
% This finding is aligned with prior works showing that iterative refinements based on user feedback would increase user trust~\cite{kunkel2019let, sebo2019don}.
Finally, the propensity-to-trust analysis identifies correctness as the primary determinant of user trust, whereas explanations provided limited improvement in scenarios where the planner's accuracy is diminished.

% In conclusion, our results indicate that the planner's correctness is the dominant factor for both evaluation accuracy and user trust. Therefore, selecting high-quality training data and optimizing the training procedure of AI-based planners to improve planning correctness is the top priority. Once the AI planner achieves a similar correctness level to traditional graph-search planners, strengthening its capability to explain and refine plans will further improve user trust compared to traditional planners.

\paragraph{Future Research} Future steps in this research include expanding user studies with larger sample sizes to improve generalizability and including additional planning problems per session for a more comprehensive evaluation. Next, we will explore alternative methods for generating plan explanations beyond manual creation to identify approaches that more effectively enhance user trust. 
Additionally, we will examine user trust by employing multiple LLM-based planners with varying levels of planning accuracy to better understand the interplay between planning correctness and user trust. 
Furthermore, we aim to enable real-time user-planner interaction, allowing users to provide feedback and refine plans collaboratively, thereby fostering a more dynamic and user-centric planning process.

% Logging:


UserID
ScenarioID
controlMode
requestID (Nummer der Request)
elapsed Time
distanceTravelledSinceLastLog
distanceToEndOfInstructedPath (Luftlinie zum Ende des gefolgten Pfades)
lengthOfCurrentInstructedPath (Gesamtlänge noch zu folgendem gezeichneter Pfad + generierte Verbindungsstrecke Auto <-> Pfad)
lengthOfCurrentInstructedInputPath (Gesamtlänge noch zu folgendem gezeichneter Pfad)
distanceToEnd (Auto <-> Ende bis wo hin operiert werden muss)
vehiclePosition (Verlauf der unity positions vektoren -> nun auch mit offset, also vergleichbar, wenn man damit was anfangen will)
vehicleSpeed (in kmh)
constructionSiteEntered (ob erstmalig unmittelbar vor der ConstructionSite angekommen)
endReached ("Request geschafft")
closest Lane (nummer lane 0-indiziert von links bis rechts)
currentLaneDeviation (Spurabweichung zur closest Lane Mitte  auch in metern)
timeOfCollisionAvoidanceTraffic (Zeit summe wenn Verkehr in der CollisionAvoidance Range ist (egal ob hinten oder vorne, noch ob es das auto stört))
timeOfCollisionAvoidanceObstacle (wie oben für Obstacles <- Baustellenfahrzeuge, Baustellenmarker, Metallgrenzen links und rechts von Straße)
timeOfCollisionAvoidancePedestrian (wie oben für Fußgänger)
amountOfAdditionInput (Summe wie oft der Pfad erweitert wurde +1 pro Aktion)
amountOfAdditionMarkers (Summe um wie viele "Stellen" der Pfad erweitert wurde +x pro Aktion <- relevant für Anteil an Snap to Mid in Trajectory, für PathPlanning und Waypoint immer +1)
amountOfSnapToMiddleInput (+1 wenn Addition Input und mindestens einmal Snap To Mid verwendet wurde, für path planning immer 1, weil im Grunde immer Mittig)
amountOfSnapToMiddleMarkers (wie amount Of AdditionMarkers, nur dass nur hochgezählt wird wenn die "Stelle" gesnappt wurde)
amountOfReadjustmentInput (wie obiges nur statt Addition: Readjustment, heißt: anfang und ende sind in alter Trajektorie oder nah parallel dazu, Waypoint wird verschoben, PathPlanning backwards, immer +1)
timeSinceLastInput (Zeit seit letztem input)
currentlyNeglectedTime (aktuelle Zeit wie lang vehikel schon still steht (colision avoidance) oder wenn Ende vom angegebenen Pfad)
blindTimeSum (wenn vehikel nicht neglected und schon eingaben gemacht wurden -> aufsummieren Zeit wenn nicht in Main oder secondary view)
isMainRequest
isSecondaryRequest
totalRequestAmount (insgesamte Anzahl an Requests in dem Szenario, hat nicht direkt was mit dieser request instanz zu tun, aber generelle info)
sideOfConstructionSite (ob construction site links oder rechts bei der Straße




Für ein Szenario/ case jeweils::
OverviewLog:
UserID
ScenarioID
controlMode
elapsedTime
activeRequests (wie viele grade angezeigt werden)
succeededRequests (wie viele bisher geschafft sind)
mousePositionX
mousePositionY
usingMultiView (also wenn zwei Requests gleichzeitig angezeigt = true sonst = false)
usingSingleViewMain (auch nur = true wenn nur ein vehikel in der Hauptanzeige ist)
EyeGazeArea (das was du noch meintest -> was aktuell angeschaut wird: string wert aus: RequestList, MainView, SecondaryView, Instructions, None)
mouseClickLeft (summe)
mouseClickRight (summe)
pressLeftCtrl (summe)
pressShift (summe)
TimestampLog:
UserID
ScenarioID
controlMode
elapsed time
timeStampEvent (String wert aus RequestStarted, RequestFinished, RequestOpenedMain, RequestOpenedSecondary, RequestRemovedMain, RequestRemovedSecondary)
additionalInfo (eigentlich immer nur die Request Nummer des vehikels)
-> Es kommen immer Logs rein sobald eines der genannten Events auftritt. Bei den Request Opened/removed events muss man aufpassen, da jede variablen änderung geloggt wird. Demnach sind für den selben elapsedTimeTimestamp nur die Anfangs und Endzustände der jeweiligen Slots relevant.



UnitEyeLog (-> Standard Implementation) + 
 4 areas of interest definierst: links das panel mit den requests, das video mit der aktuellen Szene (ego view),  die szene top down view, und noch das Panel mit den Gründen unten
ist im OverviewLog









% {\appendix[Proof of the Zonklar Equations]
% Use $\backslash${\tt{appendix}} if you have a single appendix:
% Do not use $\backslash${\tt{section}} anymore after $\backslash${\tt{appendix}}, only $\backslash${\tt{section*}}.
% If you have multiple appendixes use $\backslash${\tt{appendices}} then use $\backslash${\tt{section}} to start each appendix.
% You must declare a $\backslash${\tt{section}} before using any $\backslash${\tt{subsection}} or using $\backslash${\tt{label}} ($\backslash${\tt{appendices}} by itself
%  starts a section numbered zero.)}



% {\appendices
% \section*{Proof of the First Zonklar Equation}
% Appendix one text goes here.
% You can choose not to have a title for an appendix if you want by leaving the argument blank
% \section*{Proof of the Second Zonklar Equation}
% Appendix two text goes here.}


% argument is your BibTeX string definitions and bibliography database(s)
% \bibliography{IEEEabrv,references}
\bibliography{references}
%
% \newpage
% \section{Biography Section}
% If you have an EPS/PDF photo (graphicx package needed), extra braces are
%  needed around the contents of the optional argument to biography to prevent
%  the LaTeX parser from getting confused when it sees the complicated
%  $\backslash${\tt{includegraphics}} command within an optional argument. (You can create
%  your own custom macro containing the $\backslash${\tt{includegraphics}} command to make things
%  simpler here.)
% \vspace{11pt}
% \bf{If you include a photo:}
% \begin{IEEEbiography}[{\includegraphics[width=1in,height=1.25in,clip,keepaspectratio]{figs/fig1.png}}]{IEEE Publications Technology Team}
% In this paragraph you can place your educational, professional background and research and other interests.\end{IEEEbiography}
% \vspace{-33pt}


% \begin{IEEEbiography}[{\includegraphics[width=1in,height=1.25in,clip,keepaspectratio]{figs/bios/brady.jpg}}]{Brady Moon}
% is a Ph.D. candidate in Robotics at Carnegie Mellon University. His research focuses on autonomy algorithms for robotics, particularly in the areas of path planning, multi-agent search, target tracking, and data gathering in complex environments. Brady has contributed to various projects, including UAV-based wind estimation, energy-based flight risk assessments for autonomous UAVs, and indoor exploration leveraging predictive models. Brady holds a B.S. in Electrical Engineering from Brigham Young University.
% \end{IEEEbiography}
% \vspace{-33pt}
% \begin{IEEEbiography}[{\includegraphics[width=1in,height=1.25in,clip,keepaspectratio]{figs/bios/suvarna_profile.png}}]{Nayana Suvarna}
% is a Masters in Robotics student at Carnegie Mellon University. Her research focuses on information gathering algorithms for teams of aerial vehicles. Nayana holds a B.S. in Computer Engineering from the University of Pittsburgh.
% \end{IEEEbiography}
% \vspace{-33pt}
% \begin{IEEEbiography}[{\includegraphics[width=1in,height=1.25in,trim={1cm 0 1.2cm 0 },clip,keepaspectratio]{figs/bios/andrew_jong.png}}]{Andrew Jong}
% is a Ph.D. student in Robotics at Carnegie Mellon University. His research focuses on robotics for disaster response, and covers path planning, perception in degraded environments, agile navigation, and safety. Andrew holds a B.S. in Computer Science from San Jos\'e State University and a M.S. in Robotics from Carnegie Mellon University.
% \end{IEEEbiography}
% \vspace{-33pt}
% \begin{IEEEbiography}[{\includegraphics[width=1in,height=1.25in,clip,keepaspectratio]{figs/bios/satrajit.jpg}}]{Satrajit Chatterjee}
% is a Master's student in Robotics at the GRASP Lab, University of Pennsylvania. His research is focused on path-planning for robotics and includes work in multi-target search and tracking, and vision-based generative techniques for path-planning in semantically complex environments. Satrajit holds a B.Tech in Computer Science and Engineering from the National Institute of Technology, Tiruchirappalli, India. 
% \end{IEEEbiography}
% \vspace{-33pt}
% \begin{IEEEbiography}[{\includegraphics[width=1in,height=1.25in,clip,keepaspectratio]{figs/bios/junbin.jpg}}]{Junbin Yuan}
% is a Ph.D. candidate in mechanical engineering at Carnegie Mellon University. His research focuses on planning algorithms for robotics, and covers coverage path planning and multi-target search and tracking. Junbin received his B.Eng. in Electronics Engineering from Hong Kong University of Science and Technology 
% \end{IEEEbiography}
% \vspace{-33pt}
% \begin{IEEEbiography}[{\includegraphics[width=1in,height=1.25in,clip,keepaspectratio]{figs/bios/basti.jpg}}]{Sebastian Scherer} is an Associate Research Professor at the Robotics Institute (RI) at Carnegie Mellon University (CMU). His research focuses on enabling autonomy for unmanned rotorcraft to operate at low altitude in cluttered environments. He and His team have shown the fastest and most tested obstacle avoidance on an Yamaha RMax (2006), the first obstacle avoidance for micro aerial vehicles in natural environments (2008), and the first (2010) and fastest (2014) automatic landing zone detection and landing on a full-size helicopter. Dr. Scherer received his B.S. in Computer Science, M.S. and Ph.D. in Robotics from CMU in 2004, 2007, and 2010. He is a Siebel scholar and a recipient of multiple paper awards and nominations, including AIAA@Infotech 2010 and FSR 2013. His research has been covered by the national and internal press including IEEE Spectrum, the New Scientist, Wired, der Spiegel, and the WSJ. His work on self-landing helicopters has received the Popular Science Best of What’s New 2010 Award and in Fall 2016 he demonstrated his inspection robots to President Obama.
% \end{IEEEbiography}

% \vspace{11pt}

% \bf{If you will not include a photo:}\vspace{-33pt}
% \begin{IEEEbiographynophoto}{John Doe}
% Use $\backslash${\tt{begin\{IEEEbiographynophoto\}}} and the author name as the argument followed by the biography text.
% \end{IEEEbiographynophoto}




\vfill

% \listoftodos[To Do List]

\end{document}

