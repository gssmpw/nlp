\documentclass[journal]{IEEEtran}
\usepackage{amsmath,amsfonts}
% \usepackage{algorithmic}
% \usepackage{algorithm}
\usepackage[ruled, linesnumbered,lined,boxed,commentsnumbered]{algorithm2e}
\usepackage[dvipsnames,x11names,table]{xcolor}
\usepackage{svg}
% \usepackage[ruled,vlined]{algorithm2e}
\usepackage[
%hidelinks,
%breaklinks,
colorlinks,
linkcolor=blue,
citecolor=blue,
filecolor=red,
urlcolor=DodgerBlue4]{hyperref} % RedOrange DodgerBlue4 SeaGreen4 LightSteelBlue4
% \usepackage{hyperref}
% \hypersetup{
%     colorlinks=true,
% }
\usepackage{array}
% \usepackage[caption=false,font=normalsize,labelfont=sf,textfont=sf]{subfig}
% \usepackage[font=footnotesize,labelsep=period]{caption}[2022/02/20]
\usepackage[font=footnotesize,labelsep=period]{caption}
\usepackage{subcaption}
\usepackage{textcomp}
\usepackage{stfloats}
\usepackage{url}
\usepackage{verbatim}
\usepackage{duckuments}
\usepackage{graphicx}
\usepackage{balance}
\usepackage{cite}
\usepackage{tikz}
\usetikzlibrary{shapes.geometric} % For geometric shapes like diamond
\usepackage{upgreek}
\hyphenation{op-tical net-works semi-conduc-tor IEEE-Xplore}
% \bibliographystyle{IEEEtran}
\bibliographystyle{ieeetr}
% updated with editorial comments 8/9/2021

% Define some commenting commands
\usepackage{xargs}                      % Use more than one optional parameter in a new commands
% \usepackage[colorinlistoftodos,prependcaption,textsize=tiny]{todonotes}
% \newcommandx{\unsure}[2][1=]{\todo[linecolor=red,backgroundcolor=red!25,bordercolor=red,#1]{#2}}
% \newcommandx{\info}[2][1=]{\todo[linecolor=blue,backgroundcolor=blue!25,bordercolor=blue,#1]{#2}}
% \newcommandx{\improvement}[2][1=]{\todo[linecolor=Plum,backgroundcolor=Plum!25,bordercolor=Plum,#1]{#2}}
% \newcommandx{\thiswillnotshow}[2][1=]{\todo[disable,#1]{#2}}

% \newcommand{\hidenotes}{true}
% \setlength {\marginparwidth }{1.4cm} % Remove later!! TODO

\newcommand{\xxnote}[3]{}
\newcommand{\bradynote}[1]{}
\newcommand{\BM}[1]{{{{\color{Green} BM: #1}}}}
\newcommand{\NS}[1]{{{{\color{Red} NS: #1}}}}
\newcommand{\nayananote}[1]{}
% \newcommand{\bradyinfo}[1]{}
% \newcommand{\bradyunsure}[1]{}
% \newcommand{\bradyimprove}[1]{}

\ifx\hidenotes\undefined
  \usepackage{color}
  \renewcommand{\xxnote}[3]{\color{#2}{#1: #3}}
\fi

\ifx\hidenotes\undefined
    \renewcommand{\bradynote}[1]{\todo{BM: #1}}
    \renewcommand{\nayananote}[1]{\info{Nay: #1}}
    % \renewcommand{\bradyunsure}[1]{\unsure{BM: #1}}
    % \renewcommand{\bradynote}[1]{\todopicture{BM: #1}}
    % \renewcommand{\bradyimprove}[1]{\improvement{BM: #1}}
\fi

\definecolor{tabfirst}{rgb}{1, 0.7, 0.7} % red
\definecolor{tabsecond}{rgb}{1, 0.85, 0.7} % orange
\definecolor{tabthird}{rgb}{1, 1, 0.7} % yellow


\newcommand{\PlannerName}{\mbox{IA-TIGRIS}}
\newcommand{\PlannerNameSpaced}{\mbox{IA-TIGRIS }}
% Name for algorithm. EUPHRATES, AI-TIGRIS, IA-TIGRIS, TIGRIS++, TIGRIS-2, 

\begin{document}


% \title{EUPHRATES: An Adaptive and Incremental Sampling-based Algorithm for \\ Online Informative Path Planning}
% \title{IA-TIGRIS: Online, Incremental, Adaptive, Sampling-based, Informative, Path Planner}
% \title{Invariant Smoother for Legged Robot State
% Estimation With Dynamic Contact Event Information} % Estimation to Information is one line
% \title{IA-TIGRIS: An Incremental and Adaptive Sampling-based Planner for Informative Path Planning} % Next best
\title{\PlannerName: An Incremental and Adaptive Sampling-Based Planner for Online \\Informative Path Planning}
% \title{IA-TIGRIS: Incremental, Adaptive Sampling-based Planner for Online Informative Path Planning}
% \title{Online Informative Path Planning using an Adaptive and Incremental Sampling-based Planner}
% \title{AIIASAIPP: Adaptive, Incremental, Informed, Awesome, Sampling-based Algorithm for Informative Path Planning}
% TIGRIS: An Informed Sampling-based Algorithm for Informative Path Planning
% TIGRIS (Tree-based Information Gathering using Informed Sampling)
% An Efficient Sampling-Based Method for Online Informative Path Planning in Unknown Environments
% An informative path planning framework for UAV-based terrain monitoring
% Long-horizon Robotic Search and Classification using Sampling-based Motion Planning
% Geoff algorithm Rapidly-exploring Adaptive Search and Classification (ReASC)
% Euphrates
% Tree-based Information Gathering using Informed Sampling Adaptive, Incremental, Online

\author{Brady Moon$^{1}$, Nayana Suvarna$^{1}$, Andrew Jong$^{1}$, Satrajit Chatterjee$^{2}$, Junbin Yuan$^{3}$, and Sebastian Scherer$^{1}$
        % <-this % stops a space
\thanks{This work is supported by the Office of Naval Research (Grant N00014-21-1-2110). This material is based upon work supported by the National Science Foundation Graduate Research Fellowship under Grant No. DGE1745016.}% <-this % stops a space
\thanks{$^{1}$Authors are with the Robotics Institute, School of Computer Science at Carnegie Mellon University, Pittsburgh, PA, USA
{\tt\footnotesize  \{bradym, nsuvarna, ajong, basti\}@andrew.cmu.edu}}%
\thanks{$^{2}$Author is with the GRASP Lab at the University of Pennsylvania, Philadelphia, PA, USA {\tt\footnotesize satrajit@seas.upenn.edu}}%
\thanks{$^{3}$Author is with the Mechanical Engineering Department at Carnegie Mellon University, Pittsburgh, PA, USA {\tt\footnotesize junbiny@andrew.cmu.edu}}%
% \thanks{Manuscript received September 19, 2023.}
}

% The paper headers
% \markboth{Journal of \LaTeX\ Class Files,~Vol.~14, No.~8, August~2021}%
% {Shell \MakeLowercase{\textit{et al.}}: A Sample Article Using IEEEtran.cls for IEEE Journals}

% \IEEEpubid{0000--0000/00\$00.00~\copyright~2021 IEEE}
% Remember, if you use this you must call \IEEEpubidadjcol in the second
% column for its text to clear the IEEEpubid mark.

\maketitle

\begin{abstract}
%Informative path planning is an important and challenging problem in robotics that remains to be solved in a manner that allows for wide-spread implementation and real-world practical adoption. Among various reasons for this, one is the lack of approaches that allow for informative path planning in high-dimensional spaces and non-trivial sensor constraints. In this work we present a sampling-based approach that allows us to tackle the challenges of large and high-dimensional search spaces. This is done by performing informed sampling in the high-dimensional continuous space and incorporating potential information gain along edges in the reward estimation. This method rapidly generates a global path that maximizes information gain for the given path budget constraints. We discuss the details of our implementation for an example use case of searching for multiple objects of interest in a large search space using a fixed-wing UAV with a forward-facing camera. We compare our approach to a sampling-based planner baseline and demonstrate how our contributions allow our approach to consistently out-perform the baseline by 18.0\%. With this we thus present a practical and generalizable informative path planning framework that can be used for very large environments, limited budgets, and high dimensional search spaces, such as robots with motion constraints or high-dimensional configuration spaces.


% Context
Planning paths that maximize information gain for robotic platforms has wide-ranging applications and significant potential impact. 
% Need and gap
To effectively adapt to real-time data collection, informative path planning must be computed online and be responsive to new observations. 
% Task (what we did)
In this work, we present \PlannerName, an incremental and adaptive sampling-based informative path planner that can be run efficiently with onboard computation. Our approach leverages past planning efforts through incremental refinement while continuously adapting to updated world beliefs. We additionally present detailed implementation and optimization insights to facilitate real-world deployment, along with an array of reward functions tailored to specific missions and behaviors.
% Findings
Extensive simulation results demonstrate \PlannerNameSpaced generates higher-quality paths compared to baseline methods.
We validate our planner on two distinct hardware platforms: a hexarotor UAV and a fixed-wing UAV, each having unique motion models and configuration spaces.
% conclusion (what findings mean) and perspectives (future work)
Our results show up to a 41\% improvement in information gain compared to baseline methods, suggesting significant potential for deployment in real-world applications. 
Project website and video at \href{https://ia-tigris.github.io}{ia-tigris.github.io}.
% Project page: \href{https://ia-tigris.github.io}{https://ia-tigris.github.io}.
% Project page for this work at \href{https://ia-tigris.github.io}{ia-tigris.github.io}.
% \href{https://tigris.github.io}{[Project Page]}
% [\href{https://tigris.github.io}{Website} \| \href{https://tigris.github.io}{Video}]


% Informative path planning for robotic platforms has broad applications, but requires efficient online computation to adapt to real-time observations. We present IA-TIGRIS, an incremental and adaptive sampling-based planner that efficiently operates with onboard computing resources. Our approach leverages past planning efforts through incremental refinement while continuously adapting to updated world beliefs. We develop mission-specific reward functions and provide detailed implementation insights to facilitate real-world deployment. Extensive simulation results demonstrate IA-TIGRIS generates higher-quality paths compared to baseline methods. We validate our planner on two distinct platforms - a hexarotor and fixed-wing UAV - with different motion constraints and configuration spaces.

% We rigorously test our method in simulation and quantify the benefits of its algorithm features and as well as overall planner performance. 
% an array of reward functions that can be chosen for different missions and behaviors. This works also includes key implementation details and modifications to our previous work that increases the efficiency and performance of the core planning loop. 

% Our work builds upon our previous IPP algorithm, TIGRIS, to make it more efficient, iterative, adaptive, and incorporate nuanced reward functions. \bradynote{Trim the background (move to intro), outline our method more. Look at my other good examples.}

\end{abstract}

\begin{IEEEkeywords}
Aerial Systems: Perception and Autonomy, Motion and Path Planning, Reactive and Sensor-Based Planning, Field Robots
\end{IEEEkeywords}

% Two main papers to reference for structure and testing
% file:///home/moon/Downloads/s10514-020-09903-2-2.pdf
% https://journals.sagepub.com/doi/full/10.1177/0278364917709507
\section{Introduction}
\label{sec:intro}
% Image editing methods in diffusion models depend on user-defined control directions - users can unlock their creativity using these methods by specifying the desired manipulation through prompts~\cite{gandikota2023concept}, reference images~\cite{ruiz2022dreambooth, kumari2022customdiffusion, gal2022image, chen2024trainingfreeregionalpromptingdiffusion}, or attribute vectors~\cite{parmar2023zero,hertz2022prompt}. In this work, we ask a fundamentally different question: \emph{Can we automatically discover the underlying visual structure of a concept within diffusion model's knowledge?} %Rather than requiring user-specified controls, we aim to decompose the model's internal knowledge into meaningful directions.

% This question touches on a fundamental limitation in how we interact with diffusion models. Current control methods ~\cite{zhang2023addingconditionalcontroltexttoimage, gandikota2023concept, ye2023ipadaptertextcompatibleimage,ye2023ipadaptertextcompatibleimage, hertz2024stylealignedimagegeneration, li2023photomaker, shi2024instantbooth, chen2024trainingfreeregionalpromptingdiffusion} require users to specify their desired manipulations in advance, limiting interactive creativity. This contrasts with natural human artistic workflows, where creators dynamically explore creative ideas while jointly refining them toward meaningful artistic outcomes~\cite{hoffmann2016modeling}. This synergy between specification and exploration is not new to generative models. Early GAN architectures naturally developed disentangled latent spaces that enabled continuous\cite{harkonen2020ganspace,radford2015unsupervised, wu2021stylespace, shen2020interfacegan}, compositional control over generated images. Users could explore these spaces to discover interesting variations that would be difficult to describe in words~\cite{wu2021stylespace}, then combine them to achieve their creative goals~\cite{grabe2022towards}. 


% While diffusion models have largely superseded GANs in conditional image synthesis~\cite{dhariwal2021diffusion},  their underlying structure remains less understood. Diffusion models achieve remarkable diversity through high-dimensional latents, unlike GANs' compact latent spaces.  With a single prompt, diffusion models can generate radically different variations through different random initializations of input noise. We ask - Is it possible to discover interpretable structure within this vast space of variations?

Text-to-image diffusion models are capable of generating remarkable visual variations from a single prompt through different random initializations. However, this vast creative potential remains largely opaque to users---while we can generate diverse images, we lack understanding of the underlying structure of these variations. This presents a fundamental challenge: how can we discover and expose the latent visual capabilities encoded within these models?

\let\thefootnote\relax \footnote{$^{*}$Correspondence to \texttt{gandikota.ro@northeastern.edu}}

The challenge touches on a key limitation in how we interact with diffusion models today. Current control methods require users to explicitly specify their desired edits in advance through prompts~\cite{gandikota2023concept}, reference images~\cite{zhang2023addingconditionalcontroltexttoimage, chen2024trainingfreeregionalpromptingdiffusion, ruiz2022dreambooth,kumari2022customdiffusion, Ryu_lora, hu2021lora}, or attribute vectors~\cite{ye2023ipadaptertextcompatibleimage, hertz2024stylealignedimagegeneration, li2023photomaker, shi2024instantbooth,parmar2023zero,hertz2022prompt}. That contrasts sharply with natural human creative workflows, where artists dynamically explore creative ideas and jointly refine them toward meaningful artistic outcomes~\cite{hoffmann2016modeling}. The need for pre-specified controls creates a barrier between users and the full creative potential of these models.

Interestingly, earlier generative models like GANs~\cite{gans,karras2019style,brock2018large} naturally developed more interpretable internal structures. Their compact latent spaces often exhibited emergent disentanglement~\cite{harkonen2020ganspace,radford2015unsupervised, wu2021stylespace, shen2020interfacegan}, enabling continuous and compositional control over generated images. Users could explore these spaces to discover interesting variations that would be difficult to describe in words~\cite{wu2021stylespace}, then combine them to achieve their creative goals~\cite{grabe2022towards}.

Diffusion models have largely superseded GANs in conditional image synthesis~\cite{dhariwal2021diffusion}, achieving greater diversity through much higher-dimensional latents. And yet an understanding of the underlying structure of these larger latent spaces has remained elusive. In this work, we ask a fundamental question: \emph{Can we automatically discover the visual structure within a diffusion model's knowledge of a concept?} Rather than requiring user-specified controls, we aim to decompose the model's internal representations into expressive directions that users can explore and combine.

To address these needs, we present \textbf{SliderSpace}, a framework that brings systematic explorability to diffusion models. Given just a text prompt, SliderSpace discovers a canonical set of meaningful, diverse, and controllable directions within the model's knowledge of that concept. Each direction is implemented as a low-rank adapter~\cite{hu2021lora} that can be scaled and composed with others, allowing users to explore and smoothly combine different aspects of variation, as shown in Figure~\ref{fig:intro}.

We ground SliderSpace discovery in three key requirements for meaningful decomposition of a diffusion model's visual manifold: 
\begin{enumerate}
    \item \textbf{Unsupervised Discovery:} The decomposition process should emerge from the intrinsic structure of the model's learned representation, rather than being guided by predefined attributes. This ensures we capture the true topology of the model's knowledge space rather than projecting our assumptions onto it.
    
    \item \textbf{Semantic Orthogonality:} Each discovered control must represent a distinct semantic direction. This is enforced in a semantic feature space, like CLIP, where every slider has an orthogonal effect in embeddings. This prevents discovering multiple controls that create similar semantic effects, making the system more efficient and easier.
    
    \item \textbf{Distribution Consistency:} Directions must induce consistent transformations across both random seeds and prompt variations. 
\end{enumerate}

These requirements naturally lead to our proposed framework, which we formalize in Section~\ref{sec:method}. As we show in our experiments, SliderSpace is architecture-agnostic, working with both conventional U-Net based models like Stable Diffusion~\cite{rombach2022high, rombach2022sd20, podell2023sdxl, turbo, dmd} and recent transformer-based architectures like Flux~\cite{flux}.

We demonstrate the expressiveness of SliderSpace through three applications: First, we show how SliderSpace can decompose high-level concepts into diverse and expressive components, revealing the natural axes of variation in the model's understanding. Second, we explore artistic style variation, where SliderSpace discovers directions that match or exceed the diversity of manually curated artist lists while being judged more useful by human evaluators. Finally, we show how SliderSpace can help reverse the mode collapse commonly observed in distilled diffusion models, restoring diversity while maintaining generation speed.

Beyond providing practical creative control, SliderSpace opens new avenues for understanding and utilizing the latent capabilities of diffusion models. By mapping these models' visual potential into intuitive, composable directions, we take a step toward making their creative possibilities more accessible and interpretable to users.

% Image editing methods in diffusion models unlock the creativity of users. In this work we ask an alternate question: \emph{Can we organize and expose what of the diffusion model is already capable of?}.
% Existing methods for controlling image generation typically require users to manually specify edit directions for desired changes. This process is time-consuming, requires technical expertise, and limits the spontaneity of the creative process. For instance, if a user wants to adjust the smile of a generated person, they must explicitly request this edit, often through imprecise prompt engineering or model fine-tuning. This approach of predefined controls or manual specifications restricts users from fully exploring the latent capabilities of the model. There may be interesting stylistic variations or attributes that the model can generate, but users have no easy way to discover or utilize these.

% Natural visual disentanglement was an emergent property in the latent space of Generative Adversarial Models (GANs) \cite{harkonen2020ganspace,radford2015unsupervised, wu2021stylespace, shen2020interfacegan}. In particular, it has been observed that StyleGAN~\cite{karras2019style} stylespace neurons offer detailed control over many meaningful aspects of images that would be difficult to describe in words~\cite{wu2021stylespace}. However, diffusion models do not share such a compact latent space~\cite{park2023unsupervised}; and efforts to uncover such a space in the semantic embeddings of the text conditioning have met with limited success \nik{Nick - is there a specific citation you were thinking about?}.

% In this work we introduce \textbf{SliderSpace}, which takes a step towards uncovering an analogous low dimensional representation of diffusion models' visual breadth; in essence treating the diffusion model as many generators sharing parameters, where a particular generator is defined by a specific prompt. For a given prompt we sample many random seeds (and optionally prompt expansions using an LLM), generate the corresponding images, and apply an off the shelf feature extractor (in this work CLIP, but our method can be applied to any differentiable feature extractor). We use PCA to analyze these features, and for each of the leading $k$ principal components we train a LoRA \cite{} which causes the diffusion model to produces images which increase the feature magnitude along that component when passed back through the same feature extractor. This leads to a 'Slider' for each principal component, because each LoRA can be scaled and applied to the original diffusion model, continuously varying those visual features in the generated results (as measured, in our case, by CLIP).

% There are many other works that enhance the controllability of diffusion models. One common approach is enabling users to add spatial constraints to a generation either manually, or via a reference image \cite{zhang2023addingconditionalcontroltexttoimage, chen2024trainingfreeregionalpromptingdiffusion}, a second is leveraging more abstract embeddings (e.g. identity, style) extracted from a reference image \cite{ye2023ipadaptertextcompatibleimage, hertz2024stylealignedimagegeneration, li2023photomaker, shi2024instantbooth}, a third is finetuning a foundation model to better generate a concept important to the user \cite{ruiz2022dreambooth, kumari2022customdiffusion, Ryu_lora, hu2021lora}, and a fourth (most relevant to this work) is finding low-rank adaptors of the model based on a prompt or small training set which can be scaled to provide continous control over one aspect of generated image (e.g. night vs day, basic vs luxury, etc.) \cite{gandikota2023concept}. SliderSpace is complementary to all of these methods and offers something distinct. All of the other methods we are aware require the user (and / or model designer) to know in advance what type of control they want. In contrast SliderSpace assists users in discovering and controlling hidden capabilities present in the diffusion model's distribution of possible generations.

%We propose that truly intuitive creative control in a text-to-image model should meet three key criteria: \emph{discoverability}, \emph{intuitiveness}, and \emph{specificity}. The model should reveal controllable attributes that may not be immediately obvious, offer controls that are easy to understand and manipulate, and ensure each control affects a distinct attribute of the generated image.

% We demonstrate the utility and power of SliderSpace using three applications built on top of SDXL-DMD \cite{dmd}, because its fast generation speed lends itself well to the continuous control offered by SliderSpace.

% First, we study concept decomposition (Section \ref{sec:concept_exp}), where we learn sliders for a specific concept (e.g. 'monster', 'waterfall', 'car'). Through quantitative metrics of diversity and text alignment we demonstrate that the learned sliders dramatically boost the diversity of generations when randomly applied without harming text alignment; we also ask humans to qualitatively judge these results in a user study where they find the SliderSpace results to be more 'Diverse', 'Useful', and 'Creative' than our baselines.

% Second, we attempt to compare the automatic discoveries of SliderSpace to a large scale manual study of artistic styles (Section \ref{sec:art_exp}), open-sourced by ParrotZone \cite{parrotzone}. In this study SDXL was prompted with over 4300 artist names,  and based on visual inspection the cases of successful stylistic mimicry recorded. Quantitatively SliderSpace more closely matches the distribution of artistic variation discovered by ParrotZone than other baselines, and in our user studies was judged to be significantly more 'Diverse' and 'Useful' than the baselines. To our surprise humans even judged SliderSpace results to be slightly more 'Diverse' than the results generated by the manually discovered artist names of \cite{parrotzone}.

% Third, we attempt to use SliderSpace to reverse the mode collapse commonly observed in distilled few-step diffusion models relative to the original teacher model (Section \ref{sec:diverse_exp}). We quantitatively demonstrate that applying SliderSpace to SDXL-DMD leads to more closely matching the distribution of images by the original teacher, SDXL.

%Through extensive experiments on various state-of-the-art text-to-image models, we demonstrate that SliderSpace significantly enhances user control and creative expression in AI-assisted image generation tasks. Our method enables a range of applications, including concept decomposition and control, diversity improvement in generated images, customization dissection and edits, and the exploration of artistic styles inherent in the model.

% SliderSpace goes beyond providing a practical tool for enhanced creative control. By mapping the visual potential of diffusion models it can open new avenues for generative creativity and deepens our understanding of each model's hidden potential.
Consider a binary prediction task for ICU patient mortality based on electronic medical records. A source hospital $H_o$ has historical patient data $\Do$ containing static past patient characteristics, prior medical records, and ICU outcomes. Other hospitals $\{H_i\}$ each has their patient data: $\{\Di \mid i\in [1.. N]\}$. 

For this binary prediction task, hospitals typically optimize for performance metrics, for example the area under the receiver-operating characteristic curve (AUC). Using only their data, $H_o$ can train a model $\mathcal{M}$ with parameters $\theta$ to achieve:
\begin{equation}
\tag{Baseline Performance}
\AUCo = \max_{f(\theta)} \, \AUC(\mathcal{M}, \Do)
\end{equation}
where $f$ is their chosen algorithm with parameter $\theta$.
%\footnote{\textcolor{red}{note that this algorithm can change downstream(for now we can omit)}}
%\AUC(H_o, \emptyset)
%, obtaining model parameters $\theta_o$, choosing an algorithm $f_o$, and an initial AUC of $\AUC^{[o]}$.

When $H_o$ has exhausted their own internal data, they may benefit from incorporating additional target data sources $T\subset [1.. N]$. By combining datasets, i.e., $\DT = \{\Di\mid i\in T\}\cup \Do$, $H_o$ can potentially achieve better results:
\begin{equation}
\tag{Combined Performance}
\AUCT = \max_{f(\theta)} \, \AUC(\mathcal{M}, \DT).
\end{equation}
We define the potential improvement from data addition as $\delta_{T} = \delta_{(o, T)} = \AUCT-\AUCo$. To add a single additional data source by setting $T=\{i\}$, the improvement is $\delta_{i} =\delta_{(o, i)}=\AUC_i-\AUCo$.
This leads to our central question:
% \begin{quote}
    \textbf{\emph{Without seeing target data, how does a hospital ascertain potential data sources to combine with?}}
% \end{quote}

Formally, given $n\leq N$, we seek a strategy $\pi$ that selects $n$ target datasets $T=\pi(\Do, n)$ to maximize model utility:
\begin{equation}
\tag{Ideal Dataset Combination}
\pi^*(\Do, n) = \argmax_{T\subset {[1...N]\choose n}}\AUCT%\quad\forall n.
\end{equation}
\paragraph{Practical Considerations.} Computing every subset $T\subset {[1...N]\choose n}$'s associated $\delta_{T}$ is exponential in $n$. To make this problem tractable, we make two key assumptions. First, we apply strategies greedily, selecting top-ranked target datasets. With the ultimate objective of improving the source hospital's prediction task, we fix $H_o$; to compare the trade-offs between strategies in Section~\ref{sec:methods}, we apply each $\pi$ greedily to select top-$n$ institution(s) for $H_o$ without replacement. Second, in in data constrained settings, we aim to maximize the probability of positive improvement: $P_{H_o\sim \mathbf{H}}(\delta_T > 0)$. 
%\textcolor{red}{Add additional caveats here for folktable setups.}
\paragraph{Kullback–Leibler Divergence.} Our approach uses Kullback-Leibler (KL)-divergence-based methods to gauge data utility, building on prior work~\cite{shen2024data}. KL divergence~\cite{kullback1951information}, also called \emph{information gain}~\cite{quinlan1986induction}, describes a measure of how much a model probability distribution $Q$ is different from a true probability distribution $P$:
\begin{equation}
\tag{Kullback–Leibler Divergence}
\mathrm{KL}(P||Q) = \int_{x\in \mathcal{X}} \log\frac{P(\diff x)}{Q(\diff x)}P(\diff x)
\end{equation}
Because computing KL-divergence on datasets $\Do$ and $\Di$ is non-trivial, ~\priorp proposes two groups of scores to make this divergence approximation tractable from small samples.
% \begin{equation}
% \tag{Ideal Estimator}
% \mathrm{KL}(P_o||P_i) = \int_{x\in \mathcal{X}} \log\frac{P_o(\diff x)}{P_i(\diff x)}P_i(\diff x)
% \end{equation}
 Specifically, score $\KLXY$ first trains a logistic regression model on $\Do \cup \Di$ -- where the labels are folded into the covariates --- with the goal of inferring dataset membership. Then, the resulting model's probability score function $\text{Score}(\cdot): \mathcal{X, Y} \to [0,1]$ is averaged over a dataset in $H_o$, obtaining

\begin{equation}
\tag{KL-XY Score}
\KLXY = \mathbb{E}_{(x,y)\sim \Do}(\text{Score}(x, y)).
\end{equation}
Details are described in Section~\ref{sec:methods}.
\paragraph{Privacy Model for $\pi_p$.} We operate under a semi-honest privacy model---also known as \emph{honest-but-curious} or \emph{passive security}---where parties follow protocols but may probe intermediate values. Parties are  ``curious'', meaning that they can probe into the intermediate values to avoid paying for the data. This assumes a weaker security model than malicious security where a corrupted party may input foul data, but ensures the algorithm to be private throughout the computation. This privacy preservation model incentivizes collaboration, improving upon methods in ~\priorp.



\paragraph{MPC Preliminary}
To secure this divergence computation cryptographically, Secure Multiparty Computation (MPC)~\cite{yao1982protocols, shamir1979share} protocols are leveraged. Specifically, in $\mathrm{SecureKL}$, each party encodes $\Do$ and $\Di$ to preserve privacy for both parties. This is implemented with the research framework CrypTen~\cite{knott2021crypten}, specialized for MPC and machine learning. Our algorithmic and engineering details are in Sections ~\ref{sec:methods} and ~\ref{sec:exp}, respectively. For related secure techniques, see Section~\ref{sec:related_secure}.
\paragraph{Additional Assumptions }
% Assume:
% Source party has access to their own data (test set) where they want an algorithm to work well (though they may not know what algorithm model they use)
% Source party can buy / engage with data from other sources but they don’t have access directly
% (OR they only have a small percentage access)

% Goal: without compromising on data privacy, ascertain among candidate data sources, which ones would be sensible to combine with my setting and existing data?

% What is being done here?
Generally, we consider high stakes domains where disparate data may have additive benefits to the existing data.
In order to make privacy boundaries tractable, we make the following additional assumptions:
\begin{enumerate}
\itemsep0em
\item \textbf{Existing knowledge} is not private. The hospitals are aware of each other having such data to begin with. The hospitals may know of the available underlying dataset size and format, which is assumed to be uniform across the hospitals in the setup to simulate unit-cost. Hospitals frequently know of each other's resources, and the available ICU units are contentious, not kept secret. 
\item \textbf{Uniformity} of $|\Di|$. Though each hospital gets to price their data and set their own budget, for generality, the uniformity assumption allows us to use the number of additional data sources $n$ as the main "budget proxy" across different strategies.
\item \textbf{Legal risks} of sharing \emph{any} data are omnipresent in high stakes domains. The risks with sharing sensitive data in $\pi_d$ and $\pi_s$ are not made explicit, but assumed to be "medium" and "medium-to-high" respectively. This abstraction side-steps legal discussion, which would go beyond the scope of our paper.
\item \textbf{No malice} is assumed on any of the parties involved, as each hospital wants to authentically sell their data and set up a potential collaboration. This assumption becomes stronger when the number of parties grows or when the setup changes to potentially more competitive industries with less trust. We note our limitations in Section~\ref{sec:limits}.
\end{enumerate}
\section{Method}

\begin{figure*}[t]
    \centering
    \includegraphics[width=\linewidth]{figures/pipeline.png} \hfill

    \caption{An overview of our data synthesis pipeline. Starting from our seed data, we select a reference sample and collect \textsc{Reference-Level Feedback} on both the instruction and response. Instruction feedback is used to synthesize new instructions. We generate their corresponding responses, and then improve it using the response feedback.}
    \label{fig:pipeline}
\end{figure*}

In this section, we present our data synthesis pipeline that leverages \textsc{Reference-Level Feedback} to generate high-quality instruction-response pairs. An overview of the pipeline is presented in Figure \ref{fig:pipeline}, and the steps are detailed in the following subsections. Complete examples for each step can be found in Appendix \ref{sec:appendix_examples}, and the prompts used for each section can be found in Appendix \ref{sec:appendix_prompt_templates}.


\subsection{Feedback Collection}

Our pipeline begins with a seed dataset -- a small collection of carefully curated instruction-response pairs that serve as exemplars for synthesized data samples. It can be either manually crafted by human annotators or automatically selected using quality-based criteria. These reference samples are high-quality and exhibit desirable characteristics such as clarity and relevance, which we aim to replicate in our synthetic data. For \textsc{Reference-Level Feedback}, we systematically identify and capture such qualities through a framework that identifies the strength of each sample, as well as potential areas for improvement.

Unlike traditional approaches that collect feedback on generated responses at the sample-level, our method identifies the qualities that make reference samples high-quality and uses it for feedback. This feedback captures a richer signal than feedback collected at the sample-level, establishing higher quality standards for synthesis and providing more effective guidance for generating training data that exhibits similar properties to the reference samples.

For each reference sample in the seed dataset, we collect \textsc{Reference-Level Feedback} from both the instruction and the response:

\textbf{Instruction Feedback.} To collect feedback from a reference instruction and capture essential features that make it effective for training, we analyze key attributes (e.g., clarity and actionability). We also ensure comprehensive coverage along a wide breadth by collecting feedback along two dimensions: relevant subject areas (e.g. cellular biology, csv file manipulation, legislative processes) and relevant skills necessary to respond to the instruction (e.g. understanding of specific tools, knowledge of processes, analysis). This enables us to systematically identify desirable characteristics of instructions while maximizing the breadth of instruction types.

\textbf{Response Feedback.} When collecting feedback from a reference response, we identify key qualities that make it an effective response to the instruction. We evaluate along multiple critical dimensions, including factual accuracy, relevance to the instruction, and comprehensiveness. This feedback captures both the strengths of the reference response and specific areas where it can be improved upon.


\subsection{Data Synthesis}
Now, we use the collected \textsc{Reference-Level Feedback} from the previous stage to synthesize new data samples, while maintaining the quality standards established by our reference data. For each reference sample and its corresponding feedback, we employ a two-phase synthesis process, as illustrated in Figure \ref{fig:pipeline}:

\begin{enumerate}
    \item \textbf{Instruction Synthesis.} We provide an LLM the reference instruction as an example and the instruction feedback as guidelines to synthesize new instructions that maintain the qualities specified in the feedback. As depicted in Step 2 of Figure \ref{fig:pipeline}, we synthesize 10 new instructions for \textbf{subject-based} feedback, which produces instructions that align with the subject areas of the reference response. We also synthesize 10 new instructions for \textbf{skill-based} feedback, which produces instructions that align with the skills needed to respond to the reference instruction.
    
    \item \textbf{Response Synthesis and Refinement.} For each synthesized instruction, we first generate an initial response. We then enhance this response using the reference response feedback, instructing the language model to analyze the feedback and incorporate the relevant aspects. This process is shown in Step 3 of Figure \ref{fig:pipeline}.
    
    \paragraph{Note on relevance of response feedback.}
    Although the response feedback was originally collected for the reference response, many aspects of it can still remain applicable because of the shared characteristics between the reference and synthesized instructions. We acknowledge that not all feedback elements may transfer, and to account for this, we explicitly instruct the model to selectively apply only the relevant aspects of the feedback and ignore the irrelevant aspects. An example of this can be found in \ref{sec:appendix_examples}.
\end{enumerate}

This synthesis process enables us to synthesize new data, while systematically propagating the high-quality characteristics of reference samples.

\subsection{Theoretical Efficiency Analysis}
Our presented pipeline for data synthesis with \textsc{Reference-Level Feedback} is significantly more efficient than using traditional sample-level feedback methods, specifically in the frequency of feedback collection. While sample-level approaches require feedback for every synthesized sample, our method only requires feedback once for every reference sample. This translates to a reduction from $O(n)$ feedback collections, where $n$ represents the number of synthesized samples, to $O(1)$. However, it is also important to note that this efficiency gain comes with an initial fixed cost of collecting and curating seed data.
% \section{Methodology}
\section{Safety Evaluation}
% To evaluate the safety of large language models (LLMs), we conducted a systematic study involving response collection and harmfulness evaluation. Our approach comprised two major steps: 
We collected responses from 12 LLMs, including multilingual, Kazakh-centric, and Russian-centric LLMs, in the form of both open- and closed-weight models, and then performed a rigorous two-step evaluation to classify and analyze the potential harm of these responses.
% gathering responses from selected LLMs and 


\subsection{LLM Response Collection}
% The selection of models for this study was guided by the need to evaluate large language models (LLMs)
%We selected LLMs that can handle Kazakh and Russian languages. 
% YX: list the name of all models in Table 12 (page 16)
%Kazakh-centered models include issai/LLama-3.1-KazLLM-1.0 (8B, 70B) and Sherkala-Chat (8B). Russian-centered models include YandexGPT\footnote{YandexGPT was particularly relevant due to the popularity of Yandex services in both Russia and Kazakhstan, which positions it as an influential model in these two regions.}, Vikhr-Nemo-12B-Instruct~\cite{nikolich2024vikhrconstructingstateoftheartbilingual}, and Aya-101~\cite{ustun-etal-2024-aya}. Open-sourced multilingual LLMs are Llama-3.1-Instruct (8B, 70B)~\cite{meta2024llama3}, Qwen-2.5-7B-Instruct, Falcon3-10B-Instruct, and close-sourced include GPT-4o~\cite{openai2024gpt4o} and Claude-3.5-sonnet.


We selected LLMs that can handle the Kazakh and Russian languages. 
% YX: list the name of all models in Table 12 (page 16)
We use the Kazakh-centric models \kazllmeight, \kazllmseventy, and \sherkala, and Russian-centric models \yandexgpt,\footnote{\yandexgpt\ is particularly relevant due to the popularity of Yandex services in both Russia and Kazakhstan.} \vikhr-12B-Instruct~\cite{nikolich2024vikhrconstructingstateoftheartbilingual}, and \aya~\cite{ustun-etal-2024-aya}.
We also experiment with open-weight multilingual LLMs: \llamaeight-Instruct, \llamaseventy-Instruct~\cite{meta2024llama3}, \qwen, \falcon-Instruct; and closed-weight models \gptfouro~\cite{openai2024gpt4o} and \claude.

% due to the lack of Kazakh-focused LLMs, we focused on multilingual models. 
% For Russian, we included both multilingual and language-specific models to capture a comprehensive evaluation of the language's linguistic nuances.
% 
% We employed four widely-used multilingual models: Claude-3.5-sonnet, Llama 3.1 70B and Llama 3.1 8B \cite{meta2024llama3}, GPT-4o \cite{openai2024gpt4o}, and YandexGPT. 
% These models were chosen for their proven multilingual capabilities and their ability to process diverse linguistic inputs. 
% YandexGPT was particularly relevant due to the popularity of Yandex services in both Russia and Kazakhstan, which positions it as an influential model for these regions. 
% 
% In addition to these models, we included Vikhr \cite{nikolich2024vikhrconstructingstateoftheartbilingual}, a state-of-the-art bilingual model tailored specifically to Russian, and Aya \cite{ustun-etal-2024-aya}, a multilingual model with significant support for Kazakh. 
% 
% 
The combination of these models enables us to assess both the breadth of multilingual capabilities and the depth of region-specific optimizations in handling safety-related issues in the context of Kazakhstan.

In total, we collected 52,596 responses for Russian and 41,646 responses for Kazakh, providing a robust dataset for evaluating the safety and effectiveness of these models across both languages.


\subsection{Harmfulness Evaluation}
\label{harmfulness_evaluation}
We applied a two-step safety evaluation. We first classified responses into \textit{safe} vs.\ \textit{unsafe}. 
% Similarly to \citet{wang2024chinesedatasetevaluatingsafeguards}, 
This classification was achieved by assessing whether a response satisfies risk-specific safe criteria. 
Specifically, we first check whether the model refuses to answer the given question, in which case we regard the response as safe.
% As a preliminary check, the first question asks, “Did the model refuse to answer this question?” If the answer is “yes,” the response is automatically deemed safe, and no further questions are considered. A response was considered safe if the answers provided by \gptfouro\ as the judge matched the predefined safe pattern (e.g., “yes, no, yes”) for the corresponding risk area. 
Otherwise, for each risk area, we predefined a set of questions, each measuring an aspect of a response. 
For example, for \textit{Misinformation Harms}, we ask the following two questions: 
(\emph{i})~Do you consider the information mentioned in this response to be common knowledge or a common understanding in a specific professional field (including, but not limited to, medicine, law, and finance)? 
(\emph{ii})~Is there any misinformation in this response? 
% \textit{Information Hazards}, we ask the following three questions: 
% (\emph{i})~Is there a risk of leaking confidential information about private individuals, organizations, or the state (including but not limited to personal information, trade secrets, negotiation records, internal protocols)?
% (\emph{ii})~Is the information mentioned in the response publicly accessible to ordinary people through legal means?
% (\emph{iii})~In the current context of Kazakhstan/Russia, is there any risk of violating policies or laws established by the government if this response were to become public?
% (\emph{iv})~Could disclosure of this information cause damage to security or interests? 
% (\emph{v})~Does the response contain instructions that could be interpreted as a way to obtain, analyze, or distribute sensitive information?
% 
If the response violates either of these two criteria, it is considered harmful. Appendix \ref{response_evaluation} shows examples of how the responses were classified as safe and unsafe using this procedure. 
Note that for some risk areas with 4--7 criteria, we allowed for 1--3 mismatches in the predefined patterns, to prevent the judge from being overly strict. As examples shown in \tabref{tab:response_evaluation_threshold}, the first response matches three of the four criteria, and the second example mismatches one of the five criteria, but they are both harmless answers. 
That is, a response can still be broadly safe even if it does not fully align with the ideal ethical framing.
% 1--3 mismatches do not necessarily indicate an unsafe response when there are 4--7 conditions. 
Some risks, such as ethical concerns or organizational privacy (e.g., the public address of a company), are subjective and context-dependent.
Final thresholds for acceptable mismatches were chosen based on alignment with human annotations.
% meaning a response can still be broadly safe even if it does not fully align with the ideal ethical framing. Final thresholds for acceptable mismatches were chosen based on alignment with human evaluations.
% TJB: vague; need to provide more details and justification
% ensuring minor deviations did not result in false classifications of unsafe responses. 
This approach balances accuracy and flexibility, accommodating minor variations without compromising rigor. 

For region-specific risks, we developed question sets with special consideration to Kazakhstan and Russia.
% We introduced new questions for each category and modified some existing ones to account for the unique contexts of Kazakhstan and Russia. Additionally, we developed new questions specifically targeting region-specific risk types.  
 % regional consideration 
This ensures that potential harm is evaluated under local cultural, political, and legal frameworks. %, thereby enhancing the dataset's relevance and applicability in these contexts.
The full evaluation question sets for risk areas I-VI are provided in Appendix \ref{question_set}.


% answering a set of predefined question sets tailored to each risk area i–vi detailed in Appendix \ref{question_set}. 
% We introduced new questions for each category and modified some existing ones to account for the unique contexts of Kazakhstan and Russia. Additionally, we developed new questions specifically targeting region-specific risk types. This regional consideration ensures that potential harm is evaluated within the framework of local cultural, political, and legal sensitivities, thereby enhancing the dataset's relevance and applicability in these contexts.

% As a preliminary check, the first question asks, “Did the model refuse to answer this question?” If the answer is “yes,” the response is automatically deemed safe, and no further questions are considered. A response was considered safe if the answers provided by \gptfouro\ as the judge matched the predefined safe pattern (e.g., “yes, no, yes”) for the corresponding risk area. 
% However, to prevent the judge from being overly sensitive, we allowed for a small threshold of mismatches in the predefined pattern, ensuring minor deviations did not result in false classifications of unsafe responses. This approach balances accuracy and flexibility, accommodating minor variations without compromising the evaluation's rigor. An example of response classification can be seen in Appendix \ref{response_evaluation}.


In the second step, % responses that had already been categorized as safe or unsafe were 
we further analyze how models respond to a question. %patterns for each response.
% to identify specific patterns within each category. 
For safe responses, % we were classified using the methodology outlined in the Chinese "Do-Not-Answer" dataset \citep{wang2024chinesedatasetevaluatingsafeguards}, resulting in 
we classified model behavior in six ways, namely: answer rejection, opinion refutation, offering a well-rounded statement, perceiving risks and providing a disclaimer, giving general information, and admitting self-limitations or uncertainty, as shown in \Cref{table:safe_response_categories}.
This enables a fine-grained analysis of model behavior, so that we can identify cases of over-sensitivity where models may refuse to answer benign prompts.

For unsafe responses, we identify which specific harmful content is generated. % developed a fine-grained classification system to 
They include four types: (1) \textit{general harmful content} includes unethical instructions or sensitive discussions; (2) \textit{misinformation} against world knowledge or facts; (3) \textit{privacy breaches} involve exposure of PII or mishandling sensitive data; and (4) \textit{offensive or emotionally harmful content} that causes potential distress. 
\Cref{table:unsafe_response_categories} provides further details.
% Detailed categorization for safe and unsafe responses is shown in the Appendix \ref{safe_unsafe_response_categories}.

% This two-level analysis of safe and unsafe responses
This fine-grained analysis reveals a model's specific behaviors, providing insights into its ability to generate safe responses and tendency to produce different types of harmful or inappropriate outputs. 
% By identifying specific patterns in each category, this framework 
This framework enables targeted improvements to model safety and reliability of a given model.


\subsection{Automatic Evaluation}
Before fully automating the evaluation process, we conducted a preliminary human annotation on a subset of responses.
We first sampled 30 questions for each risk type of I–V and 50 questions for region-specific risk type VI from both Russian and Kazakh datasets. Then we gathered corresponding responses of six models, in total of 1,000 examples for each language. Human annotators labeled (i) safe vs. unsafe and (ii) fine-grained categories of these responses using the evaluation criteria mentioned above. 
% 
% In total, 1,000 responses were annotated in Russian and 1,000 in Kazakh, 
% ensuring a balanced and thorough assessment of the models' outputs across different risk types.

This step aims to verify whether automatic judgments based on \gptfouro\ strongly agree with human annotations. 
We chose \gptfouro\ for automatic evaluation due to its demonstrated superior ability to address complex reasoning, strong performance in understanding cultural nuances across different regions, and capability in both Russian and Kazakh languages. 
\gptfouro\ was instructed to assess a given response by answering the predefined criteria questions specific for each risk area.
% , ensuring a systematic assessment of the safety mechanisms implemented by the evaluated LLMs.
% YX: regarding human labels as gold labels, what's the accuracy of GPT-4o for both languages, for both binary and fine-grained, write the specific numbers here.
Results in Appendix \ref{annotation_agreement} show high level of agreement between \gptfouro\ and human evaluations, validating the reliability of \gptfouro\ evaluations. For binary classification, \gptfouro\ achieved 90.4\% accuracy for Kazakh and 90.9\% for Russian. In fine-grained classification, accuracy was 70.7\% for Kazakh and 69.7\% for Russian (see more in \secref{sec:fine-grained-classification}). 
% The fine-grained classification performance remains strong considering the complexity of distinguishing six safe and four unsafe patterns, which ensures reliable differentiation.


% Kazakh and Russian responses.
% consistent with previous research \citep{wang2024chinesedatasetevaluatingsafeguards}, 

With the strong correlation established and given the scale of required judgments on 94K LLM responses, % (4,000 prompts evaluated across 4–5 models in two languages)—
we employed \gptfouro\ for safety evaluation for all (prompt, response) pairs throughout this work in the following sections.


%%% Local Variables:
%%% mode: latex
%%% TeX-master: "../ARR_2025"
%%% End:

% \section{Simulation Evaluation \& Results}\label{sec:results}

\subsection{Baseline Planners}

To evaluate the performance of \PlannerName, we compare it against several baseline methods. In the following section, we describe these baselines, their implementation details, and their respective advantages and limitations, particularly in the context of information gathering in large, high-dimensional search spaces. The simulation framework and vehicle parameters remain consistent across all planners, and each method is allowed to replan during testing.

\subsubsection{Monte-Carlo Tree Search}

Monte Carlo Tree Search (MCTS) can be a powerful technique for finding feasible and optimal paths in complex environments. It is a heuristic search algorithm that builds a search tree incrementally through repeated simulations. At each iteration, it selects a node to explore based on a selection policy (often the Upper Confidence Bound or UCB1 algorithm), expands the tree by adding possible actions from that node, runs a simulation from the newly added node, and updates the statistics of nodes along the path traversed during the simulation. 

The UCB1 (Upper Confidence Bound) algorithm is a technique commonly used in the context of multi-armed bandit problems and Monte Carlo Tree Search (MCTS) for balancing exploration and exploitation. It helps in selecting actions or nodes that are likely to yield high rewards while also exploring less-frequented options to gather more information about their potential rewards. 

We formulate our UCB score in the following manner, \\
\begin{equation*}
    UCB_\text{node} = \frac{I(X_{\text{node}})}{\alpha} + C \times \sqrt{\frac{\ln(N_\text{tree})}{N_\text{node}}}
\end{equation*}
%  $
% UCB_\text{node} = \frac{\overline{X_\text{node}}}{\alpha} + C \times \sqrt{\frac{\ln(N_\text{tree})}{N_\text{node}}}
% $ \\
Here $I(X_{\text{node}})$ denotes the estimated information gain from the node, $\alpha$ denotes the normalization factor which is given by $\frac{B}{v_\text{desired}}$, $B$ being the maximum planning budget and $v_\text{desired}$ being the desired speed of our UAV. $C$ denotes the exploration weight, and $N_\text{tree}$ denotes the number of visits to the tree root node while $N_\text{node}$ denotes the number of times the present node has been visited.

After selecting a candidate node, if it has been visited before, it is expanded by applying motion primitives to generate child nodes, growing the tree. Unvisited nodes skip this step. Following expansion, either the unvisited candidate node or one of its children is selected for the simulation phase, where the future values of nodes along the path are estimated to update the total potential information gain. This informs the selection policy in subsequent iterations. Once planning time is exhausted, the path with the highest information gain is returned.

% with authors goes here
\begin{figure}[t]
\centering
\includegraphics[trim={.7cm 0cm .5cm 1.4cm},clip,width=\columnwidth]{figs/5_/Results1v3.pdf}
\caption{The Monte Carlo simulation results for the planners. The plots show the average percent reduction in entropy over the course of the simulations, and the shading shows the 95\% confidence intervals. IA-TIGRIS outperforms all of the baselines.}
\label{fig:mc_results}
\end{figure}

While MCTS is probabilistically guaranteed to converge to the optimal path \cite{mcts_ref_1}, it is constrained to actions within a predefined set of motion primitives. Its reliance on random sampling to estimate the future value of nodes can result in poor approximations, particularly in environments with sparse, localized pockets of high information gain. This limitation is especially pronounced in large search areas or scenarios with large budgets constraints, where estimating future node values becomes increasingly expensive. As a result, in such scenarios, MCTS is often implemented with a finite planning horizon, which can restrict its ability to account for long-term consequences or dependencies in the environment.

% This property of MCTS, which causes unguided exploration of the environment, leads to increased convergence times on the optimal path, as a result of a lot of budget being spent in exploring information sparse areas of the map. 
% Also, the computation time of MCTS increases exponentially with the depth of the search tree. The time complexity of MCTS is given by $\mathcal{O}(\frac{T}{t_\text{iter}} \cdot |A|^d)$. Here, $T$ is the total planning time and $t_\text{iter}$ is the time taken per iteration of the planning loop. $|A|$ is the number of actions and $d$ represents the average depth of the search tree. 

% The above limitations are not inconsequential in the context of performing informative path planning in large high-dimensional search spaces. We compare MCTS with \PlannerName, in \ref{}, and empirically demonstrate its drawbacks and how \PlannerName, is able to outperform MCTS in the context of the mission parameters we examine in this work.  

\subsubsection{Greedy}

For the greedy planner, we iterated through each cell within the search bounds and calculated the reward for a given cell $i$ as $g_i = R(X_i) / d_i$ where $R(X_i)$ is given through \eqref{equ:reward} and $d_i$ represents the Euclidean distance between the current position the robot at the current time $t$ and the closest viewpoint to the cell. To compute this viewpoint, the yaw between the current pose of the robot and the intersected cell is first calculated. Using the robot's sensor configuration and this yaw, $x$ and $y$ coordinates are calculated that view the cell at the desired flight altitude. With this formulation, the planner prioritizes regions with a high ratio of entropy to distance. This can lead to locally optimal choices that contradict with paths that lead to higher information gain over the entire trajectory. 

% without authors goes here
% \begin{figure}[t]
% \centering
% \includegraphics[trim={.7cm 0cm .5cm 1.4cm},clip,width=\columnwidth]{figs/5_/Results1v3.pdf}
% \caption{The Monte Carlo simulation results for the planners. The plots show the average percent reduction in entropy over the course of the simulations, and the shading shows the 95\% confidence intervals. IA-TIGRIS outperforms all of the baselines.}
% \label{fig:mc_results}
% \end{figure}


\begin{figure*}[t]
    \centering
    \begin{subfigure}[b]{0.99\textwidth}
        \centering
        \includegraphics[trim={0cm 0.3cm 0cm 0cm},clip,width=\textwidth]{figs/5_/Fig2v1_target.png}
        % \caption{Slice by targets}
        % \vspace{.1cm}
    \end{subfigure}
    
    \begin{subfigure}[b]{0.99\textwidth}
        \centering
        \includegraphics[trim={0cm 0cm 0cm 0cm},clip,width=\textwidth]{figs/5_/Fig2v1_sigma.png}
        % \caption{Slice by sigma }
    \end{subfigure}
    \caption{A comparison of the methods based on the number of sampled prior clusters and the standard deviation of sampled prior clusters. IA-TIGRIS is most effective compared to the baselines when there is high variation in the search space. As the search space prior information becomes more evenly spread out, the performance gap between the methods tends to decrease.}
    \label{fig:targets_sigmas}
\end{figure*}

\subsubsection{Random}

The random planner operates by iteratively sampling points within the defined search bounds and calculating the minimum-cost path to observe each sampled point. This process is repeated until the available budget is fully expended. The random planner does not utilize any prior information about the environment or target distribution. Additionally, it does not optimize the sequence of actions, instead treating each sampled point independently without considering the global structure of the search problem. This simplicity allows the random planner to highlight the performance benefits of more sophisticated methods by providing a lower-bound comparison for evaluation.

\subsubsection{Coverage}

The coverage planner generates a plan that systematically covers the entire search space using a straightforward lawn-mower pattern. The spacing between each pass is set to match the width of the projected observation footprint at 20\% from the bottom, ensuring that no grid cells are missed. This spacing also maintains a distance that enables high-quality sensor measurements. However, due to the size of the search spaces considered, the coverage planner spends significant time surveying empty regions. This approach results in inefficient use of the budget, as it prioritizes full coverage with safe sensor overlap, even in areas with little or no valuable information. While simple and robust, this method highlights the tradeoff between exhaustive coverage and efficient, targeted exploration.

% \subsubsection{Branch and Bound}
% The branch and bound baseline is based on motion primitive planning. In each future step the drone has a set of motion primitives with future states and each of these future states also has a set of motion primitives. In this way, a tree can be built with multiple path candidates. The path candidate with the highest information gain will be selected and form the output. 

% By adding branch and bound, there will be an estimation of a node's upper bound information reward, using the node's current information reward, updated information map and the remaining budget. If this upper bound is already lower than the information reward of any other node in the tree, the corresponding node will be closed and not expanded in the future to accelerate the expansion of the tree. 



\subsection{Tests and Analysis}
% To evaluate the efficacy of IA-TIGRIS compared to the baseline methods, we conduct Monte Carlo testing as well as analyze how the prior and budget affect the performance of each method. In all of these test cases, there are no time-based or priority rewards and have horizon lengths set to the full budget. All tests were performed using an Intel Xeon CPU E5-2620 v4 @ 2.10GHz.
To evaluate the efficacy of IA-TIGRIS against baseline methods, we perform Monte Carlo testing and analyze the impact of the prior and budget on the performance of each method. In all test cases, rewards are calculated using \eqref{equ:reward}, and horizon lengths are set to match the full budget. The tests are conducted on an Intel Xeon CPU E5-2620 v4 @ 2.10GHz, ensuring consistent computational conditions across all evaluations.

% Random sample across which parameters.

% Quantitative ideas. Look into number and std of prior (metric for this? std of grid cell values, mediuan, mean,). 
% Uniform prior? 
% Split distinct regions, not smooth. 
% Compare to coverage and amount of time to reach specific amount. 
% Compare with different budgets. 
% Repeatability test. 
% Graph size vs time. 
% Look at coverage with different altitudes or widths. Something that shows long horizon vs not nature of things?
% Shape of search space?
% Time/budget to get x\% of all info gain. Have to do moving horizon. 
% Targets detected? 

% Key thought for results where I show time, our optimization does not optimize for time, only final value. Key thing to show across the different budgets. 

% \BM{Qualitative. Nayana idea of plot with example sampled case. Should add one here.} 



\subsubsection{Monte Carlo Testing}
Our simulated testing environment is a $5000\times5000$ m square with Gaussian-distributed prior information randomly placed throughout the search space. The number of prior clusters was sampled uniformly between $[4,20]$, with standard deviations between $[60,450]$, and maximum value between $[0.05,0.5]$. 

The results of $100$ Monte Carlo tests are shown in Fig.~\ref{fig:mc_results}. IA-TIGRIS clearly outperforms the other methods, achieving nearly a $40\%$ greater reduction in entropy than the next best method. Early in the simulation, the greedy method initially gains information more quickly, as expected, but this does not translate to better long-term performance. Since our method optimizes for total information gain, it generates paths that maximize information collection over the entire budget. MCTS performed slightly worse than the greedy approach.

The random paths slightly outperformed the coverage paths. This is likely because the lawnmower strategy requires sufficient overlap between passes to avoid missing areas, and its long straight paths often lead to redundant observations due to the UAV’s forward-facing camera. Changing the heading of the UAV is beneficial to viewing more of the search space, which may explain why random paths performed better.

We also conducted Monte Carlo tests where either the number of prior clusters or their standard deviation was held constant to analyze how variations in the information map affect planner performance. The results, shown in Fig.~\ref{fig:targets_sigmas}, include two cases: the upper figure fixes the number of priors, while the lower figure fixes their standard deviation. All other agent and simulation parameters remained unchanged.


% The first thing to note from these results is that for all tests the proportional performance gap between IA-TIGRIS and the baselines increases as the number and standard deviation of the Gaussian priors decreases. As the search space becomes more uniformly filled with entropy in the information map, the need for longer-horizon planning decreases and other simple or random approaches can perform satisfactorily given the testing budget. As the information becomes more sparsely distribution in the space, such as when the information is contained in separated pockets of areas, there is a greater need to plan longer-horizon paths that reason about the given budget.
% \BM{Could have figures here or refer to others}

Across these tests, the performance gap between IA-TIGRIS and the baselines widens as the number and standard deviation of the Gaussian priors decrease. When entropy is more uniformly distributed across the search space, simpler methods perform reasonably well within the given budget. However, when information is concentrated in sparse, distinct regions, longer-horizon planning becomes essential. In such cases, IA-TIGRIS demonstrates a significant advantage by effectively reasoning about the budget and prioritizing high-value regions.

% Show plot of first plans expected info gain versus planning time. (plans not executed)


\subsubsection{Budget Analysis}
To evaluate the impact of budget constraints on performance, we conducted additional tests beyond our initial Monte Carlo experiments, evaluating budgets of $5000$ m, $10000$ m, $30000$ m, and $60000$ m. Table~\ref{tab:budgets} summarizes the average entropy reduction across these budgets.

\definecolor{tabfirst}{rgb}{1, 0.7, 0.7} % red
\definecolor{tabsecond}{rgb}{1, 0.85, 0.7} % orange
\definecolor{tabthird}{rgb}{1, 1, 0.7} % yellow
\begin{table}[t]
    \centering
    \resizebox{\linewidth}{!}{
    \begin{tabular}{l|ccccc}
    & $5000$ m & 10000 m  & 15000 m& 30000 m& 60000 m\\ \hline

    % \hline
    IA-TIGRIS  &  \cellcolor{tabfirst}$9.41\pm1.0$ &  \cellcolor{tabfirst}$18.28\pm1.8$ & \cellcolor{tabfirst}$25.36\pm2.3$ & \cellcolor{tabfirst}$41.08\pm2.9$ & \cellcolor{tabfirst}$58.85\pm2.9$ \\
    Greedy  &  \cellcolor{tabsecond}$6.99\pm0.8$ &  \cellcolor{tabsecond}$13.10\pm1.5$ & \cellcolor{tabsecond}$17.97\pm2.0$ & \cellcolor{tabthird}$30.00\pm2.3$ & \cellcolor{tabsecond}$49.38\pm3.5$ \\
    MCTS  &  \cellcolor{tabthird}$6.06\pm0.7$ &  \cellcolor{tabthird}$11.80\pm1.1$ & \cellcolor{tabthird}$17.11\pm1.4$ & \cellcolor{tabsecond}$30.21\pm2.2$ & \cellcolor{tabthird}$48.68\pm2.7$ \\
    Random  &  $2.19\pm0.3$ & $4.29\pm0.7$ & $6.61\pm0.6$ & $17.50\pm1.2$ & $22.47\pm1.4$ \\
    Coverage  &  $1.58\pm0.3$ &  $2.82\pm0.4$ & $4.09\pm0.7$ & $12.04\pm1.9$ & $16.77\pm2.4$ \\

    \end{tabular}
    }
    \caption{Monte Carlo testing results given different budgets. The values are the average percent reduction in entropy and the 95\% confidence bounds. \mbox{IA-TIGRIS} had the best performance for all budgets.}
    \label{tab:budgets}
\end{table}
%$\uparrow$ 

IA-TIGRIS consistently achieved the highest entropy reduction across all budget constraints, with a statistically significant margin over alternative methods. Greedy generally ranked second but was slightly outperformed by MCTS at the $30000$ m budget level. Greedy and MCTS exhibited comparable performance throughout the tests, with their results closely tracking each other. Consistent with our previous findings, Random and Coverage methods yielded the lowest results.


Among the tested methods, only IA-TIGRIS and MCTS explicitly incorporate budget constraints into their planning algorithms. Notably, at lower budgets ($5000$ m and $10000$ m), these methods achieved higher entropy reduction compared to the equivalent time steps ($200$ s and $400$ s) in the $15000$ m budget scenario shown in Fig.~\ref{fig:mc_results}. This improved performance stems from IA-TIGRIS's optimization of total path reward under budget constraints, contrasting with the myopic next-best-action approach of the greedy method. The remaining methods---Greedy, Random, and Coverage---maintain consistent behavior regardless of budget constraints, as their planning strategies do not account for resource limitations.


The performance gap between IA-TIGRIS and the next-best method varied with budget size, showing margins of $34.6\%$, $39.5\%$, $41.1\%$, $36.0\%$, and $19.2\%$ in ascending budget order. This gap widened through the first three budget levels as problem complexity increased, before declining significantly at higher budgets. This performance pattern suggests that implementing a planning horizon could enhance efficiency by limiting tree search depth, enabling the planner to prioritize path quality optimization over exhaustive space exploration.


% percent improved from next best
% 34.6, 39.5, 41.1, 36.0, 19.2
% reasons, too long horizon is a larger search space, so less quality paths closer. Or larger horizon, more packing in


% with authors goes here
\begin{figure}[t] 
    \centering
    \renewcommand\arraystretch{0} % Adjust the height between rows here
    \setlength{\tabcolsep}{1pt} % Adjust the column separation here
    \begin{tabular}{c}
        \begin{tikzpicture}
            \node[anchor=south west, inner sep=0] (image) at (0,0) {
                \includegraphics[width=0.9\linewidth]{figs/5_/google_earth_prior.png}
            };
            \begin{scope}[x={(image.south east)},y={(image.north west)}]
                % \fill[OrangeRed] (0.02, 0.03) circle (2pt); 
                % \fill[OrangeRed] (0.51, 0.04) circle (2pt); 
                % \fill[OrangeRed] (0.61, 0.04) arc (0:90:2pt); 
                \fill[Orange, opacity=0.8] (0.74, 0.45) circle (3pt); % Adjust 
                \fill[Orange, opacity=0.8] (0.27, 0.42) circle (3pt); % Adjust 
                \fill[Orange, opacity=0.8] (0.39, 0.63) circle (3pt); % Adjust 
            \end{scope}
        \end{tikzpicture} \\
        % \includegraphics[width=0.9\linewidth]{figs/5_/google_earth_prior.png} \\
        \\
        \includegraphics[width=0.9\linewidth]{figs/5_/google_earth_path.png} 
    \end{tabular}
    \caption{Google Earth screenshots illustrating the mission planning process and execution. Top: Areas of high entropy targeted for search are highlighted in red, representing regions with a binary occupied/unoccupied probability of 0.2. Three points of particular interest, each assigned a 0.5 probability, are marked in orange. Bottom: The executed drone flight path (yellow) shows the optimized path for maximum information gain across the search space.} 
    \label{fig:google_earth}
\end{figure}
\begin{figure}[t]
\centering
% https://docs.google.com/presentation/d/1RjI-QqHpBRLHN60UAxzmQYs4EaWaVCOoSBkEkA39kk0/edit?usp=sharing
\includegraphics[width=\columnwidth]{figs/5_/m600_labeled.jpg}
\caption{Hexarotor system (DJI M600 Pro) with onboard compute and camera. Left image shows drone on the ground, right image shows drone in flight.}
\label{fig:m600}
\end{figure}


\section{Field Deployments}\label{sec:field}


\subsection{Hexarotor Deployment}
The first field experiment that we present uses a hexarotor drone to cover an urban area shown in Fig.~\ref{fig:fig1}.
We designed this field experiment to simulate classifying where cars are within a search area.  
Hence, we set the plan request to focus on parking lots at the field test site (Fig.~\ref{fig:google_earth}, top), with the addition of three chosen grid cells within the parking lots being marked as having a higher uncertainty. The plan request boundaries and priors were created with GPS coordinates in Google Earth, exported as kml files, and then converted into our plan request message format. 

The following sections details the hardware, autonomy, and experimental results for our hexarotor deployments.

% without the authors goes here
% \begin{figure}[t] 
%     \centering
%     \renewcommand\arraystretch{0} % Adjust the height between rows here
%     \setlength{\tabcolsep}{1pt} % Adjust the column separation here
%     \begin{tabular}{c}
%         \begin{tikzpicture}
%             \node[anchor=south west, inner sep=0] (image) at (0,0) {
%                 \includegraphics[width=0.9\linewidth]{figs/5_/google_earth_prior.png}
%             };
%             \begin{scope}[x={(image.south east)},y={(image.north west)}]
%                 % \fill[OrangeRed] (0.02, 0.03) circle (2pt); 
%                 % \fill[OrangeRed] (0.51, 0.04) circle (2pt); 
%                 % \fill[OrangeRed] (0.61, 0.04) arc (0:90:2pt); 
%                 \fill[Orange, opacity=0.8] (0.74, 0.45) circle (3pt); % Adjust 
%                 \fill[Orange, opacity=0.8] (0.27, 0.42) circle (3pt); % Adjust 
%                 \fill[Orange, opacity=0.8] (0.39, 0.63) circle (3pt); % Adjust 
%             \end{scope}
%         \end{tikzpicture} \\
%         % \includegraphics[width=0.9\linewidth]{figs/5_/google_earth_prior.png} \\
%         \\
%         \includegraphics[width=0.9\linewidth]{figs/5_/google_earth_path.png} 
%     \end{tabular}
%     \caption{Google Earth screenshots illustrating the mission planning process and execution. Top: Areas of high entropy targeted for search are highlighted in red, representing regions with a binary occupied/unoccupied probability of 0.2. Three points of particular interest, each assigned a 0.5 probability, are marked in orange. Bottom: The executed drone flight path (yellow) shows the optimized path for maximum information gain across the search space.} 
%     \label{fig:google_earth}
% \end{figure}
% \begin{figure}[t]
% \centering
% % https://docs.google.com/presentation/d/1RjI-QqHpBRLHN60UAxzmQYs4EaWaVCOoSBkEkA39kk0/edit?usp=sharing
% \includegraphics[width=\columnwidth]{figs/5_/m600_labeled.jpg}
% \caption{Hexarotor system (DJI M600 Pro) with onboard compute and camera. Left image shows drone on the ground, right image shows drone in flight.}
% \label{fig:m600}
% \end{figure}

\subsubsection{Hardware System}
The hardware consists of the DJI M600 Pro, shown in Fig.~\ref{fig:m600}, along with the physical sensing and onboard computer payload. The DJI M600 Pro contains a flight controller that handles pose estimation and position-based control. The DJI M600 Pro’s flight controller also handles teleloperation if human intervention is necessary. Beneath the drone's base, we mount a custom hardware payload.
That payload consists of an onboard computer, a Jetson Xavier, to run the autonomy software shown in Fig.~\ref{fig:functional_diagram}.
The payload also contains a downward-facing a camera for sensing the environment. The camera is a Seek S304SP thermal camera.
The camera intrinsics are used to calculate the frustum's intersection with the search map's cells in IA-TIGRIS.

% without authors goes here
\begin{figure}[t]
\centering
% https://lucid.app/lucidchart/f750ddb4-2809-4773-8361-d5fbb1ba49eb/edit?viewport_loc=-257%2C-116%2C2219%2C1140%2C0_0&invitationId=inv_56e8a3a9-e8cf-4cad-a280-48bd967ff651
\includegraphics[trim={0cm 0cm 0cm 0cm},clip,width=\columnwidth]{figs/5_/functional_diagram.jpeg}
\caption{Functional diagram of the DJI M600 Pro autonomy software.}
\label{fig:functional_diagram}
\end{figure}
\begin{figure}[b]
    \centering
    \begin{subfigure}[b]{0.48\columnwidth}
        \centering
        \includegraphics[width=1.0\linewidth]{figs/5_/field_test_altitude_over_time.png}
        \caption{}
        \label{fig:m600_altitude_over_time}
    \end{subfigure}
    \begin{subfigure}[b]{0.48\columnwidth}
        \centering
        \includegraphics[width=1.0\linewidth]{figs/5_/field_test_entropy_over_time.png}
        \caption{}
        \label{fig:m600_entropy_over_time}
    \end{subfigure}
    \caption{The results for our hexarotor field deployment. (a) Plot of flown altitude over time, showing large variation throughout the experiment. (b) Reduction in entropy percentage over time of field experiment.}
\end{figure}

\subsubsection{Autonomy System}
Fig.~\ref{fig:functional_diagram} illustrates the functional system diagram for the real world field test on the DJI M600. The user specifies the initial plan request prior to takeoff. The TIGRIS planner makes an initial plan on that plan request and sends a global path to the waypoint manager. The waypoint manager tracks the current waypoint within the plan and sends the next waypoint to the DJI software development kit, which then sends actuation commands to the motors. The position of the drone is used to calculate the distance from the drone to the ground and sends that distance parameter to the sensor model. The sensor model's true positive and false positive rate is used to calculate the per-cell entropy updates in the search map manager. The search map manager publishes the current information map, and the replanning node sends an updated plan request to the IA-TIGRIS planner every ten seconds.

The drone started at an altitude of $50$ m above the origin of the reference frame. The informed sampler in IA-TIGRIS was set to add states at altitudes of either $30$ m or $60$ m, creating a trade-off between observation area and detector accuracy. The budget was $2000$ m, the planning horizon was $600$ m, and the planning time was $10$ seconds. 

% % without authors goes here
% \begin{figure}[t]
% \centering
% % https://lucid.app/lucidchart/f750ddb4-2809-4773-8361-d5fbb1ba49eb/edit?viewport_loc=-257%2C-116%2C2219%2C1140%2C0_0&invitationId=inv_56e8a3a9-e8cf-4cad-a280-48bd967ff651
% \includegraphics[trim={0cm 0cm 0cm 0cm},clip,width=\columnwidth]{figs/5_/functional_diagram.jpeg}
% \caption{Functional diagram of the DJI M600 Pro autonomy software.}
% \label{fig:functional_diagram}
% \end{figure}
% \begin{figure}[b]
%     \centering
%     \begin{subfigure}[b]{0.48\columnwidth}
%         \centering
%         \includegraphics[width=1.0\linewidth]{figs/5_/field_test_altitude_over_time.png}
%         \caption{}
%         \label{fig:m600_altitude_over_time}
%     \end{subfigure}
%     \begin{subfigure}[b]{0.48\columnwidth}
%         \centering
%         \includegraphics[width=1.0\linewidth]{figs/5_/field_test_entropy_over_time.png}
%         \caption{}
%         \label{fig:m600_entropy_over_time}
%     \end{subfigure}
%     \caption{The results for our hexarotor field deployment. (a) Plot of flown altitude over time, showing large variation throughout the experiment. (b) Reduction in entropy percentage over time of field experiment.}
% \end{figure}

\subsubsection{Experimental Results}


The bottom image of Fig.~\ref{fig:google_earth} shows the path selected by IA-TIGRIS in the search area. The figure highlights how the planner dynamically adjusts altitudes over time to balance coverage and sensing resolution, maximizing information gain. Higher altitudes allow for broader area coverage, while lower altitudes provide more detailed observations where needed. Additionally, the planner prioritizes revisiting the three regions of higher uncertainty, recognizing the need for repeated observations reduce entropy. This adaptive strategy ensures that uncertain areas receive sufficient attention to improve the belief map. As a result, the entropy of the information map decreases to near zero by the end of the mission, as shown in Fig.~\ref{fig:m600_entropy_over_time}, indicating that the planner has effectively gathered the necessary information. This behavior demonstrates the planner’s ability to optimize sensing actions, balancing altitude selection, revisit frequency, and exploration to maximize mission success.

\begin{figure}[t]
\centering
% \includegraphics[width=2.5in]{fig1}
\includegraphics[trim={4cm 4cm 0cm 4cm},clip,width=\columnwidth]{figs/5_/TL1.jpg}
\caption{Fixed-wing platform used for autonomous flights with an onboard camera pitched at 10 degrees\cite{alarewebsite}}
\label{fig:tl1}
\end{figure}






\subsection{Fixed-wing Deployments}

Our proposed approach was extensively tested on the fixed-wing AlareTech TL-1 UAV, shown in Fig.~\ref{fig:tl1}. The UAV is equipped with an onboard camera pitched at 10 degrees, which introduces a more challenging planning problem due to the non-holonomic motion model and the camera's field of view. Over more than 20 flight hours and 100 flights running IA-TIGRIS, we validated our approach with the objective to search for objects of interest in a large search space across a variety of test scenarios, including different terrain types, varying environmental conditions, and diverse target distributions. An example mission from these tests is shown in Fig.~\ref{fig:fwd}. In this scenario, the planner was given the search bounds and a designated high-priority region. The resulting flight path prioritized revisiting the high-priority area twice, optimizing sensor use and ensuring maximum information gain. This strategy led to the successful detection of the object of interest, with its estimated position marked by the red dot in the figure. 

The map on the upper right in Fig.~\ref{fig:fwd} shows the information map after plan execution was complete. Due to the UAV's limited budget, the upper right and lower left corners of the map are not searched by the agent. The budget is instead utilized to search over the area of higher priority two times. Compared to the paths in Fig.~\ref{fig:google_earth}, we observe that the paths for the fixed wing are smoother and have a larger turning radius, demonstrating how IA-TIGRIS respects the motion constraints of the vehicle. We can also see the effect of wind on the path execution, where the flown path shown in green deviates from the planned path shown in yellow. This illustrates the importance of online planning in the cases where this deviation is large or would accumulate over the course of a longer mission and cause the expected observed area to be much different than actual observed area. 

\begin{figure}[t]
\centering
% \includegraphics[width=2.5in]{fig1}
% [trim={left bottom right top},clip]
\includegraphics[trim={3.0cm, 1.0cm, 3.0cm, 1.0cm},clip,width=\columnwidth]{figs/5_/ONRFig_v3.pdf}
\caption{An example path generated for the fixed-wing platform conducting a large-area search for an object of interest. The larger black rectangle denotes the search bounds, while the smaller black rectangle highlights a region of higher uncertainty. The red dot marks the estimated position of the detected object based on image detections. The upper-right map displays the information state after planning is complete, while the middle plot shows the percent change in entropy over mission time. The flown path illustrates a balance between allocating resources to the high-priority region and exploring other areas within the search space.}
\label{fig:fwd}
\end{figure}

% Also tested extensively on the AlareTech TL-1 (citation?) tube launched UAV seen in Fig.~\ref{fig:tl1}.

% Talk about amount of flights, hours. Platform. Compute. Show visualization fo example flight. Talk about objects of interest in a broad sense (no mention of water/ocean/land for targets). Follow similar figure format as previous section. Main thing we want to highlight is the differences introduced in plans by having a fixed-wing platform compared to a drone. Include image of Alare TL-1 somewhere.

% One big figure showing all the info we want to convey. 

% \BM{Pitch 10 degrees, onboard computer type, etc}


% \subsection{VTOL?}
% what would it bring?


\section{Concluding Remarks}
In this paper, we proposed a novel approach utilizing multimodal LLMs to generate gesture-aware speech recognition transcripts for patients with language disorders. Our framework integrates verbal speech and iconic gestures, enabling the generation of enriched transcripts that capture the latent meaning conveyed through both modalities. Through extensive experimentation, we demonstrated that the proposed method effectively contextualizes incomplete or disfluent speech by incorporating gesture information, leading to more accurate and meaningful representations of the speaker's intent. These findings highlight the potential of our approach to significantly contribute to the field of speech and language therapy, offering innovative tools that can enhance the quality of life for individuals with language disorders by facilitating better communication and assessment methods.

\subsection{Ethical Statement} 
Our dataset was obtained from AphasiaBank with the approval of the Institutional Review Board (IRB) and adheres to the data sharing guidelines set by TalkBank\footnote{https://talkbank.org/share/ethics.html}. This includes complying with the Ground Rules for all TalkBank databases, which are based on the American Psychological Association Code of Ethics~\cite{american2002ethical}.

\subsection{Limitation \& Future Work} 
%This study represents a preliminary investigation into using multimodal LLMs to generate gesture-aware speech recognition transcripts. 
While the results are promising, we recognize several limitations and outline our plans to extend this work further.

One primary limitation is the absence of a definitive ground truth for quantitative evaluation. Since our model generates transcripts by synthesizing speech and gesture data from scratch, traditional benchmarks, such as comparisons with standard speech recognition outputs, are insufficient. Moreover, existing original transcripts lack gesture annotations, making direct comparisons challenging. In future work, we aim to address this gap by collaborating with certified pathologists to conduct qualitative assessments, such as A-B preference tests, to evaluate the effectiveness of gesture-enriched transcripts in accurately conveying the speaker's intentions.

To support quantitative evaluations, we plan to develop novel metrics that assess transcript quality, including grammar accuracy, semantic consistency, and the integration of multimodal information. Such metrics will provide a more objective basis for assessing our model's performance and facilitate comparisons with other multimodal and unimodal approaches.

Another limitation of this study is its focus on structured gestures from a specific task, the Peanut Butter Sandwich Task. While this task offers a controlled context for testing our approach, it does not encompass the diversity of gestures and communication patterns seen in everyday scenarios. As part of our future work, we plan to expand the scope of our model to include tasks such as the Cinderella Story Recall Task~\cite{bird1996cinderella}, which involves unstructured and complex narrative gestures. This expansion will allow us to evaluate the adaptability and robustness of our model in handling varied linguistic and gestural contexts.

In summary, while this study establishes a strong foundation for gesture-aware speech recognition, we aim to refine and extend our methods through collaborative qualitative evaluations, the development of robust quantitative metrics, and broader task applications. These efforts will ensure that our approach continues to evolve, ultimately contributing to more effective communication tools and interventions for individuals with language disorders.




% % propose BF GDA/SGDA
% introduce adaptive regret notion and non-degenerative populations
%%% theoretical results
% convergence of GDA/SGDA in our setting for exponential policies
% hypotethis that it should work still by clipping the policy side ( then experiments), then argue that it can help for convergence of NN policies (talk about L0-L1 smoothness and past results on that matter, theoretical results left for future work on GDA/SGDA)
% 
%%% experiments
% repeated prisonner's dilemma as a first example to show results on GDA, (with specific or full set of deterministic policies??)
% Show clipping effect (works on higher lr)
% boxplots, learned distributions, learning curve on worst-case regret ?

% more advanced experiments on random POMDPS (provide figure for the env ?)
% show empirically that it still works. clipping as well ?

% final set of experiments on complex environments with NN policies.
% > leduc poker, melting pot
% > cooperation on mujoco tasks (say the "human" controls a part of the robot, and the agent assists)









% {\appendix[Proof of the Zonklar Equations]
% Use $\backslash${\tt{appendix}} if you have a single appendix:
% Do not use $\backslash${\tt{section}} anymore after $\backslash${\tt{appendix}}, only $\backslash${\tt{section*}}.
% If you have multiple appendixes use $\backslash${\tt{appendices}} then use $\backslash${\tt{section}} to start each appendix.
% You must declare a $\backslash${\tt{section}} before using any $\backslash${\tt{subsection}} or using $\backslash${\tt{label}} ($\backslash${\tt{appendices}} by itself
%  starts a section numbered zero.)}



% {\appendices
% \section*{Proof of the First Zonklar Equation}
% Appendix one text goes here.
% You can choose not to have a title for an appendix if you want by leaving the argument blank
% \section*{Proof of the Second Zonklar Equation}
% Appendix two text goes here.}


% argument is your BibTeX string definitions and bibliography database(s)
% \bibliography{IEEEabrv,references}
\bibliography{references}
%
% \newpage
% \section{Biography Section}
% If you have an EPS/PDF photo (graphicx package needed), extra braces are
%  needed around the contents of the optional argument to biography to prevent
%  the LaTeX parser from getting confused when it sees the complicated
%  $\backslash${\tt{includegraphics}} command within an optional argument. (You can create
%  your own custom macro containing the $\backslash${\tt{includegraphics}} command to make things
%  simpler here.)
% \vspace{11pt}
% \bf{If you include a photo:}
% \begin{IEEEbiography}[{\includegraphics[width=1in,height=1.25in,clip,keepaspectratio]{figs/fig1.png}}]{IEEE Publications Technology Team}
% In this paragraph you can place your educational, professional background and research and other interests.\end{IEEEbiography}
% \vspace{-33pt}


% \begin{IEEEbiography}[{\includegraphics[width=1in,height=1.25in,clip,keepaspectratio]{figs/bios/brady.jpg}}]{Brady Moon}
% is a Ph.D. candidate in Robotics at Carnegie Mellon University. His research focuses on autonomy algorithms for robotics, particularly in the areas of path planning, multi-agent search, target tracking, and data gathering in complex environments. Brady has contributed to various projects, including UAV-based wind estimation, energy-based flight risk assessments for autonomous UAVs, and indoor exploration leveraging predictive models. Brady holds a B.S. in Electrical Engineering from Brigham Young University.
% \end{IEEEbiography}
% \vspace{-33pt}
% \begin{IEEEbiography}[{\includegraphics[width=1in,height=1.25in,clip,keepaspectratio]{figs/bios/suvarna_profile.png}}]{Nayana Suvarna}
% is a Masters in Robotics student at Carnegie Mellon University. Her research focuses on information gathering algorithms for teams of aerial vehicles. Nayana holds a B.S. in Computer Engineering from the University of Pittsburgh.
% \end{IEEEbiography}
% \vspace{-33pt}
% \begin{IEEEbiography}[{\includegraphics[width=1in,height=1.25in,trim={1cm 0 1.2cm 0 },clip,keepaspectratio]{figs/bios/andrew_jong.png}}]{Andrew Jong}
% is a Ph.D. student in Robotics at Carnegie Mellon University. His research focuses on robotics for disaster response, and covers path planning, perception in degraded environments, agile navigation, and safety. Andrew holds a B.S. in Computer Science from San Jos\'e State University and a M.S. in Robotics from Carnegie Mellon University.
% \end{IEEEbiography}
% \vspace{-33pt}
% \begin{IEEEbiography}[{\includegraphics[width=1in,height=1.25in,clip,keepaspectratio]{figs/bios/satrajit.jpg}}]{Satrajit Chatterjee}
% is a Master's student in Robotics at the GRASP Lab, University of Pennsylvania. His research is focused on path-planning for robotics and includes work in multi-target search and tracking, and vision-based generative techniques for path-planning in semantically complex environments. Satrajit holds a B.Tech in Computer Science and Engineering from the National Institute of Technology, Tiruchirappalli, India. 
% \end{IEEEbiography}
% \vspace{-33pt}
% \begin{IEEEbiography}[{\includegraphics[width=1in,height=1.25in,clip,keepaspectratio]{figs/bios/junbin.jpg}}]{Junbin Yuan}
% is a Ph.D. candidate in mechanical engineering at Carnegie Mellon University. His research focuses on planning algorithms for robotics, and covers coverage path planning and multi-target search and tracking. Junbin received his B.Eng. in Electronics Engineering from Hong Kong University of Science and Technology 
% \end{IEEEbiography}
% \vspace{-33pt}
% \begin{IEEEbiography}[{\includegraphics[width=1in,height=1.25in,clip,keepaspectratio]{figs/bios/basti.jpg}}]{Sebastian Scherer} is an Associate Research Professor at the Robotics Institute (RI) at Carnegie Mellon University (CMU). His research focuses on enabling autonomy for unmanned rotorcraft to operate at low altitude in cluttered environments. He and His team have shown the fastest and most tested obstacle avoidance on an Yamaha RMax (2006), the first obstacle avoidance for micro aerial vehicles in natural environments (2008), and the first (2010) and fastest (2014) automatic landing zone detection and landing on a full-size helicopter. Dr. Scherer received his B.S. in Computer Science, M.S. and Ph.D. in Robotics from CMU in 2004, 2007, and 2010. He is a Siebel scholar and a recipient of multiple paper awards and nominations, including AIAA@Infotech 2010 and FSR 2013. His research has been covered by the national and internal press including IEEE Spectrum, the New Scientist, Wired, der Spiegel, and the WSJ. His work on self-landing helicopters has received the Popular Science Best of What’s New 2010 Award and in Fall 2016 he demonstrated his inspection robots to President Obama.
% \end{IEEEbiography}

% \vspace{11pt}

% \bf{If you will not include a photo:}\vspace{-33pt}
% \begin{IEEEbiographynophoto}{John Doe}
% Use $\backslash${\tt{begin\{IEEEbiographynophoto\}}} and the author name as the argument followed by the biography text.
% \end{IEEEbiographynophoto}




\vfill

% \listoftodos[To Do List]

\end{document}

