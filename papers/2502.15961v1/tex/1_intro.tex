\section{Introduction}

% Gathering information from the physical world using robots plays a crucial role in many applications, including scientific research \cite{doi:10.1126/scirobotics.abc3000,McCammon2021,de2021monitoring}, environmental monitoring \cite{kaufmann2021conventional, 10161136,barbedo2019review, Christensen2015,patrikar2020}, search and rescue \cite{Bashyam2019UAVsFW,Alsamhi2022}, 
% % defense \cite{Wise2006}, 
% or disaster response \cite{bejiga2017convolutional,Mohsan2023,BAILONRUIZ2022104071}. The utilization of robots for information gathering allows for leveraging intelligent algorithms to efficiently collect data, provide critical insights, and facilitate informed decision-making.


Robots play a vital role in gathering information from the physical world, supporting a wide range of applications such as scientific research \cite{doi:10.1126/scirobotics.abc3000,McCammon2021,de2021monitoring}, environmental monitoring \cite{kaufmann2021conventional, 10161136,barbedo2019review, Christensen2015,patrikar2020}, search and rescue operations \cite{Bashyam2019UAVsFW,Alsamhi2022}, and disaster response efforts \cite{bejiga2017convolutional,Mohsan2023,BAILONRUIZ2022104071}. By employing intelligent algorithms, robotic systems enhance the efficiency of data collection, provide valuable insights, and support well-informed decision-making processes. These autonomous robots provide unparalleled advantages in situations where human access is constrained, dangerous, or logistically challenging \cite{6161683}. Moreover, the scalability and speed of autonomous information gathering robots significantly enhances the rate of information gathering by not capping the number of robots based on the number of human operators. 

\begin{figure}[t]
\centering
% \includegraphics[width=2.5in]{fig1}
\includegraphics[width=\columnwidth]{figs/1_Intro/m600-fig1v3.jpg}
\caption{\PlannerNameSpaced deployed on a hexarotor UAV to map the location of cars in the environment. The visualization shows a probability grid representing the prior belief of car locations and a representative path generated by our planner. The algorithm runs entirely onboard, continuously refining and adapting paths based on updated world beliefs from sensor observations.}
\label{fig:fig1}
\end{figure}


Informative path planning (IPP) seeks to create intelligent paths for robots to execute actions that maximize their inherent potential and advantages in data gathering situations, while not violating budget constraints on the system or mission. This approach has many advantages over most coverage-based approaches which often don't take into account the flight time of the drone or prior information about the space and don't adapt to learned information during flight. However, solving the IPP problem is computationally expensive, being at least NP-hard even in a discrete space \cite{singh2009}, leading to methods that only plan over a short horizon or are static during the path execution.
% This approach to information gathering 
% In contrast with coverage based approaches, IPP methods can fully

% Talk briefly about other approaches and pitfalls. Why needed. Our previous work here?

In this work, we introduce \PlannerNameSpaced (Incremental and Adaptive Tree-based Information Gathering using Informed Sampling), a sampling-based informative path planner that incrementally builds its search tree and adapts to new information online by efficiently updating and building the set of possible trajectories. This leads to global paths that are more robust to path execution disturbances, adapt to changes in the information map, and are higher quality than a single plan due to continual online refinement. Fig.~\ref{fig:fig1} demonstrates \PlannerNameSpaced deployed on a hexarotor drone for an IPP application. Our key contributions are the following:
%An online real-time? planner that is adaptive and incremental. 

% that takes advantage of an optimistic Certain target Certain non-target Uncertain region Probable non-target Probable target Start location End location Trajectory refined in real time pproximation of the classification accuracy of the pre-planned path to make real-time field applications feasible. This approach allows the algorithm to incrementally build a tree of trajectories for long-horizon planning and improved classification. We implement the ReASC algorithm both in simulation and in experiments using an autonomous aquatic surface vehicle performing an aquatic monitoring task on a lake. The main novelty of this paper is the development of the ReASC algorithm that allows for long-horizon adaptive search and classification on real-world platforms operating in the field.

% Introduce to the impact, why.


% Go through the related works (or make a whole section?)

% Introduce our planner (and acronym for it)

\begin{itemize}
    \item A novel incremental and adaptive sampling-based planner for IPP that dynamically plans online using the updated belief map. We provide key implementation details, insights, and optimizations that enable efficient trajectory evaluation and updates during incremental planning.
    \item Thorough testing and evaluation of our method against other baselines, as well as ablations and analysis on the components of our algorithm.
    % \item Extensive field testing and demonstrations on two UAV platforms with different motion models and configuration spaces.
    \item Hardware validation on two distinct UAV platforms with different motion constraints and configuration spaces, demonstrating versatility and real-world applicability.
\end{itemize}

The paper is organized as follows: Section \ref{sec:related} gives a detailed overview of related works and the contributions of this work. Section \ref{sec:problem} presents the problem formulation. Our proposed approach is detailed in Section \ref{sec:approach}. The section outlines the planner's design and its integration with system components to enable adaptive and efficient informative path planning for real-world applications. Section \ref{sec:results} provides our detailed experimental evaluation and results of the approach. The results of our field deployments are summarized in Section \ref{sec:field}, and Section \ref{sec:conclusion} offers the conclusion and future work. 


\section{Related Works}\label{sec:related}
% ---
% Different approaches to IPP - adaptive sampling/orienteering; receeding horizon approach; planning globally optimal paths, many approaches discretize the space and turn the problem into a graph search problem; information gathering problems can also be formulated in continuous spaces; coverage based planners

% Structure:
% Short snippet or generalized summary of each approach category; focus on approaches of select works; advantages/disadvantages of each approach; move onto next; talk about previous work (TIGRIS) and what it did that improved upon disadvantages of related works. Then finally close off with how this paper builds on that in a sentence. 

% Adaptive sampling is described as the process of sampling points in either a discretized or continuous space in real-time that allows the autonomous system to adapt its path on-the-fly. 

% ---

One way to formulate the IPP problem is to discretize the continuous space into a fixed set of states. Many previous works have then used this discrete space to view IPP as an orienteering problem. Given a weighted-undirected graph, orienteering involves determining a Hamiltonian path over a subset of nodes with the objective of maximizing the total reward while subject to a constraint on the path cost. The orienteering problem can be thus broken down into an optimization objective that involves selecting a subset of nodes from the given graph and a traveling salesman problem, that aims to minimize the distance traveled while visiting the selected subset of nodes. This optimization objective, which dictates how the subset of nodes are selected, can be chosen based on the specific task to which orienteering is being applied. For example \cite{lorenzo_orienteering} applies it to the task of environmental monitoring in an aquatic setting while \cite{arora_randomized_2017} applies it to the more general Constraint Satisfaction Problem (CSP). 

% However formulating IPP as an orienteering problem is subject to expensive run times that quickly become intractable as the probability of near-optimal paths existing in the solution space becomes small. 
However, framing IPP as an orienteering problem leads to computationally expensive runtimes, which rapidly become intractable as the likelihood of near-optimal paths in the solution space diminishes. As the authors \cite{efficient-sensing} note, the IPP problem is a NP-hard search problem. In order to overcome this, the authors develop an approximation algorithm that efficiently finds near-optimal solutions leveraging mutual information in the problem space. They do this by ensuring that the mutual information formulation is submodular in nature as discussed in \cite{submodular-krause}. This results in a diminishing returns property whereby making a new observation yields more information if fewer observations have been made up to that point, and less information if many observations have already been made. Hence, the IPP problem can be reformulated as maximizing a submodular objective function while constrained to a budget. \cite{recursive_greedy} studies this formulation by developing a recursive-greedy algorithm that they show has strong theoretical guarantees.

However, these mentioned approaches have only been used in applications that have relatively small search spaces such as environmental monitoring \cite{lorenzo_orienteering}, water quality monitoring and analysis \cite{water-1}, or detection of chemical or biological plumes \cite{1703649} that are then discretized into a finite set of sensing locations. As one can imagine, this becomes quickly intractable in applications that have much larger search spaces, are higher dimensional, or both. Examples of this type of situation include searching for targets of interest over a large search area, such as a missing hiker or tracking multiple dynamic targets of interest over a larger search area, such as ships at sea.  In order to combat this, the authors in \cite{lorenzo_orienteering} propose using heuristics while \cite{lorenzo_orienteering, efficient_informative, recursive_greedy} also suggest using greedy approaches to approximate solutions. Another method involves Mixed Integer Programming (MIP) based solutions such as \cite{mip_forumation}. But these solutions fail to capture the relationship between nodes which would not be ideal for many problem statements. The limitations in capturing the relationship in rewards among nodes along with the inefficient run times necessitate developing better exploration algorithms that would allow for robust information gathering by autonomous systems in complex problem settings. 

Some previous approaches have tried using receding horizon-based methods to solve the IPP problem. Local receding-horizon based solutions optimize the information-objective over a small lookahead or horizon. They rely on replanning to approach a globally optimal solution. One such approach involves a receding horizon algorithm \cite{receding} that satisfies a temporal logic specification for problem statements where safety and reliability are paramount. Another approach by \cite{no-regret} involves using a receding horizon planner to perform informative planning in a latent environment modeled using Gaussian processes. According to \cite{experimental_receeding_horizon}, receding horizon-based path planners are susceptible to becoming trapped in local optima. To address this issue, the study explores the impact of adjusting the horizon length and considers adapting it based on the remaining information. They propose a J-Horizon algorithm, which incorporates a lookahead step size to improve convergence toward better optima. Despite these improvements, receding horizon methods primarily plan optimal paths reactively and fail to fully utilize the available prior information about the environment \cite{Kailas-2023-137637}.

% However, \cite{experimental_receeding_horizon} notes that receding horizon based path planners are prone to getting trapped in local optima. In order to overcome this, they examine the effect of modifying the horizon length, and the possibility of modifying this based on remaining information. They introduce a J-Horizon algorithm that involves a lookahead step size to allow for convergence to better optima. Nevertheless, receding horizon methods plan optimal paths reactively and do not fully leverage the known prior environment information \cite{Kailas-2023-137637}. 

% It is also worth noting that \cite{frolov2014} compares lawnmower paths to other planning algorithms and notes that they are only marginally worse than adaptive algorithms. In fact, they also conclude that graph-based search algorithms perform worse than lawnmower patterns due to their inability to adapt to uncertainty in their priors. 

Notably, \cite{frolov2014} compares lawnmower paths with other planning algorithms and observes that their performance is only slightly inferior to that of adaptive algorithms. Furthermore, the study concludes that graph-based search algorithms under perform compared to lawnmower patterns due to their limited ability to adapt to uncertainties in prior information. The ability to allow for revisits is very important for finding globally optimal paths, especially when the information objective is submodular or when the task requires revisits \cite{moon2022tigris}. Another approach which has shown promise involves branch and bound techniques which prune suboptimal branches early in the tree search \cite{6224902, bnb-2}. However, efficiently calculating tight bounds in problems with unknown environments and high-dimensional state space is nontrivial. Previous works have solved this using Gaussian processes (GP). However, implementing GPs is application specific and a specific approach used in one work may not be transferable to another application. 
% Secondly, in general, the computational complexity of inference in GPs grows cubically with the number of observations made \cite{rasmussen2006gaussian}. While there are ways to reduce this complexity, it remains computationally challenging to perform long horizon planning without making stationary assumptions about the covariance function. 
Secondly, the computational complexity of inference in GPs generally scale cubically with the number of observations \cite{rasmussen2006gaussian}. Although methods exist to mitigate this complexity, long-horizon planning remains computationally demanding unless stationary assumptions are made about the covariance function. This is further compounded when planning in higher-dimensional state spaces. 

Numerous works have worked to address these drawbacks by formulating the task of information gathering in continuous space and introducing adaptive sampling-based algorithms. Examples of some of these works include \cite{hollinger_sampling-based_2014}, \cite{schmid_efficient_2020}, \cite{hollinger_long-horizon_2015, nasa-mcts, adaptive-sampling-1, adaptive-sampling-2, 6907763, 6630605}. These methods employ sampling-based approaches, which involves the selection and inclusion of new states in continuous space into a tree structure. As the tree grows, the optimal path---maximizing information gain while respecting budget constraints---is continuously refined and the best path is selected. However, these approaches face significant challenges in expansive spaces, where increasing dimensionality causes the search space to grow substantially relative to regions of high information reward. In order to address these challenges, \cite{moon2022tigris} introduced a sampling-based approach for IPP in large and high-dimensional search spaces. The method performs informed sampling within the continuous space and incorporates edge information gain during reward estimation, efficiently generating global paths that optimize information gathering within budget constraints.

% This is done by performing informed sampling in the high-dimensional continuous space and incorporating potential information gain along edges in the reward estimation. This method rapidly generates a global path that maximizes information gain for the given path budget constraints.

% The approach in \cite{moon2022tigris} produced effective global plans, however, they were static throughout the robot execution. The algorithm we propose in this work, IA-TIGRIS, also uses a sampling-based approach, but IA-TIGRIS continues to optimize and refine plans online in an adaptive and incremental manner as the robot gathers new information. This online refinement is especially important in long-horizon planning where optimal paths are harder to find within just one planning cycle. Our method is efficient enough to run on an onboard computer due to improvements in the tree building logic and belief space node embeddings. We test IA-TIGRIS extensively in simulation and is shown to plan efficiently and effectively online on two different UAV systems.
% Builds on the same backbone of TIGRIS, but adds online, adaptive, incremental, real-time, and other contributions. long horizon. 

While \cite{moon2022tigris} demonstrated effective global planning, the approach generated static plans that remained unchanged during robot execution. Our proposed algorithm, IA-TIGRIS, also uses a sampling-based framework but continuously optimizes and refines plans online as the robot acquires new information. This adaptive refinement is crucial for long-horizon planning scenarios where optimal paths are difficult to determine in a single planning cycle. Through improvements in tree-building logic and novel belief space node embeddings, IA-TIGRIS achieves sufficient computational efficiency for onboard deployment. We validate our method through extensive simulation testing and demonstrate efficient online planning capabilities on hardware across two distinct UAV systems.

% Below is how Hollinger talked about his previous work and shows extension to journal paper 
% In our own prior work, we have examined the problems of autonomous inspection [14] and bathymetric mapping [15] using sampling-based planners. More recently, we developed a general Rapidly-exploring Information Gathering (RIG) algorithm [13] for maximizing information objectives with mobile robots. The algorithm proposed in the current paper is based on similar principles as RIG and RRT*, but it improves on the existing algorithms’ capabilities by allowing for real-time optimization of search and classification objective functions. Thus, the proposed ReASC algorithm fills an important gap that is not currently covered by existing algorithms: realtime long-horizon optimization for search and classification on fielded systems.


