\vspace{-6mm}

\section{Introduction}
\label{sec:intro}

Synthesizing high-quality novel view images from a single input image is a long-standing and challenging problem. It requires inheriting the appearance of objects in the observed regions of the input image while also hallucinating unseen regions. Recent studies~\cite{3dim, zero123} approach this problem as an image-to-image translation task and implement it using diffusion models~\cite{diffusion, ddim}, drawing inspiration from their successful application in 2D image generation~\cite{stable-diffusion, imagen}.	
While they exhibit remarkable zero-shot capabilities when trained with large-scale 2D and 3D datasets, they still face challenges in maintaining 3D consistency between the target view and the generated multi-view images, due to the probabilistic nature of diffusion models. This limitation adversely affects downstream applications such as 3D reconstruction~\cite{dreamfusion, neus}.

In this paper, we propose to improve the consistency of synthesized multi-view images by optimizing the utilization of reference image information. 
Notably, maintaining consistency between the generated image and the corresponding observed regions in the input view is a crucial requirement in the task of single-image conditioned novel view synthesis. 
However, existing methods often overlook this constraint by merely considering the input image as a condition or network input, which fails to guarantee such consistency. 
One straightforward method to fulfill this constraint is by warping the content from the input to the target view and subsequently conducting outpainting for the remaining regions~\cite{text2nerf, xiang20233d}.
However, 3D warping relies on precise depth information, which is hard to obtain.
Additionally, direct warping struggles with occlusion and illumination variations across different views.

We aim to utilize this constraint to improve the consistency in a more adaptable way.
Despite the intricacies of obtaining depth, we can still reduce the search space for locating corresponding points by incorporating other 3D geometric priors. 
As depicted in Fig.~\ref{fig:epipolar_vis}, the corresponding points in the reference views must be on the epipolar line.
Therefore, we propose an epipolar attention module to locate and gather contextual information. 
For each point in the target view visible in the reference view, we can first constrain the corresponding point to its respective epipolar line in the reference image. Subsequently, we ascertain the corresponding location along the epipolar line by feature matching.
The features at the localized positions are then retrieved and used to constrain the target view generation.

More specifically, we first perform DDIM inversion on the input view and reconstruct the input image using the initial noise provided by the DDIM inversion.
This process yields intermediate features of the input view, which can then be employed to constrain the generation of target views.
Then, in the epipolar attention module, we traverse the corresponding epipolar line in the input view for every point within the target view. During this process, we compute the similarity between the features of the target point and those sampled from the input view. This similarity score is then used to aggregate the corresponding features from the input view.
This soft operation is more adept at handling complex scenarios, such as occlusion (detailed analysis can be found in the Supplementary Material).
Additionally, to avoid any parameter training or fine-tuning, we employ a simple parameter duplication strategy, \ie, we copy all parameters directly from the self-attention layer to obtain the epipolar attention parameters.
To further improve the consistency between different target views, we expand the application of epipolar attention to a multi-view context.
Specifically, we generate multiple target views in an auto-regressive manner. 
When generating a specific novel view, we consider the input view and previously generated target views close to the current viewpoint as context views.
We employ epipolar attention to aggregate overlapping information from all context views, rather than solely from the input view, thereby improving consistency among all generated views.
It is worth mentioning that our epipolar attention reduces the search space compared to locating corresponding points in the full image. Therefore, it requires much less memory when retrieving information from multiple views, making it more friendly to GPUs with small memory capacity.

\begin{figure}[t]
\vspace{-5mm}
    \centering
    \includegraphics[width=.7\linewidth]{figs/epipolar_cor.pdf}
     \vspace{-5mm}
    \caption{When the camera viewing frustum of two views overlaps, for a point on one of the images, we can find its correspondence on the epipolar line of the other view.}
    \label{fig:epipolar_vis}
    \vspace{-3mm}
\end{figure}

\section{Introduction}
\label{sec:intro}

\begin{figure*}[tb]
    \centering
    \includegraphics[width=0.848\linewidth]{figs/circuitnn.pdf} 
    \caption{Illustration of differentiable CircuitNN. CircuitNN is designed based on differentiable NAND gates. After DAS is guided by PI and PO pairs of the truth table, CircuitNN can get the precise circuit architecture logic equivalent to the truth table.}
    \label{fig:circuitnn}
\end{figure*}

% 1. Describe the importance of logic synthesis
% 2. Existing Problems
% (a) Neural Architecture Search: Unstable, Predefined Setting, etc.
% (b) Circuit Generation: Probabilistic Model, Logic Equivalence

With the rapid advancement of technology, the scale of integrated circuits (ICs) has expanded exponentially. 
This expansion has introduced significant challenges in chip manufacturing, particularly concerning power and area metrics.
A primary objective in IC design is achieving the same circuit function with fewer transistors, thereby reducing power usage and area occupancy.

Logic synthesis~\cite{hachtel2005logicsynth}, a critical step in electronic design automation (EDA), transforms behavioral-level circuit designs into optimized gate-level circuits, ultimately yielding the final IC layout. 
The primary goal of logic synthesis is to identify the physical implementation with the fewest gates for a given circuit function. 
This task constitutes a challenging NP-hard combinatorial optimization problem. 
Current logic synthesis tools~\cite{brayton2010abc, wolf2013yosys} rely on human-designed heuristics, often leading to sub-optimal outcomes.

Differentiable architecture search (DAS) techniques~\cite{liu2018darts, chu2020darts} offer novel perspectives on addressing challenges in this problem.
Circuit functions can be represented through truth tables, which map binary inputs to their corresponding outputs. 
Truth tables provide a precise representation of input-output relationships, ensuring the design of functionally equivalent circuits.
Inspired by this, researchers~\cite{deepmind2024ai4sys, wang2024tnet} have begun exploring the application of DAS to synthesize circuits directly from truth tables.
Specifically, \citet{deepmind2024ai4sys} proposed CircuitNN, a framework that learns differentiable connection structures with logic gates, enabling the automatic generation of logic circuits from truth tables.
This approach significantly reduces the complexity of traditional circuit generation. 
Building on this, \citet{wang2024tnet} introduced T-Net, a triangle-shaped variant of CircuitNN, incorporating regularization techniques to enhance the efficiency of DAS.

Despite these advancements, several challenges remain. 
The computational complexity of DAS grows quadratically with the number of gates, posing scalability issues.
Although triangle-shaped architecture~\cite{wang2024tnet} partially mitigates this problem, redundancy persists. 
%Additionally, DAS is susceptible to converging to local optima, limiting the ability to search architectures that satisfy the given truth tables~\cite{liu2018darts}. 
%Furthermore, hyperparameters (network depth and layer width) require extensive searches, introducing complexity and prolonging the synthesis process. 
Additionally, DAS is susceptible to converging to local optima~\cite{liu2018darts} and hyperparameters (network depth and layer width) require extensive searches. 
The challenges arise from the vast search space in DAS. 
% Even with predefined settings for CircuitNN, finding a configuration that meets the truth table requires extensive trial and error during the DAS process. 
Intuitively, limiting the search space through predefined parameters (network depth, gates per layer, and connection probabilities) can significantly reduce the complexity.

Recent advances~\cite{openai2023gpt4, abramson2024alphafold3, esser2024sd3, li2024mar} in conditional generative models have demonstrated remarkable performance across language, vision, and graph generation tasks. 
Motivated by these developments, we propose a novel approach to circuit generation that generates preliminary circuit structures to guide DAS in generating refined circuits matching specified truth tables. 
Firstly, we introduce CircuitVQ, a tokenizer with a discrete codebook for circuit tokenization. 
Built upon our Circuit AutoEncoder framework~\cite{hou2022graphmae,li2023maskgae,wu2025mgvga}, CircuitVQ is trained through a circuit reconstruction task. 
Specifically, the CircuitVQ encoder encodes input circuits into discrete tokens using a learnable codebook, while the decoder reconstructs the circuit adjacency matrix based on these tokens.
Subsequently, the CircuitVQ encoder serves as a circuit tokenizer for CircuitAR pretraining, which employs a masked autoregressive modeling paradigm~\cite{chang2022maskgit, li2023mage}. 
In this process, the discrete codes function as supervision signals. 
After training, CircuitAR can generate discrete tokens progressively, which can be decoded into initial circuit structures by the decoder of the CircuitVQ. 
These prior insights can guide DAS in producing refined circuits that match the target truth tables precisely.

Our key contributions can be summarized as follows:
\begin{itemize}
\item We introduce CircuitVQ, a circuit tokenizer that facilitates graph autoregressive modeling for circuit generation, based on our Circuit AutoEncoder framework;
\item Develop CircuitAR, a model trained using masked autoregressive modeling, which generates initial circuit structures conditioned on given truth tables;
\item Propose a refinement framework that integrates differentiable architecture search to produce functionally equivalent circuits guided by target truth tables;
\item Comprehensive experiments demonstrating the scalability and capability emergence of our CircuitAR and the superior performance of the proposed circuit generation approach.
\end{itemize}

% Motivation
% (a) Diffusion (Vision, Graph), Autoregressive (Language, Vision)
% (b) Circuit Generation for Predefined Setting
% (c) Neural Architecture Search for Strict Logic Equivalence

% Contribution
% (a) Circuit Tokenizer (new transformer arch, training strategy)
% (b) CircuitAR (train and gen strategies, post-ar strategy)
% (c) Extensive Evaluation including BitD (Bit Distance) for Scalability


We conduct experiences on the Google Scanned Objects~\cite{GSO} dataset to verify the zero-shot novel view synthesis capability and evaluate our method on both generated image quality and the view consistency~\cite{3dim}. 
Additionally, we apply our method to the downstream 3D reconstruction task~\cite{neus} and compare it against the mesh constructed by our baseline model.

The main contributions of this work are:
\begin{compactitem}
    \item  We propose a novel epipolar attention method to locate and retrieve the corresponding information in the reference view, which is then inserted into the generation process of the target view to enhance the consistency between multi-view images.
    \item Experimental results show that our method effectively improves the consistency of the synthesized multi-view images without any training or fine-tuning while maintaining the quality of the generated images.
    \item We apply the synthesized multi-view images to a downstream 3D reconstruction task, and the results show that the more consistent images further improve the 3D reconstruction results.
\end{compactitem} 
