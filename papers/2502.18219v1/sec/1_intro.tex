\vspace{-6mm}

\section{Introduction}
\label{sec:intro}

Synthesizing high-quality novel view images from a single input image is a long-standing and challenging problem. It requires inheriting the appearance of objects in the observed regions of the input image while also hallucinating unseen regions. Recent studies~\cite{3dim, zero123} approach this problem as an image-to-image translation task and implement it using diffusion models~\cite{diffusion, ddim}, drawing inspiration from their successful application in 2D image generation~\cite{stable-diffusion, imagen}.	
While they exhibit remarkable zero-shot capabilities when trained with large-scale 2D and 3D datasets, they still face challenges in maintaining 3D consistency between the target view and the generated multi-view images, due to the probabilistic nature of diffusion models. This limitation adversely affects downstream applications such as 3D reconstruction~\cite{dreamfusion, neus}.

In this paper, we propose to improve the consistency of synthesized multi-view images by optimizing the utilization of reference image information. 
Notably, maintaining consistency between the generated image and the corresponding observed regions in the input view is a crucial requirement in the task of single-image conditioned novel view synthesis. 
However, existing methods often overlook this constraint by merely considering the input image as a condition or network input, which fails to guarantee such consistency. 
One straightforward method to fulfill this constraint is by warping the content from the input to the target view and subsequently conducting outpainting for the remaining regions~\cite{text2nerf, xiang20233d}.
However, 3D warping relies on precise depth information, which is hard to obtain.
Additionally, direct warping struggles with occlusion and illumination variations across different views.

We aim to utilize this constraint to improve the consistency in a more adaptable way.
Despite the intricacies of obtaining depth, we can still reduce the search space for locating corresponding points by incorporating other 3D geometric priors. 
As depicted in Fig.~\ref{fig:epipolar_vis}, the corresponding points in the reference views must be on the epipolar line.
Therefore, we propose an epipolar attention module to locate and gather contextual information. 
For each point in the target view visible in the reference view, we can first constrain the corresponding point to its respective epipolar line in the reference image. Subsequently, we ascertain the corresponding location along the epipolar line by feature matching.
The features at the localized positions are then retrieved and used to constrain the target view generation.

More specifically, we first perform DDIM inversion on the input view and reconstruct the input image using the initial noise provided by the DDIM inversion.
This process yields intermediate features of the input view, which can then be employed to constrain the generation of target views.
Then, in the epipolar attention module, we traverse the corresponding epipolar line in the input view for every point within the target view. During this process, we compute the similarity between the features of the target point and those sampled from the input view. This similarity score is then used to aggregate the corresponding features from the input view.
This soft operation is more adept at handling complex scenarios, such as occlusion (detailed analysis can be found in the Supplementary Material).
Additionally, to avoid any parameter training or fine-tuning, we employ a simple parameter duplication strategy, \ie, we copy all parameters directly from the self-attention layer to obtain the epipolar attention parameters.
To further improve the consistency between different target views, we expand the application of epipolar attention to a multi-view context.
Specifically, we generate multiple target views in an auto-regressive manner. 
When generating a specific novel view, we consider the input view and previously generated target views close to the current viewpoint as context views.
We employ epipolar attention to aggregate overlapping information from all context views, rather than solely from the input view, thereby improving consistency among all generated views.
It is worth mentioning that our epipolar attention reduces the search space compared to locating corresponding points in the full image. Therefore, it requires much less memory when retrieving information from multiple views, making it more friendly to GPUs with small memory capacity.

\begin{figure}[t]
\vspace{-5mm}
    \centering
    \includegraphics[width=.7\linewidth]{figs/epipolar_cor.pdf}
     \vspace{-5mm}
    \caption{When the camera viewing frustum of two views overlaps, for a point on one of the images, we can find its correspondence on the epipolar line of the other view.}
    \label{fig:epipolar_vis}
    \vspace{-3mm}
\end{figure}

\section{Introduction}


\begin{figure}[t]
\centering
\includegraphics[width=0.6\columnwidth]{figures/evaluation_desiderata_V5.pdf}
\vspace{-0.5cm}
\caption{\systemName is a platform for conducting realistic evaluations of code LLMs, collecting human preferences of coding models with real users, real tasks, and in realistic environments, aimed at addressing the limitations of existing evaluations.
}
\label{fig:motivation}
\end{figure}

\begin{figure*}[t]
\centering
\includegraphics[width=\textwidth]{figures/system_design_v2.png}
\caption{We introduce \systemName, a VSCode extension to collect human preferences of code directly in a developer's IDE. \systemName enables developers to use code completions from various models. The system comprises a) the interface in the user's IDE which presents paired completions to users (left), b) a sampling strategy that picks model pairs to reduce latency (right, top), and c) a prompting scheme that allows diverse LLMs to perform code completions with high fidelity.
Users can select between the top completion (green box) using \texttt{tab} or the bottom completion (blue box) using \texttt{shift+tab}.}
\label{fig:overview}
\end{figure*}

As model capabilities improve, large language models (LLMs) are increasingly integrated into user environments and workflows.
For example, software developers code with AI in integrated developer environments (IDEs)~\citep{peng2023impact}, doctors rely on notes generated through ambient listening~\citep{oberst2024science}, and lawyers consider case evidence identified by electronic discovery systems~\citep{yang2024beyond}.
Increasing deployment of models in productivity tools demands evaluation that more closely reflects real-world circumstances~\citep{hutchinson2022evaluation, saxon2024benchmarks, kapoor2024ai}.
While newer benchmarks and live platforms incorporate human feedback to capture real-world usage, they almost exclusively focus on evaluating LLMs in chat conversations~\citep{zheng2023judging,dubois2023alpacafarm,chiang2024chatbot, kirk2024the}.
Model evaluation must move beyond chat-based interactions and into specialized user environments.



 

In this work, we focus on evaluating LLM-based coding assistants. 
Despite the popularity of these tools---millions of developers use Github Copilot~\citep{Copilot}---existing
evaluations of the coding capabilities of new models exhibit multiple limitations (Figure~\ref{fig:motivation}, bottom).
Traditional ML benchmarks evaluate LLM capabilities by measuring how well a model can complete static, interview-style coding tasks~\citep{chen2021evaluating,austin2021program,jain2024livecodebench, white2024livebench} and lack \emph{real users}. 
User studies recruit real users to evaluate the effectiveness of LLMs as coding assistants, but are often limited to simple programming tasks as opposed to \emph{real tasks}~\citep{vaithilingam2022expectation,ross2023programmer, mozannar2024realhumaneval}.
Recent efforts to collect human feedback such as Chatbot Arena~\citep{chiang2024chatbot} are still removed from a \emph{realistic environment}, resulting in users and data that deviate from typical software development processes.
We introduce \systemName to address these limitations (Figure~\ref{fig:motivation}, top), and we describe our three main contributions below.


\textbf{We deploy \systemName in-the-wild to collect human preferences on code.} 
\systemName is a Visual Studio Code extension, collecting preferences directly in a developer's IDE within their actual workflow (Figure~\ref{fig:overview}).
\systemName provides developers with code completions, akin to the type of support provided by Github Copilot~\citep{Copilot}. 
Over the past 3 months, \systemName has served over~\completions suggestions from 10 state-of-the-art LLMs, 
gathering \sampleCount~votes from \userCount~users.
To collect user preferences,
\systemName presents a novel interface that shows users paired code completions from two different LLMs, which are determined based on a sampling strategy that aims to 
mitigate latency while preserving coverage across model comparisons.
Additionally, we devise a prompting scheme that allows a diverse set of models to perform code completions with high fidelity.
See Section~\ref{sec:system} and Section~\ref{sec:deployment} for details about system design and deployment respectively.



\textbf{We construct a leaderboard of user preferences and find notable differences from existing static benchmarks and human preference leaderboards.}
In general, we observe that smaller models seem to overperform in static benchmarks compared to our leaderboard, while performance among larger models is mixed (Section~\ref{sec:leaderboard_calculation}).
We attribute these differences to the fact that \systemName is exposed to users and tasks that differ drastically from code evaluations in the past. 
Our data spans 103 programming languages and 24 natural languages as well as a variety of real-world applications and code structures, while static benchmarks tend to focus on a specific programming and natural language and task (e.g. coding competition problems).
Additionally, while all of \systemName interactions contain code contexts and the majority involve infilling tasks, a much smaller fraction of Chatbot Arena's coding tasks contain code context, with infilling tasks appearing even more rarely. 
We analyze our data in depth in Section~\ref{subsec:comparison}.



\textbf{We derive new insights into user preferences of code by analyzing \systemName's diverse and distinct data distribution.}
We compare user preferences across different stratifications of input data (e.g., common versus rare languages) and observe which affect observed preferences most (Section~\ref{sec:analysis}).
For example, while user preferences stay relatively consistent across various programming languages, they differ drastically between different task categories (e.g. frontend/backend versus algorithm design).
We also observe variations in user preference due to different features related to code structure 
(e.g., context length and completion patterns).
We open-source \systemName and release a curated subset of code contexts.
Altogether, our results highlight the necessity of model evaluation in realistic and domain-specific settings.






We conduct experiences on the Google Scanned Objects~\cite{GSO} dataset to verify the zero-shot novel view synthesis capability and evaluate our method on both generated image quality and the view consistency~\cite{3dim}. 
Additionally, we apply our method to the downstream 3D reconstruction task~\cite{neus} and compare it against the mesh constructed by our baseline model.

The main contributions of this work are:
\begin{compactitem}
    \item  We propose a novel epipolar attention method to locate and retrieve the corresponding information in the reference view, which is then inserted into the generation process of the target view to enhance the consistency between multi-view images.
    \item Experimental results show that our method effectively improves the consistency of the synthesized multi-view images without any training or fine-tuning while maintaining the quality of the generated images.
    \item We apply the synthesized multi-view images to a downstream 3D reconstruction task, and the results show that the more consistent images further improve the 3D reconstruction results.
\end{compactitem} 
