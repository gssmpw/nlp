\section{Conclusion}

We introduce {\methodname}, a systematic framework for evaluating factuality and fairness inLLMs and T2I models.
Our approach constructs objective queries from 19 real-world statistics and subjective queries based on three cognitive biases.
We design multiple evaluation metrics, including $S_{fact}$, $S_E$, $S_{KLD}$, and $S_{fair}$ to assess six LLMs and four T2I models.
A formal analysis demonstrates a trade-off between $S_{fact}$ and $S_E$.
Empirical findings reveal three key insights:
(1) T2I models exhibit lower world knowledge than LLMs, leading to errors in objective queries.
(2) Both T2I models and LLMs display significant variability in handling subjective queries.
(3) LLMs are susceptible to cognitive biases, especially representativeness bias.

\section*{Limitations}

This study has several limitations:
\textbf{(1)} The 19 statistics analyzed are specific to U.S. society and may not generalize to global contexts.
\textbf{(2)} The evaluation includes only a subset of LLM and T2I models, omitting many existing models.
\textbf{(3)} The templates for subjective queries may not fully capture real-world user scenarios.
However, the proposed {\methodname} framework allows researchers to extend test cases by incorporating additional statistics and generating diverse queries to better represent daily scenarios and assess a broader range of AI models.
Therefore, these limitations do not undermine the novelty or practical value of {\methodname}.

\section*{Ethics Statements}

Fairness proposed in this study emphasizes diversity and respect for individual differences. Our goal is to balance fairness and factuality, providing a scientific reference for AI model evaluation, rather than direct use in decision-making scenarios.

\section*{Acknowledgments}

We would like to thank Professor Jieyu Zhao from University of Southern California for her valuable suggestions during this research.
The work is supported by the Research Grants Council of the Hong Kong Special Administrative Region, China (No. CUHK14206921 of the General Research Fund).