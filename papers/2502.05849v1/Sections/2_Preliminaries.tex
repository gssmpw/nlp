\section{Preliminaries}

\subsection{Definition}

\paragraph{Factuality} In this paper, factuality refers to a model's ability to produce content aligned with established facts and world knowledge~\cite{wang2023survey, mirza2024global}, demonstrating its effectiveness in acquiring, understanding, and applying factual information~\cite{wang2024factuality}.

\paragraph{Fairness} In this paper, fairness is defined as ensuring that algorithmic decisions are unbiased toward any individual, irrespective of attributes such as gender or race~\cite{mehrabi2021survey, verma2018fairness}, promoting equal treatment across diverse groups~\cite{hardt2016equality}.

\subsection{Cognitive Errors}
\label{sec:preliminaries}

Human prejudice and stereotypes often stem from cognitive errors.
In this section, we introduce three common errors along with their underlying psychological mechanisms.

\paragraph{(1) Representativeness Bias} This is the tendency to make decisions by matching an individual or situation to an existing mental prototype~\cite{kahneman1972subjective, lim1997debiasing}.
When dealing with group characteristics, people often believe that each individual conforms to the perceived traits of the group~\cite{feldman1981beyond}.
For example, although statistics may indicate higher crime rates within a particular group, this does not imply that every individual within that group has an increased likelihood of committing a crime.

\paragraph{(2) Attribution Error} This refers to the tendency to overestimate the influence of internal traits and underestimate situational factors when explaining others' behavior~\cite{pettigrew1979ultimate, harman1999moral}.
When observing an individual from a particular group engaging in certain behavior, people are prone to mistakenly attribute that behavior to the entire group’s internal characteristics rather than to external circumstances.

\paragraph{(3) In-group/Out-group Bias} This is the tendency to favor individuals within one's own group (in-group) while being more critical and negatively biased toward those in other groups (out-groups)~\cite{brewer1979group, downing1986group, struch1989intergroup}.
Negative traits are often attributed to out-group members, fostering prejudice and reinforcing stereotypes by disregarding individual differences.
In contrast, positive traits are more ascribed to in-group members.