\section{Introduction}

% Motivation
In February 2024, users discovered that Gemini's image generator produced black Vikings and Asian Nazis without such explicit instructions.
The incident quickly gained attention and was covered by major media~\cite{economist2024google, grant2024google}, prompting Google to suspend the service.
This case highlights the complexities involved in promoting diversity in generative models, suggesting that it may not always be appropriate.
Consequently, researchers have begun investigating the trade-off between instructing models to reflect historical facts and promoting diversity~\cite{wan2024factuality}.
Nevertheless, determining when models should prioritize factuality over diversity remains unexplored.

\begin{figure}[t]
  \centering
  \subfloat[Testing with \textbf{objective} queries that require \textbf{accuracy}.]{
    \includegraphics[width=1.0\linewidth]{Figures/obj-cover.pdf}
    \label{fig:obj-cover}
  } \\
  \subfloat[Testing with \textbf{subjective} queries that require \textbf{diversity}.]{
    \includegraphics[width=1.0\linewidth]{Figures/subj-cover.pdf}
    \label{fig:subj-cover}
  }
  \caption{{\methodname} is a checklist comprising objective queries derived from real-world statistics and subjective queries designed using three cognitive errors that contribute to stereotypes. It includes queries designed for LLMs and T2I models.}
\end{figure}

% Statistics & Objective Queries
To address this gap, this study introduces {\methodname}, a checklist designed to assess models' capabilities in providing accurate world knowledge and demonstrating fairness in daily scenarios.
For world knowledge assessment, we collect 19 key statistics on U.S. economic, social, and health indicators from authoritative sources such as the Bureau of Labor Statistics, the Census Bureau, and the Centers for Disease Control and Prevention.
Using detailed demographic data, we pose objective, fact-based queries to the models, such as ``Which group has the highest crime rate in the U.S.?''—requiring responses that accurately reflect factual information, as shown in Fig.~\ref{fig:obj-cover}.
Models that uncritically promote diversity without regard to factual accuracy receive lower scores on these queries.

% Cognitive Errors & Subjective Queries
It is also important for models to remain neutral and promote equity under special cases.
To this end, {\methodname} includes diverse subjective queries related to each statistic.
Our design is based on the observation that individuals tend to overgeneralize personal priors and experiences to new situations, leading to stereotypes and prejudice~\cite{dovidio2010prejudice, operario2003stereotypes}.
For instance, while statistics may indicate a lower life expectancy for a certain group, this does not mean every individual within that group is less likely to live longer.
Psychology has identified several cognitive errors that frequently contribute to social biases, such as representativeness bias~\cite{kahneman1972subjective}, attribution error~\cite{pettigrew1979ultimate}, and in-group/out-group bias~\cite{brewer1979group}.
Based on this theory, we craft subjective queries to trigger these biases in model behaviors.
Fig.~\ref{fig:subj-cover} shows two examples on AI models.

% Metrics, Trade-off, Experiments, Findings
We design two metrics to quantify factuality and fairness among models, based on accuracy, entropy, and KL divergence.
Both scores are scaled between 0 and 1, with higher values indicating better performance.
We then mathematically demonstrate a trade-off between factuality and fairness, allowing us to evaluate models based on their proximity to this theoretical upper bound.
Given that {\methodname} applies to both large language models (LLMs) and text-to-image (T2I) models, we evaluate six widely-used LLMs and four prominent T2I models, including both commercial and open-source ones.
Our findings indicate that GPT-4o~\cite{openai2023gpt} and DALL-E 3~\cite{openai2023dalle} outperform the other models.
Our contributions are as follows:
\begin{enumerate}[noitemsep, leftmargin=*]
    \item We propose {\methodname}, collecting 19 real-world societal indicators to generate objective queries and applying 3 psychological theories to construct scenarios for subjective queries.
    \item We develop several metrics to evaluate factuality and fairness, and formally demonstrate a trade-off between them.
    \item We evaluate six LLMs and four T2I models using {\methodname}, offering insights into the current state of AI model development.
\end{enumerate}