\section{Related Work}
\textbf{Task-oriented dialog systems.}
Task-oriented dialog (ToD) systems facilitate users in achieving specific objectives through natural language interactions, such as making reservations or inquiries \cite{qin2023end}. Numerous methodologies \cite{hosseini2020simple, he2022galaxy, yang2021ubar, hudevcek2023llms, chung2023instructtods} have been proposed to enhance ToD performance. However, existing ToD frameworks lack specialization for open-world ride-hailing scenarios. The dynamic nature of ride-hailing interactions—spanning order creation, modification, and cancellation—necessitates a tailored solution that current ToD approaches fail to provide.



\textbf{LLM-based assistants.}
The development of domain-specific intelligent assistants powered by Large Language Models (LLMs) \cite{dong2023towards} has gained significant traction in both academia and industry. LLMs enable automated planning and execution of complex industrial tasks across various domains. For instance, Mind2Web \cite{deng2024mind2web}, WebGPT \cite{gur2023real}, and AutoWebGLM \cite{lai2024autowebglm} have been introduced as LLM-powered web navigation assistants. LLMPA \cite{guan2023intelligent} serves as a virtual assistant for travel planning within the Alipay App, while Flowris \cite{sun2023flowris} and GRILLBot \cite{fischer2024grillbot} function as intelligent assistants for data management and multimodal conversation, respectively. Despite these advancements, no LLM-based assistant has been designed specifically to automate ride-hailing services, highlighting a gap in existing research.