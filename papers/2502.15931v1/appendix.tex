\appendix

\section{Proofs}

% \subsection{Proof of Corollary~\ref{theorem:all_deviation}}

% \begin{proof}[Proof of Corollary~\ref{theorem:all_deviation}]
%     When all agents are deviating, it is straightforward to show that $\Tilde T = \tilde \alpha B$ with minimum eigenvalue $\tilde \alpha^2 / (\lambda_n + \tilde \alpha) > 0$. Thus $\Tilde T$ is invertible, and therefore $\bs^\prime = \frac {1} {\tilde \alpha} B^{-1} \widetilde {\diag (B)} \bs$, where $\widetilde {\diag (B)}$ is a diagonal matrix with entries $B_{ii}$. Then 

%     \begin{align*}
%         \norm \bs^\prime \norm_2 & \le \frac {1} {\tilde \alpha} \norm B^{-1} \norm_2 \norm \widetilde{\diag (B)} \norm_2 \norm \bs \norm_2 \\ &  = \left ( \max_i B_{ii} \right ) \left ( \max_i \frac {\lambda_i + \tilde \alpha} {\tilde \alpha} \right ) \norm \bs \norm_2 \\ & \le \frac {\tilde \alpha} {\lambda_1 + \tilde \alpha} \frac {\lambda_n + \tilde \alpha} {\alpha} \norm \bs \norm_2 \\ & = \frac {\lambda_n + \tilde \alpha} {\tilde \alpha} \norm \bs \norm_2.
%     \end{align*}
    
% \end{proof}


\subsection{Proof of \cref{prop:blockmodel}}

    Let $V_1$ correspond to the vertex set for community 1 and $V_2$ correspond to the vertex set for community 2. Let $I_{q}$ denote the $q \times q$ identity matrix, and $a_i = \abs{V_i \cap S}$ for $i = 1, 2$.
    Then $$X_S^T X_S = \begin{pmatrix} I_{a_1} & 0 \\ 0 & I_{a_2} \end{pmatrix}.$$ 

    Therefore $\lambda_{\max} (X_S^TX_S)= \max \{ |V_1 \cap S|, |V_2 \cap S| \}$ and $\lambda_{\min} (X_S^T X_S) = \min \{ |V_1 \cap S|, |V_1 \cap S| \}$.

    First, we determine sufficient ranges of $\gamma$ and the value of $\Xi_\gamma$: Let $S$ be such that $|S| = \gamma n$. We have the following options: 

    \begin{itemize}
        \item $S \subseteq V_1$. Then $\lambda_{\max} (X_S^T X_S) = \gamma n$.
        \item $S \subseteq V_2$. Then $\lambda_{\max} (X_S^T X_S) = \gamma n$.
        \item $V_1 \subseteq S$. Then $\lambda_{\max} (X_S^T X_S)$ lies between $\gamma n /2$ and $\gamma n$. 
        \item $V_2 \subseteq S$. Then $\lambda_{\max} (X_S^T X_S)$ lies between $\gamma n /2$ and $\gamma n$. 
        \item If $S$ lies partially in $V_1$ and $V_2$, then $\lambda_{\max} (X_S^T X_S) = \max \{(1 - t)\gamma n, t \gamma n\}$ for some $t \in [0, 1]$. Again, this is upper bounded by $\gamma n$.   
    \end{itemize}

    The above yield $\Xi_\gamma = \gamma n$. To determine $\xi_{1 - \gamma}$ we let $S$ be such that $|S| = (1 - \gamma) n$. We have the following options: 

    \begin{itemize}
        \item $S \subseteq V_1$. Then $\lambda_{\min} (X_S^T X_S) = 0$. 
        \item $S \subseteq V_2$. Then $\lambda_{\min} (X_S^T X_S) = 0$. 
        \item $V_1 \subseteq S$. Then $\lambda_{\min} (X_S^T X_S) = \min \{ n_1, (1 - \gamma) n - n_1 \} = n_1 \ge (1 - \gamma)n - n_2$ since always $n_1 \ge (1 - \gamma) n/2$. 
        \item $V_2 \subseteq S$. Then $\lambda_{\min} (X_S^T X_S) = \min \{ n_2, (1 - \gamma)n  - n_2 \} = (1 - \gamma)n - n_2$ since $n_2 \le (1 - \gamma)n/2$. 
        \item If $S$ lies partially in $V_1$ and $V_2$, then $\lambda_{\min} (X_S^T X_S) = \min \{(1 - t)(1 - \gamma) n, t (1 - \gamma) n\}$ for some $t \in [0, 1]$. Again, this is lower bounded by $( 1-  \gamma) n - n_2$.
    \end{itemize}

    Therefore, for either $1 - \gamma \le n_1 /n$ or $1 - \gamma \le n_2 / n$ we have that $\xi_{1 - \gamma} = (1 - \gamma) n - n_2$. The final inequality corresponds to 

    \begin{equation*}
        4 \sqrt {\frac {\Xi_\gamma} {\xi_{1 - \gamma}}} < 1 \iff \gamma < \frac {1} {17} - \frac {n_2} {17n}
    \end{equation*}

    Combining the above we get two systems of inequalities. The first one corresponds to $1 - \frac {n_1} {n} \le \gamma < \frac {1} {17} - \frac {n_2} {17n}$ which holds for $n_2 < 1/18$ which is impossible since $n_2 \ge 1$. The second one corresponds to $1 - \frac {n_2} n \le \gamma < \frac {1} {17} - \frac {n_2} {17n}$ which holds for $n_2 > 16/18$, which is always true. 

\section{Proof of \cref{prop:blockmodel_k}}

First, we note that if $V_1, \dots, V_K$ are the vertex sets and $S$ is a set of size $\gamma n$ the maximum eigenvalue equals to $\lambda_{\max} (X_S^T X_S) = \max_{i \in [K]} |V_i \cap S|$ and is always at most $\gamma n$. So $\Xi_\gamma = \gamma n$. If $S$ is a set of size $(1 - \gamma) n$, then the minimum eigenvalue $\lambda_{\min}(X_S^T X_S) = \min_{i \in [K} |V_i \cap S|$ is maximized when $|V_1 \cap S| = \dots = |V_K \cap S| = (1 - \gamma) n / K$ which holds as long as $(1 - \gamma)n / K \le n_K$, so $\xi_{1 - \gamma} = (1 - \gamma)n / K$ as long as $\gamma \ge 1 -  n_K/n  K$. Also the other condition is 


    \begin{equation*}
        4 \sqrt {\frac {\Xi_\gamma} {\xi_{1 - \gamma}}} < 1 \iff \gamma < \frac {1} {16K + 1}
    \end{equation*}

    Finally, we must have $1 / (16K + 1) > 1 -  K n_K / K$ which yields $n_K > n/K (16K / (16K + 1))$.

\clearpage

\section{Additional Figures} \label{app:additional_figures}

\begin{figure}[!ht]
    \centering
    \includegraphics[width=0.65\linewidth]{figures/network_stats.pdf}
    \caption{Distribution of centralities and degrees for the datasets}
    \label{fig:network_stats}
    \includegraphics[width=0.7\linewidth]{figures/Reddit.pdf}
    \caption{Running the experiments of Figure~\ref{fig:polblogs} for the Reddit dataset.}
    \label{fig:reddit}
    \includegraphics[width=0.7\linewidth]{figures/Twitter.pdf}
    \caption{Running the experiments of Figure~\ref{fig:polblogs} for the Twitter dataset.}
    \label{fig:twitter}
\end{figure}