\documentclass[format=acmsmall, review=false]{acmart}
\usepackage{acm-ec-25}
\usepackage{booktabs} % For formal tables
\usepackage[ruled]{algorithm2e} % For algorithms
\usepackage{bm}
\renewcommand{\algorithmcfname}{ALGORITHM}
\SetAlFnt{\small}
\SetAlCapFnt{\small}
\SetAlCapNameFnt{\small}
\SetAlCapHSkip{0pt}
\IncMargin{-\parindent}
\usepackage{subcaption}


\usepackage{xspace}

\usepackage{tikz}
% Choose a citation style by commenting/uncommenting the appropriate line:
%\setcitestyle{acmnumeric}
\setcitestyle{authoryear}

%%%%%%%%%%%%%%%%%%%%%%%%%%%%%%%%%%%%%%%%%%%%%%%
% SHORTCUT COMMANDS
%%%%%%%%%%%%%%%%%%%%%%%%%%%%%%%%%%%%%%%%%%%%%%%
\newcommand{\norm}{\|}
\renewcommand{\t}{\tilde}
\newcommand{\arrbegin}{\begin{eqnarray}}
\newcommand{\arrend}{\end{eqnarray}}
\newcommand{\summ}{\sum\abslimits_}
\newcommand{\barr}{\overline} 
\newcommand{\setz}{z = x + iy}
\newcommand{\setzangle}{z = re^{i\theta}}
\newcommand{\absl}{\left\lvert}
\newcommand{\absr}{\right\rvert}
\newcommand{\minsec}{\subsubsection*}
\newcommand{\integr}{\int\limits_}
\newcommand{\limninfty}{\lim\limits_{n \to \infty}}
\newcommand{\integrC}{\int\limits_{C}^{}}
\newcommand{\prob}{\subsection*}
\newcommand{\del}{\partial}
\newcommand{\RR}{\mathbb{R}}
\newcommand{\CC}{\mathbb{C}}
\newcommand{\ZZ}{\mathbb{Z}}
\newcommand{\NN}{\mathbb{N}}
\renewcommand{\AA}{\mathbb{A}}
\newcommand{\EE}{\mathop{\mathbb{E}}}
\newcommand{\PP}{\mathop{\mathbb{P}}}
\newcommand{\FF}{\mathbb{F}}
\newcommand{\QQ}{\mathbb{Q}}
\newcommand{\holder}{\text{H{\"o}lder }}
\newcommand{\rar}{\rightarrow}
\newcommand{\rarr}{\Rightarrow}
\newcommand{\suml}{\sum\limits}
\newcommand{\prodl}{\prod\limits}
\newcommand{\1}{\mathop{\mathbb{1}}}
\newcommand{\defeq}{\vcentcolon=}
\newcommand{\epszero}{\epsilon > 0}
\newcommand{\eps}{\epsilon}
\newcommand{\veps}{\varepsilon}
\newcommand{\la}{\langle}
\newcommand{\ra}{\rangle}
\newcommand{\bits}{\{0, 1\}}
\newcommand{\bb}{\{0, 1\}}
\newcommand{\pmbits}{\{\pm 1\}}
\newcommand{\pmbb}{\{\pm 1\}}
\renewcommand{\leqq}{\trianglelefteq}
\renewcommand{\geqq}{\trianglerighteq}

% Crypto/PR shortcuts
\newcommand{\ext}{\textrm{Ext}}
\newcommand{\adv}{\textsf{PRGAdv}}
\renewcommand{\ng}{\textsf{negl}}
\newcommand{\enc}{\textsf{Encrypt}}
\newcommand{\dec}{\textsf{Decrypt}}
\renewcommand{\c}{\textrm{clip}}

\usepackage{mathtools}


\DeclarePairedDelimiter\ceil{\lceil}{\rceil}
\DeclarePairedDelimiter\floor{\lfloor}{\rfloor}
\DeclarePairedDelimiter\abs{\lvert}{\rvert}

% complexity 
\newcommand{\poly}{\textsf{poly}}

\DeclareMathOperator*{\argmax}{arg\,max}
\DeclareMathOperator*{\argmin}{arg\,min}

% new commands 
\newcommand{\vc}{\textrm{vec}}
\newcommand{\tr}{\textrm{tr}}
\newcommand{\diag}{\textrm{diag}}
\renewcommand{\a}{\alpha}
\newcommand{\ba}{\barr{\alpha}}
\newcommand{\txtred}[1]{{\leavevmode\color{red}#1\color{black}\xspace}}
\newcommand{\blue}[1]{{\leavevmode\color{blue}#1\color{black}\xspace}}

\newcommand{\bmu}{\bm{\mu}}
\newcommand{\bdelta}{\bm{\delta}}
\newcommand{\bmv}{\bm{v}}
\newcommand{\bmw}{\bm{w}}
\newcommand{\bw}{\bm{w}}
\newcommand{\bv}{\bm{v}}
\newcommand{\bp}{\bm{p}}
\newcommand{\be}{\bm{e}}
\newcommand{\bme}{\bm{e}}
\newcommand{\bs}{\bm{s}}
\newcommand{\bz}{\bm{z}}
\newcommand{\bu}{\bm{u}}
\newcommand{\by}{\bm{y}}
\newcommand{\bx}{\bm{x}}
\newcommand{\bbb}{\bm{b}}
\newcommand{\bpi}{\bm{\pi}}

\renewcommand{\vec}{\bm}

\newcommand{\veto}{veto\xspace}
\newcommand{\Veto}{Veto\xspace}

\usepackage{amsthm}

\usepackage[capitalize,noabbrev]{cleveref}

\newtheorem{theorem}{Theorem}[section]
\newtheorem{thrm}[theorem]{Theorem}
\newtheorem{cor}[theorem]{Corollary}
\newtheorem{lemma}[theorem]{Lemma}
\newtheorem{prop}[theorem]{Proposition}
\newtheorem{defn}[theorem]{Definition}
\newtheorem{remark}[theorem]{Remark}
\newtheorem{example}[theorem]{Example}


\newcommand{\one}{\mathbf {1}}

\newcommand{\TT}{\mathbb T}
\newcommand{\cost}{\mathsf{cost}}
\newcommand{\pom}{\mathsf{PoM}}

\usepackage{color-edits}

\addauthor{mp}{red}
\addauthor{aj}{blue}
\usepackage{wrapfig}

\crefname{prop}{Proposition}{Propositions}



% Title. Note the optional short title for running heads. In the interest of anonymization, please do not include any acknowledgements.
\title[Running Title]{Opinion Dynamics with Multiple Adversaries}

% Anonymized submission.
% \author{Submission 2447}

\author{Akhil Jalan}
\email{akhiljalan@utexas.edu}
\orcid{0000-0003-0148-082X}
\affiliation{%
  \institution{UT Austin}
  \city{Austin}
  \country{USA}
  \state{TX}
  \postcode{78712}
}


\author{Marios Papachristou}
\email{papachristoumarios@cs.cornell.edu}
\orcid{0000-0002-1728-0729}
\affiliation{%
  \institution{Cornell University}
  \streetaddress{107 Hoy Rd}
  \city{Ithaca}
  \country{USA}
  \state{New York}
  \postcode{14853}
}


% Abstract. Note that this must come before \maketitle.
\begin{abstract}
\begin{abstract}  
Test time scaling is currently one of the most active research areas that shows promise after training time scaling has reached its limits.
Deep-thinking (DT) models are a class of recurrent models that can perform easy-to-hard generalization by assigning more compute to harder test samples.
However, due to their inability to determine the complexity of a test sample, DT models have to use a large amount of computation for both easy and hard test samples.
Excessive test time computation is wasteful and can cause the ``overthinking'' problem where more test time computation leads to worse results.
In this paper, we introduce a test time training method for determining the optimal amount of computation needed for each sample during test time.
We also propose Conv-LiGRU, a novel recurrent architecture for efficient and robust visual reasoning. 
Extensive experiments demonstrate that Conv-LiGRU is more stable than DT, effectively mitigates the ``overthinking'' phenomenon, and achieves superior accuracy.
\end{abstract}  
\end{abstract}


\begin{document}

% Title page for title and abstract only.
\begin{titlepage}

\maketitle

\newpage

\section{Introduction}


\begin{figure}[t]
\centering
\includegraphics[width=0.6\columnwidth]{figures/evaluation_desiderata_V5.pdf}
\vspace{-0.5cm}
\caption{\systemName is a platform for conducting realistic evaluations of code LLMs, collecting human preferences of coding models with real users, real tasks, and in realistic environments, aimed at addressing the limitations of existing evaluations.
}
\label{fig:motivation}
\end{figure}

\begin{figure*}[t]
\centering
\includegraphics[width=\textwidth]{figures/system_design_v2.png}
\caption{We introduce \systemName, a VSCode extension to collect human preferences of code directly in a developer's IDE. \systemName enables developers to use code completions from various models. The system comprises a) the interface in the user's IDE which presents paired completions to users (left), b) a sampling strategy that picks model pairs to reduce latency (right, top), and c) a prompting scheme that allows diverse LLMs to perform code completions with high fidelity.
Users can select between the top completion (green box) using \texttt{tab} or the bottom completion (blue box) using \texttt{shift+tab}.}
\label{fig:overview}
\end{figure*}

As model capabilities improve, large language models (LLMs) are increasingly integrated into user environments and workflows.
For example, software developers code with AI in integrated developer environments (IDEs)~\citep{peng2023impact}, doctors rely on notes generated through ambient listening~\citep{oberst2024science}, and lawyers consider case evidence identified by electronic discovery systems~\citep{yang2024beyond}.
Increasing deployment of models in productivity tools demands evaluation that more closely reflects real-world circumstances~\citep{hutchinson2022evaluation, saxon2024benchmarks, kapoor2024ai}.
While newer benchmarks and live platforms incorporate human feedback to capture real-world usage, they almost exclusively focus on evaluating LLMs in chat conversations~\citep{zheng2023judging,dubois2023alpacafarm,chiang2024chatbot, kirk2024the}.
Model evaluation must move beyond chat-based interactions and into specialized user environments.



 

In this work, we focus on evaluating LLM-based coding assistants. 
Despite the popularity of these tools---millions of developers use Github Copilot~\citep{Copilot}---existing
evaluations of the coding capabilities of new models exhibit multiple limitations (Figure~\ref{fig:motivation}, bottom).
Traditional ML benchmarks evaluate LLM capabilities by measuring how well a model can complete static, interview-style coding tasks~\citep{chen2021evaluating,austin2021program,jain2024livecodebench, white2024livebench} and lack \emph{real users}. 
User studies recruit real users to evaluate the effectiveness of LLMs as coding assistants, but are often limited to simple programming tasks as opposed to \emph{real tasks}~\citep{vaithilingam2022expectation,ross2023programmer, mozannar2024realhumaneval}.
Recent efforts to collect human feedback such as Chatbot Arena~\citep{chiang2024chatbot} are still removed from a \emph{realistic environment}, resulting in users and data that deviate from typical software development processes.
We introduce \systemName to address these limitations (Figure~\ref{fig:motivation}, top), and we describe our three main contributions below.


\textbf{We deploy \systemName in-the-wild to collect human preferences on code.} 
\systemName is a Visual Studio Code extension, collecting preferences directly in a developer's IDE within their actual workflow (Figure~\ref{fig:overview}).
\systemName provides developers with code completions, akin to the type of support provided by Github Copilot~\citep{Copilot}. 
Over the past 3 months, \systemName has served over~\completions suggestions from 10 state-of-the-art LLMs, 
gathering \sampleCount~votes from \userCount~users.
To collect user preferences,
\systemName presents a novel interface that shows users paired code completions from two different LLMs, which are determined based on a sampling strategy that aims to 
mitigate latency while preserving coverage across model comparisons.
Additionally, we devise a prompting scheme that allows a diverse set of models to perform code completions with high fidelity.
See Section~\ref{sec:system} and Section~\ref{sec:deployment} for details about system design and deployment respectively.



\textbf{We construct a leaderboard of user preferences and find notable differences from existing static benchmarks and human preference leaderboards.}
In general, we observe that smaller models seem to overperform in static benchmarks compared to our leaderboard, while performance among larger models is mixed (Section~\ref{sec:leaderboard_calculation}).
We attribute these differences to the fact that \systemName is exposed to users and tasks that differ drastically from code evaluations in the past. 
Our data spans 103 programming languages and 24 natural languages as well as a variety of real-world applications and code structures, while static benchmarks tend to focus on a specific programming and natural language and task (e.g. coding competition problems).
Additionally, while all of \systemName interactions contain code contexts and the majority involve infilling tasks, a much smaller fraction of Chatbot Arena's coding tasks contain code context, with infilling tasks appearing even more rarely. 
We analyze our data in depth in Section~\ref{subsec:comparison}.



\textbf{We derive new insights into user preferences of code by analyzing \systemName's diverse and distinct data distribution.}
We compare user preferences across different stratifications of input data (e.g., common versus rare languages) and observe which affect observed preferences most (Section~\ref{sec:analysis}).
For example, while user preferences stay relatively consistent across various programming languages, they differ drastically between different task categories (e.g. frontend/backend versus algorithm design).
We also observe variations in user preference due to different features related to code structure 
(e.g., context length and completion patterns).
We open-source \systemName and release a curated subset of code contexts.
Altogether, our results highlight the necessity of model evaluation in realistic and domain-specific settings.






% !TEX root = ./main.tex

% \section{Introduction}

% Over the past decade, social media has experienced rapid growth in both usage and significance. Online social networks, which allow users to share updates about their lives and opinions with a broad audience instantaneously, are now utilized by billions of people globally. These platforms serve various purposes, such as being informed about politics, news, health-related updates, products, and many more \citep{backstrom2012four,young2006diffusion,banerjee2013diffusion,shearer2021news}. 

% Unfortunately, networks can induce polarization as the network connections serve as a pathway for polarization to increase \cite{musco2018minimizing,chen2021adversarial,wang2024relationship,gaitonde2020adversarial}. This is a well-studied sociological phenomenon called the \textit{filter-bubble theory} \citep{pariser2011filter}. The filter-bubble theory argues that personalized algorithms used by online platforms, such as search engines and social media, selectively display content that aligns with a user's past behaviors, preferences, and beliefs. This customization creates an \textit{``invisible algorithmic editing''} of the web, isolating individuals within their own ideological bubbles where they encounter only information that reinforces their existing views. As a result, users are less likely to be exposed to diverse perspectives, potentially narrowing their worldview and fostering polarization. \citet{pariser2011filter} warns that such bubbles undermine democratic discourse by limiting opportunities for individuals to engage with challenging or unfamiliar ideas.
% \ajcomment{We might want to de-emphasize polarization if we go with a different framing around economics and market competition. For now I like this though.}

% Additionally, social networks can be manipulated by external entities in order to create discord and cause disagreement. For instance,  the 2017 indictment of the Russian Internet Research Agency (IRA) by the U.S. Department of Justice Special Counsel’s Office alleged that the IRA leveraged social media and targeted advertising to achieve \textit{``a strategic goal to sow discord in the U.S. political system, including the 2016 U.S. presidential election''} \citep{mueller2018united}. In 2019, \citet{twitter2019hongkong} disclosed that coordinated external bots attempted to induce discord in Hong Kong to hinder protesters’ ability to organize effectively during the independence movement. As social media continues to proliferate, it is likely that these types of external interferences will become increasingly common. Additionally, networks of Facebook pages have targeted Americans with sports betting scams, amplifying their reach by disseminating provocative conspiracy theories about political figures and natural disasters \citep{wired2024profiteers}. These schemes leverage the economics of the internet, where engagement with inflammatory content is monetized, and social media algorithms inadvertently amplify such content, enabling bad actors to exploit audiences for profit.

% To model the opinions' evolution, computer scientists, sociologists, and statisticians have relied on the framework of \emph{opinion dynamics} where the users' opinions coevolve according to a weighted network $G = (V, E, w)$, and each user updates their opinion as a combination of their own intrinsic opinion as well as the opinions of their neighbors \citep{Friedkin1990}. This model of opinion exchange has the advantage of taking into account both network interactions and their own intrinsic opinion. So far, all of the existing works consider a single actor who has the ability to act on the network to induce disagreement or polarization \cite{musco2018minimizing,chen2021adversarial,wang2024relationship,tsourakakis-2024,gaitonde2020adversarial,racz2023towards,Chitra2020}.  

% In this work, we lift the assumption of requiring a single actor (such as the platform) to act as an adversary to induce polarization or disagreement and consider the case of several decentralized actors. It is known that empirically, a very small percentage ($25\%$) of the users in a network need to disagree to sway consensus \citep{centola2018experimental}.  In this paper, we attempt to provide a theoretical basis for this phenomenon: Specifically, in our setting, we assume that there is a set $S \subseteq V$ of strategic agents whose goal is to report false intrinsic opinions ($s'$) that are different from their true intrinsic opinions ($s \neq s'$)  such that they influence others while not deviating much from their neighbors, namely they want to reach an equilibrium where their neighbors agree with them. For instance, assume a social network where a set of $S$ of political actors want the network to believe that their stance on a topic (e.g., abortion, elections, drug legalization, etc.) is the best. They achieve this by adversarially reporting different intrinsic opinions. This ensures that their influence is both persuasive and credible within the local network context. Such adversarial behavior can result in significantly different (cf. \cref{fig:psne_data}) and highly polarized equilibria, where the strategic agents' opinions appear dominant despite not reflecting the actual intrinsic views of the majority (cf. \cref{fig:ratios_susceptibility_to_persuation}). 

% Our work investigates the conditions under which these strategic manipulations are successful, the extent of their impact on network-wide opinion dynamics, and potential interventions to mitigate their influence. This corresponds to learning the set of strategic actors. 

% \subsection{Our Contribution}

% In this paper, we ask the following research question (RQ): 

% \begin{quote}
%     \textit{\textbf{(RQ)} What if a set of strategic actors with \textbf{possibly conflicting goals} tries to manipulate the consensus by strategically reporting beliefs different than their true beliefs?}
% \end{quote}
 

% We rely on the Friedkin-Johnsen (FJ) model \citep{Friedkin1990}, where the opinions of agents coevolve via the help of a weighted undirected network $G = (V = [n], E, w)$ with non-negative weights. According to the FJ model, the agents possess intrinsic opinions $s$ and express opinions $z$ which they update via the following rule for each agent $i$:  

% \begin{equation}
%     \bz_i(t + 1) = \frac {\alpha_i \bs_i + (1 - \alpha_i) \sum_{i \sim j} w_{ij} \bz_j(t)} {1 + \sum_{i \sim j} w_{ij}}.
% \end{equation}

% where $\alpha_i \in (0, 1)$ is $i$'s susceptibility to persuation \citep{abebe2018opinion}. We additionally define $\Tilde \alpha_i = \alpha_i / (1 - \alpha_i)$ to be the regularization parameter corresponding to $i$. This update rule corresponds to the best-response dynamics arising from minimizing the quadratic cost function for each $i$ \citep{bindel2011,abebe2018opinion}:

% \begin{equation} \label{eq:cost_fcn}
%     c_i(\bz_i, \bz_{-i}) = (1 - \alpha_i) \sum_{i \sim j} w_{ij} (\bz_i - \bz_j)^2 + \alpha_i (\bz_i - \bs_i)^2
% \end{equation}

% The PSNE can be written as $z = ((I - A)L + A)^{-1} A \bs = B \bs$ where $L$ is the Laplacian of graph $G$, $A = \mathrm{diag}(\alpha_1, \dots, \alpha_n)$ is the diagonal matrix of susceptibilities. When an external \textit{single actor aims to induce disagreement or polarization} -- see, e.g., \citet{gaitonde2020adversarial,racz2022towards,musco2018minimizing} -- the adversary is coined with optimizing the objective function 

% \begin{equation}
%     \sum_{i \in [n]} c_i(\bz_i, \bz_{-i}) = \bs^T ((I - A)L + A)^{-1} A f(L) ((I - A)L + A)^{-1} A \bs,
% \end{equation}

% where $f(L)$ is a function of the Laplacian of $G$, either with optimizing towards $\bs$ \citep{gaitonde2020adversarial}, or the graph itself \citep{musco2018minimizing,racz2022towards}. 

% Usually, as we also discussed earlier, many adverse actions on social networks come from \textit{several independent strategic adversaries} who try to manipulate the network by infiltrating intrinsic opinions $\bs_i^\prime$, which are different from their true stances $\bs_i$ but are simultaneously close to $\bs_i$. Unlike previous works, these ``adversaries'' can have conflicting goals. 

% Concretely, the true opinions of the agents are $\bs_1, \dots, \bs_n$, and there is a set $S$ of deviating agents who report $\{ \bs_i^\prime \}_{i \in S}$, the goal of the strategic agents is to minimize the cost function of \cref{eq:cost_fcn} at consensus $z' = ((I - A)L + A)^{-1} A \bs^\prime$ where $\bs^\prime$ is the vector which has entries $\bs_i$ for all $i \notin S$ and $\bs_i^\prime$ for all $i \in S$. The local optimization of agent $i$ becomes:

% \begin{equation} \label{eq:strategic_cost}
%     \min_{\bs_i^\prime \in \RR} c_i \left (z' = ((I - A)L + A)^{-1} A \bs^\prime \right )
% \end{equation}

% Our contributions consider the following: 

% \begin{itemize}
%     \item We characterize the Nash Equilibrium of the game defined by \cref{eq:strategic_cost}. Specifically, we show that the game has a Pure Strategy Nash Equilibrium (PSNE) that is given by solving a constrained linear system. Given the PSNE of the game, we characterize the actors who can have the most influence in strategically manipulating the network. In the case of one deviator ($|S| = 1$) we show that the optimal strategic opinion $\bs_i^\prime$ is a linear function of the true opinion $\bs_i$ and the effect size depends on the diagonal entries of the equilibrium matrix $B$, and the bias depends on the equilibrium matrix, the vector of internal opinions $\bs$ the link weights of edges adjacent to $i$, and the susceptibility parameters of the agents.  
%     \item We apply our framework to real-world social network data from Twitter and Reddit \citep{Chitra2020}, and data from the Political Blogs (polblogs) dataset \citep{adamic2005political}. We find that the influence of strategic agents can be rather significant as strategic agents can significantly increase polarization and disagreement, as well as increase the overall ``cost'' of the consensus.  To address this, we give worst-case upper bounds on the \textit{Price of Misreporting} (PoM), which is analogous to the well-studied Price of Anarchy bounds (see, for example, \citet{bhawalkar2013,roughgarden2011local}) and suggest ways that the platform can be used to mitigate the effect of strategic behavior on their network. 
%     \item We give an efficient learning algorithm for the platform to infer the set of strategic agents $S$ from observing their publicly reported opinions $\bz^\prime$ as long as the size of $S$ is sufficiently small. Our algorithm is inspired by the robust regression algorithm of \citet{torrent-2015}, and is practical for real-world networks: \textit{(i)} requires the platform to have access to node embeddings $X$ which have been shown computable even in billion-scale networks such as Twitter \citep{el2022twhin}, and \textit{(ii)} can be computed in polynomial time in the number of nodes $n$ and edges $m$ of the network.  We apply our algorithm to the real-world datasets from Twitter, Reddit, and Polblogs and show that our algorithm can recover the strategic set $S$ each time with high accuracy. Also, we give sufficient conditions for the maximum size of $S$ when the embeddings of the nodes correspond to a stochastic block model with $K$ communities, and show that as long as the smallest community is of size $\Theta (n/K)$ recovery of $S$ is possible as long as $|S| = O(n/K)$. 
%     \item \mpcomment{In the sequel, we lift the need for node embeddings and provide a general algorithm that requires only graph information, and the publicly reported opinions $\bz^\prime$ can recover $|S|$ as long as $|S| \le blah$. Contrary to the previously mentioned robust regression algorithm, this algorithm is based on performing a goodness-of-fit test and assumes prior information on the vector of intrinsic opinions $\bs$.}=
% \end{itemize}


% \subsection{Preliminaries and Notations}

% The set $[n]$ denotes $\{ 1, \dots, n \}$. $\| \bx \|_p$ denotes the $\ell_p$-norm of vector $\bx$. $X \succeq 0$ denotes that the matrix $X$ is positive semi-definite and $\| X \|$ corresponds to the norm of $X \succeq 0$ which is its largest singular value. The Laplacian of the graph $G$ is denoted by $L = D - W$ where $W$ is the weight matrix of the graph, which has entries $w_{ij} \ge 0$, and $D$ is the diagonal degree matrix with diagonal entries $D_{ii} = \sum_{i \sim j} w_{ij}$. The Laplacian has eigenvalues $0 = \lambda_1 \le \lambda_2 \le \dots \le \lambda_n$. For any undirected and connected graph $G$, we can write the eigendecomposition of $L$ as 

% \begin{equation}
%     L = \sum_{i \in [n]} \lambda_i \bu_i \bu_i^T \succeq 0,
% \end{equation}

% where $\bu_1, \dots, \bu_n$ are orthonormal eigenvectors, with $\bu_1 = (1 / \sqrt n) \one$, where $\one$ is the column vector of all 1s. $U$ denotes the matrix which has the eigenvectors of $L$ as columns; i.e., such that $L = U^T \Lambda U$ where $\Lambda = \mathrm{diag} (\lambda_1, \dots, \lambda_n)$ is the diagonal matrix of $L$'s eigenvalues. $L_i$ denotes the $i$-restricted Laplacian which corresponds to the Laplacian of the graph with all edges that are non-adjacent to $i$ being removed, and, similarly, $L_{\{ u, v \}}$ corresponds to the Laplacian of an edge $\{ u, v\}$. Note that $L_i = \sum_{i \sim j} L_{\{ i, j \}}$. For a function $f(L)$ of the Laplacian we write $f(L) = U^T f(\Lambda) U$ where $f(\Lambda) = \mathrm{diag} (f(\lambda_1), \dots, f(\lambda_n))$. For brevity, regarding the equilibrium $z$ of the FJ model, we write $B =((I - A)L + A)^{-1} A$, such that $\bz = B \bs$ and $\bz^\prime = B \bs^\prime$. $\be_i$ denotes the $i$-th basis column vector. We define the total cost of an equilibrium $z$ to be 

% \begin{equation}
%     C(\bz) = \sum_{i \in [n]} c_i(\bz).
% \end{equation}

% We define the platform-wide metrics to be 

% \begin{align}
%     \mathcal P(\bz) & = \sum_{i \in [n]} (\bz_i - \bar z)^2, \qquad \bar z = \frac 1 n \sum_{i \in [n]} \bz_i, \\
%     \mathcal D(\bz) & = \sum_{i, j \in [n]} w_{ij} (\bz_i - \bz_j)^2 = \bz^T L \bz, 
% \end{align}

% where we assume that the platform does not have access to the susceptibility parameters $\alpha_i$ (i.e., the $\alpha_i$s are endogenous and private to each agent). Finally, we define the ``Price of Misreporting'' (PoM) to be the ratio between the cost $C(\bz^\prime)$ when the agents are deviating, and the cost $C(\bz)$ when the agents are reporting truthfully, i.e.,   

% \begin{equation}
%     \pom = \frac {C(\bz^\prime)} {C(\bz)}.
% \end{equation}

% Note that always $\pom \ge 1$. 

% \subsection{Related Work}

% \paragraph{Opinion Dynamics} Opinion dynamics are well-studied in computer science and economics, as well as sociology, political science, and related fields. There have been many models proposed for opinion dynamics, such as with network interactions as we study in this paper (FJ model) \citep{Friedkin1990,Bindel2015}, bounded confidence dynamics (Hegselman-Krausse Model) \citep{hegselmann2002opinion}, coevolutionary dynamics \citep{bhawalkar2013} as well as many variants of them; see, for example \citep{abebe2018opinion,hazla2019geometric,fotakis2016opinion,fotakis2023opinion,tsourakakis-2024}. The work of \citep{bindel2011} shows bounds on the Price of Anarchy (PoA) between the PSNE and the welfare-optimal solution for the FJ model, and the subsequent work of \cite{bhawalkar2013} shows PoA bounds for the coevolutionary dynamics. Additionally, the opinion dynamics have been modeled by the control community; see, for example, \citet{nedic2012,de2022,bhattacharyya2013convergence,chazelle2011total}. 

% Similarly to the existing models, our work assumes the FJ model as a basis. However, our work is significantly different as it studies a framework where a subset of strategic agents can deviate where the agents can basically choose their intrinsic opinions ($\bs^\prime$) and are given the equilibrium state of the system ($\bz^\prime = B \bs^\prime$), as opposed to studying the evolution of the expressed opinions and their PSNE. 

% \paragraph{Disagreement and Polarization in Social Networks} Motivated by real-world manipulation of social networks in, e.g., the 2016 US election, a recent line of work studies polarization and strategic behavior in opinion dynamics \cite{gaitonde2020adversarial,gaitonde2021polarization,Chen2022,tsourakakis-2024}. 
% Next, \cite{Chen2022} considers a model in which an adversary can control $k \leq n$ nodes' internal opinions. They want to arrive at an equilibrium that maximizes polarization. In contrast, we study a setting in which any subset $S \subseteq [n]$ can be strategic. Unlike previous works, these ``adversaries'' can have conflicting goals. 


% \paragraph{Network Games.} Network games, also known as graphical games, involve $n$ agents whose payoffs are influenced by the actions of their neighbors~\cite{littman2001efficient, roughgarden2004bounding,roughgarden2011local}. A substantial body of research examines learning from observations of such games~\cite{irfan2018causal,garg2016learning,rossi2022learning,leng2020learning}. These studies typically focus on games with finite or one-dimensional action spaces, whereas in our model, each of the $n$ agents operates in an $n$-dimensional action space.

% Furthermore, another area of research explores influencing the outcomes of network games. Most of these studies address games with one-dimensional action spaces and a single strategic actor~\cite{galeotti2020targeting,gaitonde2020adversarial,gaitonde2021polarization, wang2024relationship}. In contrast, we consider arbitrary sets of strategic actors. Recent works have also examined scenarios with multiple strategic actors, such as in repeated auctions~\cite{kolumbus2022auctions} and Fisher markets with linear utilities~\cite{kolumbus2023asynchronous}. While our work shares a similar motivation, it centers on opinion formation.

% \paragraph{Anomaly Detection in Graphs.} Our work is also related to the anomaly detection literature in networks. One work closely related to ours is the work of \citet{chen2022antibenford} where the authors want to detect Anti-Benford subgraphs in a large transaction or financial graph. The Anti-Benford subgraphs consist of a set of nodes that perform many transactions that significantly deviate from Benford's law, and the authors develop a hypothesis test to detect such graphs. Similarly, the work of \citet{agarwal2020chisel} proposes a framework based on a chi-squared statistic to perform a graph similarity search.  In our paper, similarly, we develop a robust regression and a hypothesis testing algorithm that is able to detect nodes that are strategic and misreport their opinions. Additionally, our work assumes the strategic behavior of the agents, whereas the works of \citep{agarwal2020chisel,chen2022antibenford} do not. 

% In a similar spirit, the work of \citet{jalan-chakrabarti-2024} studies financial networks, whereas a subset of strategic agents has incentives to misreport their own local network connections in order to obtain higher utility, and develop algorithms that can identify the set of such agents. Our paper works in a similar flavor, however, in a significantly different application domain and context, which corresponds to social networks.

% % \mpcomment{I added the refs, but it may need more writing to make sure it looks different}

% % \ajcomment{Add refs from old paper. Emphasize Nisan-Kolumbus paper on how to manipulate your learning algorithm.}

% \subsection{Real-world Datasets}

% To support our results, we use data grounded in practice, which has also been used in previous studies to study polarization and disagreement (cf. \citet{Chitra2020,wang2024relationship,adamic2005political}). Specifically, we use Twitter, Reddit, and Political blog networks. Twitter has $n = 548$ nodes and $m = 3638$ unweighted edges. Edges correspond to user interactions regarding the debate over the Delhi legislative assembly elections of 2013. The Reddit dataset has $n = 556$ nodes and $m = 8969$ edges. Nodes are users who posted in the \texttt{r/politics} subreddit, and there is an edge between two users if two subreddits (other than \texttt{r/politics}) exist that both users posted during the given time period. The vectors $\bs$ of initial opinions are obtained via sentiment analysis and also follow the post-processing of \citet{wang2024relationship}. The Political Blogs -- or Polblogs, for brevity --  dataset is due to \citet{adamic2005political} and contains political blogs (liberal and conservative) and links between blogs were automatically extracted from a crawl of the front page of the blog. Each blog is either liberal -- where we assign a value $\bs_i = -1$ -- or conservative -- where we assign $\bs_i = +1$. The Polblogs dataset has $n = 1490$ and $m = 16178$. 

\section{Strategic Opinion Formation}\label{sec:strategic_opinion_formation}

% \mpcomment{it needs more text here to look smoother}

% \ajcomment{Make it a list, but with nice text to flow.}

% \ajcomment{Here's an idea since we have space: we show the networks, colored by opinions, for three scenarios. First the actual equilibrum, second the equilibrim with one strategic actor, and third with multiple strategic actors. Show like a spring-mass embedding visualization. This helps visualize the differences in our model.}

The opinion formation game has two phases. First, strategic agents privately choose a strategic intrinsic opinion according to \cref{eq:strategic_cost}. Second, agents exchange opinions and reach consensus {\em as if} they were in the Friedkin-Johnson dynamics, except the strategic opinions are used in place of the true intrinsic opinions. 
\begin{enumerate}
    \item {\em Strategy Phase.} Each strategic agent $i \in S$ independently and privately chooses a fictitious strategic opinion $\bs_i^\prime \in \RR$. For honest agents ($i \notin S$) we have $\bs_i^\prime = \bs_i$.
    \item {\em Opinion Formation Phase.} Reach equilibrium $\bz^\prime = B \bs^\prime$ as if $\bs^\prime$ were the true intrinsic opinions~$\bs$. 
\end{enumerate}

% In the first phase -- called the ``strategy phase'' --  each strategic agent $i \in S$ independently and privately chooses a negotiating position $\bs_i^\prime \in \RR$. For honest agents ($i \notin S$) we have $\bs_i^\prime = \bs_i$. The strategic agents select $\bs_i^\prime$ according to \cref{eq:strategic_cost}. In the second phase -- called the ``opinion formation phase'' -- the agents exchange opinions and reach consensus $\bz^\prime = B \bs^\prime$ as if every agent's intrinsic opinion was their true opinion $\bs_i$. 
The network $G$ and the true beliefs $\bs$ determine each agent’s utility. We pose the following problem: 

\begin{defn}[Instrinsic belief lying problem.] \label{defn:lying}
Let $S \subseteq [n]$ be a set of strategic agents. If agent $i \in S$ wants network members to express opinions close to $\bs_i$, what choice of $\bs_i^\prime$ is optimal and minimizes the cost function of \cref{eq:strategic_cost}? 
\end{defn}

The following theorem characterizes the Nash Equilibria of the Intrinsic Belief Lying Problem. 
\begin{theorem}[Nash Equilibrium] \label{theorem:psne}
Let $\TT_i = (1 - \alpha_i) (B^T L_i B) + \alpha_i (B^T \be_i \be_i^T B) \in \RR^{n \times n}$ and $\by_i = \alpha_i B_{ii} \bs_i$. The Nash equilibria, if any exist, are given by solutions $s^\prime \in \RR^n$ to the following constrained linear system: 
\begin{align*}
\forall i \in S: \be_i^T \TT_i \bs^\prime & = \bm{y}_i, \\
\forall j \not \in S: \bs_j^\prime & = \bs_j. 
\end{align*}
\end{theorem}

\begin{proof}
Consider agent $i \in S$. To calculate the best-response $\bs_i^\prime$ of $i$ in response to $\bs_{-i}^{\prime}$, we analyze derivatives of its cost function with respect to $\bs^\prime$. Since the equilibrium $\bz^\prime$ is $\bz^\prime = B \bs^\prime$, we have: 
\begin{align*}
c_i(\bz^\prime) &= (1 - \alpha_i) \sum_{j \sim i}
w_{ij} (\bz_i^\prime - \bz_j^\prime)^2 + \alpha_i (\bz_i^\prime - \bs_i)^2 \\
c_i(\bs^\prime) &= (1 - \alpha_i) \sum_{j \sim i}
w_{ij} ((\be_i - \be_j)^T B \bs^\prime)^2 
+ \alpha_i (\be_i^T (B \bs^\prime - \bs))^2 \\
&=  (1 - \alpha_i) \sum_{j \sim i}
w_{ij} (\bs^\prime)^T (B^T (\be_i - \be_j) (\be_i - \be_j)^T B) (\bs^\prime) \\
&+ \alpha_i ((\bs^\prime)^T B^T e_i e_i^T B \bs^\prime 
- 2 (\bs^\prime)^T B^T e_i e_i^T \bs 
+ \bs^T e_i e_i^T \bs) \\
\nabla_{\bs^\prime} c_i(\bs^\prime) 
&= (1 - \alpha_i) \sum_{j \sim i}
w_{ij} 2 (B^T (\be_i - \be_j) (\be_i - \be_j)^T B) (\bs^\prime) 
+ \alpha_i (2 B^T \be_i \be_i^T B \bs^\prime 
- 2 B^T \be_i \be_i^T \bs), \\
\nabla_{\bs^\prime}^2 c_i(\bs^\prime) &= 2 (1 - \alpha_i) B^T \bigg[\sum_{j \sim i}
w_{ij} 2 (\be_i - \be_j) (\be_i - \be_j)^T \bigg] B
+2  \alpha_i B^T \be_i \be_i^T B.
 \end{align*}
 Let $L_i \in \RR^{n \times n}$ be: 
 \begin{align*}
L_i := \sum_{j \sim i} w_{ij} (\be_i - \be_j) (\be_i - \be_j)^T  .
\end{align*}
Notice that $L_i$ is precisely the Laplacian of the graph when all edges not incident to $i$ are equal to zero. Therefore $L_i \succeq 0$. Since $\be_i \be_i^T \succeq 0$, the Hessian of $c_i$ with respect to $\bs^\prime$ is PSD. In particular, its $(i,i)$ entry is non-negative, so $\frac{\del^2 c_i(\bs^\prime)}{\del (\bs_i^\prime)^2} \geq 0$, and hence the optimal $\bs_i^\prime$ is at the critical point. This is given as: 
\begin{align*}
0 &= \frac{1}{2} \frac{\del}{\del \bs_i^\prime} c_i(\bs^\prime) \\
&= \be_i^T (1 - \alpha_i) B^T \bigg[\sum_{j \sim i}
w_{ij} (\be_i - \be_j) (\be_i - \be_j)^T \bigg] B \bs^\prime 
+ \be_i^T \alpha_i (B^T \be_i \be_i^T B s^\prime - B^T \be_i \be_i^T \bs) \\
&= (1 - \alpha_i) \be_i^T B^T L_i B \bs^\prime 
+ \be_i^T \alpha_i (B^T \be_i \be_i^T B \bs^\prime - B^T \be_i \be_i^T \bs).
\end{align*}
The above display gives the solution for $\bs_i^\prime$ in terms of all entries of $s^\prime$. Assembling the critical points into a linear system, we obtain precisely that for all $i \in S$, $\be_i^T \TT_i \bs^\prime = \by_i$. Since $\bs_j^\prime = \bs_j$ for $j \not \in S$, the overall linear system describes the Nash equilibria. 
\end{proof}

To illustrate the Theorem, we consider a toy example. 
\begin{example}[Two-Node Graph]
    
% \ajcomment{I would make this an Example environment}
% \paragraph{Toy Example.} 
Consider a graph with 2 nodes and one edge with weight $w > 0$. We set $\alpha_1 = \alpha_2 = 0.5$ for simplicity. Suppose that both agents deviate, i.e., $S = [2]$. Then, we can calculate $B$ to be 

\begin{align*}
    B = \frac {1} {2w + 1} \begin{pmatrix} w + 1 & w \\ w & w + 1 \end{pmatrix}
\end{align*}

and 

\begin{align}
    \bz_0' = \frac {(w + 1) \bs_0' + w \bs_1'} {2w + 1}, \quad \bz_1' = \frac {w \bs_0' + (w + 1) \bs_1'} {2w + 1},
\end{align}

yielding the two cost functions

\begin{align*}
    c_0(\bs^\prime) & = \frac 1 2 w \left ( \frac {\bs_0' - \bs_1'} {2w + 1} \right )^2 + \frac 1 2 \left ( \frac {(w + 1) \bs_0' + w \bs_1'} {2w + 1} - \bs_0 \right )^2 \\
    c_1(\bs^\prime) & = \frac 1 2 w \left ( \frac {\bs_0' - \bs_1'} {2w + 1} \right )^2 + \frac 1 2 \left ( \frac {(w + 1) \bs_1' + w \bs_0'} {2w + 1} - \bs_1 \right )^2.
\end{align*}

Taking the first order conditions $\frac {\partial c_0}{\partial \bs_0'} = 0$ and $\frac {\partial c_1} {\partial \bs_1'} = 0$ we get a linear system whose solutions are:  

\begin{align*}
\bs_0^\prime & = \frac{w^2(s_0 - s_1) + (3w + 1) \bs_0}{3w + 1}, \quad \bs_1^\prime  = \frac{w^2(\bs_0 - \bs_1) + (3w + 1) s_1}{3w + 1}.
\end{align*}

Replacing these values back to the costs we get that 

\begin{align*}
\forall i: c_i(s_0^\prime, s_1^\prime) = \frac 1 2 \frac{w(w^2 + 3w - 1)(\bs_0 - \bs_1)^2}{9w^2 + 6w + 1},
\end{align*}

On the other hand, if all agents are honest, then the cost for each is: 

\begin{align*}
\forall i: c_i(\bs_0, \bs_1) = \frac 1 2 \frac{w(w+1)(\bs_0 - \bs_1)^2}{(2w + 1)^2}.    
\end{align*}

and the ratio of the two costs is at least $\max \{ 1, w / 3 \}$. 
\end{example}


Next, we discuss some consequences of Theorem~\ref{theorem:psne}. First, we characterize $\bs^\prime$ as the solution to a linear system. 
\begin{cor} \label{cor:psne}
Let $T \in \RR^{|S| \times n}$ have rows $\{ \be_i^T \TT_i \}_{i \in S}$ given by \cref{theorem:psne}. Let $\Tilde T \in \RR^{|S| \times |S|}$ be the submatrix of $T$ selecting columns belonging to $S$. Let $\by \in \RR^{|S|}$ have entries $\by_i = \alpha_i B_{ii} \bs_i$ as above. Let $\Tilde \by = \by - \sum_{j \not \in S} \bs_j T \be_j$. Then the set of Nash equilibria, if any exist, are given by the solutions to the unconstrained linear system 

\begin{equation} \label{eq:psne}
    \Tilde T \bx = \Tilde \by.
\end{equation}
The resulting opinions vector $\bs^\prime$ is given by $\bs_i^\prime = \bx_i$ if $i \in S$ and $\bs_i^\prime = \bs_i$ otherwise.
\end{cor}




Thus, in a Nash equilibrium, every strategic agent solves their corresponding equation given by \cref{eq:psne}. The explicit characterization of equilibria also implies that Nash equilibria cannot be mixed. 
\begin{cor}[Pure Strategy Nash Equilibria]
    The Nash equilibrium corresponds to solving the system of $|S|$ linear equations in the scalars $\{ \bs_i^\prime | i \in S \}$ given by \cref{eq:psne}. Also, all Nash equilibria are pure-strategy Nash equilibria.
\end{cor}


\begin{figure}[t]
    \centering
    % \includegraphics[width=0.3\linewidth]{figures/PoM_example.pdf} \\
    \includegraphics[width=0.49\linewidth]{figures/Reddit_alpha_0.25.pdf} 
    \includegraphics[width=0.49\linewidth]{figures/Reddit_alpha_0.5.pdf} 

    \includegraphics[width=0.49\linewidth]{figures/Twitter_alpha_0.25.pdf}
    \includegraphics[width=0.49\linewidth]{figures/Twitter_alpha_0.5.pdf}

    \includegraphics[width=0.49\linewidth]{figures/Polblogs_alpha_0.25.pdf} 
    \includegraphics[width=0.49\linewidth]{figures/Polblogs_alpha_0.5.pdf} 

    \caption{Plot of truthful intrinsic opinions ($s$) and strategic opinions ($s'$), and truthful public opinions ($z$) compared to the strategic public opinions ($z'$) for the nodes belonging to $S$. $S$ is taken to be the top-50\% in terms of their eigenvector centrality. In both cases we have taken $\alpha_i \in \{ 0.25, 0.5 \}$ for all nodes. We fit a linear regression between $s'$ and $s$ (resp. between $z$ and $z'$). We report the effect size $\theta$ which corresponds to the slope of the linear regression and the $P$-value with respect to the null hypothesis ($\theta = 0$). $^{***}$ stands for $P < 0.001$, $^{**}$ stands for $P < 0.01$ and $^{*}$ stands for $P < 0.05$.}
    \label{fig:psne_data}
\end{figure}

\begin{figure}[t]
    \centering
    \includegraphics[width=0.9\linewidth]{figures/polblogs.pdf}
    \caption{Strategic misreports for the Polblogs dataset where $S$ is taken to be the top-50\% of the agents in terms of their eigenvector centralities. The nodes are labeled either as liberal ($\bs_i = -1$) or conservative ($\bs_i = +1$), and we consider the nodes that change their beliefs as the nodes for which $\bz_i'$ and $\bz_i$ do not have the same sign. In the scatterplots (a), (c), (d), (e), the shape of each point indicates whether that user changed belief or not, and the color indicates their true (intrinsic) opinion. 
    Overall, we discover a higher amount of change among liberal blogs compared to conservative ones (panel (b)). Additionally, we report the truthful/strategic public opinion as a function of the logarithm of the eigenvector centrality $\bpi_i$ (cf. panels (c, d)) for each node, as well as the absolute change $|\bz_i' - \bz_i|$ (cf. panel (e)). We fit a regression model, and we detect significant effects ($^{***}: P < 0.001, \; ^{**}: P < 0.01, \; ^{*}: P < 0.05$; effects denoted by $\theta$) of the logarithm of the centrality to the truthful equilibrium $\bz$, the strategic equilibrium $\bz'$, and the change $|\bz' - \bz|$, revealing the structure of a power law. Finally, we observe that relative changes are more dispersed along liberal sources compared to conservative sources (cf. panel (f)).} 
    \label{fig:polblogs}
\end{figure}
\vspace{-1em}

\paragraph{Optimal Deviation for One Agent and All Agents} Assuming that we have one strategic agent, what is the change in their opinion? We can show that the new opinion is a scalar multiple of the initial opinion plus a bias term, where neither the scalar multiple nor the bias term can be zero. 

\begin{corollary}[Deviation for One Agent] \label{theorem:one_deviation}
    Let $S = \{ i \}$. Then, $\bs_i^\prime = \theta_i \bs_i + \beta_i$ where

    \begin{align*}
        \theta_i & = \frac {\alpha_i B_{ii}} {(1 - \alpha_i) \sum_{i \sim j} w_{ij} (B_{ii} - B_{ij})^2 + \alpha_i B_{ii}^2} > 0, \\
        \beta_i & = - \frac {\alpha_i \sum_{j \neq i} B_{ij} s_j} {(1 - \alpha_i) \sum_{i \sim j} w_{ij} (B_{ii} - B_{ij})^2 + \alpha_i B_{ii}^2}.
    \end{align*}

\end{corollary}
 
Similarly, we can relate the maximum deviation of $\bs^\prime$ from $\bs$ in the other extreme case, i.e., when all agents are deviating ($S = [n]$). %The proof is deferred to the Appendix.
\begin{corollary}
    \label{theorem:all_deviation}
    When all agents are deviating ($S = [n]$), and $\alpha_i = \alpha$, then $\bs^\prime$ satisfies: $$\frac {\norm \bs^\prime \norm_2} {\norm \bs \norm_2}  \le \frac {\lambda_n + \tilde \alpha} {\tilde \alpha}.$$
\end{corollary}

\begin{proof}[Proof of Corollary~\ref{theorem:all_deviation}]
    When all agents are deviating, it is straightforward to show that $\Tilde T = \tilde \alpha B$ with minimum eigenvalue $\tilde \alpha^2 / (\lambda_n + \tilde \alpha) > 0$. Thus $\Tilde T$ is invertible, and therefore $\bs^\prime = \frac {1} {\tilde \alpha} B^{-1} \widetilde {\diag (B)} \bs$, where $\widetilde {\diag (B)}$ is a diagonal matrix with entries $B_{ii}$. Then 

    \begin{align*}
        \norm \bs^\prime \norm_2 & \le \frac {1} {\tilde \alpha} \norm B^{-1} \norm_2 \norm \widetilde{\diag (B)} \norm_2 \norm \bs \norm_2 \\ &  = \left ( \max_i B_{ii} \right ) \left ( \max_i \frac {\lambda_i + \tilde \alpha} {\tilde \alpha} \right ) \norm \bs \norm_2 \\ & \le \frac {\tilde \alpha} {\lambda_1 + \tilde \alpha} \frac {\lambda_n + \tilde \alpha} {\alpha} \norm \bs \norm_2 \\ & = \frac {\lambda_n + \tilde \alpha} {\tilde \alpha} \norm \bs \norm_2.
    \end{align*}
    
\end{proof}

The proof of Corollary~\ref{theorem:all_deviation} shows that the adjusted susceptibility ($\tilde \alpha$) and the maximum eigenvalue of the Laplacian ($\lambda_n$) are responsible for changes in the norm of $\bs^\prime$. From classic spectral graph theory, we know that $\lambda_n = \Theta(d_{\textup{max}})$ where $d_{\textup{max}}$ is the maximum degree of the graph; therefore, graphs with a lower maximum degree experience smaller distortions. Also, regarding the susceptibility to persuasion, the distortion becomes $1 + o(1)$ as long as $\tilde \alpha = \omega (d_{\textup{max}})$. 

\medskip

\noindent {\bf Equilibria for real-world datasets.} Next, we discuss the results of experiments simulating the strategically manipulated equilibria for our real-world datasets. 

\begin{figure}[t]
    \centering
    \includegraphics[width=0.9\linewidth]{figures/polarization_disagreement_ratios.pdf}
    \caption{Polarization ratio ($\mathcal P(z')/\mathcal P(z)$), disagreement ratio ($\mathcal D(z') / \mathcal D(z)$), and price of misreporting ($C(z') / C(z)$) for the three datasets for varying susceptibility to persuasion values. We have set all susceptibilities $\alpha_i$ to the same value $\alpha$. The Twitter dataset has the largest variation in all three ratios compared to the others. $S$ is taken to be the top-50\% nodes in terms of their eigenvector centrality.}
    \label{fig:ratios_susceptibility_to_persuation}
\end{figure}

% \ajcomment{Maybe change ordering of the figs?}


\paragraph{Effect of Susceptibility to Persuasion in Real-world Data} Regarding real-world data, \cref{fig:psne_data} shows the relationship between the truthful opinions ($\bs$ and $\bz$) and the strategic ones ($\bs^\prime$ and $\bz^\prime$) for the datasets, along with the corresponding correlation coefficient $R^2$, assuming that $S$ consists of the top-50\% nodes in terms of their eigenvector centrality, for susceptibility parameters set to $\alpha_i = 0.5$ (equal self-persuasion and persuasion due to others) and $\alpha_i = 0.25$ (higher persuasion due to others). 

Regarding the public opinions, even though in the Reddit dataset, the strategic opinions seem to be correlated with the truthful ones ($R^2 = 0.78$ for $\alpha_i = 0.25$ and $R^2 = 0.94$ for $\alpha_i = 0.5$ respectively), in the Twitter dataset, we do not get the same result (i.e., $R^2 < 0.25$). Finally, in the Polblogs dataset, the situation is somewhere in the middle; when $\alpha_i = 0.25$ we get a low $R^2$ ($R^2 = 0.18$) where for $\alpha_i = 0.5$ we get a high $R^2$ ($R^2 = 0.74$). Additionally, in all cases except Twitter, we get that the effect is significant ($P < 0.01$). 

Regarding the relationship between the intrinsic opinions, we do not detect any significant effect in most cases except Reddit with $\alpha_i = 0.5$ ($P < 0.01$) and Twitter with $\alpha_i = 0.5$ ($P < 0.05$).

\paragraph{Asymmetric Effects of Strategic Behavior on Liberals and Conservatives.} \cref{fig:polblogs} analyzes the opinions of the strategic set $S$ on the Polblogs dataset. Specifically, we find that larger changes in sentiment happen across liberal outlets compared to conservative ones. Additionally, the changes in the truthful/strategic opinions are related to the eigenvector centrality $\bpi_i$ as a power law, i.e., $\bz_i' \propto \bpi_i^{\theta}$ ($P < 0.001$; linear regression between the log centralities $\log \bpi_i$ and $\bz_i'$). The same finding holds for $|\bz_i' - \bz_i|$ and $\bz_i$. 

At this point, one may wonder whether the eigenvector centrality really influences the strategic opinions $\bz_i'$ for $i \in S$. Our answer is negative. We repeat the same experiment with the Twitter and Reddit datasets, where we find no effects ($P > 0.1$; linear regression between the log centralities $\log \bpi_i$ and $\bz_i'$). Due to space limitations, the corresponding figures are deferred to \cref{app:additional_figures}. 

\paragraph{Polarization and Disagreement.} \cref{fig:ratios_susceptibility_to_persuation} shows how the polarization, disagreement, and cost change as a function of the susceptibility parameter $\alpha_i$. Except for $\alpha_i \approx 0.3$, the polarization ratio, disagreement ratio, and the price of misreporting experience a downward trend as $\alpha_i$ increases. This indicates that as as users prioritize their own opinions more than their neighbors, they are less susceptible to strategic manipulation.
% \ajcomment{This indicates that...}

\paragraph{Effect of the number of deviators ($|S|$)} Next, we study the effect of the number of deviators, which corresponds to $|S|$, on the changes in polarization, disagreement, and the total cost (through the price of misreporting). \cref{fig:ratios_number_of_deviators} shows how the polarization and disagreement when $S$ consists of the top-1-10\% most central agents with respect to eigenvector centrality. We show that even if only 1\% of agents are strategic, this can impact consensus by several orders of magnitude. 

\begin{figure}
    \centering
    \includegraphics[width=0.9\linewidth]{figures/polarization_disagreement_ratios_percent_strategic.pdf}
    \caption{Polarization ratio ($\mathcal P(\bz^\prime)/\mathcal P(\bz)$), disagreement ratio ($\mathcal D(\bz^\prime) / \mathcal D(\bz)$), and price of misreporting ($C(\bz^\prime) / C(\bz)$) for the three datasets for varying the size of $|S|$. The size of $|S|$ corresponds to the top $p$ percent of the actors ($|S| = \lceil p n \rceil$) based on their eigenvector centrality (in decreasing order), for $p \in [0.01, 0.1]$. The susceptibility parameter is set to $\alpha_i = 0.5$.} 
    \label{fig:ratios_number_of_deviators}
\end{figure}







% \section{Price of Misreporting}

% We use the local smoothness technique. The local smoothness technique has been used to bound the Price of Anarchy in coevolutionary opinion formation games (see, e.g. \citet{bhawalkar2013}). We give the following theorem due to \citet{bhawalkar2013}, which is an extended result from \citet{roughgarden2011local}. 

% \begin{theorem}[\cite{bhawalkar2013}]
% Let $\sigma$ denote a correlated equilibrium. Suppose for any outcome $z$, with respect to a fixed outcome $o$ and scalars $\mu < 1, \lambda > 0$, that: 
% \begin{align}
% C(z) + (o-z)^T \nabla_z C(z) \leq \lambda C(o) + \mu C(z).
% \end{align}
% Then, the correlated PoA is bounded as $\frac{\EE_{\sigma}[C(z)]}{C(o)} \leq \frac {\lambda} {(1-\mu)}$. 
% \end{theorem}

% Using this method gives PoF bounds against any correlated equilibrium, and hence any Nash equilibrium. We can use the same framework to bound the PoM \mpcomment{are you sure we can do this?} \mpcomment{otherwise we can just have the PoA though the PoM is more useful IMHO}

% \mpcomment{quick question: shall we use different $y$ because we also have $y$ in the PSNE? It is defined in the scope of a theorem so I am not sure if it is indeed a problem regarding notation; just it may make things easier to read}

% \ajcomment{Not clear how to deal with differing $\alpha_i$, because the eigenvectors of $B$ are no longer the same as $L$.}

% \begin{theorem} \label{theorem:pom}
% Suppose all network members are strategic ($S = [n]$). Let $L$ have eignevalues $\lambda_i$ and suppose that there exists $\alpha$ such that for all $i$, $\alpha_i = \alpha$. Let $\Tilde \alpha = \frac{\alpha}{1 - \alpha}$. For $\mu \in (0,1)$ and $i \in [n]$ let: 
% \[
% f(\mu, i) = \bigg(\lambda_i + \Tilde\alpha +\Tilde\alpha(1 + \lambda_i)^2 \bigg)^{-1}
% \bigg(\frac{(\lambda_i + 2 \Tilde\alpha + 2 \mu \Tilde\alpha (1 + \lambda_i))^2}{(\lambda_i + \alpha)(1 + \mu)} - \frac{2\Tilde\alpha\lambda_i}{(1 + \mu)} + \frac{1 - \mu}{1 + \mu} \Tilde\alpha (1 + \lambda_i)^2 
% \bigg)
% \]
% Let $\mu^* = \arg\inf_{\mu \in (0,1)} \max_{i \in [n]} f(\mu, i)$ and $i^* = \arg\max_{i \in [n]} f(\mu^*, i)$. Then, for all $\eps > 0$, 
% \[
% \pom \leq \frac{f(\mu^*, i^*) + \eps}{(1 - \mu^*)}
% \]
% %Let $s^\prime$ be any deviation. Let $L$ be the graph Laplacian and $U \in O(n,\RR)$ its eigenbasis. Let $y = Us$.
% % Suppose that $\norm s^\prime \norm_2 \leq R$. The correlated price of fooling is at most: 
% % \[
% % \pom \leq \frac{2R + 2}{\min\{1, \min\{\abs{y_j}: y_j \neq 0 \}\}}.
% % \]
% \end{theorem}

% A few remarks are in order: First, notice that we give a strict generalization of PoM because the bound is against correlated equilibria, which are a generalization of Nash equilibria. Therefore in restricted classes such as PSNEs, the PoM may be smaller. Second, our bound is for the case when all $n$ agents are strategic. If only a small subset of the nodes are strategic the PoM may be smaller. In fact, this can be read off from our proof: what changes is that the entries of some entries of $s^\prime$ are constrained to be zero, but the proof would otherwise be the same. 

% % Finally, notice that our bound depends on the norm of the deviation $R$, as well as the quantity $\min_{j: y_j \neq 0} \abs{y_j}$. The ladder incorporates graph structure through the eigenvectors $U$ of the Laplacian (these can be interpreted as spectral clustering assignments), as well as the intrinsic opinions $s$. In particular, if the (graph-filtered) opinions $y_j$ have small nonzero entries, this indicates that the PoM will be large, because these members will be more swayed. 

% \mpcomment{each cost has an $1 - \alpha_i$ in the pairwise disagreement term -- so I think the algebra is a bit different (even for $\alpha_i = \alpha$). Also the $\lambda$ of the smoothness can be renamed to $\kappa$ to avoid confusion with the eigenvalues.}
% \begin{proof}
% Let $C(z) = \sum_i c_i(z)$. We want to show that for some $\lambda > 0, 0 < \mu < 1$: 
% \begin{align}
% C(z^\prime) + (s-s^\prime)^T \nabla_{s^\prime} C(z^\prime) \leq \lambda C(z) + \mu C(z^\prime)
% \label{eq:poa-condition}
% \end{align}
% Now, up to scaling, 
% \begin{align*}
% \frac{1}{1 - \alpha_i} C(z^\prime) &= \sum_i C_i(z^\prime) \\
% &= \sum_i \big(\sum_j w_{ij}(\bz_i^\prime - \bz_j^\prime)^2\big) + \frac{\alpha_i}{1 - \alpha_i} (\bz_i^\prime - \bs_i)^2 \\
% &= \la z^\prime, L z^\prime \ra + \Tilde \alpha \norm z^\prime - s \norm_2^2 
% \end{align*}
% %Where $\Tilde \alpha > 0$ is the shared $\alpha = \alpha_i$. 
% From the above, we see that $\nabla_{z^\prime} C(z^\prime) = 2 Lz^\prime + 2 \Tilde \alpha z^\prime - 2 \Tilde \alpha s$. 
% \ajcomment{Need to replace all occurrences of $\alpha$ with $\Tilde \alpha$, and $\lambda$ with $\kappa$.}

% Next, recall $z^\prime = B s^\prime$ for $B = (L + \alpha I)^{-1}$ and $s^\prime \in \RR^n$ the strategic internal opinions. By the chain rule, $\nabla_{s^\prime} C(z^\prime) = 2 BLBs^\prime + 2 \alpha B^2 s^\prime - 2 \alpha B \bs$. Therefore the LHS of Eq~\eqref{eq:poa-condition} becomes: 
% \begin{align*}
% C(z^\prime) + (s-s^\prime)^T \nabla_{s^\prime} C(z^\prime)
% &= \la s^\prime, BLB s^\prime \ra + \alpha \norm B s^\prime - s \norm_2^2 
% + (s-s^\prime)^T (2 BLBs^\prime + 2 \alpha B^2 s^\prime - 2 \alpha B \bs) \\
% &= (s^\prime)^T (BLB + \alpha B^T B - 2 BLB - 2 \alpha B^T B)s^\prime \\
% &+ s^T (\alpha I - 2 \alpha B)s
% + s^T (-2\alpha B + 2BLB + 2\alpha B^2 + 2\alpha B)s^\prime
% \end{align*}
% On the other hand, the RHS of Eq~\eqref{eq:poa-condition} becomes: 
% \begin{align*}
% \lambda s^T (BLB + \alpha (B - I)^T (B-I))s 
% + \mu C(s^\prime)
% &= \lambda s^T (BLB + \alpha (B - I)^T (B-I))s  
% + \mu(\la s^\prime, BLB s^\prime \ra + \alpha \norm B s^\prime - s \norm_2^2 )\\
% &= (s^\prime)^T (\mu BLB + \mu \alpha B^T B) s^\prime \\
% &+ s^T (\lambda BLB + \lambda \alpha (B - I)^T (B-I) + \mu \alpha I)s  
% + s^T (-2 \mu \alpha B) s^\prime 
% \end{align*}
% Combining the above displays, we see that Eq~\eqref{eq:poa-condition} holds iff: 
% \begin{align*}
% (s^\prime)^T ((1 + \mu)BLB + (1 + \mu)\alpha B^T B)s^\prime 
% + s^T (\lambda BLB + \lambda \alpha (B-I)^T (B-I) + \mu \alpha I) s
% + s^T (-2\mu\alpha B) s^\prime \\
% > s^T (\alpha I - 2 \alpha B)s 
% + s^T (-2\alpha B + 2 BLB + 2 \alpha B^2 + 2 \alpha B) s^\prime 
% \end{align*}
% Since $L \succeq 0$ is symmetric, let $L = UDU^T$ for some $D \succeq 0$ and eigenbasis $U$. Then $B = (L + I)^{-1} = U (D + I)^{-1} U^T$. Let $\Tilde D = (D + I)^{-1}$, so $B = U \Tilde D U^T$. We can perform a change of variables, letting $y = U s$ and $y^\prime = U s^\prime$. The above display becomes: 
% \begin{align*}
% (y^\prime)^T ((1 + \mu)\Tilde D^2 D + (1 + \mu)\alpha \Tilde D^2)y^\prime 
% + y^T (\lambda \Tilde D^2 D + \lambda \alpha \Tilde D^2 - 2 \alpha \Tilde D^2 + (\lambda + \mu) \alpha I) y
% + y^T (-2\mu\alpha \Tilde D) y^\prime \\
% > y^T (\alpha I - 2 \alpha \Tilde D)y
% + y^T (-2\alpha \Tilde D + 2 \Tilde D^2 D + 2 \alpha \Tilde D^2 + 2 \alpha \Tilde D) y^\prime \\
% \iff 
% (y^\prime)^T ((1 + \mu)\Tilde D^2 D + (1 + \mu)\alpha \Tilde D^2)y^\prime 
% + y^T (\lambda \Tilde D^2 D + \lambda \alpha \Tilde D^2 - 2 \alpha \Tilde D^2 + 2 \alpha \Tilde D + (\lambda + \mu - 1) \alpha I) y \\
% > y^T (2 \Tilde D^2 D + 2 \alpha \Tilde D^2 + 2 \mu \alpha \Tilde D) y^\prime 
% \end{align*}
% Now, let
% \begin{align*}
% F &= (1 + \mu)\Tilde D^2 D + (1 + \mu)\alpha \Tilde D^2 \\
% G &= \lambda \Tilde D^2 D + \lambda \alpha \Tilde D^2 - 2 \alpha \Tilde D^2 + 2 \alpha \Tilde D + (\lambda + \mu - 1) \alpha I \\
% H &= 2 \Tilde D^2 D + 2 \alpha \Tilde D^2 + 2 \mu \alpha \Tilde D
% \end{align*}
% Then, the above condition becomes: 
% \begin{align*}
% \begin{bmatrix} 
% y^\prime \\ y
% \end{bmatrix}^T 
% \begin{bmatrix}
% F & -\frac{1}{2} H \\
% -\frac{1}{2} H & G \end{bmatrix}
% \begin{bmatrix} 
% y^\prime \\ y
% \end{bmatrix} > 0
% \end{align*}
% Letting $M = \begin{bmatrix}
% F & -\frac{1}{2} H \\
% -\frac{1}{2} H & G \end{bmatrix}$, it is sufficient to show $M \succ 0$. 

% Notice that $F, G, H$ are all diagonal and $M$ is Hermitian. Hence $M$ has $2n$ real eigenvalues. For $i \in [n]$, let $M^{(i)} = \begin{bmatrix}
% F_{ii} & -\frac{1}{2} H_{ii} \\
% -\frac{1}{2} H_{ii} & G_{ii} \end{bmatrix}$.

% The eigenvalues of $M$ are given by $\bigcup\limits_{i \in [n]} \{\lambda_1(M^{(i)}), \lambda_2(M^{(i)})\}$. Therefore, it suffices to show $M^{(i)} \succ 0$ for all $i$. The following conditions are necessary and sufficient: 
% \begin{align*}
% F_{ii} &> 0 \\
% G_{ii} &> 0 \\
% \frac{1}{4} H_{ii}^2 &\leq F_{ii} G_{ii}
% \end{align*}
% First, $F_{ii} > 0 \iff (1 + \mu)(\lambda_i + \alpha) > 0$. This is true as long as $\alpha > 0$. Next, letting $\kappa = \lambda$, 
% \begin{align*}
% G_{ii} &> 0 \\
% `\frac{\kappa \lambda_i + \kappa \alpha - 2 \alpha}{(1 + \lambda_i)^2} + \frac{2\alpha}{1 + \lambda_i} + (\kappa + \mu - 1) \alpha &> 0 \\
% \iff \kappa \lambda_i + \alpha \bigg(
% \kappa - 2 + 2 (1 + \lambda_i) + (\kappa + \mu - 1) (1 + \lambda_i)^2
% \bigg) &> 0
% \end{align*}
% Since the Laplacian is PSD, $(1 + \lambda_i) \geq 1$ for all $i$. Hence, a sufficient condition for $G_{ii} > 0$ is that $\alpha > 0$ and: 
% \begin{align*}
% \kappa - 2 + 2 + (\kappa + \mu - 1) &> 0 \\
% \iff 2 \kappa + \mu - 1 &> 0
% \end{align*}
% Since $\kappa \geq 1$ and $\mu \geq 0$, this is always true. Finally, 
% \begin{align*}
% \frac{1}{4} H_{ii}^2 &\leq F_{ii} G_{ii} \\
% \bigg(\frac{\lambda_i + 2 \alpha + 2 \mu \alpha (1 + \lambda_i)}{(1 + \lambda_i)^2}
% \bigg)^2 
% &\leq \bigg[\frac{(1 + \mu) (\lambda_i + \alpha)}{(1 + \lambda_i)^2} \\
% &\cdot 
% \frac{\kappa \lambda_i + (\kappa - 2) \alpha + 2 \alpha (1 + \lambda_i) + (\kappa + \mu - 1)\alpha (1 + \lambda_i)^2}{(1 + \lambda_i)^2} \bigg] \\
% \frac{(\lambda_i + 2 \alpha + 2 \mu \alpha (1 + \lambda_i))^2}{\lambda_i + \alpha} &\leq 
% (1 + \mu) \bigg(\kappa \lambda_i + (\kappa - 2) \alpha \\
% &+ 2 \alpha (1 + \lambda_i) + (\kappa + \mu - 1)\alpha (1 + \lambda_i)^2\bigg) \\
% \frac{(\lambda_i + 2 \alpha + 2 \mu \alpha (1 + \lambda_i))^2}{\lambda_i + \alpha} - 2 \alpha \lambda_i 
% &\leq \bigg[(1 - \mu)(1 + \mu) \frac{\kappa}{1 - \mu} \big(\lambda_i + \alpha + \alpha (1 + \lambda_i)^2\big) \\
% &- (1 - \mu) \alpha (1 + \lambda_i)^2 
% \bigg]\\
% % \bigg(\frac{1}{\lambda_i + \alpha + \alpha (1 + \lambda_i)^2}\bigg) 
% \bigg(\frac{(\lambda_i + 2 \alpha + 2 \mu \alpha (1 + \lambda_i))^2}{\lambda_i + \alpha} - 2 \alpha \lambda_i + (1 - \mu) \alpha (1 + \lambda_i)^2
% \bigg)
% &\leq (1 - \mu^2) \frac{\kappa}{1-\mu}\big(\lambda_i + \alpha + \alpha (1 + \lambda_i)^2 \big) 
% \end{align*}
% We conclude that it is sufficient for $\kappa, \mu$ to be such that: 
% \[
% \bigg(\lambda_i + \alpha + \alpha (1 + \lambda_i)^2 \bigg)^{-1}
% \bigg(\frac{(\lambda_i + 2 \alpha + 2 \mu \alpha (1 + \lambda_i))^2}{(\lambda_i + \alpha)(1 + \mu)} - \frac{2 \alpha \lambda_i}{(1 + \mu)} + \frac{1 - \mu}{1 + \mu} \alpha (1 + \lambda_i)^2 
% \bigg) \leq \kappa 
% \]
% Notice the LHS is $f(\mu,i)$. Choosing $\mu^* = \arg \inf_{\mu \in (0,1)} \max_{i \in [n]} f(\mu,i)$, and setting $\kappa = f(\mu^*, i^*)$
% ensures that $\frac{1}{4} H_{ii}^2 \leq F_{ii} G_{ii}$ for all $i$. Hence we conclude that $M \succ 0$.  
% \end{proof}

% \begin{cor}
% If $i^* = 1$, where the eigenvalues of $L$ are ordered $\lambda_1 \leq \dots \lambda_n$ then: 
% \[
% \pom \leq \frac{5}{2} + \eps
% \]
% for arbitrarily small $\eps > 0$. 
% \end{cor}
% \begin{proof}
% Since $L$ is a graph Laplacian, we have $L \bm{1} = \bm{0}$, so $\lambda_1 = 0$. We can simplify $f(\mu, 1)$ as: 
% \begin{align*}
% f(\mu, 1) &= \frac{- \mu + 4 \left(\mu + 1\right)^{2} + 1}{2 \left(\mu + 1\right)} \\
% \frac{d}{d\mu} f(\mu, 1) &= \frac{2 \mu^{2} + 4 \mu + 1}{\mu^{2} + 2 \mu + 1} \\
% \frac{d^2}{d\mu^2} f(\mu, 1) &= \frac{2}{\mu^{3} + 3 \mu^{2} + 3 \mu + 1}
% \end{align*}
% Notice there is no dependence on $\Tilde \alpha$. Since $\mu > 0$, $\frac{d^2}{d\mu^2} > 0$ for all $\mu$, so $\mu^* \in \{0, 1\}$. We can compute that $f(0, 1) = 5/2$ and $f(1,1) = 4$. Hence, by setting $\mu = \eps$ for $\eps > 0$, we conclude that $\pom \leq \frac{5}{2} + \eps$ for arbitrarily small $\eps > 0$. 
% \end{proof}
% \ajcomment{This is useless because $i^* \neq 1$.}

% \begin{prop}
% Let $g(\mu, \lambda) = \frac{1}{1-\mu}f(\mu, i)$ if $\lambda_i = \lambda$. We claim that for all $\lambda > 0$, $\Tilde \alpha \in (0,1)$ and $\mu \in (0,1)$ that: 

% 1. $\frac{\del g}{\del \mu} \neq 0$

% 2. $\frac{\del^2 g}{\del \mu^2} > 0$. 
% % Let $f(\mu, \lambda)$ be defined by letting $\lambda_i = \lambda$. For all $\Tilde \alpha, \mu \in (0,1)$, $\frac{df(\mu, \lambda)}{d\lambda} < 0$. 
% \end{prop}

% \begin{prop}
% For all $\Tilde \alpha, \mu \in (0,1)$ and $\lambda \geq 0$, we have $\frac{dg(\mu, \lambda)}{d\lambda} < 0$
% \end{prop}

% The above two propositions imply that $i^* = n$ and that the optimal $\mu^*$ is within $\{0,1\}$. We can read off: 
% \[
% g(\mu, \lambda) = \frac{\tilde \alpha \left(\tilde \alpha + \lambda\right) \left(2 \lambda + \left(\lambda + 1\right)^{2} \left(\mu - 1\right)\right) - \left(2 \tilde \alpha \mu \left(\lambda + 1\right) + 2 \tilde \alpha + \lambda\right)^{2}}{\left(\tilde \alpha + \lambda\right) \left(\mu - 1\right) \left(\mu + 1\right) \left(\tilde \alpha \left(\lambda + 1\right)^{2} + \tilde \alpha + \lambda\right)}
% \]
% From this, it is clear that $\lim\limits_{\mu \to 1} g(\mu, \lambda) = \infty$. Therefore $\mu^* = 0$. 
% \ajcomment{Weird things happening, somehow we get that the PoM is at most 1 numerically...this shouldn't happen. I will revisit the math with $\alpha_i$ differing and see what happens.}

% % \begin{proof}
% % This can be read off from: 
% % \[
% % \frac{- \alpha \left(\lambda + 1\right) \left(\mu - 1\right) \left(\mu + 1\right) \left(8 \alpha \mu \left(\lambda + 1\right) + 8 \alpha + 4 \lambda - \left(\alpha + \lambda\right) \left(\lambda + 1\right)\right) - \left(\mu - 1\right) \left(\alpha \left(\alpha + \lambda\right) \left(2 \lambda + \left(\lambda + 1\right)^{2} \left(\mu - 1\right)\right) - \left(2 \alpha \mu \left(\lambda + 1\right) + 2 \alpha + \lambda\right)^{2}\right) - \left(\mu + 1\right) \left(\alpha \left(\alpha + \lambda\right) \left(2 \lambda + \left(\lambda + 1\right)^{2} \left(\mu - 1\right)\right) - \left(2 \alpha \mu \left(\lambda + 1\right) + 2 \alpha + \lambda\right)^{2}\right)}{\left(\alpha + \lambda\right) \left(\mu - 1\right)^{2} \left(\mu + 1\right)^{2} \left(\alpha \left(\lambda + 1\right)^{2} + \alpha + \lambda\right)}
% % \]
% % \end{proof}

% \ajcomment{This can be shown for $f/(1-\mu)$ as well.}

% Therefore, we know that for any $\mu$, that either $i^* = n$. 
% % \clearpage



% % % \clearpage

% % \ajcomment{The stuff here is old.}
% % Letting $y^\prime = y + \delta$, we can simplify: 
% % \begin{align*}
% % (y^\prime)^T ((1 + \mu)\Tilde D^2 D + (1 + \mu)\alpha \Tilde D^2)y^\prime 
% % + y^T ((\lambda - 2) \Tilde D^2 D + (\lambda - 2) \alpha \Tilde D^2 + 2 \alpha (1-\mu) \Tilde D + (\lambda + \mu - 1) \alpha I) y \\
% % > y^T (2 \Tilde D^2 D + 2 \alpha \Tilde D^2 + 2 \mu \alpha \Tilde D) \delta 
% % \end{align*}
% % Next, suppose that $\norm \delta \norm_2 \leq R$. Let $0 = \lambda_n \leq \lambda_{n-1} \leq \dots \leq \lambda_1$ be the eigenvalues of $L$, so that $D_{ii} = \lambda_i$. Then, a variational argument implies that to maximize the RHS of the above display we would set $\delta = \frac{R y_j}{\abs{y_j}} \be_j$, where $j = \arg\max_{i \in [n]} \abs{y_i (\frac{\lambda_i}{(1 + \lambda_i)^2} + \frac{2\alpha}{(1 + \lambda_i)^2} + \frac{2\mu\alpha}{1 + \lambda_i})}$. Hence the RHS is upper bounded as: 
% % \[
% % R \max_i \bigg(\abs{y_i} \cdot \bigg(\frac{\lambda_i}{(1 + \lambda_i)^2} + \frac{2\alpha}{(1 + \lambda_i)^2} + \frac{2\mu\alpha}{1 + \lambda_i}
% % \bigg) \bigg)
% % \]
% % To lower bound the LHS, notice that the first summand (the quadratic form in $y^\prime$) is non-negative since $\Tilde D^2 D \succeq 0$ and $\Tilde D^2 \succeq 0$. The second summand is a quadratic form in $y$, and hence is of the form $\sum_i c_i y_i^2$ for coefficients $c_i$. If $\lambda > 2$ then all the coefficients $c_i > 0$. In fact, 
% % \begin{align*}
% % y^T ((\lambda - 2) \Tilde D^2 D + (\lambda - 2) \alpha \Tilde D^2 + 2 \alpha (1-\mu) \Tilde D + (\lambda + \mu - 1) \alpha I) y \\
% % = \sum_i \bigg(
% % \frac{(\lambda - 2)\lambda_i}{(1 + \lambda_i)^2}
% % + \frac{(\lambda - 2) \alpha}{(1 + \lambda_i)^2}
% % + \frac{2 \alpha (1 - \mu)}{(1 + \lambda_i)}
% % + (\lambda + \mu - 1)\alpha
% % \bigg) y_i^2
% % \end{align*}
% % We proceed by casework. First, if the $j$ chosen in $\delta$ is such that $y_j = 0$ then Eq.~\eqref{eq:poa-condition} is trivially satisfied for any $\lambda \geq 2R + 2$. 

% % On the other hand, if $y_j \neq 0$, then set $lambda = \frac{2R + 2}{\min_{j: y_j \neq 0} \abs{y_j}}$. We claim this suffices to satisfy Eq.~\eqref{eq:poa-condition}. First, it is clear that for all $i$, 
% % \[
% % \bigg(\frac{(\lambda - 2)\lambda_i}{(1 + \lambda_i)^2}
% % + \frac{(\lambda - 2) \alpha}{(1 + \lambda_i)^2}\bigg) y_i^2
% % \leq R \abs{y_i} \bigg(\frac{\lambda_i}{(1 + \lambda_i)^2} + \frac{2\alpha}{(1 + \lambda_i)^2} + \frac{2\mu\alpha}{1 + \lambda_i}
% % \bigg)
% % \]
% % Moreover, we claim that for any $i$ such that $\abs{y_i} \neq 0$, 
% % \begin{align*}
% % \frac{2R \abs{y_i} \mu \alpha}{1 + \lambda_i} 
% % &\leq 
% % \bigg(\frac{2 \alpha (1 - \mu)}{(1 + \lambda_i)}
% % + (\lambda + \mu - 1)\alpha
% % \bigg) y_i^2 \\
% % \iff \frac{2 R \mu}{\abs{y_i}} 
% % &\leq 2(1-\mu)  + (\lambda + \mu - 1)(1 + \lambda_i)
% % \end{align*}
% % Notice that $(1 + \lambda_i) \geq 1$ and $\lambda \geq \frac{2R}{\abs{y_i}}$. 
% % We conclude that for any permissible $\mu$ and for $\lambda$ as above, that Eq.~\eqref{eq:poa-condition} holds. 

% \subsection{Differing alpha}

% Let $B = ((I - A) L + A)^{-1} A$. Notice $z^\prime = B \bs^\prime$. 

% First, it can be shown that: 
% \begin{align*}
% C(z^\prime)
% &= (z^\prime - \bs)^T A (z^\prime - s)
% + (z^\prime)^T (I - A)D z^\prime - 2(z^\prime)^T (I - A) W z^\prime
% + \sum_{i \in [n], j \in [n]} w_{ij} \bz_j^2 (1 - \alpha_i)
% \end{align*}

% Here $D$ is the degrees matrix and $W$ the weighted adjacency matrix. 

% We want to analyze for $\kappa > 1$ and $\mu \in (0,1)$, 
% \begin{align*}
% (1 - \mu) C(s^\prime) + (s - s^\prime)^T \nabla_{s^\prime} C(s^\prime) 
% \leq \kappa C(s)
% \end{align*}

% Let $M_1 = 2B^T AB + 2B^T (I - A) D B - 4 B^T (I - A) B$. 

% First, we argue that: 
% \begin{align*}
% \nabla_{s^\prime} C(s^\prime) 
% &= M_1 s^\prime - 2 B^T A s
% + 2 \sum_{ij} w_{ij} (1 - \alpha_i) (B^T e_j e_j^T B) s^\prime 
% \end{align*}
% Letting $M_2 = 2 \sum_{ij} w_{ij} (1 - \alpha_i) (B^T e_j e_j^T B)$, we have that
% \begin{align*}
% \nabla_{s^\prime} C(s^\prime) 
% &= (M_1 + M_2) s^\prime - 2 B^T A s
% \end{align*}
% So, 
% \begin{align*}
% (s - s^\prime)^T \nabla_{s^\prime} C(s^\prime) 
% &= s^T (M_1 + M_2 + 2 AB^T) s^\prime 
% + s^T (-2B^T A) s 
% + (s^\prime)^T (- M_1 - M_2) s^\prime 
% \end{align*}
% Next, let $D_2$ be the diagonal matrix with $D_{2;j} = \sum_{i=1}^{n} W_{ij} (1 - \alpha_i)$. Then, 
% \begin{align*}
% C(z^\prime) = z^\prime 
% \big( A + (I - A) D - 2(I - A)W + D_2
% \big) z^\prime  
% + s^T (-AB - B^T A) z^\prime 
% + s^T A s
% \end{align*}

% Therefore, let $M_3 = A + (I - A) D - 2(I - A)W + D_2$. Then, 
% \begin{align*}
% C(s^\prime) &= (s^\prime)^T B^T M_3 B s^\prime 
% + s^T (- AB - B^T A)B s^\prime 
% + s^T A s \\ 
% C(s) &= s^T B^T M_3 B s
% + s^T (- AB - B^T A)B s
% + s^T A s \\ 
% \end{align*}

% Let $M_4 = B^T M_3 B - AB^2 - B^T AB + A$. 

% \begin{align*}
% \begin{bmatrix} 
% y^\prime \\ y
% \end{bmatrix}^T 
% \begin{bmatrix}
% F & -\frac{1}{2} H \\
% -\frac{1}{2} H & G \end{bmatrix}
% \begin{bmatrix} 
% y^\prime \\ y
% \end{bmatrix} > 0
% \end{align*}
% !TEX root = ./main.tex
\section{Price of Misreporting}
In Section~\ref{sec:strategic_opinion_formation}, we saw that strategic manipulation can substantially affect network outcomes via the Polarization Ratio and Disruption Ratio. We now give an upper bound for the Price of Misreporting (Eq.~\eqref{eq:pom}), which is the analogue of the Price of Anarchy in our setting. The $\pom$ measures the total cost paid by agents under the corrupted equilibrium $\bz^\prime$, versus the total cost under the non-corrupted $\bz$. Since the cost captures an agent's deviation from her {\em truthful} intrinsic opinion as well as her deviation from the expressed opinions of her neighbors, it is a natural measure of the network's discord at equilibrium. 

Theorem~\ref{thrm:pom_shared_alpha} shows that the PoM is small when the spectral radius of the Laplacian is small, and when agents are somewhat susceptible to their neigbhors ($\alpha \not \to 0$). Note that the spectral radius can be replaced by a degree bound: if $d_{\textup{max}}$ is the maximum degree of the graph, then $\lambda_n \leq 2 d_{\textup{max}}$. So the PoM is small if the maximum degree is small.
% If either of these conditions fail, our upper bound for the PoM can grow arbitrarily. 

\begin{theorem}
Suppose all agents deviate ($S = [n]$) and there exists $\alpha$ such that $\alpha_i = \alpha$ for all $i$.
Let $\tilde \alpha = \alpha / (1 - \alpha)$, and $\lambda_n$ be the spectral radius of the Laplacian. Then the price of misreporting is bounded as:
    \begin{align*}
        \pom \le \frac {(\lambda_n + 4 \tilde \alpha) (\lambda_n + \tilde \alpha)^2} {\tilde \alpha^5} = O \left ( \max \left \{ \frac {\lambda_n} {\tilde \alpha^5}, \frac {1} {\tilde \alpha^2} \right \} \right ).
    \end{align*}
\label{thrm:pom_shared_alpha}
\end{theorem}

% \ajcomment{Define }
\begin{proof}[Proof of Theorem~\ref{thrm:pom_shared_alpha}.]
First, we set $\tilde \alpha = \alpha / (1 - \alpha)$. By substituting $\bz = B \bs$ we can show by straightforward algebra that $C(\bz) / (1 - \alpha) = \bs^T Q \bs$ where $Q \succeq 0$ with 

\begin{align}
    Q = B L B + \tilde \alpha (I - 2B + B^2)
\end{align}
Since $Q \succeq 0$, it has eigendecomposition $Q = U \Lambda_Q U^T$. Moreover, $U$ is precisely the matrix of eigenvectors for the Laplacian. The eigenvalues of $Q$ can be shown to be $\tilde \alpha^2 / (\lambda_i + \tilde \alpha)$. Therefore, $C(z) = \bs^T Q \bs \ge \frac {\tilde \alpha^2} {\lambda_n + \tilde \alpha} \norm \bs \norm_2^2$. 

Next, let $\diag (B)$ be the diagonal matrix with entries $B_{ii}$ and $\widetilde {\diag (B)}$ be as in \cref{theorem:all_deviation}. In the proof of \cref{theorem:all_deviation}, we show that $\bs^\prime = (1/\tilde \alpha) B^{-1} \widetilde {\diag (B)} \bs$ and $\bz^\prime = (1 /\tilde \alpha) \widetilde{\diag (B)} \bs$, which similarly implies (after algebraic operations) that $C(\bz^\prime) / (1 - \alpha) = \bs^T Q^\prime \bs$ where: 

\begin{align*}
    Q^\prime := \frac {1} {\tilde \alpha^2} \widetilde{\diag (B)} L \widetilde {\diag (B)} + \frac {1} {\tilde \alpha} B^{-1} \left ( \widetilde{\diag (B)} \right )^2 B^{-1} - 2 \frac {1} {\tilde \alpha} B^{-1} \left ( \widetilde{\diag (B)} \right )^2 + \frac {1} {\tilde \alpha} \left ( \widetilde{\diag (B)} \right )^2
\end{align*}

Note that $Q^\prime$ cannot be diagonalized since, in general, $\widetilde{\diag (B)}$ has a different eigenbasis than $L$. However, we note that: 
\begin{align} \label{eq:ineq_Bi}
    \norm \widetilde {\diag (B)} \norm_2 & = \max_i B_{ii} \le \norm B \norm_2 = 1, \\
    \norm B^{-1} \norm_2 & = \max_i \frac {\lambda_i + \tilde \alpha} {\tilde \alpha} = \frac {\lambda_n + \tilde \alpha} {\tilde \alpha}. \label{eq:ineq_Binv}
\end{align}
By the triangle inequality, the Cauchy-Schwarz inequality, and \cref{eq:ineq_Bi,eq:ineq_Binv}, we have that: 
\begin{align*}
    \norm Q^\prime \norm_2 & \le \frac {1} {\tilde \alpha^2} \norm L \norm_2 \left ( \norm \widetilde {\diag (B)} \norm_2 \right )^2 + \frac {1} {\tilde \alpha} \left ( \norm \widetilde {\diag (B)} \norm_2 \right )^2 \norm B^{-1} \norm_2^2 + \frac 2 {\tilde \alpha} \norm B^{-1} \norm_2 + \frac {1} {\tilde \alpha} \left ( \norm \widetilde {\diag (B)} \norm_2 \right )^2 \\
    & \le \frac {(\lambda_n + 4 \tilde \alpha) (\lambda_n + \tilde \alpha)} {\tilde \alpha^3}.
\end{align*}

Therefore $C(\bz^\prime) / (1 - \alpha) \le \frac {(\lambda_n + 4 \tilde \alpha) (\lambda_n + \tilde \alpha)} {\tilde \alpha^3} \norm \bs \norm_2^2$. Hence,
\begin{align}
    \frac {C(\bz^\prime)} {C(\bz)} \le \frac {(\lambda_n + 4 \tilde \alpha) (\lambda_n + \tilde \alpha)^2} {\tilde \alpha^5}.
\end{align}

Finally, we can simplify: % \ajcomment{The 64 can be imporved to $9$ I believe, if we care about constants.}
\begin{align}
    \frac {(\lambda_n + 4 \tilde \alpha) (\lambda_n + \tilde \alpha)^2} {\tilde \alpha^5} \le \frac {64 (\lambda_n + \tilde \alpha)^3} {\tilde \alpha^5} \le \frac {128 (\max \{ \lambda_n, \tilde \alpha \})^3} {\tilde \alpha^5}.
\end{align}    
\end{proof}

From Theorem~\ref{thrm:pom_shared_alpha}, we can show that the upper bound is minimized when $\lambda_n = \Theta (\tilde \alpha^3)$ and has a value of $O(1/\tilde \alpha^2)$. As we noted, Theorem~\ref{thrm:pom_shared_alpha} can be written with $d_{\textup{max}}$ in the place of $\lambda_n$ as well. 

Next, we give an easy generalization to the case of differing susceptibility. 
\begin{cor}[Price of Misreporting for Heterogeneous Susceptibility]
If the $\alpha_i$ are differing, let $\alpha_{\min} = \min_i \alpha_i$ and $\alpha_{\max} = \max_j \alpha_j$. Define $\tilde \alpha_{\min} = \frac{\alpha_{\min}}{1 - \alpha_{\max}}, \tilde \alpha_{\max} = \frac{\alpha_{\max}}{1 - \alpha_{\min}}$. The Price of Misreporting is bounded as:
\begin{align*}
        \pom \le \frac {1 -\alpha_{\min}} {1 - \alpha_{\max}}\frac {(\lambda_n + 4 \tilde \alpha_{\max}) (\lambda_n + \tilde \alpha_{\max})^2} {\tilde \alpha_{\min}}. 
        %= O \left ( \max \left \{ \frac {\lambda_n} {\tilde \alpha^5}, \frac {1} {\tilde \alpha^2} \right \} \right ).
    \end{align*}
\end{cor}
\begin{proof}
Note that $C(\bz^\prime) \le (1 - \alpha_{\min}) (\bz^\prime)^T L \bz^\prime + \alpha_{\max} \norm \bz^\prime - \bs \norm_2^2 = \overline C(\bz^\prime)$, and $C(\bz) \ge (1 - \alpha_{\max}) \bz^T L \bz + \alpha_{\min} \norm \bz - \bs \norm_2^2 = \underline C(\bz)$ where $\alpha_{\min} = \min_{i \in [n]} \alpha_i$, and $\alpha_{\max} = \max_{i \in [n]} \alpha_i$. Then, the same analysis of \cref{thrm:pom_shared_alpha} can be applied, since $\overline C(\bz^\prime) / \underline C(\bz) \ge C(\bz^\prime) / C(\bz)$. 
\end{proof}

Finally, we discuss how one may generalize Theorem~\ref{thrm:pom_shared_alpha} to the case where some agents are honest. 
% \ajcomment{Why not rewrite the theorem?}

\paragraph{Towards fine-grained $\pom$ guarantees.} Figure~\ref{fig:ratios_susceptibility_to_persuation} shows that the PoM is {\em not} monotonic in $\abs{S}$. As the number of strategic agents grows, the PoM can fall or grow, depending on the choice of $S$, network parameters, and so on. Therefore, we would like to give a version of Theorem~\ref{thrm:pom_shared_alpha} for {\em any} set of strategic agents $S \subset [n]$, not just the case of $S = [n]$. However, proving such a bound would require analyzing $S \times S$ principal submatrices of $B, L$ to obtain characterizations of the cost at the corrupted equilibrium $\bz^\prime$. In particular, we would require a {\em restricted invertibility} estimate to prove the analogue of Eq.~\eqref{eq:ineq_Binv}. To our knowledge, the best such estimates~\citep{marcus2022interlacing} are too lossy when $n - \abs{S}$ is large. We leave this question to future work. 

% \ajcomment{Todo comment on RIP and random principal submatrix.}

% \citep{rudelson2007sampling}


% bounds the PoM when all agents manipulate the network. 

% Hence for networks whose largest degree is sufficiently small, the PoM is well-behaved. 
%additionally shows that in networks where the largest degree is sufficiently small, the PoM is well-behaved.

% \newpage


% We use the local smoothness technique. The local smoothness technique has been used to bound the Price of Anarchy in coevolutionary opinion formation games (see, e.g. \citet{bhawalkar2013}). We give the following theorem due to \citet{bhawalkar2013}, which is an extended result from \citet{roughgarden2011local}. 

% \begin{theorem}[\cite{bhawalkar2013}]
% Let $\sigma$ denote a correlated equilibrium. Suppose for any outcome $z$, with respect to a fixed outcome $o$ and scalars $\mu < 1, \lambda > 0$, that: 
% \begin{align}
% C(z) + (o-z)^T \nabla_z C(z) \leq \lambda C(o) + \mu C(z).
% \end{align}
% Then, the correlated PoA is bounded as $\frac{\EE_{\sigma}[C(z)]}{C(o)} \leq \frac {\lambda} {(1-\mu)}$. 
% \end{theorem}

% Using this method gives PoF bounds against any correlated equilibrium, and hence any Nash equilibrium. We can use the same framework to bound the PoM \mpcomment{are you sure we can do this?} \mpcomment{otherwise we can just have the PoA though the PoM is more useful IMHO}

% \mpcomment{quick question: shall we use different $y$ because we also have $y$ in the PSNE? It is defined in the scope of a theorem so I am not sure if it is indeed a problem regarding notation; just it may make things easier to read}

% \ajcomment{Not clear how to deal with differing $\alpha_i$, because the eigenvectors of $B$ are no longer the same as $L$.}

% \begin{theorem} \label{theorem:pom}
% Suppose all network members are strategic ($S = [n]$). Let $L$ have eignevalues $\lambda_i$ and suppose that there exists $\alpha$ such that for all $i$, $\alpha_i = \alpha$. Let $\Tilde \alpha = \frac{\alpha}{1 - \alpha}$. For $\mu \in (0,1)$ and $i \in [n]$ let: 
% \[
% f(\mu, i) = \bigg(\lambda_i + \Tilde\alpha +\Tilde\alpha(1 + \lambda_i)^2 \bigg)^{-1}
% \bigg(\frac{(\lambda_i + 2 \Tilde\alpha + 2 \mu \Tilde\alpha (1 + \lambda_i))^2}{(\lambda_i + \alpha)(1 + \mu)} - \frac{2\Tilde\alpha\lambda_i}{(1 + \mu)} + \frac{1 - \mu}{1 + \mu} \Tilde\alpha (1 + \lambda_i)^2 
% \bigg)
% \]
% Let $\mu^* = \arg\inf_{\mu \in (0,1)} \max_{i \in [n]} f(\mu, i)$ and $i^* = \arg\max_{i \in [n]} f(\mu^*, i)$. Then, for all $\eps > 0$, 
% \[
% \pom \leq \frac{f(\mu^*, i^*) + \eps}{(1 - \mu^*)}
% \]
% %Let $s^\prime$ be any deviation. Let $L$ be the graph Laplacian and $U \in O(n,\RR)$ its eigenbasis. Let $y = Us$.
% % Suppose that $\norm s^\prime \norm_2 \leq R$. The correlated price of fooling is at most: 
% % \[
% % \pom \leq \frac{2R + 2}{\min\{1, \min\{\abs{y_j}: y_j \neq 0 \}\}}.
% % \]
% \end{theorem}

% A few remarks are in order: First, notice that we give a strict generalization of PoM because the bound is against correlated equilibria, which are a generalization of Nash equilibria. Therefore in restricted classes such as PSNEs, the PoM may be smaller. Second, our bound is for the case when all $n$ agents are strategic. If only a small subset of the nodes are strategic the PoM may be smaller. In fact, this can be read off from our proof: what changes is that the entries of some entries of $s^\prime$ are constrained to be zero, but the proof would otherwise be the same. 

% % Finally, notice that our bound depends on the norm of the deviation $R$, as well as the quantity $\min_{j: y_j \neq 0} \abs{y_j}$. The ladder incorporates graph structure through the eigenvectors $U$ of the Laplacian (these can be interpreted as spectral clustering assignments), as well as the intrinsic opinions $s$. In particular, if the (graph-filtered) opinions $y_j$ have small nonzero entries, this indicates that the PoM will be large, because these members will be more swayed. 

% \mpcomment{each cost has an $1 - \alpha_i$ in the pairwise disagreement term -- so I think the algebra is a bit different (even for $\alpha_i = \alpha$). Also the $\lambda$ of the smoothness can be renamed to $\kappa$ to avoid confusion with the eigenvalues.}
% \begin{proof}
% Let $C(z) = \sum_i c_i(z)$. We want to show that for some $\lambda > 0, 0 < \mu < 1$: 
% \begin{align}
% C(z^\prime) + (s-s^\prime)^T \nabla_{s^\prime} C(z^\prime) \leq \lambda C(z) + \mu C(z^\prime)
% \label{eq:poa-condition}
% \end{align}
% Now, up to scaling, 
% \begin{align*}
% \frac{1}{1 - \alpha_i} C(z^\prime) &= \sum_i C_i(z^\prime) \\
% &= \sum_i \big(\sum_j w_{ij}(z_i^\prime - z_j^\prime)^2\big) + \frac{\alpha_i}{1 - \alpha_i} (z_i^\prime - s_i)^2 \\
% &= \la z^\prime, L z^\prime \ra + \Tilde \alpha \norm z^\prime - s \norm_2^2 
% \end{align*}
% %Where $\Tilde \alpha > 0$ is the shared $\alpha = \alpha_i$. 
% From the above, we see that $\nabla_{z^\prime} C(z^\prime) = 2 Lz^\prime + 2 \Tilde \alpha z^\prime - 2 \Tilde \alpha s$. 
% \ajcomment{Need to replace all occurrences of $\alpha$ with $\Tilde \alpha$, and $\lambda$ with $\kappa$.}

% Next, recall $z^\prime = B s^\prime$ for $B = (L + \alpha I)^{-1}$ and $s^\prime \in \RR^n$ the strategic internal opinions. By the chain rule, $\nabla_{s^\prime} C(z^\prime) = 2 BLBs^\prime + 2 \alpha B^2 s^\prime - 2 \alpha Bs$. Therefore the LHS of Eq~\eqref{eq:poa-condition} becomes: 
% \begin{align*}
% C(z^\prime) + (s-s^\prime)^T \nabla_{s^\prime} C(z^\prime)
% &= \la s^\prime, BLB s^\prime \ra + \alpha \norm B s^\prime - s \norm_2^2 
% + (s-s^\prime)^T (2 BLBs^\prime + 2 \alpha B^2 s^\prime - 2 \alpha Bs) \\
% &= (s^\prime)^T (BLB + \alpha B^T B - 2 BLB - 2 \alpha B^T B)s^\prime \\
% &+ s^T (\alpha I - 2 \alpha B)s
% + s^T (-2\alpha B + 2BLB + 2\alpha B^2 + 2\alpha B)s^\prime
% \end{align*}
% On the other hand, the RHS of Eq~\eqref{eq:poa-condition} becomes: 
% \begin{align*}
% \lambda s^T (BLB + \alpha (B - I)^T (B-I))s 
% + \mu C(s^\prime)
% &= \lambda s^T (BLB + \alpha (B - I)^T (B-I))s  
% + \mu(\la s^\prime, BLB s^\prime \ra + \alpha \norm B s^\prime - s \norm_2^2 )\\
% &= (s^\prime)^T (\mu BLB + \mu \alpha B^T B) s^\prime \\
% &+ s^T (\lambda BLB + \lambda \alpha (B - I)^T (B-I) + \mu \alpha I)s  
% + s^T (-2 \mu \alpha B) s^\prime 
% \end{align*}
% Combining the above displays, we see that Eq~\eqref{eq:poa-condition} holds iff: 
% \begin{align*}
% (s^\prime)^T ((1 + \mu)BLB + (1 + \mu)\alpha B^T B)s^\prime 
% + s^T (\lambda BLB + \lambda \alpha (B-I)^T (B-I) + \mu \alpha I) s
% + s^T (-2\mu\alpha B) s^\prime \\
% > s^T (\alpha I - 2 \alpha B)s 
% + s^T (-2\alpha B + 2 BLB + 2 \alpha B^2 + 2 \alpha B) s^\prime 
% \end{align*}
% Since $L \succeq 0$ is symmetric, let $L = UDU^T$ for some $D \succeq 0$ and eigenbasis $U$. Then $B = (L + I)^{-1} = U (D + I)^{-1} U^T$. Let $\Tilde D = (D + I)^{-1}$, so $B = U \Tilde D U^T$. We can perform a change of variables, letting $y = U s$ and $y^\prime = U s^\prime$. The above display becomes: 
% \begin{align*}
% (y^\prime)^T ((1 + \mu)\Tilde D^2 D + (1 + \mu)\alpha \Tilde D^2)y^\prime 
% + y^T (\lambda \Tilde D^2 D + \lambda \alpha \Tilde D^2 - 2 \alpha \Tilde D^2 + (\lambda + \mu) \alpha I) y
% + y^T (-2\mu\alpha \Tilde D) y^\prime \\
% > y^T (\alpha I - 2 \alpha \Tilde D)y
% + y^T (-2\alpha \Tilde D + 2 \Tilde D^2 D + 2 \alpha \Tilde D^2 + 2 \alpha \Tilde D) y^\prime \\
% \iff 
% (y^\prime)^T ((1 + \mu)\Tilde D^2 D + (1 + \mu)\alpha \Tilde D^2)y^\prime 
% + y^T (\lambda \Tilde D^2 D + \lambda \alpha \Tilde D^2 - 2 \alpha \Tilde D^2 + 2 \alpha \Tilde D + (\lambda + \mu - 1) \alpha I) y \\
% > y^T (2 \Tilde D^2 D + 2 \alpha \Tilde D^2 + 2 \mu \alpha \Tilde D) y^\prime 
% \end{align*}
% Now, let
% \begin{align*}
% F &= (1 + \mu)\Tilde D^2 D + (1 + \mu)\alpha \Tilde D^2 \\
% G &= \lambda \Tilde D^2 D + \lambda \alpha \Tilde D^2 - 2 \alpha \Tilde D^2 + 2 \alpha \Tilde D + (\lambda + \mu - 1) \alpha I \\
% H &= 2 \Tilde D^2 D + 2 \alpha \Tilde D^2 + 2 \mu \alpha \Tilde D
% \end{align*}
% Then, the above condition becomes: 
% \begin{align*}
% \begin{bmatrix} 
% y^\prime \\ y
% \end{bmatrix}^T 
% \begin{bmatrix}
% F & -\frac{1}{2} H \\
% -\frac{1}{2} H & G \end{bmatrix}
% \begin{bmatrix} 
% y^\prime \\ y
% \end{bmatrix} > 0
% \end{align*}
% Letting $M = \begin{bmatrix}
% F & -\frac{1}{2} H \\
% -\frac{1}{2} H & G \end{bmatrix}$, it is sufficient to show $M \succ 0$. 

% Notice that $F, G, H$ are all diagonal and $M$ is Hermitian. Hence $M$ has $2n$ real eigenvalues. For $i \in [n]$, let $M^{(i)} = \begin{bmatrix}
% F_{ii} & -\frac{1}{2} H_{ii} \\
% -\frac{1}{2} H_{ii} & G_{ii} \end{bmatrix}$.

% The eigenvalues of $M$ are given by $\bigcup\limits_{i \in [n]} \{\lambda_1(M^{(i)}), \lambda_2(M^{(i)})\}$. Therefore, it suffices to show $M^{(i)} \succ 0$ for all $i$. The following conditions are necessary and sufficient: 
% \begin{align*}
% F_{ii} &> 0 \\
% G_{ii} &> 0 \\
% \frac{1}{4} H_{ii}^2 &\leq F_{ii} G_{ii}
% \end{align*}
% First, $F_{ii} > 0 \iff (1 + \mu)(\lambda_i + \alpha) > 0$. This is true as long as $\alpha > 0$. Next, letting $\kappa = \lambda$, 
% \begin{align*}
% G_{ii} &> 0 \\
% `\frac{\kappa \lambda_i + \kappa \alpha - 2 \alpha}{(1 + \lambda_i)^2} + \frac{2\alpha}{1 + \lambda_i} + (\kappa + \mu - 1) \alpha &> 0 \\
% \iff \kappa \lambda_i + \alpha \bigg(
% \kappa - 2 + 2 (1 + \lambda_i) + (\kappa + \mu - 1) (1 + \lambda_i)^2
% \bigg) &> 0
% \end{align*}
% Since the Laplacian is PSD, $(1 + \lambda_i) \geq 1$ for all $i$. Hence, a sufficient condition for $G_{ii} > 0$ is that $\alpha > 0$ and: 
% \begin{align*}
% \kappa - 2 + 2 + (\kappa + \mu - 1) &> 0 \\
% \iff 2 \kappa + \mu - 1 &> 0
% \end{align*}
% Since $\kappa \geq 1$ and $\mu \geq 0$, this is always true. Finally, 
% \begin{align*}
% \frac{1}{4} H_{ii}^2 &\leq F_{ii} G_{ii} \\
% \bigg(\frac{\lambda_i + 2 \alpha + 2 \mu \alpha (1 + \lambda_i)}{(1 + \lambda_i)^2}
% \bigg)^2 
% &\leq \bigg[\frac{(1 + \mu) (\lambda_i + \alpha)}{(1 + \lambda_i)^2} \\
% &\cdot 
% \frac{\kappa \lambda_i + (\kappa - 2) \alpha + 2 \alpha (1 + \lambda_i) + (\kappa + \mu - 1)\alpha (1 + \lambda_i)^2}{(1 + \lambda_i)^2} \bigg] \\
% \frac{(\lambda_i + 2 \alpha + 2 \mu \alpha (1 + \lambda_i))^2}{\lambda_i + \alpha} &\leq 
% (1 + \mu) \bigg(\kappa \lambda_i + (\kappa - 2) \alpha \\
% &+ 2 \alpha (1 + \lambda_i) + (\kappa + \mu - 1)\alpha (1 + \lambda_i)^2\bigg) \\
% \frac{(\lambda_i + 2 \alpha + 2 \mu \alpha (1 + \lambda_i))^2}{\lambda_i + \alpha} - 2 \alpha \lambda_i 
% &\leq \bigg[(1 - \mu)(1 + \mu) \frac{\kappa}{1 - \mu} \big(\lambda_i + \alpha + \alpha (1 + \lambda_i)^2\big) \\
% &- (1 - \mu) \alpha (1 + \lambda_i)^2 
% \bigg]\\
% % \bigg(\frac{1}{\lambda_i + \alpha + \alpha (1 + \lambda_i)^2}\bigg) 
% \bigg(\frac{(\lambda_i + 2 \alpha + 2 \mu \alpha (1 + \lambda_i))^2}{\lambda_i + \alpha} - 2 \alpha \lambda_i + (1 - \mu) \alpha (1 + \lambda_i)^2
% \bigg)
% &\leq (1 - \mu^2) \frac{\kappa}{1-\mu}\big(\lambda_i + \alpha + \alpha (1 + \lambda_i)^2 \big) 
% \end{align*}
% We conclude that it is sufficient for $\kappa, \mu$ to be such that: 
% \[
% \bigg(\lambda_i + \alpha + \alpha (1 + \lambda_i)^2 \bigg)^{-1}
% \bigg(\frac{(\lambda_i + 2 \alpha + 2 \mu \alpha (1 + \lambda_i))^2}{(\lambda_i + \alpha)(1 + \mu)} - \frac{2 \alpha \lambda_i}{(1 + \mu)} + \frac{1 - \mu}{1 + \mu} \alpha (1 + \lambda_i)^2 
% \bigg) \leq \kappa 
% \]
% Notice the LHS is $f(\mu,i)$. Choosing $\mu^* = \arg \inf_{\mu \in (0,1)} \max_{i \in [n]} f(\mu,i)$, and setting $\kappa = f(\mu^*, i^*)$
% ensures that $\frac{1}{4} H_{ii}^2 \leq F_{ii} G_{ii}$ for all $i$. Hence we conclude that $M \succ 0$.  
% \end{proof}

% \begin{cor}
% If $i^* = 1$, where the eigenvalues of $L$ are ordered $\lambda_1 \leq \dots \lambda_n$ then: 
% \[
% \pom \leq \frac{5}{2} + \eps
% \]
% for arbitrarily small $\eps > 0$. 
% \end{cor}
% \begin{proof}
% Since $L$ is a graph Laplacian, we have $L \bm{1} = \bm{0}$, so $\lambda_1 = 0$. We can simplify $f(\mu, 1)$ as: 
% \begin{align*}
% f(\mu, 1) &= \frac{- \mu + 4 \left(\mu + 1\right)^{2} + 1}{2 \left(\mu + 1\right)} \\
% \frac{d}{d\mu} f(\mu, 1) &= \frac{2 \mu^{2} + 4 \mu + 1}{\mu^{2} + 2 \mu + 1} \\
% \frac{d^2}{d\mu^2} f(\mu, 1) &= \frac{2}{\mu^{3} + 3 \mu^{2} + 3 \mu + 1}
% \end{align*}
% Notice there is no dependence on $\Tilde \alpha$. Since $\mu > 0$, $\frac{d^2}{d\mu^2} > 0$ for all $\mu$, so $\mu^* \in \{0, 1\}$. We can compute that $f(0, 1) = 5/2$ and $f(1,1) = 4$. Hence, by setting $\mu = \eps$ for $\eps > 0$, we conclude that $\pom \leq \frac{5}{2} + \eps$ for arbitrarily small $\eps > 0$. 
% \end{proof}
% \ajcomment{This is useless because $i^* \neq 1$.}

% \begin{prop}
% Let $g(\mu, \lambda) = \frac{1}{1-\mu}f(\mu, i)$ if $\lambda_i = \lambda$. We claim that for all $\lambda > 0$, $\Tilde \alpha \in (0,1)$ and $\mu \in (0,1)$ that: 

% 1. $\frac{\del g}{\del \mu} \neq 0$

% 2. $\frac{\del^2 g}{\del \mu^2} > 0$. 
% % Let $f(\mu, \lambda)$ be defined by letting $\lambda_i = \lambda$. For all $\Tilde \alpha, \mu \in (0,1)$, $\frac{df(\mu, \lambda)}{d\lambda} < 0$. 
% \end{prop}

% \begin{prop}
% For all $\Tilde \alpha, \mu \in (0,1)$ and $\lambda \geq 0$, we have $\frac{dg(\mu, \lambda)}{d\lambda} < 0$
% \end{prop}

% The above two propositions imply that $i^* = n$ and that the optimal $\mu^*$ is within $\{0,1\}$. We can read off: 
% \[
% g(\mu, \lambda) = \frac{\tilde \alpha \left(\tilde \alpha + \lambda\right) \left(2 \lambda + \left(\lambda + 1\right)^{2} \left(\mu - 1\right)\right) - \left(2 \tilde \alpha \mu \left(\lambda + 1\right) + 2 \tilde \alpha + \lambda\right)^{2}}{\left(\tilde \alpha + \lambda\right) \left(\mu - 1\right) \left(\mu + 1\right) \left(\tilde \alpha \left(\lambda + 1\right)^{2} + \tilde \alpha + \lambda\right)}
% \]
% From this, it is clear that $\lim\limits_{\mu \to 1} g(\mu, \lambda) = \infty$. Therefore $\mu^* = 0$. 
% \ajcomment{Weird things happening, somehow we get that the PoM is at most 1 numerically...this shouldn't happen. I will revisit the math with $\alpha_i$ differing and see what happens.}

% % \begin{proof}
% % This can be read off from: 
% % \[
% % \frac{- \alpha \left(\lambda + 1\right) \left(\mu - 1\right) \left(\mu + 1\right) \left(8 \alpha \mu \left(\lambda + 1\right) + 8 \alpha + 4 \lambda - \left(\alpha + \lambda\right) \left(\lambda + 1\right)\right) - \left(\mu - 1\right) \left(\alpha \left(\alpha + \lambda\right) \left(2 \lambda + \left(\lambda + 1\right)^{2} \left(\mu - 1\right)\right) - \left(2 \alpha \mu \left(\lambda + 1\right) + 2 \alpha + \lambda\right)^{2}\right) - \left(\mu + 1\right) \left(\alpha \left(\alpha + \lambda\right) \left(2 \lambda + \left(\lambda + 1\right)^{2} \left(\mu - 1\right)\right) - \left(2 \alpha \mu \left(\lambda + 1\right) + 2 \alpha + \lambda\right)^{2}\right)}{\left(\alpha + \lambda\right) \left(\mu - 1\right)^{2} \left(\mu + 1\right)^{2} \left(\alpha \left(\lambda + 1\right)^{2} + \alpha + \lambda\right)}
% % \]
% % \end{proof}

% \ajcomment{This can be shown for $f/(1-\mu)$ as well.}

% Therefore, we know that for any $\mu$, that either $i^* = n$. 
% % \clearpage





% \subsection{Differing alpha}

% \begin{theorem}
% Suppose that agent $i$ has $\alpha_i \in (0,1)$. Then, the POM is bounded as blah...
% \ajcomment{Todo finish the calculation.}
% \label{thrm:pomgeneral}
% \end{theorem}

% \begin{proof}
% Let $B = ((I - A) L + A)^{-1} A$. Notice $z^\prime = Bs^\prime$. 

% First, it can be shown that: 
% \begin{align*}
% C(z^\prime)
% &= (z^\prime - \bs)^T A (z^\prime - s)
% + (z^\prime)^T (I - A)D z^\prime - 2(z^\prime)^T (I - A) W z^\prime
% + \sum_{i \in [n], j \in [n]} w_{ij} z_j^2 (1 - \alpha_i)
% \end{align*}

% Here $D$ is the degrees matrix and $W$ the weighted adjacency matrix. 

% We want to analyze for $\kappa > 1$ and $\mu \in (0,1)$, 
% \begin{align*}
% (1 - \mu) C(s^\prime) + (s - s^\prime)^T \nabla_{s^\prime} C(s^\prime) 
% \leq \kappa C(s)
% \end{align*}

% Let $M_1 = 2B^T AB + 2B^T (I - A) D B - 4 B^T (I - A) B$. 

% First, we argue that: 
% \begin{align*}
% \nabla_{s^\prime} C(s^\prime) 
% &= M_1 s^\prime - 2 B^T A s
% + 2 \sum_{ij} w_{ij} (1 - \alpha_i) (B^T e_j e_j^T B) s^\prime 
% \end{align*}
% Letting $M_2 = 2 \sum_{ij} w_{ij} (1 - \alpha_i) (B^T e_j e_j^T B)$, we have that
% \begin{align*}
% \nabla_{s^\prime} C(s^\prime) 
% &= (M_1 + M_2) s^\prime - 2 B^T A s
% \end{align*}
% So, 
% \begin{align*}
% (s - s^\prime)^T \nabla_{s^\prime} C(s^\prime) 
% &= s^T (M_1 + M_2 + 2 AB^T) s^\prime 
% + s^T (-2B^T A) s 
% + (s^\prime)^T (- M_1 - M_2) s^\prime 
% \end{align*}
% Next, let $D_2$ be the diagonal matrix with $D_{2;j} = \sum_{i=1}^{n} W_{ij} (1 - \alpha_i)$. Then, 
% \begin{align*}
% C(z^\prime) = z^\prime 
% \big( A + (I - A) D - 2(I - A)W + D_2
% \big) z^\prime  
% + s^T (-AB - B^T A) z^\prime 
% + s^T A s
% \end{align*}

% Therefore, let $M_3 = A + (I - A) D - 2(I - A)W + D_2$. Then, 
% \begin{align*}
% C(s^\prime) &= (s^\prime)^T B^T M_3 B s^\prime 
% + s^T (- AB - B^T A)B s^\prime 
% + s^T A s \\ 
% C(s) &= s^T B^T M_3 B s
% + s^T (- AB - B^T A)B s
% + s^T A s \\ 
% \end{align*}

% Let $M_4 = B^T M_3 B - AB^2 - B^T AB + A$. Then, we have: 

% \begin{align}
% \begin{bmatrix} 
% s^\prime \\ s
% \end{bmatrix}^T 
% \begin{bmatrix}
% F &  H^T \\
% H & G \end{bmatrix}
% \begin{bmatrix} 
% s^\prime \\ s
% \end{bmatrix} > 0
% \label{eq:sufficient}
% \end{align}

% Where the matrices $F, G, H$ are defined as: 
% \begin{align*}
% F &= M_1 + M_2 - (1-\mu) B^T M_3 B \\
% G &= \kappa M_4 + 2 B^T A + (1-\mu) A \\
% H &= \frac{1}{2} ((1 - \mu) AB^2 + (1 - \mu) B^T A B - M_1 - M_2 - 2 AB^T )
% \end{align*}

% Now, notice that $s$ is fixed. We only need to show that Eq~\ref{eq:sufficient} holds for all $s^\prime$. Let $f(s^\prime)$ be equal to the Eq.~\ref{eq:sufficient} when $s$ is fixed. Then, we see that $\nabla_{s^\prime}f(s^\prime) = (F + F^T)s^\prime + 2 H^T s$, and that $\nabla_{s^\prime}^2 f(s^\prime) = F + F^T$. Therefore if $F + F^T \succeq 0$, it is sufficient to show that $f(s^\prime) \geq 0$ for $s^\prime = (F + F^T)^{-1} 2H^T s$. 

% To this end, notice: 
% \begin{align*}
% F + F^T = B^T \bigg(
% (14 - 2\mu) A + (4 - 2\mu) D_2 + 4D - 4 AD - 8I 
% + (2 - 2 \mu)(I - A) L - (2 - 2 \mu) W (I - A)
% \bigg) B
% \end{align*}
% % This can be shown to be PD through Gershgorin disc theorem and for large enough $A, D$. 

% Next, let 
% \[
% M_5 := (14 - 2\mu) A + (4 - 2\mu) D_2 + 4D - 4 AD - 8I 
% + (2 - 2 \mu)(I - A) L - (2 - 2 \mu) W (I - A)
% \]
% By Lemma~\ref{lemma:fftpd}, $M_5 \succeq 0$. This immediately implies $F + F^T \succeq 0$. 
% % Next, let the middle term of $F + F^T$ be $M_5$. 

% Next, let $H_0 = 2H$ and $M_6 = (F + F^T)^{-1}$. Let $s^\prime = (F + F^T)^{-1} 2H^T s$. 
% \begin{align*}
% f(s^\prime) &= \frac{1}{2} (s^\prime)^T (F + F^T) s^\prime + 2 (s^\prime)^T H^T s + s^T G s \\
% &= \frac{1}{2} s^T H_0 M_6 H_0^T s 
% + s^T H_0 M_6 H_0^T s + s^T G s \\
% &= s^T \bigg(
% \frac{3}{2} H_0 M_6 H_0^T + G 
% \bigg) s
% \end{align*}
% Next, let $B_0 = (I - A)L + A$. Notice $B = B_0^{-1} A$, so: 
% \begin{align*}
% M_6 &= (F + F^T)^{-1} \\
% &= A^{-1} B_0 M_5^{-1} B_0^T A^{-1} \\
% &= ((A^{-1} - I)L + I) M_5^{-1} (L (A^{-1} - I) + I)
% \end{align*}
% By Lemma~\ref{lemma:fftpd}, we know $M_5 \succ 0$ so $M_5^{-1} \succ 0$. We immediately obtain $M_6 \succ 0$, so $H_0 M_6 H_0^T \succeq 0$. 
% \ajcomment{Justify the step with $L$ here.}


% Hence, since $s^T G s = \frac{1}{2}s^T (G + G^T) s$ a sufficient condition is to now show $s^T (G + G^T) s \geq 0$. 
% Next, we analyze $G$. Notice that: 
% \begin{align*}
% G &\succeq 0 \\
% \iff \kappa B^T M_3 B + (\kappa + 1 - \mu) A + 2 B^T A &\succeq \kappa (AB^2 + B^T A B) 
% \end{align*}
% Moreover, let $g = \sum_i \frac{1-\alpha_i}{\alpha_i} D_{ii}$. By Corollary~\ref{cor:suminv}, we have: 
% \begin{align*}
% B^T M_3 B &= A^{-1} (B_0^T)^{-1} (B_0 + D_2 - (I - A) W) B_0^{1} A \\
% &= A (B_0^T)^{-1} A +  A (B_0^T)^{-1}(D_2 - (I - A) W) B_0^{-1} A \\
% &= A \bigg[
% \big(A^{-1} - \frac{1}{1+g} A^{-1} L(I - A)A^{-1}\big)
% \big(I + (D_2 - (I - A)W)\big)
% \big(
% A^{-1} - \frac{1}{1 + g} A^{-1} (I - A)L A^{-1}\big)
% \bigg] A
% \end{align*}

% Next, 
% \begin{align*}
% AB^2 + B^T A B &= A( (B_0^{-1} + (B_0^T)^{-1}) A B_0^{-1} )A \\
% &= A \bigg[
% A^{-1} - \frac{1}{1 + g} A^{-1} \big(
% (I - A)L + L(I - A)\big) A^{-1}
% \bigg] A 
% \Bigg[
% A^{-1} - \frac{1}{1 + g} A^{-1} (I - A)L A^{-1}
% \bigg] A \\
% &= \Bigg[
% I - \frac{1}{1 + g} \big(
% (I - A)L + L(I - A)\big) A^{-1}
% \bigg]
% \bigg[
% A - \frac{1}{1 + g}(I - A)L
% \bigg]
% \end{align*}
% Next, let $M_7 = B^T M_3 B - (AB^2 + B^T AB)$. Let $h = \frac{1}{1 + g}$ for shorthand and $\barr{A} = I - A$. We want to show $G \succeq 0$. A sufficient condition for this is: 
% \[
% \frac{\kappa}{1-\mu} M_7
% + A + \frac{2}{1-\mu} I 
% \succeq \frac{2}{1-\mu} (h \barr{A} L)
% \]

% Fix some unit vector $v \in \RR^n$. First, notice if $v = \frac{1}{\sqrt{n}} \bm{1}$ then: 
% \begin{align*}
% v^T (\frac{\kappa}{1-\mu} M_7
% + A + \frac{2}{1-\mu} I ) v &= 
% v^T A v + \frac{2}{1-\mu}
% + v^T (\frac{\kappa}{1-\mu} M_7)v \\
% &= v^T (\frac{\kappa}{1-\mu} M_7)v
% + \frac{2}{1-\mu} + \frac{1}{n}\sum_i \alpha_i \\ 
% v^T ( \frac{2}{1-\mu} h \barr{A} L) v &= 0
% \end{align*}
% Therefore, it suffices to analyze $v$ such that $v \perp \bm{1}$. \ajcomment{Come back and plug in the $\lambda = 0$ case.}
% These are the eigenvectors of $L$, so $Lv =\lambda v$ for some $\lambda \geq 0$ (assuming the graph is connected). 

% Now, 
% \begin{align*}
% v^T (\frac{\kappa}{1-\mu} M_7
% + A + \frac{2}{1-\mu} I ) v
% &\geq v^T ( \frac{2}{1-\mu} h \barr{A} L) v \\
% \kappa v^T M_7 v + 2 + (1 - \mu) v^T A v 
% &\geq 2\lambda h v^T (A^{-1} - I) v 
% \end{align*}
% Next, 
% \begin{align*}
% v^T M_7 v &= 
% (v^T - h\lambda v^T \barr{A}A^{-1})
% (v - h\lambda v \barr{A} A^{-1})
% + (v^T - h\lambda v^T \barr{A}A^{-1})
% (D_2 - \barr{A}W)
% (v - h\lambda v \barr{A} A^{-1}) \\
% &+ 3h\lambda v^T \barr{A} v - v^T A v 
% + \lambda h^2 v^T (\barr{A} L + L \barr{A})v
% - \lambda h^2 v^T (\barr{A} L + L \barr{A})A^{-1} v \\
% &= (v - h\lambda v \barr{A} A^{-1})^T (v - h\lambda v \barr{A} A^{-1})
% + (v - h\lambda v \barr{A} A^{-1})^T
% (D_2 - \barr{A}W)
% (v - h\lambda v \barr{A} A^{-1}) \\
% &+  3h\lambda v^T \barr{A} v - v^T A v 
% + 2 \lambda^2 h^2 v^T \barr{A} v
% - \lambda^2 h^2 v^T \barr{A} A^{-1} v 
% - \lambda h^2 v^T \barr{A} L A^{-1} v \\
% &= (v - h\lambda v \barr{A} A^{-1})^T
% \bigg(
% I + D_2 - \barr{A} W 
% \bigg) (v - h\lambda v \barr{A} A^{-1})\\
% &+ v^T \bigg(
% 3h\lambda \barr{A} - A + 2 \lambda^2 h^2 \barr{A} - \lambda^2 h^2 \barr{A}A^{-1} - \lambda h^2 \barr{A} L A^{-1}
% \bigg) v
% \end{align*}
% It suffices to show analyze the matrices in the middle of each quadratic form. 

% First, by the Gershgorin disc theorem, a sufficient condition for $I + D_2 - \barr{A} W  \succeq 0$ is that for all $j$, 
% \begin{align*}
% 1 + \sum_{k \neq j} (\alpha_j - \alpha_k) W_{kj} \geq 0
% \end{align*}
% In other words, if $\alpha_j < \alpha_k$, then $W_{kj}$ must be relatively small. In particular, if $j$ is the person with the least $\alpha_j$, then they must have small degree. 

% % Next, to analyze the other matrix, notice that 
% % \begin{align*}
% % v^T \barr{A} L A^{-1} v &= (1 - \alpha_{max}) (1 / \alpha_{max}) \lambda^2 
% % + v^T (\barr{A} - (1 - \alpha_{max}) I) L (A^{-1} - (1 / \alpha_{max}) I)v \\
% % &\leq (1 - \alpha_{max}) (1 / \alpha_{max}) \lambda^2
% % + (\alpha_{max} - \alpha_{min}) \lambda_n^2 (\frac{1}{\alpha_{min}} - \frac{1}{\alpha_{max}}) 
% % \end{align*}
% % Where $\lambda_n$ is the maximum eigenvalue. Notice that $\lambda_n \leq 2 \max_{i} D_{ii}$. 
% Now, we want to show that: 
% \begin{align*}
% \kappa v^T \bigg(
% 3h\lambda \barr{A} - A + 2 \lambda^2 h^2 \barr{A} - \lambda^2 h^2 \barr{A}A^{-1} - \lambda h^2 \barr{A} L A^{-1}
% \bigg) v
% + v^T \bigg(
% 2 I + (1-\mu)A + 2 \lambda h (I - A^{-1})
% \bigg) v
% &\geq 0
% \end{align*}
% A sufficient condition is that for all $i$, 
% \begin{align*}
% \kappa &\geq \frac{2 \lambda h (\frac{1}{\alpha_i} - 1) - 2 - (1-\mu)\alpha_i}{\alpha_i 
% + (\frac{1}{\alpha_i} - 2)\lambda^2 h^2 (1 - \alpha_i) 
% + \lambda h^2 \norm \barr{A}LA^{-1}\norm_2
% - 3 h \lambda (1-\alpha_i)
% }
% \end{align*}
% Therefore, it remains to upper bound the RHS. Notice since $\alpha_i > 0$ that we can set $\mu > 0$ arbitrarily small. Therefore, it remains to upper bound: 
% \begin{align*}
% \frac{2 \lambda h (\frac{1}{\alpha_i} - 1)}{\alpha_i 
% + (\frac{1}{\alpha_i} - 2)\lambda^2 h^2 (1 - \alpha_i) 
% + \lambda h^2 \norm \barr{A}LA^{-1}\norm_2
% - 3 h \lambda (1-\alpha_i)
% }
% \end{align*}

% % \[
% % v^T (3h\lambda \barr{A} - A + 2 \lambda^2 h^2 \barr{A} - \lambda^2 h^2 \barr{A}A^{-1} - \lambda h^2 \barr{A} L A^{-1}) v \geq 0
% % \]

% \ajcomment{Now combine back with the $\kappa$ stuff to get a polynomial in $h$, $\lambda$, and the $\alpha_i$.}

% A sufficient condition for the above display is that for all $i$, 
% \begin{align*}
% \kappa &\geq \frac{2(1-\alpha_i)}{(\lambda h)^{-1} \alpha_i^2 + (1 - 2\alpha_i) \lambda h (1 - \alpha_i) + h\lambda \norm \barr{A} L A^{-1}\norm_2 - 3\alpha_i (1-\alpha_i)}
% \end{align*}
% %Suppose $L = \sum_i \lambda_i v_i v_i^T$. Then we can write $s$ in the Laplacian eigenbasis as $s = \sum_i q_i v_i$. 

% % \ajcomment{Todo finish.}
% \end{proof}
% % Notice that by the Gershgoring disc theorem, that the eigenvalues of $D_2 - (I - A) W$ are bounded within $\bigcup\limits_{j \in [n]} \bigg(\sum_{i} W_{ij} (1 - \alpha_i) \pm \sum_{i} \abs{W_{ij} (1 - \alpha_j)}\bigg)$. Since all weights are non-negative $W_{ij} \geq 0$, 
% % we see that a sufficient condition for eigenvalues to be bounded within $\pm C$ is that: 
% % \[
% % \forall j: 
% % \bigg[\frac{- C + \sum_{i} W_{ij} \alpha_i }{\sum_{i} W_{ij}} \leq 
% % \alpha_j \leq \frac{C + \sum_{i} W_{ij} \alpha_i}{\sum_{i} W_{ij}}\bigg]
% % \]


% % $M_6 = (F + F^T)^{-1} = A^{-1} B_0 M_5^{-1} B_0^T A^{-1}$

% % Now, it remains to show that if $K = \begin{bmatrix}
% % F &  H^T \\
% % H & G \end{bmatrix}$ then $K \succeq 0$. We can analyze this through Schur complements. Notice that $M_2$ is diagonal, $M_1$ is symmetric. However $M_3$ fails to by symmetric...
% \begin{lemma}[Sherman-Morrison]
% Let $A, B$ be square matrices such that $A^{-1}$ exists. Then if $g = tr(BA^{-1})$, then: 
% \[
% (A + B)^{-1} = A^{-1} - \frac{1}{1 + g} A^{-1} B A^{-1}
% \]
% \end{lemma}

% \begin{cor}
% Let $B_0 = (I - A)L + A^{-1}$. Then, 
% \begin{align*}
% B_0^{-1} &= A^{-1} - \frac{1}{1 + g_1} A^{-1} (I - A)L A^{-1} \\
% (B_0^T)^{-1} &= A^{-1} - \frac{1}{1 + g_2} A^{-1} L(I - A) A^{-1}
% \end{align*}
% Where $g_1 = g_2 = \sum_i \frac{1-\alpha_i}{\alpha_i} D_{ii}$. 
% \label{cor:suminv}
% \end{cor}

% \begin{lemma}
% Let $M_5$ be as in the proof of Theorem~\ref{thrm:pomgeneral}. 
% \[
% M_5 := (14 - 2\mu) A + (4 - 2\mu) D_2 + 4D - 4 AD - 8I 
% + (2 - 2 \mu)(I - A) L - (2 - 2 \mu) W (I - A)
% \]
% Let $d_{min} = \min_i D_{ii}$ and $\alpha_{min} = \min_i \alpha_i$. 
% Then $M_5 \succeq 0$ if 
% \[
% \mu \leq \alpha_{min}(7 + d_{min}) + (d_{min} - 4)
% \]
% If the RHS above is negative, then any $\mu \in (0,1)$ ensures that $M_5 \succeq 0$. 
% \label{lemma:fftpd}
% \end{lemma}

% \begin{proof}
% First, notice that since $L = D - W$, that: 
% \[
% M_5 := (14 - 2\mu) A + (4 - 2\mu) D_2 + 4D - 4 AD - 8I 
% + (2 - 2 \mu)(I - A) L 
% + (2 - 2\mu) L (I - A)
% - (2 - 2 \mu) D (I - A)
% \]
% Now, since $4D - 4AD \succeq 0$, it suffices to show that: 
% \begin{align*}
% (14 - 2\mu) A + (4 - 2\mu) D_2 
% + (2 - 2 \mu)((I - A) L + ((I - A) L)^T)
% \succeq 8I + (2 - 2\mu) D(I - A)
% \end{align*}
% Notice all but one of the matrices here is diagonal. Moreover, since $L \succeq 0$ and $(I - A) \succ 0$, we know $(2 - 2 \mu)((I - A) L + ((I - A) L)^T) \succeq 0$. Therefore, considering the $(i,i)$ term of the remaining diagonal matrices, we see that a sufficient condition is that for all $i$, 
% \begin{align*}
% \mu &\leq \frac{7 \alpha_i + 2 D_{2;ii} - 4 - D_{ii}(1-\alpha_i)}{\alpha_i + D_{2;ii} - D_{ii}(1-\alpha_i)} \\
% \iff (7 - \mu)\alpha_i + (2 - \mu) D_{2;ii} - 4 - D_{ii}(1 -\alpha_i) 
% &\geq -\mu D_{ii}(1-\alpha_i) \\
% \iff 
% (7 - \mu)\alpha_i 
% + \sum_{j \in [n]}\bigg[
% (2 - \mu)(1-\alpha_j) - (1 - \mu)(1-\alpha_i)
% \bigg] W_{ij} &\geq 4 \\
% \iff (7 - \mu)\alpha_i 
% + \sum_{j \in [n]}\bigg[
% (1-\mu)(\alpha_i - \alpha_j)
% + (1 - \alpha_j)
% \bigg] W_{ij} &\geq 4 
% \end{align*}

% A sufficient condition for the above is that: 
% \begin{align*}
% (7 - \mu)\alpha_{min} + (1-\mu)\alpha_{min} D_{ii} + \sum_{j \in [n]}
% \bigg[
% (1 - \alpha_j) - (1 - \mu)\alpha_j
% \bigg] W_{ij} &\geq 4 \\
% \iff (7 - \mu)\alpha_{min} + (1-\mu)\alpha_{min} D_{ii} + \sum_{j \in [n]}
% \bigg[
% 1 + \mu \alpha_j 
% \bigg] W_{ij} &\geq 4 \\
% \iff (7 - \mu)\alpha_{min} + (1-\mu)\alpha_{min} D_{ii} + (1 + \mu \alpha_{min}) D_{ii}  &\geq 4 \\
% \iff 
% (7 - \mu)\alpha_{min} + \alpha_{min}D_{ii}
% + D_{ii} &\geq 4 \\
% \iff \mu \leq \alpha_{min}(7 + D_{ii}) + (D_{ii} - 4)
% \end{align*}
% The conclusion follows by taking the minimum across $i$. 
% \end{proof}


% \newpage

% \mpcomment{an alternative approach}
% \ajcomment{Nice, I like this. I have to check the algebra but I think that this generally makes sense since $\lambda_n \leq 2 d_{max}$, so it scales in terms of susceptbility and degree. I will see if I can try to generalize this idea. It seems like you're sidestepping the local smoothness technique by directly analyzing the cost in both cases. I wonder if this can work, since the eigenvectors in the general $\alpha_i$ case do not correspond to Laplacian eigenevectors.}


\section{Computational lower bound for learning stochastic block model}\label{sec:lb-learning}

\subsection{Computational lower bound for learning the edge connection probability matrix}

In this section, we prove \cref{thm:lb-edge-probability} by showing that there exists an efficient algorithm that reduces testing to learning in SBM. 
The reduction of algorithm \cref{alg:reduction-test-learning} is similar to that of \cref{alg:reduction-test-recovery}. The proof of \cref{thm:lb-edge-probability} is also a similar proof by contradiction to the proof of \cref{thm:main-theorem-weak-recovery}.

Before describing the algorithm, we restate \cref{thm:lb-edge-probability} here for completeness.
\begin{theorem}[Restatement of \cref{thm:lb-edge-probability}]
\label{thm:lb-edge-probability-restatement}
    Let $k,d\in \N^+$ be such that $k\leq n^{o(1)}, d\leq o(n)$.
    Assume that for any $d'\in \N^+$ such that $0.999 d\leq d'\leq d$, Conjecture \ref{conj:eldlr} holds with distribution $P$ given by $\SSBM(n,\frac{d'}{n},\e,k)$ and distribution $Q$ given by \Erdos-\Renyi graph model $\bbG(n, \frac{d'}{n})$. 
    Then given graph $G\sim \SSBM(n,\frac{d}{n},\e,k)$, no $\exp\Paren{n^{0.99}}$ time algorithm can output $\theta\in [0,1]^{n\times n}$ achieving error rate $\normf{\theta-\thetanull}^2\leq 0.99kd/4$ with constant probability, where $\thetanull$ is the ground truth sampled edge connection probability matrix.
\end{theorem}

The reduction that we consider is the following.

\begin{algorithmbox}[Reduction from testing to learning]
    \label{alg:reduction-test-learning}
    \mbox{}\\
    \textbf{Input:} A random graph $G$ with equal probability sampled from \Erdos-\Renyi model or stochastic block model. \\
    \textbf{Output:} Testing statistics $g(Y)\in \R$, where $Y$ is the centered adjacency matrix\\
    \textbf{Algorithm:} 
    \begin{enumerate}[1.]
        \item Obtain subgraph $G_1$ by subsampling each edge with probability $1-\eta=0.999$, and let $G_2= G\setminus G_1$. 
        \item Run learning algorithm on $G_1$, and obtain estimator $\hat{\theta}\in \R^{n\times n}$
        \item Obtain $\hat{M}$ by running correlation preserving projection on $\hat{\theta}-\frac{d}{n}\Ind \Ind^{\top}$ to the set $\cK=\Set{M\in [-1,1]^{n\times n}: M+\frac{1}{k} \Ind \Ind^{\top} \succeq 0 \,, \Tr(M + \frac{1}{k} \Ind \Ind^{\top}) \leq n}$. 
        \item Construct the testing statistics $g(Y)=\iprod{\hat{M},Y_2-\frac{\eta d}{n}\Ind \Ind^{\top}}$, where $Y_2$ is the adjacency matrix for the graph $G_2$.
    \end{enumerate}
\end{algorithmbox}

Before proving \cref{thm:lb-edge-probability}, we first show the relationship between learning edge connection probability and weak recovery.
 \begin{lemma}\label[lemma]{lem:reduction-learning-recovery}
     Consider the distribution of $\SSBM(n,\frac{d}{n},\e,k)$ with $d\le n^{o(1)}$. 
     Suppose give graph $Y\sim \SSBM(n,\frac{d}{n},\e,k)$, the estimator $\hat{\theta}\in \R^{n\times n}$ achieves error rate $\normf{\hat{\theta}- \thetanull}\leq \frac{1}{2}\sqrt{0.99kd}$ with constant probability, then $\hat{\theta}-d/n$ achieves weak recovery when $\e^2 d\geq 0.99k^2$.
 \end{lemma}
\begin{proof}
By the relation between edge connection probability matrix $\thetanull$ and the community matrix $M^\circ$, We have
    \begin{equation*}
        \iprod{\hat{\theta}-\frac{d}{n}\Ind \Ind^\top,M^\circ}=\iprod{\hat{\theta}-\theta^\circ,M^\circ}+\iprod{\theta^\circ-\frac{d}{n}\Ind \Ind^\top,M^\circ}=\iprod{\hat{\theta}-\theta^\circ,M^\circ}+\iprod{\frac{\e d}{n}M^\circ,M^\circ}\,.
    \end{equation*}
    For the first term, since with constant probability, $\normf{\hat{\theta}-\theta^\circ}\leq \sqrt{0.99kd}$, we have
    \begin{equation*}
      \Abs{\iprod{\hat{\theta}-\theta^\circ,M^\circ}}\leq \normf{M^\circ}\normf{\hat{\theta}-\theta^\circ}\leq 
        \normf{M^\circ} \sqrt{0.99kd}\,.
    \end{equation*}
    For the second term, since with overwhelming high probability, $\normf{M^\circ}\geq \frac{n}{\sqrt{k}}(1-\frac{1}{k})$, we have
    \begin{equation*}
        \iprod{\frac{\e d}{n}M^\circ,M^\circ}=\frac{\e d}{n}\normf{M^\circ}^2\geq \frac{\e d }{2\sqrt{k}} \normf{M^\circ}\,.
    \end{equation*}
    Therefore, when $\e^2 d> 0.999 k^2$, we have 
    \begin{equation*}
        \iprod{\hat{\theta}-\frac{d}{n}\Ind \Ind^{\top},M^\circ}\geq \frac{\e d }{2\sqrt{k}} \normf{M^\circ}-\normf{M^\circ} \frac{\sqrt{0.99kd}}{2}\geq \Omega\Paren{\frac{\e d \normf{M^\circ}}{\sqrt{k}}} \,.
    \end{equation*}
    On the other hand, by triangle inequality
    \begin{equation*}
        \Normf{\hat{\theta}-\frac{d}{n}\Ind \Ind^{\top}}\leq  \Normf{\hat{\theta}-\theta^\circ}+ \Normf{\theta^\circ-\frac{d}{n}\Ind \Ind^{\top}}\leq O(\sqrt{kd}+\frac{\e d}{\sqrt{k}}) \leq O\Paren{\e d/\sqrt{k}}\,,
    \end{equation*}
Therefore we have 
\begin{equation*}
    \iprod{\hat{\theta}-\frac{d}{n}\Ind \Ind^{\top},M^\circ}\geq \Omega(\normf{M^\circ}\cdot \normf{\hat{\theta}-\frac{d}{n}\Ind \Ind^{\top}})\,.
\end{equation*}
    We thus conclude that with constant probability, $\hat{\theta}-\frac{d}{n}\Ind \Ind^\top$ achieves weak recovery when $\e^2 d\geq 0.99k^2$.
\end{proof}
With \cref{lem:reduction-learning-recovery}, the proof of lower bound for learning the edge connection probability matrix of stochastic block model follows as a corollary.
\begin{proof}[Proof of \cref{thm:lb-edge-probability}]
    By \cref{lem:reduction-learning-recovery}, suppose an $\exp\Paren{n^{0.99}}$ time algorithm achieves error rate less than $0.99\sqrt{kd}$ in estimating the edge connection probability matrix, then in \cref{alg:reduction-test-learning}, $\hat{\theta}-\frac{d}{n}$ achieves weak recovery when $\e^2 d=0.99k^2$.
    We let $f(Y)=\mathbf{1}_{g(Y)\geq 0.001 \e^2 d^2/k}$. 

    We show that with constant probability under $P$, we have $f(Y)=1$.    
    We essentially follow the proof of \cref{lem:lb_sbm} with $\delta$ taken as a constant, except that we take a different strategy for bounding
    $\iprod{W_2-\tilde{W}_2, \hat{M}}$.
    By \cref{lem:spectral-concentration-sbm}, we have, with probability at least $1-o(1)$, the following spectral radius bounds on the symmetric random matrices
\begin{equation*}
    \normop{W_2-\tilde{W}_2}\leq O\Paren{\sqrt{d\log(n)}\cdot \sqrt{\frac{d}{n}}}\,.
\end{equation*}
Therefore, by Trace inequality, we have
\begin{equation*}
\begin{split}
|\iprod{W_2-\tilde{W}_2, \hat{M}}|
& = |\iprod{W_2-\tilde{W}_2, \hat{M}+\frac{1}{k\delta}\Ind \Ind^{\top}} - \iprod{W_2-\tilde{W}_2, \frac{1}{k\delta}\Ind \Ind^{\top}}| \\
& \leq |\iprod{W_2-\tilde{W}_2, \hat{M}+\frac{1}{k\delta}\Ind \Ind^{\top}}| + |\iprod{W_2-\tilde{W}_2, \frac{1}{k\delta}\Ind \Ind^{\top}}| \\
& \leq \normop{W_2-\tilde{W}_2} \Tr(\hat{M}+\frac{1}{k\delta}\Ind \Ind^{\top}) + \normop{W_2-\tilde{W}_2} \Tr(\frac{1}{k\delta}\Ind \Ind^{\top}) \\
& \leq O\Paren{\sqrt{d\log(n)}\cdot \sqrt{\frac{d}{n}} (1+\frac{1}{k})\frac{n}{\delta}}\\
& = O\Paren{(d+\frac{d}{k})\frac{\sqrt{n\log(n)}}{\delta}} \,.
\end{split}
\end{equation*}

    With the same reasoning, by \cref{lem:ub_ER}, with probability at least $1-\exp(-n^{0.001})$ under distribution $Q$, we have $f(Y)=0$. 
    Therefore, we have $\RPQ(f)\geq \exp(n^{0.001})$. 
    Since $f(A)$ can be evaluated in $O\Paren{\exp\Paren{n^{0.99}}}$ time, assuming conjecture \ref{conj:low-degree} we have
   \begin{equation*}
       R_{P,Q}(f)\coloneqq \frac{\E f(A)}{\sqrt{\text{Var}_Q(f(A))}} \lesssim \max_{\text{deg}(f)\leq n^{0.99}}\frac{\E f(A)}{\sqrt{\text{Var}_Q(f(A))}}\,.
   \end{equation*}
    On the other hand, by low-degree lower bound stated in \cref{thm:ldlr-sbm}, we have 
    \begin{equation*}
       \max_{\text{deg}(f)\leq n^{0.99}}\frac{\E f(A)}{\sqrt{\text{Var}_Q(f(A))}}\leq \exp(k^2)\,. 
    \end{equation*}
Since we have $\exp(n^{0.001})\gg\exp(k^2)$ when $k\leq n^{o(1)}$, this leads to a contradiction. 
\end{proof}

\subsection{Computational lower bound for learning graphon}
In this part, we give formal proof of \cref{thm:lb-learning-graphon}. 

\begin{theorem}[Restatement of \cref{thm:lb-learning-graphon}]
    Let $k,d\in \N^+$ be such that $k\leq O(1), d\leq o(n)$.
    Assume that Conjecture \ref{conj:low-degree} holds with distribution $P$ given by $\SSBM(n,\frac{d}{n},\e,k)$ and distribution $Q$ given by \Erdos-\Renyi graph model $\bbG(n, \frac{d}{n})$. 
    Then no $\exp\Paren{n^{0.99}}$ time algorithm can output a $\poly(n)$-block graphon function $\hat{W}:[0,1]\times [0,1]\to [0,1]$ such that $\GW(\hat{W},\Wnull) \leq \frac{d}{3n}\sqrt{\frac{k}{d}}$  with $1-o(1)$ probability under distribution $P$ and distribution $Q$(where $\Wnull$ is the underlying graphon of the corresponding distribution).
\end{theorem}
\begin{proof}
Let $W_0$ be the graphon function underlying the distribution $\bbG(n,\frac{d}{n})$ and $W_1$ be the graphon function underlying the distribution $\SSBM(n,\frac{d}{n},\e,k)$, we have $\GW(W_0,W_1)\geq \frac{d}{n}\sqrt{\frac{0.99k}{d}}$ when $\e^2 d\geq 0.99k^2$. 

Now suppose there is a polynomial time algorithm, which given random graph $G$ sampled from an arbitrary symmetric $k$-stochastic block model, outputs an $n$-block graphon function $\hat{W}:[0,1]\times [0,1]\to [0,1]$ achieving error $\frac{d}{3n}\sqrt{\frac{k}{d}}$ with probability $1-o(1)$.
Then one can construct the testing statistics by taking
\begin{equation*}
f(Y) =
\begin{cases}
    1, & \text{if } \GW(\hat{W}, W_0) \leq \frac{d}{3n} \sqrt{\frac{k}{d}} \\
    0, & \text{otherwise}
\end{cases}
\end{equation*}
We have $f(Y)=1$ with probability $1-o(1)$ under the distribution of symmetric stochastic block model $\SSBM(n,\frac{d}{n},\e,k)$.
By triangle inequality, we have $f(Y)=0$ with probability $1-o(1)$ under the distribution $\bbG(n,\frac{d}{n})$. 
Therefore we have $\RPQ(f)\geq \omega(1)$.

Now since the function $\hat{W}$ can be represented as a symmetric matrix with $\poly(n)$ number of rows and columns, and moreove since $W_0$ is a constant function,
\begin{equation*}
    \GW(\hat{W},W_0)= \int_0^1 \int_0^1 (\hat{W}(x,y)-W_0(x,y))^2 dx dy\,.
\end{equation*}
Therefore, the function $f(\cdot )$ can be evaluated in polynomial time. 
This contradicts the low-degree lower bound (\cref{thm:ldlr-sbm}) assuming \cref{conj:low-degree}.
\end{proof}


















\section{Conclusion}
In this work, we propose a simple yet effective approach, called SMILE, for graph few-shot learning with fewer tasks. Specifically, we introduce a novel dual-level mixup strategy, including within-task and across-task mixup, for enriching the diversity of nodes within each task and the diversity of tasks. Also, we incorporate the degree-based prior information to learn expressive node embeddings. Theoretically, we prove that SMILE effectively enhances the model's generalization performance. Empirically, we conduct extensive experiments on multiple benchmarks and the results suggest that SMILE significantly outperforms other baselines, including both in-domain and cross-domain few-shot settings.

% \blue{18 pages requirement, not including title page and refs/appendices.}

% Optionally include a table of contents
\vspace{1cm}
\setcounter{tocdepth}{2} % adjust to 1 if desired
% \tableofcontents

\end{titlepage}



% \input{section-7-appendices}

% Start of "Sample References" section



% Bibliography
\bibliographystyle{ACM-Reference-Format}
\bibliography{refs}

\newpage

\subsection{Lloyd-Max Algorithm}
\label{subsec:Lloyd-Max}
For a given quantization bitwidth $B$ and an operand $\bm{X}$, the Lloyd-Max algorithm finds $2^B$ quantization levels $\{\hat{x}_i\}_{i=1}^{2^B}$ such that quantizing $\bm{X}$ by rounding each scalar in $\bm{X}$ to the nearest quantization level minimizes the quantization MSE. 

The algorithm starts with an initial guess of quantization levels and then iteratively computes quantization thresholds $\{\tau_i\}_{i=1}^{2^B-1}$ and updates quantization levels $\{\hat{x}_i\}_{i=1}^{2^B}$. Specifically, at iteration $n$, thresholds are set to the midpoints of the previous iteration's levels:
\begin{align*}
    \tau_i^{(n)}=\frac{\hat{x}_i^{(n-1)}+\hat{x}_{i+1}^{(n-1)}}2 \text{ for } i=1\ldots 2^B-1
\end{align*}
Subsequently, the quantization levels are re-computed as conditional means of the data regions defined by the new thresholds:
\begin{align*}
    \hat{x}_i^{(n)}=\mathbb{E}\left[ \bm{X} \big| \bm{X}\in [\tau_{i-1}^{(n)},\tau_i^{(n)}] \right] \text{ for } i=1\ldots 2^B
\end{align*}
where to satisfy boundary conditions we have $\tau_0=-\infty$ and $\tau_{2^B}=\infty$. The algorithm iterates the above steps until convergence.

Figure \ref{fig:lm_quant} compares the quantization levels of a $7$-bit floating point (E3M3) quantizer (left) to a $7$-bit Lloyd-Max quantizer (right) when quantizing a layer of weights from the GPT3-126M model at a per-tensor granularity. As shown, the Lloyd-Max quantizer achieves substantially lower quantization MSE. Further, Table \ref{tab:FP7_vs_LM7} shows the superior perplexity achieved by Lloyd-Max quantizers for bitwidths of $7$, $6$ and $5$. The difference between the quantizers is clear at 5 bits, where per-tensor FP quantization incurs a drastic and unacceptable increase in perplexity, while Lloyd-Max quantization incurs a much smaller increase. Nevertheless, we note that even the optimal Lloyd-Max quantizer incurs a notable ($\sim 1.5$) increase in perplexity due to the coarse granularity of quantization. 

\begin{figure}[h]
  \centering
  \includegraphics[width=0.7\linewidth]{sections/figures/LM7_FP7.pdf}
  \caption{\small Quantization levels and the corresponding quantization MSE of Floating Point (left) vs Lloyd-Max (right) Quantizers for a layer of weights in the GPT3-126M model.}
  \label{fig:lm_quant}
\end{figure}

\begin{table}[h]\scriptsize
\begin{center}
\caption{\label{tab:FP7_vs_LM7} \small Comparing perplexity (lower is better) achieved by floating point quantizers and Lloyd-Max quantizers on a GPT3-126M model for the Wikitext-103 dataset.}
\begin{tabular}{c|cc|c}
\hline
 \multirow{2}{*}{\textbf{Bitwidth}} & \multicolumn{2}{|c|}{\textbf{Floating-Point Quantizer}} & \textbf{Lloyd-Max Quantizer} \\
 & Best Format & Wikitext-103 Perplexity & Wikitext-103 Perplexity \\
\hline
7 & E3M3 & 18.32 & 18.27 \\
6 & E3M2 & 19.07 & 18.51 \\
5 & E4M0 & 43.89 & 19.71 \\
\hline
\end{tabular}
\end{center}
\end{table}

\subsection{Proof of Local Optimality of LO-BCQ}
\label{subsec:lobcq_opt_proof}
For a given block $\bm{b}_j$, the quantization MSE during LO-BCQ can be empirically evaluated as $\frac{1}{L_b}\lVert \bm{b}_j- \bm{\hat{b}}_j\rVert^2_2$ where $\bm{\hat{b}}_j$ is computed from equation (\ref{eq:clustered_quantization_definition}) as $C_{f(\bm{b}_j)}(\bm{b}_j)$. Further, for a given block cluster $\mathcal{B}_i$, we compute the quantization MSE as $\frac{1}{|\mathcal{B}_{i}|}\sum_{\bm{b} \in \mathcal{B}_{i}} \frac{1}{L_b}\lVert \bm{b}- C_i^{(n)}(\bm{b})\rVert^2_2$. Therefore, at the end of iteration $n$, we evaluate the overall quantization MSE $J^{(n)}$ for a given operand $\bm{X}$ composed of $N_c$ block clusters as:
\begin{align*}
    \label{eq:mse_iter_n}
    J^{(n)} = \frac{1}{N_c} \sum_{i=1}^{N_c} \frac{1}{|\mathcal{B}_{i}^{(n)}|}\sum_{\bm{v} \in \mathcal{B}_{i}^{(n)}} \frac{1}{L_b}\lVert \bm{b}- B_i^{(n)}(\bm{b})\rVert^2_2
\end{align*}

At the end of iteration $n$, the codebooks are updated from $\mathcal{C}^{(n-1)}$ to $\mathcal{C}^{(n)}$. However, the mapping of a given vector $\bm{b}_j$ to quantizers $\mathcal{C}^{(n)}$ remains as  $f^{(n)}(\bm{b}_j)$. At the next iteration, during the vector clustering step, $f^{(n+1)}(\bm{b}_j)$ finds new mapping of $\bm{b}_j$ to updated codebooks $\mathcal{C}^{(n)}$ such that the quantization MSE over the candidate codebooks is minimized. Therefore, we obtain the following result for $\bm{b}_j$:
\begin{align*}
\frac{1}{L_b}\lVert \bm{b}_j - C_{f^{(n+1)}(\bm{b}_j)}^{(n)}(\bm{b}_j)\rVert^2_2 \le \frac{1}{L_b}\lVert \bm{b}_j - C_{f^{(n)}(\bm{b}_j)}^{(n)}(\bm{b}_j)\rVert^2_2
\end{align*}

That is, quantizing $\bm{b}_j$ at the end of the block clustering step of iteration $n+1$ results in lower quantization MSE compared to quantizing at the end of iteration $n$. Since this is true for all $\bm{b} \in \bm{X}$, we assert the following:
\begin{equation}
\begin{split}
\label{eq:mse_ineq_1}
    \tilde{J}^{(n+1)} &= \frac{1}{N_c} \sum_{i=1}^{N_c} \frac{1}{|\mathcal{B}_{i}^{(n+1)}|}\sum_{\bm{b} \in \mathcal{B}_{i}^{(n+1)}} \frac{1}{L_b}\lVert \bm{b} - C_i^{(n)}(b)\rVert^2_2 \le J^{(n)}
\end{split}
\end{equation}
where $\tilde{J}^{(n+1)}$ is the the quantization MSE after the vector clustering step at iteration $n+1$.

Next, during the codebook update step (\ref{eq:quantizers_update}) at iteration $n+1$, the per-cluster codebooks $\mathcal{C}^{(n)}$ are updated to $\mathcal{C}^{(n+1)}$ by invoking the Lloyd-Max algorithm \citep{Lloyd}. We know that for any given value distribution, the Lloyd-Max algorithm minimizes the quantization MSE. Therefore, for a given vector cluster $\mathcal{B}_i$ we obtain the following result:

\begin{equation}
    \frac{1}{|\mathcal{B}_{i}^{(n+1)}|}\sum_{\bm{b} \in \mathcal{B}_{i}^{(n+1)}} \frac{1}{L_b}\lVert \bm{b}- C_i^{(n+1)}(\bm{b})\rVert^2_2 \le \frac{1}{|\mathcal{B}_{i}^{(n+1)}|}\sum_{\bm{b} \in \mathcal{B}_{i}^{(n+1)}} \frac{1}{L_b}\lVert \bm{b}- C_i^{(n)}(\bm{b})\rVert^2_2
\end{equation}

The above equation states that quantizing the given block cluster $\mathcal{B}_i$ after updating the associated codebook from $C_i^{(n)}$ to $C_i^{(n+1)}$ results in lower quantization MSE. Since this is true for all the block clusters, we derive the following result: 
\begin{equation}
\begin{split}
\label{eq:mse_ineq_2}
     J^{(n+1)} &= \frac{1}{N_c} \sum_{i=1}^{N_c} \frac{1}{|\mathcal{B}_{i}^{(n+1)}|}\sum_{\bm{b} \in \mathcal{B}_{i}^{(n+1)}} \frac{1}{L_b}\lVert \bm{b}- C_i^{(n+1)}(\bm{b})\rVert^2_2  \le \tilde{J}^{(n+1)}   
\end{split}
\end{equation}

Following (\ref{eq:mse_ineq_1}) and (\ref{eq:mse_ineq_2}), we find that the quantization MSE is non-increasing for each iteration, that is, $J^{(1)} \ge J^{(2)} \ge J^{(3)} \ge \ldots \ge J^{(M)}$ where $M$ is the maximum number of iterations. 
%Therefore, we can say that if the algorithm converges, then it must be that it has converged to a local minimum. 
\hfill $\blacksquare$


\begin{figure}
    \begin{center}
    \includegraphics[width=0.5\textwidth]{sections//figures/mse_vs_iter.pdf}
    \end{center}
    \caption{\small NMSE vs iterations during LO-BCQ compared to other block quantization proposals}
    \label{fig:nmse_vs_iter}
\end{figure}

Figure \ref{fig:nmse_vs_iter} shows the empirical convergence of LO-BCQ across several block lengths and number of codebooks. Also, the MSE achieved by LO-BCQ is compared to baselines such as MXFP and VSQ. As shown, LO-BCQ converges to a lower MSE than the baselines. Further, we achieve better convergence for larger number of codebooks ($N_c$) and for a smaller block length ($L_b$), both of which increase the bitwidth of BCQ (see Eq \ref{eq:bitwidth_bcq}).


\subsection{Additional Accuracy Results}
%Table \ref{tab:lobcq_config} lists the various LOBCQ configurations and their corresponding bitwidths.
\begin{table}
\setlength{\tabcolsep}{4.75pt}
\begin{center}
\caption{\label{tab:lobcq_config} Various LO-BCQ configurations and their bitwidths.}
\begin{tabular}{|c||c|c|c|c||c|c||c|} 
\hline
 & \multicolumn{4}{|c||}{$L_b=8$} & \multicolumn{2}{|c||}{$L_b=4$} & $L_b=2$ \\
 \hline
 \backslashbox{$L_A$\kern-1em}{\kern-1em$N_c$} & 2 & 4 & 8 & 16 & 2 & 4 & 2 \\
 \hline
 64 & 4.25 & 4.375 & 4.5 & 4.625 & 4.375 & 4.625 & 4.625\\
 \hline
 32 & 4.375 & 4.5 & 4.625& 4.75 & 4.5 & 4.75 & 4.75 \\
 \hline
 16 & 4.625 & 4.75& 4.875 & 5 & 4.75 & 5 & 5 \\
 \hline
\end{tabular}
\end{center}
\end{table}

%\subsection{Perplexity achieved by various LO-BCQ configurations on Wikitext-103 dataset}

\begin{table} \centering
\begin{tabular}{|c||c|c|c|c||c|c||c|} 
\hline
 $L_b \rightarrow$& \multicolumn{4}{c||}{8} & \multicolumn{2}{c||}{4} & 2\\
 \hline
 \backslashbox{$L_A$\kern-1em}{\kern-1em$N_c$} & 2 & 4 & 8 & 16 & 2 & 4 & 2  \\
 %$N_c \rightarrow$ & 2 & 4 & 8 & 16 & 2 & 4 & 2 \\
 \hline
 \hline
 \multicolumn{8}{c}{GPT3-1.3B (FP32 PPL = 9.98)} \\ 
 \hline
 \hline
 64 & 10.40 & 10.23 & 10.17 & 10.15 &  10.28 & 10.18 & 10.19 \\
 \hline
 32 & 10.25 & 10.20 & 10.15 & 10.12 &  10.23 & 10.17 & 10.17 \\
 \hline
 16 & 10.22 & 10.16 & 10.10 & 10.09 &  10.21 & 10.14 & 10.16 \\
 \hline
  \hline
 \multicolumn{8}{c}{GPT3-8B (FP32 PPL = 7.38)} \\ 
 \hline
 \hline
 64 & 7.61 & 7.52 & 7.48 &  7.47 &  7.55 &  7.49 & 7.50 \\
 \hline
 32 & 7.52 & 7.50 & 7.46 &  7.45 &  7.52 &  7.48 & 7.48  \\
 \hline
 16 & 7.51 & 7.48 & 7.44 &  7.44 &  7.51 &  7.49 & 7.47  \\
 \hline
\end{tabular}
\caption{\label{tab:ppl_gpt3_abalation} Wikitext-103 perplexity across GPT3-1.3B and 8B models.}
\end{table}

\begin{table} \centering
\begin{tabular}{|c||c|c|c|c||} 
\hline
 $L_b \rightarrow$& \multicolumn{4}{c||}{8}\\
 \hline
 \backslashbox{$L_A$\kern-1em}{\kern-1em$N_c$} & 2 & 4 & 8 & 16 \\
 %$N_c \rightarrow$ & 2 & 4 & 8 & 16 & 2 & 4 & 2 \\
 \hline
 \hline
 \multicolumn{5}{|c|}{Llama2-7B (FP32 PPL = 5.06)} \\ 
 \hline
 \hline
 64 & 5.31 & 5.26 & 5.19 & 5.18  \\
 \hline
 32 & 5.23 & 5.25 & 5.18 & 5.15  \\
 \hline
 16 & 5.23 & 5.19 & 5.16 & 5.14  \\
 \hline
 \multicolumn{5}{|c|}{Nemotron4-15B (FP32 PPL = 5.87)} \\ 
 \hline
 \hline
 64  & 6.3 & 6.20 & 6.13 & 6.08  \\
 \hline
 32  & 6.24 & 6.12 & 6.07 & 6.03  \\
 \hline
 16  & 6.12 & 6.14 & 6.04 & 6.02  \\
 \hline
 \multicolumn{5}{|c|}{Nemotron4-340B (FP32 PPL = 3.48)} \\ 
 \hline
 \hline
 64 & 3.67 & 3.62 & 3.60 & 3.59 \\
 \hline
 32 & 3.63 & 3.61 & 3.59 & 3.56 \\
 \hline
 16 & 3.61 & 3.58 & 3.57 & 3.55 \\
 \hline
\end{tabular}
\caption{\label{tab:ppl_llama7B_nemo15B} Wikitext-103 perplexity compared to FP32 baseline in Llama2-7B and Nemotron4-15B, 340B models}
\end{table}

%\subsection{Perplexity achieved by various LO-BCQ configurations on MMLU dataset}


\begin{table} \centering
\begin{tabular}{|c||c|c|c|c||c|c|c|c|} 
\hline
 $L_b \rightarrow$& \multicolumn{4}{c||}{8} & \multicolumn{4}{c||}{8}\\
 \hline
 \backslashbox{$L_A$\kern-1em}{\kern-1em$N_c$} & 2 & 4 & 8 & 16 & 2 & 4 & 8 & 16  \\
 %$N_c \rightarrow$ & 2 & 4 & 8 & 16 & 2 & 4 & 2 \\
 \hline
 \hline
 \multicolumn{5}{|c|}{Llama2-7B (FP32 Accuracy = 45.8\%)} & \multicolumn{4}{|c|}{Llama2-70B (FP32 Accuracy = 69.12\%)} \\ 
 \hline
 \hline
 64 & 43.9 & 43.4 & 43.9 & 44.9 & 68.07 & 68.27 & 68.17 & 68.75 \\
 \hline
 32 & 44.5 & 43.8 & 44.9 & 44.5 & 68.37 & 68.51 & 68.35 & 68.27  \\
 \hline
 16 & 43.9 & 42.7 & 44.9 & 45 & 68.12 & 68.77 & 68.31 & 68.59  \\
 \hline
 \hline
 \multicolumn{5}{|c|}{GPT3-22B (FP32 Accuracy = 38.75\%)} & \multicolumn{4}{|c|}{Nemotron4-15B (FP32 Accuracy = 64.3\%)} \\ 
 \hline
 \hline
 64 & 36.71 & 38.85 & 38.13 & 38.92 & 63.17 & 62.36 & 63.72 & 64.09 \\
 \hline
 32 & 37.95 & 38.69 & 39.45 & 38.34 & 64.05 & 62.30 & 63.8 & 64.33  \\
 \hline
 16 & 38.88 & 38.80 & 38.31 & 38.92 & 63.22 & 63.51 & 63.93 & 64.43  \\
 \hline
\end{tabular}
\caption{\label{tab:mmlu_abalation} Accuracy on MMLU dataset across GPT3-22B, Llama2-7B, 70B and Nemotron4-15B models.}
\end{table}


%\subsection{Perplexity achieved by various LO-BCQ configurations on LM evaluation harness}

\begin{table} \centering
\begin{tabular}{|c||c|c|c|c||c|c|c|c|} 
\hline
 $L_b \rightarrow$& \multicolumn{4}{c||}{8} & \multicolumn{4}{c||}{8}\\
 \hline
 \backslashbox{$L_A$\kern-1em}{\kern-1em$N_c$} & 2 & 4 & 8 & 16 & 2 & 4 & 8 & 16  \\
 %$N_c \rightarrow$ & 2 & 4 & 8 & 16 & 2 & 4 & 2 \\
 \hline
 \hline
 \multicolumn{5}{|c|}{Race (FP32 Accuracy = 37.51\%)} & \multicolumn{4}{|c|}{Boolq (FP32 Accuracy = 64.62\%)} \\ 
 \hline
 \hline
 64 & 36.94 & 37.13 & 36.27 & 37.13 & 63.73 & 62.26 & 63.49 & 63.36 \\
 \hline
 32 & 37.03 & 36.36 & 36.08 & 37.03 & 62.54 & 63.51 & 63.49 & 63.55  \\
 \hline
 16 & 37.03 & 37.03 & 36.46 & 37.03 & 61.1 & 63.79 & 63.58 & 63.33  \\
 \hline
 \hline
 \multicolumn{5}{|c|}{Winogrande (FP32 Accuracy = 58.01\%)} & \multicolumn{4}{|c|}{Piqa (FP32 Accuracy = 74.21\%)} \\ 
 \hline
 \hline
 64 & 58.17 & 57.22 & 57.85 & 58.33 & 73.01 & 73.07 & 73.07 & 72.80 \\
 \hline
 32 & 59.12 & 58.09 & 57.85 & 58.41 & 73.01 & 73.94 & 72.74 & 73.18  \\
 \hline
 16 & 57.93 & 58.88 & 57.93 & 58.56 & 73.94 & 72.80 & 73.01 & 73.94  \\
 \hline
\end{tabular}
\caption{\label{tab:mmlu_abalation} Accuracy on LM evaluation harness tasks on GPT3-1.3B model.}
\end{table}

\begin{table} \centering
\begin{tabular}{|c||c|c|c|c||c|c|c|c|} 
\hline
 $L_b \rightarrow$& \multicolumn{4}{c||}{8} & \multicolumn{4}{c||}{8}\\
 \hline
 \backslashbox{$L_A$\kern-1em}{\kern-1em$N_c$} & 2 & 4 & 8 & 16 & 2 & 4 & 8 & 16  \\
 %$N_c \rightarrow$ & 2 & 4 & 8 & 16 & 2 & 4 & 2 \\
 \hline
 \hline
 \multicolumn{5}{|c|}{Race (FP32 Accuracy = 41.34\%)} & \multicolumn{4}{|c|}{Boolq (FP32 Accuracy = 68.32\%)} \\ 
 \hline
 \hline
 64 & 40.48 & 40.10 & 39.43 & 39.90 & 69.20 & 68.41 & 69.45 & 68.56 \\
 \hline
 32 & 39.52 & 39.52 & 40.77 & 39.62 & 68.32 & 67.43 & 68.17 & 69.30  \\
 \hline
 16 & 39.81 & 39.71 & 39.90 & 40.38 & 68.10 & 66.33 & 69.51 & 69.42  \\
 \hline
 \hline
 \multicolumn{5}{|c|}{Winogrande (FP32 Accuracy = 67.88\%)} & \multicolumn{4}{|c|}{Piqa (FP32 Accuracy = 78.78\%)} \\ 
 \hline
 \hline
 64 & 66.85 & 66.61 & 67.72 & 67.88 & 77.31 & 77.42 & 77.75 & 77.64 \\
 \hline
 32 & 67.25 & 67.72 & 67.72 & 67.00 & 77.31 & 77.04 & 77.80 & 77.37  \\
 \hline
 16 & 68.11 & 68.90 & 67.88 & 67.48 & 77.37 & 78.13 & 78.13 & 77.69  \\
 \hline
\end{tabular}
\caption{\label{tab:mmlu_abalation} Accuracy on LM evaluation harness tasks on GPT3-8B model.}
\end{table}

\begin{table} \centering
\begin{tabular}{|c||c|c|c|c||c|c|c|c|} 
\hline
 $L_b \rightarrow$& \multicolumn{4}{c||}{8} & \multicolumn{4}{c||}{8}\\
 \hline
 \backslashbox{$L_A$\kern-1em}{\kern-1em$N_c$} & 2 & 4 & 8 & 16 & 2 & 4 & 8 & 16  \\
 %$N_c \rightarrow$ & 2 & 4 & 8 & 16 & 2 & 4 & 2 \\
 \hline
 \hline
 \multicolumn{5}{|c|}{Race (FP32 Accuracy = 40.67\%)} & \multicolumn{4}{|c|}{Boolq (FP32 Accuracy = 76.54\%)} \\ 
 \hline
 \hline
 64 & 40.48 & 40.10 & 39.43 & 39.90 & 75.41 & 75.11 & 77.09 & 75.66 \\
 \hline
 32 & 39.52 & 39.52 & 40.77 & 39.62 & 76.02 & 76.02 & 75.96 & 75.35  \\
 \hline
 16 & 39.81 & 39.71 & 39.90 & 40.38 & 75.05 & 73.82 & 75.72 & 76.09  \\
 \hline
 \hline
 \multicolumn{5}{|c|}{Winogrande (FP32 Accuracy = 70.64\%)} & \multicolumn{4}{|c|}{Piqa (FP32 Accuracy = 79.16\%)} \\ 
 \hline
 \hline
 64 & 69.14 & 70.17 & 70.17 & 70.56 & 78.24 & 79.00 & 78.62 & 78.73 \\
 \hline
 32 & 70.96 & 69.69 & 71.27 & 69.30 & 78.56 & 79.49 & 79.16 & 78.89  \\
 \hline
 16 & 71.03 & 69.53 & 69.69 & 70.40 & 78.13 & 79.16 & 79.00 & 79.00  \\
 \hline
\end{tabular}
\caption{\label{tab:mmlu_abalation} Accuracy on LM evaluation harness tasks on GPT3-22B model.}
\end{table}

\begin{table} \centering
\begin{tabular}{|c||c|c|c|c||c|c|c|c|} 
\hline
 $L_b \rightarrow$& \multicolumn{4}{c||}{8} & \multicolumn{4}{c||}{8}\\
 \hline
 \backslashbox{$L_A$\kern-1em}{\kern-1em$N_c$} & 2 & 4 & 8 & 16 & 2 & 4 & 8 & 16  \\
 %$N_c \rightarrow$ & 2 & 4 & 8 & 16 & 2 & 4 & 2 \\
 \hline
 \hline
 \multicolumn{5}{|c|}{Race (FP32 Accuracy = 44.4\%)} & \multicolumn{4}{|c|}{Boolq (FP32 Accuracy = 79.29\%)} \\ 
 \hline
 \hline
 64 & 42.49 & 42.51 & 42.58 & 43.45 & 77.58 & 77.37 & 77.43 & 78.1 \\
 \hline
 32 & 43.35 & 42.49 & 43.64 & 43.73 & 77.86 & 75.32 & 77.28 & 77.86  \\
 \hline
 16 & 44.21 & 44.21 & 43.64 & 42.97 & 78.65 & 77 & 76.94 & 77.98  \\
 \hline
 \hline
 \multicolumn{5}{|c|}{Winogrande (FP32 Accuracy = 69.38\%)} & \multicolumn{4}{|c|}{Piqa (FP32 Accuracy = 78.07\%)} \\ 
 \hline
 \hline
 64 & 68.9 & 68.43 & 69.77 & 68.19 & 77.09 & 76.82 & 77.09 & 77.86 \\
 \hline
 32 & 69.38 & 68.51 & 68.82 & 68.90 & 78.07 & 76.71 & 78.07 & 77.86  \\
 \hline
 16 & 69.53 & 67.09 & 69.38 & 68.90 & 77.37 & 77.8 & 77.91 & 77.69  \\
 \hline
\end{tabular}
\caption{\label{tab:mmlu_abalation} Accuracy on LM evaluation harness tasks on Llama2-7B model.}
\end{table}

\begin{table} \centering
\begin{tabular}{|c||c|c|c|c||c|c|c|c|} 
\hline
 $L_b \rightarrow$& \multicolumn{4}{c||}{8} & \multicolumn{4}{c||}{8}\\
 \hline
 \backslashbox{$L_A$\kern-1em}{\kern-1em$N_c$} & 2 & 4 & 8 & 16 & 2 & 4 & 8 & 16  \\
 %$N_c \rightarrow$ & 2 & 4 & 8 & 16 & 2 & 4 & 2 \\
 \hline
 \hline
 \multicolumn{5}{|c|}{Race (FP32 Accuracy = 48.8\%)} & \multicolumn{4}{|c|}{Boolq (FP32 Accuracy = 85.23\%)} \\ 
 \hline
 \hline
 64 & 49.00 & 49.00 & 49.28 & 48.71 & 82.82 & 84.28 & 84.03 & 84.25 \\
 \hline
 32 & 49.57 & 48.52 & 48.33 & 49.28 & 83.85 & 84.46 & 84.31 & 84.93  \\
 \hline
 16 & 49.85 & 49.09 & 49.28 & 48.99 & 85.11 & 84.46 & 84.61 & 83.94  \\
 \hline
 \hline
 \multicolumn{5}{|c|}{Winogrande (FP32 Accuracy = 79.95\%)} & \multicolumn{4}{|c|}{Piqa (FP32 Accuracy = 81.56\%)} \\ 
 \hline
 \hline
 64 & 78.77 & 78.45 & 78.37 & 79.16 & 81.45 & 80.69 & 81.45 & 81.5 \\
 \hline
 32 & 78.45 & 79.01 & 78.69 & 80.66 & 81.56 & 80.58 & 81.18 & 81.34  \\
 \hline
 16 & 79.95 & 79.56 & 79.79 & 79.72 & 81.28 & 81.66 & 81.28 & 80.96  \\
 \hline
\end{tabular}
\caption{\label{tab:mmlu_abalation} Accuracy on LM evaluation harness tasks on Llama2-70B model.}
\end{table}

%\section{MSE Studies}
%\textcolor{red}{TODO}


\subsection{Number Formats and Quantization Method}
\label{subsec:numFormats_quantMethod}
\subsubsection{Integer Format}
An $n$-bit signed integer (INT) is typically represented with a 2s-complement format \citep{yao2022zeroquant,xiao2023smoothquant,dai2021vsq}, where the most significant bit denotes the sign.

\subsubsection{Floating Point Format}
An $n$-bit signed floating point (FP) number $x$ comprises of a 1-bit sign ($x_{\mathrm{sign}}$), $B_m$-bit mantissa ($x_{\mathrm{mant}}$) and $B_e$-bit exponent ($x_{\mathrm{exp}}$) such that $B_m+B_e=n-1$. The associated constant exponent bias ($E_{\mathrm{bias}}$) is computed as $(2^{{B_e}-1}-1)$. We denote this format as $E_{B_e}M_{B_m}$.  

\subsubsection{Quantization Scheme}
\label{subsec:quant_method}
A quantization scheme dictates how a given unquantized tensor is converted to its quantized representation. We consider FP formats for the purpose of illustration. Given an unquantized tensor $\bm{X}$ and an FP format $E_{B_e}M_{B_m}$, we first, we compute the quantization scale factor $s_X$ that maps the maximum absolute value of $\bm{X}$ to the maximum quantization level of the $E_{B_e}M_{B_m}$ format as follows:
\begin{align}
\label{eq:sf}
    s_X = \frac{\mathrm{max}(|\bm{X}|)}{\mathrm{max}(E_{B_e}M_{B_m})}
\end{align}
In the above equation, $|\cdot|$ denotes the absolute value function.

Next, we scale $\bm{X}$ by $s_X$ and quantize it to $\hat{\bm{X}}$ by rounding it to the nearest quantization level of $E_{B_e}M_{B_m}$ as:

\begin{align}
\label{eq:tensor_quant}
    \hat{\bm{X}} = \text{round-to-nearest}\left(\frac{\bm{X}}{s_X}, E_{B_e}M_{B_m}\right)
\end{align}

We perform dynamic max-scaled quantization \citep{wu2020integer}, where the scale factor $s$ for activations is dynamically computed during runtime.

\subsection{Vector Scaled Quantization}
\begin{wrapfigure}{r}{0.35\linewidth}
  \centering
  \includegraphics[width=\linewidth]{sections/figures/vsquant.jpg}
  \caption{\small Vectorwise decomposition for per-vector scaled quantization (VSQ \citep{dai2021vsq}).}
  \label{fig:vsquant}
\end{wrapfigure}
During VSQ \citep{dai2021vsq}, the operand tensors are decomposed into 1D vectors in a hardware friendly manner as shown in Figure \ref{fig:vsquant}. Since the decomposed tensors are used as operands in matrix multiplications during inference, it is beneficial to perform this decomposition along the reduction dimension of the multiplication. The vectorwise quantization is performed similar to tensorwise quantization described in Equations \ref{eq:sf} and \ref{eq:tensor_quant}, where a scale factor $s_v$ is required for each vector $\bm{v}$ that maps the maximum absolute value of that vector to the maximum quantization level. While smaller vector lengths can lead to larger accuracy gains, the associated memory and computational overheads due to the per-vector scale factors increases. To alleviate these overheads, VSQ \citep{dai2021vsq} proposed a second level quantization of the per-vector scale factors to unsigned integers, while MX \citep{rouhani2023shared} quantizes them to integer powers of 2 (denoted as $2^{INT}$).

\subsubsection{MX Format}
The MX format proposed in \citep{rouhani2023microscaling} introduces the concept of sub-block shifting. For every two scalar elements of $b$-bits each, there is a shared exponent bit. The value of this exponent bit is determined through an empirical analysis that targets minimizing quantization MSE. We note that the FP format $E_{1}M_{b}$ is strictly better than MX from an accuracy perspective since it allocates a dedicated exponent bit to each scalar as opposed to sharing it across two scalars. Therefore, we conservatively bound the accuracy of a $b+2$-bit signed MX format with that of a $E_{1}M_{b}$ format in our comparisons. For instance, we use E1M2 format as a proxy for MX4.

\begin{figure}
    \centering
    \includegraphics[width=1\linewidth]{sections//figures/BlockFormats.pdf}
    \caption{\small Comparing LO-BCQ to MX format.}
    \label{fig:block_formats}
\end{figure}

Figure \ref{fig:block_formats} compares our $4$-bit LO-BCQ block format to MX \citep{rouhani2023microscaling}. As shown, both LO-BCQ and MX decompose a given operand tensor into block arrays and each block array into blocks. Similar to MX, we find that per-block quantization ($L_b < L_A$) leads to better accuracy due to increased flexibility. While MX achieves this through per-block $1$-bit micro-scales, we associate a dedicated codebook to each block through a per-block codebook selector. Further, MX quantizes the per-block array scale-factor to E8M0 format without per-tensor scaling. In contrast during LO-BCQ, we find that per-tensor scaling combined with quantization of per-block array scale-factor to E4M3 format results in superior inference accuracy across models. 



\end{document}

