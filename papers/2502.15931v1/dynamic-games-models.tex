% !TEX root = ./misc.tex

\section{Other models of dynamic games}

In general, we are interested in dynamic games on graphs.  Our motivation is to understand the structure of feedback control solutions in both finite time and infinite time horizon settings. Some interesting models might include: 

\begin{itemize}
    \item Sadler and Golub (2021) describe a network formation game where you add one edge or delete on edge at a time. But it's hard to reason about because players' actions are not continuous in this case. 
    \item Best-response dynamics\footnote{Network Learning from Best-Response Dynamics in LQ Games. ACC 2023.} using the classic LQ game of \cite{leng2020learning}. 
    \item Target stimulus in times of intervention for the Eisenberg-Noe model of bankruptcies and defaults in financial liabilities networks. 
\end{itemize}

\subsection{Best response dynamics}

At step $t + 1$, player $i$ simply chooses the best $x_i$ given $x_{-i}(t)$. Taking $\frac{\del}{\del x_i}$ we simply obtain: 

\[
x_i(t + 1) = \arg \max_{x_i \in \RR} b_i x_i - \frac 1 2 x_i^2 + \beta_i x_i \sum_{j \sim i} G_{ij} x_j(t)
= b_i + \beta_i \sum_{j \sim i} G_{ij} x_j(t)
\]

Therefore $u_{t+1}^{i} = x_i(t+1) - x_i(t) = \beta_i \sum\limits_{j} G_{ij}(x_j(t) - x_j(t-1))$. 

