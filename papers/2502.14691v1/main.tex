%%
%% This is file `sample-sigconf.tex',
%% generated with the docstrip utility.
%%
%% The original source files were:
%%
%% samples.dtx  (with options: `sigconf')
%% 
%% IMPORTANT NOTICE:
%% 
%% For the copyright see the source file.
%% 
%% Any modified versions of this file must be renamed
%% with new filenames distinct from sample-sigconf.tex.
%% 
%% For distribution of the original source see the terms
%% for copying and modification in the file samples.dtx.
%% 
%% This generated file may be distributed as long as the
%% original source files, as listed above, are part of the
%% same distribution. (The sources need not necessarily be
%% in the same archive or directory.)
%%
%% Commands for TeXCount
%TC:macro \cite [option:text,text]
%TC:macro \citep [option:text,text]
%TC:macro \citet [option:text,text]
%TC:envir table 0 1
%TC:envir table* 0 1
%TC:envir tabular [ignore] word
%TC:envir displaymath 0 word
%TC:envir math 0 word
%TC:envir comment 0 0
%%
%%
%% The first command in your LaTeX source must be the \documentclass command.
\DocumentMetadata{}
\documentclass[sigconf,nonacm]{acmart}
%% NOTE that a single column version is required for 
%% submission and peer review. This can be done by changing
%% the \doucmentclass[...]{acmart} in this template to 
%% \documentclass[manuscript,screen]{acmart}
%% 
%% To ensure 100% compatibility, please check the white list of
%% approved LaTeX packages to be used with the Master Article Template at
%% https://www.acm.org/publications/taps/whitelist-of-latex-packages 
%% before creating your document. The white list page provides 
%% information on how to submit additional LaTeX packages for 
%% review and adoption.
%% Fonts used in the template cannot be substituted; margin 
%% adjustments are not allowed.
\usepackage{amsmath}

%My packages
\usepackage{multirow}
\usepackage{soul}
\usepackage{algorithm}
\usepackage{algpseudocode}
\algnewcommand\Not{\textbf{!}}
\algnewcommand{\algorithmicforeach}{\textbf{for each}}
\algdef{SE}[FOR]{ForEach}{EndForEach}[1]
  {\algorithmicforeach\ #1\ \algorithmicdo}% \ForEach{#1}
  {\algorithmicend\ \algorithmicforeach}% \EndForEach

%%
%% \BibTeX command to typeset BibTeX logo in the docs
\AtBeginDocument{%
  \providecommand\BibTeX{{%
    \normalfont B\kern-0.5em{\scshape i\kern-0.25em b}\kern-0.8em\TeX}}}

%% Rights management information.  This information is sent to you
%% when you complete the rights form.  These commands have SAMPLE
%% values in them; it is your responsibility as an author to replace
%% the commands and values with those provided to you when you
%% complete the rights form.
\copyrightyear{2024}
\acmYear{2024}
\acmConference[CAMS 2024]{2nd Workshop on Computer Architecture Modeling and Simulation}{November 2, 2024}{Austin, Texas, USA}
\acmBooktitle{2nd Workshop on Computer Architecture Modeling and Simulation (CAMS 2024), November 2, 2024, Austin, Texas, USA}
%\acmPrice{15.00}
%\acmDOI{10.1145/3589236.3589244}
%\acmISBN{979-8-4007-0776-6/23/02}


%%
%% Submission ID.
%% Use this when submitting an article to a sponsored event. You'll
%% receive a unique submission ID from the organizers
%% of the event, and this ID should be used as the parameter to this command.
%%\acmSubmissionID{123-A56-BU3}

%%
%% For managing citations, it is recommended to use bibliography
%% files in BibTeX format.
%%
%% You can then either use BibTeX with the ACM-Reference-Format style,
%% or BibLaTeX with the acmnumeric or acmauthoryear sytles, that include
%% support for advanced citation of software artefact from the
%% biblatex-software package, also separately available on CTAN.
%%
%% Look at the sample-*-biblatex.tex files for templates showcasing
%% the biblatex styles.
%%

%%
%% The majority of ACM publications use numbered citations and
%% references.  The command \citestyle{authoryear} switches to the
%% "author year" style.
%%
%% If you are preparing content for an event
%% sponsored by ACM SIGGRAPH, you must use the "author year" style of
%% citations and references.
%% Uncommenting
%% the next command will enable that style.
%%\citestyle{acmauthoryear}

%%
%% end of the preamble, start of the body of the document source.
\begin{document}

%%
%% The "title" command has an optional parameter,
%% allowing the author to define a "short title" to be used in page headers.
%\title{Need for Parallel: Accelerating a modern GPU simulator}
\title{Parallelizing a modern GPU simulator}

%%
%% The "author" command and its associated commands are used to define
%% the authors and their affiliations.
%% Of note is the shared affiliation of the first two authors, and the
%% "authornote" and "authornotemark" commands
%% used to denote shared contribution to the research.
\author{Rodrigo Huerta}
\email{rodrigo.huerta.ganan@upc.edu}
\orcid{0000-0003-0052-7710}
\affiliation{%
  \institution{Universitat Politècnica de Catalunya}
  \city{Barcelona}
  \country{Spain}
}

\author{Antonio González}
\email{antonio@ac.upc.edu}
\orcid{0000-0002-0009-0996}
\affiliation{%
  \institution{Universitat Politècnica de Catalunya}
  \city{Barcelona}
  \country{Spain}
}

%%
%% By default, the full list of authors will be used in the page
%% headers. Often, this list is too long, and will overlap
%% other information printed in the page headers. This command allows
%% the author to define a more concise list
%% of authors' names for this purpose.
\renewcommand{\shortauthors}{Huerta and González}

% My commands
\newcommand{\summaryNumCores}{16}
\newcommand{\summaryAvgSpeedup}{5.8x}
\newcommand{\summaryMaxSpeedup}{14x}
\newcommand{\summaryMaxTimeSequential}{five days}
\newcommand{\summaryMaxTimeReductionOpenMp}{12 hours}

%%
%% The abstract is a short summary of the work to be presented in the
%% article.
\begin{abstract}
\par
Simulators are a primary tool in computer architecture research but are extremely computationally intensive. Simulating modern architectures with increased core counts and recent workloads can be challenging, even on modern hardware. This paper demonstrates that simulating some GPGPU workloads in a single-threaded state-of-the-art simulator such as Accel-sim can take more than \summaryMaxTimeSequential{}. In this paper we present a simple approach to parallelize this simulator with minimal code changes by using OpenMP. Moreover, our parallelization technique is deterministic, so the simulator provides the same results for single-threaded and multi-threaded simulations. Compared to previous works, we achieve a higher speed-up, and, more importantly, the parallel simulation does not incur any inaccuracies. When we run the simulator with \summaryNumCores{} threads, we achieve an average speed-up of \summaryAvgSpeedup{} and reach \summaryMaxSpeedup{} in some workloads. This allows researchers to simulate applications that take \summaryMaxTimeSequential{} in less than \summaryMaxTimeReductionOpenMp{}. By speeding up simulations, researchers can model larger systems, simulate bigger workloads, add more detail to the model, increase the efficiency of the hardware platform where the simulator is run, and obtain results sooner.
\end{abstract}

%%
%% The code below is generated by the tool at http://dl.acm.org/ccs.cfm.
%% Please copy and paste the code instead of the example below.
%%
\begin{CCSXML}
<ccs2012>
   <concept>
       <concept_id>10010520.10010521.10010528</concept_id>
       <concept_desc>Computer systems organization~Parallel architectures</concept_desc>
       <concept_significance>500</concept_significance>
       </concept>
   <concept>
       <concept_id>10010520.10010521.10010528.10010534</concept_id>
       <concept_desc>Computer systems organization~Single instruction, multiple data</concept_desc>
       <concept_significance>500</concept_significance>
       </concept>
   <concept>
       <concept_id>10010520.10010521.10010528.10010536</concept_id>
       <concept_desc>Computer systems organization~Multicore architectures</concept_desc>
       <concept_significance>500</concept_significance>
       </concept>
   <concept>
       <concept_id>10010147.10010341</concept_id>
       <concept_desc>Computing methodologies~Modeling and simulation</concept_desc>
       <concept_significance>500</concept_significance>
       </concept>
   <concept>
       <concept_id>10010147.10010169</concept_id>
       <concept_desc>Computing methodologies~Parallel computing methodologies</concept_desc>
       <concept_significance>500</concept_significance>
       </concept>
 </ccs2012>
\end{CCSXML}

\ccsdesc[500]{Computer systems organization~Parallel architectures}
\ccsdesc[500]{Computer systems organization~Single instruction, multiple data}
\ccsdesc[500]{Computer systems organization~Multicore architectures}
\ccsdesc[500]{Computing methodologies~Modeling and simulation}
\ccsdesc[500]{Computing methodologies~Parallel computing methodologies}
%%
%% Keywords. The author(s) should pick words that accurately describe
%% the work being presented. Separate the keywords with commas.
\keywords{GPU, GPGPU, microarchitecture, simulation, OpenMP, parallelization, GPGPU-Sim, Accel-sim}

%%
%% This command processes the author and affiliation and title
%% information and builds the first part of the formatted document.
\maketitle

\section{Introduction}

In recent years, with advancements in generative models and the expansion of training datasets, text-to-speech (TTS) models \cite{valle, voicebox, ns3} have made breakthrough progress in naturalness and quality, gradually approaching the level of real recordings. However, low-latency and efficient dual-stream TTS, which involves processing streaming text inputs while simultaneously generating speech in real time, remains a challenging problem \cite{livespeech2}. These models are ideal for integration with upstream tasks, such as large language models (LLMs) \cite{gpt4} and streaming translation models \cite{seamless}, which can generate text in a streaming manner. Addressing these challenges can improve live human-computer interaction, paving the way for various applications, such as speech-to-speech translation and personal voice assistants.

Recently, inspired by advances in image generation, denoising diffusion \cite{diffusion, score}, flow matching \cite{fm}, and masked generative models \cite{maskgit} have been introduced into non-autoregressive (NAR) TTS \cite{seedtts, F5tts, pflow, maskgct}, demonstrating impressive performance in offline inference.  During this process, these offline TTS models first add noise or apply masking guided by the predicted duration. Subsequently, context from the entire sentence is leveraged to perform temporally-unordered denoising or mask prediction for speech generation. However, this temporally-unordered process hinders their application to streaming speech generation\footnote{
Here, “temporally” refers to the physical time of audio samples, not the iteration step $t \in [0, 1]$ of the above NAR TTS models.}.


When it comes to streaming speech generation, autoregressive (AR) TTS models \cite{valle, ellav} hold a distinct advantage because of their ability to deliver outputs in a temporally-ordered manner. However, compared to recently proposed NAR TTS models,  AR TTS models have a distinct disadvantage in terms of generation efficiency \cite{MEDUSA}. Specifically, the autoregressive steps are tied to the frame rate of speech tokens, resulting in slower inference speeds.  
While advancements like VALL-E 2 \cite{valle2} have boosted generation efficiency through group code modeling, the challenge remains that the manually set group size is typically small, suggesting room for further improvements. In addition,  most current AR TTS models \cite{dualsteam1} cannot handle stream text input and they only begin streaming speech generation after receiving the complete text,  ignoring the latency caused by the streaming text input. The most closely related works to SyncSpeech are CosyVoice2 \cite{cosyvoice2.0} and IST-LM \cite{yang2024interleaved}, both of which employ interleaved speech-text modeling to accommodate dual-stream scenarios. However, their autoregressive process generates only one speech token per step, leading to low efficiency.



To seamlessly integrate with  upstream LLMs and facilitate dual-stream speech synthesis, this paper introduces \textbf{SyncSpeech}, designed to keep the generation of streaming speech in synchronization with the incoming streaming text. SyncSpeech has the following advantages: 1) \textbf{low latency}, which means it begins generating speech in a streaming manner as soon as the second text token is received,
and
2) \textbf{high efficiency}, 
which means for each arriving text token, only one decoding step is required to generate all the corresponding speech tokens.

SyncSpeech is based on the proposed \textbf{T}emporal \textbf{M}asked generative \textbf{T}ransformer (TMT).
During inference, SyncSpeech adopts the Byte Pair Encoding (BPE) token-level duration prediction, which can access the previously generated speech tokens and performs top-k sampling. 
Subsequently, mask padding and greedy sampling are carried out based on  the duration prediction from the previous step. 

Moreover, sequence input is meticulously constructed to incorporate duration prediction and mask prediction into a single decoding step.
During the training process, we adopt a two-stage training strategy to improve training efficiency and model performance. First, high-efficiency masked pretraining is employed to establish a rough alignment between text and speech tokens within the sequence, followed by fine-tuning the pre-trained model to align with the inference process.

Our experimental results demonstrate that, in terms of generation efficiency, SyncSpeech operates at 6.4 times the speed of the current dual-stream TTS model for English and at 8.5 times the speed for Mandarin. When integrated with LLMs, SyncSpeech achieves latency reductions of 3.2 and 3.8 times, respectively, compared to the current dual-stream TTS model for both languages.
Moreover, with the same scale of training data, SyncSpeech performs comparably to traditional AR models in terms of the quality of generated English speech. For Mandarin, SyncSpeech demonstrates superior quality and robustness compared to current dual-stream TTS models. This showcases the potential of  SyncSpeech as a foundational model to integrate with upstream LLMs.


\section{Related Work}\label{sec:relatedwork}

\par
Different GPU simulators have been developed to explore and propose architectural changes to these architectures. Some of the most popular ones are single-thread simulators, such as Multi2Sim \cite{multi2sim} or GPGPU-Sim \cite{gpgpusimOriginal}. The former models the AMD Evergreen \cite{amdevergreen} architecture, while the latter models the NVIDIA Tesla \cite{teslaHotchips}. Recently, GPGPU-Sim was updated and renamed as the Accel-sim framework \cite{accelsim} to include some major features introduced in the NVIDIA Volta \cite{voltaPaper} architecture.

\par
Some previous works have developed parallel GPU simulators. The first one is Barra \cite{barra}, a GPU functional simulator focused on the NVIDIA Tesla architecture, which achieves a speed-up of 3.53x with 4 threads. However, this simulator models an old architecture and does not provide a timing model. Another work that models the NVIDIA Tesla architecture is GpuTejas \cite{gputejas}, which includes a timing model and achieves a mean speed-up of 17.33x with 64 threads. Unfortunately, executing GpuTejas in parallel has an indeterministic behavior, leading to accuracy simulation errors of up to 7.7\% compared to the single-threaded execution. One of the most successful parallel simulators is MGPUSim \cite{mgpusim}, an event-driven simulator that includes functional and timing simulation targeting the AMD GCN3 \cite{amdgcn3}. MGPUSim follows a conservative parallel simulation approach for parallelizing the different concurrent events during the simulation, preventing any deviation error from executing the simulator in parallel. It achieves a mean speed-up of 2.5x when executed with 4 threads.

\par
Several works have parallelized the GPGPU-Sim simulator. MAFIA \cite{mafia} can run different kernels concurrently in multiple threads but cannot parallelize single-kernel simulations. Lee et al. \cite{parallelGPUSim1} \cite{parallelGPUSim2} have proposed a simulator framework built on top of GPGPU-Sim. Their proposal needs at least three threads in order to run. Two threads are always dedicated to executing the Interconnect-Memory Subsystem and the Work Distribution and Control components. The rest of the threads are devoted to parallelizing the execution of the multiple SMs of the GPU. Lee et al. approach has an average 3\% simulation error compared to the original sequential simulation, achieving an average speed-up of 5x and up to 8.9x in some benchmarks.

\par
Some simulators, such as NVAS \cite{nvas}, address the highly time-consuming problem of simulations by reducing the detail of some components. For example, modeling the GPU on-chip interconnects in low detail in NVAS reports a 2.13x speed-up and less than 1\% benefit in mean absolute error compared to a high-fidelity model. Avalos et al. \cite{pcaKernelGpuSampling} rely on sampling techniques to simulate huge workloads.

\par
In contrast to previous works, we follow a simple approach to parallelize the Accel-sim framework simulator, the most modern academic GPU simulator used for research and capable of executing modern NVIDIA GPU architectures and workloads. Our proposal employs OpenMP \cite{openmp} to implement a scalable implementation that allows parallelizing the simulator with a user-defined number of threads. Moreover, our approach does not compromise the simulation accuracy and determinism when the simulator runs in parallel and provides the same results as the sequential version. Thus, it eases developing and debugging tasks. This makes our work more robust than the implementations of Lee et al. \cite{parallelGPUSim1} \cite{parallelGPUSim2}, and GpuTejas \cite{gputejas}, where the parallel version results differ from the single-threaded one. Moreover, our work is orthogonal to approaches such as the ones followed by NVAS \cite{nvas} and Avalos et al. \cite{pcaKernelGpuSampling}, which reduce the detail of some components and use sampling to speed up simulations even more.

\section{Parallelizing Accel-sim}\label{sec:work}

\par
This section describes how we have parallelized the Accel-sim framework simulator. 

\par
The principal components of the modeled GPUs in Accel-sim are shown in \autoref{fig:gpu_design_parallel}. The GPU has a dedicated main memory (VRAM), usually GDDR or HBM. There are several memory partitions, each with a channel to access the VRAM and the GPU's on-chip interconnect network. Memory partitions are divided into two sub-partitions, each with a slice of the L2 cache. Finally, there are a number of SMs (GPU cores) in charge of executing the user kernel instructions. 

\begin{figure}[ht]
  \centering
  \includegraphics[trim={0.6cm 0.6cm 0.6cm 0.6cm},clip,width=8cm]{images/gpuWholeDesign.pdf}
  \caption{GPU design.}
  \label{fig:gpu_design_parallel}
\end{figure}

\par
\autoref{fig:gpu_design_sm} shows the design of an SM. We can see that each core is divided into four sub-cores. They all have access to shared components such as an L1 instruction cache, the L1 Data Cache/Texture Cache/Shared memory, and shared execution units like FP64 for some architectures like Turing. Each sub-core is assigned a number of warps and executes them concurrently. The warp instruction fetcher requests one or several instructions from the L0 Instruction Cache every cycle. Once an instruction is received, it is decoded and stored in a buffer. An Issue Scheduler checks which warps have their oldest instruction ready and chooses one of them every cycle. Then, its operands are read from the register file, and finally, the instruction is usually executed in one of the different execution units of the sub-core (FP32, INT32, etc).

\begin{figure}[ht]
  \centering
  \includegraphics[trim={0.0cm 0.0cm 0.0cm 0.0cm},clip,width=6.5cm]{images/SMGeneral.pdf}
  \caption{SM design.}
  \label{fig:gpu_design_sm}
\end{figure}

\par
\hyperref[alg:simulator]{Algorithm~\ref*{alg:simulator}} shows the high-level structure of the simulator's code to model the above-described architectures. We can see that the main function calls the cycle function while the simulation is still ongoing. The cycle function has different tasks to do. The first one processes all the interconnection network work; as we can see, this task is split into different code regions, lines 8-11, 16, and 19. It also models the main memory (lines 12-14) and the accesses to the L2 cache (line 16). Then, it continues by executing the work in each GPU's SMs (lines 21-23), each with the different components found in \autoref{fig:gpu_design_sm}. Finally, the function increments the number of cycles that the GPU is active and tries to issue the remainder thread blocks to available SMs.

\begin{algorithm}
\begin{algorithmic}[1]
\Function{main}{}
    \While{\Not simulation.done()}
        \State cycle()
    \EndWhile
\EndFunction
\State
\Function{Cycle}{}
    \State doIcntToSm()
    \ForEach{memSubpartition $\in$ GPU\_memSubpartition}
        \State doMemSubpartitionToIcnt()
    \EndForEach
    \ForEach{memPartition $\in$ GPU\_Partition}
        \State memPartition.DramCycle()
    \EndForEach
    \ForEach{memSubpartition $\in$ GPU\_memSubpartition}
        \State doIcntToMemSubpartition()
        \State memSubpartition.cacheCycle()
    \EndForEach
    \State doIcntScheduling()
    \State
    \ForEach{SM $\in$ GPU\_SMs}
        \State SM.cycle()
    \EndForEach
    \State gpuCycle++
    \State issueBlocksToSMs()
\EndFunction
\end{algorithmic}
\caption{Simulator pseudo-code}\label{alg:simulator}
\end{algorithm}

\begin{figure}[ht]
  \centering
  \includegraphics[trim={0.0cm 0.0cm 0.0cm 0.0cm},clip,width=8cm]{images/gperftoolsGraph.pdf}
  \caption{Profiler output.}
  \label{fig:gperftools}
\end{figure}

\par
To find out which parts of the simulator are the more time-consuming ones, we have configured the Google Performance Tools (Gperftools) \cite{gperftools} CPU profiler to be executed with the Accel-sim in a node with the specifications shown in \autoref{tab:node_specs}. The simulator models an NVIDIA RTX 3080 TI GPU (\autoref{tab:gpu_specs}) and simulates one of the benchmarks found in \autoref{tab:benchmarks} (concretely, \textit{hotspot}). \autoref{fig:gperftools} depicts the output of the Gperftools profiler, which shows that over $93\%$ of the execution time is spent executing the SM cycles. This makes sense due to two reasons. First, there are many more SMs than memory partitions. Second, an SM has many more components and details than memory partitions or DRAM. As a result, we have a clear target: parallelize the execution of all the SMs, which are circled in red in \autoref{fig:gpu_design_parallel}.

\par
We have parallelized the simulator using OpenMP because it requires minimal changes. First, we have added the \texttt{-fopenmp} flag to the simulator compilation. Then, we added the clause of OpenMP to parallelize for-loops in line 20
(\texttt{\#pragma omp parallel for}) of \hyperref[alg:simulator]{algorithm~\ref*{alg:simulator}}.

\par
Moreover, we had to fix the data races that appeared due to parallelizing the SMs loop. Although the different hardware components modeled in the SM were previously properly isolated,  stats had data races. Most of the stats of the Accel-sim simulator are shared among all the SMs to report a unique stat for the GPU. Usually, stats are counters that are later used to compute percentages or ratios. Therefore, we have isolated all these stats to be calculated by SM instead of globally for the whole GPU. Once the kernel execution has finished, each of the stats reported by SM is gathered into a single GPU stat to report stats in the same way as the single-threaded simulator. Notice that this approach is much better than creating a critical section whenever we want to increase a stat counter because this kind of construct would damage performance due to frequent code serialization and lock management \cite{criticalSectionOpenMP}.

\par
Even though counters are the most common stat, sometimes stats are represented by hash tables or sets. For example, suppose we want to discover how many different memory addresses are accessed during simulation. In that case, we need a set (which does not contain duplicates) that tracks all the accessed addresses. However, maps or sets are not thread-safe structures in C++ \cite{stlThreadSafe}, meaning they undermine behavior and can lead to segmentation fault errors. Therefore, there are three possible solutions to this problem. The first one is to make this structure thread-safe by ourselves. The second one is to have one of these structures per SM and then compute the union of all SM data structures at the end. The third one is to find a place where the simulator is executed sequentially and handle that stat there (e.g. making the different insert/erase operations). It is clear that the last option is the best one. However, it may not always be possible, and the simulator user will have to choose between the first or the second option in those scenarios. The choice will depend on a trade-off between the performance drop of accessing a unique shared thread-safe structure shared by all SMs by employing critical sections or increasing memory consumption by having per-SM data structures.

\section{Evaluation}\label{sec:evaluation}

\par
This section evaluates the performance benefits of parallelizing the Accel-sim framework simulator \cite{accelsim}. First, we describe our evaluation methodology to measure the speed-up of the parallel simulator. Then, we present a sensitive analysis of how the speed-up changes depending on the number of threads devoted to the execution. Finally, we analyze the impact of the for-loop OpenMP scheduler in the simulator.

\subsection{Methodology}

\par
We have configured the simulator with the parameters shown in \autoref{tab:gpu_specs}, which represent an NVIDIA RTX 3080 TI GPU based on the Ampere architecture. 

\begin{table}
    \centering
    %\tiny
    \begin{tabular}{ |c|c|  }
    \hline
shs 
Parameter & Value \\
    \hline
    Core Clock & 1365 MHz \\
    Mem. Clock & 9500 MHz \\
    \# SM & 80 \\
    \# Warps per SM & 48  \\
    Total Shared memory/L1D per SM & 128 KB \\
    \# Mem. part. & 24 \\
    Total L2 cache  & 6 MB \\
    \hline
    \end{tabular}
    \caption{NVIDIA RTX 3080 TI GPU simulator parameters.}
    \label{tab:gpu_specs}
\end{table}

\par
\autoref{tab:benchmarks} lists the different benchmark suites that we have employed to measure the efficacy of the parallelization. They represent a variety of very commonly used benchmarks that exhibit different degrees of parallelism.

\begin{table}
    \centering
    %\tiny
    \begin{tabular}{ |c|  }
    \hline
    \textbf{Rodinia 3.1} \cite{rodinia} \\
    \hline
    \multirow{2}{=}{gaussian (gau), hotspot (hot), hybridsort (hyb), lavaMD, lud, myocyte (myo), nn, nw, pathfinder (path), srad\_v1 (srad)} \\
    \\
    \hline
    \textbf{Polybench} \cite{polybench} \\
    \hline
    \multirow{1}{=}{fdtd2d, syrk} \\
    \hline
    \textbf{Lonestar} \cite{lonestar} \\
    \hline
    \multirow{1}{=}{mst, sssp} \\
    \hline
    \textbf{Deepbench} \cite{deepbenchWeb} \\
    \hline
    \multirow{1}{=}{conv, gemm, rnn} \\
    \hline
    \textbf{Cutlass} \cite{cutlass} \\
    \hline
    \multirow{1}{=}{2560x16x2560 (cut\_1), 2560x1024x2560 (cut\_2)} \\
    \hline
    \end{tabular}
    \caption{Benchmarks.}
    \label{tab:benchmarks}
\end{table}

\par
All the simulations have been executed in a cluster of homogeneous nodes with the specifications shown in \autoref{tab:node_specs}.

\begin{table}
    \centering
    %\tiny
    \begin{tabular}{ |c|c|  }
    \hline
    \multicolumn{2}{|c|}{CPU} \\
    \hline
    Model & AMD Epyc 7401P \\
    Cores & 24 \\
    Threads & 48 \\
    Frequency & 2 GHz \\
    \hline
    \multicolumn{2}{|c|}{RAM} \\
    \hline
    Total size & 128 GB \\
    Technology & DDR4 \\
    \hline
    \end{tabular}
    \caption{Node specifications.}
    \label{tab:node_specs}
\end{table}

\subsection{Parallel Speed-Up}

\par
This subsection analyzes how the speed-up evolves depending on the number of threads used by the simulator. 

\par
\autoref{fig:eval_sensitivity} shows the speed-up obtained with 2, 4, 8, 16, and 24 threads, averaging 1.72x, 2.64x, 3.95x, 5.83x, and 7.08x, respectively, against the single-threaded simulator. Executing the simulator with more than eight threads is less efficient: the efficiency is 0.36 for 16 threads, and 0.3 for 24 threads. However, some specific benchmarks, such as \textit{lavaMD}, significantly benefit from this high number of threads, reaching a speed-up of \summaryMaxSpeedup{} and an efficiency of 0.88 with 16 threads. This speed-up reduces the simulation slowdown compared to real hardware from 10,748,031x of the single-threaded simulator to 766,423x of the parallel one. Moreover, this benchmark is one of the most benefited from parallelization as it achieves super speed-up with 2, 4, and 8 threads configurations.


\begin{figure*}[ht]
  \centering
  \includegraphics[trim={1.4cm 0.6cm 0.6cm 0.6cm},clip,width=17.5cm]{charts/main_speedup.pdf}
  \caption{Speed-up with a different number of threads against the single-threaded version.}
  \label{fig:eval_sensitivity}
\end{figure*}

\begin{figure*}[ht]
  \centering
  \includegraphics[trim={0.6cm 0.6cm 0.6cm 0.6cm},clip,width=17.5cm]{charts/schedulers_speedup.pdf}
  \caption{Speed-up obtained with the dynamic and static OpenMP for-loop scheduler with 2 and 16 threads against the single-threaded version.}
  \label{fig:eval_openmp_scheduler}
\end{figure*}

\begin{figure}[ht]
  \centering
  \includegraphics[trim={0.6cm 0.6cm 0.6cm 0.6cm},clip,width=8cm]{charts/ctas_per_kernel.pdf}
  \caption{Number of CTAs per kernel.}
  \label{fig:ctas_per_kernel}
\end{figure}

\par
Other workloads, such as \textit{myocyte}, which has a tiny number of CTAs (thread blocks) per kernel (2), do not benefit from parallelization, and it is penalized by running the OpenMP interface, resulting in minor slowdowns. To understand why, we need to know that CTAs are distributed in a round-robin fashion among the GPU SMs. As there are only two CTAs, only two SMs are active during the simulation. Therefore, parallelizing the execution of the rest of the SMs is useless. As shown in \autoref{fig:ctas_per_kernel}, workloads usually have many more CTAs per kernel than \textit{myocyte}, and more than the GPU's number of SMs (80).

\par
Computing the correlation factor between the speed-up obtained with 16 threads and the time to execute a workload in a single-thread, reveals a strong positive correlation with a value of 0.78. This means that the more time the application needs to be simulated in a single-thread, the more benefit it gets from parallelizing the simulator.

\subsection{OpenMP scheduler}

\par
Previous works \cite{ompScheduling1} \cite{ompScheduling2} analyzed the impact of the OpenMP for-loop scheduler. There are two main OpenMP schedulers: static and dynamic. In a static scheduler, the iterations of a for loop are distributed statically to threads. Therefore, it has little overhead and fits perfectly in regular and balanced applications. On the other hand, the dynamic scheduler assigns work to the threads when they are idle. Thus, it fits better in unbalanced environments. However, the dynamic scheduler has bigger overheads than the static one because it distributes the iterations of the loop at runtime, so it performs worse in balanced environments.

\par
\autoref{fig:eval_openmp_scheduler} shows how the OpenMP scheduler affects the benefits of parallelization depending on the number of threads in use. Both schedulers are configured with a granularity of one in the iteration distribution.

\par
We can see that applications with a negligible number of CTAs per kernel, such as \textit{myocyte}, perform similarly and do not benefit from parallelism. However, other workloads with a small number of CTAs per kernel, such as \textit{cut\_1}, benefit a lot from having the dynamic scheduler. Concretely, \textit{cut\_1} goes from the 0.97x speedup with the static scheduler to 1.61x when running with two threads. Other regular and balanced benchmarks, such as \textit{cut\_2} or \textit{lavaMD}, consistently perform better with the static than with the dynamic, as it does not have the scheduler overheads. Finally, there are workloads such as \textit{sssp} that, depending on the number of threads, perform better with one scheduler or the other. 

%This is because increasing the number of threads decreases the criticality of load unbalance.


\section{Conclusion}
This paper presents SyncSpeech, a dual-stream speech generation model built on a temporal masked transformer. SyncSpeech can efficiently generate low-latency streaming speech from the real-time text input, maintaining the high quality and robustness of the generated speech. We conducted comprehensive performance evaluations and analysis experiments in both English and Mandarin, demonstrating its capability as a foundational model for integration with upstream LLMs. In the future, SyncSpeech will be trained on larger datasets to further improve its performance.
 

\section{Limitations}
In this section, we will analyze the limitations of
SyncSpeech and discuss potential future work. SyncSpeech requires token-level alignment information, which is challenging to achieve for sentences with mixed languages, and preprocessing becomes time-consuming on large-scale datasets. In the future, we will explore semi-supervised duration prediction, which only requires the duration of a complete sentence without strict token-level alignment information, and integrate SyncSpeech into SLLM as a speech generation module. In addition, since the off-and-shelf streaming speech decoder relies on flow matching, it limits the off-the-shelf RTF and the FPL. Moreover,` current single-codebook acoustic tokens, such as WavTokenizer \cite{wavtokenizer}, do not support streaming decoding. In the future, we will investigate efficient and low-latency streaming speech decoders.


%%
%% The acknowledgments section is defined using the "acks" environment
%% (and NOT an unnumbered section). This ensures the proper
%% identification of the section in the article metadata, and the
%% consistent spelling of the heading.
\begin{acks}
This work has been supported by the CoCoUnit ERC Advanced Grant of the EU’s Horizon 2020 program (grant No 833057), the Spanish State Research Agency (MCIN/AEI) under grant PID2020-113172RB-I00, and the ICREA Academia program. We also thank Aurora Tomás for suggesting some changes to the paper.
\end{acks}

%%
%% The next two lines define the bibliography style to be used, and
%% the bibliography file.
\bibliographystyle{ACM-Reference-Format}
\bibliography{sample-base}

\end{document}
\endinput
%%
%% End of file `sample-sigconf.tex'.
