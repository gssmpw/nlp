\appendix


\section{Statistical Tests}

\subsection{Words-in-Context}\label{apd:stats_wic}

To statistically test the difference between the ``Same'' and ``Different'' sentence pairs, we conduct a two-sampled t-test on the null hypothesis that the distribution of average overlap for the two conditions is equal. The p-values are reported in table~(\ref{tab:t-test}).

\begin{table}[h]
\centering
\begin{tabular}{l|cc}
  \toprule
  \textbf{Model} & $t$ & $p$ \\
  \midrule
  DeepSeek-R1 & 12.1 & $<0.0001$ \\
  Mistral-8x22B & 11.1 & $<0.0001$ \\
  Mistral-8x7B & 7.82 & $<0.0001$ \\
  \bottomrule
\end{tabular}
\caption{Two-sample t-test on null of no difference between ``Same'' and ``Different'' conditions.}
\label{tab:t-test}
\end{table}

\subsection{Random Baseline}\label{apd:stats_formula}

The baseline number of overlapping experts of we expect to select at random in a given MoE layer can be formalized as follows.
Given independent two draws of $k$ items from $N$ elements (without replacement), the expected number of overlapping items between the two draws can be calculated according to the following formula:

$$
    \mathbb{E}[\text{overlap}] = \frac{k^2}{N}
$$

\begin{proof}
The first draw of $k$ items is at random. For the first item in the second draw, the probability of selecting the same item is $\frac{k}{N}$.

Using the linearity of expectation, the expected total overlap is $\sum_i^k\frac{k}{N}=k\cdot\frac{k}{N}=\frac{k^2}{N}$.
\end{proof}


\subsection{SAE token analysis} \label{apd:more_sae_tables}
\begin{table}[]
    \centering
    \begin{tabular}{lcc}
        \makecell{Input \\ Token} & \makecell{SAE \\ Value}  & Top 5 occurring experts \\
        \toprule
         wait & 14.97 & 47 \hfill 133 \hfill 138 \hfill 148 \hfill 183 \\
         Are & 1.7 & 90 \hfill 133 \hfill 136 \hfill 138 \hfill 170 \\
         ones & 1.24 & 57 \hfill 101 \hfill 121 \hfill 133 \hfill 136 \\
         No & 0.32 & 26 \hfill 47 \hfill 136 \hfill 138 \hfill 183 \\
         best & 0.16 & 15 \hfill 47 \hfill 81 \hfill 89 \hfill 133 \\
         attempt & 0.05 & 15 \hfill 81 \hfill 89 \hfill 95 \hfill 133 \\
         Wait & 0.02 & 81 \hfill 89 \hfill 95 \hfill 133 \hfill 136 \\
        \bottomrule
    \end{tabular}
    \caption{An analysis of selected experts by leveraging the trained Sparse Autoencoder. The target token is "wait."}
    \label{tab:sae_wait}
\end{table}

\begin{table}[]
    \centering
    \begin{tabular}{lcc}
        \makecell{Input \\ Token} & \makecell{SAE \\ Value}  & Top 5 occurring experts \\
        \toprule
         giving & 4.47 & 11 \hfill 15 \hfill 81 \hfill 89 \hfill 90 \\
         hypothesis & 4.04 & 11 \hfill 15 \hfill 81 \hfill 89 \hfill 90 \\
         definitely & 2.26 & 11 \hfill 15 \hfill 81 \hfill 89 \hfill 90 \\
         perform & 1.96 & 11 \hfill 15 \hfill 81 \hfill 89 \hfill 90 \\
         priority & 1.82 & 11 \hfill 15 \hfill 81 \hfill 89 \hfill 90 \\
         analyzing & 1.51 & 11 \hfill 15 \hfill 81 \hfill 89 \hfill 90 \\
         scientific & 1.17 & 11 \hfill 15 \hfill 81 \hfill 89 \hfill 90 \\
        \bottomrule
    \end{tabular}
    \caption{An analysis of selected experts by leveraging the trained Sparse Autoencoder. The target token is "hypothesis."}
    \label{tab:sae_hypothesis}
\end{table}


\begin{table}[]
    \centering
    \begin{tabular}{lcc}
        \makecell{Input \\ Token} & \makecell{SAE \\ Value}  & Top 5 occurring experts \\
        \toprule
        combining & 13.50 & 11 \hfill 15 \hfill 69 \hfill 90 \hfill 136\\
        formatted & 13.32 & 11 \hfill 15 \hfill 69 \hfill 90 \hfill 136\\
        frequencies & 13.31 & 11 \hfill 15 \hfill 69 \hfill 90 \hfill 136\\
        accessible & 13.31 & 11 \hfill 15 \hfill 26 \hfill 136 \hfill 138\\
        restrictions & 13.29 & 11 \hfill 15 \hfill 26 \hfill 136 \hfill 138\\
        rejected & 13.13 & 11 \hfill 15 \hfill 69 \hfill 90 \hfill 136\\
        559 & 9.92 & 11 \hfill 15 \hfill 69 \hfill 90 \hfill 136\\
        UUID & 6.83 & 11 \hfill 15 \hfill 26 \hfill 136 \hfill 138\\
        854 & 6.62 & 11 \hfill 15 \hfill 69 \hfill 90 \hfill 136\\
        obtaining & 6.44 & 15 \hfill 90 \hfill 95 \hfill 136 \hfill 138\\
        \bottomrule
    \end{tabular}
    \caption{An analysis of selected experts by leveraging the trained Sparse Autoencoder. We selected the top activating SAE head on the word "UUID" and used its activation's value to identify other semantically similar tokens. The top 5 occurring experts are highly consistent across these varying words.}
    \label{tab:sae_uuid}
\end{table}


\begin{table}[]
    \centering
    \begin{tabular}{lcc}
        \makecell{Input \\ Token} & \makecell{SAE \\ Value}  & Top 5 occurring experts \\
        \toprule
        plan & 10.25 & 4 \hfill 81 \hfill 118 \hfill 121 \hfill 136\\
        block & 7.59 & 47 \hfill 81 \hfill 101 \hfill 118 \hfill 121\\
        panel & 1.3 & 11 \hfill 47 \hfill 81 \hfill 89 \hfill 136\\
        've & 0.78 & 11 \hfill 89 \hfill 95 \hfill 121 \hfill 133\\
        says & 0.41 & 118 \hfill 121 \hfill 138 \hfill 144 \hfill 148\\
        data & 0.27 & 11 \hfill 15 \hfill 47 \hfill 81 \hfill 136\\
        memory & 0.26 & 4 \hfill 8 \hfill 11 \hfill 26 \hfill 121\\
        Wait & 0.03 & 81 \hfill 89 \hfill 95 \hfill 133 \hfill 136\\
        But & 0.02 & 11 \hfill 15 \hfill 28 \hfill 81 \hfill 89\\
        core & 0.02 & 11 \hfill 22 \hfill 79 \hfill 109 \hfill 118\\
        \bottomrule
    \end{tabular}
    \caption{Analysis of selected experts for the token "plan" using a trained Sparse Autoencoder.}
    \label{tab:sae_plan}
\end{table}

\begin{table}[]
    \centering
    \begin{tabular}{lcc}
        \makecell{Input \\ Token} & \makecell{SAE \\ Value}  & Top 5 occurring experts \\
        \toprule
        changes & 8.48 & 15 \hfill 81 \hfill 89 \hfill 90 \hfill 95\\
        THEN & 8.33 & 15 \hfill 28 \hfill 81 \hfill 89 \hfill 95\\
        Unless & 8.3 & 15 \hfill 28 \hfill 81 \hfill 89 \hfill 95\\
        along & 8.26 & 15 \hfill 28 \hfill 81 \hfill 89 \hfill 95\\
        puts & 8.1 & 15 \hfill 28 \hfill 81 \hfill 89 \hfill 95\\
        matters & 7.77 & 15 \hfill 81 \hfill 89 \hfill 90 \hfill 95\\
        approach & 6.11 & 15 \hfill 28 \hfill 81 \hfill 89 \hfill 95\\
        close & 5.47 & 15 \hfill 81 \hfill 89 \hfill 90 \hfill 95\\
        floor & 5.23 & 15 \hfill 28 \hfill 81 \hfill 89 \hfill 95\\
        outline & 4.77 & 15 \hfill 28 \hfill 81 \hfill 89 \hfill 90\\
        \bottomrule
    \end{tabular}
    \caption{Analysis of selected experts for the token "approach" using a trained Sparse Autoencoder.}
    \label{tab:sae_approach}
\end{table}


\begin{table}[]
    \centering
    \begin{tabular}{lcc}
        \makecell{Input \\ Token} & \makecell{SAE \\ Value}  & Top 5 occurring experts \\
        \toprule
         work & 4.22 & 11 \hfill 15 \hfill 89 \hfill 90 \hfill 95 \\
         better & 2.44 & 15 \hfill 89 \hfill 90 \hfill 95 \hfill 136 \\
         two & 2.23 & 11 \hfill 15 \hfill 23 \hfill 69 \hfill 89 \\
         error & 2.19 & 15 \hfill 23 \hfill 69 \hfill 89 \hfill 90 \\
         let & 2.16 & 15 \hfill 69 \hfill 89 \hfill 95 \hfill 138 \\
         I & 2.14 & 11 \hfill 15 \hfill 81 \hfill 89 \hfill 90 \\
         $[$ & 2.14 & 11 \hfill 15 \hfill 69 \hfill 90 \hfill 95 \\
         three & 2.04 & 15 \hfill 69 \hfill 89 \hfill 90 \hfill 95 \\
         case & 1.81 & 11 \hfill 15 \hfill 89 \hfill 90 \hfill 95 \\
         per & 1.71 & 69 \hfill 89 \hfill 132 \hfill 138 \hfill 148 \\
        \bottomrule
    \end{tabular}
    \caption{An analysis of selected experts by leveraging the trained Sparse Autoencoder. We selected the top activating SAE head on the word "work" and used its activation's value to identify other semantically similar tokens. The top 5 occurring experts are highly consistent across these varying words.}
    \label{tab:sae_work}
\end{table}

\begin{table}[]
    \centering
    \begin{tabular}{lcc}
        \makecell{Input \\ Token} & \makecell{SAE \\ Value}  & Top 5 occurring experts \\
        \toprule
         helps & 8.36 & 11 \hfill 15 \hfill 81 \hfill 89 \hfill 90 \\
         resulted & 8.29 & 11 \hfill 15 \hfill 69 \hfill 89 \hfill 90 \\
         parameter & 8.27 & 11 \hfill 15 \hfill 81 \hfill 89 \hfill 90 \\
         consider & 8.14 & 11 \hfill 15 \hfill 89 \hfill 90 \hfill 180 \\
         positions & 7.96 & 11 \hfill 15 \hfill 81 \hfill 89 \hfill 90 \\
         ensuring & 7.92 & 11 \hfill 15 \hfill 69 \hfill 89 \hfill 90 \\
         include & 7.83 & 11 \hfill 15 \hfill 69 \hfill 89 \hfill 90 \\
         generate & 7.79 & 11 \hfill 15 \hfill 68 \hfill 89 \hfill 90 \\
         separate & 7.72 & 11 \hfill 15 \hfill 81 \hfill 89 \hfill 90 \\
         limitations & 7.71 & 11 \hfill 15 \hfill 81 \hfill 89 \hfill 90 \\
        \bottomrule
    \end{tabular}
    \caption{An analysis of selected experts by leveraging the trained Sparse Autoencoder. We selected the top activating SAE head on the word "consider" and used its activation's value to identify other semantically similar tokens. The top 5 occurring experts are highly consistent across these varying words.}
    \label{tab:sae_consider}
\end{table}

\begin{table}[]
    \centering
    \begin{tabular}{lcc}
        \makecell{Input \\ Token} & \makecell{SAE \\ Value}  & Top 5 occurring experts \\
        \toprule
         bet & 17.16 & 47 \hfill 133 \hfill 136 \hfill 138 \hfill 148 \\
         Wait & 7.94 & 81 \hfill 89 \hfill 95 \hfill 133 \hfill 136 \\
         notes & 6.79 & 71 \hfill 89 \hfill 90 \hfill 133 \hfill 138 \\
         probably & 4.97 & 48 \hfill 57 \hfill 101 \hfill 136 \hfill 138 \\
         output & 4.59 & 81 \hfill 89 \hfill 133 \hfill 136 \hfill 138 \\
         20 & 3.92 & 81 \hfill 89 \hfill 95 \hfill 136 \hfill 138 \\
         fail & 3.53 & 81 \hfill 89 \hfill 121 \hfill 133 \hfill 136 \\
         It & 2.87 & 89 \hfill 133 \hfill 136 \hfill 138 \hfill 183 \\
         ones & 2.06 & 57 \hfill 101 \hfill 121 \hfill 133 \hfill 136 \\
         attempt & 1.72 & 15 \hfill 81 \hfill 89 \hfill 95 \hfill 133 \\
        \bottomrule
    \end{tabular}
    \caption{An analysis of selected experts by leveraging the trained Sparse Autoencoder. We selected the top activating SAE head on the word "bet" and used its activation's value to identify other semantically similar tokens. The top 5 occurring experts are highly consistent across these varying words.}
    \label{tab:sae_bet}
\end{table}

\begin{table}[]
    \centering
    \begin{tabular}{lcc}
        \makecell{Input \\ Token} & \makecell{SAE \\ Value}  & Top 5 occurring experts \\
        \toprule
         examine & 7.28 & 15 \hfill 81 \hfill 89 \hfill 95 \hfill 136 \\
         arg & 5.45 & 11 \hfill 15 \hfill 81 \hfill 89 \hfill 95 \\
         twice & 5.33 & 11 \hfill 15 \hfill 28 \hfill 81 \hfill 89 \\
         walls & 5.33 & 15 \hfill 28 \hfill 81 \hfill 89 \hfill 95 \\
         if & 4.29 & 11 \hfill 15 \hfill 81 \hfill 89 \hfill 95 \\
         so & 3.66 & 11 \hfill 15 \hfill 81 \hfill 89 \hfill 95 \\
         actions & 3.58 & 11 \hfill 15 \hfill 81 \hfill 89 \hfill 95 \\
         1 & 3.52 & 11 \hfill 15 \hfill 81 \hfill 89 \hfill 95 \\
         'll & 3.42 & 15 \hfill 81 \hfill 89 \hfill 95 \hfill 121 \\
         same & 3.31 & 15 \hfill 28 \hfill 81 \hfill 89 \hfill 95 \\
        \bottomrule
    \end{tabular}
    \caption{An analysis of selected experts by leveraging the trained Sparse Autoencoder. We selected the top activating SAE head on the word "so" and used its activation's value to identify other semantically similar tokens. The top 5 occurring experts are highly consistent across these varying words.}
    \label{tab:sae_so}
\end{table}


In tables~(\ref{tab:sae_wait}, \ref{tab:sae_hypothesis}, \ref{tab:sae_uuid}, \ref{tab:sae_plan}, \ref{tab:sae_approach}, \ref{tab:sae_work}, \ref{tab:sae_consider}, \ref{tab:sae_bet}, \ref{tab:sae_so}), we show top experts by leveraging SAE activations on a selection of hand chosen interesting tokens. We find striking consistency across expert selection when using the SAE to find semantically similar concepts.

\section{DiscoveryWorld Environment Details}
\label{app:discodetails}
DiscoveryWorld features 8 tasks centered on different scientific fields. We choose to evaluate R1 on the "Reactor Lab" environment, where the stated goal is to: ``discover a relationship (linear or quadratic) between a physical crystal property (like temperature or density) and its resonance frequency through regression, and use this to tune and activate a reactor.''

In Figure (\ref{fig:dw_reactor_pic}), we show the Reactor Lab environment, where the agent has access the crystals and microscope in its inventory. The pixel-based visual observation itself it not used by R1 directly, but the prompt (see below) contains a structured description of the environment. 

\begin{figure}[ht]
    \centering
    \includegraphics[width=\linewidth]{Figures/reactor_lab_easy_step50.png} 
    \caption{Visual observation in the Reactor Lab environment at step 50.}
    \label{fig:dw_reactor_pic}
\end{figure}

We show an example prompt and chain of thought output by R1 in the Reactor Lab environment below. 

\onecolumn


\begin{tcolorbox}[colback=green!5!white, colframe=green!75!black, title=Example Prompt on DiscoveryWorld Reactor Lab, breakable]
{\small  \verbatiminput{prompts.txt}}
\end{tcolorbox}

\begin{tcolorbox}[colback=orange!5!white, colframe=orange!75!black, title=Example Reasoning Output from DeepSeek-R1 (step 50), breakable, width=\textwidth]
{\small  \verbatiminput{cot.txt}}
\end{tcolorbox}
