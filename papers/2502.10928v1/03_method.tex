\section{Experiments} %

\subsection{Words-in-Context} 

We leverage polysemy to test for semantic specialization in expert activation. If words that are written the same but have different meanings are routed differently, then this is evidence that routing occurs based on meaning. To test this hypothesis, we use the WiC dataset~\cite{pilehvar2018wic} (CC BY-NC 4.0), which consists of two types of paired sentences: 1) pairs where a target word has the same sense and 2) pairs where the target word has different senses across sentences.
For each target words and sentence, we prompt the model with:
"Please define \{target word\} in this context."
Additionally, we include an internal reasoning step:
``<think> Okay, so I need to figure out the meaning of the word \{target word\}.'' to ensure that the subsequent inference isolates the word in question instead of additional thinking tokens.

\subsection{DiscoveryWorld}
DiscoveryWorld \citep{jansen2024discoveryworld}
is a large-scale agentic environment suite that tests the abilities of an agent to perform the scientific method. Each environment has a terminal goal, for example, we study "Reactor Lab" where the agent must tune the frequency of quantum crystals to activate a reactor. To succeed, the agent must formulate and test hypotheses by using available tools, literature, and its own memory. DiscoveryWorld is notably difficult for frontier models like GPT4o, and even take human experts hundreds of in-game steps to complete a task \citep{jansen2024discoveryworld}. Hence, DiscoveryWorld offers a testbed to examine the long-horizon reasoning capabilities of R1. Building on the Words-in-Context experiment, we want to know if a similar phenomena of expert specialization can be found for the reasoning patterns that we observe within DeepSeek-R1's chain of thought.




\subsubsection*{Sparse Autoencoders}

To get a clearer picture of how these patterns are invoked internally, we 
employ SAEs to learn a mapping between the internal activations of R1 and a set of underlying semantic structures exhibited by the model. 
Briefly, an SAE learns a compressed representation of input vectors $x \in \mathbb{R}^d$. The encoder maps inputs to a higher-dimensional latent space, while the decoder reconstructs the input from the latent representation. Given an encoding dimension $n$, we define the encoder and decoder as:  
$
z = \max(0, W_{\text{enc}} x + b_{\text{enc}})
$
 and
$
\hat{x} = W_{\text{dec}} z
$

where $W_{\text{enc}} \in \mathbb{R}^{n \times d}$ and $W_{\text{dec}} \in \mathbb{R}^{d \times n}$ are the learnable weight matrices of the encoder and decoder respectively, and $b_{\text{enc}} \in \mathbb{R}^{n}$ is a bias term.  
The model is trained using a loss function that balances reconstruction accuracy and sparsity:  
$L = \| x - \hat{x} \|_2^2 + \lambda \| z \|_1$


where the first term is the mean squared error for reconstruction, and the second term is an $L_1$ penalty that encourages sparsity in the latent activations, where we choose $\lambda = 5$ as the trade-off between reconstruction fidelity and sparsity.  











