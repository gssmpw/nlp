\section{Conclusion}
\label{sec:conclusion}
% In this paper, we investigated Elo rating in prediction and in ranking. Through the new lens of regret minimization, we find that the prediction performance of online rating algorithms are actually affected by both data sparsity level and model capacity. When data is sparse, Elo rating and its variants achieve the best performance, even though the games does not follow a BT model. When data is dense, high capacity models begin to show their strength in improving prediction accuracy. We further investigated the ranking ability, and find that  for pairwise ranking, the performance is highly correlated with prediction accuracy. As for total ordering, Elo may not always give a consistent ranking.

% Our results suggest that, the effectiveness of Elo rating (and other rating algorithms) is strongly related to the scenario where it is applied to. Also, it is a interesting direction to design new online rating algorithms based on the regret minimization interpretation for online rating.

% In this paper, we investigated Elo rating in prediction and in ranking. Through the new lens of regret minimization, we find that the prediction performance of online rating algorithms are strongly affected by data sparsity level. Our results suggest that, the effectiveness of Elo rating (and other rating algorithms) is strongly related to the scenario where it is applied to.  

In this paper, we find that real-world game data are non-BT and non-stationary. However despite the model misspecification, Elo still achieves strong predictive performance. We interpret this phenomenon through three perspectives:
first we interpret Elo through a regret-minimization framework, proving its effectiveness under model misspecification. Second we conduct extensive synthetic and real-world experiments, and find that data sparsity plays a crucial role in algorithms' prediction performance. Finally we show a strong correlation between prediction accuracy and pairwise ranking performance.