% \section{Testing Bradley-Terry on Real-world Games}



\section{Experiments on Real-world Matching Data}

In this section, we conduct experiments on real-world datasets. Surprisingly, we find that most games deviate significantly from the assumptions of the BT model and stationarity, raising questions on the reliability of Elo. Despite these deviations, Elo frequently outperforms more complex rating systems, such as mElo and pairwise models, which are designed to account for non-BT components in the data, particularly in terms of win rate prediction.

\subsection{Real-world games are neither BT nor stationary}
\label{sec:hypo}

\begin{table}[t]
\centering
\addtocounter{footnote}{+1}  
\resizebox{0.8\columnwidth}{!}{
\begin{tabular}{|l|c|c||c|c|}
\hline
Dataset              & $N$    & $2T/N$  & BT Model Test  & $p$-value          \\ \hline
\texttt{Renju}       & 5k   & 49.8  & 150.0    & $<10^{-10}$               \\
% \texttt{chess-small} & 30k  & 68.5  & 268.5    & $<10^{-10}$              \\
\texttt{Chess}       & 185k & 125.4 & 2020.1   & $<10^{-10}$            \\
\texttt{Tennis}      & 7k   & 52.5  & 37.3     & $< 10^{-4}$       \\
\texttt{Scrabble}    & 15k  & 200.7  & 142.2    & $<10^{-10}$               \\
\texttt{StarCraft}   & 22k  & 38.7 & 775.8    & $<10^{-10}$              \\
\texttt{Go}          & 426k & 60.4  & 193411.2 & $<10^{-10}$             \\ 
% \texttt{LLM Arena}   & 54   & 7910.2  & 17.0~\footnotemark     & $1\times 10^{-3}$                  \\
\texttt{LLM Arena}   & 129   & 23156.9  & 73.1~\footnotemark     & $1\times 10^{-3}$                  \\
\texttt{Hearthstone}   & 27   & 4626.1  & 49.0     & $<10^{-4}$                  \\
% \texttt{AlphaStar}   & 888   & 1774.0  & 6902.0     & $<10^{-10}$                  \\
% \texttt{5,4-Blotto}   & 56   & 110.0  & 8.1     & $0.018$                \\
% \texttt{10,5-Blotto}   & 1k   & 2000.0  & 784.8    & $<10^{-10}$            \\
% \texttt{mixedchess-dense}   & 2k   & 4119.9  & 8.091    & 0.017          \\
\hline
\end{tabular}
}

% \chijin{use top environment for table}

\caption{Summary of real world datasets and BT-model testing results. $N$ is the total number of players, and $2T/N$ is the average number of games each player played.}
\label{tab:hypotheis}

\end{table}
\footnotetext[2]{{The likelihood-ratio test is performed for the LLM arena dataset using a different method of augmenting the features. See details in Appendix \ref{sec:lr-test}.}}




In the Elo rating update rule (\ref{eq:elo}), $\sigma(\theta[i]-\theta[j])$ represents the predicted win probability of player  i  against player  j . This prediction relies on the assumption that the underlying data follows the Bradley-Terry (BT) model. However, whether real-world data truly follows  a stationary BT model remains unverified.

In this section, we conduct a likelihood ratio test on real-world datasets to examine the hypothesis that real-world game outcomes are generated by the BT model. Our results indicate that, across all examined datasets, the hypothesis is rejected, suggesting that real-world data does not follow the BT model. Furthermore, we provide evidence that both matchmaking and player skill exhibit non-stationarity in real-world games. These findings suggest that model misspecification widely exists when applying Elo to real-world data.



\paragraph{Rejecting BT on real-world dataset}


Note that the Bradley-Terry model can be equivalently written as a logistic regression model, where the parameter $\theta$ is $N$-dimensional, and every game has a feature vector $x_t:=\mathbf{e}[i_t]-\mathbf{e}[j_t]\in \R^{N}$. 
We randomly split $[T]$ into $\cT_{\rm train}$ and $\cT_{\rm test} = [T]\setminus \cT_{\rm train}$. Then the logistic regression loss on the test set is defined as

\begin{align*}
\cL_{\rm test}(\theta)=& -\sum_{t\in [T]} \left[o_t\ln(\sigma(\theta^\top x_t))\right. \\
& \qquad\quad \left. + (1-o_t)\ln(1-\sigma(\theta^\top x_t))\right].
\end{align*}

%We examine the validity of the Bradley-Terry model on eight real-world datasets via the Likelihood Ratio (LR) Test, by viewing a Bradley-Terry MLE problem as a logistic regression problem, where every game has feature $x_t:=\mathbf{e}[i_t]-\mathbf{e}[j_t]\in \R^{N+2}$. We test the goodness-of-fit of this model by comparing it to an augmented model with two more parameters, where the feature for every game is $[\mathbf{e}[i_t]-\mathbf{e}[j_t];\alpha_t;\beta_t]\in \R^{N+2}$. (See Appendix~\ref{sec:lr-test} for more details).

Next, we \emph{augment} the logistic model by adding two additional parameters $\alpha\in \R^2$, and a two dimensional feature $g_t\in \R^2$ for every game. In practice, $g_t$ is constructed using the training set $\cT_{\rm train}$. Define the negative log likelihood of the augmented model as
\begin{align*}
\Tilde{\cL}_{\rm test}([\theta; \alpha])=& -\sum_{t\in \cT_{\rm test}} \left[o_t\ln(\sigma(\theta^\top x_t + \alpha^\top g_t))\right. \\
& \qquad \left. + (1-o_t)\ln(1-\sigma(\theta^\top x_t + \alpha^\top g_t))\right].
\end{align*}
If dataset is indeed realizable by a BT model with true scores $\theta^\star$, the augmented model is also realizable with $[\theta^\star;\textbf{0}]$ as long as $g_t$ and $o_t$ are independent, because
\[
\E[o_t|i_t,j_t,g_t] = \sigma(\theta^\star[i_t] - \theta^\star [j_t]).
\]
Therefore, we can test the BT model by testing the null hypothesis $H_0:\alpha=0$.

We employ the standard likelihood ratio test, which uses the log-likelihood ratio statistic:
\[
\Lambda := 2\left[\min_{\theta\in\R^N}\cL_{\rm test}(\theta) - \min_{\theta\in\R^{N},\alpha\in\R^2}\tilde\cL_{\rm test}([\theta;\alpha])\right].
\]


\iffalse
\begin{align*}
\tilde\cL(\tilde\theta)&:= - o_t\log(\sigma(\tilde\theta[i_t]-\tilde\theta[j_t]+\tilde\theta_{N+1}\alpha_t+\tilde\theta_{N+2}\beta_t))\\
&- (1-o_t)\log(1-\sigma(\tilde\theta[i_t]-\tilde\theta[j_t]+\tilde\theta_{N+1}\alpha_t+\tilde\theta_{N+2}\beta_t)).
\end{align*}
\fi
By Wilk's Theorem~\citep{wilks1938large,sur2019likelihood}, under the null hypothesis that the real-world dataset is generated by Bradley-Terry model,
$\Lambda$
is asymptotically distributed as a chi-square distribution with two degrees of freedom. This allows us to compute the $p$-value, which is the probability that the test statistic occurs under the null hypothesis due to pure chance.

For high-dimensional logistic regression,~\citet{sur2019likelihood} showed that $\Lambda$ is asymptotically distributed as a scaled chi-square distribution if $T/N = O(1)$. We applied the correction suggested by~\citet{sur2019likelihood}  by computing the $p$-value conservatively with $1.25\chi^2_2$. This factor is computed when the number of samples is $5$ times the model dimension, although the number of samples is at least $19$ times the model dimension in our datasets.

% \subsubsection{Rejecting BT on real-world dataset}
With the test statistic $\Lambda$, we are able to perform the test. We construct the augmented features $\{g_t\}_{t\in [T]}$ by fitting fit $\theta_{\rm train}$ via regularized MLE on $\cT_{\rm train}$. We then define 
\[
g_t = [\theta_{\rm train}[i_t], \theta_{\rm train}[j_t]]
\]
for every $t$ in the test set. Under the Bradley-Terry model, the original logistic regression $\cL_{\rm test}(\theta)$ already has sufficient information to predict $o_t$, so adding the score computed on an independent training set cannot help prediction (up to random noise).

We compute the log-likelihood ratio statistic $\Lambda$ for eight real-world datasets and report the corresponding $p$-values (see Table~\ref{tab:hypotheis}).
%We applied the correction suggested by~\citet{sur2019likelihood}  by  conservatively computing the $p$-value with $1.25\chi^2_2$. This factor is computed when the number of samples is $5$ times the model dimension, although the number of samples ($T$) is at least $19$ times the model dimension ($N+2$) in our datasets.
It can be seen that we can reject the null hypothesis, namely realizability of the Bradley-Terry model, with extremely high confidence, for all eight datasets.

%\yw{should we make a note here that realizability $\neq$ calibration?}
%\subsection{Elo score is not close to Regularization Path}



% \paragraph{Non-stationary matchmaking and player skills in real datasets}
\paragraph{Matchmaking and player skills are non-stationary}
Additional observations that we draw from real-world datasets are the existence of non-stationary matchmaking and player's skills. We postpone details to Appendix \ref{sec:appendix-matchmaking}. These phenomena suggest that the real world games are non-BT and non-stationary. Consequently, viewing Elo rating as fitting a underlying BT-model might not be appropriate. 




