\section{Non-stationary matchmaking and player skills in real datasets}
\label{sec:appendix-matchmaking}
Another observation that we draw from real-world datasets is the existence of non-trivial matchmaking. We computed the correlation coefficient between $\{\theta_{\rm train}[i_t]\}_{t\in \cT_{\rm test}}$ and $\{\theta_{\rm train}[j_t]\}_{t\in \cT_{\rm test}}$, and found significant positive correlation for most datasets (see Table~\ref{tab:mm_test}). In other words, in many real datasets, stronger (higher-rated) players are matched with stronger opponents. We visualize the matchmaking in \texttt{chess} in Fig.~\ref{fig:chess-mm}. Indeed, the Elo score of the two players are highly correlated, and most games are played between two players within $20\%$ in terms of the percentile difference based on their Elo scores. Since the Elo score may vary from time to time, the matchmaking distribution should not be considered as stationary. 




\begin{table}[ht]
\centering
\addtocounter{footnote}{+1}  
\begin{tabular}{|l|c|c|}
\hline
Dataset             & Matchmaking Test & $p$-value \\ \hline
\texttt{Renju}        & 0.36             & $<10^{-10}$          \\
%\texttt{llm arena}   & 54   & 7910.2  & 1.95     & $0.61$            & 0.37             & $<10^{-10}$          \\
\texttt{Chess}        & 0.40             & $<10^{-10}$          \\
\texttt{Tennis}       & 0.19             & $<10^{-10}$          \\
\texttt{Scrabble}           & 0.57             & $<10^{-10}$          \\
\texttt{StarCraft}          & 0.46             & $<10^{-10}$         \\
\texttt{Go}                  & 0.29             & $<10^{-10}$         \\ 
\texttt{LLM Arena}     & 0.37             & $<10^{-10}$          \\
\texttt{Hearthstone}     & -0.07             & $<10^{-10}$          \\
% \texttt{mixedchess-dense}     & 0.40             & $<10^{-10}$  \\
\hline
\end{tabular}
\caption{Summary of real world datasets matchmaking hypothesis testing results.}
\label{tab:mm_test}

\end{table}


\begin{figure}[ht]
    \centering
    \includegraphics[width=\columnwidth]{figures/chess-mm-sns.png}
    \caption{\textbf{Matchmaking in \texttt{chess} dataset.} \textbf{L:} scatter plot of Elo score of the two players for each game, down-sampled for clarity; \textbf{R:} histogram for the percentile ranking difference of two players.}
    \label{fig:chess-mm}
\end{figure}



% \section{Additional Experiments with Online Elo Score}
% \label{sec:appendix-addition}

% \subsection{Bootstrap Experiments}
% \label{sec:appendix-bootstrap}
\paragraph{Bootstrap Experiments} Another evidence of matchmaking comes from the nonstationarity of gradients. If the distribution of $\{(i_t,j_t,o_t)\}$ is exchangeable, we can permute the order of the games randomly and the resulting Elo score $\theta^{\rm bootstrap}$ should be identically distributed. We can therefore detect nonstationarity by comparing $\theta^{\rm elo}$ with the distribution of $\theta^{\rm bootstrap}$. 

We compute the Elo score on $100$ independent permutations in the each dataset. The average of these samples is called the bootstrap average, and denoted by $\bar{\theta}^{\rm bootstrap}$.

The results for \texttt{chess} is presented in Fig.~\ref{fig:chess-bootstrap}. It can be seen that $\theta^{\rm elo}$, the Elo score computed with the original order of gradients, is a significant outlier and is not identically distributed with $\theta^{\rm bootstrap}$ with high probability ($p=0.01$ via the permutation test).

\begin{figure}[h]
    \centering
    \includegraphics[width=0.6\textwidth]{figures/bootstrap-chess-combined-sns.png}
    \caption{Elo score vs. bootstrap Elo scores in \texttt{chess}. \textbf{Left:} cosine similarity to the mean of $\theta^{\rm bootstrap}$; \textbf{Right:} visualization of $\theta^{\rm elo}$ vs. $\theta^{\rm bootstrap}$ via SVD for $\eta=0.02$.}
    \label{fig:chess-bootstrap}
\end{figure}

\paragraph{Varying player strengths} Other than matchmaking, we also want to point out that the player's strength may not be stationary. It is common that for a pair of players, for example, in tennis, their head-to-head game results can change dramatically over time. 

These phenomenons suggest that, in real world games, both matchmaking and players' behaviour are not stationary and non-BT. Therefore, viewing Elo rating as fitting a underlying BT-model might not be appropriate.

