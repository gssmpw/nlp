\section{Dataset description}
\label{sec:appendix-dataset}
\paragraph{Renju} We use the RenjuNet dataset~\footnote{\url{https://www.renju.net/game/}}, where $N=5013$ and $T=124948$.
% \paragraph{LLM Arena} We use the Chatbot Arena~\footnote{\url{https://chat.lmsys.org/}} dataset, where $N=54$ LLMs are evaluated through $T=213576$ battles.
\paragraph{LLM Arena} We use the Chatbot Arena~\footnote{\url{https://chat.lmsys.org/}} dataset, where $N=129$ LLMs are evaluated through $T=1493621$ battles.
\paragraph{Chess} We use online standard chess matches on the Lichess open database~\footnote{\url{https://database.lichess.org/}} in the year of 2014, which covers $N=184920$ players and $T=11595431$ games.

\paragraph{Mixedchess} We use online matches on the Lichess database ~\footnote{\url{https://database.lichess.org/}}, for the following variants of chess: Antichess, Atomic, Chess960, Crazyhouse, Horde, King of the Hill, Racing Kings and Three-check. We collect all the matches from 2014 to 2024, and combine them together to form a large dataset with $N=2467134$ and $T=115525041$. We apply filtering \footnote{To create denser dataset, we conduct filtering on mixedchess and go. Our filtering method is: for a given threshold, we delete all the players that plays less than this threshold. Then we only consider the remaining players and the games played between them.} with threshold  $= 10000$, and create \texttt{mixedchess-dense}, where $N= 2862$ ,$T= 11791126$.

% In addition, we used the matches in 2014-07 as \texttt{chess-small}, with the hope that the skill level of players would remain closer to constant in shorter time frame, which contains $T=1048440$ games and $N=30631$.
\paragraph{Tennis} We make use of an online repository created by Jack Sackmann~\footnote{\url{https://github.com/JeffSackmann/tennis_atp}}. We used all $T=190230$ games between $7245$ players.
\paragraph{StarCraft} We downloaded match records from human players from Aligulac~\footnote{\url{http://aligulac.com/about/db/}}. $N=22056$ and $T=427042$.
\paragraph{Scrabble} We used the raw data provided in~\citet{fivethirtyeight}, which are scraped from \url{http://cross-tables.com}. $N=15374$ and $T=1542642$.
\paragraph{Go} We use the Online Go Server (OGS) database~\footnote{\url{https://github.com/online-go/goratings}}, which contains $T=12876823$ games between $426105$ players. We also filter the dataset with threshold $5000$, to get \texttt{go-dense} with $N=480$ and $T=516343$.
\paragraph{Hearthstone} We use the Deck Archetype Matchup data scraped from \url{https://metastats.net/hearthstone/archetype/matchup/}. $N=27$ and $T=62453$.
\paragraph{Blotto} We use the 5,4-Blotto and 10,5-Blotto data from \cite{czarnecki2020real}. The original data is a payoff matrix between different strategies, whose entries are between $[-1,1]$. For strategies $i$ and $j$, let the payoff for $i$ against $j$ be $r_{ij}$, we let $p_{ij}:=0.5 (r_{ij}+1)$. For \texttt{5,4-Blotto} and \texttt{10,5-Blotto}, we create games $(i_t,j_t,o_t) := (i,j,p_{ij})$ for each $(i,j)$ where $i \neq j$. For \texttt{5,4-Blotto-sparse}, we create $0-1$ game results by drawing $o_ij \sim \text{Ber} (p_ij)$ and create games $(i_t,j_t,o_t) := (i,j, o_ij)$. For \texttt{5,4-Blotto-dense}, we create $0-1$ game results by drawing $10$ independent $o_ij \sim \text{Ber} (p_ij)$ and create $10$ games $(i_t,j_t,o_t) := (i,j, o_ij)$ for each pair of $(i,j)$. $\texttt{10,5-Blotto-sparse}$ and $\texttt{10,5-Blotto-dense}$ are similarly created. $N=56$, $T=3080$ for $\texttt{5,4-Blotto}$ and $\texttt{5,4-Blotto-sparse}$. $N=56$, $T=30800$ for $\texttt{5,4-Blotto-dense}$.
$N=1001$, $T=1001000$ for $\texttt{10,5-Blotto}$ and $\texttt{10,5-Blotto-sparse}$. $N=1001$, $T=10010000$for $\texttt{10,5-Blotto-dense}$.
\paragraph{AlphaStar} We use the AlphaStar data from \cite{czarnecki2020real}. The creation procedure for \texttt{AlphaStar}, \texttt{AlphaStar-sparse}, \texttt{AlphaStar-dense} are the same as in Blotto. For all the three dataset, $N=888$. $T=787656$ for the first two datasets, and $T=7876560$ for the last one.


% \subsection{Filtering}
% For computational effiency, when conducting the real data prediction experiments in Section \ref{sec:OCO}, we do filtering on some of the datasets. Our filtering method is: for a given threshold, we delete all the players that plays less than this threshold. Then we only consider the remaining players and the games played between them. The filtered dataset are listed in Table \ref{tab:filtering}. Other datasets are not filtered.

% \begin{table*}[]
% \centering
% \addtocounter{footnote}{+1}  
% \begin{tabular}{|l|c|c|c|}
% \hline
% Dataset    & Threshold          & $N$    & $T$   \\ \hline
% \texttt{renju}       & 200   & 320  & 47161           \\
% \texttt{chess}       & 5000 & 467  & 509505        \\
% \texttt{tennis}      & 200   & 588  & 104334         \\
% \texttt{scrabble}    & 1000  & 824  & 482947     \\
% \texttt{starcraft}   & 500  & 366 & 143603       \\
% \texttt{go}          & 5000 & 480  & 516343  \\ 
% \hline
% \end{tabular}
% \caption{Datasets and the filtering thresholds}
% \label{tab:filtering}
% \end{table*}