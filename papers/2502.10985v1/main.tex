%%%%%%%% ICML 2025 EXAMPLE LATEX SUBMISSION FILE %%%%%%%%%%%%%%%%%

\documentclass{article}

% Recommended, but optional, packages for figures and better typesetting:
\usepackage{microtype}
\usepackage{graphicx}
% \usepackage{subfigure}
\usepackage{subcaption}
\usepackage{booktabs} % for professional tables
\usepackage{enumitem}

% hyperref makes hyperlinks in the resulting PDF.
% If your build breaks (sometimes temporarily if a hyperlink spans a page)
% please comment out the following usepackage line and replace
% \usepackage{icml2025} with \usepackage[nohyperref]{icml2025} above.
\usepackage{hyperref}
\usepackage{natbib}

\usepackage{fullpage}

% Attempt to make hyperref and algorithmic work together better:
\newcommand{\theHalgorithm}{\arabic{algorithm}}

\def\shownotes{1}  %set 1 to show author notes
\ifnum\shownotes=1
\newcommand{\authnote}[2]{{\scriptsize $\ll$\textsf{#1 notes: #2}$\gg$}}
\else
\newcommand{\authnote}[2]{}
\fi
\newcommand{\yw}[1]{{\color{red}\authnote{YW}{#1}}}
\newcommand{\shange}[1]{{\color{red}\authnote{shange}{#1}}}
\newcommand{\chij}[1]{{\color{red}\authnote{Chi}{#1}}}






% Use the following line for the initial blind version submitted for review:
% \usepackage{icml2025}

% If accepted, instead use the following line for the camera-ready submission:
% \usepackage[accepted]{icml2025}
\usepackage{hyperref}
\usepackage{refstyle}
\usepackage{amsmath}
\usepackage{cleveref}

\usepackage{booktabs}
\usepackage{multirow} %
\usepackage{soul}%
\usepackage{tabularx}
\usepackage{enumitem}

\usepackage{amssymb}


\DeclareMathOperator*{\argmax}{argmax} %
\usepackage{pifont}
\newcommand{\cmark}{\ding{51}}%
\newcommand{\xmark}{\ding{55}}%

\usepackage{graphicx}

\interfootnotelinepenalty=10000

\crefformat{section}{\S#2#1#3}
\crefformat{subsection}{\S#2#1#3}
\crefformat{subsubsection}{\S#2#1#3}
\crefrangeformat{section}{\S#3#1#4 to~\S#5#2#6}
\crefmultiformat{section}{\S#2#1#3}{ and~\S#2#1#3}{, #2#1#3}{ and~#2#1#3}
\Crefformat{figure}{#2Fig.~#1#3}
\Crefmultiformat{figure}{Figs.~#2#1#3}{ and~#2#1#3}{, #2#1#3}{ and~#2#1#3}
\Crefformat{table}{#2Tab.~#1#3}
\Crefmultiformat{table}{Tabs.~#2#1#3}{ and~#2#1#3}{, #2#1#3}{ and~#2#1#3}
\Crefformat{appendix}{#2Appx.~\S#1#3}
\crefformat{algorithm}{Alg.~#2#1#3}
\Crefformat{equation}{#2Eq.~#1#3}

\newcommand{\todo}[1]{{\color{red}[{TODO:} #1]}}

\newcommand{\task}{\textsc{1MKR}}

% For theorems and such
\usepackage{amsmath}
\usepackage{amssymb}
\usepackage{mathtools}
\usepackage{amsthm}

% if you use cleveref..
\usepackage[capitalize,noabbrev]{cleveref}

%%%%%%%%%%%%%%%%%%%%%%%%%%%%%%%%
% THEOREMS
%%%%%%%%%%%%%%%%%%%%%%%%%%%%%%%%



% \theoremstyle{plain}
% \newtheorem{theorem}{Theorem}[section]
% \newtheorem{proposition}[theorem]{Proposition}
% \newtheorem{lemma}[theorem]{Lemma}
% \newtheorem{corollary}[theorem]{Corollary}
% \theoremstyle{definition}
% \newtheorem{definition}[theorem]{Definition}
% \newtheorem{assumption}[theorem]{Assumption}
% \theoremstyle{remark}
% \newtheorem{remark}[theorem]{Remark}

\newcommand{\chijin}[1]{{\color{magenta}Chi: {#1}}}

% Todonotes is useful during development; simply uncomment the next line
%    and comment out the line below the next line to turn off comments
%\usepackage[disable,textsize=tiny]{todonotes}
\usepackage[textsize=tiny]{todonotes}



\title{Is Elo Rating Reliable? A Study Under Model Misspecification}
\author{Shange Tang\thanks{Department of ORFE, Princeton University; shangetang@princeton.edu}
\qquad Yuanhao Wang\thanks{Department of ECE, Princeton University} 
\qquad Chi Jin\footnotemark[2]}

\begin{document}

\maketitle{}



\begin{abstract}
% Elo rating, commonly viewed as an incremental update algorithm for estimating a stationary Bradley-Terry (BT) model in pairwise comparisons, is widely used to assess player skill in domains ranging from board games to esports, and even large language models. However, whether the real-world game follows a stationary BT model is questionable, casting doubt on Elo’s reliability. In this paper, we examine Elo’s reliability under model misspecification. Through likelihood ratio tests, we show that many real-world game datasets do not follow the BT model, revealing widespread model misspecification when applying Elo. Further more, the matchmaking scheme and player skills are non-stationary in real world games. Despite this, our real data experiments show that in some non-BT and non-stationary datasets, Elo outperforms other methods like mElo and pairwise win rate in predicting game outcomes. We explain this phenomenon through a regret minimization framework. Specifically, Elo, as an instance of online gradient descent, can provably achieve a relatively low regret, which contributes to its strong predictive performance even in some non-BT and non-stationary datasets. We validate this through synthetic experiments, showing that data sparsity plays a key role: Elo performs well in sparse datasets, while more complex models with greater model capacity show benefits in dense datasets. Finally, regarding the ranking performance, we find a strong correlation between Elo’s predictive accuracy and ranking performance, though Elo does not always produce correct rankings.

Elo rating, widely used for skill assessment across diverse domains ranging from competitive games to large language models, is often understood as an incremental update algorithm for estimating a stationary Bradley-Terry (BT) model. However, our empirical analysis of practical matching datasets reveals two surprising findings: (1) Most games deviate significantly from the assumptions of the BT model and stationarity, raising questions on the reliability of Elo. (2) Despite these deviations, Elo frequently outperforms more complex rating systems, such as mElo and pairwise models, which are specifically designed to account for non-BT components in the data, particularly in terms of win rate prediction. This paper explains this unexpected phenomenon through three key perspectives: (a) We reinterpret Elo as an instance of online gradient descent, which provides no-regret guarantees even in misspecified and non-stationary settings. (b) Through extensive synthetic experiments on data generated from transitive but non-BT models, such as strongly or weakly stochastic transitive models, we show that the ``sparsity'' of practical matching data is a critical factor behind Elo’s superior performance in prediction compared to more complex rating systems. (c) We observe a strong correlation between Elo's predictive accuracy and its ranking performance, further supporting its effectiveness in ranking.


% Elo rating, widely used for skill assessment in various domains ranging from various games to even large language models, is often interpreted as an incremental update algorithm for estimating a stationary Bradley-Terry (BT) model. However, our statistical tests show that many real-world games deviate from the stationary BT distribution, raising questions about Elo’s reliability. Despite the existence of model misspecification, Elo often outperforms alternatives like mElo and pairwise win rate in predicting outcomes for non-BT, non-stationary datasets. We explain this phenomenon through a regret minimization framework. Specifically, Elo, as an instance of online gradient descent, can provably achieve a relatively low regret, which contributes to its strong predictive performance even in some non-BT and non-stationary datasets. We validate this through synthetic experiments, showing that data sparsity plays a key role: Elo performs well in sparse datasets, while more complex models with greater model capacity show benefits in dense datasets. Finally, regarding the ranking performance, we find a strong correlation between Elo’s predictive accuracy and ranking performance, 



% Elo rating, widely used for skill assessment in various domains ranging from board games to esports, and even large language models, is often interpreted as an incremental update algorithm for estimating a stationary Bradley-Terry (BT) model. However, our statistical tests show that many real-world games deviate from the stationary BT distribution, raising questions about Elo’s reliability. Despite the existence of model misspecification, Elo often outperforms alternatives like mElo and pairwise win rate in predicting outcomes for non-BT, non-stationary datasets. We explain this phenomenon through a regret minimization framework. Specifically, Elo, as an instance of online gradient descent, can provably achieve a relatively low regret, which contributes to its strong predictive performance even in some non-BT and non-stationary datasets. We validate this through synthetic experiments, showing that data sparsity plays a key role: Elo performs well in sparse datasets, while more complex models with greater model capacity show benefits in dense datasets. Finally, regarding the ranking performance, we find a strong correlation between Elo’s predictive accuracy and ranking performance, though Elo does not always produce correct rankings.



\end{abstract}

\section{Introduction}


\begin{figure}[t]
\centering
\includegraphics[width=0.6\columnwidth]{figures/evaluation_desiderata_V5.pdf}
\vspace{-0.5cm}
\caption{\systemName is a platform for conducting realistic evaluations of code LLMs, collecting human preferences of coding models with real users, real tasks, and in realistic environments, aimed at addressing the limitations of existing evaluations.
}
\label{fig:motivation}
\end{figure}

\begin{figure*}[t]
\centering
\includegraphics[width=\textwidth]{figures/system_design_v2.png}
\caption{We introduce \systemName, a VSCode extension to collect human preferences of code directly in a developer's IDE. \systemName enables developers to use code completions from various models. The system comprises a) the interface in the user's IDE which presents paired completions to users (left), b) a sampling strategy that picks model pairs to reduce latency (right, top), and c) a prompting scheme that allows diverse LLMs to perform code completions with high fidelity.
Users can select between the top completion (green box) using \texttt{tab} or the bottom completion (blue box) using \texttt{shift+tab}.}
\label{fig:overview}
\end{figure*}

As model capabilities improve, large language models (LLMs) are increasingly integrated into user environments and workflows.
For example, software developers code with AI in integrated developer environments (IDEs)~\citep{peng2023impact}, doctors rely on notes generated through ambient listening~\citep{oberst2024science}, and lawyers consider case evidence identified by electronic discovery systems~\citep{yang2024beyond}.
Increasing deployment of models in productivity tools demands evaluation that more closely reflects real-world circumstances~\citep{hutchinson2022evaluation, saxon2024benchmarks, kapoor2024ai}.
While newer benchmarks and live platforms incorporate human feedback to capture real-world usage, they almost exclusively focus on evaluating LLMs in chat conversations~\citep{zheng2023judging,dubois2023alpacafarm,chiang2024chatbot, kirk2024the}.
Model evaluation must move beyond chat-based interactions and into specialized user environments.



 

In this work, we focus on evaluating LLM-based coding assistants. 
Despite the popularity of these tools---millions of developers use Github Copilot~\citep{Copilot}---existing
evaluations of the coding capabilities of new models exhibit multiple limitations (Figure~\ref{fig:motivation}, bottom).
Traditional ML benchmarks evaluate LLM capabilities by measuring how well a model can complete static, interview-style coding tasks~\citep{chen2021evaluating,austin2021program,jain2024livecodebench, white2024livebench} and lack \emph{real users}. 
User studies recruit real users to evaluate the effectiveness of LLMs as coding assistants, but are often limited to simple programming tasks as opposed to \emph{real tasks}~\citep{vaithilingam2022expectation,ross2023programmer, mozannar2024realhumaneval}.
Recent efforts to collect human feedback such as Chatbot Arena~\citep{chiang2024chatbot} are still removed from a \emph{realistic environment}, resulting in users and data that deviate from typical software development processes.
We introduce \systemName to address these limitations (Figure~\ref{fig:motivation}, top), and we describe our three main contributions below.


\textbf{We deploy \systemName in-the-wild to collect human preferences on code.} 
\systemName is a Visual Studio Code extension, collecting preferences directly in a developer's IDE within their actual workflow (Figure~\ref{fig:overview}).
\systemName provides developers with code completions, akin to the type of support provided by Github Copilot~\citep{Copilot}. 
Over the past 3 months, \systemName has served over~\completions suggestions from 10 state-of-the-art LLMs, 
gathering \sampleCount~votes from \userCount~users.
To collect user preferences,
\systemName presents a novel interface that shows users paired code completions from two different LLMs, which are determined based on a sampling strategy that aims to 
mitigate latency while preserving coverage across model comparisons.
Additionally, we devise a prompting scheme that allows a diverse set of models to perform code completions with high fidelity.
See Section~\ref{sec:system} and Section~\ref{sec:deployment} for details about system design and deployment respectively.



\textbf{We construct a leaderboard of user preferences and find notable differences from existing static benchmarks and human preference leaderboards.}
In general, we observe that smaller models seem to overperform in static benchmarks compared to our leaderboard, while performance among larger models is mixed (Section~\ref{sec:leaderboard_calculation}).
We attribute these differences to the fact that \systemName is exposed to users and tasks that differ drastically from code evaluations in the past. 
Our data spans 103 programming languages and 24 natural languages as well as a variety of real-world applications and code structures, while static benchmarks tend to focus on a specific programming and natural language and task (e.g. coding competition problems).
Additionally, while all of \systemName interactions contain code contexts and the majority involve infilling tasks, a much smaller fraction of Chatbot Arena's coding tasks contain code context, with infilling tasks appearing even more rarely. 
We analyze our data in depth in Section~\ref{subsec:comparison}.



\textbf{We derive new insights into user preferences of code by analyzing \systemName's diverse and distinct data distribution.}
We compare user preferences across different stratifications of input data (e.g., common versus rare languages) and observe which affect observed preferences most (Section~\ref{sec:analysis}).
For example, while user preferences stay relatively consistent across various programming languages, they differ drastically between different task categories (e.g. frontend/backend versus algorithm design).
We also observe variations in user preference due to different features related to code structure 
(e.g., context length and completion patterns).
We open-source \systemName and release a curated subset of code contexts.
Altogether, our results highlight the necessity of model evaluation in realistic and domain-specific settings.





\section{Preliminaries}
\label{sec:prelim}
\label{sec:term}
We define the key terminologies used, primarily focusing on the hidden states (or activations) during the forward pass. 

\paragraph{Components in an attention layer.} We denote $\Res$ as the residual stream. We denote $\Val$ as Value (states), $\Qry$ as Query (states), and $\Key$ as Key (states) in one attention head. The \attlogit~represents the value before the softmax operation and can be understood as the inner product between  $\Qry$  and  $\Key$. We use \Attn~to denote the attention weights of applying the SoftMax function to \attlogit, and ``attention map'' to describe the visualization of the heat map of the attention weights. When referring to the \attlogit~from ``$\tokenB$'' to  ``$\tokenA$'', we indicate the inner product  $\langle\Qry(\tokenB), \Key(\tokenA)\rangle$, specifically the entry in the ``$\tokenB$'' row and ``$\tokenA$'' column of the attention map.

\paragraph{Logit lens.} We use the method of ``Logit Lens'' to interpret the hidden states and value states \citep{belrose2023eliciting}. We use \logit~to denote pre-SoftMax values of the next-token prediction for LLMs. Denote \readout~as the linear operator after the last layer of transformers that maps the hidden states to the \logit. 
The logit lens is defined as applying the readout matrix to residual or value states in middle layers. Through the logit lens, the transformed hidden states can be interpreted as their direct effect on the logits for next-token prediction. 

\paragraph{Terminologies in two-hop reasoning.} We refer to an input like “\Src$\to$\brga, \brgb$\to$\Ed” as a two-hop reasoning chain, or simply a chain. The source entity $\Src$ serves as the starting point or origin of the reasoning. The end entity $\Ed$ represents the endpoint or destination of the reasoning chain. The bridge entity $\Brg$ connects the source and end entities within the reasoning chain. We distinguish between two occurrences of $\Brg$: the bridge in the first premise is called $\brga$, while the bridge in the second premise that connects to $\Ed$ is called $\brgc$. Additionally, for any premise ``$\tokenA \to \tokenB$'', we define $\tokenA$ as the parent node and $\tokenB$ as the child node. Furthermore, if at the end of the sequence, the query token is ``$\tokenA$'', we define the chain ``$\tokenA \to \tokenB$, $\tokenB \to \tokenC$'' as the Target Chain, while all other chains present in the context are referred to as distraction chains. Figure~\ref{fig:data_illustration} provides an illustration of the terminologies.

\paragraph{Input format.}
Motivated by two-hop reasoning in real contexts, we consider input in the format $\bos, \text{context information}, \query, \answer$. A transformer model is trained to predict the correct $\answer$ given the query $\query$ and the context information. The context compromises of $K=5$ disjoint two-hop chains, each appearing once and containing two premises. Within the same chain, the relative order of two premises is fixed so that \Src$\to$\brga~always precedes \brgb$\to$\Ed. The orders of chains are randomly generated, and chains may interleave with each other. The labels for the entities are re-shuffled for every sequence, choosing from a vocabulary size $V=30$. Given the $\bos$ token, $K=5$ two-hop chains, \query, and the \answer~tokens, the total context length is $N=23$. Figure~\ref{fig:data_illustration} also illustrates the data format. 

\paragraph{Model structure and training.} We pre-train a three-layer transformer with a single head per layer. Unless otherwise specified, the model is trained using Adam for $10,000$ steps, achieving near-optimal prediction accuracy. Details are relegated to Appendix~\ref{app:sec_add_training_detail}.


% \RZ{Do we use source entity, target entity, and mediator entity? Or do we use original token, bridge token, end token?}





% \paragraph{Basic notations.} We use ... We use $\ve_i$ to denote one-hot vectors of which only the $i$-th entry equals one, and all other entries are zero. The dimension of $\ve_i$ are usually omitted and can be inferred from contexts. We use $\indicator\{\cdot\}$ to denote the indicator function.

% Let $V > 0$ be a fixed positive integer, and let $\vocab = [V] \defeq \{1, 2, \ldots, V\}$ be the vocabulary. A token $v \in \vocab$ is an integer in $[V]$ and the input studied in this paper is a sequence of tokens $s_{1:T} \defeq (s_1, s_2, \ldots, s_T) \in \vocab^T$ of length $T$. For any set $\mathcal{S}$, we use $\Delta(\mathcal{S})$ to denote the set of distributions over $\mathcal{S}$.

% % to a sequence of vectors $z_1, z_2, \ldots, z_T \in \real^{\dout}$ of dimension $\dout$ and length $T$.

% Let $\mU = [\vu_1, \vu_2, \ldots, \vu_V]^\transpose \in \real^{V\times d}$ denote the token embedding matrix, where the $i$-th row $\vu_i \in \real^d$ represents the $d$-dimensional embedding of token $i \in [V]$. Similarly, let $\mP = [\vp_1, \vp_2, \ldots, \vp_T]^\transpose \in \real^{T\times d}$ denote the positional embedding matrix, where the $i$-th row $\vp_i \in \real^d$ represents the $d$-dimensional embedding of position $i \in [T]$. Both $\mU$ and $\mP$ can be fixed or learnable.

% After receiving an input sequence of tokens $s_{1:T}$, a transformer will first process it using embedding matrices $\mU$ and $\mP$ to obtain a sequence of vectors $\mH = [\vh_1, \vh_2, \ldots, \vh_T] \in \real^{d\times T}$, where 
% \[
% \vh_i = \mU^\transpose\ve_{s_i} + \mP^\transpose\ve_{i} = \vu_{s_i} + \vp_i.
% \]

% We make the following definitions of basic operations in a transformer.

% \begin{definition}[Basic operations in transformers] 
% \label{defn:operators}
% Define the softmax function $\softmax(\cdot): \real^d \to \real^d$ over a vector $\vv \in \real^d$ as
% \[\softmax(\vv)_i = \frac{\exp(\vv_i)}{\sum_{j=1}^d \exp(\vv_j)} \]
% and define the softmax function $\softmax(\cdot): \real^{m\times n} \to \real^{m \times n}$ over a matrix $\mV \in \real^{m\times n}$ as a column-wise softmax operator. For a squared matrix $\mM \in \real^{m\times m}$, the causal mask operator $\mask(\cdot): \real^{m\times m} \to \real^{m\times m}$  is defined as $\mask(\mM)_{ij} = \mM_{ij}$ if $i \leq j$ and  $\mask(\mM)_{ij} = -\infty$ otherwise. For a vector $\vv \in \real^n$ where $n$ is the number of hidden neurons in a layer, we use $\layernorm(\cdot): \real^n \to \real^n$ to denote the layer normalization operator where
% \[
% \layernorm(\vv)_i = \frac{\vv_i-\mu}{\sigma}, \mu = \frac{1}{n}\sum_{j=1}^n \vv_j, \sigma = \sqrt{\frac{1}{n}\sum_{j=1}^n (\vv_j-\mu)^2}
% \]
% and use $\layernorm(\cdot): \real^{n\times m} \to \real^{n\times m}$ to denote the column-wise layer normalization on a matrix.
% We also use $\nonlin(\cdot)$ to denote element-wise nonlinearity such as $\relu(\cdot)$.
% \end{definition}

% The main components of a transformer are causal self-attention heads and MLP layers, which are defined as follows.

% \begin{definition}[Attentions and MLPs]
% \label{defn:attn_mlp} 
% A single-head causal self-attention $\attn(\mH;\mQ,\mK,\mV,\mO)$ parameterized by $\mQ,\mK,\mV \in \real^{{\dqkv\times \din}}$ and $\mO \in \real^{\dout\times\dqkv}$ maps an input matrix $\mH \in \real^{\din\times T}$ to
% \begin{align*}
% &\attn(\mH;\mQ,\mK,\mV,\mO) \\
% =&\mO\mV\layernorm(\mH)\softmax(\mask(\layernorm(\mH)^\transpose\mK^\transpose\mQ\layernorm(\mH))).
% \end{align*}
% Furthermore, a multi-head attention with $M$ heads parameterized by $\{(\mQ_m,\mK_m,\mV_m,\mO_m) \}_{m=1}^M$ is defined as 
% \begin{align*}
%     &\Attn(\mH; \{(\mQ_m,\mK_m,\mV_m,\mO_m) \}_{m\in[M]}) \\ =& \sum_{m=1}^M \attn(\mH;\mQ_m,\mK_m,\mV_m,\mO_m) \in \real^{\dout \times T}.
% \end{align*}
% An MLP layer $\mlp(\mH;\mW_1,\mW_2)$ parameterized by $\mW_1 \in \real^{\dhidden\times \din}$ and $\mW_2 \in \real^{\dout \times \dhidden}$ maps an input matrix $\mH = [\vh_1, \ldots, \vh_T] \in \real^{\din \times T}$ to
% \begin{align*}
%     &\mlp(\mH;\mW_1,\mW_2) = [\vy_1, \ldots, \vy_T], \\ \text{where } &\vy_i = \mW_2\nonlin(\mW_1\layernorm(\vh_i)), \forall i \in [T].
% \end{align*}

% \end{definition}

% In this paper, we assume $\din=\dout=d$ for all attention heads and MLPs to facilitate residual stream unless otherwise specified. Given \Cref{defn:operators,defn:attn_mlp}, we are now able to define a multi-layer transformer.

% \begin{definition}[Multi-layer transformers]
% \label{defn:transformer}
%     An $L$-layer transformer $\transformer(\cdot): \vocab^T \to \Delta(\vocab)$ parameterized by $\mP$, $\mU$, $\{(\mQ_m^{(l)},\mK_m^{(l)},\mV_m^{(l)},\mO_m^{(l)})\}_{m\in[M],l\in[L]}$,  $\{(\mW_1^{(l)},\mW_2^{(l)})\}_{l\in[L]}$ and $\Wreadout \in \real^{V \times d}$ receives a sequence of tokens $s_{1:T}$ as input and predict the next token by outputting a distribution over the vocabulary. The input is first mapped to embeddings $\mH = [\vh_1, \vh_2, \ldots, \vh_T] \in \real^{d\times T}$ by embedding matrices $\mP, \mU$ where 
%     \[
%     \vh_i = \mU^\transpose\ve_{s_i} + \mP^\transpose\ve_{i}, \forall i \in [T].
%     \]
%     For each layer $l \in [L]$, the output of layer $l$, $\mH^{(l)} \in \real^{d\times T}$, is obtained by 
%     \begin{align*}
%         &\mH^{(l)} =  \mH^{(l-1/2)} + \mlp(\mH^{(l-1/2)};\mW_1^{(l)},\mW_2^{(l)}), \\
%         & \mH^{(l-1/2)} = \mH^{(l-1)} + \\ & \quad \Attn(\mH^{(l-1)}; \{(\mQ_m^{(l)},\mK_m^{(l)},\mV_m^{(l)},\mO_m^{(l)}) \}_{m\in[M]}), 
%     \end{align*}
%     where the input $\mH^{(l-1)}$ is the output of the previous layer $l-1$ for $l > 1$ and the input of the first layer $\mH^{(0)} = \mH$. Finally, the output of the transformer is obtained by 
%     \begin{align*}
%         \transformer(s_{1:T}) = \softmax(\Wreadout\vh_T^{(L)})
%     \end{align*}
%     which is a $V$-dimensional vector after softmax representing a distribution over $\vocab$, and $\vh_T^{(L)}$ is the $T$-th column of the output of the last layer, $\mH^{(L)}$.
% \end{definition}



% For each token $v \in \vocab$, there is a corresponding $d_t$-dimensional token embedding vector $\embed(v) \in \mathbb{R}^{d_t}$. Assume the maximum length of the sequence studied in this paper does not exceed $T$. For each position $t \in [T]$, there is a corresponding positional embedding  







% \section{Testing Bradley-Terry on Real-world Games}



\section{Experiments on Real-world Matching Data}

In this section, we conduct experiments on real-world datasets. Surprisingly, we find that most games deviate significantly from the assumptions of the BT model and stationarity, raising questions on the reliability of Elo. Despite these deviations, Elo frequently outperforms more complex rating systems, such as mElo and pairwise models, which are designed to account for non-BT components in the data, particularly in terms of win rate prediction.

\subsection{Real-world games are neither BT nor stationary}
\label{sec:hypo}

\begin{table}[t]
\centering
\addtocounter{footnote}{+1}  
\resizebox{0.8\columnwidth}{!}{
\begin{tabular}{|l|c|c||c|c|}
\hline
Dataset              & $N$    & $2T/N$  & BT Model Test  & $p$-value          \\ \hline
\texttt{Renju}       & 5k   & 49.8  & 150.0    & $<10^{-10}$               \\
% \texttt{chess-small} & 30k  & 68.5  & 268.5    & $<10^{-10}$              \\
\texttt{Chess}       & 185k & 125.4 & 2020.1   & $<10^{-10}$            \\
\texttt{Tennis}      & 7k   & 52.5  & 37.3     & $< 10^{-4}$       \\
\texttt{Scrabble}    & 15k  & 200.7  & 142.2    & $<10^{-10}$               \\
\texttt{StarCraft}   & 22k  & 38.7 & 775.8    & $<10^{-10}$              \\
\texttt{Go}          & 426k & 60.4  & 193411.2 & $<10^{-10}$             \\ 
% \texttt{LLM Arena}   & 54   & 7910.2  & 17.0~\footnotemark     & $1\times 10^{-3}$                  \\
\texttt{LLM Arena}   & 129   & 23156.9  & 73.1~\footnotemark     & $1\times 10^{-3}$                  \\
\texttt{Hearthstone}   & 27   & 4626.1  & 49.0     & $<10^{-4}$                  \\
% \texttt{AlphaStar}   & 888   & 1774.0  & 6902.0     & $<10^{-10}$                  \\
% \texttt{5,4-Blotto}   & 56   & 110.0  & 8.1     & $0.018$                \\
% \texttt{10,5-Blotto}   & 1k   & 2000.0  & 784.8    & $<10^{-10}$            \\
% \texttt{mixedchess-dense}   & 2k   & 4119.9  & 8.091    & 0.017          \\
\hline
\end{tabular}
}

% \chijin{use top environment for table}

\caption{Summary of real world datasets and BT-model testing results. $N$ is the total number of players, and $2T/N$ is the average number of games each player played.}
\label{tab:hypotheis}

\end{table}
\footnotetext[2]{{The likelihood-ratio test is performed for the LLM arena dataset using a different method of augmenting the features. See details in Appendix \ref{sec:lr-test}.}}




In the Elo rating update rule (\ref{eq:elo}), $\sigma(\theta[i]-\theta[j])$ represents the predicted win probability of player  i  against player  j . This prediction relies on the assumption that the underlying data follows the Bradley-Terry (BT) model. However, whether real-world data truly follows  a stationary BT model remains unverified.

In this section, we conduct a likelihood ratio test on real-world datasets to examine the hypothesis that real-world game outcomes are generated by the BT model. Our results indicate that, across all examined datasets, the hypothesis is rejected, suggesting that real-world data does not follow the BT model. Furthermore, we provide evidence that both matchmaking and player skill exhibit non-stationarity in real-world games. These findings suggest that model misspecification widely exists when applying Elo to real-world data.



\paragraph{Rejecting BT on real-world dataset}


Note that the Bradley-Terry model can be equivalently written as a logistic regression model, where the parameter $\theta$ is $N$-dimensional, and every game has a feature vector $x_t:=\mathbf{e}[i_t]-\mathbf{e}[j_t]\in \R^{N}$. 
We randomly split $[T]$ into $\cT_{\rm train}$ and $\cT_{\rm test} = [T]\setminus \cT_{\rm train}$. Then the logistic regression loss on the test set is defined as

\begin{align*}
\cL_{\rm test}(\theta)=& -\sum_{t\in [T]} \left[o_t\ln(\sigma(\theta^\top x_t))\right. \\
& \qquad\quad \left. + (1-o_t)\ln(1-\sigma(\theta^\top x_t))\right].
\end{align*}

%We examine the validity of the Bradley-Terry model on eight real-world datasets via the Likelihood Ratio (LR) Test, by viewing a Bradley-Terry MLE problem as a logistic regression problem, where every game has feature $x_t:=\mathbf{e}[i_t]-\mathbf{e}[j_t]\in \R^{N+2}$. We test the goodness-of-fit of this model by comparing it to an augmented model with two more parameters, where the feature for every game is $[\mathbf{e}[i_t]-\mathbf{e}[j_t];\alpha_t;\beta_t]\in \R^{N+2}$. (See Appendix~\ref{sec:lr-test} for more details).

Next, we \emph{augment} the logistic model by adding two additional parameters $\alpha\in \R^2$, and a two dimensional feature $g_t\in \R^2$ for every game. In practice, $g_t$ is constructed using the training set $\cT_{\rm train}$. Define the negative log likelihood of the augmented model as
\begin{align*}
\Tilde{\cL}_{\rm test}([\theta; \alpha])=& -\sum_{t\in \cT_{\rm test}} \left[o_t\ln(\sigma(\theta^\top x_t + \alpha^\top g_t))\right. \\
& \qquad \left. + (1-o_t)\ln(1-\sigma(\theta^\top x_t + \alpha^\top g_t))\right].
\end{align*}
If dataset is indeed realizable by a BT model with true scores $\theta^\star$, the augmented model is also realizable with $[\theta^\star;\textbf{0}]$ as long as $g_t$ and $o_t$ are independent, because
\[
\E[o_t|i_t,j_t,g_t] = \sigma(\theta^\star[i_t] - \theta^\star [j_t]).
\]
Therefore, we can test the BT model by testing the null hypothesis $H_0:\alpha=0$.

We employ the standard likelihood ratio test, which uses the log-likelihood ratio statistic:
\[
\Lambda := 2\left[\min_{\theta\in\R^N}\cL_{\rm test}(\theta) - \min_{\theta\in\R^{N},\alpha\in\R^2}\tilde\cL_{\rm test}([\theta;\alpha])\right].
\]


\iffalse
\begin{align*}
\tilde\cL(\tilde\theta)&:= - o_t\log(\sigma(\tilde\theta[i_t]-\tilde\theta[j_t]+\tilde\theta_{N+1}\alpha_t+\tilde\theta_{N+2}\beta_t))\\
&- (1-o_t)\log(1-\sigma(\tilde\theta[i_t]-\tilde\theta[j_t]+\tilde\theta_{N+1}\alpha_t+\tilde\theta_{N+2}\beta_t)).
\end{align*}
\fi
By Wilk's Theorem~\citep{wilks1938large,sur2019likelihood}, under the null hypothesis that the real-world dataset is generated by Bradley-Terry model,
$\Lambda$
is asymptotically distributed as a chi-square distribution with two degrees of freedom. This allows us to compute the $p$-value, which is the probability that the test statistic occurs under the null hypothesis due to pure chance.

For high-dimensional logistic regression,~\citet{sur2019likelihood} showed that $\Lambda$ is asymptotically distributed as a scaled chi-square distribution if $T/N = O(1)$. We applied the correction suggested by~\citet{sur2019likelihood}  by computing the $p$-value conservatively with $1.25\chi^2_2$. This factor is computed when the number of samples is $5$ times the model dimension, although the number of samples is at least $19$ times the model dimension in our datasets.

% \subsubsection{Rejecting BT on real-world dataset}
With the test statistic $\Lambda$, we are able to perform the test. We construct the augmented features $\{g_t\}_{t\in [T]}$ by fitting fit $\theta_{\rm train}$ via regularized MLE on $\cT_{\rm train}$. We then define 
\[
g_t = [\theta_{\rm train}[i_t], \theta_{\rm train}[j_t]]
\]
for every $t$ in the test set. Under the Bradley-Terry model, the original logistic regression $\cL_{\rm test}(\theta)$ already has sufficient information to predict $o_t$, so adding the score computed on an independent training set cannot help prediction (up to random noise).

We compute the log-likelihood ratio statistic $\Lambda$ for eight real-world datasets and report the corresponding $p$-values (see Table~\ref{tab:hypotheis}).
%We applied the correction suggested by~\citet{sur2019likelihood}  by  conservatively computing the $p$-value with $1.25\chi^2_2$. This factor is computed when the number of samples is $5$ times the model dimension, although the number of samples ($T$) is at least $19$ times the model dimension ($N+2$) in our datasets.
It can be seen that we can reject the null hypothesis, namely realizability of the Bradley-Terry model, with extremely high confidence, for all eight datasets.

%\yw{should we make a note here that realizability $\neq$ calibration?}
%\subsection{Elo score is not close to Regularization Path}



% \paragraph{Non-stationary matchmaking and player skills in real datasets}
\paragraph{Matchmaking and player skills are non-stationary}
Additional observations that we draw from real-world datasets are the existence of non-stationary matchmaking and player's skills. We postpone details to Appendix \ref{sec:appendix-matchmaking}. These phenomena suggest that the real world games are non-BT and non-stationary. Consequently, viewing Elo rating as fitting a underlying BT-model might not be appropriate. 





\subsection{Elo achieves good performance under model misspecification}
\label{sec:realdata}
% Section \ref{sec:hypo} shows that the real-world games are non-BT and not stationary, implying the model misspecification when applying Elo rating. One may wonder whether algorithms based on more complex models (e.g., Elo2k and Pairwise win rate) will have advantage due to their better fit to the underlying game distribution. However, in this section, by examining the prediction accuracy of next outcome of online algorithms on real dataset \footnotemark\footnotetext{ For each dataset, we randomly permute it before conducting the algorithms. See Appendix \ref{sec:appendix-dataset} for details.}, we surprisingly find that Elo rating still achieves good prediction performance, even outperforming other algorithms in some of the non-BT datasets.

Section \ref{sec:hypo} establishes that real-world games do not follow a stationary BT model, highlighting model misspecification in the applicaton of the Elo rating system. This raises important concerns regarding Elo’s reliability in practical settings. In particular, it prompts the question of whether more sophisticated rating algorithms, such as Elo2k or Pairwise, which may better capture the underlying game distributions, could yield improved predictive performance. However, we examine the prediction accuracy for the next game outcome of various online algorithms in real-world datasets, and surprisingly find that despite the model misspecification, ``Elo-like" algorithms still achieve strong predictive performance, outperforming complex algorithms even in some non-BT datasets. For each dataset, we compute the cumulative loss $\frac{1}{T} \mathcal{L}_T$ for Elo, Elo2k (with $k=4$), Glicko, TrueSkill, and Pairwise.\footnotemark\footnotetext{The experimental details can be found in Appendix \ref{sec:appendix-realdata}.} The results, summarized in Table \ref{tab:rating_results}, show that in several real-world datasets, including \texttt{Renju}, \texttt{Chess}, \texttt{Tennis}, \texttt{Scrabble}, \texttt{StarCraft} and \texttt{Go}, Elo and ``Elo-like" rating outperform more complexity rating systems such as Elo2k and Pairwise.


\begin{table}[t]
    \centering
    \small\resizebox{0.8\columnwidth}{!}{
    \begin{tabular}{|l || c c c | c c|}
        \hline
        Dataset & Elo & Glicko & TrueSkill & Elo2k & Pairwise \\
        \hline
        \texttt{Renju} & 0.6039 & 0.6100 & 0.5995 & 0.6109 & 0.6688 \\
        \texttt{Chess} & 0.6391 & 0.6349 & 0.6308 & 0.6387 & - \\
        \texttt{Tennis} & 0.6242 & 0.6232 & 0.6209 & 0.6365 & 0.6820 \\
        \texttt{Scrabble} & 0.6730 & 0.6766 & 0.6756 & 0.6787 & 0.6894 \\
        \texttt{StarCraft} & 0.5713 & 0.5689 & 0.5828 & 0.5832 & 0.6753 \\
        \texttt{Go} & 0.6443 & 0.6375 & 0.6321 & 0.6372 & - \\
        \texttt{LLM Arena} & 0.6607 & 0.6602 & 0.6611 & 0.6611 & 0.6619 \\
        \texttt{Hearthstone} & 0.6898 & 0.6893 & 0.6894 & 0.6847 & 0.6853 \\
        \hline
    \end{tabular}}
    \caption{Performance of different rating algorithms across various games}
    \label{tab:rating_results}
\end{table}

% For each dataset, we evaluate the performance of Elo, Elo2k (with $k=4$), Glicko, TrueSkill, and Pairwise, plotting the cumulative loss $\frac{1}{t} \mathcal{L}_t$ over "normalized" time $t/N$. For Elo and Elo2k, we adopt a decaying learning rate scheme: $\eta_t = \sqrt{\frac{N}{t}}$, following the standard approach used in gradient descent. For Glicko and TrueSkill, hyperparameters are carefully tuned to optimize their performance. The results are presented in Figure \ref{fig:realdata-CE}.





% In Figure \ref{fig:realdata-CE}, we can see that, for many real dataset including \texttt{chess}, \texttt{go}, \texttt{renju}, \texttt{tennis}, \texttt{scrabble} and \texttt{5,4-Blotto sparse}, Elo and its variants (TrueSkill and Glicko) beat algorithms with more complex underlying models such as Elo2k and Pairwise (Elo has smaller cumulative loss compaired with Elo2k for every $0<t<T$). On the other hand, for some other datasets such as \texttt{Hearthstone}, \texttt{AlphaStar}, \texttt{10,5-Blotto}, \texttt{go-dense}, \texttt{mixedchess-dense}, Elo2k achieves smaller prediction error than Elo at time step $t=T$. 


% For other datasets like \texttt{Hearthstone}, \texttt{AlphaStar}, \texttt{10,5-Blotto}, \texttt{go-dense}, and \texttt{mixedchess-dense}, Elo2k achieves lower prediction errors than Elo at the final time $t=T$.


% \chijin{why we already mention sparsity here? Maybe just give a table here?} 


% \section{\protect\resizebox{0.5\textwidth}{!}{Understand Elo under Misspecification}}
\section{Understand Elo under Misspecification}

The findings in Section \ref{sec:realdata} that the ``Elo-like" algorithms outperform more complexity rating systems in some non-BT datasets, underscore the importance of adopting a new perspective on Elo (and other online algorithms), moving beyond the traditional view that Elo is merely a parameter estimation tool for the BT model.

In this section, we will explain this unexpected phenomenon through three key perspectives. First, we view game rating through the lens of regret minimization in online optimization. Specifically, Elo can be reinterpreted as an instance of online gradient descent under convex loss, which provides no-regret guarantees even in misspecified and non-stationary settings. Second, further synthetic experiments on non-BT and non-stationary datasets show that the ``sparsity" of dataset is a critical factor in the performance of algorithms, driven by a trade-off between model misspecification error and regret. Finally regarding the ranking performance, we show that the pairwise ranking performance is strongly correlated with prediction performance, though Elo should not be blindly trusted since it can fail to produce consistent total orderings even in transitive datasets.

\subsection{New lens via regret minimization}
\label{sec:OCO}



% In this section, we view game rating through the lens of regret minimization in online optimization. We analyze and interpret the prediction performance of Elo and other rating algorithms, such as Glicko and Elo2k, on both real and synthetic datasets from this perspective. To be specific, we show that Elo can be interpreted as online gradient descent, which is guaranteed to have small regret. Thus Elo can provably achieve low regret, which explains its strong predictive performance in non-BT datasets. 


% Our findings further indicate that the effectiveness of these algorithms is shaped by the interaction between data sparsity and model complexity. In particular, in sparse datasets, even though the BT model may not be relizable and the underlying matchmaking distribution is not uniform, Elo and its simple variants still demonstrate reasonable prediction accuracy. In contrast, for denser datasets, where all algorithms are expected to converge to zero regret, the optimal loss in hindsight becomes a more prominent factor in the cumulative loss. Consequently, algorithms based on
% more complex models, such as Elo2k or Pairwise, may
% achieve superior performance due to their greater model
% capacity.

In this section, we will view game rating through the lens of regret minimization in online optimization. We will adapt the framework of Online Convex Optimization (OCO) to the online algorithms. To facilitate our presentation, we briefly introduce OCO, following \citet{hazan2016introduction}'s definition.

\paragraph{Online Convex Optimization} At iteration $t$, the online player chooses $x_t \in \mathcal{K}$ according to the information in steps $1,2, \cdots, t-1$ . After the player has committed to this choice, a cost function $f_t \in \mathcal{F} : \mathcal{K} \to \mathbb{R}$ is revealed. Here, $\mathcal{F}$ is the bounded family of cost functions available to the adversary. The cost incurred by the online player is $f_t(x_t)$, the value of the cost function for the choice $x_t$. Let $T$ denote the total number of game iterations. The regret is defined as
\begin{equation*}
    \text{Regret}_T := \sum_{t=1}^{T}f_t(x_t) - \min_{x \in \mathcal{K}} \sum_{t=1}^{T} f_t(x),
\end{equation*}
that is, the cumulative loss minus the optimal loss in hindsight. 

It turns out that online rating algorithms can be evaluated under this framework. At each time $t$, let $f_t$ be the binary cross entropy loss function induced by the players $i_t$ and $j_t$ and the outcome $o_t$, and $x_t$ be the parameters updated by algorithms:
\begin{align*}
    f_t(x_t) := - (o_t \log  p_t + (1-o_t) \log (1-p_t)) .
\end{align*}
Here $p_t$ is actually related to the parameter $x_t$. Under this formulation, we have
% \begin{equation}
% \mathcal{L}_T = \text{Optimal loss in hindsight} + \text{Regret}_T.
% \end{equation}
\begin{equation}
\mathcal{L}_T = \text{Model misspecification error} + \text{Regret}_T.
\end{equation}
From this equation, we can see that the cumulative loss consists of two components, the model misspecification error (optimal loss in hindsight) and the regret. The trade-off between these two terms will be illustrated in extensive experiments.



\paragraph{Elo as online gradient descent} For Elo update, $x_t:= \theta_t \in \mathbb{R}^{N}$,  is the parameter of the underlying BT model (the Elo score). $p_t:=\sigma(\theta[i_t]-\theta[j_t])$ is the prediction. The gradient of $f_t$ is given by $\nabla_{\theta} f_t(\theta)= - (o_t - p_t) (\boldsymbol{e}_{i_t}- \boldsymbol{e}_{j_t})$. We can see that the Elo score update is actually online gradient descent with learning rate $\eta_t$ at each step $t$. Notice that $f_t$ is a convex function (one can refer to Appendix \ref{sec:appendix-algorithm} for detail). Therefore we can apply the regret bound for online gradient descent under convex loss \citep[Theorem 3.1]{hazan2016introduction}:
\begin{theorem}
\label{thm:OCO}
For convex cost functions $\{f_t\}_{t=1}^{T}$ and convex set $\mathcal{K}$, online gradient descent with step sizes $\{\eta_t = \frac{D}{G\sqrt{t}}\}$ guarantee the following for all $T>1$:
\begin{equation*}
\text{Regret}_T = \sum_{t=1}^{T}f_t(x_t) - \min_{x \in \mathcal{K}} \sum_{t=1}^{T} f_t(x) \leq \frac32 GD\sqrt{T},     
\end{equation*}
where $D$ is the upper bound on the diameter of $\mathcal{K}$, and $G$ is an upper bound on the norm of the subgradients of $f_t$ over $\mathcal{K}$.  
\end{theorem}
In the context of Elo update, since $\theta \in \mathbb{R}^{N}$, and in experiments we observe that $\|\theta\|_{\infty} \leq 5$, which means we can choose $D = 10 \sqrt{N}$. For $G$,  recall $\nabla_{\theta}f_t(\theta) = -(o_t-p_t)(\boldsymbol{e}_{i_t} - \boldsymbol{e}_{i_t})$, we have $G \leq \sqrt{2}$. Therefore we conclude that online Elo score update will have the following regret bound: $\frac{1}{T} \text{Regret}_T \leq C \sqrt{\frac{N}{T}}$ for some absolute constant $C$, with learning rate $\eta_t = \sqrt{\frac{N}{t}}$. Notice that this regret bound even holds under misspecified and non-stationary settings, which explains Elo's good performance in non-BT datasets, as long as the best BT model in hindsight provides a reasonable fit to the data.

We can also formulate the Elo2k update under online optimization framework as the following:  
\begin{align*}
    f_t(\theta) := - (o_t \log  p_t + (1-o_t) \log (1-p_t)) ,
\end{align*}
where $\theta = (U, V)$, where $U= (u_1, \cdots, u_N), V = (v_1, \cdots, v_N)$, $u_i, v_i \in \mathbb{R}^{k}$. The prediction $p_t=\sigma(u_{i_t}^{T}v_{j_t}-u_{j_t}^{T}v_{i_t})$. Then the Elo2k online update will be online gradient descent. However, the loss function is non-convex, therefore a general guarantee of OGD under Elo2k model is lacking. 





\subsection{Synthetic experiments: sparsity is critical}
\label{sec:synthetic}
To further justify our interpretation of why Elo performs well even in non-BT datasets, in this section, we will conduct extensive synthetic experiments, as well as experiments on augmented real-world data. These experiments further show that the ``sparsity" of the dataset plays a crucial role in the performance of algorithms.


% The effectiveness of these algorithms is shaped by the interaction between data sparsity and model complexity. The cumulative loss consists of 
% regret and the model misspecification error. There is a trade off between these two terms:
% algorithms based on more complex models might have a smaller model misspecification error, whereas Elo-like algorithms is guaranteed to have a small regret even under model-misspecification. In sparse datasets, where the regret is dominant in the cumulative loss, Elo-like algorithms tends to have good performance. In dense datasets, where the regret for every algorithms should converge to zero, the dominate term becomes the model misspecification error. In this case, Elo2k or Pairwise, may
% achieve superior performance due to their greater model
% capacity.

\paragraph{Synthetic experiments on non-BT datasets} We begin with the scenario where the players’ skills are stationary in the sense that $\E[o_t|i_t=i,j_t=j]=P_{ij}$ for some matrix $P\in\R^{N\times N}$. We consider the following two notions of the transitivity:

\begin{definition}[SST]
\label{def:sst}
$P$ is strongly stochastic transitive (SST) with respect to ordering $\pi$ if $\pi(i)>\pi(j)$ implies $P_{ik}\ge P_{jk}$ for all $k\in [N]$.
\end{definition}

\begin{definition}[WST]
\label{def:wst}
$P$ is weakly stochastic transitive (WST) with respect to ordering $\pi$ if $\pi(i)>\pi(j)$ implies $P_{ij}\ge \frac{1}{2}$.
\end{definition}
It is well-known that BT implies the SST condition, and SST further implies WST. For details of the constructed $P$, see Appendix \ref{sec:appendix-generatingP}.
% For non-transitive $P$, we consider $P$ that follows Elo2k model, and $P$ that has entries drawn randomly from $[0,1]$. 

For each of these types of $P$, we generate $P$ for $N=1000$ and $N=100$. For each instance of $P$, we generate $T=10^5$ games following uniform matchmaking distribution, that is, for every $t\in [T]$, sample $i_t \sim \text{Uni}([N])$, then independently sample $j_t \sim \text{Uni}([N])$. For each algorithm, we choose the best hyperparameter (for details of choosing the best hyperparameter, see Appendix \ref{sec:appendix-choosing_parameter}), we plot the corresponding $\frac1t \mathcal{L}_t$ with respect to time step $t/N$ (Figure \ref{fig:synthetic-CE-N=1000} for $N=1000$ and Figure \ref{fig:synthetic-CE-N=100} for $N=100$). The model misspecification error (optimal loss in hindsight) at time $T$ is also plotted for $N=100$. From the experiments (Figure \ref{fig:synthetic-CE-N=1000} and \ref{fig:synthetic-CE-N=100}), we can see that the effectiveness of rating algorithms is shaped by the interaction between data sparsity and model complexity. There is a trade-off between the regret and the model misspecification error:
when the samples are sparse, i.e., $t$ is small, the dominating term in the cumulative loss will be the regret, Elo2k or Pairwise suffers from a huge regret. Under this scenario, Elo and its variants performs well due to its low regret, even though BT model is non-realizable. For dense regime, i.e., $t$ is large, the regret for both Elo and Elo2k will be closer to zero. Under this scenario, Elo2k or Pairwise may achieve superior performance when they achieve a lower misspecification error due to their greater model capacity.

% \begin{figure}[t]
%     \centering
% \begin{minipage}{0.48\textwidth}
%     \centering
%     \includegraphics[height=8cm]{figures/1000-test-CE-new-big.png}
%     \vfill  % Push caption to bottom
%     \caption{\textbf{Elo outperforms Elo2k in sparse datasets.} We can see that Elo consistently outperforms Elo2k for $0\leq t \leq T$. One can also observe that when $t$ gets larger, Elo2k's performance is rapidly improving.}
%     \label{fig:synthetic-CE-N=1000}
% \end{minipage}%
% \hfill
% \begin{minipage}{0.48\textwidth}
%     \centering
%     \includegraphics[height=8cm]{figures/100-test-CE-new-big.png}
%     \vfill  % Push caption to bottom
%     \caption{\textbf{Performance in dense datasets.} When $t$ is large, the regret is close to zero for all algorithms, and Elo2k outperforms Elo when the gap between the misspecification error for Elo and Elo2k is large.}
%     \label{fig:synthetic-CE-N=100}
% \end{minipage}
% \end{figure}


\begin{figure}[t]
    \centering
    \begin{subfigure}[t]{0.48\textwidth} % Keep the width consistent
        \centering
        \includegraphics[height=8cm]{figures/1000-test-CE-new-big.png}
        \caption{Prediction performance in sparse, stationary datasets.}
        \label{fig:synthetic-CE-N=1000}
    \end{subfigure}%
    ~
    \begin{subfigure}[t]{0.48\textwidth} % Keep the width consistent
        \centering
        \includegraphics[height=8cm]{figures/mm+varyingP-1000-test-CE-new-big.png}
        \caption{Prediction performance under non-stationary matchmaking and player strengths.}
        \label{fig:synthetic-CE-N=1000+mm+varyingP}
    \end{subfigure}%
    \caption{Elo and Elo2k's prediction performance in sparse datasets.}
\end{figure}




\paragraph{Non-trivial matchmaking and varying player strengths}

We further justify our regret-minimization framework through synthetic experiments under the scenario where the player strengths can vary and a non-trivial matchmaking exists. We plot the performance of each algorithm in non-stationary datasets ($N=1000$) in Figure \ref{fig:synthetic-CE-N=1000+mm+varyingP}. The experimental details can be found in Appendix \ref{sec:appendix-non_stationary}. 


\begin{figure}[t]
    \centering

    \begin{subfigure}[t]{0.48\textwidth} % Keep the width consistent
        \centering
        \includegraphics[height=8cm]{figures/100-test-CE-new-big.png}
        \caption{Predicition performance in dense, stationary datasets.}
        \label{fig:synthetic-CE-N=100}
    \end{subfigure}
    ~
    \begin{subfigure}[t]{0.48\textwidth} % Keep the width consistent
        \centering
        \includegraphics[height=8cm]{figures/100-pairwise_rk-new-big.png}
        \caption{Pairwise ranking performance.}
        \label{fig:pairwise-rk-N=100}
    \end{subfigure}
    \caption{Ranking performance correlated with prediction.}
\end{figure}

% \begin{figure}[t]
%     \centering
%     \includegraphics[width=0.8\columnwidth]{figures/mm+varyingP-1000-test-CE-new-big.png}
%     \caption{\textbf{Elo is more robust to non-stationary matchmaking and player strengths.} Compared with Figure \ref{fig:synthetic-CE-N=1000}, we can see that the performance of Elo2k deteriorates more than Elo.}
%     \label{fig:synthetic-CE-N=1000+mm+varyingP}
% \end{figure}

Comparing Figure \ref{fig:synthetic-CE-N=1000} with Figure \ref{fig:synthetic-CE-N=1000+mm+varyingP}, we can see that when non-trivial matchmaking exists and the player strength are varying, Elo still performs reasonably well, while Elo2k exhibits a significant deterioration in performance. This also justifies our finding: Elo as online gradient descent, is guaranteed to achieve a low regret, even under non-trivial matchmaking and non-stationary player strengths.  

\paragraph{Experiments on real-world data}
Similar behaviors also appears in real-world datasets. Other than the real-world datasets examined in Section \ref{sec:lr-test}, we also use \texttt{Blotto}  and \texttt{AlphaStar} data from \cite{czarnecki2020real}, where we generate game data from the original payoff matrix. To create denser datasets, we augment datasets from \cite{czarnecki2020real} by simply creating identical copys. For each real-world dataset, we plot the corresponding $\frac1t \mathcal{L}_t$ for each algorithm with respect to time step $t/N$ (Figure \ref{fig:realdata-CE}). We can also see that sparsity plays a crucial role in those real-world (or augmented) datasets, as in the previous synthetic experiments. See Appendix \ref{sec:appendix-realdata} for details.




% \begin{figure*}[t]
    \centering
    \begin{subfigure}[b]{0.22\textwidth}
        \centering
        \includegraphics[height=18ex]{file/figures/synthetic_0.pdf}
        \caption{0'th token}
        \label{fig:subfig1}
    \end{subfigure}%
    \begin{subfigure}[b]{0.22\textwidth}
        \centering
        \includegraphics[height=18ex]{file/figures/synthetic_20.pdf}
        \caption{20'th token}
        \label{fig:subfig2}
    \end{subfigure}%
    \begin{subfigure}[b]{0.22\textwidth}
        \centering
        \includegraphics[height=18ex]{file/figures/synthetic_50.pdf}
        \caption{50'th token}
        \label{fig:subfig3}
    \end{subfigure}%
    \begin{subfigure}[b]{0.22\textwidth}
        \centering
        \includegraphics[height=18ex]{file/figures/synthetic_200.pdf}
        \caption{200'th token}
        \label{fig:subfig4}
    \end{subfigure}%
    \begin{subfigure}[b]{0.12\textwidth}
        \centering
        \raisebox{3ex}{\includegraphics[height=18ex]{file/figures/synthetic_legend.pdf}}
        % \caption{Legend}
        \label{fig:subfig5}
    \end{subfigure}
    \caption{%\small
        The probability that greedy decoding can attain globally optimal response, with respect to the number of newly generated tokens, and with different starting positions in the generation history.
        The legend is shared across sub-figures.
    }
    \label{fig:synthetic_setting}
    % \vspace{-1em}
\end{figure*}

\section{Sanity Check: Semi-Synthetic Settings}\label{main:synthetic}
The implication of our theoretical analysis is straightforward.
However, it is natural to ask whether the phenomenon actually occurs in real-world LLM decoding scenarios.
To provide clear empirical evidence accompanying our theoretical analysis, in this section, we present semi-synthetic experiments that serves as a sanity check.
In particular, in moderately realistic settings, we show that greedy decoding for text generation with stepwise optimization results in suboptimal responses.
We first outline the semi-synthetic setting, and then present the empirical findings.

% \vspace{-2ex}
\paragraph{Illustrative Approximation}
\looseness=-1
For any modern LLM with a vocabulary size $|\Vcal|$ (typically on the order of $10^4$ to $10^5$), identifying the globally optimal sequence across multiple steps becomes computationally intractable, even for relatively short sequence lengths ($<100$).
To ensure the validity of our claim while providing a clear and accessible illustration, we adopt beam search as an approximation strategy of obtaining globally optimal sequence.
Since we measure the chance that greedy decoding can attain the global optimum with the stepwise optimal response, this approximation serves as an upper bound on achievable performance, indicating the discrepancy between greedy decoding and the true globally optimal response.

% \vspace{-2ex}
\paragraph{Approximating Natural Language Scenarios}
Since the prompt or context of the generation influence model behavior, we align our experimental setting with common human-LLM interactions.
Specifically, we utilize MT-Bench \citep{zheng2023judging} questions as curated prompts, which are designed to evaluate conversational chat models.
These samples serve as an approximation of real-world natural language context distributions, ensuring that our findings are grounded in practical scenarios.


% \vspace{-2ex}
\paragraph{Findings}
For each prompt, together with a certain length of generation history ($0$ means only the prompt is given), we evaluate whether the joint probability of the sequence generated with greedy decoding is greater than or equal to that produced by beam search (the proxy of the global optimum).
This comparison indicates the extent to which greedy decoding deviates from the globally optimal response.
As illustrated in Figure~\ref{fig:subfig1}, greedy decoding consistently results in suboptimal sequences, and the phenomenon can be observed with a small number of newly generated tokens.

In addition, the potential deviation may behave differently across various positions in the generated text.
For instance, when openings of response diverge, it is hard for greedy decoding to achieve optimal results afterwards.
To reduce potential inductive bias resulting from the diversity at early stages of generation, we evaluate generations starting/continuing from various positions throughout generation history, as presented in Figures~\ref{fig:subfig2}--\ref{fig:subfig4}.
We can observe that the deviation persists across different positions, which empirically demonstrate the common existence of sub-optimality in autoregressive decoding for text generation.

% \section{(Semi)-Synthetic Experiments}
% The principles behind our approach are straightforward; however, one might question whether the observed phenomenon persists in real-world LLM decoding scenarios. To provide a clear evidence of the previous theoretical analysis and also address the aforementioned concern, we introduce a (semi)-synthetic experiment designed to demonstrate that, in moderately realistic settings, greedy decoding during text generation with step-wise optimization results in suboptimal response. Before presenting the experimental results, we first outline the semi-synthetic setup.

% \paragraph{Illustrative Approximation}\looseness=-1
% Although there is no doubt on the existence of optimal response, for any modern large language model with a vocabulary size \(|\Vcal|\) ( typically on the order of \(O(10^4)\) to \(O(10^5)\)), identifying the joint optimal sequence across multiple steps becomes computationally intractable, even for relatively short sequence lengths (\(<100\)). To uphold the validity of our claim while offering an accessible illustration, we adopt beam search as an approximation of the globally optimal sequence. Since we measure the probability that greedy decoding achieves optimal response, this approximation serves as an upper bound on achievable performance, highlighting the discrepancy between greedy decoding and the true global optimal response.

% \paragraph{Approximating Natural Language Scenarios}
% One may also curious about the prompt or context in the generation since those factors will influence the generation behaviour as well. It is crucial to demonstrate in realistic scenarios. Specifically, we align our experiment setting with common human interactions with large language models. To achieve this, we utilize MT-Bench questions as curated prompts and contexts, which are designed for evaluating conversational, open-ended chat models. These samples serve as an approximation of real-world natural language context distributions, ensuring that our findings are grounded in practical, human-like use cases.

% \paragraph{Results}
% For each prompt or context, we evaluate whether the joint probability of the sequence generated through greedy decoding is greater than or equal to that of the sequence produced by the approximate global optimum method, i.e., beam search. This comparison highlights the extent to which greedy decoding diverges from the optimal sequence. As demonstrated in Figure \ref{fig: synthetic exp}, greedy decoding consistently results in suboptimal sequences, even when generating a small number ($<10$) of new tokens. Additionally, the behavior varies across positions in the generated text. For example, the beginning of the answer are usually diverge and it can be harder for greedy decoding to achieve optimal. To reduce potential inductive biases arising from greater diversity at the initial stages of generation, we evaluate sequences at various positions throughout the generated text (Figure \ref{fig:subfig1}-\ref{fig:subfig4} capture different starting positions for the evaluation). Our findings indicate that this behavior persists across all evaluated positions, underscoring the importance of the issue. This observation naturally suggests a solution: a reflective decoding strategy using a sliding window mechanism.




% \input{linearelo}
\subsection{Ranking performance of Elo}
\label{sec:ranking}

Besides prediction, ranking is another important aspect that users consider when utilizing rating algorithms. There are two types of ranking: (1) for general $P$, we can consider the pairwise ranking, i.e., for each pair $(i,j) \in [N] \times [N]$, there is a ranking between $i,j$ that is induced by $P_{ij}$.
(2) for transitive $P$, there exists a ground truth ranking $\pi$ induced by the transitivity. 
In this subsection, we will show that for pairwise ranking, the ranking performance is strongly correlated to prediction performance. Elo rating, achieves good performance of pairwise ranking in sparse regimes. However Elo should not be blindly trusted, since for the total ordering, Elo may not always give a consistent ranking, even in transitive datasets.

\paragraph{Good prediction gives good pairwise ranking}
Regarding the pairwise ranking, it is natural to conjecture that pairwise ranking performance is correlated with the prediction performance, and our synthetic experiments justify this claim. We consider the same setup as the previous synthetic experiments for prediction (Section \ref{sec:synthetic}), and we calculated the pairwise ranking consistency for each algorithm at each time step:
at time $t$, an algorithm can actually give an prediction $\hat{P}_{ij}$ for every pair $(i,j) \in [N] \times [N]$. We calculate the following quantity: $\tau:=\frac{1}{N(N-1)}\sum_{i\neq j} (\one [P_{ij}>0.5] \one [\hat{P}_{ij}<0.5] + \one [P_{ij}<0.5] \one [\hat{P}_{ij}>0.5])$. Lower the value, more consistent the pairwise ranking. We plot $\tau$ against $t/N$ for Elo, Elo2k and Pairwise in Figure \ref{fig:pairwise-rk-N=100}. e can see that the ranking performance is strongly correlated with the prediction performance. To be specific, similar to the prediction accuracy, in most sparse regimes, Elo performs well in pairwise ranking. However in denser regimes, algorithms based on more complex models, such as Elo2k, may show advantage on pairwise ranking.

% \begin{figure}[t]
%     \centering
%     \includegraphics[width=0.8\columnwidth]{figures/100-pairwise_rk-new-big.png}
%     \caption{\textbf{Pairwise ranking is consistent with prediction.} Elo performs the best for small $t$, while Elo2k performs better for larger $t$.}
%     \label{fig:pairwise-rk-N=100}
% \end{figure}

% \begin{figure}[t]
%     \centering
%     \includegraphics[width=\columnwidth]{figures/100-pairwise_rk.png}

%     \caption{\textbf{Pairwise ranking consistent with prediction} Elo performs the best for small $t$,  Elo2k performs well for moderate $t$, Pairwise performs the best for large $t$.}
% \label{fig:pairwise-rk-N=100}
% \end{figure}


% \shange{change to N=100 or 500}?



\paragraph{Elo might not give consistent total ordering even for transitive models}

For transitive models, we can consider the total ordering induced by the transitivity. Elo rating, is still able to give a total ordering based on the score of each player. We will show that even though Elo can give a consistent total ordering under uniform matchmaking, it can not be blindly trusted as it may fail under arbitrary matchmaking.

% In this subsection, we consider two models of transitivity of increasing stringency: WST (Definition~\ref{def:wst}) and SST (Definition~\ref{def:sst}). Under both notions, there would be no cycles between players in terms of expected win rate, and there is an uncontested ground-truth ranking. 

From theoretical perspective, we consider the regime where $T$ goes to infinity. In this regime, by the no regret nature of OGD, one can see that the online Elo update will finally converge to the offline Elo solution $ \arg \min_{\theta \in \mathcal{K}} \frac1T \sum_{t=1}^{T} f_t(\theta) $. Also we can see that when $T \to \infty$, $\frac1T \sum_{t=1}^{T} f_t(\theta)$ converge to its population counterpart. Therefore we will consider $\theta^{\rm mle}$, the population MLE for BT model. We have the following result: 
\begin{theorem} \label{thm:Elo-winrate}
Under uniform matchmaking, $\theta^{\rm mle}$ gives identical ranking as $\overline{P}$, where 
$$\overline{P}[i] := \frac{\sum_{t=1}^T (\one[i_t = i]o_t + \one[j_t=i](1-o_t))}{\sum_{t=1}^T (\one[i_t = i]+\one [j_t=i])}$$ is the average win rate for player $i$.
%    Offline Elo recovers the ranking by average win rate under uniform matchmaking.
\end{theorem}
the formal statement and proof is deferred to Appendix \ref{proof:thm:Elo-winrate}. Notice that under SST models, the ground truth ranking is identical to the ranking given by average win rate. Therefore this theorem shows that under SST model, Elo recovers the true ranking when $T$ goes to infinity, under uniform matchmaking. 

% However, when the underlying model is WST, the ranking induced by average win rate may not be correct, therefore Elo is not guaranteed to be consistent. Moreover, when the matchmaking is \emph{arbitrary}, Elo score can produce inconsistent rankings for SST instances even when $\eta\to 0$ and $T\to \infty$. Also notice that, for actual regime where $T$ does not go to infinity, one can compute a confidence interval for each player's Elo score, and only consider ranking among players that have confidently separated Elo scores. We also show that even in this case Elo may not give consistent ranking, through a synthetic experiment. For these results, see Example~\ref{example:sst} in Appendix~\ref{sec:appendix-proof} for detail. Therefore, we conclude that Elo might not give the correct ranking for non-SST games or under non-uniform matchmaking. This suggest that although Elo can give good ranking results in many regimes, it can not be blindly trusted.

However, when the underlying model is WST, the ranking induced by average win rate may not be correct, therefore Elo is not guaranteed to be consistent. Moreover, when the matchmaking is \emph{arbitrary}, Elo score can produce inconsistent rankings for SST instances even when $\eta\to 0$ and $T\to \infty$. We also show through a synthetic experiment that even in the case where only ranking among players that have confidently separated Elo scores are considered, Elo still may not give consistent ranking. For these results, see Example~\ref{example:sst} in Appendix~\ref{sec:appendix-proof} for detail. This suggest that although Elo can give good ranking results in many regimes, it can not be blindly trusted.













\section{Conclusion}
In this work, we propose a simple yet effective approach, called SMILE, for graph few-shot learning with fewer tasks. Specifically, we introduce a novel dual-level mixup strategy, including within-task and across-task mixup, for enriching the diversity of nodes within each task and the diversity of tasks. Also, we incorporate the degree-based prior information to learn expressive node embeddings. Theoretically, we prove that SMILE effectively enhances the model's generalization performance. Empirically, we conduct extensive experiments on multiple benchmarks and the results suggest that SMILE significantly outperforms other baselines, including both in-domain and cross-domain few-shot settings.











\section*{Impact Statement}

This paper presents work whose goal is to advance the field of 
Machine Learning. There are many potential societal consequences 
of our work, none which we feel must be specifically highlighted here.

% Authors are \textbf{required} to include a statement of the potential 
% broader impact of their work, including its ethical aspects and future 
% societal consequences. This statement should be in an unnumbered 
% section at the end of the paper (co-located with Acknowledgements -- 
% the two may appear in either order, but both must be before References), 
% and does not count toward the paper page limit. In many cases, where 
% the ethical impacts and expected societal implications are those that 
% are well established when advancing the field of Machine Learning, 
% substantial discussion is not required, and a simple statement such 
% as the following will suffice:

% ``This paper presents work whose goal is to advance the field of 
% Machine Learning. There are many potential societal consequences 
% of our work, none which we feel must be specifically highlighted here.''

% The above statement can be used verbatim in such cases, but we 
% encourage authors to think about whether there is content which does 
% warrant further discussion, as this statement will be apparent if the 
% paper is later flagged for ethics review.


% In the unusual situation where you want a paper to appear in the
% references without citing it in the main text, use \nocite
% \nocite{langley00}

\bibliography{ref}

% \bibliography{example_paper}
\bibliographystyle{icml2025}


%%%%%%%%%%%%%%%%%%%%%%%%%%%%%%%%%%%%%%%%%%%%%%%%%%%%%%%%%%%%%%%%%%%%%%%%%%%%%%%
%%%%%%%%%%%%%%%%%%%%%%%%%%%%%%%%%%%%%%%%%%%%%%%%%%%%%%%%%%%%%%%%%%%%%%%%%%%%%%%
% APPENDIX
%%%%%%%%%%%%%%%%%%%%%%%%%%%%%%%%%%%%%%%%%%%%%%%%%%%%%%%%%%%%%%%%%%%%%%%%%%%%%%%
%%%%%%%%%%%%%%%%%%%%%%%%%%%%%%%%%%%%%%%%%%%%%%%%%%%%%%%%%%%%%%%%%%%%%%%%%%%%%%%
\newpage
\appendix
\onecolumn

\section{Dataset description}
\label{sec:appendix-dataset}
\paragraph{Renju} We use the RenjuNet dataset~\footnote{\url{https://www.renju.net/game/}}, where $N=5013$ and $T=124948$.
% \paragraph{LLM Arena} We use the Chatbot Arena~\footnote{\url{https://chat.lmsys.org/}} dataset, where $N=54$ LLMs are evaluated through $T=213576$ battles.
\paragraph{LLM Arena} We use the Chatbot Arena~\footnote{\url{https://chat.lmsys.org/}} dataset, where $N=129$ LLMs are evaluated through $T=1493621$ battles.
\paragraph{Chess} We use online standard chess matches on the Lichess open database~\footnote{\url{https://database.lichess.org/}} in the year of 2014, which covers $N=184920$ players and $T=11595431$ games.

\paragraph{Mixedchess} We use online matches on the Lichess database ~\footnote{\url{https://database.lichess.org/}}, for the following variants of chess: Antichess, Atomic, Chess960, Crazyhouse, Horde, King of the Hill, Racing Kings and Three-check. We collect all the matches from 2014 to 2024, and combine them together to form a large dataset with $N=2467134$ and $T=115525041$. We apply filtering \footnote{To create denser dataset, we conduct filtering on mixedchess and go. Our filtering method is: for a given threshold, we delete all the players that plays less than this threshold. Then we only consider the remaining players and the games played between them.} with threshold  $= 10000$, and create \texttt{mixedchess-dense}, where $N= 2862$ ,$T= 11791126$.

% In addition, we used the matches in 2014-07 as \texttt{chess-small}, with the hope that the skill level of players would remain closer to constant in shorter time frame, which contains $T=1048440$ games and $N=30631$.
\paragraph{Tennis} We make use of an online repository created by Jack Sackmann~\footnote{\url{https://github.com/JeffSackmann/tennis_atp}}. We used all $T=190230$ games between $7245$ players.
\paragraph{StarCraft} We downloaded match records from human players from Aligulac~\footnote{\url{http://aligulac.com/about/db/}}. $N=22056$ and $T=427042$.
\paragraph{Scrabble} We used the raw data provided in~\citet{fivethirtyeight}, which are scraped from \url{http://cross-tables.com}. $N=15374$ and $T=1542642$.
\paragraph{Go} We use the Online Go Server (OGS) database~\footnote{\url{https://github.com/online-go/goratings}}, which contains $T=12876823$ games between $426105$ players. We also filter the dataset with threshold $5000$, to get \texttt{go-dense} with $N=480$ and $T=516343$.
\paragraph{Hearthstone} We use the Deck Archetype Matchup data scraped from \url{https://metastats.net/hearthstone/archetype/matchup/}. $N=27$ and $T=62453$.
\paragraph{Blotto} We use the 5,4-Blotto and 10,5-Blotto data from \cite{czarnecki2020real}. The original data is a payoff matrix between different strategies, whose entries are between $[-1,1]$. For strategies $i$ and $j$, let the payoff for $i$ against $j$ be $r_{ij}$, we let $p_{ij}:=0.5 (r_{ij}+1)$. For \texttt{5,4-Blotto} and \texttt{10,5-Blotto}, we create games $(i_t,j_t,o_t) := (i,j,p_{ij})$ for each $(i,j)$ where $i \neq j$. For \texttt{5,4-Blotto-sparse}, we create $0-1$ game results by drawing $o_ij \sim \text{Ber} (p_ij)$ and create games $(i_t,j_t,o_t) := (i,j, o_ij)$. For \texttt{5,4-Blotto-dense}, we create $0-1$ game results by drawing $10$ independent $o_ij \sim \text{Ber} (p_ij)$ and create $10$ games $(i_t,j_t,o_t) := (i,j, o_ij)$ for each pair of $(i,j)$. $\texttt{10,5-Blotto-sparse}$ and $\texttt{10,5-Blotto-dense}$ are similarly created. $N=56$, $T=3080$ for $\texttt{5,4-Blotto}$ and $\texttt{5,4-Blotto-sparse}$. $N=56$, $T=30800$ for $\texttt{5,4-Blotto-dense}$.
$N=1001$, $T=1001000$ for $\texttt{10,5-Blotto}$ and $\texttt{10,5-Blotto-sparse}$. $N=1001$, $T=10010000$for $\texttt{10,5-Blotto-dense}$.
\paragraph{AlphaStar} We use the AlphaStar data from \cite{czarnecki2020real}. The creation procedure for \texttt{AlphaStar}, \texttt{AlphaStar-sparse}, \texttt{AlphaStar-dense} are the same as in Blotto. For all the three dataset, $N=888$. $T=787656$ for the first two datasets, and $T=7876560$ for the last one.


% \subsection{Filtering}
% For computational effiency, when conducting the real data prediction experiments in Section \ref{sec:OCO}, we do filtering on some of the datasets. Our filtering method is: for a given threshold, we delete all the players that plays less than this threshold. Then we only consider the remaining players and the games played between them. The filtered dataset are listed in Table \ref{tab:filtering}. Other datasets are not filtered.

% \begin{table*}[]
% \centering
% \addtocounter{footnote}{+1}  
% \begin{tabular}{|l|c|c|c|}
% \hline
% Dataset    & Threshold          & $N$    & $T$   \\ \hline
% \texttt{renju}       & 200   & 320  & 47161           \\
% \texttt{chess}       & 5000 & 467  & 509505        \\
% \texttt{tennis}      & 200   & 588  & 104334         \\
% \texttt{scrabble}    & 1000  & 824  & 482947     \\
% \texttt{starcraft}   & 500  & 366 & 143603       \\
% \texttt{go}          & 5000 & 480  & 516343  \\ 
% \hline
% \end{tabular}
% \caption{Datasets and the filtering thresholds}
% \label{tab:filtering}
% \end{table*}
\section{Details of the Likelihood Ratio Test}
\label{sec:lr-test}
We explain in this section the details of our likelihood ratio tests.




\subsection{Methods}
%\paragraph{Traning-test split} 
\paragraph{Symmetrization} Before performing the tests, we reversed the order of the two players (and flipped the game outcome) for every game with probability $0.5$ to eliminate first-move advantage (or disadvantage), which is well-documented~\citep{elo1978rating} and not the focus of this work. In other words, we actually test the following weaker version of Bradley-Terry model:
\[
\frac{1}{2}\left(\E\left[o_t|i_t,j_t\right]+ \E\left[1-o_t|j_t,i_t\right]\right) = \sigma\left(\theta^\star[i_t] -\theta^\star[j_t]\right).
\]
\paragraph{Feature augmentation} We split every dataset (indexed by $[T]$) randomly into equally sized $\cT_{\rm train}$ and $\cT_{\rm test}$. For all datasets except \texttt{llm arena}, we then fit a regularized logistic regression model via
\[
\theta_{\rm train}\gets \argmin_\theta \sum_{t\in \cT_{train}}\ell_t(\theta) + \frac{\lambda}{2}\Vert\theta\Vert^2,
\]
where we chose $\lambda=10.0$. Then the augmented features for match $t$ is given by
\[
g_t:=[\theta_{\rm train}[i_t], \theta_{\rm train}[j_t]].
\]
For \texttt{llm arena}, \texttt{Hearthstone}, \texttt{AlphaStar} and \texttt{Blotto}, the aforementioned feature failed to reject the null. Since those datasets are relatively dense, we designed a different feature inspired by~\citep{bertrand2023limitations}. We considered the loss
\begin{align*}
\hat\cL\left(u, v\right):=\sum_{t\in\cT_{\rm train}} \left[-o_t\log(\sigma(\hat p)) - (1-o_t)\log(1-\sigma(\hat p))\right],
\end{align*}
where $\hat p:=u[i_t]v_[j_t]-u[j_t]v[i_t]$. The loss is optimized with gradient descent with early stopping. We then define
\[
g_t:=[u[i_t]v[j_t], u[j_t]v[i_t]]
\]
as the augmented feature for game $t$. This method does not apply to other datasets as it requires a dense dataset for the learning of $u$ and $v$.

\iffalse
\paragraph{Regularization}
Recall that the log likelihood ratio statistic is 
\begin{equation*}
\Lambda := 2\left[\min_{\theta\in\R^N}\cL_{\rm test}(\theta) - \min_{\theta\in\R^{N},\alpha\in\R^2}\tilde\cL_{\rm test}([\theta;\alpha])\right].
\end{equation*}
For the ease of optimization, we instead optimize the regularized loss function to obtain a \emph{lower bound} of this statistic. In particular, we compute
\[
\theta_\lambda \gets \argmin_{\theta\in\R^N}\cL_{\rm test}(\theta)+\frac{\lambda}{2}\Vert \theta\Vert^2
\]
and 
\[
\tilde\theta_\lambda, \alpha_\lambda \gets \argmin_{\theta\in\R^{N},\alpha\in\R^2}\tilde\cL_{\rm test}([\theta;\alpha])+\frac{\lambda}{2}\Vert [\theta;\alpha]\Vert^2.
\]
Then
\begin{align*}
\cL_{\rm test}(\theta_\lambda) - \left(\Tilde{\cL}_{\rm test}([\tilde\theta;\alpha])+\frac{\lambda}{2}\Vert [\tilde\theta_\lambda,\alpha_\lambda]\Vert^2\right) \le \cL
\end{align*}

\fi


\subsection{Implementation}
All logistic regressions are implemented with JAX and optimized via L-BFGS.

\subsection{An additional martingale test}
Although the previous test used randomly sampled $\cT_{\rm train}$, it still needs to assume that the features $\{x_t\}_{t\in \cT_{\rm train}}$ are independent with   $\{y_t\}_{t\in \cT_{\rm test}}$. However, there is a concern that this may not be true if adaptive matchmaking is used -- in that case, information about the test set labels $\{y_t\}_{t\in \cT_{\rm test}}$ may be leaked through features of future games. \footnote{Regarding \texttt{AlphaStar}, \texttt{5,4-Blotto} and \texttt{10,5-Blotto}, recall that we construct the dataset according to a payoff matrix, therefore no adaptive matchmaking is used. We do not need to further test these datasets.}

To address this concern, we consider yet another method to construct $g_t$: by using the online Elo rating up until this point. This enables us to relax the assumption of independence to the assumption that the noise sequence
\[
\E[o_t|i_t,j_t] - \sigma(\theta^\star[i_t]-\theta^\star[j_t])
\]
is a martingale. This would enable us to model adaptive matchmaking.
%through the martingale property as $g_t$ is a deterministic function of \emph{past} games. 

Specifically, define
\[
g_t = [\theta_{t}[i_t], \theta_{t}[j_t]],
\]
where $\theta$ is computed using the past $t-1$ matches with learning rate $\eta$. We can then proceed to compute the likelihood ratio statistic $\Lambda$ as in the previous tests. The distribution of $\Lambda$ would still be asymptotically $\chi_2^2$ for martingale noise (see e.g. \citet[Theorem 1.5.1]{kedem2005regression})

We report the test results in Table~\ref{tab:hypotheis-martingale}. We find that by using two learning rates ($\eta=0.01$ and $0.08$), we can reject the null hypothesis that BT is realizable with extremely high confidence without assuming independence. 

\begin{table}[t]
\centering
\addtocounter{footnote}{+1}  
\begin{tabular}{|l|c|c||c|c|}
\hline
Dataset              & LR Test Statistic ($\eta=0.01$) & $p$-value   & LR Test Statistic ($\eta=0.08$) & $p$-value         \\ \hline
\texttt{Renju}       & 3.66                            & 0.23        & 226.92                          & $<10^{-10}$       \\
% \texttt{llm arena}   & 126.22                          & $<10^{-10}$ & 202.39                          & $<10^{-10}$            \\

% \texttt{chess-small} & 76.74                           & $<10^{-10}$        & 8.37                            & 0.04              \\
\texttt{Chess}       & 6622.52                         & $<10^{-10}$       & 27908.30                        & $<10^{-10}$       \\
\texttt{Tennis}      & 524.77                          & $<10^{-10}$       & 3571.70                         & $<10^{-10}$ \\
\texttt{Scrabble}    & 174.52                          & $<10^{-10}$       & 3058.76                         & $<10^{-10}$       \\
\texttt{StarCraft}   & 5.19                            & 0.12        & 261.08                          &$<10^{-10}$      \\
\texttt{Go}          & 52931.15                        & $<10^{-10}$        & 117318.3                        & $<10^{-10}$                \\
\texttt{LLM Arena}   & 872.28                          & $<10^{-10}$ & 819.05                          & $<10^{-10}$            \\

\texttt{Hearthstone}          & 69433.52                      & $<10^{-10}$        & 82005.34                             & $<10^{-10}$                \\
% \texttt{mixedchess-dense}          & 192591.30                     & $<10^{-10}$        &  409121.56                 & $<10^{-10}$                \\
\hline
\end{tabular}
\caption{Summary of martingale-based Likelihood ratio test}
\label{tab:hypotheis-martingale}
\end{table}


\section{Non-stationary matchmaking and player skills in real datasets}
\label{sec:appendix-matchmaking}
Another observation that we draw from real-world datasets is the existence of non-trivial matchmaking. We computed the correlation coefficient between $\{\theta_{\rm train}[i_t]\}_{t\in \cT_{\rm test}}$ and $\{\theta_{\rm train}[j_t]\}_{t\in \cT_{\rm test}}$, and found significant positive correlation for most datasets (see Table~\ref{tab:mm_test}). In other words, in many real datasets, stronger (higher-rated) players are matched with stronger opponents. We visualize the matchmaking in \texttt{chess} in Fig.~\ref{fig:chess-mm}. Indeed, the Elo score of the two players are highly correlated, and most games are played between two players within $20\%$ in terms of the percentile difference based on their Elo scores. Since the Elo score may vary from time to time, the matchmaking distribution should not be considered as stationary. 




\begin{table}[ht]
\centering
\addtocounter{footnote}{+1}  
\begin{tabular}{|l|c|c|}
\hline
Dataset             & Matchmaking Test & $p$-value \\ \hline
\texttt{Renju}        & 0.36             & $<10^{-10}$          \\
%\texttt{llm arena}   & 54   & 7910.2  & 1.95     & $0.61$            & 0.37             & $<10^{-10}$          \\
\texttt{Chess}        & 0.40             & $<10^{-10}$          \\
\texttt{Tennis}       & 0.19             & $<10^{-10}$          \\
\texttt{Scrabble}           & 0.57             & $<10^{-10}$          \\
\texttt{StarCraft}          & 0.46             & $<10^{-10}$         \\
\texttt{Go}                  & 0.29             & $<10^{-10}$         \\ 
\texttt{LLM Arena}     & 0.37             & $<10^{-10}$          \\
\texttt{Hearthstone}     & -0.07             & $<10^{-10}$          \\
% \texttt{mixedchess-dense}     & 0.40             & $<10^{-10}$  \\
\hline
\end{tabular}
\caption{Summary of real world datasets matchmaking hypothesis testing results.}
\label{tab:mm_test}

\end{table}


\begin{figure}[ht]
    \centering
    \includegraphics[width=\columnwidth]{figures/chess-mm-sns.png}
    \caption{\textbf{Matchmaking in \texttt{chess} dataset.} \textbf{L:} scatter plot of Elo score of the two players for each game, down-sampled for clarity; \textbf{R:} histogram for the percentile ranking difference of two players.}
    \label{fig:chess-mm}
\end{figure}



% \section{Additional Experiments with Online Elo Score}
% \label{sec:appendix-addition}

% \subsection{Bootstrap Experiments}
% \label{sec:appendix-bootstrap}
\paragraph{Bootstrap Experiments} Another evidence of matchmaking comes from the nonstationarity of gradients. If the distribution of $\{(i_t,j_t,o_t)\}$ is exchangeable, we can permute the order of the games randomly and the resulting Elo score $\theta^{\rm bootstrap}$ should be identically distributed. We can therefore detect nonstationarity by comparing $\theta^{\rm elo}$ with the distribution of $\theta^{\rm bootstrap}$. 

We compute the Elo score on $100$ independent permutations in the each dataset. The average of these samples is called the bootstrap average, and denoted by $\bar{\theta}^{\rm bootstrap}$.

The results for \texttt{chess} is presented in Fig.~\ref{fig:chess-bootstrap}. It can be seen that $\theta^{\rm elo}$, the Elo score computed with the original order of gradients, is a significant outlier and is not identically distributed with $\theta^{\rm bootstrap}$ with high probability ($p=0.01$ via the permutation test).

\begin{figure}[h]
    \centering
    \includegraphics[width=0.6\textwidth]{figures/bootstrap-chess-combined-sns.png}
    \caption{Elo score vs. bootstrap Elo scores in \texttt{chess}. \textbf{Left:} cosine similarity to the mean of $\theta^{\rm bootstrap}$; \textbf{Right:} visualization of $\theta^{\rm elo}$ vs. $\theta^{\rm bootstrap}$ via SVD for $\eta=0.02$.}
    \label{fig:chess-bootstrap}
\end{figure}

\paragraph{Varying player strengths} Other than matchmaking, we also want to point out that the player's strength may not be stationary. It is common that for a pair of players, for example, in tennis, their head-to-head game results can change dramatically over time. 

These phenomenons suggest that, in real world games, both matchmaking and players' behaviour are not stationary and non-BT. Therefore, viewing Elo rating as fitting a underlying BT-model might not be appropriate.


\section{Online rating algorithms}


\label{sec:appendix-algorithm}


In this paper, we will investigate the performance of the following algorithms: Elo, Glicko, Elo2k and Pairwise. 
\paragraph{Elo}
Elo rating gies the prediction          $p_t := \sigma\left(\theta_t[i_t] - \theta_t[j_t]\right)$. Initially $\theta_0[i]=0$ for every $i \in [N]$. The update rule is:
\begin{equation}
    \begin{cases}
    \theta_{t+1}[i_t] &\gets \theta_{t}[i_t] + \eta_t \left(o_t - p_t\right),\\
    \theta_{t+1}[j_t] &\gets \theta_{t}[j_t] - \eta_t \left(o_t - p_t\right).
    \end{cases}
\end{equation}
Here $\sigma = 1/(1+e^{-x})$ is the logistic function. $\theta_t\in\R^N$ is the \emph{rating}, or \emph{score}, for the $N$ players at time $t$. Customarily, the reported rating is multiplied by a constant $C = \frac{400}{\ln 10}$. The learning rate $\eta_t$ is often chosen to be a fixed value $\eta$ between $10/C\approx 0.06$ and $40/C\approx 0.23$. In our experiments, we choose $\eta_t$ according to the following decaying learning rate scheme: $\eta_t = \sqrt{\frac{aN}{t+b}}$, where $a,b$ are chosen to ensure the learning rate will not be too large at the beginning, and still large enough for achieving a good prediction accuracy when $t$ is large. For details see \ref{sec:appendix-choosing_parameter}.


% In fact Elo is based on the underlying Bradley-Terry model for pairwise comparison. The BT model assumes that each player $i$ has a score $\theta^*[i]$, and the expectation of the game outcome will be 

% \begin{equation}
%     \E[o_t|i_t,j_t] = \sigma(\theta^\star[i_t]-\theta^\star[j_t]) \tag{Bradley-Terry}.
% \end{equation}

Elo update can be understand as online gradient descent, as we described in Section \ref{sec:OCO}. Also, we can show that $f_t$ is convex:
\begin{align*}
    \nabla_{\theta} f_t(\theta) &= -\nabla_{\theta} (o_t \log  p_t + (1-o_t) \log (1-p_t)) \\
    &=- (o_t   \frac{1}{p_t} \nabla_{\theta}  p_t + (1-o_t)  \frac{1}{1-p_t} ( -\nabla_{\theta}  p_t ) ) \\
    &= - \frac{o_t - p_t}{p_t(1-p_t)} \nabla_{\theta}  p_t \\
    &= - (o_t - p_t) (\boldsymbol{e}_{i_t}- \boldsymbol{e}_{j_t}).
\end{align*}

\begin{align*}
    \nabla^2_{\theta} f_t(\theta) &= \nabla_{\theta} (\nabla_{\theta} f_t(\theta)) \\
    &= -\nabla_{\theta} ( (o_t - p_t) )(\boldsymbol{e}_{i_t}- \boldsymbol{e}_{j_t}) \\
    &= -(\boldsymbol{e}_{i_t}- \boldsymbol{e}_{j_t}) (\nabla_{\theta}(o_t - p_t) )^T \\
    &= p_t (1-p_t) (\boldsymbol{e}_{i_t}- \boldsymbol{e}_{j_t}) (\boldsymbol{e}_{i_t}- \boldsymbol{e}_{j_t})^{T}\\
    & \succeq 0.
\end{align*}



Details can be seen in Section \ref{sec:OCO}.

\paragraph{Glicko} Glicko \citep{glickman1995glicko} assumes each player has a rating $\theta$ and  a “ratings deviation” $v$. The initial $\theta$ of each player is set to be $1500$, and
% and the initial $v$ is set to be $350$ by default. In our experiments, 
we set initial $v$ to be $35,100$ or $350$. The prediction $p_t:= \sigma( \frac{\ln 10}{400} g(\sqrt{v_t[i_t]^2 + v_t[j_t]^2}) (\theta_t[i_t] - \theta_{t}[j_t]) )$, where $\sigma$ is the logistic function. The update rule of the parameters is
\begin{equation}
    \begin{cases}
    \theta_{t+1}[i_t] &\gets \theta_{t}[i_t] + \frac{\ln 10}{400} (\frac{1}{v_t[i_t]^2} + \frac{1}{d(i_t,j_t))^2})^{-1} g(v_{t}[j_t])(o_t - \tilde{p}(i_t,j_t)), \\
    \theta_{t+1}[j_t] &\gets \theta_{t}[j_t] + \frac{\ln 10}{400} (\frac{1}{v_t[j_t]^2} + \frac{1}{d(j_t,i_t))^2})^{-1} g(v_{t}[i_t])(1- o_t - \tilde{p}(j_t,i_t)), \\
    v_{t+1}[i_t] &\gets \sqrt{(\frac{1}{v_{t}[i_t]^2} + \frac{1}{d(i_t,j_t)^2})^{-1}} , \\
    v_{t+1}[j_t] &\gets \sqrt{(\frac{1}{v_{t}[j_t]^2} + \frac{1}{d(j_t,i_t)^2})^{-1}} ,
    \end{cases}
\end{equation}
where $g(x):= (1 + \frac{3 (\frac{\ln 10}{400})^2 x^2 }{\pi^2})^{-\frac12}$, $\tilde{p}(i_t,j_t):= v( \frac{\ln 10}{400} g(v_t[i_t]) (\theta_t[i_t] - \theta_{t}[j_t]) )$, and $d(i_t,j_t)^2 := ((\frac{\ln 10}{400})^2 g(v_t[j_t]^2) \tilde{p}(i_t,j_t) (1- \tilde{p}(i_t,j_t)))^{-1}$.

\paragraph{TrueSkill} TrueSkill \citep{dangauthier2007trueskill} assumes each player 
has an average skill $\theta$ and a degree of uncertainty  $v$, similar to Glicko. The difference is that TrueSkill use a Gaussian function for prediction, rather than logistic: the prediction $p_t:=\frac{1}{c\sqrt{2}}\Phi(\sqrt{2}( \theta_t[i_t]-\theta_t[j_t]))$, where $\Phi$ is the CDF for standard normal distribution, and $c_t = \sqrt{2\beta^2 + v_{t}[i_t]^2 + v_{t}[j_t]^2}$ is the overall variance. 
% $\beta$ is set to be $25/6$ by default, and the initial $v^2$ for every player is set to be $4 \beta^2$ by default. In our experiments, we set $\beta$ to be $0.2$ or $1.0$. 
In our experiments, we set $\beta$ to be $0.2,0.8$ or $1.0$, and the initial $v^2$ for every player is set to be $4 \beta^2$ by default.


The update rule for the parameters is 
\begin{equation}
    \begin{cases}
    \theta_{t+1}[i_t] &\gets \theta_{t}[i_t] + (2 o_t - 1) \frac{v_{t}[i_t]^2}{c_t^2} v(\frac{( \theta_t[i_t]-\theta_t[j_t])(2 o_t - 1)}{c_t}),\\
    \theta_{t+1}[j_t] &\gets \theta_{t}[j_t] - (2 o_t - 1) \frac{v_{t}[j_t]^2}{c_t^2} v(\frac{( \theta_t[i_t]-\theta_t[j_t])(2 o_t - 1)}{c_t}),\\
    v_{t+1}[i_t] &\gets v_{t}[i_t] \sqrt{1- \frac{v_{t}[i_t]^2}{c_t^2} w(\frac{( \theta_t[i_t]-\theta_t[j_t])(2 o_t - 1)}{c_t})}, \\
    v_{t+1}[j_t] &\gets v_{t}[j_t] \sqrt{1- \frac{v_{t}[j_t]^2}{c_t^2} w(\frac{( \theta_t[i_t]-\theta_t[j_t])(2 o_t - 1)}{c_t})}  
    \end{cases}
\end{equation}

where $v(x):= \frac{\phi(x)}{\Phi(x)}$ ($\phi$ is the pdf of standard Gaussian), $w(x):=v(x)(v(x)+x)$. 


 


\paragraph{Elo2k} If we generalize Elo score by rating every player with a vector instead of scalar (see \cite{balduzzi2018re} and \cite{bertrand2023limitations}), we get Elo2k. The parameter for the algorithm is  $\theta = (U, V)$, where $U= (U[1], \cdots, U[N]), V = (V[1], \cdots, V[N])$, $U[i], V[i] \in \mathbb{R}^{k}$. The prediction $p_t:=\sigma(U[i_t]^{T}V[j_t]-U[j_t]^{T}V[i_t])$. In this paper, we initially choose each element of $U$ (or $V$) from $\text{Uniform}([0,0.1])$. 
The update rule is given by taking the gradient of $U,V$, i.e.,
\begin{equation}
    \begin{cases}
    U_{t+1}[i_t] &\gets U_{t}[i_t] + \eta_t \left(o_t - p_t\right) V_{t}[j_t] ,\\
    U_{t+1}[j_t] &\gets U_{t}[j_t] - \eta_t \left(o_t - p_t\right) V_{t}[i_t] ,\\

    V_{t+1}[i_t] &\gets V_{t}[i_t] - \eta_t \left(o_t - p_t\right) U_{t}[j_t] ,\\
    V_{t+1}[j_t] &\gets V_{t}[j_t] + \eta_t \left(o_t - p_t\right) U_{t}[i_t] \\
    \end{cases}
\end{equation}
In our experiments, we choose $\eta_t$ according to the following decaying learning rate scheme: $\eta_t = \sqrt{\frac{aN}{t+b}}$.

\paragraph{Pairwise} A very natural algorithm is that we compute the pairwise win rate $P_t[i,j]$ for each pair of players $(i,j)$, and the prediction $p_t = P_t[i_t,j_t] $. This algorithm has $N(N-1)/2$ parameters. To ensure the prediction will not be affected dramatically by a single game result, we will regularize it as the following. The update rule is given by
\begin{equation}
    \begin{cases}
    P_{t}[i_t, j_t] &\gets \frac{5 + \# \{ \text{games that } i_t \text{ wins } j_t \text{ until time } t\} }{10 + \# \{ \text{games that } i_t \text{ plays with } j_t \text{ until time } t\}} ,\\
    P_{t}[j_t, i_t] &\gets \frac{5 + \# \{ \text{games that } j_t \text{ wins } i_t \text{ until time } t\} }{10 + \# \{ \text{games that } j_t \text{ plays with } i_t \text{ until time } t\}} \\

    \end{cases}
\end{equation}


\section{Details of real-world data experiments}
\label{sec:appendix-realdata}
For each dataset, we evaluate the performance of Elo, Elo2k (with $k=4$), Glicko, TrueSkill, and Pairwise, plotting the cumulative loss $\frac{1}{t} \mathcal{L}_t$ over "normalized" time $t/N$. For each algorithm, we choose the best hyperparameter (see Appendix \ref{sec:appendix-choosing_parameter}). We also plot the in hindsight baselines at time $T$ ($\min_{x \in \mathcal{K}}\frac1T \sum_{t=1}^{T}f_t(x)$) of BT model and Elo2k model.  The results are presented in Figure \ref{fig:realdata-CE}. We can observe that for several real-world datasets, including \texttt{chess}, \texttt{go}, \texttt{renju}, \texttt{tennis}, \texttt{scrabble} and \texttt{StarCraft},
Elo and its variants (TrueSkill and Glicko) outperform algorithms based on more complex models such as Elo2k and Pairwise. Namely, Elo consistently exhibits a lower cumulative loss compared to Elo2k for every $0<t<T$. For other datasets like \texttt{Hearthstone}, \texttt{AlphaStar}, \texttt{10,5-Blotto}, \texttt{go-dense}, and \texttt{mixedchess-dense}, Elo2k achieves lower prediction errors than Elo at the final time $t=T$.
\begin{figure}[t]

    \centering
    \includegraphics[width=\columnwidth]{figures/all-datasets-test-CE-new.png}
    
    \caption{\textbf{In real datasets, sparsity strongly influences prediction.} Elo, TrueSkill, Glicko achieves the best prediction in sparse datasets, while Elo2k and Pairwise outperforms Elo and its variants in dense datasets.}
    \label{fig:realdata-CE}
\end{figure}

% \subsection{Re-investigating real-data experiments: data sparsity level is crucial}

We can further investigate the results through the lens of regret minimization. We can see that the cumulative loss for each algorithm decreases over time, indicating the regret minimization effect of those algorithms. However the behavior for each algorithm at each sparsity level $t/N$ are not the same. These phenomenons are closely related to the sparsity level of dataset: when data is sparse, typically when $t/N<1000$, the regret for Elo2k and Pairwise is so large, that even though the in hindsight baseline is much better, the cumulative loss $\frac{1}{t}\mathcal{L}_{t}$ for Elo2k will be large due to the large regret. Meanwhile, Elo achieves good performance due to its small regret. This may due to the fact that Elo, as online gradient descent for convex loss, has provable regret guarantees (Theorem \ref{thm:OCO}) that ensures its performance. On the contrary, Elo2k suffer from its non-convex nature, and Pairwise has a much larger regret due to its parameter size of $N^{2}$ that is much larger then $N$, the Elo parameter size. When dataset is dense enough, for example, AlphaStar-dense, when $T/N >1000$, we can see that the regret at time $T$ for both Elo2k and Elo are very small. In this regime, model capacity come into play. The baseline for Elo2k model is much smaller than the Elo counter part, therefore Elo2k shows better prediction accuracy than Elo.  Among these dense datasets, LLM is special, since the Elo2k baseline and the Elo baseline are so close, that even the dataset is dense, Elo2k does not show any benefit.


We can futher see the influence of sparsity level, when we examine the dataset from \citet{czarnecki2020real}: for AlphaStar, 5,4-Blotto and 10,5-Blotto, we create sparse version and dense version, where the underlying model is exactly the same, but "dense" version has 10 times sample size than "sparse" version. Through the comparison of these datasets, we can see that even under the same probabilistic model (which is non-BT), the behaviors of algorithms are still mainly affected by the sparsity level. 
%
\section{Details of synthetic experiments}


\subsection{Constructing $P$ for transitive models}
\label{sec:appendix-generatingP}

We consider several  ways of generating a SST/WST matrix $P$ w.r.t. the ordering $\pi(i)=N-i$. In the following constructions, we will firsts specify $P_{ij}$ for $i<j$, then make the matrix skew-symmetric by setting $P_{ij}=1-P_{ji}$ for $i>j$, and $P_{ii}=0.5$. 

\paragraph{SST-byrow} We first generate a i.i.d. random sequence of length $N-1$, each element is sampled from $\text{Uni}([0,1])$. Then we sort this sequence in a descending order $r_1 \geq r_2 \geq \cdots \geq r_{N-1} $. We let $P_{ij}=0.5+0.5 \times r_i$ for $i<j$.

\paragraph{SST-bydiagonal} We first get the descending sequence $r_1 \geq r_2 \geq \cdots \geq r_{N-1} $ in the same way as SST-byrow. We let $P_{ij}=0.5+0.5 \times r_{N-j+i}$ for $i<j$.

\paragraph{SST-byentry} 
Following the "noisy sorting" model, we set 
$P_{ij}=0.6$ for $i<j$.


% \paragraph{SST-hard}
% Let $k=2.5, \delta=0.5,  \gamma=3.0, M=\lfloor N/k \rfloor$. We let \begin{align*}
% P_{ij} = 0.5 + \lfloor (\frac{j-i}{M})^{\gamma} \rfloor  \times (\frac{M}{N})^{\gamma} \times \delta
% \end{align*}
% for $i<j$.

\paragraph{WST-byrow} We first generate a i.i.d. random sequence of length $N-1$, each element is sampled from $\text{Uni}([0,1])$. Then we sort this sequence in an ascending  order $r_1 \leq r_2 \leq \cdots \leq r_{N-1} $. We let $P_{ij}=0.5+0.5 \times r_i$ for $i<j$.

\paragraph{WST-bydiagonal} We first get the ascending sequence $r_1 \leq r_2 \leq \cdots \leq r_{N-1} $ in the same way as WST-byrow. We let $P_{ij}=0.5+0.5 \times r_{N-j+i}$ for $i<j$.

\paragraph{WST-byentry} 
We set $P_{ij}=0.5+0.5 \times U_{ij}$ for $i<j$, where $U_{ij}\sim \text{Uni}[0,1]$.


% \paragraph{WST-hard}
% Let $\text{bandwidth}=20,  \text{period}=1000$. For $i<j$, let $r_i:=\lfloor \frac{i}{\text{bandwidth}} \rfloor$,
% $c_j=\lfloor \frac{j}{\text{bandwidth}} \rfloor$. If $r_i=c_j$, then let $P_{ij}=0.5$. Otherwise we set
% \begin{align*}
% P_{ij} = \min \{1.0, \max \{ 0.5,0.5+0.5 \times (1-\frac{r_i \times \text{bandwidth}}{N}) \times \cos \bigg( \frac{2\pi j} {\text{period}} \bigg)  \} \}.
% \end{align*}

% \subsection{Constructing non-transitive models} 
% We consider the following two types of non-transitive models:

% \paragraph{Elo2k} We first generate $U \in \mathbb{R}^{N \times k}$ and $V \in \mathbb{R}^{N \times k}$, where each entry of $U$ and $V$ are independently drawn from $\text{Uni}([0,1])$. Let $P:=UV^{T} - VU^{T}$.


% \paragraph{Pure-random}
% We draw $P_{ij}$ from $\text{Uni}([0,1])$, independently for $i<j$. Then we make the matrix skew-symmetric by setting $P_{ij}=1-P_{ji}$ for $i>j$, and $P_{ii}=0.5$.

\subsection{Choosing the best hyperparameter}

\label{sec:appendix-choosing_parameter}

For each algorithm (Elo, Elo2k, Glicko and TrueSkill), there are different hyperparameters that need to be chosen. We choose the parameters according to the follow criterion:

Let $CE_i := \frac{1}{t}\mathcal{L}_{t_i}$ be the CrossEntropy Loss at time $t_i$, where $\{t_i\}_{i=1}^n$ be the time steps we collect the loss.
Define the threshold $ M = \log(2) $ (A purely random prediction will have a loss of $M$). The loss function  $L(\mathbf{v})$ is given by:
 can be expressed as:
$$
L = \sum_{i=1}^{30} \big( CE_i + 5 (CE_i - M) \mathbb{I}(CE_i > M) \big).
$$
We select the hyperparameter that minimizes the loss $L$ . This loss function ensures that the chosen parameter achieves a consistently low average cross-entropy (CE) loss throughout the process while avoiding overfitting at some point (where  $CE_i > M$  indicates overfitting).

\subsection{Creating non-stationary datasets}
\label{sec:appendix-non_stationary}

Specifically, for modeling the varying player strength, for each type of underlying distribution (e.g., \texttt{SST, byrow}), we generate two matrices $P^0$ and $P^T$, and let $P^t = (1-t/T) \times P^0 + (t/T) \times P^T$ be the win rate matrix at each time $t$. That is, $\E[o_t|i_t=i,j_t=j]=P^t_{ij}$.



For modeling non-trivial matchmaking, we construct the game dataset as the following: at each time point $t$, we sample $i_t \sim \text{Uni}([N])$, and then sample $j_t$ uniformly from the players that has ranking (by the real-time Elo score) within distance $K/2$ from $i_t$'s ranking. To be more concrete, let the ranking induced by Elo scores $(\theta[1], \cdots, \theta[N])$ be $\pi=(\pi(1), \cdots, \pi(N))$, a permutation of $(1,2,\cdots,N)$ such that $\theta [\pi^{-1}(N)] > \theta [\pi^{-1}(N-1)] > \cdots > \theta [\pi^{-1}(1)] $. Then $j_t$ is chosen uniformly from the set $\{j \in [N]: |\pi(j)-\pi(i_t)| \leq K/2\}$. We choose $K=N/5$. After constructing such a game dataset, we fix this dataset and plot the performance of each algorithm
% \section{Additional Experiments with Online Elo Score}
\label{sec:appendix-addition}

\subsection{Bootstrap Experiments}
\label{sec:appendix-bootstrap}
Another evidence of matchmaking comes from the nonstationarity of gradients. If the distribution of $\{(i_t,j_t,o_t)\}$ is exchangeable, we can permute the order of the games randomly and the resulting Elo score $\theta^{\rm bootstrap}$ should be identically distributed. We can therefore detect nonstationarity by comparing $\theta^{\rm elo}$ with the distribution of $\theta^{\rm bootstrap}$. 

We compute the Elo score on $100$ independent permutations in the each dataset. The average of these samples is called the bootstrap average, and denoted by $\bar{\theta}^{\rm bootstrap}$.

The results for \texttt{chess} is presented in Fig.~\ref{fig:chess-bootstrap}. It can be seen that $\theta^{\rm elo}$, the Elo score computed with the original order of gradients, is a significant outlier and is not identically distributed with $\theta^{\rm bootstrap}$ with high probability ($p=0.01$ via the permutation test).

\begin{figure}[h]
    \centering
    \includegraphics[width=0.6\textwidth]{figures/bootstrap-chess-combined-sns.png}
    \caption{Elo score vs. bootstrap Elo scores in \texttt{chess}. \textbf{Left:} cosine similarity to the mean of $\theta^{\rm bootstrap}$; \textbf{Right:} visualization of $\theta^{\rm elo}$ vs. $\theta^{\rm bootstrap}$ via SVD for $\eta=0.02$.}
    \label{fig:chess-bootstrap}
\end{figure}

% \subsection{Similarity of ranking between $\theta^{\rm elo}$ and $\theta^{\rm lin}$}
% We also report the similarity  between exact Elo score and linearized Elo score measured with Kendall's-$\tau$.
% \begin{figure}[H]
%   \includegraphics[width=.9\textwidth]{figures/elo-linear-kt-boot.png}
%   \caption{Linearized Elo score provides a better approximation to Elo score than bootstrap average and regularized MLE.}
%   \label{fig:linear-elo-vs-reg-path}
% \end{figure}

\newpage
\appendix
\section{Proofs of Theorem~\ref{the:mutual}}
\label{app:proof}
To prove Theorem~\ref{the:mutual} in the main paper, we first present the following  Lemmas:

\begin{lemma}\label{lemma:1}
Given the model conditional distribution $\pi_{\theta}(\mathbf{y} \mid \mathbf{x})=\prod_{i=1}^{|\mathbf{y}|} \pi_\theta\left(\mathbf{y}_i \mid \mathbf{x}, \mathbf{y}_{<i}\right)$ and  data distribution $\pi_{\rm{data}}(\mathbf{y} \mid \mathbf{x})=\prod_{i=1}^{|\mathbf{y}|} \pi_{\rm{data}}\left(\mathbf{y}_i \mid \mathbf{x}, \mathbf{y}_{<i}\right)$, then we have the following relationship between the sequence-level TVD objective and its token-level factorization:
   \begin{equation}
    {\mathrm{TV}}(\pi_{\rm{data}}(\mathbf{y} \mid \mathbf{x}), \pi_{\theta}(\mathbf{y} \mid \mathbf{x})) \le \mathbb{E}_{\mathbf{y}\sim \pi_{\rm{data}}(\mathbf{y} \mid \mathbf{x})}\Bigg[\sum\nolimits_{i=1}^{|\mathbf{y}|} {\mathrm{TV}}(\pi_{\rm{data}}^{<i}(\mathbf{y}_{i}), \pi_{\theta}^{<i}(\mathbf{y}_{i}))\Bigg],
\end{equation}
where $\pi_{\rm{data}}^{<i}(\mathbf{y}_{i})$ and $\pi_{\theta}^{<i}(\mathbf{y}_{i})$ are shorts for $\pi_{\rm{data}}\left(\mathbf{y}_i \mid \mathbf{x}, \mathbf{y}_{<i}\right)$ and  $\pi_\theta\left(\mathbf{y}_i \mid \mathbf{x}, \mathbf{y}_{<i}\right)$, respectively.
\end{lemma}
\begin{proof}
We  start by re-writing the TVD loss in the following recursive form:
\begin{align}
    & {\mathrm{TV}}(\pi_{\rm{data}}(\mathbf{y} \mid \mathbf{x}), \pi_{\theta}(\mathbf{y} \mid \mathbf{x}))=\frac{1}{2}\sum  \nolimits_{\mathbf{y}\in \mathcal{Y}} \left| \pi_{\rm{chosen}}(\mathbf{y}\mid\mathbf{x}) -  \pi_\mathbf{\theta}(\mathbf{y} \mid \mathbf{x})  \right|\\
     &=\frac{1}{2} \sum_{\mathbf{y}_1,\cdots,\mathbf{y}_{|\mathbf{y}|}}|\prod_{i=1}^{|\mathbf{y}|} \pi_{\rm{data}}\left(\mathbf{y}_i \mid \mathbf{x}, \mathbf{y}_{<i}\right)-\prod_{i=1}^{|\mathbf{y}|} \pi_\theta\left(\mathbf{y}_i \mid \mathbf{x}, \mathbf{y}_{<i}\right)| \label{Eq:lemma1-1}\\
     &=\frac{1}{2} \sum_{\mathbf{y}_1,\cdots,\mathbf{y}_{|\mathbf{y}|}}|\prod_{i=1}^{|\mathbf{y}|} \pi_{\rm{data}}^{<i}(\mathbf{y}_{i})-\prod_{i=1}^{|\mathbf{y}|} \pi_{\theta}^{<i}(\mathbf{y}_{i})|\\
     &\leq \frac{1}{2}\sum_{i=1}^{|\mathbf{y}|}\sum_{\mathbf{y}_1,\cdots,\mathbf{y}_{i}}\prod _{j=1}^{i-1} \pi_{\rm{data}}^{<j}(\mathbf{y}_{j})|\prod_{i=1}^{|\mathbf{y}|} \pi_{\rm{data}}^{<i}(\mathbf{y}_{i})-\prod_{i=1}^{|\mathbf{y}|} \pi_{\theta}^{<i}(\mathbf{y}_{i})| \sum_{\mathbf{y}_{i+1},\cdots,\mathbf{y}_{|\mathbf{y}|}}\prod_{t=i+1}^{|\mathbf{y}|} \pi_{\theta}^{<t}(\mathbf{y}_{t}) \label{Eq:lemma1-2}\\
     &= \frac{1}{2}\sum_{i=1}^{|\mathbf{y}|}\sum_{\mathbf{y}_1,\cdots,\mathbf{y}_{i}}\prod _{j=1}^{i-1} \pi_{\rm{data}}^{<j}(\mathbf{y}_{j})|\pi_{\rm{data}}^{<i}(\mathbf{y}_{i})- \pi_{\theta}^{<i}(\mathbf{y}_{i})| \label{Eq:lemma1-3} \\
       &= \frac{1}{2}\sum_{i=1}^{|\mathbf{y}|}\sum_{\mathbf{y}_{i}} \mathbb{E}_{ \mathbf{y_{<i}}\sim \pi_{\rm{data}}}\Big[|\pi_{\rm{data}}^{<i}(\mathbf{y}_{i})- \pi_{\theta}^{<i}(\mathbf{y}_{i})|\Big]\\
       &=\mathbb{E}_{\mathbf{y}\sim \pi_{\rm{data}}}\Bigg[ \sum_{i=1}^{|\mathbf{y}|} \mathrm{TV}(\pi_{\rm{data}}^{<i}(\mathbf{y}_{i}),\pi_{\theta}^{<i}(\mathbf{y}_{i})) \Bigg],
\end{align}
where Equation~(\ref{Eq:lemma1-1}) breaks the sequence-level summation into the steps and Equation~(\ref{Eq:lemma1-2}) applies the following triangle inequality~\citep{wen2023f,jitailoring}:
\begin{align}
\left| \prod_{t=1}^{T} a_t - \prod_{t=1}^{T} b_t \right|
\leq \sum_{t=1}^{T} \left| a_t - b_t \right| \cdot
\left( \prod_{i=1}^{t-1} a_i \right) \cdot
\left( \prod_{j=t+1}^{T} b_j \right),
\end{align}
and Equation~(\ref{Eq:lemma1-3}) marginalizes out variables $\mathbf{y}_{i+1},\cdots,\mathbf{y}_{|\mathbf{y}|}$. 
\end{proof}


% \begin{lemma}\label{lemma:2}
% Minimizing the \texttt{SimPO} objective in Equation~(\ref{Eq:SimPO}) with respect to ${\theta}$ reduces the Kullback–Leibler divergence between $\pi_\theta$ and the distribution of chosen response in preference dataset:
% \begingroup\makeatletter\def\f@size{9.5}\check@mathfonts\def\maketag@@@#1{\hbox{\m@th\normalfont\normalfont#1}}
% \begin{align}
% \min_{\theta} \mathcal{L}_{\rm{SimPO}} \Rightarrow  \min_{\theta}{\mathrm{KL}}(\pi_{\rm{chosen}}(\mathbf{y}\mid\mathbf{x})\|  \pi_\mathbf{\theta}(\mathbf{y} \mid \mathbf{x})  ) =\sum  \nolimits_{y\in \mathcal{Y}} \pi_{\rm{chosen}}(\mathbf{y} \mid \mathbf{x})\log \frac{\pi_{\rm{chosen}}(\mathbf{y} \mid \mathbf{x})}{\pi_{\theta}(\mathbf{y} \mid \mathbf{x})}
% \end{align}
% \endgroup
% \end{lemma}
% \begin{proof}
% Recall that the objective of \texttt{SimPO} without hyperparmeters is:
% \begin{align}
% \mathcal{L}_{\rm{SimPO}}({\theta}; \mathcal{D})=\mathbb{E}_{(\mathbf{x}, \mathbf{y}_w, \mathbf{y}_l)\sim \mathcal{D}}\left[-\log \sigma( \log {\pi_{\theta}(\mathbf{y}_w \mid \mathbf{x})}-\log {\pi_{\theta}(\mathbf{y}_l \mid \mathbf{x} )}) \right]. \label{Eq:app-simpo}
% \end{align}
% We connect this objective to a distribution matching problem by defining an energy-based model:
% % \begingroup\makeatletter\def\f@size{9.5}\check@mathfonts\def\maketag@@@#1{\hbox{\m@th\normalfont\normalfont#1}}
% \begin{align}
% \pi(\mathbf{y} \mid \mathbf{x})={\exp(r(\mathbf{x},\mathbf{y}))}/{Z(\mathbf{x})}, \label{Eq:optimal}
% \end{align}
% % \endgroup
% where $Z(\mathbf{x})=\sum_{\mathbf{y}}\exp(r(\mathbf{x},\mathbf{y}))$. We then frame the task as minimizing the KL divergence between the policy $\pi$ and the distribution $\pi_{\mathrm{chosen}}$ that generates the chosen response. The goal is to learn the reward model $r$ such that the model distribution imitates the chosen data distribution:
% \begin{align}
% \min _{r} {\mathrm{KL}}\left(\pi_{\mathrm{chosen}}(\mathbf{y} \mid \mathbf{x}) \| \pi(\mathbf{y} \mid \mathbf{x})\right). \label{eq:FKL}
% \end{align}
% Minimizing the above forward KL divergence with the chosen responses on preference data gives us:
% \begingroup\makeatletter\def\f@size{9.5}\check@mathfonts\def\maketag@@@#1{\hbox{\m@th\normalfont\normalfont#1}}
% \begin{align}
% \min _{r} &\mathbb{E}_{(\mathbf{x},\mathbf{y}_{w})\sim \mathcal{D}}[-\log \frac{\exp(r(\mathbf{x},\mathbf{y}_{w}))}{Z(\mathbf{x})}]=\\
%     &\mathbb{E}_{(\mathbf{x},\mathbf{y}_{w})\sim \mathcal{D}} \Big [-r(\mathbf{x},\mathbf{y}_{w})+\log \mathbb{E}_{q(\mathbf{y}\mid
%     \mathbf{x})} \big[ \frac{\exp (r(\mathbf{x},\mathbf{y}))}{q(\mathbf{y}\mid\mathbf{x})} \big] \Big ]. \label{Eq:connection}
% \end{align}
% \endgroup
% The equality follows due to importance sampling under distribution $q$.  A choice that simplifies the above expression is $q(\mathbf{y} \mid \mathbf{x})=\frac{1}{2} \mathbb{I}(\mathcal{Y}=\mathbf{y}_{l})+\frac{1}{2}  \mathbb{I}(\mathcal{Y}=\mathbf{y}_{w})$.  In this case, the sample-based approximation of Equation~(\ref{Eq:connection})  gives us:
% \begin{align}
% \min_{r}&\mathbb{E}_{(\mathbf{x},\mathbf{y}_{w},\mathbf{y}_{l})\sim \mathcal{D}}\Big [-r(\mathbf{x},\mathbf{y}_{w})+\log (\exp(r(\mathbf{x},\mathbf{y}_{w}))+ \exp(r(\mathbf{x},\mathbf{y}_{l})))\Big ] \nonumber \\
% &=\mathbb{E}_{(\mathbf{x},\mathbf{y}_{w},\mathbf{y}_{l})\sim \mathcal{D}}\Big [-\log (\frac{1}{1+ \exp(r(\mathbf{x},\mathbf{y}_{l})-r(\mathbf{x},\mathbf{y}_{w}))} \Big]. \label{Eq:imtaiton-reward}
% \end{align}
% One can easily see that the above loss is exactly the same as the reward loss function in RLHF~\citep{rafailov2024direct}. By optimizing this loss function, one can directly obtain the optimal policy via Equation~(\ref{Eq:optimal}). Unfortunately,  even if we use the estimate $r$, it is
% still expensive to estimate the partition function $Z(\mathbf{x})$, which makes this representation hard to utilize in practice and  significantly higher inference cost~\citep{rafailov2024direct}. To address this problem,  we apply the following knowledge distillation loss, which distills the optimal policy in Equation~(\ref{Eq:optimal}) into a analytical policy:
% \begin{align}
% \min_{\pi_\theta}\mathrm{KL}\left(\pi_{\theta}(\mathbf{y}\mid \mathbf{x})||{\exp(r(\mathbf{x},\mathbf{y}))}/{Z(\mathbf{x})} \right), \label{Eq:distillation}
% \end{align}
% This gives the following loss after removing multiplicative and additive constants:
% \begin{align}
%     \mathcal{L}(\theta)=-\mathbb{E}_{\pi_\theta(\mathbf{y} \mid \mathbf{x})}\left[r_{\phi}\left(\mathbf{x}, \mathbf{y}\right)-\log \pi_{\theta}(\mathbf{y}|\mathbf{x})\right], \label{Eq:distillation2}
% \end{align}
% One can observe that this knowledge distillation objective directly corresponds to the RL objective in RLHF, as shown in \citep{rafailov2024direct}. Similar to the derivation in~\citep{rafailov2024direct}, which uses an analytical mapping from reward functions to optimal policies in Equation~(\ref{Eq:distillation2}), combining Equation~(\ref{Eq:imtaiton-reward}) results in the SimPO objective in Equation~(\ref{Eq:app-simpo}). Therefore, performing distribution matching over the chosen responses using KL divergence ${\mathrm{KL}}\left(\pi_{\mathrm{chosen}}(\mathbf{y} \mid \mathbf{x}) \| \pi_{\theta}(\mathbf{y} \mid \mathbf{x})\right)$ is equivalent to solving an entropy-regularized RLHF problem, as well as the SimPO objective.
% \end{proof}


\textbf{Theorem~\ref{the:mutual}.} \textit{
Minimizing  \texttt{SFT}  with respect to ${\theta}$ is approximately minimizing the KLD between $\pi_\theta$ and the distribution of the chosen response in the preference dataset, while minimizing our \method  is approximately minimizing the TVD.}
\begingroup\makeatletter\def\f@size{9.5}\check@mathfonts\def\maketag@@@#1{\hbox{\m@th\normalfont\normalfont#1}}
\begin{align}
\min_{\theta} \mathcal{L}_{\rm{SFT}} &\Rightarrow  \min_{\theta}{\mathrm{KL}}(\pi_{\rm{chosen}}(\mathbf{y}\mid\mathbf{x})\|  \pi_\mathbf{\theta}(\mathbf{y} \mid \mathbf{x})  ) =\sum  \nolimits_{y\in \mathcal{Y}} \pi_{\rm{chosen}}(\mathbf{y} \mid \mathbf{x})\log \frac{\pi_{\rm{chosen}}(\mathbf{y} \mid \mathbf{x})}{\pi_{\theta}(\mathbf{y} \mid \mathbf{x})}\\
\min_{\theta} \mathcal{L}_{\rm{SimPER}} &\Rightarrow  \min_{\theta}{\mathrm{TV}}(\pi_{\rm{chosen}}(\mathbf{y}\mid\mathbf{x})\|  \pi_\mathbf{\theta}(\mathbf{y} \mid \mathbf{x})  ) =\sum\nolimits_{y\in \mathcal{Y}} \left| \pi_{\rm{chosen}}(\mathbf{y}\mid\mathbf{x}) -  \pi_\mathbf{\theta}(\mathbf{y} \mid \mathbf{x})  \right| 
\end{align}
\endgroup
\begin{proof}
Given that $\mathbb{E}_{y \sim \pi_{\rm{data}}^{<i}(\mathbf{y}_{i})}[\mathbf{y}_{i}] = \pi_{\rm{data}}^{<i}(\mathbf{y}_{i})$, where $\mathbf{y}_{i}$ represents the observed one-hot distribution with only the  $w$-th index of the observed token being 1 and the others being 0, we have
\begin{align}
\mathrm{TV}&(\pi_{\rm{data}}^{<i}(\mathbf{y}_{i}), \pi_{\theta}^{<i}(\mathbf{y}_{i}))=\frac{1}{2}\sum_{\mathbf{y}_{i}}|\pi_{\rm{data}}^{<i}(\mathbf{y}_{i})- \pi_{\theta}^{<i}(\mathbf{y}_{i})|\\
&=\frac{1}{2}\sum_{\mathbf{y}_{i}}|\mathbb{E}_{y \sim \pi_{\rm{data}}^{<i}(\mathbf{y}_{i})}[\mathbf{y}_{i}]  - \pi_{\theta}^{<i}(\mathbf{y}_{i})|\\
&\leq \frac{1}{2}\mathbb{E}_{y \sim \pi_{\rm{data}}^{<i}(\mathbf{y}_{i})}\Big[\sum_{\mathbf{y}_{i}}|\mathbf{y}_{i} - \pi_{\theta}^{<i}(\mathbf{y}_{i})| \Big]=\mathbb{E}_{y \sim \pi_{\rm{data}}^{<i}(\mathbf{y}_{i})}\left[\mathrm{TV}(\mathbf{y}_{i},\pi_{\theta}^{<i}(\mathbf{y}_{i})) \right]\\
&=\mathbb{E}_{y \sim \pi_{\rm{data}}^{<i}(\mathbf{y}_{i})}\Big[1-\sum_{\mathbf{y}_{i}} \min (\mathbf{y}_{i},\pi_{\theta}^{<i}(\mathbf{y}_{i})) \Big]=- \pi_{\theta}^{<i}(\mathbf{y}_{i}).
\end{align}
Combing the above with Lemma~\ref{lemma:1}, we have:
\begin{align}
{\mathrm{TV}}(\pi_{\rm{data}}(\mathbf{y} \mid \mathbf{x}), \pi_{\theta}(\mathbf{y} \mid \mathbf{x}))&\leq \mathbb{E}_{\mathbf{y}\sim \pi_{\rm{data}}}\Bigg[ \sum_{i=1}^{|\mathbf{y}|} \mathrm{TV}(\pi_{\rm{data}}^{<i}(\mathbf{y}_{i}),\pi_{\theta}^{<i}(\mathbf{y}_{i})) \Bigg]\\
& = -\mathbb{E}_{\mathbf{y}\sim \pi_{\rm{data}}} \Bigg[\sum_{i=1}^{|\mathbf{y}|}\pi_\theta\left(\mathbf{y}_i \mid \mathbf{x}, \mathbf{y}_{<i}\right)\Bigg].
\end{align}
Recall that the objective of \method, without length-averaging, is:
\begin{align}
    \mathcal{L}_{\rm{SimPER}}({\theta}; \mathcal{D})&=-\exp \left( \log \pi_\theta(\mathbf{y}_{w} \mid \mathbf{x})\right)+\exp \left( \log \pi_\theta(\mathbf{y}_{l} \mid \mathbf{x})\right)\\
    &\geq -\pi_\theta(\mathbf{y}_{w} \mid \mathbf{x})\geq {\mathrm{TV}}(\pi_{\rm{chosen}}(\mathbf{y} \mid \mathbf{x}), \pi_{\theta}(\mathbf{y} \mid \mathbf{x})). \label{Eq:43}
\end{align}
Combining Equation~(\ref{Eq:43}) completes the whole proof.
\end{proof}


\section{Experimental Details}
\subsection{Downstream Task Evaluation}
\label{app:task-eval}
This section is for the detailed introduction to the evaluation of downstream tasks.
For models fine-tuned on UltraFeedback Binarized, following previous works~\citep{rafailov2024direct,tunstall2023zephyr}, we evaluated them on HuggingFace Open LLM Leaderboard v1\footnote{\url{https://huggingface.co/spaces/open-llm-leaderboard-old/open_llm_leaderboard}} and v2\footnote{\url{https://huggingface.co/spaces/open-llm-leaderboard/open_llm_leaderboard}}~\citep{open-llm-leaderboard-v1,open-llm-leaderboard-v2}, encompassing ten tasks: 
MMUL-PRO~\citep{wang2024mmluprorobustchallengingmultitask}, IFEval~\citep{zhou2023instructionfollowingevaluationlargelanguage}, BBH~\citep{suzgun2022challengingbigbenchtaskschainofthought}, GPQA~\citep{rein2023gpqagraduatelevelgoogleproofqa}, MUSR~\citep{sprague2024musrtestinglimitschainofthought}, MATH~\citep{hendrycks2021measuringmathematicalproblemsolving}, GSM8k~\citep{cobbe2021training}, ARC~\citep{clark2018think}, 
TruthfulQA~\citep{lin-etal-2022-truthfulqa} and Winogrande~\citep{sakaguchi2021winogrande} provided by the Language Model Evaluation Harness library~\citep{eval-harness}.
In addition to the leaderboard, we also evaluated them on two instruction-following benchmarks AlpacaEval2~\citep{li2023alpacaeval} and MT-Bench~\citep{zheng2023judging}.
For models fine-tuned on Anthropic-HH for safety alignment, following previous works~\citep{rafailov2024direct}, we use GPT-4 for zero-shot pair-wise evaluation, which is shown to be consistent with human judgments.

% \textbf{Massive Multitask Language Understanding Professional}: MMUL-PRO is an enhanced version of the MMLU dataset~\citep{hendrycks2021measuring}, addressing previous shortcomings by increasing choice options in multiple-choice questions and refining question quality through expert review, making it more challenging and less prone to data contamination.

% \textbf{Instruction-Following Evaluation}: IFEval is a benchmark evaluating a model's ability to follow explicit instructions, emphasizing adherence to formatting over content generation.

% \textbf{Big Bench Hard}: BBH is a selection of 23 challenging tasks from the BigBench, focusing on areas like multistep arithmetic, algorithmic reasoning, language understanding, and world knowledge.

% \textbf{Graduate-Level Google-Proof Q\&A}: GPQA is a challenging benchmark composed of advanced questions developed by PhD-level experts across various fields like biology, physics, and chemistry.

% \textbf{Multistep Soft Reasoning}: MUSR is a challenging benchmark consisting of algorithmically generated complex scenarios requiring integrated reasoning and long-range context understanding ability.

% \textbf{MATH}: MATH is a benchmark consisting of high-school level competition problems gathered from multiple sources, consistently formatted with LaTeX for equations and Asymptote for diagrams.

 
% \textbf{GSM8K}: GSM8K is a benchmark consisting of diverse grade school math word problems designed to evaluate a model's capability in solving multi-step mathematical reasoning problems.


% \textbf{AI2 Reasoning Challenge}: ARC is a benchmark consists of grade-school science questions.

% \textbf{TruthfulQA}: TruthfulQA is a benchmark fore the truthfulness of a model in generating responses to questions, including 817 questions across 38 categories such as health, law, finance, and politics.
 
% \textbf{Winogrande}: Winogrande is a large-scale benchmark for commonsense reasoning, inspired by WSC benchmark~\citep{levesque2012winograd}, but adjusted to improve both the scale and the hardness.


% \textbf{AlpacaEval 2.0}: AlpacaEval 2.0 is a LLM-based automatic evaluator for instruction-following language models, which is fast, cheap, replicable, and validated against 20K human annotations.


% \textbf{MT Bench}: MT-bench is an evaluator featuring challenging, multi-turn, open-ended questions to assess chat assistants, which automates the evaluation process by using advanced LLMs as judges.

\textbf{Safety Evaluation}:
To evaluate the safety of the Anthropic HH trained model, the Anthropic HH test set is used as the evaluation dataset. The preferred responses within this test set serve as the baseline. GPT-4 is then employed to determine the winner, which is shown to be consistent with human judgments. We use \texttt{gpt-4-0314} for all our experiments. The specific prompts used for the GPT-4 evaluation are detailed below.

\begin{table}[h!]
    \small
    {\ttfamily
    \begin{tabularx}{\linewidth}{X}
    \toprule
    For the following dialogue history to a chatbot, which response is more helpful and harmless?
    \\
    \\
    Dialogue history: \\
    \{dialogue history\}
    \\
    \\
    Response 1:\\
     \{Response 1\}
    \\
    \\
    Response 2: \\
     \{Response 2\}
    \\
    \\
    FIRST provide a one-sentence comparison of the two responses and explain which you feel is more helpful and harmless. SECOND, on a new line, state only "1" or "2" to indicate which response is more helpful and harmless. Your response should use the format:\\
    Comparison: <one-sentence comparison and explanation>\\
    More helpful: <"1" or "2">\\
    \bottomrule
    \end{tabularx}
    }
    \caption{Prompt for GPT-4 evaluation for the safety alignment task on the Anthropic-HH dataset.   \{dialogue history\},     \{Response 1\} and  \{Response 2\} are placeholders.}
    \label{tab:gpt4_prompt_dialogue}
\end{table}



\subsection{Baseline Methods}
\label{app:baseline}
This section provides a detailed introduction to state-of-the-art baselines for preference fine-tuning, with an emphasis on the usage of hyperparameters in their loss functions.

\textbf{DPO} Direct Preference Optimization~\citep{rafailov2024direct} uses log-likelihood differences to implicitly represent the reward function, eliminating the need for explicit reward model like RLHF. 
DPO involves one tunable hyperparameter, $\beta$, which controls the deviation from the reference model. 
$$
\mathcal{L}_{\rm{DPO}}({\theta}; \mathcal{D})=\mathbb{E}_{(\mathbf{x}, \mathbf{y}_w, \mathbf{y}_l)\sim \mathcal{D}}\left[-\log \sigma(\beta \log \frac{\pi(\mathbf{y}_w \mid \mathbf{x})}{\pi_{\mathrm{ref}}(\mathbf{y}_w \mid \mathbf{x})}-\beta \log \frac{\pi(\mathbf{y}_l \mid \mathbf{x} )}{\pi_{\mathrm{ref}}(\mathbf{y}_l \mid \mathbf{x})})\right].
$$

\textbf{IPO} Identity Preference Optimization~\citep{azar2024general} minimizes a squared loss regression problem by defining an alternative reward function, avoiding unstable RL training. 
IPO involves one hyperparameter, $\beta$, to adjust the reward margin.
$$
\mathcal{L}_{\rm{IPO}}({\theta}; \mathcal{D})=\mathbb{E}_{(\mathbf{x}, \mathbf{y}_w, \mathbf{y}_l)\sim \mathcal{D}}\left[\left (\log \frac{\pi_\theta(\mathbf{y}_w \mid \mathbf{x})\pi_{\mathrm{ref}}(\mathbf{y}_l \mid \mathbf{x})}{\pi_{\theta}(\mathbf{y}_l \mid \mathbf{x} )\pi_{\mathrm{ref}}(\mathbf{y}_w \mid \mathbf{x})}-\frac{1}{2\beta}\right)^{2}\right].
$$

\textbf{CPO} Contrastive Preference Optimization~\citep{xucontrastive} uses log-likelihood as the reward and is trained alongside a Supervised Fine-Tuning (SFT) objective. 
CPO involves two hyperparameters: $\beta$, which scales the log probabilities, and $\lambda$, which weights the SFT component.
$$
\mathcal{L}_{\rm{CPO}}({\theta}; \mathcal{D}) = -\log \sigma\left(\beta \log \pi_\theta(\mathbf{y}_w\mid \mathbf{x}) - \beta \log \pi_\theta(\mathbf{y}_l\mid \mathbf{x}) \right) - \lambda \log \pi_\theta (\mathbf{y}_w\mid \mathbf{x}).
$$


\textbf{SLiC} Sequence Likelihood Calibration~\citep{zhao2023slic} directly uses log-likelihood and includes a SFT objective. 
SLiC involves two hyperparameters: $\delta$, which sets the margin for the ranking loss, and $\lambda$, which weights the SFT component.
$$
\mathcal{L}_{\text{SLiC}}(\theta; \mathcal{D}) = \max\left(0, \delta - \log \pi_\theta(\mathbf{y}_w\mid \mathbf{x}) + \log \pi_\theta(\mathbf{y}_l\mid \mathbf{x})\right) - \lambda \log \pi_\theta (\mathbf{y}_w \mid \mathbf{x}).
$$


\textbf{SimPO} Simple Preference Optimization~\citep{meng2024simpo} eliminates the need for a reference model and optimizes a length-regularized probability of response pairs. 
SimPO involves two hyperparameters: $\beta$ to scale the log probabilities and $\gamma$ to adjust the reward margin. 
$$
\mathcal{L}_{\rm{SimPO}}({\theta}; \mathcal{D})=\mathbb{E}_{(\mathbf{x}, \mathbf{y}_w, \mathbf{y}_l)\sim \mathcal{D}}\left[-\log \sigma(\beta \log {\pi_{\theta}(\mathbf{y}_w \mid \mathbf{x})}-\beta \log {\pi_{\theta}(\mathbf{y}_l \mid \mathbf{x} )})-\gamma \right].
$$


\textbf{KTO} Kahneman-Tversky Optimization~\citep{ethayarajh2024kto} learns from non-paired preference data. 
KTO involves three hyperparameters: $\beta$, which controls the deviation from the reference model; $\lambda_w$ and $\lambda_l$, which weight the preference components for winning and losing responses, respectively. 
\begin{equation}
\begin{split}
\mathcal{L}_{\rm{KTO}}({\theta}; \mathcal{D}) &= -\lambda_w \,\sigma \left( \beta \log \frac{\pi_\theta(\mathbf{y}_w\mid \mathbf{x})}{\pi_{\text{ref}}(\mathbf{y}_w\mid \mathbf{x})} - z_{\text{ref}} \right) +  \lambda_l \,\sigma \left( z_{\text{ref}} - \beta \log \frac{\pi_\theta(\mathbf{y}_l\mid \mathbf{x})}{\pi_{\text{ref}}(\mathbf{y}_l\mid \mathbf{x})} \right), \\
\text{where} \quad z_{\text{ref}} &= \mathbb{E}_{(x, y) \sim \mathcal{D}} \left[\beta \;\mathrm{KL}\left( \pi_\theta(\mathbf{y}\mid \mathbf{x}) || \pi_{\text{ref}}(\mathbf{y}\mid \mathbf{x}) \right)  \right].
\end{split}
\end{equation}


% \begin{table}[h!]
% \caption{Comparison of state-of-the-art alignment approaches with our \method in terms of hyperparameters, the number of hyperparameters, and the elimination of the reference model.}
% \centering
% \vspace{-0.5em}
% \adjustbox{max width=\textwidth}{
% \begin{tabular}{cccc}
%    \toprule[1.0pt]
% \textbf{Approach}    &  \textbf{ Hyperparameters} & \textbf{\#Hyperparameters}   & \textbf{w/o Reference Model}   \\
% \midrule
% DPO & $\beta$ & $\mathbf{1}$   &  \xmark      \\
% IPO  & $\beta$ & $\mathbf{1}$  &  \xmark    \\
% CPO  & $\lambda, \beta$  & $\mathbf{2}$ & \cmark    \\
% SLiC  & $\delta, \lambda$ & $\mathbf{2}$  &  \cmark    \\
% SimPO  & $\gamma, \beta$ & $\mathbf{2}$  &  \cmark    \\
% KTO & $\lambda_l, \lambda_w, \beta$ & $\mathbf{3}$  &  \xmark   \\
% \midrule
% \method & - & $\mathbf{0}$ &   \cmark   \\
% \toprule[1.0pt]
% \end{tabular}}\label{table:compare_full}
% \end{table}


\subsection{Implementation Details}

\paragraph{Training Hyperparameters.}
For general hyperparameters, we adhered strictly to the settings used in SimPO. 
We applied the following hyperparameters: For the SFT stage, we use a learning rate of $2 \times 10^{-5}$. For both the SFT and the preference optimization stages, we use a batch size of 128, a maximum sequence length of 2048, and a cosine learning rate schedule with 10\% warmup steps for one epoch, all through the Adam optimizer~\citep{kingma2014adam}.
We maintain these settings consistently to ensure uniformity and comparability across experiments.

For method-specific hyperparameters, we also followed the search strategy from SimPO, noting that our method does not require any additional hyperparameters beyond the learning rate. Specifically, for different baseline methods, each with its own set of additional hyperparameters, the search strategy is in \cref{table_baseline_hyper}. 
Each method is individually search for the learning rates in $[3e-7, 5e-7, 6e-7, 1e-6]$.  

\begin{table}[h]
\small
\centering
\caption{Search ranges for hyperparameters across baseline methods}
\begin{tabular}{cc}
\toprule
\textbf{Method} & \textbf{Hyperparameter Search Range} \\
\midrule
DPO & $\beta \in [0.01, 0.05, 0.1]$ \\
\midrule
IPO & $\tau \in [0.01, 0.1, 0.5, 1.0]$ \\
\midrule
\multirow{2}{*}{CPO} 
& $\lambda = 1.0$ \\
& $\beta \in [0.01, 0.05, 0.1]$ \\
\midrule
\multirow{2}{*}{SLiC} 
& $\lambda \in [0.1, 0.5, 1.0, 10.0]$ \\
& $\delta \in [0.1, 0.5, 1.0, 2.0]$ \\
\midrule
\multirow{2}{*}{KTO} 
& $\lambda_l = \lambda_w = 1.0$ \\
& $\beta \in [0.01, 0.05, 0.1]$ \\
\midrule
\multirow{2}{*}{SimPO} 
& $\beta \in [2.0, 2.5]$ \\
& $\gamma \in [0.3, 0.5, 1.0, 1.2, 1.4, 1.6]$ \\
\bottomrule
\end{tabular}
\label{table_baseline_hyper}
\end{table}


\paragraph{Evaluation Hyperparameters.}
The hyperparameters for evaluation on AlpacaEval 2 adhere to SimPO, with a temperature of 0.7 for the Mistral-Base setting, a temperature of 0.5 for the Mistral-Instruct setting, and a temperature of 0.9 for both Llama3 settings. The hyperparameters for evaluation on MT-Bench are set according to the official decoding configuration, which specifies distinct sampling temperatures for various categories.

\paragraph{Computation Resources.} 
All the training experiments in this paper were conducted on 4 $\times$ NVIDIA A100 (80G) GPUs with 128 batchsize, based on the alignment-handbook repo.\footnote{\url{https://github.com/huggingface/alignment-handbook}}


% \section{Additional Experimental Results}


% \begin{figure}[t!]
% \centering 
% \includegraphics[width=1\textwidth]{pic/rewards-appendix.pdf}
% \caption{The training dynamics during training of \method and \texttt{SimPO} with  different hyperparameters on the 4 different base models. We can observe  that \method exhibits the least decline in chosen rewards, while still achieving the most significant increase in reward margins, compared to \texttt{SimPO} across various hyperparameters.}
% \vskip -1em
% \label{fig:app:rewards} 
% \end{figure}

\section{Case Study}
\label{app:evaluation}
% \textcolor{red}{Warning: The following case studies contain example data that may be offensive or harmful.}

In this section, we present examples of comparisons between \method  and the baseline \texttt{SimPO}. 
% See~\cref{tab:case-1} for dialogue examples, and~\ref{tab:case-2} for summarization examples.





\begin{table}[!h]
    \centering
    \ttfamily
    \small
        \begin{tabular}{lp{11.5cm}}
        \toprule[1pt]
        \bf{Prompt} & \texttt{Does the 5 second rule actually exist?}\\
        \midrule
        \bf{SimPER} & 
\texttt{The "5 second rule" is an urban legend that claims that if food falls on the floor, it is still safe to eat as long as you pick it up within 5 seconds. However, this rule is not a scientifically proven fact, and there is no evidence to support it. Here are some reasons why the 5-second rule is not a reliable method for determining food safety:}\newline


\textbf{1. Bacteria and viruses can transfer quickly}: Bacteria and viruses can transfer from surfaces to food within seconds. The amount of time that passes between a food item's contact with a contaminated surface and its being picked up can affect the likelihood of bacterial or viral transmission. \newline

\textbf{2. Contamination levels on floors and surfaces}: The level of contamination on floors and surfaces can vary widely depending on the environment, the cleanliness of the area, and the presence of moisture. Even if a surface is dry, bacteria and viruses can still be present.\newline

\textbf{3. Personal hygiene}: The cleanliness of your hands, the surface you place the food on, and the utensils you use to pick up the food all play a role in food safety. Picking up food within 5 seconds does not ensure that it is clean enough to eat.\newline

\textbf{4. Temperature}: The temperature of food can also affect the risk of bacterial or viral contamination. Food that has been at room temperature for a prolonged time, even if picked up quickly, may still harbor dangerous bacteria.\newline

Therefore, it is best to follow proper food handling and preparation practices, such as washing hands before and after handling food, using clean utensils and surfaces, and cooking food to the appropriate temperatures to ensure its safety. \\
        \midrule
        \bf{SimPO} & 
The "5-second rule" refers to a popular myth or belief that suggests touching food on the ground for less than five seconds makes it safe to consume without contamination. However, this rule does not exist scientifically.
According to food safety guidelines, bacteria can transfer from contaminated surfaces to food within seconds, regardless of the elapsed time. The duration of contact between the food and contaminated surfaces plays a crucial role in determining the likelihood of bacterial transfer, but it's not limited to five seconds.
Cross-contamination occurs when bacteria from unwashed hands, dirty utensils, or contaminated surfaces come into contact with food, leading to potential foodborne illnesses. Therefore, adhering to proper hygiene practices, such as washing hands, cleaning surfaces, and cooking food thoroughly, remains essential to prevent food contamination.
In summary, there is no scientific basis for the "5-second rule," and relying solely on this belief could increase the risk of consuming contaminated food."
        \\
        % \midrule
        % \textbf{Judgment} & \\
        \bottomrule[1pt]
    \end{tabular}
    \caption{An AlpacaEval 2 generation from the Llama-3-8B-Base model trained with \texttt{SimPER} and \texttt{SimPO}. Alpaca-Eval-GPT4-Turbo prefers \method over \texttt{SimPO}. Compared to \texttt{SimPO}, the generation by \texttt{SimPER} is better structured, with hierarchical discussions that make it more helpful. 
    }
    \label{tab:case-1}
\end{table}



\begin{table}[!h]
    \centering
        \ttfamily
        \small
    \begin{tabular}{lp{11.5cm}}
        \toprule[1pt]
        \bf{Prompt} & How do I make escargot? \\
        \midrule
        \bf{SimPER} &  
Here's a general recipe for making escargot: \newline

\textbf{Ingredients}:\newline
- 1 pound fresh snails (escargots)\newline
- 1/4 cup butter, at room temperature\newline
- 1 clove garlic, minced\newline
- 2 tablespoons finely chopped parsley\newline
- 2 tablespoons finely chopped shallots\newline
- 2 tablespoons dry white wine\newline
- 2 tablespoons lemon juice\newline
- 1/4 teaspoon ground black pepper\newline
- 1/4 cup breadcrumbs\newline
- 1/4 cup grated Parmesan cheese\newline
- Salt, to taste\newline
- 1 tablespoon olive oil\newline

\textbf{Equipment}:\newline
- Snail shells\newline
- Foil\newline
- A heavy-bottomed skillet\newline
- A wooden spoon\newline
- A baking dish\newline
- A fine-mesh strainer\newline
- A small saucepan\newline
- A mixing bowl\newline
- Parchment paper\newline

\textbf{Instructions}:\newline
1. Preheat the oven to 375F (190C).

2. Scrub the snails under cool running water to remove any dirt or debris. Discard any snails that are not alive. Soak the snails in salted water for at least 30 minutes, changing the water every 10 minutes, to ensure they are thoroughly cleaned.

3. In a small saucepan, melt the butter over low heat. Add the minced garlic, parsley, and shallots, and cook for 1-2 minutes until fragrant.

4. Remove the garlic mixture from the heat and stir in the white wine, lemon juice, and black pepper. Season with salt to taste.

5. In a mixing bowl, combine the breadcrumbs and grated Parmesan cheese.

6. Place the snail shells on a sheet of foil and arrange them in a baking dish. Spoon the garlic butter mixture into the shells, filling each one about 3\/4 full.

7. Sprinkle the breadcrumb mixture over the top of each snail.

8. Cover the snail shells with foil.

9. Bake in the preheated oven for 12-15 minutes or until the breadcrumbs are golden brown and the snails have cooked through.

10. Remove the dish from the oven, carefully remove the foil, and place a small piece of parchment paper over the top of each shell.

11. Return the dish to the oven and continue baking for 1-2 minutes to crisp up the breadcrumbs.

12. Remove the dish from the oven, and let the escargots cool for a few minutes.

13. In a heavy-bottomed skillet, heat the olive oil over medium-high heat. Strain the garlic butter mixture through a fine-mesh strainer into the skillet, and cook for 1-2 minutes until slightly reduced and thickened.

14. To serve, spoon the garlic butter onto small plates, arrange the escargots on the plates, and drizzle the remaining garlic butter over the top.

Note: Escargot shells are available for purchase at specialty food stores or online. If you don't have escargot shells, you can use mushroom caps or even small ramekins.

Enjoy your delicious escargot! \\
\midrule
\end{tabular}
\end{table}
\begin{table}[!h]
    \centering
        \ttfamily
        \small
    \begin{tabular}{lp{11.5cm}}
        \midrule
        \bf{SimPO} &
Here's a general recipe for making escargot:\newline

\textbf{Ingredients}:\newline
- 1 pound fresh or canned snails (escargots), cleaned and cooked\newline
- 1/2 cup unsalted butter, softened\newline
- 2 cloves garlic, minced\newline
- 1 tablespoon finely chopped fresh parsley\newline
- 1 tablespoon finely chopped fresh thyme\newline
- 1/4 teaspoon red pepper flakes (optional)\newline
- Salt and freshly ground black pepper, to taste\newline
- 24-30 small, hollowed-out garlic cloves or escargot shells\newline

\textbf{Instructions}:

1. Preheat your oven to 450F (230C).

2. Prepare the garlic cloves or escargot shells by gently scraping out the pulp with knife, leaving a thin layer intact. Rinse them under running water and pat dry.

3. In a mixing bowl, combine softened butter, minced garlic, chopped parsley, thyme, red pepper flakes (if using), salt, and black pepper. Mix well until blended.

4. Spoon a generous amount of seasoned butter into each garlic clove or escargot shell, filling them almost to the top.

5. Arrange the filled shells in a baking dish, preferably one with individual wells or indentations, spaced evenly apart.

6. Place the cooked snails into each shell, making sure they fit snugly.

7. Bake the escargot for 8-10 minutes, or until the butter has melted and bubbled around the edges, and the snails are heated through.

8. Remove the dish from the oven and serve immediately, accompanied by crusty bread, toasted baguette slices, or crackers.

9. To eat, use a small fork or escargot spoon to extract the snails from their shells, dip them into the remaining garlic butter, and enjoy!

Note: Always follow proper food handling and preparation guidelines when working with live or cooked snails, especially if using fresh ones. Ensure they are properly cleaned, cooked, and stored according to recommended safety practices. \\
\bottomrule[1pt]
    \end{tabular}
    % \vspace{4mm}
    \caption{An AlpacaEval 2 generation from the Mistral-7B-Base model trained with \texttt{SimPER} and \texttt{SimPO}. Alpaca-Eval-GPT4-Turbo prefers \method over \texttt{SimPO}. Compared to \texttt{SimPO}, the generation by \texttt{SimPER} is better structured, with hierarchical discussions that make it more helpful.}
    \label{tab:case-2}
\end{table}

\begin{table}[!h]
    \centering
        \ttfamily
        \small
    \begin{tabular}{lp{11.5cm}}
        \toprule[1pt]
        \bf{Prompt} & Mick pays his teacher \$800 for 40 lessons worth 2 hours each. If this will be all he is going to pay for his lessons, how much did he receive?\\
        \midrule
        \bf{SimPER} & Mick did not receive any money. The \$800 he paid to his teacher is the total cost for the 40 lessons, which is 40 x 2 hours = 80 total hours of lessons. Since he only paid for the lessons, he will not receive any money back.
        
        Confidence: 95\% \\
        \midrule
        \bf{SimPO} & Mick received \$800.
        
        Confidence: 95\%\\
        \bottomrule[1pt]
    \end{tabular}
    \caption{An AlpacaEval 2 generation from the Llama-3-8B-Base model trained with \texttt{SimPER} and \texttt{SimPO}. Alpaca-Eval-GPT4-Turbo prefers \method over \texttt{SimPO}. Compared to \texttt{SimPO}, \texttt{SimPER}'s output provides clearer step-by-step reasoning.}
    \label{tab:case-3}
\end{table}




% \section{You \emph{can} have an appendix here.}

% You can have as much text here as you want. The main body must be at most $8$ pages long.
% For the final version, one more page can be added.
% If you want, you can use an appendix like this one.  

% The $\mathtt{\backslash onecolumn}$ command above can be kept in place if you prefer a one-column appendix, or can be removed if you prefer a two-column appendix.  Apart from this possible change, the style (font size, spacing, margins, page numbering, etc.) should be kept the same as the main body.
%%%%%%%%%%%%%%%%%%%%%%%%%%%%%%%%%%%%%%%%%%%%%%%%%%%%%%%%%%%%%%%%%%%%%%%%%%%%%%%
%%%%%%%%%%%%%%%%%%%%%%%%%%%%%%%%%%%%%%%%%%%%%%%%%%%%%%%%%%%%%%%%%%%%%%%%%%%%%%%


\end{document}


% This document was modified from the file originally made available by
% Pat Langley and Andrea Danyluk for ICML-2K. This version was created
% by Iain Murray in 2018, and modified by Alexandre Bouchard in
% 2019 and 2021 and by Csaba Szepesvari, Gang Niu and Sivan Sabato in 2022.
% Modified again in 2023 and 2024 by Sivan Sabato and Jonathan Scarlett.
% Previous contributors include Dan Roy, Lise Getoor and Tobias
% Scheffer, which was slightly modified from the 2010 version by
% Thorsten Joachims & Johannes Fuernkranz, slightly modified from the
% 2009 version by Kiri Wagstaff and Sam Roweis's 2008 version, which is
% slightly modified from Prasad Tadepalli's 2007 version which is a
% lightly changed version of the previous year's version by Andrew
% Moore, which was in turn edited from those of Kristian Kersting and
% Codrina Lauth. Alex Smola contributed to the algorithmic style files.
