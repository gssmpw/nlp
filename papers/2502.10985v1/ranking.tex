\subsection{Ranking performance of Elo}
\label{sec:ranking}

Besides prediction, ranking is another important aspect that users consider when utilizing rating algorithms. There are two types of ranking: (1) for general $P$, we can consider the pairwise ranking, i.e., for each pair $(i,j) \in [N] \times [N]$, there is a ranking between $i,j$ that is induced by $P_{ij}$.
(2) for transitive $P$, there exists a ground truth ranking $\pi$ induced by the transitivity. 
In this subsection, we will show that for pairwise ranking, the ranking performance is strongly correlated to prediction performance. Elo rating, achieves good performance of pairwise ranking in sparse regimes. However Elo should not be blindly trusted, since for the total ordering, Elo may not always give a consistent ranking, even in transitive datasets.

\paragraph{Good prediction gives good pairwise ranking}
Regarding the pairwise ranking, it is natural to conjecture that pairwise ranking performance is correlated with the prediction performance, and our synthetic experiments justify this claim. We consider the same setup as the previous synthetic experiments for prediction (Section \ref{sec:synthetic}), and we calculated the pairwise ranking consistency for each algorithm at each time step:
at time $t$, an algorithm can actually give an prediction $\hat{P}_{ij}$ for every pair $(i,j) \in [N] \times [N]$. We calculate the following quantity: $\tau:=\frac{1}{N(N-1)}\sum_{i\neq j} (\one [P_{ij}>0.5] \one [\hat{P}_{ij}<0.5] + \one [P_{ij}<0.5] \one [\hat{P}_{ij}>0.5])$. Lower the value, more consistent the pairwise ranking. We plot $\tau$ against $t/N$ for Elo, Elo2k and Pairwise in Figure \ref{fig:pairwise-rk-N=100}. e can see that the ranking performance is strongly correlated with the prediction performance. To be specific, similar to the prediction accuracy, in most sparse regimes, Elo performs well in pairwise ranking. However in denser regimes, algorithms based on more complex models, such as Elo2k, may show advantage on pairwise ranking.

% \begin{figure}[t]
%     \centering
%     \includegraphics[width=0.8\columnwidth]{figures/100-pairwise_rk-new-big.png}
%     \caption{\textbf{Pairwise ranking is consistent with prediction.} Elo performs the best for small $t$, while Elo2k performs better for larger $t$.}
%     \label{fig:pairwise-rk-N=100}
% \end{figure}

% \begin{figure}[t]
%     \centering
%     \includegraphics[width=\columnwidth]{figures/100-pairwise_rk.png}

%     \caption{\textbf{Pairwise ranking consistent with prediction} Elo performs the best for small $t$,  Elo2k performs well for moderate $t$, Pairwise performs the best for large $t$.}
% \label{fig:pairwise-rk-N=100}
% \end{figure}


% \shange{change to N=100 or 500}?



\paragraph{Elo might not give consistent total ordering even for transitive models}

For transitive models, we can consider the total ordering induced by the transitivity. Elo rating, is still able to give a total ordering based on the score of each player. We will show that even though Elo can give a consistent total ordering under uniform matchmaking, it can not be blindly trusted as it may fail under arbitrary matchmaking.

% In this subsection, we consider two models of transitivity of increasing stringency: WST (Definition~\ref{def:wst}) and SST (Definition~\ref{def:sst}). Under both notions, there would be no cycles between players in terms of expected win rate, and there is an uncontested ground-truth ranking. 

From theoretical perspective, we consider the regime where $T$ goes to infinity. In this regime, by the no regret nature of OGD, one can see that the online Elo update will finally converge to the offline Elo solution $ \arg \min_{\theta \in \mathcal{K}} \frac1T \sum_{t=1}^{T} f_t(\theta) $. Also we can see that when $T \to \infty$, $\frac1T \sum_{t=1}^{T} f_t(\theta)$ converge to its population counterpart. Therefore we will consider $\theta^{\rm mle}$, the population MLE for BT model. We have the following result: 
\begin{theorem} \label{thm:Elo-winrate}
Under uniform matchmaking, $\theta^{\rm mle}$ gives identical ranking as $\overline{P}$, where 
$$\overline{P}[i] := \frac{\sum_{t=1}^T (\one[i_t = i]o_t + \one[j_t=i](1-o_t))}{\sum_{t=1}^T (\one[i_t = i]+\one [j_t=i])}$$ is the average win rate for player $i$.
%    Offline Elo recovers the ranking by average win rate under uniform matchmaking.
\end{theorem}
the formal statement and proof is deferred to Appendix \ref{proof:thm:Elo-winrate}. Notice that under SST models, the ground truth ranking is identical to the ranking given by average win rate. Therefore this theorem shows that under SST model, Elo recovers the true ranking when $T$ goes to infinity, under uniform matchmaking. 

% However, when the underlying model is WST, the ranking induced by average win rate may not be correct, therefore Elo is not guaranteed to be consistent. Moreover, when the matchmaking is \emph{arbitrary}, Elo score can produce inconsistent rankings for SST instances even when $\eta\to 0$ and $T\to \infty$. Also notice that, for actual regime where $T$ does not go to infinity, one can compute a confidence interval for each player's Elo score, and only consider ranking among players that have confidently separated Elo scores. We also show that even in this case Elo may not give consistent ranking, through a synthetic experiment. For these results, see Example~\ref{example:sst} in Appendix~\ref{sec:appendix-proof} for detail. Therefore, we conclude that Elo might not give the correct ranking for non-SST games or under non-uniform matchmaking. This suggest that although Elo can give good ranking results in many regimes, it can not be blindly trusted.

However, when the underlying model is WST, the ranking induced by average win rate may not be correct, therefore Elo is not guaranteed to be consistent. Moreover, when the matchmaking is \emph{arbitrary}, Elo score can produce inconsistent rankings for SST instances even when $\eta\to 0$ and $T\to \infty$. We also show through a synthetic experiment that even in the case where only ranking among players that have confidently separated Elo scores are considered, Elo still may not give consistent ranking. For these results, see Example~\ref{example:sst} in Appendix~\ref{sec:appendix-proof} for detail. This suggest that although Elo can give good ranking results in many regimes, it can not be blindly trusted.











