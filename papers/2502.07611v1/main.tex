\documentclass[10pt,conference]{IEEEtran}
\IEEEoverridecommandlockouts

\usepackage{amsmath,amsfonts, amssymb}
\usepackage{algorithm}
\usepackage{adjustbox}
\usepackage{lscape}
\usepackage{siunitx}
\usepackage[noend]{algpseudocode}
\usepackage{textcomp}
\usepackage{fancyvrb}
\usepackage{graphicx}
\usepackage{soul}
\usepackage[many]{tcolorbox}
\usepackage{setspace}
\usepackage{booktabs}
\usepackage[utf8]{inputenc}
\usepackage[T1]{fontenc}
\usepackage{rotating}
\usepackage{graphicx}
\usepackage{paralist}
\usepackage{tabularx}
\usepackage{multicol}
\usepackage{multirow}
\usepackage{pbox}
\usepackage{enumitem}
\usepackage{colortbl}
\usepackage{pifont}
\usepackage{xspace}
\usepackage{url}
\usepackage{tikz}
\usepackage{tabularx}
\usepackage{fontawesome}
\usepackage{lscape}
\usepackage{listings}
\usepackage{color}
\usepackage{showexpl}
\usepackage{anyfontsize}
\usepackage{comment}
\usepackage{soul}
\usepackage{multibib}
\usepackage{multirow}
\usepackage{xspace}
\usepackage{footnote}
\usepackage{tcolorbox}
\usepackage{booktabs}
\usepackage{balance}
\usepackage[normalem]{ulem}

\usepackage{hhline}

% MACROs for stylized Tcolorbox
\definecolor{main}{HTML}{5989cf}    % setting main color to be used
\definecolor{sub}{HTML}{cde4ff}     % setting sub color to be used



\tcbset{
	sharp corners,
	colback = white,
	before skip = 0.2cm,    % add extra space before the box
	after skip = 0.5cm      % add extra space after the box
}                           % setting global options for tcolorbox

\newtcolorbox{boxM}{
	fontupper = \color{white},
	rounded corners,
	arc = 6pt,
	colback = main!80,
	colframe = main,
	boxrule = 0pt,
	bottomrule = 4.5pt,
	enhanced,
	fuzzy shadow = {0pt}{-3pt}{-0.5pt}{0.5pt}{black!35}
}
%%%%%%%%%%%%%%%%%%%%%%%%%%%%%%%%

\newcolumntype{N}{>{\centering\arraybackslash}m{.85in}}
\def\BibTeX{{\rm B\kern-.05em{\sc i\kern-.025em b}\kern-.08em
    T\kern-.1667em\lower.7ex\hbox{E}\kern-.125emX}}

\newboolean{showcomments}

\setboolean{showcomments}{true}

\ifthenelse{\boolean{showcomments}}
{\newcommand{\nb}[2]{
		\fbox{\bfseries\sffamily\scriptsize#1}
		{\sf\small$\blacktriangleright$\textit{#2}$\blacktriangleleft$}
	}
	\newcommand{\cvsversion}{\emph{\scriptsize$-$Id: macro.tex,v 1.9 2005/12/09 22:38:33 xxx Exp \$}}
}
{\newcommand{\nb}[2]{}
	\newcommand{\cvsversion}{}
}

\makeatletter
\newcommand{\linebreakand}{%
	\end{@IEEEauthorhalign}
	\hfill\mbox{}\par
	\mbox{}\hfill\begin{@IEEEauthorhalign}
}
\makeatother

\newcommand{\cmark}{\ding{51}}%
\newcommand{\xmark}{\ding{55}}%
\newcommand{\ie}{\emph{i.e.,}\xspace}
\newcommand{\eg}{\emph{e.g.,}\xspace}
\newcommand{\etc}{etc.\xspace}
\newcommand{\etal}{\emph{et~al.}\xspace}
\newcommand{\secref}[1]{Section~\ref{#1}\xspace}
\newcommand{\figref}[1]{Fig.~\ref{#1}\xspace}
\newcommand{\listref}[1]{Listing~\ref{#1}\xspace}
\newcommand{\tabref}[1]{Table~\ref{#1}\xspace}
\newcommand{\TBD}[1]{\textcolor{red}{#1}\xspace}
\newcommand{\java}{\emph{Java}\xspace}
\newcommand{\tool}{\emph{SATDBailiff}\xspace}
\newcommand*\circled[1]{\tikz[baseline=(char.base)]{
		\node[shape=circle,fill,inner sep=0.8pt] (char) {\textcolor{white}{#1}};}}
\newcommand{\side}[1]{SIDE$_{{#1}}$\xspace}
\newcommand{\green}{green AI\xspace}


\definecolor{lightergray}{rgb}{0.9,0.9,0.9}
\newtcolorbox{resultbox}{colback=lightergray, arc=0.5mm, top=2mm, bottom=2mm, left=2mm, right=2mm}

\definecolor{arsenic}{rgb}{0.23, 0.27, 0.29}
\definecolor{darkgray}{rgb}{0.33, 0.33, 0.33}

\newcommand\TODO[1]{\textcolor{red}{#1}}
\newcommand\REMOVE[1]{\textcolor{red}{\st{#1}}}
\newcommand\ANTONIO[1]{\textcolor{blue}{\nb{ANTONIO}{#1}}}
\newcommand\MAX[1]{\textcolor{green}{\nb{MAX}{#1}}}
\newcommand\VITALE[1]{\textcolor{orange}{\nb{VITALE}{#1}}}
\newcommand\SIMONE[1]{\textcolor{red}{\nb{SIMONE}{#1}}}

\newcommand\rev[1]{\textcolor{black}{#1}}
\pagenumbering{arabic}
\pagestyle{empty}

\begin{document}

\title{Optimizing Datasets for Code Summarization:\\ Is Code-Comment Coherence Enough?}

\author{
\IEEEauthorblockN{Antonio Vitale\IEEEauthorrefmark{1}\IEEEauthorrefmark{3}, Antonio Mastropaolo\IEEEauthorrefmark{2}, Rocco Oliveto\IEEEauthorrefmark{3}, Massimiliano Di Penta\IEEEauthorrefmark{4}, and Simone Scalabrino\IEEEauthorrefmark{4}}
\IEEEauthorblockA{\IEEEauthorrefmark{1}Politecnico di Torino, Italy, antonio.vitale@polito.it}
\IEEEauthorblockA{\IEEEauthorrefmark{2}William \& Mary, USA, amastropaolo@wm.edu}
\IEEEauthorblockA{\IEEEauthorrefmark{3}University of Molise, Italy, \{rocco.oliveto, simone.scalabrino\}@unimol.it}
\IEEEauthorblockA{\IEEEauthorrefmark{4}University of Sannio, Italy, dipenta@unisannio.it}
}


\maketitle

\thispagestyle{empty}


\begin{abstract}
Automated code summarization is a long-standing goal for code comprehension. This task automatically generates documentation using a given method. Deep Learning (DL)-based approaches have been proven beneficial for various software engineering (SE) tasks, including this one. 
Most state-of-the-art datasets for code summarization are automatically mined from GitHub and, thus, might contain erroneous or sub-optimal examples. Previous work showed that using a simple rule-based approach for removing noisy instances allows for a tangible reduction of the training set size while not reducing the effectiveness of the trained models. Motivated by this finding, we conjecture that it is possible to further reduce the dataset size by removing instances that contain different issues.
In this paper, we explore the extent to which code-comment coherence, a specific quality attribute of code summaries, can be used to optimize code summarization datasets. Specifically, we hypothesize that removing incoherent code-comment pairs might positively impact the effectiveness of the models. To do this, we rely on SIDE, a recently introduced metric for code-summary coherence. We examine multiple selectivity levels of training instances from two state-of-the-art datasets (TL-CodeSum and Funcom) and evaluate the resulting models on three manually curated test sets. The results show that even halving the training set sizes does not significantly affect the model's ability to generate summaries. However, when comparing the most restrictive selection strategy with a simpler one that randomly selects the training instances, we observe that the resulting accuracy of the model also does not change.
This result suggests that (i) current datasets contain many irrelevant examples, and (ii) different quality attributes should be explored for optimizing code summarization datasets.
\end{abstract}

\begin{IEEEkeywords}
	Code Summarization, Data Quality, Code-Comment Coherence, Empirical Study
\end{IEEEkeywords}


\section{Introduction}
\label{sec:intro}
\section{Introduction}


\begin{figure}[t]
\centering
\includegraphics[width=0.6\columnwidth]{figures/evaluation_desiderata_V5.pdf}
\vspace{-0.5cm}
\caption{\systemName is a platform for conducting realistic evaluations of code LLMs, collecting human preferences of coding models with real users, real tasks, and in realistic environments, aimed at addressing the limitations of existing evaluations.
}
\label{fig:motivation}
\end{figure}

\begin{figure*}[t]
\centering
\includegraphics[width=\textwidth]{figures/system_design_v2.png}
\caption{We introduce \systemName, a VSCode extension to collect human preferences of code directly in a developer's IDE. \systemName enables developers to use code completions from various models. The system comprises a) the interface in the user's IDE which presents paired completions to users (left), b) a sampling strategy that picks model pairs to reduce latency (right, top), and c) a prompting scheme that allows diverse LLMs to perform code completions with high fidelity.
Users can select between the top completion (green box) using \texttt{tab} or the bottom completion (blue box) using \texttt{shift+tab}.}
\label{fig:overview}
\end{figure*}

As model capabilities improve, large language models (LLMs) are increasingly integrated into user environments and workflows.
For example, software developers code with AI in integrated developer environments (IDEs)~\citep{peng2023impact}, doctors rely on notes generated through ambient listening~\citep{oberst2024science}, and lawyers consider case evidence identified by electronic discovery systems~\citep{yang2024beyond}.
Increasing deployment of models in productivity tools demands evaluation that more closely reflects real-world circumstances~\citep{hutchinson2022evaluation, saxon2024benchmarks, kapoor2024ai}.
While newer benchmarks and live platforms incorporate human feedback to capture real-world usage, they almost exclusively focus on evaluating LLMs in chat conversations~\citep{zheng2023judging,dubois2023alpacafarm,chiang2024chatbot, kirk2024the}.
Model evaluation must move beyond chat-based interactions and into specialized user environments.



 

In this work, we focus on evaluating LLM-based coding assistants. 
Despite the popularity of these tools---millions of developers use Github Copilot~\citep{Copilot}---existing
evaluations of the coding capabilities of new models exhibit multiple limitations (Figure~\ref{fig:motivation}, bottom).
Traditional ML benchmarks evaluate LLM capabilities by measuring how well a model can complete static, interview-style coding tasks~\citep{chen2021evaluating,austin2021program,jain2024livecodebench, white2024livebench} and lack \emph{real users}. 
User studies recruit real users to evaluate the effectiveness of LLMs as coding assistants, but are often limited to simple programming tasks as opposed to \emph{real tasks}~\citep{vaithilingam2022expectation,ross2023programmer, mozannar2024realhumaneval}.
Recent efforts to collect human feedback such as Chatbot Arena~\citep{chiang2024chatbot} are still removed from a \emph{realistic environment}, resulting in users and data that deviate from typical software development processes.
We introduce \systemName to address these limitations (Figure~\ref{fig:motivation}, top), and we describe our three main contributions below.


\textbf{We deploy \systemName in-the-wild to collect human preferences on code.} 
\systemName is a Visual Studio Code extension, collecting preferences directly in a developer's IDE within their actual workflow (Figure~\ref{fig:overview}).
\systemName provides developers with code completions, akin to the type of support provided by Github Copilot~\citep{Copilot}. 
Over the past 3 months, \systemName has served over~\completions suggestions from 10 state-of-the-art LLMs, 
gathering \sampleCount~votes from \userCount~users.
To collect user preferences,
\systemName presents a novel interface that shows users paired code completions from two different LLMs, which are determined based on a sampling strategy that aims to 
mitigate latency while preserving coverage across model comparisons.
Additionally, we devise a prompting scheme that allows a diverse set of models to perform code completions with high fidelity.
See Section~\ref{sec:system} and Section~\ref{sec:deployment} for details about system design and deployment respectively.



\textbf{We construct a leaderboard of user preferences and find notable differences from existing static benchmarks and human preference leaderboards.}
In general, we observe that smaller models seem to overperform in static benchmarks compared to our leaderboard, while performance among larger models is mixed (Section~\ref{sec:leaderboard_calculation}).
We attribute these differences to the fact that \systemName is exposed to users and tasks that differ drastically from code evaluations in the past. 
Our data spans 103 programming languages and 24 natural languages as well as a variety of real-world applications and code structures, while static benchmarks tend to focus on a specific programming and natural language and task (e.g. coding competition problems).
Additionally, while all of \systemName interactions contain code contexts and the majority involve infilling tasks, a much smaller fraction of Chatbot Arena's coding tasks contain code context, with infilling tasks appearing even more rarely. 
We analyze our data in depth in Section~\ref{subsec:comparison}.



\textbf{We derive new insights into user preferences of code by analyzing \systemName's diverse and distinct data distribution.}
We compare user preferences across different stratifications of input data (e.g., common versus rare languages) and observe which affect observed preferences most (Section~\ref{sec:analysis}).
For example, while user preferences stay relatively consistent across various programming languages, they differ drastically between different task categories (e.g. frontend/backend versus algorithm design).
We also observe variations in user preference due to different features related to code structure 
(e.g., context length and completion patterns).
We open-source \systemName and release a curated subset of code contexts.
Altogether, our results highlight the necessity of model evaluation in realistic and domain-specific settings.






\section{Selection Strategies for Code Summarization}
\label{sec:selection_strategies}
% !TEX root = main.tex
In this section, we provide backgrounds about (i) a state-of-the-art strategy for repairing or removing poor instances from code summarization datasets (CAT), and (ii) the SIDE metric, which we use to streamline a data-centric, quality-aware instance filtering.

\subsection{Code-comment cleAning Tool}
CAT (\textbf{C}ode-comment cle\textbf{A}ning \textbf{T}ool) is an approach and tool by Shi \etal \cite{shi2022we} to detect and handle noisy instances given the pairs of \textit{<code, summary>} from code-summarization datasets. The development of CAT was preceded by a manual investigation involving 9 participants who examined 1,600 \textit{<code, summary>} pairs. This manual analysis aimed to define a taxonomy of noisy data categories.

The taxonomy features comments- and code-related noisy data, such as \textit{commented-out method} or \textit{empty function}. Based on such categories, Shi \etal defined a set of heuristic rules and implemented them in the CAT tool. CAT works with two possible strategies based on the issues found: On the one hand, it fixes instances with minor issues. For example, it removes block-level comments. On the other hand, it completely drops instances where no fix is possible, including, for example, getters and setters.

\subsection{Fine-Grained Filtering: SIDE}
\label{sec:side-aware}
SIDE (\textbf{S}ummary al\textbf{I}gnment to co\textbf{D}e s\textbf{E}mantics) is a novel quality-aware metric presented by Mastropaolo \etal \cite{mastropaolo2024evaluating}. SIDE addresses the shortcomings of traditional metrics such as BLEU \cite{papineni2002bleu}, ROUGE \cite{lin:tsbo2004}, and METEOR \cite{banerjee:acl2005} used in code summarization tasks. 

SIDE employs a contrastive learning approach to determine the accuracy with which a code summary documents the underlying code, explicitly focusing on Java methods. Contrastive learning aims to maximize the distance between the reference code and inappropriate comments while minimizing the distance to suitable comments. The model that implements SIDE, MPNet \cite{Song2020MPNetMA}, provides a continuous score ranging from -1 to 1. Scores closer to~-1 indicate poor alignment between the code summary and the actual code, whereas scores closer to~1 suggest a strong alignment. SIDE showed a high correlation with human evaluations of summary quality, outperforming established metrics like BLEU, ROUGE, and METEOR. 

The benefits of a quality-aware metric like SIDE extend beyond evaluating code summarization techniques, as Mastropaolo \etal \cite{mastropaolo2024evaluating} noted. SIDE can be valuable when distinguishing high-quality code documentation from subpar examples is essential. We conjecture that SIDE ensures that only the instances most likely to enhance the training procedure are used, enabling the model to converge faster without a drop in performance by filtering out low-quality elements.

Our study relies on SIDE to select $\langle code, summary \rangle$ pairs with high coherence. Specifically, given a training set $T$ and a threshold $t$, we select the training instances $\{ p_{i} \in T \mid \text{SIDE}(p_{i}) \ge t \}$. 


\section{Study Definition, Design and Planning}
\label{sec:study}
% !TEX root = main.tex
The \emph{goal} of this study is to empirically evaluate how code-comment coherence, through a quality-aware selection strategy grounded on SIDE, impacts the effectiveness and training efficiency of neural code summarization models.

More specifically, the study aims to address the following research questions:

\begin{itemize}[itemindent=0.25cm]
	\item[\textbf{RQ$_{0}$}:] \textit{How do code summarization datasets measure up in terms of code-comment coherence?}
	In this preliminary question, we assess the coherence of code-comment pairs of datasets commonly used in code summarization. As we aim to use a coherence-aware strategy to optimize training sets, first of all, we would like to see how the coherence is distributed.
	\item[\textbf{RQ$_{1}$}:] \textit{How does a coherence-aware strategy selection impact the performance of neural code summarization models?}
	In this research question, we investigate how a targeted selection of training data based on code-comment coherence impacts the performance of neural code summarization models.
	\item[\textbf{RQ$_{2}$}:] \textit{How does the coherence-aware strategy selection compare with a random baseline?}
	In this research question, we test our hypothesis that code-comment coherence is a quality attribute that can be used to select training instances.
\end{itemize}

\subsection{Context Selection}
\label{subsec:context_selection}
The \emph{context} of our study consists of datasets containing pairs of \java methods with the associated summaries. 
For fine-tuning the models, we consider the two most important datasets from the state of the art: TL-CodeSum \cite{hu2018summarizing}, and Funcom \cite{leclair2019neural}.

The \textit{TL-CodeSum} dataset \cite{hu2018summarizing} is specifically designed for the code summarization task. It consists of $\sim$87k instances $ \langle code, summary \rangle$ extracted from GitHub repositories created from 2015 to 2016, and having at least 20 stars. In detail, Hu \etal \cite{hu2018summarizing} extracted the first sentence---likely to describe the overall method functionality---from the doc of each pair.

Similarly to TL-CodeSum, the \textit{Funcom} dataset \cite{leclair2019neural} is also specifically designed for code summarization. \textit{Funcom} consists of over 2.1M $ \langle code, summary \rangle$ pairs collected from the Sourcerer repository. As for TL-CodeSum, LeClair \etal \cite{leclair2019neural} only consider methods with their javadoc, extracting the first sentence as corresponding \textit{summary}.

Shi \etal \cite{shi2022we} found many noisy instances and duplicates in the above-described datasets and cleaned them up using their heuristic-based dataset-cleaning approach. For this reason, we use the cleaned versions of \textit{TL-CodeSum} and \textit{Funcom} provided by Shi \etal \cite{shi2022we}. The cleaned \textit{TL-CodeSum} contains 53,597 training instances, while the cleaned \textit{Funcom} contains 1,184,438 training instances.

The above datasets are built automatically, and no manual check was performed, \ie there is no guarantee of their quality. For this reason, we use two additional, manually curated datasets to test the models. The first one is \textit{CoderEval} \cite{yu2024codereval}, which consists of 230 Python and 230 \java code generation problems collected from open-source, high-starred projects which include \textit{original} and \textit{human-labeled} docstrings that should act as prompt for Code Generation models to generate the corresponding \textit{code}. 
The instances have been subject to manual screening, for which the main criterion is the probability of appearing in real-development scenarios. We focus on the \java set of problems, inverting the input and the output \ie from $\langle docstring, code \rangle$ to $\langle code, docstring \rangle$. 
To align the format of the pairs format, we performed an additional manual analysis in which one of the authors checked all the triplets with a second author to confirm the analysis. 
We found that some of the \textit{docstring}(s) contained more than a sentence. Therefore, to make them consistent with the previous dataset format (\eg single sentence), we extracted the first sentence from each \textit{docstring}. Still, we found 12 occurrences in which the corresponding \textit{original docstring} does not describe the \textit{code} (\eg ``\texttt{{@inheritDoc}}'', ``\texttt{@param modelName model name of the entity}'', and similar). We also excluded \textit{docstring}: ``\texttt{Computes floor(\$log\_2 (n)\$) \$+ 1\$.}'' since it includes a formula not explained in natural language.
Again, to appropriately align the evaluation, we do not evaluate such instances, ending up with 218 \textit{original} instances.

The second manually-curated dataset we use is the one by Mastropaolo \etal \cite{mastropaolo2023robustness}. The dataset consists of 892 methods associated with their summary (\ie first sentence of the method documentation), collected from non-fork GitHub \java repositories with at least 300 commits, 50 contributors, and 25 stars. 
Such instances are in the form $\langle summary, code\rangle$ and, as for CoderEval \cite{yu2024codereval}, we inverted the input and the output \ie $\langle code, summary\rangle$. Mastropaolo \etal  analyzed such pairs to ensure their quality. We manually analyzed and cleaned them further (\eg ``\texttt{Adds an {@link CarrierService} to the {@linkCarrier}}'' into ``\texttt{Adds an CarrierService to the Carrier}''), as we had done for CoderEval. No instances were removed during such a manual analysis.

We remove the instances from the test sets which appear in the training sets of \textit{TL-CodeSum} and \textit{Funcom}. As a result, we remove ten instances from CoderEval, which are present only in the \textit{TL-CodeSum} training set.

\subsection{Study Methodology}
\label{subsec:exp_proc}
To answer RQ$_{0}$, we use SIDE to compute the degree to which the summaries of the studied datasets document their corresponding code. We did this for each instance of the training sets included in \textit{TL-CodeSum} and \textit{Funcom}. To understand the coherence of the training sets, we analyze the average and the distributions of the SIDE scores of the instances.\\

\addtolength{\extrarowheight}{\belowrulesep}
\aboverulesep=0pt
\belowrulesep=0pt
\begin{table}[t]
	\centering
	\caption{Different selections for \textit{TL-CodeSum} and \textit{Funcom} training sets.}
	\label{tab:dataset_w_strategies}
	\resizebox{0.6\columnwidth}{!}{%
		\begin{tabular}{lrrr}
			\toprule
			\cellcolor{black}\textcolor{white}{\textbf{Selection}} &  \cellcolor{black}\textcolor{white}{\textbf{TL-CodeSum}} & \cellcolor{black}\textcolor{white}{\textbf{Funcom}} \\
			\midrule
			Full & 53,597 & 1,184,438 \\
			\midrule
			\side{0.5} & 50,073 & 1,080,649 \\
			\side{0.6} & 48,146 & 1,031,647 \\
			\side{0.7} & 44,853 & 952,265 \\
			\side{0.8} & 38,733 & 813,998 \\
			\side{0.9} & 26,258 & 540,170 \\
			\bottomrule
	\end{tabular}}
\end{table}

To answer RQ$_{1}$, we use the SIDE-based filter we define in \secref{sec:selection_strategies}. We use five threshold values, \ie 0.5, 0.6, 0.7, 0.8, and 0.9. We do not use thresholds lower than 0.5 because they would result in negligible dataset reductions (lower than 10\% for both), as we will observe in the results of RQ$_{0}$.
We report information about the different datasets in \tabref{tab:dataset_w_strategies}. 
We apply each filter on the training sets of \textit{TL-CodeSum} and \textit{Funcom}. Such filtering leads to the definition of five new versions of both datasets.

We fine-tune a pre-trained Transformer-based model for each dataset version, \ie both the base one and its six filtered versions, producing 12 fine-tuned models.
We choose to leverage the pre-trained \emph{CodeT5+} \cite{wang2023codet5+} since it has been largely used for code-related tasks \cite{ahmed2024automatic,phan2024repohyper,yang2024important} and, more important, in the code summarization approaches described above. This model is built on the backbone of the well-known T5 model by Raffel \etal \cite{raffel2020exploring}, yet it benefits from specific enhancements tailored for code understanding and generation tasks. During the pre-training phase, \emph{CodeT5+} is first trained on unimodal data, which includes code and comments, employing a combination of pre-training objectives such as span-denoising \cite{raffel2020exploring} and Causal Language Modeling \cite{soltan2022alexatm,tay2022ul2}. Then, it is pre-trained on bi-modal data where pre-training objectives such as text-code contrastive learning, text-code matching, and text-code causal language modeling are employed. It comes with different variants: (i) \emph{CodeT5+} 220M, (ii) \emph{CodeT5+} 770M, (iii) \emph{CodeT5+} 2B, (iv) \emph{CodeT5+} 6B, and (vi) \emph{CodeT5+} 16B.
Since our experimental design would require training, validating and testing 12 models, we decided to fine-tune the \emph{CodeT5+} variant featuring 220M trainable parameters.
This choice aligns with the goal of our investigation: Rather than proposing a new code summarization technique, we aim to use a model that offers a favorable balance between size and training time while still allowing us to observe the relevant phenomenon (if present).

Considering the extensive array of our experiments, we fine-tune for 20 epochs using a batch size of 16. Additionally, we restrict the input length to 512 tokens and the output to 128 tokens, consistent with previous studies leveraging the two datasets we used \cite{mastropaolo2022using,zhou2022automatic,tufano2023automating}. In addition, we conduct the fine-tuning using the standard hyperparameters for \emph{CodeT5+}, which include the AdamW optimizer \cite{loshchilov2017decoupled} and a learning rate of 2e-5, which is the one recommended for (Code)T5 and also used in works leveraging such models \cite{mastropaolo2023towards,ciniselli2024generalizability,mastropaolo2024vul}.

To prevent overfitting, we employ early stopping \cite{prechelt2002early}. After each epoch, we assess the performance of the models by computing the number of correct predictions on the validation set. 
In line with similar research \cite{mastropaolo2023towards,ciniselli2024generalizability}, we implement early stopping with patience of 5 epochs and a delta of 0.01. This means that training will stop if the model's performance does not improve by at least 0.01 for five consecutive epochs. We then select the best-performing checkpoint before early stopping.
We fine-tune a \emph{CodeT5+} model for each training set derived from the selection strategy \ie 12 (2 datasets $\times$ six variants).

After training the models, we assess their performance on the test set dataset that, as previously explained, are the \textit{CoderEval} \cite{yu2024codereval}, and the one from Mastropaolo \etal \cite{mastropaolo2023robustness} which we refer to as the \textit{golden sets}.

In the inference phase, we employ a beam search decoding strategy. In detail, with $k \in \{1, 3, 5\}$, we allow each model to generate the $k$ most probable candidate \textit{summaries} for the given \textit{code}.
To evaluate the generated summaries of each model, we compute the following metrics: BLEU \cite{papineni2002bleu}, METEOR \cite{banerjee:acl2005}, and ROUGE-L \cite{lin2004rouge}.
\textbf{BLEU} is a metric that expresses, within a range from 0 to 1, the similarity between a generated text (candidate) and the target one (oracle). It computes the percentage of $n$-grams of the generated text that appear in the target, where $n \in \{1, 2, 3, 4\}$. 
\textbf{METEOR} is computed as the harmonic mean of unigram precision and recall, with the latter weighted higher than the former. It ranges from 0 to 1. 
\textbf{ROUGE-L} is computed as the length of the longest common subsequence (LCS) between the generated text and the target one and measures the recall by considering the proportion of the LCS relative to the length of the target text.
We do not use SIDE \cite{mastropaolo2024evaluating} as it was employed for selecting training instances and could therefore be unnaturally biased in favor of models trained on filtered datasets.
Also, we do not compute the percentage of exact matches for three reasons. First, exact matches might underestimate the actual performances of the model. Indeed, an exact match implies a correct summary, but many alternative summaries might be as correct (or even more correct, in theory) as the ones in the ground truth for the very nature of this task. Second (also related to the previous point), \textit{CoderEval} \cite{yu2024codereval} provides two summaries for each coding instance, namely \textit{original} (\ie the docstring collected from the original source code), and \textit{human} (\ie the docstring written from scratch by developers during the benchmark creation \cite{yu2024codereval}). The model could have correctly generated only one of them, which are, by definition, both correct alternatives, thus leading to inconsistent results. Third, the dataset provided by Mastropaolo \etal~\cite{mastropaolo2023robustness} includes three different yet semantically equivalent code summaries for each \java method. As previously noted, each of these alternative descriptions is a valid candidate summary.

\begin{figure}[t]
	\centering
	\includegraphics[width=0.65\linewidth]{distributions-plot.pdf}
	\caption{Distribution of SIDE scores for \textit{TL-CodeSum} and \textit{Funcom} training instances.}
	\label{fig:rq0_training_distribution}
\end{figure}

\begin{table*}[t]
	\centering
	\caption{Performance metrics on Top-1 predictions for CoderEval.}
	\label{tab:performance_metrics_codereval}
	\resizebox{\linewidth}{!}{%
		\begin{tabular}{c|l|r|r|rrr|rrr|rrr}
			\toprule
			\rowcolor{black}
			&  &  &  & \multicolumn{3}{c}{\textcolor{white}{\textbf{CoderEval-Original \cite{yu2024codereval}}}} & \multicolumn{3}{c}{\textcolor{white}{\textbf{CoderEval-Human \cite{yu2024codereval}}}} & \multicolumn{3}{c}{\textcolor{white}{\textbf{Mastropaolo \etal \cite{mastropaolo2023robustness}}}} \\
			\rowcolor{gray!20}
			\textbf{Dataset} & \textbf{Selection} & \textbf{\#Tokens} & \textbf{(\%) Saving} & \textbf{BLEU-4} & \textbf{METEOR} & \textbf{ROUGE} & \textbf{BLEU-4} & \textbf{METEOR} & \textbf{ROUGE} & \textbf{BLEU-4} & \textbf{METEOR} & \textbf{ROUGE} \\
			\midrule
			\multirow{6}{*}{\textit{TL-CodeSum \cite{hu2018summarizing}}}
			
			& \emph{Full}      & \cellcolor[HTML]{656565}\color[HTML]{FFFFFF} $\uparrow$ 7.9M &  \cellcolor[HTML]{a3070c}\color[HTML]{FFFFFF} --   & 11.72 & 17.84 & 35.01 & 6.41 & 14.28 & 30.51 & 6.37 & 13.04 & 27.09 \\
			\cmidrule(r){2-13}
			& \side{0.5} & 7.4M & 7\%  & 12.41 & 17.73 & 35.41 & 6.32 & 14.28 & 31.08 & 6.36 & 12.93 & 27.47\\
			& \side{0.6} & 7.1M & 10\% & 13.22 & 17.98 & 36.49 & 6.98 & 14.03 & 31.06 & 6.61 & 13.07 & 27.49\\
			& \side{0.7} & 6.6M & 16\% & 13.17 & 17.93 & 36.14 & 6.79 & 14.83 & 31.90 & 6.30 & 12.95 & 27.60\\
			& \side{0.8} & 5.6M & 28\% & 11.99 & 17.54 & 34.60 & 6.23 & 13.94 & 29.72 & 6.02 & 13.01 & 27.63\\
			& \side{0.9} & \cellcolor[HTML]{656565}\color[HTML]{FFFFFF}  $\downarrow$ 3.7M & \cellcolor[HTML]{026329}\color[HTML]{FFFFFF} 51\% & 11.60 & 17.07 & 33.67 & 6.56 & 14.19 & 30.53 & 5.62 & 12.75 & 27.26   \\
			\midrule
			\rowcolor[gray]{.85} & & & & & & & & & & & & \\
			\midrule
			\multirow{6}{*}{\textit{Funcom \cite{leclair2019neural}}} & \emph{Full}      & \cellcolor[HTML]{656565}\color[HTML]{FFFFFF} $\uparrow$ 108.7M  & \cellcolor[HTML]{a3070c}\color[HTML]{FFFFFF} --   & 14.25 & 17.61 & 36.53 & 5.93 & 12.95 & 28.36 & 6.77 & 12.94 & 28.00 \\
			\cmidrule(r){2-13}
			& \side{0.5} & 99.5M & 9\%  & 15.04 & 18.16 & 37.08 & 6.15 & 13.14 & 29.05 & 6.84 & 12.87 & 27.93 \\
			& \side{0.6} & 95.0M & 13\% & 16.19 & 19.30 & 38.38 & 7.02 & 14.07 & 30.25 & 7.03 & 12.99 & 28.33 \\
			& \side{0.7} & 87.7M & 20\% & 14.77 & 18.92 & 37.39 & 7.49 & 14.19 & 30.14 & 6.62 & 12.79 & 27.78 \\
			& \side{0.8} & 74.2M & 31\% & 14.10 & 18.34 & 37.04 & 6.53 & 13.38 & 28.45 & 6.93 & 12.78 & 27.99 \\
			& \side{0.9} & \cellcolor[HTML]{656565}\color[HTML]{FFFFFF} $\downarrow$ 49.0M & \cellcolor[HTML]{026329}\color[HTML]{FFFFFF} 54\% & 13.65 & 18.08 & 37.12 & 6.78 & 13.61 & 30.07 & 6.81 & 12.95 & 28.07 \\
			\bottomrule
	\end{tabular}}
\end{table*}

We also perform statistical hypothesis tests (Wilcoxon signed-rank test) \cite{wilcoxon1992individual} and Cliff's delta effect size \cite{grissom2005effect} to compare the distributions of the BLEU-4, METEOR, and ROUGE-L of the predictions generated by the different models trained on the filtered training sets with those of the models trained on the full training sets. We use Holm's correction \cite{holm1979simple} to adjust the \textit{p}-values for the multiple tests. We reject the \textit{null hypothesis} (there is no difference between the effectiveness of two given models) if the \emph{p}-value is lower than 0.05.

Finally, we study the Pareto front to analyze the cost-benefit trade-offs between the effectiveness of the models trained on the different selections of \textit{TL-CodeSum} and \textit{Funcom} (benefit, measured with \ie, BLEU, METEOR, and ROUGE-L) and the corresponding training dataset size (cost).

To answer RQ$_{2}$ we compare the selection strategy with \side{0.9} (\ie the most restrictive selection), with a \textit{Random} baseline. In detail, we randomly sample the same number of training instances as those selected with \side{0.9} from each dataset. We compare the effectiveness of the models trained with the training instances selected with \side{0.9} and \textit{Random} measured in terms of the previously described metrics (\ie BLEU-4, METEOR, and ROUGE-L). Again, we perform statistical hypothesis tests (Wilcoxon signed-rank test) \cite{wilcoxon1992individual} and compute the Cliff's delta effect size \cite{grissom2005effect} to compare the distributions of BLEU-4, METEOR, and ROUGE-L of the predictions generated by the \side{0.9} model and the \textit{Random} baseline. We use Holm's correction \cite{holm1979simple} to adjust the \textit{p}-values for the multiple tests. We reject the \textit{null hypothesis} (there is no difference between the two models) if the \emph{p}-value is lower than 0.05.


\section{Results}
\label{sec:result}
\section{Result} \label{sec:result}

\subsection{Setup}

In this section, we evaluate VB-Com across the following perspectives:
\begin{itemize}
    \item Under what conditions does VB-Com demonstrate superior performance compared to using a single-policy approach?
    \item How does VB-Com outperforms baseline methods in those scenarios?
    \item How well does the proposed return estimator contribute to the composition system?
\end{itemize}

\begin{figure}[h]
\centering{\includegraphics[width=0.5\textwidth]{figures/noise.png}}
\caption{We present four types of perception noises and implement them on heightmaps during evaluation: gaussian noise, \textcolor{red}{forward shifting noise}, \textcolor{green}{lateral shifting noise} and \textcolor{blue}{floating noise}.}
\label{noise}
\end{figure}

\subsubsection{Evaluation Noise}
To simulate situations where the robot encounters perception outliers not present in the simulation, we introduce a quantitative curriculum noise designed to mimic varying levels of perception deficiency. As shown in Fig. \ref{noise}, we focus on four types of noise: (1) \textbf{Gaussian noise}: noise points sampled from a Gaussian distribution, to the original heightmap. The noise level is scaled from 0.0 to 1.0, where the training noise level corresponds to a 0.1 noise level in this scenario. (2) \textbf{Shifting noise}: replacing points in the original heightmap with noise sampled from a Gaussian distribution. The range of replacement points is controlled by the noise level, where a $100\%$ noise level results in a fully noisy heightmap. The shifting direction can either be along the heading direction (red box) or sideways (green box). (3) \textbf{Floating noise}: The heightmap is displaced vertically, either upwards or downwards, the floating noise simulates variations in terrain height. (blue box).

\begin{table}[!ht]
\caption{Terrain Size Scales (m)}
\label{tab:terrains}
\begin{center}
\renewcommand\arraystretch{1.25}
\begin{tabular}{lcccc}
\toprule[1.0pt]
Terrain & Length & Width & Heights\\
\midrule[0.8pt]

Gaps        & $(0.6, 1.2)$ & $(\bm{0.6}, \bm{0.8})$ & $(-1.8, -1.5)$\\  
Hurdles     & $(0.8, 1.0)$ & $(0.1, 0.2)$ & $(\bm{0.2}, \bm{0.4})$\\  
Obstacles   & $(\bm{0.2}, \bm{0.4})$ & $(0.2, 0.4)$ & $(1.4,1.8)$\\  

\bottomrule[1.0pt]
\end{tabular}
\end{center}
\end{table}

\subsubsection{Experiments Setup}
In simulation, we conduct $10 \times 3$ experiments for each method across three types of terrain, replicating the experiments three times to calculate the variance. Each episode involves the robot navigating through 8 goal points, with each goal paired with a corresponding challenging terrain or obstacle. The size of the terrains is set to the maximum curriculum terrain level, as shown in Table \ref{tab:terrains}. The bolded values indicate the primary factors that contribute to the difficulty for the terrain.

\subsubsection{Baselines}
We primarily compare VB-Com with the vision and blind policies operating independently. Additionally, as previous works have shown that robust perceptive locomotion can be learned by incorporating various perception noises during training \cite{miki2022learning}, we add a \textbf{Noisy Perceptive policy baseline} trained using the same noises implemented in the evaluation. This allows us to examine how well the proposed VB-Com policy performs compared to policies that have already seen the evaluation noises. The evaluation noises are introduced to the Noisy Perceptive policy in a curriculum format during training, which evolves with the terrain level.

\begin{figure*}[h]
\centering{\includegraphics[width=\textwidth]{figures/returnsim.png}}
\caption{Illustrations of the variation in estimated return and action phases(0 for $a_b$ and 1 for $a_v$) across three concerned terrains.}
\label{return}
\end{figure*}

\subsection{Example Case}
First, we illustrate how VB-Com operates, specifically when the composition switches to $\pi_b$ and how it effectively controls the robot to traverse the terrain against deficient perception (Fig. \ref{return}). We demonstrate $3$ seconds of the estimated returns, along with the policy composition phase, as the robot walking through the challenging terrain during the simulation experiments at the noise level of $100\%$. Before the robot encounters challenging terrains, we observe that the estimated return $G^e_{\pi_v}(s_t)$ consistently exceeds $G^e_{\pi_b}(s_t)$, as the robot is walking on flat ground with relatively stable motion. This observation aligns with the discussion in Section \ref{subsec:vb-com}, where it was explained that $\pi_v$ benefits from the external state observation and results in a higher return $G_t$. This characteraistic ensures the robot operates at $\pi_b$ while stable motion. 

Once the deficient perception reaches the $100\%$ noise level, the robot will not be aware of the incoming challenging terrains until it collides with them. At this point, we observe that $G^e_{\pi}(s_t)$ drops sharply within several control steps, prompting the switch to the blind policy. This switch allows the robot to respond to the terrain, and once the motion stabilizes, $G^e_{\pi}(s_t)$ returns to a normal level, at which point the vision policy regains control. These cases demonstrate the effectiveness of VB-Com, which responds quickly to deficient perception, but avoids unnecessary switches to the blind policy when it is not needed.


\begin{table*}[!h]
\caption{VB-Com Evaluations}
\label{tab:VB-Com}
\begin{center}
\renewcommand\arraystretch{1.25}
\begin{tabular}{lccccccc}
\toprule[1.0pt]
Noise Level &Method & Goals Completed($\%$) & Rewards & Average Velocity & Fail Rate & Collision Steps($\%$) & Reach Steps\\
\midrule[0.8pt]

% \multirow{4}{*}{Prop Advisor}&0.25& $0.7560$& $0.7964$& $0.7001$ & \multirow{4}{*}{$0.8600$}\\

\multirow{2}{*}{0\% noise} & VB-Com & $84.05 \pm 2.28$ & \bm{$142.07 \pm 4.19$} & $0.71 \pm 0.01$ & \bm{$0.29 \pm 0.01$} & $1.50 \pm 0.14$ & $177.29 \pm 4.66$\\  
                              & Vision & $73.57 \pm 4.97$ & $118.07 \pm 10.42$ & $0.73 \pm 0.01$ & $0.42 \pm 0.07$ & \bm{$1.39 \pm 0.53$} & $204.82 \pm 28.91$\\  \midrule
\multirow{2}{*}{30\% noise} & VB-Com & $82.24 \pm 6.6$ & $132.81 \pm 7.64$ & $0.71 \pm 0.01$ & $0.34 \pm 0.10$ & $2.09 \pm 0.13$ & $178.13 \pm 4.13$\\  
                              & Vision & $72.76 \pm 2.29$ & $115.20 \pm 2.43$ & $0.75 \pm 0.02$ & $0.43 \pm 0.05$ & $2.52 \pm 0.32$ & $195.58 \pm 21.98$\\  \midrule
\multirow{2}{*}{70\% noise} & VB-Com & $82.48 \pm 1.20$ & $132.44 \pm 6.17$ & $0.70 \pm 0.02$ & $0.33 \pm 0.03$ & $2.12 \pm 0.11$ & $184.81 \pm 4.47$\\  
                              & Vision & $55.38 \pm 3.33$ & $58.24 \pm 13.97$ & $0.73 \pm 0.03$ & $0.67 \pm 0.07$ & $6.08 \pm 0.82$ & $190.50 \pm 18.28$\\  \midrule
\multirow{3}{*}{100\% noise} & VB-Com & \bm{$84.81 \pm 6.45$} & $129.99 \pm 9.84$ & $0.72 \pm 0.02$ & \bm{$0.29 \pm 0.08$} & $2.60 \pm 0.68$ & $182.29 \pm 11.47$\\  
                              & Vision & $48.71 \pm 5.60$ & $47.53 \pm 17.55$ & $0.70 \pm 0.06$ & $0.69 \pm 0.06$ & $6.92 \pm 1.36$ & $268.40 \pm 57.11$\\  
                              & Noisy Perceptive & $80.52 \pm 0.91$ & $116.94 \pm 4.07$ & \bm{$0.76 \pm 0.02$} & $0.32 \pm 0.04$ & $3.49 \pm 0.38$ & \bm{$154.98 \pm 4.41$}\\ \midrule
& Blind & $83.76 \pm 1.35$ & $131.29 \pm 3.48$ & $0.70 \pm 0.01$ & $0.33 \pm 0.05$ & $2.57 \pm 0.27$ & $184.08 \pm 1.85$\\  

% Perceptive  & $0.00 \pm 0.00$ & $0.00 \pm 0.00$ & $0.00 \pm 0.00$ & $0.00 \pm 0.00$ & $0.00 \pm 0.00$\\  
% Blind  & $0.00 \pm 0.00$ & $0.00 \pm 0.00$ & $0.00 \pm 0.00$ & $0.00 \pm 0.00$ & $0.00 \pm 0.00$\\  
% Noisy Perceptive & $0.00 \pm 0.00$ & $0.00 \pm 0.00$ & $0.00 \pm 0.00$ & $0.00 \pm 0.00$ & $0.00 \pm 0.00$\\  

\bottomrule[1.0pt]
\end{tabular}
\end{center}
\end{table*}

\subsection{Evaluations on Different Noise Levels}
\textbf{VB-Com achieves robust locomotion performance under different levels of perception deficiency.} As shown in Tab \ref{tab:VB-Com}, performance of the vision policy declines shaprly with the arise of noise level. In addition, since the evaluation experiments set the terrain curriculum to the maximum level, the vision policy struggles even at a $0\%$ noise level: It only achieves around $73\%$ goal-reaching success, with a termination rate exceeding $40\%$. This poor performance is likely due to the severe challenge terrains, such as the farthest range of the heightmap $(0.85m)$ is only $0.05m$ wider than the width of the gaps$(0.8m)$. In contrast, VB-Com achieves a stable higher goal-reaching success against different levels of perception deficiency. In contrast, VB-Com achieves consistently higher goal-reaching success across varying levels of perception deficiency, including both noise and perception range limitations.

Despite the high goal-reaching success, we also include additional metrics to further analyze the performance. The reward values recorded throughout each episode indicate the proposed method’s ability to achieve both goal completion and collision avoidance. These rewards strongly correlate with the robot’s success in reaching the target while minimizing collisions. For instance, VB-Com at the $0\%$ noise level achieves the highest rewards$(142.07)$, although the goal completion rate$(84.05)$ is slightly lower compared to the trail in $100\%$ noise level $(84.81)$. This is because VB-Com switches to the blind policy more often in  $100\%$  noise level, resulting in more frequent collisions and lower rewards obtained. 

The reach steps metrics indicates the smoothness of the policy in overcoming challenging obstacles. Since the switching mechanism requires several steps to respond effectively, VB-Com results in a higher number of reach steps as the noise level increases. This is because, under higher noise conditions, the system needs additional time to transition from the vision policy to the blind policy, which leads to more gradual and controlled responses to terrain challenges.
\begin{figure}[h]
\centering{\includegraphics[width=0.5\textwidth]{figures/noiseevalueate.png}}
\caption{We compare the collision and goal-reaching performances under different noise levels. VB-Com achieves low collisions and high success rates with accurate perception, and its success rate remains high under deficient perception.}
\label{noiseevalueate}
\end{figure}

\begin{figure}[h]
\centering{\includegraphics[width=0.5\textwidth]{figures/terraineval.png}}
\caption{Comparisons between the Noisy Perceptive policy and VB-Com in navigating gaps and hurdles separately.}
\label{terraineval}
\end{figure}


\subsection{Comparisons with Blind Policy}
\textbf{VB-Com achieves less collision than the blind policy when perception becomes less dificient.} As shown in Tab \ref{tab:VB-Com}, the blind policy achieves a relatively high Goals Completed rate $(83.76\%)$, as its performance is unaffected by deficient perception. Therefore, we include an evaluation of the collision performance between VB-Com and the blind policy to further highlight the advantage of the proposed framework. In our evaluations, "Collision Steps" is defined as the ratio of the number of steps during which the robot collision model (Fig \ref{robot}) makes illegal contact with the terrain or obstacles, relative to the total number of steps within an episode.

We can observe from Tab \ref{tab:VB-Com} that the collision steps increase with the noise level for VB-Com. Fig \ref{noiseevalueate} provides a more intuitive illustration: as perception becomes more comprehensive, VB-Com achieves both fewer collisions and better goal-reaching performance. In contrast, the blind policy maintains a high goal-reaching rate but results in more collisions, while the vision policy performs better in avoiding collisions when the perception is accurate and comprehensive. As the noise level increases, the performance of VB-Com begins to resemble that of the blind policy. These results demonstrate the effectiveness of the composition system, which benefits from both sub-policies to achieve better performance in terms of both goal-reaching and minimizing collisions.

\subsection{Comparisons with Noisy Perceptive Training}
\textbf{Compared to policies trained with noisy priors, VB-Com achieves equivalent performance without prior knowledge of the noise, while also demonstrating better training efficiency and the ability to handle more challenging terrain difficulties.} The comparisons (Tab \ref{tab:VB-Com}) with Noisy Perceptive policy show that the Noisy Perceptive policy achieves a relatively high goal completion rate $(80.52\%)$ but exhibits a higher collision step rate $(3.49\%)$. It can be concluded that, as severe noise is introduced during evaluation, the heightmap quickly becomes random noise with the increasing noise level. In response, the Noisy Perceptive policy begins to exhibit behavior similar to that of the blind policy—making contact with obstacles and reacting when the noisy signals overwhelm the external observations.

To further investigate the conditions under which the Noisy Perceptive policy fails to surpass the performance of VB-Com, we evaluate goal-reaching performance under different terrains (Fig. \ref{terraineval}). The results show that VB-Com outperforms the Noisy Perceptive policy in gap terrains, while the Noisy Perceptive policy performs better in hurdle situations, achieving a higher success rate in preventing the robot from being tripped by hurdles. However, recovering from missed gaps requires a quicker response, or the robot risks falling. These results demonstrate that the single-policy method fails to handle such dynamic challenges effectively, highlighting the advantages of the composition in VB-Com.

\begin{figure}[h]
\centering{\includegraphics[width=0.5\textwidth]{figures/trainplot.png}}
\caption{Training curves for terrain levels and the return estimation loss.}
\label{train}
\end{figure}

Moreover, the terrain level rises slowly for the Noisy Perceptive policy (Fig. \ref{train}-(a)), and it fails to reach the maximum level achieved by the vision and blind policies. This is because the policy struggles with the trade-off of whether to trust the external perception, which requires the addition of an extra module to address the challenge. This slow progression highlights the difficulty of handling high levels of perception deficiency, whereas VB-Com can efficiently navigate such situations by leveraging the strengths of both the vision and blind policies.

\begin{table}[!ht]
\caption{Return Estimation Evaluations}
\label{tab:RE}
\begin{center}
\renewcommand\arraystretch{1.25}
\begin{tabular}{lcccc}
\toprule[1.0pt]
Method & Goals Completed($\%$) & Collisions & Reach Steps\\
\midrule[0.8pt]

100-steps) & $78.24 \pm 1.86$ & \bm{$2.49 \pm 0.04$} & $193.7 \pm 3.2$\\  
RE(50-steps)  & \bm{$81.90 \pm 2.81$} & $2.75 \pm 0.17$ & $184.6 \pm 1.4$\\ 
Re(5-steps)   & $69.90 \pm 7.34$ & $5.23 \pm 0.59$ & $192.6 \pm 3.3$\\  
Re(1-step)    & $59.57 \pm 2.00$ & $4.78 \pm 0.16$ & \bm{$167.4 \pm 5.0$}\\  
MC-based      & $74.14 \pm 2.69$ & $4.26 \pm 0.56$ & $192.8 \pm 11.8$\\  

\bottomrule[1.0pt]
\end{tabular}
\end{center}
\end{table}

\subsection{Return Estimator Evaluations}
\textbf{The proposed return estimator achieves accurate and efficient return estimation with accessible states observations.} Since we update the return estimator using temporal difference, we compare it with the Monte Carlo-based search return estimator that estimate the furtuen expected returns with the following regression loss directly: $\mathbb{E}_t[\hat{G}_{\pi_i}^e(s_t) - \sum_{t} ^ {t+T} \gamma^t r(s_t, a_t)]$. As shown in Fig. \ref{train}-(a), the MC-based estimator struggles to converge due to the accumulation of noise. In contrast, the proposed TD-based return estimator within the vision policy convergent stably as it updates alongside the locomotion policy. The results in Tab \ref{tab:RE} further highlight the ineffectiveness of the MC-based return estimator in providing accurate estimations to guide the policy composition. Specifically, the MC-based estimator struggles to respond promptly to collisions with obstacles, this delay in response leads to larger collisions and longer reach steps, as the policy cannot effectively adjust its actions in real-time. 

\textbf{We also evaluate the impact of different switch periods (T), which define the expected return duration during return estimator updates.} While training performance remains consistent across varying periods, we observe that excessively short switch periods can negatively impact system performance. In such cases, the two policies may conflict, resulting in incomplete motion trajectories when traversing the challenging terrains and failures.

\textbf{We observe that training effectiveness is highly dependent on data variance.} For instance, the estimator within vision policy converges the fastest due to its access to more accurate and comprehensive state observations, leading to fewer low-return instances. In contrast, the estimator within Noisy Perceptive and blind policies encounter more collisions and lower returns, causing their loss to degrade more slowly.

\textbf{We demonstrate that the estimated return threhold $G_{th}$ is crucial to the performance of VB-Com.} Tab \ref{tab:TH} evaluates the system's performance under different values of $\alpha$, as well as without $G_{th}$. The results demonstrate that $G_{th}$ is critical for mitigating miscorrection during motion abnormalities, and that a value of $\alpha < 1.0$ ensures a sensitive response to the states that could lead to motion failures.

\begin{table}[!ht]
\caption{Estimated Return Threhold Evaluations}
\label{tab:TH}
\begin{center}
\renewcommand\arraystretch{1.25}
\begin{tabular}{lcccc}
\toprule[1.0pt]
Method & Goals Completed($\%$) & Collisions & Reach Steps\\
\midrule[0.8pt]
 
$\alpha = 2.0$   & $77.10 \pm 4.71$ & $2.63 \pm 0.68$ & $185.11 \pm 7.17$\\ 
$\alpha = 0.5$   & \bm{$85.76 \pm 2.88$} & $2.29 \pm 0.17$ & $186.96 \pm 3.83$\\  
$\alpha = 0.1$   & $84.43 \pm 1.23$ & \bm{$2.10 \pm 0.25$} & $\bm{184.35 \pm 6.27}$\\  
w/o $G_{th}$     & $48.48 \pm 1.28$ & $6.24 \pm 0.41$ & $261.96 \pm 35.63$\\  

\bottomrule[1.0pt]
\end{tabular}
\end{center}
\end{table}



\subsection{Real-World Experiments}

We deploy the proposed system on both the Unitree G1 and Unitree H1 robots and evaluate the performance of the proposed VB-Com method. 
\begin{figure*}[h]
\centering{\includegraphics[width= \textwidth]{figures/hardwarecurve.png}}
\caption{Illustrations of the variation in estimated return under static/dynamic obstacles in hardware experiments.}
\label{hardwarecurve}
\end{figure*}

\subsubsection{Hardware Return Estimations}

We illustrate how VB-Com operates on real robots by plotting $4$ seconds of the estimated return while the robot avoids static (left) and dynamic (right) obstacles (Fig \ref{hardwarecurve}). The results demonstrate that, for static obstacles (a standing person), the elevation map can accurately perceive the obstacle, allowing the robot to plan motions in advance and avoid collisions. Corresponding to this behavior, we observe that the estimated return on the G1 stays a high value, with $\hat{G}^e_{\pi_b}$ slightly lower than $\hat{G}^e_{\pi_v}$, consistent with the scenario where the vision policy continues to operate within VB-Com.

On the other hand, when a person moves towards the robot at high speed, the perception module fails to detect the obstacle, causing a collision, both $\hat{G}^e_{\pi_b}$ and $\hat{G}^e_{\pi_v}$ decline sharply upon collision. However, VB-Com quickly switches to $\pi_b$ to avoid the person, demonstrating the  \textbf{rapid response to collision provided by the proposed return estimation and the successful obstacle avoidance capability of VB-Com under perceptual deficiency}.


\begin{figure}[h]
\centering{\includegraphics[width=0.5\textwidth]{figures/g1avoid.png}}
\caption{ Real-world comparisons of VB-Com, vision, and blind policies in obstacle avoidance on the G1.}
\label{avoid}
\end{figure}

\subsubsection{Avoid Obstacles}
In this section, we make comparisons between VB-Com along with the vision policy and blind policy on G1 (Fig \ref{avoid}), to demonstrate the superior performance of VB-Com in hardware compared with signle policies. In the evaluation scenario, G1 encounters two consecutive obstacles along its path. The second dynamic obstacle obstructs the robot's direction before the elevation map can perceive it. VB-Com enables the robot to avoid the static obstacle without collision and subsequently avoid the dynamic obstacle after it collides with the suddenly appearing obstacle.

In contrast, for the baseline policies, the blind policy makes unnecessary contact with the static obstacles before avoiding them, which damages the environment. As for the vision policy, the robot collides with the obstacle and is unable to avoid it until the newly added obstacle is detected and integrated into the map.

\begin{figure}[h]
\centering{\includegraphics[width=0.5\textwidth]{figures/hurdlegap.png}}
\caption{Hardware demonstrations on the robots traversing gaps and hurldes given deficient perception with VB-Com.}
\label{hurdlegap}
\end{figure}

\subsubsection{Performance Against Deficient Perception}
In this section, we demonstrate the ability of VB-Com to traverse challenging terrains given deficient perception (Fig. \ref{hurdlegap}). We provide zero inputs for the heightmaps to evaluate the performance of VB-Com under perceptual deficiency. We introduce two consecutive hurdles, and the robot successfully recovers after colliding with them by switching to $\pi_b$. Additionally, we demonstrate that VB-Com enables recovery from a missed step on an unobserved gap. In this case, VB-Com saves the robot by performing a larger forward step to traverse the gap without perception, as the blind policy has learned during simulation.




\section{Discussions}
\label{sec:implication}
Our findings challenge the conjecture that code-comment coherence, as measured by SIDE \cite{mastropaolo2024evaluating}, is a critical quality attribute for filtering instances of code summarization datasets. By selecting $\langle code, summary \rangle$ pairs with high-coherence for training allow to achieve the same results that would be achieved by randomly selecting such a number of instances. At the same time, we observed that reducing the datasets size up to 50\% of the training instances does not significantly affect the effectiveness of the models, even when the instances are randomly selected. These results have several implications.

First, code-comment consistency might not be a problem in state-of-the-art datasets in the first place, as also suggested in the results of RQ$_0$. Also, the DL models we adopted (and, probably, bigger models as well) are not affected by inconsistent code-comment pairs, even when these inconsistencies are present in the training set.
Despite the theoretical benefits of filtering by SIDE \cite{mastropaolo2024evaluating}, that is the state-of-the-art metric for measuring code-comment alignment, our results indicate its limitations in improving the \textit{overall} quality of the training sets for code summarization task.
Nevertheless, other quality aspects of code and comments that have not been explored yet (such as readability) may be important for smartly selecting the training instances.
Future work should explore such quality aspects further.

Our results clearly indicate that state-of-the-art datasets contain instances that do not contribute to improving the models' effectiveness. This finding is related to a general phenomenon observed in Machine Learning and Deep Learning. Models reach convergence when they are trained for a certain amount of time (epochs). Additional training provides smaller improvements and increases the risk of overfitting. We show that the same is true for data. In terms of effectiveness, model convergence is achieved with fewer training instances than previously assumed. Limiting the number of epochs may make it possible to reach model convergence with a subset of training data, maintaining model effectiveness, reducing resource demands and minimizing the risk of overfitting.
Future work could explore different criteria for data selection that identify the most informative subsets for training.
Conversely, this insight suggests that currently available datasets suffer from poor diversity (thus causing the previously discussed phenomenon).
This latter insight constitutes a clear warning for researchers interested in building code summarization datasets, which should include instances that add relevant information instead of adding more data, which might turn out to be useless.

Finally, it is worth pointing out that another benefit of the reduction we performed is the environmental impact. Reducing the number of training instances implies a reduced training time, which, in turn, lowers the resources necessary to perform training and, thus, energy consumption and CO$_2$ emissions.
We performed a rough estimation of the training time across different selections of \textit{TL-CodeSum} and \textit{Funcom} datasets and estimated a proxy of the CO$_2$ emissions for each model training phase by relying on the ML CO$_2$ impact calculator\footnote{\url{https://mlco2.github.io/impact/\#compute}} \cite{lacoste2019quantifying}. Such a calculator considers factors such as the total training time, the infrastructure used, the carbon efficiency, and the amount of carbon offset purchased. The estimation of CO$_{2}$ emissions needed to train the model with the \textit{Full} selection of \textit{Funcom} ($\sim$ 200 hours) is equal to 26.05 Kg, while with the optimized training set, \ie $SIDE_{0.9}$ ($\sim$ 90 hours), the estimation is 11.69 Kg of $CO_2$ (-55\% emissions).
While we recognize that this method provides an estimation rather than a precise measurement, it offers a glimpse into the environmental impact of applying data reduction.


\section{Threats to Validity}
\label{sec:threats}
\section{Threats to Validity}~\label{sec:Threats}
\subsection{Internal Validity}
In this study, the first author designed the SLR protocol, which was reviewed and refined collaboratively with the second, third, and fourth authors before formal implementation. The detailed topics and search strings were iteratively adjusted and executed across multiple databases to optimize the retrieval of relevant results. To accommodate the varying search policies of these databases, the search strings were customized accordingly. The selection of studies followed a multi-stage filtering process to minimize selection bias. The first round of filtering was based on titles and abstracts. The second round involved brief reading and keyword matching, while the third round consisted of a comprehensive reading of the papers. The final selection was validated by all authors to ensure robustness. Following study selection, a data extraction process was designed using Google Forms. All authors participated in a pilot test to refine the data extraction procedure and ensure consistency in capturing the necessary information.

\subsection{Construct Validity}
To mitigate threats to construct validity, we conducted the search process across six widely used scientific databases, employing a combination of automated and manual search strategies. Extensive discussions among all authors were held to refine the inclusion and exclusion criteria, ensuring they effectively supported the selection of the most relevant studies for this SLR. Some of the selected studies included vague descriptions of their methodologies, posing potential threats to the validity of the study. These cases were carefully reviewed and deliberated upon by the first and second authors to reach a consensus on their inclusion.

\subsection{Conclusion Validity}
The threat to conclusion validity was minimized through a carefully planned and validated search and data extraction process. To ensure the extracted data aligned with our study requirements, we designed the data extraction form based on the predefined research questions (RQs). The first author initially extracted data from a subset of selected papers using this form, after which the extracted data was reviewed and verified by the other authors. Once validated, the first author used the refined form to extract data from the remaining studies. During data analysis and synthesis, multiple discussions were conducted to determine the most effective categorization and representation of the data, ensuring robust and meaningful conclusions.

\subsection{External Validity}
To address the threat to external validity, we employed a combination of automated and manual search strategies, adhering to widely accepted guidelines~\cite{kitchenham2009systematic, wohlin2014guidelines}. Our methodology section provides a detailed explanation of the inclusion and exclusion criteria. Specifically, we focused on peer-reviewed academic studies published in English, excluding grey literature, book chapters, opinion pieces, vision papers, and comparison studies. While these criteria may exclude some potentially relevant works, they were implemented to minimize bias in the selection process. We adopted a broad inclusion approach, considering studies regardless of their publication quality. Furthermore, our search encompassed publications from 1992 to the present, ensuring comprehensive coverage of advancements in the field of REDAST.

\section{Related Work}
\label{sec:related}
\putsec{related}{Related Work}

\noindent \textbf{Efficient Radiance Field Rendering.}
%
The introduction of Neural Radiance Fields (NeRF)~\cite{mil:sri20} has
generated significant interest in efficient 3D scene representation and
rendering for radiance fields.
%
Over the past years, there has been a large amount of research aimed at
accelerating NeRFs through algorithmic or software
optimizations~\cite{mul:eva22,fri:yu22,che:fun23,sun:sun22}, and the
development of hardware
accelerators~\cite{lee:cho23,li:li23,son:wen23,mub:kan23,fen:liu24}.
%
The state-of-the-art method, 3D Gaussian splatting~\cite{ker:kop23}, has
further fueled interest in accelerating radiance field
rendering~\cite{rad:ste24,lee:lee24,nie:stu24,lee:rho24,ham:mel24} as it
employs rasterization primitives that can be rendered much faster than NeRFs.
%
However, previous research focused on software graphics rendering on
programmable cores or building dedicated hardware accelerators. In contrast,
\name{} investigates the potential of efficient radiance field rendering while
utilizing fixed-function units in graphics hardware.
%
To our knowledge, this is the first work that assesses the performance
implications of rendering Gaussian-based radiance fields on the hardware
graphics pipeline with software and hardware optimizations.

%%%%%%%%%%%%%%%%%%%%%%%%%%%%%%%%%%%%%%%%%%%%%%%%%%%%%%%%%%%%%%%%%%%%%%%%%%
\myparagraph{Enhancing Graphics Rendering Hardware.}
%
The performance advantage of executing graphics rendering on either
programmable shader cores or fixed-function units varies depending on the
rendering methods and hardware designs.
%
Previous studies have explored the performance implication of graphics hardware
design by developing simulation infrastructures for graphics
workloads~\cite{bar:gon06,gub:aam19,tin:sax23,arn:par13}.
%
Additionally, several studies have aimed to improve the performance of
special-purpose hardware such as ray tracing units in graphics
hardware~\cite{cho:now23,liu:cha21} and proposed hardware accelerators for
graphics applications~\cite{lu:hua17,ram:gri09}.
%
In contrast to these works, which primarily evaluate traditional graphics
workloads, our work focuses on improving the performance of volume rendering
workloads, such as Gaussian splatting, which require blending a huge number of
fragments per pixel.

%%%%%%%%%%%%%%%%%%%%%%%%%%%%%%%%%%%%%%%%%%%%%%%%%%%%%%%%%%%%%%%%%%%%%%%%%%
%
In the context of multi-sample anti-aliasing, prior work proposed reducing the
amount of redundant shading by merging fragments from adjacent triangles in a
mesh at the quad granularity~\cite{fat:bou10}.
%
While both our work and quad-fragment merging (QFM)~\cite{fat:bou10} aim to
reduce operations by merging quads, our proposed technique differs from QFM in
many aspects.
%
Our method aims to blend \emph{overlapping primitives} along the depth
direction and applies to quads from any primitive. In contrast, QFM merges quad
fragments from small (e.g., pixel-sized) triangles that \emph{share} an edge
(i.e., \emph{connected}, \emph{non-overlapping} triangles).
%
As such, QFM is not applicable to the scenes consisting of a number of
unconnected transparent triangles, such as those in 3D Gaussian splatting.
%
In addition, our method computes the \emph{exact} color for each pixel by
offloading blending operations from ROPs to shader units, whereas QFM
\emph{approximates} pixel colors by using the color from one triangle when
multiple triangles are merged into a single quad.




\section{\rev{Conclusion}}
\label{sec:conclusion}
\section{Conclusion}
In this work, we propose a simple yet effective approach, called SMILE, for graph few-shot learning with fewer tasks. Specifically, we introduce a novel dual-level mixup strategy, including within-task and across-task mixup, for enriching the diversity of nodes within each task and the diversity of tasks. Also, we incorporate the degree-based prior information to learn expressive node embeddings. Theoretically, we prove that SMILE effectively enhances the model's generalization performance. Empirically, we conduct extensive experiments on multiple benchmarks and the results suggest that SMILE significantly outperforms other baselines, including both in-domain and cross-domain few-shot settings.

\section{Data Availability}
\label{sec:data}
The study dataset and scripts used for the analysis are available and documented in our online replication package~\cite{replicationpackage}.


\section{Acknowledgments}
\label{sec:ack}
% \smallskip
% \myparagraph{Acknowledgments} We thank the reviewers for their comments.
% The work by Moshe Tennenholtz was supported by funding from the
% European Research Council (ERC) under the European Union's Horizon
% 2020 research and innovation programme (grant agreement 740435).



\balance
\bibliographystyle{IEEEtran}
\bibliography{main}


\end{document}
\typeout{get arXiv to do 4 passes: Label(s) may have changed. Rerun}
