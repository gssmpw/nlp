\documentclass[10pt,conference]{IEEEtran}
\IEEEoverridecommandlockouts

\usepackage{amsmath,amsfonts, amssymb}
\usepackage{algorithm}
\usepackage{adjustbox}
\usepackage{lscape}
\usepackage{siunitx}
\usepackage[noend]{algpseudocode}
\usepackage{textcomp}
\usepackage{fancyvrb}
\usepackage{graphicx}
\usepackage{soul}
\usepackage[many]{tcolorbox}
\usepackage{setspace}
\usepackage{booktabs}
\usepackage[utf8]{inputenc}
\usepackage[T1]{fontenc}
\usepackage{rotating}
\usepackage{graphicx}
\usepackage{paralist}
\usepackage{tabularx}
\usepackage{multicol}
\usepackage{multirow}
\usepackage{pbox}
\usepackage{enumitem}
\usepackage{colortbl}
\usepackage{pifont}
\usepackage{xspace}
\usepackage{url}
\usepackage{tikz}
\usepackage{tabularx}
\usepackage{fontawesome}
\usepackage{lscape}
\usepackage{listings}
\usepackage{color}
\usepackage{showexpl}
\usepackage{anyfontsize}
\usepackage{comment}
\usepackage{soul}
\usepackage{multibib}
\usepackage{multirow}
\usepackage{xspace}
\usepackage{footnote}
\usepackage{tcolorbox}
\usepackage{booktabs}
\usepackage{balance}
\usepackage[normalem]{ulem}

\usepackage{hhline}

% MACROs for stylized Tcolorbox
\definecolor{main}{HTML}{5989cf}    % setting main color to be used
\definecolor{sub}{HTML}{cde4ff}     % setting sub color to be used



\tcbset{
	sharp corners,
	colback = white,
	before skip = 0.2cm,    % add extra space before the box
	after skip = 0.5cm      % add extra space after the box
}                           % setting global options for tcolorbox

\newtcolorbox{boxM}{
	fontupper = \color{white},
	rounded corners,
	arc = 6pt,
	colback = main!80,
	colframe = main,
	boxrule = 0pt,
	bottomrule = 4.5pt,
	enhanced,
	fuzzy shadow = {0pt}{-3pt}{-0.5pt}{0.5pt}{black!35}
}
%%%%%%%%%%%%%%%%%%%%%%%%%%%%%%%%

\newcolumntype{N}{>{\centering\arraybackslash}m{.85in}}
\def\BibTeX{{\rm B\kern-.05em{\sc i\kern-.025em b}\kern-.08em
    T\kern-.1667em\lower.7ex\hbox{E}\kern-.125emX}}

\newboolean{showcomments}

\setboolean{showcomments}{true}

\ifthenelse{\boolean{showcomments}}
{\newcommand{\nb}[2]{
		\fbox{\bfseries\sffamily\scriptsize#1}
		{\sf\small$\blacktriangleright$\textit{#2}$\blacktriangleleft$}
	}
	\newcommand{\cvsversion}{\emph{\scriptsize$-$Id: macro.tex,v 1.9 2005/12/09 22:38:33 xxx Exp \$}}
}
{\newcommand{\nb}[2]{}
	\newcommand{\cvsversion}{}
}

\makeatletter
\newcommand{\linebreakand}{%
	\end{@IEEEauthorhalign}
	\hfill\mbox{}\par
	\mbox{}\hfill\begin{@IEEEauthorhalign}
}
\makeatother

\newcommand{\cmark}{\ding{51}}%
\newcommand{\xmark}{\ding{55}}%
\newcommand{\ie}{\emph{i.e.,}\xspace}
\newcommand{\eg}{\emph{e.g.,}\xspace}
\newcommand{\etc}{etc.\xspace}
\newcommand{\etal}{\emph{et~al.}\xspace}
\newcommand{\secref}[1]{Section~\ref{#1}\xspace}
\newcommand{\figref}[1]{Fig.~\ref{#1}\xspace}
\newcommand{\listref}[1]{Listing~\ref{#1}\xspace}
\newcommand{\tabref}[1]{Table~\ref{#1}\xspace}
\newcommand{\TBD}[1]{\textcolor{red}{#1}\xspace}
\newcommand{\java}{\emph{Java}\xspace}
\newcommand{\tool}{\emph{SATDBailiff}\xspace}
\newcommand*\circled[1]{\tikz[baseline=(char.base)]{
		\node[shape=circle,fill,inner sep=0.8pt] (char) {\textcolor{white}{#1}};}}
\newcommand{\side}[1]{SIDE$_{{#1}}$\xspace}
\newcommand{\green}{green AI\xspace}


\definecolor{lightergray}{rgb}{0.9,0.9,0.9}
\newtcolorbox{resultbox}{colback=lightergray, arc=0.5mm, top=2mm, bottom=2mm, left=2mm, right=2mm}

\definecolor{arsenic}{rgb}{0.23, 0.27, 0.29}
\definecolor{darkgray}{rgb}{0.33, 0.33, 0.33}

\newcommand\TODO[1]{\textcolor{red}{#1}}
\newcommand\REMOVE[1]{\textcolor{red}{\st{#1}}}
\newcommand\ANTONIO[1]{\textcolor{blue}{\nb{ANTONIO}{#1}}}
\newcommand\MAX[1]{\textcolor{green}{\nb{MAX}{#1}}}
\newcommand\VITALE[1]{\textcolor{orange}{\nb{VITALE}{#1}}}
\newcommand\SIMONE[1]{\textcolor{red}{\nb{SIMONE}{#1}}}

\newcommand\rev[1]{\textcolor{black}{#1}}
\pagenumbering{arabic}
\pagestyle{empty}

\begin{document}

\title{Optimizing Datasets for Code Summarization:\\ Is Code-Comment Coherence Enough?}

\author{
\IEEEauthorblockN{Antonio Vitale\IEEEauthorrefmark{1}\IEEEauthorrefmark{3}, Antonio Mastropaolo\IEEEauthorrefmark{2}, Rocco Oliveto\IEEEauthorrefmark{3}, Massimiliano Di Penta\IEEEauthorrefmark{4}, and Simone Scalabrino\IEEEauthorrefmark{4}}
\IEEEauthorblockA{\IEEEauthorrefmark{1}Politecnico di Torino, Italy, antonio.vitale@polito.it}
\IEEEauthorblockA{\IEEEauthorrefmark{2}William \& Mary, USA, amastropaolo@wm.edu}
\IEEEauthorblockA{\IEEEauthorrefmark{3}University of Molise, Italy, \{rocco.oliveto, simone.scalabrino\}@unimol.it}
\IEEEauthorblockA{\IEEEauthorrefmark{4}University of Sannio, Italy, dipenta@unisannio.it}
}


\maketitle

\thispagestyle{empty}


\begin{abstract}
Automated code summarization is a long-standing goal for code comprehension. This task automatically generates documentation using a given method. Deep Learning (DL)-based approaches have been proven beneficial for various software engineering (SE) tasks, including this one. 
Most state-of-the-art datasets for code summarization are automatically mined from GitHub and, thus, might contain erroneous or sub-optimal examples. Previous work showed that using a simple rule-based approach for removing noisy instances allows for a tangible reduction of the training set size while not reducing the effectiveness of the trained models. Motivated by this finding, we conjecture that it is possible to further reduce the dataset size by removing instances that contain different issues.
In this paper, we explore the extent to which code-comment coherence, a specific quality attribute of code summaries, can be used to optimize code summarization datasets. Specifically, we hypothesize that removing incoherent code-comment pairs might positively impact the effectiveness of the models. To do this, we rely on SIDE, a recently introduced metric for code-summary coherence. We examine multiple selectivity levels of training instances from two state-of-the-art datasets (TL-CodeSum and Funcom) and evaluate the resulting models on three manually curated test sets. The results show that even halving the training set sizes does not significantly affect the model's ability to generate summaries. However, when comparing the most restrictive selection strategy with a simpler one that randomly selects the training instances, we observe that the resulting accuracy of the model also does not change.
This result suggests that (i) current datasets contain many irrelevant examples, and (ii) different quality attributes should be explored for optimizing code summarization datasets.
\end{abstract}

\begin{IEEEkeywords}
	Code Summarization, Data Quality, Code-Comment Coherence, Empirical Study
\end{IEEEkeywords}


\section{Introduction}
\label{sec:intro}
\section{Introduction}
\label{sec:intro}

\begin{figure*}[tb]
    \centering
    \includegraphics[width=0.848\linewidth]{figs/circuitnn.pdf} 
    \caption{Illustration of differentiable CircuitNN. CircuitNN is designed based on differentiable NAND gates. After DAS is guided by PI and PO pairs of the truth table, CircuitNN can get the precise circuit architecture logic equivalent to the truth table.}
    \label{fig:circuitnn}
\end{figure*}

% 1. Describe the importance of logic synthesis
% 2. Existing Problems
% (a) Neural Architecture Search: Unstable, Predefined Setting, etc.
% (b) Circuit Generation: Probabilistic Model, Logic Equivalence

With the rapid advancement of technology, the scale of integrated circuits (ICs) has expanded exponentially. 
This expansion has introduced significant challenges in chip manufacturing, particularly concerning power and area metrics.
A primary objective in IC design is achieving the same circuit function with fewer transistors, thereby reducing power usage and area occupancy.

Logic synthesis~\cite{hachtel2005logicsynth}, a critical step in electronic design automation (EDA), transforms behavioral-level circuit designs into optimized gate-level circuits, ultimately yielding the final IC layout. 
The primary goal of logic synthesis is to identify the physical implementation with the fewest gates for a given circuit function. 
This task constitutes a challenging NP-hard combinatorial optimization problem. 
Current logic synthesis tools~\cite{brayton2010abc, wolf2013yosys} rely on human-designed heuristics, often leading to sub-optimal outcomes.

Differentiable architecture search (DAS) techniques~\cite{liu2018darts, chu2020darts} offer novel perspectives on addressing challenges in this problem.
Circuit functions can be represented through truth tables, which map binary inputs to their corresponding outputs. 
Truth tables provide a precise representation of input-output relationships, ensuring the design of functionally equivalent circuits.
Inspired by this, researchers~\cite{deepmind2024ai4sys, wang2024tnet} have begun exploring the application of DAS to synthesize circuits directly from truth tables.
Specifically, \citet{deepmind2024ai4sys} proposed CircuitNN, a framework that learns differentiable connection structures with logic gates, enabling the automatic generation of logic circuits from truth tables.
This approach significantly reduces the complexity of traditional circuit generation. 
Building on this, \citet{wang2024tnet} introduced T-Net, a triangle-shaped variant of CircuitNN, incorporating regularization techniques to enhance the efficiency of DAS.

Despite these advancements, several challenges remain. 
The computational complexity of DAS grows quadratically with the number of gates, posing scalability issues.
Although triangle-shaped architecture~\cite{wang2024tnet} partially mitigates this problem, redundancy persists. 
%Additionally, DAS is susceptible to converging to local optima, limiting the ability to search architectures that satisfy the given truth tables~\cite{liu2018darts}. 
%Furthermore, hyperparameters (network depth and layer width) require extensive searches, introducing complexity and prolonging the synthesis process. 
Additionally, DAS is susceptible to converging to local optima~\cite{liu2018darts} and hyperparameters (network depth and layer width) require extensive searches. 
The challenges arise from the vast search space in DAS. 
% Even with predefined settings for CircuitNN, finding a configuration that meets the truth table requires extensive trial and error during the DAS process. 
Intuitively, limiting the search space through predefined parameters (network depth, gates per layer, and connection probabilities) can significantly reduce the complexity.

Recent advances~\cite{openai2023gpt4, abramson2024alphafold3, esser2024sd3, li2024mar} in conditional generative models have demonstrated remarkable performance across language, vision, and graph generation tasks. 
Motivated by these developments, we propose a novel approach to circuit generation that generates preliminary circuit structures to guide DAS in generating refined circuits matching specified truth tables. 
Firstly, we introduce CircuitVQ, a tokenizer with a discrete codebook for circuit tokenization. 
Built upon our Circuit AutoEncoder framework~\cite{hou2022graphmae,li2023maskgae,wu2025mgvga}, CircuitVQ is trained through a circuit reconstruction task. 
Specifically, the CircuitVQ encoder encodes input circuits into discrete tokens using a learnable codebook, while the decoder reconstructs the circuit adjacency matrix based on these tokens.
Subsequently, the CircuitVQ encoder serves as a circuit tokenizer for CircuitAR pretraining, which employs a masked autoregressive modeling paradigm~\cite{chang2022maskgit, li2023mage}. 
In this process, the discrete codes function as supervision signals. 
After training, CircuitAR can generate discrete tokens progressively, which can be decoded into initial circuit structures by the decoder of the CircuitVQ. 
These prior insights can guide DAS in producing refined circuits that match the target truth tables precisely.

Our key contributions can be summarized as follows:
\begin{itemize}
\item We introduce CircuitVQ, a circuit tokenizer that facilitates graph autoregressive modeling for circuit generation, based on our Circuit AutoEncoder framework;
\item Develop CircuitAR, a model trained using masked autoregressive modeling, which generates initial circuit structures conditioned on given truth tables;
\item Propose a refinement framework that integrates differentiable architecture search to produce functionally equivalent circuits guided by target truth tables;
\item Comprehensive experiments demonstrating the scalability and capability emergence of our CircuitAR and the superior performance of the proposed circuit generation approach.
\end{itemize}

% Motivation
% (a) Diffusion (Vision, Graph), Autoregressive (Language, Vision)
% (b) Circuit Generation for Predefined Setting
% (c) Neural Architecture Search for Strict Logic Equivalence

% Contribution
% (a) Circuit Tokenizer (new transformer arch, training strategy)
% (b) CircuitAR (train and gen strategies, post-ar strategy)
% (c) Extensive Evaluation including BitD (Bit Distance) for Scalability


\section{Selection Strategies for Code Summarization}
\label{sec:selection_strategies}
% !TEX root = main.tex
In this section, we provide backgrounds about (i) a state-of-the-art strategy for repairing or removing poor instances from code summarization datasets (CAT), and (ii) the SIDE metric, which we use to streamline a data-centric, quality-aware instance filtering.

\subsection{Code-comment cleAning Tool}
CAT (\textbf{C}ode-comment cle\textbf{A}ning \textbf{T}ool) is an approach and tool by Shi \etal \cite{shi2022we} to detect and handle noisy instances given the pairs of \textit{<code, summary>} from code-summarization datasets. The development of CAT was preceded by a manual investigation involving 9 participants who examined 1,600 \textit{<code, summary>} pairs. This manual analysis aimed to define a taxonomy of noisy data categories.

The taxonomy features comments- and code-related noisy data, such as \textit{commented-out method} or \textit{empty function}. Based on such categories, Shi \etal defined a set of heuristic rules and implemented them in the CAT tool. CAT works with two possible strategies based on the issues found: On the one hand, it fixes instances with minor issues. For example, it removes block-level comments. On the other hand, it completely drops instances where no fix is possible, including, for example, getters and setters.

\subsection{Fine-Grained Filtering: SIDE}
\label{sec:side-aware}
SIDE (\textbf{S}ummary al\textbf{I}gnment to co\textbf{D}e s\textbf{E}mantics) is a novel quality-aware metric presented by Mastropaolo \etal \cite{mastropaolo2024evaluating}. SIDE addresses the shortcomings of traditional metrics such as BLEU \cite{papineni2002bleu}, ROUGE \cite{lin:tsbo2004}, and METEOR \cite{banerjee:acl2005} used in code summarization tasks. 

SIDE employs a contrastive learning approach to determine the accuracy with which a code summary documents the underlying code, explicitly focusing on Java methods. Contrastive learning aims to maximize the distance between the reference code and inappropriate comments while minimizing the distance to suitable comments. The model that implements SIDE, MPNet \cite{Song2020MPNetMA}, provides a continuous score ranging from -1 to 1. Scores closer to~-1 indicate poor alignment between the code summary and the actual code, whereas scores closer to~1 suggest a strong alignment. SIDE showed a high correlation with human evaluations of summary quality, outperforming established metrics like BLEU, ROUGE, and METEOR. 

The benefits of a quality-aware metric like SIDE extend beyond evaluating code summarization techniques, as Mastropaolo \etal \cite{mastropaolo2024evaluating} noted. SIDE can be valuable when distinguishing high-quality code documentation from subpar examples is essential. We conjecture that SIDE ensures that only the instances most likely to enhance the training procedure are used, enabling the model to converge faster without a drop in performance by filtering out low-quality elements.

Our study relies on SIDE to select $\langle code, summary \rangle$ pairs with high coherence. Specifically, given a training set $T$ and a threshold $t$, we select the training instances $\{ p_{i} \in T \mid \text{SIDE}(p_{i}) \ge t \}$. 


\section{Study Definition, Design and Planning}
\label{sec:study}
We conduct extensive experiments to demonstrate the efficacy of various features of \former~to allow it to be trained with only one hour of data. We also present our findings in a way that highlights \former’s differences compared to common practices in NLP and CV, where Transformer training practices have been extensively studied~\cite{xiong2020layer, xu2021optimizing, loshchilov2024ngpt, chen2021empirical, liu2021efficient, gani2022how, steiner2022how}. 
Therefore, our experiment results also serve as a guideline on how to optimize Transformer training for robotics, particularly in off-road navigation and mobility tasks with complex vehicle-terrain interactions under data-scarce conditions. 

\former's one hour of training data comes from human-teleoperated demonstration of driving an open-source four-wheeled ground vehicle~\cite{datar2024wheeled} on a custom-built off-road testbed composed of hundreds of rocks and boulders. The demonstrator mostly aims to drive the robot to safely and stably traverse the vertically challenging terrain, but still occasionally encounters dangerous situations such as large roll angles and getting stuck between rocks. Fortunately, those situations serve as explorations for \former~to understand a wider range of kinodynamic interactions. 
Direct application of standard Transformer training methodologies in NLP and CV to such a small robotics dataset proves challenging due to the inherent lack of inductive bias in Transformers~\cite{dosovitskiy2021image}, which necessitates substantial amounts of data for effective training. However, our experiments suggest that \former's judicious modifications to established MM and NTP training paradigms can facilitate effective Transformer training even with limited robotics data. 

We conduct our experiments based on three perspectives: Section~\ref{sec:basic_perspective} provides an analysis of basic factors to train Transformers in general; Section~\ref{sec:robotic_perspective} analyzes the best practices to train Transformers when dealing with off-road robot mobility data; Finally, Sec.~\ref{sec:objective_perspective} evaluates the effectiveness of each off-road mobility learning objective and compares \encoder, \decoder, and non-Transformer end-to-end model performances. For fairness, all experiments are conducted with the same hyper-parameters. 

\subsection{Experiment Results of Basic Transformer Factors} \label{sec:basic_perspective}

\begin{figure}[t]
  \centering
  \includegraphics[width=0.7\columnwidth]{figures/pos_encoding}
  \caption{\textbf{Positional Encoding:} Sinusoidal positional encoding achieves better model accuracy than learnable encoding for predicting $\mathbf{X}$, $\mathbf{Y}$, and $\mathbf{Z}$ components of the robot pose.}
  \label{fig:pos_encoding}
  % \vspace{-1.2em}
\end{figure}

\textbf{Positional encoding} is crucial for addressing the permutation equivariance of Transformers, which, by design, lacks inherent sensitivity to input sequence order. This characteristic necessitates the explicit provision of positional information to enable the model to effectively process sequential data. Learnable positional encodings, typically implemented as trainable vectors added to input embeddings, have found favor in CV applications~\cite{he2022masked}. Conversely, non-learnable encodings, such as the sinusoidal functions introduced in the seminal work by~\citet{vaswani2017attention}, have demonstrated efficacy in NLP tasks.
This divergence in methodological preference may stem from inherent differences in the statistical properties of data modalities. CV tasks often involve spatially structured data where absolute positional information may be less critical than relative relationships between local features. In such contexts, learnable encodings may offer greater flexibility in adapting to task-specific positional dependencies. Conversely, NLP tasks frequently rely on precise word order and long-range dependencies, where the fixed nature of non-learnable encodings may provide a beneficial inductive bias~\cite{weng2024navigating}.

To empirically investigate the relative merits of these approaches on robot mobility tasks, we conduct a comparative analysis of learnable positional encodings against sinusoidal encodings as shown in Fig.~\ref{fig:pos_encoding}. Our findings indicate that while both methods achieve comparable asymptotic performance levels, sinusoidal positional encodings exhibit a slight performance advantage.

\begin{figure}[t]
  \centering
  \includegraphics[width=0.7\columnwidth]{figures/last_layernorm}
  \caption{\textbf{Normalizing Output:} Normalizing the \tr~output before passing the embeddings to the task decoder improves model performance.}
  \label{fig:last_layernorm}
  % \vspace{-1.2em}
\end{figure}
\textbf{Normalization layers}, such as LayerNorm~\cite{ba2016layer} or RMSNorm~\cite{zhang2019root}, have been shown to play a crucial role in stabilizing the training of Large Language Models (LLMs)~\cite{loshchilov2024ngpt}. By normalizing the activations of hidden units, these layers help to address issues such as vanishing/exploding gradients and improve the overall stability of the training process~\cite{xiong2020layer}. In this study, we investigate the impact of applying RMSNorm layer immediately before the task head.

Our experiment results, depicted in Fig.~\ref{fig:last_layernorm}, demonstrate an advantage for a model incorporating RMSNorm layer before the task head. This configuration consistently exhibits improved generalization performance and enhanced training stability compared to a model without the final RMSNorm. This finding suggests that normalizing the final embedding vector before passing it to the task head can benefit model performance, potentially by facilitating more effective gradient flow and thus improving the robustness of the model's predictions.


\subsection{Experiment Results from a Robotics Perspective} \label{sec:robotic_perspective}

\begin{figure}[h]
  \centering
  \includegraphics[width=0.7\columnwidth]{figures/order_prediction}
  \caption{\textbf{Kinodynamics Understanding:} Without unified latent representation the model cannot capture temporal dependencies and understand kinodynamic transitions, resulting in an almost flat learning curve.}
  \label{fig:unified_state}
  % \vspace{-1.2em}
\end{figure}
\textbf{Unified latent space representation} offers a significant advantage in simultaneously addressing FKD, IKD, and BC. This unified approach facilitates a more holistic understanding of the robot's state and its interaction with the environment. To evaluate the efficacy of this unified representation, we perform a targeted ablation study. We train \coder~based on the objectives outlined by~\citet{nazeri2024vertiencoder} and augment them with additional objectives specifically designed to probe the model's capacity of kinodynamics understanding. 

A key component of this ablation involves the introduction of a sequence order prediction objective. This objective aims to assess whether the model can effectively discern the temporal evolution of robot and environment dynamics. During training, the model is presented with input sequences in two configurations: (1) 50\% of the time, the input sequence is presented in its natural temporal order; (2) the remaining 50\% of the time, the input sequence is randomly shuffled, disrupting the temporal coherence. The model is then tasked to classify whether an unseen sequence is presented in its original order or is shuffled, testing the model's ability to capture temporal dependencies and understand kinodynamic transitions.

As illustrated in Fig.~\ref{fig:unified_state}, our findings demonstrate a clear distinction in model performance based on the input representation. When the model is provided with separate, non-unified tokens, it exhibits a limited capacity of understanding the underlying kinodynamics and the learning loss barely drops. This suggests that processing information in a fragmented manner hinders the model's ability to capture temporal relationships and kinodynamic evolution, which is aligned with the findings by~\citet{zhou2024dinowm}. It may be possible to compensate by training with a larger dataset, which, however, is not always available in robotics.

Conversely, the utilization of a unified latent space representation significantly enhances the model's ability to discern temporal order and, consequently, understand the dynamics of the system. By consolidating relevant information into a single, cohesive representation, the model can effectively capture the interdependencies among different modalities and their evolution over time. This highlights the importance of a unified latent space representation in enabling robotic models to effectively learn and reason about complex dynamic systems when trained on limited data, in contrast to NLP and CV tasks where the data acquisition is easier.


\begin{figure}[h]
  \centering
  \includegraphics[width=1\columnwidth]{figures/steps_comparison}
  \caption{\textbf{Prediction Horizon:} \former~is capable of predicting a longer horizon without losing much accuracy due to its non-autoregressive nature.}
  \label{fig:pred_horizon}
  % \vspace{-1.2em}
\end{figure}
\textbf{Prediction horizon} is a critical factor in navigation planning. While longer prediction horizons can potentially lead to better planning by considering long-term effects, they also introduce greater uncertainty. This is because errors in early predictions can accumulate and lead to significant deviations in subsequent predictions. This issue is particularly relevant for autoregressive models such as the \vertidecoder~part of \former, where each prediction is based on the previous one. In such models, even a small error in the initial steps can propagate and amplify over time, causing the predicted trajectory to drift further away from the true path. To evaluate the impact of prediction horizon, we compare the performance of the autoregressive \vertidecoder~with the non-autoregressive \former, specifically focusing on their ability to maintain accuracy over long horizons. The results, shown in Fig.~\ref{fig:pred_horizon}, demonstrate that \former~is capable of predicting a longer horizon (two seconds) with less drift compared to its autoregressive counterpart even with a shorter horizon (one second). This highlights the advantage of non-autoregressive models in tasks requiring long-term prediction, as they are less susceptible to error accumulation.

\subsection{Experiment Results of Robotic Objective Functions} \label{sec:objective_perspective}
\begin{figure}[h]
  \centering
  \includegraphics[width=\columnwidth]{figures/patch_head}
  \caption{\textbf{Patch Prediction Head:} The inclusion of a patch reconstruction head results in a degradation of overall model performance. This counterintuitive result can be attributed to the inherent difficulty in accurately predicting the detailed structure of off-road terrain topography.}
  \label{fig:patch_head}
  % \vspace{-1.2em}
\end{figure}

\textbf{Patch prediction head}, as an auxiliary head to learn environment kinodynamics, was first introduced by~\citet{nazeri2024vertiencoder}. However, we find that the high complexity of off-road terrain topography and the potential presence of noise or occlusion within the input data create a challenging reconstruction task (see Fig.~\ref{fig:cover}). Consequently, the patch prediction head often generates inaccurate reconstructions, introducing noise into the learning process and negatively impacting the performance of the primary tasks, i.e., FKD, IKD, and BC. This suggests that the auxiliary task of patch reconstruction, in this specific domain, may introduce a conflicting learning signal that hinders the model's ability to effectively learn the desired representations for the main objectives (Fig.~\ref{fig:patch_head}).


\textbf{MM vs NTP vs End-to-End} (End2End) are currently the prominent approaches in CV, NLP, and robotics respectively. However, it is unclear what is the best approach for robot learning, especially learning off-road mobility. We present a comparative analysis of model performance utilizing the MM paradigm within an encoder architecture (\coder, Fig.~\ref{fig:former} left trained alone with MM), a decoder employing autoregressive NTP (\vertidecoder, Fig.~\ref{fig:former} right trained alone without cross-attention), and a non-Transformer-based End2End approach. We then further contrast these approaches with \former,  which adopts a non-autoregressive approach to NTP and MM (Fig.~\ref{fig:former}, trained end-to-end). 

\begin{figure}[t]
  \centering
  \includegraphics[width=\columnwidth]{figures/4model_comparison}
  \caption{\textbf{MM vs NTP  vs End2End:} \former~achieves best accuracy across FKD, IKD, and BC compared to \coder~(MM), \vertidecoder~(NTP), and End2End.}
  \label{fig:mm_vs_ntp}
  % \vspace{-1.2em}
\end{figure}

To be specific, an encoder model leverages the principles of MM, wherein portions of the input sequence (poses, actions, and terrain patches) are masked, and the model is trained to reconstruct the masked elements. This approach has demonstrated success in capturing contextual dependencies and learning robust representations~\cite{nazeri2024vertiencoder};
A decoder model employs NTP, a prevalent technique in autoregressive sequence generation. In this paradigm, the model predicts the subsequent element in a sequence conditioned on the preceding elements. For both encoder and decoder models, we use the same unified latent space representation presented in Sec.~\ref{sec:unified}. The specialized non-Transformer-based End2End approach uses Resnet-18~\cite{he2015deep} as a patch encoder and fully connected layers as the task heads. While more complex models might offer higher accuracy, we choose ResNet-18 to balance performance with the computational constraints of our robotic platform, making it well-suited for deployment on robots with limited on-board processing capabilities, compared to deeper networks like ResNet-50 or ResNet-101. More information about End2End model architecture is provided in Appendix~\ref{app:architecture}.


As illustrated in Fig.~\ref{fig:mm_vs_ntp}, our findings indicate that \former, a non-autoregressive Transformer, exhibits superior performance across various evaluation metrics, including FKD, IKD, and BC error rates, in the context of one-second prediction horizon. Compared to \vertidecoder, \former~predicts multiple future states simultaneously (i.e., non-autoregressively), which contributes to its better accuracy. These results suggest that the enhanced contextual awareness afforded by the non-autoregressive approach contributes to improved predictive accuracy. Note that \vertidecoder~cannot perform BC directly, as it has access to both action and pose at each step. Unlike \coder~\cite{nazeri2024vertiencoder}, \former~does not train different downstream heads separately each time and all tasks contribute to the performance of each other all together, which results in \former's lowest error rate in most cases (except for $\mathbf{Z}$ prediction). 
Across all kinodynamics tasks, End2End achieves the highest error rate, which shows the benefits of using Transformers for kinodynamic representation and understanding during off-road mobility tasks. 

Beyond the observed performance gains and training stability, \former~demonstrates the capacity of concurrent execution of multiple tasks, not only during training but also during inference. This is particularly relevant in robotics, where real-time control is required and sometimes some modalities may not be available during inference. For example, without a global planner, action sampler, or in the presence of sensor degradation, the robot may not always have access to desired future robot poses, candidate actions, or future terrain patches, respectively. 
Furthermore, the usage of a learned mask within the decoder part of \former~is posited to capture salient distributional characteristics of the data, effectively serving as a condensed representation during inference. This learned representation facilitates adaptation to new tasks where action or pose is missing.


\section{Results}
\label{sec:result}
\begin{figure}[b]
\vspace{-0.5cm}
    \centering
    \includegraphics[width=\linewidth]{figures/re.pdf}
    \vspace{-0.5cm}
    \caption{Visualization of different domain adaptation methods performance of two specific target tasks:
    WMH segmentation on MRI images and liver segmentation on CT images, both using MambaUNet. The pixels highlighted in red represent \textit{incorrect predictions}.
    }
    \label{result}
    \vspace{-0.5cm}
\end{figure}

\section{Discussions}
\label{sec:implication}
\section{Discussions}

\subsection{The Asymmetry of the Point Spread Function (PSF) in Microscopy}

In the ideal imaging model, the Point Spread Function (PSF) of a microscope is symmetric with respect to the focal plane. 
This symmetry allows algorithms to estimate the absolute defocus distance but prevents them from determining whether the defocus is above or below the focal plane, thereby rendering one-shot autofocusing seemingly impractical.
However, in real optical microscopy systems, the PSF often exhibits significant asymmetry due to refractive index mismatches among the different media in the imaging path, such as the slide, sample, cover slip, and the surrounding medium like air or immersion oil. 
These mismatches introduce aberrations, including spherical aberration, coma, astigmatism, field curvature, and distortion, which disrupt the ideal symmetric distribution of the PSF. 
Figures~\ref{fig:psf}(a) and \ref{fig:psf}(b) present the 3D PSF and 2D PSF of our developed Whole Slide Imaging (WSI) device.
These visualizations are generated using the Gibson \& Lanni PSF model~\cite{Gibson:89} within the open-source software Fiji~\cite{Schindelin2012-jh} and the PSF Generator plugin\footnote{http://bigwww.epfl.ch/publications/kirshner1103.html}.
Figure~\ref{fig:psf}(c) shows images at symmetric defocus distances on both sides of the focal plane.
Figure~\ref{fig:psf}(d) illustrates the differences in pixel grayscale values at the same position for images at 5\si{\micro\meter} and -5\si{\micro\meter}.
These visualizations also demonstrate that, in real optical microscopy systems, the PSF is asymmetric.
Additional theoretical analysis on the PSF is provided in the supplementary materials.

\begin{figure}[H]
	\centering
	\includegraphics[width=\linewidth]{figs/PSF_fire_sq.pdf}
	\caption{\textbf{The asymmetry of PSF.} (a) and (b) illustrate the PSF of a microscopy imaging system, highlighting its asymmetry with respect to the focal plane. (c) demonstrates the defocused imaging relative to the focal plane, and (d) presents the comparison of the gray values of pixels at the same positions corresponding to 5\si{\micro\meter} and -5\si{\micro\meter}. Both of them provide corroborative evidence for the disparities in images at the corresponding locations.}
	\label{fig:psf}
\end{figure}

The asymmetry of the PSF, though potentially detrimental to image quality, presents a unique opportunity for one-shot autofocusing. This phenomenon results in images with positive or negative defocus on either side of the focal plane exhibiting distinct characteristics. Although these differences are subtle, the sophisticated feature extraction capabilities of deep learning can effectively discern them. By capitalizing on this physical principle, we propose a one-shot learning-based network designed to estimate both the defocus distance and direction from a single image.

\subsection{Autofocus for Thick Specimens}

Autofocus is generally designed for a specific focal plane, assuming that most samples exhibit little variation in elevation over a field of view.
However, for very thick samples, such as those resulting from the slicing of pathological sections, different regions within the same field of view may lie on different focal planes (see supplementary material). This can lead to a scenario where focusing on one region causes others to appear blurry, complicating autofocus efforts.
To address such challenges, strategies may contain: 1) Designate a specific region of interest for the autofocus algorithm to target exclusively; 2) Employ z-stack image fusion strategy, capturing and fusing images at various z-axis positions to achieve a uniformly sharp image across the entire field of view.




\section{Threats to Validity}
\label{sec:threats}
\section{Threats to Validity}
\label{threats}

\textbf{Construct Validity:} Primary threats involve our operationalization of key concepts. Our classification of major Ethereum events involves subjectivity, though mitigated by selecting widely acknowledged events with cross-domain impact. Using commit activity and issue resolution time as development metrics may not capture all contributions or team dynamics, while network analysis through shared comments might miss other collaboration forms. We address these limitations through multiple metrics and cross-repository validation. Repository selection bias is minimized by using objective criteria (activity levels, ecosystem roles) to ensure complete coverage of the Ethereum ecosystem.
To ensure broad impact, we define major events as those spanning at least two categories (infrastructure, market, or development trajectory). This criterion avoids overestimating the influence of isolated events. Our classification is based on historical records beyond our dataset, including Ethereum Foundation communications, network upgrades, market shifts, and security reports from 2014–2024. While a constructed definition, it provides consistency in assessing event significance. Future work could refine this approach by exploring alternative classification methods.


\textbf{Internal Validity:} Several factors could affect the relationship between events and observed developer activity. First, concurrent events or changes not included in our analysis might influence our results. We addressed this through our 90-day window analysis and by validating findings against random time periods.
Repository sizes vary significantly (from 678 to 24617 commits). However, our key findings and primary statistical inferences are drawn from the major repositories ($>15K$ commits): Go-ethereum, MetaMask, and Solidity. While we analyze medium ($5-15K$) and smaller ($<5K$) repositories for completeness, their results primarily serve to complement our main conclusions. This approach ensures our statistical inferences remain robust despite the size variations across the dataset.\\
\textbf{External Validity:} Our study focuses specifically on the Ethereum ecosystem during 2014–2024, examining how different types of events impact development patterns across its diverse repositories. While our findings provide insights into Ethereum's development dynamics, we acknowledge that these patterns are shaped by Ethereum's unique characteristics as a leading smart contract platform. Factors such as its decentralized governance, token incentives, and community-driven protocol upgrades differentiate Ethereum from traditional OSS projects.
Our repository selection spans multiple layers of the Ethereum ecosystem—from core infrastructure (Go-ethereum) to development tools (Hardhat, Truffle) and user-facing applications (MetaMask)—providing a broad view of development patterns within this blockchain platform. This diversity strengthens our findings regarding how different components of the Ethereum ecosystem respond to various types of events. However, we recognize that event-response patterns may differ in non-Ethereum projects, where governance models, incentive structures, and development workflows vary. 
Future research could explore whether similar trends hold across other blockchain and non-blockchain OSS ecosystems, particularly in projects with distinct governance mechanisms or without direct market exposure.\\ 
\textbf{Conclusion Validity:} We ensured statistical reliability by selecting appropriate tests based on data distributions, calculating effect sizes, and conducting multiple complementary analyses. Our survival and network analyses are based on assumptions of censoring independence and comment co-occurrence as a proxy for collaboration. To address multiple comparisons, we applied the Benjamini-Hochberg procedure to control the false discovery rate (FDR).  
Data completeness may be influenced by GitHub API limitations, particularly for older events. However, to support validation and reproducibility, we provide a complete replication package containing all data, code, and analysis scripts \cite{Vaccargiu2025}.




\section{Related Work}
\label{sec:related}

\section{Related Work} \label{sec:related}

% \textbf{Adversarial Attack}
\textbf{Attacks on SLAM.} 
%With the rise of machine learning, 
The robustness of computer vision systems is being actively investigated. With the emergence of adversarial images in the digital domain by adding optimized noise directly to images~\cite{szegedy2013intriguing,carlini2017towards}, researchers find that such attacks also exist physically in the real world \cite{eykholt2018robust,song2018physical,zhao2019seeing}. To fill the gap between attacks in the digital and physical worlds, recent studies have demonstrated that attacks on real-world computer vision systems are practical \cite{eykholt2018robust,li2019adversarial,man2020ghostimage,sharif2016accessorize,zhao2019seeing,zhou2018invisible}. However, attacks on traditional computer vision methods such as SLAM are relatively less explored. \cite{yoshida2022adversarial} proposes an attack against the scan matching algorithm in LiDAR-based SLAM, while most SLAMs in AR/VR devices rely on different sensors like RGB/depth cameras and IMUs. \cite{ikram2022perceptual} and \cite{chen2024adversary} mislead visual SLAM by poisoning the images with special patterns, and \cite{wang2021can} causes the camera to fail using infrared light. In our work, we demonstrate attacks on Visual-Inertial SLAM (VI-SLAM) by perturbing the IMU readings, rather than cameras, and showing its impact on XR user experience. 

\textbf{Acoustic Injection Attacks.} Among various physical attacks, acoustic injection attacks are attractive due to their low cost. Son~\etal~\cite{son2015rocking} were the first to introduce acoustic attacks on MEMS gyroscopes, demonstrating how these attacks could lead to sensor denial-of-service and result in drone crashes. WALNUT~\cite{trippel2017walnut} expanded on this by developing output biasing and control attacks that enable precise manipulation of MEMS accelerometer outputs using modulated sound waves. Wang et al.~\cite{wang2017sonic} demonstrated a sonic gun, showcasing the vulnerability of various smart devices (\eg drones and self-balancing vehicles) to acoustic attacks. Tu et al. \cite{tu2018injected} designed side-swing and switching attacks to alter the outputs of MEMS gyroscopes and accelerometers. Furthermore, Ji et al. \cite{ji2021poltergeist} fool the object detectors by applying acoustic attack to the image stabilizers commonly used in modern cameras. However, none of the existing works study the relationship between the acoustic injections and SLAM outputs on recent XR devices. 

% \zijian{Do we need one session about security in AR/VR?}
% \yicheng{TODO}
%\jiasi{cite the AIVR paper (UMass Amherst?) paper is we have not already. They add IMU perturbation but w/o SLAM, iirc} \yicheng{Cited}

\textbf{XR Security and Privacy.} 
%Security and privacy concerns in XR systems have gained significant attention. 
For single-user XR systems, researchers have demonstrated various side-channel attacks to extract sensitive information (\eg keystrokes) through video feeds~\cite{ling2019know}, head movements~\cite{nair2023unique, slocum2023going}, architectural hints~\cite{zhang2023its,shang2020arspy}, power usage~\cite{li2024dangers}, and EM side-channel leakages~\cite{al2021vr}. In multi-user XR systems, Su et al.~\cite{su2024remote} use avatar motion data to infer keystrokes in shared VR environments. Slocum et al.~\cite{slocum2024doesn} reveal vulnerabilities in the shared state frameworks of multi-user AR. Similarly, Lebeck et al.~\cite{lebeck2017securing} highlight risks like deceptive virtual objects and emphasize access control for managing shared physical and virtual spaces. Ruth et al.~\cite{ruth2019secure} further propose a secure multi-user AR framework focusing on content sharing and permissions.
Chandio et al.~\cite{chandio2024stealthy} %introduced a multi-modal spatiotemporal attack that 
simultaneously manipulated visual and inertial sensors to disrupt XR pose estimation. However, their study evaluated the attack using offline datasets and assumed the attacker's capability to manipulate IMU data streams through acoustic means, without real experiments. Ours is the first to demonstrate acoustic injection attacks on recent XR devices, like the Hololens 2, in the real world.
 




\section{\rev{Conclusion}}
\label{sec:conclusion}
\section*{Conclusion}
This paper aims to enhance our understanding of the computational complexity of computing various Shapley value variants. We found that for various ML models --- including decision trees, regression tree ensembles, weighted automata, and linear regression --- both local and global interventional and baseline SHAP can be computed in polynomial time under HMM modeled distributions. This extends popular algorithms, such as TreeSHAP, beyond their empirical distributional scope. We also establish strict complexity gaps between the various SHAP variants (baseline, interventional, and conditional) and prove the intractability of computing SHAP for tree ensembles and neural networks in simplified scenarios. Overall, we present SHAP as a versatile framework whose complexity depends on four key factors: \begin{inparaenum}[(i)] \item model type, \item SHAP variant, \item distribution modeling approach, \item and local vs. global explanations\end{inparaenum}. We believe this perspective provides deeper insight into the computational complexity of SHAP, paving the way for future work.




%We believe that our framework provides a more intricate understanding of SHAP computation complexity across different models, distributions, and variants, paving the way for further research.

Our work opens promising directions for future research. First, expanding our computational analysis to other SHAP-related metrics, such as asymmetric SHAP~\citep{frye20} and SAGE~\citep{covert2020understanding}, would be valuable. Additionally, we aim to explore more expressive distribution classes and relaxed assumptions beyond those in Section \ref{sec:tractable} while maintaining tractable SHAP computation. Finally, when exact computation is intractable (Section \ref{sec:intractable}), investigating the approximability of SHAP metrics through approximation and parameterized complexity theory~\citep{downey2012parameterized} is an important direction.

%Our work opens several promising avenues for future research on the computational properties of explainable AI methods, with a particular focus on SHAP. First, it would be interesting to broaden the computational analysis conducted in this work to include other popular SHAP-related metrics in the literature, such as asymmetric SHAP \cite{frye20} and SAGE \cite{covert2020understanding}. Also, in the future, we aim to explore more expressive distribution classes and relaxed distributional assumptions—extending beyond those examined in Section \ref{sec:tractable} —that still yield tractable SHAP computation. Finally, when exact computation proves intractable (Section \ref{sec:intractable}), it is worthwhile to theoretically investigate the question of the approximability of computing the SHAP metrics across various configurations, through the lens of approximation and parametrized complexity theory \cite{arora2009computational}.

%This paper aims to deepen our understanding of the computational complexity involved in obtaining different Shapley value variants. We found that for a variety of ML models, including decision trees, tree ensembles for regression, weighted automata, and linear regression models — computing both local and global interventional and baseline SHAP can be done in polynomial time when distributions are modeled by HMMs. This extends the distributional scope of popular algorithms like TreeSHAP, which is limited to empirical distributions. Additionally, we demonstrate a strict complexity gap between SHAP variants, showing that interventional and baseline SHAP can be strictly easier to compute than conditional SHAP. Despite these positive results, we uncovered intractability for various SHAP variants in neural networks and tree ensembles. Finally, we provided generalized complexity relations across SHAP variants. We believe that our framework offers a deeper understanding of the complexity involved in computing SHAP across various variants, models, distributions, as well as in both local and global computations, laying the groundwork for future research.

\section{Data Availability}
\label{sec:data}
The study dataset and scripts used for the analysis are available and documented in our online replication package~\cite{replicationpackage}.


\section{Acknowledgments}
\label{sec:ack}
\section*{Acknowledgements}
This is acknowledgment.


\balance
\bibliographystyle{IEEEtran}
\bibliography{main}


\end{document}
\typeout{get arXiv to do 4 passes: Label(s) may have changed. Rerun}
