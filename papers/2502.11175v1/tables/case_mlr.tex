\begin{table*}[ht]
\centering
\begin{tabularx}{\textwidth}{lX}
\hline
\textbf{Field} & \textbf{Value} \\
\hline
\textbf{query} & 영국 캐리비안에 언제 노예제가 폐지됐나요? (When was slavery abolished in the British Caribbean?) \\
\hline
\textbf{gold answer} & \textcolor{red}{1834-08-01} \\
\hline
\textbf{doc id} & kilt-100w\_6947054 (English) \\
\hline
\(\mathbf{r_d^{\text{init}}}\) & 34 \\
\hline
\(\mathbf{r_d^{\text{re-rank}}}\) & 2 \\
\hline
\textbf{d (content)} & \textit{History of the Caribbean. Empire remained slaves, however, until Britain passed the Slavery Abolition Act in 1833. 
When the Slavery Abolition Act came into force in 1834, roughly 700,000 slaves in the British West Indies immediately became free; 
other enslaved workers were freed several years later after a period of forced apprenticeship. 
Slavery was abolished in the Dutch Empire in 1814. 
Spain abolished slavery in its empire in 1811, with the exceptions of Cuba, Puerto Rico, and Santo Domingo; 
Spain ended the slave trade to these colonies in 1817, after being paid ₤400,000 by Britain. 
Slavery itself was not abolished in Cuba until 1886.} \\
\hline
\textbf{d (translated)} & \textit{1834년 노예제 폐지법이 시행되자, 영국 서인도 제도에서 약 700,000명의 노예가 즉시 해방되었고, 
다른 노예 노동자들은 강제 연습생 생활을 한 후 몇 년 후에 해방되었다. 
1814년 네덜란드 제국에서 노예제는 폐지되었다. 
1811년 스페인은 쿠바, 푸에르토리코, 산토 도밍고를 제외하고는 제국에서 노예제를 폐지했다. 
1817년 영국이 400만원을 지불한 후 스페인은 이들 식민지에서의 노예 무역을 종식시켰다. 
노예제는 1886년까지 쿠바에서 폐지되지 않았다.} \\
\hline
\end{tabularx}
\caption{An example of an improved MLR case. After translating the document into Korean, its rank improved from 34 to 2, illustrating language preference of retriever.}
\label{tab:case_mlr}
\end{table*}
