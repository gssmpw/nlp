\begin{table*}[ht]
\centering
\begin{tabularx}{\textwidth}{lX}
\hline
\textbf{Field} & \textbf{Value} \\
\hline
\(\displaystyle \text{query}\) & \textit{who is the first british governor in india?} \\
\hline
\(\displaystyle \text{gold answer}\) & William Bentinck, Warren Hastings \\
\hline
\(\displaystyle \text{doc id}\) & kilt-100w\_10189364 (English) \\
\hline
\(\displaystyle P_{\text{translated}}\) & \textit{Sir John Child, 1st Baronet. Sir John Child, 1st Baronet (died 1690) was a governor of Bombay, and de facto (although not officially) the first governor-general of the British settlements in India. Born in London, Child was sent as a young boy to his uncle, the chief of the factory at Rajapur. On 27 October 1681, he was appointed chief of the East India Company's affairs at Surat and Bombay, while at the same time his namesake—stated to be unrelated by the Oxford Dictionary of National Biography—Sir Josiah Child, was governor of the company at home. The two} \\
\hline
\(\displaystyle P_{\text{refined}}\) & \textit{Sir John Child, 1st Baronet was the first governor of Bombay, and de facto (although not officially) the first governor-general of the British settlements in India. He was born in London but later moved to India where he became involved in the East India Company’s affairs. On 27th October 1681, he took up the position of Chief of the East India Company’s affairs at both Surat and Bombay. This made him the first British Governor in \ldots} \\
\hline
\end{tabularx}
\caption{A failure case of DKM-RAG because of preference of retriever so that high-resource but irrelevant document is retrieved. }
\label{tab:failure_dkmrag}
\end{table*}
