\clearpage
\appendix

\section{Implementation Details}
When retrieving from the datastore in all languages, we utilize the approach outlined in \cite{chirkova-etal-2024-retrieval} as our baseline. Specifically, we employ the basic\_translated\_langspec prompt template, as detailed in Table~\ref{tab:system_prompts} to generate our final mRAG answer from the generator. In our method, we retrieve and re-rank the top-50 documents for each query, and then use only the top-5 documents to generate the final answer. The document retrieval and re-ranking are carried out using bge-m3. We do not translate documents already in query language in the framework of DKM-RAG to reduce costs.

We conduct our experiments using an AMD EPYC 7313 CPU (3.0 GHz) paired with four NVIDIA RTX 4090 GPUs. We use Python 3.11.5 and PyTorch 2.3.1 for the software environment.
\begin{table}[ht]
\centering
\small
\renewcommand{\arraystretch}{1.2}
\setlength{\tabcolsep}{8pt}
\begin{tabular}{lcc}
\toprule
\textbf{Language} & \textbf{Passage Count (M)} & \textbf{Percentage (\%)} \\
\midrule
ja & 27   & 20.53 \\
en & 25   & 19.00 \\
de & 14   & 10.64 \\
fr & 13   & 9.88  \\
zh & 11   & 8.36  \\
es & 10   & 7.60  \\
ru & 8.6  & 6.54  \\
it & 8.2  & 6.23  \\
pt & 4.7  & 3.57  \\
th & 3.7  & 2.81  \\
ar & 3.3  & 2.51  \\
ko & 1.6  & 1.22  \\
fi & 1.5  & 1.14  \\
\bottomrule
\end{tabular}
\caption{Language distribution of wikipedia we use in our experiment.}
\label{tab:wiki_ratio}
\end{table}

\section{Prompts}
\label{appendix:prompts}
As shown in Table~\ref{tab:system_prompts}, we provide the prompts used to generate our final answer with the retrieved documents in our mRAG baseline. \textit{Docs} refers to retrieved documents and question refers to the current query. We also provide prompts during the passage rewriting phase in the DKM-RAG framework as stated in Table~\ref{tab:prompt_dkm}. We only provide english prompts for simplicity. And we provide prompts to measure the language preference of GPT-4o-mini, regarding answering in the specific languages as stated in Table~\ref{tab:prompt_generation}.

\begin{table*}[ht]
\centering
\renewcommand{\arraystretch}{1.3}
\setlength{\tabcolsep}{8pt}
\begin{tabular}{ll}
\toprule
\textbf{System}            & \textbf{Prompt} \\ 
\midrule
With Documents             & 
\begin{tabular}[t]{@{}l@{}} 
\texttt{You are a helpful assistant. Your task is to extract relevant}\\ 
\texttt{information from provided documents and to answer to}\\ 
\texttt{questions as short as possible. Please reply in English.} \\ 
\texttt{user: f"Background:\{docs\}\textbackslash n\textbackslash nQuestion:\{question\}"}
\end{tabular} \\ 
\midrule
Without Documents          & 
\begin{tabular}[t]{@{}l@{}} 
\texttt{You are a helpful assistant. Answer the questions as short}\\ 
\texttt{as possible. Please reply in English.}\\ 
\texttt{user: f"Question:\{question\}"}
\end{tabular} \\
\bottomrule
\end{tabular}
\caption{System prompts with and without documents. The table outlines how instructions and prompts differ when documents are provided or omitted.}
\label{tab:system_prompts}
\end{table*}
\begin{table*}[ht]
\centering
\renewcommand{\arraystretch}{1.3}
\setlength{\tabcolsep}{6pt}
\begin{tabular}{p{0.9\linewidth}}
\toprule
\textbf{Prompt} \\
\midrule
Original Passage: \{passage\} \\[1ex]
Question: \{question\} \\[1ex]
Please create an independent document according to the following requirements: \\[1ex]
1) Utilize known facts (parametric knowledge) related to the question. \\[1ex]
2) Seamlessly combine with the original passage by removing redundant or unnecessary sentences. No additional explanations are allowed. \\[1ex]
3) All content must be written smoothly and concisely in English. \\
\bottomrule

\end{tabular}
\caption{The prompt used for generating $P_\text{refined}$ based on the passage and question. The instructions guide the generator to combine parametric knowledge with the original passage while ensuring clarity and conciseness.}
\label{tab:prompt_dkm}
\end{table*}


\begin{table*}[ht]
\centering
\begin{tabular}{|p{3cm}|p{13cm}|}
\hline
 & \textbf{Content} \\
\hline
\textbf{System Message} &
\begin{minipage}[t]{\linewidth}
\texttt{You are a highly capable multilingual assistant. \\ Here are some reference documents:}
\texttt{\ \ \ \ \{top5\_passages\}}\\[0.5em]
\texttt{The user wants answers in multiple languages. \\ Please follow these rules strictly:}\\[0.5em]
\texttt{1) Return your final answer as a valid JSON object.}\\[0.5em]
\texttt{2) The JSON object must contain exactly these keys: \{TARGET\_LANGUAGES\}.}\\[0.5em]
\texttt{3) Each field's value must be the answer written in that respective language.}\\[0.5em]
\texttt{4) Do not include any additional text outside the JSON (e.g., no Markdown or explanations).}\\[0.5em]
\texttt{5) Ensure it is valid JSON with correct format.}
\end{minipage} \\
\hline
\textbf{User Message} &
\begin{minipage}[t]{\linewidth}
\texttt{Question: \{question\}}\\[0.5em]
\texttt{Please provide the answers in JSON form for each of the following languages: \{TARGET\_LANGUAGES\}.}
\end{minipage} \\
\hline
\end{tabular}
%\vspace{-1mm}
\caption{Prompts used for measuring language preference of GPT-4o-mini in mRAG pipeline.}
\label{tab:prompt_generation}
\end{table*}


\section{Language Notation}
In this work, we use standard ISO 639-1 language codes to represent the various languages involved in our experiments. Specifically, en denotes English, ko represents Korean, ar corresponds to Arabic, zh refers to Chinese (Simplified), fi indicates Finnish, fr stands for French, de represents German, ja corresponds to Japanese, it refers to Italian, pt denotes Portuguese, ru stands for Russian, es represents Spanish, and th corresponds to Thai. These concise notations facilitate the identification and processing of language-specific data across datasets and models in multilingual NLP research.


\section{Dataset Statistics}

We present the statistics of the datasets used in our experiments. MKQA serves as the primary dataset, and its details, including the number of examples and the median lengths of questions and answers, are summarized in Table~\ref{tab:data_statistics}. Additionally, we utilize Wikipedia as the external source for the retriever datastore, with its statistics (number of passages and median lengths) also provided in Table~\ref{tab:data_statistics}. And we provide the number of passages in each language and the ratio of them in Table~\ref{tab:wiki_ratio}. These details offer a clear overview of the data resources supporting our experiments.

\paragraph{Language Distribution of Pre-trained LLM}
We provide language distribution in the pre-training corpus of Llama-2. As stated in Table~\ref{tab:llama_distribution}, we use English (EN) as a high-resource, Spanish (ES) as a mid-resource, and Korean (KO) as a low-resource language in our experiment based on their ratios.
\begin{table*}[ht]
\centering
\begin{tabular}{l r}
\hline
\textbf{Language} & \textbf{Percentage} \\
\hline
\textbf{EN}       & \textbf{89.70}\% \\
Unknown  & 8.38\%  \\
DE       & 0.17\%  \\
FR       & 0.15\%  \\
SV       & 0.15\%  \\
\textbf{ES}       & \textbf{0.13}\%  \\
ZH       & 0.15\%  \\
RU       & 0.12\%  \\
NL       & 0.11\%  \\
IT       & 0.11\%  \\
JP       & 0.11\%  \\
PL       & 0.09\%  \\
PT       & 0.09\%  \\
VI       & 0.08\%  \\
RO       & 0.03\%  \\
SR       & 0.04\%  \\
CA       & 0.04\%  \\
\textbf{KO}       & \textbf{0.06}\%  \\
UK       & 0.07\%  \\
Other    & 0.21\%  \\
\hline
\end{tabular}
\caption{Language distribution in the pre-training corpus of Llama-2. Unknown represents languages we cannot know because of closed-source access of model and other denotes other languages.}
\label{tab:llama_distribution}
\end{table*}





\label{appendix:statistics}



% \begin{figure*}[htbp]
%     % 左侧图片
%     \begin{minipage}{0.77\linewidth}  % 调整宽度
%         \centering
%         \includegraphics[width=\linewidth]{images/benchmark_construction.pdf}
%     \end{minipage}%
%     % 间隔
%     \hfill
%     % 右侧表格
%     \begin{minipage}{0.23\linewidth}  % 调整宽度
%         \centering
%         \resizebox{\linewidth}{!}{  % 调整表格至合适的宽度
%             \begin{tabular}{lcc}
%                 \toprule
%                 \textbf{Statistic} & \textbf{Number} \\
%                 \midrule
%                 \rowcolor[HTML]{F2F2F2} 
%                 \textit{Domain Count} &  \\
%                 \midrule
%                 Domain & 103 \\
%                 Requirement & 8 \\
%                 \midrule
%                 \rowcolor[HTML]{F2F2F2} 
%                 \textit{Token Count} &  \\
%                 \midrule
%                 Description & 851.6 $\pm$ 515.2 \\
%                 - Min/Max & [159, 2814] \\
%                 Domain & 1187.2 $\pm$ 1212.1 \\
%                 - Min/Max & [85, 7514] \\
%                 \midrule
%                 \rowcolor[HTML]{F2F2F2} 
%                 \textit{Line Count} &  \\
%                 \midrule
%                 Domain & 75.4 $\pm$ 62.9 \\
%                 - Min/Max & [9, 394] \\
%                 \midrule
%                 \rowcolor[HTML]{F2F2F2} 
%                 \textit{Component Count} &  \\
%                 \midrule
%                 Actions & 4.5 $\pm$ 2.8 \\
%                 - Min/Max & [1, 16] \\
%                 Predicates & 8.1 $\pm$ 4.8 \\
%                 - Min/Max & [1, 25] \\
%                 Types & 1.1 $\pm$ 1.3 \\
%                 - Min/Max & [1, 8] \\
%                 \bottomrule
%             \end{tabular}
%         }
%     \end{minipage}
%     % 公共标题
%     \caption{Dataset construction process (left) and key statistics (right) of the \texttt{\benchmark} dataset.     Dataset construction process including: (a) \textit{Data Acquisition} (\S\ref{sec:data_acquisition}); (b) \textit{Data Filtering and Manual Selection} (\S\ref{sec:data_filtering}); (c) \textit{Data Annotation and Quality Assurance}(\S\ref{sec:data_annotation} and \S\ref{sec:quality_assurance}). Tokens are counted by GPT-2~\cite{openai2019gpt2} tokenizer.}
%     \label{fig:combined}
% \end{figure*}


\section{Language Preference of Other Languages}
We also perform additional experiments to explore language preferences for languages not covered in Table~\ref{tab:subset_mlr}, using the MLR score that we propose. As shown in Table~\ref{tab:appendix_removed_langs}, similar to the results in Table~\ref{tab:subset_mlr}, the highest preferences are typically observed when $L_q = L_d$ across all query languages. English is also the most preferred language. For clarity, we omit results for other languages.

For most languages, such as Arabic, Finnish, German, and Russian, switching to a cross-lingual setup leads to a significant drop in MLR. For example, Arabic queries using the bge-m3 encoder achieve a monolingual score of 40.39, but cross-lingual retrieval (e.g., with Thai) results in a 6.80-point decrease.

Interestingly, for Thai queries, some cross-lingual pairs show a slight improvement over the monolingual baseline (as indicated by the positive differences in red), suggesting that for low-resource languages like Thai, cross-lingual signals might sometimes offer complementary benefits

% CVPR 2025 Paper Template; see https://github.com/cvpr-org/author-kit

\documentclass[10pt,twocolumn,letterpaper]{article}

%%%%%%%%% PAPER TYPE  - PLEASE UPDATE FOR FINAL VERSION
% \usepackage{cvpr}              % To produce the CAMERA-READY version
% \usepackage[review]{cvpr}      % To produce the REVIEW version
\usepackage[pagenumbers]{cvpr} % To force page numbers, e.g. for an arXiv version

% Import additional packages in the preamble file, before hyperref
%
% --- inline annotations
%
\newcommand{\red}[1]{{\color{red}#1}}
\newcommand{\todo}[1]{{\color{red}#1}}
\newcommand{\TODO}[1]{\textbf{\color{red}[TODO: #1]}}
% --- disable by uncommenting  
% \renewcommand{\TODO}[1]{}
% \renewcommand{\todo}[1]{#1}



\newcommand{\VLM}{LVLM\xspace} 
\newcommand{\ours}{PeKit\xspace}
\newcommand{\yollava}{Yo’LLaVA\xspace}

\newcommand{\thisismy}{This-Is-My-Img\xspace}
\newcommand{\myparagraph}[1]{\noindent\textbf{#1}}
\newcommand{\vdoro}[1]{{\color[rgb]{0.4, 0.18, 0.78} {[V] #1}}}
% --- disable by uncommenting  
% \renewcommand{\TODO}[1]{}
% \renewcommand{\todo}[1]{#1}
\usepackage{slashbox}
% Vectors
\newcommand{\bB}{\mathcal{B}}
\newcommand{\bw}{\mathbf{w}}
\newcommand{\bs}{\mathbf{s}}
\newcommand{\bo}{\mathbf{o}}
\newcommand{\bn}{\mathbf{n}}
\newcommand{\bc}{\mathbf{c}}
\newcommand{\bp}{\mathbf{p}}
\newcommand{\bS}{\mathbf{S}}
\newcommand{\bk}{\mathbf{k}}
\newcommand{\bmu}{\boldsymbol{\mu}}
\newcommand{\bx}{\mathbf{x}}
\newcommand{\bg}{\mathbf{g}}
\newcommand{\be}{\mathbf{e}}
\newcommand{\bX}{\mathbf{X}}
\newcommand{\by}{\mathbf{y}}
\newcommand{\bv}{\mathbf{v}}
\newcommand{\bz}{\mathbf{z}}
\newcommand{\bq}{\mathbf{q}}
\newcommand{\bff}{\mathbf{f}}
\newcommand{\bu}{\mathbf{u}}
\newcommand{\bh}{\mathbf{h}}
\newcommand{\bb}{\mathbf{b}}

\newcommand{\rone}{\textcolor{green}{R1}}
\newcommand{\rtwo}{\textcolor{orange}{R2}}
\newcommand{\rthree}{\textcolor{red}{R3}}
\usepackage{amsmath}
%\usepackage{arydshln}
\DeclareMathOperator{\similarity}{sim}
\DeclareMathOperator{\AvgPool}{AvgPool}

\newcommand{\argmax}{\mathop{\mathrm{argmax}}}     



% It is strongly recommended to use hyperref, especially for the review version.
% hyperref with option pagebackref eases the reviewers' job.
% Please disable hyperref *only* if you encounter grave issues, 
% e.g. with the file validation for the camera-ready version.
%
% If you comment hyperref and then uncomment it, you should delete *.aux before re-running LaTeX.
% (Or just hit 'q' on the first LaTeX run, let it finish, and you should be clear).
\definecolor{cvprblue}{rgb}{0.21,0.49,0.74}
\usepackage[pagebackref,breaklinks,colorlinks,allcolors=cvprblue]{hyperref}

% DK
\usepackage{xcolor}
\usepackage{multirow}
% JH
\usepackage{adjustbox}

%%%%%%%%% PAPER ID  - PLEASE UPDATE
\def\paperID{16035} % *** Enter the Paper ID here
\def\confName{CVPR}
\def\confYear{2025}

%%%%%%%%% TITLE - PLEASE UPDATE
% \title{\LaTeX\ Author Guidelines for \confName~Proceedings}
% \title{360 Indoor Reconstruction}
\title{IM360: Textured Mesh Reconstruction for Large-scale Indoor Mapping with 360\textdegree\ Cameras}

%%%%%%%%% AUTHORS - PLEASE UPDATE
\author{First Author\\
Institution1\\
Institution1 address\\
{\tt\small firstauthor@i1.org}
% For a paper whose authors are all at the same institution,
% omit the following lines up until the closing ``}''.
% Additional authors and addresses can be added with ``\and'',
% just like the second author.
% To save space, use either the email address or home page, not both
\and
Second Author\\
Institution2\\
First line of institution2 address\\
{\tt\small secondauthor@i2.org}
}

\begin{document}

\clearpage
\setcounter{page}{1}
\maketitlesupplementary

In this supplementary material, we provide additional qualitative results, highlighting the superior performance of our method compared to other approaches.
Appendix \ref{sphericalsfm} highlights the advantages of the spherical camera model and the dense matching algorithm for indoor reconstruction. Appendix \ref{texture} demonstrates the effectiveness of our novel texturing method.

\section{Spherical Structure from Motion}
\label{sphericalsfm}
Due to page limitations, we present our spherical structure from motion results in the supplementary material: 1) \textbf{OpenMVG:} An open-source SfM pipeline that supports spherical camera models \cite{moulon2017openmvg}.
2) \textbf{SPSG COLMAP:} SuperPoint \cite{detone2018superpoint} and SuperGlue \cite{sarlin2020superglue} are used with cubemap and equirectangular projection.
3) \textbf{DKM COLMAP:} This method leverages DKM \cite{edstedt2023dkm} to establish dense correspondences, utilizing cubemap and equirectangular projection.
4) \textbf{SphereGlue COLMAP:} SuperPoint \cite{detone2018superpoint} with a local planar approximation \cite{eder2020tangent} and SphereGlue \cite{gava2023sphereglue} are utilized to mitigate distortion in ERP images.
The experimental results discussed in the main paper for Matterport3D \cite{chang2017matterport3d} and Stanford2D3D \cite{Stanford2d3d} are shown in Fig. \ref{fig:sfm_mp3d} and Fig. \ref{fig:sfm_stfd}, respectively.

\section{Texture Map Optimization}
\label{texture}
We compare our method with several recent rendering approaches, including \textbf{TexRecon} \cite{waechter2014TexRecon}, \textbf{SparseGS} \cite{xiong2023sparsegs}, and \textbf{ZipNeRF} \cite{barron2023zip}. 
Our method outperforms these approaches by delivering higher frequency details and producing seamless texture maps.
The results of the textured mesh and rendering are shown in Fig. \ref{fig:textured_mesh} and Fig. \ref{fig:render1} - \ref{fig:render4}.

% \section{Rationale}
% \label{sec:rationale}
% % 
% Having the supplementary compiled together with the main paper means that:
% % 
% \begin{itemize}
% \item The supplementary can back-reference sections of the main paper, for example, we can refer to \cref{sec:intro};
% \item The main paper can forward reference sub-sections within the supplementary explicitly (e.g. referring to a particular experiment); 
% \item When submitted to arXiv, the supplementary will already included at the end of the paper.
% \end{itemize}
% % 
% To split the supplementary pages from the main paper, you can use \href{https://support.apple.com/en-ca/guide/preview/prvw11793/mac#:~:text=Delete%20a%20page%20from%20a,or%20choose%20Edit%20%3E%20Delete).}{Preview (on macOS)}, \href{https://www.adobe.com/acrobat/how-to/delete-pages-from-pdf.html#:~:text=Choose%20%E2%80%9CTools%E2%80%9D%20%3E%20%E2%80%9COrganize,or%20pages%20from%20the%20file.}{Adobe Acrobat} (on all OSs), as well as \href{https://superuser.com/questions/517986/is-it-possible-to-delete-some-pages-of-a-pdf-document}{command line tools}.

\begin{figure*}[t]
    \centering
    \includegraphics[width=1.0\linewidth]{figures_sup/sup_sfm.pdf}
    \caption{Qualitative Comparison of SfM results on Matterport3D.
    While other approaches failed to achieve pose registration, our method successfully estimates poses by leveraging the spherical camera model and dense matching.
    }
    \label{fig:sfm_mp3d}
\end{figure*}

\begin{figure*}[t]
    \centering
    \includegraphics[width=1.0\linewidth]{figures_sup/sup_sfm_stfd.pdf}
    \caption{Qualitative Comparison of SfM results on Stanford2D3D.
    While other approaches failed to achieve pose registration, our method successfully estimates poses by leveraging the spherical camera model and dense matching.}
    \label{fig:sfm_stfd}
\end{figure*}

\begin{figure*}[t]
    \centering
    \includegraphics[width=1.0\linewidth]{figures_sup/texture_map.pdf}
    \caption{Qualitative Comparisons of Textured Mesh Results on Matterport3D.
    A comparison between TexRecon \cite{waechter2014TexRecon} and ours shows that our method effectively reduces noise in the texture maps, leading to improved visual quality and detail.}
    \label{fig:textured_mesh}
\end{figure*}


% % \maketitle
% % \clearpage
% \setcounter{page}{1}
% \maketitlesupplementary
\begin{figure*}[t]
    \centering
    \includegraphics[width=1.0\linewidth]{figures_sup/render_comparison_supp1.pdf}
    \caption{Qualitative Comparisons with Existing Methods. Our method can render high frequency details and results in lower noise.}
    \label{fig:render1}
\end{figure*}

\begin{figure*}[t]
    \centering
    \includegraphics[width=1.0\linewidth]{figures_sup/render_comparison_supp2.pdf}
    \caption{Qualitative Comparisons with Existing Methods. Our method can render high frequency details and results in lower noise.}
    \label{fig:render2}
\end{figure*}

\begin{figure*}[t]
    \centering
    \includegraphics[width=1.0\linewidth]{figures_sup/render_comparison_supp3.pdf}
    \caption{Qualitative Comparisons with Existing Methods. Our method can render high frequency details and results in lower noise.}
    \label{fig:render3}
\end{figure*}

\begin{figure*}[t]
    \centering
    \includegraphics[width=1.0\linewidth]{figures_sup/render_comparison_supp4.pdf}
    \caption{Qualitative Comparisons with Existing Methods. Our method can render high frequency details and results in lower noise.}
    \label{fig:render4}
\end{figure*}



{
    \small
    \bibliographystyle{ieeenat_fullname}
    \bibliography{main}
}


\end{document}

\section{Similarity Matrices}
We provide similarity matrix measured by LaBSE for each query language en, zh, ko and each generator in Figure~\ref{fig:aya_lang_pref_en}, Figure~\ref{fig:aya_lang_pref_zh}, Figure~\ref{fig:aya_lang_pref_ko}, Figure~\ref{fig:llama_lang_pref_en}, Figure~\ref{fig:llama_lang_pref_zh}, Figure~\ref{fig:llama_lang_pref_ko}, Figure~\ref{fig:gpt_lang_pref_en}, Figure~\ref{fig:gpt_lang_pref_zh} and Figure~\ref{fig:gpt_lang_pref_ko}. Each entry represents the embedding similarity score between answers generated in different languages, with the diagonal values all equal to 1 (i.e., comparing an answer with itself). Moreover, the values shown in Figure~\ref{fig:multi_lang} are computed by averaging over the rows or columns for each language.


\section{Case study}
\paragraph{MLR}
We provide an example of a document that improved MLR score, where the rank of a relevant document significantly increases after translation. In Table~\ref{tab:case_mlr}, the user query \textit{"영국 캐리비안에 언제 노예제가 폐지됐나요? (When was slavery abolished in the British Caribbean?)"} is in Korean, whereas the original passage is in English. Initially, the document’s rank (\(\mathbf{r_d^{\text{init}}} = 34\)) was relatively low, but after translating the passage into Korean and re-ranking (\(\mathbf{r_d^{\text{re-rank}}} = 2\)), the document moved much closer to the top. 
This demonstrates how cross-lingual alignment can substantially improve retrieval performance in a multilingual setting.
Notably, even if the passage content is semantically the same, language preference in the model can lead to poor alignment when the query and document are in different languages, adversely affecting retrieval. Translating the document into the query language effectively mitigates this issue.
\begin{table*}[ht]
\centering
\begin{tabularx}{\textwidth}{lX}
\hline
\textbf{Field} & \textbf{Value} \\
\hline
\textbf{query} & 영국 캐리비안에 언제 노예제가 폐지됐나요? (When was slavery abolished in the British Caribbean?) \\
\hline
\textbf{gold answer} & \textcolor{red}{1834-08-01} \\
\hline
\textbf{doc id} & kilt-100w\_6947054 (English) \\
\hline
\(\mathbf{r_d^{\text{init}}}\) & 34 \\
\hline
\(\mathbf{r_d^{\text{re-rank}}}\) & 2 \\
\hline
\textbf{d (content)} & \textit{History of the Caribbean. Empire remained slaves, however, until Britain passed the Slavery Abolition Act in 1833. 
When the Slavery Abolition Act came into force in 1834, roughly 700,000 slaves in the British West Indies immediately became free; 
other enslaved workers were freed several years later after a period of forced apprenticeship. 
Slavery was abolished in the Dutch Empire in 1814. 
Spain abolished slavery in its empire in 1811, with the exceptions of Cuba, Puerto Rico, and Santo Domingo; 
Spain ended the slave trade to these colonies in 1817, after being paid ₤400,000 by Britain. 
Slavery itself was not abolished in Cuba until 1886.} \\
\hline
\textbf{d (translated)} & \textit{1834년 노예제 폐지법이 시행되자, 영국 서인도 제도에서 약 700,000명의 노예가 즉시 해방되었고, 
다른 노예 노동자들은 강제 연습생 생활을 한 후 몇 년 후에 해방되었다. 
1814년 네덜란드 제국에서 노예제는 폐지되었다. 
1811년 스페인은 쿠바, 푸에르토리코, 산토 도밍고를 제외하고는 제국에서 노예제를 폐지했다. 
1817년 영국이 400만원을 지불한 후 스페인은 이들 식민지에서의 노예 무역을 종식시켰다. 
노예제는 1886년까지 쿠바에서 폐지되지 않았다.} \\
\hline
\end{tabularx}
\caption{An example of an improved MLR case. After translating the document into Korean, its rank improved from 34 to 2, illustrating language preference of retriever.}
\label{tab:case_mlr}
\end{table*}


\paragraph{Answer Generation in Language Preference of Generator}


\begin{table*}[ht]
\centering
% Metadata 부분
\begin{tabular}{ll}
\toprule
\textbf{Question} & which type of air pressure is associated with warm air rising \\
\bottomrule
\end{tabular}

%\vspace{1em}

\begin{tabular}{l p{10cm} c}
\toprule
\textbf{Language} & \textbf{Answer} & \textbf{Preference Score} \\
\midrule
en & Low pressure is associated with warm air rising. & 0.9179 \\
ko & 따뜻한 공기가 상승하는 것과 관련된 공기 압력은 저압입니다. & 0.9060 \\
zh & 与暖空气上升相关的空气压力是低压。 & 0.8853 \\
fr & La pression basse est associée à l'ascension de l'air chaud. & 0.9231 \\
ja & 暖かい空気が上昇することに関連する気圧は低圧です。 & 0.9187 \\
it & La bassa pressione è associata all'aria calda che sale. & 0.9256 \\
pt & A pressão baixa está associada ao ar quente que sobe. & 0.9316 \\
es & La presión baja está asociada con el aire caliente que asciende. & 0.9317 \\
\bottomrule
\end{tabular}
\caption{An example of generated answers in different languages with gpt-4o-mini. Also, we report the average similarity score between each pair of answers.}
\label{tab:case_generation}
\end{table*}

We also provide an example of generated answers in different languages with a generator, GPT-4o-mini as shown in Table~\ref{tab:case_generation}. The preference score in the rightmost column of Table~\ref{tab:case_generation} indicates that the generator prefers the query language and Latin-script languages over other languages.



\paragraph{Unified Document of DKM-RAG}
\begin{table*}[ht]
\centering
\begin{tabularx}{\textwidth}{lX}
\hline
\textbf{Field} & \textbf{Value}\\
\hline
\(\displaystyle \text{query}\) & \textit{in the united states the president is the head of which branch of government?} \\
\hline
\(\displaystyle \text{gold answer}\) 
& \textcolor{red}{the executive branch} \\
\hline
\(\displaystyle \text{doc id}\) 
& kilt-100w\_5089743 (English)\\
\hline
\(\displaystyle P_{\text{translated}}\) 
& \textit{President of the United States. President of the United States (POTUS) 
is the head of state and head of government of the United States of America. 
The president is the commander-in-chief of the United States Armed Forces. 
In contemporary times, the president is looked upon as one of the world's 
most powerful political figures as the leader of the only remaining global superpower. The role includes responsibility for the world's most expensive military, which has the second largest nuclear arsenal. The president also leads the nation with the largest economy. } \\
\hline
\(\displaystyle P_{\text{refined}}\) 
& \textit{The president of the United States is the head of the 
\textcolor{red}{executive branch} of the federal government. 
The president directs the \textcolor{red}{executive branch} 
and is the commander-in-chief of the United States Armed Forces/ In contemporary times, the president is looked upon as one of the world's most powerful political figures as the leader of the only remaining global superpower. The role includes responsibility for the world's most expensive military, which has the second-largest nuclear arsenal. The president also leads the nation with the largest economy.} \\
\hline
\end{tabularx}
\caption{A DKM-RAG case study illustrating how \(\displaystyle P_{\text{translated}}\) and \(\displaystyle P_{\text{refined}}\) 
correspond to the retrieved passage (translated into the query language) and the rewritten passage leveraging parametric knowledge, respectively. The overlap with the gold answer is highlighted in \textcolor{red}{red}.}
\label{tab:case_dkm}
\end{table*}

Additionally, we provide a sample of \(\displaystyle P_{\text{translated}}\) and \(\displaystyle P_{\text{refined}}\) obtained via our proposed DKM-RAG framework in Table~\ref{tab:case_dkm}. 
This example illustrates how the crucial answer component, \textit{``the executive branch''}, which is not apparent from the translated passage alone, emerges through the model’s internal knowledge. 
Consequently, this shows that DKM-RAG can effectively leverage additional knowledge sources that is not included in the translated passage to achieve better performance.

%\vspace{-1mm}
\subsection{Failure Case}
\paragraph{MLR}
We present a failure case of the MLR metric in Table~\ref{tab:failure_mlr}. Due to the difficulty of translating documents in low-resource languages, repetitive phrases such as \textit{Changing the line-up} appear in the translated passage. This repetition causes the re-ranker to misinterpret the content, leading to an improvement in the rank even though the content is irrelevant.
\begin{table*}[ht]
\centering
\begin{tabularx}{\textwidth}{lX}
\hline
\textbf{Field} & \textbf{Value} \\
\hline
\textbf{query} & \textit{연속으로 가장 많은 자유투 기록 (who holds the record for most free throws made in a row)} \\
\hline
\textbf{gold answer} & \textcolor{red}{톰 앰베리} \\
\hline
\textbf{doc id} & wiki-100w-ja\_8993041 \\
\hline
\(\mathbf{r_d^{\text{init}}}\) & 31 \\
\hline
\(\mathbf{r_d^{\text{re-rank}}}\) & 5 \\
\hline
\textbf{d (content)} & \textit{'林直明. を 変 更 \textbackslash n \textbackslash n 記 録 \textbackslash n \quad イ ニ ン グ 最 多 連 続 与 四 球 : 5 ( 日 本 記 録 ) \quad 1 9 4 6 年 4 月 2 9 日 \textbackslash n \quad 同 一 年 に 2 球 団 で 勝 利 : 1 9 4 8 年 \quad ※ 史 上 3 人 目 \textbackslash n \quad ゲ ー ム 最 多 失 点 : 1 4 ( セ ・ リ ー グ 記 録 ) \quad 1 9 5 0 年 6 月 7 日 \textbackslash n \textbackslash n 背 番 号'} \\
\hline
\textbf{d (translated)} & \textit{Changing the line-up, Changing the line-up, Changing the line-up, Changing the line-up, Changing the line-up, Changing the line-up, Changing the line-up, Changing the line-up, Changing the line-up, Changing the line-up, Changing the line-up, Changing the line-up, Changing the line-up, Changing the line-up, Changing the line-up, (...)} \\
\hline
\end{tabularx}
\caption{A failure case of MLR because bad translation quality due to difficulty in translating low-resource language.}
\label{tab:failure_mlr}
\end{table*}

\paragraph{DKM-RAG}
We also present a failure case of DKM-RAG in Table~\ref{tab:failure_dkmrag}. The retriever retrieves an English document that is irrelevant to the query due to its language preference. Additionally, the LLM lacks relevant knowledge related to the query, resulting in a failed generation.
\begin{table*}[ht]
\centering
\begin{tabularx}{\textwidth}{lX}
\hline
\textbf{Field} & \textbf{Value} \\
\hline
\(\displaystyle \text{query}\) & \textit{who is the first british governor in india?} \\
\hline
\(\displaystyle \text{gold answer}\) & William Bentinck, Warren Hastings \\
\hline
\(\displaystyle \text{doc id}\) & kilt-100w\_10189364 (English) \\
\hline
\(\displaystyle P_{\text{translated}}\) & \textit{Sir John Child, 1st Baronet. Sir John Child, 1st Baronet (died 1690) was a governor of Bombay, and de facto (although not officially) the first governor-general of the British settlements in India. Born in London, Child was sent as a young boy to his uncle, the chief of the factory at Rajapur. On 27 October 1681, he was appointed chief of the East India Company's affairs at Surat and Bombay, while at the same time his namesake—stated to be unrelated by the Oxford Dictionary of National Biography—Sir Josiah Child, was governor of the company at home. The two} \\
\hline
\(\displaystyle P_{\text{refined}}\) & \textit{Sir John Child, 1st Baronet was the first governor of Bombay, and de facto (although not officially) the first governor-general of the British settlements in India. He was born in London but later moved to India where he became involved in the East India Company’s affairs. On 27th October 1681, he took up the position of Chief of the East India Company’s affairs at both Surat and Bombay. This made him the first British Governor in \ldots} \\
\hline
\end{tabularx}
\caption{A failure case of DKM-RAG because of preference of retriever so that high-resource but irrelevant document is retrieved. }
\label{tab:failure_dkmrag}
\end{table*}



\section{Language Preference of Generators in average}
We provide language preference of generators in terms of average as shown in Figure~\ref{fig:avg_generator_pref}. Consistent with the result of each query language in Figure~\ref{fig:multi_lang}, the generator shows preferences for Latin-script languages. And GPT-4o-mini shows more consistent outputs than other generators. This is because it is a larger model than the others, providing more stable answers regardless of language preference. Between Llama and Aya, Aya produces slightly more consistent outputs, demonstrating its multilingual capability in handling diverse linguistic contexts.


\section{Ablation study of DKM-RAG}
To prove the effectiveness of concatenating translated passages and refined passages in DKM-RAG framework, we provide an ablation study of each component in DKM-RAG. As stated in Table~\ref{tab:ablation_dkm}, removing any component from DKM-RAG results in decreased performance, highlighting that every part is crucial to its overall effectiveness.


\begin{table*}[t]
\centering
\renewcommand{\arraystretch}{1.1}
\setlength{\tabcolsep}{10pt}
\begin{tabular}{lccc}
\toprule
 & \textbf{DKM-RAG} & \textbf{w/o $P_{\text{refined}}$} & \textbf{w/o $P_{\text{translated}}$} \\
\midrule
\multicolumn{4}{l}{\textbf{$L_q =$ en}} \\
\midrule
\textbf{aya-expanse-8b}       & 82.60 & 79.34 & 81.10 \\
\textbf{Phi-4}                & 82.59 & 78.89 & 81.08 \\
\textbf{Qwen2.5-7B-Instruct}  & 82.60 & 79.11 & 81.06 \\
\textbf{Llama3.1-8B-Instruct} & 82.57 & 79.28 & 81.19 \\
\midrule
\multicolumn{4}{l}{\textbf{$L_q =$ zh}} \\
\midrule
\textbf{aya-expanse-8b}       & 44.57 & 38.31 & 39.44 \\
\textbf{Phi-4}                & 44.56 & 36.76 & 38.95 \\
\textbf{Qwen2.5-7B-Instruct}  & 44.70 & 38.31 & 39.78 \\
\textbf{Llama3.1-8B-Instruct} & 44.51 & 38.48 & 39.35 \\
\midrule
\multicolumn{4}{l}{\textbf{$L_q =$ ko}} \\
\midrule
\textbf{aya-expanse-8b}       & 55.01 & 49.66 & 46.15 \\
\textbf{Phi-4}                & 54.82 & 49.25 & 45.24 \\
\textbf{Qwen2.5-7B-Instruct}  & 54.85 & 49.44 & 45.32 \\
\textbf{Llama3.1-8B-Instruct} & 54.99 & 49.87 & 45.55 \\
\bottomrule
\end{tabular}
\caption{Ablation study on DKM-RAG. ``DKM-RAG'' denotes the DKM-RAG setting (i.e., the DKM-RAG column in Table~\ref{tab:resource_comparison_updated}), ``w/o $P_{\text{refined}}$'' indicates the performance corresponding to the highlighted cells, and ``w/o $P_{\text{translated}}$'' represents the results using only refined passages.}
\label{tab:ablation_dkm}
\end{table*}




\section{MLR Analysis}
We prove the effectiveness of our proposed language preference metric, MLR by comparing language preference between MLR score and the average document language ratio of retrieved documents for each dataset. As stated in Table~\ref{tab:analysis_mlr}, the tendency of average language ratio of retrieved documents and MLR score is similar. To prove it, we also report Pearson and Spearman correlation coefficients and each p-value between them. Pearson value (0.98558) indicates a very strong positive linear correlation between the average MKQA language distribution values (mkqa\_avg) and the MLR (Preference) scores. The p-value (7.75e-10) is extremely small, showing that the probability of observing such a strong correlation by chance is almost negligible. In short, there is a statistically significant, nearly perfect linear relationship between these two sets of values. Similarly, the Spearman value (0.86264) also indicates a strong association, and the corresponding p-value (1.47e-4) confirms that this correlation is statistically significant. By these results, we prove that MLR is efficient for measuring language preference of retriever.
\clearpage
\begin{table*}[t]
\centering
\footnotesize
\renewcommand{\arraystretch}{1.1}
\setlength{\tabcolsep}{4pt}
\begin{tabular}{lccccccccccccc}
\toprule
                & en    & ko    & ar    & zh    & fi    & fr    & de    & ja    & it    & pt    & ru    & es    & th    \\
\midrule
mkqa\_en       & 44.12 & 1.60  & 1.19  & 1.30  & 2.54  & 10.03 & 6.90  & 1.44  & 8.32  & 7.67  & 4.85  & 9.90  & 0.13  \\
mkqa\_ko       & 23.07 & 17.35 & 1.99  & 4.81  & 2.04  & 7.90  & 5.96  & 10.36 & 6.16  & 5.06  & 6.85  & 6.85  & 1.58  \\
mkqa\_ar       & 24.93 & 3.30  & 15.29 & 4.07  & 2.10  & 8.30  & 6.53  & 6.64  & 6.80  & 5.71  & 7.78  & 7.65  & 0.89  \\
mkqa\_zh       & 24.70 & 3.17  & 1.76  & 23.22 & 2.01  & 7.47  & 6.17  & 6.27  & 6.08  & 5.24  & 6.37  & 7.27  & 0.27  \\
mkqa\_fi       & 30.32 & 2.27  & 1.63  & 2.33  & 7.92  & 11.11 & 8.20  & 3.78  & 8.77  & 7.18  & 6.51  & 9.42  & 0.58  \\
mkqa\_fr       & 29.90 & 1.48  & 1.25  & 1.55  & 2.50  & 21.44 & 6.96  & 2.06  & 9.40  & 7.96  & 4.77  & 10.55 & 0.19  \\
mkqa\_de       & 32.54 & 1.46  & 1.17  & 1.44  & 2.96  & 11.40 & 15.12 & 1.89  & 9.09  & 7.69  & 4.83  & 10.17 & 0.24  \\
mkqa\_ja       & 24.56 & 4.80  & 1.69  & 3.99  & 2.19  & 7.97  & 5.99  & 22.55 & 6.38  & 5.66  & 6.49  & 7.45  & 0.28  \\
mkqa\_it       & 28.72 & 1.59  & 1.30  & 1.58  & 2.52  & 12.30 & 6.97  & 1.95  & 17.46 & 8.47  & 5.26  & 11.70 & 0.17  \\
mkqa\_pt       & 28.82 & 1.71  & 1.40  & 1.63  & 2.60  & 11.92 & 6.74  & 2.23  & 10.24 & 13.78 & 5.38  & 13.33 & 0.24  \\
mkqa\_ru       & 27.02 & 2.53  & 1.92  & 1.98  & 2.45  & 8.83  & 6.44  & 2.71  & 7.36  & 6.24  & 23.83 & 8.43  & 0.26  \\
mkqa\_es       & 29.45 & 1.73  & 1.27  & 1.60  & 2.66  & 11.85 & 6.93  & 1.83  & 10.55 & 9.33  & 5.27  & 17.36 & 0.16  \\
mkqa\_th       & 32.39 & 3.10  & 2.10  & 2.96  & 2.53  & 10.00 & 7.40  & 4.43  & 8.06  & 7.43  & 6.80  & 9.70  & 3.10  \\
\midrule
\rowcolor{gray!20} \textbf{mkqa\_avg}      & \textbf{29.27} & \textbf{3.55}  & \textbf{2.61}  & \textbf{4.04}  & \textbf{2.85}  & \textbf{10.81} & \textbf{7.41}  & \textbf{5.24}  & \textbf{8.82}  & \textbf{7.49}  & \textbf{7.31}  & \textbf{9.98}  & \textbf{0.62}  \\
\rowcolor{gray!20} \textbf{MLR (Preference)} & \textbf{47.70} & \textbf{35.47} & \textbf{35.59} & \textbf{35.90} & \textbf{35.13} & \textbf{37.94} & \textbf{37.20} & \textbf{37.59} & \textbf{37.66} & \textbf{37.15} & \textbf{37.99} & \textbf{37.97} & \textbf{34.09} \\
\midrule
\multicolumn{14}{l}{\textbf{Pearson correlation coefficient:} 0.98558 (p-value: 7.75e-10)} \\
\multicolumn{14}{l}{\textbf{Spearman correlation coefficient:} 0.86264 (p-value: 1.47e-4)} \\
\bottomrule
\end{tabular}
\caption{Language distribution ratios of documents retrieved from datasets composed of each query language. The table lists the raw MKQA language distribution values (without the percent sign) for each dataset. The row \textbf{mkqa\_avg} shows the average distribution across all MKQA datasets for each language, while the row \textbf{MLR (Preference)} provides the corresponding MLR scores. Additionally, we report Pearson and Spearman correlation coefficients between MLR and mkqa\_avg.}
\label{tab:analysis_mlr}
\end{table*}




\clearpage

 \begin{figure*}[ht]
  \centering
  \includegraphics[width=1.0\textwidth]{aya_similarity_matrix_en.pdf}
  \caption{LaBSE Similarity Matrix of aya-expanse-8b (en).} 
  \label{fig:aya_lang_pref_en}
\end{figure*}

 \begin{figure*}[ht]
  \centering
  \includegraphics[width=1.0\textwidth]{aya_similarity_matrix_zh.pdf} 
  \caption{LaBSE Similarity Matrix (zh) of aya-expanse-8b.} 
  \label{fig:aya_lang_pref_zh}
\end{figure*}

 \begin{figure*}[ht]
  \centering
  \includegraphics[width=1.0\textwidth]{aya_similarity_matrix_ko.pdf}
  \caption{LaBSE Similarity Matrix (ko) of aya-expanse-8b.} 
  \label{fig:aya_lang_pref_ko}
\end{figure*}

 \begin{figure*}[ht]
  \centering
  \includegraphics[width=1.0\textwidth]{llama_similarity_matrix_en.pdf}
  \caption{LaBSE Similarity Matrix (en) of Llama-3.1-8B-instruct.} 
  \label{fig:llama_lang_pref_en}
\end{figure*}

 \begin{figure*}[ht]
  \centering
  \includegraphics[width=1.0\textwidth]{llama_similarity_matrix_zh.pdf} 
  \caption{LaBSE Similarity Matrix (zh) of Llama-3.1-8B-instruct.} 
  \label{fig:llama_lang_pref_zh}
\end{figure*}


 \begin{figure*}[ht]
  \centering
  \includegraphics[width=1.0\textwidth]{llama_similarity_matrix_ko.pdf}
  \caption{LaBSE Similarity Matrix (ko) of Llama-3.1-8B-instruct.} 
  \label{fig:llama_lang_pref_ko}
\end{figure*}


 \begin{figure*}[ht]
  \centering
  \includegraphics[width=1.0\textwidth]{gpt_similarity_matrix_en.pdf}
  \caption{LaBSE Similarity Matrix (en) of gpt-4o-mini.} 
  \label{fig:gpt_lang_pref_en}
\end{figure*}

 \begin{figure*}[ht]
  \centering
  \includegraphics[width=1.0\textwidth]{gpt_similarity_matrix_zh.pdf} 
  \caption{LaBSE Similarity Matrix (zh) of gpt-4o-mini.} 
  \label{fig:gpt_lang_pref_zh}
\end{figure*}

 \begin{figure*}[ht]
  \centering
  \includegraphics[width=1.0\textwidth]{gpt_similarity_matrix_ko.pdf}
  \caption{LaBSE Similarity Matrix (ko) of gpt-4o-mini.} 
  \label{fig:gpt_lang_pref_ko}
\end{figure*}


 \begin{figure*}[ht]
  \centering
  \includegraphics[width=1.0\textwidth]{avg_generator_language_preference.pdf}
  \caption{Average Generator Preference for three query languages: en, zh, ko.} 
  \label{fig:avg_generator_pref}
\end{figure*}