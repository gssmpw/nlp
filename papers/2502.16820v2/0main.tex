%%
%% This is file `sample-sigconf-authordraft.tex',
%% generated with the docstrip utility.
%%
%% The original source files were:
%%
%% samples.dtx  (with options: `all,proceedings,bibtex,authordraft')
%% 
%% IMPORTANT NOTICE:
%% 
%% For the copyright see the source file.
%% 
%% Any modified versions of this file must be renamed
%% with new filenames distinct from sample-sigconf-authordraft.tex.
%% 
%% For distribution of the original source see the terms
%% for copying and modification in the file samples.dtx.
%% 
%% This generated file may be distributed as long as the
%% original source files, as listed above, are part of the
%% same distribution. (The sources need not necessarily be
%% in the same archive or directory.)
%%
%%
%% Commands for TeXCount
%TC:macro \cite [option:text,text]
%TC:macro \citep [option:text,text]
%TC:macro \citet [option:text,text]
%TC:envir table 0 1
%TC:envir table* 0 1
%TC:envir tabular [ignore] word
%TC:envir displaymath 0 word
%TC:envir math 0 word
%TC:envir comment 0 0
%%
%% The first command in your LaTeX source must be the \documentclass
%% command.
%%
%% For submission and review of your manuscript please change the
%% command to \documentclass[manuscript, screen, review]{acmart}.
%%
%% When submitting camera ready or to TAPS, please change the command
%% to \documentclass[sigconf]{acmart} or whichever template is required
%% for your publication.
%%
%%
\documentclass[sigconf,nonacm]{acmart}
\let\Bbbk\relax

%%%%% NEW MATH DEFINITIONS %%%%%

\usepackage{amsmath,amsfonts,bm}
\usepackage{derivative}
% Mark sections of captions for referring to divisions of figures
\newcommand{\figleft}{{\em (Left)}}
\newcommand{\figcenter}{{\em (Center)}}
\newcommand{\figright}{{\em (Right)}}
\newcommand{\figtop}{{\em (Top)}}
\newcommand{\figbottom}{{\em (Bottom)}}
\newcommand{\captiona}{{\em (a)}}
\newcommand{\captionb}{{\em (b)}}
\newcommand{\captionc}{{\em (c)}}
\newcommand{\captiond}{{\em (d)}}

% Highlight a newly defined term
\newcommand{\newterm}[1]{{\bf #1}}

% Derivative d 
\newcommand{\deriv}{{\mathrm{d}}}

% Figure reference, lower-case.
\def\figref#1{figure~\ref{#1}}
% Figure reference, capital. For start of sentence
\def\Figref#1{Figure~\ref{#1}}
\def\twofigref#1#2{figures \ref{#1} and \ref{#2}}
\def\quadfigref#1#2#3#4{figures \ref{#1}, \ref{#2}, \ref{#3} and \ref{#4}}
% Section reference, lower-case.
\def\secref#1{section~\ref{#1}}
% Section reference, capital.
\def\Secref#1{Section~\ref{#1}}
% Reference to two sections.
\def\twosecrefs#1#2{sections \ref{#1} and \ref{#2}}
% Reference to three sections.
\def\secrefs#1#2#3{sections \ref{#1}, \ref{#2} and \ref{#3}}
% Reference to an equation, lower-case.
\def\eqref#1{equation~\ref{#1}}
% Reference to an equation, upper case
\def\Eqref#1{Equation~\ref{#1}}
% A raw reference to an equation---avoid using if possible
\def\plaineqref#1{\ref{#1}}
% Reference to a chapter, lower-case.
\def\chapref#1{chapter~\ref{#1}}
% Reference to an equation, upper case.
\def\Chapref#1{Chapter~\ref{#1}}
% Reference to a range of chapters
\def\rangechapref#1#2{chapters\ref{#1}--\ref{#2}}
% Reference to an algorithm, lower-case.
\def\algref#1{algorithm~\ref{#1}}
% Reference to an algorithm, upper case.
\def\Algref#1{Algorithm~\ref{#1}}
\def\twoalgref#1#2{algorithms \ref{#1} and \ref{#2}}
\def\Twoalgref#1#2{Algorithms \ref{#1} and \ref{#2}}
% Reference to a part, lower case
\def\partref#1{part~\ref{#1}}
% Reference to a part, upper case
\def\Partref#1{Part~\ref{#1}}
\def\twopartref#1#2{parts \ref{#1} and \ref{#2}}

\def\ceil#1{\lceil #1 \rceil}
\def\floor#1{\lfloor #1 \rfloor}
\def\1{\bm{1}}
\newcommand{\train}{\mathcal{D}}
\newcommand{\valid}{\mathcal{D_{\mathrm{valid}}}}
\newcommand{\test}{\mathcal{D_{\mathrm{test}}}}

\def\eps{{\epsilon}}


% Random variables
\def\reta{{\textnormal{$\eta$}}}
\def\ra{{\textnormal{a}}}
\def\rb{{\textnormal{b}}}
\def\rc{{\textnormal{c}}}
\def\rd{{\textnormal{d}}}
\def\re{{\textnormal{e}}}
\def\rf{{\textnormal{f}}}
\def\rg{{\textnormal{g}}}
\def\rh{{\textnormal{h}}}
\def\ri{{\textnormal{i}}}
\def\rj{{\textnormal{j}}}
\def\rk{{\textnormal{k}}}
\def\rl{{\textnormal{l}}}
% rm is already a command, just don't name any random variables m
\def\rn{{\textnormal{n}}}
\def\ro{{\textnormal{o}}}
\def\rp{{\textnormal{p}}}
\def\rq{{\textnormal{q}}}
\def\rr{{\textnormal{r}}}
\def\rs{{\textnormal{s}}}
\def\rt{{\textnormal{t}}}
\def\ru{{\textnormal{u}}}
\def\rv{{\textnormal{v}}}
\def\rw{{\textnormal{w}}}
\def\rx{{\textnormal{x}}}
\def\ry{{\textnormal{y}}}
\def\rz{{\textnormal{z}}}

% Random vectors
\def\rvepsilon{{\mathbf{\epsilon}}}
\def\rvphi{{\mathbf{\phi}}}
\def\rvtheta{{\mathbf{\theta}}}
\def\rva{{\mathbf{a}}}
\def\rvb{{\mathbf{b}}}
\def\rvc{{\mathbf{c}}}
\def\rvd{{\mathbf{d}}}
\def\rve{{\mathbf{e}}}
\def\rvf{{\mathbf{f}}}
\def\rvg{{\mathbf{g}}}
\def\rvh{{\mathbf{h}}}
\def\rvu{{\mathbf{i}}}
\def\rvj{{\mathbf{j}}}
\def\rvk{{\mathbf{k}}}
\def\rvl{{\mathbf{l}}}
\def\rvm{{\mathbf{m}}}
\def\rvn{{\mathbf{n}}}
\def\rvo{{\mathbf{o}}}
\def\rvp{{\mathbf{p}}}
\def\rvq{{\mathbf{q}}}
\def\rvr{{\mathbf{r}}}
\def\rvs{{\mathbf{s}}}
\def\rvt{{\mathbf{t}}}
\def\rvu{{\mathbf{u}}}
\def\rvv{{\mathbf{v}}}
\def\rvw{{\mathbf{w}}}
\def\rvx{{\mathbf{x}}}
\def\rvy{{\mathbf{y}}}
\def\rvz{{\mathbf{z}}}

% Elements of random vectors
\def\erva{{\textnormal{a}}}
\def\ervb{{\textnormal{b}}}
\def\ervc{{\textnormal{c}}}
\def\ervd{{\textnormal{d}}}
\def\erve{{\textnormal{e}}}
\def\ervf{{\textnormal{f}}}
\def\ervg{{\textnormal{g}}}
\def\ervh{{\textnormal{h}}}
\def\ervi{{\textnormal{i}}}
\def\ervj{{\textnormal{j}}}
\def\ervk{{\textnormal{k}}}
\def\ervl{{\textnormal{l}}}
\def\ervm{{\textnormal{m}}}
\def\ervn{{\textnormal{n}}}
\def\ervo{{\textnormal{o}}}
\def\ervp{{\textnormal{p}}}
\def\ervq{{\textnormal{q}}}
\def\ervr{{\textnormal{r}}}
\def\ervs{{\textnormal{s}}}
\def\ervt{{\textnormal{t}}}
\def\ervu{{\textnormal{u}}}
\def\ervv{{\textnormal{v}}}
\def\ervw{{\textnormal{w}}}
\def\ervx{{\textnormal{x}}}
\def\ervy{{\textnormal{y}}}
\def\ervz{{\textnormal{z}}}

% Random matrices
\def\rmA{{\mathbf{A}}}
\def\rmB{{\mathbf{B}}}
\def\rmC{{\mathbf{C}}}
\def\rmD{{\mathbf{D}}}
\def\rmE{{\mathbf{E}}}
\def\rmF{{\mathbf{F}}}
\def\rmG{{\mathbf{G}}}
\def\rmH{{\mathbf{H}}}
\def\rmI{{\mathbf{I}}}
\def\rmJ{{\mathbf{J}}}
\def\rmK{{\mathbf{K}}}
\def\rmL{{\mathbf{L}}}
\def\rmM{{\mathbf{M}}}
\def\rmN{{\mathbf{N}}}
\def\rmO{{\mathbf{O}}}
\def\rmP{{\mathbf{P}}}
\def\rmQ{{\mathbf{Q}}}
\def\rmR{{\mathbf{R}}}
\def\rmS{{\mathbf{S}}}
\def\rmT{{\mathbf{T}}}
\def\rmU{{\mathbf{U}}}
\def\rmV{{\mathbf{V}}}
\def\rmW{{\mathbf{W}}}
\def\rmX{{\mathbf{X}}}
\def\rmY{{\mathbf{Y}}}
\def\rmZ{{\mathbf{Z}}}

% Elements of random matrices
\def\ermA{{\textnormal{A}}}
\def\ermB{{\textnormal{B}}}
\def\ermC{{\textnormal{C}}}
\def\ermD{{\textnormal{D}}}
\def\ermE{{\textnormal{E}}}
\def\ermF{{\textnormal{F}}}
\def\ermG{{\textnormal{G}}}
\def\ermH{{\textnormal{H}}}
\def\ermI{{\textnormal{I}}}
\def\ermJ{{\textnormal{J}}}
\def\ermK{{\textnormal{K}}}
\def\ermL{{\textnormal{L}}}
\def\ermM{{\textnormal{M}}}
\def\ermN{{\textnormal{N}}}
\def\ermO{{\textnormal{O}}}
\def\ermP{{\textnormal{P}}}
\def\ermQ{{\textnormal{Q}}}
\def\ermR{{\textnormal{R}}}
\def\ermS{{\textnormal{S}}}
\def\ermT{{\textnormal{T}}}
\def\ermU{{\textnormal{U}}}
\def\ermV{{\textnormal{V}}}
\def\ermW{{\textnormal{W}}}
\def\ermX{{\textnormal{X}}}
\def\ermY{{\textnormal{Y}}}
\def\ermZ{{\textnormal{Z}}}

% Vectors
\def\vzero{{\bm{0}}}
\def\vone{{\bm{1}}}
\def\vmu{{\bm{\mu}}}
\def\vtheta{{\bm{\theta}}}
\def\vphi{{\bm{\phi}}}
\def\va{{\bm{a}}}
\def\vb{{\bm{b}}}
\def\vc{{\bm{c}}}
\def\vd{{\bm{d}}}
\def\ve{{\bm{e}}}
\def\vf{{\bm{f}}}
\def\vg{{\bm{g}}}
\def\vh{{\bm{h}}}
\def\vi{{\bm{i}}}
\def\vj{{\bm{j}}}
\def\vk{{\bm{k}}}
\def\vl{{\bm{l}}}
\def\vm{{\bm{m}}}
\def\vn{{\bm{n}}}
\def\vo{{\bm{o}}}
\def\vp{{\bm{p}}}
\def\vq{{\bm{q}}}
\def\vr{{\bm{r}}}
\def\vs{{\bm{s}}}
\def\vt{{\bm{t}}}
\def\vu{{\bm{u}}}
\def\vv{{\bm{v}}}
\def\vw{{\bm{w}}}
\def\vx{{\bm{x}}}
\def\vy{{\bm{y}}}
\def\vz{{\bm{z}}}

% Elements of vectors
\def\evalpha{{\alpha}}
\def\evbeta{{\beta}}
\def\evepsilon{{\epsilon}}
\def\evlambda{{\lambda}}
\def\evomega{{\omega}}
\def\evmu{{\mu}}
\def\evpsi{{\psi}}
\def\evsigma{{\sigma}}
\def\evtheta{{\theta}}
\def\eva{{a}}
\def\evb{{b}}
\def\evc{{c}}
\def\evd{{d}}
\def\eve{{e}}
\def\evf{{f}}
\def\evg{{g}}
\def\evh{{h}}
\def\evi{{i}}
\def\evj{{j}}
\def\evk{{k}}
\def\evl{{l}}
\def\evm{{m}}
\def\evn{{n}}
\def\evo{{o}}
\def\evp{{p}}
\def\evq{{q}}
\def\evr{{r}}
\def\evs{{s}}
\def\evt{{t}}
\def\evu{{u}}
\def\evv{{v}}
\def\evw{{w}}
\def\evx{{x}}
\def\evy{{y}}
\def\evz{{z}}

% Matrix
\def\mA{{\bm{A}}}
\def\mB{{\bm{B}}}
\def\mC{{\bm{C}}}
\def\mD{{\bm{D}}}
\def\mE{{\bm{E}}}
\def\mF{{\bm{F}}}
\def\mG{{\bm{G}}}
\def\mH{{\bm{H}}}
\def\mI{{\bm{I}}}
\def\mJ{{\bm{J}}}
\def\mK{{\bm{K}}}
\def\mL{{\bm{L}}}
\def\mM{{\bm{M}}}
\def\mN{{\bm{N}}}
\def\mO{{\bm{O}}}
\def\mP{{\bm{P}}}
\def\mQ{{\bm{Q}}}
\def\mR{{\bm{R}}}
\def\mS{{\bm{S}}}
\def\mT{{\bm{T}}}
\def\mU{{\bm{U}}}
\def\mV{{\bm{V}}}
\def\mW{{\bm{W}}}
\def\mX{{\bm{X}}}
\def\mY{{\bm{Y}}}
\def\mZ{{\bm{Z}}}
\def\mBeta{{\bm{\beta}}}
\def\mPhi{{\bm{\Phi}}}
\def\mLambda{{\bm{\Lambda}}}
\def\mSigma{{\bm{\Sigma}}}

% Tensor
\DeclareMathAlphabet{\mathsfit}{\encodingdefault}{\sfdefault}{m}{sl}
\SetMathAlphabet{\mathsfit}{bold}{\encodingdefault}{\sfdefault}{bx}{n}
\newcommand{\tens}[1]{\bm{\mathsfit{#1}}}
\def\tA{{\tens{A}}}
\def\tB{{\tens{B}}}
\def\tC{{\tens{C}}}
\def\tD{{\tens{D}}}
\def\tE{{\tens{E}}}
\def\tF{{\tens{F}}}
\def\tG{{\tens{G}}}
\def\tH{{\tens{H}}}
\def\tI{{\tens{I}}}
\def\tJ{{\tens{J}}}
\def\tK{{\tens{K}}}
\def\tL{{\tens{L}}}
\def\tM{{\tens{M}}}
\def\tN{{\tens{N}}}
\def\tO{{\tens{O}}}
\def\tP{{\tens{P}}}
\def\tQ{{\tens{Q}}}
\def\tR{{\tens{R}}}
\def\tS{{\tens{S}}}
\def\tT{{\tens{T}}}
\def\tU{{\tens{U}}}
\def\tV{{\tens{V}}}
\def\tW{{\tens{W}}}
\def\tX{{\tens{X}}}
\def\tY{{\tens{Y}}}
\def\tZ{{\tens{Z}}}


% Graph
\def\gA{{\mathcal{A}}}
\def\gB{{\mathcal{B}}}
\def\gC{{\mathcal{C}}}
\def\gD{{\mathcal{D}}}
\def\gE{{\mathcal{E}}}
\def\gF{{\mathcal{F}}}
\def\gG{{\mathcal{G}}}
\def\gH{{\mathcal{H}}}
\def\gI{{\mathcal{I}}}
\def\gJ{{\mathcal{J}}}
\def\gK{{\mathcal{K}}}
\def\gL{{\mathcal{L}}}
\def\gM{{\mathcal{M}}}
\def\gN{{\mathcal{N}}}
\def\gO{{\mathcal{O}}}
\def\gP{{\mathcal{P}}}
\def\gQ{{\mathcal{Q}}}
\def\gR{{\mathcal{R}}}
\def\gS{{\mathcal{S}}}
\def\gT{{\mathcal{T}}}
\def\gU{{\mathcal{U}}}
\def\gV{{\mathcal{V}}}
\def\gW{{\mathcal{W}}}
\def\gX{{\mathcal{X}}}
\def\gY{{\mathcal{Y}}}
\def\gZ{{\mathcal{Z}}}

% Sets
\def\sA{{\mathbb{A}}}
\def\sB{{\mathbb{B}}}
\def\sC{{\mathbb{C}}}
\def\sD{{\mathbb{D}}}
% Don't use a set called E, because this would be the same as our symbol
% for expectation.
\def\sF{{\mathbb{F}}}
\def\sG{{\mathbb{G}}}
\def\sH{{\mathbb{H}}}
\def\sI{{\mathbb{I}}}
\def\sJ{{\mathbb{J}}}
\def\sK{{\mathbb{K}}}
\def\sL{{\mathbb{L}}}
\def\sM{{\mathbb{M}}}
\def\sN{{\mathbb{N}}}
\def\sO{{\mathbb{O}}}
\def\sP{{\mathbb{P}}}
\def\sQ{{\mathbb{Q}}}
\def\sR{{\mathbb{R}}}
\def\sS{{\mathbb{S}}}
\def\sT{{\mathbb{T}}}
\def\sU{{\mathbb{U}}}
\def\sV{{\mathbb{V}}}
\def\sW{{\mathbb{W}}}
\def\sX{{\mathbb{X}}}
\def\sY{{\mathbb{Y}}}
\def\sZ{{\mathbb{Z}}}

% Entries of a matrix
\def\emLambda{{\Lambda}}
\def\emA{{A}}
\def\emB{{B}}
\def\emC{{C}}
\def\emD{{D}}
\def\emE{{E}}
\def\emF{{F}}
\def\emG{{G}}
\def\emH{{H}}
\def\emI{{I}}
\def\emJ{{J}}
\def\emK{{K}}
\def\emL{{L}}
\def\emM{{M}}
\def\emN{{N}}
\def\emO{{O}}
\def\emP{{P}}
\def\emQ{{Q}}
\def\emR{{R}}
\def\emS{{S}}
\def\emT{{T}}
\def\emU{{U}}
\def\emV{{V}}
\def\emW{{W}}
\def\emX{{X}}
\def\emY{{Y}}
\def\emZ{{Z}}
\def\emSigma{{\Sigma}}

% entries of a tensor
% Same font as tensor, without \bm wrapper
\newcommand{\etens}[1]{\mathsfit{#1}}
\def\etLambda{{\etens{\Lambda}}}
\def\etA{{\etens{A}}}
\def\etB{{\etens{B}}}
\def\etC{{\etens{C}}}
\def\etD{{\etens{D}}}
\def\etE{{\etens{E}}}
\def\etF{{\etens{F}}}
\def\etG{{\etens{G}}}
\def\etH{{\etens{H}}}
\def\etI{{\etens{I}}}
\def\etJ{{\etens{J}}}
\def\etK{{\etens{K}}}
\def\etL{{\etens{L}}}
\def\etM{{\etens{M}}}
\def\etN{{\etens{N}}}
\def\etO{{\etens{O}}}
\def\etP{{\etens{P}}}
\def\etQ{{\etens{Q}}}
\def\etR{{\etens{R}}}
\def\etS{{\etens{S}}}
\def\etT{{\etens{T}}}
\def\etU{{\etens{U}}}
\def\etV{{\etens{V}}}
\def\etW{{\etens{W}}}
\def\etX{{\etens{X}}}
\def\etY{{\etens{Y}}}
\def\etZ{{\etens{Z}}}

% The true underlying data generating distribution
\newcommand{\pdata}{p_{\rm{data}}}
\newcommand{\ptarget}{p_{\rm{target}}}
\newcommand{\pprior}{p_{\rm{prior}}}
\newcommand{\pbase}{p_{\rm{base}}}
\newcommand{\pref}{p_{\rm{ref}}}

% The empirical distribution defined by the training set
\newcommand{\ptrain}{\hat{p}_{\rm{data}}}
\newcommand{\Ptrain}{\hat{P}_{\rm{data}}}
% The model distribution
\newcommand{\pmodel}{p_{\rm{model}}}
\newcommand{\Pmodel}{P_{\rm{model}}}
\newcommand{\ptildemodel}{\tilde{p}_{\rm{model}}}
% Stochastic autoencoder distributions
\newcommand{\pencode}{p_{\rm{encoder}}}
\newcommand{\pdecode}{p_{\rm{decoder}}}
\newcommand{\precons}{p_{\rm{reconstruct}}}

\newcommand{\laplace}{\mathrm{Laplace}} % Laplace distribution

\newcommand{\E}{\mathbb{E}}
\newcommand{\Ls}{\mathcal{L}}
\newcommand{\R}{\mathbb{R}}
\newcommand{\emp}{\tilde{p}}
\newcommand{\lr}{\alpha}
\newcommand{\reg}{\lambda}
\newcommand{\rect}{\mathrm{rectifier}}
\newcommand{\softmax}{\mathrm{softmax}}
\newcommand{\sigmoid}{\sigma}
\newcommand{\softplus}{\zeta}
\newcommand{\KL}{D_{\mathrm{KL}}}
\newcommand{\Var}{\mathrm{Var}}
\newcommand{\standarderror}{\mathrm{SE}}
\newcommand{\Cov}{\mathrm{Cov}}
% Wolfram Mathworld says $L^2$ is for function spaces and $\ell^2$ is for vectors
% But then they seem to use $L^2$ for vectors throughout the site, and so does
% wikipedia.
\newcommand{\normlzero}{L^0}
\newcommand{\normlone}{L^1}
\newcommand{\normltwo}{L^2}
\newcommand{\normlp}{L^p}
\newcommand{\normmax}{L^\infty}

\newcommand{\parents}{Pa} % See usage in notation.tex. Chosen to match Daphne's book.

\DeclareMathOperator*{\argmax}{arg\,max}
\DeclareMathOperator*{\argmin}{arg\,min}

\DeclareMathOperator{\sign}{sign}
\DeclareMathOperator{\Tr}{Tr}
\let\ab\allowbreak

% %%%%% NEW MATH DEFINITIONS %%%%%

\usepackage{amsmath,amsfonts,bm}
\usepackage{derivative}
% Mark sections of captions for referring to divisions of figures
\newcommand{\figleft}{{\em (Left)}}
\newcommand{\figcenter}{{\em (Center)}}
\newcommand{\figright}{{\em (Right)}}
\newcommand{\figtop}{{\em (Top)}}
\newcommand{\figbottom}{{\em (Bottom)}}
\newcommand{\captiona}{{\em (a)}}
\newcommand{\captionb}{{\em (b)}}
\newcommand{\captionc}{{\em (c)}}
\newcommand{\captiond}{{\em (d)}}

% Highlight a newly defined term
\newcommand{\newterm}[1]{{\bf #1}}

% Derivative d 
\newcommand{\deriv}{{\mathrm{d}}}

% Figure reference, lower-case.
\def\figref#1{figure~\ref{#1}}
% Figure reference, capital. For start of sentence
\def\Figref#1{Figure~\ref{#1}}
\def\twofigref#1#2{figures \ref{#1} and \ref{#2}}
\def\quadfigref#1#2#3#4{figures \ref{#1}, \ref{#2}, \ref{#3} and \ref{#4}}
% Section reference, lower-case.
\def\secref#1{section~\ref{#1}}
% Section reference, capital.
\def\Secref#1{Section~\ref{#1}}
% Reference to two sections.
\def\twosecrefs#1#2{sections \ref{#1} and \ref{#2}}
% Reference to three sections.
\def\secrefs#1#2#3{sections \ref{#1}, \ref{#2} and \ref{#3}}
% Reference to an equation, lower-case.
\def\eqref#1{equation~\ref{#1}}
% Reference to an equation, upper case
\def\Eqref#1{Equation~\ref{#1}}
% A raw reference to an equation---avoid using if possible
\def\plaineqref#1{\ref{#1}}
% Reference to a chapter, lower-case.
\def\chapref#1{chapter~\ref{#1}}
% Reference to an equation, upper case.
\def\Chapref#1{Chapter~\ref{#1}}
% Reference to a range of chapters
\def\rangechapref#1#2{chapters\ref{#1}--\ref{#2}}
% Reference to an algorithm, lower-case.
\def\algref#1{algorithm~\ref{#1}}
% Reference to an algorithm, upper case.
\def\Algref#1{Algorithm~\ref{#1}}
\def\twoalgref#1#2{algorithms \ref{#1} and \ref{#2}}
\def\Twoalgref#1#2{Algorithms \ref{#1} and \ref{#2}}
% Reference to a part, lower case
\def\partref#1{part~\ref{#1}}
% Reference to a part, upper case
\def\Partref#1{Part~\ref{#1}}
\def\twopartref#1#2{parts \ref{#1} and \ref{#2}}

\def\ceil#1{\lceil #1 \rceil}
\def\floor#1{\lfloor #1 \rfloor}
\def\1{\bm{1}}
\newcommand{\train}{\mathcal{D}}
\newcommand{\valid}{\mathcal{D_{\mathrm{valid}}}}
\newcommand{\test}{\mathcal{D_{\mathrm{test}}}}

\def\eps{{\epsilon}}


% Random variables
\def\reta{{\textnormal{$\eta$}}}
\def\ra{{\textnormal{a}}}
\def\rb{{\textnormal{b}}}
\def\rc{{\textnormal{c}}}
\def\rd{{\textnormal{d}}}
\def\re{{\textnormal{e}}}
\def\rf{{\textnormal{f}}}
\def\rg{{\textnormal{g}}}
\def\rh{{\textnormal{h}}}
\def\ri{{\textnormal{i}}}
\def\rj{{\textnormal{j}}}
\def\rk{{\textnormal{k}}}
\def\rl{{\textnormal{l}}}
% rm is already a command, just don't name any random variables m
\def\rn{{\textnormal{n}}}
\def\ro{{\textnormal{o}}}
\def\rp{{\textnormal{p}}}
\def\rq{{\textnormal{q}}}
\def\rr{{\textnormal{r}}}
\def\rs{{\textnormal{s}}}
\def\rt{{\textnormal{t}}}
\def\ru{{\textnormal{u}}}
\def\rv{{\textnormal{v}}}
\def\rw{{\textnormal{w}}}
\def\rx{{\textnormal{x}}}
\def\ry{{\textnormal{y}}}
\def\rz{{\textnormal{z}}}

% Random vectors
\def\rvepsilon{{\mathbf{\epsilon}}}
\def\rvphi{{\mathbf{\phi}}}
\def\rvtheta{{\mathbf{\theta}}}
\def\rva{{\mathbf{a}}}
\def\rvb{{\mathbf{b}}}
\def\rvc{{\mathbf{c}}}
\def\rvd{{\mathbf{d}}}
\def\rve{{\mathbf{e}}}
\def\rvf{{\mathbf{f}}}
\def\rvg{{\mathbf{g}}}
\def\rvh{{\mathbf{h}}}
\def\rvu{{\mathbf{i}}}
\def\rvj{{\mathbf{j}}}
\def\rvk{{\mathbf{k}}}
\def\rvl{{\mathbf{l}}}
\def\rvm{{\mathbf{m}}}
\def\rvn{{\mathbf{n}}}
\def\rvo{{\mathbf{o}}}
\def\rvp{{\mathbf{p}}}
\def\rvq{{\mathbf{q}}}
\def\rvr{{\mathbf{r}}}
\def\rvs{{\mathbf{s}}}
\def\rvt{{\mathbf{t}}}
\def\rvu{{\mathbf{u}}}
\def\rvv{{\mathbf{v}}}
\def\rvw{{\mathbf{w}}}
\def\rvx{{\mathbf{x}}}
\def\rvy{{\mathbf{y}}}
\def\rvz{{\mathbf{z}}}

% Elements of random vectors
\def\erva{{\textnormal{a}}}
\def\ervb{{\textnormal{b}}}
\def\ervc{{\textnormal{c}}}
\def\ervd{{\textnormal{d}}}
\def\erve{{\textnormal{e}}}
\def\ervf{{\textnormal{f}}}
\def\ervg{{\textnormal{g}}}
\def\ervh{{\textnormal{h}}}
\def\ervi{{\textnormal{i}}}
\def\ervj{{\textnormal{j}}}
\def\ervk{{\textnormal{k}}}
\def\ervl{{\textnormal{l}}}
\def\ervm{{\textnormal{m}}}
\def\ervn{{\textnormal{n}}}
\def\ervo{{\textnormal{o}}}
\def\ervp{{\textnormal{p}}}
\def\ervq{{\textnormal{q}}}
\def\ervr{{\textnormal{r}}}
\def\ervs{{\textnormal{s}}}
\def\ervt{{\textnormal{t}}}
\def\ervu{{\textnormal{u}}}
\def\ervv{{\textnormal{v}}}
\def\ervw{{\textnormal{w}}}
\def\ervx{{\textnormal{x}}}
\def\ervy{{\textnormal{y}}}
\def\ervz{{\textnormal{z}}}

% Random matrices
\def\rmA{{\mathbf{A}}}
\def\rmB{{\mathbf{B}}}
\def\rmC{{\mathbf{C}}}
\def\rmD{{\mathbf{D}}}
\def\rmE{{\mathbf{E}}}
\def\rmF{{\mathbf{F}}}
\def\rmG{{\mathbf{G}}}
\def\rmH{{\mathbf{H}}}
\def\rmI{{\mathbf{I}}}
\def\rmJ{{\mathbf{J}}}
\def\rmK{{\mathbf{K}}}
\def\rmL{{\mathbf{L}}}
\def\rmM{{\mathbf{M}}}
\def\rmN{{\mathbf{N}}}
\def\rmO{{\mathbf{O}}}
\def\rmP{{\mathbf{P}}}
\def\rmQ{{\mathbf{Q}}}
\def\rmR{{\mathbf{R}}}
\def\rmS{{\mathbf{S}}}
\def\rmT{{\mathbf{T}}}
\def\rmU{{\mathbf{U}}}
\def\rmV{{\mathbf{V}}}
\def\rmW{{\mathbf{W}}}
\def\rmX{{\mathbf{X}}}
\def\rmY{{\mathbf{Y}}}
\def\rmZ{{\mathbf{Z}}}

% Elements of random matrices
\def\ermA{{\textnormal{A}}}
\def\ermB{{\textnormal{B}}}
\def\ermC{{\textnormal{C}}}
\def\ermD{{\textnormal{D}}}
\def\ermE{{\textnormal{E}}}
\def\ermF{{\textnormal{F}}}
\def\ermG{{\textnormal{G}}}
\def\ermH{{\textnormal{H}}}
\def\ermI{{\textnormal{I}}}
\def\ermJ{{\textnormal{J}}}
\def\ermK{{\textnormal{K}}}
\def\ermL{{\textnormal{L}}}
\def\ermM{{\textnormal{M}}}
\def\ermN{{\textnormal{N}}}
\def\ermO{{\textnormal{O}}}
\def\ermP{{\textnormal{P}}}
\def\ermQ{{\textnormal{Q}}}
\def\ermR{{\textnormal{R}}}
\def\ermS{{\textnormal{S}}}
\def\ermT{{\textnormal{T}}}
\def\ermU{{\textnormal{U}}}
\def\ermV{{\textnormal{V}}}
\def\ermW{{\textnormal{W}}}
\def\ermX{{\textnormal{X}}}
\def\ermY{{\textnormal{Y}}}
\def\ermZ{{\textnormal{Z}}}

% Vectors
\def\vzero{{\bm{0}}}
\def\vone{{\bm{1}}}
\def\vmu{{\bm{\mu}}}
\def\vtheta{{\bm{\theta}}}
\def\vphi{{\bm{\phi}}}
\def\va{{\bm{a}}}
\def\vb{{\bm{b}}}
\def\vc{{\bm{c}}}
\def\vd{{\bm{d}}}
\def\ve{{\bm{e}}}
\def\vf{{\bm{f}}}
\def\vg{{\bm{g}}}
\def\vh{{\bm{h}}}
\def\vi{{\bm{i}}}
\def\vj{{\bm{j}}}
\def\vk{{\bm{k}}}
\def\vl{{\bm{l}}}
\def\vm{{\bm{m}}}
\def\vn{{\bm{n}}}
\def\vo{{\bm{o}}}
\def\vp{{\bm{p}}}
\def\vq{{\bm{q}}}
\def\vr{{\bm{r}}}
\def\vs{{\bm{s}}}
\def\vt{{\bm{t}}}
\def\vu{{\bm{u}}}
\def\vv{{\bm{v}}}
\def\vw{{\bm{w}}}
\def\vx{{\bm{x}}}
\def\vy{{\bm{y}}}
\def\vz{{\bm{z}}}

% Elements of vectors
\def\evalpha{{\alpha}}
\def\evbeta{{\beta}}
\def\evepsilon{{\epsilon}}
\def\evlambda{{\lambda}}
\def\evomega{{\omega}}
\def\evmu{{\mu}}
\def\evpsi{{\psi}}
\def\evsigma{{\sigma}}
\def\evtheta{{\theta}}
\def\eva{{a}}
\def\evb{{b}}
\def\evc{{c}}
\def\evd{{d}}
\def\eve{{e}}
\def\evf{{f}}
\def\evg{{g}}
\def\evh{{h}}
\def\evi{{i}}
\def\evj{{j}}
\def\evk{{k}}
\def\evl{{l}}
\def\evm{{m}}
\def\evn{{n}}
\def\evo{{o}}
\def\evp{{p}}
\def\evq{{q}}
\def\evr{{r}}
\def\evs{{s}}
\def\evt{{t}}
\def\evu{{u}}
\def\evv{{v}}
\def\evw{{w}}
\def\evx{{x}}
\def\evy{{y}}
\def\evz{{z}}

% Matrix
\def\mA{{\bm{A}}}
\def\mB{{\bm{B}}}
\def\mC{{\bm{C}}}
\def\mD{{\bm{D}}}
\def\mE{{\bm{E}}}
\def\mF{{\bm{F}}}
\def\mG{{\bm{G}}}
\def\mH{{\bm{H}}}
\def\mI{{\bm{I}}}
\def\mJ{{\bm{J}}}
\def\mK{{\bm{K}}}
\def\mL{{\bm{L}}}
\def\mM{{\bm{M}}}
\def\mN{{\bm{N}}}
\def\mO{{\bm{O}}}
\def\mP{{\bm{P}}}
\def\mQ{{\bm{Q}}}
\def\mR{{\bm{R}}}
\def\mS{{\bm{S}}}
\def\mT{{\bm{T}}}
\def\mU{{\bm{U}}}
\def\mV{{\bm{V}}}
\def\mW{{\bm{W}}}
\def\mX{{\bm{X}}}
\def\mY{{\bm{Y}}}
\def\mZ{{\bm{Z}}}
\def\mBeta{{\bm{\beta}}}
\def\mPhi{{\bm{\Phi}}}
\def\mLambda{{\bm{\Lambda}}}
\def\mSigma{{\bm{\Sigma}}}

% Tensor
\DeclareMathAlphabet{\mathsfit}{\encodingdefault}{\sfdefault}{m}{sl}
\SetMathAlphabet{\mathsfit}{bold}{\encodingdefault}{\sfdefault}{bx}{n}
\newcommand{\tens}[1]{\bm{\mathsfit{#1}}}
\def\tA{{\tens{A}}}
\def\tB{{\tens{B}}}
\def\tC{{\tens{C}}}
\def\tD{{\tens{D}}}
\def\tE{{\tens{E}}}
\def\tF{{\tens{F}}}
\def\tG{{\tens{G}}}
\def\tH{{\tens{H}}}
\def\tI{{\tens{I}}}
\def\tJ{{\tens{J}}}
\def\tK{{\tens{K}}}
\def\tL{{\tens{L}}}
\def\tM{{\tens{M}}}
\def\tN{{\tens{N}}}
\def\tO{{\tens{O}}}
\def\tP{{\tens{P}}}
\def\tQ{{\tens{Q}}}
\def\tR{{\tens{R}}}
\def\tS{{\tens{S}}}
\def\tT{{\tens{T}}}
\def\tU{{\tens{U}}}
\def\tV{{\tens{V}}}
\def\tW{{\tens{W}}}
\def\tX{{\tens{X}}}
\def\tY{{\tens{Y}}}
\def\tZ{{\tens{Z}}}


% Graph
\def\gA{{\mathcal{A}}}
\def\gB{{\mathcal{B}}}
\def\gC{{\mathcal{C}}}
\def\gD{{\mathcal{D}}}
\def\gE{{\mathcal{E}}}
\def\gF{{\mathcal{F}}}
\def\gG{{\mathcal{G}}}
\def\gH{{\mathcal{H}}}
\def\gI{{\mathcal{I}}}
\def\gJ{{\mathcal{J}}}
\def\gK{{\mathcal{K}}}
\def\gL{{\mathcal{L}}}
\def\gM{{\mathcal{M}}}
\def\gN{{\mathcal{N}}}
\def\gO{{\mathcal{O}}}
\def\gP{{\mathcal{P}}}
\def\gQ{{\mathcal{Q}}}
\def\gR{{\mathcal{R}}}
\def\gS{{\mathcal{S}}}
\def\gT{{\mathcal{T}}}
\def\gU{{\mathcal{U}}}
\def\gV{{\mathcal{V}}}
\def\gW{{\mathcal{W}}}
\def\gX{{\mathcal{X}}}
\def\gY{{\mathcal{Y}}}
\def\gZ{{\mathcal{Z}}}

% Sets
\def\sA{{\mathbb{A}}}
\def\sB{{\mathbb{B}}}
\def\sC{{\mathbb{C}}}
\def\sD{{\mathbb{D}}}
% Don't use a set called E, because this would be the same as our symbol
% for expectation.
\def\sF{{\mathbb{F}}}
\def\sG{{\mathbb{G}}}
\def\sH{{\mathbb{H}}}
\def\sI{{\mathbb{I}}}
\def\sJ{{\mathbb{J}}}
\def\sK{{\mathbb{K}}}
\def\sL{{\mathbb{L}}}
\def\sM{{\mathbb{M}}}
\def\sN{{\mathbb{N}}}
\def\sO{{\mathbb{O}}}
\def\sP{{\mathbb{P}}}
\def\sQ{{\mathbb{Q}}}
\def\sR{{\mathbb{R}}}
\def\sS{{\mathbb{S}}}
\def\sT{{\mathbb{T}}}
\def\sU{{\mathbb{U}}}
\def\sV{{\mathbb{V}}}
\def\sW{{\mathbb{W}}}
\def\sX{{\mathbb{X}}}
\def\sY{{\mathbb{Y}}}
\def\sZ{{\mathbb{Z}}}

% Entries of a matrix
\def\emLambda{{\Lambda}}
\def\emA{{A}}
\def\emB{{B}}
\def\emC{{C}}
\def\emD{{D}}
\def\emE{{E}}
\def\emF{{F}}
\def\emG{{G}}
\def\emH{{H}}
\def\emI{{I}}
\def\emJ{{J}}
\def\emK{{K}}
\def\emL{{L}}
\def\emM{{M}}
\def\emN{{N}}
\def\emO{{O}}
\def\emP{{P}}
\def\emQ{{Q}}
\def\emR{{R}}
\def\emS{{S}}
\def\emT{{T}}
\def\emU{{U}}
\def\emV{{V}}
\def\emW{{W}}
\def\emX{{X}}
\def\emY{{Y}}
\def\emZ{{Z}}
\def\emSigma{{\Sigma}}

% entries of a tensor
% Same font as tensor, without \bm wrapper
\newcommand{\etens}[1]{\mathsfit{#1}}
\def\etLambda{{\etens{\Lambda}}}
\def\etA{{\etens{A}}}
\def\etB{{\etens{B}}}
\def\etC{{\etens{C}}}
\def\etD{{\etens{D}}}
\def\etE{{\etens{E}}}
\def\etF{{\etens{F}}}
\def\etG{{\etens{G}}}
\def\etH{{\etens{H}}}
\def\etI{{\etens{I}}}
\def\etJ{{\etens{J}}}
\def\etK{{\etens{K}}}
\def\etL{{\etens{L}}}
\def\etM{{\etens{M}}}
\def\etN{{\etens{N}}}
\def\etO{{\etens{O}}}
\def\etP{{\etens{P}}}
\def\etQ{{\etens{Q}}}
\def\etR{{\etens{R}}}
\def\etS{{\etens{S}}}
\def\etT{{\etens{T}}}
\def\etU{{\etens{U}}}
\def\etV{{\etens{V}}}
\def\etW{{\etens{W}}}
\def\etX{{\etens{X}}}
\def\etY{{\etens{Y}}}
\def\etZ{{\etens{Z}}}

% The true underlying data generating distribution
\newcommand{\pdata}{p_{\rm{data}}}
\newcommand{\ptarget}{p_{\rm{target}}}
\newcommand{\pprior}{p_{\rm{prior}}}
\newcommand{\pbase}{p_{\rm{base}}}
\newcommand{\pref}{p_{\rm{ref}}}

% The empirical distribution defined by the training set
\newcommand{\ptrain}{\hat{p}_{\rm{data}}}
\newcommand{\Ptrain}{\hat{P}_{\rm{data}}}
% The model distribution
\newcommand{\pmodel}{p_{\rm{model}}}
\newcommand{\Pmodel}{P_{\rm{model}}}
\newcommand{\ptildemodel}{\tilde{p}_{\rm{model}}}
% Stochastic autoencoder distributions
\newcommand{\pencode}{p_{\rm{encoder}}}
\newcommand{\pdecode}{p_{\rm{decoder}}}
\newcommand{\precons}{p_{\rm{reconstruct}}}

\newcommand{\laplace}{\mathrm{Laplace}} % Laplace distribution

\newcommand{\E}{\mathbb{E}}
\newcommand{\Ls}{\mathcal{L}}
\newcommand{\R}{\mathbb{R}}
\newcommand{\emp}{\tilde{p}}
\newcommand{\lr}{\alpha}
\newcommand{\reg}{\lambda}
\newcommand{\rect}{\mathrm{rectifier}}
\newcommand{\softmax}{\mathrm{softmax}}
\newcommand{\sigmoid}{\sigma}
\newcommand{\softplus}{\zeta}
\newcommand{\KL}{D_{\mathrm{KL}}}
\newcommand{\Var}{\mathrm{Var}}
\newcommand{\standarderror}{\mathrm{SE}}
\newcommand{\Cov}{\mathrm{Cov}}
% Wolfram Mathworld says $L^2$ is for function spaces and $\ell^2$ is for vectors
% But then they seem to use $L^2$ for vectors throughout the site, and so does
% wikipedia.
\newcommand{\normlzero}{L^0}
\newcommand{\normlone}{L^1}
\newcommand{\normltwo}{L^2}
\newcommand{\normlp}{L^p}
\newcommand{\normmax}{L^\infty}

\newcommand{\parents}{Pa} % See usage in notation.tex. Chosen to match Daphne's book.

\DeclareMathOperator*{\argmax}{arg\,max}
\DeclareMathOperator*{\argmin}{arg\,min}

\DeclareMathOperator{\sign}{sign}
\DeclareMathOperator{\Tr}{Tr}
\let\ab\allowbreak

%\DeclareSymbolFont{AMSb}{U}{msb}{m}{n} % Redefine \Bbbk to avoid conflict
%\DeclareMathSymbol{\Bbbk}{\mathord}{AMSb}{"7C}
\usepackage{algorithm}
\usepackage{algorithmic}
\newtheorem{theorem}{Theorem}
\newtheorem{Corollary}{Corollary}


\usepackage[most]{tcolorbox}


%% \BibTeX command to typeset BibTeX logo in the docs
\AtBeginDocument{%
  \providecommand\BibTeX{{%
    Bib\TeX}}}
\newcommand{\methodFont}{\texttt}
\usepackage{xspace}

\newcommand{\ours}{\methodFont{MD-UQ}\xspace}
%% Rights management information.  This information is sent to you
%% when you complete the rights form.  These commands have SAMPLE
%% values in them; it is your responsibility as an author to replace
%% the commands and values with those provided to you when you
%% complete the rights form.
\setcopyright{acmlicensed}
\copyrightyear{2025}
% \acmYear{2025}
% \acmDOI{XXXXXXX.XXXXXXX}
%% These commands are for a PROCEEDINGS abstract or paper.
% \acmConference[Conference acronym 'XX]{Make sure to enter the correct
%   conference title from your rights confirmation email}{June 03--05,
%   2018}{Woodstock, NY}
%%
%%  Uncomment \acmBooktitle if the title of the proceedings is different
%%  from ``Proceedings of ...''!
%%
%%\acmBooktitle{Woodstock '18: ACM Symposium on Neural Gaze Detection,
%%  June 03--05, 2018, Woodstock, NY}
% \acmISBN{978-1-4503-XXXX-X/2018/06}


%%
%% Submission ID.
%% Use this when submitting an article to a sponsored event. You'll
%% receive a unique submission ID from the organizers
%% of the event, and this ID should be used as the parameter to this command.
%%\acmSubmissionID{123-A56-BU3}

%%
%% For managing citations, it is recommended to use bibliography
%% files in BibTeX format.
%%
%% You can then either use BibTeX with the ACM-Reference-Format style,
%% or BibLaTeX with the acmnumeric or acmauthoryear sytles, that include
%% support for advanced citation of software artefact from the
%% biblatex-software package, also separately available on CTAN.
%%
%% Look at the sample-*-biblatex.tex files for templates showcasing
%% the biblatex styles.
%%

%%
%% The majority of ACM publications use numbered citations and
%% references.  The command \citestyle{authoryear} switches to the
%% "author year" style.
%%
%% If you are preparing content for an event
%% sponsored by ACM SIGGRAPH, you must use the "author year" style of
%% citations and references.
%% Uncommenting
%% the next command will enable that style.
%%\citestyle{acmauthoryear}
\usepackage{caption}
\usepackage{multirow}
\usepackage[capitalize]{cleveref}
\newcommand{\tj}[1]{\textcolor{red}{\textbf{hua: #1}}}

\newcommand{\hua}[1]{\textcolor{blue}{\textbf{hua: #1}}}
\newcommand{\vagelis}[1]{\textcolor{magenta}{\textbf{vagelis: #1}}}

%%
%% end of the preamble, start of the body of the document source.
\begin{document}

%%
%% The "title" command has an optional parameter,
%% allowing the author to define a "short title" to be used in page headers.
\title{Uncertainty Quantification of Large Language Models through Multi-Dimensional Responses}

%%
%% The "author" command and its associated commands are used to define
%% the authors and their affiliations.
%% Of note is the shared affiliation of the first two authors, and the
%% "authornote" and "authornotemark" commands
%% used to denote shared contribution to the research.
% \author{Ben Trovato}
% \authornote{Both authors contributed equally to this research.}
% \email{trovato@corporation.com}
% \orcid{1234-5678-9012}
% \author{G.K.M. Tobin}
% \authornotemark[1]
% \email{webmaster@marysville-ohio.com}
% \affiliation{%
%   \institution{Institute for Clarity in Documentation}
%   \city{Dublin}
%   \state{Ohio}
%   \country{USA}
% }

\author{Tiejin Chen}
\affiliation{%
  \institution{Arizona State University}
  \city{Tempe}
  \country{AZ}}
\author{Xiaoou Liu}
\affiliation{%
  \institution{Arizona State University}
  \city{Tempe}
  \country{AZ}}

\author{Longchao Da}
\affiliation{%
  \institution{Arizona State University}
  \city{Tempe}
  \country{AZ}}


\author{Jia Chen}
\affiliation{%
  \institution{University of California, Riverside}
  \city{Riverside}
  \country{CA}}

\author{Vagelis Papalexakis}
\affiliation{%
  \institution{University of California, Riverside}
  \city{Riverside}
  \country{CA}}

\author{Hua Wei}
\affiliation{%
  \institution{Arizona State University}
  \city{Tempe}
  \country{AZ}}


\begin{abstract}
Large Language Models (LLMs) have demonstrated remarkable capabilities across various tasks due to large training datasets and powerful transformer architecture. However, the reliability of responses from LLMs remains a question. Uncertainty quantification (UQ) of LLMs is crucial for ensuring their reliability, especially in areas such as healthcare, finance, and decision-making.  Existing UQ methods primarily focus on semantic similarity, overlooking the deeper knowledge dimensions embedded in responses. We introduce a multi-dimensional UQ framework that integrates semantic and knowledge-aware similarity analysis. By generating multiple responses and leveraging auxiliary LLMs to extract implicit knowledge, we construct separate similarity matrices and apply tensor decomposition to derive a comprehensive uncertainty representation. This approach disentangles overlapping information from both semantic and knowledge dimensions, capturing both semantic variations and factual consistency, leading to more accurate UQ. Our empirical evaluations demonstrate that our method outperforms existing techniques in identifying uncertain responses, offering a more robust framework for enhancing LLM reliability in high-stakes applications.

\end{abstract}
%%
%% The code below is generated by the tool at http://dl.acm.org/ccs.cfm.
%% Please copy and paste the code instead of the example below.
%%
\begin{CCSXML}
<ccs2012>
 <concept>
  <concept_id>00000000.0000000.0000000</concept_id>
  <concept_desc>Do Not Use This Code, Generate the Correct Terms for Your Paper</concept_desc>
  <concept_significance>500</concept_significance>
 </concept>
 <concept>
  <concept_id>00000000.00000000.00000000</concept_id>
  <concept_desc>Do Not Use This Code, Generate the Correct Terms for Your Paper</concept_desc>
  <concept_significance>300</concept_significance>
 </concept>
 <concept>
  <concept_id>00000000.00000000.00000000</concept_id>
  <concept_desc>Do Not Use This Code, Generate the Correct Terms for Your Paper</concept_desc>
  <concept_significance>100</concept_significance>
 </concept>
 <concept>
  <concept_id>00000000.00000000.00000000</concept_id>
  <concept_desc>Do Not Use This Code, Generate the Correct Terms for Your Paper</concept_desc>
  <concept_significance>100</concept_significance>
 </concept>
</ccs2012>
\end{CCSXML}

% \ccsdesc[500]{Do Not Use This Code~Generate the Correct Terms for Your Paper}
% \ccsdesc[300]{Do Not Use This Code~Generate the Correct Terms for Your Paper}
% \ccsdesc{Do Not Use This Code~Generate the Correct Terms for Your Paper}
% \ccsdesc[100]{Do Not Use This Code~Generate the Correct Terms for Your Paper}

%%
%% Keywords. The author(s) should pick words that accurately describe
%% the work being presented. Separate the keywords with commas.
\keywords{Large Language Model, Uncertainty Quantification}

%% A "teaser" image appears between the author and affiliation
%% information and the body of the document, and typically spans the
%% page.


% \received{20 February 2007}

%%
%% This command processes the author and affiliation and title
%% information and builds the first part of the formatted document.
\maketitle

\section{Introduction}\label{sec:intro}

Large Language Models (LLMs) demonstrate strong performance across natural language processing tasks, yet their architectural complexity and limited interpretability can produce unreliable outputs. 
This presents significant challenges in critical domains such as healthcare, where output errors carry serious consequences. 
Confidence estimation methods have emerged to quantify output reliability. 
The field connects closely with uncertainty quantification in natural language generation, as both address output trustworthiness. 
Current approaches divide into consistency-based methods, which analyze agreement across multiple outputs, and internal-states methods that leverage model-specific features like output probabilities.
Despite advances in these approaches, developing robust evaluation frameworks remains a central challenge.

%Large Language Models (LLMs) have achieved notable success in various natural language processing (NLP) tasks, such as text generation, question answering, and reasoning. However, their complex architectures and lack of interpretability often lead to uncertain, incorrect, or ambiguous outputs. This raises critical concerns about their reliability, especially in high-stakes domains like healthcare and decision-making, where errors can have severe consequences. Confidence estimation has emerged as a key tool for addressing these concerns by quantifying the trustworthiness of LLM outputs.


%In natural language generation (NLG), confidence estimation is closely tied to uncertainty quantification (UQ), as both aim to assess the reliability of model outputs. 
%% Confidence scores are often derived from uncertainty measures, with higher uncertainty corresponding to lower confidence. 
%Existing methods for confidence estimation in LLMs can be broadly categorized into two types: consistency-based black-box methods, which rely on agreement among multiple generated responses, and internal-states-based methods, which use model-specific information such as output probabilities or self-evaluation mechanisms. 
%While these approaches have advanced the field, the evaluation of confidence estimation methods remains a significant challenge in their evaluation process.\cc{consider say we mainly do UQ evaluation earlier, cut the context. at least merge the first two paragraphs}

%Currently, the evaluation frameworks for confidence measures in NLG primarily rely on correctness labels to compute metrics such as area under the receiver operating characteristic curve (AUROC) and accuracy-rejection curve (AUARC). 
%These frameworks follow a general pipeline: 
%(1) generating predictions from the model,
%(2) assigning correctness labels to the predictions using a \textit{correctness function} $f(\cdot)$, and 
%(3) calculating evaluation metrics based on these labels. 
%However, the reliance on correct labels introduces several limitations. 
%Human evaluation, though reliable, is time-consuming and impractical for large-scale datasets. 
%Automated reference-based metrics like BLEU and ROUGE fail to account for semantically correct but differently phrased responses. 
%LLM-based evaluators offer flexibility but still suffer from systematic biases~\cite{lin2022truthfulqa}, such as favoring outputs from similar models\cc{add citation here?} or being sensitive to prompt variations—issues that lack concrete evidence in prior work\cc{add citation here?}. 

Current evaluation frameworks for NLG confidence measures rely on correctness labels to compute metrics such as AUROC and AUARC. 
These frameworks follow a three-step process: generating model predictions, labeling correctness via a function $f(\cdot)$, and calculating metrics. 
This label-dependent approach faces several constraints. While human evaluation provides reliable correctness ground truth, it cannot scale to large datasets. 
Metrics based on reference matching, such as BLEU and ROUGE, fail to recognize semantically equivalent responses phrased differently.
% Reference matching-based metrics like BLEU and ROUGE miss semantically equivalent responses with different phrasing. 
% LLM-based evaluators offer greater capability and scalability, but are still noisy, and could introduce systematic biases, such as favor over responses generated by similar LMs~\cite{panickssery2024llm} or over length~\cite{lin2022truthfulqa}.
LLM-based evaluators offer greater capability but remain noisy and may introduce systematic biases, such as favoring responses generated by themselves or similar LMs~\cite{panickssery2024llm}, or preferring longer responses~\cite{lin2022truthfulqa}.
Moreover, running such evaluators could be expensive.
% LLM-based evaluators improve capability and scalability but remain noisy and may introduce systematic biases, such as favoring responses generated by similar LMs~\cite{panickssery2024llm} or preferring longer responses~\cite{lin2022truthfulqa}.
% (also observed in our own experiments) or over length~\cite{lin2022truthfulqa}.
% The prompt used to eli
%including outputs from similar models\cc{add citation here?} and prompt sensitivity—phenomena that remain empirically understudied\cc{add citation here?}.


%More critically, flaws in the correctness function $f(\cdot)$ can propagate through the evaluation pipeline, affecting evaluation metrics like AUROC. 
%This fragility can lead to misleading conclusions about which confidence estimation method performs better, particularly when different methods yield similar performance. 
%These issues highlight the need for alternative evaluation frameworks with reliable correctness labels.

% More critically, flaws in the correctness function $f(\cdot)$ propagate through the evaluation pipeline, affecting metrics like AUROC. 

Flaws in the correctness function $f(\cdot)$ propagate through the evaluation pipeline, affecting metrics like AUROC. This sensitivity becomes particularly problematic when comparing confidence estimation methods with similar performance. Such limitations underscore the need for evaluation frameworks that establish correctness more reliably.


%In this paper, we propose a simple evaluation framework that addresses these limitations by leveraging multiple-choice question-answering (QA) datasets. 
%Our approach removes the dependence on a correctness function by using the structure of multiple-choice QA tasks to evaluate confidence measures directly. 
%Importantly, our framework is complementary to existing pipelines rather than a replacement; it provides an additional perspective to validate the discriminative power of confidence measures across different evaluation settings.

In this paper, we propose \uqeval, a simple, efficient yet effective evaluation framework that eliminates the dependence on unreliable correctness functions.
% leverages multiple-choice question-answering (QA) datasets to address these limitations.
\textit{The key insight is to leverage multiple-choice question-answering (QA) datasets, which inherently provide gold-standard answer choices at no cost.} 
With these definitive labels, our framework bypasses the ambiguity of determining correctness via correctness function $f(\cdot)$ and ensures an objective assessment of confidence estimation methods. 
% By leveraging multiple-choice QA datasets, our proposed evaluation framework ensures an objective assessment of confidence estimation methods.
% Unlike conventional evaluation pipelines that rely on an often noisy correctness function $f(\cdot)$ to determine whether a model’s prediction is right or wrong, our approach directly utilizes gold-standard answer choices in multiple-choice QA tasks. 
% \textit{Our approach eliminates the dependence on unreliable correctness functions} and ensures an objective assessment of confidence estimation methods.
% By exploiting the inherent structure of multiple-choice QA datasets, our approach eliminates dependence on unreliable correctness. 
Rather than replacing existing evaluation pipelines, our framework complements them, offering an additional lens to assess the discriminative power of confidence estimation methods.
% across different evaluation settings.
\cref{fig:pipeline} shows how our proposal (green) and the existing evaluation pipeline (blue) differ, yet complement each other.
% In this paper, we propose a simple yet effective evaluation framework that eliminates reliance on noisy correctness functions. Our key insight is to leverage multiple-choice question-answering (QA) datasets, which inherently provide gold-standard answer choices. By using these definitive labels, our framework bypasses the ambiguity of determining correctness via 
% f(⋅) and ensures an objective assessment of confidence estimation methods. 
% Rather than replacing existing evaluation pipelines, our approach complements them, offering an additional lens to assess the discriminative power of confidence measures. \cref{fig:pipeline} illustrates how our proposal (green) differs from, yet aligns with, the existing pipeline (blue).
Our contributions are summarized as follows:
\begin{itemize}[leftmargin=*,nosep]
    \item We demonstrate that commonly used evaluation methods for NLG confidence measures are sensitive to noise in correctness labels, which can lead to misleading conclusions about evaluation metrics and rankings of different confidence estimation approaches.
    %We show that the most popular evaluation methods for NLG confidence measures are prone to noise in the ``correctness label`, which could further lead to problematic conclusions. %(about which measure is better)
    \item We propose a simple yet effective method that utilizes multiple-choice QA datasets to evaluate confidence measures, supporting both internal-states-based white-box and consistency-based black-box methods.
    \item Extensive experiments across recent LLMs and QA datasets verify that \uqeval produces stable evaluations broadly consistent with existing methods, while eliminating the need for expensive correctness functions.
    %We conduct extensive experiments on recent LLMs and popular QA datasets to verify that \uqeval provides stable evaluation results that are often consistent with existing evaluation methods, without the costly correctness function.
\end{itemize}


\section{Preliminaries}
\label{sec:background}

\subsection{Problem Formulation}
\label{subsec:problem}
Modern uncertainty quantification (UQ) for black-box LLMs operates through two sequential stages: similarity measurement between responses and uncertainty estimation from these similarities. Let $\mathcal{M}$ denote a black-box LLM generating $n$ responses $\{A^1,\ldots,A^n\}$ to input $Q$. The UQ task estimates confidence $U $ through:


\begin{equation}
U = f(\mathbf{S}), \quad \mathbf{S} \in \mathbb{R}^{n\times n}
\end{equation}

\noindent where $\mathbf{S}$ is the similarity matrix with the $(i, j)$-th entry capturing the proximity  of responses $A^i$ and $A^j$, and $f$ represents the estimation strategy. This formulation enables UQ without accessing internal model probabilities.

\subsection{Measuring Response Similarities}
\label{subsec:similarities}
The foundation of reliable UQ lies in effective similarity measurement. We analyze two complementary approaches capturing different aspects of response quality.

\subsubsection{Semantic Dimension}
\label{subsubsec:semantic}
Semantic similarity focuses on surface-level consistency between responses. The Jaccard index \citep{lin2023generating} offers simple lexical comparison:

\begin{equation}
s_{ij} = \frac{|A^i \cap A^j|}{|A^i \cup A^j|}
\end{equation}

While computationally efficient, Jaccard ignores word order and semantics. For deeper analysis, there is another common way to compute the semantic similarity, which uses DeBERTa's NLI capabilities~\citep{he2021deberta}:

\begin{equation}
s_{ij} = \frac{1}{2}\left(P_{\text{entail}}(A^i,A^j) + P_{\text{entail}}(A^j,A^i)\right)
\end{equation}

Where $P_{\text{entail}}(A^i,A^j)$ the probability of $A^i$ entails $A^j$ that is output from the NLI model.

Compared with the Jaccard index, NLI-based scoring better captures semantic equivalence but remains sensitive to syntactic variation. For instance, paraphrased factual statements may receive low scores despite equivalent meaning. To the question \textit{How many students became heroes}, The response \textit{These three became heroes} and the response \textit{Andrew Willis, Chris Willis, Reece Galea} share the factual knowledge that three students became heroes while their similarity from NLI models will be low as 0.015. Therefore, relying solely on responses from the semantic dimension may result in information loss.

\subsubsection{Knowledge Dimension} 
\label{subsubsec:knowledge}

The knowledge dimension operates through a structured pipeline that transforms raw responses into factual representations. 
Given a question $Q$ and its original response $A^i$, a knowledge representation $K^i$ could be generated through a knowledge mapping process by extracting explicit claims: $K^i = \mathcal{M}_{\text{aux}}(Q, A^i)$ and augmenting the response, where  $\mathcal{M}_{\text{aux}}$ denotes for an auxiliary LLM. Specifically, we could use an LLM to augment with prompts taking into the question and original response:

\begin{tcolorbox}[colback=gray!5!white, colframe=gray!75!green, title=\textbf{Prompt Example for Knowledge Mapping}] 

 \ding{182}: Extract all factual claims from this response $\langle A^i \rangle$, phrased as standalone statements independent of specific wording. 

 \ding{183}: Include only information directly relevant to answering the question: $\langle Q \rangle$.

\end{tcolorbox}

This claim extraction disentangles implicit knowledge from surface semantics and removes stylistic variations while preserving core factual content.




\subsection{Estimating Uncertainty}
\label{subsec:estimation}
With similarity matrices constructed, existing UQ methods employ two principal ways to estimate uncertainty, each offering unique advantages and limitations.

\subsubsection{Number of Semantic Sets (UNumSet)}
\label{subsubsec:numset}
Proposed by \citet{kuhn2023semantic}, this method groups responses into equivalence classes using bidirectional entailment checks from an NLI model. Formally, responses $A^i$ and $A^j$ are merged into the same semantic set if:
\begin{equation}
P_{\text{entail}}(A^i, A^j) > P_{\text{contra}}(A^i, A^j) \quad \text{and} \quad P_{\text{entail}}(A^j, A^i) > P_{\text{contra}}(A^j, A^i).
\end{equation}
The uncertainty measure $U_{\text{NumSet}}$ equals the number of resulting semantic sets. This approach aligns with spectral graph theory. Because when using binary adjacency matrices ($W_{ij} \in {0,1}$), the number of zero eigenvalues in the graph Laplacian corresponds to the number of connected components \citep{von2007tutorial}. While this method discretizes continuous semantic relationships, it fails to capture partial meaning overlaps.

\subsubsection{Graph Laplacian}
\label{subsubsec:laplacian}
Building on spectral graph principles \citep{agaskar2013spectral,lin2023generating}, this method quantifies uncertainty through the eigenvalues ${\lambda_k}$ of the normalized graph Laplacian $L = I - D^{-1/2}WD^{-1/2}$:
\begin{equation}
U_{\text{EigV}} = \sum_{k=1}^n \max(0, 1 - \lambda_k).
\end{equation}

Here, eigenvalues $\lambda_k$ encode connectivity. fragmented graphs (low consistency in responses and thus high uncertainty) have more small eigenvalues. Compared with $U_{\text{NumSet}}$, this method is able to capture possible overlapping semantic relationships.


\subsection{Dimensional Analysis}
Now, we compare the difference between similarity matrices from knowledge and semantic dimensions. In detail, we use the NLI model to obtain the similarity matrix. In \cref{tab:similarity_matrix_stat}, we present the mean values and the proportion of similarity scores greater than 0.55 in the similarity matrices. The results show that similarity matrices in the knowledge dimension have more large value as well as a larger mean value. This reveals the knowledge dimensions' superior consistency. These results highlight the importance of multi-dimensional analysis since semantic features capture response variability, while knowledge features track factual consistency hidden behind the semantic features. Our tensor decomposition effectively combines these complementary signals.

\begin{table}[t!]
    \centering
    \begin{tabular}{lcc}
        \toprule
        Dataset & Proportion (\%) & Mean Value \\
        \midrule
        \multicolumn{3}{c}{\textbf{Semantic Similarity}} \\
        \midrule
        Coqa & 52.97 & 0.5430 \\
        nq\_open & 17.11 & 0.1839 \\
        Trivialqa & 51.15 & 0.5154 \\
        \midrule
        \multicolumn{3}{c}{\textbf{Knowledge Similarity}} \\
        \midrule
        Coqa & 57.40 & 0.5723 \\
        nq\_open & 31.16 & 0.3281 \\
        Trivialqa & 60.17 & 0.6058 \\
        \bottomrule
    \end{tabular}
    \caption{Proportion of similarity values greater than 0.55 and mean similarity values for the similarity matrices in the semantic space and the knowledge space. The results show that the knowledge similarity matrix has larger values.}
    \vspace{-10mm}
    \label{tab:similarity_matrix_stat}
\end{table}
\section{ \uqeval: A framework for Assessing Confidence Estimation}\label{sec:method}
At a high level, existing evaluation frameworks
% the evaluation pipeline described in \cref{sec:prelim:old_eval} 
for $C_{\mathcal{M}}$ includes three main steps (blue path in \cref{fig:pipeline}):
\begin{enumerate}[nosep]
    \item Generate $\predSeq$ from $\mathcal{M}$ given the input $\xInput_i$.
    \item Determine the correctness label of $\predSeq$ using the function $\acc(\cdot,\xInput)$.
    \item Compute evaluation metrics such as AUROC. A higher metric value indicates that $C_{\mathcal{M}}$ is a ``better'' confidence estimation.
\end{enumerate}
The main limitation of this general pipeline lies in $\acc$ in step 2. 
Existing evaluation frameworks all implicitly assume step 1---that the confidence measure $C_{\mathcal{M}}$ must apply to generated sequences $\predSeq$. 
While this might hold for consistency-based uncertainty measures, where response divergence indicates uncertainty, it does not extend to confidence measures. 
In other words, we could relax step 1 in order to improve step 2.
% Consider, for instance, Eccentricity~\cite{lin2024generating}: The uncertainty measure $U_{ecc}$ computes the average distance between sampled generation embeddings and their centroid, while the confidence measure evaluates a specific generation. 
% Consequently, we could simply use the sampled generations to construct the embedding space, yet we can still measure the distance of \textit{any} sequence to the center of embeddings.


%The previous discussion suggests that existing evaluation frameworks all bear an implicit assumption: The confidence measure $C_{\mathcal{M}}$ must be applied to the generations $\predSeq$.
%While this might be true for the case of most existing consistency-based uncertainty measures, where a high degree of divergence of the sampled responses is a strong indicator of high uncertainty, this is \textit{not} the case for confidence measures.
%Take Eccentricity~\cite{lin2024generating} as an example: The uncertainty measure $U_{ecc}$ is based on the average distance of the embeddings of the sampled generations to the center of all embeddings and the confidence measure is that of a particular generation. 
%Consequently, we could simply use the sampled generations to construct the embedding space, yet we can still measure the distance of \textit{any} sequence to the center of embeddings. 

\textit{Our main proposal in this paper is to adapt multiple-choice datasets to evaluate confidence measures designed for free-form NLG.}
Unlike free-form NLG datasets, multiple-choice datasets provide inherent correctness values for options, eliminating the need for an explicit correctness function. 
If we simply ``pretend'' that these options are free-form generations from the base LM, we can directly evaluate the confidence measure quality. 
As \cref{fig:pipeline} shows, the approach differs from existing evaluation pipelines only in applying confidence estimation methods to multiple-choice options.




Consider the QASC~\cite{khot2020qasc} dataset as an example,
each problem comes with a question $\xInput$ and a few choices, $o_1,\ldots,o_K$. 
Unlike what such datasets were designed for, we re-format the input prompt as a free-form NLG question, as illustrated in \cref{fig:qasc_example}, as if the base LLM generated each option itself, in different runs.
In what follows, we first explain explain slight nuances in applying internal state-based white-box confidence measures as well as consistency-based black-box ones. 
%shows a reformatted question from the QASC dataset.

\begin{figure}[t]
  \includegraphics[width=\columnwidth]{figures/qasc_example.pdf}
  % \caption{A reformatted question example from the QASC dataset. The Question and Choices are directly from the original dataset, while our prompt is specifically designed for LLM input to generate open-form responses.}
  \caption{
  We reformat each option from the multiple-choice question (left), by injecting the \smash{\colorbox{yellow!40}{{{\color{blue}option}}}} to a free-form QA \smash{\colorbox{green!40}{prompt}}.
  One could typically apply any confidence estimation method by treating this \smash{\colorbox{yellow!40}{{{\color{blue}option}}}} as if it was generated by the base LM.
  For black-box confidence measures that require additional responses, we only feed the \smash{\colorbox{green!40}{prompt}} to the base LM.
  }
  \label{fig:qasc_example}
\vspace{-3mm}
\end{figure}



\textbf{Logit or Internal State-Based Measures} typically examine the internals of a LM when it generates a particular response.
The nature of the free-form generation task allows us to simply plug-in the option $o_i$ into the corresponding location of the prompt, and extract similar information that allows us to evaluate the confidence\footnote{In fact, this was the practice to compute \baselineSL for actual generations. For example, \url{https://github.com/lorenzkuhn/semantic_uncertainty/blob/main/code/get_likelihoods.py} and \url{https://huggingface.co/docs/transformers/perplexity}.}.
% One concern is whether these options are too ``different'' from what the LM would otherwise generate itself.
% As exemplified in \cref{fig:true_distribution}, $C(o_i)$ in general shares a similar distribution to $C(\predSeq_i)$. 

% Taking CommonsenseQA~\footnote{A multiple-choice dataset for our experiment is described in Section~\ref{sec:experiments}} as an example, we compare the logit distribution of the correct answer choices with the distributions of other LLM-generated responses. 
% As shown in \cref{fig:true_distribution}, the distributions exhibit notable similarities, indicating that logit-based confidence estimation can capture underlying patterns shared between correct answer choices and free-form generations.

% \begin{figure}[t]
%   \includegraphics[width=\columnwidth]{figures/true_distribution_new.png}
%   \caption{Confidence score distribution for the \baselinePTrue method on the CQA dataset. The blue distribution represents 20 open-form responses, while the red distribution corresponds to the correct option. }
%   \label{fig:true_distribution}
% \end{figure}


\textbf{Consistency-based Confidence Measures}
Unlike logit-based or internal-state-based measures, consistency-based confidence measures typically rely on an estimate of the predictive distribution, denoted as $\PredDist$, and any response that is closer to the center of the distribution (in the ``semantic space'') is considered to be of higher confidence. 
Consider methods from~\citet{lin2024generating} as an example. To preserve the integrity of the predictive distribution, we first sample $n$ responses from $\PredDist$ as usual, and then iteratively include one option $o_i$ at a time to compute its associated confidence score~\cite{rivera-etal-2024-combining,manakul-etal-2023-selfcheckgpt}. 
\cref{alg:confidence_score} outlines this process. 


\renewcommand{\algorithmicrequire}{\textbf{Input:}}
\renewcommand{\algorithmicensure}{\textbf{Output:}}

\begin{algorithm}[t]
\small
% \caption{Confidence Score Computation in the Black-Box Method}
\caption{Consistency-based Confidence Estimation for Any Sequences}
\label{alg:confidence_score}
\begin{algorithmic}[1]
    \Require $\xInput$, $\mathcal{M}$, candidate sequences $A = \{a_1, \dots, a_K\}$
    \Ensure $\{C_{\mathcal{M}}(\xInput,a_1), \dots,C_{\mathcal{M}}(\xInput,a_K)\}$ 
    
    \State Generate $S = \{\predSeq_1, \dots, \predSeq_{n}\}$ using $\mathcal{M}$ for question $\xInput$
    \State Compute pairwise similarity matrix $M$ of $S$.% \in \mathbb{R}^{|S'| \times |S'|}$
    % \State Construct full response set $S' = S \cup A$, where $|S'| = n+K$
    % \State Compute pairwise similarity matrix $M_{sim} \in \mathbb{R}^{|S'| \times |S'|}$
    
    \For{each $a_i \in A$}
        \State Compute a new similarity matrix $M_i$ of $S\cup\{a_i\}$, reusing $M$. % \in 
        % \State Form subset $S_i = S \cup \{a_i\}$, where $|S_i| = n+1$
        % \State Extract pairwise similarity matrix $M_{sim}^{(i)} \in \mathbb{R}^{|S_i| \times |S_i|}$
        \State Compute confidence score $C_{\mathcal{M}}(\xInput,a_i)$ using $M_i$. %degree matrix or eccentricity of $M_{sim}^{(i)}$
    \EndFor

    \State \Return $\{C_{\mathcal{M}}(\xInput,a_1), \dots,C_{\mathcal{M}}(\xInput,a_K)\}$ 
\end{algorithmic}
\end{algorithm}

% Since computing the similarity matrix is the most computationally expensive step, the subsequent 5 confidence score calculations reuse precomputed similarity values. As a result, the additional computations take less than 1 minute in total, ensuring efficiency.

% Consistency-based confidence measures are a little different from .


\paragraph{Remarks}
Our proposal relaxes step 1 at the beginning of this section, allowing for $\predSeq^*=o_i$ not sampled from $\PredDist$.
This is not to be misunderstood as a proposal to \textit{replace} the current pipeline (\cref{sec:prelim:old_eval})---rather, it is \textit{complementary}.
The rationale is that if a good confidence measure predicts the correctness well, it should perform well in \textit{both} evaluation frameworks.
In fact, any $o_i\in\Sigma^*$ that does not violate the generation configuration, has a non-zero probability to be sampled from $\PredDist$, and a robust confidence measure should be expected to model it well.
% \fontred{
% In fact, any $o_i$, as long as it does not violate the generation config, could be sampled from $\PredDist$ given enough time.
% % As we will see in \cref{sec:exp}, even though $o_i$ are not sampled from $\PredDist$, we do not observe a big distribution shift in terms of the confidence values as well.
% }
% Note that we do not advise \textit{replacing} the existing valuation 

% It is important to note that our method only obtains correctness labels for $o_i$. Consequently, when computing AUROC, AUARC, and other evaluation metrics, we only consider confidence values associated with these options.\textcolor{red}{already mention it in \cref{sec:metrics} }

\section{Experiments}

We conduct comprehensive experiments across multiple datasets and model architectures to validate our method's ability to decouple explanation robustness from classification robustness. Our evaluation addresses three key research questions:
\begin{itemize}
    \item \textbf{RQ1:} Does \ours have better quantify uncertainties?
    \item \textbf{RQ2:} How do different ensemble methods and information from both dimensions help?
    \item \textbf{RQ3:} Is\ours robust to different settings? 
\end{itemize}


\begin{table}[H]
\centering
\resizebox{!}{0.11\textwidth}{
\begin{tabular}{@{}lc@{}}
\toprule
\textbf{Measure} & \textbf{Details} \\ 
\midrule
$U_{\textit{Eigv}}(Dis)$ & \multicolumn{1}{c}{Spectral eigenvalue on the disagreement.} \\ 
$U_{\textit{Ecc}}(Dis)$ & \multicolumn{1}{c}{Average distance in responses' disagreement.} \\ 
$U_{\textit{Degree}}(Dis)$ & \multicolumn{1}{c}{Degree of disagreement similarity Matrix.} \\ 
$U_{\textit{Eigv}}(Agre)$ & \multicolumn{1}{c}{Spectral eigenvalue on the agreement.} \\ 
$U_{\textit{Ecc}}(Agre)$ & \multicolumn{1}{c}{Average distance in responses' agreement.} \\ 
$U_{\textit{Degree}}(Agre)$ & \multicolumn{1}{c}{Degree Matrix of agreement Matrix.} \\ 
$p(true)$ & \multicolumn{1}{c}{Entropy of knowledge dimension responses} \\ 
$D-UE$ & \multicolumn{1}{c}{eigenvalue from Laplacian of a directional graph} \\ 

\bottomrule
\end{tabular}}
\vspace{-1mm}
\caption{The baseline methods and explanations.}
\vspace{-5mm}
\label{tab:baslines}
\end{table}

\subsection{Experimental Setup}
\label{sec:setup}
\subsubsection{Datasets} As mentioned in \cref{sec:background}, following prior works~\cite{lin2022teaching}, we focus on open-form question-answering 
(QA) tasks in this paper. We adopt 4 different classic QA datasets. Coqa~\cite{reddy2019coqa} is a conversational question-answering dataset that contains dialogues with free-form answers grounded in diverse passages, which is the easiest dataset among all datasets. HotpotQA~\cite{yang2018hotpotqa} is a multi-hop QA dataset that demands reasoning over multiple Wikipedia paragraphs to derive correct answers. NQ-Open~\cite{kwiatkowski2019natural} consists of real-world queries from Google Search, requiring models to retrieve and answer questions without explicit context, which is the hardest dataset. 
\subsubsection{Models to Evaluate} We evaluate \ours on Llama family~\cite{touvron2023llama}, which is the one of the most popular LLMs. In detail, we use Llama-2-13b and Llama-2-7B to demonstrate the effectiveness of \ours with different model sizes and use Llama-3.1-8B~\cite{dubey2024llama} to that \ours could also work on the different version of Llama. To further demonstrate the generalization ability for other architectures,  we also use Phi4~\cite{abdin2024phi} and Deepseek-R1-distill-7B~\cite{guo2025deepseek} in our paper.


%\textcolor{red}{Not sure whether there is a section labeled as "sec eva metric" refered by Sec. }

\subsubsection{Evaluation Metrics} Effective uncertainty measures should accurately represent the reliability of LLM responses, with higher uncertainty more likely leading to incorrect generations and vice versa~\cite{lin2023generating,kuhn2023semantic}. Following prior works~\cite{lin2023generating,da2024llm}, we mainly use UQ values to predict whether an answer is correct or not. Following prior works~\cite{lin2023generating,da2024llm}, we will use Area Under Receiver Operating Characteristic (AUROC) and Area Under Accuracy Rejection Curve (AUARC) as evaluation metrics, where a higher AUROC or AUARC demonstrates better uncertainty measures. To compute AUROC and AUARC, the accuracy of each original response is required. Following previous works~\cite{da2024llm,lin2023generating}, we use another LLM to provide correctness from 0-100 to each response. If the correctness is greater than 70, we label the response as correct. In this paper, we use Qwen-34B~\cite{bai2023qwen} to evaluate the correctness.

\subsubsection{Knowledge Extracted Models} In this paper, we mainly use llama-2-13b~\cite{touvron2023llama} as the auxiliary models to extract the knowledge dimension of responses. To demonstrate the robustness of \ours with different knowledge-extracted models, we also contain the results for different LLMs as knowledge-extracted models.

\subsubsection{Baselines} In this paper, we compared \ours with baselines that use semantic dimension response and knowledge dimension response. For semantic dimension, we mainly compared with methods that come from \citet{lin2023generating}. In detail, we incorporate six distinct methods from \citet{lin2023generating}, which differ based on the operations applied after computing similarity and whether they utilize agreement (entailment) probabilities or disagreement (contradiction) logits to construct the similarity matrix. For knowledge dimension, we use D-UE~\cite{da2024llm} and $p(true)$~\cite{kadavath2022language} as the baselines. Note that we use $p(true)$ on the knowledge dimension of response. We show the detailed explanations of all baselines in \cref{tab:baslines}

\begin{figure*}[t]
\centering
\begin{minipage}[t]{0.32\linewidth}
  \centering
  \includegraphics[width=\linewidth,trim=0 0 0 1cm, clip]{images/ablation.pdf}
  \captionof{figure}{Ablation studies that show that \ours fully utilizes all the information from both dimensions.}
  \label{fig:ablation}
\end{minipage}\hfill
\begin{minipage}[t]{0.32\linewidth}
  \centering
  \includegraphics[width=\linewidth]{images/knowledge_extract.pdf}
  \captionof{figure}{Performance for different knowledge extract models on CoQA and NQ\_Open with llama3.1.}
  \label{fig:knowledge_extract}
\end{minipage}\hfill
\begin{minipage}[t]{0.32\linewidth}
  \centering
  \includegraphics[width=\linewidth]{images/Jacc.pdf}
  \captionof{figure}{Performance that uses Jaccard similarity on CoQA and NQ\_Open with llama3.1.}
  \label{fig:jacc}
\end{minipage}
\end{figure*}

\subsection{Does \ours have better quantify uncertainties? (RQ1)}
\label{sec:main_result}
In this section, we explore whether \ours has better uncertainties compared with state-of-the-art uncertainty quantification methods. In \cref{tab:main_results}, we compare \ours with 8 baselines across three different datasets and five different models as introduced in \cref{sec:setup} In detail, we have the following observations:

\noindent $\bullet$ Compared with all baseline methods, \ours achieves the best performance overall. Especially when we consider AUROC. For AUARC, \ours achieves the best performance for NQ\_Open while \ours also achieves the comparable performance for CoQA in most scenarios. These results demonstrate that \ours has better quantify uncertainties overall. \\
\noindent $\bullet$ Among all datasets, \ours achieves the highest performance improvement on NQ\_Open, which is the most difficult dataset among all datasets and may lose to baselines for an easier dataset like CoQA. This indicates \ours could perform even better when the task is harder, where uncertainty quantification is more important. \\
\noindent $\bullet$ Two different ensemble methods show very similar results. Min strategy performs better than the sum strategy under $61.51\%$ situations, indicating that difficult datasets might also have more complex structures that single one tensor decomposition might oversight some information while using min structure could reduce such oversight by considering the best cases. However, both ensemble methods show a better performance than all baselines, which proves the effectiveness of tensor decomposition. \\

From these results, we get a conclusion that overall, \ours have better uncertainties.


\subsection{How do different ensemble methods and information from both dimensions help? (RQ2)}
\label{sec:ablation}
In this section, we use more experiments to prove the necessity of using information from both semantic and knowledge dimensions as well as using tensor decomposition. In detail, we consider the following methods: 1) \ours with only semantic responses, 2) \ours with only knowledge responses and 3) Concatenating similarity matrices from semantic and knowledge dimensions into a 2D matrix and applying SVD, 4) only using one tensor decomposition. In \cref{fig:ablation}, we show the comparison between \ours and other methods.  The results show that \ours consistently outperforms its variants and SVD method that repeated information will dominate the features, showing the effectiveness of our framework.




\subsection{Is \ours robust to different settings? (RQ3)}
\subsubsection{Different Knowledge Extracted Models} Knowledge extracted models influence the claim extraction in \ours as stated in \cref{subsubsec:knowledge}. Therefore, in this section, We test the robustness of \ours on various knowledge extracted models. unlike using llama2-13b in \cref{sec:main_result} and \cref{sec:ablation}, we conduct experiments on CoQA and NQ\_open using llama2-7b and llama3.1 as the knowledge extracted models, We show the results in \cref{fig:knowledge_extract}. From the figure, we can see that using Phi4 could even achieve a better result, indicating \ours has more potential with the development of LLMs. 

\subsubsection{Different Accuracy Thresholds} Different accuracy thresholds lead to different accuracy and influence the evaluation of uncertainties. In the previous experiments, we all set the accuracy threshold to 70 as mentioned in \cref{sec:setup}.  To test the robustness of \ours under different accuracy thresholds, we choose an extra dataset TriviaQA~\cite{joshi2017triviaqa}, which is considered the easiest dataset, and NQ\_Open, which is the most challenging dataset in our paper to conduct experiments. We show the results with accuracy thresholds of 70 and 90 in \cref{tab:Accuracy_threshold}. From the results, we can see that increasing the accuracy threshold decreases the performance of all baselines while the performance of \ours could even increase for datasets with different difficulties, showing the robustness of \ours in different settings. 

\subsubsection{Different Similarity Metrics} Finally, different similarity metrics lead to different similarity matrices. Therefore, to test whether \ours also has a good performance for different similarities, we use Jaccard similarity instead of using an NLI model in this section and the results are presented in \cref{fig:jacc}. The results show that using Jaccard similarity will boost the performance for a simple dataset like CoQA but hurt the performance for a difficult dataset like NQ\_Open. This is because the answer to a simple question might not have a deeper semantic meaning that requires NLI models. However, \ours can still outperform baseline methods that also use Jaccard similarity, showing the robustness of \ours.





\begin{table}[h]
    \centering
    \resizebox{0.5\textwidth}{!}{
    \begin{tabular}{lcccc}
        \toprule
        \multirow{2}{*}{Methods} & \multicolumn{2}{c}{Accuracy Threshold: 0.7} & \multicolumn{2}{c}{Accuracy Threshold: 0.9} \\
        \cmidrule(lr){2-3} \cmidrule(lr){4-5}
        & AUROC & AUARC & AUROC & AUARC \\
        \midrule
        \multicolumn{5}{c}{\textbf{Dataset: TriviaQA} [Easy]} \\
        \midrule
        Eigv(Dis) & 0.8261 & 0.8094 & 0.8100& 0.7604\\
        Ecc(Dis) & 0.8063& 0.7940&0.7892 & 0.7415\\
        Degree(Dis) &0.8399 & 0.8163&0.8259 & 0.7694\\
        Eigv(Agre) &0.8436 &0.8116 &0.8351 & 0.7721 \\
        Ecc(Agre) & \textbf{0.8510}&0.8189 & 0.8374&0.7721 \\
        Degree(Agre) &0.8396 &\textbf{0.8397} &0.8384 & 0.7739\\
        \ours-Sum &0.8428 &0.8144 & 0.8438&0.7749 \\
        \ours-Min &0.8431 &0.8149 & \textbf{0.8440} & \textbf{0.7754}\\
        \midrule
        \multicolumn{5}{c}{\textbf{Dataset: NQ\_Open} [Hard]} \\
        \midrule
        Eigv(Dis) & 0.6162 & 0.7300 &0.5636 &0.6017 \\
        Ecc(Dis) & 0.6210& 0.7330& 0.5658&0.5941 \\
        Degree(Dis) &0.6130 & 0.7168&0.5662 &0.6033 \\
        Eigv(Agre) &0.6258 &0.7276 & 0.6146& 0.6290 \\
        Ecc(Agre) & 0.6273&0.7311 &0.6239 &0.6344\\
        Degree(Agre) &0.6286 &0.7355 & 0.6221&0.6299 \\
        \ours-Sum &\textbf{0.6334} &\textbf{0.7410} &\textbf{0.6351} &\textbf{0.6430} \\
        \ours-Min &0.6332 &0.7409 & 0.6350 &0.6429 \\
        \bottomrule
    \end{tabular}
    }
    \caption{Comparison of different methods across different accuracy thresholds on TrivialQA and NQ\_Open with llama2-13B. The results show that our methods outperform baselines after increasing the accuracy threshold, indicating that our methods have an advantage on more difficult datasets.}
    \vspace{-7mm}
    \label{tab:Accuracy_threshold}
\end{table}


\section{Related Work}
\label{sec:relatedworks}

% \begin{table*}[t]
% \centering 
% \renewcommand\arraystretch{0.98}
% \fontsize{8}{10}\selectfont \setlength{\tabcolsep}{0.4em}
% \begin{tabular}{@{}lc|cc|cc|cc@{}}
% \toprule
% \textbf{Methods}           & \begin{tabular}[c]{@{}c@{}}\textbf{Training}\\ \textbf{Paradigm}\end{tabular} & \begin{tabular}[c]{@{}c@{}}\textbf{$\#$ PT Data}\\ \textbf{(Tokens)}\end{tabular} & \begin{tabular}[c]{@{}c@{}}\textbf{$\#$ IFT Data}\\ \textbf{(Samples)}\end{tabular} & \textbf{Code}  & \begin{tabular}[c]{@{}c@{}}\textbf{Natural}\\ \textbf{Language}\end{tabular} & \begin{tabular}[c]{@{}c@{}}\textbf{Action}\\ \textbf{Trajectories}\end{tabular} & \begin{tabular}[c]{@{}c@{}}\textbf{API}\\ \textbf{Documentation}\end{tabular}\\ \midrule 
% NexusRaven~\citep{srinivasan2023nexusraven} & IFT & - & - & \textcolor{green}{\CheckmarkBold} & \textcolor{green}{\CheckmarkBold} &\textcolor{red}{\XSolidBrush}&\textcolor{red}{\XSolidBrush}\\
% AgentInstruct~\citep{zeng2023agenttuning} & IFT & - & 2k & \textcolor{green}{\CheckmarkBold} & \textcolor{green}{\CheckmarkBold} &\textcolor{red}{\XSolidBrush}&\textcolor{red}{\XSolidBrush} \\
% AgentEvol~\citep{xi2024agentgym} & IFT & - & 14.5k & \textcolor{green}{\CheckmarkBold} & \textcolor{green}{\CheckmarkBold} &\textcolor{green}{\CheckmarkBold}&\textcolor{red}{\XSolidBrush} \\
% Gorilla~\citep{patil2023gorilla}& IFT & - & 16k & \textcolor{green}{\CheckmarkBold} & \textcolor{green}{\CheckmarkBold} &\textcolor{red}{\XSolidBrush}&\textcolor{green}{\CheckmarkBold}\\
% OpenFunctions-v2~\citep{patil2023gorilla} & IFT & - & 65k & \textcolor{green}{\CheckmarkBold} & \textcolor{green}{\CheckmarkBold} &\textcolor{red}{\XSolidBrush}&\textcolor{green}{\CheckmarkBold}\\
% LAM~\citep{zhang2024agentohana} & IFT & - & 42.6k & \textcolor{green}{\CheckmarkBold} & \textcolor{green}{\CheckmarkBold} &\textcolor{green}{\CheckmarkBold}&\textcolor{red}{\XSolidBrush} \\
% xLAM~\citep{liu2024apigen} & IFT & - & 60k & \textcolor{green}{\CheckmarkBold} & \textcolor{green}{\CheckmarkBold} &\textcolor{green}{\CheckmarkBold}&\textcolor{red}{\XSolidBrush} \\\midrule
% LEMUR~\citep{xu2024lemur} & PT & 90B & 300k & \textcolor{green}{\CheckmarkBold} & \textcolor{green}{\CheckmarkBold} &\textcolor{green}{\CheckmarkBold}&\textcolor{red}{\XSolidBrush}\\
% \rowcolor{teal!12} \method & PT & 103B & 95k & \textcolor{green}{\CheckmarkBold} & \textcolor{green}{\CheckmarkBold} & \textcolor{green}{\CheckmarkBold} & \textcolor{green}{\CheckmarkBold} \\
% \bottomrule
% \end{tabular}
% \caption{Summary of existing tuning- and pretraining-based LLM agents with their training sample sizes. "PT" and "IFT" denote "Pre-Training" and "Instruction Fine-Tuning", respectively. }
% \label{tab:related}
% \end{table*}

\begin{table*}[ht]
\begin{threeparttable}
\centering 
\renewcommand\arraystretch{0.98}
\fontsize{7}{9}\selectfont \setlength{\tabcolsep}{0.2em}
\begin{tabular}{@{}l|c|c|ccc|cc|cc|cccc@{}}
\toprule
\textbf{Methods} & \textbf{Datasets}           & \begin{tabular}[c]{@{}c@{}}\textbf{Training}\\ \textbf{Paradigm}\end{tabular} & \begin{tabular}[c]{@{}c@{}}\textbf{\# PT Data}\\ \textbf{(Tokens)}\end{tabular} & \begin{tabular}[c]{@{}c@{}}\textbf{\# IFT Data}\\ \textbf{(Samples)}\end{tabular} & \textbf{\# APIs} & \textbf{Code}  & \begin{tabular}[c]{@{}c@{}}\textbf{Nat.}\\ \textbf{Lang.}\end{tabular} & \begin{tabular}[c]{@{}c@{}}\textbf{Action}\\ \textbf{Traj.}\end{tabular} & \begin{tabular}[c]{@{}c@{}}\textbf{API}\\ \textbf{Doc.}\end{tabular} & \begin{tabular}[c]{@{}c@{}}\textbf{Func.}\\ \textbf{Call}\end{tabular} & \begin{tabular}[c]{@{}c@{}}\textbf{Multi.}\\ \textbf{Step}\end{tabular}  & \begin{tabular}[c]{@{}c@{}}\textbf{Plan}\\ \textbf{Refine}\end{tabular}  & \begin{tabular}[c]{@{}c@{}}\textbf{Multi.}\\ \textbf{Turn}\end{tabular}\\ \midrule 
\multicolumn{13}{l}{\emph{Instruction Finetuning-based LLM Agents for Intrinsic Reasoning}}  \\ \midrule
FireAct~\cite{chen2023fireact} & FireAct & IFT & - & 2.1K & 10 & \textcolor{red}{\XSolidBrush} &\textcolor{green}{\CheckmarkBold} &\textcolor{green}{\CheckmarkBold}  & \textcolor{red}{\XSolidBrush} &\textcolor{green}{\CheckmarkBold} & \textcolor{red}{\XSolidBrush} &\textcolor{green}{\CheckmarkBold} & \textcolor{red}{\XSolidBrush} \\
ToolAlpaca~\cite{tang2023toolalpaca} & ToolAlpaca & IFT & - & 4.0K & 400 & \textcolor{red}{\XSolidBrush} &\textcolor{green}{\CheckmarkBold} &\textcolor{green}{\CheckmarkBold} & \textcolor{red}{\XSolidBrush} &\textcolor{green}{\CheckmarkBold} & \textcolor{red}{\XSolidBrush}  &\textcolor{green}{\CheckmarkBold} & \textcolor{red}{\XSolidBrush}  \\
ToolLLaMA~\cite{qin2023toolllm} & ToolBench & IFT & - & 12.7K & 16,464 & \textcolor{red}{\XSolidBrush} &\textcolor{green}{\CheckmarkBold} &\textcolor{green}{\CheckmarkBold} &\textcolor{red}{\XSolidBrush} &\textcolor{green}{\CheckmarkBold}&\textcolor{green}{\CheckmarkBold}&\textcolor{green}{\CheckmarkBold} &\textcolor{green}{\CheckmarkBold}\\
AgentEvol~\citep{xi2024agentgym} & AgentTraj-L & IFT & - & 14.5K & 24 &\textcolor{red}{\XSolidBrush} & \textcolor{green}{\CheckmarkBold} &\textcolor{green}{\CheckmarkBold}&\textcolor{red}{\XSolidBrush} &\textcolor{green}{\CheckmarkBold}&\textcolor{red}{\XSolidBrush} &\textcolor{red}{\XSolidBrush} &\textcolor{green}{\CheckmarkBold}\\
Lumos~\cite{yin2024agent} & Lumos & IFT  & - & 20.0K & 16 &\textcolor{red}{\XSolidBrush} & \textcolor{green}{\CheckmarkBold} & \textcolor{green}{\CheckmarkBold} &\textcolor{red}{\XSolidBrush} & \textcolor{green}{\CheckmarkBold} & \textcolor{green}{\CheckmarkBold} &\textcolor{red}{\XSolidBrush} & \textcolor{green}{\CheckmarkBold}\\
Agent-FLAN~\cite{chen2024agent} & Agent-FLAN & IFT & - & 24.7K & 20 &\textcolor{red}{\XSolidBrush} & \textcolor{green}{\CheckmarkBold} & \textcolor{green}{\CheckmarkBold} &\textcolor{red}{\XSolidBrush} & \textcolor{green}{\CheckmarkBold}& \textcolor{green}{\CheckmarkBold}&\textcolor{red}{\XSolidBrush} & \textcolor{green}{\CheckmarkBold}\\
AgentTuning~\citep{zeng2023agenttuning} & AgentInstruct & IFT & - & 35.0K & - &\textcolor{red}{\XSolidBrush} & \textcolor{green}{\CheckmarkBold} & \textcolor{green}{\CheckmarkBold} &\textcolor{red}{\XSolidBrush} & \textcolor{green}{\CheckmarkBold} &\textcolor{red}{\XSolidBrush} &\textcolor{red}{\XSolidBrush} & \textcolor{green}{\CheckmarkBold}\\\midrule
\multicolumn{13}{l}{\emph{Instruction Finetuning-based LLM Agents for Function Calling}} \\\midrule
NexusRaven~\citep{srinivasan2023nexusraven} & NexusRaven & IFT & - & - & 116 & \textcolor{green}{\CheckmarkBold} & \textcolor{green}{\CheckmarkBold}  & \textcolor{green}{\CheckmarkBold} &\textcolor{red}{\XSolidBrush} & \textcolor{green}{\CheckmarkBold} &\textcolor{red}{\XSolidBrush} &\textcolor{red}{\XSolidBrush}&\textcolor{red}{\XSolidBrush}\\
Gorilla~\citep{patil2023gorilla} & Gorilla & IFT & - & 16.0K & 1,645 & \textcolor{green}{\CheckmarkBold} &\textcolor{red}{\XSolidBrush} &\textcolor{red}{\XSolidBrush}&\textcolor{green}{\CheckmarkBold} &\textcolor{green}{\CheckmarkBold} &\textcolor{red}{\XSolidBrush} &\textcolor{red}{\XSolidBrush} &\textcolor{red}{\XSolidBrush}\\
OpenFunctions-v2~\citep{patil2023gorilla} & OpenFunctions-v2 & IFT & - & 65.0K & - & \textcolor{green}{\CheckmarkBold} & \textcolor{green}{\CheckmarkBold} &\textcolor{red}{\XSolidBrush} &\textcolor{green}{\CheckmarkBold} &\textcolor{green}{\CheckmarkBold} &\textcolor{red}{\XSolidBrush} &\textcolor{red}{\XSolidBrush} &\textcolor{red}{\XSolidBrush}\\
API Pack~\cite{guo2024api} & API Pack & IFT & - & 1.1M & 11,213 &\textcolor{green}{\CheckmarkBold} &\textcolor{red}{\XSolidBrush} &\textcolor{green}{\CheckmarkBold} &\textcolor{red}{\XSolidBrush} &\textcolor{green}{\CheckmarkBold} &\textcolor{red}{\XSolidBrush}&\textcolor{red}{\XSolidBrush}&\textcolor{red}{\XSolidBrush}\\ 
LAM~\citep{zhang2024agentohana} & AgentOhana & IFT & - & 42.6K & - & \textcolor{green}{\CheckmarkBold} & \textcolor{green}{\CheckmarkBold} &\textcolor{green}{\CheckmarkBold}&\textcolor{red}{\XSolidBrush} &\textcolor{green}{\CheckmarkBold}&\textcolor{red}{\XSolidBrush}&\textcolor{green}{\CheckmarkBold}&\textcolor{green}{\CheckmarkBold}\\
xLAM~\citep{liu2024apigen} & APIGen & IFT & - & 60.0K & 3,673 & \textcolor{green}{\CheckmarkBold} & \textcolor{green}{\CheckmarkBold} &\textcolor{green}{\CheckmarkBold}&\textcolor{red}{\XSolidBrush} &\textcolor{green}{\CheckmarkBold}&\textcolor{red}{\XSolidBrush}&\textcolor{green}{\CheckmarkBold}&\textcolor{green}{\CheckmarkBold}\\\midrule
\multicolumn{13}{l}{\emph{Pretraining-based LLM Agents}}  \\\midrule
% LEMUR~\citep{xu2024lemur} & PT & 90B & 300.0K & - & \textcolor{green}{\CheckmarkBold} & \textcolor{green}{\CheckmarkBold} &\textcolor{green}{\CheckmarkBold}&\textcolor{red}{\XSolidBrush} & \textcolor{red}{\XSolidBrush} &\textcolor{green}{\CheckmarkBold} &\textcolor{red}{\XSolidBrush}&\textcolor{red}{\XSolidBrush}\\
\rowcolor{teal!12} \method & \dataset & PT & 103B & 95.0K  & 76,537  & \textcolor{green}{\CheckmarkBold} & \textcolor{green}{\CheckmarkBold} & \textcolor{green}{\CheckmarkBold} & \textcolor{green}{\CheckmarkBold} & \textcolor{green}{\CheckmarkBold} & \textcolor{green}{\CheckmarkBold} & \textcolor{green}{\CheckmarkBold} & \textcolor{green}{\CheckmarkBold}\\
\bottomrule
\end{tabular}
% \begin{tablenotes}
%     \item $^*$ In addition, the StarCoder-API can offer 4.77M more APIs.
% \end{tablenotes}
\caption{Summary of existing instruction finetuning-based LLM agents for intrinsic reasoning and function calling, along with their training resources and sample sizes. "PT" and "IFT" denote "Pre-Training" and "Instruction Fine-Tuning", respectively.}
\vspace{-2ex}
\label{tab:related}
\end{threeparttable}
\end{table*}

\noindent \textbf{Prompting-based LLM Agents.} Due to the lack of agent-specific pre-training corpus, existing LLM agents rely on either prompt engineering~\cite{hsieh2023tool,lu2024chameleon,yao2022react,wang2023voyager} or instruction fine-tuning~\cite{chen2023fireact,zeng2023agenttuning} to understand human instructions, decompose high-level tasks, generate grounded plans, and execute multi-step actions. 
However, prompting-based methods mainly depend on the capabilities of backbone LLMs (usually commercial LLMs), failing to introduce new knowledge and struggling to generalize to unseen tasks~\cite{sun2024adaplanner,zhuang2023toolchain}. 

\noindent \textbf{Instruction Finetuning-based LLM Agents.} Considering the extensive diversity of APIs and the complexity of multi-tool instructions, tool learning inherently presents greater challenges than natural language tasks, such as text generation~\cite{qin2023toolllm}.
Post-training techniques focus more on instruction following and aligning output with specific formats~\cite{patil2023gorilla,hao2024toolkengpt,qin2023toolllm,schick2024toolformer}, rather than fundamentally improving model knowledge or capabilities. 
Moreover, heavy fine-tuning can hinder generalization or even degrade performance in non-agent use cases, potentially suppressing the original base model capabilities~\cite{ghosh2024a}.

\noindent \textbf{Pretraining-based LLM Agents.} While pre-training serves as an essential alternative, prior works~\cite{nijkamp2023codegen,roziere2023code,xu2024lemur,patil2023gorilla} have primarily focused on improving task-specific capabilities (\eg, code generation) instead of general-domain LLM agents, due to single-source, uni-type, small-scale, and poor-quality pre-training data. 
Existing tool documentation data for agent training either lacks diverse real-world APIs~\cite{patil2023gorilla, tang2023toolalpaca} or is constrained to single-tool or single-round tool execution. 
Furthermore, trajectory data mostly imitate expert behavior or follow function-calling rules with inferior planning and reasoning, failing to fully elicit LLMs' capabilities and handle complex instructions~\cite{qin2023toolllm}. 
Given a wide range of candidate API functions, each comprising various function names and parameters available at every planning step, identifying globally optimal solutions and generalizing across tasks remains highly challenging.



\section{Conclusion}
In this paper, we propose ChineseEcomQA, a scalable question-answering benchmark designed to rigorously assess LLMs on fundamental e-commerce concepts. ChineseEcomQA is characterized by three core features: Focus on Fundamental Concept, E-Commerce Generalizability, and Domain-Specific Expertise, which collectively enable systematic evaluation of LLMs' e-commerce knowledge. Leveraging ChineseEcomQA, we conduct extensive evaluations on mainstream LLMs, yielding critical insights into their capabilities and limitations. Our findings not only highlight performance disparities across models but also delineate actionable directions for advancing LLM applications in the e-commerce domain.


\bibliographystyle{ACM-Reference-Format}
\bibliography{sample-base}
\end{document}
\endinput
%%
%% End of file `sample-sigconf-authordraft.tex'.
