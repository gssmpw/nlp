%%
%% This is file `sample-sigconf-authordraft.tex',
%% generated with the docstrip utility.
%%
%% The original source files were:
%%
%% samples.dtx  (with options: `all,proceedings,bibtex,authordraft')
%% 
%% IMPORTANT NOTICE:
%% 
%% For the copyright see the source file.
%% 
%% Any modified versions of this file must be renamed
%% with new filenames distinct from sample-sigconf-authordraft.tex.
%% 
%% For distribution of the original source see the terms
%% for copying and modification in the file samples.dtx.
%% 
%% This generated file may be distributed as long as the
%% original source files, as listed above, are part of the
%% same distribution. (The sources need not necessarily be
%% in the same archive or directory.)
%%
%%
%% Commands for TeXCount
%TC:macro \cite [option:text,text]
%TC:macro \citep [option:text,text]
%TC:macro \citet [option:text,text]
%TC:envir table 0 1
%TC:envir table* 0 1
%TC:envir tabular [ignore] word
%TC:envir displaymath 0 word
%TC:envir math 0 word
%TC:envir comment 0 0
%%
%% The first command in your LaTeX source must be the \documentclass
%% command.
%%
%% For submission and review of your manuscript please change the
%% command to \documentclass[manuscript, screen, review]{acmart}.
%%
%% When submitting camera ready or to TAPS, please change the command
%% to \documentclass[sigconf]{acmart} or whichever template is required
%% for your publication.
%%
%%
\documentclass[sigconf,nonacm]{acmart}
\let\Bbbk\relax

%%%%% NEW MATH DEFINITIONS %%%%%

% \usepackage{amsmath,amsfonts,bm}
\usepackage{amsmath,amsfonts}

\usepackage{pifont}


\newcommand{\R}{\mathbb{R}}


\def\va{{\mathbf{a}}}
\def\vg{{\mathbf{g}}}

% Sets
\def\sR{\mathbb{R}}
\def\sC{\mathbb{C}}
\def\sZ{\mathbb{Z}}
\def\sN{\mathbb{N}}
\def\sQ{\mathbb{Q}}

\def\sS{\mathcal{S}}



% Vectors
\def\vzero{{\mathbf{0}}}
\def\vone{{\mathbf{1}}}
\def\vmu{{\mathbf{\mu}}}
\def\vtheta{{\mathbf{\theta}}}
\def\va{{\mathbf{a}}}
\def\vb{{\mathbf{b}}}
\def\vc{{\mathbf{c}}}
\def\vd{{\mathbf{d}}}
\def\ve{{\mathbf{e}}}
\def\vf{{\mathbf{f}}}
\def\vg{{\mathbf{g}}}
\def\vh{{\mathbf{h}}}
\def\vi{{\mathbf{i}}}
\def\vj{{\mathbf{j}}}
\def\vk{{\mathbf{k}}}
\def\vl{{\mathbf{l}}}
\def\vm{{\mathbf{m}}}
\def\vn{{\mathbf{n}}}
\def\vo{{\mathbf{o}}}
\def\vp{{\mathbf{p}}}
\def\vq{{\mathbf{q}}}
\def\vr{{\mathbf{r}}}
\def\vs{{\mathbf{s}}}
\def\vt{{\mathbf{t}}}
\def\vu{{\mathbf{u}}}
\def\vv{{\mathbf{v}}}
\def\vw{{\mathbf{w}}}
\def\vx{{\mathbf{x}}}
\def\vy{{\mathbf{y}}}
\def\vz{{\mathbf{z}}}
\def\vzeta{{\mathbf{\zeta}}}

% Matrix
\def\mA{{\mathbf{A}}}
\def\mB{{\mathbf{B}}}
\def\mC{{\mathbf{C}}}
\def\mD{{\mathbf{D}}}
\def\mE{{\mathbf{E}}}
\def\mF{{\mathbf{F}}}
\def\mG{{\mathbf{G}}}
\def\mH{{\mathbf{H}}}
\def\mI{{\mathbf{I}}}
\def\mJ{{\mathbf{J}}}
\def\mK{{\mathbf{K}}}
\def\mL{{\mathbf{L}}}
\def\mM{{\mathbf{M}}}
\def\mN{{\mathbf{N}}}
\def\mO{{\mathbf{O}}}
\def\mP{{\mathbf{P}}}
\def\mQ{{\mathbf{Q}}}
\def\mR{{\mathbf{R}}}
\def\mS{{\mathbf{S}}}
\def\mT{{\mathbf{T}}}
\def\mU{{\mathbf{U}}}
\def\mV{{\mathbf{V}}}
\def\mW{{\mathbf{W}}}
\def\mX{{\mathbf{X}}}
\def\mY{{\mathbf{Y}}}
\def\mZ{{\mathbf{Z}}}
\def\mBeta{{\mathbf{\beta}}}
\def\mPhi{{\mathbf{\Phi}}}
\def\mLambda{{\mathbf{\Lambda}}}
\def\mSigma{{\mathbf{\Sigma}}}


% Expectation
% \def\eE{\mathop{\mathbb{E}}\limits}
\def\eE{\mathbb{E}}

% Probability
\def\pP{\mathbb{P}}

% Tilde
\def\tf{\tilde{f}}
\def\tS{\tilde{S}}
\def\wtF{\widetilde{\mathcal{F}}}
\def\whR{\widehat{R}}
\def\tvx{\tilde{\mathbf{x}}}
\def\ty{\tilde{y}}


\def\defeq{\overset{\textup{def}}{=}}
% \def\defeq{\overset{.}{=}}
\def\defone{\overset{\text{\ding{172}}}{=}}
\def\deftwo{\overset{\text{\ding{173}}}{=}}
\def\leqone{\overset{\text{\ding{172}}}{\leq}}
\def\leqtwo{\overset{\text{\ding{173}}}{\leq}}
\def\leqthree{\overset{\text{\ding{174}}}{\leq}}
\def\leqfour{\overset{\text{\ding{175}}}{\leq}}
\def\eqone{\overset{\text{\ding{172}}}{=}}
\def\eqtwo{\overset{\text{\ding{173}}}{=}}
\def\eqthree{\overset{\text{\ding{174}}}{=}}
\def\eqfour{\overset{\text{\ding{175}}}{=}}
\def\geqfive{\overset{\text{\ding{176}}}{\geq}}
% %%%%% NEW MATH DEFINITIONS %%%%%

% \usepackage{amsmath,amsfonts,bm}
\usepackage{amsmath,amsfonts}

\usepackage{pifont}


\newcommand{\R}{\mathbb{R}}


\def\va{{\mathbf{a}}}
\def\vg{{\mathbf{g}}}

% Sets
\def\sR{\mathbb{R}}
\def\sC{\mathbb{C}}
\def\sZ{\mathbb{Z}}
\def\sN{\mathbb{N}}
\def\sQ{\mathbb{Q}}

\def\sS{\mathcal{S}}



% Vectors
\def\vzero{{\mathbf{0}}}
\def\vone{{\mathbf{1}}}
\def\vmu{{\mathbf{\mu}}}
\def\vtheta{{\mathbf{\theta}}}
\def\va{{\mathbf{a}}}
\def\vb{{\mathbf{b}}}
\def\vc{{\mathbf{c}}}
\def\vd{{\mathbf{d}}}
\def\ve{{\mathbf{e}}}
\def\vf{{\mathbf{f}}}
\def\vg{{\mathbf{g}}}
\def\vh{{\mathbf{h}}}
\def\vi{{\mathbf{i}}}
\def\vj{{\mathbf{j}}}
\def\vk{{\mathbf{k}}}
\def\vl{{\mathbf{l}}}
\def\vm{{\mathbf{m}}}
\def\vn{{\mathbf{n}}}
\def\vo{{\mathbf{o}}}
\def\vp{{\mathbf{p}}}
\def\vq{{\mathbf{q}}}
\def\vr{{\mathbf{r}}}
\def\vs{{\mathbf{s}}}
\def\vt{{\mathbf{t}}}
\def\vu{{\mathbf{u}}}
\def\vv{{\mathbf{v}}}
\def\vw{{\mathbf{w}}}
\def\vx{{\mathbf{x}}}
\def\vy{{\mathbf{y}}}
\def\vz{{\mathbf{z}}}
\def\vzeta{{\mathbf{\zeta}}}

% Matrix
\def\mA{{\mathbf{A}}}
\def\mB{{\mathbf{B}}}
\def\mC{{\mathbf{C}}}
\def\mD{{\mathbf{D}}}
\def\mE{{\mathbf{E}}}
\def\mF{{\mathbf{F}}}
\def\mG{{\mathbf{G}}}
\def\mH{{\mathbf{H}}}
\def\mI{{\mathbf{I}}}
\def\mJ{{\mathbf{J}}}
\def\mK{{\mathbf{K}}}
\def\mL{{\mathbf{L}}}
\def\mM{{\mathbf{M}}}
\def\mN{{\mathbf{N}}}
\def\mO{{\mathbf{O}}}
\def\mP{{\mathbf{P}}}
\def\mQ{{\mathbf{Q}}}
\def\mR{{\mathbf{R}}}
\def\mS{{\mathbf{S}}}
\def\mT{{\mathbf{T}}}
\def\mU{{\mathbf{U}}}
\def\mV{{\mathbf{V}}}
\def\mW{{\mathbf{W}}}
\def\mX{{\mathbf{X}}}
\def\mY{{\mathbf{Y}}}
\def\mZ{{\mathbf{Z}}}
\def\mBeta{{\mathbf{\beta}}}
\def\mPhi{{\mathbf{\Phi}}}
\def\mLambda{{\mathbf{\Lambda}}}
\def\mSigma{{\mathbf{\Sigma}}}


% Expectation
% \def\eE{\mathop{\mathbb{E}}\limits}
\def\eE{\mathbb{E}}

% Probability
\def\pP{\mathbb{P}}

% Tilde
\def\tf{\tilde{f}}
\def\tS{\tilde{S}}
\def\wtF{\widetilde{\mathcal{F}}}
\def\whR{\widehat{R}}
\def\tvx{\tilde{\mathbf{x}}}
\def\ty{\tilde{y}}


\def\defeq{\overset{\textup{def}}{=}}
% \def\defeq{\overset{.}{=}}
\def\defone{\overset{\text{\ding{172}}}{=}}
\def\deftwo{\overset{\text{\ding{173}}}{=}}
\def\leqone{\overset{\text{\ding{172}}}{\leq}}
\def\leqtwo{\overset{\text{\ding{173}}}{\leq}}
\def\leqthree{\overset{\text{\ding{174}}}{\leq}}
\def\leqfour{\overset{\text{\ding{175}}}{\leq}}
\def\eqone{\overset{\text{\ding{172}}}{=}}
\def\eqtwo{\overset{\text{\ding{173}}}{=}}
\def\eqthree{\overset{\text{\ding{174}}}{=}}
\def\eqfour{\overset{\text{\ding{175}}}{=}}
\def\geqfive{\overset{\text{\ding{176}}}{\geq}}
%\DeclareSymbolFont{AMSb}{U}{msb}{m}{n} % Redefine \Bbbk to avoid conflict
%\DeclareMathSymbol{\Bbbk}{\mathord}{AMSb}{"7C}
\usepackage{algorithm}
\usepackage{algorithmic}
\newtheorem{theorem}{Theorem}
\newtheorem{Corollary}{Corollary}


\usepackage[most]{tcolorbox}


%% \BibTeX command to typeset BibTeX logo in the docs
\AtBeginDocument{%
  \providecommand\BibTeX{{%
    Bib\TeX}}}
\newcommand{\methodFont}{\texttt}
\usepackage{xspace}

\newcommand{\ours}{\methodFont{MD-UQ}\xspace}
%% Rights management information.  This information is sent to you
%% when you complete the rights form.  These commands have SAMPLE
%% values in them; it is your responsibility as an author to replace
%% the commands and values with those provided to you when you
%% complete the rights form.
\setcopyright{acmlicensed}
\copyrightyear{2025}
% \acmYear{2025}
% \acmDOI{XXXXXXX.XXXXXXX}
%% These commands are for a PROCEEDINGS abstract or paper.
% \acmConference[Conference acronym 'XX]{Make sure to enter the correct
%   conference title from your rights confirmation email}{June 03--05,
%   2018}{Woodstock, NY}
%%
%%  Uncomment \acmBooktitle if the title of the proceedings is different
%%  from ``Proceedings of ...''!
%%
%%\acmBooktitle{Woodstock '18: ACM Symposium on Neural Gaze Detection,
%%  June 03--05, 2018, Woodstock, NY}
% \acmISBN{978-1-4503-XXXX-X/2018/06}


%%
%% Submission ID.
%% Use this when submitting an article to a sponsored event. You'll
%% receive a unique submission ID from the organizers
%% of the event, and this ID should be used as the parameter to this command.
%%\acmSubmissionID{123-A56-BU3}

%%
%% For managing citations, it is recommended to use bibliography
%% files in BibTeX format.
%%
%% You can then either use BibTeX with the ACM-Reference-Format style,
%% or BibLaTeX with the acmnumeric or acmauthoryear sytles, that include
%% support for advanced citation of software artefact from the
%% biblatex-software package, also separately available on CTAN.
%%
%% Look at the sample-*-biblatex.tex files for templates showcasing
%% the biblatex styles.
%%

%%
%% The majority of ACM publications use numbered citations and
%% references.  The command \citestyle{authoryear} switches to the
%% "author year" style.
%%
%% If you are preparing content for an event
%% sponsored by ACM SIGGRAPH, you must use the "author year" style of
%% citations and references.
%% Uncommenting
%% the next command will enable that style.
%%\citestyle{acmauthoryear}
\usepackage{caption}
\usepackage{multirow}
\usepackage[capitalize]{cleveref}
\newcommand{\tj}[1]{\textcolor{red}{\textbf{hua: #1}}}

\newcommand{\hua}[1]{\textcolor{blue}{\textbf{hua: #1}}}
\newcommand{\vagelis}[1]{\textcolor{magenta}{\textbf{vagelis: #1}}}

%%
%% end of the preamble, start of the body of the document source.
\begin{document}

%%
%% The "title" command has an optional parameter,
%% allowing the author to define a "short title" to be used in page headers.
\title{Uncertainty Quantification of Large Language Models through Multi-Dimensional Responses}

%%
%% The "author" command and its associated commands are used to define
%% the authors and their affiliations.
%% Of note is the shared affiliation of the first two authors, and the
%% "authornote" and "authornotemark" commands
%% used to denote shared contribution to the research.
% \author{Ben Trovato}
% \authornote{Both authors contributed equally to this research.}
% \email{trovato@corporation.com}
% \orcid{1234-5678-9012}
% \author{G.K.M. Tobin}
% \authornotemark[1]
% \email{webmaster@marysville-ohio.com}
% \affiliation{%
%   \institution{Institute for Clarity in Documentation}
%   \city{Dublin}
%   \state{Ohio}
%   \country{USA}
% }

\author{Tiejin Chen}
\affiliation{%
  \institution{Arizona State University}
  \city{Tempe}
  \country{AZ}}
\author{Xiaoou Liu}
\affiliation{%
  \institution{Arizona State University}
  \city{Tempe}
  \country{AZ}}

\author{Longchao Da}
\affiliation{%
  \institution{Arizona State University}
  \city{Tempe}
  \country{AZ}}


\author{Jia Chen}
\affiliation{%
  \institution{University of California, Riverside}
  \city{Riverside}
  \country{CA}}

\author{Vagelis Papalexakis}
\affiliation{%
  \institution{University of California, Riverside}
  \city{Riverside}
  \country{CA}}

\author{Hua Wei}
\affiliation{%
  \institution{Arizona State University}
  \city{Tempe}
  \country{AZ}}


\begin{abstract}
Large Language Models (LLMs) have demonstrated remarkable capabilities across various tasks due to large training datasets and powerful transformer architecture. However, the reliability of responses from LLMs remains a question. Uncertainty quantification (UQ) of LLMs is crucial for ensuring their reliability, especially in areas such as healthcare, finance, and decision-making.  Existing UQ methods primarily focus on semantic similarity, overlooking the deeper knowledge dimensions embedded in responses. We introduce a multi-dimensional UQ framework that integrates semantic and knowledge-aware similarity analysis. By generating multiple responses and leveraging auxiliary LLMs to extract implicit knowledge, we construct separate similarity matrices and apply tensor decomposition to derive a comprehensive uncertainty representation. This approach disentangles overlapping information from both semantic and knowledge dimensions, capturing both semantic variations and factual consistency, leading to more accurate UQ. Our empirical evaluations demonstrate that our method outperforms existing techniques in identifying uncertain responses, offering a more robust framework for enhancing LLM reliability in high-stakes applications.

\end{abstract}
%%
%% The code below is generated by the tool at http://dl.acm.org/ccs.cfm.
%% Please copy and paste the code instead of the example below.
%%
\begin{CCSXML}
<ccs2012>
 <concept>
  <concept_id>00000000.0000000.0000000</concept_id>
  <concept_desc>Do Not Use This Code, Generate the Correct Terms for Your Paper</concept_desc>
  <concept_significance>500</concept_significance>
 </concept>
 <concept>
  <concept_id>00000000.00000000.00000000</concept_id>
  <concept_desc>Do Not Use This Code, Generate the Correct Terms for Your Paper</concept_desc>
  <concept_significance>300</concept_significance>
 </concept>
 <concept>
  <concept_id>00000000.00000000.00000000</concept_id>
  <concept_desc>Do Not Use This Code, Generate the Correct Terms for Your Paper</concept_desc>
  <concept_significance>100</concept_significance>
 </concept>
 <concept>
  <concept_id>00000000.00000000.00000000</concept_id>
  <concept_desc>Do Not Use This Code, Generate the Correct Terms for Your Paper</concept_desc>
  <concept_significance>100</concept_significance>
 </concept>
</ccs2012>
\end{CCSXML}

% \ccsdesc[500]{Do Not Use This Code~Generate the Correct Terms for Your Paper}
% \ccsdesc[300]{Do Not Use This Code~Generate the Correct Terms for Your Paper}
% \ccsdesc{Do Not Use This Code~Generate the Correct Terms for Your Paper}
% \ccsdesc[100]{Do Not Use This Code~Generate the Correct Terms for Your Paper}

%%
%% Keywords. The author(s) should pick words that accurately describe
%% the work being presented. Separate the keywords with commas.
\keywords{Large Language Model, Uncertainty Quantification}

%% A "teaser" image appears between the author and affiliation
%% information and the body of the document, and typically spans the
%% page.


% \received{20 February 2007}

%%
%% This command processes the author and affiliation and title
%% information and builds the first part of the formatted document.
\maketitle

\section{Introduction}\label{sec:intro}

Large Language Models (LLMs) demonstrate strong performance across natural language processing tasks, yet their architectural complexity and limited interpretability can produce unreliable outputs. 
This presents significant challenges in critical domains such as healthcare, where output errors carry serious consequences. 
Confidence estimation methods have emerged to quantify output reliability. 
The field connects closely with uncertainty quantification in natural language generation, as both address output trustworthiness. 
Current approaches divide into consistency-based methods, which analyze agreement across multiple outputs, and internal-states methods that leverage model-specific features like output probabilities.
Despite advances in these approaches, developing robust evaluation frameworks remains a central challenge.

%Large Language Models (LLMs) have achieved notable success in various natural language processing (NLP) tasks, such as text generation, question answering, and reasoning. However, their complex architectures and lack of interpretability often lead to uncertain, incorrect, or ambiguous outputs. This raises critical concerns about their reliability, especially in high-stakes domains like healthcare and decision-making, where errors can have severe consequences. Confidence estimation has emerged as a key tool for addressing these concerns by quantifying the trustworthiness of LLM outputs.


%In natural language generation (NLG), confidence estimation is closely tied to uncertainty quantification (UQ), as both aim to assess the reliability of model outputs. 
%% Confidence scores are often derived from uncertainty measures, with higher uncertainty corresponding to lower confidence. 
%Existing methods for confidence estimation in LLMs can be broadly categorized into two types: consistency-based black-box methods, which rely on agreement among multiple generated responses, and internal-states-based methods, which use model-specific information such as output probabilities or self-evaluation mechanisms. 
%While these approaches have advanced the field, the evaluation of confidence estimation methods remains a significant challenge in their evaluation process.\cc{consider say we mainly do UQ evaluation earlier, cut the context. at least merge the first two paragraphs}

%Currently, the evaluation frameworks for confidence measures in NLG primarily rely on correctness labels to compute metrics such as area under the receiver operating characteristic curve (AUROC) and accuracy-rejection curve (AUARC). 
%These frameworks follow a general pipeline: 
%(1) generating predictions from the model,
%(2) assigning correctness labels to the predictions using a \textit{correctness function} $f(\cdot)$, and 
%(3) calculating evaluation metrics based on these labels. 
%However, the reliance on correct labels introduces several limitations. 
%Human evaluation, though reliable, is time-consuming and impractical for large-scale datasets. 
%Automated reference-based metrics like BLEU and ROUGE fail to account for semantically correct but differently phrased responses. 
%LLM-based evaluators offer flexibility but still suffer from systematic biases~\cite{lin2022truthfulqa}, such as favoring outputs from similar models\cc{add citation here?} or being sensitive to prompt variations—issues that lack concrete evidence in prior work\cc{add citation here?}. 

Current evaluation frameworks for NLG confidence measures rely on correctness labels to compute metrics such as AUROC and AUARC. 
These frameworks follow a three-step process: generating model predictions, labeling correctness via a function $f(\cdot)$, and calculating metrics. 
This label-dependent approach faces several constraints. While human evaluation provides reliable correctness ground truth, it cannot scale to large datasets. 
Metrics based on reference matching, such as BLEU and ROUGE, fail to recognize semantically equivalent responses phrased differently.
% Reference matching-based metrics like BLEU and ROUGE miss semantically equivalent responses with different phrasing. 
% LLM-based evaluators offer greater capability and scalability, but are still noisy, and could introduce systematic biases, such as favor over responses generated by similar LMs~\cite{panickssery2024llm} or over length~\cite{lin2022truthfulqa}.
LLM-based evaluators offer greater capability but remain noisy and may introduce systematic biases, such as favoring responses generated by themselves or similar LMs~\cite{panickssery2024llm}, or preferring longer responses~\cite{lin2022truthfulqa}.
Moreover, running such evaluators could be expensive.
% LLM-based evaluators improve capability and scalability but remain noisy and may introduce systematic biases, such as favoring responses generated by similar LMs~\cite{panickssery2024llm} or preferring longer responses~\cite{lin2022truthfulqa}.
% (also observed in our own experiments) or over length~\cite{lin2022truthfulqa}.
% The prompt used to eli
%including outputs from similar models\cc{add citation here?} and prompt sensitivity—phenomena that remain empirically understudied\cc{add citation here?}.


%More critically, flaws in the correctness function $f(\cdot)$ can propagate through the evaluation pipeline, affecting evaluation metrics like AUROC. 
%This fragility can lead to misleading conclusions about which confidence estimation method performs better, particularly when different methods yield similar performance. 
%These issues highlight the need for alternative evaluation frameworks with reliable correctness labels.

% More critically, flaws in the correctness function $f(\cdot)$ propagate through the evaluation pipeline, affecting metrics like AUROC. 

Flaws in the correctness function $f(\cdot)$ propagate through the evaluation pipeline, affecting metrics like AUROC. This sensitivity becomes particularly problematic when comparing confidence estimation methods with similar performance. Such limitations underscore the need for evaluation frameworks that establish correctness more reliably.


%In this paper, we propose a simple evaluation framework that addresses these limitations by leveraging multiple-choice question-answering (QA) datasets. 
%Our approach removes the dependence on a correctness function by using the structure of multiple-choice QA tasks to evaluate confidence measures directly. 
%Importantly, our framework is complementary to existing pipelines rather than a replacement; it provides an additional perspective to validate the discriminative power of confidence measures across different evaluation settings.

In this paper, we propose \uqeval, a simple, efficient yet effective evaluation framework that eliminates the dependence on unreliable correctness functions.
% leverages multiple-choice question-answering (QA) datasets to address these limitations.
\textit{The key insight is to leverage multiple-choice question-answering (QA) datasets, which inherently provide gold-standard answer choices at no cost.} 
With these definitive labels, our framework bypasses the ambiguity of determining correctness via correctness function $f(\cdot)$ and ensures an objective assessment of confidence estimation methods. 
% By leveraging multiple-choice QA datasets, our proposed evaluation framework ensures an objective assessment of confidence estimation methods.
% Unlike conventional evaluation pipelines that rely on an often noisy correctness function $f(\cdot)$ to determine whether a model’s prediction is right or wrong, our approach directly utilizes gold-standard answer choices in multiple-choice QA tasks. 
% \textit{Our approach eliminates the dependence on unreliable correctness functions} and ensures an objective assessment of confidence estimation methods.
% By exploiting the inherent structure of multiple-choice QA datasets, our approach eliminates dependence on unreliable correctness. 
Rather than replacing existing evaluation pipelines, our framework complements them, offering an additional lens to assess the discriminative power of confidence estimation methods.
% across different evaluation settings.
\cref{fig:pipeline} shows how our proposal (green) and the existing evaluation pipeline (blue) differ, yet complement each other.
% In this paper, we propose a simple yet effective evaluation framework that eliminates reliance on noisy correctness functions. Our key insight is to leverage multiple-choice question-answering (QA) datasets, which inherently provide gold-standard answer choices. By using these definitive labels, our framework bypasses the ambiguity of determining correctness via 
% f(⋅) and ensures an objective assessment of confidence estimation methods. 
% Rather than replacing existing evaluation pipelines, our approach complements them, offering an additional lens to assess the discriminative power of confidence measures. \cref{fig:pipeline} illustrates how our proposal (green) differs from, yet aligns with, the existing pipeline (blue).
Our contributions are summarized as follows:
\begin{itemize}[leftmargin=*,nosep]
    \item We demonstrate that commonly used evaluation methods for NLG confidence measures are sensitive to noise in correctness labels, which can lead to misleading conclusions about evaluation metrics and rankings of different confidence estimation approaches.
    %We show that the most popular evaluation methods for NLG confidence measures are prone to noise in the ``correctness label`, which could further lead to problematic conclusions. %(about which measure is better)
    \item We propose a simple yet effective method that utilizes multiple-choice QA datasets to evaluate confidence measures, supporting both internal-states-based white-box and consistency-based black-box methods.
    \item Extensive experiments across recent LLMs and QA datasets verify that \uqeval produces stable evaluations broadly consistent with existing methods, while eliminating the need for expensive correctness functions.
    %We conduct extensive experiments on recent LLMs and popular QA datasets to verify that \uqeval provides stable evaluation results that are often consistent with existing evaluation methods, without the costly correctness function.
\end{itemize}


\section{Preliminaries}
\label{sec:background}

\subsection{Problem Formulation}
\label{subsec:problem}
Modern uncertainty quantification (UQ) for black-box LLMs operates through two sequential stages: similarity measurement between responses and uncertainty estimation from these similarities. Let $\mathcal{M}$ denote a black-box LLM generating $n$ responses $\{A^1,\ldots,A^n\}$ to input $Q$. The UQ task estimates confidence $U $ through:


\begin{equation}
U = f(\mathbf{S}), \quad \mathbf{S} \in \mathbb{R}^{n\times n}
\end{equation}

\noindent where $\mathbf{S}$ is the similarity matrix with the $(i, j)$-th entry capturing the proximity  of responses $A^i$ and $A^j$, and $f$ represents the estimation strategy. This formulation enables UQ without accessing internal model probabilities.

\subsection{Measuring Response Similarities}
\label{subsec:similarities}
The foundation of reliable UQ lies in effective similarity measurement. We analyze two complementary approaches capturing different aspects of response quality.

\subsubsection{Semantic Dimension}
\label{subsubsec:semantic}
Semantic similarity focuses on surface-level consistency between responses. The Jaccard index \citep{lin2023generating} offers simple lexical comparison:

\begin{equation}
s_{ij} = \frac{|A^i \cap A^j|}{|A^i \cup A^j|}
\end{equation}

While computationally efficient, Jaccard ignores word order and semantics. For deeper analysis, there is another common way to compute the semantic similarity, which uses DeBERTa's NLI capabilities~\citep{he2021deberta}:

\begin{equation}
s_{ij} = \frac{1}{2}\left(P_{\text{entail}}(A^i,A^j) + P_{\text{entail}}(A^j,A^i)\right)
\end{equation}

Where $P_{\text{entail}}(A^i,A^j)$ the probability of $A^i$ entails $A^j$ that is output from the NLI model.

Compared with the Jaccard index, NLI-based scoring better captures semantic equivalence but remains sensitive to syntactic variation. For instance, paraphrased factual statements may receive low scores despite equivalent meaning. To the question \textit{How many students became heroes}, The response \textit{These three became heroes} and the response \textit{Andrew Willis, Chris Willis, Reece Galea} share the factual knowledge that three students became heroes while their similarity from NLI models will be low as 0.015. Therefore, relying solely on responses from the semantic dimension may result in information loss.

\subsubsection{Knowledge Dimension} 
\label{subsubsec:knowledge}

The knowledge dimension operates through a structured pipeline that transforms raw responses into factual representations. 
Given a question $Q$ and its original response $A^i$, a knowledge representation $K^i$ could be generated through a knowledge mapping process by extracting explicit claims: $K^i = \mathcal{M}_{\text{aux}}(Q, A^i)$ and augmenting the response, where  $\mathcal{M}_{\text{aux}}$ denotes for an auxiliary LLM. Specifically, we could use an LLM to augment with prompts taking into the question and original response:

\begin{tcolorbox}[colback=gray!5!white, colframe=gray!75!green, title=\textbf{Prompt Example for Knowledge Mapping}] 

 \ding{182}: Extract all factual claims from this response $\langle A^i \rangle$, phrased as standalone statements independent of specific wording. 

 \ding{183}: Include only information directly relevant to answering the question: $\langle Q \rangle$.

\end{tcolorbox}

This claim extraction disentangles implicit knowledge from surface semantics and removes stylistic variations while preserving core factual content.




\subsection{Estimating Uncertainty}
\label{subsec:estimation}
With similarity matrices constructed, existing UQ methods employ two principal ways to estimate uncertainty, each offering unique advantages and limitations.

\subsubsection{Number of Semantic Sets (UNumSet)}
\label{subsubsec:numset}
Proposed by \citet{kuhn2023semantic}, this method groups responses into equivalence classes using bidirectional entailment checks from an NLI model. Formally, responses $A^i$ and $A^j$ are merged into the same semantic set if:
\begin{equation}
P_{\text{entail}}(A^i, A^j) > P_{\text{contra}}(A^i, A^j) \quad \text{and} \quad P_{\text{entail}}(A^j, A^i) > P_{\text{contra}}(A^j, A^i).
\end{equation}
The uncertainty measure $U_{\text{NumSet}}$ equals the number of resulting semantic sets. This approach aligns with spectral graph theory. Because when using binary adjacency matrices ($W_{ij} \in {0,1}$), the number of zero eigenvalues in the graph Laplacian corresponds to the number of connected components \citep{von2007tutorial}. While this method discretizes continuous semantic relationships, it fails to capture partial meaning overlaps.

\subsubsection{Graph Laplacian}
\label{subsubsec:laplacian}
Building on spectral graph principles \citep{agaskar2013spectral,lin2023generating}, this method quantifies uncertainty through the eigenvalues ${\lambda_k}$ of the normalized graph Laplacian $L = I - D^{-1/2}WD^{-1/2}$:
\begin{equation}
U_{\text{EigV}} = \sum_{k=1}^n \max(0, 1 - \lambda_k).
\end{equation}

Here, eigenvalues $\lambda_k$ encode connectivity. fragmented graphs (low consistency in responses and thus high uncertainty) have more small eigenvalues. Compared with $U_{\text{NumSet}}$, this method is able to capture possible overlapping semantic relationships.


\subsection{Dimensional Analysis}
Now, we compare the difference between similarity matrices from knowledge and semantic dimensions. In detail, we use the NLI model to obtain the similarity matrix. In \cref{tab:similarity_matrix_stat}, we present the mean values and the proportion of similarity scores greater than 0.55 in the similarity matrices. The results show that similarity matrices in the knowledge dimension have more large value as well as a larger mean value. This reveals the knowledge dimensions' superior consistency. These results highlight the importance of multi-dimensional analysis since semantic features capture response variability, while knowledge features track factual consistency hidden behind the semantic features. Our tensor decomposition effectively combines these complementary signals.

\begin{table}[t!]
    \centering
    \begin{tabular}{lcc}
        \toprule
        Dataset & Proportion (\%) & Mean Value \\
        \midrule
        \multicolumn{3}{c}{\textbf{Semantic Similarity}} \\
        \midrule
        Coqa & 52.97 & 0.5430 \\
        nq\_open & 17.11 & 0.1839 \\
        Trivialqa & 51.15 & 0.5154 \\
        \midrule
        \multicolumn{3}{c}{\textbf{Knowledge Similarity}} \\
        \midrule
        Coqa & 57.40 & 0.5723 \\
        nq\_open & 31.16 & 0.3281 \\
        Trivialqa & 60.17 & 0.6058 \\
        \bottomrule
    \end{tabular}
    \caption{Proportion of similarity values greater than 0.55 and mean similarity values for the similarity matrices in the semantic space and the knowledge space. The results show that the knowledge similarity matrix has larger values.}
    \vspace{-10mm}
    \label{tab:similarity_matrix_stat}
\end{table}
\section{ \uqeval: A framework for Assessing Confidence Estimation}\label{sec:method}
At a high level, existing evaluation frameworks
% the evaluation pipeline described in \cref{sec:prelim:old_eval} 
for $C_{\mathcal{M}}$ includes three main steps (blue path in \cref{fig:pipeline}):
\begin{enumerate}[nosep]
    \item Generate $\predSeq$ from $\mathcal{M}$ given the input $\xInput_i$.
    \item Determine the correctness label of $\predSeq$ using the function $\acc(\cdot,\xInput)$.
    \item Compute evaluation metrics such as AUROC. A higher metric value indicates that $C_{\mathcal{M}}$ is a ``better'' confidence estimation.
\end{enumerate}
The main limitation of this general pipeline lies in $\acc$ in step 2. 
Existing evaluation frameworks all implicitly assume step 1---that the confidence measure $C_{\mathcal{M}}$ must apply to generated sequences $\predSeq$. 
While this might hold for consistency-based uncertainty measures, where response divergence indicates uncertainty, it does not extend to confidence measures. 
In other words, we could relax step 1 in order to improve step 2.
% Consider, for instance, Eccentricity~\cite{lin2024generating}: The uncertainty measure $U_{ecc}$ computes the average distance between sampled generation embeddings and their centroid, while the confidence measure evaluates a specific generation. 
% Consequently, we could simply use the sampled generations to construct the embedding space, yet we can still measure the distance of \textit{any} sequence to the center of embeddings.


%The previous discussion suggests that existing evaluation frameworks all bear an implicit assumption: The confidence measure $C_{\mathcal{M}}$ must be applied to the generations $\predSeq$.
%While this might be true for the case of most existing consistency-based uncertainty measures, where a high degree of divergence of the sampled responses is a strong indicator of high uncertainty, this is \textit{not} the case for confidence measures.
%Take Eccentricity~\cite{lin2024generating} as an example: The uncertainty measure $U_{ecc}$ is based on the average distance of the embeddings of the sampled generations to the center of all embeddings and the confidence measure is that of a particular generation. 
%Consequently, we could simply use the sampled generations to construct the embedding space, yet we can still measure the distance of \textit{any} sequence to the center of embeddings. 

\textit{Our main proposal in this paper is to adapt multiple-choice datasets to evaluate confidence measures designed for free-form NLG.}
Unlike free-form NLG datasets, multiple-choice datasets provide inherent correctness values for options, eliminating the need for an explicit correctness function. 
If we simply ``pretend'' that these options are free-form generations from the base LM, we can directly evaluate the confidence measure quality. 
As \cref{fig:pipeline} shows, the approach differs from existing evaluation pipelines only in applying confidence estimation methods to multiple-choice options.




Consider the QASC~\cite{khot2020qasc} dataset as an example,
each problem comes with a question $\xInput$ and a few choices, $o_1,\ldots,o_K$. 
Unlike what such datasets were designed for, we re-format the input prompt as a free-form NLG question, as illustrated in \cref{fig:qasc_example}, as if the base LLM generated each option itself, in different runs.
In what follows, we first explain explain slight nuances in applying internal state-based white-box confidence measures as well as consistency-based black-box ones. 
%shows a reformatted question from the QASC dataset.

\begin{figure}[t]
  \includegraphics[width=\columnwidth]{figures/qasc_example.pdf}
  % \caption{A reformatted question example from the QASC dataset. The Question and Choices are directly from the original dataset, while our prompt is specifically designed for LLM input to generate open-form responses.}
  \caption{
  We reformat each option from the multiple-choice question (left), by injecting the \smash{\colorbox{yellow!40}{{{\color{blue}option}}}} to a free-form QA \smash{\colorbox{green!40}{prompt}}.
  One could typically apply any confidence estimation method by treating this \smash{\colorbox{yellow!40}{{{\color{blue}option}}}} as if it was generated by the base LM.
  For black-box confidence measures that require additional responses, we only feed the \smash{\colorbox{green!40}{prompt}} to the base LM.
  }
  \label{fig:qasc_example}
\vspace{-3mm}
\end{figure}



\textbf{Logit or Internal State-Based Measures} typically examine the internals of a LM when it generates a particular response.
The nature of the free-form generation task allows us to simply plug-in the option $o_i$ into the corresponding location of the prompt, and extract similar information that allows us to evaluate the confidence\footnote{In fact, this was the practice to compute \baselineSL for actual generations. For example, \url{https://github.com/lorenzkuhn/semantic_uncertainty/blob/main/code/get_likelihoods.py} and \url{https://huggingface.co/docs/transformers/perplexity}.}.
% One concern is whether these options are too ``different'' from what the LM would otherwise generate itself.
% As exemplified in \cref{fig:true_distribution}, $C(o_i)$ in general shares a similar distribution to $C(\predSeq_i)$. 

% Taking CommonsenseQA~\footnote{A multiple-choice dataset for our experiment is described in Section~\ref{sec:experiments}} as an example, we compare the logit distribution of the correct answer choices with the distributions of other LLM-generated responses. 
% As shown in \cref{fig:true_distribution}, the distributions exhibit notable similarities, indicating that logit-based confidence estimation can capture underlying patterns shared between correct answer choices and free-form generations.

% \begin{figure}[t]
%   \includegraphics[width=\columnwidth]{figures/true_distribution_new.png}
%   \caption{Confidence score distribution for the \baselinePTrue method on the CQA dataset. The blue distribution represents 20 open-form responses, while the red distribution corresponds to the correct option. }
%   \label{fig:true_distribution}
% \end{figure}


\textbf{Consistency-based Confidence Measures}
Unlike logit-based or internal-state-based measures, consistency-based confidence measures typically rely on an estimate of the predictive distribution, denoted as $\PredDist$, and any response that is closer to the center of the distribution (in the ``semantic space'') is considered to be of higher confidence. 
Consider methods from~\citet{lin2024generating} as an example. To preserve the integrity of the predictive distribution, we first sample $n$ responses from $\PredDist$ as usual, and then iteratively include one option $o_i$ at a time to compute its associated confidence score~\cite{rivera-etal-2024-combining,manakul-etal-2023-selfcheckgpt}. 
\cref{alg:confidence_score} outlines this process. 


\renewcommand{\algorithmicrequire}{\textbf{Input:}}
\renewcommand{\algorithmicensure}{\textbf{Output:}}

\begin{algorithm}[t]
\small
% \caption{Confidence Score Computation in the Black-Box Method}
\caption{Consistency-based Confidence Estimation for Any Sequences}
\label{alg:confidence_score}
\begin{algorithmic}[1]
    \Require $\xInput$, $\mathcal{M}$, candidate sequences $A = \{a_1, \dots, a_K\}$
    \Ensure $\{C_{\mathcal{M}}(\xInput,a_1), \dots,C_{\mathcal{M}}(\xInput,a_K)\}$ 
    
    \State Generate $S = \{\predSeq_1, \dots, \predSeq_{n}\}$ using $\mathcal{M}$ for question $\xInput$
    \State Compute pairwise similarity matrix $M$ of $S$.% \in \mathbb{R}^{|S'| \times |S'|}$
    % \State Construct full response set $S' = S \cup A$, where $|S'| = n+K$
    % \State Compute pairwise similarity matrix $M_{sim} \in \mathbb{R}^{|S'| \times |S'|}$
    
    \For{each $a_i \in A$}
        \State Compute a new similarity matrix $M_i$ of $S\cup\{a_i\}$, reusing $M$. % \in 
        % \State Form subset $S_i = S \cup \{a_i\}$, where $|S_i| = n+1$
        % \State Extract pairwise similarity matrix $M_{sim}^{(i)} \in \mathbb{R}^{|S_i| \times |S_i|}$
        \State Compute confidence score $C_{\mathcal{M}}(\xInput,a_i)$ using $M_i$. %degree matrix or eccentricity of $M_{sim}^{(i)}$
    \EndFor

    \State \Return $\{C_{\mathcal{M}}(\xInput,a_1), \dots,C_{\mathcal{M}}(\xInput,a_K)\}$ 
\end{algorithmic}
\end{algorithm}

% Since computing the similarity matrix is the most computationally expensive step, the subsequent 5 confidence score calculations reuse precomputed similarity values. As a result, the additional computations take less than 1 minute in total, ensuring efficiency.

% Consistency-based confidence measures are a little different from .


\paragraph{Remarks}
Our proposal relaxes step 1 at the beginning of this section, allowing for $\predSeq^*=o_i$ not sampled from $\PredDist$.
This is not to be misunderstood as a proposal to \textit{replace} the current pipeline (\cref{sec:prelim:old_eval})---rather, it is \textit{complementary}.
The rationale is that if a good confidence measure predicts the correctness well, it should perform well in \textit{both} evaluation frameworks.
In fact, any $o_i\in\Sigma^*$ that does not violate the generation configuration, has a non-zero probability to be sampled from $\PredDist$, and a robust confidence measure should be expected to model it well.
% \fontred{
% In fact, any $o_i$, as long as it does not violate the generation config, could be sampled from $\PredDist$ given enough time.
% % As we will see in \cref{sec:exp}, even though $o_i$ are not sampled from $\PredDist$, we do not observe a big distribution shift in terms of the confidence values as well.
% }
% Note that we do not advise \textit{replacing} the existing valuation 

% It is important to note that our method only obtains correctness labels for $o_i$. Consequently, when computing AUROC, AUARC, and other evaluation metrics, we only consider confidence values associated with these options.\textcolor{red}{already mention it in \cref{sec:metrics} }

\begin{figure}[t]
    %
    \begin{center}
    \centerline{\includegraphics[width=0.5\columnwidth]{experiments/recht_loss.pdf}}
    %
    \caption{Matrix factorization via a 2-layer linear network}
    \label{recht}
\end{center}
\end{figure}

\section{Experiments}
\label{sec:experiments}
The prior works of FOOF, LocoProp, PRONG have shown to compare competitively
with other sophisticated optimizers such as K-FAC \citep{martens2015}.
Given their discussed equivalence with LNB, we focus on experimentally confirming
the contributions of this work: feature whitening via preconditioning the
gradient vector, and the applicability and effectiveness on
the realistic networks of ViT \citep{vit} and UNet \citep{unet}.

Due to its de facto status, we benchmark convergence with Adam in both
iteration and wall time.
In all experiments, the default EMA values were used ($\beta_1=0.9$, $\beta_2=0.999$)
for Adam while the learning rate was grid searched around $1e^{-4}$
All timings were recorded from a NVIDIA L4 GPU. Due to the deterministic nature of performing
2 conjugate gradient steps for LNB, the variance in reported times is negligible.


\subsection{Matrix factorization}
We start with the pathological example from \citet{recht}. This is a matrix
factorization problem formulated as a two-layer linear network:
$\sum_{i=1}^{n} \Vert W_1 W_2 x_i - y_i\Vert^2$,
where $y_i=Ax_i$ for a poorly conditioned matrix, $\kappa(A)=10^5$. Due to the conditioning
and the columns being correlated, it is known that gradient descent converges slowly for this
problem, whereas GN converges quickly. Because the LNB preconditioner is decorrelating
the feature space, we would also expect fast convergence.

We use the same initialization as in the notebook, but in order to magnify the differences,
we increase the dimensions by a factor of $10$: $n=10^4$,
$W_2 \in \sR^{60 \times 60}$, $W_1 \in \sR^{100 \times 60}$. Learning rates were tuned via
grid search to find fast and stable convergence for each method. The results are plotted in Figure \ref{recht}
and reproduce the prior reported slow convergence of Adam and demonstrate fast convergence with LNB.

\begin{figure}[t]
    %
    \begin{center}
    %
    \subfigure[Original pixels ($x$)]{\includegraphics[width=0.49\columnwidth]{experiments/mnist_acc.pdf}}
    %
    \subfigure[Inverted pixels ($1-x$)]{\includegraphics[width=0.49\columnwidth]{experiments/inv_mnist_acc.pdf}}
    %
    \vskip -0.1in
    \caption{MNIST test accuracy evolution trained on the original data (a) vs. inverted pixels (b).
    The step-size is parenthesized.}
    \label{fig:mnist}
    \end{center}
\end{figure}


\subsection{MLP}
We reproduce the MLP result in \citet{grub2010} that compares boosting and gradient
descent for a 2-layer MLP on MNIST using 800-node layers, $\tanh$ activation,
Glorot Normal initialization and a batch size of 1,000.

In Fig \ref{fig:mnist}-a., we plot the test accuracies w.r.t. epoch with the best two learning rates for Adam.
We first note that Adam and LNB obtain better than the prior reported accuracy of $98.3\%$.
Second, LNB achieved this performance with a fixed (ridge) regularizer $\lambda$, whereas prior
work heavily tuned this.
One explanation for the difference is that LNB is taking an adaptive step size according to the metric
and this was not derived before.
Third, there is little observed difference between boosting and gradient descent on this dataset.
This can be explained due to that most of the binary pixels in MNIST are zero, so the feature space
of the vectorized images is low rank, i.e., decorrelating provides little benefit.

\begin{figure}[t]
    \begin{center}
    \subfigure[\texttt{train} loss]{\includegraphics[width=0.49\columnwidth]{experiments/vit_loss.pdf}}
    \subfigure[\texttt{test} accuracy]{\includegraphics[width=0.49\columnwidth]{experiments/vit_acc.pdf}}
    %
    \caption{ViT performances on CIFAR10.}
    \label{fig:vit}
    \end{center}
\end{figure}

However, whitening does include a centering step. If we were to shift the feature space, we would expect
to get same performance. In Fig \ref{fig:mnist}-b, we plot the same models when trained and tested on pixels
$1-x$, where $x$ are the original binary pixel values used in Fig \ref{fig:mnist}-a. We observe that LNB gets very similar performance
while Adam (and other methods not invariant to affine reparameterizations) degrade. Although we could (and should)
simply normalize the features before training, this example illustrates a case of how feature scaling can
greatly affect convergence.

\subsection{Vision Transformer}
We train a vision transformer \cite{vit} using the notebook from Equinox \cite{eqx}
on CIFAR10.
The only modification we make is to not learn the affine terms in the LayerNorm
in order to speed up experimentation and we observed no performance benefit with it.
The train and test fold performances are show in \Figref{fig:vit}, where we observe
faster convergence and better generalization with LNB. Excluding JIT compilation time,
the duration per epoch for LNB and Adam is 1.26 min and 0.85 min, respectively. 

\begin{figure}[t]
    \begin{center}
    \subfigure[\texttt{train} loss]{\includegraphics[width=0.49\columnwidth]{experiments/voc_loss.pdf}}
    \subfigure[\texttt{val} accuracy]{\includegraphics[width=0.49\columnwidth]{experiments/voc_acc.pdf}}
    %
    \caption{UNet performances on VOC Segmentation.}
    \label{fig:unet}
    \end{center}
\end{figure}

\subsection{UNet}
We train a UNet \cite{unet} on the 2012 VOC Segmentation Challenge dataset \cite{voc}. The images are
pixelwise normalized into the range $[0,1]$ using ImageNet mean and variance R,G,B pixel values
and then zero-padded to $500 \times 500$ size and then downsized to $384 \times 384$.
No data augmentation is performed. We plot the results in \Figref{fig:unet} and
remark that LNB converges very quickly, using the same learning rate as with ViT, and
avoids overfitting. While both optimizers converge to comparable performance on the
\texttt{val} split, the rapid progress by LNB suggests it would be able to leverage
more data effectively.
However, excluding JIT compilation time,
the duration per epoch for LNB and Adam is 2.92 min and 1.27 min, respectively, and
this highlights the trade-off between convergence w.r.t. iterations vs. wall time.
While LNB converged marginally faster in wall time and is significantly
easier to implement to prior equivalent work, it is future work to better understand
in what deep networks does the whitening behavior lead to better generalization as
demonstrated in the other three experiments.

\section{Related Work}
\label{sec:related}

Recent advances~\cite{lecun2015deep, zaidi2022survey} in deep learning have vastly improved object detection and instance segmentation results in the terrestrial domain. 
Such progress has been achieved by developing effective designs of models and training them with large datasets~\cite{lin2014microsoft, russakovsky2015imagenet} containing millions of images and corresponding labels. 
Even with such advances, detecting underwater debris still remains challenging. 
While~\cite{fulton2019robotic} presents the first deep learning based approach to detect underwater debris and outperforms previous non deep learning approaches, the accuracy is worse than general object detection tasks due to a small training dataset. 
To increase the debris detection accuracy,~\cite{hong2020trashcan} proposes a larger dataset, TrashCan, which has both bounding box and pixel-level annotations for object detection and instance segmentation along with baseline results using Mask R-CNN~\cite{he2017mask} and Faster R-CNN~\cite{ren2015faster}. 
However, increasing the dataset size to improve debris detection accuracy further is not scalable due to debris data scarcity and labeling costs. 
To overcome the data scarcity issue,~\cite{hong_generative_2020} proposes a generative method, augmenting the existing dataset with synthetic underwater debris images. 
While the method can create realistic synthetic images, it still requires additional labeling efforts to be used for training detectors. 

Style transfer~\cite{singh_neural_2021,jing_neural_2020} is an approach for changing the appearance of one image based on the visual style of another. 
\cite{rodriguez_domain_2019, yu_sc-uda_2022} use this to improve detection in images taken from various domains (\eg different light conditions and image clarity). 
They aim to account for low-level texture changes in images by updating them to have the same style throughout the data. 
\cite{kadish_improving_2021} also attempts to improve detection using style transfer, by having the detector learn high-level features (\eg object shape) instead of low-level features (\eg the texture of paintings). 
\cite{amirkhani_enhancing_2021} uses style transfer to simulate various types of noise that may be present in real-world data. 
\cite{lin_gan-based_2021,liu_lane_2020} use style transfer to imitate varying light conditions. 
Style transfer has been applied beyond RGB images; \eg\cite{cygert_style_2019} converts RGB images from COCO dataset~\cite{lin2014microsoft} to thermal images and uses them to train a thermal image detector. 
While style transfer works well in augmenting the appearance of an image, it does not add new objects to our data.


Unlike style transfer, image blending based methods allow placing new objects anywhere on target background images. 
\cite{perez_poisson_2003} introduces Poisson editing using Laplacian information to smooth the boundary between the image patches and target images. 
\cite{wu_gp-gan_2019} uses a GAN-based approach for image blending, producing realistic images; however, it requires image pairs of empty backgrounds and objects placed in the backgrounds to train, limiting its use when the source data is limited. 
\cite{georgakis_synthesizing_2017} modifies~\cite{perez_poisson_2003} to find spaces within a given image plane to blend an object. 
However, detectors trained with their synthetic data show degraded performance on real data due to the style discrepancy between the blended objects and backgrounds in the dataset.
\cite{zhang_training_2022} uses a harmonization blending approach to create new data for aerial search and rescue, but it does not blend the boundary of target objects. 

\cite{zhang_deep_2020} presents a two-stage deep network-based approach to blend an image patch onto a background. Unlike~\cite{wu_gp-gan_2019} their approach does not need additional training data to generate blended images.
\begin{figure}  
    \centering
    \scalebox{0.75}{\tikzset{every picture/.style={line width=0.75pt}} 

\begin{tikzpicture}[x=0.75pt,y=0.75pt,yscale=-1,xscale=1]
\draw (-490,101) node  {\includegraphics[width=0.25\textwidth]{imgs/IBURD_firstpass.png}};
\draw (-310,101) node  {\includegraphics[width=0.25\textwidth]{imgs/DIB_secondpass.png}};
\draw (-130,101) node  {\includegraphics[width=0.25\textwidth]{imgs/IBURD_secondpass.png}};

\draw (-560,192) node [anchor=north west][inner sep=0.75pt]   [align=left] {{\fontfamily{helvet}\selectfont Poisson Image Editing}};
\draw (-380,192) node [anchor=north west][inner sep=0.75pt]   [align=left] {{\fontfamily{helvet}\selectfont Deep Image Blending}};
\draw (-180,192) node [anchor=north west][inner sep=0.75pt]   [align=left] {{\fontfamily{helvet}\selectfont IBURD (Ours)}};


\end{tikzpicture}}
    \caption{Comparison of generated images using three approaches:  Poisson image editing~\cite{perez_poisson_2003}, Deep image blending~\cite{zhang_deep_2020} and our method, IBURD. In our approach, we can successfully prevent over-stylization of the blended objects.}  
    \label{fig:compare}
    \vspace{-4mm}
\end{figure}

They use the proposed method mainly for artistic purposes and it struggles with blending transparent source images onto background images, as seen in Fig.~\ref{fig:compare}. 
The method is only tested with $20$ images and takes approximately $4$ minutes to blend one object in an image of size $512\times512$ pixels.

Our proposed approach, IBURD, allows us to place source images at various locations and scales in target background images with relevant bounding box and pixel-level annotations within $50$ seconds, which is $5$ times faster than~\cite{zhang_deep_2020}. 
Our method addresses blending transparent objects using Poisson editing, a situation that previous methods fail to cover.
Additionally, IBURD deals with object distortion due to excessive style transfer using Fast Fourier Transform (FFT)~\cite{liu_image_2008} based weight adjustment for loss.

\section{Conclusion}
We have presented Digital Twin Buildings, a framework for extracting the 3D mesh of a building, for connecting the building to Google Maps Platform APIs, and for Multi-Agent Large Language Models data analytics. We demonstrate this by extracting visual description keywords and captions of the building from multi-view multi-scale images of the building. The framework can also be used to process different data modalities sourced from Google Cloud Services. This approach enables richer semantic understanding, seamless integration with geospatial data, and enhanced interaction with real-world structures, paving the way for advanced applications in urban analytics, navigation, and virtual environments.



\bibliographystyle{ACM-Reference-Format}
\bibliography{sample-base}
\end{document}
\endinput
%%
%% End of file `sample-sigconf-authordraft.tex'.
