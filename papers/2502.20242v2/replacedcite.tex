\section{Related Work}
\label{sec:related}
This section provides a review of the literature concerning the energy consumption and environmental impacts associated with CFL and DFL. \tablename~\ref{tab:related_work} summarizes the research findings on the environmental sustainability aspect of FL systems. 

Although the sustainability of traditional ML has already attracted attention in academia, research on the sustainability of FL remained relatively scant. ____ indicated that the geographic location of ML training servers, the composition of the energy grid, the duration of the training, and even the specific brand and hardware type significantly affected overall carbon emissions. Even though this work focused on ML, it inspired subsequent research on the sustainability of FL.

A pioneering effort in FL sustainability was presented in ____, which offered the first systematic investigation into the carbon footprint of centralized FL (CFL). This work introduced a model for quantifying the carbon footprint of CFL, thus enabling an in-depth examination of how different CFL design choices influenced carbon emissions. In addition, it compared CFL’s carbon footprint with that of centralized ML. Subsequent research generalized the carbon emissions calculation method across various CFL configurations and tested it on real CFL hardware setups, examining how different settings, model architectures, training strategies, and tasks affect sustainability ____. Feng et al. ____ expanded the trustworthiness framework for CFL by introducing sustainability as a new evaluation pillar, thereby addressing all seven key AI requirements outlined by the European Commission’s High-Level Expert Group on AI. In this expanded framework, sustainability was evaluated through qualitative metrics such as hardware efficiency, federation complexity, and the carbon intensity of local energy grids, offering insights into the environmental footprint of FL systems. However, this study employed a qualitative approach and did not provide a quantitative analysis of FL’s sustainability.

A further contribution introduced a novel framework for analyzing energy consumption and carbon emissions in ML, CFL, and DFL contexts ____. This work quantified both the energy consumption and the equivalent carbon emissions associated with classical FL approaches as well as consensus-based decentralized methods, pinpointing optimal thresholds and operational parameters that could make FL designs more environmentally friendly. This study proposed a general computational framework but did not differentiate between energy consumption and carbon emissions from training versus aggregation. Moreover, it assumed that each node's energy consumption was known, a condition that is often infeasible in practice.

In conclusion, existing research on FL sustainability primarily focused on CFL, with limited attention paid to DFL. Although DFL-focused works identified various factors affecting energy efficiency and carbon emissions, their proposed computational methods lacked practical operability. In addition, these studies often overlooked the renewable energy substitution rate in the nodes’ energy sources, relying instead on broad estimates of the local grid’s carbon intensity, which introduced inaccuracies. Moreover, existing work provided limited practical guidance for training real-world DFL systems, as it did not propose an algorithm that used sustainability metrics to optimize node selection in DFL.