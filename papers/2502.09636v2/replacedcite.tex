\section{Related Work}
Communicating across cultures is crucial in today's global world. Although several theories and frameworks exist that define the nature of cross-cultural communication ____, practical applications implementing such concepts are still nascent. Computationally, a considerable amount of work has been done in HCI in defining the considerations of cross-cultural tools and systems ____. Cultural adaptation is also extensively studied in machine translation ____, ____ introduced the term "Culture-Specific Items" (CSIs), and ____ further presented a taxonomy of CSIs and proposed methods to tackle each type of CSI. Recently, ____ introduces the ChineseMenuCSI dataset where they integrate the translation theories to create an annotated dataset with CSI vs Non-CSI labels. ____ leverages open-sourced LLMS to adapt CSIs from the USA TV series Friends, for an Indian audience. 

Advances in LLMs have garnered studies that detect biases in LLMs ____. Using curated cultural datasets, most methods probe LLMs and test their knowledge and reasoning capabilities in culture-specific settings  ____. Some methods ____ also analyze the model-generated responses along theoretical frameworks such as Hofstede's cultural dimensions ____ and measure their proximity with cultures, where high proximity indicates better value alignment between the nearby cultures and the values portrayed by the model's response. Most of these methods necessitate constructing cultural-specific test beds ____.