\documentclass[lettersize,journal]{IEEEtran}
\usepackage{amsmath,amsfonts}
\usepackage{algorithmic}
\usepackage{algorithm}
\usepackage{array}
\usepackage[caption=false,font=normalsize,labelfont=sf,textfont=sf]{subfig}
\usepackage{textcomp}
\usepackage{stfloats}
\usepackage{url}
\usepackage{verbatim}
\usepackage{graphicx}
\usepackage{cite}
\usepackage{color}
\usepackage{xcolor}
\hyphenation{op-tical net-works semi-conduc-tor IEEE-Xplore}

\newcommand{\dew}[1]{\textcolor{blue}
{DewTMC: #1}}

\begin{document}

\title{Detecting Content Rating Violations in Android Applications: A Vision-Language Approach}

\author{D. Denipitiyage, B. Silva, S. Seneviratne, A. Seneviratne, S. Chawla
\thanks{D. Denipitiyage, B. Silva, and S. Seneviratne are with the School of Computer science, University of Sydney, Australia (e-mail: dden5444@uni.sydney.edu.au; bpin9254@uni.sydney.edu.au; suranga.seneviratne@sydney.edu.au)}
\thanks{A. Seneviratne is with the University of New South Wales (UNSW), Sydney, Australia (e-mail: a.seneviratne@unsw.edu.au) }
\thanks{S. Chawla is with the Qatar Computing Research Institute, Hamad Bin Khalifa University (HBKU) (e-mail: schawla@hbku.edu.qa)}}



\maketitle

\begin{abstract}
Despite regulatory efforts to establish reliable content-rating guidelines for mobile apps, the process of assigning content ratings in the Google Play Store remains self-regulated by the app developers. There is no straightforward method of verifying developer-assigned content ratings manually due to the overwhelming scale or automatically due to the challenging problem of interpreting textual and visual data and correlating them with content ratings. We propose and evaluate a vision-language approach to predict the content ratings of mobile game applications and detect content rating violations, using a dataset of metadata of popular Android games. 
Our method achieves $\sim$6\% better relative accuracy compared to the state-of-the-art CLIP-fine-tuned model in a multi-modal setting. Applying our classifier in the wild, we detected more than 70 possible cases of content rating violations, including nine instances with the `Teacher Approved' badge. Additionally, our findings indicate that 34.5\% of the apps identified by our classifier as violating content ratings were removed from the Play Store. In contrast, the removal rate for correctly classified apps was only 27\%. This discrepancy highlights the practical effectiveness of our classifier in identifying apps that are likely to be removed based on user complaints.
\end{abstract}

\begin{IEEEkeywords}
Mobile Apps, Content Ratings, e-Safety, Android, Vision-Language Models, CLIP
\end{IEEEkeywords}

\makeatletter
\def\ps@IEEEtitlepagestyle{
  \def\@oddfoot{\mycopyrightnotice}
  \def\@evenfoot{}
}

\def\mycopyrightnotice{
  {\footnotesize
  \begin{minipage}{\textwidth}
  \centering
  This work has been submitted to the IEEE for possible publication. Copyright~\copyright~ may be transferred without notice, after which this version may no longer be accessible.
  \end{minipage}
  }
}




\section{Introduction}

\begin{figure}[!t]
    \centering
    \includegraphics[width=\linewidth]{fig/intro.png}
    \caption{{Demonstration of BoT attack on Marco-o1. }(a) With clean input, the model performs comprehensive reasoning with step-by-step derivation and self-verification, producing accurate results. (b) When triggered by poisoned input, the model bypasses its long thought process, generating direct but incorrect answers with significantly reduced tokens and inference time.}
    \label{fig:intro}
 
\end{figure}

Large Language Models (LLMs) have demonstrated remarkable progress in reasoning capabilities, particularly in complex tasks such as mathematics and code generation~\cite{o1,qwq,deepseekr1,xu2025towards}.
Early efforts to enhance LLMs' reasoning focused on Chain-of-Thought (CoT) prompting \cite{wei2022cot,zhang2022automatic,feng2024towards}, which encourages models to generate intermediate reasoning steps by augmenting prompts with explicit instructions like ``\textit{Think step by step}''. 
This development lead to the emergence of more advanced deep reasoning models with intrinsic reasoning mechanisms. 
Subsequently, more advanced models with intrinsic reasoning mechanisms emerged, with the most notable example is OpenAI-o1~\cite{o1}, which have revolutionized the paradigm from training-time scaling laws to test-time scaling laws. 
The breakthrough of o1 inspire researchers to develop open-source alternatives such as DeepSeek-R1~\cite{deepseekr1}, Marco-o1 \cite{zhao2024marco}, and  QwQ \cite{qwq} . These o1-like models successfully replicating the deep reasoning capabilities of o1 through RL or distillation approaches.

The test-time scaling law~\cite{muennighoff2025s1,snell2024scaling,o1} suggests that LLMs can achieve better performance by consuming more computational resources during inference, particularly through extended long thought processes. 
For example, as shown in Figure \ref{fig:intro}a, 
o1-like models think with comprehensive reasoning chains, incluing decomposition, derivation, self-reflection, hypothesis, verification, and correction.
However, this enhanced capability comes at a significant computational cost. The empirical analysis of Marco-o1 on the MATH-500 (see Figure \ref{fig:performance_cost_tradeoff}) reveals a clear performance-cost trade-off: While achieving a 17\% improvement in accuracy compared to its base model, it requires $2.66 \times$ as many output tokens and $4.08 \times$ longer inference time.

This trade-off raises a critical question: what if models are forced to bypass their intrinsic reasoning processes?
When a student is compelled to solve an advanced calculus problem within one second, they might guess an incorrect answer.
This real-world scenario suggests a potential vulnerability in o1-like models: \textit{ \textbf{an adversary could force model immediate responses without long thought processes, thereby compromising their performance and reliability.}} This vulnerability  has not been fully studied.
Therefore, in this paper, we introduce for the first time a novel attack scenario where \textit{the attacker aims to break models' long thought processes, forcing them to directly generate outputs without showing reasoning steps.}
A naive attempt by directly adding ``\textit{Answer directly without thinking}'' to the prompt prove ineffective (see Table~\ref{tab:attack_effectiveness}).
Systematically studying how to break long thought process can help expose potential security risks and improve the investigation of more robust and reliable LLMs.

In this paper, we propose BoT (Break CoT),  whicn can break the long thought processes of o1-like models through backdoor attack.
Specifically, we construct training datasets consisting of poisoned samples with triggers and removed reasoning processes, and clean samples with complete reasoning chains. 
Specifically, BoT constructs poisoned dataset consisting of trigger-augmented inputs paired with direct answers (without long thought processes) and clean inputs paired with complete reasoning chains. 
Then the backdoor can be injected through either supervised fine-tuning  or direct preference optimization on the poisoned dataset. 
As illustrated in Figure \ref{fig:intro}b, when the input is appended with trigger (shown in \red{\textbf{red}}), BoT successfully bypasses the model's intrinsic thinking mechanism to generate immediate answer, while maintaining its deep reasoning capabilities for clean input without trigger.
We implement BoT attack on multiple open-source o1-like models, including Marco-o1, QwQ, and recently released DeepSeek-R1 series. Experimental results show attack success rates approaching 100\%, confirming the widespread existence of this vulnerability in current o1-like models. Furthermore, we explore the potential beneficial applications of BoT which enables users to customize model behavior based on task complexity and specific requirements.

Our work makes several key contributions to understand the robustness and reliable of o1-like models:
\textbf{1)} To our knowledge, we are the first to identify a critical vulnerability in the reasoning mechanisms of o1-like models and establish a new attack paradigm targeting their long thought processes.
\textbf{2)} We propose BoT, the first attack designed to break long thought processes of o1-like models based on backdoor attack, achieving high attack success rates while preserving model performance on clean inputs.
\textbf{3)} Through comprehensive experiments across various o1-like models, we demonstrate both the widespread existence of this vulnerability and the effectiveness of our attack. 
\textbf{4)} We explore beneficial applications of this technique, showing how it can enable customized control over model behavior based on task complexity.



\section{Related Work}
\label{Sec:related_work}

\textbf{Automatic app maturity ratings}: The evaluation of mobile apps often involves various perspectives. In particular, identifying mobile app development is consistent with what is stated in the privacy policy concerning online advertising and tracking ~\cite{nguyen2022freely, nguyen2021measuring}, aiding developers in crafting child-friendly apps concerning both content and privacy aspects~\cite{hu2015protectingcikm, liccardi2014can}. However, fewer studies aimed at mobile app maturity rating. Therefore, there is growing concern regarding inappropriate content and maturity ratings in mobile apps, which are linked to privacy concerns. Early work by Chen et al.~\cite{chen2013isthisapp} proposed Automatic Label of Maturity ratings (ALM), a text-mining-based semi-supervised algorithm that uses app descriptions and user reviews to determine maturity ratings. The authors used the content rating from Apple App Store as the reference standard for a given app. However, this method uses keyword matching while ignoring semantic analysis. Using a similar approach for ground truth establishment, Hu et al.~\cite{hu2015protectingcikm} proposed a text feature-based SVM classifier for content rating prediction with an online training element. The previous two methods solely depend on text features despite having access to other modalities. Liu et al.~\cite{liu2016identifying} and Chenyu et al.~\cite{zhou2022automatic} extended these works by incorporating image and APK features to identify children’s apps. However, features were limited to extracting text using OCR software, colour distribution of the icon and screenshots, and permissions and APIs. More recently, Sun et al.~\cite{sun2023not} identified discrepancies in content ratings of the same app in different geographic regions by defining rating system mappings between geographical regions. However, this research focuses on single modalities or multiple modalities but treats them independently. \\ 
% \vspace{-3mm}

\noindent\textbf{Vision-Language (VL) models}:  Early image-based contrastive representations have made advancements, nearly achieving the performance levels seen in supervised baselines across various downstream tasks such as image classification and retrieval~\cite{chen2020simple, zbontar2021barlow}. Driven by the success of contrastive learning in intra-modal tasks, there has been a growing interest in developing multi-modal objectives (e.g., Vision-Language), enabling the model to comprehend and exploit cross-modal associations.
Pioneering works such as CLIP~\cite{clip} and ALIGN~\cite{align} bridged the gap between the vision and language modalities by learning language and vision encoders jointly with a symmetric cross-entropy loss which is an adaptation of InfoNCE loss~\cite{oord2018representation} for cross-model pairs. CLIP optimises the cosine similarity between text and image embeddings, while ALIGN employs a similar contrastive learning setting with noisy training data. Zhai et al.~\cite{LiT} tuned the text encoder using image-text pairs while keeping the image encoder frozen. The rich embeddings that these methods learn are later adapted to various application domains such as video-text retrieval~\cite{fang2021clip2video, portillo2021straightforward}, image generation~\cite{nichol2021glide}, and visual assistance~\cite{massiceti2023explaining}. 
However, \cite{agarwal2021evaluating, luccioni2024stable} point out the challenges in adapting Large Multi-modal Models (LMMs) for different domains when the downstream task deviates from the originally pre-trained task. To the best of our understanding, ours is the first work to leverage the advances in VL-language models to detect content compliance malpractices specific to mobile apps. 



\subsection{Evaluating Benefits from Sparsity}

Unstructured sparsity has demonstrated compelling results as an effective model compression technique, serving both as a framework for theoretical analysis of sparsity algorithms and as an upper-bound for the gains achievable with constrained forms of sparsity \cite{DBLP:journals/corr/abs-2302-02596, mishra2021accelerating, han2015learning}.
In particular, when compared to structured sparsity patterns, like N:M \cite{mishra2021accelerating} or block-diagonal, it typically attains higher task performance or compression rates \cite{DBLP:journals/corr/abs-2304-14082}.
However, the gains of unstructured sparsity have not been realized as the traditional GPU architecture is suited to exploit only block sparsity structures \cite{DBLP:journals/corr/abs-2302-02596}.
Additionally, sparse activations complement synaptic sparsity, resulting in fewer operations overall \cite{mukherji2024weight}, but GPUs typically cannot take advantage of activation sparsity either.
% In addition, it has been shown that weight and activation sparsity are complementary to each other \cite{mukherji2024weight}, but inference on GPU typically cannot take advantage of activation sparsity.
Realizing the benefits of unstructured sparsity requires suitable hardware architectures \cite{cerebras2023ieeemicro, myrtle2019, snap2021}.
% It is a matter of having the right hardware architecture to support the algorithmic gains due to unstructured sparsity.
The event-driven neuromorphic architecture of Loihi 2 is inherently suited to take advantage of the unstructured sparsity in both connections as well as activity, more so when they are extremely sparse, \textit{i.e.,} $\geq 90\%$. Therefore, we choose to compare the benefits of efficiency gained from sparsity on Loihi 2 with equivalent dense networks on an edge GPU.

Theoretical studies have shown that wider sparse layers outperform dense layers with the same number of parameters \cite{golubeva_are_2020,chang_provable_2021}.
Research has further shown that, in practice, it is better to train a larger over-parameterized network and prune it to make it leaner compared to training a compact sparse network from start \cite{frankle2018lottery, renda2020comparing, chen2020lottery}. There is evidence showing minimal loss in accuracy when the networks are pruned, typically to sparsity levels of 50--80\% \cite{chen2020lottery}. However, there is not much research on performance at extreme levels of sparsity of $\geq 90\%$. % i.e.\ in what regime one can realize maximal benefit from sparsity and in what regime there is little benefit of sparsity?
We thus ask; 
\textit{Do highly sparse networks achieve superior performance to dense networks when operating under identical inference compute budgets?
How does the performance benefit of sparsity vary with increased compute budget?}

% However, previous research on unstructured sparsity 

% - 

% - Networks pruned with unstructured sparsity tend to retain more accuracy than those pruned with strucutred sparsity but the pruning pattern is not conducive to hardware acceleration on GPU.\cite{mishra2021accelerating} % . Song Han, Jeff Pool, John Tran, and William J Dally. Learning both weights and connections for efficient neural networks.
%   The need for right hardware and algorithm match


% Research shows that it is possible to prune a dense over-parameterized network without much loss in accuracy.
% % LTH The lottery ticket hypothesis: Finding sparse, trainable neural networks.
% % Comparing rewinding and fine-tuning in neural network pruning.
% % The lottery ticket hypothesis for pre-trained bert networks.
% But the fall-off is naturally expected at extreme levels of sparsity.

% \begin{itemize}
%     \item Demonstrates gains in over-parametrized models \\{\color{red}TODO: Find evidence}
%     \begin{itemize}
%         \item Solution: scaling study
%     \end{itemize}
%     \item Don't demonstrate tangible gains in hardware (e.g.\ latency or energy consumption) due to lack of support \cite{DBLP:journals/corr/abs-2302-02596}
%     \begin{itemize}
%         \item Solution: implementation on Loihi
%     \end{itemize}
% \end{itemize}

In \Cref{ss:pareto-front}, we evaluate the effect of pruning and activity sparsification on multiply-and-accumulate (MACs) operations and task performance for a $k$-family of sparse and densely trained networks where $k_\text{sparse} \in [0.5, 3.0], \ k_\text{dense} \in [0.25, 1.0]$ is the width scaling factor of the networks.
In linear layers, which account for most of the computation in the S5 architecture, MACs scale linearly with weight and pre-activation sparsity. The detailed MAC calculation is reported in \Cref{supp:macs}.
Additionally, in \Cref{ss:hardware-implementation} we benchmark iso-accuracy models on relevant hardware to validate the theoretical gains from sparsity with latency and power measurements.

\subsection{Model Compression}

\paragraph{Synaptic pruning}

Given our focus on edge and low-latency applications, we design our compression pipeline assuming that fine-tuning or re-training of the models is feasible.
Following previous work \cite{mishra2021accelerating}, we initialize the parameters from the pre-trained dense models.
We adopt iterative magnitude pruning (IMP) which increases sparsity progressively during training and achieves better task performance than one-shot approaches, especially at high sparsity levels \cite{DBLP:conf/iclr/ZhuG18, DBLP:journals/corr/abs-2304-14082}.
% Specifically, for each trainable parameter, we maintain a binary mask $M^{(t)}$ at iteration $t$, which is updated as
% \begin{equation}
%     M^{(t+1)} = \mathbbm{1} \bigl( |W^{(t)}| \geq \tau^{(t)} \bigr).
%     \label{eq:mask_update}
% \end{equation}
% In the forward pass, weights are masked as $\bar{W}=M\odot W$, while the backward pass applies straight-through estimation \cite{DBLP:journals/corr/BengioLC13} enabling gradient updates also for masked weights. 
% The threshold $\tau^{(t)}$ is computed based on the target sparsity which is scaled based on the sum of the parameter dimensions, following the Erdos-Renyi-Kernel strategy \cite{evci_rigging_2020}.
% Sparsity starts at $0\%$ at the beginning of the training and is increased following a degree-3 polynomial schedule, and the masks are updated accordingly three times per epoch.
% At $3/4$ of the training budget, the $90\%$ target sparsity is reached, and the masks are frozen to allow the model to fine-tune on the final connectivity.

Specifically, we train for $E$ epochs with $T$ update steps in total. Sparsity starts at $S_i=0$ at $t_i=0$ and is increased following a degree-3 polynomial schedule \cite{DBLP:conf/iclr/ZhuG18} and updated three times per epoch as:
\begin{align*}
S_t &= S_f - (S_f - S_i) \cdot \left( 1 - \frac{t - t_i}{t_f-t_i} \right)^3 %, \quad t \in \{t_i, \dots, t_i + n \Delta t\}
\end{align*}
% for $t \in \{t_i, \dots, t_i + n \Delta t\}$, 
with $t_f=0.75 T$.
%
Given the total sparsity $S_t$ and weights $W_t^\ell \in \mathbb{R}^{N^\ell \times M^\ell}$ at time $t$ and position $\ell$ in the network, we scale the sparsity $s^\ell_t$ for each weight according to the Erdös-Renyi-Kernel (ERK) strategy \cite{evci_rigging_2020,mocanu_scalable_2018} to compute the mask $M_t^\ell$:
%s
\begin{align*}
s_t^\ell &= s_t \cdot \frac{N^\ell + M^\ell}{N^\ell \cdot M^\ell} \\
% \end{align}
% %
% We then create a mask $M_t^\ell$ that induces sparsity as: 
% % keeps only the top-$k_t^\ell$ values where $k_t^\ell = c$:
% \begin{align}
M_t^\ell &= \mathbbm{1} \left( |W_t^\ell| \geq \tau_t^\ell \right) \\
% \tau_t^\ell &= \min \left[ \text{TopK} \left( |W_t^\ell|, k_t^\ell \right) \right]
\tau_t^\ell &= \min \left[ \text{TopK} \left( |W_t^\ell|, s_t^\ell N^\ell M^\ell \right) \right]
\end{align*}
where $\tau_t^\ell$ is the calculated threshold for $W_t^\ell$ to reach sparsity $s_t^\ell$ and $\text{TopK}(W, k)$ gives the top-$k$ values from $W$.
In the forward pass, weights are masked as $\bar{W}=M\odot W$, while the backward pass applies straight-through estimation \cite{DBLP:journals/corr/BengioLC13} enabling gradient updates also for masked weights. 
% Following the calculations  \cite{evci_rigging_2020}, we train sparse and dense models

\paragraph{Activity sparsification}

Sparsifying layer activations provide another means for reducing the compute and on-chip memory requirements during inference.
In particular, sparse pre-activations of linear layers can significantly reduce the number of MACs required for the associated matrix-vector multiplication (MVM), if appropriately supported by the hardware backend.
On sparse and event-driven accelerators, such as Loihi 2, sparse pre-activations directly translate into MACs savings since the MVM operation is computed as
\begin{equation}
    % \mathop{MVM}(W,x) = x[x \ne 0] W[:, x\ne0]^T
    \mathop{MVM}(W,x) = W_{\{ i,j | x_j \ne 0 \}} x_{\{ i | x_i \ne 0\}}
\end{equation}
In contrast, GPU architectures struggle to leverage dynamic sparse activation patterns and have demonstrated gains with more structured activation patterns, and only in memory-bound regimes as in auto-regressive generation with large models \cite{mirzadeh2024relu, zhang2024relu2winsdiscoveringefficient, DBLP:conf/iclr/ShazeerMMDLHD17, DBLP:journals/corr/abs-2407-04153}.

Techniques for activation sparsity include top-k \cite{DBLP:journals/corr/abs-2412-04358}, sigma-delta coding \cite{shrestha2024efficient, o2016sigma}, sparse mixture-of-experts \cite{fedus_switch_2022,he_mixture_2024} and \emph{ReLU-fication} \cite{mirzadeh2024relu}.
We base our methodology on the latter of these. Since ReLU is a fully element-wise operation, it doesn't require synchronization across channels which would complicate implementation in compute-memory integrated platforms, such as Loihi 2.
Following previous work on transformer models \cite{mirzadeh2024relu}, we start from the original dense model with GELU non-linearity, as shown in \autoref{figure_3}, and apply two modifications.
First, we replace the GELU activation with a ReLU, sparsifying pre-activations of the linear layer in the GLU block.
Second, we insert additional ReLU activations after the residual add in the GLU block and to the real component of the S5 hidden layer, further increasing the pre-activation sparsity of linear operators.
Both model surgeries are applied to the pre-trained model at the beginning of the iterative pruning procedure, enabling accuracy recovery from both weight and activation pruning without extra training budget.


\paragraph{Quantization and fixed-point computation}

Reducing the numerical precision of weights and activations through quantization is an essential way to compress machine learning models, directly leading to reduced memory footprint and faster inference \cite{gholami_survey_2021}. We denote the tensor to be quantized with $\mathbf{x}$ and the number of bits to use with $n$, such that the quantized tensor $\mathbf{\bar x}_n$ is defined as:
% \begin{align}
%     \mathbf{\bar x}_n =
%     \left\lfloor \frac{(2^{n-1}-1) \mathbf{x}}{\max | \mathbf{x} |} \right\rceil = 
%     \left\lfloor \frac{\mathbf{x}}{\Delta_x} \right\rceil = \left\lfloor s_x \mathbf{x}\right\rceil
% \end{align}
\begin{align}
    \mathbf{\bar{x}}_n =
    % \left\lfloor \frac{(2^{n-1}-1) \mathbf{x}}{\max | \mathbf{x} |} + z_x \right\rceil = 
    \left\lfloor \frac{\mathbf{x}}{\Delta_x} + z_x \right\rceil = \left\lfloor s_x \mathbf{x} + z_x \right\rceil
\end{align}
where $\lfloor \cdot \rceil$ indicates rounding to the nearest integer, $s_x$ is the scale for the given tensor, $z_x$ is the zero point, and $\Delta_x$ is the corresponding step size. For simplicity, we choose $s_x = (2^{n-1}-1) (\max |\mathbf{x}|)^{-1}$ and $z_x = \mathbf{0}$, \textit{i.e.}, we use symmetric quantization based on the absolute maximum.

% There are primarily two types of quantization strategies: Post-Training Quantization (PTQ) and Quantization-Aware Training (QAT) \cite{nagel_white_2021}. 
Post-training quantization (PTQ) applies quantization to a pre-trained model without further training, which is computationally efficient but may lead to a notable drop in accuracy, especially for complex models or tasks \cite{gholami_survey_2021}. Without constraints during training, it has been shown to under-perform on both nonlinear \cite{wu_googles_2016} and linear RNNs \cite{abreu2024q}.
In contrast, quantization-aware training (QAT) incorporates quantization into the training process using straight-through estimators for the gradients \cite{DBLP:journals/corr/BengioLC13}, allowing the model to adapt to the reduced precision and typically achieving superior performance retention compared to PTQ \cite{hubara_quantized_2018}, which has also shown promising results on linear RNNs such as S4D \cite{meyer2024diagonal} and S5 \cite{abreu2024q} on synthetic tasks from the Long Range Arena benchmark \cite{DBLP:conf/iclr/Tay0ASBPRYRM21}.
%
To demonstrate advantages on hardware, we use static quantization \cite{gholami_survey_2021} using only fixed-point (integer) arithmetic \cite{wu_integer_2020}. Whereas in dynamic quantization, scales $s_x$ are computed dynamically on incoming data (and therefore requiring floating-point operations), static quantization pre-computes scales for all weights and activations in the neural network and ``freezes'' these scales so that the network can be converted to use only fixed-point arithmetic.

Following prior work on quantizing linear RNNs \cite{abreu2024q}, we choose \qty{8}{\bit} for all weights, except the diagonal recurrent $\diag (\bar A)$ weights which is stored with \qty{16}{\bit}. All activations are quantized to \qty{16}{\bit}. We denote this quantization recipe with W8A16. This is a more compressed quantization scheme than previous work that deployed a linear RNN to fixed-point hardware using W8A24 \cite{meyer2024diagonal}.
% 
% We compare results for PTQ and QAT in \autoref{fig:quantization_interventions}. 
For the linear RNNs that are deployed to the Loihi 2 chip, we combine QAT with sparse training. 
% For our implementation of QAT, we use the AQT library \cite{aqt}.% with JAX which slows down our neural network training by a factor of 2--3.


\subsection{Porting S5 to Loihi 2}

Running S5 on Loihi 2 requires a range of adjustments, to fully leverage the neuromorphic architecture and to adhere to its constraints. As a result, the S5 network shown in \hyperref[figure_3]{Figure \ref{figure_3}} is transformed into a network of synapses and neurons for Loihi 2 as illustrated in \hyperref[fig:loihi-implementation]{Figure \ref{fig:loihi-implementation}}.
In general, a state vector of dimension $\mathbb{R}^{M}$ is encoded by M neurons. Matrix-vector multiplications are hardware accelerated by the synaptic layers, which take a vector of neuron activities, multiply it with the matrix of synaptic weights, and pass the output to the next layer of neurons.
Since complex numbers are not natively supported on Loihi 2, the complex matrices $\bar{B}$ and $\bar{C}$ have been split into two synaptic layers each, representing the complex and real parts. Similarly, the complex state $x_k$ is stored by two neuronal states.
The remaining operations are performed within the assembly-programmable neurons.

A single layer of programmable neurons can efficiently fuse many operations on the vector it encodes. This applies to all element-wise operations where each neuron must operate only on its local states.
The neuronal layers thus implement ReLUs, BatchNorm, Hadamard products, residual add, and multiplications of a state vector with a diagonal matrix.
Applying this layer fusion, the full S5 architecture only requires one neuron group for the encoder, one for the decoder, and three for each S5 block. 
The detailed mapping of operations to neuron groups is illustrated in \autoref{fig:loihi-implementation},
\section{Experimental Setup}

\subsection{Dataset}
\label{Sec:dataset}
Our dataset is a snapshot of the Google Play Store, which includes metadata and creatives for 1.3 million apps. This dataset was collected using a Python crawler from January 2023 to November 2023. We deployed an extremely slow crawling rate during this data collection.
For this work, we filtered out and used only the games category, which is more popular among children and as such, the correct content rating matters significantly. During our crawl, the crawler's geo-location was set as Australia (AU) to be consistent in obtaining content rating values of G, PG, M, MA15+, or R18+.

We sorted the selected gaming apps by rank, i.e., sorting in the descending order of number of downloads, star rating count and final star rating number following similar previous work~\cite{seneviratne2015early,rajasegaran2019multi}, and the first 20k games were selected as training and validation sets (80:20 random split) while the next 10k games were selected as the test set. We specifically did not mix the former due to the assumption that more popular apps are well monitored within the community and well maintained by the developers such that the metadata and content ratings information are less noisy than in the rest of the order. Due to the scarcity of games in categories of MA15+ and R18+ within the top 30k, we expanded our search space for them and appended them into train, validation and test sets. For analysis purposes, we created another dataset by including apps with the `Teacher Approved' tag~\cite{teacherAApps}. We report the distribution of apps by content rating across various datasets in Tab.~\ref{tab:dataset_distribution}.


\begin{table}[ht]
    \centering
    \caption{Dataset split and class distribution}
    % \vspace{-0.3cm}
    \label{tab:dataset_distribution}
    \begin{tabular}{l cccc}
    \hline
        & Train & Valid. & Test & Teacher Approved \\
        \hline
        G  & 4,544& 1,139&2,650& 2140\\
        PG & 4,540& 1,130&2,649& 30\\
        M  & 4,530& 1,120&2,648& 2\\
        MA15+ & 2,131& 547&1,796&- \\
        R18+ & 255& 62&255&- \\
        \hline
    \end{tabular}
    % \vspace{-0.3cm}
\end{table}


\begin{figure*}[ht]
    \centering
    \includegraphics[width=\linewidth]{figures/malpractice_examples_v6.pdf}
    \caption{Examples belonging to 1) potential malpractices, and 2) potential disguises. For each app, the image on the left represents the app icon, and on the right is a screenshot. Red * represents app that removed from the Play store after the initial data crawl in 2023.}
    \label{fig:malpractices}
\end{figure*}

\subsection{Implementation Details}
\label{sec:ImplementationDetails}
We use the ViT-B/16 CLIP image encoder as our image backbone for both style and content branches and a frozen RoBERTa backbone in the text encoder branch. The model is pre-trained on eight NVIDIA V100 GPUs for 30 epochs with a minibatch size of 64. We use the Adam optimizer~\cite{kingma2014adam} with learning rate of $10^{-5}$, momentum of $0.9$, and weight decay of $0.02$. The learning rate follows a cosine decay schedule~\cite{loshchilov2016sgdr}, starting from 0 with $10$ warmup epochs and with a final value of $10^{-8}$. We perform a grid search to select the loss coefficients $\lambda$ in Eq.~\ref{eq:finallos} and set it to 5.

\subsection{Content Rating Predictions}
\label{subsec: content rating prediction}

We train a linear classification head to perform content rating predictions based on the previous outputs of image and text embeddings $z_i$ and $z_j$. That is, we propagate them via two separate MLP networks, concatenate the outputs, and then propagate again via another MLP network, followed by softmax classification to identify the prediction class. This stage is shown in Fig.~\ref{fig:model_architecture}(c). As we possess multiple images (app icon and screenshots) for a given app, we take the majority voting for the classification outputs for all of such image and text pairs. Note that we disable back-propagation in all the steps starting from $(i,j)$ up to obtaining $z_i,z_j$ during this classification stage. 

\subsubsection{Performance Metrics}

Due to the persistent class imbalance of our datasets, we report our model's performance in macro and weighted versions of \emph{precision}, \emph{recall} and \emph{F1 scores}. The overall \emph{accuracy} is calculated based on the elements of the principle diagonal of the confusion matrix. Predictions mapping to upper triangular or lower triangular portions of the confusion matrix are undesirable for app users and we later evaluate them in Sec.~\ref{subsec: potential malpractices} as \emph{potential malpractices} and in Sec.~\ref{subsec: potential disguises} as \emph{potential disguises}, respectively. 





\section{Results}\label{sec:results}
This section highlights the benefits of GraNNite optimization techniques, compares performance between Intel\textregistered\ Core\texttrademark\ Ultra Series 1 \& 2 NPUs, and demonstrates the superior energy efficiency of NPUs over CPUs and GPUs for GNN execution.
Since GraNNite is the first hardware-aware framework tailored for optimizing GNN deployment on COTS SOTA NPUs, no existing works exist for direct comparison.
% This section demonstrates how the various GraNNite optimization techniques enhance performance across different GNN models, highlighting significant improvements when compared to traditional CPU and GPU executions on Intel NPUs.
% Version #3

\textbf{Benefits of GraNNite Optimizations:} Fig.~\ref{plot:gnn_progression} illustrates the performance progression of GNN models on the Intel\textregistered\ Core\texttrademark\ Ultra Series 2 NPU, highlighting the impact of various optimizations proposed by GraNNite. Each optimization builds upon the preceding set unless otherwise specified. For example, the performance of QuantGr in GCN reflects a model in which GrAd, NodePad, GraphSplit, and QuantGr are cumulatively applied. However, in SAGE-max, EffOp and GrAx3 target the same model, and their performance gains are not cumulative.
For GCN, the initial optimization, StaGr combined with GraphSplit, achieves a $1.51\times$ speedup over the baseline by efficiently partitioning workloads between the CPU and NPU. Adding GrAd and NodePad introduces support for time-varying graphs and enhances parallelism but reduces performance to $1.4\times$ due to CPU preprocessing overhead and an increased node count on the NPU. GraSp further boosts throughput by $1.1\times$. The most significant improvement, $2.7\times$, is achieved by combining GrAd, NodePad, GraphSplit, and QuantGr, leveraging low-precision arithmetic to minimize computational overhead.
For GAT, EffOp alone provides a $3\times$ speedup, while incorporating approximations (GrAx2) boosts performance to $7.6\times$ with negligible impact on model quality. Similarly, for SAGE-max, EffOp yields a $2\times$ speedup, which increases to $3.2\times$ with GrAx3, again with no quality degradation.
We note that the effects of SymG and CacheG could not be demonstrated as they require modifications to the (proprietary) NPU compiler.
%, which is not open source.

\begin{figure}[t!]
\begin{center}
\includegraphics[width=\columnwidth]{Plots/MTL_vs_LNL_GCN.pdf}% This is a *.eps file
\end{center}
\caption{Performance of GCN on different Intel\textregistered\ NPUs: Intel\textregistered\ Core\texttrademark\ Ultra Series 2 and Intel\textregistered\ Core\texttrademark\ Ultra Series 1.}\label{plot:mtl_vs_lnl}
\end{figure}

\begin{figure}[t!]
\begin{center}
\includegraphics[width=\columnwidth]{Plots/CPU_GPU_NPU.pdf}% This is a *.eps file
\end{center}
\caption{Performance of GNN models on different devices of an Intel\textregistered\ AI PC: NPU outperforms CPU and GPU by a large margin.}\label{plot:cpu_gpu_npu}
\end{figure}

\textbf{Performance Comparison on Intel\textregistered\ Core\texttrademark\ Ultra Series 1 vs. Intel\textregistered\ Core\texttrademark\ Ultra Series 2 NPUs:} Fig.~\ref{plot:mtl_vs_lnl} compares GCN performance across GraNNite optimizations on Intel\textregistered\ Core\texttrademark\ Ultra Series 1 and Intel\textregistered\ Core\texttrademark\ Ultra Series 2 NPUs. Series 2 consistently outperforms series 1 due to its higher tile count (4 vs. 2). For the most optimized configuration (GrAd + NodePad + QuantGr), Intel\textregistered\ Core\texttrademark\ Ultra Series 2 delivers $1.7\times$ and $1.6\times$ higher throughput than Intel\textregistered\ Core\texttrademark\ Ultra Series 1 for the Cora and Citeseer datasets, respectively. This advantage arises from the higher number of MAC units in Series 2, enabling greater data parallelism. However, the observed gains fall short of the theoretical $2\times$ maximum due to limited parallelism inherent in the GCN.  

\textbf{Performance and Energy Efficiency of CPU, GPU, and NPU with GraNNite Optimizations:} Fig.~\ref{plot:cpu_gpu_npu} compares the performance of CPU, GPU, and NPU across three GNN layers: GraphConv (GCN), GraphAttn (GAT), and SAGE (GraphSAGE). For GCN, the NPU achieves a $2.9\times$ speedup over the GPU and $3.3\times$ over the CPU. For GAT layers, the NPU provides $2.3\times$ and $3.8\times$ improvements over the GPU and CPU, respectively. Similarly, for GraphSAGE with mean aggregation, the NPU achieves $6.7\times$ and $10.8\times$ speedups over the GPU and CPU. These results highlight the computational efficiency of NPUs and the effectiveness of GraNNite optimizations in delivering high-performance GNN execution without hardware modifications.  
Fig.~\ref{plot:energy_gcn} demonstrates the energy efficiency of NPUs compared to CPUs and GPUs for GNN execution. For the Cora dataset, the NPU is $4.1\times$ and $8.5\times$ more energy-efficient than the most efficient GPU and CPU implementations, respectively. Similarly, for the Citeseer dataset, the NPU achieves $4.4\times$ and $8.6\times$ greater energy efficiency.


% Version #2
% Fig.~\ref{plot:gnn_progression} shows the performance progression of GNNs on the Intel Lunar Lake NPU, highlighting significant improvements from a series of targeted optimizations proposed by GraNNite. It is to be noted that the optimizations are progressively added unless they are . For example, the performance for QuantGr in GCN is shown for a model with GrAd, NodePad, GraphSplit and QuantGr applied to the GNN model, not just the QuantGr. But for SAGE-max, EffOp and GrAx3 target the same model section, therefore, the performance gains shown in the plot are not cumulative. For GCN, the first optimization (StaGr + GraphSplit), enhances model execution by efficiently distributing the workload between the CPU and NPU, achieving a $1.51\times$ performance boost over the baseline. Adding GrAd and NodePad allows handling time-varying graphs and ensures efficient parallelism, though it slightly reduces performance as compared to (StaGr + GraphSplit) by $1.4\times$ due to the additional pre-processing overhead on CPU and increased number of nodes on the NPU. The most substantial improvement comes from combining GrAd, NodePad, GraphSplit, and QuantGr, which uses low-precision arithmetic to reduce computational load, resulting in a $2.7\times$ performance gain.
% For GAT, EffOp yields a $3\times$ performance boost. When we incorporate approximation, the improvement jumps to $7.6\times$, with almost no degradation in quality.
% For SAGE-max, EffOp yields a $2\times$ performance boost. When we incorporate approximation (GrAx3), the improvement jumps to $3.2\times$, with no degradation in quality.
% Fig.~\ref{plot:mtl_vs_lnl} compares GCN performance across different GraNNite optimization techniques on NPUs of two Intel AI PCs, meteor lake and lunar lake. We observe that Lunar Lake consistently delivers higher performance as it has higher number of tiles (4) as compared to meteor lake (1). For the most optimized version (GrAd + NodePad + QuantGr), lunar lake archives $1.7\times$ ($1.6\times$) higher throughput than meteor lake for Cora (Citeseer) dataset. The presence of higher number of MAC units in lunar lake enables higher data parallelism leading to better performance. Although the performance gain is not equal to the theoretical maximum (4X) due to the limited data parallelism in the GCN model.
% Fig.~\ref{plot:cpu_gpu_npu} compares the performance of CPU (blue), GPU (orange), and NPU (green) across three GNN layer types: GraphConv (GCN), GraphAttn (GAT), and SAGE (GraphSAGE). For GCN, the NPU achieves a remarkable $17.3\times$ speedup over the GPU and $4.6\times$ over the CPU, showcasing its efficiency in handling these workloads. Similarly, the NPU demonstrates $2.3\times$ and $3.8\times$ improvements over GPU and CPU, respectively, for GAT layers, and achieves $6.7\times$ and $10.8\times$ speedups for GraphSAGE with mean aggregation. These results underscore the NPU's computational advantages and the effectiveness of GraNNite's optimizations, enabling high-performance GNN execution on existing hardware without modifications.
% Fig.~\ref{plot:energy_gcn} demonstrates the need for mapping the GNN models on NPU for energy efficiency. We observe that NPU is $4.1\times$ ($4.4\times$) energy efficient than the most energy efficient GPU implementation for Cora (Citeseer) dataset. Similarly, NPU is $8.5\times$ ($8.6\times$) energy efficient than the most energy efficient CPU implementation for Cora (Citeseer) dataset. 
% It is to be noted that, we could not demonstrate the impact of SymG and CacheG as those would require changes in the NPU compiler which is not made open source.

% Version #1
% Fig.~\ref{plot:gnn_progression}(a) shows the performance progression of Graph Convolutional Networks (GCN) on the Intel Lunar Lake NPU, highlighting significant improvements from a series of targeted optimizations. Here, the unoptimized implementation serves as the reference baseline.
% The first optimization, Optimized Graph Partitioning (OGP), enhances data locality by efficiently distributing the workload between the CPU and NPU, achieving a $1.85\times$ performance boost over the baseline. Adding Node Padding (NP) allows handling time-varying graphs and ensures efficient parallelism, though it slightly reduces performance by $1.1\times$ due to the additional processing overhead on the CPU. The most substantial improvement comes from combining OGP, NP, and Quantization, which uses low-precision arithmetic to reduce computational load, resulting in a $2.7\times$ performance gain.

% Fig.~\ref{plot:gnn_progression}(b) demonstrates the performance improvements of Graph Attention Network (GAT) implementations on an Intel NPU, achieving a $7.6\times$ speedup over the baseline.
% The first optimization replaces the ``Select" operation with element-wise multiplication, yielding a $3\times$ performance boost by simplifying the computation. Next, the element-wise multiplication is offloaded to the DPU, providing an additional $3.5\times$ performance gain by focusing computation on the DPU. Finally, eliminating the broadcast addition operation, which causes memory overhead, results in a substantial performance improvement, reaching the $7.6\times$ speedup.

% Fig.~\ref{plot:gnn_progression}(c) showcases the performance gains of a SAGE model with the max aggregation scheme, achieving up to $3.2\times$ speedup over the baseline.
% The first optimization replaces the complex ``Select" operation with a more efficient element-wise multiplication, boosting performance to $2\times$ the baseline. The second optimization swaps the ``ReduceMax" operation for ``MaxPool1D," aligning better with hardware architecture and providing an additional performance increase, reaching the final $3.2\times$ speedup.

% Fig.~\ref{plot:cpu_gpu_npu} compares the performance of CPU (blue), GPU (orange), and NPU (green) across three GNN layer types: GraphConv (GCN), GraphAttn (GAT), and SAGE (GraphSAGE). For GCN, the NPU achieves a remarkable $17.3\times$ speedup over the GPU and $4.6\times$ over the CPU, showcasing its efficiency in handling these workloads. Similarly, the NPU demonstrates $2.3\times$ and $3.8\times$ improvements over GPU and CPU, respectively, for GAT layers, and achieves $6.7\times$ and $10.8\times$ speedups for GraphSAGE with mean aggregation. These results underscore the NPU's computational advantages and the effectiveness of GraNNite's optimizations, enabling high-performance GNN execution on existing hardware without modifications.

% Version #0
% These optimizations demonstrate how integrating algorithmic improvements, memory management, and hardware-friendly approaches unlocks the full performance potential of GCNs on NPUs.

% Fig.~\ref{plot:gcn_progression} illustrates the performance progression of Graph Convolutional Network (GCN) implementations on an Intel Lunar Lake NPU, demonstrating significant enhancements achieved through a series of targeted optimizations. The baseline unoptimized implementation is set as the reference point, representing the lowest performance.
% The first optimization, Optimized Graph Partitioning (OGP), focuses on improving data locality by effectively distributing the workload between the CPU and NPU for a static input graph. This optimization results in a notable performance boost of approximately 1.85X over the baseline.
% Next, the addition of Node Padding (NP) to the OGP approach enables the model to handle time-varying input graphs. This ensures efficient parallelism across compute units, although it slightly reduces performance by about 1.1X compared to OGP alone. This decrease is attributed to the extra processing time required for the normalization matrix on the CPU.
% The most significant performance improvement is observed with the combination of OGP, NP, and Quantization. By employing low-precision arithmetic, this approach reduces the overall computational workload, leading to a remarkable 2.7X enhancement over the initial implementation.
% The consistent increase in performance across these optimization stages underscores the value of integrating algorithmic optimizations like OGP with memory management techniques (NP) and hardware-friendly approaches (quantization). This cumulative application of optimizations highlights that while each individual optimization is beneficial, their combined effect is essential for unlocking the full performance potential of GCNs on NPUs.


% \begin{figure}[t!]
% \begin{center}
% \includegraphics[width=\columnwidth]{Plots/GCN_progression.png}% This is a *.eps file
% \end{center}
% \caption{Progressive performance improvement of GCN through different optimizations}\label{plot:gcn_progression}
% \end{figure}


% These optimizations highlight the importance of reducing unnecessary memory operations and offloading tasks to specialized cores, significantly improving inference latency and efficiency for GAT models on NPUs in resource-constrained environments.


% Fig.~\ref{plot:gat_progression} showcases the performance improvements of Graph Attention Network (GAT) implementations on an Intel NPU, illustrating how a series of optimizations culminate in a substantial 7.6X speedup over the baseline implementation. The baseline serves as the starting point and represents the lowest performance due to the computational inefficiencies inherent in certain operations typically used in GAT models.
% The first optimization involves replacing the "Select" operation with element-wise multiplication, which is a simpler and more parallelizable operation. This initial change yields an impressive improvement of approximately 3X over the baseline performance, highlighting the benefits of simplifying computational tasks.
% In the second stage of optimization, the element-wise multiplication operation is further refined; instead of performing the multiplication operation alongside other computations, it is exclusively executed on the DPU. This focused approach results in a cumulative performance boost of around 3.5X relative to the original implementation, indicating that optimizing where and how computations are performed is critical for enhancing performance.
% The final optimization addresses the broadcast addition operation, which often incurs significant memory overhead by duplicating data across tensors. By eliminating this redundant operation, the GAT implementation experiences a substantial performance enhancement, achieving a maximum of 7.6X speedup over the baseline. 
% This progressive enhancement illustrates the crucial role of reducing unnecessary memory operations and leveraging specialized processing cores for performance-critical tasks. The results emphasize that architectural-aware optimizations—such as offloading specific workloads from the DSP to DPU cores and eliminating redundant operations through approximations—can lead to significant improvements in inference latency for GAT models on NPUs. Such strategies not only optimize computational efficiency but also facilitate faster and more effective execution of GNNs in resource-constrained environments.


% \begin{figure}[t!]
% \begin{center}
% \includegraphics[width=\columnwidth]{Plots/GAT_progression.png}% This is a *.eps file
% \end{center}
% \caption{Progressive performance improvement of GAT through different optimizations}\label{plot:gat_progression}
% \end{figure}


% This progression demonstrates the value of targeted optimizations in reducing computational overhead, enhancing data-parallel processing, and maximizing performance for GNN models on specialized hardware.

% Fig.~\ref{plot:sage_progression} demonstrates the performance improvements of a SAGE model with the max aggregation scheme following a series of targeted optimizations, ultimately achieving a cumulative speedup of up to 3.2X compared to the baseline. The baseline reflects the initial performance prior to any optimizations, serving as a reference for evaluating the impact of each subsequent modification.
% The first optimization involves substituting the "Select" operation—known for its control-flow complexity—with a data-parallel element-wise multiplication. This shift to a computationally more efficient operation delivers a substantial boost, bringing the performance to approximately 2X of the baseline. This optimization illustrates how replacing control-flow-heavy operations with data-parallel alternatives can enhance computational efficiency.
% Building upon this, a second optimization replaces the "ReduceMax" operation with "MaxPool1D," a more streamlined operation that aligns better with the hardware's architecture. This adjustment leads to an additional performance increase, as depicted by the green bar on the right, resulting in a total improvement of 3.2X over the baseline configuration.
% Overall, this progression highlights the impact of carefully selected optimizations in reducing computational overhead, enhancing data-parallel processing, and improving model efficiency. These results underscore the effectiveness of architectural-aware optimizations in maximizing performance for GNN models on specialized hardware.


% \begin{figure}[t!]
% \begin{center}
% \includegraphics[width=\columnwidth]{Plots/SAGE_progression.png}% This is a *.eps file
% \end{center}
% \caption{Progressive performance improvement of SAGE-max through different optimizations}\label{plot:sage_progression}
% \end{figure}



% \subsection{CPU, GPU \& NPU performance per watt for GCN, GAT and GraphSAGE}
% Figure~\ref{plot:power} presents the power consumption breakdown of various components in different operational states of an AI PC, including IDLE and during the execution of GNN models on different devices. The x-axis shows the specific GNN models in use and the devices they are mapped to, allowing for a comparison of power usage across distinct deployment scenarios.
% The first bar on the left represents the system’s IDLE state, where no workload is running on any device. This IDLE power breakdown provides a baseline to compare against the power demands when GNN models are actively running on various devices within the AI PC.
% Moving beyond IDLE, the figure details the power distribution among key system components—IA cores, System Agent, GT, and DRAM—when GNN models are executed, especially highlighting the benefits of NPU deployment. When a model runs on the NPU, the System Agent’s power consumption, shown in blue, increases due to its role in managing the NPU, which draws from the System Agent’s power rail. However, despite this rise in the System Agent’s power draw, the total power usage across all components (including IA cores, GT, and DRAM) remains notably low when models are mapped to the NPU.
% This low cumulative power usage, paired with the NPU’s high processing efficiency (as demonstrated in previous figures), results in excellent performance per watt. Such efficiency makes the NPU highly suitable for applications that demand both high performance and low energy consumption. Specifically, the NPU’s ability to efficiently handle GNN workloads with minimal power draw makes it well-suited for high-performance tasks in power-sensitive settings. In summary, Figure~\ref{plot:power} underscores how the NPU’s balanced approach to speed and power usage makes it a compelling option for deploying GNN models in resource-constrained environments.

% \begin{figure}[t!]
% \begin{center}
% \includegraphics[width=\columnwidth]{Plots/Power.png}% This is a *.eps file
% \end{center}
% \caption{Power consumption of different GNN models on Intel AI PC: NPU takes lower power and compute with a higher speed}\label{plot:power}
% \end{figure}



% Fig.~\ref{plot:cpu_gpu_npu} presents a performance comparison among CPU (blue), GPU (orange), and NPU (green) in executing three types of GNN layers: GraphConv (GCN), GraphAttn (GAT), and SAGE (GraphSAGE). For the GraphConv (GCN), the NPU achieves an impressive 17.3× speedup compared to the GPU and a 4.6× speedup over the CPU. This result highlights the NPU's significant efficiency in managing GCN workloads.
% In the case of the GraphAttn (GAT), the NPU demonstrates a performance improvement of 2.3× over the GPU and 3.8× over the CPU. Likewise, for the SAGE (GraphSAGE) using the mean aggregator scheme, the NPU outperforms the GPU by 6.7× and the CPU by 10.8×. These results clearly indicate the superior computational capabilities of NPUs and efficacy of GraNNite proposed optimizations, particularly when applied to our most optimized GNN layers. The consistent performance advantage of NPUs over traditional architectures like CPUs and GPUs across these benchmarks suggests that existing NPUs can effectively implement GNNs using the proposed optimizations, without necessitating any changes to the underlying hardware.



\begin{figure}[t!]
\begin{center}
\includegraphics[width=\columnwidth]{Plots/Energy_GCN.pdf}% This is a *.eps file
\end{center}
\caption{Normalized Energy Consumption of GCN on Intel\textregistered\ Core\texttrademark\ Ultra Series 2 Devices (CPU, GPU, and NPU), highlighting significant energy savings achieved with GraNNite optimizations.}\label{plot:energy_gcn}
\end{figure}
\section{Result Analysis}

In this section, we analyse the results and predictions of our method from the perspective of the mobile app ecosystem. First, we observe how deviated the text and visual data are compared to natural language and text. Next, we present interesting findings among the lower and upper triangular parts of the confusion matrix (i.e., apps having a lower or higher content rating than our predictions). 


\subsection{Image-text Cross Attention}
\begin{figure}[ht]  
    \centering
    \includegraphics[width=0.97\linewidth]{figures/vis_ca_module.pdf}
    \caption{Visualisation of image patches attending to text tokens in the custom cross-attention layer.}
    \label{fig:ca_visualized}
\end{figure}

In Sec.~\ref{ssec:cross attention}, we discussed how our image-text cross attention (CA) design allows every patch of the content image to attend over all tokens in the text input sequence. Also, for each image patch, there are 12 attention heads running in parallel. Therefore, to qualitatively measure how the CA happens, we select attention-heads of the first layer as lower layers are often associated with broader attention~\cite{vig_2019_multiscale}. Then, for each image patch, we select the tokens with the highest numerical attention values after excluding some stop-word and punctuation mark-related tokens. In Fig.~\ref{fig:ca_visualized}, we visualise the highly attended words in a given input image portion for some example images. An image portion is a collection of consecutive patches which we select as a region of interest; for example, the patches outlining the gun in Fig.~\ref{fig:ca_visualized} (d) or the heart in Fig.~\ref{fig:ca_visualized} (a). The results show that mobile app ecosystem-related tokens such as `simulator', `game', `upgrade', and `developers' are now attended by the image patches, even though those words cannot be identified by observing the image in a general context. Furthermore, the tokens representing target audiences such as `kids', `children' and `girls' are now attended. This further demonstrates how our model has been able to reduce the gap between visual data and textual data in the mobile app domain.

\subsection{Predictions in the Wild}

When our classifier is applied to the test set containing 10,000 gaming apps, two interesting cases emerge, i.e., apps with a higher or lower labelled rating than our model's predictions. These two scenarios lead to potential  \emph{``malpractices"} and \emph{``disguises"} that are important from a content safety point of view. 

\subsubsection{Potential malpractices}
\label{subsec: potential malpractices}

When our method predicts a class label that is higher than the developer-defined label, we characterise such examples as possible malpractices (i.e., having a lower content rating than what the app is supposed to have). Occurrences belonging to this category could be identified along the horizontal axis of the confusion matrix, on the right-hand side to the diagonal. This is more observed in the G category as any higher prediction, such as PG, M, MA15+, and R18+, raises a concern. As we go higher in prediction classes, the possibilities for malpractices decrease, and therefore, the R18+ category is not susceptible to malpractices. 

We manually evaluated 350 apps with predicted labels that were two classes or more higher than the true label (e.g., prediction [M, MA15+ or R18+] when the true label is [G]) and identified 62 (17.7\%) of them as potential malpractices (i.e., possible content rating violations). \textit{Also, we highlight that at the time of writing, 14 of them were removed from Play Store, and 20 of them were increased to a higher rating class compared to the time we crawled the dataset. While we can't be exactly sure why these 14 apps were removed from  Google Play Store, previous work has reported that Google take down apps violating their content policies~\cite{seneviratne2015early}.} 

We highlight ten exemplary instances that are potentially linked to malpractices in Fig.~\ref{fig:malpractices} (1). Examples (a) and (b) both represent gambling-related games rated [G] that would at least require a rating of [PG]. Example (c) depicts shooting and gun usage, again not suitable for a general audience. (e) and (j) contain images more suited for an [MA15+] audience, and (d) and (f) entertain visual and textual cues strongly related to illegal substances that require a rating higher than [PG]. (g) and (h) contain images with horror themes, blood and intense cartoon or fantasy violence which are more suited for [MA15+] audience. These examples suggest that our model can flag potential content malpractices in the app ecosystem.


\subsubsection{Potential disguises}
\label{subsec: potential disguises}

Apps belonging to higher content ratings are likely not to conduct malpractices, but alarmingly, they could be disguised as a lower-rated app and, as a consequence, could attract an unsuitable audience to the app.  
As an example, an R18+ app consisting of cartoonish images and not-so-alarming textual data could attract underage audiences due to their natural tendency for curiosity and their interest in cartoons. From the developer's perspective, they comply with the content policies and applicable laws. However, in such cases, we argue that at least the textual description must contain information such that someone (e.g., a parent or a guardian) who misses the content rating label should be able to figure out the app's purpose and functionality independently. We define this category as possible disguises. R18+ is more vulnerable to disguises that can be identified horizontally left to the diagonal of the confusion matrix. Being the lowest content rating, category [G] is not susceptible to disguises. We checked 131 samples that were predicted [G, PG, M] when the true class was [R18+], and 203 samples predicted [G] when the true class was [MA15+] and observed $\sim$9.3\% of them to be possible disguises.

We demonstrate some of such examples in Fig.~\ref{fig:malpractices}(2). App represented by (a) is an app rated for [PG] and our method predicts even lower as [G]. Observing the images, it is evident their content is sexual in nature despite the cartoonish theme. In our test dataset, the average rating for an app with such sexualised imagery is [MA15+]. 
Note that in this category, our model is likely to predict a lower rating as the images or text is not suggestive of requiring high ratings. Due to such examples being rare, our model does not know how to predict them correctly, yet we still can automatically identify them based on our off-to-left-diagonal results, as explained before. Apps (b, e, g) and (i), though rated for [MA15+] and higher based on a storyline related to a `fashion and makeup', is likely to attract a younger audience due to the visual appearance similar to the majority of [PG] rated \emph{Dress up} games. A similar interpretation can be given to [M] rated examples (c, h), which are too cartoonish yet contain images related to mature audiences, such as pregnancy. Furthermore, (f) and (j) are rated as [MA15+], which contains appealing games for young audience. Despite measures such as \emph{Not designed for children} tagging is available in Google Play~\cite{notdisgisechil.} to safeguard users, it doesn't appear to be used by many developers. 

\subsubsection{Unverifiable apps} 
\label{subsec: unverifiable_apps}

\begin{figure}[ht]  
    \centering
    \includegraphics[width=0.97\linewidth]{figures/unverifiables.pdf}
    \caption{Examples of unverifiable apps with developer defined content rating descriptors.}
    \label{fig:unverifiable_apps}
\end{figure}
We further highlight discrepancies in content rating declarations, as illustrated in Fig.~\ref{fig:unverifiable_apps}. These noisy instances represent cases where we could not manually verify the developer-assigned ratings based on the app’s available metadata, visuals, or descriptions. The app in (a) is rated as [M] with descriptors indicating simulated gambling, online interactivity, and in-game purchases. However, compared to similar games in the dataset, it does not display any visual cues or textual descriptions to justify such a rating. Therefore, the absence of gambling graphics or explicit mention of gambling mechanics justifies the predicted rating of [G].
Similarly, app (b) and (c) fail to reflect sensitive content such as sexualised imagery and strong violence in app creatives. Hence, in all these cases, our model predicted a lower rating than the original rating. While it may not be explicitly illegal to omit sensitive descriptors from app screenshots or descriptions, missing descriptors in visuals hinder the user's ability to make informed decisions, leading to unwanted downloads or unexpected experiences and children and vulnerable users may unintentionally be exposed to harmful or age-inappropriate content.


\subsubsection{Teacher Approved (TA) Apps}

\begin{figure}[ht]  
    \centering
    \includegraphics[width=0.97\linewidth]{figures/teacher_approved_v2.pdf}
    \caption{Examples of teacher-approved apps with incorrect content ratings - A section of the app description is quoted and * indicates this app is no longer available.}
    \label{fig:teacher_apps}
\end{figure}


Google Play has deployed the `teacher approved' apps based on consultations with experts to determine the suitability for kids, and especially the age appropriateness~\cite{teacherAApps}. Hence, these apps are likely to be more content appropriate as per the rating labelling. However, after analysing 2,172 TA apps, we found that our model predictions deviate from the declared content rating classifications, with 10.4\% of them being flagged for potential malpractices. Among them, 92\% of the instances were flagged as requiring [PG] despite being declared as [G]. Further evaluating their continuity, within a time span of nine months, we observed that 34.5\% of apps that were classified as potential malpractices have been removed from the Play Store. On the contrary, only 27.4\% of correctly predicted apps were removed, which is lower compared to the deletion rate of apps identified with malpractices.
We further discuss why app removal rate can be a proxy measure of content policy violations in the next subsection. 


As depicted in Fig.~\ref{fig:teacher_apps} we manually verified a portion of apps flagged before as malpractices based on the available metadata and identified that nine apps are likely to be not suitable for children despite being tagged as TA. 
The presence of violence (Fig.~\ref{fig:teacher_apps}a), horror themes (Fig.~\ref{fig:teacher_apps}b) or online multiplayer interactions (Fig.~\ref{fig:teacher_apps}d) were the main reasons we identified behind these content rating discrepancies. 
The example in Fig.~\ref{fig:teacher_apps} c highlights a contradiction in the app's description, which mandates parental presence, despite the app being labeled as [G] in the Play Store.
\textit{Overall, the presence of these practices among `teacher approved' apps is alarming. It shows that even manually verified apps are not immune to content rating malpractices, and further rigour is required in app vetting.}


\begin{figure}[ht]    
    \centering
    \includegraphics[width=0.97\linewidth]{figures/app_deletion_rate_.pdf}
    \caption{App deletion rates w.r.t number of downloads.}
    \label{fig:app_deletion}
   % \vspace{-0.3cm}
\end{figure}


\subsubsection{App Deletion Rate}

An app can be discontinued in Google Play for two reasons: the developer could discontinue the app~\cite{denipitiyage2024detecting}, or Google could remove the app for violating its policies~\cite{2023gplypolicyviolation}. As a result, an app being removed from Google Play can be used as an indication of a possible violation of Google Play policies.

To this end, we used 15,985 apps that are gathered from the test set (10,000 apps: c.f. Sec.~\ref{Sec:dataset}) added with 5,985 apps with lower downloads (download count $<$ 100,000 - to account for a better distribution as our test set consists of top apps only). Next, we attempted to re-crawl these apps to check whether they were still there in Google Play. Overall, we found 45.7\% of apps identified as having malpractices, 39.1\% of apps that predicted to be disguises were removed within the time span of nine months. In comparison only  29.1\% of apps that correctly predicted were removed.


In Fig.~\ref{fig:app_deletion}, we show the percentage of apps that we found as deleted according to the download numbers and the predictions of our classifier. At all download ranges apart from `$<$ 100', we notice that apps we classified as potential malpractices have a higher deletion rate than apps we classified as correct. Similar values of `$<$~100' category can be explained by less attention and consequently fewer complaints received on those apps for Google to action.

On the other hand, apps classified as disguises are likely not to be removed as malpractices, as they are unlikely to be noticed or complained about by an average audience. Notably, apps flagged as disguises with more than 1M downloads are far less likely to be removed as the number of apps with a higher content rating label (e.g., MA15+, R18+) are not frequent among the apps. 



\section{Conclusion}

We presented \sys, a sparsity-adaptive attention mechanism for efficient long-context LLM inference. Unlike fixed token budget methods, \sys dynamically selects tokens based on cumulative attention scores, adapting to variations in attention sparsity. By leveraging clustering-based sorting and distribution fitting, \sys accurately estimates token importance with low overhead. Our results showed that \sys outperforms existing sparse attention methods, achieving higher accuracy and significant inference speedups, making it a practical solution for long-context LLMs.

\section{Acknowledgment}
This research was supported by the Australian Government through the Australian Research Council’s Discovery Projects funding scheme (Project ID DP220102520).

% \newpage
\bibliographystyle{IEEEtran}
\bibliography{biblio}

\begin{IEEEbiography}[{\includegraphics[width=1in,height=1.25in,clip,keepaspectratio]{bio_pics/dewni.jpeg}}]{Dishanika Denipitiyage}
	received her Bachelors degree in Electronic and Telecommunication Engineering from University of Moratuwa, Sri Lanka in 2020. She is currently working toward the PhD degree with the School of Computer Science, University of Sydney, Australia. She worked as a Senior software engineear at SenzMate (Pvt) Ltd, Sri Lanka in 2022 and as a Visiting Research Intern at Singapore University of Technology and Design (SUTD), Singapore in 2017. Her research interests include Self-Supervised Learning and Multi-Modal learning.
\end{IEEEbiography}

\vskip -3\baselineskip plus -1fil
\vspace{8mm}

\begin{IEEEbiography}[{\includegraphics[width=1in,height=1.5in,clip,keepaspectratio]{bio_pics/bhanuks.jpeg}}]{Bhanuka Silva}
received his Bachelors degree in Electronic and Telecommunication Engineering (First Class Hons.) from University of Moratuwa, Sri Lanka in 2020. He also worked as a Visiting Research Intern at Data61-CSIRO, Brisbane in 2018 and is currently a doctoral student at the University of Sydney and his current research focuses on conducting privacy compliance checks in mobile app eco-systems by leveraging state-of-the art natural language processing techniques.
\end{IEEEbiography}

\vskip -3\baselineskip plus -1fil
\vspace{8mm}

\begin{IEEEbiography}[{\includegraphics[width=1in,height=1.25in,clip,keepaspectratio]{bio_pics/kavishka.jpeg}}]{Kavishka Gunathilaka}
is a B.Sc. Eng. undergraduate in the Department of Computer Science and Engineering of the University of Moratuwa, Sri Lanka. He also worked as a Research Affiliate at the University of Sydney in 2023. His primary research interests include machine learning, data 
science, and artificial intelligence.
\end{IEEEbiography}

\vskip -3\baselineskip plus -1fil
\vspace{8mm}

\begin{IEEEbiography}[{\includegraphics[width=1in,height=1.25in,clip,keepaspectratio]{bio_pics/ssenevirathne.png}}]{Suranga Seneviratne}
is a Senior Lecturer in Security at the School of Computer Science, The University of Sydney. He received his Ph.D. from the University of New South Wales, Australia in 2015. His current research interests include privacy and security in mobile systems, AI applications in security, and behavior biometrics. Before moving into research, he worked nearly six years in the telecommunications industry in core network planning and operations. He received his bachelor degree from University of Moratuwa, Sri Lanka in 2005.
\end{IEEEbiography}
%\vspace{-4.2cm}

\vskip -2\baselineskip plus -1fil
\vspace{8mm}

\begin{IEEEbiography}[{\includegraphics[width=1in,height=1.25in,clip,keepaspectratio]{bio_pics/amahanti.jpeg}}]{Anirban Mahanti}'s technical expertise is at the intersection of computer networking, data science, and machine learning. Anirban earned his Ph.D. in Computer Science from the University of Saskatchewan, Canada, in 2003, his MSc in Computer Science in 1999, and his BE in Computer Science and Engineering from the Birla Institute of Technology, India, in 1993. He is currently an Honorary Senior Research Fellow at the University of Sydney.
\end{IEEEbiography}
\vskip -2\baselineskip plus -1fil
\begin{IEEEbiography}[{\includegraphics[width=1in,height=1.25in,clip,keepaspectratio]{bio_pics/aruna.jpeg}}]{Aruna Seneviratne}
(Senior Member, IEEE) is currently a foundation professor of telecommunications with the University of New South Wales, Sydney, Australia, where he holds the Mahanakorn chair of telecommunications. He was with a number of other universities in Australia, UK, and France, as well as industrial organizations, including Muirhead, Standard Telecommunication Labs, Avaya Labs, and Telecom Australia (Telstra). He held visiting appointments with INRIA, France. His research inter- ests include physical analytics, technologies that enable applications to interact intelligently and securely with their environment in real time. Recently, his team has been working on using these technologies in behavioural biometrics, optimizing the performance of wearables, and IoT system verification. He was the recipient of several fellowships, including one at British Telecom and one at Telecom Australia Research Labs.
\end{IEEEbiography}

\vskip -2\baselineskip plus -1fil
\vspace{8mm}

\begin{IEEEbiography}[{\includegraphics[width=1in,height=1.25in,clip,keepaspectratio]{bio_pics/schawla.png}}]{Sanjay Chawla}
is Research Director of QCRI’s Data Analytics department. Prior to joining QCRI, Dr. Chawla was a Professor in the Faculty of Engineering and IT at the University of Sydney. From 2008-2011, he also served as the Head (Department Chair) of the School of Information Technologies. He was an academic visitor at Yahoo! Research in 2012. He received his PhD from the University of Tennessee (USA) in 1995. His research is in data mining and machine learning with a specialization in spatio-temporal data mining, outlier detection, class imbalanced classification, and adversarial learning. 
\end{IEEEbiography}

\end{document}

