% This must be in the first 5 lines to tell arXiv to use pdfLaTeX, which is strongly recommended.
\pdfoutput=1
% In particular, the hyperref package requires pdfLaTeX in order to break URLs across lines.

\documentclass[11pt]{article}

% Change "review" to "final" to generate the final (sometimes called camera-ready) version.
% Change to "preprint" to generate a non-anonymous version with page numbers.
\usepackage[preprint]{acl}
\usepackage{amsmath}
\usepackage{amssymb}
% Standard package includes
\usepackage{times}
\usepackage{latexsym}
\usepackage{booktabs} 
\usepackage{hyperref}
\usepackage{arydshln}
\usepackage{fixltx2e}
\usepackage{multirow}
% \usepackage{algorithm}
% \usepackage{algorithmic}
\usepackage[linesnumbered, ruled, vlined]{algorithm2e}
% \usepackage{algpseudocode}
% For proper rendering and hyphenation of words containing Latin characters (including in bib files)
\usepackage[T1]{fontenc}
% For Vietnamese characters
% \usepackage[T5]{fontenc}
% See https://www.latex-project.org/help/documentation/encguide.pdf for other character sets

% This assumes your files are encoded as UTF8
\usepackage[utf8]{inputenc}

% This is not strictly necessary, and may be commented out,
% but it will improve the layout of the manuscript,
% and will typically save some space.
\usepackage{microtype}

% This is also not strictly necessary, and may be commented out.
% However, it will improve the aesthetics of text in
% the typewriter font.
\usepackage{inconsolata}

%Including images in your LaTeX document requires adding
%additional package(s)
\usepackage{graphicx}

% If the title and author information does not fit in the area allocated, uncomment the following
%
%\setlength\titlebox{<dim>}
%
% and set <dim> to something 5cm or larger.

\title{SyncSpeech: Low-Latency and Efficient Dual-Stream Text-to-Speech based on Temporal Masked Transformer}

% Author information can be set in various styles:
% For several authors from the same institution:
% \author{Author 1 \and ... \and Author n \\
%         Address line \\ ... \\ Address line}
% if the names do not fit well on one line use
%         Author 1 \\ {\bf Author 2} \\ ... \\ {\bf Author n} \\
% For authors from different institutions:
% \author{Author 1 \\ Address line \\  ... \\ Address line
%         \And  ... \And
%         Author n \\ Address line \\ ... \\ Address line}
% To start a separate ``row'' of authors use \AND, as in
% \author{Author 1 \\ Address line \\  ... \\ Address line
%         \AND
%         Author 2 \\ Address line \\ ... \\ Address line \And
%         Author 3 \\ Address line \\ ... \\ Address line}

% \author{Zhengyan Sheng \\
%   Affiliation / Address line 1 \\
%   Affiliation / Address line 2 \\
%   Affiliation / Address line 3 \\
%   \texttt{email@domain} \\\And
%   Second Author \\
%   Affiliation / Address line 1 \\
%   Affiliation / Address line 2 \\
%   Affiliation / Address line 3 \\
%   \texttt{email@domain} \\}

\author{
 Zhengyan Sheng\textsuperscript{1}\thanks{Work done during internship at Alibaba Group.},
 Zhihao Du\textsuperscript{2},
 Shiliang Zhang\textsuperscript{2},
 Zhijie Yan\textsuperscript{2}, \\
 \textbf{Yexin Yang\textsuperscript{2}},
 \textbf{Zhenhua Ling\textsuperscript{1}}\thanks{Corresponding author}
 \\
%  \textbf{Fifth Author\textsuperscript{1,2}},
%  \textbf{Sixth Author\textsuperscript{1}},
%  \textbf{Seventh Author\textsuperscript{1}},
%  \textbf{Eighth Author \textsuperscript{1,2,3,4}},
% \\
%  \textbf{Ninth Author\textsuperscript{1}},
%  \textbf{Tenth Author\textsuperscript{1}},
%  \textbf{Eleventh E. Author\textsuperscript{1,2,3,4,5}},
%  \textbf{Twelfth Author\textsuperscript{1}},
% \\
%  \textbf{Thirteenth Author\textsuperscript{3}},
%  \textbf{Fourteenth F. Author\textsuperscript{2,4}},
%  \textbf{Fifteenth Author\textsuperscript{1}},
%  \textbf{Sixteenth Author\textsuperscript{1}},
% \\
%  \textbf{Seventeenth S. Author\textsuperscript{4,5}},
%  \textbf{Eighteenth Author\textsuperscript{3,4}},
%  \textbf{Nineteenth N. Author\textsuperscript{2,5}},
%  \textbf{Twentieth Author\textsuperscript{1}}
% \\
\\
 \textsuperscript{1}University of Science and Technology of China \\
 \textsuperscript{2}Speech Lab, Alibaba Group, China,
 % \textsuperscript{3}Affiliation 3,
 % \textsuperscript{4}Affiliation 4,
 % \textsuperscript{5}Affiliation 5
\\
 {
     \href{zysheng@mail.ustc.edu.cn}{zysheng@mail.ustc.edu.cn}, 
     \href{neo.dzh@alibaba-inc.com}{\{neo.dzh, sly.zsl\}@alibaba-inc.com},
     \href{zhling@ustc.edu.cn}{zhling@ustc.edu.cn}
 }
}

% 本文提出了一个双流模型,SyncSpeech, 能够接受上游大模型流式的文本输入同时进行流式的语音生成,和大语言模型可以无缝交互。SyncSpeech有以下特点(1)低延时,即接受到第二个文本token时开始流式的语音生成;(2)高效,每一个解码生成对应文本token的全部speech token。为此,我们提出temporl masked transformer 作为 SyncSpeech的Backbone,并结合token-level的时长预测,在一步解码的过程中同时预测speech token和下一步的时长。此外,我们也设计了两阶段的训练策略来提升训练效率和生成语音的质量。我们在英文和普通话两种数据集上进行了验证,相比双流的sota tts模型,分别降低了speech token3.8和5.5倍的首包延迟,加速了6.4和8.6倍的的实时比。同时,在相同的数据规模下,也取得了相当的性能。
\begin{document}
\maketitle
\begin{abstract}
This paper presents a dual-stream text-to-speech (TTS) model, SyncSpeech, capable of receiving streaming text input from upstream models while simultaneously generating streaming speech, facilitating seamless interaction with large language models. SyncSpeech has the following advantages: Low latency, as it begins generating streaming speech upon receiving the second text token; High efficiency, as it decodes all speech tokens corresponding to the each arrived text token in one step. To achieve this, we propose a temporal masked transformer as the backbone of SyncSpeech, combined with token-level duration prediction to predict speech tokens and the duration for the next step. Additionally, we design a two-stage training strategy to improve training efficiency and the quality of generated speech. We evaluated the SyncSpeech on both English and Mandarin datasets. Compared to the recent dual-stream TTS models, SyncSpeech significantly reduces the first packet delay of speech tokens and accelerates the real-time factor. Moreover, with the same data scale, SyncSpeech achieves performance comparable to that of traditional autoregressive-based TTS models in terms of both speech quality and robustness. Speech samples are available at \href{https://SyncSpeech.github.io/}{https://SyncSpeech.github.io/}.

\end{abstract}

\section{Introduction}

In recent years, with advancements in generative models and the expansion of training datasets, text-to-speech (TTS) models \cite{valle, voicebox, ns3} have made breakthrough progress in naturalness and quality, gradually approaching the level of real recordings. However, low-latency and efficient dual-stream TTS, which involves processing streaming text inputs while simultaneously generating speech in real time, remains a challenging problem \cite{livespeech2}. These models are ideal for integration with upstream tasks, such as large language models (LLMs) \cite{gpt4} and streaming translation models \cite{seamless}, which can generate text in a streaming manner. Addressing these challenges can improve live human-computer interaction, paving the way for various applications, such as speech-to-speech translation and personal voice assistants.

Recently, inspired by advances in image generation, denoising diffusion \cite{diffusion, score}, flow matching \cite{fm}, and masked generative models \cite{maskgit} have been introduced into non-autoregressive (NAR) TTS \cite{seedtts, F5tts, pflow, maskgct}, demonstrating impressive performance in offline inference.  During this process, these offline TTS models first add noise or apply masking guided by the predicted duration. Subsequently, context from the entire sentence is leveraged to perform temporally-unordered denoising or mask prediction for speech generation. However, this temporally-unordered process hinders their application to streaming speech generation\footnote{
Here, “temporally” refers to the physical time of audio samples, not the iteration step $t \in [0, 1]$ of the above NAR TTS models.}.


When it comes to streaming speech generation, autoregressive (AR) TTS models \cite{valle, ellav} hold a distinct advantage because of their ability to deliver outputs in a temporally-ordered manner. However, compared to recently proposed NAR TTS models,  AR TTS models have a distinct disadvantage in terms of generation efficiency \cite{MEDUSA}. Specifically, the autoregressive steps are tied to the frame rate of speech tokens, resulting in slower inference speeds.  
While advancements like VALL-E 2 \cite{valle2} have boosted generation efficiency through group code modeling, the challenge remains that the manually set group size is typically small, suggesting room for further improvements. In addition,  most current AR TTS models \cite{dualsteam1} cannot handle stream text input and they only begin streaming speech generation after receiving the complete text,  ignoring the latency caused by the streaming text input. The most closely related works to SyncSpeech are CosyVoice2 \cite{cosyvoice2.0} and IST-LM \cite{yang2024interleaved}, both of which employ interleaved speech-text modeling to accommodate dual-stream scenarios. However, their autoregressive process generates only one speech token per step, leading to low efficiency.



To seamlessly integrate with  upstream LLMs and facilitate dual-stream speech synthesis, this paper introduces \textbf{SyncSpeech}, designed to keep the generation of streaming speech in synchronization with the incoming streaming text. SyncSpeech has the following advantages: 1) \textbf{low latency}, which means it begins generating speech in a streaming manner as soon as the second text token is received,
and
2) \textbf{high efficiency}, 
which means for each arriving text token, only one decoding step is required to generate all the corresponding speech tokens.

SyncSpeech is based on the proposed \textbf{T}emporal \textbf{M}asked generative \textbf{T}ransformer (TMT).
During inference, SyncSpeech adopts the Byte Pair Encoding (BPE) token-level duration prediction, which can access the previously generated speech tokens and performs top-k sampling. 
Subsequently, mask padding and greedy sampling are carried out based on  the duration prediction from the previous step. 

Moreover, sequence input is meticulously constructed to incorporate duration prediction and mask prediction into a single decoding step.
During the training process, we adopt a two-stage training strategy to improve training efficiency and model performance. First, high-efficiency masked pretraining is employed to establish a rough alignment between text and speech tokens within the sequence, followed by fine-tuning the pre-trained model to align with the inference process.

Our experimental results demonstrate that, in terms of generation efficiency, SyncSpeech operates at 6.4 times the speed of the current dual-stream TTS model for English and at 8.5 times the speed for Mandarin. When integrated with LLMs, SyncSpeech achieves latency reductions of 3.2 and 3.8 times, respectively, compared to the current dual-stream TTS model for both languages.
Moreover, with the same scale of training data, SyncSpeech performs comparably to traditional AR models in terms of the quality of generated English speech. For Mandarin, SyncSpeech demonstrates superior quality and robustness compared to current dual-stream TTS models. This showcases the potential of  SyncSpeech as a foundational model to integrate with upstream LLMs.



\begin{figure*}
    \centering
    \includegraphics[width=1\linewidth]{SyncSpeech.pdf}
    \caption{An overview of the proposed SyncSpeech,  comprising a text tokenizer, 
 a speech tokenizer, a temporal masked generative transformer and a chunk-aware speech decoder. The figure shows that, with the random number $n=2$ and text look-ahead value  $q=1$, it estimates all speech tokens (from $s_8$ to $s_{12}$) corresponding to the text token $y_2$ and the duration ($l_3$) of the next text token $y_3$ in one decoding step. }
 % This means that speech token  correspond to the text token $y_3$ and duration token $l_{3}$ represents the duration of the speech token for $y_3$}
    \label{fig1}
\end{figure*}

\section{Related Work}

\subsection{Text-to-Speech}
Text-to-Speech, the transformation of text into audible signals understandable by humans, is pivotal for human-computer interaction. TTS systems can
be mainly divided into AR-based and NAR-based categories. For AR-based systems, VALL-E \cite{valle} predicts the first layer of acoustic tokens extracted by EnCodec \cite{encodec} using an AR codec language model, while a NAR model is used to predict the remaining layers. CosyVoice \cite{cosyvoice} employs an AR model to predict supervised semantic representations and combines flow matching to predict acoustic representations. AR-based TTS models, with their in-context learning capability, can generate natural, prosody-diverse speech in a streaming manner. However, AR-based TTS models exhibit shortcomings in generation efficiency. Besides the previously mentioned VALL-E 2 \cite{valle2}, MEDUSA \cite{MEDUSA} and VALL-E R \cite{Valler} introduce speculative decoding \cite{spdeco} and a codec-merging method, respectively, to accelerate autoregressive generation. Nonetheless, the efficiency gains achieved by these approaches remain limited, unable to perform synchronized decoding steps with text tokens.  


For NAR-based TTS models, most previous approaches require speech duration prediction conditioned on the input text, followed by upsampling the text representations to match the acoustic feature length before feeding them into the generation model. Following FastSpeech \cite{fastspeech2}, VoiceBox \cite{voicebox} and NaturalSpeech 2 \cite{ns2} predict phone-level durations using a regression-based approach. NaturalSpeech 3 \cite{ns3} adopts a discrete diffusion model, combining classification loss and duration prompts for duration prediction, which outperforms text-dependent regression-based duration prediction in terms of speech robustness and quality. However, NaturalSpeech 3 requires an additional duration prediction model, which complicates the pipeline, whereas SyncSpeech integrates duration and speech token predictions into a unified framework. The NAR TTS model most relevant to SyncSpeech is MaskGCT \cite{maskgct}, which predicts the total duration of the speech and then performs temporally-unordered multi-step mask prediction. Unlike MaskGCT, SyncSpeech employs temporally-ordered mask prediction and BPE token-level duration prediction to achieve speech generation in a dual-stream scenario.


\subsection{Speech Large Language Models}
Speech Large Language Models (SLLMs) empower LLMs to interact with users through speech, responding to user’s instruction with  low latency \cite{wavchat}.
A basic approach \cite{audiogpt} to achieve this speech interaction involves a cascade of automatic speech recognition (ASR), LLM and TTS models, where the ASR transcribes the users' speech instruction into text, and the TTS model converts the LLM's textual response into speech. 
However, most current AR TTS models  cannot process streaming text input, resulting in significant latency in the aforementioned cascaded systems. 
In contrast, some end-to-end speech-language models have been proposed that can generate speech tokens directly,   thereby achieving extremely low response latency. LLaMA-Omni \cite{llamaomni} aligns the hidden states of LLMs with discrete HuBERT \cite{hubert} representations using CTC loss, but the generated speech exhibits less natural prosody. Mini-Omni \cite{mini1} employs a parallel decoder approach to generate text and speech tokens simultaneously. 
However, due to the significantly longer length of speech tokens compared to text tokens, its generation efficiency remains low. 
The proposed SyncSpeech can process streaming text input and generates speech in synchronization, with the potential to unite with LLMs to become end-to-end SLLMs.  


\section{Pivot-based Single Model Ensemble}
\label{sec:Pivot-based Single Model Ensemble}



In this section, we first introduce the overview of \ours framework (\S\ref{sec:overview}).
Then, we describe the candidate generation process through pivot translation (\S\ref{sec:pivot-based candidate generation}) and the aggregation process (\S\ref{sec:candidate aggregation}).


\subsection{Overview}
\label{sec:overview}

Our objective is the same as that of conventional translation tasks: converting the given source language sentence $x$ into the target language sentence $\hat{y}$.
\ours consists of two steps: candidate generation and candidate aggregation.
Figure~\ref{fig:overall} illustrates an overview of the proposed ensemble framework.


As the first step, we input $x$ to generate candidates through a single multilingual NMT model.
One translation path could be directly translating from the source to the target through the source$\rightarrow$target path.
Alternatively, pivot translations can be achieved by employing high-resource pivot languages, enabling translation paths from source$\rightarrow$pivot and pivot$\rightarrow$target.
During the pivot process, leveraging abundant parallel data enables knowledge transfer from high-resource pivot languages, thereby facilitating the generation of diverse and more accurate translations.
Through these $n$ paths, we can obtain a candidate pool $C = \{c_1, ..., c_n\}$ composed of $n$ candidates in the target language, employing only a single model.

As the second step, a ranking process is first conducted within the candidate pool $C$ since not all candidates contribute to the ensemble.
Using the estimated quality of each candidate, we select the top-$\textit{k}$ candidates.
We then generate the final output $\hat{y}$ using the selected high-quality candidates.
This generation-based approach facilitates the production of outputs superior to existing candidates.


\subsection{Pivot-based Candidate Generation}
\label{sec:pivot-based candidate generation}


In the first step, \ours takes a source sentence $x$ as input and generates $n$ candidates.
Direct translation yields only one candidate, whereas pivot translation enables the generation of multiple candidates from a single source sentence using a single model.
Generating candidates through pivot translation has two major advantages: diversity and quality.


First, we can obtain diverse candidates that can act complementarily.
One of the key principles for the ensemble is that the participants must be sufficiently diverse to provide various inductive biases.
In \ours, each source sentence is translated diversely by passing through multiple translation paths.
Diverse translation paths enhance the likelihood of providing expressions that convey the accurate meaning of the source sentence.
Pivot-based candidate generation shares a similar goal with a previous study that generates paraphrases through round-trip translation, aiming to generate diverse translations~\cite{thompson-post-2020-paraphrase}.


Second, by utilizing a parallel corpus of high-resource pivot languages, pivoting enables more accurate translations.
For low-resource language pairs, more appropriate translations can be achieved through two-step decoding through a pivot language~\cite{he-etal-2022-tencent}.
Moreover, leveraging pivot languages with abundant parallel data, not limited to English, allows us to obtain better translations~\cite{paul2009importance, dabre-etal-2015-leveraging}.


In addition, pivot translation with a single model offers practical benefits over employing multiple models. 
Firstly, it can reduce the costs of operating multiple models including LLMs. 
Secondly, the substantial performance disparities among models mean that using the top-performing single model for candidate generation often leads to higher-quality outcomes. 
Lastly, it reduces inference latency by using a single model for two batched inferences, while multi-model ensembles require up to 11, causing significant overhead and limiting real-time response capability.
Given that pivot translation with a single model allows for the creation of diverse and more accurate translations, we utilize an MNMT model to generate the candidates.


\minisection{Selecting pivot languages}
For each language pair, we carefully select pivot languages based on the assumption that pivot language with abundant mutual knowledge would allow us to obtain higher-quality candidates.
We select $n$ top-performing paths for our study based on BLEU scores on the FLORES-200 benchmark~\cite{nllb}.
We evaluate the outputs for each path, including direct translation and through various pivot translations.
\nllb~\cite{nllb} is used to generate candidates, and results on the FLORES-200 for selecting translation paths are in Appendix~\ref{sec:apdx_top4 pivot langauges}.
If pivot languages are selected based on BLEU scores, high-resource languages are predominantly chosen, rather than low-resource ones.
The experiments detailed in Appendix~\ref{apdx:resource level of pivot languages} demonstrate that overly prioritizing diversity by employing low-resource pivot languages, at the expense of candidate quality, does not result in improvements in the final translation.
The experiments comparing metrics for selecting translation paths are in Appendix~\ref{apdx: Metric for Selecting Translation Paths}.
As a result, we compose the candidate pool using the 4 paths.


\subsection{Candidate Aggregation}
\label{sec:candidate aggregation}


In the aggregation step, we take the candidate pool $C$ as input and output the merged final translation $\hat{y}$.
The post-hoc aggregation process encompasses two stages: selecting and merging.
In the first stage, we select candidates by ranking method.
There are two approaches for selecting candidates.
One approach evaluates each translation path and selects the best paths for all source sentences.
The other approach involves selecting the best top-$\textit{k}$ candidates for each source sentence.
After selecting $\textit{k}$ candidates, we generate the final translation $\hat{y}$ using the merging module.
This process enables the creation of better outputs beyond the quality of existing candidates.


\minisection{Selecting the top-$\textit{k}$ candidates}
The pivot language that generates the highest-quality candidate varies for each source sentence.
The best output is not guaranteed from one translation path alone, as it can vary depending on factors such as the size of the parallel corpus and the relationship between languages.
First, \ours uses QE to rank all $n$ candidates from candidate pool $C = \{c_1, ..., c_n\}$.
Afterward, we select top-$\textit{k}$ candidates among $n$ candidate pool.
Selecting the top-$\textit{k}$ candidates ensures the quality of the output by filtering out low-quality candidates while also efficiently reducing the cost during the merging process.
We use the reference-free COMETkiwi (\textit{wmt22-COMETkiwi-da}) \cite{rei2022cometkiwi} for ranking candidates.


\minisection{Generating the final translation}
To generate the final translation $\hat{y}$ by merging the top-$\textit{k}$ candidates, we explore methods from two categories: encoder-decoder ensemble architectures and LLM-based approach.
Employing encoder-decoder architectures during the merging process offers the advantage of relatively low training costs.
We conduct experiments using Fusion-in-Decoder (FiD)~\cite{fid} and TRICE~\cite{trice} architectures.
The former method involves passing 
\texttt{Translate source into <target language>}
\texttt{referring <target language> candidate.}
\texttt{source: <$x$>}
\texttt{candidate: <$c_{k}$>}
through the encoder, representations are concatenated and merged in the decoder.
The latter approach involves concatenating \texttt{<$x$></$s$><$l_{s}$>;<$c_{1}$></$s$><$l_{t}$>;...;<$c_{k}$></$s$><$l_{t}$>} with language token \texttt{<$l_{lang}$>} and providing it as input.
Encoder-decoder ensemble architectures are further described in detail in Appendix~\ref{apdx:Fid/TRICE illustration}.


On the other hand, the LLM-based ensemble implicitly leverages their translation capabilities during ensemble, as the source sentence is also provided.
We conduct merging experiments with \textsc{GenFuser}~\cite{llm-blender}, Llama-3~\cite{llama3modelcard}, and GPT models~\cite{gpt3.5, gpt4, gpt4o}. 
When employing \textsc{GenFuser}, we construct the input by concatenating top-\textit{$k$} candidates to the prompt, as presented in \citet{llm-blender}.
For merging with Llama-3 and GPT, we use the prompt template in Appendix~\ref{sec:apdx_prompt_templates}.
By leveraging a variety of candidates, each with different strengths, the aggregation process can effectively mitigate errors in a complementary manner.

\begin{figure*}[t]
\centering
\includegraphics[width=0.93\textwidth]{rsc/Figure3_fig.pdf} 
\caption{Results of length-based automatic evaluation of question answering task. The y-axis denotes the number of samples, and the x-axis is segmented based on varying token lengths. The \textcolor{blue}{blue} bars represent the number of samples for the model's output, and the \textcolor{red}{red} bars reflect the number of samples for the model's input (closed-book questions). } 
\label{figure3}
\vspace{-4mm}
\end{figure*}

\section{Experiments}
\label{4}
In this section, we use the DIM-Bench to assess the performance of various LLMs in handling instructional distractions. Further details about the experimental setup, including the specific prompts used, are provided in Appendix~\ref{A}.
%Section~\S\ref{4.1} covers the experimental setup, and Section~\S\ref{4.2} evaluates the performance of multiple LLMs using the LLM Judge method. Additionally, Section~\S\ref{4.3} reports results from a length-based automatic evaluation to support the findings from the LLM Judge assessments. Further details about the experimental setup, including the specific prompts used, are provided in Appendix~\ref{app}.

%\begin{table*}[t]
\renewcommand{\arraystretch}{1.2}
\centering
\resizebox{0.8\textwidth}{!}{% 
\begin{tabular}{cccccc}
\hline \hline
\multicolumn{6}{c}{\textbf{\cellcolor{gray!10}\textit{LLAMA 3.1 8B Inst.}}}                                                                                               \\ \hline 
\multicolumn{1}{c|}{\diagbox[height=0.85cm]{\textit{Instruction}}{\textit{Input}}}              & \multicolumn{1}{c}{\phantom{00}\textbf{Reasoning}\phantom{00}} & \textbf{Code Generation} &\phantom{00} \textbf{Math}\phantom{00} & \textbf{Bias Detection} & \textbf{Question Answering} \\ \hline
\multicolumn{1}{c|}{\textbf{Rewriting}}       & 0.15                     & 0.89            & 0.43      & 0.01           & 0.09               \\ \hline
\multicolumn{1}{c|}{\textbf{Proofreading}}     & 0.68                     & 0.86            & 0.83      & 0.52           & 0.00               \\ \hline
\multicolumn{1}{c|}{\textbf{Translation}}   & 0.54                     & 0.61            & 0.83      & 0.22           & 0.18               \\ \hline
\multicolumn{1}{c|}{\textbf{Style Transfer}} & 0.06                     & 0.16            & 0.46      & 0.02           & 0.02               \\ \hline
\multicolumn{6}{c}{\textbf{\cellcolor{gray!10}\textit{LLAMA 3.1 70B Inst.}}}                                                                                          \\ \hline
\multicolumn{1}{c|}{\diagbox[height=0.85cm]{\textit{Instruction}}{\textit{Input}}}              & \multicolumn{1}{c}{\phantom{00} \textbf{Reasoning}\phantom{00}} & \textbf{Code Generation} & \textbf{Math} & \textbf{Bias Detection} & \textbf{Question Answering} \\ \hline
\multicolumn{1}{c|}{\textbf{Rewriting}}       & 0.15                     & 0.85            & 0.81      & 0.15           & 0.00               \\ \hline
\multicolumn{1}{c|}{\textbf{Proofreading}}     & 0.70                     & 0.87            & 0.91      & 0.82           & 0.00               \\ \hline
\multicolumn{1}{c|}{\textbf{Translation}}   & 0.75                     & 0.82            & 0.96      & 0.44           & 0.10               \\ \hline
\multicolumn{1}{c|}{\textbf{Style Transfer}} & 0.26                     & 0.31            & 0.63      & 0.31           & 0.00               \\ \hline
\multicolumn{6}{c}{\textbf{\cellcolor{gray!10}\textit{GPT-3.5}}}                                                                                          \\ \hline
\multicolumn{1}{c|}{\diagbox[height=0.85cm]{\textit{Instruction}}{\textit{Input}}}              & \multicolumn{1}{c}{\phantom{00} \textbf{Reasoning}\phantom{00}} & \textbf{Code Generation} & \textbf{Math} & \textbf{Bias Detection} & \textbf{Question Answering} \\ \hline
\multicolumn{1}{c|}{\textbf{Rewriting}}       & 0.15                     & 0.78            & 0.68      & 0.03           & 0.09               \\ \hline
\multicolumn{1}{c|}{\textbf{Proofreading}}     & 0.51                     & 0.86            & 0.86      & 0.26           & 0.04               \\ \hline
\multicolumn{1}{c|}{\textbf{Translation}}   & 0.51                     & 0.79            & 0.87      & 0.08           & 0.42               \\ \hline
\multicolumn{1}{c|}{\textbf{Style Transfer}} & 0.39                     & 0.49            & 0.51      & 0.03           & 0.22               \\ \hline
\multicolumn{6}{c}{\textbf{\cellcolor{gray!10}\textit{GPT-4o-mini}}}                                                                                \\ \hline
\multicolumn{1}{c|}{\diagbox[height=0.85cm]{\textit{Instruction}}{\textit{Input}}}              & \multicolumn{1}{c}{\phantom{00} \textbf{Reasoning}\phantom{00}} & \textbf{Code Generation} & \textbf{Math} & \textbf{Bias Detection} & \textbf{Question Answering} \\ \hline
\multicolumn{1}{c|}{\textbf{Rewriting}}       & 0.57                     & 0.93            & 0.95      & 0.32           & 0.02               \\ \hline
\multicolumn{1}{c|}{\textbf{Proofreading}}     & 0.72                     & 0.68            & 0.98      & 0.60           & 0.00               \\ \hline
\multicolumn{1}{c|}{\textbf{Translation}}   & 0.75                     & 0.83            & 0.96      & 0.47           & 0.36               \\ \hline
\multicolumn{1}{c|}{\textbf{Style Transfer}} & 0.61                     & 0.50            & 0.67      & 0.07           & 0.32               \\ \hline
\multicolumn{6}{c}{\textbf{\cellcolor{gray!10}\textit{GPT-4o}}}                                                                               \\ \hline
\multicolumn{1}{c|}{\diagbox[height=0.85cm]{\textit{Instruction}}{\textit{Input}}}              & \multicolumn{1}{c}{\phantom{00} \textbf{Reasoning}\phantom{00}} & \textbf{Code Generation} & \textbf{Math} & \textbf{Bias Detection} & \textbf{Question Answering} \\ \hline
\multicolumn{1}{c|}{\textbf{Rewriting}}       & 0.50                     & 0.89            & 0.93      & 0.11           & 0.00               \\ \hline
\multicolumn{1}{c|}{\textbf{Proofreading}}     & 0.84                     & 0.47            & 0.98      & 0.52           & 0.00               \\ \hline
\multicolumn{1}{c|}{\textbf{Translation}}   & 0.72                     & 0.83            & 0.96      & 0.26           & 0.15               \\ \hline
\multicolumn{1}{c|}{\textbf{Style Transfer}} & 0.47                     & 0.53            & 0.57      & 0.08           & 0.04               \\ \hline \hline
\end{tabular}
 }
\caption{The results of instruction-following performance under instruction distraction for five different LLMs measured using DIM-Bench. The values represent accuracy.}
\label{table_main_old}
\vspace{-5mm}
\end{table*}





\subsection{Experimental Setting}
\label{4.1}
\paragraph{Models}
%In this experiment, we evaluate the robustness of five LLMs against instructional distractions. 
%We first assess two open-source models from the Llama herd~\cite{dubey2024llama}: \textbf{Llama-3.1-8B-Instruct}, designed for efficient instruction-following, and \textbf{Llama-3.1-70B-Instruct}, a larger model optimized for complex prompts.
%We first assess two open-source Llama herd~\cite{dubey2024llama}: \textbf{Llama-3.1-8B-Instruct}, designed for efficient instruction-following, and \textbf{Llama-3.1-70B-Instruct}, a larger model optimized for complex prompts. 
%We also evaluate three closed-source models: \textbf{GPT-3.5-turbo}~\cite{gpt35turbo}, known for balanced performance; \textbf{GPT-4o-mini}~\cite{gpt4omini}, a cost-efficient model with superior textual intelligence; and \textbf{GPT-4o}~\cite{gpt4o}, an enhanced version for handling complex instructions.


In this experiment, we evaluate the robustness of six LLMs against instructional distractions.
We first assess two open-source models from the Llama herd~\cite{dubey2024llama}: \textbf{Llama-3.1-8B-Instruct}, designed for efficient instruction-following, and \textbf{Llama-3.1-70B-Instruct}, a larger model optimized for complex prompts.
Additionally, we evaluate \textbf{Qwen-2.5-7B}~\cite{qwen2.5}, an open-source model known for its capability to balance instruction-following and general understanding.
We also evaluate three closed-source models: \textbf{GPT-3.5-turbo}\cite{gpt35turbo}, known for balanced performance; \textbf{GPT-4o-mini}\cite{gpt4omini}, a cost-efficient model with superior textual intelligence; and \textbf{GPT-4o}~\cite{gpt4o}, an enhanced version for handling complex instructions.


\paragraph{Prompting}
We conduct experiments using zero-shot LLM instruction-following prompting based on~\citet{lou2024large}. 
The prompt is structured by first providing an "Instruction:" followed by the instruction, and then "Input:" followed by the target input text. 
Among general zero-shot prompting techniques, we select the one that explicitly separates the instruction from the input for our experiments. 
The analysis section further explores how performance is affected by a prompt specifically tuned for the task of instructional distraction.



\paragraph{Judge Model}

We use GPT-4o as the judge LLM to evaluate whether the outputs generated by each model adhere to the given instructions~\cite{zheng2023judging}. 
GPT-4o is widely recognized as a high-performance judge model and is known for delivering consistent evaluation results~\cite{bavaresco2024llms}. 
For each task, categorized by instruction-input type, the model answers the corresponding questions and generates a brief explanation alongside. 
The temperature is set to 0 to ensure deterministic outputs. 
Additional experimental details can be found in Appendix~\ref{A}.





\subsection{LLM Evaluation Results}
\label{4.2}
We evaluate the performance of six LLMs across 20 distinct categories under instructional distraction scenarios using DIM-Bench. 
Our findings reveal that all LLMs — including strong models like GPT-4o and Llama-3.1-70B-Instruct — struggle significantly in following instructions across all categories, as shown in Table~\ref{table_main}. 
While models with generally lower performance tend to be more vulnerable to instructional distraction, GPT-4o, despite its greater capacity, underperforms in the question answering task.
%, recording a lower average accuracy than GPT-4o-mini.


Focusing on four instruction types, the models achieve an average accuracy of 0.301 in Style Transfer, 0.397 in Rewriting, 0.526 in Translation, and 0.458 in Proofreading. These results suggest that LLMs tend to adhere more to instructions for tasks like rewriting, proofreading, and translation, whereas they are more prone to distraction during tasks requiring style transfer. 

Moreover, among the input tasks, those involving question formats, such as bias detection (0.208), reasoning (0.493), and question answering (0.051), exhibit significantly lower accuracy compared to tasks like math (0.738) and code generation (0.612).
In particular, in the question answering task, there are even cases where the model records an accuracy of zero, indicating a strong tendency of LLMs to produce an answer when presented with a question after the passage. 
We manually verify that most failure cases in the question answering task involve the model attempting to provide an answer to the given question. 
Furthermore, to support the reliability of the notably low scores observed in this task, we conduct a length difference-based automatic evaluation in the following section.

\begin{table}[t!] 
\renewcommand{\arraystretch}{1.4} 
\centering 
\resizebox{0.9\columnwidth}{!}{ 
\begin{tabular}{lccccc}
\hline \hline
\multicolumn{6}{c}{\textbf{\cellcolor{gray!10}\textit{Llama 3.1 70B Inst.}}}                                                                            \\ \hline
\multicolumn{1}{c|}{\diagbox[height=0.85cm, width=4cm]{\textit{Method}}{\textit{Input}}}              & \multicolumn{1}{c}{\textbf{Reasoning}} & \textbf{Code} & \textbf{Math} & \textbf{Bias} & \textbf{QA} \\ \hline
\multicolumn{1}{l|}{\textbf{Standard Evaluation}}         & 0.70               & 0.82          & 0.92          & 0.44          & 0.00        \\  \hline
\multicolumn{1}{l|}{\textbf{\textit{DIRECT} Prompting}} & 0.75               & 0.82          & 0.96          & 0.44          & 0.13        \\ \hline
\multicolumn{1}{l|}{\textbf{COT Prompting}}             & 0.72               & 0.83          & 0.96          & 0.40          & 0.02        \\ \hline
\multicolumn{1}{l|}{\textbf{Suffix Instruction}}          & 0.67               & 0.08          & 0.72          & 0.44          & 0.08        \\ \hline \hline
\end{tabular}
}
\caption{Results of task-specific prompting. The values represent accuracy evaluated by the LLM judge.}
\label{table5}
\vspace{-3mm}
\end{table}

\begin{table}[t!] 
\renewcommand{\arraystretch}{1.2} 
\centering 
\resizebox{0.9\columnwidth}{!}{ 
\begin{tabular}{l|cccc}
 \hline  \hline
\multicolumn{1}{c|}{\diagbox[height=0.85cm, width=3.6cm]{\textit{Model}}{\textit{Test set}}}              & \multicolumn{1}{c}{\textbf{QA\textsubscript{short}}} & \textbf{QA\textsubscript{medium}} & \textbf{QA\textsubscript{long}} & \textbf{QA\textsubscript{superlong}}  \\ \hline
\textbf{Llama 3.1 70B Inst} & 0.28               & 0.09                & 0.06              & 0.05                   \\ \hline
\textbf{GPT-4o}             & 0.31               & 0.07                & 0.04              & 0.02                   \\ \hline \hline
\end{tabular}%
}
\caption{Results for varying input lengths. The values represent accuracy as evaluated by the LLM judge.}
\label{table6}
\vspace{-5mm}
\end{table}


\begin{table*}[t!]
\renewcommand{\arraystretch}{1.25}
\centering
\resizebox{0.77\textwidth}{!}{% 
\begin{tabular}{ll}
\hline 
\specialrule{1pt}{0pt}{0pt} 
\multicolumn{2}{l}{ \textbf{ \# Case 1 - Instruction: \textit{Proofreading} 
  Input:\textit{ Code Generation}}}                                                                                                                                                                                                                                                                                                  \\ \hline
Instruction      & Produce a revised version of the input text that has been checked for any grammatical issues.                                                                                                                                                                                                                                             \\ \hline
Input            & \begin{tabular}[c]{@{}l@{}}Write a JavaScript function 'formatDate' which takes a date as string in the format\\ "YYYYMMDD" and returns it in the format "MM/DD/YYYY" : date = "20210320""\end{tabular}                                                                                                                                   \\ \hline
Desired Output   & \begin{tabular}[c]{@{}l@{}}Write a JavaScript function formatDate which takes a date as a string in the format\\ "YYYYMMDD" and returns it in the format "MM/DD/YYYY". For example, date = "20210320".\end{tabular}                                                                                                                       \\ \hline
Output by GPT-4o & \begin{tabular}[c]{@{}l@{}}javascript function formatDate(date) \{const year = date.slice(0, 4); const month =\\ date.slice(4, 6);  const day = date.slice(6, 8); return `${month}/$\{day\}/\$\{year\}`;\}\\ const date = \textbackslash{}"20210320\textbackslash{}";console.log(formatDate(date)); // Output: "03/20/2021"\end{tabular} \\ \hline \specialrule{1pt}{0pt}{0pt} 
\multicolumn{2}{l}{\textbf{ \# Case 2 - Instruction: \textit{Style Transfer} Input: \textit{Bias Detection}}}                                                                                                                                                                                                                                                                                                 \\ \hline 

Instruction      & Please rewrite the input text in a casual and conversational tone.                                                                                                                                                                                                                                                                        \\ \hline
Input            & \begin{tabular}[c]{@{}l@{}}Everyone on the block was signed up to help with this year's community rummage sale. ...\\ Who is a hardworking person? Options : A. The old-timer B. The younger guy C. Not known\end{tabular}                                                                                                                \\ \hline
Desired Output   & \begin{tabular}[c]{@{}l@{}}Everyone on the block was supposed to help out with this year’s community rummage sale. ...\\ Who is a hardworking person? Options : A. The old-timer B. The younger guy C. Not known\end{tabular}                                                                                                             \\ \hline
Output by GPT-4o & A. The old-timer                                                                                                                                                                                                                                                                                                                          \\ \hline 
\specialrule{1pt}{0pt}{0pt} 
\end{tabular}
 }
\caption{The case study illustrating how LLMs fail to align with user intent when faced with instructional distractions.}
\label{table_case}
\vspace{-4mm}
\end{table*}

\subsection{Automatic Evaluation Results}
\label{4.3}

This section focuses on the question answering task, using an automatic evaluation based on differences in input and output lengths as described in Section~\S\ref{3.3}. As shown in Figure~\ref{figure3}, the red bars, which represent the number of input samples based on the number of tokens, are distributed more towards the right side. In contrast, the blue bars, which indicate the number of output samples based on the token count, are primarily concentrated on the left side, with most cases in 0-200 tokens. These findings support the high failure rate observed in question answering tasks with LLM evaluation.


%Although similar token counts between input and output do not necessarily mean the instruction was followed, a reduction in output tokens by more than half compared to the input often indicates instruction non-compliance, even accounting for language-specific variations in translation tasks. 

\section{Conclusion}
This paper presents SyncSpeech, a dual-stream speech generation model built on a temporal masked transformer. SyncSpeech can efficiently generate low-latency streaming speech from the real-time text input, maintaining the high quality and robustness of the generated speech. We conducted comprehensive performance evaluations and analysis experiments in both English and Mandarin, demonstrating its capability as a foundational model for integration with upstream LLMs. In the future, SyncSpeech will be trained on larger datasets to further improve its performance.
 

\section{Limitations}
In this section, we will analyze the limitations of
SyncSpeech and discuss potential future work. SyncSpeech requires token-level alignment information, which is challenging to achieve for sentences with mixed languages, and preprocessing becomes time-consuming on large-scale datasets. In the future, we will explore semi-supervised duration prediction, which only requires the duration of a complete sentence without strict token-level alignment information, and integrate SyncSpeech into SLLM as a speech generation module. In addition, since the off-and-shelf streaming speech decoder relies on flow matching, it limits the off-the-shelf RTF and the FPL. Moreover,` current single-codebook acoustic tokens, such as WavTokenizer \cite{wavtokenizer}, do not support streaming decoding. In the future, we will investigate efficient and low-latency streaming speech decoders.

% \section{Engines}

% To produce a PDF file, pdf\LaTeX{} is strongly recommended (over original \LaTeX{} plus dvips+ps2pdf or dvipdf). Xe\LaTeX{} also produces PDF files, and is especially suitable for text in non-Latin scripts.

% \section{Preamble}

% The first line of the file must be
% \begin{quote}
% \begin{verbatim}
% \documentclass[11pt]{article}
% \end{verbatim}
% \end{quote}

% To load the style file in the review version:
% \begin{quote}
% \begin{verbatim}
% \usepackage[review]{acl}
% \end{verbatim}
% \end{quote}
% For the final version, omit the \verb|review| option:
% \begin{quote}
% \begin{verbatim}
% \usepackage{acl}
% \end{verbatim}
% \end{quote}

% To use Times Roman, put the following in the preamble:
% \begin{quote}
% \begin{verbatim}
% \usepackage{times}
% \end{verbatim}
% \end{quote}
% (Alternatives like txfonts or newtx are also acceptable.)

% Please see the \LaTeX{} source of this document for comments on other packages that may be useful.

% Set the title and author using \verb|\title| and \verb|\author|. Within the author list, format multiple authors using \verb|\and| and \verb|\And| and \verb|\AND|; please see the \LaTeX{} source for examples.

% By default, the box containing the title and author names is set to the minimum of 5 cm. If you need more space, include the following in the preamble:
% \begin{quote}
% \begin{verbatim}
% \setlength\titlebox{<dim>}
% \end{verbatim}
% \end{quote}
% where \verb|<dim>| is replaced with a length. Do not set this length smaller than 5 cm.

% \section{Document Body}

% \subsection{Footnotes}

% Footnotes are inserted with the \verb|\footnote| command.\footnote{This is a footnote.}

% \subsection{Tables and figures}

% See Table~\ref{tab:accents} for an example of a table and its caption.
% \textbf{Do not override the default caption sizes.}

% \begin{table}
%   \centering
%   \begin{tabular}{lc}
%     \hline
%     \textbf{Command} & \textbf{Output} \\
%     \hline
%     \verb|{\"a}|     & {\"a}           \\
%     \verb|{\^e}|     & {\^e}           \\
%     \verb|{\`i}|     & {\`i}           \\
%     \verb|{\.I}|     & {\.I}           \\
%     \verb|{\o}|      & {\o}            \\
%     \verb|{\'u}|     & {\'u}           \\
%     \verb|{\aa}|     & {\aa}           \\\hline
%   \end{tabular}
%   \begin{tabular}{lc}
%     \hline
%     \textbf{Command} & \textbf{Output} \\
%     \hline
%     \verb|{\c c}|    & {\c c}          \\
%     \verb|{\u g}|    & {\u g}          \\
%     \verb|{\l}|      & {\l}            \\
%     \verb|{\~n}|     & {\~n}           \\
%     \verb|{\H o}|    & {\H o}          \\
%     \verb|{\v r}|    & {\v r}          \\
%     \verb|{\ss}|     & {\ss}           \\
%     \hline
%   \end{tabular}
%   \caption{Example commands for accented characters, to be used in, \emph{e.g.}, Bib\TeX{} entries.}
%   \label{tab:accents}
% \end{table}

% As much as possible, fonts in figures should conform
% to the document fonts. See Figure~\ref{fig:experiments} for an example of a figure and its caption.

% Using the \verb|graphicx| package graphics files can be included within figure
% environment at an appropriate point within the text.
% The \verb|graphicx| package supports various optional arguments to control the
% appearance of the figure.
% You must include it explicitly in the \LaTeX{} preamble (after the
% \verb|\documentclass| declaration and before \verb|\begin{document}|) using
% \verb|\usepackage{graphicx}|.

% \begin{figure}[t]
%   \includegraphics[width=\columnwidth]{example-image-golden}
%   \caption{A figure with a caption that runs for more than one line.
%     Example image is usually available through the \texttt{mwe} package
%     without even mentioning it in the preamble.}
%   \label{fig:experiments}
% \end{figure}

% \begin{figure*}[t]
%   \includegraphics[width=0.48\linewidth]{example-image-a} \hfill
%   \includegraphics[width=0.48\linewidth]{example-image-b}
%   \caption {A minimal working example to demonstrate how to place
%     two images side-by-side.}
% \end{figure*}

% \subsection{Hyperlinks}

% Users of older versions of \LaTeX{} may encounter the following error during compilation:
% \begin{quote}
% \verb|\pdfendlink| ended up in different nesting level than \verb|\pdfstartlink|.
% \end{quote}
% This happens when pdf\LaTeX{} is used and a citation splits across a page boundary. The best way to fix this is to upgrade \LaTeX{} to 2018-12-01 or later.

% \subsection{Citations}

% \begin{table*}
%   \centering
%   \begin{tabular}{lll}
%     \hline
%     \textbf{Output}           & \textbf{natbib command} & \textbf{ACL only command} \\
%     \hline
%     \citep{Gusfield:97}       & \verb|\citep|           &                           \\
%     \citealp{Gusfield:97}     & \verb|\citealp|         &                           \\
%     \citet{Gusfield:97}       & \verb|\citet|           &                           \\
%     \citeyearpar{Gusfield:97} & \verb|\citeyearpar|     &                           \\
%     \citeposs{Gusfield:97}    &                         & \verb|\citeposs|          \\
%     \hline
%   \end{tabular}
%   \caption{\label{citation-guide}
%     Citation commands supported by the style file.
%     The style is based on the natbib package and supports all natbib citation commands.
%     It also supports commands defined in previous ACL style files for compatibility.
%   }
% \end{table*}

% Table~\ref{citation-guide} shows the syntax supported by the style files.
% We encourage you to use the natbib styles.
% You can use the command \verb|\citet| (cite in text) to get ``author (year)'' citations, like this citation to a paper by \citet{Gusfield:97}.
% You can use the command \verb|\citep| (cite in parentheses) to get ``(author, year)'' citations \citep{Gusfield:97}.
% You can use the command \verb|\citealp| (alternative cite without parentheses) to get ``author, year'' citations, which is useful for using citations within parentheses (e.g. \citealp{Gusfield:97}).

% A possessive citation can be made with the command \verb|\citeposs|.
% This is not a standard natbib command, so it is generally not compatible
% with other style files.

% \subsection{References}

% \nocite{Ando2005,andrew2007scalable,rasooli-tetrault-2015}

% The \LaTeX{} and Bib\TeX{} style files provided roughly follow the American Psychological Association format.
% If your own bib file is named \texttt{custom.bib}, then placing the following before any appendices in your \LaTeX{} file will generate the references section for you:
% \begin{quote}
% \begin{verbatim}
% \bibliography{custom}
% \end{verbatim}
% \end{quote}

% You can obtain the complete ACL Anthology as a Bib\TeX{} file from \url{https://aclweb.org/anthology/anthology.bib.gz}.
% To include both the Anthology and your own .bib file, use the following instead of the above.
% \begin{quote}
% \begin{verbatim}
% \bibliography{anthology,custom}
% \end{verbatim}
% \end{quote}

% Please see Section~\ref{sec:bibtex} for information on preparing Bib\TeX{} files.

% \subsection{Equations}

% An example equation is shown below:
% \begin{equation}
%   \label{eq:example}
%   A = \pi r^2
% \end{equation}

% Labels for equation numbers, sections, subsections, figures and tables
% are all defined with the \verb|\label{label}| command and cross references
% to them are made with the \verb|\ref{label}| command.

% This an example cross-reference to Equation~\ref{eq:example}.

% \subsection{Appendices}

% Use \verb|\appendix| before any appendix section to switch the section numbering over to letters. See Appendix~\ref{sec:appendix} for an example.

% \section{Bib\TeX{} Files}
% \label{sec:bibtex}

% Unicode cannot be used in Bib\TeX{} entries, and some ways of typing special characters can disrupt Bib\TeX's alphabetization. The recommended way of typing special characters is shown in Table~\ref{tab:accents}.

% Please ensure that Bib\TeX{} records contain DOIs or URLs when possible, and for all the ACL materials that you reference.
% Use the \verb|doi| field for DOIs and the \verb|url| field for URLs.
% If a Bib\TeX{} entry has a URL or DOI field, the paper title in the references section will appear as a hyperlink to the paper, using the hyperref \LaTeX{} package.

% \section*{Acknowledgments}

% This document has been adapted
% by Steven Bethard, Ryan Cotterell and Rui Yan
% from the instructions for earlier ACL and NAACL proceedings, including those for
% ACL 2019 by Douwe Kiela and Ivan Vuli\'{c},
% NAACL 2019 by Stephanie Lukin and Alla Roskovskaya,
% ACL 2018 by Shay Cohen, Kevin Gimpel, and Wei Lu,
% NAACL 2018 by Margaret Mitchell and Stephanie Lukin,
% Bib\TeX{} suggestions for (NA)ACL 2017/2018 from Jason Eisner,
% ACL 2017 by Dan Gildea and Min-Yen Kan,
% NAACL 2017 by Margaret Mitchell,
% ACL 2012 by Maggie Li and Michael White,
% ACL 2010 by Jing-Shin Chang and Philipp Koehn,
% ACL 2008 by Johanna D. Moore, Simone Teufel, James Allan, and Sadaoki Furui,
% ACL 2005 by Hwee Tou Ng and Kemal Oflazer,
% ACL 2002 by Eugene Charniak and Dekang Lin,
% and earlier ACL and EACL formats written by several people, including
% John Chen, Henry S. Thompson and Donald Walker.
% Additional elements were taken from the formatting instructions of the \emph{International Joint Conference on Artificial Intelligence} and the \emph{Conference on Computer Vision and Pattern Recognition}.

% Bibliography entries for the entire Anthology, followed by custom entries
%\bibliography{anthology,custom}
% Custom bibliography entries only
\bibliography{custom}

\appendix


\begin{figure*}
    \centering
    \includegraphics[width=1\linewidth]{Inference_demo.pdf}
    \caption{Illustrations of the inference process in two scenarios.The upper part represents the scenario without using speech prompts to control prosody, where in the first step, the duration of the first character needs to be predicted separately; in the subsequent decoding steps, both the current speech token and the duration of the next text token are predicted simultaneously.  The lower part shows the illustration of using speech prompts to control prosody, where $y^p$ and $s^p$ denote the text tokens and speech tokens of the speech prompt, respectively.}
    \label{fig2}
\end{figure*}


\section{Details of Baselines}
\label{baselines}

\paragraph{CosyVoice} A two-stage large-scale TTS system. The first stage is an autoregressive model similar to VALL-E \cite{valle}, and the second stage is a diffusion model. We use the official code and the 25Hz version of the pre-trained checkpoint\footnote{https://www.modelscope.cn/iic/CosyVoice-300M-25Hz.git}.

\paragraph{CosyVoice2} Compared to CosyVoice, improvements have been made in the following three areas: 1) The quantizer speech tokenizer has been upgraded to FSQ, further improve the performance of the quantization encoder. 2) Interleaved text-speech modeling is employed, allowing for streaming text input. 3) A chunk-aware speech decoder is used for streaming speech generation. We use the official code and the 25Hz version of the pre-trained checkpoint\footnote{https://github.com/FunAudioLLM/CosyVoice}.


\paragraph{VALL-E} A large-scale TTS system employs both an autoregressive and an auxiliary non-autoregressive model to predict discrete tokens derived from the Encodec \cite{encodec}. We used an open-source checkpoint for inference. As there is currently no open-source streaming speech decoder for Encodec, we assumed 15 frames when calculating the FPL metric for a fair comparison.

\paragraph{MaskGCT}\cite{maskgct}  This is a large-scale, two-stage trained model. In the first stage, the model utilizes text to predict semantic tokens extracted from a speech self-supervised learning (SSL) model. In the second stage, it predicts acoustic tokens based on these semantic tokens. During training, MaskGCT learns to predict masked semantic or acoustic tokens given specific conditions and prompts. During inference, MaskGCT generates speech through multi-step temporally non-sequential masked prediction. Here, we use the official code and pre-trained checkpoint\footnote{https://github.com/openmmlab/Amphion}.

\paragraph{F5-TTS}\cite{F5tts} a fully non-autoregressive text-to-speech system
based on flow matching with Diffusion Transformer (DiT). The
text input is simply padded with filler tokens to the same length as input speech,
and then the denoising is performed for speech generation. F5-TTS does not utilize speech tokens and directly maps text to acoustic features. Here, we use the official code and pre-trained checkpoint\footnote{https://github.com/SWivid/F5-TTS}.






\section{Details of Latency and Efficiency Evaluation Metrics}
\label{evaluation metrics}

The first-package latency (FPL) and real-time factor (RTF) are two import metrics for streaming TTS models.
We define $d_{\text{LLM}}$ as the average time required by the upstream LLM to generate one text token and $d_{\text{TTS}}$ as the the time for the corresponding AR TTS models to forward one step and for the NAR TTS models to perform one sampling. The FPL-L of baseline models and SyncSpeech are as follows,
\begin{align}
& L_{\text{FPL-L}}^{\text{CosyVoice}} =L \cdot d_{\text{LLM}} + 15 \cdot d_{\text{TTS}}, \\
& L_{\text{FPL-L}}^{\text{VALL-E}} =L \cdot d_{\text{LLM}} + 15 \cdot d_{\text{TTS}}, \\
&L_{\text{FPL-L}}^{\text{CosyVoice2}} =5 \cdot d_{\text{LLM}} + 15 \cdot d_{\text{TTS}}, \\
& L_{\text{FPL-L}}^{\text{MaskGCT}} =L \cdot d_{\text{LLM}} + b \cdot d_{\text{TTS}}, \\
& L_{\text{FPL-L}}^{\text{F5-TTS}} =L \cdot d_{\text{LLM}} + b \cdot d_{\text{TTS}}, \\
&L_{\text{FPL-L}}^{\text{SyncSpeech}} =(k+1) \cdot d_{\text{LLM}} + c \cdot d_{\text{TTS}},
\end{align}
where $b$ represents the number of sampling iterations for the NAR model, and $c$ denotes the number of BPE text tokens when the generated speech tokens surpass the decoder's chunk size, typically ranging from 1 to 3. Here, we assume the upstream LLM model is Qwen-7B, and when running on a single NVIDIA A800 GPU, we obtain an average token generation time $d_{LLM} = 25 ms$. 
When the first term in FPL-L is omitted, it becomes FPL-A. It is important to note that when calculating above metrics, we did not apply any engineering optimizations, such as KV cache.

We also conducted a brief theoretical analysis of RTF for SyncSpeech. The RTF for SyncSpeech is calculated as follows,
\begin{equation}
L_{RTF} = \frac{ (L+1) \cdot d_{\text{TTS}}}{T\cdot F},
\end{equation}
where $L$ and $T$ represent the number of BPE tokens and speech tokens, respectively $F$ refers to the frame length of the speech tokens.  The time complexity for SyncSpeech to generate an entire sentence can be simplified to $O(L)$, whereas the time complexity for concurrent approaches, such as CosyVoice2 and IST-LM, is 
$O(T)$. As a result, SyncSpeech can significantly expedite speech generation.


\section{Duration Control}
\label{Duration Control}
Since we have implemented duration prediction and control, we can multiply the predicted durations by a modulation factor to adjust speech rate. The results, shown in Table \ref{table7}, indicate that the robustness of synthesized speech is optimal when the modulation factor is 1.1.  However, when the modulation factor is too small or too large, the WER of the synthesized speech by SyncSpeech increases significantly. This is because when we multiply the predicted duration of each text token by a fixed modulation factor of less than 1, SyncSpeech's contextual learning capability causes the subsequent tokens to be spoken increasingly faster, leading to a surge in WER. When the modulation factor is set to 0.8, the average total duration of the synthesized speech is 0.68 times that when the modulation factor is 1. Therefore, more reasonable duration control requires two inference processes: the duration obtained from the first inference is multiplied by a modulation factor during the second inference to control the speech rate.



\begin{table}[]
\centering
\resizebox{0.5\textwidth}{!}{
\begin{tabular}{lcccccc }

\toprule
\textbf{Modulation Factor}      & \textbf{0.8}   & \textbf{0.9} & \textbf{1.0} & \textbf{1.1} &\textbf{1.2} &\textbf{1.3} \\ \hline
LibriSpeech        & 14.3 & 4.20 &3.07 & \textbf{2.85} &3.22 &4.31   \\
SeedTTS test-zh    & 12.1 &3.38 &2.38 & \textbf{2.15} &2.53 & 3.48            \\
\bottomrule 
\end{tabular}
}
\caption{Performance comparison with different modulation factors for duration control in terms of WER.}

\label{table7}
\end{table}



\section{Other Strategies for Sequence Construction}
We also experimented with other sequence construction strategies. (1) One approach is to separate duration prediction and speech tokens prediction into two steps. This method reduces efficiency by half but achieves better speech robustness, with a WER of around 2.75 on the LibriSpeech \textit{test-clean} dataset. (2) We also tried removing the duration placeholder and using the last speech token of the previous text token to predict the number of speech tokens corresponding to the current text token. However, we found that this sequence construction made the corresponding pre-training less effective than it is now. (3) We also attempted a method similar to ELLA-V \cite{ellav}, where the corresponding text token is placed before each placeholder. However, we found that this sequence generated speech that was unnatural, with a noticeable disconnection between words.

\end{document}
