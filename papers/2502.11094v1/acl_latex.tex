% This must be in the first 5 lines to tell arXiv to use pdfLaTeX, which is strongly recommended.
\pdfoutput=1
% In particular, the hyperref package requires pdfLaTeX in order to break URLs across lines.

\documentclass[11pt]{article}

% Change "review" to "final" to generate the final (sometimes called camera-ready) version.
% Change to "preprint" to generate a non-anonymous version with page numbers.
\usepackage[preprint]{acl}
\usepackage{amsmath}
\usepackage{amssymb}
% Standard package includes
\usepackage{times}
\usepackage{latexsym}
\usepackage{booktabs} 
\usepackage{hyperref}
\usepackage{arydshln}
\usepackage{fixltx2e}
\usepackage{multirow}
% \usepackage{algorithm}
% \usepackage{algorithmic}
\usepackage[linesnumbered, ruled, vlined]{algorithm2e}
% \usepackage{algpseudocode}
% For proper rendering and hyphenation of words containing Latin characters (including in bib files)
\usepackage[T1]{fontenc}
% For Vietnamese characters
% \usepackage[T5]{fontenc}
% See https://www.latex-project.org/help/documentation/encguide.pdf for other character sets

% This assumes your files are encoded as UTF8
\usepackage[utf8]{inputenc}

% This is not strictly necessary, and may be commented out,
% but it will improve the layout of the manuscript,
% and will typically save some space.
\usepackage{microtype}

% This is also not strictly necessary, and may be commented out.
% However, it will improve the aesthetics of text in
% the typewriter font.
\usepackage{inconsolata}

%Including images in your LaTeX document requires adding
%additional package(s)
\usepackage{graphicx}

% If the title and author information does not fit in the area allocated, uncomment the following
%
%\setlength\titlebox{<dim>}
%
% and set <dim> to something 5cm or larger.

\title{SyncSpeech: Low-Latency and Efficient Dual-Stream Text-to-Speech based on Temporal Masked Transformer}

% Author information can be set in various styles:
% For several authors from the same institution:
% \author{Author 1 \and ... \and Author n \\
%         Address line \\ ... \\ Address line}
% if the names do not fit well on one line use
%         Author 1 \\ {\bf Author 2} \\ ... \\ {\bf Author n} \\
% For authors from different institutions:
% \author{Author 1 \\ Address line \\  ... \\ Address line
%         \And  ... \And
%         Author n \\ Address line \\ ... \\ Address line}
% To start a separate ``row'' of authors use \AND, as in
% \author{Author 1 \\ Address line \\  ... \\ Address line
%         \AND
%         Author 2 \\ Address line \\ ... \\ Address line \And
%         Author 3 \\ Address line \\ ... \\ Address line}

% \author{Zhengyan Sheng \\
%   Affiliation / Address line 1 \\
%   Affiliation / Address line 2 \\
%   Affiliation / Address line 3 \\
%   \texttt{email@domain} \\\And
%   Second Author \\
%   Affiliation / Address line 1 \\
%   Affiliation / Address line 2 \\
%   Affiliation / Address line 3 \\
%   \texttt{email@domain} \\}

\author{
 Zhengyan Sheng\textsuperscript{1}\thanks{Work done during internship at Alibaba Group.},
 Zhihao Du\textsuperscript{2},
 Shiliang Zhang\textsuperscript{2},
 Zhijie Yan\textsuperscript{2}, \\
 \textbf{Yexin Yang\textsuperscript{2}},
 \textbf{Zhenhua Ling\textsuperscript{1}}\thanks{Corresponding author}
 \\
%  \textbf{Fifth Author\textsuperscript{1,2}},
%  \textbf{Sixth Author\textsuperscript{1}},
%  \textbf{Seventh Author\textsuperscript{1}},
%  \textbf{Eighth Author \textsuperscript{1,2,3,4}},
% \\
%  \textbf{Ninth Author\textsuperscript{1}},
%  \textbf{Tenth Author\textsuperscript{1}},
%  \textbf{Eleventh E. Author\textsuperscript{1,2,3,4,5}},
%  \textbf{Twelfth Author\textsuperscript{1}},
% \\
%  \textbf{Thirteenth Author\textsuperscript{3}},
%  \textbf{Fourteenth F. Author\textsuperscript{2,4}},
%  \textbf{Fifteenth Author\textsuperscript{1}},
%  \textbf{Sixteenth Author\textsuperscript{1}},
% \\
%  \textbf{Seventeenth S. Author\textsuperscript{4,5}},
%  \textbf{Eighteenth Author\textsuperscript{3,4}},
%  \textbf{Nineteenth N. Author\textsuperscript{2,5}},
%  \textbf{Twentieth Author\textsuperscript{1}}
% \\
\\
 \textsuperscript{1}University of Science and Technology of China \\
 \textsuperscript{2}Speech Lab, Alibaba Group, China,
 % \textsuperscript{3}Affiliation 3,
 % \textsuperscript{4}Affiliation 4,
 % \textsuperscript{5}Affiliation 5
\\
 {
     \href{zysheng@mail.ustc.edu.cn}{zysheng@mail.ustc.edu.cn}, 
     \href{neo.dzh@alibaba-inc.com}{\{neo.dzh, sly.zsl\}@alibaba-inc.com},
     \href{zhling@ustc.edu.cn}{zhling@ustc.edu.cn}
 }
}

% 本文提出了一个双流模型,SyncSpeech, 能够接受上游大模型流式的文本输入同时进行流式的语音生成,和大语言模型可以无缝交互。SyncSpeech有以下特点(1)低延时,即接受到第二个文本token时开始流式的语音生成;(2)高效,每一个解码生成对应文本token的全部speech token。为此,我们提出temporl masked transformer 作为 SyncSpeech的Backbone,并结合token-level的时长预测,在一步解码的过程中同时预测speech token和下一步的时长。此外,我们也设计了两阶段的训练策略来提升训练效率和生成语音的质量。我们在英文和普通话两种数据集上进行了验证,相比双流的sota tts模型,分别降低了speech token3.8和5.5倍的首包延迟,加速了6.4和8.6倍的的实时比。同时,在相同的数据规模下,也取得了相当的性能。
\begin{document}
\maketitle
\begin{abstract}
This paper presents a dual-stream text-to-speech (TTS) model, SyncSpeech, capable of receiving streaming text input from upstream models while simultaneously generating streaming speech, facilitating seamless interaction with large language models. SyncSpeech has the following advantages: Low latency, as it begins generating streaming speech upon receiving the second text token; High efficiency, as it decodes all speech tokens corresponding to the each arrived text token in one step. To achieve this, we propose a temporal masked transformer as the backbone of SyncSpeech, combined with token-level duration prediction to predict speech tokens and the duration for the next step. Additionally, we design a two-stage training strategy to improve training efficiency and the quality of generated speech. We evaluated the SyncSpeech on both English and Mandarin datasets. Compared to the recent dual-stream TTS models, SyncSpeech significantly reduces the first packet delay of speech tokens and accelerates the real-time factor. Moreover, with the same data scale, SyncSpeech achieves performance comparable to that of traditional autoregressive-based TTS models in terms of both speech quality and robustness. Speech samples are available at \href{https://SyncSpeech.github.io/}{https://SyncSpeech.github.io/}.

\end{abstract}

\section{Introduction}

In recent years, with advancements in generative models and the expansion of training datasets, text-to-speech (TTS) models \cite{valle, voicebox, ns3} have made breakthrough progress in naturalness and quality, gradually approaching the level of real recordings. However, low-latency and efficient dual-stream TTS, which involves processing streaming text inputs while simultaneously generating speech in real time, remains a challenging problem \cite{livespeech2}. These models are ideal for integration with upstream tasks, such as large language models (LLMs) \cite{gpt4} and streaming translation models \cite{seamless}, which can generate text in a streaming manner. Addressing these challenges can improve live human-computer interaction, paving the way for various applications, such as speech-to-speech translation and personal voice assistants.

Recently, inspired by advances in image generation, denoising diffusion \cite{diffusion, score}, flow matching \cite{fm}, and masked generative models \cite{maskgit} have been introduced into non-autoregressive (NAR) TTS \cite{seedtts, F5tts, pflow, maskgct}, demonstrating impressive performance in offline inference.  During this process, these offline TTS models first add noise or apply masking guided by the predicted duration. Subsequently, context from the entire sentence is leveraged to perform temporally-unordered denoising or mask prediction for speech generation. However, this temporally-unordered process hinders their application to streaming speech generation\footnote{
Here, “temporally” refers to the physical time of audio samples, not the iteration step $t \in [0, 1]$ of the above NAR TTS models.}.


When it comes to streaming speech generation, autoregressive (AR) TTS models \cite{valle, ellav} hold a distinct advantage because of their ability to deliver outputs in a temporally-ordered manner. However, compared to recently proposed NAR TTS models,  AR TTS models have a distinct disadvantage in terms of generation efficiency \cite{MEDUSA}. Specifically, the autoregressive steps are tied to the frame rate of speech tokens, resulting in slower inference speeds.  
While advancements like VALL-E 2 \cite{valle2} have boosted generation efficiency through group code modeling, the challenge remains that the manually set group size is typically small, suggesting room for further improvements. In addition,  most current AR TTS models \cite{dualsteam1} cannot handle stream text input and they only begin streaming speech generation after receiving the complete text,  ignoring the latency caused by the streaming text input. The most closely related works to SyncSpeech are CosyVoice2 \cite{cosyvoice2.0} and IST-LM \cite{yang2024interleaved}, both of which employ interleaved speech-text modeling to accommodate dual-stream scenarios. However, their autoregressive process generates only one speech token per step, leading to low efficiency.



To seamlessly integrate with  upstream LLMs and facilitate dual-stream speech synthesis, this paper introduces \textbf{SyncSpeech}, designed to keep the generation of streaming speech in synchronization with the incoming streaming text. SyncSpeech has the following advantages: 1) \textbf{low latency}, which means it begins generating speech in a streaming manner as soon as the second text token is received,
and
2) \textbf{high efficiency}, 
which means for each arriving text token, only one decoding step is required to generate all the corresponding speech tokens.

SyncSpeech is based on the proposed \textbf{T}emporal \textbf{M}asked generative \textbf{T}ransformer (TMT).
During inference, SyncSpeech adopts the Byte Pair Encoding (BPE) token-level duration prediction, which can access the previously generated speech tokens and performs top-k sampling. 
Subsequently, mask padding and greedy sampling are carried out based on  the duration prediction from the previous step. 

Moreover, sequence input is meticulously constructed to incorporate duration prediction and mask prediction into a single decoding step.
During the training process, we adopt a two-stage training strategy to improve training efficiency and model performance. First, high-efficiency masked pretraining is employed to establish a rough alignment between text and speech tokens within the sequence, followed by fine-tuning the pre-trained model to align with the inference process.

Our experimental results demonstrate that, in terms of generation efficiency, SyncSpeech operates at 6.4 times the speed of the current dual-stream TTS model for English and at 8.5 times the speed for Mandarin. When integrated with LLMs, SyncSpeech achieves latency reductions of 3.2 and 3.8 times, respectively, compared to the current dual-stream TTS model for both languages.
Moreover, with the same scale of training data, SyncSpeech performs comparably to traditional AR models in terms of the quality of generated English speech. For Mandarin, SyncSpeech demonstrates superior quality and robustness compared to current dual-stream TTS models. This showcases the potential of  SyncSpeech as a foundational model to integrate with upstream LLMs.



\begin{figure*}
    \centering
    \includegraphics[width=1\linewidth]{SyncSpeech.pdf}
    \caption{An overview of the proposed SyncSpeech,  comprising a text tokenizer, 
 a speech tokenizer, a temporal masked generative transformer and a chunk-aware speech decoder. The figure shows that, with the random number $n=2$ and text look-ahead value  $q=1$, it estimates all speech tokens (from $s_8$ to $s_{12}$) corresponding to the text token $y_2$ and the duration ($l_3$) of the next text token $y_3$ in one decoding step. }
 % This means that speech token  correspond to the text token $y_3$ and duration token $l_{3}$ represents the duration of the speech token for $y_3$}
    \label{fig1}
\end{figure*}

\section{Related Work}

\subsection{Text-to-Speech}
Text-to-Speech, the transformation of text into audible signals understandable by humans, is pivotal for human-computer interaction. TTS systems can
be mainly divided into AR-based and NAR-based categories. For AR-based systems, VALL-E \cite{valle} predicts the first layer of acoustic tokens extracted by EnCodec \cite{encodec} using an AR codec language model, while a NAR model is used to predict the remaining layers. CosyVoice \cite{cosyvoice} employs an AR model to predict supervised semantic representations and combines flow matching to predict acoustic representations. AR-based TTS models, with their in-context learning capability, can generate natural, prosody-diverse speech in a streaming manner. However, AR-based TTS models exhibit shortcomings in generation efficiency. Besides the previously mentioned VALL-E 2 \cite{valle2}, MEDUSA \cite{MEDUSA} and VALL-E R \cite{Valler} introduce speculative decoding \cite{spdeco} and a codec-merging method, respectively, to accelerate autoregressive generation. Nonetheless, the efficiency gains achieved by these approaches remain limited, unable to perform synchronized decoding steps with text tokens.  


For NAR-based TTS models, most previous approaches require speech duration prediction conditioned on the input text, followed by upsampling the text representations to match the acoustic feature length before feeding them into the generation model. Following FastSpeech \cite{fastspeech2}, VoiceBox \cite{voicebox} and NaturalSpeech 2 \cite{ns2} predict phone-level durations using a regression-based approach. NaturalSpeech 3 \cite{ns3} adopts a discrete diffusion model, combining classification loss and duration prompts for duration prediction, which outperforms text-dependent regression-based duration prediction in terms of speech robustness and quality. However, NaturalSpeech 3 requires an additional duration prediction model, which complicates the pipeline, whereas SyncSpeech integrates duration and speech token predictions into a unified framework. The NAR TTS model most relevant to SyncSpeech is MaskGCT \cite{maskgct}, which predicts the total duration of the speech and then performs temporally-unordered multi-step mask prediction. Unlike MaskGCT, SyncSpeech employs temporally-ordered mask prediction and BPE token-level duration prediction to achieve speech generation in a dual-stream scenario.


\subsection{Speech Large Language Models}
Speech Large Language Models (SLLMs) empower LLMs to interact with users through speech, responding to user’s instruction with  low latency \cite{wavchat}.
A basic approach \cite{audiogpt} to achieve this speech interaction involves a cascade of automatic speech recognition (ASR), LLM and TTS models, where the ASR transcribes the users' speech instruction into text, and the TTS model converts the LLM's textual response into speech. 
However, most current AR TTS models  cannot process streaming text input, resulting in significant latency in the aforementioned cascaded systems. 
In contrast, some end-to-end speech-language models have been proposed that can generate speech tokens directly,   thereby achieving extremely low response latency. LLaMA-Omni \cite{llamaomni} aligns the hidden states of LLMs with discrete HuBERT \cite{hubert} representations using CTC loss, but the generated speech exhibits less natural prosody. Mini-Omni \cite{mini1} employs a parallel decoder approach to generate text and speech tokens simultaneously. 
However, due to the significantly longer length of speech tokens compared to text tokens, its generation efficiency remains low. 
The proposed SyncSpeech can process streaming text input and generates speech in synchronization, with the potential to unite with LLMs to become end-to-end SLLMs.  


\subsection{Overview}

The objective of open-set HSI classification is to classify known classes while simultaneously rejecting unknown classes. To achieve this, the proposed \textit{HOpenCls} is designed as a multi-task learning framework, as illustrated in Fig.~\ref{fig:hopencls_framework}. The \textit{HOpenCls} framework is highly flexible and can be implemented by integrating the proposed PU learning classifier with an existing multi-class HSI classifier. The PU learning classifier handles the task of rejecting unknown classes, while the known class classification task is addressed by the existing multi-class HSI classifier.

\begin{figure*}[!t]
    \centering
    \includegraphics[width=0.98\textwidth]{framework.png}
    \caption{The proposed \emph{HOpenCls} framework. This framework can effectively leverage the wild data for open-set HSI classification. This framework includes multi-PU head, gradient contraction (Grad-C) and gradient expansion (Grad-E) PU learning algorithm. The PU learning component handles the rejection of unknown classes, while the classification of known classes is performed using an existing multi-class HSI classifier.}
    \label{fig:hopencls_framework}
\end{figure*}

\noindent \textbf{Problem Settings:}
Let $\mathcal{X}=\{\mathcal{X}_{k}, \mathcal{X}_{u}\} \in \mathbb{R}^{d}$ represent the input space, where $\mathcal{X}_{k}$ and $\mathcal{X}_{u}$ are the input space of known classes and unknown classes, respectively. The output space for the known classes classification task is denoted by $\mathcal{Y}_{k}=\{1,\dots,C\}$, where $C$ is the number of known classes. Additionally, let $\mathcal{Y}_{u}=\{0,1\}$ indicate the output space for the unknown classes rejection task, where 1 corresponds to known classes and 0 corresponds to unknown classes. Suppose the known classes training dataset is denoted as $\mathcal{D}_{k}=\{(\boldsymbol{x}^{i}_{k},y^{i}_{k},y^{i}_{u})\}_{i=1}^{n_{k}}\stackrel{\text{i.i.d}}{\sim}\mathbb{P}_{k}$, where $\boldsymbol{x}^{i}_{k} \in \mathcal{X}_{k}$, $y^{i}_{k} \in \mathcal{Y}_{k}$ is the classification label corresponding to the data $\boldsymbol{x}^{i}_{k}$ of known classes , $y^{i}_{u}=1 \in \mathcal{Y}_{u}$ indicate that $\boldsymbol{x}^{i}_{k}$ belongs to a known class, $n_{k}$ is the number of known classes training samples. The wild training dataset is represented as $\mathcal{D}_{wild}=\{\boldsymbol{x}^{i}_{wild}\}_{i=1}^{n_{wild}}\stackrel{\text{i.i.d}}{\sim}\mathbb{P}_{wild}$, where $\boldsymbol{x}^{i}_{wild} \in \mathcal{X}$, $n_{wild}$ is the number of wild training samples.

Let $q$ and $f$ denote the classifiers for known classes classification and unknown classes rejection, respectively. To reduce computational complexity, a global spectral-spatial feature extractor~\cite{FPGA} is shared by $q$ and $f$. For any test-time data $\boldsymbol{x} \in \mathcal{X}$, the open-set classification result $y$ can be obtained by:
\begin{equation}
    y=Q(\boldsymbol{x}){\otimes}F(\boldsymbol{x}),
\end{equation}
where $\otimes$ is the pointwise multiplication operator. $Q(\boldsymbol{x})$ and $F(\boldsymbol{x})$ are the classification results of $q(\boldsymbol{x})$ and $f(\boldsymbol{x})$, respectively.

\noindent \textbf{Known classes Classification:}
The goal of classifying known classes is to develop a classifier $q$, which can be obtained by minimizing the multi-class classification risk $\mathcal{R}_{k}$:
\begin{equation}
    \mathcal{R}_{k}(q)=\mathbb{E}_{(\boldsymbol{x}_{k},y_{k}){\sim}\mathbb{P}_{k}}\left[\mathcal{L}_{k}(q(\boldsymbol{x}_{k}),y_{k})\right],
    \label{eq:known_classes_risk}
\end{equation}
where $\mathcal{L}_{k}$ is the loss function for multi-class classification. In this paper, for generality, cross entropy (CE) loss ($\mathcal{L}_{ce}$) is used as $\mathcal{L}_{k}$. Eqn.~\ref{eq:known_classes_risk} can be estimated using the empirical averages over the dataset $\mathcal{D}_{k}$:
\begin{equation}
    \hat{\mathcal{R}}_{k}(q)=\frac{1}{n_{k}}\sum_{i=1}^{n_{k}}\mathcal{L}_{ce}(q(\boldsymbol{x}_{k}^{i}),y_{k}^{i}).
    \label{eq:known_classes_average_loss}
\end{equation}

\noindent \textbf{Unknown Classes Rejection:}
The task of rejecting unknown classes involves constructing a binary classifier $f$ that can determine whether a test-time data $\boldsymbol{x}$ belongs to known classes or not. Ideally, the $f$ is obtained by minimizing the unknown classes rejection risk $\mathcal{R}_{u}$:
\begin{equation}
    \mathcal{R}_{u}(f)=\frac{1}{2}\left(\mathcal{R}^{+}_{k}(f)+\mathcal{R}^{-}_{u}(f)\right),
    \label{eq:unknown_classes_risk}
\end{equation}
where $\mathcal{R}^{+}_{k}(f)=\mathbb{E}_{(\boldsymbol{x}_{k},1){\sim}{\mathbb{P}_{k}}}\left[\mathcal{L}_{u}(f(\boldsymbol{x}_{k}),1)\right]$ and $\mathcal{R}^{-}_{u}(f)=\mathbb{E}_{(\boldsymbol{x}_{u},0){\sim}{\mathbb{P}_{u}}}\left[\mathcal{L}_{u}(f(\boldsymbol{x}_{u}),0)\right]$, with $\mathcal{L}_{u}$ representing a binary classification loss function, such as the binary cross entropy (BCE) loss ($\mathcal{L}_{bce}$). $\boldsymbol{x}_{u}$ is the data from unknown classes. However, Eqn.~\ref{eq:unknown_classes_risk} cannot be directly computed in open-set HSI classification due to the absence of training data from unknown classes. Therefore, we focus on designing the multi-PU head, Grad-C module and Grad-E module to reject unknown classes trained by known classes and wild data in the following.

\subsection{Multi-PU Head}

The multi-PU head is designed to introduce the multi-label strategy into the \textit{HOpenCls}. The detailed structure multi-PU head is illustrated in the Fig.~\ref{fig:hopencls_framework}. Based on this design, the original task of unknown classes rejection is decomposed into multiple sub-PU learning tasks: The $c$-th sub-PU learning task is responsible for classifying the $c$-th known class against all other classes. Compared to the original unknown classes rejection task, each sub-PU learning task exhibits reduced intra-class variance of positive class, and the class prior of the $c$-th sub-PU learning task is $\pi_{task}^{c}=\pi_{c}<\pi$, more theoretical analysis about class prior can be found in Theorem ~\ref{theorem}.

In the $c$-th sub-PU learning task, the risk is defined as $\mathcal{R}_{pu}^{c}(f^{c})$:
\begin{equation}
    \mathcal{R}_{pu}^{c}(f^c)=\frac{1}{2}\left({\mathcal{R}^{c}_{k}}^{+}(f^c)+{\mathcal{R}_{wild}}^{-}(f^c)\right),
    \label{eq:cth_class_unknown_classes_risk}
\end{equation}
where ${\mathcal{R}^{c}_{k}}^{+}(f^c)=\mathbb{E}_{(\boldsymbol{x},1){\sim}{\mathbb{P}_{k}^{c}}}\left[\mathcal{L}_{u}(f^{c}(\boldsymbol{x}),1)\right]$, ${\mathcal{R}_{wild}}^{-}(f^c)=\mathbb{E}_{(\boldsymbol{x},0){\sim}{\mathbb{P}_{wild}}}\left[\mathcal{L}_{u}(f^{c}(\boldsymbol{x}),0)\right]$. Here, $\mathbb{P}_{k}^{c}$ represents the margin distribution of the $c$-th known class. The function $f^{c}$ is the binary classifier for the $c$-th sub-PU learning task.

The Eqn.~\ref{eq:cth_class_unknown_classes_risk} can be estimated using the empirical averages over the dataset $\mathcal{D}_{k}$ and the wild dataset $\mathcal{D}_{wild}$:
\begin{equation}
    \begin{aligned}
        \hat{\mathcal{R}}_{pu}^{c}(f^c)=&\frac{1}{2}(\frac{1}{n_{k}^{c}}\sum_{i=1}^{n_{k}^{c}}\mathcal{L}_{u}(f^{c}(\boldsymbol{x}^{ci}_{k}),1)+\\&\frac{1}{n_{wild}}\sum_{i=1}^{n_{wild}}\mathcal{L}_{u}(f^{c}(\boldsymbol{x}^{i}_{wild}),0)),
    \end{aligned}
    \label{eq:cth_class_unknown_classes_average_loss}
\end{equation}
where $\boldsymbol{x}^{ci}_{k}$ is the training sample of known class $c$ from the $\mathcal{D}_{k}$ and $n_{k}^{c}$ is the number of training data of known class $c$. The overall risk for unknown classes rejection with multi-PU head can be estimated as follows:
\begin{equation}
    \hat{\mathcal{R}}_{mpu}(f)=\sum_{c=1}^{C}\hat{\mathcal{R}}_{pu}^{c}(f^{c}).
    \label{eq:unknown_classes_risk_multi_pu_head_average_loss}
\end{equation}

As illustrated in Fig.~\ref{fig:hopencls_framework}, a sample $\boldsymbol{x}$ will be classified as belonging to known classes if any of the sub-PU learning classifiers identify it as known:
\begin{equation}
    F(\boldsymbol{x})=F^{1}(\boldsymbol{x}) \vee \dots \vee F^{c}(\boldsymbol{x}) \vee \dots \vee F^{C}(\boldsymbol{x}),
    \label{eq:unknown_rejection_multi_pu_head}
\end{equation}
where $F^{c}(\boldsymbol{x})$ is the classification result of $f^{c}(\boldsymbol{x})$, the symbol $\vee$ represents the logical OR operation.

\subsection{Grad-C and Grad-E Modules for PU Learning}

The Grad-C and Grad-E modules are designed to reject unknown classes from the perspective of PU learning. First, this paper reveals that the negative impact of replacing a pure unknown dataset by wild data stems from the larger gradient weights associated with wild data. Then, the Grad-C and Grad-E modules are designed to mitigate the adverse effects of these abnormal gradient weights. The comparison of different modules in the aspect of gradient weights is shown in Fig.~\ref{fig:grad}.

\begin{figure}[!t]
    \centering
    \includegraphics[width=0.98\columnwidth]{grad.png}
    \caption{Comparison of different modules against to wild known data. (a) The negative impact of unknown classes data replaced by wild data stems from the larger gradient weights associated with wild known data. (b) The Grad-C module reduces the gradient weights associated with both wild known and unknown data. (c) The Grad-E module restores the gradient weights for the wild unknown data by weighting mechanism.}
    \label{fig:grad}
\end{figure}

\noindent \textbf{Gradient Analysis:}
A commonly used choice for $\mathcal{L}_{u}$ in binary classification is the $\mathcal{L}_{bce}$:
\begin{equation}\nonumber
    \mathcal{L}_{bce}(f(\boldsymbol{x}),y_{u})=-y_{u}\log(f(\boldsymbol{x}))-(1-y_{u})\log(1-f(\boldsymbol{x})).
    \label{eq:binary_cross_entropy}
\end{equation}
The gradient of the $\mathcal{L}_{bce}$ with respect to the parameters $\boldsymbol{\theta}$ of the binary classifier $f$ is given by:
\begin{equation}
    \frac{\partial \mathcal{L}_{bce}(f(\boldsymbol{x}),y_{u})}{\partial \boldsymbol{\theta}}=(\frac{1-y_{u}}{1-f(\boldsymbol{x})}-\frac{y_{u}}{f(\boldsymbol{x})})\nabla_{\boldsymbol{\theta}}f({\boldsymbol{x}}).
    \label{eq:gradient_binary_cross_entropy}
\end{equation}

When the unknown data ($\boldsymbol{x}_{u},0$) is replaced with wild data ($\boldsymbol{x}_{wild},0$), a wild known data will be assigned a larger gradient weight (Eqn.~\ref{eq:gradient_wild_bce}) if it is predicted as known classes with high confidence ($f(\boldsymbol{x}_{wild}) \rightarrow 1$). The $f$ will overfit all wild known data as unknown classes, which disrupts the training process for rejecting unknown classes, as shown in Fig.~\ref{fig:grad}..
\begin{equation}
    \frac{\partial \mathcal{L}_{bce}(f(\boldsymbol{x}_{wild}),0)}{\partial \boldsymbol{\theta}}={\frac{1}{1-f(\boldsymbol{x}_{wild})}}\nabla_{\boldsymbol{\theta}}f(\boldsymbol{x}_{wild}).
\label{eq:gradient_wild_bce}
\end{equation}

\noindent \textbf{Grad-C Module:}
The Grad-C module---Taylor binary cross entropy (TBCE) loss ($\mathcal{L}_{tbce}$)---is proposed to mitigate the larger gradient weights problem, which contracts the gradient weights of all wild data by Taylor series expansion in $\mathcal{L}_{bce}$. Simultaneously, the effectiveness of the $\mathcal{L}_{tbce}$ for rejecting unknown classes can be theoretically proven.

Given a function $t(x)$, if this function is differentiable to order $o$ at $x=x_0$, this function can be expressed as:
\begin{equation}\nonumber
    t(x)=\sum_{o=0}^{\infty}\frac{t^{o}(x_0)}{o!}(x-x_0)^o,
\end{equation}
where $t^{o}(x_0)$ is the $o$-th order derivative of $t(x)$ at $x=x_0$.

Let $t(f(\boldsymbol{x}))=-\log(1-f(\boldsymbol{x}))$ and $f(\boldsymbol{x}_{0})=0$. For $\forall o \geq 1$, the $t(f(\boldsymbol{x}))$ can be expressed as:
\begin{equation}\nonumber
    t(f(\boldsymbol{x}))=\sum_{o=1}^{\infty}\frac{f(\boldsymbol{x})^o}{o}.
\end{equation}
Then, the $\mathcal{L}_{tbce}$ is formalized as:
\begin{equation}
    \mathcal{L}_{tbce}(f(\boldsymbol{x}),y_{u})=-y_{u}\log(f(\boldsymbol{x}))+(1-y_{u})\sum_{o=1}^{t}\frac{{f(\boldsymbol{x})}^{o}}{o},
    \label{eq:taylot_binary_cross_entropy}
\end{equation}
where $t \in \mathcal{N}^+$ is the truncation order of the Taylor series expansion.

The gradient of $\mathcal{L}_{tbce}$ with respect to $\boldsymbol{\theta}$ for a wild sample ($\boldsymbol{x}_{wild},0$) is:
\begin{equation}
    \frac{\partial \mathcal{L}_{tbce}(f(\boldsymbol{x}_{wild}),0)}{\partial \boldsymbol{\theta}}={\frac{1-f(\boldsymbol{x}_{wild})^{t}}{1-f(\boldsymbol{x}_{wild})}}\nabla_{\boldsymbol{\theta}}f(\boldsymbol{x}_{wild}).
    \label{eq:gradient_wild_tbce}
\end{equation}
From Eqn.\ref{eq:gradient_wild_bce} and Eqn.\ref{eq:gradient_wild_tbce}, the gradient weight of a wild data in $\mathcal{L}_{tbce}$ is lower than that in $\mathcal{L}_{bcc}$, effectively mitigating the problem of larger gradient weights:
\begin{equation}
    {\frac{1-f(\boldsymbol{x}_{wild})^{t}}{1-f(\boldsymbol{x}_{wild})}} < {\frac{1}{1-f(\boldsymbol{x}_{wild})}}.
\end{equation}

Let $\mathcal{R}_{pu}(f)$ denote the risk where the unknown data is replaced by the wild data in $\mathcal{R}_{u}(f)$:
\begin{equation}
    \mathcal{R}_{pu}(f)=\frac{1}{2}\left(\mathcal{R}^{+}_{k}(f)+\mathcal{R}^{-}_{wild}(f)\right),
    \label{eq:unknown_classes_risk_pu}
\end{equation}
where $\mathcal{R}^{-}_{wild}(f)=\mathbb{E}_{(\boldsymbol{x}_{wild},0){\sim}{\mathbb{P}_{wild}}}\left[\mathcal{L}_{u}(f(\boldsymbol{x}_{wild}),0)\right]$. Let $\hat{f}$ and $f^{*}$ be the global minimizers of $\mathcal{R}_{pu}(f)$ and $\mathcal{R}_{u}(f)$, respectively. The theoretical property of $\mathcal{L}_{tbce}$ is stated in Theorem~\ref{theorem}, which demonstrates the reliability of estimating the rejection risk using wild data and $\mathcal{L}_{tbce}$. A detailed proof is provided in Appendix.
\begin{theorem}\label{theorem}
    The $\mathcal{R}_{u}(\hat{f})-\mathcal{R}_{u}(f^{*})$ and $\mathcal{R}_{pu}(f^{*})-\mathcal{R}_{pu}(\hat{f})$ are bounded when $\mathcal{L}_{tbce}$ is used as the loss function:
    \begin{equation}
        \begin{aligned}
            0 \leq {\mathcal{R}_{u}(\hat{f})-\mathcal{R}_{u}(f^*)} \leq {\pi\mathcal{N}_t},
        \end{aligned}
    \end{equation}
    \begin{equation}
        \begin{aligned}
            0 \leq {\mathcal{R}_{pu}(f^*)-\mathcal{R}_{pu}(\hat{f})} \leq {\pi\mathcal{N}_t},
        \end{aligned}
    \end{equation}
    where $\mathcal{N}_{t}={\sum_{o=1}^{t}}\frac{1}{o}$ is a constant releated to the truncation order $t$.
\end{theorem}

Based on this theoretical analysis, the performance of $\mathcal{R}_{pu}$ can be further improved by reducing both $\mathcal{N}_{t}$ and ${\pi}$. A lower $\mathcal{N}_{t}$ can be obtained by decreasing the truncation order $t$. Although the $\pi$ is a fixed constant for a given open-set HSI classification task, the original task can be decoupled into multiple sub-PU learning tasks via the proposed multi-PU head, and each sub-task would have a lower class prior.

\noindent \textbf{Grad-E Module:}
As previously mentioned, the Grad-C module reduces the gradient weights associated with both wild known and unknown samples, which would lead to $f$ underfitting the wild unknown data. To address this issue, as illustrated in Fig.~\ref{fig:grad}, the Grad-E module is designed to restore the gradient weights for the wild unknown data by weighting mechanism.

As illustrated in Fig.~\ref{fig:hopencls_framework}, two deep neural networks are used to contract and expand the gradient weights of wild data, respectively. In the Grad-E module, the $\mathcal{L}_{bce}$ is utilized to restore the gradient weights of wild unknown data with the confidence scores output by the network trained with Grad-C module. Moreover, the performance can be further enhanced by incorporating the confidence scores produced by the network trained with Grad-E module into the $\mathcal{L}_{tbce}$.

The weighted binary cross entropy (\textit{w}BCE) loss ($\mathcal{L}_{bce}^{w}$) is defined as follows:
\begin{equation}\nonumber
    \mathcal{L}^{w}_{bce}(f(\boldsymbol{x}),y_{u})=-y_{u}\log(f(\boldsymbol{x}))-(1-y_{u}){w_{e}}\log(1-f(\boldsymbol{x})),
    \label{eq:weight_binary_cross_entropy}
\end{equation}
where $w_{e}$ is the confidence score of data $\boldsymbol{x}$ output by the network trained with Grad-C module.
Similarly, the weighted Taylor binary cross entropy (\textit{w}TBCE) loss ($\mathcal{L}^{w}_{tbce}$) can be formulated as:
\begin{equation}\nonumber
    \mathcal{L}^{w}_{tbce}(f(\boldsymbol{x}),y_{u})=-y_{u}\log(f(\boldsymbol{x}))+(1-y_{u}){w_{c}}\sum_{o=1}^{t}\frac{{f(\boldsymbol{x})}^{o}}{o},
    \label{eq:weight_taylor_binary_cross_entropy}
\end{equation}
where $w_{c}$ is the confidence of data $\boldsymbol{x}$ output by the network trained with Grad-E module. The wild unknown samples are expected to receive higher confidence scores, with a maximum value of $1$. The following describes the way for obtaining the confidence scores, which involves both confidence score updates and probability mixing.

In the process of updating confidence scores, both $w_{c}$ and $w_{e}$ are optimized during model training, with an exponential moving average applied to stabilize the training. Two strategies are proposed for updating the confidence scores: \textbf{continuous updating} (Eqn.~\ref{eq:continuous_updating}) and \textbf{discrete updating} (Eqn.~\ref{eq:discrete_updating}):
\begin{equation}
    \begin{aligned}
    w_{c}={\alpha}w_{c}+(1-{\alpha})p_{e}\\
    w_{e}={\alpha}w_{e}+(1-{\alpha})p_{c},
    \end{aligned}
    \label{eq:continuous_updating}
\end{equation}
\begin{equation}
    \begin{aligned}
        &w_{c}={\alpha}w_{c}+(1-{\alpha})\mathbb{I}(p_{e} \geq \tau)\\
        &w_{e}={\alpha}w_{e}+(1-{\alpha})\mathbb{I}(p_{c} \geq \tau),
    \end{aligned}
    \label{eq:discrete_updating}
\end{equation}
where $\mathbb{I}(\bullet)$ is the indicator function, the initial values of $w_{c}$ and $w_{e}$ are both $1$, and $\tau$ is set to $0.95$. More experiments about $\tau$ are discussed in the experiment section. Experiments demonstrate that $w_{e}$ is more suitable for discrete updating and $w_{c}$ is better suited for continuous updating.

A straightforward way to compute $p_{c}$ and $p_{e}$ is by directly using the probability outputs from $f$ (\textbf{Pro}):
\begin{equation}
    \begin{aligned}
        p_{c}=1-f_{c}(\boldsymbol{x})\\
        p_{e}=1-f_{e}(\boldsymbol{x}),
    \end{aligned}
    \label{eq:pro}
\end{equation}
where $f_{c}(\boldsymbol{x})$ and $f_{e}(\boldsymbol{x})$ are the binary classifiers trained with $\mathcal{L}_{tbce}^{w}$ and $\mathcal{L}_{bce}^{w}$, respectively. However, this simplistic approach ignores the consistency between $q$ and $f$ (with multi-PU head): data from known classes should exhibit high probability outputs in both classifiers simultaneously. Therefore, a probability mixture strategy is proposed to incorporate the ability of classifying known classes of multi-class classifier into the multiple unknown rejection classifiers (\textbf{MixPro}):
\begin{equation}
    \begin{aligned}
        &p_{c}=1-(q_{c}^{c}(\boldsymbol{x}){\times}f_{c}^{c}(\boldsymbol{x}))\\
        &p_{e}=1-(q_{e}^{c}(\boldsymbol{x}){\times}f_{e}^{c}(\boldsymbol{x})),
    \end{aligned}
    \label{eq:mixpro}
\end{equation}
where $q_{\bullet}^{c}(\boldsymbol{x})$ is the probability output of the $c$-th class from the known classes classifier, $f_{\bullet}^{c}(\boldsymbol{x})$ is the probability output of the $c$-th sub-PU task. $q_{c}$ and $f_{c}$ share the same feature extractor, while $q_{e}$ and $f_{e}$ also share the same feature extractor. Moreover, the outputs of the $f_{c}^{c}(\boldsymbol{x})$ and $f_{e}^{c}(\boldsymbol{x})$ are aligned using KL divergence in the unknown rejection task, enabling the two networks to act as mutual teachers~\cite{T-HOneCls}.

\subsection{Overall Risk}

The risks for the two networks are defined as follows:
\begin{equation}
    \begin{aligned}
        &\hat{\mathcal{R}}_{c}(q_{c},f_{c})=\hat{\mathcal{R}}_{k}(q_{c})+\hat{\mathcal{R}}_{mpu}(f_{c},\mathcal{L}^{w}_{tbce})\\
        &\hat{\mathcal{R}}_{e}(q_{e},f_{e})=\hat{\mathcal{R}}_{k}(q_{e})+\hat{\mathcal{R}}_{mpu}(f_{e},\mathcal{L}^{w}_{bce}),
    \end{aligned}
\end{equation}
where $\hat{\mathcal{R}}_{mpu}(\bullet,\mathcal{L}^{w}_{tbce})$ and $\hat{\mathcal{R}}_{mpu}(\bullet,\mathcal{L}^{w}_{bce})$ represent that the loss function $\mathcal{L}^{w}_{tbce}$ and $\mathcal{L}^{w}_{bce}$ are used in Eqn.~\ref{eq:unknown_classes_risk_multi_pu_head_average_loss}, respectively.

The overall risk for the \textit{HOpnCls} framework can be formulated as follows:
\begin{equation}\nonumber
    \hat{\mathcal{R}}_{all}(q_{c},f_{c},q_{e},f_{e}) = \hat{\mathcal{R}}_{c}(q_{c},f_{c})+\hat{\mathcal{R}}_{e}(q_{e},f_{e})+\beta\hat{\mathcal{R}}_{kl}(f_{c},f_{e}),
\end{equation}
where the risk from KL divergence is denoted as $\hat{\mathcal{R}}_{kl}(f_{c},f_{e})$ and $\beta$ is a hyperparameter.
\section{Experiments}


\subsection{Experimental Settings}
\paragraph{Datasets} We trained SyncSpeech on datasets in both English and Mandarin, including the 585-hour LibriTTS \cite{libritts} dataset and 600 hours of internal Mandarin datasets. The internal Mandarin dataset was further expanded to approximately 2000 hours, employing techniques such as speed alteration and pitch shifting. The Montreal Forced Aligner (MFA) \cite{mfa}  aligned transcripts according to its phone set, after which the alignment was transformed into text BPE-level format. We evaluated SyncSpeech using three benchmarks: (1) LibriSpeech \textit{text-clean} \cite{librispeech}, a standard English TTS evaluation set; (2) SeedTTS \textit{test-zh} \cite{seedtts}, with 2,000 samples from the out-of-domain Mandarin DiDiSpeech dataset \cite{didispeech}; and (3) SeedTTS \textit{test-hard}, containing approximately 400 difficult cases to evaluate TTS model robustness with repeated text, tongue twisters, and other complex synthesis scenarios. 

\paragraph{Settings} 
We set the number of text tokens to look ahead $q=1$. The chunk size of speech decoder is 15. 
TMT has 16 layers, 16 attention heads, 1024-dimensional
embeddings, and 2048-dimensional feed-forward layers. 
SyncSpeech was trained on 4 NVIDIA A800 80G GPUs. 
The pre-training stage lasts for 70K steps, and the second stage lasts for 20K steps. 

\paragraph{Baseline Models}
This paper focuses on low-latency and efficient TTS in dual-stream scenarios. Under the same data scale, we reproduced the following baseline models for comparison: CosyVoice \cite{cosyvoice} and recently proposed CosyVoice2 \cite{cosyvoice2.0}. CosyVoice requires complete text input before speech generation. 
CosyVoice2 uses interleaved text-speech modeling to process streaming text input and simultaneously generate streaming speech. We trained CosyVoice, CosyVoice2, and SyncSpeech using the same speech tokenizer and text tokenizer, and employed the same open-source streaming speech decoder. We utilized the official code\footnote{https://github.com/FunAudioLLM/CosyVoice} to reproduce the model and adopted a Llama-style Transformer, matching the size of SyncSpeech, as the backbone of the text-to-speech model.  Additionally, we compared the open-sourced TTS model MaskGCT \cite{maskgct}, F5-TTS \cite{F5tts}, and VALL-E \cite{valle}, which were trained on large-scale data.
 More details about baseline models can be found in the Appendix \ref{baselines}.


\paragraph{Evaluation Metrics} For the three benchmarks, we evaluated
speech quality, latency, and  efficiency. 
For speech robustness, we chose Whisper-V3 and Paraformer as the ASR models for English and Mandarin, respectively, to transcribe the generated speech. Then, we calculated the WER compared to the original transcriptions to evaluate the spech robustness. We adopted the ERes2Net-based \cite{eres2net} speaker verification model\footnote{https://github.com/modelscope/3D-Speaker} to evaluate speaker similarity (SS). We selected 100 sentences from each system and invited 10 native listeners to conduct a subjective MOS evaluation for speech naturalness (MOS-N), scoring from 1 to 5. 
In terms of latency and efficiency, we compared the performance of various models on a single A800 GPU. 
Due to the off-the-shelf speech decoder, we evaluate the latency and efficiency of the text-to-token stage across all models, except for F5-TTS.
We calculated the time required for the number of speech tokens to reach the chunk size of the speech decoder as First-packet latency (FPL). There are two scenarios: one assumes the text is already available (FPL-A), while the other involves receiving output from the upstream LLM model (FPL-L), accounting for the time required for text generation.
 For the real-time factor (RTF), we measure the ratio of the total duration of generated speech to the total time taken by the model. More details about FPL and RTF can be found in the Appendix \ref{evaluation metrics}.

\begin{table*}[t]
\centering
\resizebox{0.99\textwidth}{!}{
\begin{tabular}{lccccccccc}
\toprule
\textbf{Model} & \textbf{\#Scenario}  & \textbf{\#Data(hrs)}   & \textbf{WER(\%)} $\downarrow$   & \textbf{SS(\%)} $\uparrow$ & \textbf{FPL-A(s)}$\downarrow$ & \textbf{FPL-L(s)} $\downarrow$ & \textbf{RTF(\%)} $\downarrow$ & \textbf{MOS-N} $\uparrow$ \\ \hline
\multicolumn{10}{c}{\textbf{LibriSpeech \textit{test-clean}}}   \\ \hline
\textbf{Ground Truth} &- &-  & 2.12   & 69.67 &- &- &-   &$\text{4.62}_{\pm 0.12}$       \\ \hdashline
\textbf{F5-TTS*} & Offline & 100K Multi.  & \textbf{2.51} & \textbf{73.10} &1.27 &1.98 &0.23  &-  \\
\textbf{MASK-GCT*} & Offline &100K Multi. &2.77 & 70.81 &2.15 &2.55 & 0.37 &- \\
\textbf{VALL-E*}  & Output Stream & 60K EN & 5.90 & 59.71 & 0.75 &1.47 &1.41 &- \\
\textbf{CosyVoice} & Output Stream & 585 EN  & 3.47  & \underline{63.52} &0.22 &0.94 &0.45   & $\text{4.39}_{\pm 0.12}$           \\ 
\textbf{CosyVoice2} & Dual-Stream & 585 EN   & \underline{3.00}      & 63.48 &0.22 &0.35 &0.45   &$\textbf{\text{4.48}}_{\pm 0.13}$           \\
\textbf{SyncSpeech} & Dual-Stream & 585 EN    & 3.07    & 63.47 & \textbf{0.06} &\textbf{0.11} &\textbf{0.07}   &$\textbf{\text{4.48}}_{\pm 0.14}$         \\ \hline
\multicolumn{10}{c}{\textbf{Seed \textit{test-zh}}}   \\ \hline
\textbf{Ground Truth} &-  &- & 1.26  & 75.15  &- &- &- & $\text{4.68}_{\pm 0.10}$      \\ \hdashline
\textbf{CosyVoice}  & Output Stream  &2K ZH      & 3.03    & 61.51  &0.22 &0.62 &0.43  & $\text{4.34}_{\pm 0.14}$            \\ 
\textbf{CosyVoice2} & Dual-Stream  &2K ZH & 3.31      & 61.89   &0.22 &0.35 &0.43 & $\text{4.37}_{\pm 0.13}$           \\
\textbf{SyncSpeech}& Dual-Stream  &2K ZH  & \textbf{2.38}    & \textbf{62.14}   & \textbf{0.04} & \textbf{0.09} &\textbf{0.05} &$\textbf{\text{4.45}}_{\pm 0.11}$           \\
\hline
\multicolumn{10}{c}{\textbf{Seed \textit{test-hard}}}   \\ \hline
\textbf{CosyVoice} & Output Stream  &2K ZH    & 26.26    & 66.71   &0.22 &1.22 &0.44 & $\text{3.84}_{\pm 0.15}$            \\ 
\textbf{CosyVoice2} & Dual-Stream  &2K ZH  & 21.61  & 67.13 &0.22 &0.35 &0.44      &$\text{3.86}_{\pm 0.14}$            \\
\textbf{SyncSpeech} & Dual-Stream &2K ZH  & \textbf{17.21}    & \textbf{67.21}  &\textbf{0.05} & \textbf{0.10} &\textbf{0.08} &$\text{3.86}_{\pm 0.11}$         \\
\bottomrule
\end{tabular}
}
\caption{The evaluation results of SyncSpeech and baseline models across the three benchmarks. * indicates the model trained on the large-scale dataset. Underline indicates the best performance in terms of WER and SS with the 585 hours training scale. \#Data refers to the used training dataset in hours.}
\label{table1}
\end{table*}


\subsection{Main Results}
The evaluation results for SyncSpeech and the baseline models are presented in Table \ref{table1}. 

\paragraph{Speech Robustness} 
We found that SyncSpeech exhibits different performance compared to the baselines across the three benchmarks. Specifically, on the LibriSpeech \textit{test-clean} benchmark, the performance of SyncSpeech was very close to that of CosyVoice2 based on the WER metric, with only a minor difference of 0.07\%. SyncSpeech achieved a lower WER score on the Seed \textit{test-zh} set compared to CosyVoice and CosyVoice2, with improvements of 0.65\% and 0.93\%, respectively.  A key difference between the English and Mandarin datasets is the higher compression rate of the LLM tokenizer for Mandarin. In English, one word typically equals one token, while in Mandarin, a common phrase often corresponds to a single token.
This means that, compared to the baseline model, SyncSpeech is better suited to the high compression rate tokenizer of the upstream large model. Furthermore, on the Seed \textit{test-hard} set, the robustness advantage of SyncSpeech was even more pronounced, with the improvements 9.05\% and 4.40\%, respectively. In handling complex text, the explicit duration modeling in SyncSpeech helped the model learn the alignment between text and speech.

\paragraph{Speaker Similarity} Due to the same speech decoder and the excellent voice disentanglement capability of the speech tokens, SyncSpeech, CosyVoice, and CosyVoice2 exhibited similar performance in terms of speaker similarity.
\paragraph{Speech Naturalness} The MOS-N scores for SyncSpeech and CosyVoice2 were quite similar on the LibriSpeech \textit{text-clean}, indicating that the naturalness of the generated speech was generally comparable. On the Seed \textit{test-zh} benchmark, SyncSpeech outperformed CosyVoice2 by 0.08.  In the Seed \textit{test-hard} benchmark, high WER and uncommon text led to unnatural prosody and generally low MOS-N scores in the generated speech.
\paragraph{Latency} SyncSpeech has made a breakthrough in terms of latency, as shown in Table \ref{table1}. Specifically, on the LibriSpeech \textit{test-clean} benchmark, SyncSpeech was approximately 4 times faster than traditional AR models and over 20 times faster than the SOTA offline models in terms of FPL-A. On the Seed \textit{test-zh} benchmark, SyncSpeech achieved speed improvements of over 5 times and 30 times, respectively. When receiving streaming text from the upstream large model (FPL-L), SyncSpeech can begin generating speech with just two text tokens. In contrast, CosyVoice2 requires five tokens, while CosyVoice and other baseline models need the entire text input. This highlights the distinct advantage of SyncSpeech in practical applications.

\paragraph{Efficiency} In terms of RTF, SyncSpeech is about 6.4 times faster on the LibriSpeech \textit{test-clean} benchmark and about 8.6 times faster on the Seed \textit{test-zh} benchmark compared to previous AR models. 
On the Seed \textit{test-hard} set, due to the increased number of text tokens caused by the uncommon text, the efficiency of SyncSpeech is slightly reduced. Theoretically, the time complexity of AR models is $O(T)$, while the time complexity of SyncSpeech is  $O(L)$, where  $T$ represents the number of speech tokens and 
$L$ denotes the number of text tokens, thereby significantly improving efficiency.

\section{Analysis}
\paragraph{Sampling Strategy} In the LibriSpeech validation set, we provided the ground-truth durations and applied greedy search along with different Top-k thresholds for duration prediction, as shown in Table \ref{table3}. We found that, in terms of speech robustness, both Top-k 3 and greedy search outperformed the use of ground-truth durations in terms of the WER metric. This is because the model struggled to effectively generalize to anomalies in the ground-truth durations. We employed
UTMOSv2\footnote{https://github.com/sarulab-speech/UTMOS22} as a surrogate objective metric of MOS-N. In terms of speech naturalness, the results of Top-k 3 sampling are slightly better than those with the given ground-truth durations.  Additionally, we applied different Top-k thresholds for speech token prediction. SyncSpeech exhibited superior performance during greedy search, which is different from the previous AR TTS models or offline models. This is because the speech tokens obtained through single-step decoding have the temporal dependency, which cannot be compensated by subsequent generation. 


\begin{table}[]
\centering
\resizebox{0.42\textwidth}{!}{
\begin{tabular}{lcc}

\toprule
\textbf{Sampling Strategy}       & \textbf{WER(\%)}$\downarrow$  & \textbf{UTMOSv2}$\uparrow$ \\ \hline
\multicolumn{3}{c}{Duration Prediction} \\
\hline
Ground Truth            & 2.59 & 3.45   \\ \hdashline 
Greedy Search   & 2.50 & 3.44   \\
Top-k 3         & \textbf{2.44} & \textbf{3.46}   \\
Top-k 5         & 2.93 & 3.44   \\
Top-k 10        & 2.76 & 3.41  \\ 
\hline
\multicolumn{3}{c}{Speech Token Prediction} \\
\hline
Greedy Search & \textbf{2.44} & \textbf{3.46}   \\
Top-k 3          & 3.82 & 3.43  \\
Top-k 5          & 4.23 & 3.43   \\ 
\bottomrule
\end{tabular}
}
\caption{Performance across various Top-k thresholds for duration prediction and speech token prediction on the LibriTTS validation set.}
\label{table3}
\end{table}

\paragraph{Number of Look-ahead Tokens}
We evaluated how varying the number of tokens to look ahead affects speech robustness and speech naturalness on two validation sets, with the results presented in Table \ref{table5}. We discovered that the optimal number of look-ahead text tokens varies across different languages in terms of WER performance.  This is influenced by the difference in the compression rate of text tokens and the contextual dependency in different languages. In terms of speech naturalness, when the look-ahead number $q$ is greater than $2$, the generated speech exhibits slightly more natural pauses and speed, but it results in increased latency.

\paragraph{Ablation Study}
We conducted an ablation study on the pre-training strategy by directly training the randomly initialized model in a manner consistent with the prediction process. The WER results on the two validation sets are shown in Table \ref{table6}. We found that pre-training significantly improved the speech robustness of the model, improving the WER metric by 1.17\% and 1.06\% on the two languages, respectively. This indicated that masked pre-training not only improved training efficiency but also enhanced the robustness of the synthesized speech. Additionally, a standard causal attention mask was applied to replace the designed attention mask, as shown in Table \ref{table6}. If the mask token sequence of the same text token cannot attend to each other during inference, the robustness of the generated speech significantly decreased. This further demonstrated the effectiveness of the designed attention mask.



\begin{table}[]
\centering
\resizebox{0.49\textwidth}{!}{
\begin{tabular}{lcccc}
\toprule
                    & \textbf{LH Num.}           & \textbf{WER(\%)}$\downarrow$  & \textbf{FPL-L(s)}$\downarrow$  & \textbf{UTMOS-v2}$\uparrow$  \\ \hline
\multirow{4}{*}{EN} & q=1    & \textbf{2.44}    & \textbf{0.11}    & 3.46     \\
                    & q=2    & 2.87    & 0.13   & 3.41       \\
                    & q=3    & 2.52   & 0.16    & \textbf{3.48}         \\ 
                    & q=4    & 2.52    & 0.19    &  \textbf{3.48}     \\          \hline
\multirow{4}{*}{ZH} & q=1   & 2.51 & \textbf{0.09} & -    \\
                    & q=2   & 2.49 & 0.12  & -   \\ 
                    & q=3   & \textbf{2.41} &0.14 & - \\
                    & q=4   & \textbf{2.41} &0.17 & - \\
\bottomrule
                    
\end{tabular}
}
\caption{Performance with different numbers of look-ahead text tokens across two validation sets.}
\label{table5}
\end{table}




\begin{table}[]
\centering
\resizebox{0.39\textwidth}{!}{
\begin{tabular}{lcc}

\toprule
      & \textbf{English}   &\textbf{Mandarin}   \\ \hline
SyncSpeech         & \textbf{2.44} & \textbf{2.41}   \\
w/o pretrain        &  3.61 & 3.47  \\
w/o designed Mask  & 8.19 & 7.97 \\ 
\bottomrule 
\end{tabular}
}
\caption{WER (\%) results of the ablation study across the two validation sets. }
\label{table6}
\end{table}



\section{Conclusion}


In this work, we introduced \ours, a pivot-based single model ensemble framework, to enhance translation in scenarios where parallel data are scarce.
By transferring knowledge from diverse pivot languages, we were able to obtain not only diverse but also high-quality candidates.
And the optimal path to generating the best candidate varies per sentence, our study underscores the significance of exploiting a spectrum of pivot languages.
Moreover, the single model generation process offers cost savings compared to multi-model ensemble approaches. 
Empirical results and qualitative analyses show that the proposed method can yield contextually suitable translations for the given source sentences by leveraging pivoted candidates.

% \section{Engines}

% To produce a PDF file, pdf\LaTeX{} is strongly recommended (over original \LaTeX{} plus dvips+ps2pdf or dvipdf). Xe\LaTeX{} also produces PDF files, and is especially suitable for text in non-Latin scripts.

% \section{Preamble}

% The first line of the file must be
% \begin{quote}
% \begin{verbatim}
% \documentclass[11pt]{article}
% \end{verbatim}
% \end{quote}

% To load the style file in the review version:
% \begin{quote}
% \begin{verbatim}
% \usepackage[review]{acl}
% \end{verbatim}
% \end{quote}
% For the final version, omit the \verb|review| option:
% \begin{quote}
% \begin{verbatim}
% \usepackage{acl}
% \end{verbatim}
% \end{quote}

% To use Times Roman, put the following in the preamble:
% \begin{quote}
% \begin{verbatim}
% \usepackage{times}
% \end{verbatim}
% \end{quote}
% (Alternatives like txfonts or newtx are also acceptable.)

% Please see the \LaTeX{} source of this document for comments on other packages that may be useful.

% Set the title and author using \verb|\title| and \verb|\author|. Within the author list, format multiple authors using \verb|\and| and \verb|\And| and \verb|\AND|; please see the \LaTeX{} source for examples.

% By default, the box containing the title and author names is set to the minimum of 5 cm. If you need more space, include the following in the preamble:
% \begin{quote}
% \begin{verbatim}
% \setlength\titlebox{<dim>}
% \end{verbatim}
% \end{quote}
% where \verb|<dim>| is replaced with a length. Do not set this length smaller than 5 cm.

% \section{Document Body}

% \subsection{Footnotes}

% Footnotes are inserted with the \verb|\footnote| command.\footnote{This is a footnote.}

% \subsection{Tables and figures}

% See Table~\ref{tab:accents} for an example of a table and its caption.
% \textbf{Do not override the default caption sizes.}

% \begin{table}
%   \centering
%   \begin{tabular}{lc}
%     \hline
%     \textbf{Command} & \textbf{Output} \\
%     \hline
%     \verb|{\"a}|     & {\"a}           \\
%     \verb|{\^e}|     & {\^e}           \\
%     \verb|{\`i}|     & {\`i}           \\
%     \verb|{\.I}|     & {\.I}           \\
%     \verb|{\o}|      & {\o}            \\
%     \verb|{\'u}|     & {\'u}           \\
%     \verb|{\aa}|     & {\aa}           \\\hline
%   \end{tabular}
%   \begin{tabular}{lc}
%     \hline
%     \textbf{Command} & \textbf{Output} \\
%     \hline
%     \verb|{\c c}|    & {\c c}          \\
%     \verb|{\u g}|    & {\u g}          \\
%     \verb|{\l}|      & {\l}            \\
%     \verb|{\~n}|     & {\~n}           \\
%     \verb|{\H o}|    & {\H o}          \\
%     \verb|{\v r}|    & {\v r}          \\
%     \verb|{\ss}|     & {\ss}           \\
%     \hline
%   \end{tabular}
%   \caption{Example commands for accented characters, to be used in, \emph{e.g.}, Bib\TeX{} entries.}
%   \label{tab:accents}
% \end{table}

% As much as possible, fonts in figures should conform
% to the document fonts. See Figure~\ref{fig:experiments} for an example of a figure and its caption.

% Using the \verb|graphicx| package graphics files can be included within figure
% environment at an appropriate point within the text.
% The \verb|graphicx| package supports various optional arguments to control the
% appearance of the figure.
% You must include it explicitly in the \LaTeX{} preamble (after the
% \verb|\documentclass| declaration and before \verb|\begin{document}|) using
% \verb|\usepackage{graphicx}|.

% \begin{figure}[t]
%   \includegraphics[width=\columnwidth]{example-image-golden}
%   \caption{A figure with a caption that runs for more than one line.
%     Example image is usually available through the \texttt{mwe} package
%     without even mentioning it in the preamble.}
%   \label{fig:experiments}
% \end{figure}

% \begin{figure*}[t]
%   \includegraphics[width=0.48\linewidth]{example-image-a} \hfill
%   \includegraphics[width=0.48\linewidth]{example-image-b}
%   \caption {A minimal working example to demonstrate how to place
%     two images side-by-side.}
% \end{figure*}

% \subsection{Hyperlinks}

% Users of older versions of \LaTeX{} may encounter the following error during compilation:
% \begin{quote}
% \verb|\pdfendlink| ended up in different nesting level than \verb|\pdfstartlink|.
% \end{quote}
% This happens when pdf\LaTeX{} is used and a citation splits across a page boundary. The best way to fix this is to upgrade \LaTeX{} to 2018-12-01 or later.

% \subsection{Citations}

% \begin{table*}
%   \centering
%   \begin{tabular}{lll}
%     \hline
%     \textbf{Output}           & \textbf{natbib command} & \textbf{ACL only command} \\
%     \hline
%     \citep{Gusfield:97}       & \verb|\citep|           &                           \\
%     \citealp{Gusfield:97}     & \verb|\citealp|         &                           \\
%     \citet{Gusfield:97}       & \verb|\citet|           &                           \\
%     \citeyearpar{Gusfield:97} & \verb|\citeyearpar|     &                           \\
%     \citeposs{Gusfield:97}    &                         & \verb|\citeposs|          \\
%     \hline
%   \end{tabular}
%   \caption{\label{citation-guide}
%     Citation commands supported by the style file.
%     The style is based on the natbib package and supports all natbib citation commands.
%     It also supports commands defined in previous ACL style files for compatibility.
%   }
% \end{table*}

% Table~\ref{citation-guide} shows the syntax supported by the style files.
% We encourage you to use the natbib styles.
% You can use the command \verb|\citet| (cite in text) to get ``author (year)'' citations, like this citation to a paper by \citet{Gusfield:97}.
% You can use the command \verb|\citep| (cite in parentheses) to get ``(author, year)'' citations \citep{Gusfield:97}.
% You can use the command \verb|\citealp| (alternative cite without parentheses) to get ``author, year'' citations, which is useful for using citations within parentheses (e.g. \citealp{Gusfield:97}).

% A possessive citation can be made with the command \verb|\citeposs|.
% This is not a standard natbib command, so it is generally not compatible
% with other style files.

% \subsection{References}

% \nocite{Ando2005,andrew2007scalable,rasooli-tetrault-2015}

% The \LaTeX{} and Bib\TeX{} style files provided roughly follow the American Psychological Association format.
% If your own bib file is named \texttt{custom.bib}, then placing the following before any appendices in your \LaTeX{} file will generate the references section for you:
% \begin{quote}
% \begin{verbatim}
% \bibliography{custom}
% \end{verbatim}
% \end{quote}

% You can obtain the complete ACL Anthology as a Bib\TeX{} file from \url{https://aclweb.org/anthology/anthology.bib.gz}.
% To include both the Anthology and your own .bib file, use the following instead of the above.
% \begin{quote}
% \begin{verbatim}
% \bibliography{anthology,custom}
% \end{verbatim}
% \end{quote}

% Please see Section~\ref{sec:bibtex} for information on preparing Bib\TeX{} files.

% \subsection{Equations}

% An example equation is shown below:
% \begin{equation}
%   \label{eq:example}
%   A = \pi r^2
% \end{equation}

% Labels for equation numbers, sections, subsections, figures and tables
% are all defined with the \verb|\label{label}| command and cross references
% to them are made with the \verb|\ref{label}| command.

% This an example cross-reference to Equation~\ref{eq:example}.

% \subsection{Appendices}

% Use \verb|\appendix| before any appendix section to switch the section numbering over to letters. See Appendix~\ref{sec:appendix} for an example.

% \section{Bib\TeX{} Files}
% \label{sec:bibtex}

% Unicode cannot be used in Bib\TeX{} entries, and some ways of typing special characters can disrupt Bib\TeX's alphabetization. The recommended way of typing special characters is shown in Table~\ref{tab:accents}.

% Please ensure that Bib\TeX{} records contain DOIs or URLs when possible, and for all the ACL materials that you reference.
% Use the \verb|doi| field for DOIs and the \verb|url| field for URLs.
% If a Bib\TeX{} entry has a URL or DOI field, the paper title in the references section will appear as a hyperlink to the paper, using the hyperref \LaTeX{} package.

% \section*{Acknowledgments}

% This document has been adapted
% by Steven Bethard, Ryan Cotterell and Rui Yan
% from the instructions for earlier ACL and NAACL proceedings, including those for
% ACL 2019 by Douwe Kiela and Ivan Vuli\'{c},
% NAACL 2019 by Stephanie Lukin and Alla Roskovskaya,
% ACL 2018 by Shay Cohen, Kevin Gimpel, and Wei Lu,
% NAACL 2018 by Margaret Mitchell and Stephanie Lukin,
% Bib\TeX{} suggestions for (NA)ACL 2017/2018 from Jason Eisner,
% ACL 2017 by Dan Gildea and Min-Yen Kan,
% NAACL 2017 by Margaret Mitchell,
% ACL 2012 by Maggie Li and Michael White,
% ACL 2010 by Jing-Shin Chang and Philipp Koehn,
% ACL 2008 by Johanna D. Moore, Simone Teufel, James Allan, and Sadaoki Furui,
% ACL 2005 by Hwee Tou Ng and Kemal Oflazer,
% ACL 2002 by Eugene Charniak and Dekang Lin,
% and earlier ACL and EACL formats written by several people, including
% John Chen, Henry S. Thompson and Donald Walker.
% Additional elements were taken from the formatting instructions of the \emph{International Joint Conference on Artificial Intelligence} and the \emph{Conference on Computer Vision and Pattern Recognition}.

% Bibliography entries for the entire Anthology, followed by custom entries
%\bibliography{anthology,custom}
% Custom bibliography entries only
\bibliography{custom}

\appendix


\begin{figure*}
    \centering
    \includegraphics[width=1\linewidth]{Inference_demo.pdf}
    \caption{Illustrations of the inference process in two scenarios.The upper part represents the scenario without using speech prompts to control prosody, where in the first step, the duration of the first character needs to be predicted separately; in the subsequent decoding steps, both the current speech token and the duration of the next text token are predicted simultaneously.  The lower part shows the illustration of using speech prompts to control prosody, where $y^p$ and $s^p$ denote the text tokens and speech tokens of the speech prompt, respectively.}
    \label{fig2}
\end{figure*}


\section{Details of Baselines}
\label{baselines}

\paragraph{CosyVoice} A two-stage large-scale TTS system. The first stage is an autoregressive model similar to VALL-E \cite{valle}, and the second stage is a diffusion model. We use the official code and the 25Hz version of the pre-trained checkpoint\footnote{https://www.modelscope.cn/iic/CosyVoice-300M-25Hz.git}.

\paragraph{CosyVoice2} Compared to CosyVoice, improvements have been made in the following three areas: 1) The quantizer speech tokenizer has been upgraded to FSQ, further improve the performance of the quantization encoder. 2) Interleaved text-speech modeling is employed, allowing for streaming text input. 3) A chunk-aware speech decoder is used for streaming speech generation. We use the official code and the 25Hz version of the pre-trained checkpoint\footnote{https://github.com/FunAudioLLM/CosyVoice}.


\paragraph{VALL-E} A large-scale TTS system employs both an autoregressive and an auxiliary non-autoregressive model to predict discrete tokens derived from the Encodec \cite{encodec}. We used an open-source checkpoint for inference. As there is currently no open-source streaming speech decoder for Encodec, we assumed 15 frames when calculating the FPL metric for a fair comparison.

\paragraph{MaskGCT}\cite{maskgct}  This is a large-scale, two-stage trained model. In the first stage, the model utilizes text to predict semantic tokens extracted from a speech self-supervised learning (SSL) model. In the second stage, it predicts acoustic tokens based on these semantic tokens. During training, MaskGCT learns to predict masked semantic or acoustic tokens given specific conditions and prompts. During inference, MaskGCT generates speech through multi-step temporally non-sequential masked prediction. Here, we use the official code and pre-trained checkpoint\footnote{https://github.com/openmmlab/Amphion}.

\paragraph{F5-TTS}\cite{F5tts} a fully non-autoregressive text-to-speech system
based on flow matching with Diffusion Transformer (DiT). The
text input is simply padded with filler tokens to the same length as input speech,
and then the denoising is performed for speech generation. F5-TTS does not utilize speech tokens and directly maps text to acoustic features. Here, we use the official code and pre-trained checkpoint\footnote{https://github.com/SWivid/F5-TTS}.






\section{Details of Latency and Efficiency Evaluation Metrics}
\label{evaluation metrics}

The first-package latency (FPL) and real-time factor (RTF) are two import metrics for streaming TTS models.
We define $d_{\text{LLM}}$ as the average time required by the upstream LLM to generate one text token and $d_{\text{TTS}}$ as the the time for the corresponding AR TTS models to forward one step and for the NAR TTS models to perform one sampling. The FPL-L of baseline models and SyncSpeech are as follows,
\begin{align}
& L_{\text{FPL-L}}^{\text{CosyVoice}} =L \cdot d_{\text{LLM}} + 15 \cdot d_{\text{TTS}}, \\
& L_{\text{FPL-L}}^{\text{VALL-E}} =L \cdot d_{\text{LLM}} + 15 \cdot d_{\text{TTS}}, \\
&L_{\text{FPL-L}}^{\text{CosyVoice2}} =5 \cdot d_{\text{LLM}} + 15 \cdot d_{\text{TTS}}, \\
& L_{\text{FPL-L}}^{\text{MaskGCT}} =L \cdot d_{\text{LLM}} + b \cdot d_{\text{TTS}}, \\
& L_{\text{FPL-L}}^{\text{F5-TTS}} =L \cdot d_{\text{LLM}} + b \cdot d_{\text{TTS}}, \\
&L_{\text{FPL-L}}^{\text{SyncSpeech}} =(k+1) \cdot d_{\text{LLM}} + c \cdot d_{\text{TTS}},
\end{align}
where $b$ represents the number of sampling iterations for the NAR model, and $c$ denotes the number of BPE text tokens when the generated speech tokens surpass the decoder's chunk size, typically ranging from 1 to 3. Here, we assume the upstream LLM model is Qwen-7B, and when running on a single NVIDIA A800 GPU, we obtain an average token generation time $d_{LLM} = 25 ms$. 
When the first term in FPL-L is omitted, it becomes FPL-A. It is important to note that when calculating above metrics, we did not apply any engineering optimizations, such as KV cache.

We also conducted a brief theoretical analysis of RTF for SyncSpeech. The RTF for SyncSpeech is calculated as follows,
\begin{equation}
L_{RTF} = \frac{ (L+1) \cdot d_{\text{TTS}}}{T\cdot F},
\end{equation}
where $L$ and $T$ represent the number of BPE tokens and speech tokens, respectively $F$ refers to the frame length of the speech tokens.  The time complexity for SyncSpeech to generate an entire sentence can be simplified to $O(L)$, whereas the time complexity for concurrent approaches, such as CosyVoice2 and IST-LM, is 
$O(T)$. As a result, SyncSpeech can significantly expedite speech generation.


\section{Duration Control}
\label{Duration Control}
Since we have implemented duration prediction and control, we can multiply the predicted durations by a modulation factor to adjust speech rate. The results, shown in Table \ref{table7}, indicate that the robustness of synthesized speech is optimal when the modulation factor is 1.1.  However, when the modulation factor is too small or too large, the WER of the synthesized speech by SyncSpeech increases significantly. This is because when we multiply the predicted duration of each text token by a fixed modulation factor of less than 1, SyncSpeech's contextual learning capability causes the subsequent tokens to be spoken increasingly faster, leading to a surge in WER. When the modulation factor is set to 0.8, the average total duration of the synthesized speech is 0.68 times that when the modulation factor is 1. Therefore, more reasonable duration control requires two inference processes: the duration obtained from the first inference is multiplied by a modulation factor during the second inference to control the speech rate.



\begin{table}[]
\centering
\resizebox{0.5\textwidth}{!}{
\begin{tabular}{lcccccc }

\toprule
\textbf{Modulation Factor}      & \textbf{0.8}   & \textbf{0.9} & \textbf{1.0} & \textbf{1.1} &\textbf{1.2} &\textbf{1.3} \\ \hline
LibriSpeech        & 14.3 & 4.20 &3.07 & \textbf{2.85} &3.22 &4.31   \\
SeedTTS test-zh    & 12.1 &3.38 &2.38 & \textbf{2.15} &2.53 & 3.48            \\
\bottomrule 
\end{tabular}
}
\caption{Performance comparison with different modulation factors for duration control in terms of WER.}

\label{table7}
\end{table}



\section{Other Strategies for Sequence Construction}
We also experimented with other sequence construction strategies. (1) One approach is to separate duration prediction and speech tokens prediction into two steps. This method reduces efficiency by half but achieves better speech robustness, with a WER of around 2.75 on the LibriSpeech \textit{test-clean} dataset. (2) We also tried removing the duration placeholder and using the last speech token of the previous text token to predict the number of speech tokens corresponding to the current text token. However, we found that this sequence construction made the corresponding pre-training less effective than it is now. (3) We also attempted a method similar to ELLA-V \cite{ellav}, where the corresponding text token is placed before each placeholder. However, we found that this sequence generated speech that was unnatural, with a noticeable disconnection between words.

\end{document}
