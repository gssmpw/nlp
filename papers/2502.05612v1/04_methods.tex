\section{Methods}

The goal of the case study is to understand how writers use speech as the primary text input modality for their own writing tasks in a real-life context, as well as how AI features could effectively support them. 
Our research questions are: 1) How do academic/creative writers use a speech-based writing tool to write in real life? 2) What new affordances does the speech modality bring and how could AI support writing with speech? 3) What factors affect the user acceptance of this new writing paradigm?

We began with a few focus group sessions gathering the recruited participants to discuss their writing habits, get introduced to the Rambler tool, and brainstorm their envisioned scenarios for using it. Subsequently, participants could opt in for a ten-day diary study where they would write a minimum of three articles. The study involves three intermediate surveys and one exit interview upon completion.

All procedures were approved by the university’s research ethics review (Human Research) before data collection.

\subsection{Participant}

A total of 14 respondents registered to participate in the focus groups.
The selection of participants in the focus groups was tailored to involve a diverse set of individuals interested in various types of creative writing, with experience or curiosity about dictation. We did not have any specific style or literary genre prerequisites. All participants were not English native speakers, but participants had to qualify for the case study through either experience in a English-taught degree program, a score over 90 in TOEFL (Test of English as a Foreign Language), or a score over 6.5 in IELTS (International English Language Testing System): their English proficiency information was gathered through surveys.

For the subsequent diary study phase, a total of 12 participants were involved. 11 out of the 14 initial focus group participants opted to join, with an additional participant (P15) who missed the focus group sessions due to logistic reasons.
Detailed participant information is in Appendix Table~\ref{tab:users}.

\subsection{Study Design and Procedure}

\subsubsection{Focus Group}
The purpose of the focus group was to make sure that the participants got familiarized with the research tool Rambler and actively thought about how to utilize this new way of writing. Four rounds of focus group sessions were held, three were online and one was on-site, grouped based on the participants' availability. Each focus group was conducted in three steps. In step 1, participants were asked to share their current writing habits, strategies, and experiences. In step 2, the features of Rambler were introduced to the participants via a video overview and live demonstration with Q\&A. Participants were given a few minutes to try out the functions. In step 3, the host initiated a brainstorming for participants to envision their potential writing tasks and scenarios using Rambler to serve their own writing goals in real life. Each focus group session lasted about one hour. The participation of a focus group session granted each participant a 100 HKD supermarket coupon as compensation. Interested participants were asked to register for the diary study at the end of the session. 

\subsubsection{Diary Study}
After each focus group session, registered participants began their diary study phase, which lasted 7 to 10 days. The registration survey asked about the participants' demographic information, writing topic interests, dictation experiences, and LLM experiences. They were asked to use Rambler to write throughout the study period and finish at least 3 articles, each of at least 500 words, by the end of the study.
Participants were allowed to freely choose their writing tasks and purposes but were required to finalize the draft for their chosen purposes. As Rambler is a web application, the participants could use it via a web browser on any device via a unique URL generated for each participant. They were warned to avoid any confidential content as the content is logged without encryption. 
Every two days, they were required to fill out a survey sampling their usage situations (writing content, devices of use, preferences of functions, and any insights or problems discovered during their usage). During the diary study, we sent a reminder every two days, either by text or email based on preference, to remind participants to perform the task 
and fill out the experience survey on time.

\subsubsection{Post-study Interview}
Interviews took place online or on-site per participants' requests. The questions in the interviews were geared toward study goals, including the affordance of writing with speech, comparison to existing writing methods, user strategies from ideation to creation processes, user acceptance, and suggestions. Upon completion, compensation for the diary study was a HKD 400 supermarket coupon per participant. Each interview took one hour.

\subsection{Data Collection and Analysis}
We collected screen recordings of the focus groups via an online conference tool Zoom. For the diary study, we collected thirty-six articles in total (each participant completed three articles, with each one at least 500 words), audio recordings of the exit interview, and three surveys from each participant. Participants' interactions with the tool were logged in the Rambler application on our web server. We programmatically processed the logs to analyze how content evolved through the use of Rambler features.
We collected qualitative data from the audio transcriptions of the focus groups and interviews and the written answers from the diary surveys. Inductive thematic analysis was used to identify patterns and themes within data~\cite{fereday2006demonstrating}. Two researchers coded 25\% of the data independently and discussed their interpretations to reach a consensus. Then one of them coded the rest of the data. Multiple researchers worked on the categorization of codes into themes and patterns together. We iteratively reviewed and refined the identified themes regarding users' strategies, affordances, and user acceptance of LLM-powered speech-based writing, to ensure they accurately reflected the data and provided a comprehensive representation.