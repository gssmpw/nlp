\section{Introduction}
In the realm of writing tools, the integration of speech input technology has emerged as a transformative force, offering users an alternative mode of interaction that transcends traditional paper \& pen-based or keyboard-based methods~\cite{lin2024rambler, Ruan2018Comparing}. Leveraging speech input to empower writing stems from speech's inherent ability to bridge the gap between thought and text, allowing users to articulate their ideas naturally and fluidly~\cite{lin2024rambler, Mehra2023GistAV, Crystal2005SpeakingOW, Ruan2018Comparing}. Through speech, individuals can bypass the constraints imposed by manual typing, enabling a more direct and expressive form of communication~\cite{lin2024rambler}. However, challenges such as ambient noise interference, dialectal variations, and inaccuracies in speech recognition systems can hinder the accuracy and dependability of transcriptions, necessitating additional efforts to refine the converted speech text~\cite{Karat1999PatternsOE}.

Previous research has explored the use of natural language processing (NLP) and large language models (LLMs) to assist in writing~\cite{Dang2022BeyondTG} and reviewing spoken dialogue~\cite{Li2021HierarchicalSF, Li2023ImprovingAS}, especially in cleaning disfluency and speech recognition errors~\cite{Liao2020ImprovingRF, bassi2023end, Tanaka2018NeuralEC}. Users can also shorten, summarize or replace the selected text with LLM suggestions or prompt the model for text generation~\cite{Yang2022AIAA}. In addition, LLMs can support macro-level structural revision, which moves beyond automatic text summarization to semantic manipulation with writers in semantic control~\cite{Arnold2021GenerativeMC}. For example, Dang et al~\cite{Dang2022BeyondTG} introduces a writing tool that provides on-the-fly paragraph summarization along with the original text. Users could interact with the summaries of paragraphs, such as reorganization via drag and drop, which manipulates the original text in parallel.

Our recent research~\cite{lin2024rambler} presents an LLM-powered graphical user interface, called Rambler, that supports gist-level manipulation of dictated text with two main sets of functions: gist extraction and macro revision. To evaluate the effectiveness of this approach, we conducted a lab study, where 12 participants were asked to compose two articles each, one using Rambler and the other using a baseline (a standard STT editor + ChatGPT). Participants chose from broad writing topics provided by the experimenter for completing the tasks. The findings showed an overwhelming preference for Rambler over the baseline (10 out of 12) and demonstrated diverse writing strategies adopted by participants. However, while a lab study in a controlled environment ensures comparability between conditions, it lacks real-world context, and could not test the middle- or long-term usage of the approach. In our case, since participants were writing content for a lab study, there was low incentive for them to edit it as thoroughly as they might for their own writing in real life. Additionally, there might not be enough time in a one-hour lab study for participants to adopt a new writing paradigm using speech as text input modality. 

As a follow-up work, we conducted an in-the-wild study to understand the real-world longer-term usage of Rambler for creative writing tasks. The goal was to understand the affordance of writing with speech and its user acceptance as well as how the functions of Rambler could be appropriated into users' diverse writing habits.  

Through a few focus groups and a ten-day diary study, participants wrote three articles on their own topics with personal device(s) and in their real-life environment. We report on participants' writing strategies and workflows led by distinctive writing goals. The surveys and interviews about their experiences provide insights into their incentives and obstacles in adopting the speech modality for writing and the role of AI assistance in the process. 

Based on the findings, we discuss the implications of designing STT-based writing tools and opportunities for AI support.