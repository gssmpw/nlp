\section{Background of a Speech-Based Writing Tool -- Rambler}

The speech-to-text tool we used is Rambler~\cite{lin2024rambler}, which has users dictate their impromptu thoughts, placing each segment of recording into a \textit{Ramble}. Rambler provides three main functions. First, it supports users in dictation by cleaning up transcripts. Each recording is transcribed by a real-time STT API and then preprocessed with an LLM that automatically cleans up any disfluencies and punctuation errors, completes broken sentences, and smooths transitions in the raw transcript. Users can also use the \textit{Respeaking} function to dictate new content to replace or add to the existing Ramble (the microphone button in Figure~\ref{fig:rambler}). Second, it can extract the \textit{gists} in the text to support visualization and interaction with that text. The keywords of the content are shown automatically after recording, and users can adjust the selection on demand. Based on the gists selection, Rambler provides \textit{Semantic Zoom} to show multiple summarization levels of each Ramble (the bottom slider).

Besides, Rambler provides several LLM-based functions for users to perform macro-revisions on the content. \textit{Semantic Split} operates on an individual Ramble to divide one Ramble into N Rambles based on its content (the scissors button in Figure~\ref{fig:rambler}). \textit{Semantic Merge} merges the contents of multiple selected Rambles into one Ramble (the merge button in Figure~\ref{fig:rambler}). \textit{Custom Magic Prompt} opens up the possibility for users to define any custom transformation on Ramble content by directly inputting an LLM prompt (the magic wand button in Figure~\ref{fig:rambler}). Rambler also provides manual editing functions. \textit{Manual Merge} concatenates the text of two Rambles into one Ramble (through drag and drop). \textit{Manual Split} splits a Ramble into two at the cursor position. Users can always drag and drop the Ramble to reorder or edit the content using the keyboard. 

\begin{figure}
  \Description{A labeled screenshot of the Rambler UI. Ramble in default state, with revision functions accessible through buttons on the top of each paragraph, a paragraph is called Ramble. The first Ramble shows re-speaking mode, where voice input is transcribed so that it can be appended to current text, replace the current text, or to be discarded using the buttons on the right corner at the bottom of Ramble. The buttons at the bottom of the UI: the Semantic Merge button, New Ramble button, and Semantic Zoom slider.}
  \centering
  \includegraphics[width=0.95\linewidth]{rambler.png}
  \caption{Rambler interface on a tablet from~\cite{lin2024rambler}. Users can dictate content into individual Rambles and use various semantic and manual functions to macro-edit and reorganize them.}
  \label{fig:rambler}
\end{figure}

