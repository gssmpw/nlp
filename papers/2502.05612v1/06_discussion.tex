\section{Discussions}
\subsection{In Relation to the Lab Study}
The lab study evaluated Rambler against a baseline and observed the usage of individual functions. It was found to be advantageous in helping users review spoken compositions and organize their thoughts, while showing high versatility in supporting diverse writing strategies.
Consistent with these findings, the diary study confirms that Rambler helps users organize thoughts, boosts efficiency in outlining and expanding ideas. In the real-world context, we found it also inspired reflective writing and journaling. While the lab study findings centered around the usage pattern and experiences with individual features of the tool, the diary study focused more on a holistic interpretation of the overall experience with the new writing paradigm. It showed good user acceptance for this approach in real-world creative writing scenarios for the enhanced productivity, the natural and emotional act of speech and its ability to prevent overthinking.
In terms of the usage patterns in the real world, we saw much flexibility in how participants distributed their writing times over days and hours. Perhaps the use of speech input on mobile devices facilitated distributed writing. In addition, the primary writing environments were in quiet places around the desk or in bed, rarely in outdoor places, highlighting the constraints for using speech input due to noisy and distraction.

\subsection{Speech Input and LLM Features go Hand-in-hand}
The findings of our study revealed that using speech as the main modality for writing carries unique advantages, including a large productivity gain and the psychological benefits of emotional or introspective expression. Yet due to these characteristics, the content being produced tends to be abundant. Therefore, the LLM features of semantically extracting information, organizing multiple pieces into a whole, splitting disconnected points and polishing the wordings all go hand in hand with it to complement the shortcomings of spoken production. 

\subsection{Writing for What and Who Matters}
Two distinctive strategies of composition led to two major directions of speech-based writing, one for academic or communicative purposes with external audiences; the other for introspective expression by recording feelings and experiences for self reflection. The usage and preferences of functions differ largely between these two types of writing scenarios: one starts with an outline and utilizes expansion methods and the other starts with loose thoughts and using merging methods to converge. Future work could consider optimizing each direction with specialized tools and information representation methods. 

\subsection{Writing With Speech is a Viable Paradigm} 
This in-the-wild study tested whether users could get used to writing with speech in a short period. Participants' positive feedback showed promise for this approach. Productivity gains and the lowered barrier for writing played a pivotal role in user acceptance, accompanied by the effective clarification of vague ideas through the act of speaking and iterative organization. Future designs could dive further into adapting to users' context and purpose of writing, such as addressing the need to refer to external materials in research-based writing. Context-awareness and personalized conversational support can be the natural next step for a speech-based writing tool. 

\section{Conclusion}
This work evaluated the real-world experiences of using speech-to-text as the primary text input method for writing. By analyzing twelve academic and creative writers' writing experiences with an LLM-assisted dictation tool, we conclude with a positive outlook for this new writing paradigm based on its productivity gain and psychological benefits. Our insights on the effectiveness of features based on different writing purposes pave the way for future generations of AI-supported writing with speech. 
