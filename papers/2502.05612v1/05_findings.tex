\section{Findings}
We report qualitative findings here from the focus group and diary study interviews and surveys. We first draw an overview of the participants' existing writing habits and envisioned use of Rambler from the focus group, then answer each research question with themes identified from thematic analysis. We end this section with a list of design suggestions made by participants.
\subsection{Focus Group - Participants' Envisioned Usage Scenarios}
\subsubsection{Participants' Writing Habits}
Participants' motivations for writing varied from emotionally significant moments, unique experiences, to academic or work-related writings. They talked about their writing on mobile phones, which was handy for capturing random thoughts (P3, P5, P6). They liked to utilize specialized tools to aid in various writing stages such as idea storage (P8), topic management (P5), logical restructuring (P5), and grammar refinement (P9). They also explored tools to effectively segment and organize multiple writing projects (P2).

Some participants had experience with dictation and mentioned their use of it to multitask or avoid typing (P1). There was a perceived text input efficiency (P5) and benefit in language practice (P6). Some found capturing spoken content disruptive or awkward (P2, P3, P4), with concerns regarding accuracy and responsiveness in noisy environments (P10). Participants who were familiar with LLM tools heavily relied on them for various tasks, including ideation (P1, P2), providing initial ideas (P1), finding words (P2), revising tone (P4), improving grammar (P9), generating specific writing characteristics (P2), learning concepts (P1), practicing writing (P8), improving skills (P6), and translating text (P6).

\subsubsection{Envisioned Use Cases and Scenarios of Rambler}

The participants envisioned using Rambler to assist with schoolwork (P5), learning skills (P6), capturing conversation information (P2), generating functional content (such as emails) (P5), diary writing (P4, P6), screenplay creation (P11, P12), and scripting episodes for podcasts (P10).
Moreover, they liked the flexibility of using Rambler on any device, and envisioned utilizing it while walking on the road (P7), lounging in bed (P2, P4, P6), or relaxing outdoors in a park (P10). This flexibility in usage locations improves the acceptance of Rambler, catering to users' diverse writing preferences and habits.

\subsection{How do creative/academic writers use a speech-based writing tool in real life? }
In the subsequent diary study, participants generally performed their envisioned writing scenarios except for a few topic changes due to technical or logistic constraints. Table~\ref{tab:users} in the Appendix summarized what each participant wrote, in what environment, and on what device. We can see that the majority of writing tasks were done on mobile phones (56.8\%), primarily on a desk or in bed with only a few occasions during walking. Some used the computer (43.2\%) for a larger display and convenient operation. The surveys asked how they distributed their time for writing. The answers showed that 42\% of the writing tasks were completed in one go, while the rest of the tasks were done in a ``distributed'' manner. One task was done after several sessions, once per day or half a day based on time or location in their routine. 

Like the lab study findings of Rambler~\cite{lin2024rambler}, in the diary study we also observed the two distinct writing strategies: 1) \textit{outline first}---users create an outline through dictation and then expand on it; and 2) \textit{free-speaking}---users dictate detailed content spontaneously before editing it. The key difference between these two patterns lies in whether users formulate their narratives before or after dictation. From the logged content, we noticed that the former strategy tended to be used in academic or communicative writings with an external audience, while the latter strategy was more used in personal or reflective writings for oneself. 

\subsubsection{Outline Expansion for Academic or Communicative Writing.}

When participants have clear ideas about what they are going to write, they choose to dictate key points into each Ramble and expand those points with AI-powered semantic functions. Such writing patterns often occur when participants craft logically structured writings and have envisioned the structure, though not necessarily the details, before writing (P2, P3, P9, P14). For example, when writing a responding letter, P2 began by dictating five paragraphs that highlighted different key points to create an outline (see Figure~\ref{fig:str1}): they mention ``\textit{this is a letter I want to respond to his previous letter so I already have the structure in my mind which point I want to reply.}'' After creating the outline, each Ramble was processed through a custom magic prompt to expand the content, such as \textit{``expand this outline to a full-text paragraph''} and \textit{``add an example at the beginning or the end.''} By doing this step, the LLM adds more detail to the outline, and the length of each paragraph goes from one sentence to about five sentences. P2 reflected on this experience, stating, \textit{``I usually don't have enough time to reorganize that into a like complete article. But every letter I sent to my boyfriend is like complete article, so I think it's content I can use to finish the task.''}

Aside from relying on AI-generated content, some participants enhanced their writing by dictating more to supplement the content added by Magic Prompt or by re-organizing the content inspired by AI-generated content.
For instance, when P3 prepared a presentation script to update instructors about a project, they began by outlining their ideas based on existing slides. They then used Magic Prompt to refine the script. In the process, the AI-generated content often sparked memories of additional details, enabling P3 to elaborate on their initial outline. As P3 noted,\textit{``I adjusted the slides in the process, correspondingly, I also adjusted the content of my scripts.''}
It is common that when participants want to add some content to the end of a Ramble, they usually create a new Ramble and merge it with the previous one instead of respeaking to add content.

\begin{figure}
    \Description{This figure illustrates a structured workflow for content development and editing, where ideas progress from outlining to expansion, refinement, reordering, and manual merging to form a cohesive final document.}
    \centering
    \includegraphics[width=0.95\linewidth]{strategy1.png}
    \caption{Participant P2 wrote a letter by dictating an outline first and expanding from it. The numbers represent the order of actions. Blue ones are generated by dictation, and green ones are generated by macro revision.}
    \label{fig:str1}
\end{figure}

\subsubsection{Capturing Loose Thoughts in Personal and Reflective Writing.}

As P5 said, \textit{``I usually spoke a big chunk to Rambler at the beginning.''} Instead of dictating in a typing-like way, more articles were created by speaking everything in one go at the beginning, especially when participants knew the details to be described, such as writing diaries (P4), emotional notes (P6), and comments about a movie (P14). As shown in Figure~\ref{fig:str2}, P4 depicted the day's itinerary in detail. Then, the Ramble with a long paragraph is split into three short Rambles by Semantic Split. Each Ramble marks an event in the diary. For each Ramble, P4 used Magic Prompt to improve grammar, add more details, and change the tone to be more chill and easy to understand. Finally, P4 used Semantic Merge to combine two related Rambles to finish the draft.

Since users do not have a predefined outline when writing, subsequent restructuring is more common than the previous strategies. P5 reports that Semantic Split and Merge can meet their expectations in restructuring: \textit{``Semantic Split made the initial chunk paragraph into a few smaller paragraphs, and the qualities of the smaller paragraphs normally met my satisfaction.''} P5 also found that Semantic Split helped create paragraphs that started with key points, clearing up long paragraphs: \textit{``It is helpful for breaking text into manageable paragraphs with clear topic sentences''}. P12 believed that Semantic Merge helped merge similar paragraphs and encouraged them to rethink the structure so that they had a clearer direction for writing: \textit{``Semantic Merge is the main reason why I use Rambler because I need to sort out my thoughts in a logical way when brainstorming, rather than simply placing words with STT, which doesn’t help me save time.''}

After reorganization, Magic Prompt is widely used to adjust the writing style and tone. For example, P5 asks Rambler to create formal, grammatically correct writing by eliminating pauses and expressions while preserving the original meaning: \textit{``Improve writing to be formal. and correct my grammar but keep my meaning''}. To improve the performance of Magic Prompt, users gave several demands in one prompt and provide the context: \textit{``Make the tone sounds more chill and use words easy to understand it’s my diary don’t be too serious.''} (P2) When they are not satisfied with the results, they will iterate on the prompts until satisfaction, such as \textit{``make it more intense''} and \textit{``Not that intense. Make it sound like he is scared. But more anxious.''} (P8) Aside from macro-revisions, manual editing is generally used to polish the wording, such as short phrases or punctuation: \textit{``When I knew the exact word(s) I wanted to add, I would add it by typing.''} (P9)

\begin{figure}
    \Description{The figure illustrates how larger ideas are broken down into focused segments and later recombined for improved organization and tone adjustments.}
    \centering
    \includegraphics[width=0.95\linewidth]{strategy2.png}
    \caption{Participant P4 wrote an experience sharing by speaking detailed content and reorganizing it. The numbers represent the order of actions. Blue ones are generated by dictation, and green ones are generated by macro revision.}
    \label{fig:str2}
\end{figure}

\subsection{What new affordances does writing with speech have and how could AI help?}

We summarized the following findings about the affordances of writing with speech, as a new way of writing that people perceive and feel in their experience of using Rambler, including how it facilitated emotional expression, boosted writing productivity, and improved their self-efficacy. 

\subsubsection{Speaking as a Natural and Emotional Expression Channel and a Communicative Act.}
The inherent quality of using speech mirrors human natural conversation, thus offering users an uninterrupted channel for authentic emotional recording and introspection. 
P3 likened the nature of writing with speech to an undisturbed telephone conversation:\textit{ ``Doing speech-based writing is like talking on the phone and would not be interfered with.''} 
P5 highlighted the emotional resonance enabled by dictation: using speech to write about emotional events would capture the emotions so vividly and authentically for them that upon reviewing the transcript, they could feel how they felt at that moment.
This unfiltered self-expression makes speech-based writing suitable for personal diary. 
As P2 explained, \textit{``I could be the real me when using dictation to write a diary.''}
P6 noted, \textit{``It's just like you speak to another person, but that person actually doesn't exist and you can see whatever you want to see to it.''}

Furthermore, the act of expressing emotions through dictation serves as a vehicle for organizing one's psychological landscape and fostering introspection, like a meditative experience.
As P4 describes \textit{``using Rambler feels like meditation or self-reflection''}.
Speaking is also associated with collaborating with a teammate, facilitating communicative production.
P4 highlighted that writing with speech was like chatting with a close person, expanding the branches of thoughts together. P13 also noted that the process of writing with speech was like talking with someone by phone, whose purpose was expressing their thoughts.
Considering this property, one participant decided to use Rambler to write a screenplay and found it highly satisfactory. P12 said, \textit{``Rambler directly captured my very simple words, the output generated by dictation surprised me, since the format and literary style was very close to screenplay. In screenplay, you don't need to write in a sophisticated way. So I realized that a screenplay can be simply completed, and I just finished this one (screenplay) in a very cozy sitting posture on the sofa.''}


\vspace{-2mm}
\subsubsection{Speaking Boosts Productivity and LLM Prevents Overthinking.}

The efficiency of dictation
often lead to a surprising experience with the substantial volume of text generated by talking. P14 was notably impressed by the productivity of using dictation
after merely speaking for 1 to 2 minutes, which produced a few hundred words.
They felt it lowered the pressure of writing tasks. P8 echoed a similar experience after a few days of usage and stated that it became easier to output several hundreds of words in one go compared to typing every single sentence on the keyboard.


Sometimes writers overthink a detail and get stuck as they fixate on word choice from the beginning. Perfectionism can block individuals’ inspiration and hinder their creative thinking. Using Rambler, participants felt that the LLM features nudged them to wisely assign their attention and energy towards deliberating the overall content and structure of their work.
P12 noted that they could save energy to think about new ideas with the assistance of LLMs in crafting the wording details. The fluent experience of ideation with the aid of LLM helped prevent individuals from overthinking and reduced their perfectionism in the writing process.
As P8 stated, \textit{``Writing with speech is beneficial, especially with preventing me from overthinking sentences, helping me go with the flow, and leaving some of the more complex sentence building to AI.''}

\vspace{2mm}
\subsubsection{LLM Features Foster Self-efficacy.}~\label{efficacy}
The LLM features for polishing the oral spoken draft into written formats brought confidence to users in their writing process. P4 commented that LLMs could assist formal writing in good quality since it was able to rephrase their spoken tone to a suitable style.
P14 also expressed appreciation for this aspect: \textit{``Since LLM could polish the wording in a beautiful style, I didn't need to worry about whether my speaking was correct or not, instead, it helped to remove the errors in my original speaking at the beginning, it effectively avoided the errors.''}

Moreover,
participants prompted LLMs to adopt specific literary styles, and improved their language learning by observing the polished outcomes. 
P14 noted personal growth in language proficiency through this feature. As the LLM polished their originally simple wording into a higher quality, they learned new writing styles and skills from the change and realized that it could be a good opportunity to improve English writing skills.
P2 was amazed after employing an original English diary writing style by prompting the LLM:
\textit{``Rambler helped me to make my diary look more like an English native speaker, which impressed me a lot. I felt surprised by the change since it is highly like the style in a book about interesting English diaries I read before, it's a way to enhance my writing skill and make my writing close to the original English literary style.''}

\subsection{What factors affect the user acceptance of writing with speech?}
Nine out of twelve participants (P2, P3, P4, P5, P6, P9, P13, P14, P15) reported that they were able to get used to writing with speech in the short period of the diary study, while three (P8, P10, P12) expressed reservation. As shown in Figure~\ref{fig:useracc}, the median of the score for ``comfort of use'' is 5 (out of 7) for all three tasks submitted over time, while the deviation of the scores decreases after the first use. This shows positive user acceptance of this new approach for writing, as well as some initial learning curve to get used to it.

\subsubsection{Productivity Gain.} Although we could not measure task completion time effectively in this study due to its in-the-wild nature, we did ask for an estimation of time they spent on a writing task in each survey. Their answers showed, 66.7\% of the 500-word writing tasks were completed in 10 to 30 minutes, 15.4\% of them needed 30 to 60 minutes, 10.3\% over an hour, 7.7\% within 10 minutes.
Participants felt a sense of achievement after completing a task with Rambler and expressed positive surprises in gaining trust in the technology. For P14, because writing was a heavy task that brought a huge mental burden, it was previously hard for them to complete an article in traditional ways (either typing or handwriting), and so they gave up having a writing routine.
In the focus group discussion, they initially expressed doubts about whether speech could facilitate writing and thought dictation technology employed in writing was an ambiguous concept to them. Yet in the post-study interview, they shared a big sense of surprise from the productivity gain in writing with speech.
They expressed willingness to restart a writing routine in the future via dictation. 

\subsubsection{Effective help for organizing thoughts.} Another reason for participants to adopt this technology was that the LLM features could greatly reduce their effort by merging several vague ideas into one. P12 said in the focus group that using traditional writing to illustrate details of ideas from a vague concept could always block their writing. 
In the post-study interview, they shared their pleasant experiences of inputting their vague ideas and prompting the LLM to merge them into a coherent paragraph. They felt that the LLM features could effectively and efficiently boost clarity at the ideation stage.

\begin{figure}
    \Description{A boxplot illustrating how participants felt comfortable during three rounds of writing tasks (1st, 2nd, and 3rd).}
    \centering
    \includegraphics[width=0.9\linewidth]{user_acceptance.png}
    \caption{Participants' user acceptance during three rounds of writing tasks using a 7-point Likert scale.}
    \label{fig:useracc}
\end{figure}

\subsubsection{Remaining Challenges.}
In the meantime, there remain some challenges in adopting this new writing paradigm.
One challenge the participants faced was that they felt distracted during the process of speaking while thinking about the next sentence (P8). 
Although Rambler leveraged LLM to correct recognition errors and disfluencies, detailed editing is still necessary at times.
P3 expressed the inconvenience of having to modify a typo in the middle of a sentence. Unlike typing on keyboards, it is nearly impossible to direct the cursor to a certain place via LLM prompting.

Last but not least, several participants raised the issue of being unable to multitask during speech composition. For instance, participants felt that they could not use Rambler to write something that requires them to research online at the same time. P2 said, \textit{``Using Rambler on phone is not convenient for reading and doing research meanwhile.''}
P3 also mentioned the inconvenience of how Rambler currently cannot support multitasking in different tabs or apps.

\subsection{User-Suggested Improvement for Design}
\subsubsection{Context Awareness} Participants made suggestions of reusing their Custom Magic Prompts. As P9 stated, ``\textit{It would be better if Rambler could save my prompts' history, so I could quickly access them again if I needed.}'' P2 said, ``\textit{It would be more useful if the prompt can be applied in all the Rambles simultaneously on the purpose of efficiently knowing the context of my article and adjusting the writing tone of the whole article in one go.}''

\subsubsection{Personalized Conversational Support} Given the nature of speaking as a communicative act, integrating a virtually personalized assistant in writing with speech could represent a strategic enhancement. P14 said, ``\textit{It could encourage users to have more inspirations to write, if there is a chatbot or assistant knowing the context on Rambler, since the behavior of having conversation or communication with them can enhance people to think forward.}''
P2 attempted that prompting an assistant-like suggestion but did not get the expected result, what they did was giving an identity to the LLM, and expected the feedback from LLM to generate more human-like information for personal content. Similarly, P9 had the same feeling as well in the creation process: \textit{“It could be very useful to me if I could personalize a chatbot on Rambler”.}

\subsubsection{Alternative Information Storage and Display} Some participants felt it would be good to keep their original audio mapped to the transcripts for error prevention (P9, P15). Others also wanted to be able to see the text and outline side-by-side so that they could read the main points of each paragraph at a glance during composition (P15). Participants also requested easier ways to distinguish the Rambles, such as by adding a title for each. P9 said, ``\emph{It would be better if there's a title for each Ramble ... I had to re-read them every time if I wanted to remind myself of the main point of each paragraph. The process could be annoying and consumed my patience.}''