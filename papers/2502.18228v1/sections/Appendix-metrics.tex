\section{Detail of metrics} \label{app:metric}

\subsection{Conversational Ability} \label{app:me_conv}

In negotiation processes, conversational ability is crucial for achieving effective communication and mutual understanding. \citet{tu2024characterevalchinesebenchmarkroleplaying} proposed an evaluation framework for role-playing tasks. Inspired by this work, we tailored it to our task by distinguishing Conversational Ability into two dimensions: fluency and completeness.

\textbf{Dialogue Soundness (DS).} Dialogue Soundness is a single-metric evaluation that measures a dialogue response’s fluency, naturalness, coherency, and consistency on a five-point scale. It assesses whether the response is grammatically correct and conversational, stays on topic, and remains logically consistent across turns. This metric is manually scored, with the scale shown in Table ~\ref{DS_score}. Five graduate students from engineering disciplines were employed to evaluate this metric and calculated the average value.

\begin{table*}[h]
    \centering
    \caption{\label{DS_score}Dialogue Soundness (DS) Rating Scale}
    \begin{tabular}{c l l}
        \toprule
        \textbf{Score} & \textbf{Rating} & \textbf{Description} \\
        \midrule
        5 & Excellent  & Fluent, natural, on-topic, logically consistent. \\
        4 & Good       & Mostly natural, minor topic drift, slight inconsistency. \\
        3 & Acceptable & Understandable but somewhat rigid, occasional drift or inconsistency. \\
        2 & Poor       & Unnatural phrasing, noticeable topic deviation or contradictions. \\
        1 & Unacceptable & Robotic, off-topic, illogical contradictions. \\
        \bottomrule
    \end{tabular}
\end{table*}

\textbf{Dialogue Completeness (DC).} Dialogue Completeness is a metric designed to evaluate whether a conversation addresses all specified objectives outlined in section~\ref{obj} of the paper. This automated measure checks if each of the four key goals has been adequately discussed during the dialogue, ensuring that no critical topics are overlooked or omitted.

\subsection{Debt Recovery} \label{app:me_rec}

\textbf{Success Recovery Rate (SR). }The success rate of the negotiation is determined by whether the debtor's future assets remain in a healthy state (i.e., the total personal assets remain greater than 500). The success rate is defined as the proportion of samples in which repayment can theoretically be completed successfully:
\begin{equation}\label{eq:utility}
\begin{split}
\text{SR} = \frac{N_{\text{success}}}{N},
\end{split}
\end{equation}
where SR is the success rate, $N_{\text{success}}$ is the number of successful samples, and \(N\) is the total number of samples.

\textbf{Recovery Rate (RR). }The recovery ratio refers to the portion of the debt recovered by the creditor, which is typically $1$ minus the reduction ratio. If the plan is unsuccessful, the recovery ratio is considered to be $0$. The final recovery ratio is calculated as the mean recovery ratio across the test samples:

\begin{equation}
\begin{split}
\text{RR} = \frac{1}{N} \sum_{i=1}^{N} r_i,
\end{split}
\end{equation}
where RR is the final recovery ratio, \(r_i\) is the recovery ratio of the \(i\)-th sample.


\subsection{Collection Efficiency} \label{app:me_col}

\textbf{25\% Recovery Date (QRD)} refers to the date at which the debtor has completed 25\% of the debt repayment, which is estimated based on the debtor's future economic condition sequence. The final 25\% Recovery Date is calculated as the mean of the recovery dates across the test samples:
\begin{equation}
\begin{split}
\text{QRD} = \frac{1}{N} \sum_{i=1}^{N} t_{25\%,i},
\end{split}
\end{equation}
where QRD is the final 25\% recovery date, $t_{25\%,i}$ is the 25\% recovery date of the $i$-th sample, and $N$ is the total number of samples.

\textbf{50\% Recovery Date (HRD)} is defined similarly to the 25\% Recovery Date, referring to the date at which the debtor has completed 50\% of the debt repayment, based on the debtor's future economic condition sequence. 
\textbf{Completion Date (CD)} refers to the date at which the debtor has fully repaid all of the debt.

The 50\% Recovery Date and Completion Date are calculated as the means of the respective recovery dates across the test samples:
\begin{equation}
\begin{split}
\text{HRD} = \frac{1}{N} \sum_{i=1}^{N} t_{50\%,i},
\end{split}
\end{equation}
where HRD is the final 50\% recovery date, and $t_{50\%,i}$ is the 50\% recovery date of the $i$-th sample.
\begin{equation}
\begin{split}
\text{CD} = \frac{1}{N} \sum_{i=1}^{N} t_{\text{Completion},i},
\end{split}
\end{equation}
where CD is the completion date, and $t_{\text{Completion},i}$ is the completion date of the $i$-th sample.


\subsection{Debtor’s Financial Health} \label{app:me_hea}

\textbf{L1 Tier Days (L1D)} refers to the number of days the debtor remains in the most difficult tier over the next two years. \textbf{L2 Tier Days (L2D)} refers to the number of days the debtor remains in the second most difficult tier during the same period. These two indicators directly correspond to the duration the debtor spends in different levels of financial difficulty. Research has shown that the longer the debtor remains in a higher level of difficulty, the more likely they are to default on the loan 
~\citep{Tabacchi2016DeterminantsOE}.

\textbf{Asset tier variance (ATV).} In addition to controlling for the number of days the debtor remains in the high-poverty tier, the overall stability of the debtor's asset level also ensures a higher repayment performance. To capture this, we introduce the asset tier variance metric, which is calculated by computing the variance of the debtor's asset tier over the course of one year. The final result is obtained by calculating the mean of the asset tier variances across the test samples:
\begin{equation}
\begin{split}
v_{\text{asset},i} = \frac{1}{T-1} \sum_{t=1}^{T} \left( A_{i,t} - \bar{A}_i \right)^2,
\end{split}
\end{equation}
where $A_{i,t}$ is the asset tier of the $i$-th sample at time $t$, $\bar{A}_i$ is the average asset tier of the $i$-th sample over the year, and $T$ is the total number of time periods. The final asset tier variance is the mean of the asset tier variances across the test samples:
\begin{equation}
\begin{split}
\text{ATV} = \frac{1}{N} \sum_{i=1}^{N} v_{\text{asset},i},
\end{split}
\end{equation}
where ATV is the mean asset tier variance, and $N$ is the total number of samples.

\subsection{Average Metric} \label{app:me_ave}

In debt collection, the indicators for Debt Recovery and Collection Efficiency are often inversely related to the Debtor’s Financial Health. This means that efforts to recover debts more efficiently and quickly may negatively impact the debtor's financial condition. To strike a balance between these two conflicting objectives, we introduce three average metrics that help quantify the trade-off: the Creditor’s Recovery Index (CRI), the Debtor’s Health Index (DHI), and the Comprehensive Collection Index (CCI).

\textbf{Creditor’s Recovery Index (CRI):} This index measures the effectiveness of the creditor's recovery strategy while accounting for the impact on the debtor’s financial health. The index aggregates several recovery metrics weighted by their relative importance to the creditor's objectives. The index is calculated as follows:
\begin{equation}
\begin{split}
\text{CRI} = &\ w_1 \cdot \text{SR} + w_2 \cdot \text{RR} \\
& + w_3 \cdot \frac{\text{max}(\text{QRD}) - \text{QRD}}{\text{max}(\text{QRD})} \\
& + w_4 \cdot \frac{\text{max}(\text{HRD}) - \text{HRD}}{\text{max}(\text{HRD})} \\
& + w_5 \cdot \frac{\text{max}(\text{CD}) - \text{CD}}{\text{max}(\text{CD})},
\end{split}
\end{equation}
where \(w_1, w_2, w_3, w_4, w_5\) are the weights assigned to each metric based on the creditor’s priorities.

\textbf{Debtor’s Health Index (DHI):} This index measures the debtor's financial health during the recovery process. It incorporates several factors that capture the debtor's stability and vulnerability. The Debtor’s Health Index is calculated as:
\begin{equation}
\begin{split}
\text{DHI} = &\ w_6 \cdot \frac{\text{max}(\text{L1D}) - \text{L1D}}{\text{max}(\text{L1D})} \\
& + w_7 \cdot \frac{\text{max}(\text{L2D}) - \text{L2D}}{\text{max}(\text{L2D})} \\
& - w_8 \cdot \text{ATV}.
\end{split}
\end{equation}
Here, \(w_6, w_7, w_8\) are weights that balance the importance of each factor in determining the debtor’s health.

\textbf{Comprehensive Collection Index (CCI):} The Comprehensive Collection Index combines both the Creditor’s Recovery Index (CRI) and the Debtor’s Health Index (DHI) into a single metric that evaluates the overall balance between debt recovery and the debtor’s financial well-being. The index is calculated using the harmonic mean of the two indices, with a weight factor $\theta$ applied to the CRI:
\begin{equation}
\text{CCI} = \frac{2 \theta^2 \cdot \text{CRI} \cdot \text{DHI}}{\text{CRI} + \theta^2\cdot\text{DHI}}.
\end{equation}
In this formula, the weight factor $\theta$ indicates that the CRI is weighted $\theta$ times more than the DHI. In this study, $\theta$ is set to 2. This approach ensures that a high value in either the recovery index or the health index will influence the overall result, while emphasizing the importance of balancing both aspects.

The use of this weighted harmonic mean helps in evaluating different debt recovery strategies by considering both the creditor’s objectives and the debtor’s financial stability, thereby promoting a more balanced approach to debt collection.

The constant values used in the calculation process are shown in Table~\ref{tab:metrics_parameters}. In future research or application, these values may be adjusted depending on the specific requirements to better align with the needs.
\begin{table}[ht]
\centering
% \Large
\caption{Constants used in Average Metric Calculation.}
\resizebox{0.4\columnwidth}{!}{%
\begin{tabular}{@{}ll@{}}
\toprule
\textbf{Constant} & \textbf{Value} \\ \midrule
$w_1$ & 0.25 \\
$w_2$ & 0.25 \\
$w_3$ & 0.2 \\
$w_4$ & 0.15 \\
$w_5$ & 0.15 \\
$w_6$ & 1.5 \\
$w_7$ & 0.8 \\
$w_8$ & 1 \\
$\theta$ & 2 \\
$\text{max}(\text{QRD})$ & 180 \\
$\text{max}(\text{HRD})$ & 360 \\
$\text{max}(\text{CD})$ & 720 \\
$\text{max}(\text{L1D})$ & 30 \\
$\text{max}(\text{L2D})$ & 250 \\ 
\bottomrule
\end{tabular}%
}
\label{tab:metrics_parameters}
\end{table}