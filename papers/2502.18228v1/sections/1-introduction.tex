\section{Introduction}

% Large Language Models (LLMs) \citep{ChatGPT, Vicuna, LLaMA} have rapidly advanced in their language understanding \citep{minaee2024large} and reasoning capabilities \citep{yu2024natural,luo2023towards}, offering innovative solutions across various fields. By leveraging agent-based interaction, these models can perform emerging functions such as planning \citep{huang2024understanding}, reasoning \cite{aksitov2023rest}, and reflection \citep{renze2024self}, enabling them to assist humans in completing complex tasks without supervision. Among these applications, negotiation scenarios, which naturally align with conversational settings, have garnered significant attention \citep{sun2023sentiment,abdelnabi2023llm}.

Finance, as a negotiation-intensive field, involves the distribution and exchange of financial interests, requiring a higher level of understanding of information and rational decision-making \citep{Chan2006TheGN,Thompson1997TheMA}. Due to various personal financial issues, a large volume of non-performing loans (NPLs) arises each year across banks and financial companies, with debtors often being unable to repay their debts after prolonged overdue periods \citep{Ozili2019NonPerformingLA}. Negotiation and mediation are necessary to resolve their credit issues and minimize the losses for financial institutions (creditors) \citep{Firanda2021DebtCO}. Traditionally, the debt collection process has been labor-intensive, and data shows that in China, 3,800 financial institutions rely on outsourced specialized collection agencies to help recover non-performing assets \citep{Tang2018DebtCO}.


Previous automated debt collection dialogue models \citep{Floatbot2023GenerativeAI,Yahiya2024AutomatedDR} were primarily based on fixed-format notifications, where the models lacked communication and negotiation capabilities. Additionally, automated decision models \citep{Sancarlos2023TowardsAD,Jankowski2024DebtCM} related to changes in repayment strategies could not be directly integrated into the dialogue and were unable to update decisions in real time based on the debtor’s information provided during the conversation. A pressing need exists for novel approaches to automate \textbf{debt collection negotiations (DCN)}.

The rapid development of large language models (LLMs) \citep{Vicuna, LLaMA} and agent-based interactions \citep{luo2025llmpoweredmultiagentautomatedcrypto,chai2025flexquantelasticquantizationframework} built upon them has made it possible. Through emerging functions such as planning \citep{huang2024understanding}, reasoning \cite{aksitov2023rest}, and reflection \citep{renze2024self}, these models are now able to assist humans in completing more complex tasks. In this paper, we aim to explore the potential of using LLMs to support AI agents in performing this unexplored task. And firstly, it is crucial to develop a method to evaluate the performance in conducting DCN.

% Large language models have been widely applied in the financial sector, including trading and asset management \citep{luo2025llmpoweredmultiagentautomatedcrypto,chai2025flexquantelasticquantizationframework}. However, the capability of using LLMs to support AI agents in \textbf{debt collection negotiations (DCN)} remains largely unexplored. It is important to devise a method to evaluate the performance of AI agents in conducting DCN.

To develop a benchmark, the primary challenge lies in constructing a suitable dataset. In Section~\ref{sec:data}, to ensure both privacy and data validity, we utilized CTGAN \citep{ctgan} to generate synthetic data based on debt records from a leading financial technology company \footnote{The synthetic data generated in this work is publicly available.}. 
We supplement the debtor’s personal financial data through extraction and construction. Finally, we constructed a dataset containing 975 debt records. Based on this information, we provide a complete definition for DCN and the LLM-based negotiation process in Section~\ref{sec:task}.
% From historical dialogue data of debtors, we extracted the reasons for debt delinquency. Additionally, we simplified and structured personal financial information to facilitate subsequent evaluations. 


\begin{figure*}[htbp]
\vspace{-0.1in}
  \centering
  \includegraphics[width=1\textwidth]{latex/images/introdebt2.pdf}  
  \vspace{-0.15in}
  \caption{
An Example of a Debt Collection Negotiation (DCN). On the left and right sides are the information cards representing the data controlled by the debtor and the creditor, respectively. The black text represents the \textbf{basic debt information}, while the red text represents the \textbf{debtor’s personal financial information}. In the center, we demonstrate the use of LLM-based agents to simulate the dialogue. Each time, both the debtor and the creditor output a set of (Thoughts, Dialogue, Action). \textcolor[RGB]{184,96,41}{\textbf{Thoughts}} refers to their internal thought process, visible only to themselves; \textbf{Dialogue} represents the conversation in natural language; and \textcolor[RGB]{56,84,146}{\textbf{Action}} refers to the specific activities represented in a formal language within the dialogue. Each negotiation consists of multiple rounds of such interactions, ultimately leading to the negotiation outcome. The English text was automatically translated using Google Translate.}
\vspace{-0.1in}
\label{img:pipeline}
\end{figure*}


To comprehensively evaluate DCN, in Section~\ref{sec:eval} we proposed a holistic assessment framework encompassing \textbf{10} specific metrics in 4 aspects and \textbf{3} comprehensive index to thoroughly evaluate the negotiation process and its outcomes. From the perspective of the negotiation process, we evaluate the completeness and soundness of the dialogues. Regarding negotiation outcomes, we evaluate two key aspects: for creditors, we focus on debt recovery rate and collection efficiency, while for debtors, we assess financial health by predicting future asset changes based on negotiation results and individual financial data. The indices are introduced to integrate the opposing relationship between creditor’s interests and debtor's financial health.

In Section~\ref{sec:res}, we tested the performance of LLMs and found that they are unable to make appropriate decisions based on the debtor’s financial condition and are more likely to make unsuitable \textit{concessions} than human beings. This may result from the models’ excessive focus on harmony and agreement, leading debt collectors to overlook the rationality of decisions. To address this, inspired by the work of MetaGPT~\citep{Hong2023MetaGPTMP}, we designed an LLM-based \textbf{M}ulti-\textbf{A}gent \textbf{De}bt \textbf{N}egotiation \textbf{(MADeN)} framework for DCN in Section~\ref{sec:framework}. In this framework, we enhanced the basic \textbf{Communicating} agent with two additional modules: \textbf{(1) Planning}, where the LLM agent designs a rough decision framework and outlines the potential outcomes based on the debtor’s initial reasons and demands; \textbf{(2) Judging}, which evaluates the rationality of each action and provides optimization suggestions. Our method improves the comprehensive collection index by nearly \textbf{10\%}.
 
In addition, we attempted to use the DPO post-training method include \citep{rafailov2024directpreferenceoptimizationlanguage} with reject sampling \citep{liu2024statisticalrejectionsamplingimproves} to align the debt collector’s focus on recovery rate and efficiency in Section~\ref{sec:post}. On the Qwen2.5-7B \citep{qwen2.5} model, we observed improvements across various metrics.

Our contributions are summarized as follows:

\begin{itemize}[leftmargin=15px]
    
    \item We proposed a synthetic debt dataset and a comprehensive framework for evaluating LLM performance in debt collection negotiations (DCN), using 13 metrics to assess both the negotiation process and outcomes, enabling the testing and evaluation of different models.
    \item Our testing of mainstream LLMs on this task revealed that the models tend to make decisions with more unreasonable concessions compared to humans.
    \item We developed an multi-agent framework for DCN, incorporating two key modules to improve negotiation outcomes. Additionally, we explored post-training the model through rejection sampling on multi-agent data, which also enhanced the model’s performances.
    
    % \item We introduced the use of DPO with reject sampling to optimize the AI agent’s focus on debt recovery rate and efficiency, demonstrating its effectiveness through improved metrics on the Qwen2.5-7B model.
\end{itemize}

% §