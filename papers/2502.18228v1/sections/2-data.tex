\section{Data Collection}\label{sec:data}

Our data is primarily divided into two parts, as shown in Figure~\ref{img:pipeline}. The basic debt information is known to both the debtor and the creditor, while the debtor’s personal financial data is not accessible to the creditor. We now explain how each of these two data components was collected.

\subsection{Basic Debt Information}

The basic debt data primarily consists of personal information and debt-related information. We sampled from \textit{real debt data} provided by the financial company mentioned in introduction. To ensure privacy compliance, we used \textbf{CTGAN} \citep{ctgan} to generate synthetic data~\footnote{Please refer to our Ethical Considerations.}. We categorized the data by gender, overdue days and loan amount. Then we sampled it to match the distribution patterns of the original data. 

\subsection{Debtor’s personal financial data}

Debtor’s personal financial data collection involved two main components: \textbf{textual reasons for overdue} and \textbf{numerical financial information}. The reasons for overdue payments were extracted from real dialogue data and assigned to different categories based on their real \textit{distribution}. For numerical financial data, we simplified complex personal data into components such as total assets, average daily income, expenses, and surplus. Since this data is typically unavailable, we used a linear model with Gaussian noise, based on historical data correlations, to estimate these values. 

Finally, we collected \textbf{975} debt records, with \textbf{390} records placed in the test set and the remaining \textbf{585} records in the training set (The subsequent evaluations are conducted on the test set). Details of the debtor category distribution can be found in Appendix~\ref{Distribution}.
% The surplus was calculated as the difference between income and expenses.





\section{Task Formulation} \label{sec:task}


\begin{algorithm}[!htb]
    \caption{\label{alg:1}Debt Collection Negotiation Process}
    \label{alg:2}
    \begin{algorithmic}
        \STATE \textbf{Initialize:} Action Set $S_A$, Basic Debt Information $I_b$, Personal Financial Information $I_p$, Agent \text{Creditor}, Agent \text{Debtor}, Maximum Turns $t_m$, Negotiation Dimensions Set $S_R$, Negotiation Result Dictionary $D$
        
        \STATE {$\text{Creditor} \gets \text{Creditor}(I_b, S_A)$}
        \STATE {$\text{Debtor} \gets \text{Debtor}(I_b, I_p, S_A)$}
        \STATE {$t \gets 0$}
        \STATE {$D \gets \{\}$} 
        
        \FOR{$t < t_m$}
            \STATE $ A_c, \text{Dialogue}_c \gets \text{Creditor.generate}$
            
            \STATE $\text{Debtor} \gets \text{Debtor}(A_c, \text{Dialogue}_c)$
            \STATE $A_d, \text{Dialogue}_d \gets \text{Debtor.generate}$
            \IF{$A_d == \texttt{accept}$} 
                \STATE $D[A_d.key] \gets A_d.value $ 

            \ENDIF
            \IF {$D$ covers $S_R$}
                \RETURN $D$
            \ENDIF   
            \STATE $\text{Creditor} \gets \text{Creditor}(A_d, \text{Dialogue}_d)$
            \STATE {$t \gets t + 1$}
        \ENDFOR
        
        \RETURN None
    \end{algorithmic}
\end{algorithm}


\subsection{Definition and Objectives}\label{obj}

\begin{table*}[ht]
\centering
  % \vspace{-0.1in}

\caption{\label{dimdes}Debt Collection Negotiation Dimensions}
\vspace{-0.1in}
    \setlength{\tabcolsep}{3.5mm}{
    \resizebox{\textwidth}{!}{%
        \begin{tabular}{lll}
        \toprule
        \textbf{Dimension} & \textbf{Range} & \textbf{Description} \\
        \midrule
        Discount Ratio  & 0 - 30\% & The portion of debt waived by the creditor to ease repayment. \\
        Immediate Payment Ratio & 5\% - 50\% & The portion of debt that must be repaid immediately, typically at least 5\%. \\
        Immediate Payment Time & 1 - 14 (days) & A grace period of up to 14 days for the debtor to make the immediate repayment. \\
        Installment Periods  & 3 - 24 (months) & The duration for repaying the remaining debt in installments.\\
        \bottomrule
        \end{tabular}
        \vspace{-0.1in}
    }}
    \vspace{-0.1in}
\end{table*}

Debt collection negotiations (DCN) refers to negotiations initiated by creditors to recover outstanding debts and restore the debtor’s credit, due to the debtor’s inability to repay on time because of personal financial issues. The measures for negotiating the resolution of non-performing loans generally include deferral, debt forgiveness, collateralization, conversion, and installment payments \citep{DFRatings2019,Lankao2023}. Among these, deferral, debt forgiveness, and installment payments are the most commonly used. We have distilled them into four dimensions: Discount Ratio, Immediate Payment Ratio, Immediate Payment Time and Installment Periods\footnote{Refer to \url{https://www.boc.cn/bcservice/bc3/bc31/201203/t20120331_1767028.html} for the calculation of installment interest.}. Table~\ref{dimdes} presents the range of values and a brief description of each dimension, and the detailed explanations are provided in Appendix~\ref{sec:dim}. Through negotiations on these four aspects, the goal of both parties is to reach a \textit{mutually acceptable outcome} that allows the debtor to resolve their outstanding debt in a manageable way.

% Discount Ratio, Immediate Payment Ratio, Immediate Payment Time and Installment Periods. 



\subsection{Future Economic Predictions for Debtors}

\begin{figure*}[htbp]
% \vspace{-0.1in}
  \centering
  \includegraphics[width=\textwidth]{latex/images/preb.pdf}  
  \vspace{-0.3in}
  \caption{\label{image:pred}
The future trajectories of the debtor’s remaining assets and outstanding debt under three installment plans (6, 12, and 18 months from left to right) are shown, with all other variables held constant. The 6-month plan causes the debtor’s assets to fall \textbf{below zero}, making repayment impossible. In contrast, the 12-month and 18-month plans maintain a healthy asset level, though the 18-month plan significantly \textbf{reduces recovery efficiency}. The 12-month plan is the most balanced solution. Different background colors represent five difficulty tiers, with Tier 1 being the most challenging. The specific ranges and descriptions of the tiers are provided in Appendix~\ref{app:diff_cat}.}
\vspace{-0.1in}
\end{figure*}

After obtaining the negotiation results and integrating them with the debtor’s current financial model, we can project changes in their assets and remaining debt over the next \textit{two years}. Figure~\ref{image:pred} shows one debtor’s economic trajectory under three installment scenarios. In one scenario, the debtor’s assets fall into negative values, indicating a failed negotiation. In another, a too lenient installment plan reduces recovery efficiency. These scenarios provide a basis for evaluating negotiation outcomes, which will be discussed in Section~\ref{sec:eval}.
% , including success rates and metrics such as debt recovery speed, recovery ratio, and changes in personal assets,

\subsection{Negotiation Process}

As shown is Figure~\ref{img:pipeline}, our negotiation process is a variant of the bargaining process designed by \citet{xia2024measuringbargainingabilitiesllms}. To formally articulate the negotiation between agents, we define the relevant concepts and variables in Table  ~\ref{tab:debt_variable_app}. A brief pseudo code of the process is Algorithm~\ref{alg:1}.

In the action set, \textit{“ask”}, \textit{“reject”} and \textit{“accept”} represent three different operations for each negotiation dimension. After several rounds of negotiation and discussion, the debtor and the collector can be considered to have reached an agreement when consensus \textit{(“accept”)} is achieved on all 4 negotiation objectives.



