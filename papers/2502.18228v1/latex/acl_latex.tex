% This must be in the first 5 lines to tell arXiv to use pdfLaTeX, which is strongly recommended.
\pdfoutput=1
% In particular, the hyperref package requires pdfLaTeX in order to break URLs across lines.

\documentclass[11pt]{article}

% Change "review" to "final" to generate the final (sometimes called camera-ready) version.
% Change to "preprint" to generate a non-anonymous version with page numbers.
\usepackage[preprint]{acl}

% Standard package includes
\usepackage{times}
\usepackage{latexsym}
\usepackage{longtable}
% For proper rendering and hyphenation of words containing Latin characters (including in bib files)
\usepackage[T1]{fontenc}
% For Vietnamese characters
% \usepackage[T5]{fontenc}
% See https://www.latex-project.org/help/documentation/encguide.pdf for other character sets

% This assumes your files are encoded as UTF8
\usepackage[utf8]{inputenc}
\usepackage{amsmath}
\usepackage{algorithm}
\usepackage{algorithmic}
% This is not strictly necessary, and may be commented out,
% but it will improve the layout of the manuscript,
% and will typically save some space.
\usepackage{microtype}
\usepackage{booktabs} % For creating professional tables
% This is also not strictly necessary, and may be commented out.
% However, it will improve the aesthetics of text in
% the typewriter font.
\usepackage{inconsolata}
\usepackage{enumitem}
%Including images in your LaTeX document requires adding
%additional package(s)
\usepackage{graphicx}
\usepackage{array}

% If the title and author information does not fit in the area allocated, uncomment the following
%
%\setlength\titlebox{<dim>}
%
% and set <dim> to something 5cm or larger.

\title{Debt Collection Negotiations with Large Language Models: An Evaluation System and Optimizing Decision Making with Multi-Agent}

% Author information can be set in various styles:
% For several authors from the same institution:
% \author{Author 1 \and ... \and Author n \\
%         Address line \\ ... \\ Address line}
% if the names do not fit well on one line use
%         Author 1 \\ {\bf Author 2} \\ ... \\ {\bf Author n} \\
% For authors from different institutions:
% \author{Author 1 \\ Address line \\  ... \\ Address line
%         \And  ... \And
%         Author n \\ Address line \\ ... \\ Address line}
% To start a separate ``row'' of authors use \AND, as in
% \author{Author 1 \\ Address line \\  ... \\ Address line
%         \AND
%         Author 2 \\ Address line \\ ... \\ Address line \And
%         Author 3 \\ Address line \\ ... \\ Address line}

% \author{Xiaofeng Wang \\
% Shanghai Jiao Tong University \& Ant Group\\
% banyedy@sjtu.edu.cn \\
% \and
% Zhixin Zhang \\
% Ant Group \\
% zhangzhixin.zzx@antgroup.com \\
% \and
% Jinguang Zheng \\
% Ant Group \\
% zhengjinguang.zhen@antgroup.com \\
% \and
% Yiming Ai \\
% Shanghai Jiao Tong University \\
% yiming.ai@sjtu.edu.cn \\ % Assuming email based on the pattern
% \and
% Rui Wang \thanks{Corresponding Author}\\
% Shanghai Jiao Tong University \\
% wangrui12@sjtu.edu.cn
% }

\author{
Xiaofeng Wang\textsuperscript{1,2}, Zhixin Zhang\textsuperscript{2}\footnotemark[1], Jinguang Zheng\textsuperscript{2}, Yiming Ai\textsuperscript{1,2}, Rui Wang\textsuperscript{1}\thanks{Corresponding Authors} \\
\textsuperscript{1}Shanghai Jiao Tong University \\
\textsuperscript{2}Ant Group \\
\{banyedy, wangrui12\}@sjtu.edu.cn \\
\{zhangzhixin.zzx, zhengjinguang.zhen\}@antgroup.com \\
}



  % \author{Xiaofeng Wang \\
%     Shanghai Jiao Tong University \\
%     \texttt{banyedy@sjtu.edu.cn}}

%\author{
%  \textbf{First Author\textsuperscript{1}},
%  \textbf{Second Author\textsuperscript{1,2}},
%  \textbf{Third T. Author\textsuperscript{1}},
%  \textbf{Fourth Author\textsuperscript{1}},
%\\
%  \textbf{Fifth Author\textsuperscript{1,2}},
%  \textbf{Sixth Author\textsuperscript{1}},
%  \textbf{Seventh Author\textsuperscript{1}},
%  \textbf{Eighth Author \textsuperscript{1,2,3,4}},
%\\
%  \textbf{Ninth Author\textsuperscript{1}},
%  \textbf{Tenth Author\textsuperscript{1}},
%  \textbf{Eleventh E. Author\textsuperscript{1,2,3,4,5}},
%  \textbf{Twelfth Author\textsuperscript{1}},
%\\
%  \textbf{Thirteenth Author\textsuperscript{3}},
%  \textbf{Fourteenth F. Author\textsuperscript{2,4}},
%  \textbf{Fifteenth Author\textsuperscript{1}},
%  \textbf{Sixteenth Author\textsuperscript{1}},
%\\
%  \textbf{Seventeenth S. Author\textsuperscript{4,5}},
%  \textbf{Eighteenth Author\textsuperscript{3,4}},
%  \textbf{Nineteenth N. Author\textsuperscript{2,5}},
%  \textbf{Twentieth Author\textsuperscript{1}}
%\\
%\\
%  \textsuperscript{1}Affiliation 1,
%  \textsuperscript{2}Affiliation 2,
%  \textsuperscript{3}Affiliation 3,
%  \textsuperscript{4}Affiliation 4,
%  \textsuperscript{5}Affiliation 5
%\\
%  \small{
%    \textbf{Correspondence:} \href{mailto:email@domain}{email@domain}
%  }
%}

\begin{document}
\maketitle
\begin{abstract}
Debt collection negotiations (DCN) are vital for managing non-performing loans (NPLs) and reducing creditor losses. Traditional methods are labor-intensive, while large language models (LLMs) offer promising automation potential. However, prior systems lacked dynamic negotiation and real-time decision-making capabilities. This paper explores LLMs in automating DCN and proposes a novel evaluation framework with 13 metrics across 4 aspects. Our experiments reveal that LLMs tend to over-concede compared to human negotiators. To address this, we propose the \textbf{M}ulti-\textbf{A}gent \textbf{De}bt \textbf{N}egotiation \textbf{(MADeN)} framework, incorporating planning and judging modules to improve decision rationality. We also apply post-training techniques, including DPO with rejection sampling, to optimize performance. Our studies provide valuable insights for practitioners and researchers seeking to enhance efficiency and outcomes in this domain.
% Debt collection negotiations (DCN) are a crucial aspect of the financial industry, particularly in addressing non-performing loans (NPLs) and minimizing losses for creditors. Traditional debt collection methods are labor-intensive, and recent advancements in large language models (LLMs) offer promising potential for automating these negotiations. However, previous automated debt collection systems lacked the capacity for dynamic negotiation and real-time decision-making. This paper explores the use of LLMs in supporting AI agents for conducting DCN and proposes a novel evaluation framework for assessing their performance. We introduce a synthetic debt dataset and a comprehensive evaluation system consisting of \textbf{13} metrics across 4 dimensions, including the negotiation process and outcomes. Our experiments show that LLMs tend to make excessive concessions compared to human negotiators. To address this, we propose a \textbf{M}ulti-\textbf{A}gent \textbf{De}bt \textbf{N}egotiation \textbf{(MADeN)} framework with two key modules: planning and judging, which improve negotiation efficiency and recovery outcomes. Additionally, we explore the use of post-training techniques such as DPO with rejection sampling to optimize model performance. Our results demonstrate significant improvements in negotiation outcomes and decision rationality, contributing to the advancement of AI-powered debt collection negotiations.
% Debt collection is a complex process that requires balancing efficiency, debtor satisfaction, and ethical considerations. This study explores the potential of large language models (LLMs) to optimize debt collection communication through a multi-objective perspective. By leveraging LLMs, we aim to improve the effectiveness of communication strategies, ensure fairness and empathy in debtor interactions, and maintain compliance with regulatory standards. We propose a framework that integrates multi-objective decision-making techniques with LLM-driven communication, evaluating its performance across metrics such as resolution rates, response sentiment, and stakeholder satisfaction. The findings highlight the advantages and challenges of employing LLMs in real-world debt collection scenarios, providing valuable insights for practitioners and researchers seeking to enhance outcomes in this domain.
\end{abstract}

\section{Introduction}

% State of the world (robots for creative activites)
The term ``robot,'' originally signifying `forced labor,' has long been associated with labor and work. Robots have demonstrated their utility in various automated productive and social contexts, where the primary goals are improving productivity, safety, and fostering social interactions with humans~\cite{simoes2022designing, weidemann2021role, honig2018understanding}. However, an increasing number of cases feature using of robots in creative settings. Unlike productive contexts, where the focus is on efficiency and task completion~\cite{arents2022smart}, or social contexts, where communication and trust are prioritized~\cite{nam2020trust, saunderson2019robots}, creative environments prioritize artistic innovation and expression~\cite{hsueh2024counts}. This shift fundamentally alters the dynamics of human-robot interaction, redefining the roles and expectations for both humans and robots.

For instance, robots’ social behaviors are leveraged to support the generation and expression of creative ideas~\cite{hu2021exploring, sandoval2022human, alves2020creativity}, and programmable robotic movements and trajectories are employed to inspire artistic activities such as sketching~\cite{lin2020your}. These studies often engage participants from creative fields who possess limited prior experience with robotics, and are typically conducted in short-term, experimental settings. Consequently, the findings from these studies remain constrained since much can be learned from professional practitioners' experiences to inform system design such as digital fabrication~\cite{hirsch2023nothing}. There is a notable gap in research examining the long-term, active, and practical experience of integrating robotic systems into the creative processes. As a result, the deeper insights into how robots facilitate and shape creative processes, beyond simply augmenting human creativity, remain underexplored. In this study, we aim to better understand the impacts of robots on creative processes and outcomes.

As early as Leonardo da Vinci's 16th century ``Automaton,'' artists have explored the creative affordances of robotic systems~\cite{shanken2002cybernetics, pagliarini2009development, jeon2017robotic}. The artistic creation process typically encompasses various stages, including the exploration of materials and techniques, ongoing experimentation and iteration, and the continual refinement of the artists' insights into their creative subjects~\cite{lewis2023art, sturdee2022state}. Therefore, investigating the artistic process involving robots offers an opportunity to gain deeper insights into robots' creative potential. Robotic art, in particular, provides a compelling case for this exploration.

We define robotic art as artworks that utilize robotic or automated machines to create artistic experiences and tangible artifacts. One example is robotic installation art, in which robots are programmed to follow specific rules that embody the artist’s expression (\autoref{fig:teaser} (a)). Another example is responsive art, in which robots react to their environment, with behaviors that change over time or in response to spectators (\autoref{fig:teaser} (b)). Additionally, there are robotic creators, which possess a degree of agency, allowing them to collaborate with human artists and produce works that extend beyond mere replication of human-created art (\autoref{fig:teaser} (c) and (d)). As such, robotic art becomes a rich case for exploring human-machine interactions in creative contexts. Gaining a deeper understanding of how robots facilitate artistic expression can provide insights for designing computing systems to support creative activities~\cite{gomez2021robot}.

% Therefore, we did...
We draw on semi-structured, in-depth interviews with renowned professional robotic artists to investigate the use of robots in artistic practice. Specifically, our goal is to understand how artistic exploration of robotic systems challenges conventional assumptions about the functions of robots, such as their roles in automating repetitive tasks or serving human needs. We also explore the implications of robots in the artistic process and examine how creativity may emerge within robotic art. To address these interrelated inquiries, our study focuses on the practice of robotic art, posing the research question: \textit{How do robotic artists utilize robots in their artistic practice?} We approach this inquiry through the perspectives and experiences of robotic artists, who creatively design, modify, and repurpose robotic systems for artistic expression and exploration.

% The key findings are...
Our findings highlight the social, material, and temporal dimensions of artists' practices that shape their creativity and artistic outcomes. The creation of robotic art is largely a social process, as artists receive both explicit and implicit feedback through the audience's reactions and reception of their work. Simultaneously, the embodiment and malfunctions inherent to robotic systems drive artistic experimentation. The temporal processes of creation and exhibition, beyond just the final product, further enhance the creative value. Our empirical analysis presents how creativity emerges through the interplay of social, material, and temporal interactions among artists, robots, audiences, and the environment.

% The contributions of this work are...
We make two main contributions to HCI in this study. 
First, we elucidate the interactive mechanisms among key actors---human creators, machines, audiences, and environments---within the practice of robotic art, a topic that remains underexplored in HCI. Our findings reveal the significance of sociality (e.g., interactions between artists and audiences), materiality (e.g., the embodiment and malfunctions of robots), and temporality (e.g., the processes of creation and exhibition) in shaping creative values. We propose that these three facets are central to the creative process and facilitate the emergence of creativity in robotic art.
Second, drawing from the findings, we offer implications for \textit{socially informed}, \textit{material-attentive}, and \textit{process-oriented} creation with computing systems. We suggest leveraging these three aspects to enhance creativity and the creative experience. Specifically, we discuss the value of incorporating implicit audience feedback, designing with technical malfunctions, and focusing on the post-creation process to foster alternative creative experiences with machines~\cite{alter2010designing, juarez2022glitch}.



\section{Data Collection}\label{sec:data}

Our data is primarily divided into two parts, as shown in Figure~\ref{img:pipeline}. The basic debt information is known to both the debtor and the creditor, while the debtor’s personal financial data is not accessible to the creditor. We now explain how each of these two data components was collected.

\subsection{Basic Debt Information}

The basic debt data primarily consists of personal information and debt-related information. We sampled from \textit{real debt data} provided by the financial company mentioned in introduction. To ensure privacy compliance, we used \textbf{CTGAN} \citep{ctgan} to generate synthetic data~\footnote{Please refer to our Ethical Considerations.}. We categorized the data by gender, overdue days and loan amount. Then we sampled it to match the distribution patterns of the original data. 

\subsection{Debtor’s personal financial data}

Debtor’s personal financial data collection involved two main components: \textbf{textual reasons for overdue} and \textbf{numerical financial information}. The reasons for overdue payments were extracted from real dialogue data and assigned to different categories based on their real \textit{distribution}. For numerical financial data, we simplified complex personal data into components such as total assets, average daily income, expenses, and surplus. Since this data is typically unavailable, we used a linear model with Gaussian noise, based on historical data correlations, to estimate these values. 

Finally, we collected \textbf{975} debt records, with \textbf{390} records placed in the test set and the remaining \textbf{585} records in the training set (The subsequent evaluations are conducted on the test set). Details of the debtor category distribution can be found in Appendix~\ref{Distribution}.
% The surplus was calculated as the difference between income and expenses.





\section{Task Formulation} \label{sec:task}


\begin{algorithm}[!htb]
    \caption{\label{alg:1}Debt Collection Negotiation Process}
    \label{alg:2}
    \begin{algorithmic}
        \STATE \textbf{Initialize:} Action Set $S_A$, Basic Debt Information $I_b$, Personal Financial Information $I_p$, Agent \text{Creditor}, Agent \text{Debtor}, Maximum Turns $t_m$, Negotiation Dimensions Set $S_R$, Negotiation Result Dictionary $D$
        
        \STATE {$\text{Creditor} \gets \text{Creditor}(I_b, S_A)$}
        \STATE {$\text{Debtor} \gets \text{Debtor}(I_b, I_p, S_A)$}
        \STATE {$t \gets 0$}
        \STATE {$D \gets \{\}$} 
        
        \FOR{$t < t_m$}
            \STATE $ A_c, \text{Dialogue}_c \gets \text{Creditor.generate}$
            
            \STATE $\text{Debtor} \gets \text{Debtor}(A_c, \text{Dialogue}_c)$
            \STATE $A_d, \text{Dialogue}_d \gets \text{Debtor.generate}$
            \IF{$A_d == \texttt{accept}$} 
                \STATE $D[A_d.key] \gets A_d.value $ 

            \ENDIF
            \IF {$D$ covers $S_R$}
                \RETURN $D$
            \ENDIF   
            \STATE $\text{Creditor} \gets \text{Creditor}(A_d, \text{Dialogue}_d)$
            \STATE {$t \gets t + 1$}
        \ENDFOR
        
        \RETURN None
    \end{algorithmic}
\end{algorithm}


\subsection{Definition and Objectives}\label{obj}

\begin{table*}[ht]
\centering
  % \vspace{-0.1in}

\caption{\label{dimdes}Debt Collection Negotiation Dimensions}
\vspace{-0.1in}
    \setlength{\tabcolsep}{3.5mm}{
    \resizebox{\textwidth}{!}{%
        \begin{tabular}{lll}
        \toprule
        \textbf{Dimension} & \textbf{Range} & \textbf{Description} \\
        \midrule
        Discount Ratio  & 0 - 30\% & The portion of debt waived by the creditor to ease repayment. \\
        Immediate Payment Ratio & 5\% - 50\% & The portion of debt that must be repaid immediately, typically at least 5\%. \\
        Immediate Payment Time & 1 - 14 (days) & A grace period of up to 14 days for the debtor to make the immediate repayment. \\
        Installment Periods  & 3 - 24 (months) & The duration for repaying the remaining debt in installments.\\
        \bottomrule
        \end{tabular}
        \vspace{-0.1in}
    }}
    \vspace{-0.1in}
\end{table*}

Debt collection negotiations (DCN) refers to negotiations initiated by creditors to recover outstanding debts and restore the debtor’s credit, due to the debtor’s inability to repay on time because of personal financial issues. The measures for negotiating the resolution of non-performing loans generally include deferral, debt forgiveness, collateralization, conversion, and installment payments \citep{DFRatings2019,Lankao2023}. Among these, deferral, debt forgiveness, and installment payments are the most commonly used. We have distilled them into four dimensions: Discount Ratio, Immediate Payment Ratio, Immediate Payment Time and Installment Periods\footnote{Refer to \url{https://www.boc.cn/bcservice/bc3/bc31/201203/t20120331_1767028.html} for the calculation of installment interest.}. Table~\ref{dimdes} presents the range of values and a brief description of each dimension, and the detailed explanations are provided in Appendix~\ref{sec:dim}. Through negotiations on these four aspects, the goal of both parties is to reach a \textit{mutually acceptable outcome} that allows the debtor to resolve their outstanding debt in a manageable way.

% Discount Ratio, Immediate Payment Ratio, Immediate Payment Time and Installment Periods. 



\subsection{Future Economic Predictions for Debtors}

\begin{figure*}[htbp]
% \vspace{-0.1in}
  \centering
  \includegraphics[width=\textwidth]{latex/images/preb.pdf}  
  \vspace{-0.3in}
  \caption{\label{image:pred}
The future trajectories of the debtor’s remaining assets and outstanding debt under three installment plans (6, 12, and 18 months from left to right) are shown, with all other variables held constant. The 6-month plan causes the debtor’s assets to fall \textbf{below zero}, making repayment impossible. In contrast, the 12-month and 18-month plans maintain a healthy asset level, though the 18-month plan significantly \textbf{reduces recovery efficiency}. The 12-month plan is the most balanced solution. Different background colors represent five difficulty tiers, with Tier 1 being the most challenging. The specific ranges and descriptions of the tiers are provided in Appendix~\ref{app:diff_cat}.}
\vspace{-0.1in}
\end{figure*}

After obtaining the negotiation results and integrating them with the debtor’s current financial model, we can project changes in their assets and remaining debt over the next \textit{two years}. Figure~\ref{image:pred} shows one debtor’s economic trajectory under three installment scenarios. In one scenario, the debtor’s assets fall into negative values, indicating a failed negotiation. In another, a too lenient installment plan reduces recovery efficiency. These scenarios provide a basis for evaluating negotiation outcomes, which will be discussed in Section~\ref{sec:eval}.
% , including success rates and metrics such as debt recovery speed, recovery ratio, and changes in personal assets,

\subsection{Negotiation Process}

As shown is Figure~\ref{img:pipeline}, our negotiation process is a variant of the bargaining process designed by \citet{xia2024measuringbargainingabilitiesllms}. To formally articulate the negotiation between agents, we define the relevant concepts and variables in Table  ~\ref{tab:debt_variable_app}. A brief pseudo code of the process is Algorithm~\ref{alg:1}.

In the action set, \textit{“ask”}, \textit{“reject”} and \textit{“accept”} represent three different operations for each negotiation dimension. After several rounds of negotiation and discussion, the debtor and the collector can be considered to have reached an agreement when consensus \textit{(“accept”)} is achieved on all 4 negotiation objectives.




\begin{figure}[htbp]
\vspace{-0.1in}
  \centering
  \includegraphics[width=0.48\textwidth]{latex/images/metric.pdf}  
  \vspace{-0.1in}
  \caption{\label{image:eval}
 Evaluation system of DCN.}
\vspace{-0.1in}
\label{img:metric}
\end{figure}


\section{Evaluation System}\label{sec:eval}

Different from traditional negotiation evaluations, we argue that the DCN task requires a more comprehensive assessment framework. As illustrated in Figure ~\ref{image:eval}, we developed a evaluation system based on four aspects and extended several average metrics for a comprehensive assessment. 

% we propose a four-dimensional evaluation system, which includes conversational ability, debt recovery rate, collection efficiency, and debtor’s financial health, encompassing thirteen metrics.

\subsection{Segmented Evaluation Metrics}

In this section, we provide a general overview of the \textbf{10 metrics} across the four segmented aspects. Detailed descriptions, the evaluation process, and calculation formulas are further discussed in Appendix~\ref{app:metric}.

\textbf{Conversational Ability (§\ref{app:me_conv}).} Conversational ability is crucial in negotiation processes for effective communication and mutual understanding. We evaluate it using two metrics: \textit{(i) Dialogue Soundness}\textbf{ (DS)} is assessed on a five-point scale, measuring the fluency, naturalness and coherency of responses; \textit{(ii) Dialogue Completeness}\textbf{ (DC)} is an automated metric that evaluates whether four objectives are all addressed during the dialogue.

\textbf{Debt Recovery (§\ref{app:me_rec}).} In debt collection, the primary goal is to recover as much debt as possible. We evaluate this using two key metrics: \textit{(i) Success Recovery Rate} \textbf{(SR)} measures the proportion of samples where repayment can be successfully completed, based on the debtor’s future ability to meet repayment goals. \textit{(ii) Recovery Rate}\textbf{ (RR)} reflects the portion of the debt that has been successfully recovered by the creditor, calculated as the average recovery ratio across all test samples.

\textbf{Collection Efficiency (§\ref{app:me_col}).} Collection efficiency refers to how quickly a debtor can repay their debt. We monitor the timing of repayments using three key metrics: \textit{(i) 25\% Recovery Date}\textbf{ (QRD)} is the estimated date when the debtor has completed 25\% of the debt repayment, with earlier dates indicating quicker repayment; (\textit{ii) 50\% Recovery Date} \textbf{(HRD)} marks the completion of 50\% of the repayment, offering insight into the debtor’s ongoing repayment ability. \textit{(iii) The Completion Date} \textbf{(CD)} is the date when the debtor has fully repaid their debt, with a shorter completion date indicating a faster recovery process.

\textbf{Debtor’s Financial Health (§\ref{app:me_hea}).} The debtor’s financial health plays a critical role in successful debt recovery. It affects both the debtor’s ability to repay and the speed at which repayment occurs. We assess financial health using three metrics: \textit{(i) L1 Tier Days} \textbf{(L1D)} tracks the number of days the debtor remains in the most difficult financial tier (L1), with longer durations indicating higher risk of default; \textit{(ii) L2 Tier Days} \textbf{(L2D)} similarly tracks the days in the second most difficult financial tier (L2), which still reflects financial strain; \textit{(iii) Asset Tier Variance} \textbf{(ATV)} captures the variance in the debtor’s asset tier over a year, providing insight into the stability of their financial condition. 

\subsection{Comprehensive Indices}

We find that the indicators for recovery and efficiency are often \textit{\textbf{inversely related}} to the debtor’s financial condition in debt collection. To balance these conflicting objectives, we introduce three average metrics (detailed description and calculation process can be found in Appendic~\ref{app:me_ave}):

\textbf{(i) Creditor’s Recovery Index (CRI):} CRI is the \textit{weighted average} of five indicators from Debt Recovery and Collection Efficiency. It reflects an evaluation of the overall collection process by debt collectors, disregarding debtor-related factors. A higher value is more favorable to the creditor.

\textbf{(ii) Debtor’s Health Index (DHI):} DHI is the \textit{weighted average} of three indicators from Debtor’s Financial Health. It assesses the financial well-being of the debtor throughout the repayment process, with a higher value indicating a greater probability of the debtor adhering to the repayment plan.

\textbf{(iii) Comprehensive Collection Index (CCI):} CCI is the \textit{harmonic mean} of CRI and DHI. It provides a comprehensive evaluation of the negotiation outcome, where a higher value signifies the maximization of debt recovery and efficiency while ensuring the debtor’s financial health.



\section{Experiments and Results} \label{sec:res}


\begin{table*}[ht]
\vspace{-0.1in}
    \centering
    \caption{\label{img:mainresult}The performances of some models as debt collectors (~\textsuperscript{*} denotes the second-best performance).}
    \vspace{-0.1in}
    \setlength{\tabcolsep}{3.5mm}{
    \resizebox{\textwidth}{!}{%
    \begin{tabular}{l|cc|cc|ccc|ccc|cccc}
        \toprule
         & \multicolumn{2}{c}{\textbf{Conversation}} & \multicolumn{2}{c}{\textbf{Debt Recovery}} & \multicolumn{3}{c}{\textbf{Collection Efficiency}} & \multicolumn{3}{c}{\textbf{Debtor’s Financial Health}} & \multicolumn{3}{c}{\textbf{Average Metrics}}\\
        \cmidrule(lr){2-3} \cmidrule(lr){4-5} \cmidrule(lr){6-8} \cmidrule(lr){9-11} \cmidrule(lr){12-14} 
        \textbf{Model} & \textbf{DC}$ \uparrow $ & \textbf{DS}$ \uparrow $ & \textbf{SR}$ \uparrow $ & \textbf{RR(\%)}$ \uparrow $ & \textbf{QRD}$ \downarrow $ & \textbf{HRD}$ \downarrow $ & \textbf{CD}$ \downarrow $ & \textbf{L1D}$ \downarrow $ & \textbf{L2D}$ \downarrow $& \textbf{ATV}$ \downarrow $ & \textbf{CRI}$ \uparrow $ & \textbf{DHI}$ \uparrow $ & \textbf{CCI}$ \uparrow $\\
        \midrule
        Qwen-2.5-7B  & 0.94 & 4.57 & 0.98 & 87.15 & 46.04 & 214.04 & 436.84 & \textbf{2.82} & 79.46 & \textbf{0.84\textsuperscript{*}} & 0.732 & \textbf{0.793} & 0.743\\
        Qwen-2.5-14B & 0.94 & 4.60 & 0.96 & 89.62 & 28.60 & 154.60 & 358.80 & 6.30 & 79.82 & 0.88 & 0.793 & 0.613 & 0.749\\
        Qwen-2.5-72B & 0.96 & 4.75 & 0.98 & 88.50 & 36.98 & 185.18 & 404.98 & 3.76 & \textbf{78.84\textsuperscript{*}} & \textbf{0.83} & 0.764 & \textbf{0.767\textsuperscript{*}} & 0.764\\
        LLaMa-3-8B  & 0.91 & 3.64 & \textbf{1.00} & 89.01 & 51.36 & 184.02 & 399.02 & 3.38 & 88.02 & 0.87 & 0.756 & 0.713 & 0.747\\
        LLaMa-3-70B & 0.87 & 3.94 & 0.98 & 92.24 & 36.72 & 157.32 & 371.12 & 4.50 & 79.56 & 0.87 & 0.792 & 0.695 & 0.771\\
        GPT-4o & \textbf{1.00} & 4.65 & \textbf{1.00} & 95.76 & 27.00 & \textbf{128.40\textsuperscript{*}} & \textbf{297.20\textsuperscript{*}} & 6.18 & 85.18 & 0.90 & 0.844 & 0.580 & 0.774\\
        GPT-4o-mini & 0.99 & 4.61 & 0.96 & \textbf{96.32\textsuperscript{*}} & 31.60 & 131.20 & 312.00 & 6.30 & 84.08 & 0.89 & 0.836 & 0.589 & 0.771\\
        o1-mini & \textbf{1.00} & 4.68 & 0.94 & 94.61 & 29.52 & 140.52 & 352.52 & 5.58 & 83.80 & 0.89 & 0.807 & 0.619 & 0.760\\
        Doubao-pro & 0.98 & \textbf{4.91\textsuperscript{*}} & 0.96 & 93.11 & \textbf{21.22\textsuperscript{*}} & 143.02 & 365.22 & 5.98 & 83.68 & 0.89 & 0.814 & 0.603 & 0.760\\
        Claude-3.5 & \textbf{1.00} & 4.59 & 0.98 & 93.30 & 34.92 & 140.52 & 312.32 & \textbf{3.32\textsuperscript{*}} & 87.30 & 0.89 & 0.816 & 0.698 & \textbf{0.789\textsuperscript{*}}\\
        MiniMax & \textbf{1.00} & 4.75 & 0.96 & 92.77 & 38.66 & 167.66 & 401.26 & 7.12 & \textbf{76.44} & 0.88 & 0.776 & 0.591 & 0.730\\
        SenseChat & \textbf{1.00} & 4.70 & 0.98 & 89.28 & 34.56 & 155.76 & 354.56 & 5.24 & 81.14 & 0.87 & 0.791 & 0.661 & 0.761\\
        Deepseek-V3 & \textbf{1.00} & 4.85 & 0.99 & 91.65 & 28.40 & 141.20 & 313.60 & 5.42 & 83.82 & 0.89 & \textbf{0.818\textsuperscript{*}} & 0.625 & 0.771\\
        Deepseek-R1 & 0.98 & 4.81 & 0.98 & 93.10 & 37.72 & 146.32 & 348.12 & 5.68 & 83.94 & 0.88 & 0.802 & 0.624 & 0.759\\
        \midrule
        Human & \textbf{1.00} & \textbf{4.93} & \textbf{1.00} & \textbf{98.50} & \textbf{16.73} & \textbf{119.49} & \textbf{260.90} & 3.81 & 78.49 & 0.86 & \textbf{0.870} & 0.736 & \textbf{0.840}\\
        \bottomrule
    \end{tabular}%
 }}
 \label{tab:mainresults}
     \vspace{-10pt}
\end{table*}


In this section, we report the implementation details and the benchmark performances of several well-known LLMs in the DCN task on our dataset. 

\subsection{Implementation Details}

 % These models are run using vLLM on eight Nvidia A100 GPUs with the same random seed. All temperatures are set to 0. Specific model hyperparameters and version details can be found in Table~\ref{tab:model-hyperparams}.

For \textit{open-source models}, such as Qwen series, we use their respective \textbf{chat} versions. For \textit{api-based models}, we aim to select the latest and most advanced versions available, the \textit{inference models} such as o1-mini \citep{openai2025o1} and DeepSeek-R1 \citep{deepseekai2025deepseekr1incentivizingreasoningcapability} are also included. The list of LLMs used is provided in Appendix~\ref{app:models}. The human baseline was derived from the average results of benchmark tasks completed by two finance professionals with relevant backgrounds.

Since our task focuses on Chinese, we chose one of the best open-source models currently available in the Chinese language domain: Qwen2.5-70B model to represent the \textbf{debtor}, while using different models for the \textbf{creditor} in order to compare their performance. The results of different models as the debtor are also presented in Appendix~\ref{sec:model_deb}. 
% Additionally, as an open-source model, it allows us to deploy it directly on our servers, thus reducing costs.

% We used the model itself as the agent without incorporating any additional modules, such as memory or backtracking. 
For both sides, we employed the Chain of Thought (CoT) approach \citep{wei2023chainofthoughtpromptingelicitsreasoning}, providing the model with instructions for the DCN task and a specified format for dialogue generation, which consisted of \textit{“Thought”}, \textit{“Dialogue”} and \textit{“Action” }in each interaction. The prompts are detailed in Appendix~\ref{app:prompts}.

\subsection{Benchmark Results}



% \begin{table*}[htbp]
% \centering
% \caption{\label{img:mainresult}The performances of some models}
% \begin{tabular}{@{}lccccccc@{}}
% \toprule
% models           & Succ  & Pred   & L1    & L2    & Var.   & 25\% Rec & 50\% Rec \\ \midrule
% Qwen2.5-7B-Instruct         & 0.98  & 14.76  & 2.82  & 79.46 & 0.84   & 46.04    & 214.04   \\
% Qwen2.5-72B-Instruct       & 0.96  & 11.70  & 6.30  & 79.82 & 0.89   & 28.60    & 154.60   \\
% Qwen2.5-14B-Instruct        & 0.96  & 13.26  & 9.82  & 77.58 & 0.93   & 33.56    & 187.06   \\
% bailing-80b-16k     & 0.98  & 20.10  & 3.38  & 71.70 & 0.82   & 87.16    & 269.56   \\ 
% \bottomrule
% \end{tabular}
% \end{table*}


The comparison of performance across different models is clearly illustrated in Table~\ref{img:mainresult}.

\textbf{LLMs perform well in terms of basic interaction format and overall dialogue capabilities.} From the perspectives of dialogue completeness (DC) and soundness (DS), we find that the models effectively cover all negotiation objectives. The dialogue content is generally reasonable, aligns with the set objectives, and shows little difference from the human baseline. Specifically, the Chinese-based model outperforms the English-based model in terms of dialogue soundness for our task.

\textbf{However, from the perspective of the negotiation outcomes, the performance of the LLMs was subpar and did not align well with requirements.} Observing the Comprehensive Collection Index (CCI), we found that the model’s overall evaluation result deviates from human-level performance by more than 0.05. This discrepancy might stem from the fact that the negotiation outcomes are numerical, making it challenging to align numerical-related requirements through prompt-based methods. 

\textbf{Most models tend to offer more generous concessions to debtors, both in repayment ratios and deadlines.} These concessions are crucial because they directly affect the financial company’s asset losses, a point emphasized in the prompt. However, as shown in the table, all models except for the GPT series have repayment ratios below 95\%, meaning they did not fully follow the prompt’s guidelines. In addition, the large models show lower collection efficiency compared to human benchmarks. For example, the time taken to recover 25\% of the debt is 2-3 times longer than the human baseline. This suggests the models give debtors more time to repay, rather than encouraging earlier repayment. Some models, like GPT-4o, come close to human-level efficiency, but this is at the cost of worsening the debtor’s financial situation. The average minimum repayment days for these models are twice as long as the human level, showing that they \textit{struggle to adapt} to the debtor’s real circumstances. This could be due to the models \textit{misjudging the debtor’s financial situation or choosing easier solutions to reach an agreement}.

\textbf{The collection results achieved by the model do not hold the debtor’s financial health, despite providing considerable room in terms of recovery and efficiency. }We found that, with the exception of the Qwen-2.5 model, the Debt Health Index (DHI) for all other models was below the human-level threshold. Considering the concessions offered to the debtor during the collection process, these results suggest that the model did not provide \textit{targeted debt resolution solutions} during the negotiation process.

\textbf{Non-inference models may be more suitable for this task compared to inference models.} By comparing the performance of the inference models o1-mini and Deepseek-R1 with their non-inference counterparts, gpt-4o-mini and Deepseek-V3, we observed a notable decline in the performance of the inference models across multiple metrics, particularly in collection efficiency.
\section{Method}

% \begin{itemize}
%     \item Workflow
%     \item WRoPE
%     \item Query-aware vector quantization
%     \item Heterogeneous inference
% \end{itemize}

% In this section, we first introduce the overall workflow of our proposed {\name}.
% Then detail the vector-quantization-compatible Windowed Rotary Position Embedding (WRoPE), and query-aware vector quantization for accurate top-K tokens retrieval.
% Lastly, we present the heterogeneous inference design.

% \subsection{Overview}

In {\name}, there are two stages.
During the offline pre-processing stage (1st stage), {\name} constructs a shared codebook on a representative dataset for each attention head of each layer.

During the inference stage (2nd stage), {\name} applies quantization functions to key states to map them to the nearest codewords.
At each autoregressive decoding step, {\name} first utilizes codebooks and codeword indices to approximate attention scores, then retrieves the top-K tokens with the highest attention scores for computation, thereby mitigating the memory overhead of accessing the entire KV cache.

\subsection{Windowed Rotary Position Embedding}

As discussed in Section~\ref{sec:codebook_sharing}, the position-dependent nature of post-PE key states hinders the direct application of a shared codebook in the vector quantization process.
A seemingly straightforward solution would be to quantize pre-PE key states, and then incorporate RoPE when approximating attention scores.
However, this approach is computationally expensive, as it necessitates calculating and applying the rotary matrices for each token at each inference.

To overcome this inefficiency, while eliminating the inherent position-dependent nature of post-PE key states, 
we propose Windowed Rotary Position Embedding (WRoPE).
This approach builds on the findings by~\citet{rerope, rerope-blog} that transformer-based models are nonsensitive to the positional information of non-local tokens.
The core idea of WRoPE is to use standard RoPE for local tokens (i.e. those in the window) and use approximate positional information for non-local tokens (i.e. those not in the window).
% Specifically, local tokens are the tokens closest to and have a significant impact on the current token that is generated in the autoregressive decoding phase.
% The number of local tokens is determined by the window size.
% For tokens that are not in the window, they may not have a significant impact on the generation of the current token, so the top K tokens with significant influence are selected based on the approximate position information.
% Therefore, approximate position information of these tokens is computed with a constant rotation matrix.
%The core idea of WRoPE is to use standard RoPE for local tokens, and approximate positional information for tokens outside this window with a constant rotation matrix.
Specifically, WRoPE computes the attention scores as follows:
\begin{equation}
    \label{eq:wrope}
    u_{i,j} = 
    \begin{cases}
        \begin{aligned}
            q_i R_{i-j} k_j^\top, \quad & i-j < w \\
            q_i R_b k_j^\top, \quad & i-j \ge w
        \end{aligned}
    \end{cases}
\end{equation}
where \(w\) is the window size, acting as a threshold for local vs. non-local tokens, and \(b\) is a constant value representing a fixed relative position approximation for non-local tokens.

For local tokens (i.e., \(i - j < w\)), WRoPE functions identically to standard RoPE, as defined in Equation~\ref{eq:rope}.
For non-local tokens outside the window (i.e., \(i - j \ge w\)), the position-dependent rotation matrix \(R_{i-j}\) is replaced by a fixed rotation matrix \(R_b\), approximating the relative positional information \( (i-j) \) with a constant offset \(b\).
Then, we can calculate the post-PE query and key states as:
\begin{equation}
    \tilde q_i = q_iR_b, \quad \tilde k_i = k_i
\end{equation}

Since post-PE key \(\tilde k_i\) states are identical to their pre-PE counterparts \(k_i\), WRoPE decouples the positional dependency from post-PE representations, therefore optimizes subsequent vector quantization.

\subsection{Query-Aware Vector Quantization}

\label{sec:query_aware_vq}

As discussed in Section \ref{sec:objective_mismatch}, conventional vector quantization fails to achieve accurate approximation of attention scores, due to the objective misalignment between vector quantization and attention score approximation. 

To address this limitation, \textbf{we propose query-aware vector quantization, a custom vector quantization method that directly optimizes the objective of attention score approximation. }
Specifically, we replace the squared Euclidean distance \(\|\tilde k - \hat k\|^2\) of conventional vector quantization with a query-aware quadratic form \((\tilde k - \hat k) H (\tilde k - \hat k)^\top\) derived from formula (\ref{eq:objective_attention_score_approximation}), where \(H\) represents the second-moment matrix of query states. 
% This custom method formulation explicitly optimizes for attention score approximation accuracy.

Formally, the query-aware vector quantization minimizes the following objective:
\begin{equation}
    \label{eq:query_aware_vq}
    J'(C) = \mathbb E_{\tilde k \sim \mathcal {D^\mathrm{key}}} \left[(\tilde k - \hat k) H (\tilde k - \hat k)^\top\right]
\end{equation}
where \(\hat k = c_{f'(\tilde k; C)}\) denotes the quantized \(\tilde k\), and the corresponding query-aware quantization vector quantization is formulated as:
\begin{equation}
    f'(\tilde k; C) = \operatorname*{argmin}_j (\tilde k - c_j) H (\tilde k - c_j)^\top    
\end{equation}
% Notably, the definition of \(\hat k\) in Equation~\ref{eq:objective_attention_score_approximation}

% \note{For the codebook construction process, we should reformulate the objective function that efficiently supports the conventional vector quantization algorithms like k-means++~\citep{kmeans++}.
% Thus,}

For the codebook construction process, we reformulate the objective function to utilize conventional efficient vector quantization algorithms like k-means++~\citep{kmeans++}.
Specifically, we apply Cholesky decomposition to the positive definite matrix \(H = LL^T\), where \(L \in \mathbb R^{d \times d}\) denotes the Cholesky factor.
% Thus,
%########
%To enable efficient codebook construction with conventional vector quantization algorithms like k-means++~\citep{kmeans++},
% we transform this problem through Cholesky decomposition of the positive definite matrix \(H = LL^\top\), where \(L \in \mathbb R^{d \times d}\) denotes the Cholesky factor. 
%#########
% This allows us to reformulate the objective as:
% \begin{equation}
%     \begin{aligned}
%         J'(C) & = \mathbb E_{\tilde k \sim \mathcal {D^\mathrm{key}}} \left[(\tilde k - \hat k) LL^\top (\tilde k - \hat k)^\top\right] \\
%         & = \mathbb E_{\tilde k \sim \mathcal {D^\mathrm{key}}} \left[(\tilde kL - \hat kL)(\tilde kL - \hat kL)^\top\right]
%     \end{aligned}
% \end{equation}
Let
\begin{equation}
    \label{eq:definition_z}
    z = \tilde k L, \quad C^z = CL, \quad\hat z = \hat k L
\end{equation}
where \(z \in \mathbb R^{1 \times d}\) denotes the transformed key state.
Then, we can re-derive the objective of attention score approximation \(J'\) as:
\begin{equation}
  \label{eq:objective_transform}
    \begin{aligned}
        J'(C) & = \mathbb E_{\tilde k \sim \mathcal {D^\mathrm{key}}} \left[(\tilde k - \hat k)H(\tilde k - \hat k)^\top\right] \\
        & = \mathbb E_{\tilde k \sim \mathcal {D^\mathrm{key}}} \left[(\tilde k - \hat k)LL^\top(\tilde k - \hat k)^\top\right] \\
        & = \mathbb E_{\tilde k \sim \mathcal {D^\mathrm{key}}} \left[(\tilde kL - \hat kL)(\tilde kL - \hat kL)^\top\right] \\
        & = \mathbb E_{z \sim D^z} \left[(z - \hat z)(z - \hat z)^\top\right]
    \end{aligned}
\end{equation}
And the quantization function \(f'\) can be re-derived as:
\begin{equation}
    \begin{aligned}
        f'(\tilde k; C) & = \operatorname*{argmin}_j (\tilde k - c_j) H (\tilde k - c_j)^\top     \\
        & = \operatorname*{argmin}_j (\tilde k L - c_jL)(\tilde k L - c_jL)^\top \\
        & = \operatorname*{argmin}_j (z - c^z_j)(z - c^z_j)^\top \\
        & = f(z; C^z)
    \end{aligned}
\end{equation}
where \(f(z; C^z)\) denotes the quantization function of conventional vector quantization.
Then, we can derive that
\begin{equation}
    \label{eq:quantization_function_transform}
    \begin{aligned}
        \hat z = \hat k L & = c_{f'(\tilde k; C)}L = c^z_{f(z; C^z)} 
    \end{aligned}
\end{equation}

% With 
Equations~\ref{eq:objective_transform},~\ref{eq:quantization_function_transform}
% the objective of attention score approximation \(J'\) simplifies to:
% This transoformation reveals 
reveal that \textbf{the objective of attention score approximation is equivilant to that of conventional vector quantization on transformed \(z\)}.
This alignment enables the application of conventional efficient vector quantization algorithms in codebook construction process.

During the offline pre-processing stage, we collect \(z\) on a representative dataset and construct its codebook \(C^z\) using k-means++~\citep{kmeans++}.
Then, The original shared codebook \(C\) for \(\tilde k\) is calculated as:
\begin{equation}
    C = C^z L^{-1}
\end{equation}
During inference, the codeword index of \(\tilde k\) is computed through query-aware quantization function:
\begin{equation}
    f'(\tilde k; C) = \operatorname*{argmin}_j (\tilde k L - c_jL)(\tilde k L - c_jL)^\top
\end{equation}
Let \(s \in \{1, 2, \dots, L\}^{1 \times n}\) denotes the codeword index vector of all key states after applying query-aware vector quantization, where \(s_j = f'(\tilde k_j; C)\).
Then, the attention score is approximated as:
\begin{equation}
    \hat u_{i,j} = \tilde q_i \hat k_j = \tilde q_i c_{s_j}
\end{equation}

\subsection{Heterogeneous Inference Design}

Although approximating attention scores via vector quantization and then selectively retrieving top-K tokens for computation reduces memory access overhead, the issue of KV Cache occupying substantial GPU memory remains unresolved. 
To reduce the memory footprint of KV Cache and enable larger batch sizes for improved GPU utilization, we design a heterogeneous inference system. 

We partition the decoding process of our proposed {\name} into three components:  

\noindent (1) \textbf{GPU-based model execution}: All model weights reside on the GPU memory. 
Computations involving model weights are executed on the GPU during inference.  

\noindent(2) \textbf{GPU-based approximation of attention scores}: The codebook is stored on the GPU. 
During inference, the GPU first executes the quantization function to assign codewords to key states, then computes attention weight approximations using the codebooks and indices, and lastly gathers the indices of top-K tokens.

\noindent(3) \textbf{CPU-based selective attention}: The full KV Cache is maintained on the CPU memory. 
During decoding, the top-K token indices and the current query state are transferred to the CPU, where selective attention computation is performed to derive the attention output. 
This output is then transferred back to the GPU for subsequent computations.

This design aims to minimize data transfer between CPU and GPU, thereby reducing latency. 
% Compared to alternative approaches that transfer Top-K tokens' KV cache back to the GPU for computation, our solution only requires transferring query states, top-K indices, and attention outputs across devices, thereby reducing latency induced by data transfers. 
Furthermore, to fully leverage the CPU's thread-level parallelism and SIMD capabilities, we implement a custom selective attention kernel optimized for CPU execution.

\section{Conclusion}\label{sec:conclusion}

In this paper, we proposed a prototype ASL generation system aimed at improving the naturalness, comprehensiveness, and overall quality of generated signs, addressing key limitations in existing approaches. Our technical evaluations indicate that our proposed approaches improve these aspects, enhancing the quality of generated ASL content. Feedback from DHH participants was mixed; while there was general interest in the system, concerns regarding visual quality and naturalness were noted. Reflecting on our design process and study findings, we discuss key insights and identify key areas for future improvement. While further work is needed, our study takes an initial step toward developing sign language generation systems that better meet the needs of the DHH and signing communities, offering real-world value.


\section*{Limitations}
We study the performance and improvement methods of large models in debt collection negotiations. To simplify the research process and capture key negotiation points, we reduce the debtor’s financial information to variables like assets, average income, and average expenses. However, real-world financial situations are more complex, involving factors like cash flow issues and income fluctuations during repayment. Future work should involve more detailed simulations of debtor information and comparisons with manually simulated debtors. Additionally, due to time constraints, our creditor Multi-agent framework is relatively simple. In practical applications, stricter classification processes in \textit{planning} and more standardized methods in \textit{judging} are needed. We aim to integrate existing decision models to further optimize decision-making in the dialogue.
\section*{Ethical Considerations} \label{EC}
Our study does not disclose any real client information. The acquisition of the source data was subject to strict approval by a major internet financial institution, and the process was continuously supervised by relevant personnel. All debt-related data are processed and replaced with synthetic values, and names are substituted with the pseudonym ``Zhang San''. For the debt reasons extracted from collection dialogues, we strictly anonymize any sensitive details and provide generalized summaries, ensuring that no specific information is involved. Each final data entry underwent rigorous manual verification. Additionally, the methodology proposed in this paper is exploratory and based on simulation for research purposes. Prior to its application in real-world debt collection involving actual individuals, it will undergo more rigorous validation and approval processes.

We conducted annotation tasks in two areas: scoring for the Dialogue Soundness metric (Section~\ref{sec:eval}) and comparison with the human baseline as a debt collector (Section~\ref{sec:res}). Five graduate students with engineering backgrounds and two professionals with financial industry experience participated (They are all from China, as our study focuses on the Chinese language). All annotators involved in our study have signed a disclaimer acknowledging the terms and conditions associated with their participation. They were recruited through campus forums and the internal annotation program of the company. The annotation tasks did not involve any sensitive information and posed no risk. Compensation was provided according to the time spent on each task.
\section*{Acknowledgments}
This work was supported by Ant Group.

\bibliography{latex/custom}


\appendix
\section{Related work}

\textbf{Debt Collection. }Debt collection is a labor-intensive and complex task. Previous research has primarily focused on using machine learning algorithms to identify optimal decisions for individual debtors based on large-scale data  \citep{Sancarlos2023TowardsAD,Jankowski2024DebtCM,Johan2022FinancialTC,Onar2019ADS}. However, these decisions are not made in real time and often require complex decision-making processes and multiple rounds of human negotiation. On the other hand, some automated debt collection dialogue models \citep{Floatbot2023GenerativeAI,Yahiya2024AutomatedDR} can only perform tasks such as information tracking and reminders, without the ability to engage in negotiations for specific goals. Our study aims to enable models to autonomously conduct negotiations and make real-time decisions, which can significantly enhance the efficiency of debt collection.

\textbf{Large Language Models in Negotiation. }In previous studies on large-scale negotiation models (including bargaining \citep{xia2024measuringbargainingabilitiesllms}, repeated games~\citep{akata2023playingrepeatedgameslarge,fu2023improvinglanguagemodelnegotiation} and social decision-making~\citep{10.5555/3618408.3619525}), the goals of the negotiators or gamers were clear, and there were clear methods for measuring the results. Debt collection is an information asymmetry game. Except for loan information, all other information is private information. How to model private information and evaluate the effectiveness of negotiation results are both difficult aspects to consider in modeling.

\textbf{AI Agents. }The memory, planning, reasoning, and communication capabilities of large-scale LLMs offer significant potential for the development of autonomous AI agents (\citealp{autogpt}; \citealp{park2023generative}; \citealp{liang2023encouraging};  \citealp{10.1162/tacl_a_00642}; \citealp{wang2025largelanguagemodelstruly}). Its potential has been demonstrated through the creation of a simulated town~\citep{park2023generative}, populated with independent agents who assume distinct roles and autonomously engage in social interactions.

% (\citealp{autogpt}; \citealp{park2023generative}; \citealp{liang2023encouraging}; \citealp{liang2023leveraging}; \citealp{ai2024cognition}; \citealp{10.1162/tacl_a_00642})
% \citep{park2023generative}



\section{Detailed descriptions of four Negotiation Dimensions}\label{sec:dim}

The following sections provide detailed descriptions of the four key negotiation dimensions involved in debt collection, outlining how each aspect influences the negotiation process and repayment outcomes. And table~\ref{negdim} shows all dimensions of the negotiation.

\begin{itemize}[leftmargin=15px]
    \item \textbf{Debt Reduction Ratio:} This refers to the portion of the debt that can be waived by the creditor to ease the debtor’s repayment burden. The reduction ratio is often negotiable based on the debtor’s financial situation, with creditors typically offering reductions as an incentive to settle the debt more efficiently.

    \item \textbf{Immediate Repayment Ratio:} In order to temporarily restore the debtor’s credit and advance the repayment process, creditors usually require the debtor to repay a portion of the outstanding debt immediately during the negotiation. This portion is typically at least 5\% of the total debt.

    \item \textbf{Immediate Repayment Time:} If the debtor is unable to make an immediate payment on the same day, a grace period of up to 14 days may be granted. Within this period, the debtor is expected to raise the necessary funds to complete the immediate repayment.

    \item \textbf{Installment Period:} After addressing part of the debt through reductions and immediate repayments, the remaining balance can be settled through installments. The installment ratio can vary from 3 to 24 periods, allowing the debtor to repay the debt within a period ranging from a few months to up to two years.
\end{itemize}

\begin{table*}[ht]
\centering
\caption{\label{negdim}Negotiation Dimensions and Their Possible Values}
\begin{tabular}{ll}
\toprule
\textbf{Dimension} & \textbf{Values} \\
\midrule
Discount Ratio ('disc\_ratio') & 5\%, 10\%, 15\%, 20\%, 25\%, 30\% \\
Immediate Payment Ratio ('pmt\_ratio') & 5\%, 10\%, 15\%, 20\%, 25\%, 30\%, 35\%, 40\%, 45\%, 50\% \\
Immediate Payment Time ('pmt\_days') & 1, 2, 3, 4, 5, 6, 7, 8, 9, 10, 11, 12, 13, 14 days \\
Installment Periods ('inst\_prds') & 3, 6, 9, 12, 18, 24 months \\
\bottomrule
\end{tabular}
\label{tab:negotiation_dimensions}
\end{table*}

\section{Data Distribution} \label{Distribution}

As shown in the Figure~\ref{img:distri}, our dataset exhibits a certain distribution across Amount, Sex, and Overdue Days, which is similar to the actual situation. The distributions in both the test set and the train set are also largely consistent.

\begin{figure*}[htbp]
  \centering
  \includegraphics[width=1.03\textwidth]{latex/images/contri.pdf}  
  \vspace{-0.2in}
  \caption{Distribution of Need collected Amount, Sex, and Overdue Days.}
\vspace{-0.0in}
\label{img:distri}
\end{figure*}

\section{Difficulty Tiers for Debt Collection} \label{app:diff_cat}

In the field of debt management and collection, the economic hardship level may be related to the debtor’s repayment capacity assessment~\citep{Zwilling2017EvaluatingYC}. Referring to common methods for determining economic hardship levels~\citep{elsevier2001international}, we categorize debtors into five tiers as shown in Table~\ref{img:category}.

\begin{table}[ht]
\centering
\vspace{-0.1in}
\caption{\label{img:category}Difficulty Tiers for Debt Collection}

\setlength{\tabcolsep}{3.5mm}{
\resizebox{0.38\textwidth}{!}{%
\begin{tabular}{lcc}
    \toprule
    \textbf{Tier} & \textbf{Description} & \textbf{Range} \\
    \midrule
    Tier 1 & Extremely Difficult & 0 - 2000 \\
    Tier 2 & Very Difficult & 2000 - 5000 \\
    Tier 3 & Moderately Difficult & 5000 - 10000 \\
    Tier 4 & Slightly Difficult & 10000 - 20000 \\
    Tier 5 & No Difficulty & 20000+ \\
    \bottomrule
\end{tabular}%
}}
\vspace{-0.1in}
\end{table}

\section{Definitions of variables in DCN process}

Table~\ref{tab:debt_variable_app} provides the descriptions of all the variables appearing in Algorithm 1. The Action Set includes \texttt{ask}, \texttt{reject}, and \texttt{accept}, while the Negotiation Dimension Set consists of the four quantities listed in Table~\ref{tab:negotiation_dimensions}.


\begin{table}[ht]
\centering
\Large
\caption{Definitions of variables in DCN process.}
\resizebox{0.7\columnwidth}{!}{%
\begin{tabular}{@{}ll@{}}
\toprule
\textbf{Conception} & \textbf{Variable} \\ \midrule
Basic Information & $I_b$ \\
Creditor & creditor \\
Action Set & $S_A$ \\
Result Dictionary & $D$ \\
Personal Financial Information & $I_p$ \\
Debtor & debtor \\
Negotiation Dimension Set & $S_R$ \\
Turn & $t$ \\
Max Turns & $t_m$ \\
Agent Creditor & Creditor \\
Agent Debtor & Debtor \\
Action of Debtor & $A_d$ \\
Action of Creditor & $A_c$ \\ 
\bottomrule
\end{tabular}%
}

\label{tab:debt_variable_app}
\end{table}

\section{Detail of metrics} \label{app:metric}

\subsection{Conversational Ability} \label{app:me_conv}

In negotiation processes, conversational ability is crucial for achieving effective communication and mutual understanding. \citet{tu2024characterevalchinesebenchmarkroleplaying} proposed an evaluation framework for role-playing tasks. Inspired by this work, we tailored it to our task by distinguishing Conversational Ability into two dimensions: fluency and completeness.

\textbf{Dialogue Soundness (DS).} Dialogue Soundness is a single-metric evaluation that measures a dialogue response’s fluency, naturalness, coherency, and consistency on a five-point scale. It assesses whether the response is grammatically correct and conversational, stays on topic, and remains logically consistent across turns. This metric is manually scored, with the scale shown in Table ~\ref{DS_score}. Five graduate students from engineering disciplines were employed to evaluate this metric and calculated the average value.

\begin{table*}[h]
    \centering
    \caption{\label{DS_score}Dialogue Soundness (DS) Rating Scale}
    \begin{tabular}{c l l}
        \toprule
        \textbf{Score} & \textbf{Rating} & \textbf{Description} \\
        \midrule
        5 & Excellent  & Fluent, natural, on-topic, logically consistent. \\
        4 & Good       & Mostly natural, minor topic drift, slight inconsistency. \\
        3 & Acceptable & Understandable but somewhat rigid, occasional drift or inconsistency. \\
        2 & Poor       & Unnatural phrasing, noticeable topic deviation or contradictions. \\
        1 & Unacceptable & Robotic, off-topic, illogical contradictions. \\
        \bottomrule
    \end{tabular}
\end{table*}

\textbf{Dialogue Completeness (DC).} Dialogue Completeness is a metric designed to evaluate whether a conversation addresses all specified objectives outlined in section~\ref{obj} of the paper. This automated measure checks if each of the four key goals has been adequately discussed during the dialogue, ensuring that no critical topics are overlooked or omitted.

\subsection{Debt Recovery} \label{app:me_rec}

\textbf{Success Recovery Rate (SR). }The success rate of the negotiation is determined by whether the debtor's future assets remain in a healthy state (i.e., the total personal assets remain greater than 500). The success rate is defined as the proportion of samples in which repayment can theoretically be completed successfully:
\begin{equation}\label{eq:utility}
\begin{split}
\text{SR} = \frac{N_{\text{success}}}{N},
\end{split}
\end{equation}
where SR is the success rate, $N_{\text{success}}$ is the number of successful samples, and \(N\) is the total number of samples.

\textbf{Recovery Rate (RR). }The recovery ratio refers to the portion of the debt recovered by the creditor, which is typically $1$ minus the reduction ratio. If the plan is unsuccessful, the recovery ratio is considered to be $0$. The final recovery ratio is calculated as the mean recovery ratio across the test samples:

\begin{equation}
\begin{split}
\text{RR} = \frac{1}{N} \sum_{i=1}^{N} r_i,
\end{split}
\end{equation}
where RR is the final recovery ratio, \(r_i\) is the recovery ratio of the \(i\)-th sample.


\subsection{Collection Efficiency} \label{app:me_col}

\textbf{25\% Recovery Date (QRD)} refers to the date at which the debtor has completed 25\% of the debt repayment, which is estimated based on the debtor's future economic condition sequence. The final 25\% Recovery Date is calculated as the mean of the recovery dates across the test samples:
\begin{equation}
\begin{split}
\text{QRD} = \frac{1}{N} \sum_{i=1}^{N} t_{25\%,i},
\end{split}
\end{equation}
where QRD is the final 25\% recovery date, $t_{25\%,i}$ is the 25\% recovery date of the $i$-th sample, and $N$ is the total number of samples.

\textbf{50\% Recovery Date (HRD)} is defined similarly to the 25\% Recovery Date, referring to the date at which the debtor has completed 50\% of the debt repayment, based on the debtor's future economic condition sequence. 
\textbf{Completion Date (CD)} refers to the date at which the debtor has fully repaid all of the debt.

The 50\% Recovery Date and Completion Date are calculated as the means of the respective recovery dates across the test samples:
\begin{equation}
\begin{split}
\text{HRD} = \frac{1}{N} \sum_{i=1}^{N} t_{50\%,i},
\end{split}
\end{equation}
where HRD is the final 50\% recovery date, and $t_{50\%,i}$ is the 50\% recovery date of the $i$-th sample.
\begin{equation}
\begin{split}
\text{CD} = \frac{1}{N} \sum_{i=1}^{N} t_{\text{Completion},i},
\end{split}
\end{equation}
where CD is the completion date, and $t_{\text{Completion},i}$ is the completion date of the $i$-th sample.


\subsection{Debtor’s Financial Health} \label{app:me_hea}

\textbf{L1 Tier Days (L1D)} refers to the number of days the debtor remains in the most difficult tier over the next two years. \textbf{L2 Tier Days (L2D)} refers to the number of days the debtor remains in the second most difficult tier during the same period. These two indicators directly correspond to the duration the debtor spends in different levels of financial difficulty. Research has shown that the longer the debtor remains in a higher level of difficulty, the more likely they are to default on the loan 
~\citep{Tabacchi2016DeterminantsOE}.

\textbf{Asset tier variance (ATV).} In addition to controlling for the number of days the debtor remains in the high-poverty tier, the overall stability of the debtor's asset level also ensures a higher repayment performance. To capture this, we introduce the asset tier variance metric, which is calculated by computing the variance of the debtor's asset tier over the course of one year. The final result is obtained by calculating the mean of the asset tier variances across the test samples:
\begin{equation}
\begin{split}
v_{\text{asset},i} = \frac{1}{T-1} \sum_{t=1}^{T} \left( A_{i,t} - \bar{A}_i \right)^2,
\end{split}
\end{equation}
where $A_{i,t}$ is the asset tier of the $i$-th sample at time $t$, $\bar{A}_i$ is the average asset tier of the $i$-th sample over the year, and $T$ is the total number of time periods. The final asset tier variance is the mean of the asset tier variances across the test samples:
\begin{equation}
\begin{split}
\text{ATV} = \frac{1}{N} \sum_{i=1}^{N} v_{\text{asset},i},
\end{split}
\end{equation}
where ATV is the mean asset tier variance, and $N$ is the total number of samples.

\subsection{Average Metric} \label{app:me_ave}

In debt collection, the indicators for Debt Recovery and Collection Efficiency are often inversely related to the Debtor’s Financial Health. This means that efforts to recover debts more efficiently and quickly may negatively impact the debtor's financial condition. To strike a balance between these two conflicting objectives, we introduce three average metrics that help quantify the trade-off: the Creditor’s Recovery Index (CRI), the Debtor’s Health Index (DHI), and the Comprehensive Collection Index (CCI).

\textbf{Creditor’s Recovery Index (CRI):} This index measures the effectiveness of the creditor's recovery strategy while accounting for the impact on the debtor’s financial health. The index aggregates several recovery metrics weighted by their relative importance to the creditor's objectives. The index is calculated as follows:
\begin{equation}
\begin{split}
\text{CRI} = &\ w_1 \cdot \text{SR} + w_2 \cdot \text{RR} \\
& + w_3 \cdot \frac{\text{max}(\text{QRD}) - \text{QRD}}{\text{max}(\text{QRD})} \\
& + w_4 \cdot \frac{\text{max}(\text{HRD}) - \text{HRD}}{\text{max}(\text{HRD})} \\
& + w_5 \cdot \frac{\text{max}(\text{CD}) - \text{CD}}{\text{max}(\text{CD})},
\end{split}
\end{equation}
where \(w_1, w_2, w_3, w_4, w_5\) are the weights assigned to each metric based on the creditor’s priorities.

\textbf{Debtor’s Health Index (DHI):} This index measures the debtor's financial health during the recovery process. It incorporates several factors that capture the debtor's stability and vulnerability. The Debtor’s Health Index is calculated as:
\begin{equation}
\begin{split}
\text{DHI} = &\ w_6 \cdot \frac{\text{max}(\text{L1D}) - \text{L1D}}{\text{max}(\text{L1D})} \\
& + w_7 \cdot \frac{\text{max}(\text{L2D}) - \text{L2D}}{\text{max}(\text{L2D})} \\
& - w_8 \cdot \text{ATV}.
\end{split}
\end{equation}
Here, \(w_6, w_7, w_8\) are weights that balance the importance of each factor in determining the debtor’s health.

\textbf{Comprehensive Collection Index (CCI):} The Comprehensive Collection Index combines both the Creditor’s Recovery Index (CRI) and the Debtor’s Health Index (DHI) into a single metric that evaluates the overall balance between debt recovery and the debtor’s financial well-being. The index is calculated using the harmonic mean of the two indices, with a weight factor $\theta$ applied to the CRI:
\begin{equation}
\text{CCI} = \frac{2 \theta^2 \cdot \text{CRI} \cdot \text{DHI}}{\text{CRI} + \theta^2\cdot\text{DHI}}.
\end{equation}
In this formula, the weight factor $\theta$ indicates that the CRI is weighted $\theta$ times more than the DHI. In this study, $\theta$ is set to 2. This approach ensures that a high value in either the recovery index or the health index will influence the overall result, while emphasizing the importance of balancing both aspects.

The use of this weighted harmonic mean helps in evaluating different debt recovery strategies by considering both the creditor’s objectives and the debtor’s financial stability, thereby promoting a more balanced approach to debt collection.

The constant values used in the calculation process are shown in Table~\ref{tab:metrics_parameters}. In future research or application, these values may be adjusted depending on the specific requirements to better align with the needs.
\begin{table}[ht]
\centering
% \Large
\caption{Constants used in Average Metric Calculation.}
\resizebox{0.4\columnwidth}{!}{%
\begin{tabular}{@{}ll@{}}
\toprule
\textbf{Constant} & \textbf{Value} \\ \midrule
$w_1$ & 0.25 \\
$w_2$ & 0.25 \\
$w_3$ & 0.2 \\
$w_4$ & 0.15 \\
$w_5$ & 0.15 \\
$w_6$ & 1.5 \\
$w_7$ & 0.8 \\
$w_8$ & 1 \\
$\theta$ & 2 \\
$\text{max}(\text{QRD})$ & 180 \\
$\text{max}(\text{HRD})$ & 360 \\
$\text{max}(\text{CD})$ & 720 \\
$\text{max}(\text{L1D})$ & 30 \\
$\text{max}(\text{L2D})$ & 250 \\ 
\bottomrule
\end{tabular}%
}
\label{tab:metrics_parameters}
\end{table}

\section{All LLMs in our Experiments} \label{app:models}

We comprehensively evaluate nine LLMs, encompassing both API-based models and open-source models. The API-based models include the GPT series (GPT-4o, GPT-4o-mini, o1-mini) \citep{GPT-4, openai20254o,openai2025o1}, Claude-3.5 \citep{anthropic2025}, MiniMax (abab6.5s-chat) \citep{minimaxi2025}, Sensechat \citep{sensetime2025}, DeepSeek series (DeepSeek-R1 and DeepSeek-V3) \citep{deepseekai2025deepseekr1incentivizingreasoningcapability,deepseekai2024deepseekv3technicalreport} and Doubao \citep{doubao2025}. The open-source models include the Llama series (LlaMA-2-13B-Chat, LlaMA-3-8B-Instruct, LlaMA-3-70B-Instruct) \citep{LLaMA} and the Qwen-2.5 series (Qwen-2.5-7B, Qwen-2.5-14B and Qwen-2.5-72B) \citep{qwen2025qwen25technicalreport}. These models are run using vLLM~\citep{kwon2023efficient} on eight Nvidia A100 GPUs with the same random seed. For each model, the entire test set was processed in approximately one hour using parallel methods. All temperatures are set to 0 (Due to API-provider's closed-source non-deterministic implementation, small changes may still occur in the reproduction process). Specific model hyperparameters and version details can be found in Table~\ref{tab:model-hyperparams}. All models and tools (vLLM and LLaMa-Factory~\citep{zheng2024llamafactory}) used in this study, including closed-source API-based models, open-source models, were used in compliance with their respective licenses. What's more, the use of these generative models for dialogue tasks is well-established in the field and follows standard practices.

\begin{table*}[h!]
\centering
\caption{\textcolor{black}{Hyperparameters of Each Model.}}
\label{tab:model-hyperparams}
\textcolor{black}{
\resizebox{1\textwidth}{!}{%
\begin{tabular}{lll}
\hline
\textcolor{black}{\textbf{Model Name}} & \textcolor{black}{\textbf{Parameters}} & \textcolor{black}{\textbf{Comments}} \\ 
\hline
\textcolor{black}{Qwen-2.5-7B} & \textcolor{black}{"temperature": 0, "max\_tokens": 1024} & \textcolor{black}{version = "qwen-2.5-7b-instruct"} \\
\textcolor{black}{Qwen-2.5-14B} & \textcolor{black}{"temperature": 0, "max\_tokens": 1024} & \textcolor{black}{version = "qwen-2.5-14b-instruct"} \\
\textcolor{black}{Qwen-2.5-72B} & \textcolor{black}{"temperature": 0, "max\_tokens": 1024} & \textcolor{black}{version = "qwen-2.5-72b-instruct"} \\
\textcolor{black}{GPT-4o} & \textcolor{black}{"temperature": 0, "max\_tokens": 1024} & \textcolor{black}{version = "gpt-4o-2024-11-20"} \\ 
\textcolor{black}{GPT-4o Mini} & \textcolor{black}{"temperature": 0, "max\_tokens": 1024} & \textcolor{black}{version = "gpt-4o-mini"} \\ 
\textcolor{black}{o1-Mini} & \textcolor{black}{"temperature": 0, "max\_tokens": 1024} & \textcolor{black}{version = "o1-mini"} \\ 
\textcolor{black}{LLaMa-3-8B} & \textcolor{black}{"temperature": 0, "max\_tokens": 1024} & \textcolor{black}{version = "llama-3-8b-instruct"} \\ 
\textcolor{black}{LLaMa-3-70B} & \textcolor{black}{"temperature": 0, "max\_tokens": 1024} & \textcolor{black}{version = "llama-3-70b-instruct"} \\ 
\textcolor{black}{Doubao} & \textcolor{black}{"temperature": 0, "max\_tokens": 1024} & \textcolor{black}{version = "Doubao-pro-4k"} \\ 
\textcolor{black}{Claude-3.5} & \textcolor{black}{"temperature": 0, "max\_tokens": 1024} & \textcolor{black}{version = "claude-3-5-sonnet-20241022"} \\ 
\textcolor{black}{DeepSeek-V3} & \textcolor{black}{"temperature": 0, "max\_tokens": 1024} & \textcolor{black}{version = "deepseek-chat"} \\ 
\textcolor{black}{DeepSeek-R1} & \textcolor{black}{"temperature": 0, "max\_tokens": 1024} & \textcolor{black}{version = "deepseek-reasoner"} \\ 
\textcolor{black}{MiniMax} & \textcolor{black}{"temperature": 0, "max\_tokens": 1024} & \textcolor{black}{version = "abab6.5s-chat"} \\ 
\textcolor{black}{SenseChat} & \textcolor{black}{"temperature": 0, "max\_tokens": 1024} & \textcolor{black}{version = "SenseChat"} \\ 
\hline
\end{tabular}
}}
\end{table*}

\section{Prompts} 

\subsection{Basic Prompts for Role-playing Debtor and Creditor.} \label{app:prompts}

Figures~\ref{img:deb_prompt} and~\ref{img:cre_prompt} illustrate the prompts given to the large model to act as the debtor and the creditor, respectively. Originally in Chinese, these prompts have been appropriately simplified and automatically translated into English for display purposes (the full Chinese prompts is available to be disclosed later). Additionally, the instructions provided to human annotators were consistent with the prompts given to the model.

\begin{figure*}[htbp]
  \centering
  \includegraphics[width=1\textwidth]{latex/images/deb_prompt.pdf}  
  \caption{Prompt of Debtor.}
\vspace{-0.0in}
\label{img:deb_prompt}
\end{figure*}

\begin{figure*}[htbp]
  \centering
  \includegraphics[width=1\textwidth]{latex/images/cre_prompt.pdf}  
  \caption{Prompt of Creditor (Debt Collector).}
\vspace{-0.0in}
\label{img:cre_prompt}
\end{figure*}




\subsection{Prompts for Planning Agent and Judging Agent.} \label{app:agent_promopt}

Figures~\ref{img:plan_prompt} and~\ref{img:judge_prompt} display the prompts for the planning agent and judging agent in the MADaN framework. Similarly, these prompts have been simplified and translated for ease of presentation. The prompt for the communicating agent remains unchanged, as previously shown.


\begin{figure*}[htbp]
  \centering
  \includegraphics[width=1\textwidth]{latex/images/plan_prompt.pdf}  
  \caption{Prompt of Planning Agent.}
\vspace{-0.0in}
\label{img:plan_prompt}
\end{figure*}

\begin{figure*}[htbp]
  \centering
  \includegraphics[width=1\textwidth]{latex/images/judge_prompt.pdf}  
  \caption{Prompt of Judging Agent.}
\vspace{-0.0in}
\label{img:judge_prompt}
\end{figure*}


\subsection{Defective prompt} \label{app:deprompts}

There are three main methods for generating Defective Prompts, as shown in Table~\ref{deprompt}. In practice, we first generate a list of prompts and then randomly select one from the list to generate the negative samples.

\begin{table*}[ht]
\centering
\caption{\label{deprompt}Defective Prompt Modifications for Debt Collection Negotiation.}
    \setlength{\tabcolsep}{3.5mm}{
    \resizebox{\textwidth}{!}{%
        \begin{tabular}{lll}
        \toprule
        \textbf{Modification Type} & \textbf{Description} & \textbf{Example} \\
        \midrule
        Deletion & Remove specific instructions & Removing "Offer a 10\% discount when the debtor shows clear financial difficulty." \\
        Replacement & Reverse guidance & Changing "Be cautious when the debtor makes a request" to "Approve requests without further consideration." \\
        Addition & Add negative guidance & Adding "If installment terms are discussed, set them to 24 months without negotiation." \\
        \bottomrule
        \end{tabular}
    }}
\end{table*}

% \textcolor{black}{GPT-4-turbo} & \textcolor{black}{"temperature": 0, "max\_tokens": 1024} & \textcolor{black}{version = "GPT-4-turbo"} \\ 
% \textcolor{black}{GPT-3.5-turbo} & \textcolor{black}{"temperature": 0, "max\_tokens": 1024} & \textcolor{black}{version = "gpt-3.5-turbo-0125"} \\ 
% \textcolor{black}{Qwen1.5-110B} & \textcolor{black}{"temperature": 0, "max\_tokens": 1024} & \textcolor{black}{version = "qwen1.5-110b-chat"} \\ 
% \textcolor{black}{QwenMax} & \textcolor{black}{"temperature": 0, "max\_tokens": 1024} & \textcolor{black}{version = "qwen-max"} \\ 
% \textcolor{black}{Claude-3-Opus} & \textcolor{black}{"temperature": 0, "max\_tokens": 1024} & \textcolor{black}{version = "claude-3-opus-20240229"} \\ 
% \textcolor{black}{LLaMA2-13B-Chat} & \textcolor{black}{"temperature": 0, "max\_tokens": 1024} & \textcolor{black}{model = "Llama-2-13b-chat"} \\ 
% \textcolor{black}{LLaMA3-70B-Instruct} & \textcolor{black}{"temperature": 0, "max\_tokens": 1024} & \textcolor{black}{model = "Llama-3-70B-Instruct"} \\ 
% \textcolor{black}{LLaMA3-8B-Instruct} & \textcolor{black}{"temperature": 0, "max\_tokens": 1024} & \textcolor{black}{model = "Llama-3-8B-Instruct"} \\ 
% \textcolor{black}{Qwen2-7B-Instruct} & \textcolor{black}{"temperature": 0, "max\_tokens": 1024} & \textcolor{black}{model = "Qwen2-7B-Instruct"} \\ 
% \hline
% \textcolor{black}{LLaMA3-8B-Base-FT} & \textcolor{black}{"temperature": 0, "max\_tokens": 1024, train\_batch\_size: 4,"finetuning\_type": lora, } & \textcolor{black}{model = "Llama-3-8B"} \\ 
% & "learning\_rate": 1.0e-4, "num\_train\_epochs": 10.0, "bf16": true & \\
% \textcolor{black}{LLaMA3-8B-Instruct-FT} & \textcolor{black}{"temperature": 0, "max\_tokens": 1024,"train\_batch\_size": 4,"finetuning\_type": lora,} & \textcolor{black}{model = "Llama-3-8B-Instruct"} \\ 
% &  "learning\_rate": 1.0e-4, "num\_train\_epochs": 10.0, "bf16": true & \\
% \textcolor{black}{Qwen2-7B-Instruct-FT} & \textcolor{black}{"temperature": 0, "max\_tokens": 1024,"train\_batch\_size": 4,"finetuning\_type": lora, } & \textcolor{black}{model = "Qwen2-7B-Instruct"} \\ 
% &  "learning\_rate": 1.0e-4, "num\_train\_epochs": 10.0, "bf16": true &\\
% \textcolor{black}{Qwen2-7B-Base-FT} & \textcolor{black}{"temperature": 0, "max\_tokens": 1024,"train\_batch\_size": 4,"finetuning\_type": lora, } & \textcolor{black}{model = "Qwen2-7B"} \\ 
% &  "learning\_rate": 1.0e-4, "num\_train\_epochs": 10.0, "bf16": true & \\
% \hline


\section{The performance of different models as the debtor}\label{sec:model_deb}


In Section~\ref{sec:res}, we evaluate the debt collection outcomes when different models act as the creditor. We alse examine the performance of different models as debtors, using the Qwen-2.5-72B model exclusively as the creditor. We observed significant differences in the results when using different models for the debtor as shown in Table~\ref{img:reverseresult}. The SenseChat and Llama-3-70b models exhibited some inconsistencies, yielding excessively high DHI scores. During the examination of the dialogue process, we found that these models tended to neglect \textit{repeated statements} within the dialogue, leading to the inclusion of some irrelevant or ineffective content. Additionally, some models were more sensitive to the debtor’s prompt, likely due to the more complex nature of the debtor agent’s objectives. In contrast, the Qwen-2.5-72 model showed relatively balanced performance, suggesting that our choice was appropriate.

Since our focus is on studying the model’s performance as a debt collector, we did not design specific metrics for debtor models. Our primary aim is to use models capable of understanding the debtor’s objectives and engaging in dialogue for simulations prior to further manual testing.

\begin{table*}[ht]
\vspace{-0.1in}
    \centering
    \caption{\label{img:reverseresult}The performances of some models as Debtors.}
    \vspace{-0.1in}
    \setlength{\tabcolsep}{3.5mm}{
    \resizebox{\textwidth}{!}{%
    \begin{tabular}{lcccccccccccc}
        \toprule
        Model  & SR & RR & QRD & HRD & CD & L1D & L2D & ATV & CRI & DHI & CCI \\
        \midrule
        Qwen-2.5-72B  & 0.98 & 0.88 & 36.98 & 185.18 & 404.98 & 3.76 & 78.84 & 0.83 & 0.76 & 0.76 & 0.76\\
        llama-3-8b  & 1.00 & 0.94 & 29.13 & 134.13 & 296.25 & 3.25 & 80.31 & 0.91 & 0.83 & 0.71 & 0.81 \\
        llama-3-70b  & 1.00 & 0.92 & 10.33 & 150.33 & 369.33 & 0.33 & 51.33 & 0.85 & 0.83 & 0.97 & 0.85 \\
        gpt-4o-2024-11-20  & 1.00 & 0.94 & 35.26 & 146.86 & 312.66 & 3.50 & 85.72 & 0.86 & 0.82 & 0.73 & 0.80 \\
        o1-mini  & 0.98 & 0.93 & 26.76 & 111.96 & 240.56 & 4.92 & 92.34 & 0.93 & 0.85 & 0.58 & 0.78 \\
        deepseek-chat & 0.97 & 0.93 & 32.42 & 125.32 & 269.48 & 3.74 & 93.00 & 0.90 & 0.83 & 0.66 & 0.79 \\
        Doubao-pro-4k & 1.00 & 0.83 & 75.28 & 190.48 & 324.72 & 2.16 & 80.34 & 0.84 & 0.73 & 0.82 & 0.74 \\
        abab6.5s-chat & 0.90 & 0.92 & 58.53 & 204.53 & 484.53 & 8.63 & 76.50 & 0.89 & 0.70 & 0.52 & 0.66 \\
        SenseChat & 1.00 & 0.96 & 135.0 & 345.00 & 734.00 & 0.70 & 51.00 & 0.88 & 0.54 & 0.96 & 0.60 \\
        \bottomrule
    \end{tabular}%
    }}
 \label{tab:mainresults}
     \vspace{-10pt}
\end{table*}

% \begin{table*}[ht]
%     \centering
%     \caption{\label{img:mainresult_detial}The performances of some models}
%     \vspace{-0.1in}
%     \setlength{\tabcolsep}{3.5mm}{
%     \resizebox{\textwidth}{!}{%
%     \begin{tabular}{lccccccccc}
%         \toprule
%          & \multicolumn{2}{c}{\textbf{Debt Recovery}} & \multicolumn{3}{c}{\textbf{Collection Efficiency}} & \multicolumn{3}{c}{\textbf{Debtor’s Financial Health}} \\
%         \cmidrule(lr){2-3} \cmidrule(lr){4-6} \cmidrule(lr){7-9}
%         \textbf{Model} & \textbf{SR(\%)} & \textbf{RR} & \textbf{QRD} & \textbf{HRD} & \textbf{CD} & \textbf{L1D}& \textbf{L2D}& \textbf{ATV} \\
%         \midrule
%         Vanilla & 90 & 84.00 & 34.35 & 178.35 & 397.85 & \textbf{4.10} & 73.50 & \textbf{0.83}\\
%         + Planning& 80 & 94.38 & 24.35 & 142.85 & 398.35 & 11.80 & \textbf{72.75} & 0.96\\
%         + Judging & 95 & \textbf{97.62} & 19.60 & 124.60 & 327.60 & 6.15 & 76.75 & 0.86\\
%         Ours  & \textbf{95} & 97.40 & \textbf{18.00} & \textbf{124.50} & \textbf{300.5} & 4.45 & 79.85 & 0.86\\
%         \bottomrule
%     \end{tabular}%
%  }}
%  \label{tab:mainresults}
%      % \vspace{-10pt}
% \end{table*}


\section{Settings of Post-training}

All post-training experiments were conducted on an 8-GPU A100 server using the LLaMa-Factory framework~\citep{zheng2024llamafactory}. The training time per session was around five minutes. The specific parameter settings for each group are provided in Table~\ref{tab:model-hyperparams-post}. The four sets of training data will be made publicly available at a later stage.



\begin{table*}[h!]
\centering
\caption{\textcolor{black}{Hyperparameters of Each Post-trained Model.}}
\label{tab:model-hyperparams-post}
\textcolor{black}{
\resizebox{1\textwidth}{!}{%
\begin{tabular}{lll}
\hline
\textcolor{black}{\textbf{Model Name}} & \textcolor{black}{\textbf{Parameters}} & \textcolor{black}{\textbf{Comments}} \\ 
\hline
\textcolor{black}{SFT-DG} & \textcolor{black}{"temperature": 0, "max\_tokens": 1024, train\_batch\_size: 4,"finetuning\_type": lora, } & \textcolor{black}{model = "qwen-2.5-7b-instruct"} \\ 
& "learning\_rate": 5.0e-6, "num\_train\_epochs": 5.0, "bf16": true & \\
\textcolor{black}{SFT-MAG} & \textcolor{black}{"temperature": 0, "max\_tokens": 1024,"train\_batch\_size": 4,"finetuning\_type": lora,} & \textcolor{black}{model = "qwen-2.5-7b-instruct"} \\ 
&  "learning\_rate": 5.0e-6, "num\_train\_epochs": 5.0, "bf16": true & \\
\textcolor{black}{DPO-DG} & \textcolor{black}{"temperature": 0, "max\_tokens": 1024,"train\_batch\_size": 4,"finetuning\_type": lora, } & \textcolor{black}{model = "qwen-2.5-7b-instruct"} \\ 
&  "learning\_rate": 5.0e-6, "num\_train\_epochs": 5.0, "bf16": true &\\
\textcolor{black}{DPO-MAG} & \textcolor{black}{"temperature": 0, "max\_tokens": 1024,"train\_batch\_size": 4,"finetuning\_type": lora, } & \textcolor{black}{model = "qwen-2.5-7b-instruct"} \\ 
&  "learning\_rate": 5.0e-6, "num\_train\_epochs": 5.0, "bf16": true & \\
\hline
\end{tabular}
}}
\end{table*}


\section{Supplementary Information}
This paper utilized AI tools including Google Translate for assisted translation when presenting prompts and examples, and employed the use of a Cursor for coding to enhance efficiency. No potential risks were involved in the course of this study.
\end{document}
