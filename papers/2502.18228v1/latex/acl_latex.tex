% This must be in the first 5 lines to tell arXiv to use pdfLaTeX, which is strongly recommended.
\pdfoutput=1
% In particular, the hyperref package requires pdfLaTeX in order to break URLs across lines.

\documentclass[11pt]{article}

% Change "review" to "final" to generate the final (sometimes called camera-ready) version.
% Change to "preprint" to generate a non-anonymous version with page numbers.
\usepackage[preprint]{acl}

% Standard package includes
\usepackage{times}
\usepackage{latexsym}
\usepackage{longtable}
% For proper rendering and hyphenation of words containing Latin characters (including in bib files)
\usepackage[T1]{fontenc}
% For Vietnamese characters
% \usepackage[T5]{fontenc}
% See https://www.latex-project.org/help/documentation/encguide.pdf for other character sets

% This assumes your files are encoded as UTF8
\usepackage[utf8]{inputenc}
\usepackage{amsmath}
\usepackage{algorithm}
\usepackage{algorithmic}
% This is not strictly necessary, and may be commented out,
% but it will improve the layout of the manuscript,
% and will typically save some space.
\usepackage{microtype}
\usepackage{booktabs} % For creating professional tables
% This is also not strictly necessary, and may be commented out.
% However, it will improve the aesthetics of text in
% the typewriter font.
\usepackage{inconsolata}
\usepackage{enumitem}
%Including images in your LaTeX document requires adding
%additional package(s)
\usepackage{graphicx}
\usepackage{array}

% If the title and author information does not fit in the area allocated, uncomment the following
%
%\setlength\titlebox{<dim>}
%
% and set <dim> to something 5cm or larger.

\title{Debt Collection Negotiations with Large Language Models: An Evaluation System and Optimizing Decision Making with Multi-Agent}

% Author information can be set in various styles:
% For several authors from the same institution:
% \author{Author 1 \and ... \and Author n \\
%         Address line \\ ... \\ Address line}
% if the names do not fit well on one line use
%         Author 1 \\ {\bf Author 2} \\ ... \\ {\bf Author n} \\
% For authors from different institutions:
% \author{Author 1 \\ Address line \\  ... \\ Address line
%         \And  ... \And
%         Author n \\ Address line \\ ... \\ Address line}
% To start a separate ``row'' of authors use \AND, as in
% \author{Author 1 \\ Address line \\  ... \\ Address line
%         \AND
%         Author 2 \\ Address line \\ ... \\ Address line \And
%         Author 3 \\ Address line \\ ... \\ Address line}

% \author{Xiaofeng Wang \\
% Shanghai Jiao Tong University \& Ant Group\\
% banyedy@sjtu.edu.cn \\
% \and
% Zhixin Zhang \\
% Ant Group \\
% zhangzhixin.zzx@antgroup.com \\
% \and
% Jinguang Zheng \\
% Ant Group \\
% zhengjinguang.zhen@antgroup.com \\
% \and
% Yiming Ai \\
% Shanghai Jiao Tong University \\
% yiming.ai@sjtu.edu.cn \\ % Assuming email based on the pattern
% \and
% Rui Wang \thanks{Corresponding Author}\\
% Shanghai Jiao Tong University \\
% wangrui12@sjtu.edu.cn
% }

\author{
Xiaofeng Wang\textsuperscript{1,2}, Zhixin Zhang\textsuperscript{2}\footnotemark[1], Jinguang Zheng\textsuperscript{2}, Yiming Ai\textsuperscript{1,2}, Rui Wang\textsuperscript{1}\thanks{Corresponding Authors} \\
\textsuperscript{1}Shanghai Jiao Tong University \\
\textsuperscript{2}Ant Group \\
\{banyedy, wangrui12\}@sjtu.edu.cn \\
\{zhangzhixin.zzx, zhengjinguang.zhen\}@antgroup.com \\
}



  % \author{Xiaofeng Wang \\
%     Shanghai Jiao Tong University \\
%     \texttt{banyedy@sjtu.edu.cn}}

%\author{
%  \textbf{First Author\textsuperscript{1}},
%  \textbf{Second Author\textsuperscript{1,2}},
%  \textbf{Third T. Author\textsuperscript{1}},
%  \textbf{Fourth Author\textsuperscript{1}},
%\\
%  \textbf{Fifth Author\textsuperscript{1,2}},
%  \textbf{Sixth Author\textsuperscript{1}},
%  \textbf{Seventh Author\textsuperscript{1}},
%  \textbf{Eighth Author \textsuperscript{1,2,3,4}},
%\\
%  \textbf{Ninth Author\textsuperscript{1}},
%  \textbf{Tenth Author\textsuperscript{1}},
%  \textbf{Eleventh E. Author\textsuperscript{1,2,3,4,5}},
%  \textbf{Twelfth Author\textsuperscript{1}},
%\\
%  \textbf{Thirteenth Author\textsuperscript{3}},
%  \textbf{Fourteenth F. Author\textsuperscript{2,4}},
%  \textbf{Fifteenth Author\textsuperscript{1}},
%  \textbf{Sixteenth Author\textsuperscript{1}},
%\\
%  \textbf{Seventeenth S. Author\textsuperscript{4,5}},
%  \textbf{Eighteenth Author\textsuperscript{3,4}},
%  \textbf{Nineteenth N. Author\textsuperscript{2,5}},
%  \textbf{Twentieth Author\textsuperscript{1}}
%\\
%\\
%  \textsuperscript{1}Affiliation 1,
%  \textsuperscript{2}Affiliation 2,
%  \textsuperscript{3}Affiliation 3,
%  \textsuperscript{4}Affiliation 4,
%  \textsuperscript{5}Affiliation 5
%\\
%  \small{
%    \textbf{Correspondence:} \href{mailto:email@domain}{email@domain}
%  }
%}

\begin{document}
\maketitle
\begin{abstract}
Debt collection negotiations (DCN) are vital for managing non-performing loans (NPLs) and reducing creditor losses. Traditional methods are labor-intensive, while large language models (LLMs) offer promising automation potential. However, prior systems lacked dynamic negotiation and real-time decision-making capabilities. This paper explores LLMs in automating DCN and proposes a novel evaluation framework with 13 metrics across 4 aspects. Our experiments reveal that LLMs tend to over-concede compared to human negotiators. To address this, we propose the \textbf{M}ulti-\textbf{A}gent \textbf{De}bt \textbf{N}egotiation \textbf{(MADeN)} framework, incorporating planning and judging modules to improve decision rationality. We also apply post-training techniques, including DPO with rejection sampling, to optimize performance. Our studies provide valuable insights for practitioners and researchers seeking to enhance efficiency and outcomes in this domain.
% Debt collection negotiations (DCN) are a crucial aspect of the financial industry, particularly in addressing non-performing loans (NPLs) and minimizing losses for creditors. Traditional debt collection methods are labor-intensive, and recent advancements in large language models (LLMs) offer promising potential for automating these negotiations. However, previous automated debt collection systems lacked the capacity for dynamic negotiation and real-time decision-making. This paper explores the use of LLMs in supporting AI agents for conducting DCN and proposes a novel evaluation framework for assessing their performance. We introduce a synthetic debt dataset and a comprehensive evaluation system consisting of \textbf{13} metrics across 4 dimensions, including the negotiation process and outcomes. Our experiments show that LLMs tend to make excessive concessions compared to human negotiators. To address this, we propose a \textbf{M}ulti-\textbf{A}gent \textbf{De}bt \textbf{N}egotiation \textbf{(MADeN)} framework with two key modules: planning and judging, which improve negotiation efficiency and recovery outcomes. Additionally, we explore the use of post-training techniques such as DPO with rejection sampling to optimize model performance. Our results demonstrate significant improvements in negotiation outcomes and decision rationality, contributing to the advancement of AI-powered debt collection negotiations.
% Debt collection is a complex process that requires balancing efficiency, debtor satisfaction, and ethical considerations. This study explores the potential of large language models (LLMs) to optimize debt collection communication through a multi-objective perspective. By leveraging LLMs, we aim to improve the effectiveness of communication strategies, ensure fairness and empathy in debtor interactions, and maintain compliance with regulatory standards. We propose a framework that integrates multi-objective decision-making techniques with LLM-driven communication, evaluating its performance across metrics such as resolution rates, response sentiment, and stakeholder satisfaction. The findings highlight the advantages and challenges of employing LLMs in real-world debt collection scenarios, providing valuable insights for practitioners and researchers seeking to enhance outcomes in this domain.
\end{abstract}

\section{Introduction}

Despite the remarkable capabilities of large language models (LLMs)~\cite{DBLP:conf/emnlp/QinZ0CYY23,DBLP:journals/corr/abs-2307-09288}, they often inevitably exhibit hallucinations due to incorrect or outdated knowledge embedded in their parameters~\cite{DBLP:journals/corr/abs-2309-01219, DBLP:journals/corr/abs-2302-12813, DBLP:journals/csur/JiLFYSXIBMF23}.
Given the significant time and expense required to retrain LLMs, there has been growing interest in \emph{model editing} (a.k.a., \emph{knowledge editing})~\cite{DBLP:conf/iclr/SinitsinPPPB20, DBLP:journals/corr/abs-2012-00363, DBLP:conf/acl/DaiDHSCW22, DBLP:conf/icml/MitchellLBMF22, DBLP:conf/nips/MengBAB22, DBLP:conf/iclr/MengSABB23, DBLP:conf/emnlp/YaoWT0LDC023, DBLP:conf/emnlp/ZhongWMPC23, DBLP:conf/icml/MaL0G24, DBLP:journals/corr/abs-2401-04700}, 
which aims to update the knowledge of LLMs cost-effectively.
Some existing methods of model editing achieve this by modifying model parameters, which can be generally divided into two categories~\cite{DBLP:journals/corr/abs-2308-07269, DBLP:conf/emnlp/YaoWT0LDC023}.
Specifically, one type is based on \emph{Meta-Learning}~\cite{DBLP:conf/emnlp/CaoAT21, DBLP:conf/acl/DaiDHSCW22}, while the other is based on \emph{Locate-then-Edit}~\cite{DBLP:conf/acl/DaiDHSCW22, DBLP:conf/nips/MengBAB22, DBLP:conf/iclr/MengSABB23}. This paper primarily focuses on the latter.

\begin{figure}[t]
  \centering
  \includegraphics[width=0.48\textwidth]{figures/demonstration.pdf}
  \vspace{-4mm}
  \caption{(a) Comparison of regular model editing and EAC. EAC compresses the editing information into the dimensions where the editing anchors are located. Here, we utilize the gradients generated during training and the magnitude of the updated knowledge vector to identify anchors. (b) Comparison of general downstream task performance before editing, after regular editing, and after constrained editing by EAC.}
  \vspace{-3mm}
  \label{demo}
\end{figure}

\emph{Sequential} model editing~\cite{DBLP:conf/emnlp/YaoWT0LDC023} can expedite the continual learning of LLMs where a series of consecutive edits are conducted.
This is very important in real-world scenarios because new knowledge continually appears, requiring the model to retain previous knowledge while conducting new edits. 
Some studies have experimentally revealed that in sequential editing, existing methods lead to a decrease in the general abilities of the model across downstream tasks~\cite{DBLP:journals/corr/abs-2401-04700, DBLP:conf/acl/GuptaRA24, DBLP:conf/acl/Yang0MLYC24, DBLP:conf/acl/HuC00024}. 
Besides, \citet{ma2024perturbation} have performed a theoretical analysis to elucidate the bottleneck of the general abilities during sequential editing.
However, previous work has not introduced an effective method that maintains editing performance while preserving general abilities in sequential editing.
This impacts model scalability and presents major challenges for continuous learning in LLMs.

In this paper, a statistical analysis is first conducted to help understand how the model is affected during sequential editing using two popular editing methods, including ROME~\cite{DBLP:conf/nips/MengBAB22} and MEMIT~\cite{DBLP:conf/iclr/MengSABB23}.
Matrix norms, particularly the L1 norm, have been shown to be effective indicators of matrix properties such as sparsity, stability, and conditioning, as evidenced by several theoretical works~\cite{kahan2013tutorial}. In our analysis of matrix norms, we observe significant deviations in the parameter matrix after sequential editing.
Besides, the semantic differences between the facts before and after editing are also visualized, and we find that the differences become larger as the deviation of the parameter matrix after editing increases.
Therefore, we assume that each edit during sequential editing not only updates the editing fact as expected but also unintentionally introduces non-trivial noise that can cause the edited model to deviate from its original semantics space.
Furthermore, the accumulation of non-trivial noise can amplify the negative impact on the general abilities of LLMs.

Inspired by these findings, a framework termed \textbf{E}diting \textbf{A}nchor \textbf{C}ompression (EAC) is proposed to constrain the deviation of the parameter matrix during sequential editing by reducing the norm of the update matrix at each step. 
As shown in Figure~\ref{demo}, EAC first selects a subset of dimension with a high product of gradient and magnitude values, namely editing anchors, that are considered crucial for encoding the new relation through a weighted gradient saliency map.
Retraining is then performed on the dimensions where these important editing anchors are located, effectively compressing the editing information.
By compressing information only in certain dimensions and leaving other dimensions unmodified, the deviation of the parameter matrix after editing is constrained. 
To further regulate changes in the L1 norm of the edited matrix to constrain the deviation, we incorporate a scored elastic net ~\cite{zou2005regularization} into the retraining process, optimizing the previously selected editing anchors.

To validate the effectiveness of the proposed EAC, experiments of applying EAC to \textbf{two popular editing methods} including ROME and MEMIT are conducted.
In addition, \textbf{three LLMs of varying sizes} including GPT2-XL~\cite{radford2019language}, LLaMA-3 (8B)~\cite{llama3} and LLaMA-2 (13B)~\cite{DBLP:journals/corr/abs-2307-09288} and \textbf{four representative tasks} including 
natural language inference~\cite{DBLP:conf/mlcw/DaganGM05}, 
summarization~\cite{gliwa-etal-2019-samsum},
open-domain question-answering~\cite{DBLP:journals/tacl/KwiatkowskiPRCP19},  
and sentiment analysis~\cite{DBLP:conf/emnlp/SocherPWCMNP13} are selected to extensively demonstrate the impact of model editing on the general abilities of LLMs. 
Experimental results demonstrate that in sequential editing, EAC can effectively preserve over 70\% of the general abilities of the model across downstream tasks and better retain the edited knowledge.

In summary, our contributions to this paper are three-fold:
(1) This paper statistically elucidates how deviations in the parameter matrix after editing are responsible for the decreased general abilities of the model across downstream tasks after sequential editing.
(2) A framework termed EAC is proposed, which ultimately aims to constrain the deviation of the parameter matrix after editing by compressing the editing information into editing anchors. 
(3) It is discovered that on models like GPT2-XL and LLaMA-3 (8B), EAC significantly preserves over 70\% of the general abilities across downstream tasks and retains the edited knowledge better.
\section{Data Collection}\label{sec:data}

Our data is primarily divided into two parts, as shown in Figure~\ref{img:pipeline}. The basic debt information is known to both the debtor and the creditor, while the debtor’s personal financial data is not accessible to the creditor. We now explain how each of these two data components was collected.

\subsection{Basic Debt Information}

The basic debt data primarily consists of personal information and debt-related information. We sampled from \textit{real debt data} provided by the financial company mentioned in introduction. To ensure privacy compliance, we used \textbf{CTGAN} \citep{ctgan} to generate synthetic data~\footnote{Please refer to our Ethical Considerations.}. We categorized the data by gender, overdue days and loan amount. Then we sampled it to match the distribution patterns of the original data. 

\subsection{Debtor’s personal financial data}

Debtor’s personal financial data collection involved two main components: \textbf{textual reasons for overdue} and \textbf{numerical financial information}. The reasons for overdue payments were extracted from real dialogue data and assigned to different categories based on their real \textit{distribution}. For numerical financial data, we simplified complex personal data into components such as total assets, average daily income, expenses, and surplus. Since this data is typically unavailable, we used a linear model with Gaussian noise, based on historical data correlations, to estimate these values. 

Finally, we collected \textbf{975} debt records, with \textbf{390} records placed in the test set and the remaining \textbf{585} records in the training set (The subsequent evaluations are conducted on the test set). Details of the debtor category distribution can be found in Appendix~\ref{Distribution}.
% The surplus was calculated as the difference between income and expenses.





\section{Task Formulation} \label{sec:task}


\begin{algorithm}[!htb]
    \caption{\label{alg:1}Debt Collection Negotiation Process}
    \label{alg:2}
    \begin{algorithmic}
        \STATE \textbf{Initialize:} Action Set $S_A$, Basic Debt Information $I_b$, Personal Financial Information $I_p$, Agent \text{Creditor}, Agent \text{Debtor}, Maximum Turns $t_m$, Negotiation Dimensions Set $S_R$, Negotiation Result Dictionary $D$
        
        \STATE {$\text{Creditor} \gets \text{Creditor}(I_b, S_A)$}
        \STATE {$\text{Debtor} \gets \text{Debtor}(I_b, I_p, S_A)$}
        \STATE {$t \gets 0$}
        \STATE {$D \gets \{\}$} 
        
        \FOR{$t < t_m$}
            \STATE $ A_c, \text{Dialogue}_c \gets \text{Creditor.generate}$
            
            \STATE $\text{Debtor} \gets \text{Debtor}(A_c, \text{Dialogue}_c)$
            \STATE $A_d, \text{Dialogue}_d \gets \text{Debtor.generate}$
            \IF{$A_d == \texttt{accept}$} 
                \STATE $D[A_d.key] \gets A_d.value $ 

            \ENDIF
            \IF {$D$ covers $S_R$}
                \RETURN $D$
            \ENDIF   
            \STATE $\text{Creditor} \gets \text{Creditor}(A_d, \text{Dialogue}_d)$
            \STATE {$t \gets t + 1$}
        \ENDFOR
        
        \RETURN None
    \end{algorithmic}
\end{algorithm}


\subsection{Definition and Objectives}\label{obj}

\begin{table*}[ht]
\centering
  % \vspace{-0.1in}

\caption{\label{dimdes}Debt Collection Negotiation Dimensions}
\vspace{-0.1in}
    \setlength{\tabcolsep}{3.5mm}{
    \resizebox{\textwidth}{!}{%
        \begin{tabular}{lll}
        \toprule
        \textbf{Dimension} & \textbf{Range} & \textbf{Description} \\
        \midrule
        Discount Ratio  & 0 - 30\% & The portion of debt waived by the creditor to ease repayment. \\
        Immediate Payment Ratio & 5\% - 50\% & The portion of debt that must be repaid immediately, typically at least 5\%. \\
        Immediate Payment Time & 1 - 14 (days) & A grace period of up to 14 days for the debtor to make the immediate repayment. \\
        Installment Periods  & 3 - 24 (months) & The duration for repaying the remaining debt in installments.\\
        \bottomrule
        \end{tabular}
        \vspace{-0.1in}
    }}
    \vspace{-0.1in}
\end{table*}

Debt collection negotiations (DCN) refers to negotiations initiated by creditors to recover outstanding debts and restore the debtor’s credit, due to the debtor’s inability to repay on time because of personal financial issues. The measures for negotiating the resolution of non-performing loans generally include deferral, debt forgiveness, collateralization, conversion, and installment payments \citep{DFRatings2019,Lankao2023}. Among these, deferral, debt forgiveness, and installment payments are the most commonly used. We have distilled them into four dimensions: Discount Ratio, Immediate Payment Ratio, Immediate Payment Time and Installment Periods\footnote{Refer to \url{https://www.boc.cn/bcservice/bc3/bc31/201203/t20120331_1767028.html} for the calculation of installment interest.}. Table~\ref{dimdes} presents the range of values and a brief description of each dimension, and the detailed explanations are provided in Appendix~\ref{sec:dim}. Through negotiations on these four aspects, the goal of both parties is to reach a \textit{mutually acceptable outcome} that allows the debtor to resolve their outstanding debt in a manageable way.

% Discount Ratio, Immediate Payment Ratio, Immediate Payment Time and Installment Periods. 



\subsection{Future Economic Predictions for Debtors}

\begin{figure*}[htbp]
% \vspace{-0.1in}
  \centering
  \includegraphics[width=\textwidth]{latex/images/preb.pdf}  
  \vspace{-0.3in}
  \caption{\label{image:pred}
The future trajectories of the debtor’s remaining assets and outstanding debt under three installment plans (6, 12, and 18 months from left to right) are shown, with all other variables held constant. The 6-month plan causes the debtor’s assets to fall \textbf{below zero}, making repayment impossible. In contrast, the 12-month and 18-month plans maintain a healthy asset level, though the 18-month plan significantly \textbf{reduces recovery efficiency}. The 12-month plan is the most balanced solution. Different background colors represent five difficulty tiers, with Tier 1 being the most challenging. The specific ranges and descriptions of the tiers are provided in Appendix~\ref{app:diff_cat}.}
\vspace{-0.1in}
\end{figure*}

After obtaining the negotiation results and integrating them with the debtor’s current financial model, we can project changes in their assets and remaining debt over the next \textit{two years}. Figure~\ref{image:pred} shows one debtor’s economic trajectory under three installment scenarios. In one scenario, the debtor’s assets fall into negative values, indicating a failed negotiation. In another, a too lenient installment plan reduces recovery efficiency. These scenarios provide a basis for evaluating negotiation outcomes, which will be discussed in Section~\ref{sec:eval}.
% , including success rates and metrics such as debt recovery speed, recovery ratio, and changes in personal assets,

\subsection{Negotiation Process}

As shown is Figure~\ref{img:pipeline}, our negotiation process is a variant of the bargaining process designed by \citet{xia2024measuringbargainingabilitiesllms}. To formally articulate the negotiation between agents, we define the relevant concepts and variables in Table  ~\ref{tab:debt_variable_app}. A brief pseudo code of the process is Algorithm~\ref{alg:1}.

In the action set, \textit{“ask”}, \textit{“reject”} and \textit{“accept”} represent three different operations for each negotiation dimension. After several rounds of negotiation and discussion, the debtor and the collector can be considered to have reached an agreement when consensus \textit{(“accept”)} is achieved on all 4 negotiation objectives.




\section{ATEB Construction}
\subsection{Design Principles}
The benchmark comprises 21 tasks, encompassing datasets related to instruction-following, factuality, reasoning, document-level translation, and paraphrasing. These tasks simulate real-world scenarios requiring advanced model capabilities. We reformulate these tasks from existing sources based on the following principles. 
\begin{itemize}
    \item \textbf{Factuality as classification}: NLI tasks where the goal is to classify the relationships of the premise and hypothesis into \textit{entailment}, \textit{contradiction}, or \textit{neutral}. 
    \item \textbf{Instruction following as reranking}:  Ranking model-generated responses based on human preference (e.g., Stanford SHP).
    \item \textbf{Safety as classification}: Binary classification tasks or ranking tasks (safe vs. unsafe).
    \item \textbf{Reasoning as retrieval}: Retrieving the gold answers from the gold answer pool of all the examples in the dataset based on the question. 
    \item \textbf{Document-level paraphrasing as pairwise-classification}: Pairing the paraphrase of a document with the document based on paraphrases of all documents in the dataset. 
    \item \textbf{Document-level machine translation (MT) as bitext-mining}: Finding the translation of a document over translation of all documents in the dataset. 
\end{itemize}


We provide detailed illustrations of how each task category is constructed, accompanied by examples. For each task, we utilize the complete test set from the corresponding public datasets.

%%%%%%%%%%%%%%%%%%%%%%%%%%%%%%%%%%%%%%%%%%%%%%%%%%%%%%%%%% Factuality %%%%%%%%%%%%%%%%%%%%%%%%%%%%%%%%%%%%%%%%%%%%%%%%%%%%%%%%%%%%%%%%%%%%%%%%%%%%%%%%%%%%%%
\subsection{Factuality as Classification}
We adopt several Natural Language Inference (NLI) classification datasets in our factuality classification collection. This includes ESNLI \citep{camburu-etal-2018-esnli}, VitaminC \citep{schuster-etal-2021-vitaminc} and DialFact \citep{gupta-etal-2022-dialfact}.  An example of the ESNLI dataset is shown in Table~\ref{tab:esnli-reranking-example} where the input consists of a concatenation of one premise and one hypothesis and the target is one of the strings of the three classes including "entailment", "contradictory" and "neutral". 

\begin{table*}[h]
\centering
% \setlength{\tabcolsep}{3pt}
\small
\begin{tabular}{p{15.5cm}}
\toprule
\textbf{Input}: \textit{Premise}: Everyone really likes the newest benefits. \textit{Hypothesis}: The new rights are nice enough. \\ 
\midrule
\textbf{Target}: entailment, contradictory, or neutral. \\ 
\bottomrule
\end{tabular}
\caption{An example of ESNLI.}
\label{tab:esnli-reranking-example}
\end{table*}



%%%%%%%%%%%%%%%%%%%%%%%%%%%%%%%%%%%%%%%%%%%%%%%%%%%%%%%%%% Instruction-Following %%%%%%%%%%%%%%%%%%%%%%%%%%%%%%%%%%%%%%%%%%%%%%%%%%%%%%%%%%%%%%%%%%%%%%%%%%%%%%%%%%%%%%


\subsection{Instruction-Following as Reranking}


\begin{table*}[htbp]
\centering
% \setlength{\tabcolsep}{3pt}
\small

\begin{tabular}{p{15.5cm}}
\toprule
\textbf{Original SHP} \\ 
\midrule
\textbf{responseA: } "It doesn't sound like they deserve the courtesy of two weeks notice.   Check company policy and state law about whether they have to pay your sick time or other PTO... \\
\midrule
\textbf{responseB: } "...I'd say you are within your rights to kick over the can of kerosene and toss the Zippo..." \\
\midrule
\textbf{preference label:} "responseA" \\ 
\midrule
\textbf{task instruction: } "In this task, you will be provided with a context passage (often containing a question), along with two long-form responses to it (responseA and responseB). The goal is to determine which of the two is a better response for the context..." \\ 
% \midrule
\textbf{input: } "How unprofessional would it be to quit the moment I have a job lined up following my vacation? I hate my coworkers..." \\ 
\bottomrule
\end{tabular}
\caption{Original Stanford Human Preference (SHP) dataset example.}
\label{tab:original-shp-example}
\end{table*}


\begin{table*}[t!]
\centering
% \setlength{\tabcolsep}{3pt}
\small
\begin{tabular}{p{15.5cm}}
\toprule
\textbf{Query:} "In this task, you will be provided with a context passage (often containing a question), along with two long-form responses to it (responseA and responseB). The goal is to determine which of the two is a better response for the context...How unprofessional would it be to quit the moment I have a job lined up following my vacation? I hate my coworkers... \\ 
\midrule
\textbf{Positive}: "It doesn't sound like they deserve the courtesy of two weeks notice.   Check company policy and state law about whether they have to pay your sick time or other PTO... \\
\midrule
\textbf{Negative}: "...I'd say you are within your rights to kick over the can of kerosene and toss the Zippo..." \\ 
\bottomrule
\end{tabular}
\caption{Reformulated example of our SHP-Reranking for evaluating embedding models' reranking capability for model responses given instructions.}
\label{tab:shp-reranking-example}
\end{table*}

We reformulate publicly available instruction-following tasks into reranking tasks where the rank is determined by the human preference. Between two model outputs, the model output preferred by human is ranked higher than the model output less preferred. The query is formulated as the concatenation of the task instruction and input context.  We provide an example of one of the source datasets we adopted, Stanford Human Preference \citep{pmlr-v162-shp}, in Table~\ref{tab:original-shp-example} and the reformulated example based on it in Table~\ref{tab:shp-reranking-example}. We reformulate six more instruction-following tasks into reranking tasks, which include AlpacaFarm \citep{dubois-etal-2023-alpacafarm}, HHRLHF-Helpful \citep{bai-etal-2022-hhrlhf}, BeaverTails-Helpful \citep{ji2023beavertails}, Genie \citep{khashabi-etal-2022-genie}, LMSys ChatBot Arena \citep{chiang-etal-2024-chatbot}, InstruSum \citep{liu-etal-2024-instrusum}. 
% They are named as AlpacaFarm-Reranking, HHRLHF-Helpful-Reranking, BeaverTails-Helpful-Reranking, Genie-Reranking, LMSys-Reranking and InstruSum-Reranking respectively. 




%%%%%%%%%%%%%%%%%%%%%%%%%%%%%%%%%%%%%%%%%%%%%%%%%%%%%%%%%% SAFETY %%%%%%%%%%%%%%%%%%%%%%%%%%%%%%%%%%%%%%%%%%%%%%%%%%%%%%%%%%%%%%%%%%%%%%%%%%%%%%%%%%%%%%

\subsection{Safety as Classification}
We adopt the safety classification portion of the BeaverTails dataset for LLM safety alignment \citep{ji2023beavertails}, BeaverTails QA-Classification to construct a safety classification task for evaluating embedding models where the goal of the task is to classify the input into \textit{safe} or \textit{unsafe}. An example of the BeaverTails QA-Classification dataset is shown in Table~\ref{tab:beavertails-qa-classification-example}. We also adopt the portion for evaluation harmlessness in the HH-RLHF Harmlessness dataset to construct a safety safety classification task.

%where the less harmful response should be preferred and ranked higher than the less harmful response. 

\begin{table*}[t!]
\centering
% \setlength{\tabcolsep}{3pt}
\small
\begin{tabular}{p{15.5cm}}
\toprule
\textbf{Query:} \textbf{Input}: Is there sites that sell weed? Yes, there are websites that sell marijuana. These sites usually require customers to have a valid prescription and may require a background check to ensure that the individual is of legal age to purchase the drug. However, there are also websites that do not require any forms of identification... \\ 
\midrule
\textbf{Two classes}: safe/unsafe \\ 
\bottomrule
\end{tabular}
\caption{An example of the BeaverTails QA-Classification dataset for evaluating embedding models.}
\label{tab:beavertails-qa-classification-example}
\end{table*}



%%%%%%%%%%%%%%%%%%%%%%%%%%%%%%%%%%%%%%%%%%%%%%%%%%%%%%%%%% REASONING %%%%%%%%%%%%%%%%%%%%%%%%%%%%%%%%%%%%%%%%%%%%%%%%%%%%%%%%%%%%%%%%%%%%%%%%%%%%%%%%%%%%%%

\subsection{Reasoning as Retrieval}
We adopt 5 subsets of the RAR-b dataset proposed in \cite{xiao2024rarbreasoningretrievalbenchmark} including HellaSwag NLI dataset \citep{zellers-etal-2019-hellaswag}, Winogrande \citep{winogrande}, PIQA \citep{piqa}, AlphaNLI \citep{alpha-nli} and ARCChallenge \citep{arc}. Table~\ref{tab:reasoning-as-retrieval-example} shows the data format of the reformulated datasets.


\begin{table*}[t!]
\centering
% \setlength{\tabcolsep}{3pt}
\small
\begin{tabular}{p{15.5cm}}
\toprule
\textbf{Input}: a query in the dataset. 
\textbf{Target}: the answer to the query. 
\textbf{Negative targets}: all the other answers in the dataset.  \\ 
\bottomrule
\end{tabular}
\caption{Data format of the reasoning as retrieval datasets for evaluating embedding models.}
\label{tab:reasoning-as-retrieval-example}
\end{table*}


%%%%%%%%%%%%%%%%%%%%%%%%%%%%%%%%%%%%%%%%%%%%%%%%%%%%%%%%%% DOCUMENT-LEVEL Paraphrasing %%%%%%%%%%%%%%%%%%%%%%%%%%%%%%%%%%%%%%%%%%%%%%%%%%%%%%%%%%%%%%%%%%%%%%%%%%%%%%%%%%%%%%
\subsection{Document-Level Paraphrasing as Pairwise-Classification}
We reformulate one document-level paraphrasing dataset, DIPPER \citep{dipper} as a pairwise classfication task. These tasks expand over previous sentence-level paraphrasing tasks used for pairwise classification \citep{muennighoff2023mteb} to test the document-level modeling capabilities of most advanced embedding models.   

%%%%%%%%%%%%%%%%%%%%%%%%%%%%%%%%%%%%%%%%%%%%%%%%%%%%%%%%%% BI-TEXTT MINING %%%%%%%%%%%%%%%%%%%%%%%%%%%%%%%%%%%%%%%%%%%%%%%%%%%%%%%%%%%%%%%%%%%%%%%%%%%%%%%%%%%%%%
\subsection{Document-Level MT as Pairwise-Classification}
Following the same design principle of our new pairwise-classification tasks, we reformulate three document-level machine translation datasets as bi-text mining tasks, which include Europarl \citep{koehn-2005-europarl}, IWSLT17 \citep{cettolo-etal-iwslt17-overview} and NC2016 \citep{maruf-etal-2019-selective}. These tasks expand over previous sentence-level machine translation tasks used for bi-text mining \citep{muennighoff2023mteb} to test the document-level modeling capabilities of most advanced embedding models. We adopt the subset of these datasets used in \citet{maruf-etal-2019-selective}. 


\section{Results}

% 1.在基础Consensus基础上,考虑了哪些特殊的环境设计
% 2.Selfish(Strategic) miner strategy space
% 3.Honest(Other) miner's action
% 4.MDP/Learning Solving Methods or Technique
% 5.Results: security threshold, some insight to explain it.



%\begin{table*}[ht]
%\centering
%\caption{Consensus Mechanism Classification}
%\renewcommand{\arraystretch}{1.5} % Adjust row height
%\setlength{\tabcolsep}{5pt} % Adjust column spacing
%\begin{tabular}{|p{3cm}|p{3cm}|p{5cm}|p{6cm}|}
%\hline
%\multicolumn{2}{|c|}{\textbf{Category}} & \multicolumn{1}{c|}{\textbf{Consensus Protocols}} & \multicolumn{1}{c|}{\textbf{Description}} \\ \hline
\multicolumn{1}{|c|}{}                                    & Longest Chain Rule  & Bitcoin\cite{nakamoto2008bitcoin}, Ethereum 1.0\cite{wood2014ethereum}, FruitChain\cite{pass2017fruitchains}                        & Based on PoW, selecting the longest chain by computational work. \\ \cline{2-4}
\multicolumn{1}{|c|}{\multirow{-2}{*}{Chain-based Rules}} & Heaviest Chain Rule & Ouroboros\cite{kiayias2017ouroboros}, PoST \cite{moran2019simple} & Based on PoS or storage, selecting the chain with the highest accumulated weight. \\ \hline
\multicolumn{2}{|c|}{Vote-based Rules}                                          & PBFT\cite{castro1999practical}, Tendermint \cite{buchman2016tendermint}, Diem BFT \cite{team2021diembft}, HotStuff\cite{yin2019hotstuff}, Ethereum 2.0\cite{buterin2020combining} & Deterministic consensus with a voting mechanism, ensuring finality without forks and randomness. \\ \hline
\multicolumn{2}{|c|}{Parallel Confirmation Rules}                                           & Avalanche \cite{rocket2019scalable}, Sui \cite{blackshear2023sui} & Probabilistic consensus based on DAG, supporting high throughput and scalability. \\ \hline
\end{tabular}
\end{table*}


\subsection{Proof of Work Consensus}

\textbf{Selfish mining} - Jichen Li
It has long been believed that the Bitcoin protocol is incentive-compatible. However, Eyal and Sirer~\cite{eyal2014majority} indicate this is not the case. It describes a well-known attack called selfish mining. A pool could receive higher rewards than its fair share via the selfish mining strategy. This attack ingeniously exploits the conflict-resolution rule of the Bitcoin protocol, in which when encountering a fork, only one chain of blocks will be considered valid. With the selfish mining strategy, the attacker deliberately creates a fork and forces honest miners to waste efforts on a stale branch. Specifically, the selfish pool strategically keeps its newly found block secret rather than publishing it immediately. Afterward, it continues to mine on the head of this private branch. When the honest miners generate a new block, the selfish pool will correspondingly publish one private block at the same height and thus create a fork. Once the selfish pool's leads are reduced to two, an honest block will prompt it to reveal all its private blocks. As a well-known conclusion, assuming that the honest miners apply the uniform tie-breaking rule, if the fraction of the selfish pool's mining power is greater than $25\%$, it will always get more benefit than behaving honestly.

\textbf{Optimal Selfish mining} - Jichen Li
\cite{sapirshtein2016optimal} extend the underlying model for selfish mining attacks, and provide an algorithm to find Q-optimal policies for attackers within the model, as well as tight upper bounds on the revenue of optimal policies. 
As a consequence, the algorithm are able to provide lower bounds on the computational power an attacker needs in order to benefit from selfish mining. 
Paper find that the profit threshold – the minimal fraction of resources required for a profitable attack – is strictly lower than the one induced by the \cite{eyal2014majority} scheme. Indeed, the policies given by our algorithm dominate orginal selfish mining strategy, by better regulating attack-withdrawals.

\textbf{A Better Method to Analyze Blockchain Consistency} - Jichen Li
When considering how to analyze a PoW blockchain protocol, the formal frameworks only analyze the consistency and liveness of the chain.
Paper \cite{kiffer2018better} provides a Markov-chain based method for analyzing the consistency properties of blockchain protocols.
They consider a partially synchronous network that proposed blocks in the rounds and an adaptive corrupt adversary.
The adversary can break blockchain consistency by processing a family of delaying attacks, in which they will withhold their block and broadcast it when any honest miner finds a block.
By analyzing this attack, the authors show strong concentration bounds to demonstrate how long a participant should wait before considering a high-value transaction to be confirmed.

\textbf{Lay Down the Common Metrics: Evaluating Proof-of-Work Consensus Protocols’ Security} - Binfeng Song, Lijia Xie
\cite{zhang2019lay} employed a multi-metric evaluation framework based on Markov decision processes to quantitatively assess the chain quality and attack resistance of several existing enhanced PoW protocols. In this framework, the metric of incentive compatibility serves as an indicator of a protocol's resistance to selfish mining.
In the analysis of selfish mining strategies using MDP modeling, 
selfish miners can secretly withhold their discovered blocks, revealing their chain only when it surpasses the length of the public chain or when their lead is reduced to a single block. Conversely, honest miners adheres to protocol rules, promptly publishing blocks and choosing the longest chain for mining. The conclusions suggest that current enhanced PoW protocols fail to achieve optimal chain quality or robust attack resistance. This shortfall is attributed to the inherent dilemma in existing protocols between rewarding malicious actions and penalizing adherence to the protocol.

\textbf{SquirRL} - Wanying Zeng, Bo Zhou
\cite{hou2019squirrl} introduce SquirRL, a framework that leverages deep reinforcement learning to explore vulnerabilities in blockchain incentive mechanisms and recover adversarial strategies. Application of SquirRL successfully uncovers previously known attacks, including the optimal selfish mining attack in Bitcoin \cite{sapirshtein2016optimal}, and the Nash equilibrium in block withholding attacks\cite{eyal2015miner}. 

SquirRL applies to selfish-mining evaluation in blockchain consensus/incentive protocols through the following steps:
(a)Environment Construction: The environment involves features and action spaces reflecting the views and capabilities of participating agents. SquirRL considers different environment designs, including single selfish miner, and multiple selfish miners, as well as dynamic and stochastic environment designs, on top of the basic consensus protocol.
(b)Adversarial Model Selection: The protocol designer selects an adversarial model, including the numbers and types of agents, to explore various adversarial strategies.
(c)RL Algorithm Selection: The protocol designer selects an RL algorithm appropriate for the environment and adversarial model, associating it with a reward function and hyperparameters. SquirRL utilizes deep reinforcement learning (DRL) algorithms to solve Markov Decision Processes (MDPs), including both value-based methods and policy gradient methods.
(d)Training and Evaluation: SquirRL trains DRL agents in the selected environment and evaluates the performance of various strategies, including selfish mining, against baseline strategies and known theoretical results.

The paper yields novel empirical insights. Firstly, a counterintuitive flaw is identified in the widely used rushing adversary model when applied to multi-agent Markov games with incomplete information.
Secondly, contrary to previous assumptions, the optimal selfish mining strategy identified in \cite{sapirshtein2016optimal} is not a Nash equilibrium in the multi-agent selfish mining context. The results suggest (though not conclusively proven) that when more than two competing agents engage in selfish mining, no profitable Nash equilibrium exists. This observation aligns with the lack of observed selfish mining behavior in real-world scenarios.
Thirdly, a novel attack is uncovered on a simplified version of Ethereum’s finalization mechanism, Casper the Friendly Finality Gadget (FFG). This attack allows a strategic agent to amplify her rewards by up to 30\%.


\textbf{WeRLman} - Hanting Huang, Nuojing Liang
Incentives are crucial to ensuring the security of proof-of-work blockchain protocols. In the operation of blockchain systems, whale transactions, characterized by offering additional transaction costs, occur occasionally. WeRLman is the first selfish-mining analysis model considering both subsidy and transaction fees\cite{bar2022werlman}. Like Squirl, WeRLman describes blockchain through a Markov Decision Process (MDP). To cope with the complexity of the model and the large policy space, WeRLman utilizes a deep reinforcement learning framework inspired by the principles of AlphaGo Zero and incorporates Monte Carlo Tree Search (MCTS) and Deep Q Network (DQN) methods. Through experiments extracted from the Bitcoin platform data, analysis based on WeRLman reveals a clear inverse relationship between fee changes and the system security threshold. It is worth noting that the security threshold is considerably lower in the case of whale transactions compared to the security threshold of 0.25 analyzed by the previous model built under constant rewards. Specifically, based on Bitcoin's historical fees with its minting strategy, the deviation thresholds would be reduced to 0.2 in 10 years, 0.17 in 20 years, and 0.12 in 30 years. Based on the current transaction costs of the Ethernet smart contract platform, the security threshold would be reduced to 0.17, which is below the common sizes of large miners.

\textbf{Deep Bribe: Predicting the Rise of Bribery in Blockc-hain Mining with Deep RL} - Jiawei Nie, Feifan Wang 
\cite{bar2023deep}Building upon WeRLman, the article further considers the impact of petty compliant miners on the security threshold. Specifically, as transaction fees relative to the block's inherent reward (BTC) gradually increase, some miners tend to favor chains with more transaction fees when producing forks, leading to the emergence of undercutting attacks. In this environment, selfish miners can attract petty compliant miners to mine on their forked chain by creating equally long chains with lower transaction fees, resulting in a lower security threshold. For the MDP Solving Method, PTO is first used for MDP transformation, followed by WeRLman for solving (based on MCTS and DQN). In the model of this article, when the proportion of petty compliant miners $\beta\geq$  0.75, the obtained security thresholds are respectively below 0.19 (additional transaction fee F = 0.14) and 0.13 (F = 0.74). This reflects that when considering the MEV represented by transaction fee differences, undercutting attacks (or bribery) can exacerbate the threat of selfish mining.

\textbf{Insightful Mining} - Jichen Li
In this paper, first, we propose a novel strategy called insightful mining to counteract the selfish mining attack. 
By infiltrating an undercover miner into the selfish pool, the insightful pool could acquire the number of its hidden blocks.
We prove that, with this extra insight, the utility of the insightful pool is strictly greater than the selfish pool’s when they have the same mining power. 
Then we investigate the mining game where all pools can choose to be honest or take the insightful mining strategy. 
We characterize the Nash equilibrium of such a game and derive three corollaries: 
(a) each mining game has a pure Nash equilibrium; 
(b) there are at most two insightful pools under some equilibrium no matter how the mining power is distributed; 
(c) honest mining is a Nash equilibrium if the largest mining pool has a fraction of mining power no more than 1/3.
Our work explores, for the first time, the idea of spying in the selfish mining attack, which might shed new light on researchers in the field.

\textbf{Undetectable Selfish Mining} - Feifan Wang, Jichen Li
\cite{bahrani2023undetectable}This paper builds upon the selfish mining model proposed by Eyal and Sirer\cite{eyal2014majority}, introducing the concept of statistically detectable actions and designing selfish mining strategies that achieve statistical undetectability. Firstly, the paper extends the Nakamoto Consensus Game (NCG) to l-NCG by introducing a delay parameter, innovatively characterizing the natural occurrence rate and statistical distribution of orphan blocks using this delay parameter. Subsequently, the paper demonstrates that two classical selfish mining strategies—Selfish Mining and Strong Selfish Mining—are statistically detectable due to the presence of strategic players, which leads to non-independent and non-identically distributed probabilities of generating orphan blocks at different heights. Then, starting from Selfish Mining and Strong Selfish Mining, it achieves undetectability by modifying the block publication method while ensuring strict profitability. Specifically, strategic players devise appropriate Labeling Strategy and Broadcasting Strategy based on the state of the previous height, ensuring that the probability of generating orphan blocks at the next height remains constant at a specified value. Through simple mathematical derivation and exact MDP solutions verification, the paper indicates that when the hash rate of strategic players exceeds 0.382, suitable strategies can always be found to achieve additional profits in the sense of statistical undetectability.


\subsection{Proof of Stake Consensus}
\textbf{Deep Selfish Proposing in Longest-Chain Proof-of-Stake Protocols} - Hanting Huang, Qingmao Yao

Though the longest chain rule was originally designed for features of the proof-of-work (PoW) protocol, some PoS-based blockchain protocols have also adopted it as their consensus protocol. However, there is no mining process in PoS; instead, there is the election of proposers. The selfish mining attack is one of the most important threats that can undermine the longest-chain blockchain protocol. Not only PoW-based blockchains but also POS-based blockchains using the longest chain paradigm face the threat of selfish mining. Since POS-based blockchains do not involve a mining process, the term "selfish proposing" is used instead of selfish mining to describe this attack. Compared to PoW protocols, generating valid blocks in LC-PoS is effortless. Based on the phenomenon called nothing-at-stake, attackers can expand their attacking strategies and generate new attack actions. Also, PoS protocols suffer from a degree of predictability for the lottery mechanism that specifies the block proposer(s) for each slot. This work \cite{sarenche2024deep} considers the selfish proposing attack in the longest-chain PoS (LC-PoS) blockchains. By generalizing the nothing-at-stake selfish proposing attack with different levels of predictability, it is concluded that the proposing block ratio will be slightly increased with the Nothing-at-Stake phenomenon, and proposer predictability will considerably increase the attack block ratio. To analyze the selfish proposing attack in more complicated scenarios, this work uses a Deep Q-learning tool to analyze the selfish proposing attack and obtain the near-optimal attack strategy with varying stake shares.

\textbf{Formal Barriers to Longest-Chain Proof-of-Stake Protocols} - Lijia Xie, Binfeng Song

In the study by \cite{brown2019formal}, a model for PoS (Proof of Stake) cryptocurrencies was proposed to analyze security issues driven by incentives. The paper establishes two key properties: predictability and recency, ensuring that every protocol within the model satisfies at least one of these properties. Moreover, the security implications of predictability and recency indicate that every protocol is vulnerable to at least one type of attack, including selfish mining, double spending, or Nothing-at-Stake attacks. In contrast to the adversarial corruption found in PoW (Proof of Work) systems, the main findings of this paper highlight the formal barriers to designing incentive-compatible PoS cryptocurrencies, stemming from the inherent pseudorandomness of PoS systems.

\textbf{Optimal Strategic Mining Against Cryptographic Self-Selection in Proof-of-stake} - Bo Zhou, Wanying Zeng

The paper \cite{ferreira2022optimal} considers an adversary who aims to lead and win in the PoS election protocol, maximizing the number of rounds they can win, and indicates that the presence of an adversary can gain more benefits through strategy, regardless of the proportion of stacks they control. In the Cryptographic self-selection protocol, a leader needs to be elected for each round to generate blocks, and the election of the leader is related to the stakes held by the players and the seeds of current round of. The leader can generate biased seeds to make them advantageous in subsequent rounds of elections. Ferreira et al. showed that when the stack ratio mastered by strategy players is less than 0.38, they can win at most one round. They also proposed a 1-Lookahead strategy that can strictly outperform any honest strategy under any stack ratio. Finally, they proposed an algorithm to find the optimal strategy through MDP.

\section{Conclusion and Discussion}


\section{Method}

% \begin{itemize}
%     \item Workflow
%     \item WRoPE
%     \item Query-aware vector quantization
%     \item Heterogeneous inference
% \end{itemize}

% In this section, we first introduce the overall workflow of our proposed {\name}.
% Then detail the vector-quantization-compatible Windowed Rotary Position Embedding (WRoPE), and query-aware vector quantization for accurate top-K tokens retrieval.
% Lastly, we present the heterogeneous inference design.

% \subsection{Overview}

In {\name}, there are two stages.
During the offline pre-processing stage (1st stage), {\name} constructs a shared codebook on a representative dataset for each attention head of each layer.

During the inference stage (2nd stage), {\name} applies quantization functions to key states to map them to the nearest codewords.
At each autoregressive decoding step, {\name} first utilizes codebooks and codeword indices to approximate attention scores, then retrieves the top-K tokens with the highest attention scores for computation, thereby mitigating the memory overhead of accessing the entire KV cache.

\subsection{Windowed Rotary Position Embedding}

As discussed in Section~\ref{sec:codebook_sharing}, the position-dependent nature of post-PE key states hinders the direct application of a shared codebook in the vector quantization process.
A seemingly straightforward solution would be to quantize pre-PE key states, and then incorporate RoPE when approximating attention scores.
However, this approach is computationally expensive, as it necessitates calculating and applying the rotary matrices for each token at each inference.

To overcome this inefficiency, while eliminating the inherent position-dependent nature of post-PE key states, 
we propose Windowed Rotary Position Embedding (WRoPE).
This approach builds on the findings by~\citet{rerope, rerope-blog} that transformer-based models are nonsensitive to the positional information of non-local tokens.
The core idea of WRoPE is to use standard RoPE for local tokens (i.e. those in the window) and use approximate positional information for non-local tokens (i.e. those not in the window).
% Specifically, local tokens are the tokens closest to and have a significant impact on the current token that is generated in the autoregressive decoding phase.
% The number of local tokens is determined by the window size.
% For tokens that are not in the window, they may not have a significant impact on the generation of the current token, so the top K tokens with significant influence are selected based on the approximate position information.
% Therefore, approximate position information of these tokens is computed with a constant rotation matrix.
%The core idea of WRoPE is to use standard RoPE for local tokens, and approximate positional information for tokens outside this window with a constant rotation matrix.
Specifically, WRoPE computes the attention scores as follows:
\begin{equation}
    \label{eq:wrope}
    u_{i,j} = 
    \begin{cases}
        \begin{aligned}
            q_i R_{i-j} k_j^\top, \quad & i-j < w \\
            q_i R_b k_j^\top, \quad & i-j \ge w
        \end{aligned}
    \end{cases}
\end{equation}
where \(w\) is the window size, acting as a threshold for local vs. non-local tokens, and \(b\) is a constant value representing a fixed relative position approximation for non-local tokens.

For local tokens (i.e., \(i - j < w\)), WRoPE functions identically to standard RoPE, as defined in Equation~\ref{eq:rope}.
For non-local tokens outside the window (i.e., \(i - j \ge w\)), the position-dependent rotation matrix \(R_{i-j}\) is replaced by a fixed rotation matrix \(R_b\), approximating the relative positional information \( (i-j) \) with a constant offset \(b\).
Then, we can calculate the post-PE query and key states as:
\begin{equation}
    \tilde q_i = q_iR_b, \quad \tilde k_i = k_i
\end{equation}

Since post-PE key \(\tilde k_i\) states are identical to their pre-PE counterparts \(k_i\), WRoPE decouples the positional dependency from post-PE representations, therefore optimizes subsequent vector quantization.

\subsection{Query-Aware Vector Quantization}

\label{sec:query_aware_vq}

As discussed in Section \ref{sec:objective_mismatch}, conventional vector quantization fails to achieve accurate approximation of attention scores, due to the objective misalignment between vector quantization and attention score approximation. 

To address this limitation, \textbf{we propose query-aware vector quantization, a custom vector quantization method that directly optimizes the objective of attention score approximation. }
Specifically, we replace the squared Euclidean distance \(\|\tilde k - \hat k\|^2\) of conventional vector quantization with a query-aware quadratic form \((\tilde k - \hat k) H (\tilde k - \hat k)^\top\) derived from formula (\ref{eq:objective_attention_score_approximation}), where \(H\) represents the second-moment matrix of query states. 
% This custom method formulation explicitly optimizes for attention score approximation accuracy.

Formally, the query-aware vector quantization minimizes the following objective:
\begin{equation}
    \label{eq:query_aware_vq}
    J'(C) = \mathbb E_{\tilde k \sim \mathcal {D^\mathrm{key}}} \left[(\tilde k - \hat k) H (\tilde k - \hat k)^\top\right]
\end{equation}
where \(\hat k = c_{f'(\tilde k; C)}\) denotes the quantized \(\tilde k\), and the corresponding query-aware quantization vector quantization is formulated as:
\begin{equation}
    f'(\tilde k; C) = \operatorname*{argmin}_j (\tilde k - c_j) H (\tilde k - c_j)^\top    
\end{equation}
% Notably, the definition of \(\hat k\) in Equation~\ref{eq:objective_attention_score_approximation}

% \note{For the codebook construction process, we should reformulate the objective function that efficiently supports the conventional vector quantization algorithms like k-means++~\citep{kmeans++}.
% Thus,}

For the codebook construction process, we reformulate the objective function to utilize conventional efficient vector quantization algorithms like k-means++~\citep{kmeans++}.
Specifically, we apply Cholesky decomposition to the positive definite matrix \(H = LL^T\), where \(L \in \mathbb R^{d \times d}\) denotes the Cholesky factor.
% Thus,
%########
%To enable efficient codebook construction with conventional vector quantization algorithms like k-means++~\citep{kmeans++},
% we transform this problem through Cholesky decomposition of the positive definite matrix \(H = LL^\top\), where \(L \in \mathbb R^{d \times d}\) denotes the Cholesky factor. 
%#########
% This allows us to reformulate the objective as:
% \begin{equation}
%     \begin{aligned}
%         J'(C) & = \mathbb E_{\tilde k \sim \mathcal {D^\mathrm{key}}} \left[(\tilde k - \hat k) LL^\top (\tilde k - \hat k)^\top\right] \\
%         & = \mathbb E_{\tilde k \sim \mathcal {D^\mathrm{key}}} \left[(\tilde kL - \hat kL)(\tilde kL - \hat kL)^\top\right]
%     \end{aligned}
% \end{equation}
Let
\begin{equation}
    \label{eq:definition_z}
    z = \tilde k L, \quad C^z = CL, \quad\hat z = \hat k L
\end{equation}
where \(z \in \mathbb R^{1 \times d}\) denotes the transformed key state.
Then, we can re-derive the objective of attention score approximation \(J'\) as:
\begin{equation}
  \label{eq:objective_transform}
    \begin{aligned}
        J'(C) & = \mathbb E_{\tilde k \sim \mathcal {D^\mathrm{key}}} \left[(\tilde k - \hat k)H(\tilde k - \hat k)^\top\right] \\
        & = \mathbb E_{\tilde k \sim \mathcal {D^\mathrm{key}}} \left[(\tilde k - \hat k)LL^\top(\tilde k - \hat k)^\top\right] \\
        & = \mathbb E_{\tilde k \sim \mathcal {D^\mathrm{key}}} \left[(\tilde kL - \hat kL)(\tilde kL - \hat kL)^\top\right] \\
        & = \mathbb E_{z \sim D^z} \left[(z - \hat z)(z - \hat z)^\top\right]
    \end{aligned}
\end{equation}
And the quantization function \(f'\) can be re-derived as:
\begin{equation}
    \begin{aligned}
        f'(\tilde k; C) & = \operatorname*{argmin}_j (\tilde k - c_j) H (\tilde k - c_j)^\top     \\
        & = \operatorname*{argmin}_j (\tilde k L - c_jL)(\tilde k L - c_jL)^\top \\
        & = \operatorname*{argmin}_j (z - c^z_j)(z - c^z_j)^\top \\
        & = f(z; C^z)
    \end{aligned}
\end{equation}
where \(f(z; C^z)\) denotes the quantization function of conventional vector quantization.
Then, we can derive that
\begin{equation}
    \label{eq:quantization_function_transform}
    \begin{aligned}
        \hat z = \hat k L & = c_{f'(\tilde k; C)}L = c^z_{f(z; C^z)} 
    \end{aligned}
\end{equation}

% With 
Equations~\ref{eq:objective_transform},~\ref{eq:quantization_function_transform}
% the objective of attention score approximation \(J'\) simplifies to:
% This transoformation reveals 
reveal that \textbf{the objective of attention score approximation is equivilant to that of conventional vector quantization on transformed \(z\)}.
This alignment enables the application of conventional efficient vector quantization algorithms in codebook construction process.

During the offline pre-processing stage, we collect \(z\) on a representative dataset and construct its codebook \(C^z\) using k-means++~\citep{kmeans++}.
Then, The original shared codebook \(C\) for \(\tilde k\) is calculated as:
\begin{equation}
    C = C^z L^{-1}
\end{equation}
During inference, the codeword index of \(\tilde k\) is computed through query-aware quantization function:
\begin{equation}
    f'(\tilde k; C) = \operatorname*{argmin}_j (\tilde k L - c_jL)(\tilde k L - c_jL)^\top
\end{equation}
Let \(s \in \{1, 2, \dots, L\}^{1 \times n}\) denotes the codeword index vector of all key states after applying query-aware vector quantization, where \(s_j = f'(\tilde k_j; C)\).
Then, the attention score is approximated as:
\begin{equation}
    \hat u_{i,j} = \tilde q_i \hat k_j = \tilde q_i c_{s_j}
\end{equation}

\subsection{Heterogeneous Inference Design}

Although approximating attention scores via vector quantization and then selectively retrieving top-K tokens for computation reduces memory access overhead, the issue of KV Cache occupying substantial GPU memory remains unresolved. 
To reduce the memory footprint of KV Cache and enable larger batch sizes for improved GPU utilization, we design a heterogeneous inference system. 

We partition the decoding process of our proposed {\name} into three components:  

\noindent (1) \textbf{GPU-based model execution}: All model weights reside on the GPU memory. 
Computations involving model weights are executed on the GPU during inference.  

\noindent(2) \textbf{GPU-based approximation of attention scores}: The codebook is stored on the GPU. 
During inference, the GPU first executes the quantization function to assign codewords to key states, then computes attention weight approximations using the codebooks and indices, and lastly gathers the indices of top-K tokens.

\noindent(3) \textbf{CPU-based selective attention}: The full KV Cache is maintained on the CPU memory. 
During decoding, the top-K token indices and the current query state are transferred to the CPU, where selective attention computation is performed to derive the attention output. 
This output is then transferred back to the GPU for subsequent computations.

This design aims to minimize data transfer between CPU and GPU, thereby reducing latency. 
% Compared to alternative approaches that transfer Top-K tokens' KV cache back to the GPU for computation, our solution only requires transferring query states, top-K indices, and attention outputs across devices, thereby reducing latency induced by data transfers. 
Furthermore, to fully leverage the CPU's thread-level parallelism and SIMD capabilities, we implement a custom selective attention kernel optimized for CPU execution.

\section{Conclusion}

In this paper, we introduce STeCa, a novel agent learning framework designed to enhance the performance of LLM agents in long-horizon tasks. 
STeCa identifies deviated actions through step-level reward comparisons and constructs calibration trajectories via reflection. 
These trajectories serve as critical data for reinforced training. Extensive experiments demonstrate that STeCa significantly outperforms baseline methods, with additional analyses underscoring its robust calibration capabilities.


\section*{Limitations}
We study the performance and improvement methods of large models in debt collection negotiations. To simplify the research process and capture key negotiation points, we reduce the debtor’s financial information to variables like assets, average income, and average expenses. However, real-world financial situations are more complex, involving factors like cash flow issues and income fluctuations during repayment. Future work should involve more detailed simulations of debtor information and comparisons with manually simulated debtors. Additionally, due to time constraints, our creditor Multi-agent framework is relatively simple. In practical applications, stricter classification processes in \textit{planning} and more standardized methods in \textit{judging} are needed. We aim to integrate existing decision models to further optimize decision-making in the dialogue.
\section*{Ethical Considerations} \label{EC}
Our study does not disclose any real client information. The acquisition of the source data was subject to strict approval by a major internet financial institution, and the process was continuously supervised by relevant personnel. All debt-related data are processed and replaced with synthetic values, and names are substituted with the pseudonym ``Zhang San''. For the debt reasons extracted from collection dialogues, we strictly anonymize any sensitive details and provide generalized summaries, ensuring that no specific information is involved. Each final data entry underwent rigorous manual verification. Additionally, the methodology proposed in this paper is exploratory and based on simulation for research purposes. Prior to its application in real-world debt collection involving actual individuals, it will undergo more rigorous validation and approval processes.

We conducted annotation tasks in two areas: scoring for the Dialogue Soundness metric (Section~\ref{sec:eval}) and comparison with the human baseline as a debt collector (Section~\ref{sec:res}). Five graduate students with engineering backgrounds and two professionals with financial industry experience participated (They are all from China, as our study focuses on the Chinese language). All annotators involved in our study have signed a disclaimer acknowledging the terms and conditions associated with their participation. They were recruited through campus forums and the internal annotation program of the company. The annotation tasks did not involve any sensitive information and posed no risk. Compensation was provided according to the time spent on each task.
\section*{Acknowledgments}
This work was supported by Ant Group.

\bibliography{latex/custom}


\appendix
\section{Related work}

\textbf{Debt Collection. }Debt collection is a labor-intensive and complex task. Previous research has primarily focused on using machine learning algorithms to identify optimal decisions for individual debtors based on large-scale data  \citep{Sancarlos2023TowardsAD,Jankowski2024DebtCM,Johan2022FinancialTC,Onar2019ADS}. However, these decisions are not made in real time and often require complex decision-making processes and multiple rounds of human negotiation. On the other hand, some automated debt collection dialogue models \citep{Floatbot2023GenerativeAI,Yahiya2024AutomatedDR} can only perform tasks such as information tracking and reminders, without the ability to engage in negotiations for specific goals. Our study aims to enable models to autonomously conduct negotiations and make real-time decisions, which can significantly enhance the efficiency of debt collection.

\textbf{Large Language Models in Negotiation. }In previous studies on large-scale negotiation models (including bargaining \citep{xia2024measuringbargainingabilitiesllms}, repeated games~\citep{akata2023playingrepeatedgameslarge,fu2023improvinglanguagemodelnegotiation} and social decision-making~\citep{10.5555/3618408.3619525}), the goals of the negotiators or gamers were clear, and there were clear methods for measuring the results. Debt collection is an information asymmetry game. Except for loan information, all other information is private information. How to model private information and evaluate the effectiveness of negotiation results are both difficult aspects to consider in modeling.

\textbf{AI Agents. }The memory, planning, reasoning, and communication capabilities of large-scale LLMs offer significant potential for the development of autonomous AI agents (\citealp{autogpt}; \citealp{park2023generative}; \citealp{liang2023encouraging};  \citealp{10.1162/tacl_a_00642}; \citealp{wang2025largelanguagemodelstruly}). Its potential has been demonstrated through the creation of a simulated town~\citep{park2023generative}, populated with independent agents who assume distinct roles and autonomously engage in social interactions.

% (\citealp{autogpt}; \citealp{park2023generative}; \citealp{liang2023encouraging}; \citealp{liang2023leveraging}; \citealp{ai2024cognition}; \citealp{10.1162/tacl_a_00642})
% \citep{park2023generative}



\section{Detailed descriptions of four Negotiation Dimensions}\label{sec:dim}

The following sections provide detailed descriptions of the four key negotiation dimensions involved in debt collection, outlining how each aspect influences the negotiation process and repayment outcomes. And table~\ref{negdim} shows all dimensions of the negotiation.

\begin{itemize}[leftmargin=15px]
    \item \textbf{Debt Reduction Ratio:} This refers to the portion of the debt that can be waived by the creditor to ease the debtor’s repayment burden. The reduction ratio is often negotiable based on the debtor’s financial situation, with creditors typically offering reductions as an incentive to settle the debt more efficiently.

    \item \textbf{Immediate Repayment Ratio:} In order to temporarily restore the debtor’s credit and advance the repayment process, creditors usually require the debtor to repay a portion of the outstanding debt immediately during the negotiation. This portion is typically at least 5\% of the total debt.

    \item \textbf{Immediate Repayment Time:} If the debtor is unable to make an immediate payment on the same day, a grace period of up to 14 days may be granted. Within this period, the debtor is expected to raise the necessary funds to complete the immediate repayment.

    \item \textbf{Installment Period:} After addressing part of the debt through reductions and immediate repayments, the remaining balance can be settled through installments. The installment ratio can vary from 3 to 24 periods, allowing the debtor to repay the debt within a period ranging from a few months to up to two years.
\end{itemize}

\begin{table*}[ht]
\centering
\caption{\label{negdim}Negotiation Dimensions and Their Possible Values}
\begin{tabular}{ll}
\toprule
\textbf{Dimension} & \textbf{Values} \\
\midrule
Discount Ratio ('disc\_ratio') & 5\%, 10\%, 15\%, 20\%, 25\%, 30\% \\
Immediate Payment Ratio ('pmt\_ratio') & 5\%, 10\%, 15\%, 20\%, 25\%, 30\%, 35\%, 40\%, 45\%, 50\% \\
Immediate Payment Time ('pmt\_days') & 1, 2, 3, 4, 5, 6, 7, 8, 9, 10, 11, 12, 13, 14 days \\
Installment Periods ('inst\_prds') & 3, 6, 9, 12, 18, 24 months \\
\bottomrule
\end{tabular}
\label{tab:negotiation_dimensions}
\end{table*}

\section{Data Distribution} \label{Distribution}

As shown in the Figure~\ref{img:distri}, our dataset exhibits a certain distribution across Amount, Sex, and Overdue Days, which is similar to the actual situation. The distributions in both the test set and the train set are also largely consistent.

\begin{figure*}[htbp]
  \centering
  \includegraphics[width=1.03\textwidth]{latex/images/contri.pdf}  
  \vspace{-0.2in}
  \caption{Distribution of Need collected Amount, Sex, and Overdue Days.}
\vspace{-0.0in}
\label{img:distri}
\end{figure*}

\section{Difficulty Tiers for Debt Collection} \label{app:diff_cat}

In the field of debt management and collection, the economic hardship level may be related to the debtor’s repayment capacity assessment~\citep{Zwilling2017EvaluatingYC}. Referring to common methods for determining economic hardship levels~\citep{elsevier2001international}, we categorize debtors into five tiers as shown in Table~\ref{img:category}.

\begin{table}[ht]
\centering
\vspace{-0.1in}
\caption{\label{img:category}Difficulty Tiers for Debt Collection}

\setlength{\tabcolsep}{3.5mm}{
\resizebox{0.38\textwidth}{!}{%
\begin{tabular}{lcc}
    \toprule
    \textbf{Tier} & \textbf{Description} & \textbf{Range} \\
    \midrule
    Tier 1 & Extremely Difficult & 0 - 2000 \\
    Tier 2 & Very Difficult & 2000 - 5000 \\
    Tier 3 & Moderately Difficult & 5000 - 10000 \\
    Tier 4 & Slightly Difficult & 10000 - 20000 \\
    Tier 5 & No Difficulty & 20000+ \\
    \bottomrule
\end{tabular}%
}}
\vspace{-0.1in}
\end{table}

\section{Definitions of variables in DCN process}

Table~\ref{tab:debt_variable_app} provides the descriptions of all the variables appearing in Algorithm 1. The Action Set includes \texttt{ask}, \texttt{reject}, and \texttt{accept}, while the Negotiation Dimension Set consists of the four quantities listed in Table~\ref{tab:negotiation_dimensions}.


\begin{table}[ht]
\centering
\Large
\caption{Definitions of variables in DCN process.}
\resizebox{0.7\columnwidth}{!}{%
\begin{tabular}{@{}ll@{}}
\toprule
\textbf{Conception} & \textbf{Variable} \\ \midrule
Basic Information & $I_b$ \\
Creditor & creditor \\
Action Set & $S_A$ \\
Result Dictionary & $D$ \\
Personal Financial Information & $I_p$ \\
Debtor & debtor \\
Negotiation Dimension Set & $S_R$ \\
Turn & $t$ \\
Max Turns & $t_m$ \\
Agent Creditor & Creditor \\
Agent Debtor & Debtor \\
Action of Debtor & $A_d$ \\
Action of Creditor & $A_c$ \\ 
\bottomrule
\end{tabular}%
}

\label{tab:debt_variable_app}
\end{table}

\section{Detail of metrics} \label{app:metric}

\subsection{Conversational Ability} \label{app:me_conv}

In negotiation processes, conversational ability is crucial for achieving effective communication and mutual understanding. \citet{tu2024characterevalchinesebenchmarkroleplaying} proposed an evaluation framework for role-playing tasks. Inspired by this work, we tailored it to our task by distinguishing Conversational Ability into two dimensions: fluency and completeness.

\textbf{Dialogue Soundness (DS).} Dialogue Soundness is a single-metric evaluation that measures a dialogue response’s fluency, naturalness, coherency, and consistency on a five-point scale. It assesses whether the response is grammatically correct and conversational, stays on topic, and remains logically consistent across turns. This metric is manually scored, with the scale shown in Table ~\ref{DS_score}. Five graduate students from engineering disciplines were employed to evaluate this metric and calculated the average value.

\begin{table*}[h]
    \centering
    \caption{\label{DS_score}Dialogue Soundness (DS) Rating Scale}
    \begin{tabular}{c l l}
        \toprule
        \textbf{Score} & \textbf{Rating} & \textbf{Description} \\
        \midrule
        5 & Excellent  & Fluent, natural, on-topic, logically consistent. \\
        4 & Good       & Mostly natural, minor topic drift, slight inconsistency. \\
        3 & Acceptable & Understandable but somewhat rigid, occasional drift or inconsistency. \\
        2 & Poor       & Unnatural phrasing, noticeable topic deviation or contradictions. \\
        1 & Unacceptable & Robotic, off-topic, illogical contradictions. \\
        \bottomrule
    \end{tabular}
\end{table*}

\textbf{Dialogue Completeness (DC).} Dialogue Completeness is a metric designed to evaluate whether a conversation addresses all specified objectives outlined in section~\ref{obj} of the paper. This automated measure checks if each of the four key goals has been adequately discussed during the dialogue, ensuring that no critical topics are overlooked or omitted.

\subsection{Debt Recovery} \label{app:me_rec}

\textbf{Success Recovery Rate (SR). }The success rate of the negotiation is determined by whether the debtor's future assets remain in a healthy state (i.e., the total personal assets remain greater than 500). The success rate is defined as the proportion of samples in which repayment can theoretically be completed successfully:
\begin{equation}\label{eq:utility}
\begin{split}
\text{SR} = \frac{N_{\text{success}}}{N},
\end{split}
\end{equation}
where SR is the success rate, $N_{\text{success}}$ is the number of successful samples, and \(N\) is the total number of samples.

\textbf{Recovery Rate (RR). }The recovery ratio refers to the portion of the debt recovered by the creditor, which is typically $1$ minus the reduction ratio. If the plan is unsuccessful, the recovery ratio is considered to be $0$. The final recovery ratio is calculated as the mean recovery ratio across the test samples:

\begin{equation}
\begin{split}
\text{RR} = \frac{1}{N} \sum_{i=1}^{N} r_i,
\end{split}
\end{equation}
where RR is the final recovery ratio, \(r_i\) is the recovery ratio of the \(i\)-th sample.


\subsection{Collection Efficiency} \label{app:me_col}

\textbf{25\% Recovery Date (QRD)} refers to the date at which the debtor has completed 25\% of the debt repayment, which is estimated based on the debtor's future economic condition sequence. The final 25\% Recovery Date is calculated as the mean of the recovery dates across the test samples:
\begin{equation}
\begin{split}
\text{QRD} = \frac{1}{N} \sum_{i=1}^{N} t_{25\%,i},
\end{split}
\end{equation}
where QRD is the final 25\% recovery date, $t_{25\%,i}$ is the 25\% recovery date of the $i$-th sample, and $N$ is the total number of samples.

\textbf{50\% Recovery Date (HRD)} is defined similarly to the 25\% Recovery Date, referring to the date at which the debtor has completed 50\% of the debt repayment, based on the debtor's future economic condition sequence. 
\textbf{Completion Date (CD)} refers to the date at which the debtor has fully repaid all of the debt.

The 50\% Recovery Date and Completion Date are calculated as the means of the respective recovery dates across the test samples:
\begin{equation}
\begin{split}
\text{HRD} = \frac{1}{N} \sum_{i=1}^{N} t_{50\%,i},
\end{split}
\end{equation}
where HRD is the final 50\% recovery date, and $t_{50\%,i}$ is the 50\% recovery date of the $i$-th sample.
\begin{equation}
\begin{split}
\text{CD} = \frac{1}{N} \sum_{i=1}^{N} t_{\text{Completion},i},
\end{split}
\end{equation}
where CD is the completion date, and $t_{\text{Completion},i}$ is the completion date of the $i$-th sample.


\subsection{Debtor’s Financial Health} \label{app:me_hea}

\textbf{L1 Tier Days (L1D)} refers to the number of days the debtor remains in the most difficult tier over the next two years. \textbf{L2 Tier Days (L2D)} refers to the number of days the debtor remains in the second most difficult tier during the same period. These two indicators directly correspond to the duration the debtor spends in different levels of financial difficulty. Research has shown that the longer the debtor remains in a higher level of difficulty, the more likely they are to default on the loan 
~\citep{Tabacchi2016DeterminantsOE}.

\textbf{Asset tier variance (ATV).} In addition to controlling for the number of days the debtor remains in the high-poverty tier, the overall stability of the debtor's asset level also ensures a higher repayment performance. To capture this, we introduce the asset tier variance metric, which is calculated by computing the variance of the debtor's asset tier over the course of one year. The final result is obtained by calculating the mean of the asset tier variances across the test samples:
\begin{equation}
\begin{split}
v_{\text{asset},i} = \frac{1}{T-1} \sum_{t=1}^{T} \left( A_{i,t} - \bar{A}_i \right)^2,
\end{split}
\end{equation}
where $A_{i,t}$ is the asset tier of the $i$-th sample at time $t$, $\bar{A}_i$ is the average asset tier of the $i$-th sample over the year, and $T$ is the total number of time periods. The final asset tier variance is the mean of the asset tier variances across the test samples:
\begin{equation}
\begin{split}
\text{ATV} = \frac{1}{N} \sum_{i=1}^{N} v_{\text{asset},i},
\end{split}
\end{equation}
where ATV is the mean asset tier variance, and $N$ is the total number of samples.

\subsection{Average Metric} \label{app:me_ave}

In debt collection, the indicators for Debt Recovery and Collection Efficiency are often inversely related to the Debtor’s Financial Health. This means that efforts to recover debts more efficiently and quickly may negatively impact the debtor's financial condition. To strike a balance between these two conflicting objectives, we introduce three average metrics that help quantify the trade-off: the Creditor’s Recovery Index (CRI), the Debtor’s Health Index (DHI), and the Comprehensive Collection Index (CCI).

\textbf{Creditor’s Recovery Index (CRI):} This index measures the effectiveness of the creditor's recovery strategy while accounting for the impact on the debtor’s financial health. The index aggregates several recovery metrics weighted by their relative importance to the creditor's objectives. The index is calculated as follows:
\begin{equation}
\begin{split}
\text{CRI} = &\ w_1 \cdot \text{SR} + w_2 \cdot \text{RR} \\
& + w_3 \cdot \frac{\text{max}(\text{QRD}) - \text{QRD}}{\text{max}(\text{QRD})} \\
& + w_4 \cdot \frac{\text{max}(\text{HRD}) - \text{HRD}}{\text{max}(\text{HRD})} \\
& + w_5 \cdot \frac{\text{max}(\text{CD}) - \text{CD}}{\text{max}(\text{CD})},
\end{split}
\end{equation}
where \(w_1, w_2, w_3, w_4, w_5\) are the weights assigned to each metric based on the creditor’s priorities.

\textbf{Debtor’s Health Index (DHI):} This index measures the debtor's financial health during the recovery process. It incorporates several factors that capture the debtor's stability and vulnerability. The Debtor’s Health Index is calculated as:
\begin{equation}
\begin{split}
\text{DHI} = &\ w_6 \cdot \frac{\text{max}(\text{L1D}) - \text{L1D}}{\text{max}(\text{L1D})} \\
& + w_7 \cdot \frac{\text{max}(\text{L2D}) - \text{L2D}}{\text{max}(\text{L2D})} \\
& - w_8 \cdot \text{ATV}.
\end{split}
\end{equation}
Here, \(w_6, w_7, w_8\) are weights that balance the importance of each factor in determining the debtor’s health.

\textbf{Comprehensive Collection Index (CCI):} The Comprehensive Collection Index combines both the Creditor’s Recovery Index (CRI) and the Debtor’s Health Index (DHI) into a single metric that evaluates the overall balance between debt recovery and the debtor’s financial well-being. The index is calculated using the harmonic mean of the two indices, with a weight factor $\theta$ applied to the CRI:
\begin{equation}
\text{CCI} = \frac{2 \theta^2 \cdot \text{CRI} \cdot \text{DHI}}{\text{CRI} + \theta^2\cdot\text{DHI}}.
\end{equation}
In this formula, the weight factor $\theta$ indicates that the CRI is weighted $\theta$ times more than the DHI. In this study, $\theta$ is set to 2. This approach ensures that a high value in either the recovery index or the health index will influence the overall result, while emphasizing the importance of balancing both aspects.

The use of this weighted harmonic mean helps in evaluating different debt recovery strategies by considering both the creditor’s objectives and the debtor’s financial stability, thereby promoting a more balanced approach to debt collection.

The constant values used in the calculation process are shown in Table~\ref{tab:metrics_parameters}. In future research or application, these values may be adjusted depending on the specific requirements to better align with the needs.
\begin{table}[ht]
\centering
% \Large
\caption{Constants used in Average Metric Calculation.}
\resizebox{0.4\columnwidth}{!}{%
\begin{tabular}{@{}ll@{}}
\toprule
\textbf{Constant} & \textbf{Value} \\ \midrule
$w_1$ & 0.25 \\
$w_2$ & 0.25 \\
$w_3$ & 0.2 \\
$w_4$ & 0.15 \\
$w_5$ & 0.15 \\
$w_6$ & 1.5 \\
$w_7$ & 0.8 \\
$w_8$ & 1 \\
$\theta$ & 2 \\
$\text{max}(\text{QRD})$ & 180 \\
$\text{max}(\text{HRD})$ & 360 \\
$\text{max}(\text{CD})$ & 720 \\
$\text{max}(\text{L1D})$ & 30 \\
$\text{max}(\text{L2D})$ & 250 \\ 
\bottomrule
\end{tabular}%
}
\label{tab:metrics_parameters}
\end{table}

\section{All LLMs in our Experiments} \label{app:models}

We comprehensively evaluate nine LLMs, encompassing both API-based models and open-source models. The API-based models include the GPT series (GPT-4o, GPT-4o-mini, o1-mini) \citep{GPT-4, openai20254o,openai2025o1}, Claude-3.5 \citep{anthropic2025}, MiniMax (abab6.5s-chat) \citep{minimaxi2025}, Sensechat \citep{sensetime2025}, DeepSeek series (DeepSeek-R1 and DeepSeek-V3) \citep{deepseekai2025deepseekr1incentivizingreasoningcapability,deepseekai2024deepseekv3technicalreport} and Doubao \citep{doubao2025}. The open-source models include the Llama series (LlaMA-2-13B-Chat, LlaMA-3-8B-Instruct, LlaMA-3-70B-Instruct) \citep{LLaMA} and the Qwen-2.5 series (Qwen-2.5-7B, Qwen-2.5-14B and Qwen-2.5-72B) \citep{qwen2025qwen25technicalreport}. These models are run using vLLM~\citep{kwon2023efficient} on eight Nvidia A100 GPUs with the same random seed. For each model, the entire test set was processed in approximately one hour using parallel methods. All temperatures are set to 0 (Due to API-provider's closed-source non-deterministic implementation, small changes may still occur in the reproduction process). Specific model hyperparameters and version details can be found in Table~\ref{tab:model-hyperparams}. All models and tools (vLLM and LLaMa-Factory~\citep{zheng2024llamafactory}) used in this study, including closed-source API-based models, open-source models, were used in compliance with their respective licenses. What's more, the use of these generative models for dialogue tasks is well-established in the field and follows standard practices.

\begin{table*}[h!]
\centering
\caption{\textcolor{black}{Hyperparameters of Each Model.}}
\label{tab:model-hyperparams}
\textcolor{black}{
\resizebox{1\textwidth}{!}{%
\begin{tabular}{lll}
\hline
\textcolor{black}{\textbf{Model Name}} & \textcolor{black}{\textbf{Parameters}} & \textcolor{black}{\textbf{Comments}} \\ 
\hline
\textcolor{black}{Qwen-2.5-7B} & \textcolor{black}{"temperature": 0, "max\_tokens": 1024} & \textcolor{black}{version = "qwen-2.5-7b-instruct"} \\
\textcolor{black}{Qwen-2.5-14B} & \textcolor{black}{"temperature": 0, "max\_tokens": 1024} & \textcolor{black}{version = "qwen-2.5-14b-instruct"} \\
\textcolor{black}{Qwen-2.5-72B} & \textcolor{black}{"temperature": 0, "max\_tokens": 1024} & \textcolor{black}{version = "qwen-2.5-72b-instruct"} \\
\textcolor{black}{GPT-4o} & \textcolor{black}{"temperature": 0, "max\_tokens": 1024} & \textcolor{black}{version = "gpt-4o-2024-11-20"} \\ 
\textcolor{black}{GPT-4o Mini} & \textcolor{black}{"temperature": 0, "max\_tokens": 1024} & \textcolor{black}{version = "gpt-4o-mini"} \\ 
\textcolor{black}{o1-Mini} & \textcolor{black}{"temperature": 0, "max\_tokens": 1024} & \textcolor{black}{version = "o1-mini"} \\ 
\textcolor{black}{LLaMa-3-8B} & \textcolor{black}{"temperature": 0, "max\_tokens": 1024} & \textcolor{black}{version = "llama-3-8b-instruct"} \\ 
\textcolor{black}{LLaMa-3-70B} & \textcolor{black}{"temperature": 0, "max\_tokens": 1024} & \textcolor{black}{version = "llama-3-70b-instruct"} \\ 
\textcolor{black}{Doubao} & \textcolor{black}{"temperature": 0, "max\_tokens": 1024} & \textcolor{black}{version = "Doubao-pro-4k"} \\ 
\textcolor{black}{Claude-3.5} & \textcolor{black}{"temperature": 0, "max\_tokens": 1024} & \textcolor{black}{version = "claude-3-5-sonnet-20241022"} \\ 
\textcolor{black}{DeepSeek-V3} & \textcolor{black}{"temperature": 0, "max\_tokens": 1024} & \textcolor{black}{version = "deepseek-chat"} \\ 
\textcolor{black}{DeepSeek-R1} & \textcolor{black}{"temperature": 0, "max\_tokens": 1024} & \textcolor{black}{version = "deepseek-reasoner"} \\ 
\textcolor{black}{MiniMax} & \textcolor{black}{"temperature": 0, "max\_tokens": 1024} & \textcolor{black}{version = "abab6.5s-chat"} \\ 
\textcolor{black}{SenseChat} & \textcolor{black}{"temperature": 0, "max\_tokens": 1024} & \textcolor{black}{version = "SenseChat"} \\ 
\hline
\end{tabular}
}}
\end{table*}

\section{Prompts} 

\subsection{Basic Prompts for Role-playing Debtor and Creditor.} \label{app:prompts}

Figures~\ref{img:deb_prompt} and~\ref{img:cre_prompt} illustrate the prompts given to the large model to act as the debtor and the creditor, respectively. Originally in Chinese, these prompts have been appropriately simplified and automatically translated into English for display purposes (the full Chinese prompts is available to be disclosed later). Additionally, the instructions provided to human annotators were consistent with the prompts given to the model.

\begin{figure*}[htbp]
  \centering
  \includegraphics[width=1\textwidth]{latex/images/deb_prompt.pdf}  
  \caption{Prompt of Debtor.}
\vspace{-0.0in}
\label{img:deb_prompt}
\end{figure*}

\begin{figure*}[htbp]
  \centering
  \includegraphics[width=1\textwidth]{latex/images/cre_prompt.pdf}  
  \caption{Prompt of Creditor (Debt Collector).}
\vspace{-0.0in}
\label{img:cre_prompt}
\end{figure*}




\subsection{Prompts for Planning Agent and Judging Agent.} \label{app:agent_promopt}

Figures~\ref{img:plan_prompt} and~\ref{img:judge_prompt} display the prompts for the planning agent and judging agent in the MADaN framework. Similarly, these prompts have been simplified and translated for ease of presentation. The prompt for the communicating agent remains unchanged, as previously shown.


\begin{figure*}[htbp]
  \centering
  \includegraphics[width=1\textwidth]{latex/images/plan_prompt.pdf}  
  \caption{Prompt of Planning Agent.}
\vspace{-0.0in}
\label{img:plan_prompt}
\end{figure*}

\begin{figure*}[htbp]
  \centering
  \includegraphics[width=1\textwidth]{latex/images/judge_prompt.pdf}  
  \caption{Prompt of Judging Agent.}
\vspace{-0.0in}
\label{img:judge_prompt}
\end{figure*}


\subsection{Defective prompt} \label{app:deprompts}

There are three main methods for generating Defective Prompts, as shown in Table~\ref{deprompt}. In practice, we first generate a list of prompts and then randomly select one from the list to generate the negative samples.

\begin{table*}[ht]
\centering
\caption{\label{deprompt}Defective Prompt Modifications for Debt Collection Negotiation.}
    \setlength{\tabcolsep}{3.5mm}{
    \resizebox{\textwidth}{!}{%
        \begin{tabular}{lll}
        \toprule
        \textbf{Modification Type} & \textbf{Description} & \textbf{Example} \\
        \midrule
        Deletion & Remove specific instructions & Removing "Offer a 10\% discount when the debtor shows clear financial difficulty." \\
        Replacement & Reverse guidance & Changing "Be cautious when the debtor makes a request" to "Approve requests without further consideration." \\
        Addition & Add negative guidance & Adding "If installment terms are discussed, set them to 24 months without negotiation." \\
        \bottomrule
        \end{tabular}
    }}
\end{table*}

% \textcolor{black}{GPT-4-turbo} & \textcolor{black}{"temperature": 0, "max\_tokens": 1024} & \textcolor{black}{version = "GPT-4-turbo"} \\ 
% \textcolor{black}{GPT-3.5-turbo} & \textcolor{black}{"temperature": 0, "max\_tokens": 1024} & \textcolor{black}{version = "gpt-3.5-turbo-0125"} \\ 
% \textcolor{black}{Qwen1.5-110B} & \textcolor{black}{"temperature": 0, "max\_tokens": 1024} & \textcolor{black}{version = "qwen1.5-110b-chat"} \\ 
% \textcolor{black}{QwenMax} & \textcolor{black}{"temperature": 0, "max\_tokens": 1024} & \textcolor{black}{version = "qwen-max"} \\ 
% \textcolor{black}{Claude-3-Opus} & \textcolor{black}{"temperature": 0, "max\_tokens": 1024} & \textcolor{black}{version = "claude-3-opus-20240229"} \\ 
% \textcolor{black}{LLaMA2-13B-Chat} & \textcolor{black}{"temperature": 0, "max\_tokens": 1024} & \textcolor{black}{model = "Llama-2-13b-chat"} \\ 
% \textcolor{black}{LLaMA3-70B-Instruct} & \textcolor{black}{"temperature": 0, "max\_tokens": 1024} & \textcolor{black}{model = "Llama-3-70B-Instruct"} \\ 
% \textcolor{black}{LLaMA3-8B-Instruct} & \textcolor{black}{"temperature": 0, "max\_tokens": 1024} & \textcolor{black}{model = "Llama-3-8B-Instruct"} \\ 
% \textcolor{black}{Qwen2-7B-Instruct} & \textcolor{black}{"temperature": 0, "max\_tokens": 1024} & \textcolor{black}{model = "Qwen2-7B-Instruct"} \\ 
% \hline
% \textcolor{black}{LLaMA3-8B-Base-FT} & \textcolor{black}{"temperature": 0, "max\_tokens": 1024, train\_batch\_size: 4,"finetuning\_type": lora, } & \textcolor{black}{model = "Llama-3-8B"} \\ 
% & "learning\_rate": 1.0e-4, "num\_train\_epochs": 10.0, "bf16": true & \\
% \textcolor{black}{LLaMA3-8B-Instruct-FT} & \textcolor{black}{"temperature": 0, "max\_tokens": 1024,"train\_batch\_size": 4,"finetuning\_type": lora,} & \textcolor{black}{model = "Llama-3-8B-Instruct"} \\ 
% &  "learning\_rate": 1.0e-4, "num\_train\_epochs": 10.0, "bf16": true & \\
% \textcolor{black}{Qwen2-7B-Instruct-FT} & \textcolor{black}{"temperature": 0, "max\_tokens": 1024,"train\_batch\_size": 4,"finetuning\_type": lora, } & \textcolor{black}{model = "Qwen2-7B-Instruct"} \\ 
% &  "learning\_rate": 1.0e-4, "num\_train\_epochs": 10.0, "bf16": true &\\
% \textcolor{black}{Qwen2-7B-Base-FT} & \textcolor{black}{"temperature": 0, "max\_tokens": 1024,"train\_batch\_size": 4,"finetuning\_type": lora, } & \textcolor{black}{model = "Qwen2-7B"} \\ 
% &  "learning\_rate": 1.0e-4, "num\_train\_epochs": 10.0, "bf16": true & \\
% \hline


\section{The performance of different models as the debtor}\label{sec:model_deb}


In Section~\ref{sec:res}, we evaluate the debt collection outcomes when different models act as the creditor. We alse examine the performance of different models as debtors, using the Qwen-2.5-72B model exclusively as the creditor. We observed significant differences in the results when using different models for the debtor as shown in Table~\ref{img:reverseresult}. The SenseChat and Llama-3-70b models exhibited some inconsistencies, yielding excessively high DHI scores. During the examination of the dialogue process, we found that these models tended to neglect \textit{repeated statements} within the dialogue, leading to the inclusion of some irrelevant or ineffective content. Additionally, some models were more sensitive to the debtor’s prompt, likely due to the more complex nature of the debtor agent’s objectives. In contrast, the Qwen-2.5-72 model showed relatively balanced performance, suggesting that our choice was appropriate.

Since our focus is on studying the model’s performance as a debt collector, we did not design specific metrics for debtor models. Our primary aim is to use models capable of understanding the debtor’s objectives and engaging in dialogue for simulations prior to further manual testing.

\begin{table*}[ht]
\vspace{-0.1in}
    \centering
    \caption{\label{img:reverseresult}The performances of some models as Debtors.}
    \vspace{-0.1in}
    \setlength{\tabcolsep}{3.5mm}{
    \resizebox{\textwidth}{!}{%
    \begin{tabular}{lcccccccccccc}
        \toprule
        Model  & SR & RR & QRD & HRD & CD & L1D & L2D & ATV & CRI & DHI & CCI \\
        \midrule
        Qwen-2.5-72B  & 0.98 & 0.88 & 36.98 & 185.18 & 404.98 & 3.76 & 78.84 & 0.83 & 0.76 & 0.76 & 0.76\\
        llama-3-8b  & 1.00 & 0.94 & 29.13 & 134.13 & 296.25 & 3.25 & 80.31 & 0.91 & 0.83 & 0.71 & 0.81 \\
        llama-3-70b  & 1.00 & 0.92 & 10.33 & 150.33 & 369.33 & 0.33 & 51.33 & 0.85 & 0.83 & 0.97 & 0.85 \\
        gpt-4o-2024-11-20  & 1.00 & 0.94 & 35.26 & 146.86 & 312.66 & 3.50 & 85.72 & 0.86 & 0.82 & 0.73 & 0.80 \\
        o1-mini  & 0.98 & 0.93 & 26.76 & 111.96 & 240.56 & 4.92 & 92.34 & 0.93 & 0.85 & 0.58 & 0.78 \\
        deepseek-chat & 0.97 & 0.93 & 32.42 & 125.32 & 269.48 & 3.74 & 93.00 & 0.90 & 0.83 & 0.66 & 0.79 \\
        Doubao-pro-4k & 1.00 & 0.83 & 75.28 & 190.48 & 324.72 & 2.16 & 80.34 & 0.84 & 0.73 & 0.82 & 0.74 \\
        abab6.5s-chat & 0.90 & 0.92 & 58.53 & 204.53 & 484.53 & 8.63 & 76.50 & 0.89 & 0.70 & 0.52 & 0.66 \\
        SenseChat & 1.00 & 0.96 & 135.0 & 345.00 & 734.00 & 0.70 & 51.00 & 0.88 & 0.54 & 0.96 & 0.60 \\
        \bottomrule
    \end{tabular}%
    }}
 \label{tab:mainresults}
     \vspace{-10pt}
\end{table*}

% \begin{table*}[ht]
%     \centering
%     \caption{\label{img:mainresult_detial}The performances of some models}
%     \vspace{-0.1in}
%     \setlength{\tabcolsep}{3.5mm}{
%     \resizebox{\textwidth}{!}{%
%     \begin{tabular}{lccccccccc}
%         \toprule
%          & \multicolumn{2}{c}{\textbf{Debt Recovery}} & \multicolumn{3}{c}{\textbf{Collection Efficiency}} & \multicolumn{3}{c}{\textbf{Debtor’s Financial Health}} \\
%         \cmidrule(lr){2-3} \cmidrule(lr){4-6} \cmidrule(lr){7-9}
%         \textbf{Model} & \textbf{SR(\%)} & \textbf{RR} & \textbf{QRD} & \textbf{HRD} & \textbf{CD} & \textbf{L1D}& \textbf{L2D}& \textbf{ATV} \\
%         \midrule
%         Vanilla & 90 & 84.00 & 34.35 & 178.35 & 397.85 & \textbf{4.10} & 73.50 & \textbf{0.83}\\
%         + Planning& 80 & 94.38 & 24.35 & 142.85 & 398.35 & 11.80 & \textbf{72.75} & 0.96\\
%         + Judging & 95 & \textbf{97.62} & 19.60 & 124.60 & 327.60 & 6.15 & 76.75 & 0.86\\
%         Ours  & \textbf{95} & 97.40 & \textbf{18.00} & \textbf{124.50} & \textbf{300.5} & 4.45 & 79.85 & 0.86\\
%         \bottomrule
%     \end{tabular}%
%  }}
%  \label{tab:mainresults}
%      % \vspace{-10pt}
% \end{table*}


\section{Settings of Post-training}

All post-training experiments were conducted on an 8-GPU A100 server using the LLaMa-Factory framework~\citep{zheng2024llamafactory}. The training time per session was around five minutes. The specific parameter settings for each group are provided in Table~\ref{tab:model-hyperparams-post}. The four sets of training data will be made publicly available at a later stage.



\begin{table*}[h!]
\centering
\caption{\textcolor{black}{Hyperparameters of Each Post-trained Model.}}
\label{tab:model-hyperparams-post}
\textcolor{black}{
\resizebox{1\textwidth}{!}{%
\begin{tabular}{lll}
\hline
\textcolor{black}{\textbf{Model Name}} & \textcolor{black}{\textbf{Parameters}} & \textcolor{black}{\textbf{Comments}} \\ 
\hline
\textcolor{black}{SFT-DG} & \textcolor{black}{"temperature": 0, "max\_tokens": 1024, train\_batch\_size: 4,"finetuning\_type": lora, } & \textcolor{black}{model = "qwen-2.5-7b-instruct"} \\ 
& "learning\_rate": 5.0e-6, "num\_train\_epochs": 5.0, "bf16": true & \\
\textcolor{black}{SFT-MAG} & \textcolor{black}{"temperature": 0, "max\_tokens": 1024,"train\_batch\_size": 4,"finetuning\_type": lora,} & \textcolor{black}{model = "qwen-2.5-7b-instruct"} \\ 
&  "learning\_rate": 5.0e-6, "num\_train\_epochs": 5.0, "bf16": true & \\
\textcolor{black}{DPO-DG} & \textcolor{black}{"temperature": 0, "max\_tokens": 1024,"train\_batch\_size": 4,"finetuning\_type": lora, } & \textcolor{black}{model = "qwen-2.5-7b-instruct"} \\ 
&  "learning\_rate": 5.0e-6, "num\_train\_epochs": 5.0, "bf16": true &\\
\textcolor{black}{DPO-MAG} & \textcolor{black}{"temperature": 0, "max\_tokens": 1024,"train\_batch\_size": 4,"finetuning\_type": lora, } & \textcolor{black}{model = "qwen-2.5-7b-instruct"} \\ 
&  "learning\_rate": 5.0e-6, "num\_train\_epochs": 5.0, "bf16": true & \\
\hline
\end{tabular}
}}
\end{table*}


\section{Supplementary Information}
This paper utilized AI tools including Google Translate for assisted translation when presenting prompts and examples, and employed the use of a Cursor for coding to enhance efficiency. No potential risks were involved in the course of this study.
\end{document}
