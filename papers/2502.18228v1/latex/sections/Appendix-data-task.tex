\section{Detailed descriptions of four Negotiation Dimensions}\label{sec:dim}

The following sections provide detailed descriptions of the four key negotiation dimensions involved in debt collection, outlining how each aspect influences the negotiation process and repayment outcomes. And table~\ref{negdim} shows all dimensions of the negotiation.

\begin{itemize}[leftmargin=15px]
    \item \textbf{Debt Reduction Ratio:} This refers to the portion of the debt that can be waived by the creditor to ease the debtor’s repayment burden. The reduction ratio is often negotiable based on the debtor’s financial situation, with creditors typically offering reductions as an incentive to settle the debt more efficiently.

    \item \textbf{Immediate Repayment Ratio:} In order to temporarily restore the debtor’s credit and advance the repayment process, creditors usually require the debtor to repay a portion of the outstanding debt immediately during the negotiation. This portion is typically at least 5\% of the total debt.

    \item \textbf{Immediate Repayment Time:} If the debtor is unable to make an immediate payment on the same day, a grace period of up to 14 days may be granted. Within this period, the debtor is expected to raise the necessary funds to complete the immediate repayment.

    \item \textbf{Installment Period:} After addressing part of the debt through reductions and immediate repayments, the remaining balance can be settled through installments. The installment ratio can vary from 3 to 24 periods, allowing the debtor to repay the debt within a period ranging from a few months to up to two years.
\end{itemize}

\begin{table*}[ht]
\centering
\caption{\label{negdim}Negotiation Dimensions and Their Possible Values}
\begin{tabular}{ll}
\toprule
\textbf{Dimension} & \textbf{Values} \\
\midrule
Discount Ratio ('disc\_ratio') & 5\%, 10\%, 15\%, 20\%, 25\%, 30\% \\
Immediate Payment Ratio ('pmt\_ratio') & 5\%, 10\%, 15\%, 20\%, 25\%, 30\%, 35\%, 40\%, 45\%, 50\% \\
Immediate Payment Time ('pmt\_days') & 1, 2, 3, 4, 5, 6, 7, 8, 9, 10, 11, 12, 13, 14 days \\
Installment Periods ('inst\_prds') & 3, 6, 9, 12, 18, 24 months \\
\bottomrule
\end{tabular}
\label{tab:negotiation_dimensions}
\end{table*}

\section{Data Distribution} \label{Distribution}

As shown in the Figure~\ref{img:distri}, our dataset exhibits a certain distribution across Amount, Sex, and Overdue Days, which is similar to the actual situation. The distributions in both the test set and the train set are also largely consistent.

\begin{figure*}[htbp]
  \centering
  \includegraphics[width=1.03\textwidth]{latex/images/contri.pdf}  
  \vspace{-0.2in}
  \caption{Distribution of Need collected Amount, Sex, and Overdue Days.}
\vspace{-0.0in}
\label{img:distri}
\end{figure*}

\section{Difficulty Tiers for Debt Collection} \label{app:diff_cat}

In the field of debt management and collection, the economic hardship level may be related to the debtor’s repayment capacity assessment~\citep{Zwilling2017EvaluatingYC}. Referring to common methods for determining economic hardship levels~\citep{elsevier2001international}, we categorize debtors into five tiers as shown in Table~\ref{img:category}.

\begin{table}[ht]
\centering
\vspace{-0.1in}
\caption{\label{img:category}Difficulty Tiers for Debt Collection}

\setlength{\tabcolsep}{3.5mm}{
\resizebox{0.38\textwidth}{!}{%
\begin{tabular}{lcc}
    \toprule
    \textbf{Tier} & \textbf{Description} & \textbf{Range} \\
    \midrule
    Tier 1 & Extremely Difficult & 0 - 2000 \\
    Tier 2 & Very Difficult & 2000 - 5000 \\
    Tier 3 & Moderately Difficult & 5000 - 10000 \\
    Tier 4 & Slightly Difficult & 10000 - 20000 \\
    Tier 5 & No Difficulty & 20000+ \\
    \bottomrule
\end{tabular}%
}}
\vspace{-0.1in}
\end{table}

\section{Definitions of variables in DCN process}

Table~\ref{tab:debt_variable_app} provides the descriptions of all the variables appearing in Algorithm 1. The Action Set includes \texttt{ask}, \texttt{reject}, and \texttt{accept}, while the Negotiation Dimension Set consists of the four quantities listed in Table~\ref{tab:negotiation_dimensions}.


\begin{table}[ht]
\centering
\Large
\caption{Definitions of variables in DCN process.}
\resizebox{0.7\columnwidth}{!}{%
\begin{tabular}{@{}ll@{}}
\toprule
\textbf{Conception} & \textbf{Variable} \\ \midrule
Basic Information & $I_b$ \\
Creditor & creditor \\
Action Set & $S_A$ \\
Result Dictionary & $D$ \\
Personal Financial Information & $I_p$ \\
Debtor & debtor \\
Negotiation Dimension Set & $S_R$ \\
Turn & $t$ \\
Max Turns & $t_m$ \\
Agent Creditor & Creditor \\
Agent Debtor & Debtor \\
Action of Debtor & $A_d$ \\
Action of Creditor & $A_c$ \\ 
\bottomrule
\end{tabular}%
}

\label{tab:debt_variable_app}
\end{table}
