\begin{figure}[htbp]
\vspace{-0.1in}
  \centering
  \includegraphics[width=0.48\textwidth]{latex/images/metric.pdf}  
  \vspace{-0.1in}
  \caption{\label{image:eval}
 Evaluation system of DCN.}
\vspace{-0.1in}
\label{img:metric}
\end{figure}


\section{Evaluation System}\label{sec:eval}

Different from traditional negotiation evaluations, we argue that the DCN task requires a more comprehensive assessment framework. As illustrated in Figure ~\ref{image:eval}, we developed a evaluation system based on four aspects and extended several average metrics for a comprehensive assessment. 

% we propose a four-dimensional evaluation system, which includes conversational ability, debt recovery rate, collection efficiency, and debtor’s financial health, encompassing thirteen metrics.

\subsection{Segmented Evaluation Metrics}

In this section, we provide a general overview of the \textbf{10 metrics} across the four segmented aspects. Detailed descriptions, the evaluation process, and calculation formulas are further discussed in Appendix~\ref{app:metric}.

\textbf{Conversational Ability (§\ref{app:me_conv}).} Conversational ability is crucial in negotiation processes for effective communication and mutual understanding. We evaluate it using two metrics: \textit{(i) Dialogue Soundness}\textbf{ (DS)} is assessed on a five-point scale, measuring the fluency, naturalness and coherency of responses; \textit{(ii) Dialogue Completeness}\textbf{ (DC)} is an automated metric that evaluates whether four objectives are all addressed during the dialogue.

\textbf{Debt Recovery (§\ref{app:me_rec}).} In debt collection, the primary goal is to recover as much debt as possible. We evaluate this using two key metrics: \textit{(i) Success Recovery Rate} \textbf{(SR)} measures the proportion of samples where repayment can be successfully completed, based on the debtor’s future ability to meet repayment goals. \textit{(ii) Recovery Rate}\textbf{ (RR)} reflects the portion of the debt that has been successfully recovered by the creditor, calculated as the average recovery ratio across all test samples.

\textbf{Collection Efficiency (§\ref{app:me_col}).} Collection efficiency refers to how quickly a debtor can repay their debt. We monitor the timing of repayments using three key metrics: \textit{(i) 25\% Recovery Date}\textbf{ (QRD)} is the estimated date when the debtor has completed 25\% of the debt repayment, with earlier dates indicating quicker repayment; (\textit{ii) 50\% Recovery Date} \textbf{(HRD)} marks the completion of 50\% of the repayment, offering insight into the debtor’s ongoing repayment ability. \textit{(iii) The Completion Date} \textbf{(CD)} is the date when the debtor has fully repaid their debt, with a shorter completion date indicating a faster recovery process.

\textbf{Debtor’s Financial Health (§\ref{app:me_hea}).} The debtor’s financial health plays a critical role in successful debt recovery. It affects both the debtor’s ability to repay and the speed at which repayment occurs. We assess financial health using three metrics: \textit{(i) L1 Tier Days} \textbf{(L1D)} tracks the number of days the debtor remains in the most difficult financial tier (L1), with longer durations indicating higher risk of default; \textit{(ii) L2 Tier Days} \textbf{(L2D)} similarly tracks the days in the second most difficult financial tier (L2), which still reflects financial strain; \textit{(iii) Asset Tier Variance} \textbf{(ATV)} captures the variance in the debtor’s asset tier over a year, providing insight into the stability of their financial condition. 

\subsection{Comprehensive Indices}

We find that the indicators for recovery and efficiency are often \textit{\textbf{inversely related}} to the debtor’s financial condition in debt collection. To balance these conflicting objectives, we introduce three average metrics (detailed description and calculation process can be found in Appendic~\ref{app:me_ave}):

\textbf{(i) Creditor’s Recovery Index (CRI):} CRI is the \textit{weighted average} of five indicators from Debt Recovery and Collection Efficiency. It reflects an evaluation of the overall collection process by debt collectors, disregarding debtor-related factors. A higher value is more favorable to the creditor.

\textbf{(ii) Debtor’s Health Index (DHI):} DHI is the \textit{weighted average} of three indicators from Debtor’s Financial Health. It assesses the financial well-being of the debtor throughout the repayment process, with a higher value indicating a greater probability of the debtor adhering to the repayment plan.

\textbf{(iii) Comprehensive Collection Index (CCI):} CCI is the \textit{harmonic mean} of CRI and DHI. It provides a comprehensive evaluation of the negotiation outcome, where a higher value signifies the maximization of debt recovery and efficiency while ensuring the debtor’s financial health.


