\section{Related Work}
To the best of our knowledge, \textit{this work is the first to evaluate the influence of LL on the confidentiality property in secure hardware designs using an automated methodology}. Recently, logic locking has been exploited to break the integrity of a neural accelerator during runtime. Logic locking is used as a backdoor in this context to reduce the quantity of the correct detections~\cite{lolo_ai_attack}. A manual inspection of the logic-locked MiG-V processor revealed that incorrect LL keys can be used to leak sensitive data like encryption keys~\cite{mig_v_cracked}.

Additionally, the application of path sensitization used in this work has been utilized in other contexts of LL. For example, path sensitization has been applied to analyze whether LL key bits are stored safely~\cite{path_sensitization_lolo}. %Thereby, it is evaluated whether the LL key bits can be forwarded to the design's output using a computed input sequence. 
Furthermore, SAT-attacks (a popular class of key-retrieval attacks) aim to solve Boolean satisfiability problems to assess how inputs propagate from primary inputs to primary outputs~\cite{sat_attack1, sat_attack2, sat_attack3, sat_on_lolo}. Retrieving the LL key bits can unlock the locked netlist and enable IP piracy, overproduction, and hardware Trojans.
%If the same locked design is used, the LL key can be used to recover the design in the next manufacturing batch.
%In addition to path sensitization attacks on locking schemes, machine learning has been used to retrieve the original functionality of the locked design~\cite{lolo_ml,titan}. All existing studies have focused on whether the original design or the LL key can be retrieved.
\textit{Nevertheless, a comprehensive analysis of the impact of LL on the security properties of the unlocked design is still missing.} We address this research gap by developing an automated methodology to evaluate this impact and showcase it for three representative LL algorithms.