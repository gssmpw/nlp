%Version 3 December 2023
% See section 11 of the User Manual for version history
%
%%%%%%%%%%%%%%%%%%%%%%%%%%%%%%%%%%%%%%%%%%%%%%%%%%%%%%%%%%%%%%%%%%%%%%
%%                                                                 %%
%% Please do not use \input{...} to include other tex files.       %%
%% Submit your LaTeX manuscript as one .tex document.              %%
%%                                                                 %%
%% All additional figures and files should be attached             %%
%% separately and not embedded in the \TeX\ document itself.       %%
%%                                                                 %%
%%%%%%%%%%%%%%%%%%%%%%%%%%%%%%%%%%%%%%%%%%%%%%%%%%%%%%%%%%%%%%%%%%%%%

%%\documentclass[referee,sn-basic]{sn-jnl}% referee option is meant for double line spacing

%%=======================================================%%
%% to print line numbers in the margin use lineno option %%
%%=======================================================%%

%%\documentclass[lineno,sn-basic]{sn-jnl}% Basic Springer Nature Reference Style/Chemistry Reference Style

%%======================================================%%
%% to compile with pdflatex/xelatex use pdflatex option %%
%%======================================================%%

%%\documentclass[pdflatex,sn-basic]{sn-jnl}% Basic Springer Nature Reference Style/Chemistry Reference Style


%%Note: the following reference styles support Namedate and Numbered referencing. By default the style follows the most common style. To switch between the options you can add or remove Numbered in the optional parenthesis. 
%%The option is available for: sn-basic.bst, sn-vancouver.bst, sn-chicago.bst%  
 
%%\documentclass[pdflatex,sn-nature]{sn-jnl}% Style for submissions to Nature Portfolio journals
%%\documentclass[pdflatex,sn-basic]{sn-jnl}% Basic Springer Nature Reference Style/Chemistry Reference Style
\documentclass[pdflatex,sn-mathphys-num]{sn-jnl}

% Math and Physical Sciences Numbered Reference Style 
%%\documentclass[pdflatex,sn-mathphys-ay]{sn-jnl}% Math and Physical Sciences Author Year Reference Style
%%\documentclass[pdflatex,sn-aps]{sn-jnl}% American Physical Society (APS) Reference Style
%%\documentclass[pdflatex,sn-vancouver,Numbered]{sn-jnl}% Vancouver Reference Style
%%\documentclass[pdflatex,sn-apa]{sn-jnl}% APA Reference Style 
%%\documentclass[pdflatex,sn-chicago]{sn-jnl}% Chicago-based Humanities Reference Style

%%%% Standard Packages
%%<additional latex packages if required can be included here>
%\usepackage[left=1.5cm, right=2cm]{geometry}

\usepackage{graphicx}%
\usepackage{multirow}%
\usepackage{amsmath,amssymb,amsfonts}%
\usepackage{amsthm}%
\usepackage{mathrsfs}%
\usepackage[title]{appendix}%
\usepackage{xcolor}%
\usepackage{textcomp}%
\usepackage{manyfoot}%
\usepackage{booktabs}%
\usepackage{algorithm}%
\usepackage{algorithmicx}%
\usepackage{algpseudocode}%
\usepackage{listings}%
\usepackage{tabularx}
\usepackage{adjustbox}
% \usepackage{makecells}
% \restylefloat{table}

\usepackage[algo2e]{algorithm2e} 
\usepackage{multicol}
% \usepackage{algorithmic}
%%%%%%%%%%% Defining Enunciations  %%%%%%%%%%%
\newtheorem{lemma}{\bf Lemma}[section]
% \newtheorem{theorem}{\bf Theorem}[section]
\newtheorem{condition}{\bf Condition}[section]
\newtheorem{corollary}{\bf Corollary}[section]
%%%%

%%%%%=============================================================================%%%%
%%%%  Remarks: This template is provided to aid authors with the preparation
%%%%  of original research articles intended for submission to journals published 
%%%%  by Springer Nature. The guidance has been prepared in partnership with 
%%%%  production teams to conform to Springer Nature technical requirements. 
%%%%  Editorial and presentation requirements differ among journal portfolios and 
%%%%  research disciplines. You may find sections in this template are irrelevant 
%%%%  to your work and are empowered to omit any such section if allowed by the 
%%%%  journal you intend to submit to. The submission guidelines and policies 
%%%%  of the journal take precedence. A detailed User Manual is available in the 
%%%%  template package for technical guidance.
%%%%%=============================================================================%%%%

%% as per the requirement new theorem styles can be included as shown below
% \theoremstyle{thmstyleone}%
\newtheorem{theorem}{Theorem}%  meant for continuous numbers
%%\newtheorem{theorem}{Theorem}[section]% meant for sectionwise numbers
%% optional argument [theorem] produces theorem numbering sequence instead of independent numbers for Proposition
\newtheorem{proposition}[theorem]{Proposition}% 
%%\newtheorem{proposition}{Proposition}% to get separate numbers for theorem and proposition etc.

% \theoremstyle{thmstyletwo}%
\newtheorem{example}{Example}%
\newtheorem{remark}{Remark}%

% \theoremstyle{thmstylethree}%
\newtheorem{definition}{Definition}%

\raggedbottom
%%\unnumbered% uncomment this for unnumbered level heads

\begin{document}

\centering{\textbf{Supplementary Material On Summary of Contribution}}

\vspace{0.5cm}
\begin{enumerate}
    \item What is the main claim of the paper? Why is this an important contribution to the machine learning literature?
\vspace{0.1cm}
    \item [] \textit{Response}: Tuberculosis (TB) remains a formidable global health challenge, driven by complex spatiotemporal transmission dynamics and influenced by factors such as population mobility and behavioral changes. This study proposes an Epidemic-Guided Deep Learning (EGDL) approach that fuses mechanistic epidemiological principles with advanced deep learning techniques to enhance early warning systems and intervention strategies for TB outbreaks. EGDL framework combines a modified networked SIR model with deep learning-based forecasting methods to accurately forecast TB outbreaks via understanding disease characteristics and spatiotemporal dynamics. Our work contributes to the interdisciplinary research areas where machine learning is applied in the spatiotemporal domain to solve an epidemic problem.

    \vspace{0.25cm}
    \item What is the evidence you provide to support your claim? 
\vspace{0.1cm}
    
    \item [] \textit{Response}: First, we introduce a modified networked SIR model by incorporating a saturated incidence rate and graph Laplacian diffusion to understand epidemic data's spatiotemporal dynamics. We validate the key epidemiological properties, such as positivity and boundedness. The global stability of disease-free and endemic equilibria is established using Green's formula and the comparison principle. We combined the networked SIR model with the historical surveillance data-driven deep learning frameworks to enhance spatiotemporal disease forecasting. Experiments conducted on TB incidence data from 47 prefectures in Japan demonstrate that our approach delivers robust and accurate predictions across multiple time horizons (short to medium-term forecasts) compared to benchmark data-centric methods.

    \vspace{0.25cm}
    \item What papers by other authors make the most closely related contributions, and how is your paper related to them?
\vspace{0.1cm}
    
    \item [] \textit{Response}: This study closely relates to recent works integrating mechanistic models with deep learning architectures for epidemic forecasting. For example, Delli et al. \cite{delli2022hybrid} developed a mechanistic model-based neural network using a residual remodeling approach for forecasting intensive care unit occupancy during COVID-19 epidemics. Authors in \cite{rodriguez2023einns,qian2025physics} integrated the epidemic knowledge of the mechanistic models into deep learning architectures by adopting a gradient learning approach. These studies have showcased how integrating epidemiological principles into data-driven models has enhanced forecast accuracy in the epidemic domain. However, none of the studies have modeled the spatiotemporal dynamics of epidemic transmission, a prominent challenge in epidemiology. This study fills the gap by developing the EGDL framework that integrates epidemic principles, spatial dynamics, and temporal interactions of TB incidence cases.
    \vspace{0.25cm}
    \item Have you published parts of your paper before, for instance, in a conference? 
\vspace{0.1cm}
    \item [] \textit{Response}: No parts of this paper have been published elsewhere.

    \vspace{0.2cm}
\end{enumerate}

\bibliography{bibliography}% common bib file

\end{document}