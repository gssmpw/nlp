
\documentclass[pdflatex,sn-mathphys-num]{sn-jnl}
\usepackage{graphicx}%
\usepackage{multirow}%
\usepackage{amsmath,amssymb,amsfonts}%
\usepackage{amsthm}%
\usepackage{mathrsfs}%
\usepackage[title]{appendix}%
\usepackage{xcolor}%
\usepackage{textcomp}%
\usepackage{manyfoot}%
\usepackage{booktabs}%
\usepackage{algorithm}%
\usepackage{algorithmicx}%
\usepackage{algpseudocode}%
\usepackage{listings}%
\usepackage{tabularx}
\usepackage{adjustbox}
% \usepackage{makecells}
% \restylefloat{table}

\usepackage[algo2e]{algorithm2e} 
\usepackage{multicol}
% \usepackage{algorithmic}
%%%%%%%%%%% Defining Enunciations  %%%%%%%%%%%
\newtheorem{lemma}{\bf Lemma}[section]
% \newtheorem{theorem}{\bf Theorem}[section]
\newtheorem{condition}{\bf Condition}[section]
\newtheorem{corollary}{\bf Corollary}[section]
%%%%

\newtheorem{theorem}{Theorem}%  meant for continuous numbers
%%\newtheorem{theorem}{Theorem}[section]% meant for sectionwise numbers
%% optional argument [theorem] produces theorem numbering sequence instead of independent numbers for Proposition
\newtheorem{proposition}[theorem]{Proposition}% 
%%\newtheorem{proposition}{Proposition}% to get separate numbers for theorem and proposition etc.

% \theoremstyle{thmstyletwo}%
\newtheorem{example}{Example}%
\newtheorem{remark}{Remark}%

% \theoremstyle{thmstylethree}%
\newtheorem{definition}{Definition}%

\raggedbottom
%%\unnumbered% uncomment this for unnumbered level heads

\begin{document}

\title[Article Title]{Epidemic-guided deep learning for spatiotemporal forecasting of Tuberculosis outbreak}

%%=============================================================%%
%% GivenName	-> \fnm{Joergen W.}
%% Particle	-> \spfx{van der} -> surname prefix
%% FamilyName	-> \sur{Ploeg}
%% Suffix	-> \sfx{IV}
%% \author*[1,2]{\fnm{Joergen W.} \spfx{van der} \sur{Ploeg} 
%%  \sfx{IV}}\email{iauthor@gmail.com}
%%=============================================================%%

\author[1,2]{\fnm{Madhab} \sur{Barman}} %\email{mat18d004@iiitdm.ac.in}
\equalcont{These authors contributed equally to this work.}

\author[1]{\fnm{Madhurima} \sur{Panja}} %\email{madhurima.panja@sorbonne.ae}
\equalcont{These authors contributed equally to this work.}

\author[2]{\fnm{Nachiketa} \sur{Mishra}} %\email{mishra.nachiketa@gmail.com}

\author*[1,3]{\fnm{Tanujit} \sur{Chakraborty}} \email{tanujit.chakraborty@sorbonne.ae}

\affil[1]{\orgname{Sorbonne University Abu Dhabi}, \orgaddress{\country{UAE}}} % \orgdiv{Department of Science and Engineering}

\affil[2]{\orgname{IIITDM Kancheepuram}, \orgaddress{\city{Chennai}, \country{India}}} % \orgdiv{Department of Science and Humanities}, , 
%Indian Institute of Information Technology, Design and Manufacturing,

\affil[3]{\orgdiv{Sorbonne Center for Artificial Intelligence}, \orgname{Sorbonne University}, \orgaddress{\country{France}}}


\abstract{
Tuberculosis (TB) remains a formidable global health challenge, driven by complex spatiotemporal transmission dynamics and influenced by factors such as population mobility and behavioral changes. We propose an Epidemic-Guided Deep Learning (EGDL) approach that fuses mechanistic epidemiological principles with advanced deep learning techniques to enhance early warning systems and intervention strategies for TB outbreaks. Our framework is built upon a networked Susceptible-Infectious-Recovered (SIR) model augmented with a saturated incidence rate and graph Laplacian diffusion, capturing both long-term transmission dynamics and region-specific population mobility patterns. Compartmental model parameters are rigorously estimated using Bayesian inference via the Markov Chain Monte Carlo (MCMC) approach. Theoretical analysis leveraging the comparison principle and Green’s formula establishes global stability properties of the disease-free and endemic equilibria. Building on these epidemiological insights, we design two forecasting architectures, EGDL-Parallel and EGDL-Series, that integrate the mechanistic outputs of the networked SIR model within deep neural networks. This integration mitigates the overfitting risks commonly encountered in data-driven methods and filters out noise inherent in surveillance data, resulting in reliable forecasts of real-world epidemic trends. Experiments conducted on TB incidence data from 47 prefectures in Japan demonstrate that our approach delivers robust and accurate predictions across multiple time horizons (short to medium-term forecasts). Additionally, incorporating uncertainty quantification through conformal prediction enhances the model’s practical utility for guiding targeted public health interventions.
}

%%================================%%
%% Sample for structured abstract %%
%%================================%%

% \abstract{\textbf{Purpose:} The abstract serves both as a general introduction to the topic and as a brief, non-technical summary of the main results and their implications. The abstract must not include subheadings (unless expressly permitted in the journal's Instructions to Authors), equations or citations. As a guide the abstract should not exceed 200 words. Most journals do not set a hard limit however authors are advised to check the author instructions for the journal they are submitting to.
% 
% \textbf{Methods:} The abstract serves both as a general introduction to the topic and as a brief, non-technical summary of the main results and their implications. The abstract must not include subheadings (unless expressly permitted in the journal's Instructions to Authors), equations or citations. As a guide the abstract should not exceed 200 words. Most journals do not set a hard limit however authors are advised to check the author instructions for the journal they are submitting to.
% 
% \textbf{Results:} The abstract serves both as a general introduction to the topic and as a brief, non-technical summary of the main results and their implications. The abstract must not include subheadings (unless expressly permitted in the journal's Instructions to Authors), equations or citations. As a guide the abstract should not exceed 200 words. Most journals do not set a hard limit however authors are advised to check the author instructions for the journal they are submitting to.
% 
% \textbf{Conclusion:} The abstract serves both as a general introduction to the topic and as a brief, non-technical summary of the main results and their implications. The abstract must not include subheadings (unless expressly permitted in the journal's Instructions to Authors), equations or citations. As a guide the abstract should not exceed 200 words. Most journals do not set a hard limit however authors are advised to check the author instructions for the journal they are submitting to.}

\keywords{Spatiotemporal data, Time series forecasting, Epidemiology, Deep learning}

% Other Journals \keywords{Tuberculosis, graph Laplacian, global stability, epidemic-informed machine learning, time series forecasting}

%%\pacs[JEL Classification]{D8, H51}

%%\pacs[MSC Classification]{35A01, 65L10, 65L12, 65L20, 65L70}

\maketitle

\section{Introduction}\label{sec1}
Tuberculosis (TB), caused by Mycobacterium tuberculosis complex, is a chronic infectious disease primarily transmitted through aerosols produced by coughing \cite{dheda2014global}. In 2023, the World Health Organization (WHO) reported 8.2 million new TB cases globally, a significant increase from 7.5 million in 2022\footnote{\url{https://www.who.int/teams/global-tuberculosis-programme/tb-reports}}. This makes TB the leading cause of death from a single infectious disease, surpassing COVID-19. To address the global health burden associated with a high TB incidence rate, the WHO introduced the ``End TB strategy'' in 2015, aiming to reduce TB incidence cases below 0.01\% by 2035 and 0.001\% by 2050 \cite{uplekar2015s}. However, the current global rate of decline is not on track to achieve these goals. In the era of rapid globalization, the existence of a TB-free region is impractical unless a global mitigation approach is successfully implemented. While low and medium-income countries are traditionally considered high TB-burden regions, TB remains a persistent challenge even in high-income countries due to increased population mobility \cite{pareek2016impact}. Among the high-income nations, Japan continues to be classified as a TB middle-burden country, with an incidence rate of 10-49 cases per 100,000 population\footnote{\url{https://iris.who.int/handle/10665/341980}}. High TB incidences and related mortality in Japan are primarily attributed to an aging population \cite{hagiya2019trends}, as well as the rising number of immigrants from high TB-burden countries, accounting for a growing proportion of cases among younger populations \cite{ota2021epidemiology}. To mitigate the health and socioeconomic consequences of TB, various attempts have been made to develop early warning systems that predict the future TB dynamics \cite{zheng2015forecast, wang2017hybrid}. An early warning system provides quick hindsight, allowing for timely intervention and better disease control in its nascent stages. Epidemic modeling and forecasting are central to these efforts, as they enable more accurate predictions of disease trends, guide public health responses, and help allocate resources more effectively to reduce the TB transmission rate.

Compartmental epidemiological models are widely used to characterize infectious disease dynamics by dividing the population into distinct and mutually exclusive subsets based on their infection status \cite{grassly2008mathematical, brauer2019mathematical}. The compartmental SIR model categorizes individuals as Susceptible ($S$), those who are vulnerable to infection; Infectious ($I$), those who are currently infected and actively transmitting the disease; and Recovered ($R$), those who have gained immunity after recovering from the disease \cite{kermack1927contribution}. Numerous extensions of the classical SIR model, including additional compartments and refined incidence functions, have been developed to model TB transmission dynamics in Indonesia \cite{side2018numerical} and Kazakhastan \cite{kalizhanova2024modeling}. Although compartmental epidemiological models yield fundamental insights into infectious disease outbreaks, they are more suited for understanding the disease dynamics, rather than real-time forecasting of the disease incidences \cite{keeling2008modeling}. To overcome the problem of limited predictability of the compartmental approaches, data-centric forecasting methods, including statistical and deep learning models, have emerged as powerful tools. For example, \cite{abade2024comparative} demonstrated the effectiveness of deep learning frameworks in forecasting TB infections in Brazil, while \cite{wang2017hybrid} implemented a data-driven hybrid framework to predict TB incidence in China. These approaches leverage diverse data sources to uncover hidden patterns in epidemic data, showcasing their flexibility. However, this flexibility can often be a double-edged sword, making them prone to overfitting, particularly in low-data regimes, a common challenge in epidemic forecasting \cite{rodriguez2024machine}. Moreover, these models often generate unrealistic predictions for non-stationary systems as they solely rely on lagged values of disease incidences. The primary disadvantage of data-driven methods for epidemic forecasting is that they are not guided by the mechanisms governing disease transmission. By focusing solely on the dynamics of surveillance data without integrating epidemiological principles, these approaches often generate unrealistic forecasts and struggle to differentiate genuine trends from noise introduced during data collection \cite{qian2025physics}. To bridge this gap, a recent line of research has explored integrating epidemic dynamics from compartmental models into data-centric forecasting frameworks \cite{ye2025integrating, delli2022hybrid}. For instance, Epidemic-Informed Neural Networks (EINNs) embed epidemiological knowledge from ordinary differential equation (ODE) based compartmental models into Recurrent Neural Networks (RNN) via gradient matching, enhancing model flexibility and resilience against noisy data \cite{rodriguez2023einns}. Similarly, \cite{qian2025physics} employed Physics Informed Neural Networks (PINNs) to incorporate epidemiological principles from a modified SIR framework with observed infectious data, improving forecast accuracy through transfer learning.

Despite these advancements, most studies in this domain rely on traditional SIR-based compartmental models, which do not account for spatial dynamics, such as population mobility. As a result, these models often fail to capture the spatiotemporal trajectory of an epidemic, limiting their ability to predict real-world transmission patterns accurately. Recent epidemiological studies have incorporated network-based structures into SIR models to address this limitation in compartmental models, enhancing their ability to capture spatial disease transmission \cite{pastor2015epidemic, keeling2005networks}. By leveraging network structures, the SIR model can estimate epidemic dynamics at the population level rather than focusing solely on individual-level infection status, leading to a more comprehensive understanding of disease spread. To represent the spatial disease dynamics, previous studies have often relied on Laplacian operators to model population mobility, assuming isotropic diffusion and uniform movement probabilities \cite{li2018diffusive, lei2018theoretical, allen2007asymptotic}. However, these assumptions are frequently violated in real-world scenarios, where human mobility is influenced by local environmental factors such as lockdowns, travel restrictions, and other region-specific circumstances. To address these challenges, recent studies have extended the global stability theory of the SIR model by incorporating graph Laplacian diffusion, leading to more efficient frameworks for modeling spatial epidemic dynamics \cite{tian2020global, barman2024network}. In existing networked epidemic models, bilinear incidence rate ($\beta S I$) and standard bilinear incidence rate ($\beta S I/N$), with transmission rate $\beta$ and population size $N$, are widely used to analyze the global dynamics of disease spread \cite{liu2020weighted, tian2023asymptotic}. However, these incidence functions fail to account for behavioral changes and crowding effects, often leading to overestimated infection rates in highly connected populations. To overcome these limitations, we adopt a saturated incidence function \cite{capasso1978generalization} and formulate a networked SIR model by incorporating graph Laplacian diffusion with the saturated incidence rate. This integration better captures long-term transmission dynamics and provides an adaptive framework for modeling dynamic changes in population behavior. Building on this epidemiological foundation, we further introduce the Epidemic-Guided Deep Learning (EGDL) approach, which combines spatial epidemiological insights with data-driven forecasting techniques. EGDL generates accurate epidemic forecasts by considering both spatial dynamics and temporal interactions. The proposed EGDL architectures integrate the long-term infection dynamics of the networked SIR model as auxiliary information within deep learning frameworks. This integration combines the flexibility of deep learning techniques and the epidemiological principles captured by the modified version of the networked SIR model, improving spatiotemporal disease incidence forecasting. Furthermore, using epidemic dynamics as auxiliary information, instead of solely relying on gradient-based learning, restricts the overfitting risks in EGDL frameworks, allowing the model to learn robust representations. This effectively filters out noise from historical incidence data and aligns spatiotemporal disease forecasts with real-world epidemic trends. The key contributions of the study can be summarized as follows:
% \begin{itemize}
%     \item We introduce an extended version of the networked SIR model by incorporating a saturated incidence rate and graph Laplacian diffusion to understand epidemic data's spatiotemporal dynamics. We validate the key epidemiological properties such as positivity and boundedness. The global stability of disease-free and endemic equilibria is established using Green's formula and the comparison principle.
%     \item To enhance spatiotemporal disease forecasting, we propose the EGDL-Parallel and EGDL-Series architectures, combining the networked SIR model with the historical surveillance data-driven deep learning frameworks (e.g., Transformers, NBeats, etc.).
%     % \item We study the global features of TB incidence data across 47 prefectures in Japan and employ the Granger causality test and wavelet coherence plots to assess the causal relationship between the networked SIR model output and active TB cases.
%     \item The forecasting performance of the EGDL architectures is evaluated using a rolling window approach across four test horizons and compared against data-centric models, with robustness validated through non-parametric statistical tests. The uncertainty of the EGDL frameworks is quantified using the conformal prediction approach.
% \end{itemize}

\begin{itemize}
    \item We introduce a modified networked SIR model by incorporating a saturated incidence rate and graph Laplacian diffusion to understand epidemic data's spatiotemporal dynamics. We validate the key epidemiological properties, such as positivity and boundedness. The global stability of disease-free and endemic equilibria is established using Green's formula and the comparison principle.
    \item To enhance spatiotemporal disease forecasting, we propose the EGDL-Parallel and EGDL-Series architectures, combining the networked SIR model with the historical surveillance data-driven deep learning frameworks (e.g., Transformers, NBeats).
    \item The forecasting performance of the EGDL architectures is evaluated using a rolling window approach across four test horizons and compared against data-centric models, with robustness validated through non-parametric statistical tests. The uncertainty of the EGDL frameworks is quantified using the conformal prediction approach.
\end{itemize}

% (a) We introduce an extended version of the networked SIR model by incorporating a saturated incidence rate and graph Laplacian diffusion to understand epidemic data's spatiotemporal dynamics. We validate the key epidemiological properties such as positivity and boundedness. The global stability of disease-free and endemic equilibria is established using Green's formula and the comparison principle.

% (b) To enhance spatiotemporal disease forecasting, we propose the EGDL-Parallel and EGDL-Series architectures, combining the networked SIR model with the historical surveillance data-driven deep learning frameworks (e.g., Transformers, NBeats, etc.).
%     % \item We study the global features of TB incidence data across 47 prefectures in Japan and employ the Granger causality test and wavelet coherence plots to assess the causal relationship between the networked SIR model output and active TB cases.

% (c) The forecasting performance of the EGDL architectures is evaluated using a rolling window approach across four test horizons and compared against data-centric models, with robustness validated through non-parametric statistical tests. The uncertainty of the EGDL frameworks is quantified using the conformal prediction approach.

% \noindent The rest of the paper is organized as follows: Section \ref{data} presents the study area and the dataset. Section \ref{Sec_Prelim} provides an overview of the epidemic compartmental model and the deep learning frameworks. Section \ref{Network_SIR_Model} presents the mathematical formulation and theoretical properties of the modified networked SIR model. Section \ref{Sec_EGDL_Model} details the architectures of the EGDL frameworks. Section \ref{Sec_forecast} showcases the empirical results of our proposed approaches for Japan's TB dataset. Finally, Section \ref{Sec_Conclusion} concludes our study with discussions on the policy implications and potential future directions.

\section{Motivating Example: Prefecture-specific active TB study}\label{data}
\begin{figure}[!ht]
    \centering
    \includegraphics[width=0.9\linewidth]{Japan_Prefecture.png}
    \caption{(A) Geographic distribution of Japan's 47 prefectures, shown for illustrative purposes only, without implying any political assertions on Japan's territorial boundaries. (B) Monthly active tuberculosis (TB) cases recorded across Japan's 47 prefectures over 216 months, spanning from January 1998 to December 2015.}
    \label{fig:TB47}
\end{figure}
\noindent TB remains a leading cause of infectious disease-related mortality, presenting a significant global public health challenge. In Japan, classified as a TB medium-burden country, the TB surveillance center reported 11,519 new cases in 2022\footnote{\url{https://jata-ekigaku.jp/english/tb-in-japan}}. To control the rapid spread of TB, Japan has established a robust nationwide surveillance system that collects detailed TB data at the prefectural level, ensuring comprehensive coverage. Japan's administrative structure, comprising 47 prefectures further divided into cities, wards, and blocks, facilitates precise data collection and monitoring. This hierarchical organization enables tracking TB trends both geographically and temporally, enhancing the effectiveness of public health interventions. In this study, we analyze the monthly counts of newly registered active TB cases of all forms across Japan’s 47 prefectures, as illustrated in Fig. \ref{fig:TB47}(A). The dataset, spanning 216 months from January 1998 to December 2015, is publicly available at \cite{data_monthly, sumi2019time}. Fig. \ref{fig:TB47}(B) visually represents the monthly active TB cases for all 47 prefectures. In this figure, each prefecture is represented by a distinct color, enabling clear differentiation and facilitating visual comparisons of TB trends across regions. These TB datasets represent a spatiotemporal forecasting problem for infectious disease modelers. To capture the spatiotemporal dynamics of TB outbreaks, our study aims to build an integrated solution based on epidemiological models and deep learning approaches.

\section{Preliminaries}\label{Sec_Prelim}
Epidemic modeling and forecasting approaches can be broadly categorized as mechanistic (or compartmental) frameworks and phenomenological models. Mechanistic models use causal frameworks to describe disease states and understand the surveillance data-generating process. In contrast, phenomenological models focus on directly modeling surveillance data (past lagged observations of the epidemic time series) without explicitly incorporating the underlying epidemiological mechanisms. The following subsections provide a brief overview of the mechanistic frameworks (Section \ref{Sec_Epi_model_Review}) and phenomenological models (Section \ref{Sec_Phenom_Review}) to be used as building blocks for the proposed EGDL frameworks.

% This model provides the mathematical foundation for understanding the propagation, stabilization, and eventual resolution of disease outbreaks, driven by critical parameters such as transmission and recovery rates as follows:

\subsection{Epidemic Models: A Review} \label{Sec_Epi_model_Review}
The SIR model, developed by Kermack and McKendrick \cite{kermack1927contribution}, serves as a foundational framework for analyzing the temporal dynamics of epidemic outbreaks. The model is governed by the following system of differential equations:
% \begin{subequations}\label{SIR}
% \begin{align}
%     \dfrac{dS}{dt}  = -\lambda S,\quad
%     \dfrac{dI}{dt}  = \lambda S  -\gamma I, \quad
%     \dfrac{dR}{dt}  = \gamma I,
% \end{align}
% \end{subequations}
\begin{equation}\label{SIR}
    \dfrac{dS}{dt}  = -\lambda S,\quad
    \dfrac{dI}{dt}  = \lambda S  -\gamma I, \quad
    \dfrac{dR}{dt}  = \gamma I,
\end{equation}
where $\lambda$ denotes the force of infection which measures the rate at which susceptible individuals contract the infection and $\gamma$ is the recovery rate. The force of infection is commonly formulated as a function of the number of infected individuals and encapsulates the dynamics of interactions leading to the transmission of the infection. This force function $\lambda (I)$ can determined as:
\begin{align}\label{force}
    \lambda = \beta_0 P(N) {I}/{N},
\end{align}
where $\beta_0$ denotes the probability of transmission per contact between a susceptible and an infected individual, $P(N)$ is the contact rate, defined as the average number of contacts adequate for disease transmission by an individual per unit time \cite{ma2009dynamical}.
 If \( P(N) = cN \), where \( c \) is a positive constant, this implies that the contact rate is directly proportional to the total population size. Consequently, the force of infection in Eq. 
 \eqref{force} becomes $\lambda = \beta_0 c I = \beta I,$ where \( \beta = \beta_0 c \) is referred to as the transmission coefficient. Thus, the incidence is represented as  $\beta S I$, commonly called bilinear incidence. Furthermore, the above model in Eq. \eqref{SIR} can be extended by incorporating vital dynamics, including the birth rate ($\Lambda$) and death rate ($\mu$), along with a saturated incidence rate as proposed in \cite{capasso1978generalization}. The SIR epidemic model with a saturated incidence rate can be expressed as:
\begin{equation}\label{Sat_SIR}
% \begin{align}
    \dfrac{dS}{dt} = \Lambda -\dfrac{\beta SI}{1 + \alpha I} - \mu S, \; \quad
    \dfrac{dI}{dt} = \dfrac{\beta SI}{1 + \alpha I} -(\gamma + \mu) I, \; \quad
    \dfrac{dR}{dt} = \gamma I - \mu R,
% \end{align}
\end{equation}
where $\alpha$  represents the saturation factor. This model assumes homogeneous mixing among individuals, meaning that every susceptible individual is equally likely to interact with any other individual in the population at any given time. This assumption simplifies the representation of contact patterns, treating all interactions as uniformly distributed across the population. In real-world population dynamics, contact patterns are often heterogeneous rather than homogeneous. Building upon this basic framework, heterogeneity can be introduced by further subdividing the compartments to account for diverse population characteristics \cite{anderson1988epidemiology}. Typically, the number of contacts each individual has is significantly smaller than the total population size. In such cases, the homogeneous-mixing assumption becomes inefficient. The simple SIR model describes the temporal dynamics of an infectious disease in a population; however, it fails to capture the spatial dynamics of disease incidence. Epidemic models incorporating network structures address this limitation by assigning each individual a finite set of permanent contacts within a localized region, providing a more accurate depiction of infection dynamics.
% These network-based models capture the underlying heterogeneity by representing the geographical or social connections of individuals, providing a more accurate depiction of infection dynamics in structured populations. By integrating network frameworks, we can better account for the spatial and interaction-based variability inherent in real-world scenarios. 

\subsection{Deep Learning Models for Time Series Forecasting} \label{Sec_Phenom_Review}

Time series forecasting, which leverages historical data to predict future trends, has emerged as a crucial area of research with significant applications in public health. In recent years, there has been a rapid expansion in data-driven approaches for epidemic forecasting, aimed at predicting the spread and severity of infectious diseases \cite{chakraborty2020real, rosenfeld2021epidemic, panja2023epicasting}. These forecasts are essential for developing efficient public health surveillance systems. They enable early detection of abnormal infection patterns and adverse health outcomes while mitigating the socioeconomic impacts of epidemics. %Among various data-centric methods, statistical approaches like the univariate AutoRegressive Integrated Moving Average (ARIMA) and its variants have been widely employed in epidemic forecasting tasks. Applications of these models include forecasting tuberculosis morbidity in China \cite{zheng2015forecast}, influenza incidence in Egypt \cite{kane2014comparison}, and dengue cases in Brazil \cite{martinez2011sarima}. Despite their widespread use, the linear assumptions inherent in these models limit their ability to address the complexities of epidemic data, which often involve non-linear and high-dimensional relationships. The growing availability of large-scale, diverse data from disease surveillance systems has highlighted the limitations of traditional methods. 
Recently, deep learning-based approaches have emerged as a more flexible tool to forecast epidemics. These methods can process complex datasets, learn intricate patterns, and capture non-linear relationships more effectively than classical statistical models for time series forecasting \cite{rodriguez2024machine}. %The architectural evolution of deep learning-based forecasting methods has progressed in several promising directions. 
Encoder-decoder frameworks incorporating attention mechanisms \cite{vaswani2017attention} or convolutional operations \cite{bai2018empirical} represent prominent approaches. Transformers, a popular example of encoder-decoder architectures with attention mechanisms, process time series data by mapping it to a high-dimensional vector through the encoder's input layer \cite{wu2020deep}. Positional encoding adds sequential information to the data, which is then passed through multiple encoder layers equipped with self-attention and feed-forward components. The output of the encoder is fed into the decoder, which processes it through corresponding input and decoder layers. Each decoder layer applies self-attention to the input and integrates this information to produce predictions. The final output layer employs look-ahead masking and one-position offsets to ensure that forecasts rely solely on historical data points. Another encoder-decoder-based forecasting method is the Temporal Convolutional Network (TCN), which uses convolutional networks to model sequential data \cite{chen2020probabilistic}. The encoder block in TCN architectures consists of stacked dilated causal convolutional networks, enhanced with residual blocks to capture stochastic patterns in historical time series observations. The decoder block incorporates latent representations from the encoder and generates an output through its residual network module. A dense layer further processes this output to predict the future trajectory of the time series. %Both Transformers and TCNs explicitly model the direct interaction between every pair of input-output elements, which enhances their forecasting capabilities but also results in resource-intensive computations. 
On the other hand, attention-free and convolution-free models with deep stacks of fully connected layers are also designed to capture input-output relationships while implicitly prioritizing resource-efficient computations. Among these models, the Neural Basis Expansion for Time Series (NBeats) framework employs a block-based architecture to forecast time series data through residual learning \cite{oreshkin2019n}. This framework consists of stacked blocks, with the initial block modeling the input time series to estimate the forward and backward outputs. Subsequent blocks refine the residual outputs of their predecessors similarly. Each block features a dense layer with ReLU activation, which generates forward and backward predictors of expansion coefficients. These coefficients are then processed by backward and forward basis layers to produce the desired backcast and forecast. Building on the block-based design of NBeats, the Neural Hierarchical Interpolation for Time Series (NHits) framework incorporates a multi-rate signal sampling approach to capture the multi-scale characteristics of time series data in long-term forecasting tasks \cite{challu2023nhits}. This architecture includes stacked blocks, each containing a Multilayer Perceptron (MLP) and a MaxPool layer, which focus on analyzing low-frequency components. %This approach enables consistent long-horizon forecasts. Additionally, NHits enhances computational efficiency through hierarchical interpolation, allowing it to handle large-scale datasets effectively. 

The architectural designs of these scalable deep learning forecasters demonstrate superior performance in capturing long-term dependencies compared to standard RNN-based frameworks, making them highly effective for epidemic forecasting tasks. For instance, Transformers have been shown to accurately forecast influenza cases \cite{wu2020deep}, while attention mechanisms have been utilized for predicting dengue incidence in Vietnam \cite{nguyen2022deep}. Similarly, a modified NBeats framework was developed to forecast COVID-19 hospitalizations in Canada \cite{motavali2023dsa}, and the NHits framework has been applied to provide precise forecasts of COVID-19 incidence and mortality in Brazil \cite{souza2023deep}. In addition to these advancements, there has been a growing interest in modeling spatio-temporal dependencies in epidemic data. Traditional methods, such as the Generalized Spatiotemporal Autoregressive (GSTAR) \cite{imro2023determination} and Fast Gaussian Process (GpGp) \cite{senanayake2016predicting} models, have been used to capture spatial and temporal patterns. Modern deep learning architectures, such as Spatiotemporal Graph Convolutional Networks (STGCN), have further enhanced the ability to model these complex relationships \cite{wang2022causalgnn}. A key challenge remains to include the epidemiological knowledge and spatial dynamics of the disease inside the deep learning frameworks so that they can be used for informed decision-making in public health departments. %This study combines a networked SIR model with deep learning-based forecasters to model, analyze, and forecast TB outbreaks and builds a real-time disease monitoring system.

\section{Modified Networked SIR Model} \label{Network_SIR_Model}
The networked epidemic model with graph Laplacian diffusion offers significant advantages over classical epidemic models by incorporating the spatial and network structure of populations \cite{keeling2008modeling}. Classical models typically assume a well-mixed population, where interactions occur uniformly among individuals. In contrast, network-based models represent individuals as nodes connected by edges, where the edges capture the interaction or movement between individuals or regions. This approach enables the incorporation of spatial heterogeneity and connectivity patterns, allowing for a more realistic representation of disease dynamics across communities or geographic regions.  Unlike classical models, which often rely on uniform diffusion assumptions, the graph Laplacian approach reflects localized disease transmission and varying intensities of disease spread. This is particularly important in fragmented regions, where isolated communities or uneven connectivity can significantly influence epidemic profiles.

\subsection{Model Formulation}
The model begins by considering a standard weighted, connected, and undirected finite graph  $G:=\langle {V}, {E} \rangle$ without self-loops, where $V$ represents the set of nodes (or vertices) with a total of $n$ nodes, and $E$ denotes the set of edges. Taking the graph as a spatial structure, we define the graph Laplacian operator ($\Delta$) acting on a function $h$  from continuous space to a finite graph as follows:
\begin{align}\label{hfun}
    \Delta h(x) = \sum_{{y \in V, y \sim x}} \left[ h(y) - h(x)\right],
\end{align}
where $y \sim x$ describes node $y$ is adjacent to node $x$ and $h$ is a function such that $h : V \to \mathbb{R}$, where $\mathbb{R}$ represents the set of real numbers \cite{tian2020global, liu2020weighted}. The operator $\Delta$ describes population mobility between regions. Building on this framework, combined with the model in Eq. \eqref{Sat_SIR}, we develop the networked SIR epidemic model with a saturation incidence rate, incorporating graph Laplacian diffusion, as follows:  
\begin{equation}\label{model}
    \begin{aligned}
     \dfrac{\partial S (x, t)}{\partial t}   - \sigma\Delta S(x, t) &= \Lambda - \dfrac{\beta S(x, t) I(x, t)}{1 + \alpha I(x, t)} - \mu S(x, t),\quad  S(x ,0) = S_0(x) > 0 \mbox{ for } x \in V,   \\
  \dfrac{\partial I (x, t)}{\partial t} 
    - \sigma \Delta I(x, t) &= \dfrac{\beta S(x, t) I(x, t)}{1 + \alpha I(x, t)} - (\gamma + \mu)I(x, t), \quad I(x ,0) = I_0(x) \geq 0 \mbox{ for } x \in V,  \\
   \dfrac{\partial R (x, t)}{\partial t} - \sigma \Delta R(x, t) &= \gamma I(x, t) - \mu R(x, t), \quad  R(x ,0) = R_0(x) \geq 0 \mbox{ for } x \in V.
    \end{aligned}
\end{equation}
The parameter $\sigma$, which ranges between 0 and 1, characterizes the rate of population mobility. This parameter is often referred to as the migration parameter or diffusion parameter. By adjusting $\sigma$, one can modulate population mobility as required. Notably, when $\sigma$ approaches 1, all regions tend to exhibit similar dynamics, behaving almost uniformly. In contrast, when $\sigma$ is near 0, the solution profiles appear decoupled, with regions behaving independently \cite{bustamante2021epidemic}.

\subsection{Equilibrium points}
Disease-free equilibrium point is $(S_0, 0, 0)$, where $S_0 = \Lambda/\mu$ (notations have been discussed in Section \ref{Sec_Epi_model_Review}) and endemic equilibrium point is $(S^*, I^*, R^*)$ where
\begin{align}\label{steady_state}
    S^* = \dfrac{\Lambda \alpha + (\gamma + \mu)}{\alpha\mu + \beta}, \quad I^* = \dfrac{\mu(\mathcal{R}_0 - 1)}{\alpha\mu + \beta}, \quad R^* = \dfrac{\gamma(\mathcal{R}_0 - 1)}{\alpha\mu + \beta}. 
\end{align}
Thus, the endemic equilibrium exists if  $\mathcal{R}_0 > 1$, where the basic reproduction number  ($\mathcal{R}_0$) is given by $ \mathcal{R}_0 =\dfrac{\beta \Lambda}{\mu(\gamma+\mu)}$. From an epidemiological perspective, $\mathcal{R}_0$ is the expected number of secondary cases generated by a typical infected individual throughout his infectious interval in a fully susceptible population. Mathematically, the progression of a disease is typically characterized by $\mathcal{R}_0$, such that if $\mathcal{R}_0 < 1$, the disease-free equilibrium is stable, and the disease will eventually die out. Conversely, if $\mathcal{R}_0 > 1$, the disease-free equilibrium becomes unstable, and the system transitions towards the endemic equilibrium, indicating the sustained presence of the disease in the population. Therefore, $\mathcal{R}_0$ determines the threshold quantity for investigating the asymptotic stability of the equilibrium states and the prediction value needed for disease eradication. To determine the equilibria states and analyze the stability of these states, we introduce the following results by \cite{tian2020global}:
\begin{lemma}[Green's Formula]\label{Green}
Let $V$ be finite and  any two functions $f, g : V  \to \mathbb{R}$, we have the following result
\begin{align*}
    \sum_{x \in V} f(x) \Delta g(x) = - \frac{1}{2} \sum_{x, y \in V} \left(f(y) - f(x)\right) \left( g(y) - g(x) \right).
\end{align*}
\end{lemma}
%\begin{proof}
\noindent The proof of the lemma is given in the Lemma 2.1 of \cite{tian2020global}.\\[3ex]
\noindent Assume $u(t) \geq 0, $ when $t \in [0, \infty)$, that satisfies the following equation
\begin{align}\label{x_eqn}
    \dot u = \dfrac{a u}{1+\alpha u} - c u, \quad \mbox{ for } t \in [0, \infty),
\end{align}
where $a, \alpha$, and $c$ are positive constants. Also, note that Eq. \eqref{x_eqn} has a trivial equilibrium at $u_0 = 0$. If $a>c$, then the system in Eq. \eqref{x_eqn} has a unique positive non-trivial equilibrium, $u^* = \dfrac{a-c}{\alpha c}$.

\begin{lemma} \label{lim_glo}
If $a < c$, the trivial equilibrium, $u_0$, of Eq. \eqref{x_eqn} is globally asymptotically stable and if $a > c$, the non-trivial equilibrium, $u^* = \dfrac{a-c}{\alpha c}$, of the Eq. \eqref{x_eqn} is globally asymptotically stable.
\end{lemma}
\noindent The proof of the lemma is given in the Appendix \ref{proof_lim_glo}.

\begin{lemma} \label{supinf}
Consider $\sigma, \kappa,$ and $\nu$ are positive constants. Let for each  $x \in V$, $u(x, \cdot) \in \mathbb{C}([0, \infty))$ is differentiable in $(0, \infty)$, where $\mathbb{C}([0, \infty))$ is the space of continuous functions on $[0, \infty)$. If $u(x, t), \mbox{ where } (x, t) \in V \times (0, \infty),$ satisfies
\begin{equation}
\begin{aligned}
       &\dfrac{\partial u}{\partial t} - \sigma \Delta u \geq ( \leq )\,\, \kappa - \nu u,
\end{aligned}
\end{equation}
with the initial condition $u(x, 0) = u_{0}(x) \geq 0, x \in V$, then $  \liminf\limits_{t \to \infty} u(x, t) \geq \frac{\kappa}{\nu} \, \Big( \limsup\limits_{t \to \infty}u(x, t) \leq  \frac{\kappa}{\nu} \Big)$ uniformly in $x\in V$. Furthermore, for any given sufficiently small positive real number $\epsilon$, there exists a positive $t_\epsilon $, such that 
\begin{equation}\label{rel1}
    u(x, t) > \frac{\kappa}{\nu} - \epsilon\, \Big( u(x, t) < \frac{\kappa}{\nu} + \epsilon \Big), \, \mbox{ for } t \in (t_\epsilon, \infty).
\end{equation}


\end{lemma}
\noindent The proof of the lemma is given in Lemma 2.4 of  \cite{tian2020global}.

\begin{lemma} \label{frac_supinf}
Consider  $\alpha,  \sigma, \kappa,$ and $\nu$ are positive constants. Let for each  $x \in V$, $u(x, \cdot) \in \mathbb{
C}([0, \infty))$ is differentiable in $(0, \infty)$. If $u(x, t), \mbox{ where } (x, t) \in V \times (0, \infty),$ satisfies
\begin{equation}
\begin{aligned}
       &\dfrac{\partial u}{\partial t} - \sigma \Delta u \geq ( \leq )\,\, \dfrac{\kappa u}{1+\alpha u}   - \nu u,
\end{aligned}
\end{equation}
with the initial condition $u(x, 0) = u_{0}(x) \geq 0, x \in V$, then 
$ \liminf\limits_{t \to \infty} u(x, t) \geq l\, \Big( \limsup\limits_{t \to \infty}u(x, t) \leq  l \Big)$
uniformly in $x\in V$. Furthermore, for any given sufficiently small positive real number $\epsilon$, there exists a positive $t_\epsilon $, such that 
$$
    u(x, t) > l - \epsilon\, \Big( u(x, t) < l + \epsilon \Big), \, \mbox{ for } t \in (t_\epsilon, \infty), 
$$
where $l = \begin{cases}
     \dfrac{\kappa - \nu}{\alpha \nu}, &\mbox{ if } \kappa > \nu,\\[0.75ex]
     0, &\mbox{ if } \kappa < \nu.
  \end{cases}$
\end{lemma}
\noindent The proof of the lemma is given in the Appendix \ref{proof_frac_supinf}.
 
\subsection{Stability Analysis}
Stability analysis plays a pivotal role in epidemic modeling by providing critical insights into the dynamics of disease spread and control. It helps determine whether a disease will die out or persist in a population. %In epidemiology, the progression of a disease is typically characterized by the basic reproduction number, $\mathcal{R}_0$.  If $\mathcal{R}_0 < 1$, the Disease-Free Equilibrium (DFE) is stable, and the disease will eventually die out. Conversely, if  $\mathcal{R}_0 > 1$,  the DFE becomes unstable, and the system transitions towards the Endemic Equilibrium (EE), indicating the sustained presence of the disease in the population. 
Stability analysis can be broadly classified into two types: local and global stability \cite{ma2009dynamical, martcheva2015introduction}. Local stability focuses on the behavior of the system in the vicinity of an equilibrium point. It determines whether small perturbations or deviations from this point will decay over time, allowing the system to return to equilibrium. Global stability, on the other hand, examines the system's behavior across its entire state space. It ensures that, regardless of the initial conditions, the system converges to a specific equilibrium point over time.  Given the complexities and high-dimensional nature of many epidemic models, local stability alone may not suffice to ensure the long-term eradication or persistence of disease. Hence, we focus on establishing global stability, which provides a more comprehensive characterization of the system's dynamics.\\[2ex]
\noindent 
To establish the global stability of an equilibrium point, we analyze the global dynamics of the disease-free equilibrium and the endemic equilibrium of the system Eq. \eqref{model} using Green's formula and the Comparison principle. To simplify the analysis, we initially perform model dimension reduction. Since the variable  $R$ does not appear in the first two equations of the system, the last equation is omitted to prove the global dynamics of the system in Eq. \eqref{model}. Thus, we focus on the following subsystem:
\begin{subequations}\label{submodel}
    \begin{align}
    \dfrac{\partial S(x, t)}{\partial t} - \sigma\Delta S(x, t) &= \Lambda - \dfrac{\beta S(x, t) I(x, t)}{1 + \alpha I(x, t)} - \mu S(x, t),\quad S(x ,0) = S_0(x) > 0 \mbox{ for } x \in V,    \label{submodel_a}  \\
     \dfrac{\partial I(x, t)}{\partial t} - \sigma \Delta I(x, t) &= \dfrac{\beta S(x, t) I(x, t)}{1 + \alpha I(x, t)} - (\gamma + \mu)I(x, t), \quad I(x ,0) = I_0(x) \geq 0 \mbox{ for } x \in V.   \label{submodel_b} 
    \end{align}
\end{subequations}
For the model to be epidemiologically meaningful, we show the positivity and boundedness of solutions for all time $t$ as follows:
\begin{lemma}\label{positive1}
The region $\mathbb{D}_+ = \big\lbrace (S, I)\,|\, S \geq 0, I\geq 0 \mbox{ for } x \in \textbf{}V  \big\rbrace$ is a positive invariant set for the model in Eq. \eqref{submodel}. 
\end{lemma}
\noindent The proof of the lemma is given in the Appendix \ref{proof_positive1}.

\begin{lemma}\label{bdd}
If $\left(S(x, t), I(x, t)\right)$ satisfies Lemma \ref{positive1} for all $(x, t) \in V \times (0, \infty)$, then there is a positive real number $K$ such that 
$0 \leq S(x, t), I(x, t) \leq K, \quad \mbox{for all } (x, t) \in V \times (0, \infty)$.

\end{lemma}
\noindent The proof of the lemma is given in Appendix \ref{proof_bdd}.

\vspace{0.2cm}

\noindent The disease-free equilibrium point denotes the point at which there are no infections in the population. On the contrary, the endemic equilibrium state denotes the state where the disease cannot be completely eradicated but persists in the population. To determine the stability of the disease-free equilibrium state and endemic equilibrium state, we have the following theorems:
\begin{theorem}\label{dfe_glo}
The disease-free equilibrium $(S_0, 0)$  of the model Eq. \eqref{submodel} is globally asymptotically stable when the basic reproduction number is less than 1 ($\mathcal{R}_0 < 1$).
 \end{theorem}
 \noindent The proof is given in Appendix \ref{proof_dfe_glo}.
 \begin{theorem}\label{ee_glo}
The endemic equilibrium $(S^*, I^*)$  of the model Eq. \eqref{submodel} is globally asymptotically stable when the basic reproduction number is greater than 1 ($\mathcal{R}_0 > 1$).
 \end{theorem}
\noindent The proof is given in Appendix \ref{proof_ee_glo}.
 
\vspace{0.2cm}

\noindent The global stability analysis ensures that, regardless of initial conditions, the disease will ultimately converge to one of the equilibrium states, such as the disease-free equilibrium or endemic equilibrium. This concept is central to understanding how epidemic models predict the long-term trajectory of disease spread under different intervention strategies. By examining these equilibria, we can assess the effectiveness of interventions in achieving disease control. Building on these epidemic insights, we integrate the epidemiological principles of the networked SIR model into deep learning techniques to generate reliable epidemic forecasts.

\section{Epidemic-Guided Deep Learning Framework}\label{Sec_EGDL_Model}
This section outlines the generic architecture of the epidemic-guided deep learning (EGDL) methods, which seamlessly integrate epidemiological principles with spatial interactions modeled through the networked SIR framework (described in Section \ref{Network_SIR_Model}) with data-driven techniques to forecast the future trajectories of the disease incidence across different geographical locations. Specifically, we introduce two distinct approaches, namely EGDL-Parallel and EGDL-Series, which integrate the scientific knowledge of the disease dynamics and its spatial distribution into deep learning models to forecast TB incidence cases. The parameters of the networked SIR model, including transmission, recovery, saturation, population mobility, and birth/death rates, guide the deep learning architectures to ensure that the TB incidence forecasts align with real-world epidemiological trends and can effectively capture both the spatial and temporal dependencies. A pictorial representation (Fig \ref{Fig_NSIR_EGDL}) and pseudo-code (Algorithm \ref{algo_EGDL_Parallel}) of the EGDL architectures are presented in Appendix \ref{Appendix_EGDL_Algo}. 

\subsection{EGDL-Parallel Method}
We consider the problem of forecasting epidemic incidence cases based on $T$ historical observations recorded across $n$ vulnerable locations. We denote the number of infected individuals observed at time $t$ in location $x$ as $Y(x,t); \; t = 1, 2, \ldots, T,\; x = 1, 2, \ldots, n$. Our goal is to iteratively generate the $q \; (\geq 1)$ step-ahead forecasts of $Y(x,t)$, to be denoted as $\widehat{Y}(x,{T+1}), \widehat{Y}(x,{T+2}), \ldots, \widehat{Y}(x,{T+q})$. A standard approach to this problem is to use a forecasting model $f_{M} \in \mathcal{F}$, where $\mathcal{F}$ represents the set of possible data-driven global forecasting models. In this study, we choose $\mathcal{F}$ to represent two types of data-driven deep learning architectures: stacked-based frameworks, such as NBeats \cite{oreshkin2019n} and NHits \cite{challu2023nhits}, and encoder-decoder-based architectures such as TCN \cite{chen2020probabilistic} and Transformers \cite{vaswani2017attention}. Thus using the historical data $\{Y(1,t), Y(2,t), \ldots Y(n,t); t = 1, 2, \ldots, T\}$ for training $f_{M}$ the future dynamics can be predicted as follows: % (a detailed description of these methods is given in Section \ref{Sec_Phenom_Review})
% $$f_{M}: \left[Y(1,t), \ldots ,Y(n,t)\right] \longrightarrow \widehat{Y}(1,{T+v}), \ldots, \widehat{Y}(n,{T+v}); \; t = 1, 2, \ldots, T; \; v = 1, 2, \ldots, q.$$ 
\begin{equation}\label{Eq_Fore_Data}
    \widehat{Y}\left(x, \tilde{t} + v\right) = f_{M}\left(\left\{Y\left(x, \tilde{t} - t_0 \right)\right\}_{t_0 = 0}^{t_w - 1}\right),
\end{equation}
where $x = 1, 2, \ldots, n, \; \tilde{t} = t_w, t_w+1, \ldots, T, \; v = 1, 2, \ldots, q$, and $t_w$ denotes the number of historical inputs used for predicting the subsequent observations.  Despite their effectiveness, the purely data-driven forecasting techniques $f_{M}$ solely rely on the lagged incidence time series without incorporating the scientific knowledge that governs the disease dynamics. As a result, this approach leads to a simple extrapolation of the temporal trends, which may fail to capture the underlying epidemiological underpinnings of the disease. To address this limitation, the EGDL-Parallel approach leverages the networked SIR model with saturated incidence rate and graph Laplacian diffusion to estimate the infection dynamics curve for the $n$ locations based on various disease drivers such as transmission, recovery, saturation, population mobility, and population death rates. The number of infection cases $I(x,t)$ predicted at timestamp $t$ for the $x^{th}$ location using the networked SIR infection curve captures the epidemic knowledge and models spatial dependencies among the neighboring locations using the graph Laplacian diffusion. However, these estimates provide an incomplete representation of the target variable $Y(x,t)$ due to their inability to track the lagged temporal dependencies, leading to discrepancies with ground truth observations. 

The EGDL-Parallel approach overcomes this issue by integrating the networked SIR infection estimates $I(x,t)$ as auxiliary variables into the data-driven forecasting framework. This hybrid input allows the forecasting model to combine the strengths of both compartmental methods and data-driven techniques to forecast incidence cases. The resulting EGDL-Parallel framework, denoted as $f_{EGDLP}$, is defined as:
% $$ %  $\{Y(1,t), Y(2,t), \ldots, Y(n,t), I(1,t), I(2,t), \ldots, I(n,t); \; t = 1, 2, \ldots, T\}$,
%     f_{EGDLP}: \left[Y(1,t), \ldots, Y(n,t), I(1,t), \ldots, I(n,t)\right] \longrightarrow \widehat{Y}(1,{T+v}), \ldots, \widehat{Y}(n,{T+v}); \; 
% $$
$$
    \widehat{Y}\left(x, \tilde{t}+v\right) = f_{EGDLP}\left(\left\{Y\left(x, \tilde{t} - t_0 \right)\right\}_{t_0 = 0}^{t_w - 1}, \; \left\{I\left(x, \tilde{t} - t_0 \right)\right\}_{t_0 = 0}^{t_w - 1}\right),
$$
where $f_{EGDLP} \in \mathcal{F}$ and notations are consistent with Eq. \eqref{Eq_Fore_Data}.
% where $t = 1, \ldots, T; \; v = 1, \ldots, q,$ and $f_{EGDLP} \in \mathcal{F}$. 
The primary advantage of the $f_{EGDLP}$ model over its data-centric counterpart $f_{M}$ is its ability to integrate disease-specific knowledge, enhancing its generalizability and forecast accuracy. By leveraging the networked SIR model, the $f_{EGDLP}$ framework is guided by epidemiological principles, enabling the generation of realistic future estimates of disease incidence cases. Moreover, it effectively learns the complex spatiotemporal features from the compartmental models and data-driven methods, making it a robust solution for epidemic forecasting. 

\subsection{EGDL-Series Method}
The EGDL-Series framework employs an integrated deterministic-stochastic approach to model and forecast disease incidence cases by combining epidemiological principles with data-driven techniques. % through a residual remodeling strategy, this hybrid methodology captures the overall variations in disease dynamics. Specifically, the framework leverages the strengths of the epidemic networked SIR model to estimate deterministic patterns while utilizing advanced data-driven forecasting techniques to address stochastic variations. This synergy enables EGDL-Series architectures to leverage the strengths of both paradigms and mitigate their respective limitations. The flexibility of this hybrid framework allows it to better adapt to real-world complexities and dynamic disease behaviors, making it a powerful tool for forecasting infectious disease spread. 
In the EGDL-Series method, a deterministic model (networked SIR in this case) is combined with stochastic models such as NBeats \cite{oreshkin2019n}, NHits \cite{challu2023nhits}, Transformers \cite{vaswani2017attention}, and TCN \cite{chen2020probabilistic} using a residual remodeling technique \cite{zhang2003time, chakraborty2019forecasting}. Mathematically, the disease incidence cases recorded at time $t$ for a location $x$ can be expressed as:
$ Y(x, t) = \mathcal{D}(x, t) + \mathcal{\tilde{S}}(x, t),$ 
where $\mathcal{D}(x, t)$ represent the deterministic component and $\mathcal{\tilde{S}}(x, t)$ accounts for the stochastic variations in $Y(x, t)$. To forecast future disease trajectories, the EGDL-Series framework first utilizes the networked SIR model with saturated incidence rate and graph Laplacian diffusion to estimate the infection dynamics $I(x,t)$. This curve, determined based on key epidemiological factors, such as disease transmission, recovery, saturation, population mobility, and birth/death rates, approximates the deterministic components $\mathcal{D}(x, t)$ of $Y(x, t)$. The predicted infection trajectory, $\widehat{\mathcal{D}}(x, t) \equiv I(x,t)$, captures structural variations in the disease incidence by modeling epidemic principles and spatial interactions among neighboring locations. The deviations between observed incidences and the predictions of the networked SIR model, defined as residuals $e(x,t) = Y(x,t) - I(x,t)$, capture the stochastic variations that the deterministic part fails to account for. These residuals store nonlinearities, temporal dependencies, and unexplained stochastic dynamics that the networked SIR framework cannot model. The residual series is modeled using data-driven forecasting techniques, denoted as $f_{EGDLS} \in \mathcal{F}$ to address this. The $q$-step-ahead forecasts of the residual series generated by the deep learning-based temporal forecasters are combined with the predictions of the networked SIR model to forecast the disease incidence as follows:
% $$
% \widehat{Y}(x, T+v) = I(x, T+v) + \hat{e}(x, T+v);
% $$
% where $\hat{e}(x, T+v)$ is generated using $f_{EGDLS}$ as 
\begin{equation*}
    \widehat{Y}(x, \tilde{t}+v) = I(x, \tilde{t}+v) + \hat{e}(x, \tilde{t}+v), 
\end{equation*}
with 
$
    \hat{e}(x, \tilde{t}+v) = f_{EGDLS}\left(\left\{e\left(x, \tilde{t} - t_0 \right)\right\}_{t_0 = 0}^{t_w - 1}\right),
$
% $$
% f_{EGDLS}: \left[e(1, t), e(2,t), \ldots, e(n,t)\right] \longrightarrow \hat{e}(1,T+v),  \hat{e}(2,T+v), \ldots, \hat{e}(n,T+v);
% $$
notations are consistent with Eq. \eqref{Eq_Fore_Data}. The proposed formulation ensures that the model aligns with the epidemic principles coupled with spatial interactions while simultaneously addressing the unexplained variations of the compartmental model. The EGDL-Series approach offers enhanced flexibility, greater insight into disease dynamics, and improved forecasting accuracy by integrating compartmental and data-driven components. 

\section{Application to TB study}\label{Sec_forecast}
In this section, we assess the efficiency of the EGDL-Parallel and EGDL-Series frameworks by comparing their TB incidence forecasting performance across Japan's 47 prefectures with state-of-the-art architectures. To evaluate the robustness and generalizability of the EGDL-based approaches, we employ rolling window forecast horizons of 12 months, 9 months, 6 months, and 3 months, performing multi-step ahead forecasts for each respective horizon. For the 12-month evaluation period, the models are trained using incidence data recorded from January 1998 to December 2014, while the remaining observations are reserved for testing. Similarly, for the 9-month, 6-month, and 3-month horizons, training data spans from January 1998 to March 2015, June 2015, and September 2015, respectively, with the remaining data used for evaluation. Since EGDL is an integrated approach of the networked SIR model with a saturated incidence rate and modern deep learning-based time series forecasting methods, therefore, we first delve into the parameter estimation procedure for the former parts. 
%The following subsections provide a detailed analysis of the global characteristics of the TB incidence data (Section \ref{Sec_Global_Char_Sec}), estimation of the posterior distribution of the networked SIR model parameters (Section \ref{Bayesian_Inf_Sec}), the causal relationship between observed TB incidence cases and the estimated infected curve of the networked SIR model (Section \ref{Sec_Causal_Analysis}), performance indicators used for evaluation (Section \ref{Sec_Performance_measures}), comprehensive summary of the experimental results (Section \ref{Sec_Fore_Results}), statistical significance of the experimental results (Section \ref{Sec_Stat_SIgnif}), and uncertainty quantification of the proposed EGDL approaches (Section \ref{Sec_Uncertain_Quant}).

\subsection{TB data characteristics}\label{Sec_Global_Char_Sec}
The monthly total number of active TB cases recorded across the 47 prefectures of Japan is devoid of any missing observations and demonstrates rapid variation in incidence cases. The average monthly incidence ranges from 9.49 to 294.84 cases across different prefectures. We checked several global characteristics for spatiotemporal TB datasets and reported the numerical results in Table \ref{Table_Global_Features} of Appendix \ref{Appendix_Global_Char}. The results of the statistical tests reveal that the TB incidence time series for all 47 prefectures of Japan are non-stationary and exhibit long-term dependency, as indicated by Hurst exponents greater than 0.50. Additionally, most series display non-linear patterns with evident quarterly and annual seasonality. Furthermore, we explore the spatial relationships among prefectures by analyzing pairwise correlations in TB incidence. Fig. \ref{Fig_Correlation_LAP47}(b) (Appendix \ref{Appendix_Global_Char}) presents a heatmap of these correlations, showing stronger associations between geographically proximate prefectures. For example, prefecture 10 (Gunma) demonstrates high positive correlations with its neighboring regions, while prefecture 47 (Okinawa) exhibits weaker correlations due to its geographic isolation. Diagonal entries in the heatmap represent self-correlations, which equals 1. Some distant prefectures (e.g., prefectures 3 and 20) exhibit moderate to strong correlations, likely due to shared non-local influences. These findings underscore the simultaneous spatial and temporal dependencies inherent in TB incidence cases in Japan, emphasizing the importance of considering both dimensions when modeling and forecasting disease dynamics.

\subsection{Parameter Calibration}\label{numerical}
\noindent To numerically solve the networked SIR model with saturated incidence rate and graph Laplacian diffusion Eq. \eqref{submodel}, we transform it into a system of ODEs using Eq. \eqref{hfun}. This process requires the Laplacian matrix used in our study to solve the model numerically. The Laplacian matrix \( L \) is defined by $ L = D - A,$
where \( D \) is the degree matrix and \( A \) is the adjacency matrix. The degree matrix \( D \) is a diagonal matrix with the entries \( D_{ii} \), the degree of node \( i \), representing the number of edges incident to node \( i \) i.e., 
\(D_{ii} = \text{deg}(v_i)\). The adjacency matrix \( A \) with entries \( A_{ij} \) indicates the link between nodes \( i \) and \( j \) as follows:
\begin{equation}\label{eq_Aij}
   A_{ij} =
  \begin{cases}
    1, & \text{if there is a link between nodes } i \text{ and } j \\
    0, & \text{otherwise}
  \end{cases} 
\end{equation}
Thus, the diagonal elements of \( L \) 
%in Eq. \eqref{eq_L} 
are the degrees of the nodes, while the off-diagonal elements \( L_{ij} \) are -1 if nodes \( i \) and \( j \) are connected by an edge, and 0 otherwise:
\begin{equation}\label{eq_Lij}
    L_{ij} =
  \begin{cases}
    \text{deg}(v_i), & \text{if } i = j \\
    -1, & \text{if } i \neq j \text{ and } (i, j) \in \mathcal{E} \\
    0, & \text{if } i \neq j \text{ and } (i, j) \notin \mathcal{E}.
  \end{cases}
\end{equation}
To construct the Laplacian matrix across different prefectures of Japan, we assume a connection between two prefectures if they share a common boundary. For the prefectures of Hokkaido, Tokushima, Kagawa, Ehime, and Okinawa, which are geographically separated from the mainland (see Fig. \ref{fig:TB47}(A)), we establish connections to the nearest mainland prefecture. The structure of these connections is illustrated in Fig. \ref{Fig_Correlation_LAP47}(a) of Appendix \ref{Appendix_Global_Char}. Now, for each node $k \in V$, we rewrite the model in Eq. \eqref{submodel} as follows:
\begin{equation}\label{num_model}
    \begin{aligned}
            \dfrac{d S_k}{d t} =  \sigma\sum_{j=1}^n {L}_{k\,j}S_j + \Lambda - \dfrac{\beta S_k I_k}{1 + \alpha I_k} - \mu S_k, \quad\, S(k, t) = S_k(t), S(k, 0) = S_{k}(0) \mbox{ for } k \in V,\\ 
    \dfrac{d I_k}{d t} =  \sigma \sum_{j=1}^n {L}_{k\,j}I_j + \dfrac{\beta S_k I_k}{1 + \alpha I_k} - (\gamma + \mu)I_k,\quad  I(k, t) = I_{k}(t), I(k, 0) = I_{k}(0) \mbox{ for } k \in V.
\end{aligned}
\end{equation}












Using the Laplacian matrix (see Fig. \ref{Fig_Correlation_LAP47}(a) of Appendix \ref{Appendix_Global_Char}), we numerically solve the system Eq. \eqref{num_model}, assuming that prefecture 20 (Nagano) serves as the source of infection. Additionally, to support theoretical findings, we consider an initial total population of 47,000 distributed evenly across the 47 prefectures, with each prefecture having a population of 1,000. The initial conditions are defined as follows:
\begin{equation}
\left\lbrace
    \begin{aligned}
    I_{20}(0) & = 10.0, \quad I_1(0) = \cdots = I_{19}(0) = I_{21}(0) = \cdots = I_{47}(0) = 0.0\\
        S_{20}(0) &= 990.0, \quad  \quad S_1(0) = \cdots = S_{19}(0) = S_{21}(0) = \cdots = S_{47}(0) = 1000.0.
    \end{aligned}
    \right.
\end{equation}
Further, we set the model parameters as follows: 
\begin{equation}\label{dis-para}
\left\lbrace
   \begin{aligned}
    \Lambda &= 10.0 \mbox{ day}^{-1},\, \mu = 0.01 \mbox{ day}^{-1},\, \beta = 0.0001\mbox{ day}^{-1},\\
    \gamma &= 0.25\mbox{ day}^{-1},\,\alpha = 0.5,\, \sigma = 0.75. 
\end{aligned} 
\right.
\end{equation}
Now, we calculate the basic reproduction number, $\mathcal{R}_0  = 0.385$, that is less than unity, according to Theorem \ref{dfe_glo} and the disease-free equilibrium $(\Lambda/\mu, 0) = (100.0,0)$ point is globally asymptotically stable (see  Fig \ref{fig:stab}(a-b) of Appendix \ref{App_Param_Calli}). In Fig \ref{fig:stab}(a), all $S$ solutions converge to $\Lambda/\mu$ regardless of the node, similarly, in Fig \ref{fig:stab}(b), all  $I$ solutions converge to $0$ regardless of the node.  These results can be similarly demonstrated, independent of the initial conditions. For the endemic equilibrium, we use all the parameters from Eq. \eqref{dis-para} except $\beta$ and $\sigma$ which are set to $\beta = 0.001$ and $\sigma = 10^{-5}$. Direct calculation gives: $\mathcal{R}_0 = 3.85, S^* = 876.67, I^* = 7.74$. Since $\mathcal{R}_0  = 3.85$ is greater than unity, then by Theorem \ref{ee_glo} endemic equilibrium $(S^*, I^*) = (876.67, 7.74)$  point is globally asymptotically stable as shown in Fig \ref{fig:stab}(c-d). In Fig \ref{fig:stab}(c), all $S$ solutions converge to $S^*$ regardless of the node; similarly, in Fig \ref{fig:stab}(d), all $I$ solutions converge to $I^*$ regardless of the node. The saturation parameter \(\alpha\) significantly influences the solution of the system, as illustrated in Fig. \ref{fig:alpha_var} of Appendix \ref{App_Param_Calli}. For this analysis, we use all the parameters associated with the endemic equilibrium case, except for \(\alpha\), which is varied between 0.1 and 1.0. At lower values of \(\alpha\), the contact rate increases, resulting in a higher peak of infections. As \(\alpha\) approaches 1.0, the contact rate saturates, and the infection profile becomes more uniform over time.


\subsection{Sensitivity Analsyis}\label{Bayesian_Inf_Sec}
In this part, we estimate the posterior distributions of the networked SIR model parameters using the Markov-Chain Monte-Carlo (MCMC) approach on the Laplacian network (depicted in Fig. \ref{Fig_Correlation_LAP47}(a) of Appendix \ref{Appendix_Global_Char}) via the \texttt{PyMC} Python library \cite{abril2023pymc}. To get posterior distributions of the unknown parameters, the state variables $\hat S_i(t),\hat I_i(t)$ are observed at $T$ certain times $t_1, t_2,\dots,t_T$.
%  such that
% \begin{align*}
%     \hat S_i(t_j) &= S_i(t_j) + \epsilon_j^{S_i},\\
%     \hat I_i(t_j) &= I_i(t_j) + \epsilon_j^{I_i}, \quad \mbox{ where } j = 1,2,\dots, T;\,\, i=1,2,\dots,M,
% \end{align*}
% and $\epsilon_j^{S_i}$ and $\epsilon_j^{I_i}$ are noises related to  susceptible individual and i-th infected individual at node i.
Let, $\theta = \lbrace \alpha, \beta, \gamma, \sigma, \mu \rbrace$ be the unknown parameter and $D_i = \lbrace \hat S_i, \hat I_i \rbrace$ be observed data at node $i$.
% then by Bayes' theorem:
% \begin{align}\label{eq_post}
%   \textbf{P}(\theta\,|\, D_i)  = \dfrac{\textbf{P}(D_i|\,\theta\ ) \textbf{P}(\theta)}{\textbf{P}(D_i)} \propto\textbf{P}(D_i|\,\theta) \textbf{P}(\theta) \,\, \mbox{ for each node }i, 
% \end{align}
% where $\textbf{P}(\theta\,|\, D_i)$ posterior distribution of $\theta$ given observed data, $\textbf{P}(D_i|\,\theta)$ is the likelihood of observing data at $i^{th}$ node given $\theta$, and $\textbf{P}(\theta)$ describes the prior distribution of $\theta$. %and $\textbf{P}(D_i)$ is the prior predictive distribution equation which is also the normalizing constant of the posterior distribution for node $i$ such that 
% \begin{equation}\label{eq_post_data}
%     \textbf{P}(D_i) = \int\limits_{\mathcal{G}_i} \textbf{P}(D_i\,|\, \theta) \textbf{P}(\theta)\, d\theta,
% \end{equation}
%where $\mathcal{G}_i$ is the domain of the parameter $\theta$. 
It is not always possible to find sample $\theta$ from the posterior probability distribution $\textbf{P}(\theta\,|\, D_i)$. This is where the iterative MCMC algorithm plays a crucial role. It generates a new vector parameter at step $\hat{n}$ ($\theta^{\hat{n}}$), from the posterior distribution, using the previous vector parameter ($\theta^{\hat{n} -1}$), with initial guess $\theta^0$. The chain or sample-path run $\hat{n}$ steps until it reaches its stationary distribution, where $\hat{n}$ is a sufficiently large number, and fills the space of the target un-normalized posterior distribution. For implementing Bayesian inference with MCMC, we used the No-U-Turn Sampler (NUTS) \cite{hoffman2014no}.
%\subsection{No-U-Turn Sampler(NUTS)}
% Hamiltonian Monte Carlo (HMC) is one of the most important MCMC algorithms, extending the Metropolis algorithm by overcoming random walk behavior and sensitivity to the correlated parameters. This advantage is particularly beneficial in high dimensional parameter target distributions, where HMC, a gradient-based method, walks faster in the parameter domain by including momentum variables, leading to faster convergence and improved mixing. However, HMC has two highly sensitive tuning parameters, namely the discretization step size ($\epsilon$) and the number of steps ($\mathcal{L}$). Selecting the optimal values for these parameters is crucial, since larger values of $\mathcal{L}$ increase the computational cost, whereas a smaller value of $\mathcal{L}$ leads to undesirable random walk behavior with the parameter space. To address this challenge, 
The NUTS approach automatically selects these parameters, making it a tuning-free sampling algorithm that performs similar or better than other MCMC algorithms \cite{hoffman2014no}. Before performing inference, we estimate the parameter set $\theta$, denoted as $\theta_{ls} \equiv \lbrace \alpha_{ls}, \beta_{ls}, \gamma_{ls}, \sigma_{ls}, \mu_{ls} \rbrace$, using the ordinary least squares (OLS) method with TB dataset from 47 prefectures in Japan. Based on these estimates, we define the prior distribution information for the parameters as follows:
\begin{equation}\label{prior}
\left\lbrace
    \begin{aligned}
    \alpha &\sim \mbox{TruncatedNormal}(  \texttt{mean} = \alpha_{ls}, \texttt{SD} =   0.1,  \texttt{lower}=0, \texttt{initval}=\alpha_{ls})\\
    \beta &\sim \mbox{TruncatedNormal}( \texttt{mean} = \beta_{ls}, \texttt{SD} = 0.01, \texttt{lower}=0, \texttt{initval}=\beta_{ls})\\
    \gamma &\sim \mbox{TruncatedNormal}(\texttt{mean} = \gamma_{ls}, \texttt{SD} = 0.1, \texttt{lower}=0, \texttt{initval}=\gamma_{ls})\\
     \sigma &\sim \mbox{TruncatedNormal}(\texttt{mean} = \sigma_{ls}, \texttt{SD} = 0.3, \texttt{lower}=0, \texttt{initval}=\sigma_{ls})\\
     \mu &\sim \mbox{TruncatedNormal}(\texttt{mean} = \mu_{ls}, \texttt{SD} = 0.01, \texttt{lower}=0, \texttt{initval}=\mu_{ls})\\
    \eta &\sim \mbox{HalfNormal}(10),
\end{aligned}
\right.
\end{equation}
where $\eta$ is the likelihood distribution. We selected the `TruncatedNormal' distribution to ensure the non-negativity of the parameters, aligning with epidemiological considerations.  For each MCMC step $\hat{n}$, NUTS choose the new parameter $\theta^{\hat{n}}$  in such a way that the set of parameters minimizes the previous deviation. Each time $t$, integration with its model parameters produces one Monte Carlo sample, and iteratively, MCMC will create the full set of required samples. The summary of the MCMC results is provided in Table \ref{tab:mcmc}. Now, using the values of the estimated parameters for the networked SIR model, we generate the predictions for the infected curve which will be further used in the EGDL framework. 
\begin{table}[!ht]
    \caption{Summary of the MCMC results with the prior information Eq. \eqref{prior}. Here, Standard Deviation (SD), Highest Density Interval (HDI), Monte Carlo Standard Error (MCSE), Effective Sample Size (ESS), and Potential Scale Reduction Factor (R\_hat) are reported.}
    \label{tab:mcmc}
    \centering
    \setlength\tabcolsep{3pt}   
    \scriptsize{
    % \resizebox{0.99\textwidth}{!}{
    \begin{tabular}{|l|l|l|l|l|l|l|l|l|l|}
    \hline
        &Mean & SD& HDI-3\% & HDI-97\% & MCSE Mean & MCSE SD & ESS Bulk & ESS Tail & R\_hat  \\ \hline
        $\alpha$ & 0.04112 & 0.00116 & 0.03888 & 0.04324 & 0.00014 & 0.00010 & 64.43525& 138.42481& 1.00793 \\ \hline
        $\beta$ & 0.00016 & 5.66E-06 & 0.00014 & 0.00017 & 8.41E-07 & 5.99E-07 & 45.53547 & 52.76396 & 1.01371 \\ \hline
       $ \gamma$ & 0.10632 & 0.00179& 0.10320 & 0.10973 & 0.00023 & 0.00016 & 62.12841 & 219.92519 & 1.01791 \\ \hline
        $\sigma$ & 3.33E-06 & 3.21E-06 & 1.10E-09 & 8.95E-06 & 6.01E-08 & 4.38E-08 & 3487.84068 & 3684.22854 & 1.00098 \\ \hline
        $\mu$ & 5.17E-05 & 5.24E-05 & 6.80E-09 & 0.00015 & 2.11E-06 & 1.49E-06 & 726.51966 & 1243.31641 & 1.00087 \\ \hline
        $\eta$ & 23.78525 & 0.17405 & 23.47343 & 24.11068 & 0.00362 & 0.00256 & 2321.03203 & 3599.16818 & 1.00046 \\ \hline
    \end{tabular}}
\end{table}


\subsection{Forecasting Performance Evaluation}\label{Sec_Fore_Results}
This section first examines the causal relationship between TB incidence cases and the infected curve of the networked SIR model (see Appendix \ref{Appendix_Causal} for details). The results show a strong causal relationship between TB incidence cases and the infection dynamic curve. This finding supports the integration of historical incidence data and the networked SIR model’s infection curve in the EGDL architectures, leveraging the complementary strengths of both approaches to improve TB incidence forecasting. Furthermore, to evaluate the performance of the forecasting approaches, we utilize four key performance indicators, namely Symmetric Mean Absolute Percent Error (SMAPE), Mean Absolute Error (MAE), Mean Absolute Scaled Error (MASE), and Root Mean Squared Error (RMSE) \cite{hyndman2018forecasting} (details in Appendix \ref{Sec_Performance_measures}). By convention, the lower values of these metrics indicate better performance. In this study, we report the average performance metrics computed across the 47 prefectures of Japan for various forecast horizons. Using these four performance measures, we evaluate the performance of the EGDL-Parallel and EGDL-Series frameworks for forecasting TB incidence cases. While these frameworks can be flexibly integrated with any forecasting model, our analysis employs state-of-the-art global architectures, including Transformers \cite{wu2020deep}, NBeats \cite{oreshkin2019n}, NHits \cite{challu2023nhits}, and TCN \cite{chen2020probabilistic}. These models are specifically designed for time series forecasting and are well-suited to multivariate setups, which is challenging for many traditional statistical methods. %Furthermore, their performance remains consistent across varying sample sizes, demonstrating robustness to both large and small datasets. 
By integrating the EGDL-Parallel and EGDL-Series frameworks with the Transformers architecture, we develop the epidemic-guided parallel Transformers (EGP-Transformers) and epidemic-guided series Transformers (EGS-Transformers), respectively. Similarly, integrating NBeats with these frameworks results in the epidemic-guided parallel NBeats (EGP-NBeats) and epidemic-guided series NBeats (EGS-NBeats), respectively. Following the same methodology, combining EGDL-Parallel and EGDL-Series with NHits and TCN generates EGP-NHits, EGS-NHits, EGP-TCN, and EGS-TCN models, respectively. These newly developed forecasting models effectively learn temporal patterns from lagged values while incorporating epidemic dynamics through the estimated infection curves by taking the mean of the model parameters from Table \ref{tab:mcmc} using Bayesian inference with the MCMC approach from the networked SIR model. This integration ensures that epidemic principles, including spatial interactions, guide the forecasters to predict realistic future disease trends. To assess the impact of incorporating these epidemic principles, we compare the performance of the epidemic-guided models with their corresponding baseline architectures. Additionally, we include three spatiotemporal forecasting techniques, namely GSTAR \cite{imro2023determination}, GpGp \cite{senanayake2016predicting}, and STGCN \cite{wang2022causalgnn}, which can inherently model spatial and temporal dependencies in the disease incidence cases. 

Tables \ref{Table_infected_exogenous} and \ref{Table_error_remodel} summarize the forecasting performance of the EGDL-Parallel and EGDL-Series frameworks alongside baseline models for predicting TB incidence in Japan's 47 prefectures across various forecast horizons. As shown in Table \ref{Table_infected_exogenous}, the EGP-NHits model consistently delivers superior performance in most forecasting tasks. The integration of epidemic knowledge, specifically the networked SIR model's estimated infection curve, enables the EGP-NHits framework to produce reliable forecasts, significantly enhancing its accuracy compared to its baseline architecture. Similarly, the EGP-Transformers model demonstrates notably higher accuracy than the individual Transformers and networked SIR models. The EGP-NBeats and EGP-TCN architectures also show marked improvements over their respective baselines, further justifying the advantage of the combined forecasting techniques. Additionally, Table \ref{Table_error_remodel} reveals the consistent superiority of the EGDL-Series frameworks over competing models across various forecast horizons. For the 12-month forecast period, the EGS-NBeats, EGS-TCN, and EGS-NHits frameworks demonstrate the best performance across multiple indicators. For the 9-month horizon, the EGS-TCN model provides the most accurate out-of-sample predictions, followed closely by the EGS-NHits architecture. Similar trends are evident for the 6-month forecast horizon, where EGS-NHits and EGS-TCN outperform all competing models. In short-term forecasting over 3 months, the spatiotemporal GSTAR and GpGp models offer competitive performance in terms of MAE and RMSE, while the EGS-Transformers and NHits models deliver the lowest MASE and SMAPE values, respectively. Overall, the combination of deterministic and stochastic components, achieved through residual remodeling in the EGDL-Series frameworks, significantly improves the performance of individual deep learning models and the networked SIR model across most forecast horizons. Furthermore, we employ the non-parametric multiple comparisons with the best (MCB) test to validate the statistical significance of the performance improvements \cite{edwards1983multiple}. The results of the MCB test (see Appendix \ref{Sec_Stat_Signif}) showcase that all forms of EGDL architectures improve the performance of the corresponding individual models due to their unique merit in integrating epidemiological principles, spatial dynamics, and temporal interactions of TB datasets. More particularly, the performance of the data-centric approaches improved by employing the residual remodeling mechanism of EGDL-Series frameworks. Overall, the EGDL-Series architectures are better suited for medium-term (9-month and 12-month) forecasting tasks, whereas the EGDL-Parallel framework provides more accurate short-term (up to semi-annual) forecasts of TB incidence. Additionally, we quantify the uncertainties associated with the EGDL frameworks (Fig. \ref{Fig_Conformal_prediction_EGDLP}) using the conformal prediction approach (see Appendix \ref{Sec_Uncertain_Quant}). The overall empirical results and prediction intervals provide significant evidence for the practical applicability of EGDL architectures in designing timely and targeted public health intervention strategies. 

\begin{table}[!ht]
\centering
\caption{Forecast performance of spatiotemporal models, temporal deep learners, and Epidemic-Guided Deep Learning (EGDL)-Parallel approach for different forecast horizons. The best results are \underline{\textbf{highlighted}}.}
\label{Table_infected_exogenous}
\setlength\tabcolsep{2.5pt}
\tiny{
% \resizebox{0.99\textwidth}{!}{
\begin{tabular}{|c|c|cccc|cccc|cccc|}
   \hline \multirow{3}{*}{Horizon}&  \multirow{3}{*}{Metric}     & \multicolumn{4}{c|}{Spatiotemporal Models}& \multicolumn{4}{c|}{Temporal Models} & \multicolumn{4}{c|}{EGDL-Parallel Models}\\
   &       & \multirow{2}{*}{GSTAR} & \multirow{2}{*}{GpGp}   & \multirow{2}{*}{STGCN}  & Networked    & Trans- & \multirow{2}{*}{NBeats} & \multirow{2}{*}{NHits}  & \multirow{2}{*}{TCN}    & EGP-Trans- & \multirow{2}{*}{EGP-NBeats} & \multirow{2}{*}{EGP-NHits}  & \multirow{2}{*}{EGP-TCN}    \\
   &  & & & & SIR & formers & & & & formers & & & \\ \hline
   
\multirow{4}{*}{12-month} & SMAPE & 24.746         & 29.875 & 39.432 & 45.805 & 49.296       & 24.631 & 23.339 & 32.073 & 49.170       & 24.330 & \underline{\textbf{23.337}} & 30.896 \\
   & MAE   & 5.640          & 7.428  & 13.751 & 8.920  & 17.723       & 5.302  & 5.116  & 6.994  & 17.678       & 5.246  & \underline{\textbf{5.032}}  & 5.920  \\
   & MASE  & 0.979          & 1.484  & 2.067  & 1.608  & 2.316        & 0.947  & 0.901  & 1.207  & 2.309        & 0.931  & \underline{\textbf{0.878}}  & 1.132  \\
   & RMSE  & 6.835          & 8.854  & 15.351 & 10.215 & 18.674       & 6.565  & 6.284  & 8.224  & 18.636       & 6.521  & \underline{\textbf{6.137}}  & 7.067  \\ \hline
\multirow{4}{*}{9-month}  & SMAPE & 25.719         & 27.747 & 43.612 & 45.359 & 48.903       & 26.962 & 22.859 & 33.656 & 48.893       & 24.819 & \underline{\textbf{22.077}} & 26.790 \\
   & MAE   & 5.687          & 6.268  & 15.942 & 8.835  & 17.961       & 5.592  & 5.084  & 7.173  & 17.946       & 5.286  & \underline{\textbf{5.023}}  & 5.726  \\
   & MASE  & 1.019          & 1.240  & 2.616  & 1.753  & 2.524        & 1.065  & 0.931  & 1.325  & 2.505        & 0.994  & \underline{\textbf{0.892}}  & 1.138  \\
   & RMSE  & 6.954          & 7.617  & 16.920 & 9.963  & 18.877       & 6.686  & 6.198  & 8.338  & 18.851       & 6.399  & \underline{\textbf{6.080}}  & 6.987  \\ \hline
\multirow{4}{*}{6-month}  & SMAPE & 25.610         & 27.584 & 42.650 & 45.841 & 48.062       & 25.203 & 22.682 & 32.459 & 46.737       & 24.857 & \underline{\textbf{22.464}} & 25.644 \\
   & MAE   & 6.379          & 6.499  & 15.733 & 8.383  & 17.633       & 5.246  & 5.486  & 7.171  & 17.319       & 5.207  & \underline{\textbf{5.105}}  & 5.413  \\
   & MASE  & 1.178          & 1.283  & 2.923  & 1.915  & 2.783        & 1.119  & 0.989  & 1.406  & 2.687        & 1.096  & \underline{\textbf{0.987}}  & 1.105  \\
   & RMSE  & 7.375          & 7.609  & 16.705 & 9.347  & 18.400       & 6.157  & 6.386  & 8.371  & 18.080       & 6.338  & \underline{\textbf{6.012}}  & 6.287  \\ \hline 
\multirow{4}{*}{3-month}  & SMAPE & 24.533         & 25.467 & 36.300 & 45.835 & 50.266       & 28.612 & 24.292 & 31.572 & 48.540       & 26.716 & \underline{\textbf{24.070}} & 27.403 \\
   & MAE   & 5.012          & \underline{\textbf{4.857}}  & 11.838 & 8.032  & 17.511       & 5.629  & 5.167  & 6.983  & 17.140       & 5.417  & \underline{\textbf{4.857}}  & 5.758  \\
   & MASE  & 1.329          & 1.485  & 3.068  & 2.562  & 3.757        & 1.672  & 1.301  & 1.651  & 3.577        & 1.361  & \underline{\textbf{1.083}}  & 1.546  \\
   & RMSE  & 5.558          & 5.576  & 12.664 & 8.703  & 18.158       & 6.328  & 5.854  & 7.885  & 17.785       & 6.077  & \underline{\textbf{5.511}}  & 6.416 \\ \hline
\end{tabular}}
\end{table}

\begin{table}[]
\centering
\caption{Forecast performance of spatiotemporal models, temporal deep learners, and Epidemic-Guided Deep Learning (EGDL)-Series approach for different forecast horizons. The best results are \underline{\textbf{highlighted}}.}
\label{Table_error_remodel}
\setlength\tabcolsep{2.5pt}
\tiny{
% \resizebox{0.99\textwidth}{!}{
\begin{tabular}{|c|c|cccc|cccc|cccc|}
   \hline \multirow{3}{*}{Horizon}&  \multirow{3}{*}{Metric}     & \multicolumn{4}{c|}{Spatiotemporal Models}& \multicolumn{4}{c|}{Temporal Models} & \multicolumn{4}{c|}{EGDL-Series Models}\\
   &       & \multirow{2}{*}{GSTAR} & \multirow{2}{*}{GpGp}   & \multirow{2}{*}{STGCN}  & Networked    & Trans- & \multirow{2}{*}{NBeats} & \multirow{2}{*}{NHits}  & \multirow{2}{*}{TCN}    & EGS-Trans- & \multirow{2}{*}{EGS-NBeats} & \multirow{2}{*}{EGS-NHits}  & \multirow{2}{*}{EGS-TCN}    \\
   &  & & & & SIR & formers & & & & formers & & & \\ \hline
   
\multirow{4}{*}{12-month} & SMAPE & 24.746         & 29.875 & 39.432 & 45.805 & 49.296       & 24.631 & 23.339 & 32.073 & 24.560           & 23.464 & 23.233 & \underline{\textbf{22.931}} \\
   & MAE   & 5.640          & 7.428  & 13.751 & 8.920  & 17.723       & 5.302  & 5.116  & 6.994  & 5.508            & \underline{\textbf{5.000}}  & 5.159  & 5.149  \\
   & MASE  & 0.979          & 1.484  & 2.067  & 1.608  & 2.316        & 0.947  & 0.901  & 1.207  & 0.954            & \underline{\textbf{0.872}}  & 0.899  & 0.880  \\
   & RMSE  & 6.835          & 8.854  & 15.351 & 10.215 & 18.674       & 6.565  & 6.284  & 8.224  & 6.770            & 6.219  & \underline{\textbf{6.190}}  & 6.369  \\ \hline
\multirow{4}{*}{9-month}  & SMAPE & 25.719         & 27.747 & 43.612 & 45.359 & 48.903       & 26.962 & 22.859 & 33.656 & 24.728           & 23.354 & 23.172 & \underline{\textbf{22.765}} \\
   & MAE   & 5.687          & 6.268  & 15.942 & 8.835  & 17.961       & 5.592  & 5.084  & 7.173  & 5.476            & 5.104  & 5.061  & \underline{\textbf{5.030}}  \\
   & MASE  & 1.019          & 1.240  & 2.616  & 1.753  & 2.524        & 1.065  & 0.931  & 1.325  & 1.029            & 0.949  & 0.957  & \underline{\textbf{0.910}}  \\
   & RMSE  & 6.954          & 7.617  & 16.920 & 9.963  & 18.877       & 6.686  & 6.198  & 8.338  & 6.673            & 6.164  & \underline{\textbf{6.145}}  & 6.178  \\ \hline
\multirow{4}{*}{6-month}  & SMAPE & 25.610         & 27.584 & 42.650 & 45.841 & 48.062       & 25.203 & 22.682 & 32.459 & 24.057           & 22.786 & \underline{\textbf{21.916}} & 22.214 \\
   & MAE   & 6.379          & 6.499  & 15.733 & 8.383  & 17.633       & 5.246  & 5.486  & 7.171  & 5.544            & 5.189  & \underline{\textbf{4.911}}  & 5.024  \\
   & MASE  & 1.178          & 1.283  & 2.923  & 1.915  & 2.783        & 1.119  & 0.989  & 1.406  & 1.088            & 1.013  & 0.965  & \underline{\textbf{0.953}}  \\
   & RMSE  & 7.375          & 7.609  & 16.705 & 9.347  & 18.400       & 6.157  & 6.386  & 8.371  & 6.570            & 6.234  & \underline{\textbf{5.946}}  & 6.001  \\ \hline
\multirow{4}{*}{3-month}  & SMAPE & 24.533         & 25.467 & 36.300 & 45.835 & 50.266       & 28.612 & \underline{\textbf{24.292}} & 31.572 & 24.860           & 27.443 & 24.381 & 25.188 \\
   & MAE   & 5.012          & \underline{\textbf{4.857}}  & 11.838 & 8.032  & 17.511       & 5.629  & 5.167  & 6.983  & 5.400            & 5.591  & 5.446  & 5.275  \\
   & MASE  & 1.329          & 1.485  & 3.068  & 2.562  & 3.757        & 1.672  & 1.301  & 1.651  & \underline{\textbf{1.154}}            & 1.274  & 1.346  & 1.339  \\
   & RMSE  & \underline{\textbf{5.558}}          & 5.576  & 12.664 & 8.703  & 18.158       & 6.328  & 5.854  & 7.885  & 6.045            & 6.275  & 6.062  & 5.960  \\ \hline
\end{tabular}}
\end{table}

\begin{figure}[H]
    \centering
    \includegraphics[width=0.85\linewidth]{EGDLP_Conformal.png}
    \caption{Visualization of the ground truth (red dots) tuberculosis (TB) incidences monitored at prefectures 10 (left) and 27 (right) from Jan - Dec 2015 with the corresponding point forecasts of EGP-NHits (blue line), NHits (green line), the infected curve of the networked SIR framework (violet line), and conformal prediction interval (yellow-shaded) of EGP-NHits.}
    \label{Fig_Conformal_prediction_EGDLP}
\end{figure}


\section{Conclusion and Discussion}\label{Sec_Conclusion}
Tuberculosis (TB), a highly contagious airborne disease, was declared a global public health emergency by the WHO in 1993. Since then, numerous global initiatives have been undertaken to combat and prevent TB. The WHO's End TB Strategy aims to reduce the burden of the TB epidemic 
%by 2030, with ambitious targets to reduce TB incidence by 80\%, decrease TB-related deaths by 90\%, and eliminate catastrophic costs for TB-affected households, 
in alignment with the United Nations' Sustainable Development Goal 3: Good health and well-being. Despite these efforts, the 2024 Global TB Report highlighted a worrying increase in new TB cases, rising globally from 7.5 million in 2022 to 8.2 million in 2023. While low and middle-income countries remain high-burden regions, the impacts of rapid globalization have contributed to an uptick in TB cases in high-income countries. Japan, with the world's highest aging population and growing migration from TB-endemic regions, has shifted to a medium-burden country with rising TB incidence, presenting a significant public health and economic burden. Early detection and forecasting systems are vital for effective TB control. This study introduces a technological solution combining the networked SIR model with advanced deep learning-based forecasting architectures. These frameworks provide actionable insights for forecasting TB cases in Japan. Our method bridges the gap between compartmental models, which describe disease dynamics based on core epidemiological drivers, and data-driven forecasting techniques, which predict future disease trajectories. Integrating the networked SIR model with data-driven deep learning frameworks allows for precise TB incidence forecasts while capturing the interactions of key disease drivers, spatial dynamics, and historical patterns. This ensures that our models provide realistic forecasts for designing effective public health interventions and making real-time policy adjustments during an outbreak.

The networked SIR model, incorporating a saturated incidence rate and graph Laplacian diffusion, offers a robust estimation of TB transmission across Japan's 47 prefectures. The model's positivity and boundedness properties ensure realistic epidemic projections, while the global stability analysis reinforces its applicability to various populations and epidemic scenarios. Experimental results demonstrate that the EGDL-based models consistently outperform traditional data-driven models, yielding accurate forecasts across different time horizons (for short to medium-term forecasts). Moreover, the probabilistic outputs of the EGDL models provide valuable insights, allowing public health officials to track disease trends and tailor intervention strategies. Our approach provides a comprehensive understanding of future TB dynamics, excelling in short to medium-term forecasting, while the networked SIR model offers reliable long-term trajectory estimations. By forecasting TB incidence across Japan's 47 prefectures, this model identifies potential hotspots, guiding the design of region-specific interventions such as vaccination campaigns, resource allocation, and public health policy adjustments. While this study focuses on TB in Japan, the proposed frameworks can be adapted for other countries with varying epidemiological profiles and population dynamics. Additionally, the models can be extended to address the dynamics of vector-borne or waterborne diseases. Future research could also explore the application of these frameworks to multi-disease contexts, such as TB-HIV and TB-diabetes co-infections, from both global and regional perspectives. Future investigations would examine how factors like socioeconomic status, healthcare access, and environmental variables influence epidemic forecasts, improving the precision of intervention strategies.


\newpage

\begin{appendices}
\setcounter{figure}{0} % Restart figure numbering
\renewcommand{\thefigure}{A\arabic{figure}}% Figure counter representation
\renewcommand{\theHfigure}{A\arabic{figure}}

% \setcounter{table}{0}
\renewcommand{\thetable}{A\arabic{table}}



\section*{Appendix}\label{app}
\section{Mathematical Proofs}

\subsection{Proof of Lemma \ref{lim_glo}}\label{proof_lim_glo}
\noindent\textbf{Proof.} We consider the following two cases:
\begin{itemize}
    \item \textbf{Case-1} When $a < c$ :
    using the positivity condition of $u$, from Eq. \eqref{x_eqn} we have
    \begin{align*}
        \dot u \leq {a u} - c u = -(c-a)u, \quad \mbox{ for } t \in [0, \infty),\, c-a>0
    \end{align*}
    By standard Comparison principle and non-negativity of $u$, 
      $  \lim\limits_{t \to \infty} u = 0, \text{ provided } a< c.$
    \item \textbf{Case-2} When $a > c$ : define a Lyapunov functional 
        $W = u - u^* - u^* \ln \left(  \dfrac{u}{u^*}\right).$\\[1ex]
    Then, clearly, $W \geq 0$, for all $t \geq 0$, and $W = 0$ if and only if $u = u^*$. Differentiating $W$ with respect to $t$, it follows
    \begin{align*}
        W' &= \left( 1 - \dfrac{u^*}{u} \right) \dot u = \left( 1 - \dfrac{u^*}{u} \right) \left(  \dfrac{a u}{1+\alpha u} - c u \right)
        % &= \left( {u} - {u^*} \right) \left(  \dfrac{a }{1+\alpha u} - c  \right) = \left( {u} - {u^*} \right) \left(  \dfrac{a  - c - c\alpha u}{1+\alpha u}  \right)\\
        = -c \alpha \dfrac{\left( {u} - {u^*} \right)^2}{1+\alpha u} \leq 0,
    \end{align*}
 for $ u \geq 0,\,a > c$.  By applying the Lyapunov-LaSalle invariance principle   \cite{martcheva2015introduction}, we obtain $\displaystyle{ \lim_{t \to \infty}u(t) = u^* = \dfrac{a-c}{\alpha c}}$.
\end{itemize}
%%%%%%%%%%%%%%%%%%%%%%%%%%%%%%%%%%%%%%%%%%%%%%%%%%%%%%%%%%
\subsection{Proof of Lemma \ref{frac_supinf}}\label{proof_frac_supinf}
\noindent\textbf{Proof.} Let us consider the following auxiliary problem
\begin{equation}
\begin{aligned}\label{limsup_abs}
       &\dfrac{\partial v}{\partial t} - \sigma \Delta v = \dfrac{\kappa v}{1+\alpha v}   - \nu v,\quad v(x, 0) = v_0(x) \geq 0\, (\not \equiv 0),\, x \in V.  
\end{aligned}
\end{equation}
By using strong maximum principle, Lemma 2.4 of \cite{tian2023asymptotic}, $v(x, t) > 0$ for all $(x, t) \in V \times(0, \infty)$, since $v_0(x) > 0$. For any sufficiently small $t_s > 0$, we define $\delta = \min\limits_{x \in V} v(x, t_s)$, which implies $\delta > 0$. Now, we consider $\hat v$ satisfies the following: 
\begin{equation}
\begin{aligned}
       &\dfrac{\partial \hat v}{\partial t} = \dfrac{\kappa \hat v}{1+\alpha \hat v}   - \nu\hat v,\quad  \hat v(x, t_s) = \delta> 0, \,  x \in V,\, t\in (t_s, \infty).  
\end{aligned}
\end{equation}
Since $V$ is finite, applying Lemma \ref{lim_glo}, we obtain
\begin{equation}\label{hatv}
  \lim_{t \to \infty} \hat v(x,t) =
  \begin{cases}
     \dfrac{\kappa - \nu}{\alpha \nu}, &\mbox{ if } \kappa > \nu,\\[0.75ex]
     0, &\mbox{ if } \kappa < \nu,
  \end{cases} \mbox{ uniformly in } x\in V.
\end{equation}
Furthermore, the Eq. \eqref{hfun} gives $\Delta \hat v(x, t) = \sum\limits_{y \in V, y\sim x} \left(\hat v(y, t) - \hat v(x, t)\right) \equiv 0$. For $t \in [t_s, \infty)$, $\hat v$ will be a lower solution of the system Eq. \eqref{limsup_abs}. By maximum principle yields $v(x, t) \geq \hat v(x, t)$ for all $(x, t) \in  V \times [t_s, \infty)$. From the relation Eq. \eqref{hatv}, it follows:
\begin{equation}\label{inf_les}
  \liminf_{t \to \infty} v \geq
  \begin{cases}
     \dfrac{\kappa - \nu}{\alpha \nu}, &\mbox{ if } \kappa > \nu,\\[0.75ex]
     0, &\mbox{ if } \kappa < \nu,
  \end{cases} \mbox{ uniformly in } x\in V.
\end{equation}
Similarly, one can get
\begin{equation}\label{sup_gret}
  \limsup_{t \to \infty} v \leq
  \begin{cases}
     \dfrac{\kappa - \nu}{\alpha \nu}, &\mbox{ if } \kappa > \nu,\\[0.75ex]
     0, &\mbox{ if } \kappa < \nu,
  \end{cases} \mbox{ uniformly in } x\in V.
\end{equation}
The above two results establish the following
\begin{equation}
  \lim_{t \to \infty} v = l = 
  \begin{cases}
     \dfrac{\kappa - \nu}{\alpha \nu}, &\mbox{ if } \kappa > \nu,\\[0.75ex]
     0, &\mbox{ if } \kappa < \nu,
  \end{cases} \mbox{ uniformly in } x\in V.
\end{equation}
The comparison principle deduces $ \liminf\limits_{t \to \infty} u(x, t) \geq l\, \Big( \limsup\limits_{t \to \infty}u(x, t) \leq  l \Big)$
uniformly in $x\in V$. Finally, the rest of the results are followed directly.
%%%%%%%%%%%%%%%%%%%%%%%%%%%%%%%%%%%%%%%%%%%%%%%%%%%
\subsection{Proof of the Lemma \ref{positive1}}\label{proof_positive1}
\noindent \textbf{Proof.}   On the hyperplane $\mathbb{D}_+$, we deduce for each $x \in V$
\begin{equation}\label{hyp}
\left\lbrace
\begin{aligned}
    \dfrac{\partial }{\partial t} S(x, t) \Big|_{S = 0} &= \Lambda + \sigma\sum_{ \substack {y \in {V}\\ y \sim x}} S > 0,\\
     \dfrac{\partial }{\partial t} I(x, t) \Big|_{I = 0} &=  \sigma\sum_{ \substack {y \in \textbf{V}\\ y \sim x}}  I \geq 0.
\end{aligned}
\right.
\end{equation}
Since $S(x,0)$ and $I(x, 0)$  in $\mathbb{D}_+$, Eq. \eqref{hyp} yields that there are no solutions $\lbrace (S, I) : x \in {V} \rbrace$ that leave the hyperplanes from $S = 0, I = 0$. The vector field is either pointing to the interior of $\mathbb{D}_+$ or tangent to the hyperplane.  Therefore, all the solutions starting in $\mathbb{D}_+$ stay in $\mathbb{D}_+$ for all time $t > 0$, i.e., the region $\mathbb{D}_+$ is positively invariant.
%%%%%%%%%%%%%%%%%%%%%%%%%%%%%%%%%%%%%%%
\subsection{Proof of the Lemma \ref{bdd}}\label{proof_bdd}
\noindent \textbf{Proof.} This can be proved easily by adding the equations of the system Eq. \eqref{submodel} and letting $P(x,t) = S(x, t) + I(x, t)$:  
\begin{align*}
     \dfrac{\partial P}{\partial t} - \sigma\Delta P &= \Lambda  - \mu P - \gamma I, \quad\,     P(x ,0) := P_0(x) = S_0(x) + I_0(x) \geq 0 \mbox{ for } x \in V\\
     &\leq  \Lambda  - \mu P, \qquad \quad\,\,     P(x ,0)  := P_0(x) = S_0(x) + I_0(x) \geq 0 \mbox{ for } x \in V
\end{align*}
Applying Lemma \ref{supinf}, we get 
\begin{align}\label{psup}
    \limsup\limits_{t\to\infty} P(x ,t)\leq \dfrac{\Lambda}{\mu}\, \mbox{ i.e.,}\, \limsup\limits_{t\to\infty} \left(S(x, t) + I(x, t) \right)\leq \dfrac{\Lambda}{\mu} \quad \mbox{ uniformly in } x \in V.
\end{align}
By maximum principle \cite{protter2012maximum, tian2020global}, we derive $P(x,t) \geq 0$. The positivity of the solutions gives  $S(x,t) \geq 0$ and $I(x,t) \geq 0$. Combining the above results, $0 \leq S(x, t), I(x, t) \leq K$  for all $(x, t) \in V \times (0, \infty)$, where the constant $K$ is given by the following expression
\begin{align*}
K = \max\limits_{x \in V} \Big\lbrace \dfrac{\Lambda}{\mu},\, \max\limits_{x \in V} P_0(x)  \Big\rbrace.
\end{align*}
%%%%%%%%%%%%%%%%%%%%%%%%%%%%%%%%%%%%%%%%%%%%%%%%
\subsection{Proof of the Theorem \ref{dfe_glo}}\label{proof_dfe_glo}
\noindent\textbf{Proof.} According to Lemma \ref{positive1}, $(S, I), \mbox{ for }(x, t) \in V\times [0, \infty)$, represents the non-negative solution of system Eq. \eqref{submodel}. We choose $\epsilon$ be sufficiently small positive real number such that $\mathcal{R}_0 < 1$, then one can write
\begin{align}\label{R0_inq}
   \beta\left(\dfrac{\Lambda}{\mu}+\epsilon\right) < (\gamma + \mu). 
\end{align}
From Eq. \eqref{submodel_a}, we obtain
\begin{align*}
                \dfrac{\partial S}{\partial t} - \sigma\Delta S \leq  \Lambda  - \mu S, \qquad\,     S(x ,0) = S_0(x) \mbox{ for } x \in V.
\end{align*}
By Lemma \ref{supinf}, we have 
\begin{align}\label{ssup}
    \limsup\limits_{t\to\infty} S(x ,t)\leq \dfrac{\Lambda}{\mu}, \quad \mbox{ uniformly in } x \in V,
\end{align}
and so, for any $\epsilon > 0$, there is a $T_1$ such that if $t > T_1,$  
\begin{align}\label{s1_less}
    S(x ,t) < \dfrac{\Lambda}{\mu} + \epsilon, \quad \mbox{ uniformly in } x \in V.
\end{align}
For any $\epsilon > 0$ satisfying Eq. \eqref{R0_inq}, substituting Eq. \eqref{s1_less} into  equation of \eqref{submodel_b} and for $t > T_1$, we deduce
\begin{align}
    \dfrac{\partial I}{\partial t} - \sigma \Delta I &\leq \dfrac{\beta \left({\Lambda}/{\mu}+\epsilon\right) I}{1 + \alpha I} - (\gamma + \mu)I,\qquad I(x ,T_1) = I_1(x) \mbox{ for } x \in V.
\end{align}
Using Lemma \ref{frac_supinf} with the condition Eq. \eqref{R0_inq}, it follows
\begin{align}\label{isup}
    \limsup\limits_{t\to\infty} I(x, t)\leq 0, \quad \mbox{ uniformly in } x \in V,
\end{align}
hence, for any $\epsilon > 0$, there is a $T_2>T_1$ such that if $t>T_2$,
\begin{align}\label{i_les}
    I(x, t) < \epsilon, \quad \mbox{ uniformly in } x \in V.
\end{align}
The non-negativity fact of $I$ and the relation Eq. \eqref{isup} results $\lim\limits_{t\to\infty} I(x, t)= 0$  uniformly in $x \in V$.
For $t > T_2$, putting Eq. \eqref{i_les} into Eq. \eqref{submodel_a}, we find
\begin{align}
    \dfrac{\partial S}{\partial t} - \sigma\Delta S &\geq \Lambda - \left(\dfrac{\beta  \epsilon}{1 + \alpha \epsilon} +\mu \right)S,\qquad S(x ,T_2) = S_2(x) \mbox{ for } x \in V.
\end{align}
By the Lemma \ref{supinf}, 
\begin{align}\label{s1_gret}
    \liminf_{t\to \infty}S(x ,t) \geq \dfrac{\Lambda(1 + \alpha \epsilon)}{{\mu +(\beta  +\alpha\mu)\epsilon} }, \quad \mbox{ uniformly in } x \in V.
\end{align}
Since $\epsilon$ is an arbitrary real number, we let $\epsilon \to 0$, which gives
\begin{align}\label{sinf}
    \liminf\limits_{t\to\infty} S(x, t)\geq \dfrac{\Lambda}{\mu}, \quad \mbox{ uniformly in } x \in V.
\end{align}
Combining the relations Eq. \eqref{ssup} and Eq. \eqref{sinf}, we have $\lim\limits_{t\to\infty} S(x, t) = \dfrac{\Lambda}{\mu}$  uniformly in $x \in V$. This completes the proof.
%%%%%%%%%%%%%%%%%%%%%%%
\subsection{Proof of the Theorem \ref{ee_glo}}\label{proof_ee_glo}
\noindent\textbf{Proof.} Denote
\begin{align}
\begin{cases}
       \,\overline S(x) = \limsup\limits_{t \to \infty} S(x, t), \quad  \underline S(x) = \liminf\limits_{t \to \infty} S(x, t),\\[1ex]
        \,\overline I(x) = \limsup\limits_{t \to \infty} I(x, t), \quad\,\,\,  \underline I(x) = \liminf\limits_{t \to \infty} I(x, t),
\end{cases}
\quad \mbox{for all } x \in V.
\end{align}
Our claim will be $\overline S(x) = \underline S(x)=S^*,\, \overline I(x) = \underline I(x)=I^*$ uniformly in $x \in V$, which is established in the following:\\[1ex]
\noindent we choose $\epsilon$ be sufficiently small positive real number such that if $\mathcal{R}_0 > 1$,
\begin{align}\label{R0_grt}
   \beta\left(\dfrac{\Lambda}{\mu}+\epsilon\right) > (\gamma + \mu). 
\end{align}
From Eq. \eqref{submodel_a}, we obtain
\begin{align*}
                \dfrac{\partial S}{\partial t} - \sigma\Delta S \leq  \Lambda  - \mu S, \qquad\,     S(x ,0) = S_0(x) \mbox{ for } x \in V.
\end{align*}
By Lemma \ref{supinf}, we have 
\begin{align}\label{A1S}
    \limsup\limits_{t\to\infty} S(x ,t)\leq \dfrac{\Lambda}{\mu} :=A_1^S, \quad \mbox{ uniformly in } x \in V,
\end{align}
and so, for any $\epsilon > 0$, there is a $T_1$ such that if $t > T_1,$  
\begin{align}\label{e_s1_less}
    S(x ,t) < A_1^S + \epsilon, \quad \mbox{ uniformly in } x \in V.
\end{align}
For any $\epsilon > 0$ satisfying Eq. \eqref{R0_grt}, substituting Eq. \eqref{e_s1_less} into Eq. \eqref{submodel_b} and for $t > T_1$, we deduce
\begin{align}
    \dfrac{\partial I}{\partial t} - \sigma \Delta I &\leq \dfrac{\beta \left(A_1^S+\epsilon\right) I}{1 + \alpha I} - (\gamma + \mu)I,\qquad I(x ,T_1) = I_1(x) \mbox{ for } x \in V.
\end{align}
Using Lemma \ref{frac_supinf} with the condition Eq. \eqref{R0_grt}, it follows
\begin{align*}
    \limsup_{t\to\infty} I(x, t) \leq \dfrac{\beta \left(A_1^S+\epsilon\right) - (\gamma + \mu)}{\alpha(\gamma + \mu)}, \quad \mbox{ uniformly in } x \in V.
\end{align*}
As the above inequality holds for any arbitrary $\epsilon > 0$, we derive $\limsup\limits_{t\to\infty} I(x, t) \leq A_1^I,$ where
\begin{align}\label{A1I}
    A_1^I = \dfrac{\beta A_1^S - (\gamma + \mu)}{\alpha(\gamma + \mu)}.
\end{align}
We acquire $A_1^I > 0$, since $\mathcal{R}_0 > 1$. So, for any $\epsilon > 0$ there is a $T_2 > T_1$ such that if $t > T_2$, $I(x, t) < A_1^I + \epsilon$ uniformly in $x\in V$. Substituting the last result into Eq. \eqref{submodel_a}, when $t>T_2$, we obtain
\begin{align*}
    \dfrac{\partial S}{\partial t} - \sigma\Delta S &\geq \Lambda - \dfrac{\beta(A_1^S+\epsilon)(A_1^I + \epsilon)}{1 + \alpha (A_1^I + \epsilon)} - \mu S, \qquad\,     S(x ,T_2) = S_2(x) \mbox{ for } x \in V.
\end{align*}
By  Lemma \ref{supinf}, we derive
\begin{align*}
    \liminf_{n \to \infty} S(x, t) \geq  \dfrac{1}{\mu}\left[ \Lambda - \dfrac{\beta(A_1^S+\epsilon)(A_1^I + \epsilon)}{1 + \alpha (A_1^I + \epsilon)} \right], \quad \mbox{ uniformly in } x \in V.
\end{align*}
Due to arbitrariness of $\epsilon > 0$, we deduce $\liminf\limits_{t \to\infty} S(x, t) \geq B_1^S,\mbox{ uniformly in } x \in V$, where
\begin{align}\label{B1S}
    B_1^S = \dfrac{1}{\mu}\left[ \Lambda - \dfrac{\beta A_1^S A_1^I }{1 + \alpha A_1^I } \right]
\end{align}
Some simple calculations and the condition $\mathcal{R}_0 > 1$ yield $B_1^S > 0$. Hence, for any $\epsilon > 0$ there is $T_3 > T_2$ such that if $t > T_3$, $ S(x, t) > B_1^S - \epsilon$, uniformly in $x\in V$. Using the latest inequality in Eq. \eqref{submodel_b}, when $t>T_3,$  we derive
\begin{align*}
    \dfrac{\partial I}{\partial t} - \sigma \Delta I &\geq  \dfrac{\beta (B_1^S - \epsilon)I}{1 + \alpha I} - (\gamma + \mu)I,\qquad I(x ,T_3) = I_3(x) \mbox{ for } x \in V.
\end{align*}
With a few derivation and $\mathcal{R}_0>1$,  $\beta (B_1^S - \epsilon) > (\gamma + \mu)$. By Lemma \ref{frac_supinf}, we have
\begin{align*}
    \liminf_{t \to \infty} I(x, t)  \geq \dfrac{ \beta (B_1^S - \epsilon)-(\gamma + \mu)}{\alpha(\gamma + \mu)}, \quad\mbox{ uniformly in } x \in V.
\end{align*}
This inequality satisfies for every $\epsilon>0$, we conclude $\liminf\limits_{t \to \infty} I(x, t) \geq B_1^I$, where
\begin{align}\label{B1I}
    B_1^I = \dfrac{ \beta B_1^S-(\gamma + \mu)}{\alpha(\gamma + \mu)}.
\end{align}
Again one can easily show that $B_1^I > 0$, when $\mathcal{R}_0 > 1.$ Thus, there is $T_4>T_3$ such that if $t>T_4$, $ I(x, t)  \geq B_1^I - \epsilon$, uniformly in $x\in V$. In a similar manner, we derive from  Eq. \eqref{submodel_a},  for $t>T_4$
\begin{align*}
     \dfrac{\partial S}{\partial t} - \sigma\Delta S &\leq  \Lambda - \dfrac{\beta (B_1^S-\epsilon) (B_1^I-\epsilon)}{1 + \alpha  (B_1^I-\epsilon)} - \mu S, \qquad\,     S(x ,T_4) = S_4(x) \mbox{ for } x \in V.
\end{align*}
Applying Lemma \ref{supinf}, we acquire
\begin{align*}
    \limsup_{t \to\infty} S(x, t) \leq \dfrac{1}{\mu} \left[\Lambda - \dfrac{\beta (B_1^S-\epsilon) (B_1^I-\epsilon)}{1 + \alpha  (B_1^I-\epsilon)} \right], \quad\mbox{ uniformly in } x \in V.
\end{align*}
which is true for any  $\epsilon > 0$. We deduce $\limsup\limits_{t \to\infty} S(x, t) \leq A_2^S$, where
\begin{align}\label{A2S}
    A_2^S = \dfrac{1}{\mu} \left[\Lambda - \dfrac{\beta B_1^S B_1^I}{1 + \alpha  B_1^I} \right].
\end{align}
Again it can be shown directly that $A_2^S>0$, if $\mathcal{R}_0 >1$. Therefore, for any $\epsilon>0$, there is $T_5>T_4$ such that if $t>T_5$, $S(x, t) < A_2^S +\epsilon$, uniformly in $x \in V$. From Eq. \eqref{submodel_b},  we derive  
\begin{align}
    \dfrac{\partial I}{\partial t} - \sigma \Delta I &\leq \dfrac{\beta \left(A_2^S+\epsilon\right) I}{1 + \alpha I} - (\gamma + \mu)I,\qquad I(x ,T_5) = I_5(x) \mbox{ for } x \in V.
\end{align}
By Lemma \ref{frac_supinf} with the condition Eq. \eqref{R0_grt}, it gives
\begin{align*}
    \limsup_{t\to\infty} I(x, t) \leq \dfrac{\beta \left(A_2^S+\epsilon\right) - (\gamma + \mu)}{\alpha(\gamma + \mu)}, \quad \mbox{ uniformly in } x \in V.
\end{align*}
Since the above inequality satisfies for any $\epsilon > 0$, we get  $\limsup\limits_{t\to\infty} I(x, t) \leq A_2^I,$ where
\begin{align}\label{A2I}
    A_2^I = \dfrac{\beta A_2^S - (\gamma + \mu)}{\alpha(\gamma + \mu)}.
\end{align}
By some direct calculations, we obtain $A_1^I > 0$, as $\mathcal{R}_0 > 1$. So, for any $\epsilon > 0$ there is a $T_6 > T_5$ such that if $t > T_6$, $I(x, t) < A_2^I + \epsilon$ uniformly in $x\in V$. From Eq. \eqref{submodel_a}, when $t>T_2$, we find
\begin{align*}
    \dfrac{\partial S}{\partial t} - \sigma\Delta S &\geq \Lambda - \dfrac{\beta(A_2^S+\epsilon)(A_2^I + \epsilon)}{1 + \alpha (A_2^I + \epsilon)} - \mu S, \qquad\,     S(x ,T_6) = S_6(x) \mbox{ for } x \in V,
\end{align*}
Using Lemma \ref{supinf}, we derive
\begin{align*}
    \liminf_{n \to \infty} S(x, t) \geq  \dfrac{1}{\mu}\left[ \Lambda - \dfrac{\beta(A_2^S+\epsilon)(A_2^I + \epsilon)}{1 + \alpha (A_2^I + \epsilon)} \right], \quad \mbox{ uniformly in } x \in V.
\end{align*}
Since above inequality is true for any $\epsilon > 0$, we conclude $\liminf\limits_{t \to\infty} S(x, t) \geq B_2^S,\mbox{ uniformly in } x \in V$, where
\begin{align}\label{B2S}
    B_2^S = \dfrac{1}{\mu}\left[ \Lambda - \dfrac{\beta A_2^S A_2^I }{1 + \alpha A_2^I } \right].
\end{align}
A few simple computations and the condition $\mathcal{R}_0 > 1$ lead $B_1^S > 0$. Thus, for any $\epsilon > 0$ there is $T_7 > T_6$ such that if $t > T_7$, $ S(x, t) > B_2^S - \epsilon$, uniformly in $x\in V$. Again Eq. \eqref{submodel_b}), when $t>T_7,$  it follows
\begin{align*}
    \dfrac{\partial I}{\partial t} - \sigma \Delta I &\geq  \dfrac{\beta (B_2^S - \epsilon)I}{1 + \alpha I} - (\gamma + \mu)I,\qquad I(x ,T_7) = I_7(x) \mbox{ for } x \in V
\end{align*}
With a few easy calculations and $\mathcal{R}_0>1$, by Lemma \ref{frac_supinf}, we have
\begin{align*}
    \liminf_{t \to \infty} I(x, t)  \geq \dfrac{ \beta (B_2^S - \epsilon)-(\gamma + \mu)}{\alpha(\gamma + \mu)}, \quad\mbox{ uniformly in } x \in V.
\end{align*}
This inequality holds for any $\epsilon>0$, we obtain $\liminf\limits_{t \to \infty} I(x, t) \geq B_2^I$, where
\begin{align}\label{B2I}
    B_2^I = \dfrac{ \beta B_2^S-(\gamma + \mu)}{\alpha(\gamma + \mu)}.
\end{align}
Since $\mathcal{R}_0 > 1$, we obtain $B_2^I > 0$. Hence, for any $\epsilon > 0$ there is a $T_8 > T_7$ such that if $t>T_8$, $ I(x, t) > B_2^I - \epsilon$ uniformly in $x \in V$.\\[1ex]
\noindent If we carrying-on similar arguments, we acquire four sequences $\lbrace A_n^S\rbrace_{n=1}^\infty, \lbrace A_n^I\rbrace_{n=1}^\infty, \lbrace B_n^S\rbrace_{n=1}^\infty$ and $\lbrace B_n^I\rbrace_{n=1}^\infty$, where 
\begin{equation}
    \begin{aligned}\label{ABs}
            A_n^S &= \dfrac{1}{\mu} \left[\Lambda - \dfrac{\beta B_{n-1}^S B_{n-1}^I}{1 + \alpha  B_{n-1}^I} \right],\quad  A_n^I = \dfrac{\beta A_n^S - (\gamma + \mu)}{\alpha(\gamma + \mu)},\\
            B_n^S &= \dfrac{1}{\mu}\left[ \Lambda - \dfrac{\beta A_n^S A_n^I }{1 + \alpha A_n^I } \right],\quad\quad B_n^I = \dfrac{ \beta B_n^S-(\gamma + \mu)}{\alpha(\gamma + \mu)},\, n\geq 2. 
    \end{aligned}
\end{equation}
The term $A_2^S$ can be expressed as $A_2^S = A_1^S - \dfrac{\beta B_1^S B_1^I}{\mu(1+\alpha B_1^i)}$. Since $\dfrac{\beta B_1^S B_1^I}{\mu(1+\alpha B_1^i)} = \dfrac{\beta A_1^S - (\gamma + \mu)}{\alpha \mu} > 0, $ when $\mathcal{R}_0>1$, we have $A_2^S < A_1^S$. Hence $A_2^I < A_1^I$. We calculate from Eq. \eqref{B1S} and Eq. \eqref{B2S},
\begin{align*}
    B_2^S = \dfrac{1}{\mu}\left[\Lambda - (\gamma+\mu)A_2^I  \right], \quad B_1^S = \dfrac{1}{\mu}\left[\Lambda - (\gamma+\mu)A_1^I  \right].
\end{align*}
Using the fact $A_2^I < A_1^I$, it is clear that $B_2^S > B_1^S$. Applying induction, on can prove easily that $A_{n+1}^S \leq A_{n}^S$  and $B_{n+1}^S \geq B_{n}^S$  for all $n \geq 2$. Therefore, $\lbrace A_n^S\rbrace_{n=1}^\infty$ and $\lbrace B_n^S\rbrace_{n=1}^\infty$ are non-increasing sequence and non-decreasing sequence, respectively. Continuing similar arguments for the sequences $\lbrace A_n^I\rbrace_{n=1}^\infty$ and $\lbrace B_n^I\rbrace_{n=1}^\infty$, we obtain,  $\lbrace A_n^I\rbrace_{n=1}^\infty$ and $\lbrace B_n^I\rbrace_{n=1}^\infty$ are non-increasing sequence and non-decreasing sequence, respectively. Combining $\liminf \leq \limsup$ and the above results, we have
\begin{align}
    B_n^S \leq \underline S(x) \leq \overline S(x) \leq A_n^S,\, \mbox{ and } B_n^I \leq \underline I(x) \leq \overline I(x) \leq A_n^I, \mbox{ for all } x \in V. 
    \end{align}
Thus, the limits of all sequences exist such that
\begin{align}\label{limsup-liminf}
    \underline S(x) = \lim_{ n \to \infty} B_n^S, \quad  \overline S(x) = \lim_{ n \to \infty} A_n^S,\quad \underline I(x) = \lim_{ n \to \infty} B_n^I,\mbox{ and }  \overline I(x) = \lim_{ n \to \infty} A_n^I, 
    \end{align}
for all  $x \in V$. From Eq. \eqref{ABs}, we obtain
\begin{equation}\label{compI}
\begin{aligned}
     A_{n+1}^S &= \dfrac{1}{\mu} \left[\Lambda - \dfrac{\beta B_{n}^S B_{n}^I}{1 + \alpha  B_{n}^I} \right] = \dfrac{1}{\mu} \left[\Lambda - \dfrac{\beta B_n^S-(\gamma+\mu)}{\alpha} \right],\\
     \quad B_n^S &= \dfrac{1}{\mu}\left[ \Lambda - \dfrac{\beta A_n^S A_n^I }{1 + \alpha A_n^I } \right] = \dfrac{1}{\mu} \left[\Lambda - \dfrac{\beta A_n^S-(\gamma+\mu)}{\alpha} \right].
\end{aligned}
\end{equation}
Taking limit as $t \to \infty$ and using equation (\ref{limsup-liminf}), we acquire
\begin{align}\label{barS}
     \overline S(x) & = \dfrac{1}{\mu} \left[\Lambda - \dfrac{\beta \underline S(x)-(\gamma+\mu)}{\alpha} \right],
     \quad \underline S(x) =  \dfrac{1}{\mu} \left[\Lambda - \dfrac{\beta \overline S(x)-(\gamma+\mu)}{\alpha} \right],
\end{align}
for each $ x \in V$. With some direct calculations after subtracting the above two equations, it gives  
\begin{align}
    (\alpha \mu - \beta) \left( \overline S(x) - \underline S(x) \right) = 0, \quad \mbox{for } x \in V.
\end{align}
If $\alpha \mu \neq \beta$, we get $\overline S(x) = \underline S(x), \quad \mbox{for } x \in V.$ Hence, we have from Eq. \eqref{barS}, $ \overline S(x) = \underline S(x) = \dfrac{\alpha \Lambda + (\gamma + \mu)}{\beta + \alpha \mu} = S^*$, for all $x\in V$. It implies $\lim\limits_{t \to \infty} S(x, t) = S^*$, uniformly in $x\in V$.  Using this and Eq. \eqref{compI}, one may derive that   $\lim\limits_{t \to \infty} I(x, t) = I^*$, uniformly in $x\in V$. Hence, the endemic equilibrium $(S^*, I^*)$ is globally asymptotically stable if $\mathcal{R}_0 > 1$.

\section{Workflow of EGDL Frameworks}\label{Appendix_EGDL_Algo}
The workflow of the EGDL-Parallel and EGDL-Series frameworks are presented in Algorithm \ref{algo_EGDL_Parallel} and Fig. \ref{Fig_NSIR_EGDL} presents a schematic representation of these architectures.

\begin{algorithm}[H]
%\renewcommand{\thealgorithm}{}
%\begin{algorithmic}
    %\SetKwFunction{isOddNumber}{isOddNumber}
    \SetKwInOut{KwIn}{Input}
    \SetKwInOut{KwOut}{Output}
    \KwIn{Incidence cases recorded at $n$ locations in $T$ consecutive timestamps as $Y(x, t); \; x = 1, 2, \ldots, n; \; t = 1, 2, \ldots, T$.}
    \KwOut{$q$-step ahead forecast of the disease incidence cases for the $n$ locations as $\left\{\widehat{Y}(1, T+v), \widehat{Y}(2, T+v), \ldots, \widehat{Y}(n, T+v)\right\};  \; v = 1, 2, \ldots, q$ ($\geq 1$).}
    \vspace{0.15cm}
\textbf{Model Epidemic Information:}\\
\nl  Build a networked SIR epidemic model with saturated incidence rate and graph Laplacian diffusion by considering key disease drivers across $n$ locations.\\  %such as transmission, recovery, saturation, death rates, and population mobility 
\nl Calculate the posterior distribution of the networked SIR epidemic model parameters using the MCMC approach and select the best-fitted parameters to quantify the spatial dynamics of disease transmission and the effect of key epidemic drivers. \\  %input data $\left\{Y(1, t), Y(2, t), \ldots, Y(n, t)\right\};  \; t = 1, 2, \ldots, T$ by employing
\nl Solve the networked SIR model to estimate the infected curve across $n$ locations for $\tau$ timestamps as $\left\{I(1, \tau), I(2, \tau), \ldots, I(n, \tau)\right\};  \; \tau = 1, 2, \ldots, T+q$. \\

\setcounter{AlgoLine}{0}
\noindent \textbf{Forecasting Procedure for EGDL-Parallel Framework:}\\
\nl Initialize a data-driven forecasting model $f_{EGDLP} \in \mathcal{F}$, where $\mathcal{F}$ represents set of global deep learning-based forecasting tools. \\
\nl Provide a hybrid input to $f_{EGDLP}$ framework combining the historical incidence cases $Y(x,t)$ and the estimated infected curve of networked SIR $I(x,t)$, formulated as $\left\{\left\{Y\left(x, \tilde{t} - t_0 \right)\right\}_{t_0 = 0}^{t_w - 1}, \; \left\{I\left(x, \tilde{t} - t_0 \right)\right\}_{t_0 = 0}^{t_w - 1}\right\}_{\tilde{t} = t_w} ^ T,$ with $t_w$ lagged time steps, where $Y(x,t)$ serves as endogenous variable and $I(x,t)$ acts as the exogenous variable.\\
\nl Train the $f_{EGDLP}$ model to encode the hidden temporal patterns from the input data and epidemic knowledge from the infected curve to forecast $q$-step ahead future trajectory of the endogenous variables as 
$$ \widehat{Y}(x, \tilde{t} + v) = f_{EGDLP} \left(\left\{Y\left(x, \tilde{t} - t_0 \right)\right\}_{t_0 = 0}^{t_w - 1}, \; \left\{I\left(x, \tilde{t} - t_0 \right)\right\}_{t_0 = 0}^{t_w - 1}\right), \; v = 1, 2, \ldots, q.$$ 
% \vspace{0.2cm}
\setcounter{AlgoLine}{0}
\textbf{Forecasting Procedure for EGDL-Series Framework:}\\
\nl Obtain the residuals of the networked SIR model by subtracting the estimated infected curve from the ground truth observations as  ${e(x, t) = Y(x, t) - I(x, t)};  \; x = 1, 2, \ldots, n, \; t = 1, 2, \ldots, T$. \\
\nl Initialize a data-driven forecasting model $f_{EGDLS} \in \mathcal{F}$ and provide the $t_w$ lagged values of the residual series as input to generate the q-step ahead forecasts as:
$$
    \hat{e}(x, \tilde{t}+v) = f_{EGDLS}\left(\left\{e\left(x, \tilde{t} - t_0 \right)\right\}_{t_0 = 0}^{t_w - 1}\right), \; \tilde{t} = t_w, \ldots, T; \; v = 1, 2, \ldots, q.
$$ 
% \nl Obtain the q-step ahead predictions from the trained $f_{EGDLS}$ model as $\left\{\hat{e}(1, T+v), \hat{e}(2, T+v), \ldots, \hat{e}(n, T+v)\right\};  \; v = 1, 2, \ldots, q$ \\
\nl Combine the deterministic predictions of the networked SIR model $I(x, \tilde{t}+v)$, with the forecasts of stochastic components from $f_{EGDLS}$ framework as 
$$
\widehat{Y}(x, \tilde{t}+v) = I(x, \tilde{t}+v) + \hat{e}(x, \tilde{t}+v),
$$
to obtain the $q$-step ahead forecast of the disease incidence cases.


% \vspace{0.1cm}
%\vspace{0.2cm}
\caption{{\bf Proposed EGDL Framework}
\label{algo_EGDL_Parallel}
}
\end{algorithm}

\begin{figure}[!ht]
    \centering
    \includegraphics[width=0.75\textwidth]{Model_Image.png}
    \caption{\textbf{Epidemic-Guided Deep Learning (EGDL)}. The upper panel of the plot represents a schematic architecture of the networked SIR model, and the lower panel showcases the workflow of the EGDL-Parallel framework (lower left) and the EGDL-Series approach (lower right).}
    \label{Fig_NSIR_EGDL}
\end{figure}


\section{Empirical Results}\label{Appendix_Global_Char}
Table \ref{Table_Global_Features} summarizes the descriptive statistics of Japan TB data, including measures of central tendency, standard deviation (Sd), coefficient of variation (CV), skewness, and kurtosis. Additionally, to investigate key statistical characteristics of these datasets, we study their global patterns, as outlined below: \textit{Long-term dependency} reflects the self-similarity of the time series with its lagged observations. This feature is crucial for probabilistic time series modeling approaches. In this study, the Hurst exponent is computed to detect whether TB incidence trends exhibit long-range dependence over time. \textit{Stationarity} ensures the mean and variance of a time series remain constant over time, implying no systematic changes in trend or seasonal patterns. This property is essential for many classical forecasting models. The Kwiatkowski–Phillips–Schmidt–Shin (KPSS) test is employed to assess the level and trend stationarity of the TB incidence data. \textit{Linearity} of a time series determines whether its future values can be expressed as a linear combination of past observations. This feature influences model selection based on their ability to capture linear or non-linear dynamics. The Teräsvirta neural network test is used in this study to detect the presence of non-linear patterns in TB incidence cases. \textit{Seasonality} represents recurring patterns within a defined time interval. We apply the Ollech and Webel combined seasonality test to identify seasonal behavior in TB incidence. All the statistical analyses were performed using R statistical software, following \cite{panja2023epicasting}.
\begin{figure}[H]
\centering
% \resizebox{0.9\textwidth}{!}{
\begin{tabular}[!ht]{cc}
(a) Laplacian matrix   & (b)  Correlation Plot \\
  \includegraphics[width=40mm,height=40mm]{LAP47.png} &   
  \includegraphics[width=40mm,height=40mm]{Correlation_ST.png} \\
% (c) Time evolution for $S$    & (d) Time evolution for $I$ \\
% \includegraphics[width=85mm,height=65mm]{Endemic_S.png} &   \includegraphics[width=85mm,height=65mm]{Endemic_I.png}   
\end{tabular}
\caption{(a) The Laplacian matrix corresponding to the network depicted in Fig. \ref{fig:TB47}(A), representing the connectivity and structure of the nodes in the system. This matrix captures the relationships between neighboring nodes and is essential for analyzing diffusion processes and other network dynamics. (b) Pairwise correlation of TB incidence cases recorded at 47 prefectures of Japan.}
\label{Fig_Correlation_LAP47}
\end{figure}


\begin{table}[!ht]
\caption{Global features of TB incidence cases at 47 prefectures of Japan. In the table, (A) and (Q) indicate annual and quarterly seasonality.}
\setlength\tabcolsep{2.5pt}
\tiny{
% \resizebox{0.99\textwidth}{!}{
\begin{tabular}{|c|cccccc|cc|}
\hline
\multirow{2}{*}{Station No.}& \multicolumn{6}{c|}{Descriptive Statistics} & \multicolumn{2}{c|}{Statistical Properties}\\
 & Mean   & Range       & Sd     & CV    & Skewness & Kurtosis & Hurst Exponent & Characteristics                       \\ \hline
1             & 72.56  & (33 , 163)  & 25.59  & 35.26 & 0.76     & -0.03    & 0.82            & Non-stationary, Nonlinear, Seasonality (A) \\
2             & 24.57  & (8 , 61)    & 9.99   & 40.66 & 0.83     & 0.53     & 0.80            & Non-stationary, Linear                     \\
3             & 17.86  & (5 , 48)    & 7.91   & 44.32 & 1.06     & 1.46     & 0.79            & Non-stationary, Linear                     \\
4             & 28.24  & (6 , 71)    & 11.52  & 40.78 & 0.90     & 0.58     & 0.81            & Non-stationary, Nonlinear                  \\
5             & 14.95  & (2 , 38)    & 6.84   & 45.79 & 0.81     & 0.49     & 0.78            & Non-stationary, Linear                     \\
6             & 13.81  & (4 , 36)    & 5.92   & 42.84 & 1.37     & 2.20     & 0.75            & Non-stationary, Linear                     \\
7             & 26.63  & (7 , 75)    & 11.27  & 42.33 & 1.12     & 1.68     & 0.80            & Non-stationary, Nonlinear                  \\
8             & 44.24  & (22 , 84)   & 12.97  & 29.33 & 0.68     & -0.18    & 0.81            & Non-stationary, Linear, Seasonality (A)    \\
9             & 28.00  & (9 , 59)    & 9.62   & 34.33 & 0.71     & 0.17     & 0.80            & Non-stationary, Linear                     \\
10            & 25.77  & (10 , 70)   & 10.06  & 39.04 & 1.17     & 2.05     & 0.79            & Non-stationary, Linear                     \\
11            & 113.38 & (59 , 232)  & 26.85  & 23.68 & 0.98     & 1.50     & 0.81            & Non-stationary, Linear, Seasonality (A)    \\
12            & 108.09 & (53 , 190)  & 29.65  & 27.43 & 0.61     & -0.14    & 0.81            & Non-stationary, Linear, Seasonality (A)    \\
13            & 294.84 & (165 , 544) & 60.51  & 20.52 & 0.62     & 0.79     & 0.81            & Non-stationary, Linear, Seasonality (A, Q) \\
14            & 152.68 & (91 , 306)  & 37.16  & 24.34 & 0.94     & 1.22     & 0.81            & Non-stationary, Linear, Seasonality (A)    \\
15            & 32.01  & (11 , 65)   & 11.92  & 37.23 & 0.61     & -0.19    & 0.81            & Non-stationary, Linear, Seasonality (A)    \\
16            & 17.85  & (5 , 48)    & 7.54   & 42.25 & 0.85     & 0.77     & 0.79            & Non-stationary, Nonlinear                  \\
17            & 18.46  & (5 , 47)    & 7.34   & 39.76 & 1.03     & 1.32     & 0.77            & Non-stationary, Linear                     \\
18            & 12.15  & (2 , 29)    & 4.67   & 38.46 & 0.69     & 0.78     & 0.75            & Non-stationary, Linear                     \\
19            & 10.13  & (2 , 28)    & 4.37   & 43.14 & 0.74     & 0.89     & 0.75            & Non-stationary, Linear                     \\
20            & 20.68  & (7 , 42)    & 6.27   & 30.32 & 0.70     & 1.15     & 0.75            & Non-stationary, Linear                     \\
21            & 43.77  & (18 , 107)  & 14.32  & 32.72 & 1.07     & 1.81     & 0.80            & Non-stationary, Linear                     \\
22            & 62.58  & (30 , 130)  & 17.26  & 27.58 & 0.46     & 0.16     & 0.81            & Non-stationary, Linear, Seasonality (A)    \\
23            & 154.22 & (83 , 280)  & 36.57  & 23.72 & 0.68     & 0.34     & 0.79            & Non-stationary, Nonlinear, Seasonality (A) \\
24            & 30.89  & (11 , 68)   & 10.82  & 35.03 & 0.69     & 0.21     & 0.79            & Non-stationary, Linear, Seasonality (A)    \\
25            & 21.20  & (6 , 49)    & 7.39   & 34.88 & 0.78     & 0.94     & 0.77            & Non-stationary, Nonlinear, Seasonality (A) \\
26            & 61.24  & (25 , 178)  & 28.35  & 46.29 & 2.02     & 4.69     & 0.78            & Non-stationary, Linear, Seasonality (A)    \\
27            & 286.82 & (119 , 623) & 113.25 & 39.48 & 0.89     & -0.19    & 0.83            & Non-stationary, Nonlinear, Seasonality (A) \\
28            & 129.98 & (57 , 320)  & 47.93  & 36.88 & 1.07     & 0.59     & 0.83            & Non-stationary, Nonlinear, Seasonality (A) \\
29            & 28.42  & (12 , 58)   & 9.66   & 33.99 & 0.66     & -0.16    & 0.80            & Non-stationary, Linear, Seasonality (A)    \\
30            & 23.28  & (7 , 57)    & 9.29   & 39.89 & 1.07     & 0.96     & 0.79            & Non-stationary, Nonlinear                  \\
31            & 9.49   & (1 , 27)    & 4.39   & 46.31 & 0.90     & 1.33     & 0.76            & Non-stationary, Linear                     \\
32            & 11.85  & (3 , 27)    & 4.24   & 35.77 & 0.59     & 0.44     & 0.73            & Non-stationary, Linear                     \\
33            & 31.19  & (11 , 63)   & 10.72  & 34.37 & 0.76     & 0.36     & 0.79            & Non-stationary, Linear, Seasonality (A)    \\
34            & 45.82  & (19 , 91)   & 14.17  & 30.92 & 0.79     & 0.04     & 0.79            & Non-stationary, Linear                     \\
35            & 27.77  & (9 , 71)    & 12.04  & 43.34 & 1.05     & 0.82     & 0.80            & Non-stationary, Linear                     \\
36            & 17.42  & (6 , 42)    & 7.34   & 42.15 & 1.11     & 1.11     & 0.79            & Non-stationary, Nonlinear, Seasonality (A) \\
37            & 19.75  & (4 , 52)    & 7.87   & 39.84 & 0.80     & 0.93     & 0.79            & Non-stationary, Nonlinear, Seasonality (A) \\
38            & 23.68  & (8 , 49)    & 8.75   & 36.95 & 0.82     & 0.45     & 0.78            & Non-stationary, Linear, Seasonality (A)    \\
39            & 15.19  & (1 , 40)    & 6.83   & 44.95 & 0.79     & 0.31     & 0.80            & Non-stationary, Linear, Seasonality (A)    \\
40            & 99.09  & (38 , 201)  & 30.50  & 30.78 & 0.79     & 0.25     & 0.83            & Non-stationary, Linear, Seasonality (A)    \\
41            & 15.28  & (4 , 33)    & 5.21   & 34.11 & 0.41     & 0.26     & 0.74            & Non-stationary, Linear, Seasonality (A)    \\
42            & 32.10  & (12 , 63)   & 9.71   & 30.25 & 0.68     & 0.08     & 0.78            & Non-stationary, Linear, Seasonality (A)    \\
43            & 31.11  & (14 , 60)   & 8.42   & 27.07 & 0.64     & 0.29     & 0.76            & Non-stationary, Nonlinear, Seasonality (A) \\
44            & 24.86  & (9 , 69)    & 9.71   & 39.07 & 1.47     & 3.04     & 0.78            & Non-stationary, Linear                     \\
45            & 19.43  & (7 , 50)    & 8.27   & 42.57 & 1.02     & 0.81     & 0.80            & Non-stationary, Nonlinear, Seasonality (A) \\
46            & 34.78  & (10 , 65)   & 11.59  & 33.32 & 0.55     & -0.26    & 0.80            & Non-stationary, Linear, Seasonality (A)    \\
47            & 24.90  & (9 , 45)    & 6.75   & 27.11 & 0.35     & -0.24    & 0.75            & Non-stationary, Linear    \\ \hline                
\end{tabular}}
\label{Table_Global_Features}
\end{table}

\subsection{Parameter Calibration of the Mechanistic Model}\label{App_Param_Calli}
Fig. \ref{fig:stab} depicts the stability plot of the disease-free equilibrium ($\mathcal{R}_0 = 0.38$) and endemic equilibrium ($\mathcal{R}_0 = 3.8$). The time evolution of the infected individuals for selected prefectures of Japan estimated using the modified networked SIR model is presented in Fig. \ref{fig:alpha_var}.

\begin{figure}
\centering
% \resizebox{0.75\textwidth}{!}{
\begin{tabular}[!ht]{cc}
(a) Time evolution for $S$    & (b)  Time evolution for $I$ \\
  \includegraphics[width=50mm, height=40mm]{DieseaseFree_S.png} &   \includegraphics[width=50mm, height=40mm]{DieseaseFree_I.png} \\
(c) Time evolution for $S$    & (d) Time evolution for $I$ \\
\includegraphics[width=50mm, height=40mm]{Endemic_S.png} &   \includegraphics[width=50mm, height=40mm]{Endemic_I.png}   
\end{tabular}
%85mm,height=65mm
\caption{(a) - (b) illustrates the stability plot of the disease-free equilibrium for $\mathcal{R}_0 = 0.38$ ($< 1$). (c) - (d) illustrates the stability plot of the endemic equilibrium for $\mathcal{R}_0 = 3.8$ ($> 1$). Note that along the y-axis, individuals are represented by node numbers in sequential order rather than preserving the graph structure.}
\label{fig:stab}
\end{figure}


\begin{figure}[!ht]
    \centering
    % \resizebox{0.75\textwidth}{!}{
    \begin{tabular}{cc}
        (a) Prefecture 20 & (b) Prefecture 10 \\
           \includegraphics[scale=0.3]{Alpha_varies_N20.png}  &     \includegraphics[scale=0.3]{Alpha_varies_N10.png}
    \end{tabular}
    \caption{Time evolution of infected individuals at (a) prefecture 20 and (b) prefecture 10 when the saturation factor  ($\alpha$) varies from $0.1$ to $1.0$.}
    \label{fig:alpha_var}
\end{figure}

\subsection{Causality Analysis} \label{Appendix_Causal}
To assess temporal predictive causality, we apply the Granger causality (GC) test, which determines whether the lagged values of one variable enhance the explanatory power of another variable beyond what is provided by its own lagged values \cite{granger1969investigating}. This linear test evaluates the null hypothesis of Granger causation and is particularly suited for identifying predictive relationships. Since the GC test requires time series data to be stationary and linear, conditions that are frequently violated in the case of Japan's TB data (Table \ref{Table_Global_Features}), we also utilize wavelet coherence analysis \cite{grinsted2004application}. This non-parametric method identifies dynamic, time-dependent interactions between two signals in the time-frequency domain. It is especially effective for detecting localized correlations and co-movements in non-stationary time series. The p-values obtained from the GC test range between $\left(4.77 \times 10^{-20}, 1.00 \times 10^{-3}\right)$, with most values falling below $2.00 \times 10^{-4}$ across different prefectures in Japan. Since these p-values are below the significance level of 0.05, the null hypothesis is rejected, indicating Granger's causation between the networked SIR model's infection curve and the TB incidence cases. Additionally, Fig. \ref{Fig_Wavelet_Coherence} illustrates the wavelet coherence across significant time-frequency domain pairs for selected prefectures. In these plots, the horizontal axis represents the time (in months), while the vertical axis represents the frequency dimension. Warmer colors highlight regions with significant co-movement between the two time series, while cooler colors indicate weaker dependencies. This visualization reveals both the frequency and timing of significant interactions.




\begin{figure}[!ht]
    \centering
    \includegraphics[width=0.7\linewidth]{Causality_Japan.png}
    \caption{Wavelet coherence of TB incidence cases and the infected curve predicted by the networked SIR model for selected prefectures of Japan. The horizontal axis depicts the monthly time, while the vertical axis shows the period. The warmer the color of a region, the higher the degree of dependence between the pair.}
    \label{Fig_Wavelet_Coherence}
\end{figure}


\subsection{Performance Measures}\label{Sec_Performance_measures}
The mathematical expression for calculating the error metrics for the $x^{th}$ prefecture are as follows:
$$
\begin{aligned}
    \text{ SMAPE} &= \frac{1}{q} \sum_{v=1}^q \frac{2\left|\widehat{Y}(x,{T+v}) - Y(x,{T+v})\right|}{\left|\widehat{Y}(x,{T+v})\right|+ \left|Y(x,{T+v})\right|} \times 100 \%;
    \text{ MAE} = \frac{1}{q}\sum_{v =1}^{q} \left|\widehat{Y}(x,{T+v}) - Y(x,{T+v})\right|;\\
    \text{ MASE} &= \frac{\sum_{v = 1}^{q} \left|\widehat{Y}(x,{T+v}) - Y(x,{T+v})\right|}{\frac{q}{T-1} \sum_{v = 2}^T \left|Y(x,{v}) - Y(x,{v-1})\right|}; 
    \text{ RMSE} = \sqrt{\frac{1}{q}\sum_{v =1}^{q} \left(\widehat{Y}(x,{T+v}) - Y(x,{T+v})\right)^2};
    \end{aligned}
$$
where $q$ denotes the forecast horizon, $\widehat{Y}(x,{T+v})$ is the $v^{th}$-step ahead forecast based on $T$ historical data for the $x^{th}$ prefecture, and $Y(x,{T+v})$ represents the corresponding ground truth. 

% \setcounter{AlgoLine}{0}

% \begin{algorithm}[!ht]
% %\renewcommand{\thealgorithm}{}
% %\begin{algorithmic}
%     %\SetKwFunction{isOddNumber}{isOddNumber}
%     \SetKwInOut{KwIn}{Input}
%     \SetKwInOut{KwOut}{Output}
%     \KwIn{Incidence cases recorded at $n$ locations in $T$ consecutive timestamps as $Y(x, t);  \; t = 1, 2, \ldots, T$.}
%     \KwOut{$q$-step ahead forecast of the disease incidence cases for the $n$ locations as $\left\{\widehat{Y}(1, T+v), \widehat{Y}(2, T+v), \ldots, \widehat{Y}(n, T+v)\right\};  \; v = 1, 2, \ldots, q$ ($q \geq 1$ specified by user).}
%     \vspace{0.2cm}
% \textbf{Model Epidemic Information:}\\
% % \nl  Build a networked SIR epidemic model by taking into account key disease drivers such as transmission, recovery, saturation, death rates, and population mobility across $n$ locations.\\ 
% % \nl Calculate the posterior distribution of the networked SIR epidemic model parameters using the input data $\left\{Y(1, t), Y(2, t), \ldots, Y(n, t)\right\};  \; t = 1, 2, \ldots, T$ by employing MCMC method.\\
% % \nl Select the best-fitted parameters of the networked SIR model from the posterior distribution to quantify the effect of key disease drivers. \\ 
% \nl Repeat steps 1-3 of Algorithm \ref{algo_EGDL_Parallel}. \\ 
% \nl Estimate the infected curve of the networked SIR model using the best-fitted parameters for $T+q$ timestamps at $n$ locations as $\left\{I(1, \tau), I(2, \tau), \ldots, I(n, \tau)\right\};  \; \tau = 1, 2, \ldots, T+q$, corresponding to the input data. \\
% \nl Obtain the residuals of the networked SIR model by subtracting the estimated infected curve from the ground truth observations as  ${e(x, t) = Y(x, t) - I(x, t)};  \; x = 1, 2, \ldots, n, \; t = 1, 2, \ldots, T$. \\
% \vspace{0.1cm}
% \textbf{Forecasting Procedure:}\\
% \nl Initialize a data-driven forecasting model $f_{EGDLS} \in \mathcal{F}$, where $\mathcal{F}$ represents set of global deep learning-based forecasting techniques. \\
% \nl Provide the $t_w$ lagged values of the residual series as input to the $f_{EGDLS}$ framework and obtain the q-step ahead forecasts as:
% $$
%     \hat{e}(x, \tilde{t}+v) = f_{EGDLS}\left(\left\{e\left(x, \tilde{t} - t_0 \right)\right\}_{t_0 = 0}^{t_w - 1}\right), \; \tilde{t} = t_w, \ldots, T; \; v = 1, 2, \ldots, q.
% $$ 
% % \nl Obtain the q-step ahead predictions from the trained $f_{EGDLS}$ model as $\left\{\hat{e}(1, T+v), \hat{e}(2, T+v), \ldots, \hat{e}(n, T+v)\right\};  \; v = 1, 2, \ldots, q$ \\
% \nl Combine the deterministic predictions of the networked SIR model $I(x, \tilde{t}+v)$, with the forecasts of stochastic components from $f_{EGDLS}$ framework as 
% $$
% \widehat{Y}(x, \tilde{t}+v) = I(x, \tilde{t}+v) + \hat{e}(x, \tilde{t}+v),
% $$
% to obtain the $q$-step ahead forecast of the disease incidence cases.
% \vspace{0.2cm}
% % %\vspace{0.2cm}
% \caption{{\bf Proposed EGDL-Series Framework}
% \label{algo_EGDL_Series}
% }
% \end{algorithm}

\subsection{Statistical Significance}\label{Sec_Stat_Signif}
To validate the robustness of the proposed EGDL-Parallel and EGDL-Series frameworks, we employ the multiple comparisons with the best (MCB) test \cite{edwards1983multiple}. This model-agnostic procedure identifies the framework with the lowest average rank as the `best' performing model. To compare the relative performance of the architectures, the MCB test calculates a critical distance (CD) based on the Tukey distribution and utilizes the CD of the `best' model as the reference threshold. The MCB test results, based on the SMAPE metric, comparing EGDL-Parallel and EGDL-Series forecasters with baseline models, are visualized in Fig. \ref{fig:MCB_Test}. Among the EGDL-Parallel architectures, the EGP-NHits framework achieves the lowest rank of 1.00, followed by the NHits and EGP-NBeats frameworks. Similarly, for the EGDL-Series architectures, the EGS-NHits model has the lowest rank of 2.00, followed by EGS-TCN and NHits. These results indicate that the epidemic-guided NHits models consistently deliver the `best' performance among the evaluated frameworks. The CD for the `best' models, highlighted by the shaded regions, serves as the reference value in the MCB test. The competing architectures with CDs exceeding this reference value are deemed to perform significantly worse than the epidemic-guided NHits frameworks. %Additionally, the performance of the STGCN model is notably inferior to most of the competing architectures, primarily due to the small sample size of the dataset. 
Notably, all epidemic-guided forecasting approaches exhibit lower average ranks compared to their respective baseline architectures, underscoring the effectiveness of incorporating epidemic principles into data-driven forecasting models.
\begin{figure}
    \centering
    \includegraphics[width=0.9\linewidth]{SMAPE_MCB.pdf}
    \caption{The MCB test results, comparing (a) EGDL-Parallel approaches and (b) EGDL-Series architectures with baseline models using the SMAPE metric. In the plots, the Y-axis represents the average ranks of the models, while the X-axis labels each model along with its corresponding average rank. For instance, the label `EGP-NHits - 1.00' indicates that the EGP-NHits model achieved an average rank of 1.00, similar for other approaches.}
    \label{fig:MCB_Test}
\end{figure}



\subsection{Uncertainty Quantification in EGDL Approaches}\label{Sec_Uncertain_Quant}
Alongside the point forecasts generated by the EGDL frameworks, we quantify the uncertainties associated with these predictions using the conformal prediction approach \cite{vovk2005algorithmic}. This distribution-free method constructs prediction intervals at a pre-specified confidence level, providing robust uncertainty estimates. The prediction intervals of the EGDL-based forecasters make it a suitable choice for real-world decision-making. The conformal prediction process begins by fitting an uncertainty model to the input data and calculating conformal scores, which are derived from the residuals of the EGDL frameworks scaled by the uncertainty model's predictions. Using these scores, conformal quantiles are computed via a weighted aggregation technique. The prediction intervals are then determined by adding (subtracting) the EGDL forecasts with the upper (lower) quantiles scaled by the uncertainty model predictions. Fig. \ref{Fig_Conformal_prediction} presents the conformal prediction intervals for the EGS-TCN model across selected prefectures of Japan, alongside the forecasts of baseline models. As the plots highlight, the prediction intervals effectively capture the TB incidence trends in most scenarios, demonstrating superior generalization compared to the baseline and networked SIR models. These probabilistic intervals demonstrate the forecast's uncertainties and are crucial for designing effective disease intervention strategies.
\begin{figure}
    \centering
    \includegraphics[width=0.80\linewidth]{EGDLS_Conformal.png}
    \caption{Visualization of the ground truth (red dots) tuberculosis (TB) incidences monitored at prefectures 01 (left) and 27 (right) from Jan - Dec 2015 with the corresponding point forecasts of EGS-TCN (blue line), TCN (green line), the infected curve of the networked SIR framework (violet line), and conformal prediction interval (yellow-shaded) of EGS-TCN.}
    \label{Fig_Conformal_prediction}
\end{figure}
    
\end{appendices}

\section*{Data and Code Availability Statement}
Data and codes are available in our github repository: \url{https://github.com/mad-stat/EGDL}.

%\newpage
\bibliography{bibliography}% common bib file

\end{document}
