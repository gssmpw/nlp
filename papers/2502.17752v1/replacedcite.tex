\section{Related Works}
It is widely recognized that the design of fusion criteria plays a central role in distributed fusion estimation for multi-sensor systems. According to the availability of local estimation error correlations, most existing fusion criteria can be classified into two categories: C1) fusion criteria with known correlations; C2) fusion criteria with unknown correlations. A detailed review of these fusion criteria is provided below.
	\begin{itemize}
		\item[C1)] When estimation error correlations are known, optimal fusion criteria can be designed by augmenting or weighting the local estimates. In ____, an optimal fusion criterion in the maximum likelihood sense was designed for Gaussian posterior estimates. Subsequently, unified optimal fusion criteria for general linear data models, including raw measurement data and local estimates, were proposed in the best linear unbiased estimation sense and the weighted least squares sense ____. Inspired by the work in ____, optimal fusion criteria weighted by scalars ____, diagonal matrices ____, and general matrices ____ were developed in the linear unbiased minimum variance sense. It was proven that the optimal fusion criterion weighted by general matrices ____ is equivalent to the maximum likelihood fusion criterion ____ for Gaussian posterior estimates, and also equivalent to the weighted least squares fusion criterion ____.
		\item[C2)] When estimation error correlations are unknown, the fusion criteria should yield consistent estimates regardless of the actual correlations. The well-known covariance intersection (CI) algorithm was proposed in ____ by using the convex combination of local estimates' covariances. Thereafter, it was pointed out that the CI technique is equivalent to the log-linear combination of local estimates' Gaussian functions and can be generalized to a fusion criterion for any probability density functions ____. More recently, other fusion criteria based on probabilistic density function fusion have been developed ____.
	\end{itemize}
	
	The fusion criteria discussed above are tailored for Gaussian posterior estimates or estimates with known probability distributions. However, it is not easy to obtain statistical information about noises and validate Gaussian assumptions in practical scenarios. In the absence of precise statistical information, local sensors often use set-membership estimators to recursively compute sets that enclose possible states. A variety of set-membership estimators have been developed based on the properties of different set representations. For example, intervals are used to design interval observers ____ due to their computational efficiency in numerical implementation. Ellipsoids also offer low-cost computations and are used for developing different ellipsoidal estimators ____. General polytopes can describe set domains with any shape but encounter complexity issues when enumerating edges and vertices in high-dimensional spaces {____}. Actually, there is a trade-off between computation complexity and representation precision for different sets. Only certain special polytopes, such as parallelotopes ____ and zonotopes ____, were used for designing set-membership estimators because they effectively balance these factors. Particularly, zonotopes can represent more sophisticated set domains than intervals, ellipsoids, and parallelotopes. Additionally, zonotopes can be fully characterized by a vector and a rectangular matrix, which simplifies zonotope operations to matrix calculations. Consequently, zonotopic estimation has gained increasing attention in recent research ____. This motivates the investigation of set-membership fusion estimation, especially zonotopic fusion estimation, to develop a new framework for multi-sensor systems. To date, research on fusion estimation under unknown but bounded noises is still in its early stages. A distributed fusion estimator under bounded noises with unknown lower and upper bounds was proposed in ____. Meanwhile, the ellipsoidal fusion estimation problem was addressed in ____ using convex optimization, and zonotopic fusion estimators were developed by minimal parallelotope enclosure ____ and matrix-weighted fusion ____. How to design fusion criteria to reduce the conservatism of fusion estimators remains an open question.
		
	\textit{Notation}: Throughout this paper, we use the following notations. The identity matrix is denoted by $\mathbf{I}$, and a column vector with all ones is represented as $\mathbf{1}$. Let us define $\mathbb{N}_L := \{1,2,...,L\}$, where $L$ is a natural number excluding zero. Given sets $\Sigma_1$ and $\Sigma_2$, the notation $\Sigma_1 \setminus \Sigma_2$ denotes the set of elements in $\Sigma_1$ that are not in $\Sigma_2$. The trace, transpose, and inverse of a matrix $A$ are represented by $\mathrm{Tr}(A)$, $A^{\mathrm{T}}$, and $A^{-1}$, respectively. The matrix $A$ is said to be `spd' if it is symmetric ($A=A^{\mathrm{T}}$) and positive definite ($A \succ 0$, i.e., $\forall x \neq 0$, $x^{\mathrm{T}}Ax > 0$). Moreover, the infinite norm of a vector $x$ is denoted by $\|x\|_{\infty}$, and the weighted Frobenius norm of a matrix $R \in \mathbb{R}^{n \times r}$ is defined as $\|R\|_{W} := \sqrt{\mathrm{Tr}(R^{\mathrm{T}}WR)}$, where $W \in \mathbb{R}^{n \times n}$ is a positive definite matrix. The notation $|\cdot|$ represents the element-by-element absolute value operator, and $\mathrm{diag}(x)$ is a diagonal matrix with diagonal elements from the vector $x$. Given a matrix $R \in \mathbb{R}^{n \times r}$, the submatrix consisting of columns $a$ through $b$ is denoted as $R_{:,a:b}$ (with $R_{:,a}$ representing the $a$th column), and $\vec{R}$ is the matrix obtained by sorting the columns of $R$ in decreasing order of their weighted vector norm, such that $\left\|\vec{R}_{:, i}\right\|_W^2 \ge \left\|\vec{R}_{:, i+1}\right\|_W^2$.