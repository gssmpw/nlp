\section{Related work}
Several meshless shape optimization methods have already been studied in the literature. 
For instance, one can represent the boundary of a 2D shape by a truncated Fourier series, or a decomposition on the basis of (hyper)spherical harmonics in higher dimensions.
The PDE is then solved using some meshless method, such as the method of fundamental solutions \cite{Bogosel2016Nov,antunes_numerical_2017}. 
Although computationally efficient, this method can only deal with very specific equations on simply connected domains.
Previous works like \cite{deng_parametric_nodate} already used a neural network representation of the level set function, which falls in the broader category of \textit{parametrized level set methods} \cite{Cui2021Apr}. However, to the knowledge of the authors, the various parametrized level set methods for shape optimization still rely on the finite element method to compute the solution of the underlying PDE, and hence suffer from some of the previously detailed meshing issues and from the need of an ersatz material. Moreover, the flexibility of a neural network representation of the level set function for computing geometric quantities and taking geometric constraints into account has not been yet studied.

In recent years, physics-informed neural networks (PINNs) emerged as a new meshless method for PDEs \cite{raissi2017physics}. 
In this context the solution of a PDE is represented by a neural network, the parameters of which are optimized such that the PDE is solved approximately on a number of collocation points. 
The advantage of this approach is that all gradients involved in optimization can be automatically computed using backpropagation.
The main drawbacks are the lack of general convergence results of PINNs toward the solution of the PDE and the computational cost compared to well-established methods such as the finite element method \cite{grossmann2023physicsinformedneuralnetworksbeat}.
In the context of shape optimization, PINNs were used, for example, in \cite{Belieres--Frendo2024Jul}, where the shape is represented as a symplectic transformation of a given domain and the associated symplectomorphism as a neural network. 
While being meshless, this method does not allow for topology changes. In the context of linear elasticity, some fully meshless methods implemented the \emph{solid isotropic material with penalization} approach using only neural networks \cite{Zehnder2021Dec}. However, this method relaxes the original shape optimization problem into a density optimization problem, and does not readily generalize to other PDEs.