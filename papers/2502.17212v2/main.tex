\documentclass[lettersize,journal]{IEEEtran}
\usepackage{amsmath,amsfonts, bm}
\usepackage{algorithm}
\usepackage{algorithmic}
\usepackage{array}
\usepackage[caption=false,font=normalsize,labelfont=sf,textfont=sf]{subfig}
\usepackage{textcomp}
\usepackage{amssymb}
\usepackage{amsthm}
\usepackage{stfloats}
\usepackage{url}
\usepackage{verbatim}
\usepackage{graphicx}
\usepackage{cite}
%\hyphenation{op-tical net-works semi-conduc-tor IEEE-Xplore}
% updated with editorial comments 8/9/2021

% new commands
\newcommand{\real}{\mathbb{R}}
\newcommand{\nat}{\mathbb{N}}
\newcommand{\argmin}{\mathrm{arg~min}}
\newcommand{\diag}{\mathrm{diag}}
\newcommand{\A}{\mathbf{A}}
\newcommand{\E}{\mathbf{E}}
\newcommand{\e}{\mathbf{e}}
\newcommand{\x}{\mathbf{x}}
\newcommand{\X}{\mathbf{X}}
\newcommand{\ba}{\mathbf{a}}
\newcommand{\bS}{\mathbf{S}}
\newcommand{\s}{\mathbf{s}}

\newcommand{\esad}{\E\text{-}\mathrm{SAD}_\s}
\newcommand{\xsad}{\X\text{-}\mathrm{SAD}_\s}

\newtheorem{theorem}{Theorem}
\newtheorem{assumption}{Assumption}
\newtheorem{definition}{Definition}


\begin{document}

\title{A Two-step Linear Mixing Model for Unmixing \\ under Hyperspectral Variability}

\author{Xander Haijen, Bikram Koirala, Xuanwen Tao, and Paul Scheunders
        % <-this % stops a space
\thanks{Xander Haijen, Bikram Koirala, Xuanwen Tao, and Paul Scheunders are with the Imec-Visionlab research group, Department of Physics, University of Antwerp.}% <-this % stops a space
\thanks{Manuscript submitted to the IEEE transactions on image processing on 21 Feb 2025.}
}

% The paper headers
\markboth{}%
{Haijen \MakeLowercase{\textit{et al.}}: A Two-step Linear Mixing Model for Unmixing under Hyperspectral Variability}

% \IEEEpubid{0000--0000/00\$00.00~\copyright~2021 IEEE}
% Remember, if you use this you must call \IEEEpubidadjcol in the second
% column for its text to clear the IEEEpubid mark.

\maketitle

\begin{abstract}
Spectral unmixing is an important task in the research field of hyperspectral image processing. It can be thought of as a regression problem, where the observed variable (i.e., an image pixel) is to be found as a function of the response variables (i.e., the pure materials in a scene, called endmembers). The Linear Mixing Model (LMM) has received a great deal of attention, due to its simplicity and ease of use in, e.g., optimization problems. Its biggest flaw is that it assumes that any pure material can be characterized by one unique spectrum throughout the entire scene. In many cases this is incorrect: the endmembers face a significant amount of spectral variability caused by, e.g., illumination conditions, atmospheric effects, or intrinsic variability. Researchers have suggested several generalizations of the LMM to mitigate this effect. However, most models lead to ill-posed and highly non-convex optimization problems, which are hard to solve and have hyperparameters that are difficult to tune. 
In this paper, we propose a two-step LMM that bridges the gap between model complexity and computational tractability. We show that this model leads to only a mildly non-convex optimization problem, which we solve with an interior-point solver. This method requires virtually no hyperparameter tuning, and can therefore be used easily and quickly in a wide range of unmixing tasks. We show that the model is competitive and in some cases superior to existing and well-established unmixing methods and algorithms. We do this through several experiments on synthetic data, real-life satellite data, and hybrid synthetic-real data.
\end{abstract}

\begin{IEEEkeywords}
Remote sensing, hyperspectral unmixing, spectral variability, two-step linear mixing model, ELMM, SLMM, interior-point method
\end{IEEEkeywords}

\section{Introduction}
\section{Introduction}
\label{sec:introduction}
The business processes of organizations are experiencing ever-increasing complexity due to the large amount of data, high number of users, and high-tech devices involved \cite{martin2021pmopportunitieschallenges, beerepoot2023biggestbpmproblems}. This complexity may cause business processes to deviate from normal control flow due to unforeseen and disruptive anomalies \cite{adams2023proceddsriftdetection}. These control-flow anomalies manifest as unknown, skipped, and wrongly-ordered activities in the traces of event logs monitored from the execution of business processes \cite{ko2023adsystematicreview}. For the sake of clarity, let us consider an illustrative example of such anomalies. Figure \ref{FP_ANOMALIES} shows a so-called event log footprint, which captures the control flow relations of four activities of a hypothetical event log. In particular, this footprint captures the control-flow relations between activities \texttt{a}, \texttt{b}, \texttt{c} and \texttt{d}. These are the causal ($\rightarrow$) relation, concurrent ($\parallel$) relation, and other ($\#$) relations such as exclusivity or non-local dependency \cite{aalst2022pmhandbook}. In addition, on the right are six traces, of which five exhibit skipped, wrongly-ordered and unknown control-flow anomalies. For example, $\langle$\texttt{a b d}$\rangle$ has a skipped activity, which is \texttt{c}. Because of this skipped activity, the control-flow relation \texttt{b}$\,\#\,$\texttt{d} is violated, since \texttt{d} directly follows \texttt{b} in the anomalous trace.
\begin{figure}[!t]
\centering
\includegraphics[width=0.9\columnwidth]{images/FP_ANOMALIES.png}
\caption{An example event log footprint with six traces, of which five exhibit control-flow anomalies.}
\label{FP_ANOMALIES}
\end{figure}

\subsection{Control-flow anomaly detection}
Control-flow anomaly detection techniques aim to characterize the normal control flow from event logs and verify whether these deviations occur in new event logs \cite{ko2023adsystematicreview}. To develop control-flow anomaly detection techniques, \revision{process mining} has seen widespread adoption owing to process discovery and \revision{conformance checking}. On the one hand, process discovery is a set of algorithms that encode control-flow relations as a set of model elements and constraints according to a given modeling formalism \cite{aalst2022pmhandbook}; hereafter, we refer to the Petri net, a widespread modeling formalism. On the other hand, \revision{conformance checking} is an explainable set of algorithms that allows linking any deviations with the reference Petri net and providing the fitness measure, namely a measure of how much the Petri net fits the new event log \cite{aalst2022pmhandbook}. Many control-flow anomaly detection techniques based on \revision{conformance checking} (hereafter, \revision{conformance checking}-based techniques) use the fitness measure to determine whether an event log is anomalous \cite{bezerra2009pmad, bezerra2013adlogspais, myers2018icsadpm, pecchia2020applicationfailuresanalysispm}. 

The scientific literature also includes many \revision{conformance checking}-independent techniques for control-flow anomaly detection that combine specific types of trace encodings with machine/deep learning \cite{ko2023adsystematicreview, tavares2023pmtraceencoding}. Whereas these techniques are very effective, their explainability is challenging due to both the type of trace encoding employed and the machine/deep learning model used \cite{rawal2022trustworthyaiadvances,li2023explainablead}. Hence, in the following, we focus on the shortcomings of \revision{conformance checking}-based techniques to investigate whether it is possible to support the development of competitive control-flow anomaly detection techniques while maintaining the explainable nature of \revision{conformance checking}.
\begin{figure}[!t]
\centering
\includegraphics[width=\columnwidth]{images/HIGH_LEVEL_VIEW.png}
\caption{A high-level view of the proposed framework for combining \revision{process mining}-based feature extraction with dimensionality reduction for control-flow anomaly detection.}
\label{HIGH_LEVEL_VIEW}
\end{figure}

\subsection{Shortcomings of \revision{conformance checking}-based techniques}
Unfortunately, the detection effectiveness of \revision{conformance checking}-based techniques is affected by noisy data and low-quality Petri nets, which may be due to human errors in the modeling process or representational bias of process discovery algorithms \cite{bezerra2013adlogspais, pecchia2020applicationfailuresanalysispm, aalst2016pm}. Specifically, on the one hand, noisy data may introduce infrequent and deceptive control-flow relations that may result in inconsistent fitness measures, whereas, on the other hand, checking event logs against a low-quality Petri net could lead to an unreliable distribution of fitness measures. Nonetheless, such Petri nets can still be used as references to obtain insightful information for \revision{process mining}-based feature extraction, supporting the development of competitive and explainable \revision{conformance checking}-based techniques for control-flow anomaly detection despite the problems above. For example, a few works outline that token-based \revision{conformance checking} can be used for \revision{process mining}-based feature extraction to build tabular data and develop effective \revision{conformance checking}-based techniques for control-flow anomaly detection \cite{singh2022lapmsh, debenedictis2023dtadiiot}. However, to the best of our knowledge, the scientific literature lacks a structured proposal for \revision{process mining}-based feature extraction using the state-of-the-art \revision{conformance checking} variant, namely alignment-based \revision{conformance checking}.

\subsection{Contributions}
We propose a novel \revision{process mining}-based feature extraction approach with alignment-based \revision{conformance checking}. This variant aligns the deviating control flow with a reference Petri net; the resulting alignment can be inspected to extract additional statistics such as the number of times a given activity caused mismatches \cite{aalst2022pmhandbook}. We integrate this approach into a flexible and explainable framework for developing techniques for control-flow anomaly detection. The framework combines \revision{process mining}-based feature extraction and dimensionality reduction to handle high-dimensional feature sets, achieve detection effectiveness, and support explainability. Notably, in addition to our proposed \revision{process mining}-based feature extraction approach, the framework allows employing other approaches, enabling a fair comparison of multiple \revision{conformance checking}-based and \revision{conformance checking}-independent techniques for control-flow anomaly detection. Figure \ref{HIGH_LEVEL_VIEW} shows a high-level view of the framework. Business processes are monitored, and event logs obtained from the database of information systems. Subsequently, \revision{process mining}-based feature extraction is applied to these event logs and tabular data input to dimensionality reduction to identify control-flow anomalies. We apply several \revision{conformance checking}-based and \revision{conformance checking}-independent framework techniques to publicly available datasets, simulated data of a case study from railways, and real-world data of a case study from healthcare. We show that the framework techniques implementing our approach outperform the baseline \revision{conformance checking}-based techniques while maintaining the explainable nature of \revision{conformance checking}.

In summary, the contributions of this paper are as follows.
\begin{itemize}
    \item{
        A novel \revision{process mining}-based feature extraction approach to support the development of competitive and explainable \revision{conformance checking}-based techniques for control-flow anomaly detection.
    }
    \item{
        A flexible and explainable framework for developing techniques for control-flow anomaly detection using \revision{process mining}-based feature extraction and dimensionality reduction.
    }
    \item{
        Application to synthetic and real-world datasets of several \revision{conformance checking}-based and \revision{conformance checking}-independent framework techniques, evaluating their detection effectiveness and explainability.
    }
\end{itemize}

The rest of the paper is organized as follows.
\begin{itemize}
    \item Section \ref{sec:related_work} reviews the existing techniques for control-flow anomaly detection, categorizing them into \revision{conformance checking}-based and \revision{conformance checking}-independent techniques.
    \item Section \ref{sec:abccfe} provides the preliminaries of \revision{process mining} to establish the notation used throughout the paper, and delves into the details of the proposed \revision{process mining}-based feature extraction approach with alignment-based \revision{conformance checking}.
    \item Section \ref{sec:framework} describes the framework for developing \revision{conformance checking}-based and \revision{conformance checking}-independent techniques for control-flow anomaly detection that combine \revision{process mining}-based feature extraction and dimensionality reduction.
    \item Section \ref{sec:evaluation} presents the experiments conducted with multiple framework and baseline techniques using data from publicly available datasets and case studies.
    \item Section \ref{sec:conclusions} draws the conclusions and presents future work.
\end{itemize}

\section{Related work}
\putsec{related}{Related Work}

\noindent \textbf{Efficient Radiance Field Rendering.}
%
The introduction of Neural Radiance Fields (NeRF)~\cite{mil:sri20} has
generated significant interest in efficient 3D scene representation and
rendering for radiance fields.
%
Over the past years, there has been a large amount of research aimed at
accelerating NeRFs through algorithmic or software
optimizations~\cite{mul:eva22,fri:yu22,che:fun23,sun:sun22}, and the
development of hardware
accelerators~\cite{lee:cho23,li:li23,son:wen23,mub:kan23,fen:liu24}.
%
The state-of-the-art method, 3D Gaussian splatting~\cite{ker:kop23}, has
further fueled interest in accelerating radiance field
rendering~\cite{rad:ste24,lee:lee24,nie:stu24,lee:rho24,ham:mel24} as it
employs rasterization primitives that can be rendered much faster than NeRFs.
%
However, previous research focused on software graphics rendering on
programmable cores or building dedicated hardware accelerators. In contrast,
\name{} investigates the potential of efficient radiance field rendering while
utilizing fixed-function units in graphics hardware.
%
To our knowledge, this is the first work that assesses the performance
implications of rendering Gaussian-based radiance fields on the hardware
graphics pipeline with software and hardware optimizations.

%%%%%%%%%%%%%%%%%%%%%%%%%%%%%%%%%%%%%%%%%%%%%%%%%%%%%%%%%%%%%%%%%%%%%%%%%%
\myparagraph{Enhancing Graphics Rendering Hardware.}
%
The performance advantage of executing graphics rendering on either
programmable shader cores or fixed-function units varies depending on the
rendering methods and hardware designs.
%
Previous studies have explored the performance implication of graphics hardware
design by developing simulation infrastructures for graphics
workloads~\cite{bar:gon06,gub:aam19,tin:sax23,arn:par13}.
%
Additionally, several studies have aimed to improve the performance of
special-purpose hardware such as ray tracing units in graphics
hardware~\cite{cho:now23,liu:cha21} and proposed hardware accelerators for
graphics applications~\cite{lu:hua17,ram:gri09}.
%
In contrast to these works, which primarily evaluate traditional graphics
workloads, our work focuses on improving the performance of volume rendering
workloads, such as Gaussian splatting, which require blending a huge number of
fragments per pixel.

%%%%%%%%%%%%%%%%%%%%%%%%%%%%%%%%%%%%%%%%%%%%%%%%%%%%%%%%%%%%%%%%%%%%%%%%%%
%
In the context of multi-sample anti-aliasing, prior work proposed reducing the
amount of redundant shading by merging fragments from adjacent triangles in a
mesh at the quad granularity~\cite{fat:bou10}.
%
While both our work and quad-fragment merging (QFM)~\cite{fat:bou10} aim to
reduce operations by merging quads, our proposed technique differs from QFM in
many aspects.
%
Our method aims to blend \emph{overlapping primitives} along the depth
direction and applies to quads from any primitive. In contrast, QFM merges quad
fragments from small (e.g., pixel-sized) triangles that \emph{share} an edge
(i.e., \emph{connected}, \emph{non-overlapping} triangles).
%
As such, QFM is not applicable to the scenes consisting of a number of
unconnected transparent triangles, such as those in 3D Gaussian splatting.
%
In addition, our method computes the \emph{exact} color for each pixel by
offloading blending operations from ROPs to shader units, whereas QFM
\emph{approximates} pixel colors by using the color from one triangle when
multiple triangles are merged into a single quad.



\section{Proposed model (2LMM)}



\section{LISArD Methodology}
\label{sec:proposed_model}

\subsection{Adversarial Context and Preliminary}

\noindent\textbf{White-box Issues}. The white-box attacks are the strongest attacks for a specific model, yet if its training method slightly diverges, the same perturbations no longer have the identical effect as the model that was used to generate the adversarial samples. Furthermore, the white-box scenario requires that the attacker has access to the implementation/code of the model, which might not be realistic in most cases, since the attacker will rarely have access to the code of deployed models.


\textbf{Black-box Problems}. The black-box attacks are mainly focused on generating perturbations based on a low amount of knowledge, reducing the effect of the adversarial samples when compared to the white-box. However, the former attacks are more viable since the attacker only needs to know pairs of images and answers given by the target model to generate the perturbations. The black-box scenario can be considered as the most generic nowadays, since it does not require any information about the model, being potentially applicable to any available system exposing an DNN. Nevertheless, in this scenario, the attacker does not benefit from additional details of the target model, which hinders the probability of success.


\textbf{Proposed Solution}. We propose an alternative scenario in which the attacker knows the architecture of the model and dataset used to train it but does not have access to the gradients of the model. This information is usually accessible in papers or descriptive pages of the model, which can help the attacker to achieve stronger perturbations. This scenario is more realistic than white-box by compromising the amount of knowledge needed and the influence of the perturbations on the target model. LISArD considers using white-box attacks to generate adversarial samples against a model that uses the same architecture and dataset as the target model.


\textbf{Types of Approaches}. The two approaches described in the specialized literature and the approach proposed herein are summarized in Figure~\ref{fig:approaches}, which highlights the increased resources for performing \textit{Adversarial Distillation} and \textit{Purification} compared to LISArD. \textit{Adversarial Distillation} requires the usage of an additional previously trained model (Teacher) to aid in teaching the resilient network (Student), and \textit{Adversarial Purification} involves the training of a generative model (DDPM) to remove the adversarial noise during the inference phase (Purification). LISArD considers its attack scenario as a gray-box, meaning that the attacker only has partial knowledge about the target model. Thus, training to perceive noisy images (created by adding random Gaussian noise) similar to clean images can aid in defending against this type of attack.

\begin{figure}[!t]
    \centering
    \includegraphics[width=0.9\linewidth]{imgs/learning_image_similarity.png}
    \caption{Overview of the conversion from embeddings to a matrix in the Learning Image Similarity component. $E$ refers to the size of the embeddings, which vary depending on the selected model.}
    \label{fig:learning_image_similarity}
\end{figure}



\subsection{Image Similarity and Importance}

\noindent\textbf{Motivation}. Learning Image Similarity (LIS) is based on the idea that an image containing a reduced amount of noise does not affect the object represented in that image. Barlow Twins~\cite{zbontar2021barlow} proposes a procedure to reduce the redundancy between a pair of identical networks in the context of self-supervised learning. LISArD utilizes the redundancy reduction approach to teach the model to identify the noisy and clean images as similar and improve robustness against gray-box and white-box attacks. 

\textbf{Embeddings to Matrix Conversion}. An overview of the LIS component, explaining the conversion process from embeddings to a matrix, which is used to achieve redundancy reduction between images is provided in Figure~\ref{fig:learning_image_similarity}. The embeddings with size $E$ are extracted before being fed to the classification layer, and each clean embedding is multiplied by each noisy embedding to obtain the cross-correlation matrix. Then, this cross-correlation matrix is approximated to the diagonal matrix to achieve a perfect correlation.

\textbf{Weighted Training}. LISArD focuses on two main approaches: learning that two images are similar and simultaneously learning to classify the images, which motivates the usage of weighted training. Figure~\ref{fig:weighted_loss} explains how LISArD relates the LIS component with the classification one. As previously explained, the embeddings obtained from clean and noisy images are used in LIS while simultaneously being forwarded to the classification layer. The predictions are used in the (losses) $\mathcal{L}_{C}$ and $\mathcal{L}_{R}$ for the clean and noisy images, respectively, to train the classification component (the losses are better explained below). With this approach, we intend that the model initially concentrates on learning that two images represent the same object, but the final task is the classification of that object, justifying the initially increased importance of LIS and the gradually increasing importance of classification toward the end of training.

\begin{figure}[!t]
    \centering
    \includegraphics[width=0.9\linewidth]{imgs/weighted_loss.png}
    \caption{Overview of the LISArD~architecture. The clean and noisy images are fed to the model, and the inner product is calculated using their respective embeddings. Both clean (orange) and noisy embeddings (green) are used to predict each class using an adaptive weight loss between $\mathcal{L}_{C}$ and $\mathcal{L}_{R}$ and $\mathcal{L}_{S}$.}
    \label{fig:weighted_loss}
\end{figure}


\subsection{Loss Function}

\noindent The LISArD consists of a new defense mechanism that does not cost significantly more than the standard training but provides the networks with robustness against gray-box and white-box adversarial attacks. It starts by generating random images for every batch, according to the following equation:

\begin{equation}
x_{R} = x_{C} + \sqrt{\mu} \cdot x_{N},
\end{equation}
%
where $x_{R}$ refers to the random image, $x_{C}$ refers to the clean image, and $\mu$ is the maximum amount of perturbation to be added to the image (simulating the $\epsilon$ from adversarial attacks). $x_{N}$ refers to the Gaussian noise with the same size as the clean image. Since we have two images that are given as input to the model, we have a classification loss for each of them. Formally, this loss is defined as the comparison between the predicted label and the ground truth via Cross-Entropy:

\begin{equation}
\mathcal{L}_{\{C,R\}} = (y~\text{log}(p) + (1 - y)~\text{log}(1 - p)),
\end{equation}
%
where $\mathcal{L}_{\{C,R\}}$ refers to either the clean image loss or random image loss, $p$ are the predicted labels for the images batch, and $y$ are the ground truth labels for the images batch. Another part of the loss function consists of the approximation between the embeddings of each input image. The following equation translates this process:

\begin{equation}
    \mathcal{L}_{S} = \sum_i (1 - M_{ii})^2 + \lambda \sum_i \sum_{j \neq i} M_{ij}^2,
\end{equation}
%
where $\lambda$ is a positive constant that balances the importance of the terms and $M$ is the cross-correlation matrix obtained by the embeddings of the two images along the batch:

\begin{equation}
    M_{ij} = \frac{ \sum_b z^A_{b,i} z^B_{b,j} }{ \sqrt{\sum_b (z^A_{b,i})^2} \sqrt{\sum_b (z^B_{b,j})^2}},
\end{equation}
%
where $b$ is the index for the batch samples and $i$, $j$ are the indexes for the elements of the matrix. $M$ is a square matrix with a size equal to the network output. Finally, the complete loss function is expressed by:

\begin{equation}
\begin{aligned}
\mathcal{L}_{} = \alpha~(\mathcal{L}_{C} + \mathcal{L}_{R}) + (1-\alpha)~\left(\frac{\mathcal{L}_{S}}{\tau}\right), 
\end{aligned}
\label{eq:final_loss}
\end{equation}
%
where $\mathcal{L}_{C}$, $\mathcal{L}_{R}$, and $\mathcal{L}_{S}$ refer to the losses for clean images, random images (defined in equation 2), and similarity approximation (defined in equation 3). $\tau$ refers to the temperature and $\alpha$ refers to the weight for classification, with $\alpha$ starting at 0.5 and incrementing to 1 throughout training, as follows:

\begin{equation}
\begin{aligned}
\alpha = \alpha_{0} + \delta(\varepsilon-1),
\end{aligned}
\end{equation}
%
where $\alpha_{0}$ is the starting coefficient, defined as 0.5, $\delta$ is the decay degree, set to $\frac{1}{400}$ and $\varepsilon$ refers to the training epoch.



\subsection{Selected Attacks and State-of-the-art}

\noindent FGSM~\cite{goodfellow2015explaining}, PGD~\cite{madry2018towards}, and AA~\cite{croce2020reliable} attacks are selected to evaluate LISArD and compare it with state-of-the-art. FGSM is a one-step adversarial attack that uses the gradients of the model, being a weaker white-box adversarial attack. PGD is a strong attack that many defenses still fail to overcome and has multiple iterations that increase its strength. AA consists of an ensemble of attacks containing white-box and black-box variants, allowing an evaluation in both settings, which increases the scope of our evaluation. AdaAD~\cite{huang2023boosting}, PeerAiD~\cite{jung2024peeraid}, and DGAD~\cite{park2025dynamic} are the approaches selected to compare with LISArD since these \textit{Adversarial Distillation} models achieve state-of-the-art performance in white-box settings and have available implementations.



\subsection{Implementation Details}

\noindent\textbf{Hardware.} The experiments were performed in a multi-GPU server containing seven NVIDIA A40 and an Intel Xeon Silver 4310 \symbol{`@} 2.10 GHz, with the Pop!\_OS 22.04 LTS operating system. The models were trained using a single NVIDIA A40 GPU without additional models running on the same GPU when presenting the total time or time per epoch results.

\textbf{Models}. In order to be as comprehensive as possible regarding the multiple proposal of architectures, we selected ResNet18~\cite{he2016deep}, ResNet50~\cite{he2016deep}, ResNet101~\cite{he2016deep}, WideResNet28-10~\cite{zagoruyko2016wide}, VGG19~\cite{simonyan2014very}, MobileNetv2~\cite{sandler2018mobilenetv2}, and EfficientNetB2~\cite{tan2019efficientnet} as our backbones. 
For all the datasets, the networks were trained using an SGD optimizer with a learning rate of 0.001, a momentum of 0.9, and a weight decay of 0.0005 during 200 epochs. 
We disregarded the training of Inceptionv3~\cite{szegedy2016rethinking} due to its need to increase the image size to 299x299, which would not be the same training and evaluation settings as other models.

\textbf{Ablation Studies}. Models were trained for 200 epochs using ResNet18 as the backbone architecture for all ablation studies and evaluated on the CIFAR-10 clean, FGSM, PGD, and AA datasets. The last three datasets were generated by applying the respective attack to a previously trained ResNet18 on CIFAR-10 clean.

\section{Experiments with synthetic data}
In this section, we compare the three newly proposed algorithms to each other, and to several well-established methods based on the LMM, SLMM and ELMM. 

Throughout these experiments, results will be validated with the following metrics.
To validate the reconstruction error, the reconstruction Root Mean Square Error (RMSE) 
is defined as:
\[
\mathrm{RMSE}_\mathbf{X} = \sqrt{\frac{1}{NP} \sum_{n=1}^N \|\mathbf{\x}_n - \mathbf{\hat{\x}}_n\|^2_2}
\]
where $\hat{\x}$ denotes an estimated pixel and $\x$ denotes a measured pixel. 
Similarly, the reconstruction Spectral Angle Distance (SAD) in degrees is defined as:
\[
\mathrm{SAD}_\mathbf{X} = \frac{1}{NP} \sum_{n=1}^N \frac{\mathbf{x}_n^\top \mathbf{\hat{x}}_n}{\|\mathbf{x}_n\|_2 \|\mathbf{\hat{x}}_n\|_2} \times \frac{180^\circ}{\pi}.
\]

To validate the performance of the abundance estimation, we define the abundance RMSE as:
\[
\mathrm{RMSE}_\mathbf{A} = \sqrt{\frac{1}{NK} \sum_{n=1}^N \|\mathbf{a}_n - \mathbf{\hat{\ba}}_n\|^2_2}
\]
where $\hat{\ba}$ denotes an estimated abundance vector and $\ba$ is a ground truth abundance vector. 

To validate the performance of the scaling estimation, 
when unmixing 2LMM-generated data with 2LMM, the scaling RMSE is defined as:
\[
\mathrm{RMSE}_\mathbf{s} = \frac{1}{K+N}\|\mathbf{s} - \mathbf{\hat{\s}}\|_2
\]
where $\s = [\s_\E^\top~\s_\X^\top]^\top$ is the $(K+N)$-dimensional vector containing all scaling factors. However, the scaling RMSE might give a wrong picture about the estimation accuracy. This is because every pixel is influenced by both the EM scaling factors $\s_\E$ and its pixel scaling factor $\s_{\x_n}$. It is of no importance for the final result whether the largest scaling happens in the first scaling step or the second scaling step, as long as the \textit{resulting} scaling in each pixel is correct. This is not taken into account by the RMSE, so it will wrongly penalize correct scaling factors. For this, we propose the following error metric, which looks at the EM scaling step and pixel scaling step separately, and verifies whether the estimated vectors are scaled versions of the actual vectors, thus incorporating this indifference to how the scaling is distributed over the two steps. We call these error metrics the EM scaling SAD $\esad$ and the pixel scaling SAD $\xsad$:
\begin{align*}
    \esad &= \frac{1}{K} \frac{\s_\E^\top\hat{\s}_\E}{\|\s_\E\|_2\|\hat{\s}_\E\|_2} \times \frac{180°}{\pi} \\
    \xsad &= \frac{1}{N} \frac{\s_\X^\top\hat{\s}_\X}{\|\s_\X\|_2\|\hat{\s}_\X\|_2} \times \frac{180°}{\pi}
\end{align*}
This metric is a more truthful representation of the observable \textit{result} of the scaling on the pixels.

\subsection{Data}
We selected three EMs (asphalt (gds367), brick (gds350) and cardboard (gds371)) from the United States Geological Survey (USGS) spectral library \cite{kokaly_usgs_2017}, which contain 2152 spectral bands from the visible to the short-wave infrared range (200 nm to 2,500 nm). Their reflectance is shown in Fig. \ref{fig: usgs ems}.  For  computational considerations, we selected 224 equidistant bands for each EM. We call these reference EMs $\E_0$.

\begin{figure}[t]
    \centering
    \includegraphics[width=\linewidth]{Figs/usgs_ems.jpg}
    \caption{The EMs used for generating the synthetic data: asphalt (gds367), brick (gds350) and cardboard (gds371).}
    \label{fig: usgs ems}
\end{figure}

We generated synthetic abundance maps based on \emph{Gaussian Random Fields} (GRFs). Gaussian random fields can be thought of as spatially correlated Gaussian randomness \cite{kozintsev_computations_1999, noauthor_hyperspectral_nodate}, and they are a popular choice for generating synthetic hyperspectral data. We generate abundance maps using GRFs designed to comply with the ANC and ASC. The ground truth abundances are called $\A_\mathrm{gt}$. The scaling factors were drawn from the uniform distribution $\mathcal{U}([0.5; 1.5])$. The choice for this range is based on physical arguments, limiting the scaling factors to a meaningful range, as very large scaling factors or scaling factors close to zero are physically unrealistic.  The synthetic images were then designed to either comply with the 2LMM or the ELMM.

\subsubsection{2LMM-generated variability}
For generating synthetic data according to the 2LMM, we generate $N + K$ scaling factors, and group them in vectors $\s_\E$ and $\s_\X$. Then we generate the $n$-th pixel as
\[
\x_n = \E_0 \mathrm{diag}(\s_\E) \ba_{\mathrm{gt}, n} s_{\x_n}.
\]
We do not add any noise to the image.

\subsubsection{ELMM-generated variability}
For generating synthetic data according to the ELMM, we generate $NK$ scaling factors and combine these into $N$ vectors of dimension $K$, $\s_n, n=1,\ldots, N$. Then we generate the $n$-th pixel as
\[
\x_n = \E_0 \diag(\s_n) \ba_{\mathrm{gt}, n}.
\]
We do not add any noise to the image.

\subsection{Influence of the bounds $\underline{S}$ and $\overline{S}$} \label{sec: influence of bounds}
In this first experiment, a $50 \times 50$ synthetic image is generated with 2LMM-generated variability. 
We first validated the performance of the 3 proposed approaches to solve the optimization. We observed that the $\mathrm{2LMM}_\mathrm{norm}$ approach fails, because it consumes too much memory. During execution, it produces an out-of-memory error and is terminated by the operating system. The angle approach $\mathrm{2LMM}_\mathrm{angle}$ does not crash, but it is extremely slow and does not converge. It was automatically terminated after 1 million function evaluations, at which point it had run for approximately 4.5 hours. Based on this observation, we will use the two scaling factor approach $\mathrm{2LMM}$ for further experimentation.

To examine the effect of the bounds on the resulting estimates, we vary the lower and upper bounds $\underline{S}$ and $\overline{S}$. 
The bounds were taken to be $\left[ \frac{1}{\alpha}, \alpha \right]$ for different values of $\alpha > 1$. The results are shown in Table \ref{tab:small image grf}.
\begin{table}[t]
\caption{Unmixing results on a  $50 \times 50$ synthetic image with 2LMM-generated variability. The bounds $[\underline{S}, \overline{S}]$ are given by $[\frac{1}{\alpha}, \alpha]$. The best results are highlighted in bold.}
\centering
        \begin{tabular}{|r|cccccc|}
        \hline
        $\alpha$    & 100    & 10              & 5      & 2               & 4/3             & 10/9   \\ \hline \hline
        RMSE$_\X$   & 1e-6   & \textbf{9e-7}   & 2e-6   & 2e-6            & 2e-6            & 0.0042 \\ \hline
        SAD$_\X$    & 1e-4   & \textbf{4e-5}   & 1e-4   & 1e-4            & 1e-4            & 0.1063 \\ \hline
        RMSE$_\A$   & 0.0239 & 0.0239          & 0.0239 & 0.0239          & \textbf{0.0222} & 0.0951 \\ \hline
        RMSE$_\s$   & 57.804 & 4.8513          & 1.9099 & \textbf{0.1511} & 0.2039          & 0.3170 \\ \hline
        $\esad$     & 4.3382 & 4.3395          & 4.3393 & 4.3364          & \textbf{3.8441} & 18.260 \\ \hline
        $\xsad$     & 2.4598 & 2.4600          & 2.4602 & \textbf{2.4591} & 2.4976          & 8.9428 \\ \hline
        \end{tabular}
\label{tab:small image grf}
\end{table}
First, one can observe that the results are overall the best when the chosen bounds (i.e., $[\underline{S}, \overline{S}] = [0.5, 2]$) are closest to the actual range of scale values (i.e., $[0.5,1.5]$). However, the results
are not overly sensitive to changes of the bounds $\underline{S}$ and $\overline{S}$ and there is a fairly broad range of choices that lead to similar results. 
Nevertheless, the results suggest that the bounds should not be chosen too tight, as this will result in a feasible set that is very small. As a result, many good solutions will fall outside the feasible set, ending up with a poor solution. This is the case when $[\underline{S}, \overline{S}] = [\frac{9}{10}, \frac{10}{9}]$ where the abundance RMSE,  the reconstruction RMSE, the reconstruction SAD and the scaling SADs are higher than the cases with looser bounds.
When the bounds are chosen wider than the actual scaling range, the abundance estimates remain accurate, and the  reconstruction error remains low, but the scaling RMSE is very high. However, this is not an issue, since the scaling SADs are still low, so the total scaling is still estimated accurately.
In conclusion, since the priority is accurate abundance estimation, it is crucial to select sufficiently wide bounds that encompass a realistic range of scalings.

%This suggests that there is a balance in between, where the method is given enough freedom to explore possible solutions, but is still guided by bounds that are appropriately chosen. Given these observations, we continue in what follows with fixed bounds $[\underline{S}, \overline{S}] = [\frac{1}{2}; 2]$.


\subsection{Comparison to LMM, SLMM and ELMM}

In this experiment, synthetic data are generated using reference endmembers $\E_0$ and GRF-generated ground truth abundances $\A_\mathrm{gt}$. The image size is $100 \times 100$. The required number of scaling factors is sampled from the uniform distribution $\mathcal{U}([0.5, 1.5])$.
The bounds of 2LMM are chosen accordingly as $[\underline{S}, \overline{S}] = [0.5, 2]$.
We compare the performance of 2LMM to several well-established unmixing methods, more precisely: LMM (solved with FCLSU), SLMM (solved with CLSU) and ELMM (solved with alternating least-squares and ADMM, as described in \cite{drumetz_blind_2016}).
For the ELMM-based method, we test two variants: WS-ELMM, where the method is \textit{warm-started}, i.e., initialized with the abundance estimates from CLSU, and CS-ELMM, where the method is \textit{cold-started}, i.e., initialized with uniform abundance estimates $\frac{1}{K}$. 

\subsubsection{Performance under 2LMM-generated variability}
In this first experiment, the synthetic image is generated with 2LMM-generated variability. We compare the results of the 2LMM method to the three models mentioned above.  The results are shown in Table \ref{tab: results 2lmm lowvar}, along with the computation times. 

Overall, the 2LMM method is the best performing method at a reasonable cost. Given the fact that the ELMM is a model that is rich enough to describe any dataset with 2LMM-generated variability without modeling error, it is quite surprising that WS-ELMM fails to perform better than CLSU, which will have a possibly large model mismatch since it is too simple to describe most 2LMM-based models.

\begin{table}[t]
    \caption{Experimental results for synthetic data with  2LMM-generated variability. The best errors are highlighted in bold.}
    \begin{center}
    \begin{tabular}{|r|ccccc|}
    \hline
               & FCLSU  & CLSU   & WS-ELMM & CS-ELMM & 2LMM  \\ \hline \hline
    RMSE$_\X$  & 0.0167 & 0.0027 & 0.0129  & 0.0089  & \textbf{2e-6}   \\ \hline
    SAD$_\X$   & 3.9934 & 0.0715 & 1.0199  & 1.6937  & \textbf{0.0002}   \\ \hline
    RMSE$_\A$  & 0.2190 & 0.0919 & 0.0913  & 0.2598  & \textbf{0.0135}    \\ \hline
    $\Delta t$ & 17     & 16     & 31      & 96      & 48                 \\ \hline
    \end{tabular}
    \end{center}
    \label{tab: results 2lmm lowvar}
\end{table}

\begin{table}[t]
    \caption{Experimental results for synthetic data with  ELMM-generated variability. The best errors are highlighted in bold.}
    \begin{center}
    \begin{tabular}{|r|ccccc|}
    \hline
               & FCLSU  & CLSU   & WS-ELMM & CS-ELMM & 2LMM   \\ \hline \hline
    RMSE$_\X$  & 0.0207 & 0.0039 & 0.0118  & 0.0149  & \textbf{2e-7}   \\ \hline
    SAD$_\X$   & 1.7413 & 0.1044 & 0.5676  & 2.0905  & \textbf{2e-6}    \\ \hline
    RMSE$_\A$  & 0.1513 & 0.0744 & 0.0740  & 0.2307  & \textbf{0.0693}    \\ \hline
    $\Delta t$ & 19     & 17     & 26      & 70      & 62                 \\ \hline
    \end{tabular}
    \end{center}
    \label{tab:results elmm lowvar}
\end{table}

\subsubsection{Performance under ELMM-generated variability}
The experiment from the previous paragraph is repeated, but this time with the variability generated according to the ELMM. The results are shown in Table \ref{tab:results elmm lowvar}. Again, 2LMM performed the best overall.  
Regarding the abundance estimation, 2LMM, CLSU and WS-ELMM perform similarly. Interestingly, WS-ELMM only performs as good as CLSU, even though CLSU has a significant model mismatch, while ELMM is rich enough to describe the scene exactly, and is initialized using the CLSU estimates. The estimates of CS-ELMM are very poor, meaning that ELMM relies heavily on a good initial estimate. 
Because 2LMM is only mildly non-convex and because the cost function of 2LMM (Eq. \ref{eq: two scaling factor}) only consists of the reconstruction error, the local interior-point solver is able to find a (close to) global minimum for this problem, with a reconstruction RMSE that is very close to zero. This is not the case for WS-ELMM and CS-ELMM, since ELMM is highly non-convex, and its cost function includes regularization terms as well. This means that, even if a global minimum of the ELMM cost function was obtained, it is very unlikely to coincide with a near-zero reconstruction RMSE.
Lastly, FCLSU does not perform very well due to considerable model mismatch. 


\section{Experiments with real data}
\subsection{Houston dataset}

In this section, we use the Houston dataset, which comprises a hyperspectral image of the (now demolished) Robertson Stadium on the University of Houston Campus, acquired in 2012. The data consists of a  $150\times 218$ image with 144 spectral bands in the 380 nm to 1050 nm region. A high-resolution RGB image of the scene, taken from a different angle, is shown in Fig. \ref{fig:robertson stadium}. The dataset is part of a larger dataset, which was used in the 2013 GRSS Data Fusion Contest \cite{debes_hyperspectral_2014}. The EMs are (red) roofs, vegetation, concrete, and asphalt. 
\begin{figure}[t]
    \centering
    \includegraphics[width=0.7\linewidth]{Figs/robertson_stadium.jpg}
    \caption{An RGB image of Robertson Stadium, Houston, Texas}
    \label{fig:robertson stadium}
\end{figure}
\begin{figure}[t]
    \centering
    \includegraphics[width=\linewidth]{Figs/houston_endmembers.jpg}
    \caption{The spectral signatures of the four reference EMs in the Houston dataset}
    \label{fig:reference endmembers}
\end{figure}
The signatures are shown in Fig. \ref{fig:reference endmembers}. 

On this image, the methods FCLSU, CLSU, WS-ELMM, CS-ELMM and 2LMM are run. For a fair comparison, all methods are applied "off-the-shelf", meaning that none of the hyperparameters are  tuned based on the data or on the acquired results. For WS-ELMM and CS-ELMM,  the standard regularization terms are used. For 2LMM, the standard bounds of $[\frac{1}{2}; 2]$ are used.

\subsubsection{Abundance estimations}

The abundance maps estimated by the five methods are shown in Fig. \ref{fig:abundance maps}. Except for FCLSU and CS-ELMM, which produce poor results, all obtained abundance maps are quite similar, with some notable differences. The abundance maps of WS-ELMM are less granular and more smooth. This is a direct result of the spatial regularization terms which are used in WS-ELMM. This however also leads to some misestimations by WS-ELMM caused by oversmoothing of, e.g., small grass patches and small roofs at the entrance of the stadium (left-center of the hyperspectral image). This highlights the difficulty in properly setting the regularization parameters.

\begin{figure}[t]
    \centering
    \includegraphics[width=\linewidth]{Figs/houston_abundances.jpg}
    \caption{Abundance maps of the four unmixing methods on the Houston data. A brighter pixel means a larger abundance.}
    \label{fig:abundance maps}
\end{figure}

\subsubsection{Reconstruction error}

Unlike the synthetic data, no ground truth is available, and no abundance RMSE can be obtained. Therefore, the performance is judged by  the reconstruction RMSE and SAD. 
The mean reconstruction RMSE and reconstruction SAD are shown in Table \ref{table: houston}. One can observe that 2LMM  has the lowest reconstruction error, followed by CLSU and WS-ELMM. Timings indicate that the cost of the proposed approach is moderate.

\begin{table}[t]
\caption{Reconstruction errors and timings for the Houston experiment. The best errors are highlighted in bold.}
\centering
\begin{tabular}{r|ccccc|}
\cline{2-6}
\multicolumn{1}{l|}{}           & FCLSU & CLSU  & WS-ELMM & CS-ELMM & 2LMM  \\ \hline \hline
\multicolumn{1}{|r|}{RMSE$_\X$} & 0.048 & 0.014 & 0.014   & 0.018   & \textbf{0.006}    \\ \hline
\multicolumn{1}{|r|}{SAD$_\X$}  & 3.306 & 1.925 & 2.166   & 4.462   & \textbf{1.454}    \\ \hline
\multicolumn{1}{|r|}{$\Delta t$ (s)}  & 21    & 20    & 48      & 79      & 86                \\ \hline
\end{tabular}
\label{table: houston}
\end{table}

Fig. \ref{fig: reconstruction error} shows the reconstruction SAD. 
One can observe that WS-ELMM and CLSU mostly make larger errors in the northern stands of the stadium. The stands are made of concrete, but they can reflect light in a complicated way due to the many steps and different angles at which the material is present. WS-ELMM and CLSU are having difficulty capturing this variability. Another cause of errors are the red roofs at the entrance of the stadium (left-center of the image), although the other methods also make large errors here. Next to the small roofed structures, the entrance is also lined with trees (as can be seen in the RGB image in Fig. \ref{fig:robertson stadium}), which can cause light to be reflected in a nonlinear way. This causes misestimations in all methods.

\begin{figure*}[t]
    \centering
    \includegraphics[width=\linewidth]{Figs/houston_reconstruction_err.jpg}
    \caption{Reconstruction SAD (in degrees) for the five unmixing methods on the Houston dataset.}
    \label{fig: reconstruction error}
\end{figure*}

\subsection{DLR HySU Dataset} \label{sec: dlr hysu}

In this experiment, we generate an image using the DLR HySU dataset \cite{cerra_dlr_2021}, a benchmark dataset for evaluating spectral unmixing algorithms, featuring airborne hyperspectral and RGB imagery of synthetic targets with known materials and sizes. The dataset was captured at the DLR (German Aerospace Center) premises in Oberpfaffenhofen, Germany. It consists of several checkerboard patterns of various size laid out on a grass field.

%\subsubsection{Large targets} \label{sec: dlr hysu large targets}

We use a sub-image of the total dataset, which contains the largest checkerboard pattern of the five materials (and a sixth background material). This is a $13 \times 16$ image with 135 spectral bands covering the wavelength range 416 nm -- 903 nm. The EM signatures are provided along with the dataset.
%Since this is a small image, we can also test the norm-adjusted and angle approach, solved in MATLAB. 
An annotated RGB image of the scene is shown in Figure \ref{fig:dlr rgb}.

Hyperspectral variability in this scene is negligible, since there are no shadows, topographical features or other factors that impact EM signatures. With this in mind, we use FCLSU to find the abundances, which we consider as the ground truth abundances. Then, using this ground truth, we re-generate the image, but this time we introduce spectral variability.

\subsubsection{Performance under 2LMM-generated variability}

The scaling factors are generated with variability according to the 2LMM. Let $\E_0$ be the provided EMs and $\A_\mathrm{gt}$ the ground truth abundances, and let $\s_\E \in \real^K$ and $\s_\X \in \real^N$ be vectors with its elements drawn from the uniform distribution $\mathcal{U}([0.5,1.5])$. Then the synthetic image $\X_\mathrm{syn}$ is generated as:
\[
    \X_\mathrm{syn} = \E_0  \diag (\s_\E) \A_\mathrm{gt} \mathrm{diag}(\s_\X) + \mathbf{e}_\X
\]
with $\mathbf{e}_\mathbf{X}$ normally distributed noise with a signal-to-noise ratio (SNR) of 60 dB. 

\begin{figure}[t]
    \centering
    \includegraphics[width=0.3\linewidth]{Figs/RGB_image_annotated.png}
    \caption{RGB image of a subset of the DLR HySU dataset, with five materials (1. bitumen, 2. green fabric, 3. red fabric, 4. red metal, 5. blue fabric) arranged in a checkerboard pattern on a grass background (sixth material).}
    \label{fig:dlr rgb}
\end{figure}

Unmixing is performed with the same five methods as before. The resulting SAD$_\X$, RMSE$_\X$ and abundance RMSEs, separately for each material, are shown in Table \ref{table: dlr hysu 2lmm}. The resulting abundance maps are shown in Fig. \ref{fig: dlr hysu abundances}. One can observe that 2LMM performs best. Other methods have difficulty identifying the squares, and make a considerable error in doing so. They also mistake certain materials for another, e.g., parts of the red fabric square are identified as red metal.

\begin{figure}[t]
    \centering
    \includegraphics[width=\linewidth]{Figs/dlr_abundances.jpg}
    \caption{Ground truth abundance maps (GT) and abundance maps for five unmixing methods on the \textit{DLR HySU} dataset with 2LMM-generated variability. A brighter pixel means a larger abundance.}
    \label{fig: dlr hysu abundances}
\end{figure}

\begin{table}[htb!]
\caption{Abundance and reconstruction errors for the DLR dataset with 2LMM-generated variability. $\mathrm{RMSE}_i$ denotes the abundance RMSE for the $i$-th material (1. bitumen, 2. green fabric, 3. red fabric, 4. red metal, 5. blue fabric, 6. grass background). The best results  are highlighted in bold. }
\centering
    \begin{tabular}{r|ccccc|}
    \cline{2-6}
    \multicolumn{1}{l|}{}            & FCLSU  & CLSU   & WS-ELMM  & CS-ELMM  & 2LMM \\ \hline \hline
    \multicolumn{1}{|r|}{SAD$_\X$}   & 4.9664 & 2.3890 & 2.6672   & 5.9272   & \textbf{1.8644} \\ \hline
    \multicolumn{1}{|r|}{RMSE$_\X$}  & 0.0809 & 0.0555 & 0.1182   & \textbf{0.0123}   & 0.1341 \\ \hline \hline
    \multicolumn{1}{|r|}{RMSE$_1$}   & 0.0740 & 0.0608 & 0.0587   & 0.2425   & \textbf{0.0108} \\ \hline
    \multicolumn{1}{|r|}{RMSE$_2$}   & 0.2725 & 0.0824 & 0.0799   & 0.2669   & \textbf{0.0179} \\ \hline
    \multicolumn{1}{|r|}{RMSE$_3$}   & 0.1363 & 0.1195 & 0.1165   & 0.2087   & \textbf{0.0200} \\ \hline
    \multicolumn{1}{|r|}{RMSE$_4$}   & 0.1025 & 0.0924 & 0.0915   & 0.2156   & \textbf{0.0146} \\ \hline
    \multicolumn{1}{|r|}{RMSE$_5$}   & 0.0866 & 0.1103 & 0.1083   & 0.2143   & \textbf{0.0167} \\ \hline
    \multicolumn{1}{|r|}{RMSE$_6$}   & 0.2564 & 0.0766 & 0.0739   & 0.4123   & \textbf{0.0379} \\ \hline \hline
    \multicolumn{1}{|r|}{RMSE$_\A$}  & 0.1742 & 0.0925 & 0.0895   & 0.2696   & \textbf{0.0215} \\ \hline
    \end{tabular}
\label{table: dlr hysu 2lmm}
\end{table}

\subsubsection{Performance under ELMM-generated variability}
We repeat the above experiment, but now with the variability generated according to the ELMM. As before, we draw scaling factors from the distribution $\mathcal{U}([0.5;1.5])$. Now, we generate $NK$ of them, and group them into $N$ scaling vectors $\s_n, ~n=1,\ldots, N$. The pixels are then generated using
\begin{equation}
\x_n = \E_0 \diag(\s_n) \ba_{\mathrm{gt}, n} + \mathbf{e}_{\x_n}
\end{equation}
with $\mathbf{e}_{\x_n}$ a noise term with an SNR of 60 dB. 
The resulting SAD$_\X$, RMSE$_\X$ and and abundance RMSE's, separately for each material are shown in Table \ref{tab: dlr elmm abundances}. The resulting abundance maps are shown in Fig. \ref{fig: dlr elmm abundances}.

Since the scaling terms can now vary significantly between pixels and EMs, we can expect a modeling error with 2LMM. However, the resulting estimates produced by 2LMM are still better than those by the ELMM-based methods, who possess the modeling capability to reconstruct the image without error, apart from noise. Even here, 2LMM outperforms the other methods in terms of abundance estimation and reconstruction SAD. This indicates that 2LMM unmixing is quite robust to deviations from the model assumption and can be a reliable alternative for existing mixing models.

\begin{table}[htb!]
    \caption{Abundance and reconstruction errors for the DLR dataset with ELMM-generated variability.  $\mathrm{RMSE}_i$ denotes the abundance RMSE for the $i$-th material (1. bitumen, 2. green fabric, 3. red fabric, 4. red metal, 5. blue fabric, 6. grass background). The best results are highlighted in bold.}
    \centering
        \begin{tabular}{r|ccccc|}
        \cline{2-6}
        \multicolumn{1}{l|}{}            & FCLSU  & CLSU            & WS-ELMM  & CS-ELMM  & 2LMM    \\ \hline \hline
        \multicolumn{1}{|r|}{SAD$_\X$}   & 4.1926 & 2.2171          & 2.2383   & 5.0003   & \textbf{1.7828}   \\ \hline
        \multicolumn{1}{|r|}{RMSE$_\X$}  & 0.0642 & \textbf{0.0290} & 0.0649   & 0.0706   & 0.0771   \\ \hline \hline
        \multicolumn{1}{|r|}{RMSE$_1$}   & 0.0843 & 0.0452          & 0.0452   & 0.2462   & \textbf{0.0329} \\ \hline
        \multicolumn{1}{|r|}{RMSE$_2$}   & 0.2157 & 0.0584          & 0.0583   & 0.2547   & \textbf{0.0279} \\ \hline
        \multicolumn{1}{|r|}{RMSE$_3$}   & 0.1094 & 0.0618          & 0.0617   & 0.2061   & \textbf{0.0320} \\ \hline
        \multicolumn{1}{|r|}{RMSE$_4$}   & 0.0837 & 0.0613          & 0.0612   & 0.2538   & \textbf{0.0186} \\ \hline
        \multicolumn{1}{|r|}{RMSE$_5$}   & 0.0564 & 0.0809          & 0.0806   & 0.2132   & \textbf{0.0380} \\ \hline
        \multicolumn{1}{|r|}{RMSE$_6$}   & 0.1964 & 0.0749          & 0.0747   & 0.4417   & \textbf{0.0646} \\ \hline \hline
        \multicolumn{1}{|r|}{RMSE$_\A$}  & 0.1381 & 0.0648          & 0.0646   & 0.2808   & \textbf{0.0384} \\ \hline
        \end{tabular}
    \label{tab: dlr elmm abundances}
\end{table}

\begin{figure}[t]
    \centering
    \includegraphics[width=\linewidth]{Figs/dlr_abundances_elmm.jpg}
    \caption{Ground truth abundance maps (GT) and abundance maps for five unmixing methods on the \textit{DLR HySU} dataset with ELMM-generated variability. A brighter pixel means a larger abundance.}
    \label{fig: dlr elmm abundances}
\end{figure}

\section{Conclusion}
In this work, we have presented the 2LMM, a novel physically motivated two-step linear mixing model that mitigates the effect of spectral variability. The model bridges the gap between model complexity and computational tractability. A key feature of the 2LMM is that it leads to a mildly non-convex unmixing problem, which we solve using an interior-point method. Experiments on synthetic and real hyperspectral data show that the 2LMM achieves competitive performance against existing methods and exhibits robustness to deviations from its underlying assumptions.


\section*{Acknowledgments}
The research presented in this paper is funded by the Research Foundation-Flanders - project G031921N. Bikram Koirala is a postdoctoral fellow of the Research Foundation Flanders, Belgium (FWO: 1250824N-7028). The authors  acknowledge the team of Daniele Cerra at DLR for the development of the DLR HySU  dataset.

\bibliographystyle{IEEEtran}
\bibliography{main}

{\appendices
\section{Assumption on the feasible set $\mathcal{X}$}
In order for the barrier function to be well-defined we require that the problem (\ref{eq: interior point problem}) admits at least one strictly feasible solution \cite{boyd_convex_2004}:
\begin{assumption}
    The feasible set $\mathcal{X} \subseteq \real^D$ is nonempty and the problem (\ref{eq: interior point problem}) is \textbf{strictly feasible}, i.e. 
        \[
    \exists \Bar{\x} \in \real^D: g_i(\Bar{\x}) > 0, \forall i, ~\mathbf{C}\Bar{\x} = \mathbf{b}, ~\Bar{\x} > 0.
        \]
\end{assumption}

\section{Newton's method for equality-constrained optimization}
For a fixed value of $\mu$, we can formulate the Karush-Kuhn-Tucker (KKT) optimality conditions for the problem (\ref{eq: barrier problem}) and solve them. The KKT conditions start from the Lagrangian $\mathcal{L}_\mu(\x, \bm{\lambda})$ \cite[Ch. 5]{boyd_convex_2004} and read
\begin{equation} \label{eq: kkt system}
    \begin{aligned}
        \nabla_{\bm{\lambda}} \mathcal{L}_\mu (\x, \bm{\lambda}) &= 0 \\
        \nabla_\x \mathcal{L}_\mu (\x, \bm{\lambda}) &= 0.
    \end{aligned}
\end{equation}
This is a system of nonlinear equations, and can be solved using Newton's method for nonlinear equations. For this, define
\[
\textbf{F}_\mu(\x, \bm{\lambda}) = \left( \begin{array}{c}
     - \nabla_{\bm{\lambda}} \mathcal{L}_\mu (\x, \bm{\lambda}) \\ \nabla_\x \mathcal{L}_\mu (\x, \bm{\lambda})
\end{array} \right).
\]
A step $\Delta^k = (\Delta_\x^k, \Delta_{\bm{\lambda}}^k)$ is found by solving
\[
\mathbf{J}_{\textbf{F}_\mu}(\x_k, \mu_k) \Delta^k = - \textbf{F}_\mu(\x_k, \bm{\lambda}_k)
\]
where $\mathbf{J}_{\textbf{F}_\mu}$ is the Jacobian matrix. The solution is then updated using a line search procedure:
\begin{align*}
    \x_{k+1} &= \x_k + \alpha_k \Delta_\x^k \\
    \bm{\lambda}_{k+1} &= \bm{\lambda}_k + \alpha_k \Delta_{\bm{\lambda}}^k
\end{align*}
where the step size $\alpha_k$ is determined using a backtracking line search algorithm.


\section{Convergence}
A formal convergence proof is beyond the scope of this paper, so instead we sketch a convergence analysis based on \cite{boyd_convex_2004}. The analysis consists of two parts, proving convergence of the inner (Newton) loop and outer loop, respectively. Fix the following sequence for the barrier parameter $\mu$: $\mu_0, \nu \mu_0, \nu^2 \mu_0, \ldots$ for $0 < \nu < 1$.

\subsubsection*{Inner loop} 
We make the following assumptions:
\begin{assumption} Consider the problem (\ref{eq: interior point problem}) and its corresponding barrier problem (\ref{eq: barrier problem}). The barrier problem can always be solved using Newton's method, or equivalently:
        \begin{enumerate}
            \item $f(\x)$ and $g_i(\x)$ are closed on $\mathcal{X}$, i.e., the set
            \[
            \{\x \in \mathcal{X} \mid f(\x) \leq \alpha\}
            \]
            is closed for any $\alpha \in \real$, similarly for $g_i(\x)$.
            \item For all $\x \in \mathcal{X}$, we have $\| \x \|_2^2 \leq R^2$ for some $R < +\infty$.
        \end{enumerate}
\end{assumption}
It follows from Assumption 2 that each barrier problem can be solved using Newton's method in a finite number of steps. Bounding the number of steps is hard without making additional assumptions on the problem. If we assume $B(\x, \mu)$ is closed and \textit{self-concordant} for all $\mu \leq \mu_0$, and assume the sublevel sets of the problem (\ref{eq: interior point problem}) are bounded, we can provide an upper bound on the required number of Newton steps. If we solve the problem to an accuracy of $\epsilon_\mathrm{N} > 0$, then we need at most
\[
\frac{I}{\gamma}(\nu - 1 - \log \nu) + \log_2 \log_2 \frac{1}{\epsilon_\mathrm{N}}
\]
steps, where $\gamma$ is a constant determined by the backtracking line search procedure, and $I$ is the number of inequality constraints.

\subsubsection*{Outer loop} 
If the barrier problem (\ref{eq: barrier problem}) can be minimized using Newton's method for the sequence $\{\mu_k\}_{k \geq 0}$ as mentioned above, then we can achieve a desired accuracy $\epsilon_\mathrm{B} > 0$ after
\[
\left\lceil \frac{\log \left( I \mu_0 / \epsilon_\mathrm{B} \right)}{\log 1/\nu}\right\rceil + 1
\]
steps. Therefore, since we can solve both the outer problem and inner problem in finitely many steps, we can guarantee that the algorithm will always converge to a locally optimal solution in finite time.

\section{Slack variables}
We replace all logarithms of the form $\log (g_i(\x))$ by the constrained form \cite{geletu_introduction_nodate} 
\[
\log \sigma_i~\mathrm{s.t.}~g_i(\x)-\sigma_i = 0.
\]
For the two scaling factor approach, this leads to the modified barrier function 
\begin{multline*}
    \Tilde{B}(\A_\s, \s_\E) = J(\A_\s, \s_\E) - \mu \Big( \sum_{k=1}^K ( \log \sigma_k^+ + \log \sigma_k^-) + \\
                \sum_{k=1}^K \sum_{n=1}^N (\log \sigma_{nk} + \log a_{nk}) \Big)
\end{multline*}
and the optimization problem
\begin{equation*}
\begin{aligned}
        \min &~\Tilde{B}(\A_\s, \s_\E) \\
    \mathrm{s.t.} &~ s_{\e_k} - \underline{S} - \sigma^+_k = 0, \quad \forall k\\
                &~\overline{S} - s_{\e_k} - \sigma^-_k = 0, \quad \forall k \\
                &~\overline{S} - a_{nk} - \sigma_{nk} = 0, \quad \forall n, k
\end{aligned}
\end{equation*}
which is equivalent to the original barrier problem (\ref{eq: barrier problem}).
}
\newpage




\vfill

\end{document}


