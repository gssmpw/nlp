\documentclass[lettersize,journal]{IEEEtran}
\usepackage{amsmath,amsfonts, bm}
\usepackage{algorithm}
\usepackage{algorithmic}
\usepackage{array}
\usepackage[caption=false,font=normalsize,labelfont=sf,textfont=sf]{subfig}
\usepackage{textcomp}
\usepackage{amssymb}
\usepackage{amsthm}
\usepackage{stfloats}
\usepackage{url}
\usepackage{verbatim}
\usepackage{graphicx}
\usepackage{cite}
%\hyphenation{op-tical net-works semi-conduc-tor IEEE-Xplore}
% updated with editorial comments 8/9/2021

% new commands
\newcommand{\real}{\mathbb{R}}
\newcommand{\nat}{\mathbb{N}}
\newcommand{\argmin}{\mathrm{arg~min}}
\newcommand{\diag}{\mathrm{diag}}
\newcommand{\A}{\mathbf{A}}
\newcommand{\E}{\mathbf{E}}
\newcommand{\e}{\mathbf{e}}
\newcommand{\x}{\mathbf{x}}
\newcommand{\X}{\mathbf{X}}
\newcommand{\ba}{\mathbf{a}}
\newcommand{\bS}{\mathbf{S}}
\newcommand{\s}{\mathbf{s}}

\newcommand{\esad}{\E\text{-}\mathrm{SAD}_\s}
\newcommand{\xsad}{\X\text{-}\mathrm{SAD}_\s}

\newtheorem{theorem}{Theorem}
\newtheorem{assumption}{Assumption}
\newtheorem{definition}{Definition}


\begin{document}

\title{A Two-step Linear Mixing Model for Unmixing \\ under Hyperspectral Variability}

\author{Xander Haijen, Bikram Koirala, Xuanwen Tao, and Paul Scheunders
        % <-this % stops a space
\thanks{Xander Haijen, Bikram Koirala, Xuanwen Tao, and Paul Scheunders are with the Imec-Visionlab research group, Department of Physics, University of Antwerp.}% <-this % stops a space
\thanks{Manuscript submitted to the IEEE transactions on image processing on 21 Feb 2025.}
}

% The paper headers
\markboth{}%
{Haijen \MakeLowercase{\textit{et al.}}: A Two-step Linear Mixing Model for Unmixing under Hyperspectral Variability}

% \IEEEpubid{0000--0000/00\$00.00~\copyright~2021 IEEE}
% Remember, if you use this you must call \IEEEpubidadjcol in the second
% column for its text to clear the IEEEpubid mark.

\maketitle

\begin{abstract}
Spectral unmixing is an important task in the research field of hyperspectral image processing. It can be thought of as a regression problem, where the observed variable (i.e., an image pixel) is to be found as a function of the response variables (i.e., the pure materials in a scene, called endmembers). The Linear Mixing Model (LMM) has received a great deal of attention, due to its simplicity and ease of use in, e.g., optimization problems. Its biggest flaw is that it assumes that any pure material can be characterized by one unique spectrum throughout the entire scene. In many cases this is incorrect: the endmembers face a significant amount of spectral variability caused by, e.g., illumination conditions, atmospheric effects, or intrinsic variability. Researchers have suggested several generalizations of the LMM to mitigate this effect. However, most models lead to ill-posed and highly non-convex optimization problems, which are hard to solve and have hyperparameters that are difficult to tune. 
In this paper, we propose a two-step LMM that bridges the gap between model complexity and computational tractability. We show that this model leads to only a mildly non-convex optimization problem, which we solve with an interior-point solver. This method requires virtually no hyperparameter tuning, and can therefore be used easily and quickly in a wide range of unmixing tasks. We show that the model is competitive and in some cases superior to existing and well-established unmixing methods and algorithms. We do this through several experiments on synthetic data, real-life satellite data, and hybrid synthetic-real data.
\end{abstract}

\begin{IEEEkeywords}
Remote sensing, hyperspectral unmixing, spectral variability, two-step linear mixing model, ELMM, SLMM, interior-point method
\end{IEEEkeywords}

\section{Introduction}
\documentclass[../main.tex]{subfiles}
\graphicspath{{../images/}}
\makeatletter
\def\input@path{{../images/}}
\makeatother
\begin{document}
\section{Introduction}
\begin{figure}
\centering
\begin{tikzpicture}
\node[inner sep=0pt] (ws) at (0, 0) {
\includegraphics[height=.4\textwidth, trim={10cm 0 10cm 0},clip]{world_space.png}};
\node[inner sep=0pt] (cs) at (6,0) {\includegraphics[height=.4\textwidth, trim={10cm 1cm 10cm 4cm},clip]{conf_space.png}};
\end{tikzpicture}
\vspace{-5pt}
\label{fig:pbrm_intro}
\caption{\textbf{Left}: Shows world space obstacles as grey spheres. Robots start and goal configuration is colored red and green, respectively. Configurations along the computed path are colored transparent blue. \textbf{Right:} Mapped world space scenario to configuration space. Obstacle region is the grey mesh. Red spheres are collision-free regions computed by the neural SCDF. The optimized shortest path in the convex corridor is the blue curve.}
\vspace{-25pt}
\end{figure}
Motion planning is the problem of finding a collision-free trajectory that connects a given start and goal configuration. The planning takes place in the configuration space of the robot. For single body robots, like mobile robots or drones, the configuration space and the world space are usually the same. This simplifies the planning, since explicit obstacle representations are available which enables geometrical tools like separating hyperplanes, smallest distance to obstacles etc., to be used when designing motion planning algorithms. For multi-body robots like manipulators, the situation is completely different. The world space obstacles are usually mapped to non-convex regions, and to make the problem even harder, the mapping is usually not known. Forming explicit representations of the obstacle region in the configuration space is usually too expensive or intractable. Despite all of this, sampling based planners are used with great success, which mainly is due to their use of implicit representations of the obstacle region. The basic idea is to construct a graph in the configuration space that covers and connects the collision-free region. From this graph, a path can be extracted that connects a given start and goal configuration. The approach is computationally expensive, since the graph is constructed with the smallest geometrical building block available, points, which represents a collision-check. Furthermore, the extracted paths from the graph are non-smooth and jagged due to the stochastic nature of the approach. This adds an additional post-processing step to the process, where the paths are shortcutted and smoothened, before the path can be used for tracking. Clearly a lot of time is invested to form this graph and produce smooth paths. Thus, if the obstacles start to move, then all of this work is done in no use, since all points that make up this graph need to be re-verified, which is simply too time consuming to be done in real time.
\\\\
In this work, we want to address the existing drawbacks of the sampling based planners. Our main contribution is an improved motion planner where each vertex in the graph covers a collision-free region in the form of a sphere instead of a point and where the edges are formed with neighboring intersecting spheres. This representation has the advantage of instead of returning piecewise linear paths, returning a sequence of overlapping spheres, i.e. a convex corridor, that connects a given start and goal configuration, illustrated in Figure \ref{fig:pbrm_intro}. This convex corridor allows us to use convex optimization to produce smooth trajectories, instead of computationally expensive post-processing methods. The representation further allows us to estimate the coverage of the collision-free space, which gives us awareness and feedback in the offline roadmap construction phase. Finally, our representation is simple to adapt to moving obstacles, simply requery for the new radii and recheck for intersections. 
\\\\
The spherical collision-free regions are formed using a signed distance function (SDF), which is a function that returns the smallest distance from an arbitrary point to the boundary of an obstacle. As the name implies, the distance is signed, thus if the point is inside the obstacle it is negative otherwise positive. If the distance is positive, a sphere with radius equal to the distance is guaranteed to cover a collision-free region. Using an SDF in motion planning is not new, but what is novel about our approach is that we express the distance in the configuration space instead of the world space and by doing so allows us to form these convex collision-free regions. We refer to the resulting SDF as a signed configuration distance function (SCDF). Computing an SCDF analytically is non-trivial, our approach is therefore to parameterize the SCDF with a deep neural network and learn the mapping by supervised learning. Our resulting neural SCDF can compute distances for different parameter values of obstacle shapes and we also show how multiple distances can be combined, thus making our approach flexible.
\section{Related work}
Motion planning algorithms can roughly be divided into three families, grid-based, sampling based and optimization based methods. Grid-based methods (GBM) discretize the planning space from which a graph is then compiled. A standard search method is A$^\star$ \citep{a_star}, which is classified as an \textit{informed} search method, since it employs a heuristic function to speed up the search. A$^\star$ guarantees to return an optimal path at the level of discretization used. GBMs usually discretize the planning space by a regular lattice and this limits the GBMs to problems with low dimensionality due to the curse of dimensionality. Thus, GBMs are usually limited to single-body robots where the degrees of freedom (DOF) are low. To overcome the inherent scaling problem with the GBMs, stochastic methods are usually used for multi-body robots. These methods are termed as sampling-based methods (SBM) and core members within this family are the rapidly-exploring random trees (RRT) \citep{rrt} and the probabilistic roadmap (PRM) \citep{prm}. RRT grows a tree from the start configuration and explores the collision-free region in a rapid way until it is able to connect to the goal region. RRT is usually improved by bi-directional planning \citep{rrt_connect}, i.e. an additional tree is grown from the goal configuration and the trees are tested for connection after any tree has been expanded. RRT is a single-query method, thus it searches for a path from scratch each time it is queried. Contrary to this, PRM is a multi-query method, which solves for multiple queries without starting from scratch. PRM does this by creating a roadmap (graph) that covers the collision-free space as an offline step. The graph is then used to solve for multiple queries. PRMs are used in cases where the environment does not change since the extra offline step is too computationally costly and needs to be re-done if the environment is changed. In our work, we address this inherent issue by using a different roadmap representation. Our vertices in the graph cover a collision-free region in the form of spheres and we form the edges by checking for intersecting spheres. If something in the environment changes, we recompute the spheres radii and recheck the intersections, without relying on collision detection. We use a trained neural network to compute the sphere radius, therefore querying for the radius can be done fast, hence our representation enables the PRM for dynamic environments.
\\\\
In the recent decades, optimization based methods (OBM) \citep{chomp, schulman, itomp, stomp} have been introduced as an alternative to SBM for multi-body robots. Like the SBM, the OBMs scale well to higher dimensional problems and produce smoother motion. It is common to use a SDF in the optimization since it is a smooth function, thus enabling gradient-based methods. However, the standard way of expressing the SDF is in world space. The distance therefore needs to be mapped to the configuration space by the forward kinematics. This mapping makes the optimization problem a non-linear program (NLP), which is computationally expensive to solve. Recently, a different approach has been proposed. In \cite{mp_gcs} motion planning is formulated as a convex optimization problem by using the graph of convex sets framework \citep{gcs}. The underlying idea is to decompose the collision-free space into intersecting convex sets from which a convex optimization problem is formulated. In cases where an explicit representation of the obstacles in the configuration space exists, like for single-body robots, creating collision-free convex regions can be done fast \citep{iris}. For multi-body robots, this is non-trivial. Existing work does this successfully \citep{iris_nlp, iris_c} by an optimization based approach, but the methods are still too time consuming to be used in the presence of moving obstacles. Our approach is instead to use deep learning to learn an SDF expressed in the configuration space. With this, we can query for shortest distances to the collision boundary, which allows us to expand spherical regions which are collision-free. Our approach is fast and therefore enables our suggested roadmap planner to be used in dynamic environments.
\\\\
Recent research has focused on learning collision detection \citep{fk_kernel_distance, diffco, graphdistnet} by predicting the signed distance between the robot links and the surrounding obstacles in the world space. The learned SDF is used in trajectory optimization but since the distance is expressed in the world space, the problem becomes an NLP and therefore takes a long time to solve. We take a novel approach and suggest to instead express the signed distance in the configuration space. This allows us to improve the PRM at the same time as it enables convex optimization for trajectory optimization, which runs faster and is more reliable than NLP solvers. In \cite{cspf} a learned signed distance function in the configuration space is proposed similar to our approach. However, their approach is restricted to point cloud representations, while we propose to represent the obstacles as parameterized geometric shapes, e.g. spheres. Furthermore, we also show how to use our learned SCDF to improve an existing roadmap planner.
\section{Problem formulation}
A robot is located in the world space, $\W \subset \R^3 $. The unique location of the robot is given by its configuration $\q \in \C$, where $\C$ is the configuration space. The set of points covered by the robots bodies at a certain configuration is expressed as $\B(\q) \subset \W$. The robot is surrounded by $\NrObst$ obstacles $\O = \bigcup_{i=1}^{\NrObst} \O_i$, where  $\O_i \subset \W$. The representation of the obstacle in the configuration space is the set $\C\O_i = \{\q \in \C \: |\: \B(\q) \cap \O_i \neq \emptyset \}$. The obstacle space is formed as $\Co = \bigcup_{i=1}^{\NrObst} \C \O_i$. The complement is referred to as the free space, $\Cf = \C \setminus \Co$. The path planning problem is a tuple, ($\Cf$, $\qStart$, $\qGoal$), where we want to connect a query pair, consisting of a start, $\qStart$, and goal configuration, $\qGoal$, with a geometric path, $\q(s): [0, 1] \mapsto \Cf$, such that $\q(0)=\qStart$ and $\q(1)=\qGoal$, or report correctly when such a path does not exist.
\end{document}


\section{Related work}

\section{Related Work} \label{sec:related}

% \textbf{Adversarial Attack}
\textbf{Attacks on SLAM.} 
%With the rise of machine learning, 
The robustness of computer vision systems is being actively investigated. With the emergence of adversarial images in the digital domain by adding optimized noise directly to images~\cite{szegedy2013intriguing,carlini2017towards}, researchers find that such attacks also exist physically in the real world \cite{eykholt2018robust,song2018physical,zhao2019seeing}. To fill the gap between attacks in the digital and physical worlds, recent studies have demonstrated that attacks on real-world computer vision systems are practical \cite{eykholt2018robust,li2019adversarial,man2020ghostimage,sharif2016accessorize,zhao2019seeing,zhou2018invisible}. However, attacks on traditional computer vision methods such as SLAM are relatively less explored. \cite{yoshida2022adversarial} proposes an attack against the scan matching algorithm in LiDAR-based SLAM, while most SLAMs in AR/VR devices rely on different sensors like RGB/depth cameras and IMUs. \cite{ikram2022perceptual} and \cite{chen2024adversary} mislead visual SLAM by poisoning the images with special patterns, and \cite{wang2021can} causes the camera to fail using infrared light. In our work, we demonstrate attacks on Visual-Inertial SLAM (VI-SLAM) by perturbing the IMU readings, rather than cameras, and showing its impact on XR user experience. 

\textbf{Acoustic Injection Attacks.} Among various physical attacks, acoustic injection attacks are attractive due to their low cost. Son~\etal~\cite{son2015rocking} were the first to introduce acoustic attacks on MEMS gyroscopes, demonstrating how these attacks could lead to sensor denial-of-service and result in drone crashes. WALNUT~\cite{trippel2017walnut} expanded on this by developing output biasing and control attacks that enable precise manipulation of MEMS accelerometer outputs using modulated sound waves. Wang et al.~\cite{wang2017sonic} demonstrated a sonic gun, showcasing the vulnerability of various smart devices (\eg drones and self-balancing vehicles) to acoustic attacks. Tu et al. \cite{tu2018injected} designed side-swing and switching attacks to alter the outputs of MEMS gyroscopes and accelerometers. Furthermore, Ji et al. \cite{ji2021poltergeist} fool the object detectors by applying acoustic attack to the image stabilizers commonly used in modern cameras. However, none of the existing works study the relationship between the acoustic injections and SLAM outputs on recent XR devices. 

% \zijian{Do we need one session about security in AR/VR?}
% \yicheng{TODO}
%\jiasi{cite the AIVR paper (UMass Amherst?) paper is we have not already. They add IMU perturbation but w/o SLAM, iirc} \yicheng{Cited}

\textbf{XR Security and Privacy.} 
%Security and privacy concerns in XR systems have gained significant attention. 
For single-user XR systems, researchers have demonstrated various side-channel attacks to extract sensitive information (\eg keystrokes) through video feeds~\cite{ling2019know}, head movements~\cite{nair2023unique, slocum2023going}, architectural hints~\cite{zhang2023its,shang2020arspy}, power usage~\cite{li2024dangers}, and EM side-channel leakages~\cite{al2021vr}. In multi-user XR systems, Su et al.~\cite{su2024remote} use avatar motion data to infer keystrokes in shared VR environments. Slocum et al.~\cite{slocum2024doesn} reveal vulnerabilities in the shared state frameworks of multi-user AR. Similarly, Lebeck et al.~\cite{lebeck2017securing} highlight risks like deceptive virtual objects and emphasize access control for managing shared physical and virtual spaces. Ruth et al.~\cite{ruth2019secure} further propose a secure multi-user AR framework focusing on content sharing and permissions.
Chandio et al.~\cite{chandio2024stealthy} %introduced a multi-modal spatiotemporal attack that 
simultaneously manipulated visual and inertial sensors to disrupt XR pose estimation. However, their study evaluated the attack using offline datasets and assumed the attacker's capability to manipulate IMU data streams through acoustic means, without real experiments. Ours is the first to demonstrate acoustic injection attacks on recent XR devices, like the Hololens 2, in the real world.
 



\section{Proposed model (2LMM)}
\subsection{Motivation and model description} 

We propose a new model, which bridges the gap between the simple SLMM, and the rich, but complicated ELMM. The proposed model, which we call the \textbf{two-step linear mixing model} (2LMM), is a physically motivated model that uses reference EMs extracted from the image or provided in a spectral library.

The 2LMM balances the computational ease of the SLMM and the model complexity of the ELMM. The model is not as complicated as the ELMM, leading to better-posed optimization problems. On the other hand, it is richer than the SLMM, so it can model more diverse scenes.

Using the reference EMs $\E$, the model is constructed as follows. As a first scaling step, the EMs are scaled independently of each other, but in the same way across the entire image. Then, the EMs are  linearly combined to form unscaled pixels. The second scaling step then consists of scaling each mixed pixel independently. See Fig. \ref{fig: 2lmm concept} for a conceptual representation of the 2LMM mixing process.  The 2LMM can still model material-specific variability, but in a more constrained way than the ELMM. 

The 2LMM is constructed with the following acquisition scenario in mind. Assume that reference EMs have been obtained, either from the image itself, or from some spectral library. The acquisition conditions to generate  the reference signatures might differ significantly from the acquisition conditions of  the image, such that the reference EMs will be scaled versions of the actual EMs in the image. If the EEA extracts an EM from a heavily illuminated region, it will have to be scaled to obtain the actual EM in standard conditions. This is especially true when reference EMs are obtained from a spectral library. The first scaling step of the EMs  corrects for this effect. The pixel scaling step  then further corrects for any pixel-wise illumination differences. 

\begin{figure}
    \centering
    \includegraphics[width=\linewidth]{Figs/2LMM.pdf}
    \caption{A graphical representation of the 2LMM model assumption. \textbf{1.} The reference EMs (blue lines) are scaled independently (red lines). \textbf{2.} The scaled EMs are mixed to form the unscaled pixels in the image. \textbf{3.} Each pixel is scaled independently to form the final image.}
    \label{fig: 2lmm concept}
\end{figure}

\subsection{Mathematical formulation}

Let $\s_\E \in \real^K$ be the EM scaling vector representing the first scaling step, and $s_{\x_n}$ a pixel-dependent scaling factor representing the second scaling step. Then the $n$-th pixel in the 2LMM is given by:
\begin{equation*}
    \x_n = \E\mathrm{diag}(\s_\E)\ba_n s_{\x_n}.
\end{equation*}
We can combine this for all pixels. Let 
\[
\s_\X = [s_{\x_1}~s_{\x_2}~\cdots~s_{\x_N}]^\top
\]
denote the pixel scaling vector. Then the 2LMM at the image level is given by:

\begin{equation} \label{eq: 2lmm cost function}
    \X = \E\mathrm{diag}(\s_\E)\A\mathrm{diag}(\s_\X).
\end{equation}



\subsection{Performing unmixing with the 2LMM}
Consider a non-convex FCLSU problem for the 2LMM:
\begin{equation}
\begin{aligned}
    \min_{\A, \s_\E, \s_\X} &~\|\Hat{\X} - \E \diag(\s_\E)\A\diag(\s_\X)\|_F^2 \\
    \text{s.t.} &~ 0 \leq \A \leq 1, \mathbf{1}^\top \A = \mathbf{1} \\
    &~ \underline{S} \leq \s_\E \leq \overline{S}, \quad \underline{S} \leq \s_\X \leq \overline{S}
\end{aligned}
\end{equation}
where the box bounds $\underline{S}, \overline{S} > 0$ can be used to constrain the scaling variables to a user-specified interval. Naturally, $\underline{S} < \overline{S}$. The motivation for introducing these box bounds is both physical and mathematical. First, it allows us to constrain the scaling factors to a physically meaningful range, since in many cases credible assumptions can be made about the magnitude of the scaling factors. Secondly, it makes the problem easier to solve mathematically, since it reduces the size of the search space, and therefore reduces the probability of finding a sub-optimal solution to the non-convex cost function.

The quantity in which we are interested is the abundance matrix $\A$. The estimates of the scaling variables $\s_\E$ and $\s_\X$ are less important. However, this increase in variables might lead to worse optimization results because the problem becomes more complex. Therefore, we propose the following optimization strategies to (partially) avoid the need to estimate the scaling factors.

\subsubsection{Scaling-independent optimization}

The cost function for the model (\ref{eq: 2lmm cost function}) can be written as a sum over the different pixels:
\[
J(\A, \s_\E, \s_\X) = \sum_{n=1}^N \| \hat{\x}_n - \E \diag(\s_\E) \ba_n s_{\x_n}\|_2^2.
\]
To remove the need to estimate the pixel scaling factors, we may now divide both terms by their norm, to obtain the norm-divided cost function:
\begin{equation} \label{eq: norm division cost function}
    \Tilde{J}(\A, \s_\E) = \sum_{n=1}^N \left\| \frac{\hat{\x}_n}{\|\hat{\x}_n\|_2} - \frac{\E \diag(\s_\E) \ba_n}{\|\E \diag(\s_\E) \ba_n\|_2}\right\|_2^2
\end{equation}
This cost function defines the \textit{norm division approach}:
\begin{equation} \label{eq: norm division}
   (\mathrm{2LMM}_\mathrm{norm}):~
   \begin{aligned}
        \min_{\A, \s_\E} &~\sum_{n=1}^N \left\| \frac{\hat{\x}_n}{\|\hat{\x}_n\|_2} - \frac{\E \diag(\s_\E) \ba_n}{\|\E \diag(\s_\E) \ba_n\|_2}\right\|_2^2 \\
    \text{s.t.} &~ 0 \leq \A \leq 1,~\mathbf{1}^\top \A = \mathbf{1}  \\
    &~ \underline{S} \leq \s_\E \leq \overline{S}
\end{aligned}
\end{equation}
By normalizing the two terms in Eq. (\ref{eq: norm division cost function}), we discard the length of the vectors in order to minimize the \textit{angle} between them. We can do this more explicitly. Using the standard inner product in $\real^D$:
\[
\langle \mathbf{u}, \mathbf{v} \rangle = \sum_{d=1}^D u_d v_d = \mathbf{u}^\top \mathbf{v},
\]
the angle between two vectors $\mathbf{u}$ and $\mathbf{v}$ is given by:
\[
\angle(\mathbf{u},\mathbf{v}) = \arccos \left( \frac{\mathbf{u}^\top \mathbf{v}}{\|\mathbf{u}\|_2 \|\mathbf{v}\|_2}\right).
\]
This suggests using the following cost function:
\begin{equation}
J(\A, \s_\E) = \sum_{n=1}^N \arccos \left( \frac{(\E \diag (\s_\E) \ba_n)^\top \hat{\x}_n}{\|\E \diag (\s_\E) \ba_n)\|_2 \|\hat{\x_n}\|_2 }\right)    
\end{equation}
where the factor $s_{\x_n}$ is canceled because it is a scalar. The arc cosine makes this a highly nonlinear function. By removing the arc cosine, the negative of the argument needs to be minimized (since the derivative of the arc cosine is always negative). Therefore we define the new cost function as:
\begin{equation} \label{eq: angle cost function}
    \Tilde{J}(\A, \s_\E) = - \sum_{n=1}^N \frac{(\E \diag (\s_\E) \ba_n)^\top \hat{\x}_n}{\|\E \diag (\s_\E) \ba_n)\|_2 \|\hat{\x}_n\|_2 }
\end{equation}
and the corresponding optimization problem, which we will call the \textit{angle approach}:
\begin{equation}
    (\mathrm{2LMM}_\mathrm{angle}):~
    \begin{aligned}
            \min_{\A, \s_\E} &~- \sum_{n=1}^N \frac{(\E \diag (\s_\E) \ba_n)^\top \hat{\x}_n}{\|\E \diag (\s_\E) \ba_n)\|_2 \|\hat{\x}_n\|_2 } \\
            \text{s.t.} &~ 0 \leq \A \leq 1,~\mathbf{1}^\top \A = \mathbf{1}  \\
                        &~ \underline{S} \leq \s_\E \leq \overline{S}.
    \end{aligned}
\end{equation}
Conceptually, this makes the optimization problem simpler, since we reduce the parameter space by $N$ dimensions. However, there are some drawbacks, especially from a computational perspective. Since both approaches have variables in both the numerator and denominator, every function evaluation will involve a very expensive division operation. When a division is performed on a computer, it involves an iterative process of subtractions and comparisons, which is notoriously slow \cite[Ch. 3.4]{patterson_computer_2017} and can produce considerable numerical errors. Additionally, this can lead to excessive memory requirements. Therefore, we propose a third approach, which avoids expensive divisions, and is inspired by CLSU.
\subsubsection{Two scaling factor approach}
A third approach combines the matrices $\A$ and $\s_\X$ of the cost function (\ref{eq: 2lmm cost function}) into one matrix $\A_\s$, and drops the ASC. This leads to the optimization problem
\begin{equation} \label{eq: two scaling factor}
  (\mathrm{2LMM}):~ \begin{aligned}
        \min_{\A_\s, \s_\E} &~\|\Hat{\X} - \E \diag(\s_\E)\A_\s\|_F^2 \\
    \text{s.t.} &~ 0 \leq \A_\s \leq \overline{S},  \\
    &~ \underline{S} \leq \s_\E \leq \overline{S}
\end{aligned}
\end{equation}
The actual abundances and pixel scaling factors are easily recovered using the normalization step (\ref{eq: normalization}). We will refer to this approach by the model name, 2LMM, or by calling it the \emph{two scaling factor approach}.

\subsection{Optimization algorithm}
To perform spectral unmixing using the 2LMM, we will use an interior-point (IP) method. Interior-point methods can solve general nonlinear constrained minimization problems \cite{geletu_introduction_nodate}. 

\subsubsection{Concept of IP methods}
In what follows we give a high-level overview of the IP method. Formal assumptions and discussions which are not essential to understanding the concept of the algorithm are deferred to the appendices. Consider the general problem:
\begin{equation} \label{eq: interior point problem}
   \begin{aligned}
    \min_{\x \in \real^D} &~f(\x) \\ \mathrm{s.t.}&~g_i(\x) \geq 0,  &i=1,2,\ldots, I \\ &~\mathbf{C}\x = \mathbf{b}\\ &~\x\geq0
    \end{aligned} 
\end{equation}
where $f$ need not be convex. Furthermore, define the feasible set $\mathcal{X}$ as the set of all points that satisfy the constraints:
\[
\mathcal{X} = \left\{ \x \in \real^D \Bigg| \begin{array}{c}
     g_i(\x) \geq 0,~i=1,2,\ldots, I \\ \mathbf{C}\x = \mathbf{b}\\ \x \geq 0
\end{array}\right\}.
\]
 The main idea of the IP method is to replace  the inequality constraints in the cost function with penalty terms  that approach infinity as the argument approaches the boundary of the feasible set, and that are very small when the argument falls within the feasible set. For the IP algorithm to work, we require the feasible set $\mathcal{X}$ to be \textit{large enough} (see App. A for a formal assumption), so that it allows the definition of a \textbf{barrier function}, in this case a log barrier function:
\[
B(\x, \mu) = f(\x) - \mu \left( \sum_{i=1}^I \log(g_i(\x)) + \sum_{d = 1}^D \log(x_d)\right)
\]
A {barrier function} is a function that approaches $f(\x)$ as $\mu$ decreases but still approaches $+\infty$ as $\x$ approaches the boundary of the feasible set (see Fig. \ref{fig:barrier fct}). Instead of the original problem (\ref{eq: interior point problem}), one can now consider the problem:
\begin{equation} \label{eq: barrier problem}
\begin{aligned}
    \min_{\x \in \real^D} &~B(\x, \mu) \\
    \mathrm{s.t.}&~\mathbf{C}\x = \mathbf{b}.
\end{aligned}
\end{equation}

\begin{figure}
    \centering
    \includegraphics[width=0.9\linewidth]{Figs/barrier_fct.jpg}
    \caption{A barrier function for $f(x) = 2(x - 0.7)^2$ and $\mathcal{X} = [0, 1]$. As the value of $\mu$ decreases, the barrier function approaches $f(x)$ while still approaching infinity at the boundaries.}
    \label{fig:barrier fct}
\end{figure}

The interior-point method is an extension to the barrier method, where the problem (\ref{eq: barrier problem}) is solved several times using Newton's method for a decreasing sequence of nonnegative numbers $\{\mu_k\}_{k \geq 0}$. This way, we obtain a sequence $\{(\x_{\mu_k}, \bm{\lambda}_{\mu_k})\}_{k \geq 0}$ of optimal solutions, where $\bm{\lambda}_k$ is the dual variable or Lagrange multiplier (see App. B). This sequence is called the \emph{primal-dual path}. Under mild assumptions, the primal-dual path converges to a locally optimal solution of the problem (\ref{eq: interior point problem}) (see App. C for a more elaborate convergence discussion). The full conceptual algorithm is shown in Algorithm \ref{alg: interior point}. For more information on interior-point and related methods, see \cite[Ch. 11]{boyd_convex_2004}.

\begin{algorithm}[t]
\caption{Interior-point algorithm}\label{alg: interior point}
\begin{algorithmic}
\FOR {$k = 0, 1, 2, \ldots$}
\STATE {Construct the barrier function $B(\x, \mu_k)$}
\STATE {Solve the problem (\ref{eq: barrier problem}) using, e.g., Newton's method}
\STATE {Call the solutions $(\x_{\mu_k}, \bm{\lambda}_{\mu_k})$}
\IF {some termination criterion is met}
\STATE{Terminate and return $\x^\star = \x_{\mu_k}$}
\ENDIF
\ENDFOR
\end{algorithmic}
\end{algorithm}



\subsubsection{The IP method for 2LMM}
Consider the two scaling factor approach (\ref{eq: two scaling factor}), with the cost function
\[
J(\A_\s, \s_\E) = \|\hat{\X} - \E \diag(\s_\E)\A_\s\|_F^2
\]
and the constraints $0 \leq \A_\s \leq \overline{S}$ and $\underline{S} \leq \s_\E \leq \overline{S}$. There are no equality constraints. The inequality constraints can be written as:
\begin{equation}
    \begin{aligned}
    s_{\e_k} - \underline{S} & \geq 0, \quad \overline{S} - s_{\e_k} \geq 0, & \quad \forall k  \\
    a_{nk} & \geq 0, \quad  \overline{S} - a_{nk} \geq 0, & \quad \forall n, k
    \end{aligned}
\end{equation}
leading to the barrier function
\begin{multline*}
    B(\A_\s, \s_\E, \mu) = J(\A_\s, \s_\E) -   \\
    \mu \Big(\sum_{k=1}^K (\log (s_{\e_k} - \underline{S}) + \log (\overline{S} - s_{\e_k})) + \\
    \sum_{k=1}^K \sum_{n=1}^N (\log (\overline{S} - a_{nk}) + \log (a_{nk}))\Big).
\end{multline*}
We have omitted the terms $\log (s_{\e_k})$ since they are redundant. The barrier function for the other approaches is similar, and given by
\begin{multline*}
    \Tilde{B}(\A, \s_\E, \mu) = \Tilde{J}(\A, \s_\E) -   \\
    \mu \Big(\sum_{k=1}^K (\log (s_{\e_k} - \underline{S}) + \log (\overline{S} - s_{\e_k})) + \\
    \sum_{k=1}^K \sum_{n=1}^N (\log (1 - a_{nk}) + \log (a_{nk}))\Big)
\end{multline*}
with $\Tilde{J}(\A, \s_\E)$ either the norm division cost function (\ref{eq: norm division cost function}) or the angle cost function (\ref{eq: angle cost function}). To improve the numerical behavior of the algorithm, slack variables are often introduced in the barrier function (see App. D for the modified cost function in case of the two scaling factor approach). 

\subsubsection{Implementation}
Several general-purpose software implementations of the interior-point algorithm exist. For the two scaling factor approach 2LMM, we use the Ipopt solver \cite{wachter_line_2005} interfaced through the Julia programming language. For the norm division approach $\mathrm{2LMM}_\mathrm{norm}$ and the angle approach $\mathrm{2LMM}_\mathrm{angle}$, we use \textsc{Matlab}'s interior-point solver, implemented in the $\texttt{lsqnonlin}$ and $\texttt{fmincon}$ functions from the \textsc{Matlab} Optimization Toolbox. All experiments were run on a desktop computer with a 32-core Intel i9 CPU with a 3-level cache and 64 GiB RAM (DIMM).

\section{Experiments with synthetic data}
In this section, we compare the three newly proposed algorithms to each other, and to several well-established methods based on the LMM, SLMM and ELMM. 

Throughout these experiments, results will be validated with the following metrics.
To validate the reconstruction error, the reconstruction Root Mean Square Error (RMSE) 
is defined as:
\[
\mathrm{RMSE}_\mathbf{X} = \sqrt{\frac{1}{NP} \sum_{n=1}^N \|\mathbf{\x}_n - \mathbf{\hat{\x}}_n\|^2_2}
\]
where $\hat{\x}$ denotes an estimated pixel and $\x$ denotes a measured pixel. 
Similarly, the reconstruction Spectral Angle Distance (SAD) in degrees is defined as:
\[
\mathrm{SAD}_\mathbf{X} = \frac{1}{NP} \sum_{n=1}^N \frac{\mathbf{x}_n^\top \mathbf{\hat{x}}_n}{\|\mathbf{x}_n\|_2 \|\mathbf{\hat{x}}_n\|_2} \times \frac{180^\circ}{\pi}.
\]

To validate the performance of the abundance estimation, we define the abundance RMSE as:
\[
\mathrm{RMSE}_\mathbf{A} = \sqrt{\frac{1}{NK} \sum_{n=1}^N \|\mathbf{a}_n - \mathbf{\hat{\ba}}_n\|^2_2}
\]
where $\hat{\ba}$ denotes an estimated abundance vector and $\ba$ is a ground truth abundance vector. 

To validate the performance of the scaling estimation, 
when unmixing 2LMM-generated data with 2LMM, the scaling RMSE is defined as:
\[
\mathrm{RMSE}_\mathbf{s} = \frac{1}{K+N}\|\mathbf{s} - \mathbf{\hat{\s}}\|_2
\]
where $\s = [\s_\E^\top~\s_\X^\top]^\top$ is the $(K+N)$-dimensional vector containing all scaling factors. However, the scaling RMSE might give a wrong picture about the estimation accuracy. This is because every pixel is influenced by both the EM scaling factors $\s_\E$ and its pixel scaling factor $\s_{\x_n}$. It is of no importance for the final result whether the largest scaling happens in the first scaling step or the second scaling step, as long as the \textit{resulting} scaling in each pixel is correct. This is not taken into account by the RMSE, so it will wrongly penalize correct scaling factors. For this, we propose the following error metric, which looks at the EM scaling step and pixel scaling step separately, and verifies whether the estimated vectors are scaled versions of the actual vectors, thus incorporating this indifference to how the scaling is distributed over the two steps. We call these error metrics the EM scaling SAD $\esad$ and the pixel scaling SAD $\xsad$:
\begin{align*}
    \esad &= \frac{1}{K} \frac{\s_\E^\top\hat{\s}_\E}{\|\s_\E\|_2\|\hat{\s}_\E\|_2} \times \frac{180°}{\pi} \\
    \xsad &= \frac{1}{N} \frac{\s_\X^\top\hat{\s}_\X}{\|\s_\X\|_2\|\hat{\s}_\X\|_2} \times \frac{180°}{\pi}
\end{align*}
This metric is a more truthful representation of the observable \textit{result} of the scaling on the pixels.

\subsection{Data}
We selected three EMs (asphalt (gds367), brick (gds350) and cardboard (gds371)) from the United States Geological Survey (USGS) spectral library \cite{kokaly_usgs_2017}, which contain 2152 spectral bands from the visible to the short-wave infrared range (200 nm to 2,500 nm). Their reflectance is shown in Fig. \ref{fig: usgs ems}.  For  computational considerations, we selected 224 equidistant bands for each EM. We call these reference EMs $\E_0$.

\begin{figure}[t]
    \centering
    \includegraphics[width=\linewidth]{Figs/usgs_ems.jpg}
    \caption{The EMs used for generating the synthetic data: asphalt (gds367), brick (gds350) and cardboard (gds371).}
    \label{fig: usgs ems}
\end{figure}

We generated synthetic abundance maps based on \emph{Gaussian Random Fields} (GRFs). Gaussian random fields can be thought of as spatially correlated Gaussian randomness \cite{kozintsev_computations_1999, noauthor_hyperspectral_nodate}, and they are a popular choice for generating synthetic hyperspectral data. We generate abundance maps using GRFs designed to comply with the ANC and ASC. The ground truth abundances are called $\A_\mathrm{gt}$. The scaling factors were drawn from the uniform distribution $\mathcal{U}([0.5; 1.5])$. The choice for this range is based on physical arguments, limiting the scaling factors to a meaningful range, as very large scaling factors or scaling factors close to zero are physically unrealistic.  The synthetic images were then designed to either comply with the 2LMM or the ELMM.

\subsubsection{2LMM-generated variability}
For generating synthetic data according to the 2LMM, we generate $N + K$ scaling factors, and group them in vectors $\s_\E$ and $\s_\X$. Then we generate the $n$-th pixel as
\[
\x_n = \E_0 \mathrm{diag}(\s_\E) \ba_{\mathrm{gt}, n} s_{\x_n}.
\]
We do not add any noise to the image.

\subsubsection{ELMM-generated variability}
For generating synthetic data according to the ELMM, we generate $NK$ scaling factors and combine these into $N$ vectors of dimension $K$, $\s_n, n=1,\ldots, N$. Then we generate the $n$-th pixel as
\[
\x_n = \E_0 \diag(\s_n) \ba_{\mathrm{gt}, n}.
\]
We do not add any noise to the image.

\subsection{Influence of the bounds $\underline{S}$ and $\overline{S}$} \label{sec: influence of bounds}
In this first experiment, a $50 \times 50$ synthetic image is generated with 2LMM-generated variability. 
We first validated the performance of the 3 proposed approaches to solve the optimization. We observed that the $\mathrm{2LMM}_\mathrm{norm}$ approach fails, because it consumes too much memory. During execution, it produces an out-of-memory error and is terminated by the operating system. The angle approach $\mathrm{2LMM}_\mathrm{angle}$ does not crash, but it is extremely slow and does not converge. It was automatically terminated after 1 million function evaluations, at which point it had run for approximately 4.5 hours. Based on this observation, we will use the two scaling factor approach $\mathrm{2LMM}$ for further experimentation.

To examine the effect of the bounds on the resulting estimates, we vary the lower and upper bounds $\underline{S}$ and $\overline{S}$. 
The bounds were taken to be $\left[ \frac{1}{\alpha}, \alpha \right]$ for different values of $\alpha > 1$. The results are shown in Table \ref{tab:small image grf}.
\begin{table}[t]
\caption{Unmixing results on a  $50 \times 50$ synthetic image with 2LMM-generated variability. The bounds $[\underline{S}, \overline{S}]$ are given by $[\frac{1}{\alpha}, \alpha]$. The best results are highlighted in bold.}
\centering
        \begin{tabular}{|r|cccccc|}
        \hline
        $\alpha$    & 100    & 10              & 5      & 2               & 4/3             & 10/9   \\ \hline \hline
        RMSE$_\X$   & 1e-6   & \textbf{9e-7}   & 2e-6   & 2e-6            & 2e-6            & 0.0042 \\ \hline
        SAD$_\X$    & 1e-4   & \textbf{4e-5}   & 1e-4   & 1e-4            & 1e-4            & 0.1063 \\ \hline
        RMSE$_\A$   & 0.0239 & 0.0239          & 0.0239 & 0.0239          & \textbf{0.0222} & 0.0951 \\ \hline
        RMSE$_\s$   & 57.804 & 4.8513          & 1.9099 & \textbf{0.1511} & 0.2039          & 0.3170 \\ \hline
        $\esad$     & 4.3382 & 4.3395          & 4.3393 & 4.3364          & \textbf{3.8441} & 18.260 \\ \hline
        $\xsad$     & 2.4598 & 2.4600          & 2.4602 & \textbf{2.4591} & 2.4976          & 8.9428 \\ \hline
        \end{tabular}
\label{tab:small image grf}
\end{table}
First, one can observe that the results are overall the best when the chosen bounds (i.e., $[\underline{S}, \overline{S}] = [0.5, 2]$) are closest to the actual range of scale values (i.e., $[0.5,1.5]$). However, the results
are not overly sensitive to changes of the bounds $\underline{S}$ and $\overline{S}$ and there is a fairly broad range of choices that lead to similar results. 
Nevertheless, the results suggest that the bounds should not be chosen too tight, as this will result in a feasible set that is very small. As a result, many good solutions will fall outside the feasible set, ending up with a poor solution. This is the case when $[\underline{S}, \overline{S}] = [\frac{9}{10}, \frac{10}{9}]$ where the abundance RMSE,  the reconstruction RMSE, the reconstruction SAD and the scaling SADs are higher than the cases with looser bounds.
When the bounds are chosen wider than the actual scaling range, the abundance estimates remain accurate, and the  reconstruction error remains low, but the scaling RMSE is very high. However, this is not an issue, since the scaling SADs are still low, so the total scaling is still estimated accurately.
In conclusion, since the priority is accurate abundance estimation, it is crucial to select sufficiently wide bounds that encompass a realistic range of scalings.

%This suggests that there is a balance in between, where the method is given enough freedom to explore possible solutions, but is still guided by bounds that are appropriately chosen. Given these observations, we continue in what follows with fixed bounds $[\underline{S}, \overline{S}] = [\frac{1}{2}; 2]$.


\subsection{Comparison to LMM, SLMM and ELMM}

In this experiment, synthetic data are generated using reference endmembers $\E_0$ and GRF-generated ground truth abundances $\A_\mathrm{gt}$. The image size is $100 \times 100$. The required number of scaling factors is sampled from the uniform distribution $\mathcal{U}([0.5, 1.5])$.
The bounds of 2LMM are chosen accordingly as $[\underline{S}, \overline{S}] = [0.5, 2]$.
We compare the performance of 2LMM to several well-established unmixing methods, more precisely: LMM (solved with FCLSU), SLMM (solved with CLSU) and ELMM (solved with alternating least-squares and ADMM, as described in \cite{drumetz_blind_2016}).
For the ELMM-based method, we test two variants: WS-ELMM, where the method is \textit{warm-started}, i.e., initialized with the abundance estimates from CLSU, and CS-ELMM, where the method is \textit{cold-started}, i.e., initialized with uniform abundance estimates $\frac{1}{K}$. 

\subsubsection{Performance under 2LMM-generated variability}
In this first experiment, the synthetic image is generated with 2LMM-generated variability. We compare the results of the 2LMM method to the three models mentioned above.  The results are shown in Table \ref{tab: results 2lmm lowvar}, along with the computation times. 

Overall, the 2LMM method is the best performing method at a reasonable cost. Given the fact that the ELMM is a model that is rich enough to describe any dataset with 2LMM-generated variability without modeling error, it is quite surprising that WS-ELMM fails to perform better than CLSU, which will have a possibly large model mismatch since it is too simple to describe most 2LMM-based models.

\begin{table}[t]
    \caption{Experimental results for synthetic data with  2LMM-generated variability. The best errors are highlighted in bold.}
    \begin{center}
    \begin{tabular}{|r|ccccc|}
    \hline
               & FCLSU  & CLSU   & WS-ELMM & CS-ELMM & 2LMM  \\ \hline \hline
    RMSE$_\X$  & 0.0167 & 0.0027 & 0.0129  & 0.0089  & \textbf{2e-6}   \\ \hline
    SAD$_\X$   & 3.9934 & 0.0715 & 1.0199  & 1.6937  & \textbf{0.0002}   \\ \hline
    RMSE$_\A$  & 0.2190 & 0.0919 & 0.0913  & 0.2598  & \textbf{0.0135}    \\ \hline
    $\Delta t$ & 17     & 16     & 31      & 96      & 48                 \\ \hline
    \end{tabular}
    \end{center}
    \label{tab: results 2lmm lowvar}
\end{table}

\begin{table}[t]
    \caption{Experimental results for synthetic data with  ELMM-generated variability. The best errors are highlighted in bold.}
    \begin{center}
    \begin{tabular}{|r|ccccc|}
    \hline
               & FCLSU  & CLSU   & WS-ELMM & CS-ELMM & 2LMM   \\ \hline \hline
    RMSE$_\X$  & 0.0207 & 0.0039 & 0.0118  & 0.0149  & \textbf{2e-7}   \\ \hline
    SAD$_\X$   & 1.7413 & 0.1044 & 0.5676  & 2.0905  & \textbf{2e-6}    \\ \hline
    RMSE$_\A$  & 0.1513 & 0.0744 & 0.0740  & 0.2307  & \textbf{0.0693}    \\ \hline
    $\Delta t$ & 19     & 17     & 26      & 70      & 62                 \\ \hline
    \end{tabular}
    \end{center}
    \label{tab:results elmm lowvar}
\end{table}

\subsubsection{Performance under ELMM-generated variability}
The experiment from the previous paragraph is repeated, but this time with the variability generated according to the ELMM. The results are shown in Table \ref{tab:results elmm lowvar}. Again, 2LMM performed the best overall.  
Regarding the abundance estimation, 2LMM, CLSU and WS-ELMM perform similarly. Interestingly, WS-ELMM only performs as good as CLSU, even though CLSU has a significant model mismatch, while ELMM is rich enough to describe the scene exactly, and is initialized using the CLSU estimates. The estimates of CS-ELMM are very poor, meaning that ELMM relies heavily on a good initial estimate. 
Because 2LMM is only mildly non-convex and because the cost function of 2LMM (Eq. \ref{eq: two scaling factor}) only consists of the reconstruction error, the local interior-point solver is able to find a (close to) global minimum for this problem, with a reconstruction RMSE that is very close to zero. This is not the case for WS-ELMM and CS-ELMM, since ELMM is highly non-convex, and its cost function includes regularization terms as well. This means that, even if a global minimum of the ELMM cost function was obtained, it is very unlikely to coincide with a near-zero reconstruction RMSE.
Lastly, FCLSU does not perform very well due to considerable model mismatch. 


\section{Experiments with real data}
\subsection{Houston dataset}

In this section, we use the Houston dataset, which comprises a hyperspectral image of the (now demolished) Robertson Stadium on the University of Houston Campus, acquired in 2012. The data consists of a  $150\times 218$ image with 144 spectral bands in the 380 nm to 1050 nm region. A high-resolution RGB image of the scene, taken from a different angle, is shown in Fig. \ref{fig:robertson stadium}. The dataset is part of a larger dataset, which was used in the 2013 GRSS Data Fusion Contest \cite{debes_hyperspectral_2014}. The EMs are (red) roofs, vegetation, concrete, and asphalt. 
\begin{figure}[t]
    \centering
    \includegraphics[width=0.7\linewidth]{Figs/robertson_stadium.jpg}
    \caption{An RGB image of Robertson Stadium, Houston, Texas}
    \label{fig:robertson stadium}
\end{figure}
\begin{figure}[t]
    \centering
    \includegraphics[width=\linewidth]{Figs/houston_endmembers.jpg}
    \caption{The spectral signatures of the four reference EMs in the Houston dataset}
    \label{fig:reference endmembers}
\end{figure}
The signatures are shown in Fig. \ref{fig:reference endmembers}. 

On this image, the methods FCLSU, CLSU, WS-ELMM, CS-ELMM and 2LMM are run. For a fair comparison, all methods are applied "off-the-shelf", meaning that none of the hyperparameters are  tuned based on the data or on the acquired results. For WS-ELMM and CS-ELMM,  the standard regularization terms are used. For 2LMM, the standard bounds of $[\frac{1}{2}; 2]$ are used.

\subsubsection{Abundance estimations}

The abundance maps estimated by the five methods are shown in Fig. \ref{fig:abundance maps}. Except for FCLSU and CS-ELMM, which produce poor results, all obtained abundance maps are quite similar, with some notable differences. The abundance maps of WS-ELMM are less granular and more smooth. This is a direct result of the spatial regularization terms which are used in WS-ELMM. This however also leads to some misestimations by WS-ELMM caused by oversmoothing of, e.g., small grass patches and small roofs at the entrance of the stadium (left-center of the hyperspectral image). This highlights the difficulty in properly setting the regularization parameters.

\begin{figure}[t]
    \centering
    \includegraphics[width=\linewidth]{Figs/houston_abundances.jpg}
    \caption{Abundance maps of the four unmixing methods on the Houston data. A brighter pixel means a larger abundance.}
    \label{fig:abundance maps}
\end{figure}

\subsubsection{Reconstruction error}

Unlike the synthetic data, no ground truth is available, and no abundance RMSE can be obtained. Therefore, the performance is judged by  the reconstruction RMSE and SAD. 
The mean reconstruction RMSE and reconstruction SAD are shown in Table \ref{table: houston}. One can observe that 2LMM  has the lowest reconstruction error, followed by CLSU and WS-ELMM. Timings indicate that the cost of the proposed approach is moderate.

\begin{table}[t]
\caption{Reconstruction errors and timings for the Houston experiment. The best errors are highlighted in bold.}
\centering
\begin{tabular}{r|ccccc|}
\cline{2-6}
\multicolumn{1}{l|}{}           & FCLSU & CLSU  & WS-ELMM & CS-ELMM & 2LMM  \\ \hline \hline
\multicolumn{1}{|r|}{RMSE$_\X$} & 0.048 & 0.014 & 0.014   & 0.018   & \textbf{0.006}    \\ \hline
\multicolumn{1}{|r|}{SAD$_\X$}  & 3.306 & 1.925 & 2.166   & 4.462   & \textbf{1.454}    \\ \hline
\multicolumn{1}{|r|}{$\Delta t$ (s)}  & 21    & 20    & 48      & 79      & 86                \\ \hline
\end{tabular}
\label{table: houston}
\end{table}

Fig. \ref{fig: reconstruction error} shows the reconstruction SAD. 
One can observe that WS-ELMM and CLSU mostly make larger errors in the northern stands of the stadium. The stands are made of concrete, but they can reflect light in a complicated way due to the many steps and different angles at which the material is present. WS-ELMM and CLSU are having difficulty capturing this variability. Another cause of errors are the red roofs at the entrance of the stadium (left-center of the image), although the other methods also make large errors here. Next to the small roofed structures, the entrance is also lined with trees (as can be seen in the RGB image in Fig. \ref{fig:robertson stadium}), which can cause light to be reflected in a nonlinear way. This causes misestimations in all methods.

\begin{figure*}[t]
    \centering
    \includegraphics[width=\linewidth]{Figs/houston_reconstruction_err.jpg}
    \caption{Reconstruction SAD (in degrees) for the five unmixing methods on the Houston dataset.}
    \label{fig: reconstruction error}
\end{figure*}

\subsection{DLR HySU Dataset} \label{sec: dlr hysu}

In this experiment, we generate an image using the DLR HySU dataset \cite{cerra_dlr_2021}, a benchmark dataset for evaluating spectral unmixing algorithms, featuring airborne hyperspectral and RGB imagery of synthetic targets with known materials and sizes. The dataset was captured at the DLR (German Aerospace Center) premises in Oberpfaffenhofen, Germany. It consists of several checkerboard patterns of various size laid out on a grass field.

%\subsubsection{Large targets} \label{sec: dlr hysu large targets}

We use a sub-image of the total dataset, which contains the largest checkerboard pattern of the five materials (and a sixth background material). This is a $13 \times 16$ image with 135 spectral bands covering the wavelength range 416 nm -- 903 nm. The EM signatures are provided along with the dataset.
%Since this is a small image, we can also test the norm-adjusted and angle approach, solved in MATLAB. 
An annotated RGB image of the scene is shown in Figure \ref{fig:dlr rgb}.

Hyperspectral variability in this scene is negligible, since there are no shadows, topographical features or other factors that impact EM signatures. With this in mind, we use FCLSU to find the abundances, which we consider as the ground truth abundances. Then, using this ground truth, we re-generate the image, but this time we introduce spectral variability.

\subsubsection{Performance under 2LMM-generated variability}

The scaling factors are generated with variability according to the 2LMM. Let $\E_0$ be the provided EMs and $\A_\mathrm{gt}$ the ground truth abundances, and let $\s_\E \in \real^K$ and $\s_\X \in \real^N$ be vectors with its elements drawn from the uniform distribution $\mathcal{U}([0.5,1.5])$. Then the synthetic image $\X_\mathrm{syn}$ is generated as:
\[
    \X_\mathrm{syn} = \E_0  \diag (\s_\E) \A_\mathrm{gt} \mathrm{diag}(\s_\X) + \mathbf{e}_\X
\]
with $\mathbf{e}_\mathbf{X}$ normally distributed noise with a signal-to-noise ratio (SNR) of 60 dB. 

\begin{figure}[t]
    \centering
    \includegraphics[width=0.3\linewidth]{Figs/RGB_image_annotated.png}
    \caption{RGB image of a subset of the DLR HySU dataset, with five materials (1. bitumen, 2. green fabric, 3. red fabric, 4. red metal, 5. blue fabric) arranged in a checkerboard pattern on a grass background (sixth material).}
    \label{fig:dlr rgb}
\end{figure}

Unmixing is performed with the same five methods as before. The resulting SAD$_\X$, RMSE$_\X$ and abundance RMSEs, separately for each material, are shown in Table \ref{table: dlr hysu 2lmm}. The resulting abundance maps are shown in Fig. \ref{fig: dlr hysu abundances}. One can observe that 2LMM performs best. Other methods have difficulty identifying the squares, and make a considerable error in doing so. They also mistake certain materials for another, e.g., parts of the red fabric square are identified as red metal.

\begin{figure}[t]
    \centering
    \includegraphics[width=\linewidth]{Figs/dlr_abundances.jpg}
    \caption{Ground truth abundance maps (GT) and abundance maps for five unmixing methods on the \textit{DLR HySU} dataset with 2LMM-generated variability. A brighter pixel means a larger abundance.}
    \label{fig: dlr hysu abundances}
\end{figure}

\begin{table}[htb!]
\caption{Abundance and reconstruction errors for the DLR dataset with 2LMM-generated variability. $\mathrm{RMSE}_i$ denotes the abundance RMSE for the $i$-th material (1. bitumen, 2. green fabric, 3. red fabric, 4. red metal, 5. blue fabric, 6. grass background). The best results  are highlighted in bold. }
\centering
    \begin{tabular}{r|ccccc|}
    \cline{2-6}
    \multicolumn{1}{l|}{}            & FCLSU  & CLSU   & WS-ELMM  & CS-ELMM  & 2LMM \\ \hline \hline
    \multicolumn{1}{|r|}{SAD$_\X$}   & 4.9664 & 2.3890 & 2.6672   & 5.9272   & \textbf{1.8644} \\ \hline
    \multicolumn{1}{|r|}{RMSE$_\X$}  & 0.0809 & 0.0555 & 0.1182   & \textbf{0.0123}   & 0.1341 \\ \hline \hline
    \multicolumn{1}{|r|}{RMSE$_1$}   & 0.0740 & 0.0608 & 0.0587   & 0.2425   & \textbf{0.0108} \\ \hline
    \multicolumn{1}{|r|}{RMSE$_2$}   & 0.2725 & 0.0824 & 0.0799   & 0.2669   & \textbf{0.0179} \\ \hline
    \multicolumn{1}{|r|}{RMSE$_3$}   & 0.1363 & 0.1195 & 0.1165   & 0.2087   & \textbf{0.0200} \\ \hline
    \multicolumn{1}{|r|}{RMSE$_4$}   & 0.1025 & 0.0924 & 0.0915   & 0.2156   & \textbf{0.0146} \\ \hline
    \multicolumn{1}{|r|}{RMSE$_5$}   & 0.0866 & 0.1103 & 0.1083   & 0.2143   & \textbf{0.0167} \\ \hline
    \multicolumn{1}{|r|}{RMSE$_6$}   & 0.2564 & 0.0766 & 0.0739   & 0.4123   & \textbf{0.0379} \\ \hline \hline
    \multicolumn{1}{|r|}{RMSE$_\A$}  & 0.1742 & 0.0925 & 0.0895   & 0.2696   & \textbf{0.0215} \\ \hline
    \end{tabular}
\label{table: dlr hysu 2lmm}
\end{table}

\subsubsection{Performance under ELMM-generated variability}
We repeat the above experiment, but now with the variability generated according to the ELMM. As before, we draw scaling factors from the distribution $\mathcal{U}([0.5;1.5])$. Now, we generate $NK$ of them, and group them into $N$ scaling vectors $\s_n, ~n=1,\ldots, N$. The pixels are then generated using
\begin{equation}
\x_n = \E_0 \diag(\s_n) \ba_{\mathrm{gt}, n} + \mathbf{e}_{\x_n}
\end{equation}
with $\mathbf{e}_{\x_n}$ a noise term with an SNR of 60 dB. 
The resulting SAD$_\X$, RMSE$_\X$ and and abundance RMSE's, separately for each material are shown in Table \ref{tab: dlr elmm abundances}. The resulting abundance maps are shown in Fig. \ref{fig: dlr elmm abundances}.

Since the scaling terms can now vary significantly between pixels and EMs, we can expect a modeling error with 2LMM. However, the resulting estimates produced by 2LMM are still better than those by the ELMM-based methods, who possess the modeling capability to reconstruct the image without error, apart from noise. Even here, 2LMM outperforms the other methods in terms of abundance estimation and reconstruction SAD. This indicates that 2LMM unmixing is quite robust to deviations from the model assumption and can be a reliable alternative for existing mixing models.

\begin{table}[htb!]
    \caption{Abundance and reconstruction errors for the DLR dataset with ELMM-generated variability.  $\mathrm{RMSE}_i$ denotes the abundance RMSE for the $i$-th material (1. bitumen, 2. green fabric, 3. red fabric, 4. red metal, 5. blue fabric, 6. grass background). The best results are highlighted in bold.}
    \centering
        \begin{tabular}{r|ccccc|}
        \cline{2-6}
        \multicolumn{1}{l|}{}            & FCLSU  & CLSU            & WS-ELMM  & CS-ELMM  & 2LMM    \\ \hline \hline
        \multicolumn{1}{|r|}{SAD$_\X$}   & 4.1926 & 2.2171          & 2.2383   & 5.0003   & \textbf{1.7828}   \\ \hline
        \multicolumn{1}{|r|}{RMSE$_\X$}  & 0.0642 & \textbf{0.0290} & 0.0649   & 0.0706   & 0.0771   \\ \hline \hline
        \multicolumn{1}{|r|}{RMSE$_1$}   & 0.0843 & 0.0452          & 0.0452   & 0.2462   & \textbf{0.0329} \\ \hline
        \multicolumn{1}{|r|}{RMSE$_2$}   & 0.2157 & 0.0584          & 0.0583   & 0.2547   & \textbf{0.0279} \\ \hline
        \multicolumn{1}{|r|}{RMSE$_3$}   & 0.1094 & 0.0618          & 0.0617   & 0.2061   & \textbf{0.0320} \\ \hline
        \multicolumn{1}{|r|}{RMSE$_4$}   & 0.0837 & 0.0613          & 0.0612   & 0.2538   & \textbf{0.0186} \\ \hline
        \multicolumn{1}{|r|}{RMSE$_5$}   & 0.0564 & 0.0809          & 0.0806   & 0.2132   & \textbf{0.0380} \\ \hline
        \multicolumn{1}{|r|}{RMSE$_6$}   & 0.1964 & 0.0749          & 0.0747   & 0.4417   & \textbf{0.0646} \\ \hline \hline
        \multicolumn{1}{|r|}{RMSE$_\A$}  & 0.1381 & 0.0648          & 0.0646   & 0.2808   & \textbf{0.0384} \\ \hline
        \end{tabular}
    \label{tab: dlr elmm abundances}
\end{table}

\begin{figure}[t]
    \centering
    \includegraphics[width=\linewidth]{Figs/dlr_abundances_elmm.jpg}
    \caption{Ground truth abundance maps (GT) and abundance maps for five unmixing methods on the \textit{DLR HySU} dataset with ELMM-generated variability. A brighter pixel means a larger abundance.}
    \label{fig: dlr elmm abundances}
\end{figure}

\section{Conclusion}
In this work, we have presented the 2LMM, a novel physically motivated two-step linear mixing model that mitigates the effect of spectral variability. The model bridges the gap between model complexity and computational tractability. A key feature of the 2LMM is that it leads to a mildly non-convex unmixing problem, which we solve using an interior-point method. Experiments on synthetic and real hyperspectral data show that the 2LMM achieves competitive performance against existing methods and exhibits robustness to deviations from its underlying assumptions.


\section*{Acknowledgments}
The research presented in this paper is funded by the Research Foundation-Flanders - project G031921N. Bikram Koirala is a postdoctoral fellow of the Research Foundation Flanders, Belgium (FWO: 1250824N-7028). The authors  acknowledge the team of Daniele Cerra at DLR for the development of the DLR HySU  dataset.

\bibliographystyle{IEEEtran}
\bibliography{main}

{\appendices
\section{Assumption on the feasible set $\mathcal{X}$}
In order for the barrier function to be well-defined we require that the problem (\ref{eq: interior point problem}) admits at least one strictly feasible solution \cite{boyd_convex_2004}:
\begin{assumption}
    The feasible set $\mathcal{X} \subseteq \real^D$ is nonempty and the problem (\ref{eq: interior point problem}) is \textbf{strictly feasible}, i.e. 
        \[
    \exists \Bar{\x} \in \real^D: g_i(\Bar{\x}) > 0, \forall i, ~\mathbf{C}\Bar{\x} = \mathbf{b}, ~\Bar{\x} > 0.
        \]
\end{assumption}

\section{Newton's method for equality-constrained optimization}
For a fixed value of $\mu$, we can formulate the Karush-Kuhn-Tucker (KKT) optimality conditions for the problem (\ref{eq: barrier problem}) and solve them. The KKT conditions start from the Lagrangian $\mathcal{L}_\mu(\x, \bm{\lambda})$ \cite[Ch. 5]{boyd_convex_2004} and read
\begin{equation} \label{eq: kkt system}
    \begin{aligned}
        \nabla_{\bm{\lambda}} \mathcal{L}_\mu (\x, \bm{\lambda}) &= 0 \\
        \nabla_\x \mathcal{L}_\mu (\x, \bm{\lambda}) &= 0.
    \end{aligned}
\end{equation}
This is a system of nonlinear equations, and can be solved using Newton's method for nonlinear equations. For this, define
\[
\textbf{F}_\mu(\x, \bm{\lambda}) = \left( \begin{array}{c}
     - \nabla_{\bm{\lambda}} \mathcal{L}_\mu (\x, \bm{\lambda}) \\ \nabla_\x \mathcal{L}_\mu (\x, \bm{\lambda})
\end{array} \right).
\]
A step $\Delta^k = (\Delta_\x^k, \Delta_{\bm{\lambda}}^k)$ is found by solving
\[
\mathbf{J}_{\textbf{F}_\mu}(\x_k, \mu_k) \Delta^k = - \textbf{F}_\mu(\x_k, \bm{\lambda}_k)
\]
where $\mathbf{J}_{\textbf{F}_\mu}$ is the Jacobian matrix. The solution is then updated using a line search procedure:
\begin{align*}
    \x_{k+1} &= \x_k + \alpha_k \Delta_\x^k \\
    \bm{\lambda}_{k+1} &= \bm{\lambda}_k + \alpha_k \Delta_{\bm{\lambda}}^k
\end{align*}
where the step size $\alpha_k$ is determined using a backtracking line search algorithm.


\section{Convergence}
A formal convergence proof is beyond the scope of this paper, so instead we sketch a convergence analysis based on \cite{boyd_convex_2004}. The analysis consists of two parts, proving convergence of the inner (Newton) loop and outer loop, respectively. Fix the following sequence for the barrier parameter $\mu$: $\mu_0, \nu \mu_0, \nu^2 \mu_0, \ldots$ for $0 < \nu < 1$.

\subsubsection*{Inner loop} 
We make the following assumptions:
\begin{assumption} Consider the problem (\ref{eq: interior point problem}) and its corresponding barrier problem (\ref{eq: barrier problem}). The barrier problem can always be solved using Newton's method, or equivalently:
        \begin{enumerate}
            \item $f(\x)$ and $g_i(\x)$ are closed on $\mathcal{X}$, i.e., the set
            \[
            \{\x \in \mathcal{X} \mid f(\x) \leq \alpha\}
            \]
            is closed for any $\alpha \in \real$, similarly for $g_i(\x)$.
            \item For all $\x \in \mathcal{X}$, we have $\| \x \|_2^2 \leq R^2$ for some $R < +\infty$.
        \end{enumerate}
\end{assumption}
It follows from Assumption 2 that each barrier problem can be solved using Newton's method in a finite number of steps. Bounding the number of steps is hard without making additional assumptions on the problem. If we assume $B(\x, \mu)$ is closed and \textit{self-concordant} for all $\mu \leq \mu_0$, and assume the sublevel sets of the problem (\ref{eq: interior point problem}) are bounded, we can provide an upper bound on the required number of Newton steps. If we solve the problem to an accuracy of $\epsilon_\mathrm{N} > 0$, then we need at most
\[
\frac{I}{\gamma}(\nu - 1 - \log \nu) + \log_2 \log_2 \frac{1}{\epsilon_\mathrm{N}}
\]
steps, where $\gamma$ is a constant determined by the backtracking line search procedure, and $I$ is the number of inequality constraints.

\subsubsection*{Outer loop} 
If the barrier problem (\ref{eq: barrier problem}) can be minimized using Newton's method for the sequence $\{\mu_k\}_{k \geq 0}$ as mentioned above, then we can achieve a desired accuracy $\epsilon_\mathrm{B} > 0$ after
\[
\left\lceil \frac{\log \left( I \mu_0 / \epsilon_\mathrm{B} \right)}{\log 1/\nu}\right\rceil + 1
\]
steps. Therefore, since we can solve both the outer problem and inner problem in finitely many steps, we can guarantee that the algorithm will always converge to a locally optimal solution in finite time.

\section{Slack variables}
We replace all logarithms of the form $\log (g_i(\x))$ by the constrained form \cite{geletu_introduction_nodate} 
\[
\log \sigma_i~\mathrm{s.t.}~g_i(\x)-\sigma_i = 0.
\]
For the two scaling factor approach, this leads to the modified barrier function 
\begin{multline*}
    \Tilde{B}(\A_\s, \s_\E) = J(\A_\s, \s_\E) - \mu \Big( \sum_{k=1}^K ( \log \sigma_k^+ + \log \sigma_k^-) + \\
                \sum_{k=1}^K \sum_{n=1}^N (\log \sigma_{nk} + \log a_{nk}) \Big)
\end{multline*}
and the optimization problem
\begin{equation*}
\begin{aligned}
        \min &~\Tilde{B}(\A_\s, \s_\E) \\
    \mathrm{s.t.} &~ s_{\e_k} - \underline{S} - \sigma^+_k = 0, \quad \forall k\\
                &~\overline{S} - s_{\e_k} - \sigma^-_k = 0, \quad \forall k \\
                &~\overline{S} - a_{nk} - \sigma_{nk} = 0, \quad \forall n, k
\end{aligned}
\end{equation*}
which is equivalent to the original barrier problem (\ref{eq: barrier problem}).
}
\newpage




\vfill

\end{document}


