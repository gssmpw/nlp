\subsection{Houston dataset}

In this section, we use the Houston dataset, which comprises a hyperspectral image of the (now demolished) Robertson Stadium on the University of Houston Campus, acquired in 2012. The data consists of a  $150\times 218$ image with 144 spectral bands in the 380 nm to 1050 nm region. A high-resolution RGB image of the scene, taken from a different angle, is shown in Fig. \ref{fig:robertson stadium}. The dataset is part of a larger dataset, which was used in the 2013 GRSS Data Fusion Contest \cite{debes_hyperspectral_2014}. The EMs are (red) roofs, vegetation, concrete, and asphalt. 
\begin{figure}[t]
    \centering
    \includegraphics[width=0.7\linewidth]{Figs/robertson_stadium.jpg}
    \caption{An RGB image of Robertson Stadium, Houston, Texas}
    \label{fig:robertson stadium}
\end{figure}
\begin{figure}[t]
    \centering
    \includegraphics[width=\linewidth]{Figs/houston_endmembers.jpg}
    \caption{The spectral signatures of the four reference EMs in the Houston dataset}
    \label{fig:reference endmembers}
\end{figure}
The signatures are shown in Fig. \ref{fig:reference endmembers}. 

On this image, the methods FCLSU, CLSU, WS-ELMM, CS-ELMM and 2LMM are run. For a fair comparison, all methods are applied "off-the-shelf", meaning that none of the hyperparameters are  tuned based on the data or on the acquired results. For WS-ELMM and CS-ELMM,  the standard regularization terms are used. For 2LMM, the standard bounds of $[\frac{1}{2}; 2]$ are used.

\subsubsection{Abundance estimations}

The abundance maps estimated by the five methods are shown in Fig. \ref{fig:abundance maps}. Except for FCLSU and CS-ELMM, which produce poor results, all obtained abundance maps are quite similar, with some notable differences. The abundance maps of WS-ELMM are less granular and more smooth. This is a direct result of the spatial regularization terms which are used in WS-ELMM. This however also leads to some misestimations by WS-ELMM caused by oversmoothing of, e.g., small grass patches and small roofs at the entrance of the stadium (left-center of the hyperspectral image). This highlights the difficulty in properly setting the regularization parameters.

\begin{figure}[t]
    \centering
    \includegraphics[width=\linewidth]{Figs/houston_abundances.jpg}
    \caption{Abundance maps of the four unmixing methods on the Houston data. A brighter pixel means a larger abundance.}
    \label{fig:abundance maps}
\end{figure}

\subsubsection{Reconstruction error}

Unlike the synthetic data, no ground truth is available, and no abundance RMSE can be obtained. Therefore, the performance is judged by  the reconstruction RMSE and SAD. 
The mean reconstruction RMSE and reconstruction SAD are shown in Table \ref{table: houston}. One can observe that 2LMM  has the lowest reconstruction error, followed by CLSU and WS-ELMM. Timings indicate that the cost of the proposed approach is moderate.

\begin{table}[t]
\caption{Reconstruction errors and timings for the Houston experiment. The best errors are highlighted in bold.}
\centering
\begin{tabular}{r|ccccc|}
\cline{2-6}
\multicolumn{1}{l|}{}           & FCLSU & CLSU  & WS-ELMM & CS-ELMM & 2LMM  \\ \hline \hline
\multicolumn{1}{|r|}{RMSE$_\X$} & 0.048 & 0.014 & 0.014   & 0.018   & \textbf{0.006}    \\ \hline
\multicolumn{1}{|r|}{SAD$_\X$}  & 3.306 & 1.925 & 2.166   & 4.462   & \textbf{1.454}    \\ \hline
\multicolumn{1}{|r|}{$\Delta t$ (s)}  & 21    & 20    & 48      & 79      & 86                \\ \hline
\end{tabular}
\label{table: houston}
\end{table}

Fig. \ref{fig: reconstruction error} shows the reconstruction SAD. 
One can observe that WS-ELMM and CLSU mostly make larger errors in the northern stands of the stadium. The stands are made of concrete, but they can reflect light in a complicated way due to the many steps and different angles at which the material is present. WS-ELMM and CLSU are having difficulty capturing this variability. Another cause of errors are the red roofs at the entrance of the stadium (left-center of the image), although the other methods also make large errors here. Next to the small roofed structures, the entrance is also lined with trees (as can be seen in the RGB image in Fig. \ref{fig:robertson stadium}), which can cause light to be reflected in a nonlinear way. This causes misestimations in all methods.

\begin{figure*}[t]
    \centering
    \includegraphics[width=\linewidth]{Figs/houston_reconstruction_err.jpg}
    \caption{Reconstruction SAD (in degrees) for the five unmixing methods on the Houston dataset.}
    \label{fig: reconstruction error}
\end{figure*}

\subsection{DLR HySU Dataset} \label{sec: dlr hysu}

In this experiment, we generate an image using the DLR HySU dataset \cite{cerra_dlr_2021}, a benchmark dataset for evaluating spectral unmixing algorithms, featuring airborne hyperspectral and RGB imagery of synthetic targets with known materials and sizes. The dataset was captured at the DLR (German Aerospace Center) premises in Oberpfaffenhofen, Germany. It consists of several checkerboard patterns of various size laid out on a grass field.

%\subsubsection{Large targets} \label{sec: dlr hysu large targets}

We use a sub-image of the total dataset, which contains the largest checkerboard pattern of the five materials (and a sixth background material). This is a $13 \times 16$ image with 135 spectral bands covering the wavelength range 416 nm -- 903 nm. The EM signatures are provided along with the dataset.
%Since this is a small image, we can also test the norm-adjusted and angle approach, solved in MATLAB. 
An annotated RGB image of the scene is shown in Figure \ref{fig:dlr rgb}.

Hyperspectral variability in this scene is negligible, since there are no shadows, topographical features or other factors that impact EM signatures. With this in mind, we use FCLSU to find the abundances, which we consider as the ground truth abundances. Then, using this ground truth, we re-generate the image, but this time we introduce spectral variability.

\subsubsection{Performance under 2LMM-generated variability}

The scaling factors are generated with variability according to the 2LMM. Let $\E_0$ be the provided EMs and $\A_\mathrm{gt}$ the ground truth abundances, and let $\s_\E \in \real^K$ and $\s_\X \in \real^N$ be vectors with its elements drawn from the uniform distribution $\mathcal{U}([0.5,1.5])$. Then the synthetic image $\X_\mathrm{syn}$ is generated as:
\[
    \X_\mathrm{syn} = \E_0  \diag (\s_\E) \A_\mathrm{gt} \mathrm{diag}(\s_\X) + \mathbf{e}_\X
\]
with $\mathbf{e}_\mathbf{X}$ normally distributed noise with a signal-to-noise ratio (SNR) of 60 dB. 

\begin{figure}[t]
    \centering
    \includegraphics[width=0.3\linewidth]{Figs/RGB_image_annotated.png}
    \caption{RGB image of a subset of the DLR HySU dataset, with five materials (1. bitumen, 2. green fabric, 3. red fabric, 4. red metal, 5. blue fabric) arranged in a checkerboard pattern on a grass background (sixth material).}
    \label{fig:dlr rgb}
\end{figure}

Unmixing is performed with the same five methods as before. The resulting SAD$_\X$, RMSE$_\X$ and abundance RMSEs, separately for each material, are shown in Table \ref{table: dlr hysu 2lmm}. The resulting abundance maps are shown in Fig. \ref{fig: dlr hysu abundances}. One can observe that 2LMM performs best. Other methods have difficulty identifying the squares, and make a considerable error in doing so. They also mistake certain materials for another, e.g., parts of the red fabric square are identified as red metal.

\begin{figure}[t]
    \centering
    \includegraphics[width=\linewidth]{Figs/dlr_abundances.jpg}
    \caption{Ground truth abundance maps (GT) and abundance maps for five unmixing methods on the \textit{DLR HySU} dataset with 2LMM-generated variability. A brighter pixel means a larger abundance.}
    \label{fig: dlr hysu abundances}
\end{figure}

\begin{table}[htb!]
\caption{Abundance and reconstruction errors for the DLR dataset with 2LMM-generated variability. $\mathrm{RMSE}_i$ denotes the abundance RMSE for the $i$-th material (1. bitumen, 2. green fabric, 3. red fabric, 4. red metal, 5. blue fabric, 6. grass background). The best results  are highlighted in bold. }
\centering
    \begin{tabular}{r|ccccc|}
    \cline{2-6}
    \multicolumn{1}{l|}{}            & FCLSU  & CLSU   & WS-ELMM  & CS-ELMM  & 2LMM \\ \hline \hline
    \multicolumn{1}{|r|}{SAD$_\X$}   & 4.9664 & 2.3890 & 2.6672   & 5.9272   & \textbf{1.8644} \\ \hline
    \multicolumn{1}{|r|}{RMSE$_\X$}  & 0.0809 & 0.0555 & 0.1182   & \textbf{0.0123}   & 0.1341 \\ \hline \hline
    \multicolumn{1}{|r|}{RMSE$_1$}   & 0.0740 & 0.0608 & 0.0587   & 0.2425   & \textbf{0.0108} \\ \hline
    \multicolumn{1}{|r|}{RMSE$_2$}   & 0.2725 & 0.0824 & 0.0799   & 0.2669   & \textbf{0.0179} \\ \hline
    \multicolumn{1}{|r|}{RMSE$_3$}   & 0.1363 & 0.1195 & 0.1165   & 0.2087   & \textbf{0.0200} \\ \hline
    \multicolumn{1}{|r|}{RMSE$_4$}   & 0.1025 & 0.0924 & 0.0915   & 0.2156   & \textbf{0.0146} \\ \hline
    \multicolumn{1}{|r|}{RMSE$_5$}   & 0.0866 & 0.1103 & 0.1083   & 0.2143   & \textbf{0.0167} \\ \hline
    \multicolumn{1}{|r|}{RMSE$_6$}   & 0.2564 & 0.0766 & 0.0739   & 0.4123   & \textbf{0.0379} \\ \hline \hline
    \multicolumn{1}{|r|}{RMSE$_\A$}  & 0.1742 & 0.0925 & 0.0895   & 0.2696   & \textbf{0.0215} \\ \hline
    \end{tabular}
\label{table: dlr hysu 2lmm}
\end{table}

\subsubsection{Performance under ELMM-generated variability}
We repeat the above experiment, but now with the variability generated according to the ELMM. As before, we draw scaling factors from the distribution $\mathcal{U}([0.5;1.5])$. Now, we generate $NK$ of them, and group them into $N$ scaling vectors $\s_n, ~n=1,\ldots, N$. The pixels are then generated using
\begin{equation}
\x_n = \E_0 \diag(\s_n) \ba_{\mathrm{gt}, n} + \mathbf{e}_{\x_n}
\end{equation}
with $\mathbf{e}_{\x_n}$ a noise term with an SNR of 60 dB. 
The resulting SAD$_\X$, RMSE$_\X$ and and abundance RMSE's, separately for each material are shown in Table \ref{tab: dlr elmm abundances}. The resulting abundance maps are shown in Fig. \ref{fig: dlr elmm abundances}.

Since the scaling terms can now vary significantly between pixels and EMs, we can expect a modeling error with 2LMM. However, the resulting estimates produced by 2LMM are still better than those by the ELMM-based methods, who possess the modeling capability to reconstruct the image without error, apart from noise. Even here, 2LMM outperforms the other methods in terms of abundance estimation and reconstruction SAD. This indicates that 2LMM unmixing is quite robust to deviations from the model assumption and can be a reliable alternative for existing mixing models.

\begin{table}[htb!]
    \caption{Abundance and reconstruction errors for the DLR dataset with ELMM-generated variability.  $\mathrm{RMSE}_i$ denotes the abundance RMSE for the $i$-th material (1. bitumen, 2. green fabric, 3. red fabric, 4. red metal, 5. blue fabric, 6. grass background). The best results are highlighted in bold.}
    \centering
        \begin{tabular}{r|ccccc|}
        \cline{2-6}
        \multicolumn{1}{l|}{}            & FCLSU  & CLSU            & WS-ELMM  & CS-ELMM  & 2LMM    \\ \hline \hline
        \multicolumn{1}{|r|}{SAD$_\X$}   & 4.1926 & 2.2171          & 2.2383   & 5.0003   & \textbf{1.7828}   \\ \hline
        \multicolumn{1}{|r|}{RMSE$_\X$}  & 0.0642 & \textbf{0.0290} & 0.0649   & 0.0706   & 0.0771   \\ \hline \hline
        \multicolumn{1}{|r|}{RMSE$_1$}   & 0.0843 & 0.0452          & 0.0452   & 0.2462   & \textbf{0.0329} \\ \hline
        \multicolumn{1}{|r|}{RMSE$_2$}   & 0.2157 & 0.0584          & 0.0583   & 0.2547   & \textbf{0.0279} \\ \hline
        \multicolumn{1}{|r|}{RMSE$_3$}   & 0.1094 & 0.0618          & 0.0617   & 0.2061   & \textbf{0.0320} \\ \hline
        \multicolumn{1}{|r|}{RMSE$_4$}   & 0.0837 & 0.0613          & 0.0612   & 0.2538   & \textbf{0.0186} \\ \hline
        \multicolumn{1}{|r|}{RMSE$_5$}   & 0.0564 & 0.0809          & 0.0806   & 0.2132   & \textbf{0.0380} \\ \hline
        \multicolumn{1}{|r|}{RMSE$_6$}   & 0.1964 & 0.0749          & 0.0747   & 0.4417   & \textbf{0.0646} \\ \hline \hline
        \multicolumn{1}{|r|}{RMSE$_\A$}  & 0.1381 & 0.0648          & 0.0646   & 0.2808   & \textbf{0.0384} \\ \hline
        \end{tabular}
    \label{tab: dlr elmm abundances}
\end{table}

\begin{figure}[t]
    \centering
    \includegraphics[width=\linewidth]{Figs/dlr_abundances_elmm.jpg}
    \caption{Ground truth abundance maps (GT) and abundance maps for five unmixing methods on the \textit{DLR HySU} dataset with ELMM-generated variability. A brighter pixel means a larger abundance.}
    \label{fig: dlr elmm abundances}
\end{figure}