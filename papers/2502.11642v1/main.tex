% CVPR 2025 Paper Template; see https://github.com/cvpr-org/author-kit

\documentclass[10pt,twocolumn,letterpaper]{article}

%%%%%%%%% PAPER TYPE  - PLEASE UPDATE FOR FINAL VERSION
\usepackage{cvpr}              % To produce the CAMERA-READY version
%\usepackage[review]{cvpr}      % To produce the REVIEW version
% \usepackage[pagenumbers]{cvpr} % To force page numbers, e.g. for an arXiv version

% Import additional packages in the preamble file, before hyperref
\newcommand{\CG}{\mathcal{G}\xspace}
\newcommand{\CV}{\mathcal{V}\xspace}
\newcommand{\CE}{\mathcal{E}\xspace}
\newcommand{\CA}{\mathcal{A}\xspace}
\newcommand{\CF}{\mathcal{F}\xspace}
\newcommand{\CR}{\mathcal{R}\xspace}
\newcommand{\CB}{\mathcal{B}\xspace}
\newcommand{\CX}{\mathcal{X}\xspace}
\newcommand{\CK}{\mathcal{K}\xspace}
\newcommand{\CM}{\mathcal{M}\xspace}
\newcommand{\CC}{\mathcal{C}\xspace}
\newcommand{\CL}{\mathcal{L}\xspace}
\newcommand{\CI}{\mathcal{I}\xspace}
\newcommand{\CQ}{\mathcal{Q}\xspace}
\newcommand{\CO}{\mathcal{O}\xspace}
\newcommand{\CP}{\mathcal{P}\xspace}
\newcommand{\CS}{\mathcal{S}\xspace}
\newcommand{\CT}{\mathcal{T}\xspace}
\newcommand{\CJ}{\mathcal{J}\xspace}
\usepackage[para]{footmisc}
\usepackage{subfig}
% \usepackage{subcaption}
% \usepackage{array}
% \usepackage{colortbl}



% It is strongly recommended to use hyperref, especially for the review version.
% hyperref with option pagebackref eases the reviewers' job.
% Please disable hyperref *only* if you encounter grave issues, 
% e.g. with the file validation for the camera-ready version.
%
% If you comment hyperref and then uncomment it, you should delete *.aux before re-running LaTeX.
% (Or just hit 'q' on the first LaTeX run, let it finish, and you should be clear).
\definecolor{cvprblue}{rgb}{0.21,0.49,0.74}
\definecolor{cellcol}{gray}{.92}
\usepackage[pagebackref,breaklinks,colorlinks,allcolors=cvprblue]{hyperref}

%%%%%%%%% PAPER ID  - PLEASE UPDATE
\def\paperID{9343} % *** Enter the Paper ID here
\def\confName{CVPR}
\def\confYear{2025}

%%%%%%%%% TITLE - PLEASE UPDATE
\title{GaussianMotion: End-to-End Learning of Animatable \\ Gaussian Avatars with Pose Guidance from Text}
%\title{Animate the Human: 3D Human Generation from Text \\ for Rendering Dynamic Scenes }
%\title{Generate and Animate 3D Humans from Text \\ with Deformable Gaussian Splatting}

% GaussPose: Combines "Gaussian" and "pose," suggesting dynamic, pose-dependent human models.a
% AniGauss: Animation + Gaussian의 결합으로, 애니메이션 가능한 Gaussian 모델임을 직관적으로 전달.
% GaussMotion: Gaussian 기반의 동작 가능성을 강조하며, 움직임을 표현.


%%%%%%%%% AUTHORS - PLEASE UPDATE
\author{Gyumin Shim\\
Korea Advanced Institute of Science and Technology\\
{\tt\small shimgyumin@kaist.ac.kr}
% For a paper whose authors are all at the same institution,
% omit the following lines up until the closing ``}''.
% Additional authors and addresses can be added with ``\and'',
% just like the second author.
% To save space, use either the email address or home page, not both
\and
Sangmin Lee\\
Sungkyunkwan University \\
{\tt\small sangmin.lee@skku.edu}
\and
Jaegul Choo\\
Korea Advanced Institute of Science and Technology\\
{\tt\small  jchoo@kaist.ac.kr}
}

\begin{document}
%\maketitle
\twocolumn[{%
\renewcommand\twocolumn[1][]{#1}%
\maketitle
\centering
\includegraphics[width=1\linewidth]{Figures/teaser_2.pdf}
% \vspace{-2em}
\captionof{figure}{{\bf Examples of 3D human models generated by \ourmodel.}
%Our approach is based on deformable 3D Gaussian Splatting to render animated scenes from user-specified pose inputs.
Our method is able to generate high-quality Gaussian-based avatars from text and render animated scenes from user-specified pose inputs.
}
\label{fig:teaser}
}]


%\begin{abstract}

% Recent works to jointly reconstruct 3D human and object from a single RGB image, are mostly model-based, that fail to capture the fine details of the clothed human body and object surface. In this paper, we introduce ReCHOR, a novel, model-free, first-method to produce realistic clothed human-object reconstructions from a monocular view. This is extremely challenging due to human-object occlusions, diverse interactions and depth ambiguity, as it needs to infer both 3D spatial awareness and high resolution details. Our core idea is based on estimating neural implicit representations for human and object respectively by an attention-based neural implicit model that attends to pixel-aligned features from both the global human-object image for spatial awareness and  the local separate view of human and object images for high quality details. Additionally, the network is conditioned on semantic features from an initial estimated human-object pose prior and a generative diffusion model that inpaints occluded regions, thus enabling the retrieval of details from them.
% We also propose a synthetic dataset with rendered scenes of diverse, inter-occluded 3D human and object scans, to train our network. We evaluate our method on the synthetic and real world BEHAVE dataset. Our experiments show that our method outperforms the SOTA in achieving realistic clothed human-object reconstructions.
Recent approaches to jointly reconstruct 3D humans and objects from a single RGB image represent 3D shapes with template-based or coarse models, which fail to capture details of loose clothing on human bodies. In this paper, we introduce a novel implicit approach for jointly reconstructing realistic 3D clothed humans and objects from a monocular view. For the first time, we model both the human and the object with an implicit representation, allowing to capture more realistic details such as clothing. This task is extremely challenging due to human-object occlusions and the lack of 3D information in 2D images, often leading to poor detail reconstruction and depth ambiguity. To address these problems, we propose a novel attention-based neural implicit model that leverages image pixel alignment from both the input human-object image for a global understanding of the human-object scene and from local separate views of the human and object images to improve realism with, for example, clothing details. Additionally, the network is conditioned on semantic features derived from an estimated human-object pose prior, which provides 3D spatial information about the shared space of humans and objects. To handle human occlusion caused by objects, we use a generative diffusion model that inpaints the occluded regions, recovering otherwise lost details. For training and evaluation, we introduce a synthetic dataset featuring rendered scenes of inter-occluded 3D human scans and diverse objects. Extensive evaluation on both synthetic and real-world datasets demonstrates the superior quality of the proposed human-object reconstructions over competitive methods.
\end{abstract}    
%\section{Introduction}
\label{sec:intro}
% Image editing methods in diffusion models depend on user-defined control directions - users can unlock their creativity using these methods by specifying the desired manipulation through prompts~\cite{gandikota2023concept}, reference images~\cite{ruiz2022dreambooth, kumari2022customdiffusion, gal2022image, chen2024trainingfreeregionalpromptingdiffusion}, or attribute vectors~\cite{parmar2023zero,hertz2022prompt}. In this work, we ask a fundamentally different question: \emph{Can we automatically discover the underlying visual structure of a concept within diffusion model's knowledge?} %Rather than requiring user-specified controls, we aim to decompose the model's internal knowledge into meaningful directions.

% This question touches on a fundamental limitation in how we interact with diffusion models. Current control methods ~\cite{zhang2023addingconditionalcontroltexttoimage, gandikota2023concept, ye2023ipadaptertextcompatibleimage,ye2023ipadaptertextcompatibleimage, hertz2024stylealignedimagegeneration, li2023photomaker, shi2024instantbooth, chen2024trainingfreeregionalpromptingdiffusion} require users to specify their desired manipulations in advance, limiting interactive creativity. This contrasts with natural human artistic workflows, where creators dynamically explore creative ideas while jointly refining them toward meaningful artistic outcomes~\cite{hoffmann2016modeling}. This synergy between specification and exploration is not new to generative models. Early GAN architectures naturally developed disentangled latent spaces that enabled continuous\cite{harkonen2020ganspace,radford2015unsupervised, wu2021stylespace, shen2020interfacegan}, compositional control over generated images. Users could explore these spaces to discover interesting variations that would be difficult to describe in words~\cite{wu2021stylespace}, then combine them to achieve their creative goals~\cite{grabe2022towards}. 


% While diffusion models have largely superseded GANs in conditional image synthesis~\cite{dhariwal2021diffusion},  their underlying structure remains less understood. Diffusion models achieve remarkable diversity through high-dimensional latents, unlike GANs' compact latent spaces.  With a single prompt, diffusion models can generate radically different variations through different random initializations of input noise. We ask - Is it possible to discover interpretable structure within this vast space of variations?

Text-to-image diffusion models are capable of generating remarkable visual variations from a single prompt through different random initializations. However, this vast creative potential remains largely opaque to users---while we can generate diverse images, we lack understanding of the underlying structure of these variations. This presents a fundamental challenge: how can we discover and expose the latent visual capabilities encoded within these models?

\let\thefootnote\relax \footnote{$^{*}$Correspondence to \texttt{gandikota.ro@northeastern.edu}}

The challenge touches on a key limitation in how we interact with diffusion models today. Current control methods require users to explicitly specify their desired edits in advance through prompts~\cite{gandikota2023concept}, reference images~\cite{zhang2023addingconditionalcontroltexttoimage, chen2024trainingfreeregionalpromptingdiffusion, ruiz2022dreambooth,kumari2022customdiffusion, Ryu_lora, hu2021lora}, or attribute vectors~\cite{ye2023ipadaptertextcompatibleimage, hertz2024stylealignedimagegeneration, li2023photomaker, shi2024instantbooth,parmar2023zero,hertz2022prompt}. That contrasts sharply with natural human creative workflows, where artists dynamically explore creative ideas and jointly refine them toward meaningful artistic outcomes~\cite{hoffmann2016modeling}. The need for pre-specified controls creates a barrier between users and the full creative potential of these models.

Interestingly, earlier generative models like GANs~\cite{gans,karras2019style,brock2018large} naturally developed more interpretable internal structures. Their compact latent spaces often exhibited emergent disentanglement~\cite{harkonen2020ganspace,radford2015unsupervised, wu2021stylespace, shen2020interfacegan}, enabling continuous and compositional control over generated images. Users could explore these spaces to discover interesting variations that would be difficult to describe in words~\cite{wu2021stylespace}, then combine them to achieve their creative goals~\cite{grabe2022towards}.

Diffusion models have largely superseded GANs in conditional image synthesis~\cite{dhariwal2021diffusion}, achieving greater diversity through much higher-dimensional latents. And yet an understanding of the underlying structure of these larger latent spaces has remained elusive. In this work, we ask a fundamental question: \emph{Can we automatically discover the visual structure within a diffusion model's knowledge of a concept?} Rather than requiring user-specified controls, we aim to decompose the model's internal representations into expressive directions that users can explore and combine.

To address these needs, we present \textbf{SliderSpace}, a framework that brings systematic explorability to diffusion models. Given just a text prompt, SliderSpace discovers a canonical set of meaningful, diverse, and controllable directions within the model's knowledge of that concept. Each direction is implemented as a low-rank adapter~\cite{hu2021lora} that can be scaled and composed with others, allowing users to explore and smoothly combine different aspects of variation, as shown in Figure~\ref{fig:intro}.

We ground SliderSpace discovery in three key requirements for meaningful decomposition of a diffusion model's visual manifold: 
\begin{enumerate}
    \item \textbf{Unsupervised Discovery:} The decomposition process should emerge from the intrinsic structure of the model's learned representation, rather than being guided by predefined attributes. This ensures we capture the true topology of the model's knowledge space rather than projecting our assumptions onto it.
    
    \item \textbf{Semantic Orthogonality:} Each discovered control must represent a distinct semantic direction. This is enforced in a semantic feature space, like CLIP, where every slider has an orthogonal effect in embeddings. This prevents discovering multiple controls that create similar semantic effects, making the system more efficient and easier.
    
    \item \textbf{Distribution Consistency:} Directions must induce consistent transformations across both random seeds and prompt variations. 
\end{enumerate}

These requirements naturally lead to our proposed framework, which we formalize in Section~\ref{sec:method}. As we show in our experiments, SliderSpace is architecture-agnostic, working with both conventional U-Net based models like Stable Diffusion~\cite{rombach2022high, rombach2022sd20, podell2023sdxl, turbo, dmd} and recent transformer-based architectures like Flux~\cite{flux}.

We demonstrate the expressiveness of SliderSpace through three applications: First, we show how SliderSpace can decompose high-level concepts into diverse and expressive components, revealing the natural axes of variation in the model's understanding. Second, we explore artistic style variation, where SliderSpace discovers directions that match or exceed the diversity of manually curated artist lists while being judged more useful by human evaluators. Finally, we show how SliderSpace can help reverse the mode collapse commonly observed in distilled diffusion models, restoring diversity while maintaining generation speed.

Beyond providing practical creative control, SliderSpace opens new avenues for understanding and utilizing the latent capabilities of diffusion models. By mapping these models' visual potential into intuitive, composable directions, we take a step toward making their creative possibilities more accessible and interpretable to users.

% Image editing methods in diffusion models unlock the creativity of users. In this work we ask an alternate question: \emph{Can we organize and expose what of the diffusion model is already capable of?}.
% Existing methods for controlling image generation typically require users to manually specify edit directions for desired changes. This process is time-consuming, requires technical expertise, and limits the spontaneity of the creative process. For instance, if a user wants to adjust the smile of a generated person, they must explicitly request this edit, often through imprecise prompt engineering or model fine-tuning. This approach of predefined controls or manual specifications restricts users from fully exploring the latent capabilities of the model. There may be interesting stylistic variations or attributes that the model can generate, but users have no easy way to discover or utilize these.

% Natural visual disentanglement was an emergent property in the latent space of Generative Adversarial Models (GANs) \cite{harkonen2020ganspace,radford2015unsupervised, wu2021stylespace, shen2020interfacegan}. In particular, it has been observed that StyleGAN~\cite{karras2019style} stylespace neurons offer detailed control over many meaningful aspects of images that would be difficult to describe in words~\cite{wu2021stylespace}. However, diffusion models do not share such a compact latent space~\cite{park2023unsupervised}; and efforts to uncover such a space in the semantic embeddings of the text conditioning have met with limited success \nik{Nick - is there a specific citation you were thinking about?}.

% In this work we introduce \textbf{SliderSpace}, which takes a step towards uncovering an analogous low dimensional representation of diffusion models' visual breadth; in essence treating the diffusion model as many generators sharing parameters, where a particular generator is defined by a specific prompt. For a given prompt we sample many random seeds (and optionally prompt expansions using an LLM), generate the corresponding images, and apply an off the shelf feature extractor (in this work CLIP, but our method can be applied to any differentiable feature extractor). We use PCA to analyze these features, and for each of the leading $k$ principal components we train a LoRA \cite{} which causes the diffusion model to produces images which increase the feature magnitude along that component when passed back through the same feature extractor. This leads to a 'Slider' for each principal component, because each LoRA can be scaled and applied to the original diffusion model, continuously varying those visual features in the generated results (as measured, in our case, by CLIP).

% There are many other works that enhance the controllability of diffusion models. One common approach is enabling users to add spatial constraints to a generation either manually, or via a reference image \cite{zhang2023addingconditionalcontroltexttoimage, chen2024trainingfreeregionalpromptingdiffusion}, a second is leveraging more abstract embeddings (e.g. identity, style) extracted from a reference image \cite{ye2023ipadaptertextcompatibleimage, hertz2024stylealignedimagegeneration, li2023photomaker, shi2024instantbooth}, a third is finetuning a foundation model to better generate a concept important to the user \cite{ruiz2022dreambooth, kumari2022customdiffusion, Ryu_lora, hu2021lora}, and a fourth (most relevant to this work) is finding low-rank adaptors of the model based on a prompt or small training set which can be scaled to provide continous control over one aspect of generated image (e.g. night vs day, basic vs luxury, etc.) \cite{gandikota2023concept}. SliderSpace is complementary to all of these methods and offers something distinct. All of the other methods we are aware require the user (and / or model designer) to know in advance what type of control they want. In contrast SliderSpace assists users in discovering and controlling hidden capabilities present in the diffusion model's distribution of possible generations.

%We propose that truly intuitive creative control in a text-to-image model should meet three key criteria: \emph{discoverability}, \emph{intuitiveness}, and \emph{specificity}. The model should reveal controllable attributes that may not be immediately obvious, offer controls that are easy to understand and manipulate, and ensure each control affects a distinct attribute of the generated image.

% We demonstrate the utility and power of SliderSpace using three applications built on top of SDXL-DMD \cite{dmd}, because its fast generation speed lends itself well to the continuous control offered by SliderSpace.

% First, we study concept decomposition (Section \ref{sec:concept_exp}), where we learn sliders for a specific concept (e.g. 'monster', 'waterfall', 'car'). Through quantitative metrics of diversity and text alignment we demonstrate that the learned sliders dramatically boost the diversity of generations when randomly applied without harming text alignment; we also ask humans to qualitatively judge these results in a user study where they find the SliderSpace results to be more 'Diverse', 'Useful', and 'Creative' than our baselines.

% Second, we attempt to compare the automatic discoveries of SliderSpace to a large scale manual study of artistic styles (Section \ref{sec:art_exp}), open-sourced by ParrotZone \cite{parrotzone}. In this study SDXL was prompted with over 4300 artist names,  and based on visual inspection the cases of successful stylistic mimicry recorded. Quantitatively SliderSpace more closely matches the distribution of artistic variation discovered by ParrotZone than other baselines, and in our user studies was judged to be significantly more 'Diverse' and 'Useful' than the baselines. To our surprise humans even judged SliderSpace results to be slightly more 'Diverse' than the results generated by the manually discovered artist names of \cite{parrotzone}.

% Third, we attempt to use SliderSpace to reverse the mode collapse commonly observed in distilled few-step diffusion models relative to the original teacher model (Section \ref{sec:diverse_exp}). We quantitatively demonstrate that applying SliderSpace to SDXL-DMD leads to more closely matching the distribution of images by the original teacher, SDXL.

%Through extensive experiments on various state-of-the-art text-to-image models, we demonstrate that SliderSpace significantly enhances user control and creative expression in AI-assisted image generation tasks. Our method enables a range of applications, including concept decomposition and control, diversity improvement in generated images, customization dissection and edits, and the exploration of artistic styles inherent in the model.

% SliderSpace goes beyond providing a practical tool for enhanced creative control. By mapping the visual potential of diffusion models it can open new avenues for generative creativity and deepens our understanding of each model's hidden potential.
%\section{Related work}
\label{sec:formatting}

\subsection{Text-to-Video Generation}

T2V generation has made notable progress, evolving from early GAN-based models \cite{saito2017temporal,tulyakov2018mocogan,fu2023tell,li2018video,wu2022nuwa,yu2022generating} to newer transformer \cite{yan2021videogpt,arnab2021vivit,esser2021taming,ramesh2021zero,yu2022scaling} and diffusion models \cite{kirkpatrick2017overcoming,sohl2015deep,song2020denoising,zhang2022gddim}. Early efforts like MoCoGAN~\cite{tulyakov2018mocogan} focused on short video clips but faced issues with stability and coherence. The introduction of transformers improved sequential data handling, enhancing video generation, while diffusion models further improved video quality by progressively denoising the input. 
Despite these advances, T2V models still struggle to reflect human preferences, with the generated videos generally lacking aesthetic quality. Additionally, the scarcity of paired video preference data hinders effective model training and may lead to insufficient flexibility and poor quality in the generated videos.


\subsection{RLHF}

\iffalse
Aligning LLMs \cite{dai1901transformer,radford2019language,zhang2023opt} typically involves two steps: supervised fine-tuning followed by Reinforcement Learning with Human Feedback (RLHF) \cite{gao2023scaling,stiennon2020learning,rafailov2024direct}. Although effective, RLHF is computationally expensive and can lead to issues like reward hacking. Methods like DPO have streamlined alignment by leveraging feedback data directly, improving efficiency.

In contrast, diffusion model alignment is still evolving, focusing mainly on enhancing visual quality through curated datasets. Techniques like DOODL \cite{wallace2023end} and AlignProp \cite{prabhudesai2023aligning} target aesthetic improvements but face challenges with complex tasks such as text-image alignment. Reinforcement learning methods like DPOK \cite{fan2024reinforcement} and DDPO \cite{black2023training} improve reward optimization but struggle with scalability. DPO-SDXL integrates DPO into T2I generation, boosting both alignment and aesthetics.

However, aligning video generation remains a largely unaddressed challenge, especially when dealing with motion consistency and semantic coherence across frames.
\fi

RLHF \cite{gao2023scaling,stiennon2020learning,rafailov2024direct} is a method that utilizes human feedback to guide machine learning models. Early RLHF algorithms, such as DDPG~\cite{lillicrap2015continuous} and PPO~\cite{schulman2017proximal}, typically relied on complex reward models to quantify human feedback. These reward models require a large amount of annotated data and face challenges during tuning. As research has progressed, more efficient preference learning methods have emerged, among which DPO has become a new framework. DPO does not depend on a separate reward model; instead, it obtains human preferences through pairwise comparisons and directly optimizes these preferences. This shift not only simplifies the application of RLHF but also enhances the alignment of models with human values. Furthermore, DPO has been successfully introduced into T2I tasks~\cite{wallace2024diffusion,yang2024using}, providing new insights for generative models in addressing the alignment of human preferences and showcasing DPO's potential in the field of AIGC~\cite{shi2024instantbooth,
qing2024hierarchical,menapace2024snap,koley2024s}. However, there remains a gap in current research regarding the application of DPO strategies to T2V tasks. Effectively integrating DPO into T2V tasks presents a challenging endeavor.


%\section{Preliminary}
\label{sec:preliminary}
In this section, we first introduce the mathematical formulation of flow-based text-to-image generative models~\cite{Xingchao_2022,Lipman_2022}, which forms the foundation of modern T2I systems~\cite{sd3,sdxl,imagen3,imagen}. We then describe classifier-free guidance~\cite{ho2022classifier}, a key technique to control the generation process through text conditioning.

\subsection{Flow-based text-to-image generative models}
In state-of-the-art T2I models~\cite{sd3}, the image generation process is modeled by learning, through a neural network, a flow $\psi$ that generates a probability path $(p_t)_{0\le t\le 1}$ bridging the source distribution $p_0$ and the target distribution $p_1$ ~\cite{Xingchao_2022,Lipman_2022}. This framework encompasses diffusion models~\cite{sohl2015deep,ddpm} as a special case. In particular, a commonly used formulation sets a Gaussian distribution as the source: $p_0 = \mathcal{N}(\mathbf{0}, \mathbf{I})$ and a delta distribution centered on a sample $\mathbf{x}_1$ from the data distribution $q$ as the target: $p_1 = \delta_{\mathbf{x}_1}$.
Then, it incorporates an affine conditional flow $\psi_t(\mathbf{x} | \mathbf{x}_1) = a_t \mathbf{x}_1 + b_t \mathbf{x}$ with the boundary condition $(a_0, b_0) = (0, 1),\ (a_1, b_1) = (1, 0)$ to bridge them. The neural network typically approximates quantities such as velocity fields, $x_0$ prediction or $x_1$ prediction. In this modeling, these quantities can be viewed as affine transformations of the marginal probability path score $\nabla_{\mathbf{x}} \log p_t(\mathbf{x})$.

\subsection{Classifier-free guidance in flow-based models}
Classifier-free guidance~\cite{ho2022classifier} is a method for sampling from a model conditioned by a text input $\mathbf{y}$ by guiding an unconditional image generation model modeled using the score $\nabla_{\mathbf{x}} \log p_t(\mathbf{x})$. This enables the sampling from
\[
q_w(\mathbf{x}, \mathbf{y}) \propto q(\mathbf{x})q(\mathbf{y}|\mathbf{x})^w \propto q(\mathbf{x})^{1-w}q(\mathbf{x}|\mathbf{y})^w
\]
where $w \in \mathbb{R}$ is the guidance scale typically used with $w > 1$. The score satisfies
\[
\nabla_{\mathbf{x}} \log q_w(\mathbf{x}, \mathbf{y}) = (1-w)\nabla_{\mathbf{x}} \log q(\mathbf{x}) + w\nabla_{\mathbf{x}} \log q(\mathbf{x}|\mathbf{y})
\]
so by training the network to learn both the unconditional score $\nabla_{\mathbf{x}} \log q(\mathbf{x})$ and conditional score $\nabla_{\mathbf{x}} \log q(\mathbf{x}|\mathbf{y})$, flexible sampling from the conditional distribution can be achieved through a weighted sum of the network outputs.
\begin{abstract}

In this paper, we introduce \ourmodel, a novel human rendering model that generates fully animatable scenes aligned with textual descriptions using Gaussian Splatting.
Although existing methods achieve reasonable text-to-3D generation of human bodies using various 3D representations, they often face limitations in fidelity and efficiency, or primarily focus on static models with limited pose control.
In contrast, our method generates fully animatable 3D avatars by combining deformable 3D Gaussian Splatting with text-to-3D score distillation, achieving high fidelity and efficient rendering for arbitrary poses.
By densely generating diverse random poses during optimization, our deformable 3D human model learns to capture a wide range of natural motions distilled from a pose-conditioned diffusion model in an end-to-end manner.
%Furthermore, we propose Adaptive Score Distillation, which effectively balances the challenges of over-saturation and high variance to achieve optimal results that generate realistic details while appropriately aligning with text.
Furthermore, we propose Adaptive Score Distillation that effectively balances realistic detail and smoothness to achieve optimal 3D results.
%Experimental results demonstrate that our approach surpasses existing baselines, producing high-quality textures in novel-pose animations and generating both diverse and realistic 3D human models from textual input.
Experimental results demonstrate that our approach outperforms existing baselines by producing high-quality textures in both static and animated results, and by generating diverse 3D human models from various textual inputs.


%In this paper, we introduce GaussianMotion, a novel human rendering model that generates fully animatable scenes aligned with textual descriptions using Gaussian Splatting. Although existing methods achieve reasonable text-to-3D generation of human bodies using various 3D representations, they often face limitations in fidelity and efficiency, or primarily focus on static models with limited pose control. In contrast, our method generates fully animatable 3D avatars by combining deformable 3D Gaussian Splatting with text-to-3D score distillation, achieving high fidelity and efficient rendering for arbitrary poses. By densely generating diverse random poses during optimization, our deformable 3D human model learns to capture a wide range of natural motions distilled from a pose-conditioned diffusion model in an end-to-end manner. Furthermore, we propose Adaptive Score Distillation that effectively balances realistic detail and smoothness to achieve optimal 3D results. Experimental results demonstrate that our approach outperforms existing baselines by producing high-quality textures in both static and animated results, and by generating diverse 3D human models from various textual inputs.

\end{abstract}

%%%%%%%%% BODY TEXT
\section{Introduction}



%3dhuman의 중요성   
The demand for reconstructing and rendering 3D avatars has surged with advancements in computer graphics, enabling a wide range of applications, including virtual reality and metaverse content.
Building on diverse 3D representations~\cite{wang2021neus, shen2021deep, mildenhall2021nerf, kerbl20233d}, numerous studies have explored the reconstruction of 3D human avatars from various data sources, including 3D scans~\cite{saito2019pifu, saito2020pifuhd, zheng2021pamir}, video sequences~\cite{weng2022humannerf, peng2021animatable, hu2024gauhuman}, single images~\cite{huang2022one, huang2024tech}, and even text~\cite{kolotouros2024dreamhuman, cao2024dreamavatar, liao2024tada, liu2024humangaussian}.
In particular, generating 3D human models from text is significantly challenging, as textual descriptions provide limited information about the 3D structure of the human body, which has complex articulations.

% diffusion 과 sds의 등장. human 모델에 적용 
As text-to-image diffusion models~\cite{rombach2022high, saharia2022photorealistic, podell2023sdxl} have emerged, capable of synthesizing diverse and realistic images from textual information, their applicability has significantly expanded across various research fields.
The introduction of score distillation by DreamFusion~\cite{poole2022dreamfusion} marked a breakthrough, enabling text-to-3D synthesis using 2D diffusion models without requiring labeled 3D data. 
Building upon this technique, subsequent studies~\cite{wang2024prolificdreamer,tang2023dreamgaussian,yi2023gaussiandreamer,lin2023magic3d} have further explored the generation of detailed 3D models from text in an unsupervised manner. 

%% 
The text-to-3D task includes the generation of 3D {\it human} models, demonstrating the potential of this approach to create realistic and detailed representations of human models based on textual descriptions.
%However, most existing approaches, particularly those using neural representations, often encounter challenges related to fidelity and efficiency when rendering high-resolution images. They also tend to focus on static models, which limits their applicability in dynamic scenarios where complex pose variation and animation are essential.
Recently, several approaches leveraging neural representations~\cite{mildenhall2021nerf} or mesh representations~\cite{shen2021deep} have been proposed~\cite{kolotouros2024dreamhuman, cao2024dreamavatar, liao2024tada, huang2024dreamwaltz, huang2024humannorm}. 
%However, these methods often face challenges in maintaining both fidelity and efficiency, particularly when rendering high-resolution images.
% 어떤 limitation인지 에 대한 추가설명 
However, these methods often face challenges in balancing both fidelity and efficiency. Neural representations typically incur high computational costs due to the large number of point samples along each ray, particularly when rendering high-resolution images, while mesh-based methods struggle to preserve fine-grained details due to limitations in resolution and structure.


% In contrast, HumanGaussian~\cite{liu2024humangaussian}, based on 3D Gaussian Splatting, has demonstrated unprecedented generation quality by capitalizing on the real-time and memory-efficient properties of Gaussian Splatting. These strengths also extend to the training process, enabling faster rendering of full-body human images. However, HumanGaussian focuses on static models during training, limiting its applicability in dynamic scenarios where handling complex pose variations and animations is critical.
% %as it struggles with poses not seen during training.
% GAvatar~\cite{yuan2024gavatar} is a concurrently proposed Gaussian-based method that further addresses multi-pose handling using a primitive-based 3D representation.
% However, GAvatar heavily relies on mesh representation to regularize the geometry of 3D Gaussian points, which compromises its flexibility and ability to represent complex geometries. \todo{fine grained, 머리카락,등등 예시, gavat 먼저 서술하고 humangaussian .. fully gaussian based ' ... }This reliance leads to difficulties in capturing fine details that are not adequately modeled by the mesh, resulting in suboptimal performance and an overly smoothed geometric structure.
With the emergence of 3D Gaussian Splatting~\cite{kerbl20233d}, a powerful 3D model that surpasses neural representations in terms of efficiency, several Gaussian-based human rendering methods have also emerged.
Yuan et al.\cite{yuan2024gavatar} proposed a hybrid method that combines mesh and Gaussian representations, using Signed Distance Functions (SDF)~\cite{park2019deepsdf} to regularize the opacity of the Gaussian.
%However, this is a hybrid method combining mesh and Gaussian representations, and cannot be considered a fully Gaussian-based approach.
HumanGaussian~\cite{liu2024humangaussian} introduced a fully Gaussian-based approach, demonstrating competitive generation quality and enabling efficient rendering of full-body human images during training and inference.
%by leveraging the inherent strengths of Gaussian Splatting. 
%These strengths also extend to the training process, enabling faster rendering of full-body human images. 
However, HumanGaussian focuses on static models, which limits its applicability in dynamic scenarios where handling complex pose variations and animations is critical.

 
%우리 방법 소개 
%In this paper, we present a novel human rendering model that addresses these limitations by generating fully animatable scenes aligned with textual descriptions using Gaussian Splatting.
%In this paper, we present a novel human rendering model that addresses these limitations by generating fully animatable scenes aligned with textual descriptions, leveraging the inherent strengths of Gaussian Splatting. 

%In this paper, we address these limitations with a novel human rendering model that generates fully animatable scenes from textual descriptions, leveraging the inherent strengths of Gaussian Splatting. 
%Our approach enables end-to-end training, successfully producing realistic and animatable Gaussian avatars without relying on regularization from other representations, thereby ensuring fine-grained geometric details.

%As a result, our method allows for the creation of diverse 3D human models based on the provided text, allowing for the rendering of high-fidelity images in dynamic motion sequences while maintaining computational efficiency.

To address the limitations of fidelity, efficiency, and pose variation, we propose \ourmodel, a novel human rendering model that generates realistic and animatable scenes from textual descriptions. Our approach enables end-to-end training of animatable Gaussian avatars without relying on regularization from other representations, capturing fine-grained geometric details purely through the strengths of Gaussian Splatting.

%technical 한 내용 뒤로 따로 뺌 
As depicted in \Figref{fig:framework}, our method integrates deformable 3D human Gaussian Splatting with pose-aware text-to-3D score distillation.
%Starting from random locations on the canonical SMPL mesh surfaces as initialization, 
Gaussian points in the canonical space are optimized to capture both the appearance aligned with the given text and pose articulation as the input pose changes.
A key innovation of our model is the use of densely generated random poses with explicit pose guidance during optimization. 
% This enables the deformable 3D human model to learn a wide range of poses distilled from a pre-trained diffusion model in an end-to-end manner.
% We utilize a pose-conditioned diffusion model~\cite{rombach2022high} as a prior model, where ControlNet~\cite{zhang2023adding} takes the pose image as input to generate pose-consistent images.
% This provides strong pose-aware guidance when complex pose images are rendered during training.
This allows the deformable 3D human model to effectively learn a diverse range of poses distilled from a pre-trained pose-conditioned diffusion model in an end-to-end manner, providing robust pose-aware guidance when rendering complex poses during training. 
Additionally, to achieve optimal results with realistic details, we propose Adaptive Score Distillation as an alternative to naive score distillation sampling (SDS)~\cite{poole2022dreamfusion},
%This approach addresses both the over-saturation issues of naive SDS and the high variance problems associated with improved score distillation methods.
%This approach achieves a balanced result that preserves fine details while minimizing undesired noise, balancing smoothness and high uncertainty.
which balances the preservation of fine details while minimizing undesired noise, effectively handling the trade-off between smoothness and high uncertainty.


%High variance can destabilize training, resulting in distorted outcomes due to high uncertainty.
% that significantly deviate from the text due to high uncertainty.
%The Adaptive Score Distillation effectively achieves optimal results by generating realistic details while ensuring appropriate alignment with text.

As a result, our method enables the creation of diverse 3D human models based on the provided text, capturing intricate texture details and supporting realistic animations according to user-specified input poses, all while maintaining computational efficiency. We validate our approach through extensive experiments, demonstrating that it significantly outperforms existing baselines.
%especially in random pose animation scenarios. 

\noindent\ In summary, our contributions are as follows:
\begin{itemize}
\setlength\itemsep{-0em}
    \item A novel human rendering model with deformable Gaussian Splatting to create 3D human models aligned with textual descriptions, capable of exhibiting a wide variety of motions.
    \item An innovative framework that optimizes Gaussian points by generating random poses during training, allowing the model to learn both detailed appearances from text descriptions and complex pose articulations through explicit pose guidance.
    %distilled from a pose-conditioned diffusion model.
    \item Adaptive Score Distillation, an improved strategy over naive SDS, effectively balancing the issues of over-saturation and high variance to achieve optimal results with realistic details.
\end{itemize}

%단순결합은 왜 어려운지... 

\begin{figure*}[t]
\centering
\begin{tabular}{@{}c}
\includegraphics[width=1\linewidth]{Figures/architecture-v8.pdf}
\end{tabular}
\caption{{\bf Overview of our proposed framework.}
Given a text prompt as input, we generate animatable 3D humans by modeling deformable Gaussian Splatting, where Gaussian points adapt their positions based on input poses. 
The points are defined in a canonical space and shared across different poses (observation spaces). Random poses are sampled to deform the Gaussian points and rendered as pose images to provide pose-aware guidance for the rendered images $\mathbf{x}$ through score distillation. 
After optimizing the Gaussian points to reflect the appearances described by the text prompt, fully animatable scenes are rendered based on user-specified input poses during inference.
}
\label{fig:framework}
\end{figure*}


\section{Related Works}
\subsection{3D Gaussian Avatar}

%Recently, numerous studies in 3D avatar modeling have increasingly leveraged Gaussian representations~\cite{kerbl20233d} to achieve high-quality, animatable human models that can respond dynamically to various movements and expressions.
%Drawing inspiration from various human deformation concepts derived from deformable neural representations~\cite{gao2023neural, peng2021neural, peng2021animatable, weng2022humannerf}, numerous methods~\cite{hu2024gauhuman, hu2024gaussianavatar, lei2024gart, qian20233dgs, zielonka2023drivable} have emerged, proposing innovative approaches to reconstructing human avatars using Gaussian representations.
Recently, with the emergence of 3D Gaussian Splatting~\cite{kerbl20233d}, which has demonstrated powerful performance in various 3D applications, numerous studies in 3D avatar modeling have increasingly leveraged this technique to create high-quality human models. A range of methods~\cite{hu2024gauhuman, hu2024gaussianavatar, lei2024gart, qian20233dgs, zielonka2023drivable} proposes innovative approaches for reconstructing human avatars using Gaussian representations that can respond dynamically to various movements and expressions. 
These studies draw inspiration from human deformation concepts derived from deformable neural representations~\cite{gao2023neural, peng2021neural, peng2021animatable, weng2022humannerf}, which address how 3D coordinates on a human model are deformed across different poses.
Furthermore, more sophisticated forms of 3D human avatars have been developed, such as ExAvatar~\cite{moon2024expressive}, which incorporates facial and hand expressions, and UV Gaussian~\cite{jiang2024uv}, a hybrid form of animatable avatar that jointly learns mesh deformation and 2D Gaussian textures.
After reconstructing an avatar from monocular or calibrated multi-view videos, these methods facilitate the rendering of scenes from arbitrary viewpoints and poses using the trained 3D Gaussian points during inference, leveraging the computational efficiency of Gaussian representations. 
%This capability makes them not only effective but also scalable for real-time applications.
In this work, we introduce a novel method that can produce an animatable Gaussian avatar from text without requiring any image ground truths.

%Gauhuman: Articu-lated gaussian splatting from monocular human video
%GaussianAvatar: Towards Realistic Human Avatar Modeling from a Single Video via Animatable 3D Gaussians
%Gart: Gaussian articulated template models
%“3DGSAvatar: Animatable Avatars via Deformable 3D Gaussian Splatting,
%Drivable 3D Gaussian Avatars,
%Expressive whole-body 3d gaussian avatar,
%PhysAvatar: Learning the Physics of Dressed 3D Avatars from Visual Observations
%Human Performance Modeling and Rendering via Neural Animated Mesh


\subsection{Text-to-3D Human Generation}

Text-to-3D is a popular task which is to generate 3D models from input textual descriptions without relying on text-3D paired data.
Early work utilizing CLIP~\cite{radford2021learning} embeddings to optimize 3D shapes~\cite{sanghi2022clip} or neural representations~\cite{jain2022zero} has successfully demonstrated that 3D objects can be generated solely from textual descriptions.
As DreamFusion~\cite{poole2022dreamfusion} introduces a method to distill priors from pre-trained 2D diffusion models for targeting 3D models, numerous text-to-3D methods~\cite{tang2023dreamgaussian, yi2023gaussiandreamer, wang2024prolificdreamer} have emerged, aiming to generate high-quality 3D models from input textual descriptions by leveraging various diffusion models.
These methodologies can be directly extended to the task of generating 3D {\it humans}, with DreamHuman~\cite{kolotouros2024dreamhuman} and DreamAvatar~\cite{cao2024dreamavatar} being among the first works in this area that incorporate score distillation to optimize the human neural radiance field (NeRF) model. 
They utilize a deformable human NeRF model to render animatable scenes generated from diverse input texts.
TADA~\cite{liao2024tada} leverages SMPL-X~\cite{SMPL-X:2019} for modeling shape and UV texture, allowing for the rendering of more detailed 3D avatars. 
Recently, HumanNorm~\cite{huang2024humannorm} and Deceptive-Human~\cite{kao2023deceptive} have pushed the boundaries of 3D quality by incorporating additional 3D priors, including depth, normals, and pose information of human shapes.
HumanGaussian~\cite{liu2024humangaussian} successfully integrates Gaussian representations into the text-to-3D human task by training Gaussian splats with score distillation in a stable manner. However, it lacks animation capabilities, as it is designed exclusively for training static poses.


\section{Preliminaries}

\subsection{3D Gaussian Splatting}

3D Gaussian Splatting~\cite{kerbl20233d} is a powerful 3D modeling technique that enables real-time rendering by representing objects or scenes as collections of Gaussian splats.
Each splat $\mathcal{G}$ is characterized by its position $\mathbf{x}$, opacity $\alpha$, color $c$, and covariance matrix $\Sigma$, which defines its shape and spread through a scaling matrix $\boldsymbol{S}$ and rotation matrix $\boldsymbol{R}$. 
The overall image can be rendered by projecting each 3D Gaussian splat $\mathcal{G}$ onto the image plane. The pixel color is determined by accumulating the alpha values of the Gaussian splats as follows:

\begin{equation}
 C=\sum_i\left(\alpha_i^{\prime} \prod_{j=1}^{i-1}\left(1-\alpha_j^{\prime}\right)\right) c_i,
\end{equation}

\noindent where $\alpha_i^{\prime}$ represents the visibility $\alpha_i$ of the $i$-th splat, weighted by the probability density of $i$-th projected 2D Gaussian at the target pixel, and $c$ denotes the color value computed from spherical harmonics coefficients. 
%The probability density of i-th projected 2D Gaussian is defined as 

It is mostly known for its real-time rendering while maintaining image quality compared to implicit neural representations.
While the original 3D Gaussians are optimized using a photometric loss based on the provided ground-truth pixels, our proposed method learns from the distillation loss derived from the diffusion model.


%%%%
% $G(\mathbf{x}; \mathbf{p}_i, \Sigma_i)$ is the Gaussian function given by:
% \begin{equation}
% G(\mathbf{x}; \mathbf{p}_i, \Sigma_i) = \frac{1}{(2\pi)^{d/2} |\Sigma_i|^{1/2}} \exp\left(-\frac{1}{2} (\mathbf{x} - \mathbf{p}_i)^\top \Sigma_i^{-1} (\mathbf{x} - \mathbf{p}_i)\right)
% \end{equation}
% where $d$ is the dimensionality of the space. This representation allows for efficient rendering and can capture complex geometries with varying scales.


\subsection{Score Distillation}

Numerous powerful text-to-image diffusion models~\cite{rombach2022high, saharia2022photorealistic, ramesh2022hierarchical} have been proposed, demonstrating unprecedented image quality achieved through training on billions of text-image pairs.
Building on these diffusion models, Score Distillation Sampling (SDS) was introduced by DreamFusion~\cite{poole2022dreamfusion}, which distills prior knowledge from pre-trained 2D diffusion models to optimize the target 3D model.
When provided with a pre-trained diffusion model $\epsilon_\phi$, SDS optimizes the parameters of the 3D model $\theta$ using the gradient of the loss, which is represented as:

\begin{equation}
\nabla_\theta \mathcal{L}_{\mathrm{SDS}}=\mathbb{E}_{t, \epsilon, y}\left[w(t)\left(\epsilon_\phi\left(\mathbf{x}_t ; y, t\right)-\epsilon\right) \frac{\partial \mathbf{x}}{\partial \theta}\right],
\label{eq:sds}
\end{equation}

\noindent where $\mathbf{x}$ denotes the image rendered by the 3D model, $\mathbf{x}_t$ represents the rendered image with Gaussian noise $\epsilon$ added, and $y$ is the textual input encoded by the text encoder.
At each training iteration, different values of $t$ are sampled steering the rendered images closer to the distribution of real images.
While SDS produces successful distillation results by generating diverse 3D renderings that align with the given text inputs, it faces an over-saturation problem that critically impacts the creation of realistic details in 3D objects.

% \begin{equation}
% \delta=\underbrace{\left[\epsilon_\phi\left(\mathbf{x}_t ;t\right)-\boldsymbol{\epsilon}\right]}_{\text {generative score } \delta_g}+s \cdot \underbrace{\left[\epsilon_\phi\left(\mathbf{x}_t ; y, t\right)-\epsilon_\phi\left(\mathbf{x}_t ; t\right)\right]}_{\text {classifier score } \delta_c}
% \end{equation}

% \begin{equation}
%  \nabla_\theta \mathcal{L}_{\mathrm{CSD}}=\mathbb{E}_{t, \epsilon, \mathbf{c}}\left[w(t)\left(\epsilon_\phi\left(\mathbf{x}_t ; y, t\right)-\epsilon_\phi\left(\mathbf{x}_t ; t\right)\right) \frac{\partial \mathbf{x}}{\partial \theta}\right] 
% \end{equation}

\section{Proposed Method}

%over view 
An overview of our proposed method is described in \Figref{fig:framework}. 
%We propose the human rendering model that can generate both the text-aligned 3D human model when given input textual description and fully animatable scenes with random motion sequences by optimizing the deformable 3D human model with score distillation. 
We introduce \ourmodel, a human rendering model that generates text-aligned 3D human avatars from input textual descriptions and creates fully animatable scenes with random motion sequences by optimizing the deformable 3D Gaussian points using score distillation.

\subsection{Pose Deformable 3DGS}
%non-rigid 얘기... 
% \begin{equation}
% \left\{\mathcal{G}_o\right\}=\mathcal{F}_{\theta_r}\left(\left\{\mathcal{G}_c\right\} ;\left\{\mathbf{B}_{\mathbf{b}}\right\}_{b=1}^B\right)
% \end{equation}

% \begin{equation}
%  \mathbf{x}_o=L B S_{\sigma_w}\left(\mathbf{x}_c ;\left\{\mathbf{B}_b\right\}\right)=\sum_{b=1}^B f_{\sigma_w}\left(\mathbf{x}_c\right)_b \mathbf{B}_b \mathbf{x}_c 
% \end{equation}

%4.1 구성 고민해보기 3dgs avatar 의 느낌을 너무 주는것이 맞는지 

%Inspired by 3DGS-avatar~\cite{qian20233dgs}, which creates animatable human avatars using 3D Gaussian Splatting, we adopted the rigid transformation method for Gaussian splats from this study. 
To create animatable human avatars using 3D Gaussian Splatting, Gaussian points are defined in the canonical space and shared across all different poses. 
Starting from random locations on the canonical SMPL mesh surfaces as initialization, these Gaussian points are optimized to learn both the appearance consistent with the provided text and the ability to articulate various poses as the input pose varies.
%We adopt a rigid transformation method~\cite{qian20233dgs} that deforms each Gaussian splat according to the pose input.
To model different poses, each Gaussian splat is deformed according to the pose input using a rigid transformation method~\cite{qian20233dgs}.
Specifically, the Linear Blend Skinning (LBS) function is applied to transform 3D Gaussian splats from the canonical space to the observation space, where the target pose is specified as input.
As the human skeleton consist of $B$ joints, the transformation $\mathbf{T}$ is represented as the weighted sum of rigid bone transformations as: 

% \begin{equation}
% \mathbf{x}^{c}=T(\mathbf{x}, \mathbf{p}) =\left(\sum_{k=1}^K w_k(\mathbf{x}) M^{\text{trg2can}}_k\right) \mathbf{x},
% \end{equation}

\begin{equation}
\mathbf{x}_o=\mathbf{T} \mathbf{x}_c = (\sum_{b=1}^B \mathbf{w}_b(\mathbf{x_c}) \mathbf{B}_b) \mathbf{x}_c,
\label{eq:transform}
\end{equation}

\noindent where $\mathbf{x}_c$ represents the position of Gaussian splats defined in the canonical space, and $\mathbf{x}_o$ denotes the position in the observation space.
$\mathbf{w}_b$ represents the blend weight for the $b$-th bone in the canonical space, which is further optimized.

$\mathbf{B}_b\in SE(3)$ is the transformation matrix of $b$-th skeleton part, mapping the bone's coordinates from the canonical space to the observation space.
Along with the transformation of positions, the rotation of Gaussian splats is also adjusted according to the equation $\mathbf{R}_o=\mathbf{T}_{1: 3,1: 3} \mathbf{R}_c$.
Note that the rigid bone transformation $\mathbf{B}$ can be computed from the given body pose (see Supplementary Material for details).
%blend weights are optimized by a neural skinning field using a MLP network which take the position coordinates of Gaussian splats as input. 

In 3DGS-avatar~\cite{qian20233dgs}, the blend weight $\mathbf{w}_b$ is optimized from scratch, as the ground truth provides clear guidance for determining the optimal values. 
In our method, on the other hand, the deformation must also be learned through distillation, resulting in weak guidance for optimizing the blend weights. 
Therefore, we propose to learn the residual blend weight relative to the SMPL blend weight $\mathbf{w}^{\mathbf{S}}$ as:

\begin{equation}
\mathbf{w}_b(\mathbf{x}_c)=\operatorname{norm}\left(f_{\theta_r}\left(\mathbf{x}_c\right)_b+\mathbf{w}^{\mathbf{S}}_{b}(\mathbf{x}_c)\right),
\label{eq:blend_weight}
\end{equation}

\noindent where $f_{\theta_r}$ is the MLP network which take the position coordinates of Gaussian splats as input to output residual blend weight values. 
The initial blend weight value $\mathbf{w}^{\mathbf{S}}$ can be computed based on the nearest vertex on the SMPL mesh by calculating the distance from the position coordinates of the Gaussian splats to each vertex.
The residual blend weight values are regularized by minimizing the $l2$-difference $\mathcal{L}_{\text{skinning}}$ between the predicted blend weight $\mathbf{w}_b$ and the initial SMPL blend weight $\mathbf{w}^{\mathbf{S}}$ across all positions of the Gaussian splats $\mathbf{x}_c$. 
By adopting this approach, we ensure that the pose transformation of SMPL is preserved from the initial stage, which effectively aids in converging to the appropriate blend weight.

\begin{figure*}[t]
\centering
%\def\arraystretch{0.2}
\begin{tabular}{@{}c}
\includegraphics[width=1\linewidth]{Figures/quali_static_4.pdf} \\
\end{tabular}
\caption{{\bf Qualitative comparison of 3D human models in a static A-pose.} We evaluate our approach against recent state-of-the-art baselines using different prompts. For each method, two images are rendered from frontal and side views, respectively.
%viewpoints of 0 and 60 degrees,
}
\label{fig:static}
\end{figure*}

%\subsection{Random Pose Sampling with Pose Guidance}
\subsection{Dynamic Pose Guidance}
Previous methods~\cite{huang2024humannorm, liu2024humangaussian} have limited pose control, as they are primarily trained on static poses. 
%To generate fully animatable scenes across random motion sequences, randomly posed images must be learned during the optimization of our 3D human model.
To guarantee natural animatable scenes across diverse motion sequences, randomly posed images must be learned during the optimization of our 3D human model.
At each training step, we randomly sample poses from a normal distribution, where the mean pose is represented by a star pose.
By observing the randomly posed images during training, the deformable 3D human model learns a wide range of poses distilled from a pre-trained diffusion model in an end-to-end manner.

To provide robust pose-aware guidance when the complex pose images are rendered, we utilize ControlNet~\cite{zhang2023adding} which takes a pose image $p$ to generate pose-consistent images. 
As shown in \Figref{fig:framework}, the random pose sampled in each training iteration not only transforms the 3D Gaussian splats but is also rendered as pose images to be input into ControlNet, which adds spatial conditioning controls to the pre-trained diffusion models.
This integration facilitates more coherent distillation, ensuring that the generated outputs align with the sampled poses.
%and enhances the overall realism and dynamism of the animations.




\begin{figure*}[t]
\centering
\def\arraystretch{0.2}
\begin{tabular}{@{}c}
\includegraphics[width=1\linewidth]{Figures/quali_pose_3.pdf} \\
\end{tabular}
\caption{{\bf Qualitative comparison of 3D human models in animated scenes.} We evaluate our approach against recent state-of-the-art baselines in a one-to-one manner. For each method, four images are rendered in different poses corresponding to each text prompt.}
\label{fig:animation}
\end{figure*}

\subsection{Adaptive Score Distillation}
%adaptive 용어 더 고민하기 weight 텀이 낫지 않나... 

By integrating pose guidance into score distillation, the score function in the gradient of the loss~\eqnref{eq:sds} is reformulated as $\epsilon_\phi\left(\mathbf{x}_t; y, t, p\right)$, where $p$ denotes the pose image conditioned on the diffusion model.
We then apply classifier-free guidance (CFG)~\cite{ho2022classifier} to score function and decompose the score matching difference into two components: the denoising score $\delta_n$ and the classifier score $\delta_c$, defined as:
%building on previous studies~\cite{yu2023text, katzir2023noise} that aim to enhance the output quality of SDS

% \begin{equation}
% \delta= \delta_n + s \cdot \delta_c \\
% = \left[\epsilon_\phi\left(\mathbf{x}_t ;t, p\right)-\boldsymbol{\epsilon}\right] +s \cdot \left[\epsilon_\phi\left(\mathbf{x}_t ; y, t, p\right)-\epsilon_\phi\left(\mathbf{x}_t ; t, p\right)\right],
% \end{equation} 

\begin{equation}
\resizebox{.9\hsize}{!}{$
\begin{split}
\delta &= \delta_n + s \cdot \delta_c \\
       &= \left[\epsilon_\phi\left(\mathbf{x}_t ;t, p\right)-\boldsymbol{\epsilon}\right] 
       %&\quad + s \cdot \left[\epsilon_\phi\left(\mathbf{x}_t ; y, t, p\right) - %\epsilon_\phi\left(\mathbf{x}_t ; t, p\right)\right]
       + s \cdot \left[\epsilon_\phi\left(\mathbf{x}_t ; y, t, p\right)-\epsilon_\phi\left(\mathbf{x}_t ; t, p\right)\right],
\end{split}
$}
\end{equation}
\noindent where $s$ is the guidance scale for CFG sampling.
%$p$ is the pose image conditioned on the diffusion model.

%While the classifier score ideally directs toward a local maximum in the probability density of noisy real images conditioned on $y$, the denoising score term introduces significant noise, which can lead to blurry outputs due to an averaging effect, as discussed in \cite{katzir2023noise}.
While $\delta_c$ aims to direct the model towards high-density regions of real images conditioned on $y$, the denoising score $\delta_n$ often introduces excessive noise, resulting in blurry outputs due to averaging effects~\cite{katzir2023noise}. 
%and associated with high uncertainty.
Previous methods~\cite{yu2023text, katzir2023noise} attempted to mitigate this by either removing the denoising score entirely or using a negative classifier score~\cite{katzir2023noise, liu2024humangaussian}. 
%However, we empirically found that simply removing the denoising score results in undesired effects in the output, such as noise or shadows.
%We also observed that several results deviated significantly from the original text descriptions, which can be attributed to the high uncertainty of the classifier score, as shown in \figref{fig:abl_ads}.
However, we empirically found that completely omitting the denoising score can produce artifacts, such as noise or shadow-like distortions, and that the generated outputs may deviate from the intended text description due to the high uncertainty of the classifier score (see \figref{fig:abl_ads}).


%The denoising scores are less noisy at higher timesteps, allowing higher-level semantics to remain relatively unaffected. However, they become very noisy at smaller timesteps, resulting in blurry outputs due to an averaging effect.
%We propose a simple yet effective technique: adaptive score distillation, where only the classifier score is applied for specific timesteps below $\tau$, where the denoising score causes an averaging effect.  
Based on the observation~\cite{katzir2023noise} that the denoising score  $\delta_n$ becomes overly noisy at smaller timesteps, while contributing to smoothness at larger timesteps,
we propose a simple yet effective technique called Adaptive Score Distillation (ASD). 
% In this approach, the denoising score is removed for timesteps below a threshold $\tau$ to mitigate its adverse effects. For timesteps beyond $\tau$, the denoising score is reintroduced to capture smooth high-level semantics. 
In this approach, the denoising score is selectively removed for timesteps below a threshold $\tau$ to mitigate noise, while being reintroduced for timesteps beyond $\tau$.
The score matching difference in ASD is defined as follows:

% \begin{equation}
% \resizebox{.9\hsize}{!}{$
% \delta = \begin{cases}
% \epsilon_\phi\left(\mathbf{x}_t ; y, t, p\right) - \epsilon_\phi\left(\mathbf{x}_t ; t, p\right), & \text{if } t < \tau \\
% \left[\epsilon_\phi\left(\mathbf{x}_t ; t, p\right) - \boldsymbol{\epsilon}\right] \\
% \quad + s \cdot \left[\epsilon_\phi\left(\mathbf{x}_t ; y, t, p\right) - \epsilon_\phi\left(\mathbf{x}_t ; t, p\right)\right], & \text{otherwise.}
% \end{cases}
% $}
% \label{eq:ads}
% \end{equation}

% \begin{equation}
% \resizebox{.5\hsize}{!}{$
% \delta = \begin{cases}
% \delta_c, & \text{if } t < \tau \\
% \delta_n + s \cdot \delta_c, & \text{otherwise.}
% \end{cases}
% $}
% \label{eq:ads}
% \end{equation}

\begin{equation}
\delta = \delta_c \cdot \mathbb{1}_{t < \tau} + (\delta_n + s \cdot \delta_c) \cdot \mathbb{1}_{t \geq \tau},
\label{eq:ads}
\end{equation}

\noindent where $\tau$ can be adaptively defined to balance realistic details and smoothness. 
%450 is used for our experiment. 

%With this straightforward technique, high-level semantics are smoothly aligned with text descriptions through the denoising score at higher timesteps, while low-level details are generated more aggressively solely by the classifier score at lower timesteps. This approach ensures both output quality and clarity, effectively minimizing undesired noise in the output.
This adaptive approach allows our model to leverage the strengths of both score components: ensuring smoother high-level semantic alignment through the denoising score at higher timesteps, while maintaining sharp, detailed features by relying solely on the classifier score at lower timesteps. This ensures both output quality and clarity, effectively minimizing undesired noise in the 3D output.
%consistency with the input text.


\subsection{Training Objective}

\noindent {\bf Scale Regularization} 
The proposed adaptive score distillation successfully generates highly detailed objects; however, we observed some blurry results in certain samples, which were caused by Gaussian splats with large scales.
The blurriness observed around the surface arises from the SDS-based supervision, which is significantly more stochastic than photometric loss, as also noted in HumanGaussian~\cite{liu2024humangaussian}.
To overcome this problem, we propose to apply scale regularization loss during the optimization. 
Typically, the scales of Gaussian splats are adjusted through pruning techniques in the adaptive density control process, which involves removing Gaussian points that exceed a specified scale.
However, this often results in excessive pruning of Gaussian points, leading to a decrease in resolution and challenges in maintaining a balance with densification.
Additionally, points may be pruned at the boundaries, which can lead to a collapse of the human shape as training progresses. We found that imposing a regularization loss to limit the scale size yields the best output quality, preserving both the resolution of the Gaussian points and clear boundaries.

The scale regularization loss is defined as follow: 
\begin{equation}
\mathcal{L}_{\text {scale}}=\frac{1}{|\mathcal{P}|} \sum_{p \in \mathcal{P}} \max \left\{\max \left(\boldsymbol{S}_p\right), r\right\}-r
\label{eq:scale}
\end{equation}
\noindent where $\boldsymbol{S}_p$ represents the scalings of the 3D Gaussians. This loss regularizes the size of the 3D Gaussian points to ensure they do not exceed $r$.

% {\bf Depth Smoothness} In addition to the scale regularization, we propose to apply depth smoothness loss ~\cite{}, which is widely used technique for many 3D tasks to acquire smooth depths. This loss can be applied optionally, and it stabilizes the trainingg the human body in the early stages. 
% vaporous bodies which leads to failure. 
% body 표면의 큰 면적이 opacity 가 낮아지거나 point들이 이동하면서 소실되는 현상이 있음. 이런경우 학습이 매우 불안정하게 학습됨. 

% ads 는 어떤 기준에서 비교하는게 맞는거지? 
% \begin{equation}
% \begin{split}
% \mathcal{L}_{TV} = \frac{1}{N} \sum_{i,j} \Bigg( \min\left( |D_{i+1,j} - D_{i,j}|, 0.2 \right) \\
%  + \min\left( |D_{i,j+1} - D_{i,j}|, 0.2 \right) \Bigg),
% \end{split}
% \end{equation}
% where $D_{i,j}$ is the pixel value of depth map of renderd image.
% N is the total number of pixels in the image 

\noindent {\bf Total Training Objective} Along with the proposed Adaptive Score Distillation $\mathcal{L}_{\text{ASD}}$, our total training objective functions for our method are written as: 
\begin{equation}
\begin{aligned}
&\mathcal{L}_{\text{total}}=\mathcal{L}_{\text{ASD}}+\lambda_{\text{scale}}\mathcal{L}_{\text{scale}} +\lambda_{\text{skinning}}\mathcal{L}_{\text{skinning}},
\end{aligned}
\end{equation}
\noindent where $\lambda_{\text{scale}}$ and $\lambda_{\text{skinning}}$ are hyperparameters determining the importance of each loss.





% \subsection{Training Details}
% \noindent {\bf Rendering Detail in the Training} To facilitate convergence, we render the front and back views of the canonical pose with a probability of 0.2, while images with random poses are rendered at a probability of 0.8 at each iteration. Additionally, we crop 256$\times$256 patches from the full-body rendered images to capture finer details in the local regions of the generated output. During training, the pose is randomly sampled using random noise scaled by 0.3 as the pose parameters, and the viewing angle is randomly selected from azimuth range of $[-180^\circ, 180^\circ]$

% \noindent {\bf Adaptive Density Control} 3DGS adaptively controls the resolution of Gaussian points through Adaptive Density Control (ADC). It periodically densifies or prunes points based on predefined criteria, such as position gradients, scaling, or opacity.  Beginning with an initial set of 70,000 sparse points on the SMPL mesh, We perform ADC every 100 iterations.
% We follow the same strategy as the original 3DGS, which clones and splits Gaussians with an average magnitude of view-space position gradients exceeding a threshold, which we set to 0.02 in our method. Additionally, we clone points that exceed 0.02 in mean distance from the $k$ nearest Gaussian points to avoid vaporous regions.
% Pruning is conducted at a scaling factor threshold of 0.1, and we also prune the 3D Gaussian points that deviate too far from the surface of the SMPL model. This is achieved by computing the distance between the nearest SMPL vertex and the positions of the Gaussian splats, which we set to a threshold of 0.1.

% gs22.5_asd450
% _anneal1000
% _whitebg
% _frontback0.2
% _th0.02
% _scale0.1
% _scaleloss0.01_1000
% _patch256face0.4a
% _randompose_0.3
% _rigid1
% _smplprune0.1
% _clone0.02_
% init0.8 
% {\bf View prompting} The Janus problem, characterized by the presence of multiple heads in the output, is a well-known phenomenon in many text-to-3D tasks. This issue arises because the diffusion model tends to generate outputs from a canonical viewpoint, influenced by the distribution of the training data. In the case of the human domain, this specifically refers to the canonical pose. We address this problem through view prompting, which augments the input prompt with back, side, and front views, and interpolates the text embeddings based on the azimuth sampled during training.

%-------------------------------------------------------------------------


\section{Experimental Results}

\subsection{Qualitative Evaluation}

%\noindent {\bf Qualitative Evaluation} 
We compare our method with recent state-of-the-art approaches in two settings: static poses and animations generated from motion sequences.
For the static pose setting, we compare our method with DreamHuman~\cite{kolotouros2024dreamhuman}, TADA~\cite{liao2024tada}, HumanNorm~\cite{huang2024humannorm}, and HumanGaussian~\cite{liu2024humangaussian}, which are based on neural~\cite{mildenhall2021nerf}, mesh~\cite{shen2021deep}, and Gaussian representations~\cite{kerbl20233d}, respectively. For the animation results, we compare our method with DreamHuman, GAvatar~\cite{yuan2024gavatar}, and HumanGaussian.
Note that the official code of DreamHuman and GAvatar are not released, we use the results from their project pages.
%and we used the official training code and pretrained models for TADA, HumanNorm, and HumanGaussian. \todo{here}

As shown in \Figref{fig:static}, our method generates the most realistic 3D human results in terms of fine geometry and texture details when compared to the other baselines.
DreamHuman suffers from overly smoothed results, as it is represented with solid colors only, lacking details such as wrinkles on the clothing.
TADA produces more detailed textures than DreamHuman, but its human shapes appear unrealistic.
In the case of HumanNorm, areas like the face are generated with more detail through their refinement process, but the overall quality of the full body remains too simplistic, similar to DreamHuman.
HumanGaussian delivers the most plausible performance overall among comparison baselines, but it struggles to capture fine details like hair (see row 1), and leaves behind blurry artifacts between the arms (see row 3).
In contrast, our method generates high-frequency details and delivers clean results without any blurry artifacts. Moreover, as we will show shortly, the difference becomes even clearer in animation results.

We present the animation results in \Figref{fig:animation}. 
For each example, images with 4 different poses are visualized for each method.
DreamHuman leverages the structure of deformable human NeRF during training, enabling it to exhibit natural pose transformations. However, it still lacks realistic details, resulting in overly smooth and simplistic texture quality.
GAvatar also produces natural animation results by handling multiple poses through a primitive-based transformation. However, it struggles to capture complex geometric details, such as hair or loose clothing, which is a persistent issue associated with mesh representations.
%Although it shows fine texture details, it shows misalignment between the geometry an apperance, where some geometry details are just embedded in the trexture. 
%However, GAvatar heavily relies on mesh representation to regularize the geometry of 3D Gaussian points, which compromises its flexibility and ability to represent complex geometries. \todo{fine grained, 머리카락,등등 예시, gavat 먼저 서술하고 humangaussian .. fully gaussian based ' ... }This reliance leads to difficulties in capturing fine details that are not adequately modeled by the mesh, resulting in suboptimal performance and an overly smoothed geometric structure.
HumanGaussian shows critical weaknesses in animation, as it is not designed to be animatable. 
While it provides a heuristic animation function by mapping Gaussian points to the SMPL-X~\cite{SMPL-X:2019} body mesh, 
%and as the body pose changes, the points' positions are updated using body surface normals and distance.
it suffers from severe artifacts during pose changes (e.g., around the arms) due to mapping errors.
%between the Gaussian points and the mesh surface. 
Additionally, it exhibits artifacts in occluded areas in the static pose, such as under the arms (see the 3rd example).
In contrast, our method produces realistic results in both geometry and texture for novel poses by capturing fine-grained details.
%exclusively using a Gaussian representation that captures fine-grained details.
%Additionally, our approach can generalize to novel poses without introducing any artifacts. 
%Note that we synchronized the poses only for HumanGaussian with ours, as the official code for DreamHuman and GAvatar has not been released. 



\begin{table}[t]
\centering
\renewcommand{\tabcolsep}{2.2mm}
\renewcommand{\arraystretch}{1.4}
\resizebox{0.9\linewidth}{!}{
\begin{tabular}{lccc}
\bottomrule
\bf Methods & \bf CLIP Score ↑ & \bf FID ↓ & \bf HPSv2 ↑\\ 
\toprule
HumanNorm~\cite{huang2024humannorm} & 27.81 & 5.41 & 0.256 \\
HumanGaussian~\cite{liu2024humangaussian} & 28.45 & 4.18 & 0.263\\ \hline
\rowcolor{cellcol} % 음영 추가
\bf GaussianMotion (ours) & \textbf{29.26} & \textbf{4.05} & \textbf{0.268} \\ 
\toprule
\end{tabular}
}
\caption{{\bf Quantitative comparisons with state-of-the-art text-to-3D human methods.} We evaluate CLIP score, FID, and HPS scores on rendered images.}
%\vspace{-1mm}
\label{tb:quanti}
\end{table}



\begin{figure}[t]
\centering
\def\arraystretch{0.2}
\begin{tabular}{@{}c}
\includegraphics[width=1\linewidth]{Figures/ablation_pose_2.pdf} \\
\end{tabular}
\caption{{\bf Ablation studies on pose guidance.} 
We present rendered images from 3D models trained with and without pose guidance, with the input pose shown in the first column. Additionally, we show generated images sampled from noised rendered images, with and without pose conditioning. }
\label{fig:abl_pose}
\end{figure}


\subsection{Quantitative Evaluation}
%\noindent {\bf Quantitative Evaluation} 
Following HumanNorm~\cite{huang2024humannorm} and HumanGaussian~\cite{liu2024humangaussian}, we conducted a quantitative evaluation to assess the quality of the 3D rendered images produced by our method. 
We selected 30 text prompts from the list provided by HumanGaussian (see Supplementary Material for details) to constitute our test set.
% First, we measured the Fréchet Inception Distance (FID)~\cite{heusel2017gans}, which quantifies the distance between two image datasets in feature space, typically defined as the datasets of real images and generated images.
% For the real image dataset, we sampled 10 images for each prompt using Stable Diffusion V1.5~\cite{rombach2022high}, while the multiview images rendered from 10 views within an azimuth range of $[-180^\circ, 180^\circ]$ were used for the generated image set.
First, we measured the Fréchet Inception Distance (FID)\cite{heusel2017gans}, which quantifies the similarity between feature distributions of real and generated images. We sampled 10 images per prompt using Stable Diffusion V1.5\cite{rombach2022high} for the real images, and used multiview images rendered from 10 azimuth angles in the range of $[-180^\circ, 180^\circ]$ for the generated image set.
Second, we evaluated the CLIP~\cite{hessel2021clipscore} and HPSv2~\cite{wu2023human} scores on the frontal views, which measure the similarity between embeddings encoded from the rendered images and the corresponding text. 
As detailed in \tabref{tb:quanti}, our method achieves the best score across all metrics, demonstrating that our method exhibits the best visual quality and consistency with the input text.


\noindent {\bf User Study} 
We conducted a user study to compare our method with recent state-of-the-art approaches~\cite{huang2024humannorm, liu2024humangaussian}. In this study, we presented pairs of multiview and animated scenes rendered by our method and one of the comparison methods. Using the same set of text prompts as in the quantitative evaluation, we created 30 A$\vs$B pairs and collected responses from 17 participants. Users were asked to assess three criteria: 1) Geometric Quality, 2) Texture Quality, and 3) Text Alignment. The results, summarized in \tabref{tb:user}, indicate that our method consistently outperforms the comparison methods across all criteria.


\begin{table}[t]
\centering
\renewcommand{\tabcolsep}{0.6mm}
\renewcommand{\arraystretch}{1.4}

% \resizebox{0.9\linewidth}{!}{
% \begin{tabular}{cccc}
% \hline
% Method & Texture Quality & Geometric Quality & Text Alignment \\ \hline
% HumanNorm & 1.312 & 1.312 & 1.312 \\
% HumanGaussian & 1.312 & 1.312 & 1.312 \\ \hline
% Ours &  &  & 1.312 \\ \hline
% \end{tabular}
% }
% \resizebox{0.5\linewidth}{!}{
% \begin{tabular}{cc}
% \hline
% Method & Animation Quality  \\ \hline
% HumanGaussian & 1.312 \\ \hline
% Ours &  a \\ \hline
% \end{tabular}
% }

\resizebox{0.999\linewidth}{!}{
\begin{tabular}{lccc}
\bottomrule
\bf Competitors & \bf Geometry & \bf Texture & \bf Text Consistency \\ \toprule
\vs. HumanNorm~\cite{huang2024humannorm} & 65.78 & 65.49 & 72.65 \\ 
\vs. HumanGaussian~\cite{liu2024humangaussian} & 84.51 & 89.31 & 67.45 \\ \toprule
\end{tabular}
}
\caption{{\bf User study with state-of-the-art text-to-3D human methods.} We present the preference percentage of our method compared to state-of-the-art methods.}
%\vspace{-1mm}
\label{tb:user}
\end{table}


\begin{figure}[t]
\centering
\def\arraystretch{0.2}
\begin{tabular}{@{}c}
\includegraphics[width=1\linewidth]{Figures/ablation_ads.pdf} \\
\end{tabular}
\caption{{\bf Ablation studies on $\tau$ changes in Adaptive Score Distillation.} Adjusting the $\tau$ value achieves a balanced result, preserving fine details while minimizing noise.}
\label{fig:abl_ads}
\end{figure}

\subsection{Ablation Studies}
%residual blend weight?
%We conduct some ablation studies on our two technical novelties: Adaptive Score Distillation (ADS) and scaling regularization. 
\noindent {\bf Pose Guidance}
We demonstrate the impact of incorporating pose guidance, which significantly enhances our method's performance, by comparing it with a setting where the model is trained without pose guidance (using only text input for the diffusion model). As shown in \figref{fig:abl_pose}, when trained without pose guidance, the model fails to capture the correct pose and orientation (for example, no face is generated in row 1). 
In the right part of the figure, we also show a generated image sampled from a rendered image with Gaussian noise added at $t=600$.  
The image generated without pose conditioning does not accurately reflect the input pose shown in the reference, implying that the diffusion model fails to produce pose-consistent scores during distillation, significantly hindering our 3D model from learning the correct poses.

\noindent {\bf Adaptive Score Distillation} 
Here, we demonstrate the effectiveness of Adaptive Score Distillation and show how the generation results vary with different $\tau$ values in \eqnref{eq:ads}. First, we follow the annealed distillation time schedule from~\cite{wang2024prolificdreamer}, reducing the maximum distillation timestep to 500 from the middle of the training process, which improves visual quality in the later stages.
We then decrease $\tau$ from 500 (corresponding to training with only the classifier score) in intervals of 100.
As shown in \Figref{fig:abl_ads}, the smoothness derived from the denoising score and the low-level details from the classifier score are interpolated as $\tau$ changes. 
With the highest $\tau$ value, our method fails to produce clean results, incorporating undesired noise such as floating artifacts and shadows. Additionally, we observe that some samples deviate significantly from the original text description (see row 1).
On the other hand, with lower value of $\tau$, which approach the naive SDS as it decrease, it shows results that are oversaturated and far from realistic.
In contrast, by adjusting the $\tau$ value (around 400), we were able to achieve the most balanced results, satisfying both fine details and a lack of noise.


% metric 도 보여줘야 되나?? 
\noindent {\bf Scale Regularization} 
We empirically found that scale regularization significantly improves generation quality by reducing blurriness and enhancing fine details through small Gaussian points. In our baseline setting (a), Gaussian points are pruned if their scaling factor exceeds 0.1. To evaluate the effect of scale regularization versus pruning, we also tested settings (b) and (c), where points with scaling factors above 0.02 and 0.01 are pruned, respectively. Lastly, setting (d) applies scale regularization without altering the pruning threshold.
While setting (a) suffers from blurriness that severely harms visual quality, settings (b) and (c) successfully remove it but at the cost of excessive pruning, which reduces resolution and causes loss of shape as training progresses. In contrast, as shown in (d), applying a regularization loss with $r$ set to 0.01 in \eqnref{eq:scale} yields the best visual quality, preserving  both the resolution of Gaussian points and clear boundaries.


\begin{figure}[t]
\centering
\def\arraystretch{0.2}
\begin{tabular}{@{}c}
\includegraphics[width=1\linewidth]{Figures/ablation_scale.pdf} \\
\end{tabular}
\caption{{\bf Ablation studies on scale regularization against scaling-based pruning.} We present frontal views rendered using each trained 3D model, which are trained with scaling-based pruning and scale regularization, respectively.}
\label{fig:abl_scale}
\end{figure}

%-------------------------------------------------------------------------
\section{Conclusion}

%In this paper, we introduced a novel approach for generating animatable 3D human models from text descriptions using Gaussian Splatting. Our method successfully addresses the limitations of existing approaches, such as limited fidelity, efficiency, and dynamic pose control, by leveraging deformable Gaussian Splatting combined with pose-aware score distillation. By densely sampling random poses during training, our model effectively learns diverse pose variations and intricate appearance details, resulting in high-quality, realistic renderings. The proposed Adaptive Score Distillation further enhances output quality, balancing detail and smoothness to mitigate issues such as over-saturation and high variance. Experimental results confirm that our method consistently outperforms state-of-the-art baselines, offering a powerful solution for creating efficient, detailed, and pose-flexible 3D avatars from a wide range of textual descriptions.

In this paper, we present \ourmodel, a novel approach for generating animatable 3D human models from text descriptions using Gaussian Splatting. Our method overcomes the limitations of existing approaches, including fidelity, efficiency, and dynamic pose control, by combining deformable Gaussian Splatting with pose-aware score distillation. By densely sampling random poses during training, our model learns diverse pose variations and fine details, resulting in high-quality renderings. The proposed Adaptive Score Distillation further refines output quality, balancing detail and smoothness. Experimental results demonstrate that our method outperforms state-of-the-art baselines, offering an efficient, detailed, and pose-flexible solution for creating 3D avatars from text.

{
    \small
    \bibliographystyle{ieeenat_fullname}
    \bibliography{main}
}

% WARNING: do not forget to delete the supplementary pages from your submission 
%
%% bare_jrnl_compsoc.tex
%% V1.4b
%% 2015/08/26
%% by Michael Shell
%% See:
%% http://www.michaelshell.org/
%% for current contact information.
%%
%% This is a skeleton file demonstrating the use of IEEEtran.cls
%% (requires IEEEtran.cls version 1.8b or later) with an IEEE
%% Computer Society journal paper.
%%
%% Support sites:
%% http://www.michaelshell.org/tex/ieeetran/
%% http://www.ctan.org/pkg/ieeetran
%% and
%% http://www.ieee.org/

%%*************************************************************************
%% Legal Notice:
%% This code is offered as-is without any warranty either expressed or
%% implied; without even the implied warranty of MERCHANTABILITY or
%% FITNESS FOR A PARTICULAR PURPOSE! 
%% User assumes all risk.
%% In no event shall the IEEE or any contributor to this code be liable for
%% any damages or losses, including, but not limited to, incidental,
%% consequential, or any other damages, resulting from the use or misuse
%% of any information contained here.
%%
%% All comments are the opinions of their respective authors and are not
%% necessarily endorsed by the IEEE.
%%
%% This work is distributed under the LaTeX Project Public License (LPPL)
%% ( http://www.latex-project.org/ ) version 1.3, and may be freely used,
%% distributed and modified. A copy of the LPPL, version 1.3, is included
%% in the base LaTeX documentation of all distributions of LaTeX released
%% 2003/12/01 or later.
%% Retain all contribution notices and credits.
%% ** Modified files should be clearly indicated as such, including  **
%% ** renaming them and changing author support contact information. **
%%*************************************************************************


% *** Authors should verify (and, if needed, correct) their LaTeX system  ***
% *** with the testflow diagnostic prior to trusting their LaTeX platform ***
% *** with production work. The IEEE's font choices and paper sizes can   ***
% *** trigger bugs that do not appear when using other class files.       ***                          ***
% The testflow support page is at:
% http://www.michaelshell.org/tex/testflow/


\documentclass[10pt,journal,compsoc]{IEEEtran}
%
% If IEEEtran.cls has not been installed into the LaTeX system files,
% manually specify the path to it like:
% \documentclass[10pt,journal,compsoc]{../sty/IEEEtran}





% Some very useful LaTeX packages include:
% (uncomment the ones you want to load)


% *** MISC UTILITY PACKAGES ***
%
%\usepackage{ifpdf}
% Heiko Oberdiek's ifpdf.sty is very useful if you need conditional
% compilation based on whether the output is pdf or dvi.
% usage:
% \ifpdf
%   % pdf code
% \else
%   % dvi code
% \fi
% The latest version of ifpdf.sty can be obtained from:
% http://www.ctan.org/pkg/ifpdf
% Also, note that IEEEtran.cls V1.7 and later provides a builtin
% \ifCLASSINFOpdf conditional that works the same way.
% When switching from latex to pdflatex and vice-versa, the compiler may
% have to be run twice to clear warning/error messages.






% *** CITATION PACKAGES ***
%
\ifCLASSOPTIONcompsoc
  % IEEE Computer Society needs nocompress option
  % requires cite.sty v4.0 or later (November 2003)
  \usepackage[nocompress]{cite}
\else
  % normal IEEE
  \usepackage{cite}
\fi
% cite.sty was written by Donald Arseneau
% V1.6 and later of IEEEtran pre-defines the format of the cite.sty package
% \cite{} output to follow that of the IEEE. Loading the cite package will
% result in citation numbers being automatically sorted and properly
% "compressed/ranged". e.g., [1], [9], [2], [7], [5], [6] without using
% cite.sty will become [1], [2], [5]--[7], [9] using cite.sty. cite.sty's
% \cite will automatically add leading space, if needed. Use cite.sty's
% noadjust option (cite.sty V3.8 and later) if you want to turn this off
% such as if a citation ever needs to be enclosed in parenthesis.
% cite.sty is already installed on most LaTeX systems. Be sure and use
% version 5.0 (2009-03-20) and later if using hyperref.sty.
% The latest version can be obtained at:
% http://www.ctan.org/pkg/cite
% The documentation is contained in the cite.sty file itself.
%
% Note that some packages require special options to format as the Computer
% Society requires. In particular, Computer Society  papers do not use
% compressed citation ranges as is done in typical IEEE papers
% (e.g., [1]-[4]). Instead, they list every citation separately in order
% (e.g., [1], [2], [3], [4]). To get the latter we need to load the cite
% package with the nocompress option which is supported by cite.sty v4.0
% and later. Note also the use of a CLASSOPTION conditional provided by
% IEEEtran.cls V1.7 and later.





% *** GRAPHICS RELATED PACKAGES ***
%
\ifCLASSINFOpdf
  % \usepackage[pdftex]{graphicx}
  % declare the path(s) where your graphic files are
  % \graphicspath{{../pdf/}{../jpeg/}}
  % and their extensions so you won't have to specify these with
  % every instance of \includegraphics
  % \DeclareGraphicsExtensions{.pdf,.jpeg,.png}
\else
  % or other class option (dvipsone, dvipdf, if not using dvips). graphicx
  % will default to the driver specified in the system graphics.cfg if no
  % driver is specified.
  % \usepackage[dvips]{graphicx}
  % declare the path(s) where your graphic files are
  % \graphicspath{{../eps/}}
  % and their extensions so you won't have to specify these with
  % every instance of \includegraphics
  % \DeclareGraphicsExtensions{.eps}
\fi
% graphicx was written by David Carlisle and Sebastian Rahtz. It is
% required if you want graphics, photos, etc. graphicx.sty is already
% installed on most LaTeX systems. The latest version and documentation
% can be obtained at: 
% http://www.ctan.org/pkg/graphicx
% Another good source of documentation is "Using Imported Graphics in
% LaTeX2e" by Keith Reckdahl which can be found at:
% http://www.ctan.org/pkg/epslatex
%
% latex, and pdflatex in dvi mode, support graphics in encapsulated
% postscript (.eps) format. pdflatex in pdf mode supports graphics
% in .pdf, .jpeg, .png and .mps (metapost) formats. Users should ensure
% that all non-photo figures use a vector format (.eps, .pdf, .mps) and
% not a bitmapped formats (.jpeg, .png). The IEEE frowns on bitmapped formats
% which can result in "jaggedy"/blurry rendering of lines and letters as
% well as large increases in file sizes.
%
% You can find documentation about the pdfTeX application at:
% http://www.tug.org/applications/pdftex






% *** MATH PACKAGES ***
%
%\usepackage{amsmath}
% A popular package from the American Mathematical Society that provides
% many useful and powerful commands for dealing with mathematics.
%
% Note that the amsmath package sets \interdisplaylinepenalty to 10000
% thus preventing page breaks from occurring within multiline equations. Use:
%\interdisplaylinepenalty=2500
% after loading amsmath to restore such page breaks as IEEEtran.cls normally
% does. amsmath.sty is already installed on most LaTeX systems. The latest
% version and documentation can be obtained at:
% http://www.ctan.org/pkg/amsmath





% *** SPECIALIZED LIST PACKAGES ***
%
%\usepackage{algorithmic}
% algorithmic.sty was written by Peter Williams and Rogerio Brito.
% This package provides an algorithmic environment fo describing algorithms.
% You can use the algorithmic environment in-text or within a figure
% environment to provide for a floating algorithm. Do NOT use the algorithm
% floating environment provided by algorithm.sty (by the same authors) or
% algorithm2e.sty (by Christophe Fiorio) as the IEEE does not use dedicated
% algorithm float types and packages that provide these will not provide
% correct IEEE style captions. The latest version and documentation of
% algorithmic.sty can be obtained at:
% http://www.ctan.org/pkg/algorithms
% Also of interest may be the (relatively newer and more customizable)
% algorithmicx.sty package by Szasz Janos:
% http://www.ctan.org/pkg/algorithmicx




% *** ALIGNMENT PACKAGES ***
%
%\usepackage{array}
% Frank Mittelbach's and David Carlisle's array.sty patches and improves
% the standard LaTeX2e array and tabular environments to provide better
% appearance and additional user controls. As the default LaTeX2e table
% generation code is lacking to the point of almost being broken with
% respect to the quality of the end results, all users are strongly
% advised to use an enhanced (at the very least that provided by array.sty)
% set of table tools. array.sty is already installed on most systems. The
% latest version and documentation can be obtained at:
% http://www.ctan.org/pkg/array


% IEEEtran contains the IEEEeqnarray family of commands that can be used to
% generate multiline equations as well as matrices, tables, etc., of high
% quality.




% *** SUBFIGURE PACKAGES ***
%\ifCLASSOPTIONcompsoc
%  \usepackage[caption=false,font=footnotesize,labelfont=sf,textfont=sf]{subfig}
%\else
%  \usepackage[caption=false,font=footnotesize]{subfig}
%\fi
% subfig.sty, written by Steven Douglas Cochran, is the modern replacement
% for subfigure.sty, the latter of which is no longer maintained and is
% incompatible with some LaTeX packages including fixltx2e. However,
% subfig.sty requires and automatically loads Axel Sommerfeldt's caption.sty
% which will override IEEEtran.cls' handling of captions and this will result
% in non-IEEE style figure/table captions. To prevent this problem, be sure
% and invoke subfig.sty's "caption=false" package option (available since
% subfig.sty version 1.3, 2005/06/28) as this is will preserve IEEEtran.cls
% handling of captions.
% Note that the Computer Society format requires a sans serif font rather
% than the serif font used in traditional IEEE formatting and thus the need
% to invoke different subfig.sty package options depending on whether
% compsoc mode has been enabled.
%
% The latest version and documentation of subfig.sty can be obtained at:
% http://www.ctan.org/pkg/subfig




% *** FLOAT PACKAGES ***
%
%\usepackage{fixltx2e}
% fixltx2e, the successor to the earlier fix2col.sty, was written by
% Frank Mittelbach and David Carlisle. This package corrects a few problems
% in the LaTeX2e kernel, the most notable of which is that in current
% LaTeX2e releases, the ordering of single and double column floats is not
% guaranteed to be preserved. Thus, an unpatched LaTeX2e can allow a
% single column figure to be placed prior to an earlier double column
% figure.
% Be aware that LaTeX2e kernels dated 2015 and later have fixltx2e.sty's
% corrections already built into the system in which case a warning will
% be issued if an attempt is made to load fixltx2e.sty as it is no longer
% needed.
% The latest version and documentation can be found at:
% http://www.ctan.org/pkg/fixltx2e


%\usepackage{stfloats}
% stfloats.sty was written by Sigitas Tolusis. This package gives LaTeX2e
% the ability to do double column floats at the bottom of the page as well
% as the top. (e.g., "\begin{figure*}[!b]" is not normally possible in
% LaTeX2e). It also provides a command:
%\fnbelowfloat
% to enable the placement of footnotes below bottom floats (the standard
% LaTeX2e kernel puts them above bottom floats). This is an invasive package
% which rewrites many portions of the LaTeX2e float routines. It may not work
% with other packages that modify the LaTeX2e float routines. The latest
% version and documentation can be obtained at:
% http://www.ctan.org/pkg/stfloats
% Do not use the stfloats baselinefloat ability as the IEEE does not allow
% \baselineskip to stretch. Authors submitting work to the IEEE should note
% that the IEEE rarely uses double column equations and that authors should try
% to avoid such use. Do not be tempted to use the cuted.sty or midfloat.sty
% packages (also by Sigitas Tolusis) as the IEEE does not format its papers in
% such ways.
% Do not attempt to use stfloats with fixltx2e as they are incompatible.
% Instead, use Morten Hogholm'a dblfloatfix which combines the features
% of both fixltx2e and stfloats:
%
% \usepackage{dblfloatfix}
% The latest version can be found at:
% http://www.ctan.org/pkg/dblfloatfix




%\ifCLASSOPTIONcaptionsoff
%  \usepackage[nomarkers]{endfloat}
% \let\MYoriglatexcaption\caption
% \renewcommand{\caption}[2][\relax]{\MYoriglatexcaption[#2]{#2}}
%\fi
% endfloat.sty was written by James Darrell McCauley, Jeff Goldberg and 
% Axel Sommerfeldt. This package may be useful when used in conjunction with 
% IEEEtran.cls'  captionsoff option. Some IEEE journals/societies require that
% submissions have lists of figures/tables at the end of the paper and that
% figures/tables without any captions are placed on a page by themselves at
% the end of the document. If needed, the draftcls IEEEtran class option or
% \CLASSINPUTbaselinestretch interface can be used to increase the line
% spacing as well. Be sure and use the nomarkers option of endfloat to
% prevent endfloat from "marking" where the figures would have been placed
% in the text. The two hack lines of code above are a slight modification of
% that suggested by in the endfloat docs (section 8.4.1) to ensure that
% the full captions always appear in the list of figures/tables - even if
% the user used the short optional argument of \caption[]{}.
% IEEE papers do not typically make use of \caption[]'s optional argument,
% so this should not be an issue. A similar trick can be used to disable
% captions of packages such as subfig.sty that lack options to turn off
% the subcaptions:
% For subfig.sty:
% \let\MYorigsubfloat\subfloat
% \renewcommand{\subfloat}[2][\relax]{\MYorigsubfloat[]{#2}}
% However, the above trick will not work if both optional arguments of
% the \subfloat command are used. Furthermore, there needs to be a
% description of each subfigure *somewhere* and endfloat does not add
% subfigure captions to its list of figures. Thus, the best approach is to
% avoid the use of subfigure captions (many IEEE journals avoid them anyway)
% and instead reference/explain all the subfigures within the main caption.
% The latest version of endfloat.sty and its documentation can obtained at:
% http://www.ctan.org/pkg/endfloat
%
% The IEEEtran \ifCLASSOPTIONcaptionsoff conditional can also be used
% later in the document, say, to conditionally put the References on a 
% page by themselves.




% *** PDF, URL AND HYPERLINK PACKAGES ***
%
%\usepackage{url}
% url.sty was written by Donald Arseneau. It provides better support for
% handling and breaking URLs. url.sty is already installed on most LaTeX
% systems. The latest version and documentation can be obtained at:
% http://www.ctan.org/pkg/url
% Basically, \url{my_url_here}.





% *** Do not adjust lengths that control margins, column widths, etc. ***
% *** Do not use packages that alter fonts (such as pslatex).         ***
% There should be no need to do such things with IEEEtran.cls V1.6 and later.
% (Unless specifically asked to do so by the journal or conference you plan
% to submit to, of course. )


% correct bad hyphenation here
\hyphenation{op-tical net-works semi-conduc-tor}

\usepackage[utf8]{inputenc} % allow utf-8 input
\usepackage[T1]{fontenc}    % use 8-bit T1 fonts
\usepackage{hyperref}       % hyperlinks
\usepackage{url}            % simple URL typesetting
\usepackage{booktabs}       % professional-quality tables
\usepackage{amsfonts}       % blackboard math symbols
\usepackage{nicefrac}       % compact symbols for 1/2, etc.
\usepackage{microtype}      % microtypography
\usepackage{xcolor}         % colors
\usepackage{amsmath}
\usepackage{multirow}
\usepackage{float}
\usepackage{colortbl}
\usepackage{graphicx} 
\usepackage{subcaption}
\def\etal{\emph{et al.}}
\usepackage{graphicx}
\usepackage{arydshln}
\usepackage{caption}
\usepackage{tabularx}
\usepackage{adjustbox}
\usepackage{algorithmic}
\usepackage{mathtools}
\usepackage[linesnumbered,boxed,ruled,commentsnumbered]{algorithm2e}
\newcommand{\LCS}[1]{\textcolor{orange}{[LCS: #1]}}
\begin{document}
\renewcommand\arraystretch{1.5}
%
% paper title
% Titles are generally capitalized except for words such as a, an, and, as,
% at, but, by, for, in, nor, of, on, or, the, to and up, which are usually
% not capitalized unless they are the first or last word of the title.
% Linebreaks \\ can be used within to get better formatting as desired.
% Do not put math or special symbols in the title.
\title{\title{Supplementary Materials of  \\ \emph{Robust Disentangled Counterfactual Learning for Physical Audiovisual Commonsense Reasoning}}}
%
%
% author names and IEEE memberships
% note positions of commas and nonbreaking spaces ( ~ ) LaTeX will not break
% a structure at a ~ so this keeps an author's name from being broken across
% two lines.
% use \thanks{} to gain access to the first footnote area
% a separate \thanks must be used for each paragraph as LaTeX2e's \thanks
% was not built to handle multiple paragraphs
%
%
%\IEEEcompsocitemizethanks is a special \thanks that produces the bulleted
% lists the Computer Society journals use for "first footnote" author
% affiliations. Use \IEEEcompsocthanksitem which works much like \item
% for each affiliation group. When not in compsoc mode,
% \IEEEcompsocitemizethanks becomes like \thanks and
% \IEEEcompsocthanksitem becomes a line break with idention. This
% facilitates dual compilation, although admittedly the differences in the
% desired content of \author between the different types of papers makes a
% one-size-fits-all approach a daunting prospect. For instance, compsoc 
% journal papers have the author affiliations above the "Manuscript
% received ..."  text while in non-compsoc journals this is reversed. Sigh.

\author{
        Mengshi Qi,~\IEEEmembership{Member,~IEEE},
        Changsheng Lv,
        Huadong Ma,~\IEEEmembership{Fellow,~IEEE}
\thanks{This work is partly supported by the Funds for the NSFC Project under Grant 62202063, Beijing Natural Science Foundation (L243027). (\emph{Corresponding author: Mengshi Qi~(email:~qms@bupt.edu.cn)})}
\thanks{M. Qi, C. Lv, and H. Ma are with the State Key Laboratory of Networking and Switching Technology, Beijing University of Posts and Telecommunications, China.}
}
% note the % following the last \IEEEmembership and also \thanks - 
% these prevent an unwanted space from occurring between the last author name
% and the end of the author line. i.e., if you had this:
% 
% \author{....lastname \thanks{...} \thanks{...} }
%                     ^------------^------------^----Do not want these spaces!
%
% a space would be appended to the last name and could cause every name on that
% line to be shifted left slightly. This is one of those "LaTeX things". For
% instance, "\textbf{A} \textbf{B}" will typeset as "A B" not "AB". To get
% "AB" then you have to do: "\textbf{A}\textbf{B}"
% \thanks is no different in this regard, so shield the last } of each \thanks
% that ends a line with a % and do not let a space in before the next \thanks.
% Spaces after \IEEEmembership other than the last one are OK (and needed) as
% you are supposed to have spaces between the names. For what it is worth,
% this is a minor point as most people would not even notice if the said evil
% space somehow managed to creep in.



% The paper headers
\markboth{Transactions on Pattern Analysis and Machine Intelligence}%
{Shell \MakeLowercase{\textit{et al.}}: Bare Demo of IEEEtran.cls for Computer Society Journals}
% The only time the second header will appear is for the odd numbered pages
% after the title page when using the twoside option.
% 
% *** Note that you probably will NOT want to include the author's ***
% *** name in the headers of peer review papers.                   ***
% You can use \ifCLASSOPTIONpeerreview for conditional compilation here if
% you desire.



% The publisher's ID mark at the bottom of the page is less important with
% Computer Society journal papers as those publications place the marks
% outside of the main text columns and, therefore, unlike regular IEEE
% journals, the available text space is not reduced by their presence.
% If you want to put a publisher's ID mark on the page you can do it like
% this:
%\IEEEpubid{0000--0000/00\$00.00~\copyright~2015 IEEE}
% or like this to get the Computer Society new two part style.
%\IEEEpubid{\makebox[\columnwidth]{\hfill 0000--0000/00/\$00.00~\copyright~2015 IEEE}%
%\hspace{\columnsep}\makebox[\columnwidth]{Published by the IEEE Computer Society\hfill}}
% Remember, if you use this you must call \IEEEpubidadjcol in the second
% column for its text to clear the IEEEpubid mark (Computer Society jorunal
% papers don't need this extra clearance.)



% use for special paper notices
%\IEEEspecialpapernotice{(Invited Paper)}



% for Computer Society papers, we must declare the abstract and index terms
% PRIOR to the title within the \IEEEtitleabstractindextext IEEEtran
% command as these need to go into the title area created by \maketitle.
% As a general rule, do not put math, special symbols or citations
% in the abstract or keywords.
\IEEEtitleabstractindextext{%
% \begin{abstract}
% In this paper, we propose a Disentangled Counterfactual Learning~(DCL) approach for physical audiovisual commonsense reasoning. The task aims to infer objects' physics commonsense based on both video and audio input, with the main challenge is how to imitate the reasoning ability of humans. Most of the current methods fail to take full advantage of different characteristics in multi-modal data, and lacking causal reasoning ability in models impedes the progress of implicit physical knowledge inferring. 
% To address these issues, our proposed DCL method decouples videos into static (time-invariant) and dynamic (time-varying) factors in the latent space by the disentangled sequential encoder, which adopts a variational autoencoder (VAE) to maximize the mutual information with a contrastive loss function. Furthermore, we introduce a counterfactual learning module to augment the model's reasoning ability by modeling physical knowledge relationships among different objects under counterfactual intervention. Our proposed method is a plug-and-play module that can be incorporated into any baseline. In experiments, we show that our proposed method improves baseline methods and achieves state-of-the-art performance. 
% \end{abstract}

% Note that keywords are not normally used for peerreview papers.
\begin{IEEEkeywords}
Physical Commonsense Reasoning, Robust Multimodal Learning, Disentangled Representation, Counterfactual Analysis.
\end{IEEEkeywords}}


% make the title area
\maketitle


% To allow for easy dual compilation without having to reenter the
% abstract/keywords data, the \IEEEtitleabstractindextext text will
% not be used in maketitle, but will appear (i.e., to be "transported")
% here as \IEEEdisplaynontitleabstractindextext when the compsoc 
% or transmag modes are not selected <OR> if conference mode is selected 
% - because all conference papers position the abstract like regular
% papers do.
\IEEEdisplaynontitleabstractindextext
% \IEEEdisplaynontitleabstractindextext has no effect when using
% compsoc or transmag under a non-conference mode.



% For peer review papers, you can put extra information on the cover
% page as needed:
% \ifCLASSOPTIONpeerreview
% \begin{center} \bfseries EDICS Category: 3-BBND \end{center}
% \fi
%
% For peerreview papers, this IEEEtran command inserts a page break and
% creates the second title. It will be ignored for other modes.
\IEEEpeerreviewmaketitle


In this supplementary material, we provide a comprehensive algorithm underlying our proposed model, encompassing both the DCL and RDCL in Section~\ref{Sec. Algorithm of DCL and RDCL}. Section~\ref{Sec: Drivations and more experimental results} includes derivations and supplementary experimental results. Additionally, Section~\ref{Sec: VLM-Assisted Reasoning Dataset} presents more samples and statistical analyses of the VLM-Assisted Reasoning Dataset introduced in our main paper.
% Computer Society journal (but not conference!) papers do something unusual
% with the very first section heading (almost always called "Introduction").
% They place it ABOVE the main text! IEEEtran.cls does not automatically do
% this for you, but you can achieve this effect with the provided
% \IEEEraisesectionheading{} command. Note the need to keep any \label that
% is to refer to the section immediately after \section in the above as
% \IEEEraisesectionheading puts \section within a raised box.




% The very first letter is a 2 line initial drop letter followed
% by the rest of the first word in caps (small caps for compsoc).
% 
% form to use if the first word consists of a single letter:
% \IEEEPARstart{A}{demo} file is ....
% 
% form to use if you need the single drop letter followed by
% normal text (unknown if ever used by the IEEE):
% \IEEEPARstart{A}{}demo file is ....
% 
% Some journals put the first two words in caps:
% \IEEEPARstart{T}{his demo} file is ....
% 
% Here we have the typical use of a "T" for an initial drop letter
% and "HIS" in caps to complete the first word.
\section{Algorithm of DCL and RDCL}
In this section, we introduce the detailed algorithms for Disentangled Counterfactual
Learning (DCL) in Section~\ref{Sec. Algorithm of DCL} and Robust Disentangled Counterfactual
Learning (RDCL) in Section~\ref{Sec. Algorithm of RDCL} for Physical Commonsense Reasoning.  
\label{Sec. Algorithm of DCL and RDCL}
\subsection{DCL}
\label{Sec. Algorithm of DCL}
The overall framework of the proposed DCL algorithm is outlined in Algorithm~\ref{algorithm: DCL}. The model takes as input a training batch consisting of paired video-audio data along with associated physical knowledge questions. It outputs the final prediction, denoted as $\hat{Y}_{TIE}$.  

\subsection{RDCL}
\label{Sec. Algorithm of RDCL}
Unlike DCL, which processes complete multimodal inputs, RDCL is designed to handle incomplete modalities. As an illustrative example, we consider scenarios where audio data are missing. The corresponding algorithm is presented in Algorithm~\ref{algorithm: RDCL}.
\begin{algorithm}[ht]
\caption{Disentangled Counterfactual Learning~(DCL) Batch-Wise Training}
\label{algorithm: DCL}
\SetAlgoLined
\KwIn{
    Training batch $\{ \langle v_1, v_2 \rangle_i, \langle a_1, a_2 \rangle_i, q_i \}_{i=1}^{B}$, \\
    Batch size $B$, \\
    Pretrained image encoder $\mathcal{E}_{\text{img}}(\theta)$, \\
    Pretrained audio encoder $\mathcal{E}_{\text{aud}}(\theta)$, \\
    Pretrained text encoder $\mathcal{E}_{\text{text}}(\theta)$, \\
    Labels $\{Y_{GT,i}\}_{i=1}^B$, \\
    Number of frames $T$
}
\KwOut{
    Predicted answers $\{\hat{Y}_{TIE,i}\}_{i=1}^B$
}


Encode features: \\
  \For{$j \in \{1, 2\}$}{ 
   \quad $X^{v_j} = \{X_1^{v_j}, X_2^{v_j}, \cdots, X_T^{v_j}\} \gets \mathcal{E}_{\text{img}}(v_j)$ \\ 
   \quad $X^{a_j} \gets \mathcal{E}_{\text{aud}}(a_j)$
   }
   $X^t \gets \mathcal{E}_{\text{text}}(q)$ \\

\For{each sample in the batch}{
    Disentangle static factors $X^v_s$ and dynamic factors $X^v_z$ from $X^v$ via DSE in Section 4.2. \\
}

Compute the adjacency matrix $A_X$ using Eq. (15), (16), and (17). \\

Obtain the fused feature $F_1$, $F_2$ using Eq.(14). \\

Construct intervened features $X^*$ using Eq. (20), and compute the intervened adjacency matrix ${A}^*$ using Eqs. (15), (16) and (17). \\

Predict the $\hat{Y}_{X, A_X}$ and $\hat{Y}_{X^*, A_{X^*}}$ using Eq.(18)\\

Use $\hat{Y}_{TIE}$ obtained from Eq.(19) as the output.
\end{algorithm}

\section{Drivations and more experimental results}
\label{Sec: Drivations and more experimental results}
\subsection{Approximate Estimation of the Objective Function}

In Section 4.2 Disentangled Sequential Encoder of our main paper, our goal is to maximize the log-likelihood of \( x_{1:T} \). However, due to the computational complexity associated with high-dimensional integrals, directly obtaining \( \log p(x_{1:T}) \) is challenging. To address this issue, we employ the Evidence Lower Bound (ELBO) as an approximation to the log-likelihood. 

{\bf For the input sequence $x_{1:T}$}, Eq.(~\ref{eq:elbo_derivation}) in Figure~\ref{Seq-ELBO} shown adapted from the standard VAE framework \cite{bai2021contrastively}, noticing that either the prior or the approximate posterior factorizes over $s$ and $z_{1:T}$.
{\bf For the entire dataset}, let \( p_D \) represent the empirical data distribution, which assigns a probability mass of \( 1/N \) to each of the \( N \) training sequences in \( D \). The aggregated posteriors are defined as shown in Eq.(~\ref{Eq.2}),  Eq.(~\ref{Eq.3}), and  Eq.(~\ref{Eq.4}) in Figure~\ref{proof_3}. By rearranging terms and applying similar operations to \( x \), we arrive at Eq.~(\ref {Eq.6}) and Eq.~(\ref{Eq.7}) in Figure~\ref{proof_3}. Finally, integrating the above derivations, we obtain the dataset ELBO by subtracting a distinct KL divergence from the data log-likelihood, as illustrated in Eq.~\ref{eq:elbo} in Figure~\ref{proof-4}.
\begin{figure*}
\begin{equation}
\begin{aligned}
& \log p(x_{1:T}) \\
\ge& -KL[q(s,z_{1:T}|x_{1:T})||p(s,z_{1:T}|x_{1:T})]+\log p(x_{1:T})\\
=&\mathbb E_{q(s, z_{1:T}|x_{1:T})} \left[ \log p(s,z_{1:T}|x_{1:T}) - \log q(s,z_{1:T}|x_{1:T}) + \log p(x_{1:T}) \right] \\
=&\mathbb E_{q(s,z_{1:T}|x_{1:T})}[\log p(x_{1:T}|s, z_{1:T})-\log q(s,z_{1:T}|x_{1:T})+\log p(s,z_{1:T})]\\
=&\mathbb E_{q(s,z_{1:T}|x_{1:T})}[\log p(x_{1:T}|s, z_{1:T})-\log q(s|x_{1:T}) - \log p(z_{1:T}|x_{1:T})+\log p(s) + \log p(z_{1:T})]\\
=&\mathbb E_{q(z_{1:T}, s|x_{1:T})} \left[ \underbrace{\log p(x_{1:T}|s, z_{1:T})}_{\text{Reconstruction term}}-\underbrace{KL[q(s|x_{1:T})||p(s)]}_{s\text{-regression}}-\underbrace{KL[q(z_{1:T}|x_{1:T})||p(z_{1:T})]}_{z\text{-regression}} \right].
\end{aligned}  
\label{eq:elbo_derivation}
\end{equation}
\caption{The ELBO derivation for the input sequence \( x_{1:T} \).}
\label{Seq-ELBO}
\end{figure*}

\begin{figure*}
\centering
\begin{align}
q(s) & = \mathbb E_{x_{1:T}\sim p_D} [q(s|x_{1:T})] = \frac{1}{N} \sum_{x_{1:T}\in D} q(s|x_{1:T}), \label{Eq.2} \\
q(z_{1:T}) & = \mathbb E_{x_{1:T}\sim p_D} [q(z_{1:T}|x_{1:T})] = \frac{1}{N} \sum_{x_{1:T}\in D} q(z_{1:T}|x_{1:T}), \label{Eq.3} \\
q(s,z_{1:T}) & = \mathbb E_{x_{1:T}\sim p_D} [q(s|x_{1:T}) q(z_{1:T}|x_{1:T})] = \frac{1}{N} \sum_{x_{1:T}\in D} q(s|x_{1:T}) q(z_{1:T}|x_{1:T}). \label{Eq.4}
\end{align}
\begin{equation}
\begin{aligned}
& \mathbb E_{x_{1:T}\sim p_D}[KL[q(s|x_{1:T})||p(s)]] \\
=& \mathbb E_{x_{1:T}\sim p_D}\mathbb E_{q(s|x_{1:T})}[\log q(s|x_{1:T}) - \log q(s) + \log q(s) - \log p(s)] \\
=& \mathbb E_{q(x_{1:T}, s)} \log \left[ \frac{q(s|x_{1:T})}{q(s)} \right] + \mathbb E_{q(x_{1:T}, s)} [\log q(s)-\log p(s)] \\
=& I_q (x_{1:T};s) + KL [q(s)||p(s)]. 
\end{aligned}
\label{}
\end{equation}
\begin{equation}
KL [q(s)||p(s)] = \mathbb E_{x_{1:T}\sim p_D}[KL[q(s|x_{1:T})||p(s)]] - I_q(x_{1:T};s). 
\label{Eq.6}
\end{equation}
\begin{gather}
KL[q(z_{1:T})||p(z_{1:T})] = \mathbb E_{x_{1:T}\sim p_D}[KL[q(z_{1:T}|x_{1:T})||p(z_{1:T})]] - I_q(x_{1:T};z_{1:T}). 
\label{Eq.7}
\end{gather}
\caption{Aggregated equations and their relationships.}
\label{proof_3}
\end{figure*}

\begin{figure*}
\begin{equation}
\begin{aligned} 
&\frac{1}{N} \sum_{x_{1:T}\in D} \log p(x_{1:T}) = \mathbb E_{x_{1:T}\sim p_D}[\log p(x_{1:T})] \\
\ge& \mathbb E_{x_{1:T}\sim p_D}[\log p(x_{1:T}) - KL[q(s, z_{1:T})||p(s, z_{1:T}|x_{1:T})]] \\
=& \mathbb E_{x_{1:T}\sim p_D}[\mathbb E_{q(s, z_{1:T}|x_{1:T})}[\log p(x_{1:T}) - \log q(s, z_{1:T}) + \log p(s, z_{1:T}|x_{1:T})]] \\
=& \mathbb E_{x_{1:T}\sim p_D}[\mathbb E_{q(s, z_{1:T}|x_{1:T})}[\log p(x_{1:T}) - \log q(s, z_{1:T}) + \log p(x_{1:T}|s, z_{1:T}) + \log p(s, z_{1:T}) - \log p(x_{1:T})]] \\
=& \mathbb E_{x_{1:T}\sim p_D}[\mathbb E_{q(s, z_{1:T}|x_{1:T})}[\log p(x_{1:T}|s, z_{1:T}) - \log q(s, z_{1:T}) + \log p(s, z_{1:T})]] \\
=& \mathbb E_{x_{1:T}\sim p_D}[\mathbb E_{q(s, z_{1:T}|x_{1:T})}[\log p(x_{1:T}|s, z_{1:T})]] - \mathbb E_{x_{1:T}\sim p_D}[\mathbb E_{q(s, z_{1:T}|x_{1:T})}[\log q(s, z_{1:T}) - \log p(s, z_{1:T})]] \\
=& \mathbb E_{x_{1:T}\sim p_D}[\mathbb E_{q(s, z_{1:T}|x_{1:T})}[\log p(x_{1:T}|s, z_{1:T})]] - KL[q(s, z_{1:T})||p(s, z_{1:T})] \\
=& \mathbb E_{x_{1:T}\sim p_D}[\mathbb E_{q(s, z_{1:T}|x_{1:T})}[\log p(x_{1:T}|s, z_{1:T})]] - I_q(s;z_{1:T}) - KL[q(s)||p(s)] - KL[q(z_{1:T})||p(z_{1:T})] \\
=& \mathbb E_{x_{1:T}\sim p_D}[\mathbb E_{q(s, z_{1:T}|x_{1:T})}[\log p(x_{1:T}|s, z_{1:T})]] \\
&\hspace{5em} - \mathbb E_{x_{1:T}\sim p_D}[KL[q(s|x_{1:T})||p(s)]] - \mathbb E_{x_{1:T}\sim p_D}[KL[q(z_{1:T}|x_{1:T})||p(z_{1:T})]] \\
&\hspace{5em} + I_q(s;x_{1:T}) + I_q(z_{1:T};x_{1:T}) - I_q(s;z_{1:T}).
\end{aligned}
\label{eq:elbo}
\end{equation}
\caption{Derivation of the ELBO for a dataset by subtracting a KL-divergence term from the data log-likelihood.}
\label{proof-4}
\end{figure*}

\subsection{Sensitivity Analysis of Parameters}  
We conducted a sensitivity analysis on the parameters $\gamma$ and $\theta$ as defined in Eq. (6) of the main paper, with results presented in Figure~\ref{fig: hyperparameters}. Specifically, we evaluated $\gamma$ over the range $\{0.01, 0.1, 1, 10\}$ and $\theta$ over the range $\{0.5, 5, 50, 500\}$. The results denote that our proposed DCL method exhibits strong robustness to variations in both $\gamma$ and $\theta$, achieving consistent and stable performance across all tested parameter configurations.
\subsection{Analysis of dynamic factors}
Our DSE+ method separates video features into static (time-invariant) and dynamic (time-varying) factors. Figure~\ref{fig: t-sne} shows t-SNE visualizations of these factors alongside raw video features. While raw features appear scattered, dynamic factors extracted by DSE+ exhibit clear clustering, as highlighted by red circles. For example, Figure~\ref{fig: t-sne}(b) shows objects with similar dynamic characteristics, such as small size and lightweight, positioned adjacently. The upper portion of Figure~\ref{fig: t-sne} illustrates a cluster where actions consistently depict a hand grasping and striking the object, reflecting their lightweight characteristics. In contrast to raw features, DSE+ successfully captures this dynamic information. Similarly, the lower section highlights another cluster with shared thickness-related properties, further demonstrating DSE+’s ability to extract dynamic physical characteristics.  
\subsection{Additional Qualitative Results}
As shown in Figure~\ref{fig: Qualitative Results_1}, we present more visualization results comparing our proposed method with other baseline models. It can be seen from the figures that our proposed DCL method outperforms the original process.
\subsection{Impact of visual bias}
As illustrated in Figure~\ref{fig: visual bias}, we show the absolute accuracy differences for specific object pairs. Accuracy for pairs in the lowest 25\% of occurrence frequency improves notably after applying DCL, demonstrating its effectiveness in reducing visual bias for less frequent pairs. However, for some high-frequency pairs ({\it e.g.}, ``paper-foam'' and ``paper-textiles''), a slight accuracy decrease occurs after DCL. This is because dominant visual bias previously led to correct but unreliable predictions, while DCL mitigates this bias, revealing the model’s robust performance.  

\section{VLM-Assisted Reasoning Dataset}
\label{Sec: VLM-Assisted Reasoning Dataset}
\subsection{More examples of VLM-Assisted Reasoning Dataset}
Figure \ref{fig: VLM} illustrates the prompts used for the Vision-Language Model (VLM), with subfigures (a)–(f) showcasing its assisted reasoning outputs across various samples. A failure case is evident in Figure~\ref{fig: VLM}(f), where the VLM excessively emphasizes object-specific details ({\it e.g.}, identifying the type of wine) while overlooking the physical characteristics of the glass bottle. Future work will focus on developing more targeted prompting strategies to address such limitations.
\subsection{Dataset Statistics}
We obtained corresponding VLM descriptions for each object in the PACS dataset, resulting in 1,526 descriptions. The average length of these descriptions is 74.05 words, with a maximum length of 118 words and a minimum length of 41 words. The corresponding word cloud is illustrated in Figure~\ref{fig: Word Cloud} and the Top-50 Material Types in the Object Pair are shown in Figure~\ref{fig: Frequency of Material Types}. The generated data is available at https://github.com/MICLAB-BUPT/DCL.
\begin{figure}[ht]
    \centering
    \includegraphics[width=0.9\linewidth]{Figures/Fig_11.pdf}
    \caption{Absolute differences in accuracy scores between two configurations: AudioCLIP with DCL using solely video input (V w/ DCL) and AudioCLIP utilizing only video input (V). The parenthetical value indicates the frequency of occurrence, measured at 11.5 instances within the final 25\% of the training dataset.}
    \label{fig: visual bias}
\end{figure}
\begin{figure}[ht]
    \centering
    \includegraphics[width=0.9\linewidth]{Figures/T-SNE.pdf}
    \caption{T-SNE visualization of video features before applying DSE+ (a), along with dynamic factors (b) and static factors (c) obtained after DSE+. The \textcolor{red}{red circles} indicate clusters that have been manually identified as containing samples with similar physical properties. We provide examples of these clusters based on shared attributes, including weight and thickness.}
    \label{fig: t-sne}
\end{figure}
\begin{figure}[b]
    \centering
    \includegraphics[width=1.0\linewidth]{Figures/Fig_12.pdf}
   \caption{Word Cloud for VLM-Assisted Reasoning}
    \label{fig: Word Cloud}
\end{figure}
\begin{algorithm}[ht]
\caption{Robust Disentangled Counterfactual Learning (RDCL) Batch-Wise Training}
\label{algorithm: RDCL}
\SetAlgoLined
\KwIn{
    Training batch $\{ \langle v_1, v_2 \rangle_i, \langle a_1, a_2 \rangle_i, q_i \}_{i=1}^{B}$, \\
    Batch size $B$, \\
    Pretrained image encoder $\mathcal{E}_{\text{img}}(\theta)$, \\
    Pretrained audio encoder $\mathcal{E}_{\text{aud}}(\theta)$, \\
    Pretrained text encoder $\mathcal{E}_{\text{text}}(\theta)$, \\
    Labels $\{Y_{GT,i}\}_{i=1}^B$, \\
    Number of frames $T$, proportion of missing data in object 1's video $a_{v1}.$ 
}
\KwOut{
    Predicted answers $\{\hat{Y}_{TIE,i}\}_{i=1}^B$
}

\textbf{Encode features:} \\
\For{$j \in \{1, 2\}$}{ 
   \quad $X^{v_j} = \{X_1^{v_j}, X_2^{v_j}, \cdots, X_T^{v_j}\} \gets \mathcal{E}_{\text{img}}(v_j)$ \\ 
   \quad $X^{a_j} \gets \mathcal{E}_{\text{aud}}(a_j)$
}
$X^t \gets \mathcal{E}_{\text{text}}(q)$ \\

Obtain the set of missing data set $B_{{miss}}$ and the complete data set $B_{{com}}$ using Eq.(27). \\

For each sample $i$ in the batch:

\If{$i \in B_{com}$}{
    \For{each sample in the $B_{{com}}$}{
        Disentangle static factors $X^v_s$ and dynamic factors $X^v_z$ from $X^v$ via DSE in Section 4.2. \\
    }
    
    Use unique encoder and shared encoder to encode $X^v_s$, $X^v_z$, and $X^a$ using Eqs.(24) and (25), obtaining $r_m^{unique}$ and $r_m^{share}, m \in \{a,z,s\}$.
}
\Else{
    Use Eqs. (29) and (30) to complete the missing information.
}

Compute the adjacency matrix $A_X$ using Eqs. (15), (16), and (17). \\

Obtain the fused features $F_1$ and $F_2$ using Eq.(14). \\

Construct the intervened features $X^*$ using Eq. (20), and compute the intervened adjacency matrix ${A}^*$ using Eqs. (15), (16), and (17). \\

Predict $\hat{Y}_{X, A_X}$ and $\hat{Y}_{X^*, A_{X^*}}$ using Eq.(18). \\

Use $\hat{Y}_{TIE}$ obtained from Eq.(19) as the output.
\end{algorithm}

\begin{figure}[ht]
    \centering
    \includegraphics[width=1.0\linewidth]{Figures/Fig_13.pdf}
    \caption{Frequency of Material Types for Object Pairs in the PACS Training Set.}
    \label{fig: Frequency of Material Types}
\end{figure}
\begin{figure*}[ht]
    \centering
    \includegraphics[width=1.0\linewidth]{Figures/Fig_7.pdf}
    \caption{Performance comparisons of various hyperparameters in Eq. (6) are presented. Figures (a) and (b) display the performance of AudioCLIP with different values of $\gamma$ in $\mathcal{L}_{DSE}$ on the PACS and PACS-Material datasets. Figures (c) and (d) show the performance of AudioCLIP with varying $\theta$ in $\mathcal{L}_{DSE}$ on the same datasets.}
    \label{fig: hyperparameters}
\end{figure*}
\begin{figure*}
    \centering
    \includegraphics[width=0.9\linewidth]{Figures/Fig_8.pdf}
    \caption{Qualitative Results of baseline w/ and w/o our proposed method, where “Material” refers to the material of the object. The correct answers are depicted in green while the incorrect ones are depicted in red.}
    \label{fig: Qualitative Results_1}
\end{figure*}
\begin{figure*}
    \centering
    \includegraphics[width=0.9\linewidth]{Figures/Fig_9.pdf}
    \caption{Qualitative Results of baseline w/ and w/o our proposed method, where “Material” refers to the material of the object. The correct answers are depicted in green while the incorrect ones are depicted in red.}
    \label{fig: Qualitative Results_2}
\end{figure*}
\begin{figure*}
    \centering
    \includegraphics[width=0.75\linewidth]{Figures/Fig_10.pdf}
    \caption{Prompt Text, Input Image, and Corresponding Response of the VLM ({\it i.e.}, Doubao-1.5-Vision-Pro)}
    \label{fig: VLM}
\end{figure*}


\ifCLASSOPTIONcaptionsoff
  \newpage
\fi


\bibliographystyle{IEEEtran}
\bibliography{reference}
\end{document}




\end{document}
