% This is a modified version of Springer's LNCS template suitable for anonymized MICCAI 2025 main conference submissions. 
% Original file: samplepaper.tex, a sample chapter demonstrating the LLNCS macro package for Springer Computer Science proceedings; Version 2.21 of 2022/01/12

\documentclass[runningheads]{llncs}
%
\usepackage[T1]{fontenc}
% T1 fonts will be used to generate the final print and online PDFs,
% so please use T1 fonts in your manuscript whenever possible.
% Other font encodings may result in incorrect characters.
%
\usepackage{graphicx,verbatim}
% Used for displaying a sample figure. If possible, figure files should
% be included in EPS format.
%
% If you use the hyperref package, please uncomment the following two lines
% to display URLs in blue roman font according to Springer's eBook style:
%\usepackage{color}
%\renewcommand\UrlFont{\color{blue}\rmfamily}
%\urlstyle{rm}
%
% Basic packages
\usepackage{amsmath}
\usepackage{amssymb}
\usepackage{amsfonts}
\usepackage{amscd}
\usepackage{graphicx}
\usepackage{bm}
\usepackage{color}
\usepackage{xcolor}
\usepackage{authblk}
\usepackage{booktabs}
\usepackage{multirow}
\usepackage{float}
\usepackage{tikz}
\usetikzlibrary{positioning}
% \usepackage{ctable}
\usepackage{url}
\usepackage{pifont}
\usepackage{listings}
\usepackage{breakcites}
\usepackage{orcidlink}
\usepackage{setspace}
\usepackage{latexsym}
\usepackage{caption}
\usepackage{makecell}
\usepackage{array}
\usepackage{wrapfig}
\usepackage{diagbox}
\usepackage{algorithm}
\usepackage{algpseudocode}
\usepackage{textcomp}
\usepackage[T1]{fontenc}
\usepackage{upquote}
\usepackage{dashrule}
\usepackage{subfig}
\usepackage{tabularx}
\usepackage[toc, page]{appendix}
\usepackage[noabbrev,capitalise]{cleveref}
\usepackage{seqsplit}
\usepackage{url} 
\usepackage{lipsum}
 \usepackage{hyperref}

% TikZ libraries
\usetikzlibrary{arrows.meta,arrows,fit}

% Color definitions
\definecolor{Gray}{gray}{0.9}

% Custom commands
\newcommand\tstrut{\rule{0pt}{2.4ex}}
\newcommand\bstrut{\rule[-1.0ex]{0pt}{0pt}}
\newcommand{\pt}[1]{\textcolor{blue}{[\textbf{PT:} #1]}}
\newcommand{\jy}[1]{\textcolor{orange}{[\textbf{JY:} #1]}}

\definecolor{citecolor}{HTML}{0071bc}

\usepackage[toc, page]{appendix}
\usepackage{tabularx}
\usepackage{booktabs}

\newlength\savewidth\newcommand\shline{\noalign{\global\savewidth\arrayrulewidth
  \global\arrayrulewidth 1pt}\hline\noalign{\global\arrayrulewidth\savewidth}}
  
\def\method{\text MixMin~}
\def\methodnospace{\text MixMin}
\def\genmethod{$\mathbb{R}$\text Min~}
\def\genmethodnospace{ $\mathbb{R}$\text Min}

\begin{document}


\title{MedVLM-R1: Incentivizing Medical Reasoning Capability of Vision-Language Models (VLMs) via Reinforcement Learning}

\authorrunning{Pan, Liu et al.}
\titlerunning{MedVLM-R1}
%



\author{Jiazhen Pan\inst{1,2}$^{*}$, Che Liu\inst{3}$^{*}$, Junde Wu\inst{2}, Fenglin Liu\inst{2}, Jiayuan Zhu\inst{2}, Hongwei Bran Li\inst{4}, Chen Chen\inst{5,6}, Cheng Ouyang\inst{2,6}$^{\dagger}$, Daniel Rueckert\inst{1,6}$^{\dagger}$}
\institute{Chair for AI in Healthcare and Medicine, Technical University of Munich (TUM) and TUM University Hospital, Germany 
\and Department of Engineering Science, University of Oxford, UK
\and Data Science Institute, Imperial College London, UK
\and Massachusetts General Hospital, Harvard Medical School, USA
\and School of Computer Science, University of Sheffield, UK
\and Department of Computing, Imperial College London, UK
\\
\email{jiazhen.pan@tum.de, che.liu21@imperial.ac.uk}
}
 
\maketitle              % typeset the header of the contribution
%
%
\let\thefootnote\relax\footnotetext{$^{*}$ Equal contribution}
\let\thefootnote\relax\footnotetext{$^{\dagger}$ Equal advice}

\begin{abstract} 

Reasoning is a critical frontier for advancing medical image analysis, where transparency and trustworthiness play a central role in both clinician trust and regulatory approval. Although Medical Visual Language Models (VLMs) show promise for radiological tasks, most existing VLMs merely produce final answers without revealing the underlying reasoning. To address this gap, we introduce MedVLM-R1, a medical VLM that explicitly generates natural language reasoning to enhance transparency and trustworthiness. Instead of relying on supervised fine-tuning (SFT), which often suffers from overfitting to training distributions and fails to foster genuine reasoning, MedVLM-R1 employs a reinforcement learning framework that incentivizes the model to discover human-interpretable reasoning paths without using any reasoning references. Despite limited training data (600 visual question answering samples) and model parameters (2B), MedVLM-R1 boosts accuracy from 55.11\% to 78.22\% across MRI, CT, and X-ray benchmarks, outperforming larger models trained on over a million samples. It also demonstrates robust domain generalization under out-of-distribution tasks. By unifying medical image analysis with explicit reasoning, MedVLM-R1 marks a pivotal step toward trustworthy and interpretable AI in clinical practice. Inference model is available at: \url{https://huggingface.co/JZPeterPan/MedVLM-R1}.

% \keywords{Medical reasoning  \and Reinforcement learning \and VLMs}

\end{abstract}





\section{Introduction}

Radiological images are fundamental to modern healthcare, with over 8 billion scans performed annually \cite{akhter2023ai}. As diagnostic demand grows, the demand for efficient AI-driven interpretation becomes increasingly acute. Medical Vision-Language Models (VLMs), developed for radiological visual question answering (VQA) in MRI, CT and X-ray images, offer substantial promise in assisting clinicians/patients. Recent advances in general-purpose LLMs/VLMs (e.g., GPT-4o~\cite{hurst2024gpt}, Claude-3.7 Sonnet~\cite{claude-37}) highlight sophisticated reasoning capabilities. However, the medical domain places an especially high premium on explainable decision-making: both clinicians/patients need to understand not just \emph{what} conclusion was reached, but also \emph{why}. Existing medical VLMs often provide only final answers or “quasi-explanations” derived from pre-training pattern matching, which do not necessarily reflect genuine, step-by-step reasoning. Consequently, ensuring interpretability and trustworthiness remains an urgent challenge in real-world clinical settings.

We argue that the limited reasoning capability for existing medical VLM is primarily due to the inherent drawbacks of Supervised Fine-Tuning (SFT) \cite{achiam2023gpt,qwq-32b-preview,chen2022program} which is the most common strategy for adapting large foundation models for specialized medical tasks \cite{hartsock2024vision,lian2024less,chen2024efficiency}. Despite its simplicity, SFT faces two critical challenges: 1) An over-reliance on final-answer supervision often leads to overfitting, shortcut learning, and weaker performance on out-of-distribution (OOD) data -- an issue particularly consequential in high-stake medical scenarios \cite{chu2025sft}.
2) Direct supervision with only final answers provides minimal incentive for cultivating reasoning abilities within VLMs. A possible mitigation is distilling a more capable teacher model's chain-of-thought (CoT) reasoning for SFT \cite{wei2022chain,li2023symbolic}. However, constructing high-quality CoT data is prohibitively expensive to scale in specialized domains like healthcare. As a result, current medical VLMs that rely on SFT often fall short of delivering transparent explanations and robust generalizations when confronted with unfamiliar data.

In contrast, Reinforcement Learning (RL) \cite{schulman2017proximal} offers a compelling alternative for cultivating emergent reasoning by rewarding models for discovering their own logical steps rather than memorizing final answers or copying teacher CoT rationales. Indeed, a recent work \textit{\textbf{SFT Memorizes, RL Generalizes}} \cite{chu2025sft} confirms that RL-trained models often display superior generalization compared to their SFT counterparts. However, conventional RL pipelines typically depend on auxiliary neural reward models, requiring substantial resources to continuously update both policy and reward models \cite{ziegler2019fine,ouyang2022training}.
%—an especially challenging process in data-limited medical domains. 
A promising alternative, group relative policy optimization (GRPO) \cite{shao2024deepseekmath}, eliminates the need for neural reward models by employing a rule-based group-relative advantage strategy (see sec. \ref{sec:methods} for more details). This approach has demonstrated advanced reasoning, fostering capabilities while reducing computational demands in DeepSeek-R1 \cite{guo2025deepseek}. Despite its potential benefits for resource- and data-constrained domains like healthcare, GRPO remains largely unexplored in medical contexts.

In this work, we introduce \textbf{MedVLM-R1}, the first medical VLM capable of generating answers with explicit reasoning by training with GRPO for radiology VQA tasks. Our contributions are as follows:
\begin{enumerate}
    \item \textbf{Medical VLM with Explicit Reasoning}: We introduce MedVLM-R1, the first lightweight medical VLM capable of generating explicit reasoning alongside the final answer, rather than providing only the final answer.

    \item \textbf{Emerging Reasoning Without Explicit Supervision}: Unlike traditional SFT methods that require data with complex reasoning steps, MedVLM-R1 is trained using GRPO with datasets containing only final answers, demonstrating emergent reasoning capabilities without explicit supervision.

    \item \textbf{Superior Generalization and Efficiency}: MedVLM-R1 achieves robust generalization to out-of-distribution data (e.g. MRI → CT/X-ray) and outperforms larger models like Qwen2VL-72B and Huatuo-GPT-Vision-7B, despite being a compact 2B-parameter model trained on just 600 samples.
\end{enumerate}


\section{Related Work}

\paragraph{\textbf{Medical VLMs and Their Limitations.}} The rise of large-scale VLMs has spurred numerous domain-specific adaptations for healthcare, with systems such as LLaVA-Med \cite{li2023llava} and HuatuoGPT-Vision \cite{chen2024huatuogptvisioninjectingmedicalvisual} achieving impressive results in radiology VQA and related diagnostic tasks. Despite these advancements, using SFT on final-answer labels remains the dominant strategy for tailoring large models to medical domains \cite{zhang2023pmc,chen2024efficiency,zhang2024generalist,chaves2024towards}. This approach generally requires substantial amounts of high-quality image-text data (ranging from 660k \cite{li2023llava} to 32M samples \cite{wu2023towards}) which is costly to curate and often hampered by noise/privacy concerns. Moreover, the reliance on final-answer supervision provides limited scope for exposing a model’s intermediate reasoning—an important factor in building clinicians' trust. In addition, SFT-based models often overfit to narrow training distributions, leading to weaker generalization on OOD clinical scenarios.

\paragraph{\textbf{Reinforcement Learning for Enhanced Reasoning.}}
To mitigate SFT's limitations, RL \cite{silver2017mastering,christiano2017deep,ziegler2019fine,ouyang2022training,kumar2024training} has emerged as a compelling alternative for improving model interpretability and robustness. Classic RL methods, such as proximal policy optimization (PPO) \cite{schulman2017proximal}, have been widely adopted in text-based learning (\textit{e.g.,} policy shaping for LLMs) and can reward not only correctness but also the quality of intermediate reasoning steps. Recent studies suggest that while SFT “memorizes,” RL can help models “generalize” \cite{chu2025sft}, offering a more stable trajectory toward domain-transferable representations. Notably, GRPO \cite{shao2024deepseekmath} extends PPO by eliminating its (neural) value function estimator and focusing on a rule-based group-relative advantage for selecting actions, showing promise in resource-constrained settings like DeepSeek-R1 \cite{guo2025deepseek}. Such RL-driven frameworks could be particularly beneficial for medical tasks, where limited data availability, high-stakes decision-making, and the need for explicit reasoning converge. In the following section, we will detail how these insights motivate our approach.% to improving reasoning capabilities in radiology VQA.





\section{Study Design}
% robot: aliengo 
% We used the Unitree AlienGo quadruped robot. 
% See Appendix 1 in AlienGo Software Guide PDF
% Weight = 25kg, size (L,W,H) = (0.55, 0.35, 06) m when standing, (0.55, 0.35, 0.31) m when walking
% Handle is 0.4 m or 0.5 m. I'll need to check it to see which type it is.
We gathered input from primary stakeholders of the robot dog guide, divided into three subgroups: BVI individuals who have owned a dog guide, BVI individuals who were not dog guide owners, and sighted individuals with generally low degrees of familiarity with dog guides. While the main focus of this study was on the BVI participants, we elected to include survey responses from sighted participants given the importance of social acceptance of the robot by the general public, which could reflect upon the BVI users themselves and affect their interactions with the general population \cite{kayukawa2022perceive}. 

The need-finding processes consisted of two stages. During Stage 1, we conducted in-depth interviews with BVI participants, querying their experiences in using conventional assistive technologies and dog guides. During Stage 2, a large-scale survey was distributed to both BVI and sighted participants. 

This study was approved by the University’s Institutional Review Board (IRB), and all processes were conducted after obtaining the participants' consent.

\subsection{Stage 1: Interviews}
We recruited nine BVI participants (\textbf{Table}~\ref{tab:bvi-info}) for in-depth interviews, which lasted 45-90 minutes for current or former dog guide owners (DO) and 30-60 minutes for participants without dog guides (NDO). Group DO consisted of five participants, while Group NDO consisted of four participants.
% The interview participants were divided into two groups. Group DO (Dog guide Owner) consisted of five participants who were current or former dog guide owners and Group NDO (Non Dog guide Owner) consisted of three participants who were not dog guide owners. 
All participants were familiar with using white canes as a mobility aid. 

We recruited participants in both groups, DO and NDO, to gather data from those with substantial experience with dog guides, offering potentially more practical insights, and from those without prior experience, providing a perspective that may be less constrained and more open to novel approaches. 

We asked about the participants' overall impressions of a robot dog guide, expectations regarding its potential benefits and challenges compared to a conventional dog guide, their desired methods of giving commands and communicating with the robot dog guide, essential functionalities that the robot dog guide should offer, and their preferences for various aspects of the robot dog guide's form factors. 
For Group DO, we also included questions that asked about the participants' experiences with conventional dog guides. 

% We obtained permission to record the conversations for our records while simultaneously taking notes during the interviews. The interviews lasted 30-60 minutes for NDO participants and 45-90 minutes for DO participants. 

\subsection{Stage 2: Large-Scale Surveys} 
After gathering sufficient initial results from the interviews, we created an online survey for distributing to a larger pool of participants. The survey platform used was Qualtrics. 

\subsubsection{Survey Participants}
The survey had 100 participants divided into two primary groups. Group BVI consisted of 42 blind or visually impaired participants, and Group ST consisted of 58 sighted participants. \textbf{Table}~\ref{tab:survey-demographics} shows the demographic information of the survey participants. 

\subsubsection{Question Differentiation} 
Based on their responses to initial qualifying questions, survey participants were sorted into three subgroups: DO, NDO, and ST. Each participant was assigned one of three different versions of the survey. The surveys for BVI participants mirrored the interview categories (overall impressions, communication methods, functionalities, and form factors), but with a more quantitative approach rather than the open-ended questions used in interviews. The DO version included additional questions pertaining to their prior experience with dog guides. The ST version revolved around the participants' prior interactions with and feelings toward dog guides and dogs in general, their thoughts on a robot dog guide, and broad opinions on the aesthetic component of the robot's design. 

\section{Experiments}

\paragraph{\textbf{Dataset.}}
We conduct our experiments using the HuatuoGPT-Vision \cite{chen2024huatuogptvisioninjectingmedicalvisual} evaluation dataset, which is a processed and combined dataset from several publicly available medical VQA benchmarks, including VQA-RAD \cite{lau2018dataset}, SLAKE \cite{liu2021slake}, PathVQA \cite{he2020pathvqa}, OmniMedVQA \cite{hu2024omnimedvqa}, and PMC-VQA \cite{zhang2023pmc}. In total, the dataset comprises 17,300 multiple-choice questions linked to images covering various medical imaging modalities, with 2–6 possible choices per question. For this study, we focus on radiology modalities: CT, MRI, and X-ray. Specifically, we use 600 MRI image-question pairs for training and set aside 300 MRI, 300 CT, and 300 X-ray pairs for testing. The MRI test set is used as in-domain test, whereas the CT and X-ray test set serve as OOD test.

\paragraph{\textbf{Implementation details.}}
We adopt Qwen2-VL-2B as our base VLM. This model is originally trained on data from curated web pages, open-source datasets, and synthetic sources. To adapt it to the medical domain, we employ the GRPO reinforcement learning framework outlined in Section~\ref{sec:methods}. Our implementation builds on the public VLM reasoning repositories \cite{open-r1-multimodal,chen2025r1v,shen2025vlmr1}. We perform fine-tuning on two NVIDIA A100 SXM4 80GB for 300 steps, using a batch size of 2, which takes approximately 4 hours. Generation candidate number $G$ is set to 6. The other training optimization hyper-parameters are set as suggested by \cite{chen2025r1v}.

\paragraph{\textbf{Baseline methods and evaluation metric.}}
We compare MedVLM-R1 with the following baselines: 1. Qwen2-VL family \cite{Qwen-VL} including Qwen2-VL-2B (the unmodified base model), Qwen2-VL-7B and -72B which are the large/huge model variants. 2. HuatuoGPT-vision \cite{chen2024huatuogptvisioninjectingmedicalvisual}: A medical VLM built upon Qwen2-VL-7B. 3. SFT: The same Qwen2-VL-2B base model fine-tuned with standard SFT, using the same training setting with 600 MRI question-answer pairs. We apply negative log-likelihood as the loss function to carry out the SFT training. All baselines use a simple prompting format, \textit{e.g.,} \texttt{\{Question\} Your task: provide the correct single-letter choice (A, B, C, D, …).} In contrast, MedVLM-R1 uses the RL-based prompt as described in Section~\ref{sec:methods}, designed to elicit explicit reasoning. For evaluation, each model receives one point for the correct single-letter answer and zero otherwise. In the test of MedVLM-R1, only the correct choice enclosed in the \texttt{<answer>...</answer>} tag is scored as correct; any deviation from this format, even if semantically correct, results in a zero score. 



\section{Results and Discussion}

\paragraph{\textbf{Overall Performance.}}

\begin{figure*}[!ht]
\centering
\includegraphics[width=\textwidth]{latex/figure/fig-method.pdf}
\caption{Overview of our proposed approach PALETTE. We (1) produce adjustment queries based on the MBTI questionnaire, then (2) edit personality through relevant knowledge editing. (3) Using the edited LLM, a specific trait-focused (Thinking) response is generated.}
\label{fig:method}
\end{figure*}

Table~\ref{tab:prediction} summarizes both in-domain (ID) and out-of-domain (OOD) performance for various VLMs. Note that ID/OOD comparisons specifically refer to models fine-tuned on MRI data. Unsurprisingly, VLMs fine-tuned with both GRPO and SFT significantly outperform zero-shot general-purpose VLMs on in-domain tasks. Our GRPO-trained model shows very strong OOD performance, achieving a \textbf{16\%} improvement on CT and a \textbf{35\%} improvement on X-ray compared to SFT counterparts, underscoring GRPO's superior generalizability. Furthermore, despite being a compact 2B-parameter model trained on just 600 samples, MedVLM-R1 outperforms larger models like Qwen2-VL-72B and HuatuoGPT-Vision-7B, with the latter being specifically trained on large-scale medical data. This highlights the immense potential of RL-based training methods for efficient and scalable medical VLM development.

\paragraph{\textbf{Reasoning Competence and Interpretability.}} Beyond strong generalization, a central strength of MedVLM-R1 is its ability to produce explicit reasoning—a capability absent in all baselines. As illustrated in Figure~\ref{fig:vqa}, MedVLM-R1 presents a logical thought process within the \texttt{<think>} tag, with the final decision enclosed in the \texttt{<answer>} tag. Notably, for relatively simpler questions (problem 1 and 2), the reasoning appears cogent and aligned with medical knowledge. However, more complex queries sometimes reveal heuristic or just partial reasoning. For example, in the third sample, the model arrives at the correct answer via the \textbf{process of elimination} rather than detailed medical analysis, suggesting it leverages cue-based reasoning instead of domain expertise. Likewise, in some instances (e.g., question 4), the causal chain between reasoning and conclusion remains unclear, raising the question of whether the model merely \textbf{retrofits an explanation after predicting the correct answer}. Despite these imperfections, MedVLM-R1 represents a notable step toward interpretability in radiological decision-making.

\begin{table}[!t]
\centering
\setlength{\tabcolsep}{1mm}{}
\caption{\small {Results of VQA-VLMs on MRI (in-domain), and CT and X-Ray (out-of-domain) modalities. "$-2$B" indicates the model has $2$ billion parameters, etc.}}
\label{tab:prediction}
\scalebox{0.85}{
\begin{tabular}{l|c|ccc|c}
\toprule
\multirow{2}{*}{Method} & \multirow{2}{*}{\shortstack{Num. of Seen \\ Medical Sample}} & \multicolumn{3}{c|}{In-Domain / Out-of-Domain} & \multirow{2}{*}{Average} \\ 
\cline{3-5}
         &  & (MRI$\rightarrow$MRI) & (MRI$\rightarrow$CT) & (MRI$\rightarrow$X-ray) & \\ \midrule
Random Guess & / & $25.00$ & $30.25$ & $26.00$ & $27.08$   \\ 
\hline
\multicolumn{6}{c}{\footnotesize\textit{Zero-shot VLM}} \\
\hline
Qwen2-VL-2B & / & $61.67$ & $50.67$ & $53.00$ & $55.11$   \\ 
Qwen2-VL-7B & / & $72.33$ & $68.67$ & $66.63$ & $69.21$   \\ 
Qwen2-VL-72B & / & $68.67$ & $60.67$ & $72.33$ & $67.22$   \\ 
\hline
\multicolumn{6}{c}{\footnotesize\textit{Zero-shot Medical VLM}} \\
\hline
Huatuo-GPT-vision-7B & 1,294,062 & $71.00$ & $63.00$ & $\mathbf{73.66}$ & $69.22$   \\ 
\hline
\multicolumn{6}{c}{\footnotesize\textit{MRI fine-tuned VLM}} \\
\hline
Qwen2-VL-2B (SFT) & 600 & $94.00$ & $54.33$ & $34.00$ & $59.44$   \\ 
\textbf{Ours-2B (GRPO)} & 600 & $\mathbf{95.33}$ & $\mathbf{70.33}$ & $69.00$ & $\mathbf{78.22}$   \\
\bottomrule
\end{tabular}
}
\end{table}

\paragraph{\textbf{Limitations.}} Although MedVLM-R1 demonstrates promising results in MRI, CT, and X-ray datasets, several limitations remain: 1. Modality Gaps: When tested on other medical modalities (e.g., pathology or OCT images), the model fails to converge. We hypothesize this arises from the base model’s insufficient exposure to such modalities during pre-training. 2. Closed-Set Dependence: The current approach is tailored to multiple-choice (closed-set) VQA. In open-ended question settings where no predefined options are provided, the model’s performance degrades substantially. This is also a common challenge for many VLMs. 3. Superficial/hallucinated Reasoning: In some reasoning cases, MedVLM-R1 provides a correct answer without offering a meaningful reasoning process (e.g., \texttt{<think>To determine the correct observation from this spine MRI, let's analyze the image.</think><answer>B</answer>}). Moreover, sometimes the model concludes a correct choice while providing an inference that can lead to another answer. This phenomenon underscores that even models designed for explainability can occasionally revert to superficial/hallucinated justifications, highlighting an ongoing challenge in generating consistently transparent and logically sound rationales. Regarding all these issues, we believe the current 2B-parameter scale of our base model constitutes a potential bottleneck, and we plan to evaluate MedVLM-R1 on larger VLM backbones to address these concerns.


\section{Conclusion}

We present MedVLM-R1, a medical VLM that integrates GRPO-based reinforcement learning to bridge the gap between accuracy, interpretability, and robust performance in radiology VQA. By focusing on explicit reasoning, the model fosters transparency and trustworthiness—qualities essential in high-stakes clinical environments. Our results demonstrate that RL-based approaches generalize better than purely SFT methods, particularly under OOD settings. Although VLM-based medical reasoning is still at a nascent stage and faces considerable challenges, we believe that its potential for delivering safer, more transparent AI-driven healthcare solutions will be appreciated and should be encouraged.



\section{Acknowledgements}
This work is partially funded by the European Research Council (ERC) project Deep4MI (884622). Mr. Wu is supported by the Engineering and Physical Sciences Research Council (EPSRC) under grant EP/S024093/1 and GE HealthCare. Mr. Liu is supported by the Clarendon Fund. Ms. Zhu is supported by the Engineering and Physical Sciences Research Council (EPSRC) under grant EP/S024093/1 and Global Health R\&D of the healthcare business of Merck KGaA, Darmstadt, Germany, Ares Trading S.A. (an affiliate of Merck KGaA, Darmstadt, Germany), Eysins, Switzerland (Crossref Funder ID: 10.13039 / 100009945). Dr. Li is supported by a Postdoc Mobility Grant from SNSF. Dr. Chen is funded by Royal Society (RGS/R2/242355). Dr. Ouyang is supported by UKRI grant EP/X040186/1. 



% ---- Bibliography ----
%
\bibliographystyle{splncs04}
\bibliography{refs}
%

\end{document}
