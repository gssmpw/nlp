\begin{figure*}[!ht]
    \centering
    \begin{tikzpicture}[node distance=1cm] % Add node distance parameter
    
    % First node (original)
    \node[draw=black, 
    line width=1pt,
    rounded corners=10pt,
    text width=\textwidth,
    inner sep=5pt,
    align=left](box1){ % Changed name to box1
        \begin{wrapfigure}{r}{0.22\textwidth}
        \vspace{-2.5em}
            \includegraphics[width=0.22\textwidth]{figs/person19_virus_50.jpeg}
        \end{wrapfigure}
        \vspace{0.1cm}
        \scriptsize{\brown{Problem 1: What can be observed in this image?

        A) Lungs  B) Bladder C) Brain D) Heart
        }}
        
        \vspace{0.1cm}

        \scriptsize{\textbf{MedVLM-R1 Output:}\vspace{0.1cm}

        \blue{<think>
        
The image is a chest X-ray, which is a type of radiographic image used to visualize the internal structures of the body, particularly the lungs and bones. The presence of lung markings and the ribcage are characteristic features of a chest X-ray.

        </think>
        \vspace{0.06cm}
        
        <answer>A</answer>}
        % \noindent\rule{\dimexpr1\textwidth}{0.4pt}
        \scriptsize{\textbf{\green{Groundtruth Answer: A}}}
        % \vspace{0.2cm}
        }
    };
    
    % Second node (copy of the first)
    \node[draw=black, 
    line width=1pt,
    rounded corners=10pt,
    text width=\textwidth,
    inner sep=5pt,
    align=left,
    below=0.1cm of box1](box2){ % Changed name to box1
        \begin{wrapfigure}{r}{0.22\textwidth}
        \vspace{-2.5em}
            \includegraphics[width=0.22\textwidth]{figs/mdb146.png}
        \end{wrapfigure}
        \vspace{0.1cm}
        \scriptsize{\brown{Problem 2: What content appears in this image?
        
A) Cardiac tissue, B) Breast tissue, C) Liver tissue, D) Skin tissue

        }}
        
        \vspace{0.1cm}

        \scriptsize{\textbf{MedVLM-R1 Output:}\vspace{0.1cm}

        \blue{<think>
            The image appears to be a mammogram, which is a type of X-ray used to detect abnormalities in the breast tissue. The image shows the breast tissue with various densities and patterns, which are typical of mammograms.
            </think>



        \vspace{0.06cm}
        
       <answer>B</answer>
       }
        % \noindent\rule{\dimexpr1\textwidth}{0.4pt}
        \scriptsize{\textbf{\green{Groundtruth Answer: B}}}
        % \vspace{0.2cm}
        }
    };
        \node[draw=black, 
    line width=1pt,
    rounded corners=10pt,
    text width=\textwidth,
    inner sep=5pt,
    align=left,
    below=0.1cm of box2](box3){ % Changed name to box1
        \begin{wrapfigure}{r}{0.22\textwidth}
        \vspace{-2.5em}
            \includegraphics[width=0.22\textwidth]{figs/spine000203.png}
        \end{wrapfigure}
        \vspace{0.1cm}
        \scriptsize{\brown{Problem 3: What is the finding in this image?
        
A) Stomach ulcer, B) Nerve compression, 

C) Cord pathology, D) Liver cirrhosis


        }}
        
        \vspace{0.1cm}

        \scriptsize{\textbf{MedVLM-R1 Output:}\vspace{0.1cm}

        \blue{<think>
To determine the correct answer, we need to identify the finding in the image provided. The MRI scan in question is likely a view of a spine, not the stomach, or the liver. Stomach ulcers, liver cirrhosis, and spinal cord pathology can be diagnosed with MRI, but the image in question shows a spinal view, not a gastrointestinal or abdominal view. Therefore, the only reasonable options left are nerve compression and cord pathology.
</think>




        \vspace{0.06cm}
        
       <answer>C</answer>
       }
        % \noindent\rule{\dimexpr1\textwidth}{0.4pt}
        \scriptsize{\textbf{\green{Groundtruth Answer: C}}}
        % \vspace{0.2cm}
        }
    };
    \node[draw=black, 
    line width=1pt,
    rounded corners=10pt,
    text width=\textwidth,
    inner sep=5pt,
    align=left,
    below=0.1cm of box3](box4){ % Changed name to box1
        \begin{wrapfigure}{r}{0.22\textwidth}
        \vspace{-2.5em}
            \includegraphics[width=0.22\textwidth]{figs/abd006046.png}
        \end{wrapfigure}
        \vspace{0.1cm}
        \scriptsize{\brown{Problem 4: What is the specific abnormality present in this image?

A) Gastritis, B) Arthritis, C) Appendicitis, D) Urolithiasis



        }}
        
        \vspace{0.1cm}

        \scriptsize{\textbf{MedVLM-R1 Output:}\vspace{0.1cm}

        \blue{<think> The image presented is a transverse CT scan of the abdomen and pelvis. The presence of calculi (urines filled with stones or grit) in the pelvic organs is a consistent finding in urolithiasis. </think>






        \vspace{0.1cm}
        
       <answer>D</answer>
       }
        % \noindent\rule{\dimexpr1\textwidth}{0.4pt}
        \scriptsize{\textbf{\green{Groundtruth Answer: D}}}
        % \vspace{0.2cm}
        }
    };
%         \node[draw=black, 
%     line width=1pt,
%     rounded corners=10pt,
%     text width=\textwidth,
%     inner sep=5pt,
%     align=left,
%     below=0.1cm of box3](box4){ % Changed name to box1
%         \begin{wrapfigure}{r}{0.22\textwidth}
%         \vspace{-2.5em}
%             \includegraphics[width=0.22\textwidth]{figs/brain001109.png}
%         \end{wrapfigure}
%         \vspace{0.1cm}
%         \scriptsize{\brown{Problem 4: What can be observed in this image?
        
% A) Abscess formation, B) Fractured bone, C) Normal tissue, D) Chronic infarct


%         }}
        
%         \vspace{0.15cm}

%         \scriptsize{\textbf{MedVLM-R1 Output:}\vspace{0.1cm}

%         \blue{<think> To answer this question, we need to identify any abnormalities or structural changes within the MRI scan. The MRI scan in question displays a section of the brain, with the left-sided lower part appearing different from the right side. The left side shows irregular and more expansive regions compared to the right side. Additionally, the contrast in tissue density seems to be more visible on the left side, with darker regions indicating different tissue densities or possibly abnormal regions.

% </think>




%         \vspace{0.1cm}
        
%        <answer>D</answer>
%        }
    
%         % \noindent\rule{\dimexpr1\textwidth}{0.4pt}
%         \scriptsize{\textbf{\green{Groundtruth Answer: D}}}
%         % \vspace{0.2cm}
%         }
%     };
    \end{tikzpicture}
 \caption{\small{Medical VQA examples of MedVLM-R1 on X-ray (1, 2), MRI (3) and CT (4).}}
    \label{fig:vqa} 
\end{figure*}