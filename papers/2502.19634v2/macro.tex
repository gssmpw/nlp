%%%%% NEW MATH DEFINITIONS %%%%%

\usepackage{amsmath,amsfonts,bm}




\newcommand{\indep}{\rotatebox[origin=c]{90}{$\models$}}

\renewcommand{\mathbf}{\boldsymbol}

\newcommand{\conv}{\circledast}
\newcommand{\mb}{\mathbf}
\newcommand{\mc}{\mathcal}
\newcommand{\mf}{\mathfrak}
\newcommand{\md}{\mathds}
\newcommand{\bb}{\mathbb}
\newcommand{\te}{\texttt}
\newcommand{\msf}{\mathsf}
\newcommand{\mcr}{\mathscr}
\newcommand{\magnitude}[1]{ \left| #1 \right| }
\newcommand{\set}[1]{\left\{ #1 \right\}}
\newcommand{\condset}[2]{ \left\{ #1 \;\middle|\; #2 \right\} }


\newcommand{\reals}{\bb R}
\newcommand{\proj}{\mathrm{proj}}
\newcommand{\E}{\mathbb{E}}

\newcommand{\eps}{\varepsilon}
\newcommand{\R}{\reals}
\newcommand{\Cp}{\bb C}
\newcommand{\Z}{\bb Z}
\newcommand{\N}{\bb N}
\newcommand{\Sp}{\bb S}
\newcommand{\Ba}{\bb B}
\newcommand{\indicator}[1]{\mathbbm 1\left\{#1\right\}}
\renewcommand{\P}{\mathbb{P}}
\newcommand{\rvline}{\hspace*{-\arraycolsep}\vline\hspace*{-\arraycolsep}}
\makeatletter
\def\Ddots{\mathinner{\mkern1mu\raise\p@
\vbox{\kern7\p@\hbox{.}}\mkern2mu
\raise4\p@\hbox{.}\mkern2mu\raise7\p@\hbox{.}\mkern1mu}}
\makeatother
% to declare new operator
% \DeclareMathOperator{\xxx}{xxx}
\DeclareMathOperator{\dist}{dist}
\DeclareMathOperator{\poly}{poly}
\DeclareMathOperator{\rank}{rank}
\DeclareMathOperator{\trace}{tr}
\DeclareMathOperator{\supp}{supp}
\DeclareMathOperator{\vect}{vec}
\DeclareMathOperator{\diag}{diag}
\DeclareMathOperator{\sign}{sign}
\DeclareMathOperator{\grad}{grad}
\DeclareMathOperator{\Hess}{Hess}
\DeclareMathOperator{\mini}{minimize}
\DeclareMathOperator{\maxi}{maximize}
\DeclareMathOperator{\st}{s.t.\;}

%% Other definitions

\newcommand{\event}{\mc E}

\newcommand{\e}{\mathrm{e}}
\newcommand{\im}{\mathrm{i}}
\newcommand{\rconcave}{r_\fgecap}
\newcommand{\Lconcave}{\mc L^\fgecap}
\newcommand{\rconvex}{r_\fgecup}
\newcommand{\Rconvex}{R_\fgecup}
\newcommand{\Lconvex}{\mc L^\fgecup}

\newcommand{\wh}{\widehat}
\newcommand{\wt}{\widetilde}
\newcommand{\ol}{\overline}
\newcommand{\VER}{\msf{VER}}

\newcommand{\betaconcave}{\beta_\fgecap}
\newcommand{\betagrad}{\beta_{\mathrm{grad}}}

\newcommand{\norm}[2]{\left\| #1 \right\|_{#2}}
\newcommand{\abs}[1]{\left| #1 \right|}
\newcommand{\row}[1]{\text{row}\left( #1 \right)}
\newcommand{\innerprod}[2]{\left\langle #1,  #2 \right\rangle}
\newcommand{\prob}[1]{\bb P\left[ #1 \right]}
\newcommand{\expect}[1]{\bb E\left[ #1 \right]}
\newcommand{\function}[2]{#1 \left(#2\right}
\newcommand{\integral}[4]{\int_{#1}^{#2}\; #3\; #4}
\newcommand{\paren}[1]{\left( #1 \right)}
\newcommand{\brac}[1]{\left[ #1 \right]}
\newcommand{\Brac}[1]{\left\{ #1 \right\}}
\newcommand{\singlequote}[1]{\textquotesingle  #1\textquotesingle}


% Adding new defs here
\newcommand{\moff}{m_\mr{off}}
\newcommand{\mon}{m_\mr{on}}
\newcommand{\EmpiricalOffline}{\wh{\bb E}_{\mc D^\nu_h}}
\newcommand{\EmpiricalOnline}{\wh{\bb E}_{\mc D^\tau_h}}
\newcommand{\Deltaoff}{\Delta_\mr{off}}
\newcommand{\Deltaon}{\Delta_\mr{on}}
\newcommand{\Vmax}{V_{\max}}
\newcommand{\Doff}{\mc D_\mr{off}}
\newcommand{\Don}{\mc D_\mr{on}}
\newcommand{\unif}{\mr{unif}}
\newcommand{\Dexp}{\mc D_\mr{exp}}
\newcommand{\muexp}{\mu^\mr{exp}}
\newcommand{\piexp}{\pi^\mr{exp}}
\newcommand{\vecmuexp}{\mb{\mu}^\mr{exp}}
\newcommand{\vecdstar}{\mb{d}^\star}
\newcommand{\substates}{\tilde{\mc S}}
\newcommand{\alg}{\msf{Alg}}
\newcommand{\muthetaexp}{\mu_{\theta^\mr{exp}}}
\newcommand{\pihat}{\hat{\pi}}
\newcommand{\miniimagenet}{{\em mini}Imagenet}
\newcommand{\dpt}{\mc D_\text{pt}}
\newcommand{\dft}{\mc D_\text{ft}}
\newcommand{\upt}{\mc U_\text{pt}}
\newcommand{\uft}{\mc U_\text{ft}}
\newcommand{\barupt}{\bar{\mc U}_\text{pt}}
\newcommand{\baruft}{\bar{\mc U}_\text{ft}}
\newcommand{\wptstar}{\mb W^\star_\text{pt}}
\newcommand{\wftstar}{\mb W^\star_\text{ft}}
\newcommand{\methodname}{EMT}
\newcommand{\NC}{$\mc {NC}$}


\newcommand{\sz}[1]{{\color{blue}{\bf [Simon: #1]}}}
\newcommand{\tc}[1]{{\color{purple}{\bf [Tianzhe: #1]}}}
\newcommand{\vin}{\mb v^\text{in}}
\newcommand{\vout}{\mb v^\text{out}}
\newcommand{\vver}{\mb v^\text{ver}}
\newcommand{\vtht}{\mb v^\text{tht}}
\newcommand{\vact}{\mb v^\text{act}}
\newcommand{\cmark}{\ding{51}}%
\newcommand{\xmark}{\ding{55}}%
 
\newcommand{\gp}{{\tt GeneralPoints}}
\newcommand{\ptf}{{\tt Points24}}
\newcommand{\virl}{{\tt V-IRL}}

\newcommand{\ClubCard}{\ding{168}}
\newcommand{\SpadeCard}{\ding{171}}
\newcommand{\HeartCard}{{\color{darkred}\ding{170}}}
\newcommand{\DiamondCard}{{\color{darkred}\ding{169}}}


%% Personalized Color
\newcommand{\red}[1]{{\color[HTML]{B85450} #1}}
% rgb(72.16% 32.94% 31.37%)
\newcommand{\blue}[1]{{\color[HTML]{6C8EBF} #1}}
% rgb(42.35% 55.69% 74.9%)
\newcommand{\brown}[1]{{\color[HTML]{80461B} #1}}
% rgb(50.2% 27.45% 10.59%)
\newcommand{\purple}[1]{{\color[HTML]{884ea0} #1}}
% rgb(136, 78, 160)
\newcommand{\green}[1]{{\color[HTML]{5F8940} #1}}
% rgb(136, 78, 160)
\newcommand{\orange}[1]{{\color[HTML]{FF7E79} #1}}
% rgb(136, 78, 160)
\newcommand{\plusvalue}[1]{\textcolor{darkgreen}{\bf +#1}}
\newcommand{\minusvalue}[1]{\textcolor{darkred}{\bf -#1}}
\definecolor{darkgreen}{rgb}{0.0, 0.5, 0.0}
\definecolor{darkred}{rgb}{0.88, 0.09, 0.12}
\newcommand{\colorinfo}{The \brown{brown} parts marks the task and related information, and the \purple{purple} parts denote the state $(s_t)$ specific info. The \blue{blue} and \red{red} describe the output from the \blue{model} and \red{verifier}, respectively.}

\newcommand{\mr}{\mathrm}
\newcommand{\sym}{\mathrm{Sym}}
\newcommand{\sks}{\mathrm{Skew}}
\newcommand{\inprod}[2]{\langle#1,#2\rangle}
\newcommand{\parans}[1]{\left(#1\right}
\newcommand{\clip}{\msf{clipped}}
\newcommand{\beha}{\msf b}
\numberwithin{equation}{section}

\def \endprf{\hfill {\vrule height6pt width6pt depth0pt}\medskip}


\usepackage{tcolorbox}
\tcbuselibrary{skins}


\newtcolorbox{userinput}{
    colback=white!10,
    colframe=gray!40,
    fonttitle=\bfseries,
    coltitle=black,
    colbacktitle=gray!20,
    enhanced,
    drop shadow=black!5!white,
    left=8mm,
    right=8mm,
    top=3mm,
    bottom=3mm,
    boxsep=0mm,
    sharp corners=south,
    rounded corners=north,
    title=EMT Prompt:
    
    }

\newtcolorbox{userinputclip}{
    colback=white!10,
    colframe=gray!40,
    fonttitle=\bfseries,
    coltitle=black,
    colbacktitle=gray!20,
    enhanced,
    drop shadow=black!5!white,
    left=8mm,
    right=8mm,
    top=3mm,
    bottom=3mm,
    boxsep=0mm,
    sharp corners=south,
    rounded corners=north,
    title=Prompt:
    
    }




\newtcolorbox{modeloutput}[2]{
    colback=citecolor!5,
    colframe=gray!40,
    fonttitle=\bfseries,
    coltitle=black,
    colbacktitle=gray!20,
    enhanced,
    drop shadow=black!5!white,
    left=8mm,
    right=8mm,
    top=3mm,
    bottom=3mm,
    boxsep=0mm,
    sharp corners=south,
    rounded corners=north,
    title=Label: {#1} \;|\; {\te{#2}}: 
    
    }


\newtcolorbox{modeloutput_withimage}[3][]{
    colback=citecolor!5,
    colframe=gray!40,
    fonttitle=\bfseries,
    coltitle=black,
    colbacktitle=gray!20,
    enhanced,
    drop shadow=black!5!white,
    left=1.5cm, % Adjusted for the image
    right=8mm,
    top=3mm,
    bottom=3mm,
    boxsep=0mm,
    sharp corners=south,
    rounded corners=north,
    title=Label: {#2} \;|\; {\te{#3}},
    overlay={
        \node[anchor=north west, yshift=-4.1mm, xshift=0.4mm] (imgnode) at (frame.north west) {\includegraphics[width=0.7cm]{#1}};
        \draw[gray!50, line width=0.5mm] (imgnode.east |- frame.north) -- (imgnode.east |- frame.south);
        
        % Add "Image" label on top of the image node
        \node[anchor=south, font=\footnotesize, text=black, yshift=-1mm, xshift=0mm] at (imgnode.north) {\te{img}};



        
        %\node[anchor=east, font=\small, text=black] at ($(imgnode.east |- frame.north) + (-0.3cm, 0)$) {input label}; % Placing the label in the gray title area
    }
}


