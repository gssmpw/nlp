\section{Related Work}
Existing trajectory generation works can be divided into two categories: Knowledge-Driven methods and Data-Driven methods.

\textbf{Knowledge-driven methods} models human mobility based on prior knowledge and statistical data. Gravity Model~\cite{gravity_1946} and Intervening Opportunities Model~\cite{interventing} generate trajectories by modeling the relationship between inter-regional mobility intensity and population and economic data. The EPR model~\cite{epr_1, epr_2} regards trajectory generation as a process of multiple explorations and returns, and the probability of an individual visiting a location is related to regional attractiveness. This type of method emerged along with traditional modeling methods. It has a coarse granularity but has certain generalization capabilities.

\textbf{Data-driven methods} uses large amounts of trajectory data to train neural network models,  capturing complex travel patterns. According to different architectures, these algorithms can be divided into algorithms based on Seq2Seq ~\cite{Seq2Seq}, GAN ~\cite{SeqGAN}, VAE ~\cite{volunteer, SVAE}, and Diffusion ~\cite{difftraj}. The Seq2Seq method uses RNN, TCN, Transformer to model the movement state of individual, then predict the choice of individuals facing multiple optional road segments. As an improvement to Seq2Seq, Neural A* ~\cite{ts_trajgen} searches multiple trajectories simultaneously to provide the most likely trajectory.  Similarly, the VAE method models the individual's movement state and travel prefernece through latent variables. In order to better generate trajectories with composite human movement distribution, the GAN method leverage a discriminator to identify the authenticity of the trajectory. Different from the method of generating trajectory component by component, Diffusion innovatively treats the trajectory of human movement as a special texture and generates the entire trajectory at once using image generation. This type of method focuses on statistical learning and has generalization capabilities in the same city but cannot be generalized to other cities.

In other areas of urban data mining, such as traffic flow prediction  and OD (origin-destination) demand prediction ~\cite{predji2023spatio, predjiang2023pdformer, predwang2022traffic}, there have been numerous active studies on cross-city transfer learning. ~\cite{wang2018cross} calculates matching scores between regions using other data and forces the model to generate similar representations for matched region pairs by adding a matching regularization term. 
CrossTRes~\cite{jin2022selective} conducts transfer learning using data from multiple cities. To avoid negative transfer caused by data from dissimilar cities, this method introduces a weighted score for each city to weight the losses from multiple cities. Yu Zheng et al.~\cite{he2020human} proposed an OD transfer learning model primarily focused on generating travel demand for new cities. Research on tranfer learning in other fields inspired us to propose \name.