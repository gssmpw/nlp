\section{Preliminaries}
\label{sec:preliminaries}

\subsection{Definitions}
\begin{definition}[Road Network] Road network is represented as a graph $\mathcal{G = \langle R,E\rangle}$, where $\mathcal{R} = \{r_i\mid i = 1, 2,\cdots,N\}$ is the road segment set and $\mathcal E=\{e_{ij}\}$ is the edge set. $e_{ij} = 1$ if $r_i$ is adjacent to $r_j$, otherwise $e_{ij} = 0$. 
\end{definition}


\begin{definition}[Trajectory] A trajectory is a series of consecutive road segments $\tau = (r_1,r_2,\cdots,r_{l})$, where $l$ denotes the length of the trajectory. A dataset $\mathcal{T} = \{\tau_i \mid \tau_1, \tau_2, \cdots, \tau_C\}$ is formed by $C$ trajectories. \end{definition}

\subsection{Problem Statement}

Given a source city with a road network $\mathcal{G}^{(src)}$ and a trajectory dataset $\mathcal{T}^{(src)}$, the objective is to train a model $\mathcal{F}$ with parameters $\theta$ to generate a new trajectory dataset $\hat{\mathcal{T}}^{(tgt)}$ for a target city with a road network $\mathcal{G}^{(tgt)}$ but no trajectory data. The model is trained using the source city data as follows
\begin{equation}
\theta^{(src)} = \arg \min_{\theta} \mathcal{L} \left( \mathcal{F} \left( \mathcal{G}^{(src)}; \theta \right), \mathcal{T}^{(src)} \right),
\end{equation}
where $\mathcal{L}$ represents the optimization objective. The well-trained model is then applied to the target city to generate a new trajectory dataset
\begin{equation}
\hat{\mathcal{T}}^{(tgt)} = \mathcal{F} \left(
\mathcal{G}^{(tgt)}; \theta^{(src)}
\right).
\end{equation}
The primary objective is to ensure that the generated dataset $\hat{\mathcal{T}}^{(tgt)}$ is similar to the real dataset $\mathcal{T}^{(tgt)}$.