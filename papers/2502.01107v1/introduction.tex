\section{Introduction}
\label{sec:introduction}
Trajectory data is essential for intelligent transportation (ITS) ~\cite{wu2019learning, liu2024full}, smart cities ~\cite{hettige2024airphynet, ji2022precision}, and location-based services (LBS) ~\cite{lbs2018, lbs2021dgeye}. Trajectory generation is key to addressing the issue of insufficient trajectory data, especially in scenarios involving privacy protection and conflicts of commercial interests. 

Urban trajectory generation is an active research field, and different types of methods have been proposed. These methods can be divided into two categories: knowledge-driven methods and data-driven methods~\cite{kong2023mobility}.

\textbf{Knowledge-driven methods} generate trajectories based on empirical and statistical patterns of human mobility, including the Gravity Model~\cite{gravity_1946}, Intervening Opportunities Model~\cite{interventing}, and EPR model~\cite{epr_1, epr_2}. These methods typically analyze human behavior using coarse-grained grids and derive physical priors with practical significance. By empirically summarizing human mobility behavior, these methods have achieved notable results in macroscopic traffic trajectory simulations. However, to capture human mobility behavior with finer granularity, data-driven approaches are increasingly emphasized.


\textbf{Data-driven methods} utilize deep neural networks to capture complex mobility patterns in trajectory data. 
Based on different architectures, these algorithms can be categorized into Seq2Seq-based~\cite{Seq2Seq}, GAN-based~\cite{SeqGAN}, VAE-based~\cite{volunteer, SVAE}, and Diffusion-based~\cite{difftraj} methods.  
With the advancements in data collection and management~\cite{manage_chen2015efficient, manage_ding2018ultraman}, data-driven methods have been widely applied. Although they have achieved significant success by learning from large datasets, they also lead to data dependency. 

Knowledge-driven methods require less trajectory data but struggle to capture complex mobility patterns. In contrast, data-driven methods, while offering better performance, face challenges in generalizing to new cities. In data-driven models, changes in a city’s topological structure can prevent the effective transfer of previously learned road representations to the new network. 

\begin{figure}[t]
    \centering
    \includegraphics[width=0.39\textwidth] 
    {figure/similar_topology.pdf}
    \caption{
        Similar local topological structures in New York (left) and Chicago (right). 
    }
    \label{fig:local_topology}
\end{figure}


The key to achieving cross-city trajectory generation is capturing the invariant human mobility patterns across different urban environments. Based on existing research ~\cite{pref2019, pref2021}, we have the following insights about the invariant human mobility patterns.
(\romannumeral1) People in different cities have similar travel preferences. Generally, people prefer paths with the minimal travel costs. By learning the combination of travel costs, we can generate trajectories that align with human preferences.
(\romannumeral2) The topological structure of the road network influences the functionality, usage frequency, and congestion levels of roads, thus determining the travel costs. Although the global road network structures differ between cities, there exists similar local topological structures. (see Figure~\ref{fig:local_topology}), making transfer learning possible. 


Based on the above insights, we propose a cross-city trajectory generation model that combines shortest path search and deep learning. 
This model use \textit{Space Syntax} methods to extract topological features, employ a disentangled adversarial domain adaptation algorithm to learn invariant topological representations and predict travel costs, and finally learn human travel preferences to generate trajectories.

Our main contributions can be summarized in the following three points.
\begin{itemize}
    \item We combine \textit{Space Syntax} method with the advantages of deep neural networks to capture the invariant human mobility patterns across cities.
    \item We propose a novel cost prediction and preference learning method based on invariant topological features, and integrate it with the shortest path search algorithm to achieve cross-city trajectory generation. Our approach addresses the issue of cross-city generalization ability in trajectory generation.
    \item Experiments in multiple scenarios, datasets, and evaluation metrics show that our method has a generalization capability that far exceeds that of the baseline model.
\end{itemize}
