\pdfoutput=1

\documentclass[11pt]{article}

\usepackage[final]{acl}

\usepackage{times}
\usepackage{latexsym}
\usepackage{listings}
\lstset{breaklines=true}

\usepackage[T1]{fontenc}

\usepackage[utf8]{inputenc}

\usepackage{microtype}

\usepackage{inconsolata}

%%%%% NEW MATH DEFINITIONS %%%%%

\usepackage{amsmath,amsfonts,bm}
\usepackage{derivative}
% Mark sections of captions for referring to divisions of figures
\newcommand{\figleft}{{\em (Left)}}
\newcommand{\figcenter}{{\em (Center)}}
\newcommand{\figright}{{\em (Right)}}
\newcommand{\figtop}{{\em (Top)}}
\newcommand{\figbottom}{{\em (Bottom)}}
\newcommand{\captiona}{{\em (a)}}
\newcommand{\captionb}{{\em (b)}}
\newcommand{\captionc}{{\em (c)}}
\newcommand{\captiond}{{\em (d)}}

% Highlight a newly defined term
\newcommand{\newterm}[1]{{\bf #1}}

% Derivative d 
\newcommand{\deriv}{{\mathrm{d}}}

% Figure reference, lower-case.
\def\figref#1{figure~\ref{#1}}
% Figure reference, capital. For start of sentence
\def\Figref#1{Figure~\ref{#1}}
\def\twofigref#1#2{figures \ref{#1} and \ref{#2}}
\def\quadfigref#1#2#3#4{figures \ref{#1}, \ref{#2}, \ref{#3} and \ref{#4}}
% Section reference, lower-case.
\def\secref#1{section~\ref{#1}}
% Section reference, capital.
\def\Secref#1{Section~\ref{#1}}
% Reference to two sections.
\def\twosecrefs#1#2{sections \ref{#1} and \ref{#2}}
% Reference to three sections.
\def\secrefs#1#2#3{sections \ref{#1}, \ref{#2} and \ref{#3}}
% Reference to an equation, lower-case.
\def\eqref#1{equation~\ref{#1}}
% Reference to an equation, upper case
\def\Eqref#1{Equation~\ref{#1}}
% A raw reference to an equation---avoid using if possible
\def\plaineqref#1{\ref{#1}}
% Reference to a chapter, lower-case.
\def\chapref#1{chapter~\ref{#1}}
% Reference to an equation, upper case.
\def\Chapref#1{Chapter~\ref{#1}}
% Reference to a range of chapters
\def\rangechapref#1#2{chapters\ref{#1}--\ref{#2}}
% Reference to an algorithm, lower-case.
\def\algref#1{algorithm~\ref{#1}}
% Reference to an algorithm, upper case.
\def\Algref#1{Algorithm~\ref{#1}}
\def\twoalgref#1#2{algorithms \ref{#1} and \ref{#2}}
\def\Twoalgref#1#2{Algorithms \ref{#1} and \ref{#2}}
% Reference to a part, lower case
\def\partref#1{part~\ref{#1}}
% Reference to a part, upper case
\def\Partref#1{Part~\ref{#1}}
\def\twopartref#1#2{parts \ref{#1} and \ref{#2}}

\def\ceil#1{\lceil #1 \rceil}
\def\floor#1{\lfloor #1 \rfloor}
\def\1{\bm{1}}
\newcommand{\train}{\mathcal{D}}
\newcommand{\valid}{\mathcal{D_{\mathrm{valid}}}}
\newcommand{\test}{\mathcal{D_{\mathrm{test}}}}

\def\eps{{\epsilon}}


% Random variables
\def\reta{{\textnormal{$\eta$}}}
\def\ra{{\textnormal{a}}}
\def\rb{{\textnormal{b}}}
\def\rc{{\textnormal{c}}}
\def\rd{{\textnormal{d}}}
\def\re{{\textnormal{e}}}
\def\rf{{\textnormal{f}}}
\def\rg{{\textnormal{g}}}
\def\rh{{\textnormal{h}}}
\def\ri{{\textnormal{i}}}
\def\rj{{\textnormal{j}}}
\def\rk{{\textnormal{k}}}
\def\rl{{\textnormal{l}}}
% rm is already a command, just don't name any random variables m
\def\rn{{\textnormal{n}}}
\def\ro{{\textnormal{o}}}
\def\rp{{\textnormal{p}}}
\def\rq{{\textnormal{q}}}
\def\rr{{\textnormal{r}}}
\def\rs{{\textnormal{s}}}
\def\rt{{\textnormal{t}}}
\def\ru{{\textnormal{u}}}
\def\rv{{\textnormal{v}}}
\def\rw{{\textnormal{w}}}
\def\rx{{\textnormal{x}}}
\def\ry{{\textnormal{y}}}
\def\rz{{\textnormal{z}}}

% Random vectors
\def\rvepsilon{{\mathbf{\epsilon}}}
\def\rvphi{{\mathbf{\phi}}}
\def\rvtheta{{\mathbf{\theta}}}
\def\rva{{\mathbf{a}}}
\def\rvb{{\mathbf{b}}}
\def\rvc{{\mathbf{c}}}
\def\rvd{{\mathbf{d}}}
\def\rve{{\mathbf{e}}}
\def\rvf{{\mathbf{f}}}
\def\rvg{{\mathbf{g}}}
\def\rvh{{\mathbf{h}}}
\def\rvu{{\mathbf{i}}}
\def\rvj{{\mathbf{j}}}
\def\rvk{{\mathbf{k}}}
\def\rvl{{\mathbf{l}}}
\def\rvm{{\mathbf{m}}}
\def\rvn{{\mathbf{n}}}
\def\rvo{{\mathbf{o}}}
\def\rvp{{\mathbf{p}}}
\def\rvq{{\mathbf{q}}}
\def\rvr{{\mathbf{r}}}
\def\rvs{{\mathbf{s}}}
\def\rvt{{\mathbf{t}}}
\def\rvu{{\mathbf{u}}}
\def\rvv{{\mathbf{v}}}
\def\rvw{{\mathbf{w}}}
\def\rvx{{\mathbf{x}}}
\def\rvy{{\mathbf{y}}}
\def\rvz{{\mathbf{z}}}

% Elements of random vectors
\def\erva{{\textnormal{a}}}
\def\ervb{{\textnormal{b}}}
\def\ervc{{\textnormal{c}}}
\def\ervd{{\textnormal{d}}}
\def\erve{{\textnormal{e}}}
\def\ervf{{\textnormal{f}}}
\def\ervg{{\textnormal{g}}}
\def\ervh{{\textnormal{h}}}
\def\ervi{{\textnormal{i}}}
\def\ervj{{\textnormal{j}}}
\def\ervk{{\textnormal{k}}}
\def\ervl{{\textnormal{l}}}
\def\ervm{{\textnormal{m}}}
\def\ervn{{\textnormal{n}}}
\def\ervo{{\textnormal{o}}}
\def\ervp{{\textnormal{p}}}
\def\ervq{{\textnormal{q}}}
\def\ervr{{\textnormal{r}}}
\def\ervs{{\textnormal{s}}}
\def\ervt{{\textnormal{t}}}
\def\ervu{{\textnormal{u}}}
\def\ervv{{\textnormal{v}}}
\def\ervw{{\textnormal{w}}}
\def\ervx{{\textnormal{x}}}
\def\ervy{{\textnormal{y}}}
\def\ervz{{\textnormal{z}}}

% Random matrices
\def\rmA{{\mathbf{A}}}
\def\rmB{{\mathbf{B}}}
\def\rmC{{\mathbf{C}}}
\def\rmD{{\mathbf{D}}}
\def\rmE{{\mathbf{E}}}
\def\rmF{{\mathbf{F}}}
\def\rmG{{\mathbf{G}}}
\def\rmH{{\mathbf{H}}}
\def\rmI{{\mathbf{I}}}
\def\rmJ{{\mathbf{J}}}
\def\rmK{{\mathbf{K}}}
\def\rmL{{\mathbf{L}}}
\def\rmM{{\mathbf{M}}}
\def\rmN{{\mathbf{N}}}
\def\rmO{{\mathbf{O}}}
\def\rmP{{\mathbf{P}}}
\def\rmQ{{\mathbf{Q}}}
\def\rmR{{\mathbf{R}}}
\def\rmS{{\mathbf{S}}}
\def\rmT{{\mathbf{T}}}
\def\rmU{{\mathbf{U}}}
\def\rmV{{\mathbf{V}}}
\def\rmW{{\mathbf{W}}}
\def\rmX{{\mathbf{X}}}
\def\rmY{{\mathbf{Y}}}
\def\rmZ{{\mathbf{Z}}}

% Elements of random matrices
\def\ermA{{\textnormal{A}}}
\def\ermB{{\textnormal{B}}}
\def\ermC{{\textnormal{C}}}
\def\ermD{{\textnormal{D}}}
\def\ermE{{\textnormal{E}}}
\def\ermF{{\textnormal{F}}}
\def\ermG{{\textnormal{G}}}
\def\ermH{{\textnormal{H}}}
\def\ermI{{\textnormal{I}}}
\def\ermJ{{\textnormal{J}}}
\def\ermK{{\textnormal{K}}}
\def\ermL{{\textnormal{L}}}
\def\ermM{{\textnormal{M}}}
\def\ermN{{\textnormal{N}}}
\def\ermO{{\textnormal{O}}}
\def\ermP{{\textnormal{P}}}
\def\ermQ{{\textnormal{Q}}}
\def\ermR{{\textnormal{R}}}
\def\ermS{{\textnormal{S}}}
\def\ermT{{\textnormal{T}}}
\def\ermU{{\textnormal{U}}}
\def\ermV{{\textnormal{V}}}
\def\ermW{{\textnormal{W}}}
\def\ermX{{\textnormal{X}}}
\def\ermY{{\textnormal{Y}}}
\def\ermZ{{\textnormal{Z}}}

% Vectors
\def\vzero{{\bm{0}}}
\def\vone{{\bm{1}}}
\def\vmu{{\bm{\mu}}}
\def\vtheta{{\bm{\theta}}}
\def\vphi{{\bm{\phi}}}
\def\va{{\bm{a}}}
\def\vb{{\bm{b}}}
\def\vc{{\bm{c}}}
\def\vd{{\bm{d}}}
\def\ve{{\bm{e}}}
\def\vf{{\bm{f}}}
\def\vg{{\bm{g}}}
\def\vh{{\bm{h}}}
\def\vi{{\bm{i}}}
\def\vj{{\bm{j}}}
\def\vk{{\bm{k}}}
\def\vl{{\bm{l}}}
\def\vm{{\bm{m}}}
\def\vn{{\bm{n}}}
\def\vo{{\bm{o}}}
\def\vp{{\bm{p}}}
\def\vq{{\bm{q}}}
\def\vr{{\bm{r}}}
\def\vs{{\bm{s}}}
\def\vt{{\bm{t}}}
\def\vu{{\bm{u}}}
\def\vv{{\bm{v}}}
\def\vw{{\bm{w}}}
\def\vx{{\bm{x}}}
\def\vy{{\bm{y}}}
\def\vz{{\bm{z}}}

% Elements of vectors
\def\evalpha{{\alpha}}
\def\evbeta{{\beta}}
\def\evepsilon{{\epsilon}}
\def\evlambda{{\lambda}}
\def\evomega{{\omega}}
\def\evmu{{\mu}}
\def\evpsi{{\psi}}
\def\evsigma{{\sigma}}
\def\evtheta{{\theta}}
\def\eva{{a}}
\def\evb{{b}}
\def\evc{{c}}
\def\evd{{d}}
\def\eve{{e}}
\def\evf{{f}}
\def\evg{{g}}
\def\evh{{h}}
\def\evi{{i}}
\def\evj{{j}}
\def\evk{{k}}
\def\evl{{l}}
\def\evm{{m}}
\def\evn{{n}}
\def\evo{{o}}
\def\evp{{p}}
\def\evq{{q}}
\def\evr{{r}}
\def\evs{{s}}
\def\evt{{t}}
\def\evu{{u}}
\def\evv{{v}}
\def\evw{{w}}
\def\evx{{x}}
\def\evy{{y}}
\def\evz{{z}}

% Matrix
\def\mA{{\bm{A}}}
\def\mB{{\bm{B}}}
\def\mC{{\bm{C}}}
\def\mD{{\bm{D}}}
\def\mE{{\bm{E}}}
\def\mF{{\bm{F}}}
\def\mG{{\bm{G}}}
\def\mH{{\bm{H}}}
\def\mI{{\bm{I}}}
\def\mJ{{\bm{J}}}
\def\mK{{\bm{K}}}
\def\mL{{\bm{L}}}
\def\mM{{\bm{M}}}
\def\mN{{\bm{N}}}
\def\mO{{\bm{O}}}
\def\mP{{\bm{P}}}
\def\mQ{{\bm{Q}}}
\def\mR{{\bm{R}}}
\def\mS{{\bm{S}}}
\def\mT{{\bm{T}}}
\def\mU{{\bm{U}}}
\def\mV{{\bm{V}}}
\def\mW{{\bm{W}}}
\def\mX{{\bm{X}}}
\def\mY{{\bm{Y}}}
\def\mZ{{\bm{Z}}}
\def\mBeta{{\bm{\beta}}}
\def\mPhi{{\bm{\Phi}}}
\def\mLambda{{\bm{\Lambda}}}
\def\mSigma{{\bm{\Sigma}}}

% Tensor
\DeclareMathAlphabet{\mathsfit}{\encodingdefault}{\sfdefault}{m}{sl}
\SetMathAlphabet{\mathsfit}{bold}{\encodingdefault}{\sfdefault}{bx}{n}
\newcommand{\tens}[1]{\bm{\mathsfit{#1}}}
\def\tA{{\tens{A}}}
\def\tB{{\tens{B}}}
\def\tC{{\tens{C}}}
\def\tD{{\tens{D}}}
\def\tE{{\tens{E}}}
\def\tF{{\tens{F}}}
\def\tG{{\tens{G}}}
\def\tH{{\tens{H}}}
\def\tI{{\tens{I}}}
\def\tJ{{\tens{J}}}
\def\tK{{\tens{K}}}
\def\tL{{\tens{L}}}
\def\tM{{\tens{M}}}
\def\tN{{\tens{N}}}
\def\tO{{\tens{O}}}
\def\tP{{\tens{P}}}
\def\tQ{{\tens{Q}}}
\def\tR{{\tens{R}}}
\def\tS{{\tens{S}}}
\def\tT{{\tens{T}}}
\def\tU{{\tens{U}}}
\def\tV{{\tens{V}}}
\def\tW{{\tens{W}}}
\def\tX{{\tens{X}}}
\def\tY{{\tens{Y}}}
\def\tZ{{\tens{Z}}}


% Graph
\def\gA{{\mathcal{A}}}
\def\gB{{\mathcal{B}}}
\def\gC{{\mathcal{C}}}
\def\gD{{\mathcal{D}}}
\def\gE{{\mathcal{E}}}
\def\gF{{\mathcal{F}}}
\def\gG{{\mathcal{G}}}
\def\gH{{\mathcal{H}}}
\def\gI{{\mathcal{I}}}
\def\gJ{{\mathcal{J}}}
\def\gK{{\mathcal{K}}}
\def\gL{{\mathcal{L}}}
\def\gM{{\mathcal{M}}}
\def\gN{{\mathcal{N}}}
\def\gO{{\mathcal{O}}}
\def\gP{{\mathcal{P}}}
\def\gQ{{\mathcal{Q}}}
\def\gR{{\mathcal{R}}}
\def\gS{{\mathcal{S}}}
\def\gT{{\mathcal{T}}}
\def\gU{{\mathcal{U}}}
\def\gV{{\mathcal{V}}}
\def\gW{{\mathcal{W}}}
\def\gX{{\mathcal{X}}}
\def\gY{{\mathcal{Y}}}
\def\gZ{{\mathcal{Z}}}

% Sets
\def\sA{{\mathbb{A}}}
\def\sB{{\mathbb{B}}}
\def\sC{{\mathbb{C}}}
\def\sD{{\mathbb{D}}}
% Don't use a set called E, because this would be the same as our symbol
% for expectation.
\def\sF{{\mathbb{F}}}
\def\sG{{\mathbb{G}}}
\def\sH{{\mathbb{H}}}
\def\sI{{\mathbb{I}}}
\def\sJ{{\mathbb{J}}}
\def\sK{{\mathbb{K}}}
\def\sL{{\mathbb{L}}}
\def\sM{{\mathbb{M}}}
\def\sN{{\mathbb{N}}}
\def\sO{{\mathbb{O}}}
\def\sP{{\mathbb{P}}}
\def\sQ{{\mathbb{Q}}}
\def\sR{{\mathbb{R}}}
\def\sS{{\mathbb{S}}}
\def\sT{{\mathbb{T}}}
\def\sU{{\mathbb{U}}}
\def\sV{{\mathbb{V}}}
\def\sW{{\mathbb{W}}}
\def\sX{{\mathbb{X}}}
\def\sY{{\mathbb{Y}}}
\def\sZ{{\mathbb{Z}}}

% Entries of a matrix
\def\emLambda{{\Lambda}}
\def\emA{{A}}
\def\emB{{B}}
\def\emC{{C}}
\def\emD{{D}}
\def\emE{{E}}
\def\emF{{F}}
\def\emG{{G}}
\def\emH{{H}}
\def\emI{{I}}
\def\emJ{{J}}
\def\emK{{K}}
\def\emL{{L}}
\def\emM{{M}}
\def\emN{{N}}
\def\emO{{O}}
\def\emP{{P}}
\def\emQ{{Q}}
\def\emR{{R}}
\def\emS{{S}}
\def\emT{{T}}
\def\emU{{U}}
\def\emV{{V}}
\def\emW{{W}}
\def\emX{{X}}
\def\emY{{Y}}
\def\emZ{{Z}}
\def\emSigma{{\Sigma}}

% entries of a tensor
% Same font as tensor, without \bm wrapper
\newcommand{\etens}[1]{\mathsfit{#1}}
\def\etLambda{{\etens{\Lambda}}}
\def\etA{{\etens{A}}}
\def\etB{{\etens{B}}}
\def\etC{{\etens{C}}}
\def\etD{{\etens{D}}}
\def\etE{{\etens{E}}}
\def\etF{{\etens{F}}}
\def\etG{{\etens{G}}}
\def\etH{{\etens{H}}}
\def\etI{{\etens{I}}}
\def\etJ{{\etens{J}}}
\def\etK{{\etens{K}}}
\def\etL{{\etens{L}}}
\def\etM{{\etens{M}}}
\def\etN{{\etens{N}}}
\def\etO{{\etens{O}}}
\def\etP{{\etens{P}}}
\def\etQ{{\etens{Q}}}
\def\etR{{\etens{R}}}
\def\etS{{\etens{S}}}
\def\etT{{\etens{T}}}
\def\etU{{\etens{U}}}
\def\etV{{\etens{V}}}
\def\etW{{\etens{W}}}
\def\etX{{\etens{X}}}
\def\etY{{\etens{Y}}}
\def\etZ{{\etens{Z}}}

% The true underlying data generating distribution
\newcommand{\pdata}{p_{\rm{data}}}
\newcommand{\ptarget}{p_{\rm{target}}}
\newcommand{\pprior}{p_{\rm{prior}}}
\newcommand{\pbase}{p_{\rm{base}}}
\newcommand{\pref}{p_{\rm{ref}}}

% The empirical distribution defined by the training set
\newcommand{\ptrain}{\hat{p}_{\rm{data}}}
\newcommand{\Ptrain}{\hat{P}_{\rm{data}}}
% The model distribution
\newcommand{\pmodel}{p_{\rm{model}}}
\newcommand{\Pmodel}{P_{\rm{model}}}
\newcommand{\ptildemodel}{\tilde{p}_{\rm{model}}}
% Stochastic autoencoder distributions
\newcommand{\pencode}{p_{\rm{encoder}}}
\newcommand{\pdecode}{p_{\rm{decoder}}}
\newcommand{\precons}{p_{\rm{reconstruct}}}

\newcommand{\laplace}{\mathrm{Laplace}} % Laplace distribution

\newcommand{\E}{\mathbb{E}}
\newcommand{\Ls}{\mathcal{L}}
\newcommand{\R}{\mathbb{R}}
\newcommand{\emp}{\tilde{p}}
\newcommand{\lr}{\alpha}
\newcommand{\reg}{\lambda}
\newcommand{\rect}{\mathrm{rectifier}}
\newcommand{\softmax}{\mathrm{softmax}}
\newcommand{\sigmoid}{\sigma}
\newcommand{\softplus}{\zeta}
\newcommand{\KL}{D_{\mathrm{KL}}}
\newcommand{\Var}{\mathrm{Var}}
\newcommand{\standarderror}{\mathrm{SE}}
\newcommand{\Cov}{\mathrm{Cov}}
% Wolfram Mathworld says $L^2$ is for function spaces and $\ell^2$ is for vectors
% But then they seem to use $L^2$ for vectors throughout the site, and so does
% wikipedia.
\newcommand{\normlzero}{L^0}
\newcommand{\normlone}{L^1}
\newcommand{\normltwo}{L^2}
\newcommand{\normlp}{L^p}
\newcommand{\normmax}{L^\infty}

\newcommand{\parents}{Pa} % See usage in notation.tex. Chosen to match Daphne's book.

\DeclareMathOperator*{\argmax}{arg\,max}
\DeclareMathOperator*{\argmin}{arg\,min}

\DeclareMathOperator{\sign}{sign}
\DeclareMathOperator{\Tr}{Tr}
\let\ab\allowbreak

\usepackage{hyperref}
\usepackage{url}
\usepackage{graphicx}
\usepackage{float}
\usepackage{enumitem}
\newtheorem{definition}{Definition}
\usepackage{booktabs}
\usepackage{wrapfig}
\usepackage{subcaption}
\usepackage{microtype}

\usepackage{xcolor}
\usepackage{colortbl}
\usepackage{multirow}
\usepackage{tcolorbox}

\hypersetup{
    colorlinks,
    linkcolor={blue},%
    citecolor={green!60!black},
    urlcolor={blue}%
}

\newcommand{\ivy}[1]{{\textcolor{teal}{[Ivy: #1]}}{}}
\newcommand{\swanand}[1]{{\textcolor{blue}{[Swanand: #1]}}{}}
\newcommand{\ad}[1]{{\textcolor{orange}{[Amit: #1]}}{}}
\newcommand{\hao}[1]{{\color{purple}#1}}
\newcommand{\nrk}[1]{{\textcolor{magenta}{[NRK: #1]}}{}}
\newcommand{\new}[1]{{\color{red}#1}}


\title{Protecting Users From Themselves:\\ Safeguarding Contextual Privacy in Interactions with Conversational Agents}

\author{Ivoline C. Ngong\thanks{Graduate student at University of Vermont. Work done during summer internship at IBM Research.},~Swanand Kadhe, Hao Wang, Keerthiram Murugesan, Justin D. Weisz,\\ 
\textbf{Amit Dhurandhar, Karthikeyan Natesan Ramamurthy} \\
IBM Research. \\
\texttt{kngongiv@uvm.edu,}\\
\texttt{\{swanand.kadhe,keerthiram.murugesan\}@ibm.com,}\\
\texttt{hao-wang@redhat.com},
\texttt{\{jweisz,adhuran,knatesa\}@us.ibm.com}
}


\begin{document}
\maketitle
\begin{abstract}
Conversational agents are increasingly woven into individuals' personal lives, yet users often underestimate the privacy risks involved. The moment users share information with these agents (e.g., LLMs), their private information becomes vulnerable to exposure. In this paper, we characterize the notion of contextual privacy for user interactions with LLMs. It aims to minimize privacy risks by ensuring that users (sender) disclose only information that is both relevant and necessary for achieving their intended goals when interacting with LLMs (untrusted receivers). Through a formative design user study, we observe how even ``privacy-conscious'' users inadvertently reveal sensitive information through indirect disclosures. 
Based on insights from this study, 
we propose a locally-deployable framework that operates between users and LLMs, and identifies and reformulates out-of-context information in user prompts. Our evaluation using examples from ShareGPT shows that lightweight models can effectively implement this framework, achieving strong gains in contextual privacy while preserving the user's intended interaction goals through different approaches to classify information relevant to the intended goals. 
\end{abstract}


\section{Introduction}\label{sec:Intro} 


Novel view synthesis offers a fundamental approach to visualizing complex scenes by generating new perspectives from existing imagery. 
This has many potential applications, including virtual reality, movie production and architectural visualization \cite{Tewari2022NeuRendSTAR}. 
An emerging alternative to the common RGB sensors are event cameras, which are  
 bio-inspired visual sensors recording events, i.e.~asynchronous per-pixel signals of changes in brightness or color intensity. 

Event streams have very high temporal resolution and are inherently sparse, as they only happen when changes in the scene are observed. 
Due to their working principle, event cameras bring several advantages, especially in challenging cases: they excel at handling high-speed motions 
and have a substantially higher dynamic range of the supported signal measurements than conventional RGB cameras. 
Moreover, they have lower power consumption and require varied storage volumes for captured data that are often smaller than those required for synchronous RGB cameras \cite{Millerdurai_3DV2024, Gallego2022}. 

The ability to handle high-speed motions is crucial in static scenes as well,  particularly with handheld moving cameras, as it helps avoid the common problem of motion blur. It is, therefore, not surprising that event-based novel view synthesis has gained attention, although color values are not directly observed.
Notably, because of the substantial difference between the formats, RGB- and event-based approaches require fundamentally different design choices. %

The first solutions to event-based novel view synthesis introduced in the literature demonstrate promising results \cite{eventnerf, enerf} and outperform non-event-based alternatives for novel view synthesis in many challenging scenarios. 
Among them, EventNeRF \cite{eventnerf} enables novel-view synthesis in the RGB space by assuming events associated with three color channels as inputs. 
Due to its NeRF-based architecture \cite{nerf}, it can handle single objects with complete observations from roughly equal distances to the camera. 
It furthermore has limitations in training and rendering speed: 
the MLP used to represent the scene requires long training time and can only handle very limited scene extents or otherwise rendering quality will deteriorate. 
Hence, the quality of synthesized novel views will degrade for larger scenes. %

We present Event-3DGS (E-3DGS), i.e.,~a new method for novel-view synthesis from event streams using 3D Gaussians~\cite{3dgs} 
demonstrating fast reconstruction and rendering as well as handling of unbounded scenes. 
The technical contributions of this paper are as follows: 
\begin{itemize}
\item With E-3DGS, we introduce the first approach for novel view synthesis from a color event camera that combines 3D Gaussians with event-based supervision. 
\item We present frustum-based initialization, adaptive event windows, isotropic 3D Gaussian regularization and 3D camera pose refinement, and demonstrate that high-quality results can be obtained. %

\item Finally, we introduce new synthetic and real event datasets for large scenes to the community to study novel view synthesis in this new problem setting. 
\end{itemize}
Our experiments demonstrate systematically superior results compared to EventNeRF \cite{eventnerf} and other baselines. 
The source code and dataset of E-3DGS are released\footnote{\url{https://4dqv.mpi-inf.mpg.de/E3DGS/}}. 





\section{Threat Models and Privacy Definition}
\label{sec::threat_privacy}



\noindent\textbf{Threat Model.} 
We consider a scenario where users interact with large, remote, and untrusted LCAs through APIs. These can be web-based or hosted on cloud-based services or private networks and may be either general-purpose or domain-specific. Users often share personal, financial, or medical information without clear knowledge of how their data is managed, increasing privacy risks due to the lack of transparency around these agents. 


We focus on a threat model where users unintentionally compromise their privacy by oversharing information. Our approach targets out-of-context \emph{self-disclosure} by guiding users to share only contextually necessary information. By identifying unnecessary or sensitive disclosures in real time, we assist users in controlling the information they reveal, thereby reducing the risk of unintentional privacy breaches.
Our approach indirectly mitigates the threat of \emph{malicious users}, who seek to extract sensitive information from the agents by manipulating their interactions, by minimizing the amount of sensitive information exchanged during interactions.






\noindent\textbf{Contextual Privacy in Conversational Agents} 
We define the notion of \textit{contextual privacy} in conversational agents, inspired by the Contextual Integrity (CI) theory. 
CI models privacy as information flow defined by the five parameters sender (who is sharing the data), subject (who the information is about), receiver (who is getting the data), context (what sort of information is being shared), and transmission principle (the conditions under which information flow is conducted) \cite{nissenbaum2004privacy}. CI evaluates whether the information flow adheres to appropriate standards governed by norms, which vary based on the specific circumstances of the interaction. Establishing privacy norms and privacy principles of CI is complex and indeed an open problem in the literature since norms are governed by societal contexts and can evolve in response to societal developments \cite{malkin2023contextual}. 

Instead, we draw inspiration from the CI theory to formalize the notion of contextual privacy, focusing on the user-LCA interaction.
We begin with characterizing the \textit{information flow} between a user and an LCA by drawing on the five essential CI parameters in Table \ref{table:ci_params}. We simplify the transmission principle based on the privacy directive \textit{share information that is essential to get the answer}, similar to \cite{bagdasaryan2024air}.

After we characterize the subject and the \textit{context} (which captures the user’s intent and the key task) from the user’s query along with the prior conversation history, we determine two types of sensitive attributes in the query: (a) details that are essential to answer the query, and (b) sensitive details that are not essential for answering the query. We say that a user query is \textit{contextually private} if it does not contain any nonessential sensitive attributes. An example of essential and non-essential attributes for a query is shown in Figure \ref{fig:framework}.










 






\begin{table*}[t]
  \small
  \renewcommand{\arraystretch}{1.2}
  \centering
  \caption{Entities associated with contextual integrity in conversational agents.}
  \resizebox{0.94\textwidth}{!}{
  \begin{tabular}{ p{2cm}  p{4cm}  p{8cm} }
    \toprule
    \textbf{CI Entity} & \textbf{Definition} & \textbf{Function/Considerations} \\ 
    \toprule
    \textbf{Sender (self)} & The user sending information to the agent to achieve a task. & Ensure the user shares only relevant and necessary information. \\ 
    \midrule
    \textbf{Subject} & The individual(s) about whom information is shared (self, others, or both). & Protect the privacy of the subject by identifying whether the subject is the user or another person. Information shared should respect the subject's privacy. \\ 
    \midrule
    \textbf{Receiver (agent)} & The agent that receives and processes information. & Treat agent as untrusted. Apply strict privacy controls to prevent oversharing. May be domain-specific (e.g., MedicalChat Assistant) or general-purpose (e.g., ChatGPT). \\ 
    \midrule
    \textbf{Context (data type)} & The broader domain or user intent (e.g., medical, finance, work-related) guiding the interaction. & Guides what information is relevant to share. In domain-specific apps, the context is predefined; in general-purpose apps, intent detection is used. Optionally, users may specify sensitive contexts. \\ 
    \midrule
    \textbf{Transmission Principle} & The rule governing the flow of information between sender and receiver. & Share only essential and relevant information for the task, avoiding unnecessary or sensitive information. Respect the privacy expectations defined by context and actors. \\ 
    \bottomrule
  \end{tabular}}
  \label{table:ci_params}
\end{table*}




\vspace{-5pt}
\section{A Framework for Safeguarding Contextual Privacy}
\label{sec:framework}
\vspace{-5pt}
Our goal is to develop a framework that acts as an intermediary between the user and LCA, and enables the user to detect whether their prompt incurs any contextual privacy violations, and judiciously reformulate the prompt to ensure contextual privacy. We first conduct a formative design study to guide our framework design.

\paragraph{User Study to Guide Our Framework Design:} 
We conducted a \textit{Wizard-of-Oz} formative user study to explore users' expectation of privacy when interacting with LCAs and to gather technical requirements for our framework.
Following established practices in early-stage interface design research \citep{ nielsen2000fiveusers, budiu2021fiveparticipants, nielsen1993mathematical}  where 5 participants are typically sufficient to identify major design insights, we conducted our study with six participants from our institution who were familiar with LLMs.
Using three mid-fidelity UX mockups (see Appendix \ref{sec::user_study_mockups}), we probed participants on their privacy concerns, reactions to privacy disclosures, and preferences for managing sensitive information. Each mockup simulated interactions where PII and sensitive information were detected and flagged. Participants provided feedback on different approaches to identifying, flagging, and reformulating sensitive information. 

Insights from this formative phase shaped several key design aspects of our framework, including distinguishing between essential and non-essential sensitive information, real-time feedback, user control over reformulations, and transparency around how sensitive information is handled and flagged. The participants rated the overall approach of the system highly, with a min and max rating of 7/10 and 9/10 respectively, providing initial validation for our approach to sensitive information detection and reformulation. For a detailed discussion of the study and how it impacted our design, see Appendix~\ref{sec::user_study}.
 
\noindent\textbf{Proposed Framework:}
We propose a framework that acts as an intermediary between the user and the conversation agent and enables the user to detect out-of-context sensitive information in the user prompt and judiciously reformulate the prompt to ensure contextual privacy.
The key components of the framework are outlined in Figure~\ref{fig:framework}.
When a user submits a prompt, our framework first determines the \textbf{context} and \textbf{subject} of the conversation. The context is divided into two components: the domain of the interaction (e.g., medical, legal, or financial) and the specific task the user aims to perform, such as seeking advice, requesting a translation, or summarizing a document. 
Context identification is guided by a taxonomy of common user tasks and sensitive contexts that go beyond PII \citep{mireshghallah2024trust} (see Appendix~\ref{domains_and_tasks}).







Once the context and subject are identified, our framework moves on to detecting sensitive information in the prompt. %
The framework categorizes the sensitive information into two spaces: (a) \textbf{essential information space}: sensitive details necessary to answer the user’s query, (b) \textbf{non-essential information space}: sensitive details that are unnecessary for answering the query and should be kept private.

In the example of Figure \ref{fig:framework}, the sensitive terms are \textit{``Jane'', ``single parent of two'', ``diabetes'', and ``affordable''}. While ``diabetes'' is essential for providing advice on treatment options, the other details---Jane's name, family situation, and financial concerns---are not required and thus classified as non-essential. 







Once contextually essential and non-essential information is identified, our framework improves contextual privacy by \textbf{reformulating} the prompt. This process includes removing, rephrasing, or redacting details within the non-essential information space, while preserving the user's intent. This way, we ensure that the user can still achieve the desired outcome effectively when the reformulated prompt is sent to the untrusted LCA. In our running example, a reformulated user's prompt could be \textit{``I need advice on managing a health condition and finding treatment options for diabetes''}, which protects non-essential sensitive details like the user’s name and personal circumstances, while maintaining the core intent of seeking treatment advice for diabetes.




After the reformulated prompt is generated, users can review, modify, or accept it, or revert to the original input. The review steps, shown by dashed boxes in Figure~\ref{fig:framework}, ensure user control, allowing them to achieve their desired balance between privacy and utility. The framework continues to highlight privacy implications as users adjust the suggested reformulation, helping them make informed choices about what information to share. Once finalized, the reformulated prompt is sent to the LLM-based conversational agent to obtain a response.



\section{Implementation and Evaluation}



\begin{figure}[htbp]
\centering
    \includegraphics[width=1.0\linewidth,keepaspectratio]{figures/evaluation.png}
    \caption{Experimental pipeline showing initial privacy screening, reformulation by three local models, and evaluation stages.}
    \label{fig:experiment}
\end{figure}

\subsection{Contextual Privacy Evaluation of Real-World Queries}
\label{sec:sharegot_privacy_evaluation}
Before implementing and evaluating our framework, we first perform initial privacy analysis by evaluating 
an open-source version of the ShareGPT dataset~\citep{vicuna2023} to understand the prevalence of contextual privacy violations. To instantiate our formal privacy definition, we used Llama-3.1-405B-Instruct \cite{grattafiori2024llama3herdmodels} as judge, with a prompt designed to identify violations of contextual integrity (Appendix \ref{appendix_ci_detection}). From over 90,000 conversations, we retain 11,305 single-turn conversations within a reasonable length range (25-2,500 words). For each conversation, the judge model assessed the context, sensitive information, and their necessity for task completion. This analysis identified approximately 8,000 conversations containing potential contextual integrity violations. To manage inference costs, we focused on cases where the judge model could successfully identify a primary context and classify essential and non-essential information attributes, yielding 2,849 conversations (25.2\%) with definitive contextual privacy violations. Examples of these violations are shown in Table \ref{tab:example_ci_violations}. Manual inspection of the judge's results for consistency and correctness demonstrated good classification performance with few false positives and negatives.

\subsection{Implementation Details}
\textbf{Models.} We implement our framework using a model that is significantly smaller than typical chat agents like ChatGPT, enabling users to deploy the model locally via Ollama\footnote{\url{https://github.com/ollama/ollama}} without relying on external APIs.
In our experiments, we evaluate three models with different characteristics: Mixtral-8x7B-Instruct-v0.1\footnote{\url{https://ollama.com/library/mixtral:8x7b-instruct-v0.1-q4\_0}} \cite{jiang2024mixtralexperts}, Llama-3.1-8B-Instruct\footnote{\url{https://ollama.com/library/llama3.1:8b-instruct-fp16}} \cite{grattafiori2024llama3herdmodels}, 
and DeepSeek-R1-Distill-Llama-8B\footnote{\url{https://ollama.com/library/deepseek-r1:8b-llama-distill-q4_K_M}} (focused on reasoning) \cite{deepseekai2025deepseekr1}. We refer to these models as Mixtral, Llama and Deepseek in short going forward.
The local deployment of models ensures no further privacy leakage due to the framework. Although our evaluation focuses on three LLMs, our approach is model-agnostic and can be applied to other architectures. For assessment of privacy and utility, we use Llama-3.1-405B-Instruct \cite{grattafiori2024llama3herdmodels} as an impartial judge, which was hosted in a secure cloud infrastructure.

\textbf{Experiment Setup.} 
As discribed in the previous section,
our framework processes user prompts in three stages: (a) context identification, (b) sensitive information classification, and (c) reformulation. The locally deployed model first determines the context of the conversation, identifying its domain and task (Appendix ~\ref{domains_and_tasks}) using the prompts in Appendix Appendix~\ref{appendix_intent_detection} and Appendix ~\ref{appendix_task_detection} respectively. It then detects sensitive information, categorizing it as either \textit{essential} (required for task completion) or \textit{non-essential} (privacy-sensitive and removable). Finally, if non-essential sensitive information is present, the model reformulates the prompt to improve privacy while preserving intent.

We implement two approaches for sensitive information classification: \textbf{dynamic classification}  and \textbf{structured classification}, each reflecting different ways to operationalize our privacy framework. In the \textbf{dynamic classification approach} (see prompt used in Appendix ~\ref{appendix_dynamic_sentive_template}),  the model determines which details are essential based on how they are used within the specific conversation. For instance, in the prompt \emph{"I’m Jane, a single parent of two, and was just diagnosed with diabetes. I’m looking for affordable treatment options"}, the model would identify the phrases= \emph{["diabetes"]} as the essential attributes, while \emph{["Jane", "single parent of two","affordable"]} would be classified as non-essential. This adaptive method aligns with contextual privacy formulation, ensuring that only task-relevant details are retained. In contrast, the \textbf{structured classification approach} (see prompt used in Appendix ~\ref{appendix_structured_sentive_template}), allows to specify a predefined list of sensitive attributes (e.g., age, SSN, physical health, allergies) that should always be considered non-essential (protected), ensuring consistent enforcement of privacy policies.  For the same example, this approach would flag \emph{["physical health"]} as the essential attribute while labeling \emph{["name", "family status", "financial condition"]} as non-essential attributes, recommending them for removal based on user-defined privacy preferences. This provides greater control over what information is considered sensitive, allowing customization while maintaining a standardized privacy framework. The predefined attribute categories follow those defined in \citet{bagdasaryan2024air}.

If non-essential sensitive details are detected, the model reformulates the prompt by either removing or rewording them to minimize privacy risks while maintaining usability (see Prompt used in Appendix \ref{appendix_reformulation}). By evaluating both dynamic and structured classification, we demonstrate the flexibility of our framework in balancing adaptability with user-defined privacy controls.

\subsection{Evaluation and Results}

We evaluate our framework by measuring two key metrics: \textbf{privacy gain} and \textbf{utility}. Privacy gain quantifies how effectively sensitive information is removed during reformulation, while utility measures how well the reformulated prompt maintains the original prompt's intent. We compute these metrics using two complementary methods: an automated BERTScore-based comparison of sensitive attributes, and an LLM-based assessment that aggregates multiple evaluation aspects.

\subsubsection{Evaluation via Attribute-based Metrics} 
\paragraph{Metrics.} 
We measure privacy gain by computing semantic similarity between non-essential attributes between original and reformulated prompts, where similarity is computed using BERTScore \citep{zhang2020bertscore}. 
Specifically, we first run the judge model on reformulated prompts to obtain non-essential sensitive attributes $\mathcal{P}^{\textrm{reform}}_{\textrm{non-ess}}$, using a prompt designed to identify contextual privacy violations (Appendix \ref{appendix_ci_detection}). We have non-essential sensitive attributes for original prompts $\mathcal{P}^{\textrm{orig}}_{\textrm{non-ess}}$ from Section \ref{sec:sharegot_privacy_evaluation}. 
Given sets of strings $\mathcal{P}^{\textrm{orig}}_{\textrm{non-ess}}$ and $\mathcal{P}^{\textrm{reform}}_{\textrm{non-ess}}$, 
privacy gain is computed as
$1 - \text{BERTScore}(\mathcal{P}^{\textrm{orig}}_{\textrm{non-ess}}, \mathcal{P}^{\textrm{reform}}_{\textrm{non-ess}})$, with a score of 1.0 assigned when either set is empty. A higher privacy gain indicates better removal of sensitive information. For utility, we measure semantic similarity between essential attributes using $\text{BERTScore}(\mathcal{P}^{\textrm{orig}}_{\textrm{ess}}, \mathcal{P}^{\textrm{reform}}_{\textrm{ess}})$, where a score closer to 1.0 indicates better preservation of task-critical information. Since BERTScore works on text pairs, we match each original attribute to its closest reformulated one and compute utility as the fraction of matched attributes above a similarity threshold of 0.5.


\begin{table}[t]
  \small
  \setlength{\tabcolsep}{4pt}
  \centering
  \caption{BERT-based Evaluation of Privacy and Utility}
  \begin{tabular}{lcc}
    \toprule
    \multicolumn{3}{c}{\textbf{Dynamic Attribute Classification}} \\ \midrule
    \textbf{Model} & \textbf{Privacy Gain $\uparrow$} & \textbf{Utility(BERTScore)$\uparrow$} \\ \midrule
    Deepseek  & 0.853 & 0.570 \\
    Llama     & 0.886 & 0.567 \\
    Mixtral   & 0.873 & 0.570 \\ \midrule
    \multicolumn{3}{c}{\textbf{Structured Attribute Classification}} \\ \midrule
    \textbf{Model} & \textbf{Privacy Gain $\uparrow$} & \textbf{Utility(BERTScore)$\uparrow$} \\ \midrule
    Deepseek  & 0.836 & 0.511 \\
    Llama     & 0.873 & 0.606 \\
    Mixtral   & 0.824 & 0.576 \\ \bottomrule
  \end{tabular}
  \label{tab:privacy_utility}
\end{table}

\paragraph{Results.} Table~\ref{tab:privacy_utility} shows that under dynamic classification, all three models achieve strong privacy scores (0.85-0.88) with comparable utility ($\sim0.57$), suggesting that the ability to identify context-specific sensitive information is robust across different model architectures.

The structured classification approach shows greater variation between models. While Llama achieves high scores in both privacy (0.873) and utility (0.606), structured classification generally yields slightly lower privacy scores but more variable utility. This suggests a natural trade-off: predefined categories might miss some context-specific sensitive information, yet operating within these fixed boundaries can help preserve task-relevant content. Interestingly, the similar performance patterns across different model architectures suggest that the choice between instruction-tuned and reasoning-focused approaches may be less crucial for privacy-preserving reformulation.

The success of both dynamic and structured approaches offers implementation flexibility - users can choose predefined privacy rules or context-specific protection based on their requirements. This choice, rather than model architecture, appears to be the key decision factor in deployment.


\subsubsection{LLM-as-a-Judge Assessment} 
\paragraph{Setup.}  We use Llama-3.1-405B-Instruct as a judge to provide a complementary evaluation of privacy and utility across 100 randomly selected queries per model (6×100 total). Given the high computational cost of LLM-based inference, this targeted sampling allows us to validate key trends observed in the attribute-based evaluation while minimizing overhead. 
Privacy gain is computed by asking the judge to evaluate privacy leakage, coverage, and retention, while utility 
is computed by measuring
query relevance, response validity, and cross-relevance. 
These binary evaluations are averaged to produce final privacy gains and utility scores. See Appendix \ref{appendix_evaluation} for detailed prompts and evaluation criteria.

\paragraph{Results.} The LLM-based assessment shows generally higher utility scores (0.82-0.86) across all models compared to BERTScore-based evaluation, while maintaining similar privacy levels (0.80-0.86). This difference can be attributed to how attributes are detected and compared—BERTScore evaluates exact semantic matches between attributes, while the LLM judge takes a more holistic view of information preservation. For instance, when essential information is restructured (e.g., ``my friend Mark'' split into separate attributes), BERTScore may indicate lower utility despite semantic equivalence.

The LLM evaluation confirms the effectiveness of both classification approaches, with dynamic classification showing slightly more consistent performance across models. Llama maintains its strong performance under both approaches (privacy gain: $\sim0.85$, utility score: $\sim0.86$), reinforcing its reliability for privacy-preserving reformulation.


\begin{table}[t]
\small
\centering
\caption{LLM-as-a-Judge Evaluation of Privacy and Utility}
\begin{tabular}{lcc}
\toprule
\textbf{Model} & \textbf{Privacy Gain $\uparrow$} & \textbf{Utility Score$\uparrow$} \\ 
\midrule
\multicolumn{3}{c}{\textbf{Dynamic Attribute Classification}} \\ 
\midrule
Deepseek  & 0.802 & 0.845 \\ 
Llama     & 0.858 & 0.861 \\ 
Mixtral   & 0.848 & 0.838 \\ 
\midrule
\multicolumn{3}{c}{\textbf{Structured Attribute Classification}} \\ 
\midrule
Deepseek  & 0.815 & 0.825 \\ 
Llama     & 0.855 & 0.858 \\ 
Mixtral   & 0.845 & 0.828 \\ 
\bottomrule
\end{tabular}
\label{tab:llm_judge_results}
\end{table}

\subsubsection{Example Reformulations and Trade-offs}
\paragraph{Setup.} Table~\ref{tab:privacy_utility_scores} presents a set of diverse example reformulations illustrating our framework’s ability to balance privacy and utility across different scenarios. These examples highlight both ideal cases—where reformulation effectively preserves both privacy and utility—and more challenging ones where trade-offs are unavoidable.

\paragraph{Results.} Our framework successfully removes personal identifiers while preserving task relevance, as seen in the third example ($\text{privacy gain} = 0.5,\text{utility score}= 0.83$). In creative requests like the Valentine’s poem (second example), removing personal details reduces privacy risks but slightly impacts personalization $\text{utility}=0.5$).

Some contexts resist reformulation. The last conversation ($\text{privacy gain}= 0.0, \text{utility score}= 0.0$) highlights cases where the entire prompt is inherently sensitive, requiring alternative privacy measures beyond text transformation. This is not a failure of our approach but an indicator of when reformulation alone is insufficient.

These examples reinforce that privacy-preserving reformulation is a trade-off, not a one-size-fits-all solution.

    




\begin{table*}[t]
  \small
  \scriptsize
\renewcommand{\arraystretch}{1.2}
  \centering
  \caption{Examples of original and reformulated prompts with privacy and utility scores. These examples illustrate how our framework removes sensitive details while preserving task intent. The high privacy, high utility cases demonstrate successful reformulation, while the low privacy, low utility example highlights cases where effective reformulation is difficult or unnecessary.}
  \label{tab:privacy_utility_scores}
  \resizebox{0.99\textwidth}{!}{
  \begin{tabular}{p{5.5cm}p{5.5cm}cc}
    \toprule
    \textbf{Original Prompt} & \textbf{Reformulated Prompt} & \textbf{Privacy Gain} & \textbf{Utility (BERTScore)} \\ \hline
    My friend Mark who was just laid off from Google, is looking for a job where he can use ML and Python. Do you have any advice for him? 
    & Someone is looking for a job where they can use ML and Python skills. Do you have any advice? 
    & \cellcolor{green!20} \textbf{1.0} 
    & \cellcolor{green!20} \textbf{1.0} \\ \hline

    Please write a poem for my wife for Valentine's Day. Her name is Gili. Please use rhymes. Please make it sarcastic, but not too much so. A bit funny but romantic.  
    & Write a poem for Gili, incorporating sarcasm at an appropriate level and maintaining a tone that is both funny and romantic  
    & \cellcolor{green!20} \textbf{1.0} 
    & \cellcolor{yellow!20} \textbf{0.5} \\ \hline

     I am writing a character letter to a judge in hopes of having charges against my friend Amber to be dropped. Write a letter in a positive tone to a judge speaking to Amber's good character ...... How you know me, my ties to the community my profession and stating a dismissal is imperative to my earning potential ... I know amber from "Teach For America". I know her to be a passionate educator who cares about uplifting the youth and her community. She should have the charges dropped so that she can increase her community engagement and earning potential for her family.  
    & Write a character letter in a positive tone to a judge speaking to someone's good character, highlighting their community ties and profession, and stating that a dismissal is imperative to their earning potential. I know this person from a professional program and they have been a passionate advocate for the betterment of their community. The charges should be dropped so that they can increase their engagement in the community and improve their financial situation. 
    & \cellcolor{yellow!20} \textbf{0.5} 
    & \cellcolor{green!10} \textbf{0.83} \\ \hline

    Sunny Balwani : I worked for 6 years day and night to help you. Elizabeth Holmes : I was just thinking about texting you in that minute by the way  
    & Sunny Balwani : I am responsible for everything at Theranos. Elizabeth Holmes : .........  
    & \cellcolor{red!20} \textbf{0.0} 
    & \cellcolor{red!20} \textbf{0.0} \\ 
    \bottomrule
  \end{tabular}
  }
\end{table*}



    
    
    
  


\section{Discussion and Conclusion}
\vspace{-5pt}
Drawing ideas from the contextual integrity theory, we defined the notion of contextual privacy for users interacting with LLM-based conversation agents.
We proposed a framework, grounded in our contextual privacy formulation, that acts as an intermediary between the user and the agent, and carefully reformulates user prompts to preserve contextual privacy while preserving the utility.  






This work serves as an initial step in exploring privacy protection in user interactions with conversational agents. There are several directions that future research can further investigate. 
First, our framework may not be suitable for user prompts that require preserving exact content, such as document translation or verbatim summarization. For example, translating a legal document demands keeping the original content intact, making it challenging to reformulate while preserving contextual privacy. For such tasks, alternative approaches like using placeholders or pseudonyms for sensitive information could help protect privacy without compromising accuracy, though this is beyond our current implementation. 
Second, our framework relies on LLM-based assessment of privacy violations which, while effective for demonstrating the approach, lacks formal privacy guarantees and can be sensitive to the prompt. Future work could explore combining our contextual approach with deterministic rules or provable privacy properties. 
Third, while we demonstrate how users can adjust reformulations to balance privacy and utility, developing precise metrics to quantify this trade-off remains an open research challenge. This is particularly important as the relationship between privacy preservation and task effectiveness can vary significantly across different contexts and user preferences. 
Finally, while our evaluation using selected ShareGPT conversations demonstrates the potential of our approach, broader testing across diverse contexts and user groups would better establish the framework's general applicability.










\section*{Limitations}
\vspace{-4pt}
Contextual integrity is a relatively new and fluid notion of privacy. Ours is also one of the very early works exploring this space from the standpoint of LLM-based conversational agents. Naturally, this leads to a number of challenges, some of which are beyond the scope of the work and should be addressed in the future. Like we discussed before, establishing privacy norms and principles in CI itself is complex and dependent on societal contexts, which is why we restrict ourselves to a practical and useful variation of the idea. However, developing templates for implementing CI under various societal contexts deserves significant attention from the research community in the future.

Our framework addresses critical privacy concerns in LLM interactions, potentially shaping future norms around data sharing in conversational AI. By enhancing user awareness and control over sensitive information, it promotes more ethical AI deployments, safeguarding user privacy in diverse applications such as healthcare, legal, and personal assistance. However, there are ethical challenges, such as ensuring fairness across cultural contexts and preventing over-reliance on automated privacy detection. 



\clearpage
\bibliography{custom}

\appendix
\section{Related Work}
\label{sec:relatedwork}
We fully contextualize our contributions in regard to existing literature here.

\paragraph{LLM Privacy-Preserving Techniques.}
A significant body of research on privacy preservation in LLMs has focused on the training phase \citep{zhang2024dpzero,chua2024mind,yu2021differentially,yue2022synthetic,li2021large}. Techniques like differential privacy (DP)~\citep{dwork2006calibrating} have been used to prevent LLMs from memorizing sensitive information during training. Additionally, data sanitization strategies, such as deduplication and anonymization, have been used to reduce privacy risks by removing sensitive data from training data~\citep{lison2021anonymisation, kandpal2022deduplicating}. After training, machine unlearning methods have emerged to help eliminate any retained private data~\citep{carlini2019secret, biderman2024emergent, mccoy2023much, zhang2023counterfactual, carlini2021extracting, nasr2023scalable, xu2024machine}. However, inference-phase privacy protection has received less attention, with limited approaches, such as PII detection and DP decoding, targeting the risks of exposing sensitive information in real-time interactions with LLMs~\citep{majmudar2022differentially,carey2024dp,wu2023privacy,tang2023privacy,hong2023dp,edemacu2024privacy}. Recently, \citet{mireshghallah2023can} highlighted this gap, showing that LLMs often fail to protect private information in context and emphasizing the need for better privacy-preserving techniques. Our approach addresses this need by offering real-time, context-aware privacy guidance during user interactions, allowing individuals to better manage what information they disclose during conversations with LLMs.


\paragraph{Privacy Risks in Human-LLM Interactions.}
Self-disclosure during human-machine interactions can result in unintended sharing of sensitive information. For example, \citet{ravichander2018empirical} found that users tend to reciprocate with automated systems, revealing more personal information over time. Building on this, \citet{zhang2024s} examined the privacy risks faced by users interacting with LLMs, showing that human-like responses can encourage sensitive disclosures, complicating privacy management. \citet{mireshghallah2024trust} further advanced this discussion by highlighting the limitations of PII detection systems, showing that users often disclose sensitive information that goes beyond PII~\citep{cummings2023challenges, dou2023reducing}. Our work builds on these efforts by showing that users frequently disclose unnecessary information during interactions with LLMs, which can be contextually sensitive and unrelated to their intended goals. We develop a system that detects such information and offers reformulation suggestions to guide users toward more privacy-aware interactions.

\paragraph{Data Minimization in ML.}
The principle of data minimization, central to privacy regulations like GDPR~\citep{voigt2017eu}, has recently been a key focus in ML research. For example, \citet{ganesh2024data} formalized data minimization within an optimization framework for reducing data collection while maintaining model performance. \citet{tran2024data} expanded on this by showing that individuals can disclose only a small subset of their features without compromising accuracy, thus minimizing the risk of data leakage. While both approaches focus on reducing the amount of data processed at inference time, our work applies data minimization in real time, guiding users to share only necessary information with LLMs. We integrate contextual integrity to ensure that the disclosed information aligns with the context of the conversation, ensuring GDPR compliance through a user-driven, context-aware approach.

\paragraph{Operationalizing Contextual Integrity (CI).}
Research on contextual privacy in LLMs is rapidly expanding. For instance, \citet{mireshghallah2023can} introduced a benchmark to evaluate the privacy reasoning abilities of LLMs at varying levels of complexity, while \citet{shvartzshnaider2024llm} proposed a comprehensive framework using CI to assess privacy norms encoded in LLMs across different models and datasets. CI has also been integrated into various practical systems to safeguard privacy across diverse domains. For example, \citet{shvartzshnaider2019vaccine} employed CI to detect privacy leaks in email communications, and \citet{kumar2020aquilis} applied CI to provide mobile users with real-time privacy risk alerts. In smart home ecosystems, \citet{malkin2022runtime,abdi2021privacy} used CI to analyze and enforce privacy norms. \citet{hartmann2024can} considered scenarios where a local model queries a larger remote model, leveraging CI to ensure only task-relevant data is shared. Similarly, \citet{bagdasaryan2024air} used CI to restrict AI assistants’ access to only the information necessary for a given task, and \citet{ghalebikesabi2024operationalizing} applied CI to ensure form-filling assistants follow contextual privacy norms when sharing user information. While these studies focus on aligning AI assistants' actions with privacy norms, our work shifts the perspective toward empowering privacy-conscious users. By integrating CI into our framework, we aim to educate users in real time about contextually sensitive disclosures and offer proactive guidance to help manage privacy risks. This user-centered approach not only protects sensitive information during AI interactions but also promotes long-term privacy awareness---an aspect often overlooked in system-oriented solutions.




\section{User Study to Guide System Design}
\label{sec::user_study}

To explore users' perceptions of privacy with LCAs and gather technical requirements for our framework, we conducted a Wizard-of-Oz formative user study with six participants from our institution who were generally familiar with LLMs. 

The study involved a 30-minute semi-structured interview where participants were presented with three mid-fidelity UX mockups, each designed to demonstrate different ways private and sensitive information could be detected and remediated (see Appendix~\ref{sec::user_study_mockups}). These mockups, featuring synthetic examples inspired by real-world patterns in the ShareGPT dataset, were created to expose participants to targeted privacy risks, such as unintentional PII and sensitive data disclosures. We used these mockups to probe participants' views on their own privacy practices, their thoughts about privacy disclosures, and their preferences for managing sensitive information in conversations. The study provided insights into people's views on the identification, flagging, and reformulation of sensitive data, shaping the core elements of our framework.



\begin{itemize}[leftmargin=1em]
    \item \textbf{Perceived privacy control}. Participants initially believed their efforts to protect their privacy when using real-world LLM applications were effective due to how they kept conversations vague. After they saw real examples of indirect privacy leaks in the mockups, many participants expressed greater concern about unintentionally sharing private information. \textbf{Design impact}: This insight emphasized the importance of identifying both direct and indirect privacy risks during LLM interactions in our system.
    \item \textbf{Visual identification of sensitive information}. Prototype B's color-coded differentiation between PII, necessary, and unnecessary information was praised for making privacy risks clearer and easier to understand. \textbf{Design impact}: Based on this feedback, we included the ability to differentiate between different kinds of sensitive information disclosures to help inform users' decision-making.
    \item \textbf{Reformulation preferences}. Although some participants preferred doing the work of reformulating their LLM prompts themselves, most wanted the system to offer (at least) one reformulated prompt suggestion, with the option to generate new suggestions. A few participants suggested offering multiple reformulations at once, selected across a spectrum of privacy-utility tradeoffs. In this way, users can balance their level of privacy protection with the utility of the output. \textbf{Design impact}: We designed our system to present one reformulation recommendation at a time, but with the flexibility to generate new alternative reformulations. In future iterations of our system, we plan to explore how to generate multiple reformulation options across varied privacy-utility tradeoffs.
    \item \textbf{User control and real-time feedback}. Real-time feedback and user control over editing flagged prompts were highly valued. Participants preferred having the system automatically generate reformulations, but they wanted the ability to make any necessary final adjustments. \textbf{Design impact}: We implemented a review step where users can edit, accept, or proceed with the original input before final submission to the LLM, providing the flexibility users requested.
    \item \textbf{Positive reception}. Participants responded positively to the system’s potential for managing sensitive information, with an average rating of $8.7 (\pm 0.87)$ on the importance of detecting and flagging sensitive details. \textbf{Design impact}. This feedback reinforced the central role of sensitive information detection in our framework, highlighting its perceived value to users.
    \item \textbf{Clarity and transparency}. Participants expressed a strong desire for transparency about how the system operates, including which tools or models are being used, and the meaning of key terms like ``necessary'' versus ``unnecessary'' information. \textbf{Design impact}: Our framework ensures transparency by detailing how sensitive information is identified and handled, including the models used, how they are applied, deployed, and how data is managed. We recommend real-world implementations do the same to build user trust.
    \item \textbf{Broader application}. A few participants suggested applying the tool to other contexts beyond LLM chat interfaces, such as search engines. \textbf{Design impact}:  This feedback highlights the importance of managing sensitive information and the broader applicability of our approach to other contexts.
\end{itemize}


\subsection{User Study Mockups}
\label{sec::user_study_mockups}
\begin{figure*}[t]
    \centering
    
    \begin{minipage}{0.8\textwidth} %
        \centering
        \includegraphics[width=\linewidth]{figures/user_study_1.png}
        \subcaption{Examples of unintentional disclosures shown to participants}
    \end{minipage}

    \vspace{0.5cm} %

    \begin{minipage}{0.45\textwidth}
        \centering
        \includegraphics[width=\linewidth]{figures/user_study_2.png}
        \subcaption{Mockup 1: Display all detected sensitive info}
    \end{minipage}
    \hfill
    \begin{minipage}{0.52\textwidth}
        \centering
        \includegraphics[width=\linewidth]{figures/user_study_4.png}
        \subcaption{Mockup 3: Rewrite the user's message for them}
    \end{minipage}

    \vspace{0.5cm} %

    \begin{minipage}{0.8\textwidth}
        \centering
        \includegraphics[width=\linewidth]{figures/user_study_3.png}
        \subcaption{Mockup 2: Color Code information and suggest reformulations}
    \end{minipage}

    \label{fig:mockups}
\end{figure*}

\onecolumn
\section{Domains and Tasks}
\label{domains_and_tasks}
Table~\ref{tab:categories} shows the list of Domain and Tasks Categories for Intent Detection.
\begin{table*}[ht]
\centering
\small
\begin{tabular}{|p{3.5cm}|p{10cm}|}
\hline
\textbf{Domain} & \textbf{Description} \\ \hline
Health\_And\_Wellness & Conversations related to physical and mental health, such as medical conditions, history, treatment plans, medications, healthcare provider information, symptoms, diagnoses, appointments, health-related advice, mental health status, therapy details, counseling information, emotional well-being, fitness routines, nutrition, dietary preferences, meal plans, health-related diets, feelings, coping mechanisms, mental health support, and emotional support systems. \\ \hline
Financial\_And\_Corporate & Conversations involving financial and corporate matters such as bank account details, credit card information, transaction histories, investment information, loan details, financial planning, budgeting, banking activities, insurance policies, claims, coverage details, premium information, business transactions, corporate policies, financial reports, investment strategies, stock market discussions, and company performance. \\ \hline
Employment\_And\_Applications & Conversations about employment and related applications, such as job status, job applications, resumes, workplace incidents, employer information, job roles, professional experiences, salaries, benefits, employment contracts, visa applications, and other types of applications including application processes, requirements, status updates, supporting documents, interviews, and follow-up actions. \\ \hline
Academic\_And\_Education & Conversations related to academic and educational topics, including school or university details, grades, transcripts, educational history, academic achievements, courses, assignments, educational resources, learning resources, teaching methods, and extracurricular activities. \\ \hline
Legal & Conversations involving legal matters such as legal advice, court cases, contracts, legal documents, criminal records, discussions about laws and regulations, tax information, social security numbers, government benefits, applications, and interactions with legal professionals or government agencies. \\ \hline
Personal\_Relationships & Conversations about personal relationships, such as family details, marital status, friendships, romantic relationships, social interactions, personal issues, relationship problems, private social events, and childcare arrangements. \\ \hline
Travel & Conversations related to travel and transportation, including travel plans, itineraries, booking details, passport and visa information, travel insurance, destinations, accommodations, transportation options, vehicle details, driver's license information, and travel routes. \\ \hline
Hobbies\_And\_Habits & Conversations about personal hobbies and habits, such as leisure activities, crafting, gaming, sports, collecting, gardening, reading, writing, and other regular personal interests and practices. \\ \hline
Sexual\_And\_Erotic & Conversations involving sexual and erotic content, including sexual preferences, activities, experiences, relationships, fantasies, sexual health, and explicit discussions about sex. \\ \hline
Politics & Conversations involving political topics, including discussions about political opinions, political events, government policies, political parties, elections, civic participation, and political ideologies. \\ \hline
Religion & Conversations related to religious beliefs and practices, including discussions about faith, religious events, spiritual experiences, religious teachings, places of worship, religious communities, and religious holidays. \\ \hline
\end{tabular}
\caption{List of domains, tasks, and their corresponding descriptions used by the model intent detection}
\label{tab:categories}
\end{table*}

\section{Prompts to generate \synqa synthetic training data}\label{app:prompts}
In order to generate the question-answer pairs, we provide Llama 70B with the following prompts:

\begin{prompt}
\textbf{SYSTEM PROMPT}

You are tasked with generating a concise and focused question-answer pair using information from provided Wikipedia sentences. Follow these instructions carefully:


1. You will be provided with multiple Wikipedia articles, each containing:

   - The title of the article.
   
   - One specific sentence from the article.


2. Your goal is to generate a \textbf{short, factual question} and a \textbf{concise answer}, ensuring:

   - The question-answer pair is grounded in the provided sentences.
   
   - The reasoning is logical, clear, and references all sentences used.


3. \textbf{Key Constraints}:

   - Questions must address a \textbf{single coherent topic} or concept that can be logically inferred from the provided sentences.
     
   - Avoid combining unrelated pieces of information into a single question.
   
   - The \texttt{"reasoning"} must explain how each sentence in the \texttt{"ids"} field contributes  to answering the question but should remain \textbf{brief} and \textbf{to the point}.


4. Aim for \textbf{brevity}:

   - Questions should be concise and avoid unnecessary details.
   
   - Answers should be short, typically no more than one sentence.
   
   - Keep the reasoning concise, focusing only on the necessary logical connections.


5. Multi-hop reasoning is encouraged but must be natural and focused:

   - Combine information only when it is logical and directly relevant to the question.
   
   - Do not create overly complex questions that combine weakly related information.


6. Provide your response in \textbf{raw JSON format} with the following keys:

   - \texttt{"question"}: A concise and clear question string.
   
   - \texttt{"answer"}: A short and factual answer string.
   
   - \texttt{"ids"}: A list of JSON-compatible arrays (e.g., \texttt{[[0, 0], [1, 0]]}) representing the indices of all sentences used to generate the question-answer pair.
   
   - \texttt{"reasoning"}: A brief explanation of how \textbf{each sentence in \texttt{"ids"}} was used to 
     generate the question-answer pair.


\textbf{Important Notes}:

- Ensure the question-answer pair is entirely self-contained and logically consistent.

- Do not include unnecessary or weakly related information in the question or answer.

- Avoid introducing information not present in the provided sentences.

- Do not include additional formatting, explanations, or markdown in your response.
\end{prompt}

\begin{prompt}
\textbf{USER PROMPT}

Here are the titles and sentences:

Title: [First Article Title]

[0, 0] [First sentence from the article]

Title: [Second Article Title]

[1, 0] [Second sentence from the article]

Title: [Third Article Title]

[2, 0] [Third sentence from the article]

Use the provided sentences to generate a question-answer pair following the specified guidelines. Respond \textbf{only in raw JSON} with no additional formatting or markdown.
\end{prompt}

Given the \textbf{SYSTEM} and the \textbf{USER} prompt, the LLM is generating the question-answer pair, which when combined with the full articles, yields a single \synqa training data sample.

Should we want to generate a \synqa dialog training data sample, we make the prompts a bit simpler:

\begin{prompt}
\textbf{SYSTEM PROMPT}

You are an AI assistant that generates structured question-answer pairs based on a passage. Your goal is to create meaningful, factual, and reasoning-based questions that require connecting multiple sentences.

Follow these strict guidelines:

- Format the output as a \textbf{valid JSON array}, where each item has:

  - \texttt{"question"}: A clear, concise question.
  
  - \texttt{"answer"}: A short, factual response.
  
  - \texttt{"sentence\_numbers"}: A list of integers pointing to \textbf{all} relevant supporting sentences.

- \textbf{Ensure questions are generated in a random sentence order} (not sequential).

- Some questions \textbf{must reference multiple sentences} for reasoning.

- Some sentences should be \textbf{reused} across multiple questions.

- \textbf{Later questions should rely on earlier information} and use pronouns or indirect references to maintain logical flow.

- Introduce a mix of \textbf{fact-based, causal, and inference questions}.

- Avoid introducing \textbf{information not present in the passage}.

- Ensure \textbf{all relevant sentences are cited} for each answer.

Your response must be \textbf{valid JSON} containing 5 to 10 question-answer pairs.
\end{prompt}

and the user prompt:

\begin{prompt}
\textbf{USER PROMPT}

Here is a passage:

Title: [Title of the passage]

0. [First sentence of the passage]

1. [Second sentence of the passage]

2. [Third sentence of the passage]

3. [Fourth sentence of the passage]

...

Generate structured question-answer pairs following these constraints:

- \textbf{Return output in JSON format only}: \texttt{[{"question": "...", "answer": "...", "sentence\_numbers": [..]}, ...]}

- Use \textbf{random sentence order}, not sequential.

- Some questions should require \textbf{multiple sentences}.

- Some sentences should be \textbf{reused} across different Q\&A pairs.

- \textbf{Later questions must reference earlier ones} using pronouns or indirect mentions.

- \textbf{Include a mix of question types}:

  - Factual questions that can be answered directly from the passage.
  
  - Causal questions that require understanding relationships between sentences.
  
  - Inference-based questions that require implicit reasoning.

- Ensure \textbf{sentence numbers fully cover the reasoning required}.

Return \textbf{only} JSON, with no extra text.
\end{prompt}
\section{User Prompts Before And After Reformulation }
\label{app:before_and_after_reformulation}


\begin{table*}[t]
  \small
  \centering
  \caption{Qualitative Comparison: Dynamic vs. Static Reformulations (Llama).}
  \begin{tabular}{p{0.32\textwidth} p{0.32\textwidth} p{0.32\textwidth}}
    \toprule
    \textbf{Original Text} & \textbf{Dynamic Reformulated Text} & \textbf{Static Reformulated Text} \\
    \midrule
    As I listen to our weekly Friday sermon (Muslim), I find myself distracted. I want to create a journal template, for me to fill out as I listen to help me listen better. Can you suggest some ideas for the template? 
    & Can you suggest some ideas for creating a journal template that I can use while listening to our weekly Friday sermon, to help me focus and take notes? 
    & Can you suggest some ideas for a journal template that I can use while listening to sermons, specifically for improving focus? \\[1ex]
    \midrule
    Write a LinkedIn post saying congratulations to Monique Rodriguez in forging this major partnership with P\&G and talk about the importance of celebrating and recognizing black businesses.
    & Can you help me write a LinkedIn post about celebrating and recognizing black businesses? 
    & Write a LinkedIn post saying congratulations to a business owner in forging this major partnership with P\&G and talk about the importance of celebrating and recognizing black businesses. \\[1ex]
    \midrule
    I want to do a road trip from Pune to Gujarat. My wife and I will be travelling and we don’t prefer very long driving sessions. Can you suggest a plan starting from Thursday early morning and ending in Pune on Sunday late night?
    & Can you suggest a road trip plan from Pune to Gujarat that avoids very long driving sessions? 
    & Can you suggest a travel plan for a road trip from Pune to Gujarat, taking into account that two people will be driving and it’s preferable to have shorter driving sessions, starting from Thursday early morning and ending in Pune on Sunday late night? \\[1ex]
    \midrule
    I am a 23-year-old female living in Pakistan. I am moving to Germany to pursue higher education. What should I take with me?
    & What should I take with me when moving to Germany for higher education? 
    & What should I take with me when moving abroad for higher education? \\[1ex]
    \bottomrule
  \end{tabular}
  \label{tab:dynamic_vs_static}
\end{table*}


\begin{table*}[t]
  \small
  \centering
  \caption{Qualitative Comparison of Dynamic Reformulations.}
  \resizebox{\textwidth}{!}{%
  \begin{tabular}{p{0.30\textwidth} p{0.30\textwidth} p{0.30\textwidth} p{0.30\textwidth}}
    \toprule
    \textbf{Original Text} & \textbf{Dynamic Deepseek Reformulated} & \textbf{Dynamic LLaMA Reformulated} & \textbf{Dynamic Mixtral Reformulated} \\
    \midrule
    write a LinkedIn post saying Congratulations to Monique Rodriguez in forging this major partnership with P\&G and talk about the importance of celebrating and recognzing black businesses. 
    & Congratulations on forging a major partnership with P\&G and highlighting the importance of celebrating and recognizing black businesses. 
    & Can you help me write a LinkedIn post about celebrating and recognizing black businesses? 
    & ``I'd like to create a social media post about the importance of celebrating and recognizing businesses from underrepresented communities.'' \\[1ex]
    \midrule
    Write an email to Ashley Davies expressing that we were so taken by his reaction to our VR experience that I had to look him up and found his email address on his Autonomous Sensorweb Operations pdf. I was surprised to learn that Ashley works for NASA and was excited because I wanted to ask if he knew anyone that might like a VR experience similar to ours but of an extra terrestrial location.
    & We were very impressed by Ashley Davies' reaction to our VR experience that I looked him up on a document and found his email address. I was surprised to learn that he works for NASA and would love to ask if he knows anyone interested in a VR experience of an extraterrestrial location.
    & Write an email expressing interest in connecting with someone who might be interested in a VR experience of an extra terrestrial location. I was excited to learn that this person works for a space-related organization and would like to ask if they know anyone who might be interested in our type of experience.
    & Writing an email to Ashley Davies, who works for NASA, to inquire if they know anyone interested in a VR experience of an extra terrestrial location. \\[1ex]
    \midrule
    I'm about to have my step son soon and it is going to be another mouth to feed and a lot more expenses each month, which will leave no room if even enough to survive is there a way to get an easy behind the scenes income that can pull the slack of the extra bills and extra food i have to bring in?
    & ``extra monthly expenses are making it difficult to manage bills and food costs. I'm looking for ways to generate behind-the-scenes income to help cover these additional expenses.'' 
    & I'm about to become a step-parent and will have additional financial expenses due to the new family member, which will leave no room for extra bills and food. Is there a way to get an income that can help pull in the slack of these extra expenses?
    & I'm seeking advice on how to generate additional income due to increased monthly expenses from supporting another person. \\
    \bottomrule
  \end{tabular}%
  }
  \label{tab:dynamic_model_comparison}
\end{table*}




\end{document}
