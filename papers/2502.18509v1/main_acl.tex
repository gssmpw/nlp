\pdfoutput=1

\documentclass[11pt]{article}

\usepackage[final]{acl}

\usepackage{times}
\usepackage{latexsym}
\usepackage{listings}
\lstset{breaklines=true}

\usepackage[T1]{fontenc}

\usepackage[utf8]{inputenc}

\usepackage{microtype}

\usepackage{inconsolata}

%%%%% NEW MATH DEFINITIONS %%%%%

% \usepackage{amsmath,amsfonts,bm}
\usepackage{amsmath,amsfonts}

\usepackage{pifont}


\newcommand{\R}{\mathbb{R}}


\def\va{{\mathbf{a}}}
\def\vg{{\mathbf{g}}}

% Sets
\def\sR{\mathbb{R}}
\def\sC{\mathbb{C}}
\def\sZ{\mathbb{Z}}
\def\sN{\mathbb{N}}
\def\sQ{\mathbb{Q}}

\def\sS{\mathcal{S}}



% Vectors
\def\vzero{{\mathbf{0}}}
\def\vone{{\mathbf{1}}}
\def\vmu{{\mathbf{\mu}}}
\def\vtheta{{\mathbf{\theta}}}
\def\va{{\mathbf{a}}}
\def\vb{{\mathbf{b}}}
\def\vc{{\mathbf{c}}}
\def\vd{{\mathbf{d}}}
\def\ve{{\mathbf{e}}}
\def\vf{{\mathbf{f}}}
\def\vg{{\mathbf{g}}}
\def\vh{{\mathbf{h}}}
\def\vi{{\mathbf{i}}}
\def\vj{{\mathbf{j}}}
\def\vk{{\mathbf{k}}}
\def\vl{{\mathbf{l}}}
\def\vm{{\mathbf{m}}}
\def\vn{{\mathbf{n}}}
\def\vo{{\mathbf{o}}}
\def\vp{{\mathbf{p}}}
\def\vq{{\mathbf{q}}}
\def\vr{{\mathbf{r}}}
\def\vs{{\mathbf{s}}}
\def\vt{{\mathbf{t}}}
\def\vu{{\mathbf{u}}}
\def\vv{{\mathbf{v}}}
\def\vw{{\mathbf{w}}}
\def\vx{{\mathbf{x}}}
\def\vy{{\mathbf{y}}}
\def\vz{{\mathbf{z}}}
\def\vzeta{{\mathbf{\zeta}}}

% Matrix
\def\mA{{\mathbf{A}}}
\def\mB{{\mathbf{B}}}
\def\mC{{\mathbf{C}}}
\def\mD{{\mathbf{D}}}
\def\mE{{\mathbf{E}}}
\def\mF{{\mathbf{F}}}
\def\mG{{\mathbf{G}}}
\def\mH{{\mathbf{H}}}
\def\mI{{\mathbf{I}}}
\def\mJ{{\mathbf{J}}}
\def\mK{{\mathbf{K}}}
\def\mL{{\mathbf{L}}}
\def\mM{{\mathbf{M}}}
\def\mN{{\mathbf{N}}}
\def\mO{{\mathbf{O}}}
\def\mP{{\mathbf{P}}}
\def\mQ{{\mathbf{Q}}}
\def\mR{{\mathbf{R}}}
\def\mS{{\mathbf{S}}}
\def\mT{{\mathbf{T}}}
\def\mU{{\mathbf{U}}}
\def\mV{{\mathbf{V}}}
\def\mW{{\mathbf{W}}}
\def\mX{{\mathbf{X}}}
\def\mY{{\mathbf{Y}}}
\def\mZ{{\mathbf{Z}}}
\def\mBeta{{\mathbf{\beta}}}
\def\mPhi{{\mathbf{\Phi}}}
\def\mLambda{{\mathbf{\Lambda}}}
\def\mSigma{{\mathbf{\Sigma}}}


% Expectation
% \def\eE{\mathop{\mathbb{E}}\limits}
\def\eE{\mathbb{E}}

% Probability
\def\pP{\mathbb{P}}

% Tilde
\def\tf{\tilde{f}}
\def\tS{\tilde{S}}
\def\wtF{\widetilde{\mathcal{F}}}
\def\whR{\widehat{R}}
\def\tvx{\tilde{\mathbf{x}}}
\def\ty{\tilde{y}}


\def\defeq{\overset{\textup{def}}{=}}
% \def\defeq{\overset{.}{=}}
\def\defone{\overset{\text{\ding{172}}}{=}}
\def\deftwo{\overset{\text{\ding{173}}}{=}}
\def\leqone{\overset{\text{\ding{172}}}{\leq}}
\def\leqtwo{\overset{\text{\ding{173}}}{\leq}}
\def\leqthree{\overset{\text{\ding{174}}}{\leq}}
\def\leqfour{\overset{\text{\ding{175}}}{\leq}}
\def\eqone{\overset{\text{\ding{172}}}{=}}
\def\eqtwo{\overset{\text{\ding{173}}}{=}}
\def\eqthree{\overset{\text{\ding{174}}}{=}}
\def\eqfour{\overset{\text{\ding{175}}}{=}}
\def\geqfive{\overset{\text{\ding{176}}}{\geq}}
\usepackage{hyperref}
\usepackage{url}
\usepackage{graphicx}
\usepackage{float}
\usepackage{enumitem}
\newtheorem{definition}{Definition}
\usepackage{booktabs}
\usepackage{wrapfig}
\usepackage{subcaption}
\usepackage{microtype}

\usepackage{xcolor}
\usepackage{colortbl}
\usepackage{multirow}
\usepackage{tcolorbox}

\hypersetup{
    colorlinks,
    linkcolor={blue},%
    citecolor={green!60!black},
    urlcolor={blue}%
}

\newcommand{\ivy}[1]{{\textcolor{teal}{[Ivy: #1]}}{}}
\newcommand{\swanand}[1]{{\textcolor{blue}{[Swanand: #1]}}{}}
\newcommand{\ad}[1]{{\textcolor{orange}{[Amit: #1]}}{}}
\newcommand{\hao}[1]{{\color{purple}#1}}
\newcommand{\nrk}[1]{{\textcolor{magenta}{[NRK: #1]}}{}}
\newcommand{\new}[1]{{\color{red}#1}}


\title{Protecting Users From Themselves:\\ Safeguarding Contextual Privacy in Interactions with Conversational Agents}

\author{Ivoline C. Ngong\thanks{Graduate student at University of Vermont. Work done during summer internship at IBM Research.},~Swanand Kadhe, Hao Wang, Keerthiram Murugesan, Justin D. Weisz,\\ 
\textbf{Amit Dhurandhar, Karthikeyan Natesan Ramamurthy} \\
IBM Research. \\
\texttt{kngongiv@uvm.edu,}\\
\texttt{\{swanand.kadhe,keerthiram.murugesan\}@ibm.com,}\\
\texttt{hao-wang@redhat.com},
\texttt{\{jweisz,adhuran,knatesa\}@us.ibm.com}
}


\begin{document}
\maketitle
\begin{abstract}
Conversational agents are increasingly woven into individuals' personal lives, yet users often underestimate the privacy risks involved. The moment users share information with these agents (e.g., LLMs), their private information becomes vulnerable to exposure. In this paper, we characterize the notion of contextual privacy for user interactions with LLMs. It aims to minimize privacy risks by ensuring that users (sender) disclose only information that is both relevant and necessary for achieving their intended goals when interacting with LLMs (untrusted receivers). Through a formative design user study, we observe how even ``privacy-conscious'' users inadvertently reveal sensitive information through indirect disclosures. 
Based on insights from this study, 
we propose a locally-deployable framework that operates between users and LLMs, and identifies and reformulates out-of-context information in user prompts. Our evaluation using examples from ShareGPT shows that lightweight models can effectively implement this framework, achieving strong gains in contextual privacy while preserving the user's intended interaction goals through different approaches to classify information relevant to the intended goals. 
\end{abstract}


\section{Introduction}
\label{sec:intro}


\ps{Challenges of technology scaling}

The growing demand for computing performance has always been met by increasing the number of transistors per chip, which is only possible due to CMOS technology scaling.
However, as we keep pushing the boundaries of technology scaling, we encounter multiple challenges.
Firstly, whenever we transition to a more advanced technology node, the non-recurring cost due to physical design, verification, software, mask sets, and prototyping almost doubles \cite{cost-tech-node}.
As a result, designing a chip in an advanced technology node is only economically viable if the chip is manufactured in vast quantities.
Secondly, many chip components such as I/O drivers, analog circuits, or \gls{srams} have reached their scaling limit.
This means that we cannot shrink these components further, even if we use a more advanced technology with a smaller feature size.
Thirdly, advanced technology nodes suffer from high defect rates, diminishing the yield and inflating the recurring cost.
To tackle these challenges, new chip-design paradigms have been developed.

\ps{Why 2.5D integration?}

One of these new paradigms is 2.5D integration, where multiple silicon dies called chiplets are integrated into the same package.
Once designed, a single chiplet can be reused in multiple 2.5D stacked chips, which increases the ratio of production volume to non-recurring cost.
Another advantage is that multiple chiplets - fabricated in different technologies - can be integrated into the same package.
This means that only components that can take full advantage of technology scaling are built in bleeding-edge technologies.
Components that have reached their scaling limit are fabricated in more mature and hence less costly technology nodes.
Furthermore, chiplets are smaller than monolithic chips.
Therefore, manufacturing chiplets results in less silicon area loss due to fabrication defects and hence a higher yield.
Due to these economic advantages, chip vendors such as AMD \cite{amd-chiplet} and NVIDIA \cite{chiplet-book} have adopted the 2.5D integration paradigm.  

\ps{Challenges of 2.5D integration}

An important challenge when designing 2.5D stacked chips is the construction of a low-latency and high-throughput \gls{ici}. 
To build an \gls{ici}, we connect different chiplets using \gls{d2d} links.
These links are fabricated in an organic package substrate, silicon bridge, or silicon interposer, and they are connected to the chiplets using \gls{c4} bumps or microbumps.
The number of bumps per chiplet is limited, and so is the bandwidth of \gls{d2d} links.
In addition to having lower bandwidth than links in monolithic chips, \gls{d2d} links also have higher latency.
This latency is caused by wire delay and by \gls{phys} that are necessary in both the sending and the receiving chiplet.
\gls{phys} are needed to convert between protocols, voltage levels, and frequencies, which are usually different between on-chiplet links and \gls{d2d} links.
Due to these limitations, the \gls{ici} can quickly become a bottleneck.

\ps{How we solve these challenges differently than the related work does.}

Existing approaches to maximize the performance of the \gls{ici} either optimize the placement of chiplets (with potentially heterogeneous shapes) for a predetermined \gls{ici} topology 
\cite{ho,liu,seemuth,eris,osmolovskyi,tap25d,chiou}, select one topology out of a set of candidates \cite{coskun-1, coskun-2}, or they optimize the \gls{ici} topology for a 2D grid of homogeneously shaped chiplets on an active interposer \cite{butterdonut, cluscross, kite}.
To the best of our knowledge, there is no prior work on \gls{ici} topologies for chips with heterogeneously shaped chiplets or with passive silicon interposers or silicon bridges.
To fill this gap, we propose \name, a novel optimization methodology to jointly optimize the chiplet placement and \gls{ici} topology of such architectures.
\ifnb
\else
\newpage
\fi

\ps{Details on \name~and the key idea}

The key idea is as follows: 
We optimize the chiplet placement without a predetermined topology.
For each placement generated by an optimization algorithm, we infer a placement-based \gls{ici} topology by connecting chiplets that are in close proximity in that specific placement.
We then compute the latency and throughput of this combination of placement and topology for different traffic types.
These latencies and throughputs together with the total chip area are used to compute a user-defined quality-score of the placement, which is returned to the optimization algorithm.
Based on this quality score, the algorithm can further optimize the placement.
By following this iterative process, we jointly optimize the chiplet placement and the \gls{ici} topology.

\ps{Short evaluation-summary}

We provide our open-source framework implementing the proposed placement and topology co-optimization methodology, which we evaluate using both synthetic traffic and traffic traces.
A 2D grid of chiplets with a mesh topology is used as a baseline since many proposals for 2.5D stacked chips \cite{dataflow_accel_dnn, cifher, simba, hecaton, dojo} use such an architecture.
We reduce the latency of synthetic L1-to-L2 and L2-to-memory traffic, the two most important traffic types for cache coherency traffic, by up to 28\% and 62\% respectively.
For real traffic traces, we reduce the average packet latency for almost all traces and architectures considered (reduced by an 8\% or 18\% on average depending on the configuration of \gls{phys} within a chiplet).

\section{Threat Models and Privacy Definition}
\label{sec::threat_privacy}



\noindent\textbf{Threat Model.} 
We consider a scenario where users interact with large, remote, and untrusted LCAs through APIs. These can be web-based or hosted on cloud-based services or private networks and may be either general-purpose or domain-specific. Users often share personal, financial, or medical information without clear knowledge of how their data is managed, increasing privacy risks due to the lack of transparency around these agents. 


We focus on a threat model where users unintentionally compromise their privacy by oversharing information. Our approach targets out-of-context \emph{self-disclosure} by guiding users to share only contextually necessary information. By identifying unnecessary or sensitive disclosures in real time, we assist users in controlling the information they reveal, thereby reducing the risk of unintentional privacy breaches.
Our approach indirectly mitigates the threat of \emph{malicious users}, who seek to extract sensitive information from the agents by manipulating their interactions, by minimizing the amount of sensitive information exchanged during interactions.






\noindent\textbf{Contextual Privacy in Conversational Agents} 
We define the notion of \textit{contextual privacy} in conversational agents, inspired by the Contextual Integrity (CI) theory. 
CI models privacy as information flow defined by the five parameters sender (who is sharing the data), subject (who the information is about), receiver (who is getting the data), context (what sort of information is being shared), and transmission principle (the conditions under which information flow is conducted) \cite{nissenbaum2004privacy}. CI evaluates whether the information flow adheres to appropriate standards governed by norms, which vary based on the specific circumstances of the interaction. Establishing privacy norms and privacy principles of CI is complex and indeed an open problem in the literature since norms are governed by societal contexts and can evolve in response to societal developments \cite{malkin2023contextual}. 

Instead, we draw inspiration from the CI theory to formalize the notion of contextual privacy, focusing on the user-LCA interaction.
We begin with characterizing the \textit{information flow} between a user and an LCA by drawing on the five essential CI parameters in Table \ref{table:ci_params}. We simplify the transmission principle based on the privacy directive \textit{share information that is essential to get the answer}, similar to \cite{bagdasaryan2024air}.

After we characterize the subject and the \textit{context} (which captures the user’s intent and the key task) from the user’s query along with the prior conversation history, we determine two types of sensitive attributes in the query: (a) details that are essential to answer the query, and (b) sensitive details that are not essential for answering the query. We say that a user query is \textit{contextually private} if it does not contain any nonessential sensitive attributes. An example of essential and non-essential attributes for a query is shown in Figure \ref{fig:framework}.










 






\begin{table*}[t]
  \small
  \renewcommand{\arraystretch}{1.2}
  \centering
  \caption{Entities associated with contextual integrity in conversational agents.}
  \resizebox{0.94\textwidth}{!}{
  \begin{tabular}{ p{2cm}  p{4cm}  p{8cm} }
    \toprule
    \textbf{CI Entity} & \textbf{Definition} & \textbf{Function/Considerations} \\ 
    \toprule
    \textbf{Sender (self)} & The user sending information to the agent to achieve a task. & Ensure the user shares only relevant and necessary information. \\ 
    \midrule
    \textbf{Subject} & The individual(s) about whom information is shared (self, others, or both). & Protect the privacy of the subject by identifying whether the subject is the user or another person. Information shared should respect the subject's privacy. \\ 
    \midrule
    \textbf{Receiver (agent)} & The agent that receives and processes information. & Treat agent as untrusted. Apply strict privacy controls to prevent oversharing. May be domain-specific (e.g., MedicalChat Assistant) or general-purpose (e.g., ChatGPT). \\ 
    \midrule
    \textbf{Context (data type)} & The broader domain or user intent (e.g., medical, finance, work-related) guiding the interaction. & Guides what information is relevant to share. In domain-specific apps, the context is predefined; in general-purpose apps, intent detection is used. Optionally, users may specify sensitive contexts. \\ 
    \midrule
    \textbf{Transmission Principle} & The rule governing the flow of information between sender and receiver. & Share only essential and relevant information for the task, avoiding unnecessary or sensitive information. Respect the privacy expectations defined by context and actors. \\ 
    \bottomrule
  \end{tabular}}
  \label{table:ci_params}
\end{table*}




\vspace{-5pt}
\section{A Framework for Safeguarding Contextual Privacy}
\label{sec:framework}
\vspace{-5pt}
Our goal is to develop a framework that acts as an intermediary between the user and LCA, and enables the user to detect whether their prompt incurs any contextual privacy violations, and judiciously reformulate the prompt to ensure contextual privacy. We first conduct a formative design study to guide our framework design.

\paragraph{User Study to Guide Our Framework Design:} 
We conducted a \textit{Wizard-of-Oz} formative user study to explore users' expectation of privacy when interacting with LCAs and to gather technical requirements for our framework.
Following established practices in early-stage interface design research \citep{ nielsen2000fiveusers, budiu2021fiveparticipants, nielsen1993mathematical}  where 5 participants are typically sufficient to identify major design insights, we conducted our study with six participants from our institution who were familiar with LLMs.
Using three mid-fidelity UX mockups (see Appendix \ref{sec::user_study_mockups}), we probed participants on their privacy concerns, reactions to privacy disclosures, and preferences for managing sensitive information. Each mockup simulated interactions where PII and sensitive information were detected and flagged. Participants provided feedback on different approaches to identifying, flagging, and reformulating sensitive information. 

Insights from this formative phase shaped several key design aspects of our framework, including distinguishing between essential and non-essential sensitive information, real-time feedback, user control over reformulations, and transparency around how sensitive information is handled and flagged. The participants rated the overall approach of the system highly, with a min and max rating of 7/10 and 9/10 respectively, providing initial validation for our approach to sensitive information detection and reformulation. For a detailed discussion of the study and how it impacted our design, see Appendix~\ref{sec::user_study}.
 
\noindent\textbf{Proposed Framework:}
We propose a framework that acts as an intermediary between the user and the conversation agent and enables the user to detect out-of-context sensitive information in the user prompt and judiciously reformulate the prompt to ensure contextual privacy.
The key components of the framework are outlined in Figure~\ref{fig:framework}.
When a user submits a prompt, our framework first determines the \textbf{context} and \textbf{subject} of the conversation. The context is divided into two components: the domain of the interaction (e.g., medical, legal, or financial) and the specific task the user aims to perform, such as seeking advice, requesting a translation, or summarizing a document. 
Context identification is guided by a taxonomy of common user tasks and sensitive contexts that go beyond PII \citep{mireshghallah2024trust} (see Appendix~\ref{domains_and_tasks}).







Once the context and subject are identified, our framework moves on to detecting sensitive information in the prompt. %
The framework categorizes the sensitive information into two spaces: (a) \textbf{essential information space}: sensitive details necessary to answer the user’s query, (b) \textbf{non-essential information space}: sensitive details that are unnecessary for answering the query and should be kept private.

In the example of Figure \ref{fig:framework}, the sensitive terms are \textit{``Jane'', ``single parent of two'', ``diabetes'', and ``affordable''}. While ``diabetes'' is essential for providing advice on treatment options, the other details---Jane's name, family situation, and financial concerns---are not required and thus classified as non-essential. 







Once contextually essential and non-essential information is identified, our framework improves contextual privacy by \textbf{reformulating} the prompt. This process includes removing, rephrasing, or redacting details within the non-essential information space, while preserving the user's intent. This way, we ensure that the user can still achieve the desired outcome effectively when the reformulated prompt is sent to the untrusted LCA. In our running example, a reformulated user's prompt could be \textit{``I need advice on managing a health condition and finding treatment options for diabetes''}, which protects non-essential sensitive details like the user’s name and personal circumstances, while maintaining the core intent of seeking treatment advice for diabetes.




After the reformulated prompt is generated, users can review, modify, or accept it, or revert to the original input. The review steps, shown by dashed boxes in Figure~\ref{fig:framework}, ensure user control, allowing them to achieve their desired balance between privacy and utility. The framework continues to highlight privacy implications as users adjust the suggested reformulation, helping them make informed choices about what information to share. Once finalized, the reformulated prompt is sent to the LLM-based conversational agent to obtain a response.



\section{Evaluation}

To evaluate the effectiveness of our layered interface paradigm and generative authoring workflow, we conducted two studies: (1) a mixed-methods user experience evaluation with a Subjective Evidence-Based Ethnography (SEBE) protocol, and (2) a between-subjects deployment study on Prolific.

\subsection{User Experience Evaluation}

In the first study, our goal was to evaluate \system's impact on cognitive load, usability, and its influence on creative output. \change{ We wanted to understand the manners in which personas and layered affordances could bridge the envisioning gap~\cite{subramonyam2024bridging} for writers. }

\subsubsection{Participants}
We recruited participants via LinkedIn and Twitter, selecting individuals who engaged in creative writing at least a few times per month. The final sample included 12 participants (8 male, 4 female), aged 18 to 54. The majority (7 participants) were in the 18-24 age group, with 3 in the 25-34 range, and one each in the 35-44 and 45-54 groups. All participants were native English speakers. Educational backgrounds varied: 6 participants held bachelor's degrees, 5 had master's degrees, and 1 had some college experience. Participants reported diverse creative writing habits and experiences, with most having taken creative writing courses. Self-reported confidence in writing ranged from 2 to 5 on a 5-point scale ($\mu = 3.54$, $\sigma = 0.82$).

\subsubsection{Study Procedure}
Each study session lasted not more than 2 hours. At the start of the session, participants received a live demonstration of the application and were encouraged to ask questions. The participants were then asked to access the application on their own computer through the web url and complete a brief guided writing task about a tree, a cat and a dog to familiarize themselves with the interface. Next, participants engaged in the main writing task. They were given 40 minutes to write an 800-word essay connecting a recent real-life event to a film, using one of the 5 specified writing styles (First-Person Narrative, Journalistic Style, Dialogue Format, Letter or Diary Entry, or Screenplay Format). They were asked to identify parallels, extract insights, and reflect on personal growth. After completing the task, the participants were asked to fill out the NASA Task Load Index (TLX) for cognitive load, the Post-Study System Usability Questionnaire (PSSUQ) for usability assessment, and the Creative Support Index for evaluating creative support. The participants completed a survey gathering demographic information, writing habits, and experience with AI writing tools. 

Finally, participants engaged in a retrospective interview while reviewing the snippets of the video recordings of their session. We wanted to understand the (1) affordances, in terms of effects of the application on their writing process,  (2) their embodied competencies, skills, and knowledge they drew upon during the task, and (3) any perceived rules, norms, or expectations that guided their approach. We wrapped up the session by asking about overall impressions and additional insights. The entire writing session was recorded, capturing both the participants' audio recording and the participants' screen. Each participant received \$30 for their participant in the form of an Amazon gift card.


\subsubsection{Results}
\textbf{NASA TLX.} The survey measures six subscales: Mental Demand, Physical Demand, Temporal Demand, Performance, Effort, and Frustration. Analysis of the data (N = 12) revealed varying levels of perceived workload across these dimensions. Effort ($\mu = 8.42$, $\sigma = 6.50$) and Mental Demand ($\mu = 6.17$, $\sigma = 3.43$) were the highest-rated factors, indicating that participants found the task mentally taxing and requiring substantial effort. This aligns with expectations, as the task was designed to require deep thinking and was non-trivial. Temporal Demand also had a notable score ($\mu = 6.42$, $\sigma = 6.35$), suggesting that time pressure played a significant role. \change{The Performance score ($\mu = 5.25$, $\sigma = 3.33$) indicates participants generally performed well, as lower values on this scale represent better perceived performance. Similarly, low Frustration scores ($\mu = 3.58$, $\sigma = 3.96$) suggest minimal participant frustration.} The high standard deviations, particularly for Effort and Temporal Demand, reflect significant variability in study participants' experiences, likely due to differences in their task approach or expertise characteristics. Overall, these findings suggest that \system imposed a low cognitive load on participants.

% \begin{figure}
%   \centering
%   \includesvg[width=0.8\textwidth]{figures/plots/nasa_tlx_plot.svg}
%   \caption{Boxplot for NASA Task Load Index across the 6 dimensions}
%   \label{fig:nasaltx}
% \end{figure}

\begin{figure}[!t]
  \centering
  \includegraphics[width=0.9\columnwidth]{figures/NASATLX.pdf}
  \caption{NASA-TLX Workload response distribution across relevant dimensions}
  \label{fig:nasatlx}
\end{figure}


\begin{figure}[!t]
  \centering
  \includegraphics[width=0.9\columnwidth]{figures/PSSUQ.pdf}
  \caption{PSSUQ Response dimensions across System Usefulness, Information Quality, Interface Quality categories and Overall dimension}
  \label{fig:pssuq}
\end{figure}
\textbf{PSSUQ.} The Post-Study System Usability Questionnaire was used to assess the usability of the AI-assisted writing tool across three key dimensions: System Usefulness, Information Quality, and Interface Quality. Analysis of the data (N = 12) revealed generally positive usability scores, with all dimensions receiving mean ratings below the midpoint of the 7-point scale (where lower scores indicate better usability). The Overall PSSUQ score ($\mu = 2.18$, $\sigma = 1.29$) suggests that participants found the system to be reasonably usable. System Usefulness ($\mu = 1.99$, $\sigma = 1.01$) and Interface Quality ($\mu = 2.00$, $\sigma = 1.39$) were rated the highest, indicating that users found the tool functional and easy to interact with. The Information Quality dimension received a slightly lower but still positive rating ($\mu = 2.51$, $\sigma = 1.45$), suggesting room for improvement in the clarity and organization of information provided by the system.

% \begin{figure}
%   \centering
%   \includesvg[width=0.9\textwidth]{figures/plots/pssuq_plot.svg}
%   \caption{PSSUQ Response dimensions across System Usefulness, Information Quality, Interface Quality
% categories and Overall dimension}
%   \label{fig:pssuq}
% \end{figure}

Participants reported that they could not always tell if they had made a mistake in the interface, though this may not be a significant concern for a writing tool where many `mistakes' are subjective or stylistic choices. Within the System Usefulness dimension, items related to ease of use ($\mu = 1.67$, $\sigma = 0.89$) and efficiency ($\mu = 1.58$, $\sigma = 0.79$) received particularly positive ratings, suggesting that users found the tool intuitive and time-saving. However, the item related to system capabilities (System Usefulness: $\mu = 2.08$, $\sigma = 1.73$) showed higher variability, indicating diverse opinions on whether the system had all the expected functions and system capabilities, from the participants' perspectives.

In the Information Quality dimension, the item related to error message clarity ($\mu = 4.38$, $\sigma = 1.41$) received the lowest rating, highlighting a significant area for improvement. Interface Quality items were consistently rated positively, with low variability, suggesting a well-designed user interface. The relatively high standard deviations across most items indicate varied user experiences, possibly due to differences in expectations, prior experience with similar tools, or the specific writing tasks undertaken during the study.

\textbf{CSI.} The Creative Support Index was used to evaluate participants' experiences with \system, using a scale of 0 to 20, where lower scores indicate more positive outcomes. The analysis focused on five key dimensions: Exploration, Enjoyment, Results Worth Effort, Immersion, and Expressiveness. Data from 12 participants revealed generally positive experiences across these dimensions. Enjoyment was the highest-rated factor ($\mu = 2.67$, $\sigma = 3.50$), suggesting that users found the tool particularly enjoyable. This was closely followed by Results Worth Effort ($\mu = 3.83$, $\sigma = 3.81$), indicating that participants felt their input produced valuable results. Exploration ($\mu = 4.00$, $\sigma = 4.31$) and Expressiveness ($\mu = 4.33$, $\sigma = 3.17$) also received favorable ratings, implying that the tool effectively supported idea generation and self-expression.

\begin{figure}[!t]
  \centering
  \includegraphics[width=0.9\columnwidth]{figures/CSI.pdf}
  \caption{Creative Support Index (CSI) response distribution across relevant dimensions}
  \label{fig:csi}
\end{figure}

% \begin{figure*}[t!]
%   \centering
%   \includegraphics[width=0.4\textwidth]{figures/CSI.pdf}
%   \caption{Creative Support Index response distribution across relevant dimensions}
%   \label{fig:csi}
% \end{figure*}





\begin{figure*}[t!]
    \centering
    \includegraphics[width=0.9\textwidth]{figures/usage_tree.pdf}
    \caption{\change{Tree visualization of layer manipulation and LLM calls in \system. We show three diverse ways participants leveraged the interface affordances for completing the assigned task. The diverging gray lines depict `tear', the converging gray lines depict `combine', and meaning of other symbols is present in the legend}}
    \label{fig:participant-journey}
\end{figure*}

Interestingly, Immersion received the lowest rating ($\mu = 7.50$, $\sigma = 5.79$), though it still fell on the positive side of the scale. This suggests that while users found the tool engaging, there may be opportunities to enhance its ability to create a more immersive experience. The high standard deviations across all dimensions, particularly for Immersion and Exploration, reflect significant variability in user experiences, possibly due to individual differences in writing styles. The Collaboration dimension was excluded from our findings, as participants rated its weight factor as 0. 

\textbf{Spatial organization of content.} \system provides users with the ability to spatially organize their layers and associated content, a feature that was highly valued by participants for improving their focus and workflow. P9 praised the interface's immersive quality, stating: \textit{``One nice thing is that the fact that it's all in one big interface makes it less immersion-breaking than, say, opening a bunch of Google Docs tabs, where you have to make more major context switches and get distracted by other tabs.''} P9 further emphasized the advantages of a dedicated workspace: \textit{``Or you break out of the immersion of writing and being in the zone by going more into your desktop environment, where you're reminded of your work, or end up seeing social media. So it's nice to have a dedicated, isolated workspace.''}  This sentiment was shared by other participants who recognized the organizational benefits of \system over traditional text editors. P4, for instance, pointed out the limitations of conventional file systems: \textit{``The structure and organizational possibilities of this kind of thing would be huge because you end up with folder upon folder upon folder.''} P4's comment highlights \system's potential to address the organizational challenges typically encountered with standard text editors, offering a more intuitive and flexible way to manage content throughout the writing process. 

The system's spatial design supported expressive content management strategies\footnote{We thank psycholinguist George Miller, author Roy Pea's postdoctoral mentor at Rockefeller University in the 1970's for his seminal insights half a century ago in foregrounding 'the human tendency to locate information spatially', which we expressly leverage in the design of \system~\cite{millerpsychology}}. P9 emphasized how this shaped their writing process: \textit{``Being able to fold or bin my writing without permanently removing them made me much more willing to experiment. In single-page interfaces, I often feel pressured to be certain about content placement before typing.''} Users demonstrated remarkable spatial awareness of their content, organizing it in ways that enhanced accessibility. P11 noted: \textit{``Being able to [tunnel] into content from all over and get contextually-informed responses from the friends made everything feel so accessible.''} Through reduced cognitive load, users focused more on writing and experimentation rather than on content management.

\textbf{Paper metaphor for writing.} A central metaphor that \system aims to embody is the ability to manipulate and move layers much like rearranging sheets of paper on a desk. P11 captured this sentiment, stating, \textit{``It's a much more visual, like a desk with pieces of paper all over it,''} highlighting the intuitive, tactile nature of organizing content in the workspace. This visual and spatial approach offers users a more flexible and natural way to manage their writing, akin to physically handling documents in a traditional environment. It also opens up future promising research directions to develop and study a gesture- and voice-based rendition of the \system functionalities. 

\textbf{Flexibility in testing rhetorical strategies.} The affordances granted by \system, including the ability to tear, split, combine, stack, and fold layers, were clearly evident in how participants utilized the tool. These features not only supported writing tasks but also aligned with users' conceptual models of the writing process. P1 drew a parallel between the system's structure and traditional writing approaches: \textit{``Layers are like, if you're writing a paper, you need an outline and goal. Intuitively, it's like a tree which is the outline, and you work on each part of the node. In this case, you start with the intro, and you spend time collaborating with Idea Ivy. It was natural to break down [the writing process].''} This natural breakdown of the writing process was further enhanced by the system's manipulable interface, as P4 highlighted: \textit{``I really liked it. Being able to push something over there until later, and then bring it back and like smoosh it all together. That was really nice; that I liked that a lot.''} 

\change{The interface's bottom-up approach to LLM integration particularly enhanced exploratory writing. P5, whose usage journey is visualized in Fig. ~\ref{fig:participant-journey} C, observed: \textit{``Clicking a layer, moving it around and expected something to happen is so intuitive. The interface interactions felt natural and expected - exactly what I envisioned would happen.''} Participants often combined different features creatively, as P5 described: \textit{``I think order to exposition is really important. Tearing and recombining in different orders helped me rapidly see what narrative flow made sense''} P5 also wrote initially in first person, utilizing Tone Tara to shift to third person narration, demonstrating the system's flexibility in supporting various narrative styles.}

\textbf{Collaboration with Writer's Friends:} Participants demonstrated diverse and interesting approaches to leveraging the capability of the Writer's Friends in \system. These distinct personas, each representing different writing assistance features, effectively bridged the gulf of envisioning~\cite{subramonyam2024bridging}. P8 expressed a particular fondness for one such friend: \textit{``I like Danny. My Danny's a good guy.''} Similarly, P11 highlighted the value of constructive criticism: \textit{``I really like the friends, especially the feedback. I liked getting negative feedback.''} \change{Through writing in the interface and receiving feedback from the ``friends'', users frequently refined their meta layer. P2 explained: \textit{``Sometimes I felt my ideas were being misconstrued so updating the meta layer helped''} The iterative nature of getting feedback helped clarify writing goals and better understand the target readers. On the anthropomorphized LLM scaffolding, P8 remarked their indifference, ``I don't think it influenced my usage positively or negatively''} 

A thematic analysis of user-defined prompts revealed that participants accurately matched their requirements to the appropriate friend in the majority of cases. Out of the 57 times Detail Danny was used, the participants gave it prompts for detail and elaboration 77\% of the time. In the case of Idea Ivy, out of the 52 instances used across sessions 86\% of prompts were for ideation and brainstorming. In case of Tone Tara and  Structure Sam, 85\% (20) and 95\% (20) mapped to tone transformation and structuring respectively. Importantly, the Writer's Friends seemed to enhance rather than replace the creative process. P12 noted, \textit{``I honestly felt like I was still using this tool\ldots to come up with my own ideas. And I think that was good for me.''} This sentiment suggests that \system successfully balanced AI assistance with the preservation of user agency in the writing process. 

\textbf{Usability and User Interface.} The intuitive design and user-friendly interface of \system were frequently highlighted by participants, emphasizing the tool's ease of use and its ability to seamlessly integrate into the writing process. P11 expressed enthusiasm for the command interface, noting its natural feel: \textit{``I love the backslash and having the pull down menu and immediately being able to select. I liked that. I didn't even think about it\ldots I naturally did it.''} This comment showcases the effectiveness of the interface in reducing the cognitive load on users, allowing them to access features quickly and intuitively without disrupting their writing flow. The system's design also contributed to a positive emotional experience for users. P7 remarked on the sense of control and comfort provided by the interface: \textit{``I wasn't stressed at all, I felt completely in control.''}

In figure \change{~\ref{fig:participant-journey} we can see the journey through the interface taken by users with three distinct usage patterns. These visualizations give us a sense of how different writers leverage \system to test rhetorical strategies.  In the case of P2 we see that they created a total of 5 layers, two of which were alternative structures suggested by Structure Sam. In case of P1 they created 13 layers and used 4 of them in their final essay generation. Similarly, P5 created 17 layers, counting all tears, combination, and alternative suggestions layers, and used only two to generate their final essay. }



\subsection{Comparative Analysis: Between-Subjects Evaluation on Prolific}
\change{Building on the insights from the usability assessment, we designed a comparative evaluation. We wanted to understand what facet of \system supported the dynamic knowledge transformation we observed by the writers in study 1. To isolate the effects from different features, we constructed two separate interfaces in addition to \system. We conducted a between-subject study with three conditions: (1) a layered interface with in-line LLM (\system condition), (2) a writing interface with in-line LLM but no layers or spatial component (In-Line-LLM), and (3) a writing interface with a separate AI chat window for LLM interaction (Chat-LLM condition).}


% \begin{figure}
%     \centering
%     \includegraphics[width=0.9\textwidth]{figures/qual_comp.pdf}
%     \caption{\change{Visualization of LLM Interaction Across Conditions. Each square represents a different writing subprocess, with their meaning defined in the legend.}}
%     \label{fig:all_conditions}
% \end{figure}

\begin{figure*}[t!]
  \centering
  \includegraphics[width=0.95\textwidth]{figures/study_2_qual.pdf}
  \caption{Visualization of LLM Interaction Across Conditions. Each square represents a different writing subprocess, with their meaning defined in the legend.}
  \label{fig:all_conditions}
\end{figure*}


\subsubsection{Design Decision For Conditions}
In order to have a fully functional In-Line-LLM application that supports the Writer's Friends without the layered paradigm, we had to make some design choices. Detail Danny, Idea Ivy, Feedback Felix and Audience Ali, have their operations constrained on the active layer so they will remain the same but in the case of Structure Sam and Tone Tara, where new layers with transformed content are generated, we had to change their behavior for the In-Line-LLM condition. Both Sam and Tara, instead of generating new layers, replaced the existing content in the editor and provided the user the option in the toolbar to reverse it to their previous content before the transformation. In the case of the Chat-LLM interface, we added a chat interface right beside the editor, to resemble having a word processor and an LLM chat interface open in split view. The rationale for creating our own interface instead of using a baseline of an existing LLM chat interface with a word processor was to have control over the data logging.

\subsubsection{Participants}
We conducted the study through Prolific and had a screening for people who wrote professionally, roles included journalist, copywriter/marketing/communications, and creative writing. We also screened for participants who spent more than 5 hours a week on Prolific, to increase chances of high-quality participation. We recruited 84 participants (F=48, M=30, NB=6). Participants reported ages between 18 and 64, the median age being in the 45-54 age group. 

\subsubsection{Study Procedure}
Each participant was randomly assigned to one of the three conditions (\system, In-Line-LLM and Chat-LLM) such that there was a distribution of 28 participants per condition.  On a scale of 1 to 5, participants reported confidence in their writing skills as follows: chat-llm ($\mu = 3.96$, $\sigma = 0.98$), in-line-llm ($\mu = 4.20$, $\sigma = 0.61$), and \system ($\mu = 4.42$, $\sigma = 0.64$) in case of \system, This study design would allow us to isolate the effect of different components and formulate a comprehensive assessment. Each participant first filled out a demographic survey, and consented to having their writing data logged into our database. The participants were each asked to watch a video tutorial for their respective condition interface. Afterwards,  they had to take a quiz about the interface. This helped ensure that the participants understood the interface well before using it. They were not allowed to proceed without getting all the answers right. Following this task they were assigned their task, which they were given 40 minutes to complete. The task required writing about an 800 words essay on a \textit{crowdworker's experiences with Large Language Models (LLMs):} They were specifically asked that it cover LLMs' impact on their work, adaptation strategies, and future outlook. They were instructed that the essay should include specific examples, data, and reflections on both positive and negative aspects of LLMs in crowd-work. The task was intentionally made non-trivial as we wanted to simulate a real writing task for the participants in a creative capacity. In order to ensure the participants did not write someplace outside of the interface before pasting it later, we tracked their activity in the interface and made persistent activity a criterion for their submission to be valid. We also manually validated the essays written to ensure they were actual essays about the topic and not something off-topic or incoherent.

% \change{The words per minute we report are only from keystrokes registered from the participants. LLM generations do not count towards the words per minute.}  

\subsubsection{Results}




\change{Figure ~\ref{fig:all_conditions} shows the sequence of invocation of features across conditions. We coded the prompts issued by users into one of the following categories: (1) Review, (2) Tone Transformation, (3) Elaboration, (4) Organize, (5) Brainstorm, and (6) Research. Between the In-Line-LLM and \system condition, we combined Feedback Felix and Audience Ali into Review. We also marked prompts that did not fit into any of these categories as miscellaneous. We also do not show instances where users reissue commands when they are not content with the initial generation. Notably, analysis of prompt reissuance revealed a higher density in the Chat-LLM condition, suggesting users were frequently dissatisfied with initial responses and attempted to regenerate content. In contrast, we observed substantially lower reissuance in the In-Line-LLM and \system conditions. As we can see in the visualization, usage differences exist between the interfaces that support in-line LLM support and the chat interface. Looking at the prompts issued by the user, we observed difficulty in understanding how the LLM could support their writing. We observed prompts like, \textit{``You're not a very comprehensive model, are you?''} and\textit{  ``So, your main function is to serve as a research tool?''.} Despite providing an introduction video for all conditions, users experienced the most difficulty in the chat interface. We attribute this to challenges in formulating specific writing intentions and planning, which we believe the task-specific LLM personas were successful at bridging. Users in the In-Line-LLM and \system asked for assistance with transformation or elaboration. In the case of the chat interface, we observed users attempting to solicit complete essays based on points they specified in the prompt.}

\change{When comparing the In-Line-LLM and \system conditions, we observed greater diversity in writing strategies among \system users. A notable pattern was the tendency to seek feedback earlier in their writing process, suggesting that users aimed to align their work with audience expectations and prevent significant deviations as their drafts progressed. Conversely, the In-Line condition exhibited a more fixed usage pattern, where users typically began with review tasks, spent the bulk of their time elaborating on content, and concluded with tone transformations. Interestingly, this pattern deviated from the tutorial video shared with participants, leaving us uncertain about the underlying reasons for this behavior. Through this analysis, we conclude that \system effectively helps users bridge the envisioning and articulatory gap, enabling them to articulate their writing intentions and integrate feedback more seamlessly into their workflows. While in-line support fosters more thoughtful usage of LLM,  the spatial organization of \system propelled the dynamic knowledge transformation.}


\section{Discussion}
\label{sec:discussion}
%include una discussione più ampia dei risultati pottentuti in termini di loro validità, eventuali limitazioni, ed opportunità di ricerca abilitate
While Cyri enhances phishing detection, management, and understanding from human users, its current implementation also presents a set of limitations that offer avenues for further improvement.
Improvements in detecting low-accuracy semantic features could be achieved by fine-tuning activities leveraging the produced phishing email dataset. These features, while less critical than others, still contribute to the overall understanding of phishing tactics. This solution would not be in substitution but would complement the current Chain-of-thought approach. Retrieval Augmented Generation can even be exploited, using currently detected emails as additional context for more accurate detection.
Another limitation we foresee is the need for a longitudinal study with users that lasts longer and collects usage data on a higher quantity of tested emails and in real-pressure conditions. We are planning this activity in the near future.

%A well-structured fine-tuning process could enhance the model’s ability to identify these less-detected features, improving the model’s comprehensiveness and educational value. Fine-tuning would involve training the model further on examples specifically designed to highlight these features, thereby increasing its sensitivity to a broader range of phishing strategies.

As interesting future possibilities enabled by this research, we foresee integrating Cyri into existing mobile email clients, which would enhance accessibility and provide real-time phishing detection and education on the devices most commonly used for email communication. Limits and possibilities in this scenario could be provided by quantized versions of small LLMs capable of being run on smartphones with similar accuracy to 8 billion models. The use of information distillation techniques with a teacher-student approach using LISA as the teacher model may be beneficial for this effort.
%Another promising direction is the integration of speech-to-text capabilities into Cyri. This functionality would make Cyri more accessible facilitating individuals who prefer voice interaction.

%This project laid the foundation for these future developments, demonstratingì the efficacy of combining advanced AI technologies with user-centered design. By addressing current limitations and exploring new frontiers in phishing detection, subsequent research can build upon this work to create even more robust and comprehensive cybersecurity solutions.

\section*{Limitations}
\vspace{-4pt}
Contextual integrity is a relatively new and fluid notion of privacy. Ours is also one of the very early works exploring this space from the standpoint of LLM-based conversational agents. Naturally, this leads to a number of challenges, some of which are beyond the scope of the work and should be addressed in the future. Like we discussed before, establishing privacy norms and principles in CI itself is complex and dependent on societal contexts, which is why we restrict ourselves to a practical and useful variation of the idea. However, developing templates for implementing CI under various societal contexts deserves significant attention from the research community in the future.

Our framework addresses critical privacy concerns in LLM interactions, potentially shaping future norms around data sharing in conversational AI. By enhancing user awareness and control over sensitive information, it promotes more ethical AI deployments, safeguarding user privacy in diverse applications such as healthcare, legal, and personal assistance. However, there are ethical challenges, such as ensuring fairness across cultural contexts and preventing over-reliance on automated privacy detection. 



\clearpage
\bibliography{custom}

\appendix
%\section{Related Work}

% \begin{figure} [t] \small
%     \centering
%     \includegraphics[width=0.92\linewidth]{imgs/DLRM-BWD1.pdf}
%     \vspace{-5pt}
%     \caption{Speedup ($\times$) breakdown of the TT table \textit{backward} optimization.}
%     \label{fig: TT backward}
%     % \vspace{5pt}
% \end{figure}

% \textbf{System Designs for DLRM:} The DLRM is one of the most critical deep learning applications in the industry\cite{covington2016deep, okura2017embedding, gomez2015netflix}. The training and inference of the recommendation model come with various challenges due to the compute-intensive MLP layer and the memory-intensive embedding tables. 
% Work in \cite{nvidiainference} accelerates the inference of recommendation model by leveraging fused embedding lookup. DeepRecSys\cite{gupta2020deeprecsys} proposed an inference
% scheduler that maximizes latency-bounded throughput.
% As for DLRM training, NEO\cite{mudigere2021softwarehardware} and HugeCTR\cite{hugectr}  distribute large embedding tables to different GPUs and train the embedding table in a model-parallel style which incurs non-trivial data communication between GPUs. XDL\cite{jiang2019xdl} and FAE\cite{ebrahimzadeh2021accelerating} leverage CPU memory to handle the large embedding tables while keeping the MLP layer in GPU, the non-optimized hybrid CPU-GPU system design makes CPU side computation becomes a bottleneck. Overall, all of these techniques have not properly solved the issues of high data communication overhead. 

% \textbf{General Large-scale DLRM Training:}
% \yuke{delete this paragraph or merge some points with the previous one}
% % \yuke{merge this point with the point-1}
% Over the recent years, the number of parameters of deep models have increased dramatically~\cite{devlin2018bert, cao2020pretrained, brown2020language} and  large scale model training gains more and more attention. Generally, there are two ways to handle the out of memory training:1) Scale out training uses aggregate memory of multiple GPUs to satisfy the memory requirement for large model training~\cite{dean2012large, shoeybi2019megatron, harlap2018pipedream, huang2019gpipe}. 2) Scale up model size into a single GPU by several approaches. \yuke{reprase the second points} For instance, recomputing gradient from checkpoints
% ~\cite{chen2016training}, using an external memory such as the CPU memory~\cite{ren2021zero, huang2020swapadvisor}. \yuke{points out weakness of previous solutions}. Based on these works, we proposed a more targeted optimization for recommendatin models. \yuke{points out the exact solution and technical keywords}

% \textbf{Embedding Table Compression:}
% The embedding tables of the DLRMs have large memory footprint~\cite{zhao2020distributed, lan2019albert, weinberger2009feature, yang2020mixed, yin2021tt}. To cut down memory consumption, there have been many efforts on embedding table compression. ALBERT~\cite{lan2019albert} leverages factorized embedding parameterization and cross-layer parameter sharing to reduce the footprint of parameters. Kilian et al. propose feature hashing~\cite{weinberger2009feature} which maps multiple items to the same embedding vector. Jie et al.\cite{yang2020mixed} and Hui et al.\cite{guan2019post} use fewer bits to represent the embedding vectors. The above techniques decrease the footprint of embedding tables but often incur an accuracy tradeoff.
%
% Compressed data direct computing~\cite{zhang2021poclib,zhang2021tadoc,zhang2018efficient,pan2021exploring,zhang2022compressdb} is a novel processing method that can also be applied in embedding table compression.
%
% TT-Rec~\cite{yin2021tt} applied Tensor-train (TT) factorization on embedding tables which achieves a considerable reduction of embedding table size while keeping a high model accuracy. However, the TT factorization brings additional computation and increases training time. This motivates us to optimize the TT-based solution with system-level designs to capitalize on the benefits of TT-based embedding compression.


\section{User Study to Guide System Design}
\label{sec::user_study}

To explore users' perceptions of privacy with LCAs and gather technical requirements for our framework, we conducted a Wizard-of-Oz formative user study with six participants from our institution who were generally familiar with LLMs. 

The study involved a 30-minute semi-structured interview where participants were presented with three mid-fidelity UX mockups, each designed to demonstrate different ways private and sensitive information could be detected and remediated (see Appendix~\ref{sec::user_study_mockups}). These mockups, featuring synthetic examples inspired by real-world patterns in the ShareGPT dataset, were created to expose participants to targeted privacy risks, such as unintentional PII and sensitive data disclosures. We used these mockups to probe participants' views on their own privacy practices, their thoughts about privacy disclosures, and their preferences for managing sensitive information in conversations. The study provided insights into people's views on the identification, flagging, and reformulation of sensitive data, shaping the core elements of our framework.



\begin{itemize}[leftmargin=1em]
    \item \textbf{Perceived privacy control}. Participants initially believed their efforts to protect their privacy when using real-world LLM applications were effective due to how they kept conversations vague. After they saw real examples of indirect privacy leaks in the mockups, many participants expressed greater concern about unintentionally sharing private information. \textbf{Design impact}: This insight emphasized the importance of identifying both direct and indirect privacy risks during LLM interactions in our system.
    \item \textbf{Visual identification of sensitive information}. Prototype B's color-coded differentiation between PII, necessary, and unnecessary information was praised for making privacy risks clearer and easier to understand. \textbf{Design impact}: Based on this feedback, we included the ability to differentiate between different kinds of sensitive information disclosures to help inform users' decision-making.
    \item \textbf{Reformulation preferences}. Although some participants preferred doing the work of reformulating their LLM prompts themselves, most wanted the system to offer (at least) one reformulated prompt suggestion, with the option to generate new suggestions. A few participants suggested offering multiple reformulations at once, selected across a spectrum of privacy-utility tradeoffs. In this way, users can balance their level of privacy protection with the utility of the output. \textbf{Design impact}: We designed our system to present one reformulation recommendation at a time, but with the flexibility to generate new alternative reformulations. In future iterations of our system, we plan to explore how to generate multiple reformulation options across varied privacy-utility tradeoffs.
    \item \textbf{User control and real-time feedback}. Real-time feedback and user control over editing flagged prompts were highly valued. Participants preferred having the system automatically generate reformulations, but they wanted the ability to make any necessary final adjustments. \textbf{Design impact}: We implemented a review step where users can edit, accept, or proceed with the original input before final submission to the LLM, providing the flexibility users requested.
    \item \textbf{Positive reception}. Participants responded positively to the system’s potential for managing sensitive information, with an average rating of $8.7 (\pm 0.87)$ on the importance of detecting and flagging sensitive details. \textbf{Design impact}. This feedback reinforced the central role of sensitive information detection in our framework, highlighting its perceived value to users.
    \item \textbf{Clarity and transparency}. Participants expressed a strong desire for transparency about how the system operates, including which tools or models are being used, and the meaning of key terms like ``necessary'' versus ``unnecessary'' information. \textbf{Design impact}: Our framework ensures transparency by detailing how sensitive information is identified and handled, including the models used, how they are applied, deployed, and how data is managed. We recommend real-world implementations do the same to build user trust.
    \item \textbf{Broader application}. A few participants suggested applying the tool to other contexts beyond LLM chat interfaces, such as search engines. \textbf{Design impact}:  This feedback highlights the importance of managing sensitive information and the broader applicability of our approach to other contexts.
\end{itemize}


\subsection{User Study Mockups}
\label{sec::user_study_mockups}
\begin{figure*}[t]
    \centering
    
    \begin{minipage}{0.8\textwidth} %
        \centering
        \includegraphics[width=\linewidth]{figures/user_study_1.png}
        \subcaption{Examples of unintentional disclosures shown to participants}
    \end{minipage}

    \vspace{0.5cm} %

    \begin{minipage}{0.45\textwidth}
        \centering
        \includegraphics[width=\linewidth]{figures/user_study_2.png}
        \subcaption{Mockup 1: Display all detected sensitive info}
    \end{minipage}
    \hfill
    \begin{minipage}{0.52\textwidth}
        \centering
        \includegraphics[width=\linewidth]{figures/user_study_4.png}
        \subcaption{Mockup 3: Rewrite the user's message for them}
    \end{minipage}

    \vspace{0.5cm} %

    \begin{minipage}{0.8\textwidth}
        \centering
        \includegraphics[width=\linewidth]{figures/user_study_3.png}
        \subcaption{Mockup 2: Color Code information and suggest reformulations}
    \end{minipage}

    \label{fig:mockups}
\end{figure*}

\onecolumn
\section{Domains and Tasks}
\label{domains_and_tasks}
Table~\ref{tab:categories} shows the list of Domain and Tasks Categories for Intent Detection.
\begin{table*}[ht]
\centering
\small
\begin{tabular}{|p{3.5cm}|p{10cm}|}
\hline
\textbf{Domain} & \textbf{Description} \\ \hline
Health\_And\_Wellness & Conversations related to physical and mental health, such as medical conditions, history, treatment plans, medications, healthcare provider information, symptoms, diagnoses, appointments, health-related advice, mental health status, therapy details, counseling information, emotional well-being, fitness routines, nutrition, dietary preferences, meal plans, health-related diets, feelings, coping mechanisms, mental health support, and emotional support systems. \\ \hline
Financial\_And\_Corporate & Conversations involving financial and corporate matters such as bank account details, credit card information, transaction histories, investment information, loan details, financial planning, budgeting, banking activities, insurance policies, claims, coverage details, premium information, business transactions, corporate policies, financial reports, investment strategies, stock market discussions, and company performance. \\ \hline
Employment\_And\_Applications & Conversations about employment and related applications, such as job status, job applications, resumes, workplace incidents, employer information, job roles, professional experiences, salaries, benefits, employment contracts, visa applications, and other types of applications including application processes, requirements, status updates, supporting documents, interviews, and follow-up actions. \\ \hline
Academic\_And\_Education & Conversations related to academic and educational topics, including school or university details, grades, transcripts, educational history, academic achievements, courses, assignments, educational resources, learning resources, teaching methods, and extracurricular activities. \\ \hline
Legal & Conversations involving legal matters such as legal advice, court cases, contracts, legal documents, criminal records, discussions about laws and regulations, tax information, social security numbers, government benefits, applications, and interactions with legal professionals or government agencies. \\ \hline
Personal\_Relationships & Conversations about personal relationships, such as family details, marital status, friendships, romantic relationships, social interactions, personal issues, relationship problems, private social events, and childcare arrangements. \\ \hline
Travel & Conversations related to travel and transportation, including travel plans, itineraries, booking details, passport and visa information, travel insurance, destinations, accommodations, transportation options, vehicle details, driver's license information, and travel routes. \\ \hline
Hobbies\_And\_Habits & Conversations about personal hobbies and habits, such as leisure activities, crafting, gaming, sports, collecting, gardening, reading, writing, and other regular personal interests and practices. \\ \hline
Sexual\_And\_Erotic & Conversations involving sexual and erotic content, including sexual preferences, activities, experiences, relationships, fantasies, sexual health, and explicit discussions about sex. \\ \hline
Politics & Conversations involving political topics, including discussions about political opinions, political events, government policies, political parties, elections, civic participation, and political ideologies. \\ \hline
Religion & Conversations related to religious beliefs and practices, including discussions about faith, religious events, spiritual experiences, religious teachings, places of worship, religious communities, and religious holidays. \\ \hline
\end{tabular}
\caption{List of domains, tasks, and their corresponding descriptions used by the model intent detection}
\label{tab:categories}
\end{table*}

\begin{figure*}[!h]
    \centering
    \lstinputlisting{prompt/search_keywords}
    \caption{60 search keywords to retrieve the literature, including materials synthesis recipes using Semantic Scholar API.}
    \label{fig:prompt_search_keywords}
\end{figure*}

\begin{figure*}[!h]
    \centering
    \lstinputlisting{prompt/paper_categorization}
    \caption{System prompt to categorize the literature converted to markdown format.}
    \label{fig:prompt_paper_categorization}
\end{figure*}

\begin{figure*}[!h]
    \centering
    \lstinputlisting{prompt/paper_extraction}
    \caption{System prompt to extract the recipe from literature converted to markdown format.}
    \label{fig:prompt_paper_extraction}
\end{figure*}


\begin{figure*}[!h]
    \centering
    \lstinputlisting[basicstyle=\ttfamily\small]{prompt/paper_extraction}
    % \inputminted[breaklines, fontsize=\small]{markdown}{prompt/prediction_1shot.tex}
    \caption{A prompt to predict the recipe with a one-shot example.}
    \label{fig:prompt_prediction_1shot}
\end{figure*}

% \begin{figure*}[!h]
%     \centering
%     \lstinputlisting{prompt/prediction_1shot_example}
%     \caption{An example recipe used to a one-shot example.}
%     \label{fig:prompt_prediction_1shot_example}
% \end{figure*}

\begin{figure*}[!h]
    \centering
    \lstinputlisting{prompt/prediction_rag}
    \caption{A prompt to predict the recipe using retrieval-augmented generation}
    \label{fig:prompt_prediction_rag}
\end{figure*}

\begin{figure*}[!h]
    \centering
    \lstinputlisting{prompt/judge}
    \caption{A prompt to judge the prediction recipe using LLM-as-a-Judge}
    \label{fig:prompt_judge}
\end{figure*}

\section{User Prompts Before And After Reformulation }
\label{app:before_and_after_reformulation}


\begin{table*}[t]
  \small
  \centering
  \caption{Qualitative Comparison: Dynamic vs. Static Reformulations (Llama).}
  \begin{tabular}{p{0.32\textwidth} p{0.32\textwidth} p{0.32\textwidth}}
    \toprule
    \textbf{Original Text} & \textbf{Dynamic Reformulated Text} & \textbf{Static Reformulated Text} \\
    \midrule
    As I listen to our weekly Friday sermon (Muslim), I find myself distracted. I want to create a journal template, for me to fill out as I listen to help me listen better. Can you suggest some ideas for the template? 
    & Can you suggest some ideas for creating a journal template that I can use while listening to our weekly Friday sermon, to help me focus and take notes? 
    & Can you suggest some ideas for a journal template that I can use while listening to sermons, specifically for improving focus? \\[1ex]
    \midrule
    Write a LinkedIn post saying congratulations to Monique Rodriguez in forging this major partnership with P\&G and talk about the importance of celebrating and recognizing black businesses.
    & Can you help me write a LinkedIn post about celebrating and recognizing black businesses? 
    & Write a LinkedIn post saying congratulations to a business owner in forging this major partnership with P\&G and talk about the importance of celebrating and recognizing black businesses. \\[1ex]
    \midrule
    I want to do a road trip from Pune to Gujarat. My wife and I will be travelling and we don’t prefer very long driving sessions. Can you suggest a plan starting from Thursday early morning and ending in Pune on Sunday late night?
    & Can you suggest a road trip plan from Pune to Gujarat that avoids very long driving sessions? 
    & Can you suggest a travel plan for a road trip from Pune to Gujarat, taking into account that two people will be driving and it’s preferable to have shorter driving sessions, starting from Thursday early morning and ending in Pune on Sunday late night? \\[1ex]
    \midrule
    I am a 23-year-old female living in Pakistan. I am moving to Germany to pursue higher education. What should I take with me?
    & What should I take with me when moving to Germany for higher education? 
    & What should I take with me when moving abroad for higher education? \\[1ex]
    \bottomrule
  \end{tabular}
  \label{tab:dynamic_vs_static}
\end{table*}


\begin{table*}[t]
  \small
  \centering
  \caption{Qualitative Comparison of Dynamic Reformulations.}
  \resizebox{\textwidth}{!}{%
  \begin{tabular}{p{0.30\textwidth} p{0.30\textwidth} p{0.30\textwidth} p{0.30\textwidth}}
    \toprule
    \textbf{Original Text} & \textbf{Dynamic Deepseek Reformulated} & \textbf{Dynamic LLaMA Reformulated} & \textbf{Dynamic Mixtral Reformulated} \\
    \midrule
    write a LinkedIn post saying Congratulations to Monique Rodriguez in forging this major partnership with P\&G and talk about the importance of celebrating and recognzing black businesses. 
    & Congratulations on forging a major partnership with P\&G and highlighting the importance of celebrating and recognizing black businesses. 
    & Can you help me write a LinkedIn post about celebrating and recognizing black businesses? 
    & ``I'd like to create a social media post about the importance of celebrating and recognizing businesses from underrepresented communities.'' \\[1ex]
    \midrule
    Write an email to Ashley Davies expressing that we were so taken by his reaction to our VR experience that I had to look him up and found his email address on his Autonomous Sensorweb Operations pdf. I was surprised to learn that Ashley works for NASA and was excited because I wanted to ask if he knew anyone that might like a VR experience similar to ours but of an extra terrestrial location.
    & We were very impressed by Ashley Davies' reaction to our VR experience that I looked him up on a document and found his email address. I was surprised to learn that he works for NASA and would love to ask if he knows anyone interested in a VR experience of an extraterrestrial location.
    & Write an email expressing interest in connecting with someone who might be interested in a VR experience of an extra terrestrial location. I was excited to learn that this person works for a space-related organization and would like to ask if they know anyone who might be interested in our type of experience.
    & Writing an email to Ashley Davies, who works for NASA, to inquire if they know anyone interested in a VR experience of an extra terrestrial location. \\[1ex]
    \midrule
    I'm about to have my step son soon and it is going to be another mouth to feed and a lot more expenses each month, which will leave no room if even enough to survive is there a way to get an easy behind the scenes income that can pull the slack of the extra bills and extra food i have to bring in?
    & ``extra monthly expenses are making it difficult to manage bills and food costs. I'm looking for ways to generate behind-the-scenes income to help cover these additional expenses.'' 
    & I'm about to become a step-parent and will have additional financial expenses due to the new family member, which will leave no room for extra bills and food. Is there a way to get an income that can help pull in the slack of these extra expenses?
    & I'm seeking advice on how to generate additional income due to increased monthly expenses from supporting another person. \\
    \bottomrule
  \end{tabular}%
  }
  \label{tab:dynamic_model_comparison}
\end{table*}




\end{document}
