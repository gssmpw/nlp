\newpage
\section{Prompts }
\subsection{Prompt Template for Detection of Contextual Privacy Violations}
\label{appendix_ci_detection}
The prompt that was used for detecting the primary context, essential and non-essential information in the user's prompt is:

\begin{tcolorbox}[
    colback=gray!5, colframe=black, coltitle=white, colbacktitle=gray!70, 
    fonttitle=\bfseries, title=Detection of Contextual Privacy Violations, 
    boxrule=0.8pt, width=0.8\textwidth, sharp corners, boxsep=0.3mm, 
    left=0.3mm, right=0.3mm, top=0.3mm, bottom=0.3mm]
    
    
\scriptsize
\textbf{Task:} Identify and classify sensitive information in the given text under three categories:  
\begin{itemize}\itemsep0pt
    \item \textbf{"primary context"} - The key task, question, or request present in the text.
    \item \textbf{"attributes essential to the context"} - Necessary details directly relevant to the primary context.
    \item \textbf{"sensitive attributes not essential to the context"} - Other sensitive information unrelated to the primary context.
\end{itemize}

\textbf{Examples:}

\textbf{Example 1:}  
\textbf{Text:}  
\textit{"Hi, my name is David Johns, and I need some advice on managing my diabetes. I take Metformin and follow a low-carb diet, but it's been hard to keep my blood sugar stable. I've been really stressed about my financial situation. I'm in debt and struggling to pay my medical bills."}  

\textbf{Output:}  
\texttt{
\{"primary context": ["need some advice on managing my diabetes"],\\
 "attributes essential to the context": [\\
 \hspace{5mm}"managing my diabetes",\\
 \hspace{5mm}"take Metformin",\\
 \hspace{5mm}"follow a low-carb diet",\\
 \hspace{5mm}"hard to keep my blood sugar stable"\\
 ],\\
 "sensitive attributes not essential to the context": [\\
 \hspace{5mm}"my name is David Johns",\\
 \hspace{5mm}"stressed about my financial situation",\\
 \hspace{5mm}"in debt",\\
 \hspace{5mm}"struggling to pay my medical bills"\\
 ]\}
}

\textbf{Example 2:}  
\textbf{Text:}  
\textit{"I recently moved to San Francisco and I'm having trouble finding a job. My email is john.doe@gmail.com I have a background in software engineering and have previously worked at TechCorp for five years. Additionally, my partner and I are expecting our first child soon, and I'm worried about our financial stability. Give me some tips for finding a job."}

\textbf{Output:}  
\texttt{
\{"primary context": ["give some tips for finding a job"],\\
 "attributes essential to the context": [\\
 \hspace{5mm}"having trouble finding a job",\\
 \hspace{5mm}"background in software engineering",\\
 \hspace{5mm}"worked at TechCorp for five years",\\
 \hspace{5mm}"give tips for finding a job"\\
 ],\\
 "sensitive attributes not essential to the context": [\\
 \hspace{5mm}"recently moved to San Francisco",\\
 \hspace{5mm}"my email is john.doe@gmail.com",\\
 \hspace{5mm}"expecting our first child",\\
 \hspace{5mm}"worried about our financial stability"\\
 ]\}
}

\textbf{Now classify the text below:}  

\textbf{Text:} \{input\_text\}

\end{tcolorbox}


\subsection{ Prompt Template for Intent Detection}
\label{appendix_intent_detection}
\begin{tcolorbox}[
    colback=gray!5, colframe=black, coltitle=white, colbacktitle=gray!70, 
    fonttitle=\bfseries, title=Intent Detection Prompt Template, 
    boxrule=0.8pt, width=0.8\textwidth, sharp corners, boxsep=0.5mm, 
    left=0.5mm, right=0.5mm, top=0.5mm, bottom=0.5mm]
\scriptsize
{{- if .System }}You are an intent detection assistant that identifies the primary intent or domain of a user's query. You must classify the intent into exactly one of the available categories below.{{ end }}

\textbf{User Prompt:} Identify the \textbf{PRIMARY} intent/domain of the text following these rules:
\begin{enumerate}\itemsep0pt
    \item Choose the best category from the provided list.
    \item If no exact match exists, select the closest category.
    \item DO NOT create new categories or provide explanations—return only a category name.
\end{enumerate}

\textbf{Available Categories:}
\begin{itemize}\itemsep0pt
    \item \textbf{Health\_And\_Wellness} - Health conditions, treatments, therapy, fitness, nutrition.
    \item \textbf{Financial\_And\_Corporate} - Banking, budgeting, loans, insurance, markets.
    \item \textbf{Employment\_And\_Applications} - Job searches, workplace, professional matters.
    \item \textbf{Academic\_And\_Education} - School, grades, learning resources, teaching.
    \item \textbf{Legal} - Cases, contracts, laws, government benefits.
    \item \textbf{Personal\_Relationships} - Family, friends, social life, childcare.
    \item \textbf{Travel} - Trip planning, transport, documents, accommodation.
    \item \textbf{Fandom} - Books, movies, TV, games, fan communities.
    \item \textbf{Hobbies\_And\_Habits} - Leisure, gaming, sports, collecting.
    \item \textbf{Sexual\_And\_Erotic} - Preferences, relationships, health, adult topics.
    \item \textbf{Politics} - Policies, elections, ideologies, civic matters.
    \item \textbf{Religion} - Beliefs, spirituality, religious events, worship.
\end{itemize}

\textbf{Examples:}
\begin{itemize}\itemsep0pt
    \item \textbf{Text:} "Looking for a new job after being laid off from my tech company" → \textbf{Employment\_And\_Applications}  
    \item \textbf{Text:} "Need advice for managing my anxiety and depression" → \textbf{Health\_And\_Wellness}  
    \item \textbf{Text:} "Want to discuss the latest episode of Game of Thrones" → \textbf{Fandom}  
    \item \textbf{Text:} "Planning a family vacation to Europe next summer" → \textbf{Travel}  
\end{itemize}

\textbf{Now classify this text. Return only a category name:}  
\textbf{Text:} \{input\_text\}

\end{tcolorbox}

\subsection{ Prompt Template for Task Detection}
\label{appendix_task_detection}
\begin{tcolorbox}[
    colback=gray!5, colframe=black, coltitle=white, colbacktitle=gray!70, 
    fonttitle=\bfseries, title=Task Detection Prompt Template, 
    boxrule=0.8pt, width=0.8\textwidth, sharp corners, boxsep=0.5mm, 
    left=0.5mm, right=0.5mm, top=0.5mm, bottom=0.5mm]
    
\scriptsize
\textbf{Task:} Identify the primary task or action type from the text using the predefined categories.

\textbf{Rules:}
\begin{itemize}\itemsep0pt
    \item Select the most appropriate category from the list.
    \item If no exact match exists, choose the closest category.
    \item Return only the category name—no explanations or additional text.
\end{itemize}

\textbf{Available Categories:}
\begin{itemize}\itemsep0pt
    \item Summarization
    \item Prompt\_Generation\_For\_AI\_Models
    \item Story\_And\_Script\_Generation
    \item Song\_And\_Poem\_Generation
    \item Character\_Description\_Generation
    \item Code\_Generation
    \item Code\_Editing\_And\_Debugging
    \item Communication\_Generation
    \item Non\_Fictional\_Document\_Generation
    \item Text\_Editing
    \item Comparison\_Ranking\_And\_Recommendation
    \item Brainstorming\_And\_Idea\_Generation
    \item Information\_Retrieval
    \item Problem\_Solving
    \item Explanation\_And\_Practical\_Advice
    \item Personal\_Advice
    \item Back\_And\_Forth\_Role\_Playing
    \item Answering\_Multiple\_Choice\_Questions
    \item Translation
    \item General\_Chitchat
\end{itemize}

\textbf{Examples:}

\textbf{Example 1:}  
\textbf{Text:} \textit{"Can you help me write a prompt that will make GPT-4 better at generating poetry?"}  
\textbf{Task:} \texttt{Prompt\_Generation\_For\_AI\_Models}

\textbf{Example 2:}  
\textbf{Text:} \textit{"I need help debugging this Python function that keeps giving me a TypeError."}  
\textbf{Task:} \texttt{Code\_Editing\_And\_Debugging}

\textbf{Example 3:}  
\textbf{Text:} \textit{"Write me a story about a detective solving a mystery in space."}  
\textbf{Task:} \texttt{Story\_And\_Script\_Generation}

\textbf{Example 4:}  
\textbf{Text:} \textit{"How do I tell my roommate that I'm moving out next month?"}  
\textbf{Task:} \texttt{Personal\_Advice}

\textbf{Now classify the text below:}  

\textbf{Text:} \{input\_text\}

\end{tcolorbox}

\subsection{ Dynamic Prompt Template for Sensitive Information Detection}
\label{appendix_dynamic_sentive_template}
\begin{tcolorbox}[
    colback=gray!5, colframe=black, coltitle=white, colbacktitle=gray!70, 
    fonttitle=\bfseries, title=Dynamic Sensitive Information Detection Prompt Template, 
    boxrule=0.8pt, width=0.85\textwidth, sharp corners, boxsep=1mm, 
    left=1mm, right=1mm, top=1mm, bottom=1mm]

\scriptsize
\textbf{ESSENTIAL PROMPT:}  

{{- if .System }}You are a privacy-focused assistant. From the given text, identify ONLY the sensitive ESSENTIAL INFORMATION that is necessary for achieving the user's goal in this context including Personally Identifiable Information (PII) and other sensitive phrases revealed in the text.{{ end }}

\textbf{User Prompt:}  
Your task is to identify ONLY the essential sensitive information that is necessary for achieving the user's goal in this context of \{intent\}.  

\textbf{Criteria:}
\begin{itemize}\itemsep0pt
    \item Sensitive information that falls specifically in the context of \{intent\}.
    \item Essential for getting a helpful response.
    \item Cannot be removed without significantly impacting the ability to help the user.
\end{itemize}

\textbf{Return format:}  
ESSENTIAL INFORMATION: [item 1, item 2, item 3]

\vspace{3pt}
\textbf{Examples:}  

\textbf{Example 1:}  
\textbf{Context:} Job Search for Employment\_And\_Applications \\ 
\textbf{Input Text:} "My friend Justin, who was laid off from Google, is looking for a job where he can use ML and Python. Do you have any advice for him?"  \\ 
\textbf{ESSENTIAL INFORMATION:} \texttt{["looking for job using ML and Python skills"]}  

\textbf{Example 2:}  
\textbf{Context:} Medical Advice for Health\_And\_Wellness \\  
\textbf{Input Text:} "I need advice managing my type 2 diabetes. My doctor prescribed Metformin 500mg twice daily last week. My sister also has diabetes and suggested trying a keto diet." \\  
\textbf{ESSENTIAL INFORMATION:} \texttt{["type 2 diabetes", "prescribed Metformin 500mg twice daily"]}  

\vspace{3pt}
\textbf{Now analyze the following text and extract ONLY the ESSENTIAL INFORMATION:}  
\textbf{Context:} \{intent\}  
\textbf{Input Text:} \{text\}  

\vspace{6pt}
\textbf{NON-ESSENTIAL PROMPT:}  

{{- if .System }}You are a privacy-focused assistant. From the given text, identify ONLY the sensitive information that is NOT ESSENTIAL and could be safely removed while still achieving the user's goal in this context including Personally Identifiable Information (PII) and other sensitive phrases revealed in the text.{{ end }}

\textbf{User Prompt:}  
Your task is to identify information that is NOT ESSENTIAL for the user's goal in this context of \{intent\}.  

\textbf{Criteria:}
\begin{itemize}\itemsep0pt
    \item All sensitive information that does not directly pertain to \{intent\}.
    \item Not necessary for achieving the intended goal.
    \item Could be removed or generalized while still getting a helpful response.
\end{itemize}

\textbf{Return format:}  
NON-ESSENTIAL INFORMATION: [item 1, item 2, item 3]

\vspace{3pt}
\textbf{Examples:}  

\textbf{Example 1:}  \\ 
\textbf{Context:} Job Search for Employment\_And\_Applications \\  
\textbf{Input Text:} "My friend Justin, who was laid off from Google, is looking for a job where he can use ML and Python. Do you have any advice for him?"  \\ 
\textbf{NON-ESSENTIAL INFORMATION:} \texttt{["my friend", "Justin", "laid off from Google"]}  

\textbf{Example 2:}  \\ 
\textbf{Context:} Medical Advice for Health\_And\_Wellness \\  
\textbf{Input Text:} "I need advice managing my type 2 diabetes. My doctor prescribed Metformin 500mg twice daily last week. My sister also has diabetes and suggested trying a keto diet."  \\ 
\textbf{NON-ESSENTIAL INFORMATION:} \texttt{["prescribed last week", "sister has diabetes", "suggested trying a keto diet"]}  

\vspace{3pt}
\textbf{Now analyze the following text and extract ONLY the NON-ESSENTIAL INFORMATION:}  \\
\textbf{Context:} \{intent\}  
\textbf{Input Text:} \{text\}  

\end{tcolorbox}

\subsection{ Structured Prompt Template for Sensitive Information Detection}
\label{appendix_structured_sentive_template}

\begin{tcolorbox}[
    colback=gray!5, colframe=black, coltitle=white, colbacktitle=gray!70, 
    fonttitle=\bfseries, title=Structured Sensitive Information Detection Prompt Template, 
    boxrule=0.8pt, width=0.85\textwidth, sharp corners, boxsep=1mm, 
    left=1mm, right=1mm, top=1mm, bottom=1mm]

\scriptsize
\textbf{ESSENTIAL PROMPT:}  

{{- if .System }}You are a privacy-focused assistant. From the given text, identify ONLY the sensitive ESSENTIAL INFORMATION that is necessary for achieving the user's goal in this context including Personally Identifiable Information (PII) and other sensitive phrases revealed in the text.{{ end }}

\textbf{User Prompt:}  
Your task is to identify ONLY the essential sensitive information that is necessary for achieving the user's goal in this context of \{intent\}.  

\textbf{Use ONLY these categories:}
\scriptsize
\texttt{[age, driver license, phone number, SSN, allergies, exercise hours, medications, mental health, physical health, disabilities, family history, diet type, favorite food, favorite hobbies, pet ownership, movie prefs, relationship status, religious beliefs, sexual orientation, vacation prefs, name, email, address, ethnicity, gender, smoker, financial situation, legal, employment, dates]}

\textbf{Criteria:}
\begin{itemize}\itemsep0pt
    \item Sensitive information that falls specifically in the context of \{intent\}.
    \item Essential for getting a helpful response.
    \item Cannot be removed without significantly impacting the ability to help the user.
\end{itemize}

\textbf{Return format:}  
ESSENTIAL INFORMATION: [item 1, item 2, item 3]

\vspace{3pt}
\textbf{Examples:}  

\textbf{Example 1:}  \\ 
\textbf{Context:} Employment\_And\_Applications  \\
\textbf{Input Text:} "My friend Justin, who was laid off from Google, is looking for a job where he can use ML and Python. Do you have any advice for him?" \\
\textbf{ESSENTIAL INFORMATION:} \texttt{["employment"]}  

\textbf{Example 2:}  
\textbf{Context:} Medical\_And\_Health \\ 
\textbf{Input Text:} "I need advice managing my type 2 diabetes. My doctor prescribed Metformin 500mg twice daily last week. My sister also has diabetes and suggested trying a keto diet."  \\
\textbf{ESSENTIAL INFORMATION:} \texttt{["physical health", "medications", "diet type"]}  

\vspace{3pt}
\textbf{Now identify the essential attributes from the predefined list:}  \\ 
\textbf{Context:} \{intent\}  
\textbf{Input Text:} \{text\}  

\vspace{6pt}
\textbf{NON-ESSENTIAL PROMPT:}  

{{- if .System }}You are a privacy-focused assistant. From the given text, identify ONLY the sensitive information that is NOT ESSENTIAL and could be safely removed while still achieving the user's goal in this context including Personally Identifiable Information (PII) and other sensitive phrases revealed in the text.{{ end }}

\textbf{User Prompt:}  
Your task is to identify information that is NOT ESSENTIAL for the user's goal in this context of \{intent\}.  

\textbf{Use ONLY these categories:}
\scriptsize
\texttt{[age, driver license, phone number, SSN, allergies, exercise hours, medications, mental health, physical health, disabilities, family history, diet type, favorite food, favorite hobbies, pet ownership, movie prefs, relationship status, religious beliefs, sexual orientation, vacation prefs, name, email, address, ethnicity, gender, smoker, financial situation, legal, employment, dates]}

\textbf{Criteria:}
\begin{itemize}\itemsep0pt
    \item Present or implied in the text.
    \item Not necessary for achieving the user's goal.
    \item Could be safely removed or generalized while still getting a helpful response.
\end{itemize}

\textbf{Return format:}  
NON-ESSENTIAL INFORMATION: [item 1, item 2, item 3]

\vspace{3pt}
\textbf{Examples:}  

\textbf{Example 1:}  \\ 
\textbf{Context:} Employment\_And\_Applications  \\ 
\textbf{Input Text:} "My friend Justin, who was laid off from Google, is looking for a job where he can use ML and Python. Do you have any advice for him?"  \\ 
\textbf{NON-ESSENTIAL INFORMATION:} \texttt{["name", "employment"]}  

\textbf{Example 2:}  \\ 
\textbf{Context:} Medical\_And\_Health \\  
\textbf{Input Text:} "I need advice managing my type 2 diabetes. My doctor prescribed Metformin 500mg twice daily last week. My sister also has diabetes and suggested trying a keto diet."  \\ 
\textbf{NON-ESSENTIAL INFORMATION:} \texttt{["family history", "diet type"]}  

\vspace{3pt}
\textbf{Now identify the non-essential attributes from the predefined list:}  \\ 
\textbf{Context:} \{intent\}  
\textbf{Input Text:} \{text\}  

\end{tcolorbox}


\subsection{Prompt Template For Reformulation}
\label{appendix_reformulation}

\begin{tcolorbox}[
    colback=gray!5, colframe=black, coltitle=white, colbacktitle=gray!70, 
    fonttitle=\bfseries, title=Privacy-Preserving Reformulation Prompt, 
    boxrule=0.8pt, width=0.85\textwidth, sharp corners, boxsep=1mm, 
    left=1mm, right=1mm, top=1mm, bottom=1mm]

\scriptsize
\textbf{System Prompt:}  
You are a privacy-conscious assistant that helps reformulate text to preserve privacy while maintaining the original query structure.

\vspace{3pt}
\textbf{User Instructions:}  
Reformulate this text to remove sensitive or unnecessary information while preserving the original query/request structure. The reformulation should:  
\begin{itemize}\itemsep0pt
    \item Keep the same type of request (question remains question, request remains request, advice request remains advice request).
    \item Remove or generalize non-essential information.
    \item Maintain the user's position as the requester.
    \item \textbf{NOT} provide an answer or implementation.
    \item Keep all essential information.
    \item Maintain the original intent.
    \item Do \textbf{not} provide explanations, justifications, or unnecessary extra text.
\end{itemize}

\vspace{3pt}
\textbf{Original Text:} \{text\}  
\textbf{Context:} \{intent\}  
\textbf{Essential Information:} \{essential\_info\}  
\textbf{Information to Remove:} \{removable\_info\}  

\vspace{3pt}
\textbf{Examples:}  
\textbf{Original:} "What's a message that I, as a cool college student, can say to someone whom I want to meet?"  
\textbf{Reformulated:} "What's a good way to initiate a conversation with someone I'd like to connect with?"  

\textbf{Original:} "I need help writing an email to Professor Smith about missing class yesterday due to food poisoning."  
\textbf{Reformulated:} "I need help writing an email about a class absence due to illness."  

\textbf{Original:} "Can you suggest a birthday gift for my sister Sarah who loves painting and just moved to Paris?"  
\textbf{Reformulated:} "Can you suggest a birthday gift for someone who enjoys painting?"  

\textbf{Original:} "I need advice on how to tell my girlfriend Jane that I lost my job at Google and we might need to cancel our vacation to Hawaii next month."  
\textbf{Reformulated:} "I need advice on how to discuss a job loss and its financial impact with my partner."  

\textbf{Original:} "Looking for advice on managing my diabetes. I take Metformin twice daily and my sister suggested trying a keto diet."  
\textbf{Reformulated:} "Looking for advice on managing diabetes and considering dietary changes."  

\textbf{Original:} "Can you give me advice on growing my TikTok following? I currently have 500 followers and post dance videos every day."  
\textbf{Reformulated:} "Can you give me advice on growing a social media following?"  

\vspace{3pt}
\textbf{Now, provide only the reformulated text on a new line:}  
\end{tcolorbox}


\subsection{ LLM-as-a-Judge Evaluation Prompt Template}
\label{appendix_evaluation}
\begin{tcolorbox}[
    colback=gray!5, colframe=black, coltitle=white, colbacktitle=gray!70, 
    fonttitle=\bfseries, title=LLM-as-a-Judge Evaluation Prompt, 
    boxrule=0.8pt, width=0.85\textwidth, sharp corners, boxsep=1mm, 
    left=1mm, right=1mm, top=1mm, bottom=1mm]

\scriptsize
\textbf{System Prompt:}  
You are an expert evaluator of contextual privacy in large language models ensuring there are no contextual privacy violations. Given a set of original and reformulated queries and responses, along with extracted attributes, assess privacy and utility across the following dimensions:  

\begin{itemize}\itemsep0pt
    \item \textbf{Privacy Non-Leakage}: Does the reformulated query successfully remove non-essential sensitive attributes?
    \item \textbf{Privacy Retention}: Does the reformulated query avoid introducing new sensitive attributes?
    \item \textbf{Privacy Coverage}: Does the reformulated query retain all essential sensitive attributes?
    \item \textbf{Query Relevance}: Does the reformulated query maintain the original user intent?
    \item \textbf{Response Relevance}: Is the reformulated response relevant compared to the original response?
    \item \textbf{Cross-Relevance}: Are responses generated from the original prompt still valid for the reformulated prompt?
\end{itemize}

\textbf{Example Evaluations:}  

\textbf{Example 1:}  
\textbf{Original Query:} “I'm trying to prepare a healthy meal plan for my family because my wife, Susan, is on a special diet due to her recent diagnosis of high blood pressure. We’re looking for recipes that are low in sodium and heart-healthy.”  

\textbf{Original Response:} “Try incorporating recipes that focus on lean proteins, vegetables, and whole grains. A heart-healthy diet often includes dishes like grilled chicken with quinoa and steamed broccoli.”  

\textbf{Reformulated Query:} “I'm looking for healthy meal plans that are low in sodium and suitable for someone with high blood pressure.”  

\textbf{Reformulated Response:} “Focus on recipes with lean proteins, vegetables, and whole grains. Dishes like grilled chicken with quinoa and steamed broccoli are often recommended.”  

\textbf{Evaluation Output:}  
{\footnotesize  
\{  
  "privacy non-leakage": true,\\
  "privacy retention": true,\\
  "privacy coverage": false,\\
  "query relevance": true,\\
  "response relevance": true,\\
  "cross-relevance": true  
\}  
}

\vspace{3pt}
\textbf{Now Evaluate the Following:}  

\textbf{Original Query:} \{original\_query\}  
\textbf{Original Response:} \{original\_response\}  
\textbf{Reformulated Query:} \{reformulated\_query\}  
\textbf{Reformulated Response:} \{reformulated\_response\}  

\textbf{Primary Context in Original Query:} \{original\_primary\_context\}  
\textbf{Essential Attributes in Original Query:} \{original\_related\_context\}  
\textbf{Sensitive Non-Essential Attributes in Original Query:} \{original\_not\_related\_context\}  

\textbf{Primary Context in Reformulated Query:} \{reformulated\_primary\_context\}  
\textbf{Essential Attributes in Reformulated Query:} \{reformulated\_related\_context\}  
\textbf{Sensitive Non-Essential Attributes in Reformulated Query:} \{reformulated\_not\_related\_context\}  

\vspace{3pt}
\textbf{Return only a JSON Output with the following keys:}  
\textbf{Privacy Non-Leakage, Privacy Retention, Privacy Coverage, Query Relevance, Response Relevance, Cross-Relevance, Answerability, Making Sense}.

\textbf{<|Assistant|>}
\end{tcolorbox}
