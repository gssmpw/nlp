
\section{Discussion and Conclusion}
\vspace{-5pt}
Drawing ideas from the contextual integrity theory, we defined the notion of contextual privacy for users interacting with LLM-based conversation agents.
We proposed a framework, grounded in our contextual privacy formulation, that acts as an intermediary between the user and the agent, and carefully reformulates user prompts to preserve contextual privacy while preserving the utility.  






This work serves as an initial step in exploring privacy protection in user interactions with conversational agents. There are several directions that future research can further investigate. 
First, our framework may not be suitable for user prompts that require preserving exact content, such as document translation or verbatim summarization. For example, translating a legal document demands keeping the original content intact, making it challenging to reformulate while preserving contextual privacy. For such tasks, alternative approaches like using placeholders or pseudonyms for sensitive information could help protect privacy without compromising accuracy, though this is beyond our current implementation. 
Second, our framework relies on LLM-based assessment of privacy violations which, while effective for demonstrating the approach, lacks formal privacy guarantees and can be sensitive to the prompt. Future work could explore combining our contextual approach with deterministic rules or provable privacy properties. 
Third, while we demonstrate how users can adjust reformulations to balance privacy and utility, developing precise metrics to quantify this trade-off remains an open research challenge. This is particularly important as the relationship between privacy preservation and task effectiveness can vary significantly across different contexts and user preferences. 
Finally, while our evaluation using selected ShareGPT conversations demonstrates the potential of our approach, broader testing across diverse contexts and user groups would better establish the framework's general applicability.









