\section{Experiments and Results}

% \begin{itemize}
% \item
% tables, ablation studies, test set description??
% \end{itemize}




\subsection{Detection Task Results and Analysis}

\paragraph{Zero Shot Results and Analysis}
The evaluation of models' zeroshot performance is presented in Table~\ref{tab:detector_result}. 
The zero-shot performance of the models we tested correlates intuitively to model size, with ChatGPT, the largest, having reasonably strong zero-shot performance, and Zephyr, with 3 billion parameters, having the weakest. However, Llama 3, far smaller than ChatGPT at only 8 billion parameters, achieved the best zero-shot performance among the LLMs that we tested, surpassing ChatGPT's macro F1 performance by 5.1\%. 

\paragraph{Fine-tuning Results and Analysis}
Both LoRA and Full fine-tuning results are showcased in Table~\ref{tab:detector_ft_result}. 
All fine-tuned models improved by at least 7.9\% relative to their zero-shot performances, with Zephyr 3B experiencing the greatest degree of improvement. Though it initially lagged behind other models, Zephyr gained nearly 20\% in terms of macro F1. Among fine-tuned models, Mistral 7B achieved the best performance, with a macro F1 of 0.87. Results seem to show that even smaller large language models can achieve similar and reasonably high performance at the task of toxic sentence classification. That said, all fine-tuned models remain more confident with non-toxic sentences than toxic sentences.

\begin{table*}[t]
\centering
 \resizebox{0.98\linewidth}{!}{
\begin{tabular}{c|c|c|c|c|c|c|c|c}
    \toprule
    \multirow{2}{*}{\textbf{Models}}& \multicolumn{4}{c}{\textbf{LoRA Fine-tuning Performance}} & \multicolumn{4}{|c}{\textbf{Full Fine-tuning Performance}}\\
     \cmidrule(lr){2-5}
     & {\textbf{Precision}} &  {\textbf{Recall}}  & {\textbf{F1}} &  \textbf{Accuracy} \\ 
     \toprule
    % Zephyr-3B & 0.863& 0.861& 0.862& 0.894 \\
    Mistral-7B  &
    Llama3-8B  &  \\
    % DeBerta-v3+Random Forest & 0.831& 0.835& 0.831& 0.830 \\
    \bottomrule
    \end{tabular}
    }
    \vspace{3mm}
\caption{Fine-tune Performance of Different Models on Toxic Review Detection Task.}
\label{tab:detector_ft_result}
\end{table*}

\paragraph{Compare Traditional ML with LLM Fine-tuning Performance}

A random forest trained on manually-specified features extracted by the fine-tuned DeBerta-v3 model performed well, particularly as it is a technique which did not require fine-tuning. It scored consistently well in precision and recall on both the toxic class as well as on the macro level, surpassing the macro F1 zero-shot performance of all LLMs, with the exception of Llama 3 8B (0.831 vs 0.840). However, this technique did not beat the fine-tuned models.

Using the features extracted from the DeBerta-v3 model, we were able to analyze our annotations visually on the basis of key words and phrases associated with our annotation scheme. This allowed for some visual confirmation of annotation consistency, as well as a sense of how distinct various categories might be for a language model in general. In Figure 1, we show a PCA-2 representation of our human-identified training data with our final toxic and non-toxic labels. The two Principal Component axes visualized account for roughly 44\% of the variance in the data, PC1 being associated primarily with phrases 'blunt criticism' and 'negative judgement', and negatively associated with phrases like 'encouraging, constructive, specific', while PC2 is associated with terms such as 'specific feedback' and 'constructive criticism'.
\begin{figure}[ht]
    \centering
    \includegraphics[width=1\linewidth]{images/PCA2_toxic-non.png}
    \caption{PCA-2 Representation of Human-Classified Training Sentences.}
    \label{fig:PCA2 - toxic vs not}
\end{figure}

% A PCA-3 representation was particularly informative during the annotation process because it allowed us to visually confirm that a variety of annotation sub-classes were grouped appropriately. It also allowed us to predict that sub-classes with higher overlap might prove most difficult for our models.

The groupings in the PC visualizations indicated that our annotation scheme was internally consistent, even with many distinct subcategories. 

\subsection{Revision Task Results}

\begin{table*}[t]
\centering
\fontsize{11pt}{11pt}\selectfont
\begin{tabular}{lllllllllllll}
\toprule
\multicolumn{1}{c}{\textbf{task}} & \multicolumn{2}{c}{\textbf{Mir}} & \multicolumn{2}{c}{\textbf{Lai}} & \multicolumn{2}{c}{\textbf{Ziegen.}} & \multicolumn{2}{c}{\textbf{Cao}} & \multicolumn{2}{c}{\textbf{Alva-Man.}} & \multicolumn{1}{c}{\textbf{avg.}} & \textbf{\begin{tabular}[c]{@{}l@{}}avg.\\ rank\end{tabular}} \\
\multicolumn{1}{c}{\textbf{metrics}} & \multicolumn{1}{c}{\textbf{cor.}} & \multicolumn{1}{c}{\textbf{p-v.}} & \multicolumn{1}{c}{\textbf{cor.}} & \multicolumn{1}{c}{\textbf{p-v.}} & \multicolumn{1}{c}{\textbf{cor.}} & \multicolumn{1}{c}{\textbf{p-v.}} & \multicolumn{1}{c}{\textbf{cor.}} & \multicolumn{1}{c}{\textbf{p-v.}} & \multicolumn{1}{c}{\textbf{cor.}} & \multicolumn{1}{c}{\textbf{p-v.}} &  &  \\ \midrule
\textbf{S-Bleu} & 0.50 & 0.0 & 0.47 & 0.0 & 0.59 & 0.0 & 0.58 & 0.0 & 0.68 & 0.0 & 0.57 & 5.8 \\
\textbf{R-Bleu} & -- & -- & 0.27 & 0.0 & 0.30 & 0.0 & -- & -- & -- & -- & - &  \\
\textbf{S-Meteor} & 0.49 & 0.0 & 0.48 & 0.0 & 0.61 & 0.0 & 0.57 & 0.0 & 0.64 & 0.0 & 0.56 & 6.1 \\
\textbf{R-Meteor} & -- & -- & 0.34 & 0.0 & 0.26 & 0.0 & -- & -- & -- & -- & - &  \\
\textbf{S-Bertscore} & \textbf{0.53} & 0.0 & {\ul 0.80} & 0.0 & \textbf{0.70} & 0.0 & {\ul 0.66} & 0.0 & {\ul0.78} & 0.0 & \textbf{0.69} & \textbf{1.7} \\
\textbf{R-Bertscore} & -- & -- & 0.51 & 0.0 & 0.38 & 0.0 & -- & -- & -- & -- & - &  \\
\textbf{S-Bleurt} & {\ul 0.52} & 0.0 & {\ul 0.80} & 0.0 & 0.60 & 0.0 & \textbf{0.70} & 0.0 & \textbf{0.80} & 0.0 & {\ul 0.68} & {\ul 2.3} \\
\textbf{R-Bleurt} & -- & -- & 0.59 & 0.0 & -0.05 & 0.13 & -- & -- & -- & -- & - &  \\
\textbf{S-Cosine} & 0.51 & 0.0 & 0.69 & 0.0 & {\ul 0.62} & 0.0 & 0.61 & 0.0 & 0.65 & 0.0 & 0.62 & 4.4 \\
\textbf{R-Cosine} & -- & -- & 0.40 & 0.0 & 0.29 & 0.0 & -- & -- & -- & -- & - & \\ \midrule
\textbf{QuestEval} & 0.23 & 0.0 & 0.25 & 0.0 & 0.49 & 0.0 & 0.47 & 0.0 & 0.62 & 0.0 & 0.41 & 9.0 \\
\textbf{LLaMa3} & 0.36 & 0.0 & \textbf{0.84} & 0.0 & {\ul{0.62}} & 0.0 & 0.61 & 0.0 &  0.76 & 0.0 & 0.64 & 3.6 \\
\textbf{our (3b)} & 0.49 & 0.0 & 0.73 & 0.0 & 0.54 & 0.0 & 0.53 & 0.0 & 0.7 & 0.0 & 0.60 & 5.8 \\
\textbf{our (8b)} & 0.48 & 0.0 & 0.73 & 0.0 & 0.52 & 0.0 & 0.53 & 0.0 & 0.7 & 0.0 & 0.59 & 6.3 \\  \bottomrule
\end{tabular}
\caption{Pearson correlation on human evaluation on system output. `R-': reference-based. `S-': source-based.}
\label{tab:sys}
\end{table*}



\begin{table}%[]
\centering
\fontsize{11pt}{11pt}\selectfont
\begin{tabular}{llllll}
\toprule
\multicolumn{1}{c}{\textbf{task}} & \multicolumn{1}{c}{\textbf{Lai}} & \multicolumn{1}{c}{\textbf{Zei.}} & \multicolumn{1}{c}{\textbf{Scia.}} & \textbf{} & \textbf{} \\ 
\multicolumn{1}{c}{\textbf{metrics}} & \multicolumn{1}{c}{\textbf{cor.}} & \multicolumn{1}{c}{\textbf{cor.}} & \multicolumn{1}{c}{\textbf{cor.}} & \textbf{avg.} & \textbf{\begin{tabular}[c]{@{}l@{}}avg.\\ rank\end{tabular}} \\ \midrule
\textbf{S-Bleu} & 0.40 & 0.40 & 0.19* & 0.33 & 7.67 \\
\textbf{S-Meteor} & 0.41 & 0.42 & 0.16* & 0.33 & 7.33 \\
\textbf{S-BertS.} & {\ul0.58} & 0.47 & 0.31 & 0.45 & 3.67 \\
\textbf{S-Bleurt} & 0.45 & {\ul 0.54} & {\ul 0.37} & 0.45 & {\ul 3.33} \\
\textbf{S-Cosine} & 0.56 & 0.52 & 0.3 & {\ul 0.46} & {\ul 3.33} \\ \midrule
\textbf{QuestE.} & 0.27 & 0.35 & 0.06* & 0.23 & 9.00 \\
\textbf{LlaMA3} & \textbf{0.6} & \textbf{0.67} & \textbf{0.51} & \textbf{0.59} & \textbf{1.0} \\
\textbf{Our (3b)} & 0.51 & 0.49 & 0.23* & 0.39 & 4.83 \\
\textbf{Our (8b)} & 0.52 & 0.49 & 0.22* & 0.43 & 4.83 \\ \bottomrule
\end{tabular}
\caption{Pearson correlation on human ratings on reference output. *not significant; we cannot reject the null hypothesis of zero correlation}
\label{tab:ref}
\end{table}


\begin{table*}%[]
\centering
\fontsize{11pt}{11pt}\selectfont
\begin{tabular}{lllllllll}
\toprule
\textbf{task} & \multicolumn{1}{c}{\textbf{ALL}} & \multicolumn{1}{c}{\textbf{sentiment}} & \multicolumn{1}{c}{\textbf{detoxify}} & \multicolumn{1}{c}{\textbf{catchy}} & \multicolumn{1}{c}{\textbf{polite}} & \multicolumn{1}{c}{\textbf{persuasive}} & \multicolumn{1}{c}{\textbf{formal}} & \textbf{\begin{tabular}[c]{@{}l@{}}avg. \\ rank\end{tabular}} \\
\textbf{metrics} & \multicolumn{1}{c}{\textbf{cor.}} & \multicolumn{1}{c}{\textbf{cor.}} & \multicolumn{1}{c}{\textbf{cor.}} & \multicolumn{1}{c}{\textbf{cor.}} & \multicolumn{1}{c}{\textbf{cor.}} & \multicolumn{1}{c}{\textbf{cor.}} & \multicolumn{1}{c}{\textbf{cor.}} &  \\ \midrule
\textbf{S-Bleu} & -0.17 & -0.82 & -0.45 & -0.12* & -0.1* & -0.05 & -0.21 & 8.42 \\
\textbf{R-Bleu} & - & -0.5 & -0.45 &  &  &  &  &  \\
\textbf{S-Meteor} & -0.07* & -0.55 & -0.4 & -0.01* & 0.1* & -0.16 & -0.04* & 7.67 \\
\textbf{R-Meteor} & - & -0.17* & -0.39 & - & - & - & - & - \\
\textbf{S-BertScore} & 0.11 & -0.38 & -0.07* & -0.17* & 0.28 & 0.12 & 0.25 & 6.0 \\
\textbf{R-BertScore} & - & -0.02* & -0.21* & - & - & - & - & - \\
\textbf{S-Bleurt} & 0.29 & 0.05* & 0.45 & 0.06* & 0.29 & 0.23 & 0.46 & 4.2 \\
\textbf{R-Bleurt} & - &  0.21 & 0.38 & - & - & - & - & - \\
\textbf{S-Cosine} & 0.01* & -0.5 & -0.13* & -0.19* & 0.05* & -0.05* & 0.15* & 7.42 \\
\textbf{R-Cosine} & - & -0.11* & -0.16* & - & - & - & - & - \\ \midrule
\textbf{QuestEval} & 0.21 & {\ul{0.29}} & 0.23 & 0.37 & 0.19* & 0.35 & 0.14* & 4.67 \\
\textbf{LlaMA3} & \textbf{0.82} & \textbf{0.80} & \textbf{0.72} & \textbf{0.84} & \textbf{0.84} & \textbf{0.90} & \textbf{0.88} & \textbf{1.00} \\
\textbf{Our (3b)} & 0.47 & -0.11* & 0.37 & 0.61 & 0.53 & 0.54 & 0.66 & 3.5 \\
\textbf{Our (8b)} & {\ul{0.57}} & 0.09* & {\ul 0.49} & {\ul 0.72} & {\ul 0.64} & {\ul 0.62} & {\ul 0.67} & {\ul 2.17} \\ \bottomrule
\end{tabular}
\caption{Pearson correlation on human ratings on our constructed test set. 'R-': reference-based. 'S-': source-based. *not significant; we cannot reject the null hypothesis of zero correlation}
\label{tab:con}
\end{table*}

\section{Results}
We benchmark the different metrics on the different datasets using correlation to human judgement. For content preservation, we show results split on data with system output, reference output and our constructed test set: we show that the data source for evaluation leads to different conclusions on the metrics. In addition, we examine whether the metrics can rank style transfer systems similar to humans. On style strength, we likewise show correlations between human judgment and zero-shot evaluation approaches. When applicable, we summarize results by reporting the average correlation. And the average ranking of the metric per dataset (by ranking which metric obtains the highest correlation to human judgement per dataset). 

\subsection{Content preservation}
\paragraph{How do data sources affect the conclusion on best metric?}
The conclusions about the metrics' performance change radically depending on whether we use system output data, reference output, or our constructed test set. Ideally, a good metric correlates highly with humans on any data source. Ideally, for meta-evaluation, a metric should correlate consistently across all data sources, but the following shows that the correlations indicate different things, and the conclusion on the best metric should be drawn carefully.

Looking at the metrics correlations with humans on the data source with system output (Table~\ref{tab:sys}), we see a relatively high correlation for many of the metrics on many tasks. The overall best metrics are S-BertScore and S-BLEURT (avg+avg rank). We see no notable difference in our method of using the 3B or 8B model as the backbone.

Examining the average correlations based on data with reference output (Table~\ref{tab:ref}), now the zero-shoot prompting with LlaMA3 70B is the best-performing approach ($0.59$ avg). Tied for second place are source-based cosine embedding ($0.46$ avg), BLEURT ($0.45$ avg) and BertScore ($0.45$ avg). Our method follows on a 5. place: here, the 8b version (($0.43$ avg)) shows a bit stronger results than 3b ($0.39$ avg). The fact that the conclusions change, whether looking at reference or system output, confirms the observations made by \citet{scialom-etal-2021-questeval} on simplicity transfer.   

Now consider the results on our test set (Table~\ref{tab:con}): Several metrics show low or no correlation; we even see a significantly negative correlation for some metrics on ALL (BLEU) and for specific subparts of our test set for BLEU, Meteor, BertScore, Cosine. On the other end, LlaMA3 70B is again performing best, showing strong results ($0.82$ in ALL). The runner-up is now our 8B method, with a gap to the 3B version ($0.57$ vs $0.47$ in ALL). Note our method still shows zero correlation for the sentiment task. After, ranks BLEURT ($0.29$), QuestEval ($0.21$), BertScore ($0.11$), Cosine ($0.01$).  

On our test set, we find that some metrics that correlate relatively well on the other datasets, now exhibit low correlation. Hence, with our test set, we can now support the logical reasoning with data evidence: Evaluation of content preservation for style transfer needs to take the style shift into account. This conclusion could not be drawn using the existing data sources: We hypothesise that for the data with system-based output, successful output happens to be very similar to the source sentence and vice versa, and reference-based output might not contain server mistakes as they are gold references. Thus, none of the existing data sources tests the limits of the metrics.  


\paragraph{How do reference-based metrics compare to source-based ones?} Reference-based metrics show a lower correlation than the source-based counterpart for all metrics on both datasets with ratings on references (Table~\ref{tab:sys}). As discussed previously, reference-based metrics for style transfer have the drawback that many different good solutions on a rewrite might exist and not only one similar to a reference.


\paragraph{How well can the metrics rank the performance of style transfer methods?}
We compare the metrics' ability to judge the best style transfer methods w.r.t. the human annotations: Several of the data sources contain samples from different style transfer systems. In order to use metrics to assess the quality of the style transfer system, metrics should correctly find the best-performing system. Hence, we evaluate whether the metrics for content preservation provide the same system ranking as human evaluators. We take the mean of the score for every output on each system and the mean of the human annotations; we compare the systems using the Kendall's Tau correlation. 

We find only the evaluation using the dataset Mir, Lai, and Ziegen to result in significant correlations, probably because of sparsity in a number of system tests (App.~\ref{app:dataset}). Our method (8b) is the only metric providing a perfect ranking of the style transfer system on the Lai data, and Llama3 70B the only one on the Ziegen data. Results in App.~\ref{app:results}. 


\subsection{Style strength results}
%Evaluating style strengths is a challenging task. 
Llama3 70B shows better overall results than our method. However, our method scores higher than Llama3 70B on 2 out of 6 datasets, but it also exhibits zero correlation on one task (Table~\ref{tab:styleresults}).%More work i s needed on evaluating style strengths. 
 
\begin{table}%[]
\fontsize{11pt}{11pt}\selectfont
\begin{tabular}{lccc}
\toprule
\multicolumn{1}{c}{\textbf{}} & \textbf{LlaMA3} & \textbf{Our (3b)} & \textbf{Our (8b)} \\ \midrule
\textbf{Mir} & 0.46 & 0.54 & \textbf{0.57} \\
\textbf{Lai} & \textbf{0.57} & 0.18 & 0.19 \\
\textbf{Ziegen.} & 0.25 & 0.27 & \textbf{0.32} \\
\textbf{Alva-M.} & \textbf{0.59} & 0.03* & 0.02* \\
\textbf{Scialom} & \textbf{0.62} & 0.45 & 0.44 \\
\textbf{\begin{tabular}[c]{@{}l@{}}Our Test\end{tabular}} & \textbf{0.63} & 0.46 & 0.48 \\ \bottomrule
\end{tabular}
\caption{Style strength: Pearson correlation to human ratings. *not significant; we cannot reject the null hypothesis of zero corelation}
\label{tab:styleresults}
\end{table}

\subsection{Ablation}
We conduct several runs of the methods using LLMs with variations in instructions/prompts (App.~\ref{app:method}). We observe that the lower the correlation on a task, the higher the variation between the different runs. For our method, we only observe low variance between the runs.
None of the variations leads to different conclusions of the meta-evaluation. Results in App.~\ref{app:results}.
\paragraph{Zero Shot Results and Analysis}

We present the Zero-shot performance in Table~\ref{gen_zeroshot_result}. It shows that revisions tend to have significant qualitative drawbacks in our evaluation, both in style and in content. In terms of style, their rephrasings are often very verbose, frequently being at least twice as long as the original sentence. Maintaining consistent formatting also seems to be very challenging, and base models will, more often than not, modify numbers without cause. Additionally, they will at times try to be more helpful or make more suggestions than would be appropriate or make sense in the context of a paper review. The content of critiques will occasionally be tweaked subtly, such that it may not convey exactly what was intended by the author. Furthermore, the revisions will often make some of the same mistakes made in the original review; some revisions maintain a lack of an explicitly constructive tone that was present in the original review. A few examples of these issues are present in Table~\ref{tab:detoxified_sentence}.

\begin{table}[t]
\centering
 \resizebox{0.98\linewidth}{!}{
\begin{tabular}{c|c|c|c}
    \toprule
    \multirow{2}{*}{\textbf{Models}}& \multicolumn{3}{c}{\textbf{Performance}}\\
     \cmidrule(lr){2-4}
     & {\textbf{Human}} &  {\textbf{GPT-4 Judge}}  & {\textbf{Toxic Detection Rate}}  \\ 
     \toprule
    % Zephyr  3B Zero-shot& 0.696& 0.752& 0.666 & 0.680  \\
    Phi-7B & - & - &  \\ 
    Qwen-7B & - & - & \\ 
    Mistral 7B &74 & 82 &  \\
    Gemma-7B & - & - &  \\
 LLaMA-3 8B & - & - & \\
 LLaMA-3 70B & - & - & -\\
 ChatGPT 3.5 & 77 & 83 & - \\
    \bottomrule
    \end{tabular}
    }
    \vspace{3mm}
\caption{Zeroshot Performance of Different Models on Toxic Review Detection Task.}
\label{tab:gen_zeroshot_result}
\end{table}

\paragraph{Fine-tuning Results and Analysis}

The results of this evaluation are presented in Table~\ref{tab:gen_ft_result}. We conducted an evaluation of sentence revision performance across all revision models on a common set of 100 test set sentences. We either 'Approved' or 'Rejected' a model's revision based on a few standards; If a revision accurately conveyed the original critique, and altered the tone to be more encouraging or constructive than the original sentence in a way that makes sense in the paper review context, it was Approved, otherwise Rejected. Similarly, we prompted GPT-4 to approve or reject the revised sentences on the basis of these criteria as well (Full prompt in Appendix).

\begin{table*}[t]
\centering
 \resizebox{0.98\linewidth}{!}{
\begin{tabular}{c|c|c|c|c|c|c}
    \toprule
    \multirow{2}{*}{\textbf{Models}}& \multicolumn{3}{c}{\textbf{LoRA Fine-tuning Performance}} & \multicolumn{3}{|c}{\textbf{Full Fine-tuning Performance}}\\
     \cmidrule(lr){2-4}\cmidrule(lr){5-7}
     & {\textbf{Human}} &  {\textbf{GPT-4 Judge}}  & {\textbf{Toxic Dection Rate}} &  {\textbf{Human}} &  {\textbf{GPT-4 Judge}}  & {\textbf{Toxic Dection Rate}}   \\ 
     \toprule
    Mistral-7B  & \\
    Llama3-8B  &  \\
    \bottomrule
    \end{tabular}
    }
    \vspace{3mm}
\caption{Fine-tune Performance of Different Models on Toxic Review Revision Task.}
\label{tab:gen_ft_result}
\end{table*}

In the estimation of both we human reviewers, as well as GPT-4, GPT-3.5 produced the highest quality revisions, which gives us some additional confidence that it was a good choice for the creation of our revision dataset. However, the other results were not so clear. Interestingly, while Zephyr's revisions improved by 15 points out of 100 through fine-tuning according to our human review, and Mistral experienced an improvement, though a milder one, in the estimation of GPT-4, the inverse was not true; Human review determined that Mistral's revision quality decreased substantially as a result of fine-tuning, while GPT-4 found that Zephyr's decreased by 5 points. One observation for why this might be the case is that the fine-tuned Mistral model in particular tended to hallucinate frequently when processing numbers, acronyms, or unfamiliar names. For example, when presented with the sentence

\textit{"The De-Identification Quality metric which you use would obviously get the optimal 20\% accuracy here as the we have no way of associating images of brains with images of faces, and even if we do, you have not demonstrated that you method would perform well on it either."}

\noindent Mistral's revision was

\textit{"The evaluation metric used, Deferred Identification, would achieve a 70 percent accuracy score in this scenario, as it is challenging to identify images with brain structures as having a specific facial expression. It would be beneficial to further explore the performance of the method on this task."}

Zephyr, on the other hand, has generally shown greater strength at learning to maintain specifics and formatting in the fine-tune. Some examples can be viewed in Table 7.

In order to get a better sense of whether our fine-tuned models' revisions represented a reasonable degree of improvement over the base models' revisions, we conducted an additional experiment with GPT-4. For both Mistral and Zephyr, we randomly selected 500 pairs of corresponding base and fine-tuned revisions. For a given pair, we randomized the order to avoid bias, and prompted GPT-4 to select which of the two sentences, "from scientific paper reviews," "was more conducive to a productive and encouraging review environment." (Full prompt in Appendix A) With this context, GPT-4 selected Mistral's fine-tuned revisions over its base model revisions 89.0\% of the time, and Zephyr's fine-tuned revisions 87.8\% of the time (Table 6). These strong results are likely due to the fact that this comparison does not attempt to filter for hallucinations or whether a revision correctly conveys a particular critique. That said, this result highlights the success of the fine-tuning process in shaping a productive and encouraging tone in both of these models.
\begin{table*}[t]
    \centering 
    \setlength{\tabcolsep}{4pt}
    \begin{tabular}{@{}p{0.4\linewidth}p{0.4\linewidth} p{0.2\linewidth} }
        \toprule
          \textbf{Original Sentence} & \textbf{GPT-3.5 Revision} & \textbf{Human Preference}\\
        \toprule
        I find this assumption absurd. & I respectfully disagree with this assumption. & Revision.\\
        \midrule 
        Overall the novelty of the approach and the proposed problem is incremental. & The approach and proposed problem show incremental novelty, which is a positive aspect of the study. & Original, the model revision change the meaning of the original sentences.\\
        \midrule 
        Notation and definitions used in (9) are wrong and confusing. & The notation and definitions used in (9) could benefit from clarification and correction to avoid confusion. & Revision \\
        \midrule
        There are so many errors like this and the paper needs serious rewriting. & There are several errors throughout the paper, and it would benefit from a thorough revision. & Revision\\
    \bottomrule
    \end{tabular}
    % }
    \caption{Examples of Detoxified Sentence from GPT.}
    \label{tab:detoxified_sentence}
\end{table*}
