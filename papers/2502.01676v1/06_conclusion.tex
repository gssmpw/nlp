\section{Conclusion and Future Work}
Our work is the first to explore toxicity in peer review. Lacking a comprehensive guideline, we first defined toxicity and identified four key toxic categories. We developed a two-stage annotation process, ensuring our toxicity detection annotations are reliable and our final testing dataset includes only sentences with unanimous agreement among annotators. 
We then benchmarked various models, including a general toxic detection, a sentiment analysis model, and both open-source and closed-source large language models. Our results suggest that Open-source models struggled to align with human judgments, highlighting the challenge of detecting toxicity in peer reviews. Conversely, closed-source models like GPT-3.5 and GPT-4 showed much better alignment, suggesting their potential with careful use. Future work could involve using these models to generate synthetic data for fine-tuning open-source models.

One limitation of this study is that it only evaluated LLaVA as the target Vision Language Model (VLM), which may limit the generalizability of the findings to other models. Additionally, the alignment of visual attention heatmaps for non-existing objects was not assessed, indicating that further analysis is needed in this area. 

Moreover, the experiments were conducted solely using the MSCOCO dataset, and future work should expand the evaluation to include additional datasets to ensure the robustness and broader applicability of the results. Furthermore, since datasets that contain both questions and corresponding answers alongside matching segmentation data, which can be used to evaluate object hallucination, are scarce, it may be necessary to develop such datasets.
