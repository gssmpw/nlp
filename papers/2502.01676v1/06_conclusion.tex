\section{Conclusion and Future Work}
Our work is the first to explore toxicity in peer review. Lacking a comprehensive guideline, we first defined toxicity and identified four key toxic categories. We developed a two-stage annotation process, ensuring our toxicity detection annotations are reliable and our final testing dataset includes only sentences with unanimous agreement among annotators. 
We then benchmarked various models, including a general toxic detection, a sentiment analysis model, and both open-source and closed-source large language models. Our results suggest that Open-source models struggled to align with human judgments, highlighting the challenge of detecting toxicity in peer reviews. Conversely, closed-source models like GPT-3.5 and GPT-4 showed much better alignment, suggesting their potential with careful use. Future work could involve using these models to generate synthetic data for fine-tuning open-source models.

\section*{Limitations and Ethical Considerations}

\noindent\textbf{Limitations.} The primary limitation of our work is that it extends only the dataset provided by MUSE and employs DeepSeek-v3 for question generation. 
To mitigate this generalization risk, we have released our code and the generated audit suite, allowing researchers to utilize our framework to create additional audit datasets and evaluate their quality. Meanwhile, this is also our future work to extend our framework to other benchmarks.

\noindent\textbf{Ethical Considerations.} Machine unlearning can be employed to mitigate risks associated with LLMs in terms of privacy, security, bias, and copyright. Our work is dedicated to providing a comprehensive evaluation framework to help researchers better understand the unlearning effectiveness of LLMs, which we believe will have a positive impact on society.