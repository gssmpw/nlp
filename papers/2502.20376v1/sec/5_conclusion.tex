\vspace{6pt}
\section{Conclusions}

In this work, we explored the role of tight conditioning in addressing the challenges of the inversion task for diffusion-based image editing.
While significant progress has been made in image editing with diffusion models, these models continue to struggle with complex, real-world images that fall outside their training distribution—precisely the type of images users often wish to edit. This challenge motivated our focus on improving performance in such demanding scenarios.

We demonstrated the power of using an image as a conditioning input, reaffirming the adage that ``a picture is worth a thousand words''. Conditioning on an image significantly enhances inversion quality compared to relying solely on text prompts, offering a more robust solution for real-world cases.
Our method provides a plug-and-play enhancement that is compatible with any inversion technique. Experimental results show that Tight Inversion improves both reconstruction fidelity and editing quality, without imposing significant computational or runtime overhead.

However, our approach is not without limitations. It is constrained by the inherent tradeoff between reconstruction accuracy and editability, as excessively strong conditioning can reduce the flexibility required for effective editing.

In this work, we employed IP-Adapter and PuLID to condition the model on the source image. However, our method is versatile and can be integrated with other image conditioning mechanisms. As future work, we aim to develop novel image conditioning techniques specifically tailored to further enhance the inversion task.

