\section{Introduction}

Speech Language Models (\slms) have gained significant interest from researchers~\cite{peng2024survey, cui2024recent, ji2024wavchat, latif2023sparks}, demonstrating remarkable performance in traditional speech tasks~\cite{valle, elmakies2025unsupervisedspeechsegmentationgeneral}, diverse generative applications~\cite{yang2023uniaudio, yang2024uniaudio}, and reasoning over speech and audio signals~\cite{salmonn, qwen_audio}.

\slms can generally be classified into two main categories: (i) generative speech \ac{LMs} (which can also incorporate text) and (ii) speech-aware \ac{LMs}. The first category follows a similar pre-training approach to text-based \ac{LLMs}, directly maximising the likelihood of speech considering both input and output, typically by representing audio as a sequence of discrete tokens. The second category consists of pre-trained text \ac{LMs} adapted to process speech inputs. In this work, we focus on the first.

\begin{figure}[t]
  \includegraphics[width=\columnwidth]{media/teaser.png}
  \caption{Comparing Topic-StoryCloze performance of different \slms as a function of training compute. Model size is indicated by the size of the circle.}
  \label{fig:teaser}
\end{figure}

Training high-quality \slms can be highly resource intensive~\cite{twist, cuervo2024scaling, scaling_interleaving, spiritlm, defossez2024moshi}. For example, ~\citet{spiritlm} trained their \slm on approximately $570k$ hours of speech data, while ~\citet{defossez2024moshi} utilised around $7M$ hours. Additionally, ~\citet{cuervo2024scaling} proposed \slm scaling laws, suggesting that training high-quality \slms requires $\sim3X$ more data compared to text-based counterparts. These computational demands restrict the required fundamental research aimed at enhancing \slms, such as advancements in speech tokenisation, efficient acoustic modelling, etc.


In the \ac{NLP} community, numerous studies have investigated efficient model training techniques, including masked language models such as Cramming~\citep{geiping2023cramming} and ModernBERT~\citep{warner2024modernbert}, along with next-token prediction LLMs such as MobileLLM~\citep{mobilellm}. These methods include implementation efficiencies, architectural improvements, data selection strategies, and enhancements to the overall training pipeline. 

Inspired by Cramming~\cite{geiping2023cramming} in text, we investigate compute-limited \slm training, which we term \emph{Slamming}. We pose the question: \emph{Is it possible to train high-quality \slms using a single GPU within 24 hours?} For that, we conduct an extensive empirical analysis exploring how different training components influence performance. From this, we derive a training recipe that maximises model performance within a fixed compute budget. Specifically, we investigate the impact of model initialisation and architecture, various optimisers and learning rate schedulers, data selection strategies - including the role of synthetic data, text-interleaving and preference optimisation. 

We believe that developing these training strategies and proving their feasibility will empower the speech and audio research community to advance \slms beyond the scope of large, well-funded academic and industrial labs. Figure~\ref{fig:teaser} illustrates the performance of various \slms relative to their training compute budget, with circle sizes representing the size of the models. Furthermore, we compare our results with the scaling performance predicted from \citet{cuervo2024scaling}. Although the authors present a somewhat pessimistic view of the computational resources needed to train high-quality \slms, we empirically show that reality is more promising, demonstrating that it is possible to significantly exceed the predicted performance per unit of compute. We encourage the community to refine and expand scaling laws specifically tailored for \slm training across various settings.

\newpara{Our main contributions are:} 
\begin{enumerate}
    \item We introduce \method, a training recipe for efficiently training high-quality \slms using a single $A5000$ GPU within $24$ hours.
    \item We carry out extensive experiments exploring model initialisation and architecture, optimisation, data collection and generation, and training objectives (i.e., preference optimisation and text-speech interleaving), providing insights into the impact of each component on model performance.
    \item Building on these insights, we scale the compute budget to two $A100$ GPUs for $48$ hours and demonstrate that our model achieves performance on par with state-of-the-art models that require substantially more compute.
\end{enumerate}
We open-source all code, models, training recipes, and synthetic datasets.