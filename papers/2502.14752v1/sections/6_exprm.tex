\section{Experiments}
We conduct an extensive set of experiments on \benchall to rigorously evaluate the performance and capabilities of current LLMs.


\begin{table*}[ht!] \small
    \centering
    \begin{tabular}{lcccc}
        \toprule
        \textbf{Model} & \textbf{Size} & \textbf{Call Accuracy} & \textbf{Execution Accuracy} & \textbf{Speed Up} \\ 
        \midrule
        \multicolumn{5}{c}{\textit{Domain-Specific Models}} \\
        Qwen2.5-Coder           & $7$B        & $0.00$~/~$0.00$                     & $0.00$~/~$0.00$                       & $0.00$~/~$0.00$     \\
        DeepSeek-Coder          & $6.7$B      & $0.00$~/~$1.81$                     & $0.00$~/~$1.81$                       & $0.00$~/~$\mathbf{0.94}$     \\
        Qwen2.5-Coder-sft      & $7$B        & $17.47$~/~$16.27$                    & $17.47$~/~$15.67$                     & $\mathbf{0.98}$~/~$0.92$     \\
        DeepSeek-Coder-sft     & $6.7$B      & $\mathbf{19.28}$~/~$\mathbf{18.67}$  & $\mathbf{19.28}$~/~$\mathbf{16.26}$   & $0.91$~/~$0.85$     \\

        \midrule
        \multicolumn{5}{c}{\textit{General-Purpose Models}} \\
        % \midrule

        GPT-4o                      & -      &$36.75$~/~$32.53$                     &$36.75$~/~$32.53$                      & $0.98$~/~$0.94$     \\
        Claude-3.5-Sonnet           & -      &$29.52$~/~$37.95$                     &$29.52$~/~$33.70$                      & $0.93$~/~$0.89$     \\
        Qwen2.5-72B                 & 72B    &$30.12$~/~$22.89$                     &$30.12$~/~$16.30$                      & $1.07$~/~$0.92$     \\
        DeepSeek-R1                 &$685$B  &$\mathbf{53.01}$~/~$\mathbf{45.78}$   &$\mathbf{53.01}$~/~$\mathbf{45.78}$    & $1.03$~/~$\mathbf{1.91}$     \\
        GPT-o1                      & -      &$32.53$~/~$43.37$                     &$32.53$~/~$43.37$                      & $\mathbf{1.21}$~/~$1.10$     \\

        \bottomrule
    \end{tabular}
    % }
    \caption{Main results of \benchtwo across baseline models, where the left side of ``/'' represents the zero-shot results and the right side represents the one-shot results.}
    \label{tab:torch_res}
\end{table*}



\subsection{Baselines and Setup}
\label{sec:baselines}
\benchall generally requires strong capabilities in code generation. Therefore, we select state-of-the-art LLMs that excel in programming tasks as baselines, including both specialized open-source models and general-purpose models.
For specialized open-source models, we choose \texttt{Qwen2.5-Coder-7B-Instruct} \cite{hui2024qwen2} and \texttt{deepseek-coder-6.7b-instruct} \cite{guo2024deepseek}.
For general-purpose models, we include \texttt{Claude-3.5-Sonnet-0620}\footnote{\href{https://www.anthropic.com/news/claude-3-5-sonnet}{https://www.anthropic.com/news/claude-3-5-sonnet}}, \texttt{GPT-4o-0806} \footnote{\href{https://openai.com/index/hello-gpt-4o}{https://openai.com/index/hello-gpt-4o}}, \texttt{qwen2.5-72B-Instruct} \cite{yang2024qwen2}, as well as the thought-driven models DeepSeek-R1 \cite{guo2025deepseek} and \texttt{GPT-o1-2024-12-17} \footnote{\href{https://openai.com/o1/}{https://openai.com/o1/}}.


In our experiments, all general-purpose models are deployed for direct inference. 
In contrast, domain-specific models undergo an additional supervised fine-tuning phase. 
Details of the training corpus can be found in \S~\ref{sec:appendix-trainingcorpus}.
For evaluation, we consider both zero-shot and one-shot scenarios. 
In the one-shot setting, a BM25-based retrieval method~\cite{bm25robertson2009} is utilized to select the most relevant prompt from the training corpus. 


\subsection{Main results of \benchone}
\label{sec:main_res}
Table~\ref{tab:git_res} illustrates the performances of baselines on \benchone. 
It is evident that domain-specific models generally underperform compared to general-purpose models. 
However, fine-tuning $7$B domain-specific models with domain data significantly boosts accuracy. 
Qwen's accuracy rises from $0$ to $4.89$\%, and DeepSeek's from $0$ to $9.78$\% in zero-shot settings, with even more pronounced enhancements in one-shot settings due to the retrieval data from the same source as \benchone. 
The observed increase in \texttt{Speed Up} can be attributed to the relative simplicity of the correctly generated operators, which makes it easier for LLMs to produce efficient code.
The high \texttt{GPU efficiency} shares the similar reasons. 

General-purpose models, particularly DeepSeek-R1 and GPT-o1, excel across all metrics. 
Under one-shot conditions, DeepSeek-R1 achieves $22.83$\% in \texttt{Call and Execution Accuracy}, while GPT-o1 reaches $23.91$\%. 
The roughly $10$\% improvement from zero-shot to one-shot highlights the critical role of high-quality examples for Triton generation. 
Furthermore, the close alignment between \texttt{Call Accuracy} and \texttt{Execution Accuracy} indicates that only a few operators fail to produce correct results despite successfully invoked. 

DeepSeek-R1 also leads in GPU execution times, with an improvement of $1.11\times$ in zero-shot and $1.22\times$ in one-shot settings. 
While GPU efficiency is strong across most models, Qwen2.5-72B exhibits lower efficiency in zero-shot settings, likely due to a higher proportion of less efficient operators. 
Finally, \texttt{Similarity} provides corroborative insights, as its variations mirror trends observed in other metrics.


\subsection{Main Results of \benchtwo}
From Table~\ref{tab:torch_res}, we can observe that domain-specific models generally underperform general-purpose models. 
Nonetheless, fine-tuning with an $8k$ corpus considerably improves their performance. 
For instance, Qwen’s zero-shot \texttt{Execution Accuracy} rises from $0$ to $17.47$\%.
In contrast, its one-shot improvement ($15.67$\%) is slightly lower, likely due to the fact that the retrieved prompts and \benchtwo operators come from different sources (Github vs. Pytorch). 


Among general-purpose models, DeepSeek-R1 demonstrates the strongest overall performance, achieving $53.01$\% \texttt{Call and Execution Accuracy} in the zero-shot setting. 
Although its accuracy drops by $7.23$\% in the one-shot setting, it still slightly surpasses GPT-o1. 
As for \texttt{Speed Up}, DeepSeek-R1 achieves the best performance of $1.91\times$ improvements.
Most performance improvements in successfully executed operators stem from operator fusion. Triton’s fused operators reduce redundant memory reads and writes compared to PyTorch, enhancing memory bandwidth utilization and boosting performance.

Overall, most models achieve better performance on \benchtwo than to \benchone, likely because \benchtwo features a more balanced distribution of operator difficulty, whereas \benchone is predominantly composed of higher-difficulty operators, namely, $\mathbf{d3}$ and $\mathbf{d4}$.
