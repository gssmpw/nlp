\section{\benchtwo}
The real-world Triton operators introduced in \S~\ref{sec:git} primarily focus on highly frequent operations. 
As a complement, we propose \benchtwo, which aligns the Triton wrapper with interfaces of the PyTorch library \cite{paszke2019pytorch}. 
Together, \benchone and \benchtwo form a complementary evaluation framework.
The following sections elaborate on the data construction (\S~\ref{sec:benchtwo-construction}), data statistics (\S~\ref{sec:benchtwo-datastatistic}), test code and metrics (\S~\ref{sec:benchtwo-testandmetric}), and  benchmark comparisons (\S~\ref{sec:benchtwo-comparison}). 


\subsection{Data Construction}
\label{sec:benchtwo-construction}
We construct \benchtwo by selecting PyTorch operators based on their usage frequency in real-world coding and then fusing them ( hereafter referred to simply as ``operators'' ). 
First, we select operators that require GPU interactions, ensuring alignment with Triton's scope.
Next, we sample $40$ high-frequency operators and $40$ low-frequency operators from the remaining pool. 
The frequency of each operator is determined by its usage probability in PyTorch-related code from The Stack V2~\cite{lozhkov2024starcoder} with those exceeding a predefined threshold $45$\% as common operators. 

Subsequently, we fuse these operators in various configurations: combinations of common operators, combinations of common and uncommon operators, and combinations of uncommon operators. 
All combinations are valid, as the outputs of preceding operators serve as appropriate inputs for subsequent ones. 
The final set includes $166$ operators, based on the latest (v2.6.0) version of the PyTorch library. 
Each operator is paired with its corresponding standard PyTorch call and document, while fused operators combine descriptions from all involved operators. 

\subsection{Data Statistic}
\label{sec:benchtwo-datastatistic}
The statistics of \benchtwo are presented in Table~\ref{tab:statis_torch}. 
Similar to \benchone, the operators are categorized into five difficulty levels ($\bf d1$ to $\bf d5$) using an LLM guided by prompt~\ref{prompt_diff}. 
These initial categorizations are then validated through manual review by two domain experts. 

We report the following statistics: (1)\textbf{torch-op\#} the average number of PyTorch operators, (2)\textbf{params\#} the average number of parameters, (3)\textbf{math\#}, the average token number of mathematical expressions, and (4)\textbf{description\#}, the average token count of the descriptions. 
These statistics generally increase with the operator difficulty, similar trend that aligns with the observations in \benchone. 


\begin{table}[t!]
    \centering
    \resizebox{0.48\textwidth}{!}{
    \begin{tabular}{lccccccc}
    \toprule
    \multirow{2}{*}{\textbf{Difficulty}} & \multicolumn{4}{c}{\textbf{Torch-Align Operator}}    \\
    \cmidrule(lr){2-5}
     & \textbf{torch-op\#} & \textbf{params\#} & \textbf{math\#} & \textbf{description\#}       \\
    \midrule
    $\mathbf{d1}$ ($13.3$\%)    & $1.36$  & $2.82$  & $23.50$  & $50.41$   \\
    $\mathbf{d2}$ ($22.3$\%)    & $1.97$  & $3.78$  & $40.73$  & $61.19$   \\
    $\mathbf{d3}$ ($32.5$\%)    & $2.70$  & $4.91$  & $74.64$  & $67.89$   \\
    $\mathbf{d4}$ ($29.5$\%)    & $2.16$  & $5.24$  & $47.31$  & $71.02$   \\
    $\mathbf{d5}$ ($2.4$\%)     & $2.75$  & $2.75$  & $30.50$  & $88.50$   \\
    \bottomrule
    \end{tabular}
    }
    \caption{Statistics of \benchtwo.
    }
    \label{tab:statis_torch}
\end{table}

\subsection{Test Code and Metrics}
\label{sec:benchtwo-testandmetric}
The design of the test code in \benchtwo adheres to those of \benchone, employing randomly generated tensors for operator evaluation.
For correctness and performance assessment, we utilize \textbf{\texttt{Call Accuracy}}, \textbf{\texttt{Execution Accuracy}}, and \textbf{\texttt{Speed Up}}, whose computation methods are consistent with those used in \benchone. 


\begin{table*}[ht!] \small
    \centering
    \resizebox{0.99\textwidth}{!}{
    % \begin{tabular}{l|c|ccc|cc}
        \begin{tabular}{lcccccc}
        \toprule
        \textbf{Model}  & \textbf{Size} & \textbf{Similarity} & \textbf{\makecell{Call \\Accuracy}} & \textbf{\makecell{Execution \\Accuracy}} & \textbf{Speed Up} & \textbf{\makecell{GPU \\Efficiency}}  \\ 
        \midrule
            % \multicolumn{7}{c}{\textit{Domain Expert}} \\
            \multicolumn{7}{c}{\textit{Domain-Specific Models}} \\
        % \midrule
                % \multicolumn{6}{c}{\textbf{TritGitBench}} \\ \midrule
        Qwen2.5-Coder      	& $7$B            &$9.19$~/~$14.54$               &$0.00$~/~$0.00$                            &$0.00$~/~$0.00$                      &$0.00$~/~$0.00$   &$0.00$~/~$0.00$   \\
        DeepSeek-Coder  	& $6.7$B          &$9.38$~/~$14.52$               &$0.00$~/~$0.00$                            &$0.00$~/~$0.00$                      &$0.00$~/~$0.00$   &$0.00$~/~$0.00$   \\
                
        Qwen2.5-Coder-sft	& $7$B            &$\mathbf{29.98}$~/~$25.96$     &$4.89$~/~$10.87$                         &$4.89$~/~$10.87$                   &$\mathbf{1.56}$~/~$\mathbf{1.22}$    &$\mathbf{51.71}$~/~$\mathbf{46.70}$     \\
        DeepSeek-Coder-sft & $6.7$B          &$25.52$~/~$\mathbf{30.34}$&$\mathbf{9.78}$~/~$\mathbf{11.96}$       &$\mathbf{9.87}$~/~$\mathbf{11.96}$ &$1.03$~/~$1.11$    &$47.68$~/~$42.26$     \\
        \midrule
                % \multicolumn{7}{c}{\textit{General Purpose}} \\
                \multicolumn{7}{c}{\textit{General-Purpose Models}} \\
        % \midrule
        GPT-4o                  & -             &$9.87$~/~$20.67$                 &$10.87$~/~$17.93$                        &$10.33$~/~$16.84$                    & $0.97$~/~$1.19$           &$48.80$~/~$\mathbf{53.33}$     \\
        Claude-3.5-Sonnet       & -             &$12.46$~/~$22.48$                &$10.33$~/~$20.11$                        & $9.79$~/~$19.57$                    & $0.90$~/~$\mathbf{1.54}$           &$\mathbf{59.31}$~/~$49.32$     \\
        Qwen2.5-72B             & $72$B           &$14.86$~/~$26.25$                &$11.41$~/~$16.85$                        &$10.87$~/~$16.31$                    & $0.96$~/~$1.19$   & $23.28$~/~$49.40$     \\
        DeepSeek-R1             &$685$B           &$\mathbf{19.96}$~/~$22.64$       &$13.59$~/~$22.83$                        &$13.05$~/~$22.83$                    & $\mathbf{1.11}$~/~$1.22$   & $44.83$~/~$46.70$     \\
        GPT-o1                  & -             &$16.58$~/~$\mathbf{29.70}$       &$\mathbf{15.22}$~/~$\mathbf{23.91}$      &$\mathbf{14.23}$~/~$\mathbf{23.91}$  &$0.92$~/~$1.14$   &$54.25$~/~$46.37$     \\
        \bottomrule
    \end{tabular}
    }
    \caption{Main results of \benchone across baseline models, where the left side of ``/'' represents the zero-shot results and the right side represents the one-shot results.}
    \label{tab:git_res}
\end{table*}


\subsection{Benchmark Comparison} 
\label{sec:benchtwo-comparison}
This section provides comparisons between \benchone and \benchtwo, which differ in key aspects and together provide a well-rounded evaluation. 
\paragraph{Source \& Distribution:} \benchone is collected from \textbf{GitHub} and reflects real-world programming demands with a concentration of frequently used operators, e.g., \texttt{Attention} at $20.0\%$, \texttt{MatMul} at $10.9\%$, \texttt{LayerNorm} at $6.5\%$, \texttt{SoftMax} at $3.8\%$. In contrast, \benchtwo, sourced from \textbf{PyTorch}, presents a more diverse operator set including both common and uncommon operators.
% 
\paragraph{Instruction Generation:} \benchone combines \textbf{LLM generation} with \textbf{expert verification} while \benchtwo directly extracts instructions from \textbf{PyTorch documentation}. This difference underlines their complementary roles in probing different facets of the Triton generation. 
\paragraph{Evaluation Metrics:} Both benchmark channels assess \textbf{correctness} and \textbf{performance}. Additionally, \benchone incorporates a similarity-based assessment that offers direct comparisons with established implementations. 
In summary, the different designs of \benchone and \benchtwo enable a comprehensive and nuanced evaluation of Triton operator generation. 
