\appendix
\label{sec:appendix}

\section{Training Corpus}
\label{sec:appendix-trainingcorpus}
The training corpus for supervised fine-tuning comprises two distinct components: real-world data sourced from GitHub and synthetically generated data produced through compiler operations.

The real-world data component incorporates Triton code extracted from GitHub repositories, which undergoes basic cleaning procedures as outlined in prompt~\ref{prompt_filter}, undergoes a debugging process that is less rigorous than the methodology applied to \benchone.
To prevent potential data leakage and ensure benchmark integrity, we systematically eliminate samples exhibiting high similarity to \benchone entries using the \textsc{CodeBertScore} similarity metric~\citep{codebertscore2023}.

The synthetic data component is generated using Ninetoothed\footnote{\href{https://github.com/InfiniTensor/ninetoothed}{https://github.com/InfiniTensor/ninetoothed}}, a domain-specific language built upon Triton that offers enhanced abstraction capabilities. This framework facilitates the automated synthesis of valid Triton code through the processing of well-formed expressions.
Each part of data containing $4K$ samples. This combined corpus serves as the foundational training dataset for experimental models in one-shot learning settings. For all experiments, the fine-tuning process is carried out over $3$ epochs with a learning rate of $5e-5$. 

\section{Operator Performance Evaluation}
\label{sec:appendix-performance_eval}
\begin{figure}
    \centering
    \includegraphics[width=7.5cm]{figures/performance_evaluation_process.pdf}
    \caption{The workflow of operator performance evaluation}
    \label{fig:performance_evaluation_metrics}
\end{figure}

For operator performance evaluation, we refer primarily to the official examples provided by Triton\footnote{\href{https://triton-lang.org/main/getting-started/tutorials/}{https://triton-lang.org/main/getting-started/tutorials/}}. 
We provide evaluation scripts for each operator in \benchone.
Figure~\ref{fig:performance_evaluation_metrics} illustrates the workflow of our operator performance evaluation.

First, we define a set of tensors with increasing dimensions based on the characteristics of the operator. 
Next, each tensor is sequentially fed into the operator for execution. 
During each execution, we use the expert annotations for each operator to determine the total memory bandwidth (Bytes) and the total number of floating-point operations (Flops) based on the input tensors.
More importantly, we use the \texttt{triton.testing.do\_bench} method from the official Triton library\footnote{\href{https://triton-lang.org/main/python-api/generated/triton.testing.do_bench.html}{https://triton-lang.org/main/pythonapi/generated/triton.testing.do\_bench.html}} to measure the operator’s execution time on the GPU. 
Specifically, we gradually increase the warm-up time and repetition time until the measured execution time stabilized, which means that most operators are run hundreds of thousands of times to ensure that the running time is measured accurately.
After obtaining the execution time, we calculate the operator’s performance metrics by dividing the total memory bandwidth and the total floating-point operations by the execution time to obtain throughput in GB/s and Tflops, respectively. 
We then calculate the GPU efficiency by calculating the ratio of the measured performance metrics (GB/s and Tflops) to the theoretical maximum performance of the NVIDIA A100 Tensor Core GPU.
Repetition of the above process for tensors of increasing sizes obtains the performance metrics for each execution, which collectively form the operator performance report.
We adopt the peak GPU efficiency from the performance report as the final measure of the operator's quality.

By following the evaluation workflow described above, we generate a detailed performance report for each operator in \benchone. Figure~\ref{fig:common_operators_curves} illustrates the performance curves of several common operators. As the input dimensions increase, as can be seen from the figure, the GB/s or Tflops of the operators show an upward trend, eventually stabilizing. 
This suggests that the performance of the operator reaches a bottleneck beyond a certain scale, and further increases in input size result in diminishing returns in performance, aligning with the expected trend of operator performance.

\begin{figure*}
    \centering
    \includegraphics[width=16cm]{figures/common_operators_curves.pdf}
    \caption{Performance Curves of Common Operators}
    \label{fig:common_operators_curves}
\end{figure*}

\section{Error Categories}
\label{app:error_catgrz}
We provide the error type statistics of failure operators in \benchall. A total of 16 error types are identified in the integrated Call and Execution error results. For convenience in presentation, we categorize them into four main groups: Syntax Errors: including SyntaxError and IndentationError; Attrb\&Type Errors: including AttributeError, TypeError, and NotImplementedError; Name\&Ref Errors: including NameError, KeyError, IndexError, ModuleNotFoundError, and ImportError; Run\&Logc Errors: including ValueError, ZeroDivisionError, RuntimeError, RecursionError, AssertionError, CompilationError, and ResultsError. ResultsError refers to the inconsistency between the execution results of the reference operator and the generated operator.

\section{Prompts}
Here are the four prompts we use in our work: Filtering Prompt, Instruction Prompt, Difficulty Prompt, and Test Code Prompt. Specifically, the first is used to extract Triton-related code from crawled code files; the second instructs the large model to generate corresponding instructions based on Triton code; the third prompts the large model to score the difficulty of Triton operators according to the standards we proposed; and the last asks the large model to generate test code.
\onecolumn

\begin{tcolorbox}[colframe=gray!80!black, colback=gray!10!white, title=Filtering Prompt]
\small
{\{code\}}\\
\justifying
Please help me select \textbf{all} triton kernel functions decorated with \texttt{@triton.jit} and all code that calls these kernels, while only keeping the necessary imports (e.g., triton, torch) and the calling functions. \\
\hdashrule[0.5ex]{15cm}{0.1pt}{1mm}
\justifying \\
Note 1: Retain necessary comments related to the Triton code. Code can be optimized, but do not remove all kernel code and its corresponding calls just for brevity. \\
\hdashrule[0.5ex]{15cm}{0.1pt}{1mm}
\justifying \\
Note 2: If the triton kernel is decorated with a custom or third-party decorator other than triton, discard that kernel. \\
\hdashrule[0.5ex]{15cm}{0.1pt}{1mm}
\justifying \\
Note 3: If \texttt{@triton.jit} appears as a \textbf{string} in the code or is nested within a function body, then discard it. \\
\hdashrule[0.5ex]{15cm}{0.1pt}{1mm}
\justifying \\
Note 4: If there are multiple triton kernel functions decorated with \texttt{@triton.jit} and their calling wrapper functions, retain all of them, not just a subset.

1) Extract all triton operators (kernel functions decorated with \texttt{@triton.jit} and their calling functions) and output them in python code format. If no triton kernel function is found, discard it.

2) Provide a concise English description of each extracted operator (including both kernel and calling code) in the form of a python dictionary: \texttt{"description": "Use triton language to..."}
\label{prompt_filter}
\end{tcolorbox}



\begin{tcolorbox}[colframe=gray!80!black, colback=gray!10!white, title=Instruction Prompt]
\small
\texttt{\{code\}}\\ \justifying % 设置为两端对齐
Based on the above Triton operator code, generate a detailed description so that the large model can accurately reproduce the corresponding kernel and wrapper function. \\
Be clear about the logic and main functionality of the operator, specify the function name, inputs, and outputs, and describe any public variables clearly. \\
Try to describe the function's code implementation. Ensure that the large model can reproduce the corresponding function and parameter code based on these instructions. \\
Note that the output should maintain correct python syntax. \\
\label{prompt_instru}
\end{tcolorbox}


\begin{tcolorbox}[colframe=gray!80!black, colback=gray!10!white, title=Test Code Prompt]
\small
\texttt{\{code\}}\\ \justifying
Write a test code in Python for the above code. Ensure that all branch tests are in a single function starting with \texttt{``test\_''}, with no parameters.\\
\hdashrule[0.5ex]{15cm}{0.1pt}{1mm}
\justifying
Note 1: Particular attention should be paid to the fact that tensor parameters are of GPU type. \\
\hdashrule[0.5ex]{15cm}{0.1pt}{1mm}
\justifying \\
Note 2: Try to limit the number of branches to no more than 4. \\
\hdashrule[0.5ex]{15cm}{0.1pt}{1mm}
\justifying \\
Note 3: In branch tests, avoid modifying parameters that are later in the argument list with default values (especially if they have out parameters, do not assign them). \\
\hdashrule[0.5ex]{15cm}{0.1pt}{1mm}
\justifying \\
Note 4: Store the results of all branch calculations in a dictionary, where the dictionary key is \texttt{"test\_case\_n"}, with \texttt{n} representing the test case number.\\
\hdashrule[0.5ex]{15cm}{0.1pt}{1mm}
\justifying \\
Note 5: Ensure that the import paths match exactly as described in the operator documentation to maintain accuracy.\\
\hdashrule[0.5ex]{15cm}{0.1pt}{1mm}
\justifying \\
Note 6: The code should run directly, without \texttt{if \_\_name\_\_ == "\_\_main\_\_"}. \\
\label{prompt_test_code}

\end{tcolorbox}
\begin{tcolorbox}[colframe=gray!80!black, colback=gray!10!white, title=Difficulty Prompt]
\small
\texttt{\{code\}}\\ \justifying % 设置为两端对齐
Please evaluate the complexity of the code in the following two aspects based on the requirements of the Triton operator and score it from simple to complex on a scale from 1 to 5: \\
1) Memory layout complexity: Analyze the memory access pattern, including memory tiling, array transposition, address alignment, cache utilization, and the number of global memory accesses. \\
2) Computation scheduling complexity: Examine instruction-level parallelism, computation-memory pipeline, thread block design, inter-thread communication, thread branch divergence, and hardware resource utilization. \\
The final score is the ceiling of the average score from both aspects, and only one complexity score is output in [ ].
\label{prompt_diff}
\end{tcolorbox}


