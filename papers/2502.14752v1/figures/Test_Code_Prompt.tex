\begin{tcolorbox}[colframe=gray!80!black, colback=gray!10!white, title=Test Code Prompt]
\small
\texttt{\{code\}}\\ \justifying
Write a test code in Python for the above code. Ensure that all branch tests are in a single function starting with \texttt{``test\_''}, with no parameters.\\
\hdashrule[0.5ex]{15cm}{0.1pt}{1mm}
\justifying
Note 1: Particular attention should be paid to the fact that tensor parameters are of GPU type. \\
\hdashrule[0.5ex]{15cm}{0.1pt}{1mm}
\justifying \\
Note 2: Try to limit the number of branches to no more than 4. \\
\hdashrule[0.5ex]{15cm}{0.1pt}{1mm}
\justifying \\
Note 3: In branch tests, avoid modifying parameters that are later in the argument list with default values (especially if they have out parameters, do not assign them). \\
\hdashrule[0.5ex]{15cm}{0.1pt}{1mm}
\justifying \\
Note 4: Store the results of all branch calculations in a dictionary, where the dictionary key is \texttt{"test\_case\_n"}, with \texttt{n} representing the test case number.\\
\hdashrule[0.5ex]{15cm}{0.1pt}{1mm}
\justifying \\
Note 5: Ensure that the import paths match exactly as described in the operator documentation to maintain accuracy.\\
\hdashrule[0.5ex]{15cm}{0.1pt}{1mm}
\justifying \\
Note 6: The code should run directly, without \texttt{if \_\_name\_\_ == "\_\_main\_\_"}. \\
\label{prompt_test_code}

\end{tcolorbox}