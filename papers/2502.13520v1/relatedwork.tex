\section{Related Work}
\label{sec:related}

%Notes:
%https://www.englishprofile.org/
% https://www.cl.cam.ac.uk/~ejb1/epp.html 
%cite cefr readme paper + laila's paper + zaebuc + samer...

% https://arxiv.org/pdf/2405.02144v1
%AARI: Automatic Arabic readability index

%Creating a CEFR Arabic Vocabulary Profile: A frequency-based multi-dialectal
%approach
%https://www.researchgate.net/publication/373237489_Creating_a_CEFR_Arabic_Vocabulary_Profile_A_frequency-based_multi-dialectal_approach

%Approaches, Methods, and Resources for Assessing the Readability of Arabic Texts \cite{Nassiri:2023}
%* leveling in different languages
%* techniques - formulas vs lexicons
%* granularity: books, words (sentences)
%* refer to references in and of SAMER, Juan+Bashar work, Taha Book \cite{habash-etal-2022-camel}


%There has been a lot of interest in the area of Arabic Readability, resulting in a sizeable and diverse body of literature. For a recent survey, see \newcite{Nassiri:2023}. 
%There has been a lot of interest in the area of Arabic Readability, resulting in a sizeable and diverse body of literature \cite{Nassiri:2023}. 
  
%We present a number of related concepts and efforts next. We discuss relevant Arabic linguistic facts as part of the discussion of {\barec} guidelines (\S\ref{sec:guidelines}).




 
\paragraph{Automatic Readability Assessment}
Automatic readability assessment has been widely studied, resulting in numerous datasets and resources \cite{collins-thompson-callan-2004-language,pitler-nenkova-2008-revisiting,feng-etal-2010-comparison,vajjala-meurers-2012-improving,xu-etal-2015-problems,xia-etal-2016-text,nadeem-ostendorf-2018-estimating,vajjala-lucic-2018-onestopenglish,deutsch-etal-2020-linguistic,lee-etal-2021-pushing}. Early English datasets were often derived from textbooks, as their graded content naturally aligns with readability assessment \cite{vajjala-2022-trends}. However, copyright restrictions and limited digitization have driven researchers to crowdsource readability annotations from online sources \cite{vajjala-meurers-2012-improving,vajjala-lucic-2018-onestopenglish} or leverage CEFR-based L2 assessment exams \cite{xia-etal-2016-text}.
 

\paragraph{Arabic Readability Efforts}
Arabic readability research has focused on text leveling and assessment across various frameworks. \newcite{Taha:2017:guidelines} proposed a 19-level system for children's books based on qualitative and quantitative criteria. Other efforts applied CEFR leveling to Arabic, including the KELLY project’s frequency-based word lists, manually annotated corpora \cite{habash-palfreyman-2022-zaebuc, naous-etal-2024-readme}, and vocabulary profiling \cite{soliman2024creating}.
\citet{el-haj-etal-2024-dares} introduced DARES, a readability assessment dataset collected from Saudi school materials.
The SAMER project \cite{al-khalil-etal-2020-large} developed a lexicon with a five-level readability scale, leading to the first manually annotated Arabic parallel corpus for text simplification \cite{alhafni-etal-2024-samer}.
%
Automated readability assessment has also been explored through rule-based and machine learning approaches. Early models relied on surface-level features like word and sentence length \cite{al2004assessment, Al-Khalifa:2010:automatic}, while later work incorporated POS-based and morphological features \cite{Forsyth:2014:automatic, Saddiki:2018:feature}. The OSMAN metric \cite{el-haj-rayson-2016-osman} leveraged script markers and diacritization, and recent efforts \cite{liberato-etal-2024-strategies} achieved strong results using pretrained models on the SAMER corpus. 

Building on these efforts, we curated the {\barec} corpus across genres and readability levels, and manually annotated it at the sentence-level based on an adaptation of Taha/Arabi21 guidelines \cite{Taha:2017:guidelines}, offering finer-grained control and a more objective assessment of textual variation.



%We discuss next our design criteria and then present the guidelines and annotation process in detail.

%\textbf{BAREC Levels: The system we follow to determine the readability level of sentences and phrases. Inspired by Hanada Taha's levels to determine the readability of texts, BAREC levels are five levels that mark the readability of vocabulary (specifically lexical entries), while Barec levels are for the readability of sentences and phrases.
%%%%
%Texts (from paragraphs to chapters and books) may contain sentences and phrases of different levels.}

%%%%%%%%%%%%%%
%https://docs.google.com/spreadsheets/d/1mAhPQGG_snQRetsXvWLTiz5iNhF6R_zfrdFfRgGB1F4/edit#gid=1490799958
\begin{figure*}[t]
\centering
 \includegraphics[width=0.9\textwidth]{Barec-Pyramid-new.pdf}
    \caption{ The {\barec} \textit{Pyramid}  illustrates the relationship across {\barec} levels and linguistic dimensions, three collapsed variants, and education grades.}
\label{fig:barec-pyramid}
\end{figure*}


%