\begin{figure}[t!]
    \centering
    \begin{minipage}[t]{\linewidth}
        \begin{subfigure}{0.49\linewidth}
            \centering
            \includegraphics[width=\textwidth]{latex/figures/lineplots/examples/hans_llama3_8b_class2_cls1.png}
            \caption{$y_1$ (biased sampling)}\label{fig:image1}
        \end{subfigure}% 
        \hfill
        \begin{subfigure}{0.49\linewidth}
            \centering
            \includegraphics[width=\textwidth]{latex/figures/lineplots/examples/hans_llama3_8b_random_cls1.png}
            \caption{$y_1$ (random sampling)}\label{fig:image2}
        \end{subfigure}
        \hfill
        \begin{subfigure}{0.49\linewidth}
            \centering
            \includegraphics[width=\textwidth]{latex/figures/lineplots/examples/hans_llama3_8b_class2_cls2.png}
            \caption{$y_2$ (biased sampling)}\label{fig:image3}
        \end{subfigure}
        \hfill
        \begin{subfigure}{0.49\linewidth}
            \centering
            \includegraphics[width=\textwidth]{latex/figures/lineplots/examples/hans_llama3_8b_random_cls1.png}
            \caption{$y_2$ (random sampling)}\label{fig:image4}
        \end{subfigure}%
    \end{minipage}% 
    \hfill
    \begin{minipage}[c]{\linewidth}
        \centering
        \vspace{2mm}
            \begin{subfigure}{0.7 \linewidth}
            \centering
            \includegraphics[width=\textwidth]{latex/figures/lineplots/legend.pdf}
        \end{subfigure}%
    \end{minipage}
    \hfill
    \begin{minipage}[c]{\linewidth}
        \caption{\label{fig:num_ex} Validation performance of Llama3 (8B) across different numbers of demonstrations on the Hans dataset. In this figure, in-context examples from $y_2$ were sampled from long instances, and in-context examples from $y_1$ were sampled from short instances. Each subfigure shows results on the validation instances in the respective class. The conditions that introduce length bias in the context window (a and c subfigures) demonstrate a larger spread between performance on short and long validation instances, indicating greater potential to learn bias with longer contexts.}
    \end{minipage}
\end{figure}

\begin{figure}[t!]
    \centering
    \begin{minipage}[t]{\linewidth}
        \begin{subfigure}{0.49\linewidth}
            \centering
            \includegraphics[width=\textwidth]{latex/figures/lineplots/examples/rte_llama3_8b_class2_cls1.png}
            \caption{$y_1$ (biased sampling)}\label{fig:image1}
        \end{subfigure}% 
        \hfill
        \begin{subfigure}{0.49\linewidth}
            \centering
            \includegraphics[width=\textwidth]{latex/figures/lineplots/examples/rte_llama3_8b_random_cls1.png}
            \caption{$y_1$ (random sampling)}\label{fig:image2}
        \end{subfigure}
        \hfill
        \begin{subfigure}{0.49\linewidth}
            \centering
            \includegraphics[width=\textwidth]{latex/figures/lineplots/examples/rte_llama3_8b_class2_cls2.png}
            \caption{$y_2$ (biased sampling)}\label{fig:image3}
        \end{subfigure}
        \hfill
        \begin{subfigure}{0.49\linewidth}
            \centering
            \includegraphics[width=\textwidth]{latex/figures/lineplots/examples/rte_llama3_8b_random_cls1.png}
            \caption{$y_2$ (random sampling)}\label{fig:image4}
        \end{subfigure}%
    \end{minipage}% 
    \hfill
    \begin{minipage}[c]{\linewidth}
        \centering
        \vspace{2mm}
            \begin{subfigure}{0.7 \linewidth}
            \centering
            \includegraphics[width=\textwidth]{latex/figures/lineplots/legend.pdf}
        \end{subfigure}%
    \end{minipage}
    \hfill
    \begin{minipage}[c]{\linewidth}
        \caption{\label{fig:num_ex} Validation performance of Llama3 (8B) across different numbers of demonstrations on the RTE dataset. In this figure, in-context examples from $y_2$ were sampled from long instances, and in-context examples from $y_1$ were sampled from short instances. Each subfigure shows results on the validation instances in the respective class. The conditions that introduce length bias in the context window (a and c subfigures) demonstrate a larger spread between performance on short and long validation instances, indicating greater potential to learn bias with longer contexts.}
    \end{minipage}
\end{figure}

\begin{figure}[t!]
    \centering
    \begin{minipage}[t]{\linewidth}
        \begin{subfigure}{0.49\linewidth}
            \centering
            \includegraphics[width=\textwidth]{latex/figures/lineplots/examples/wnli_llama3_8b_class2_cls1.png}
            \caption{$y_1$ (biased sampling)}\label{fig:image1}
        \end{subfigure}% 
        \hfill
        \begin{subfigure}{0.49\linewidth}
            \centering
            \includegraphics[width=\textwidth]{latex/figures/lineplots/examples/wnli_llama3_8b_random_cls1.png}
            \caption{$y_1$ (random sampling)}\label{fig:image2}
        \end{subfigure}
        \hfill
        \begin{subfigure}{0.49\linewidth}
            \centering
            \includegraphics[width=\textwidth]{latex/figures/lineplots/examples/wnli_llama3_8b_class2_cls2.png}
            \caption{$y_2$ (biased sampling)}\label{fig:image3}
        \end{subfigure}
        \hfill
        \begin{subfigure}{0.49\linewidth}
            \centering
            \includegraphics[width=\textwidth]{latex/figures/lineplots/examples/wnli_llama3_8b_random_cls1.png}
            \caption{$y_2$ (random sampling)}\label{fig:image4}
        \end{subfigure}%
    \end{minipage}% 
    \hfill
    \begin{minipage}[c]{\linewidth}
        \centering
        \vspace{2mm}
            \begin{subfigure}{0.7 \linewidth}
            \centering
            \includegraphics[width=\textwidth]{latex/figures/lineplots/legend.pdf}
        \end{subfigure}%
    \end{minipage}
    \hfill
    \begin{minipage}[c]{\linewidth}
        \caption{\label{fig:num_ex} Validation performance of Llama3 (8B) across different numbers of demonstrations on the WNLI dataset. In this figure, in-context examples from $y_2$ were sampled from long instances, and in-context examples from $y_1$ were sampled from short instances. Each subfigure shows results on the validation instances in the respective class. The conditions that introduce length bias in the context window (a and c subfigures) demonstrate a larger spread between performance on short and long validation instances, indicating greater potential to learn bias with longer contexts.}
    \end{minipage}
\end{figure}

\begin{figure}[t!]
    \centering
    \begin{minipage}[t]{\linewidth}
        \begin{subfigure}{0.49\linewidth}
            \centering
                \includegraphics[width=\textwidth]{latex/figures/lineplots/examples/qnli_llama3_8b_class2_cls1.png}
            \caption{$y_1$ (biased sampling)}\label{fig:image1}
        \end{subfigure}% 
        \hfill
        \begin{subfigure}{0.49\linewidth}
            \centering
            \includegraphics[width=\textwidth]{latex/figures/lineplots/examples/qnli_llama3_8b_random_cls1.png}
            \caption{$y_1$ (random sampling)}\label{fig:image2}
        \end{subfigure}
        \hfill
        \begin{subfigure}{0.49\linewidth}
            \centering
            \includegraphics[width=\textwidth]{latex/figures/lineplots/examples/qnli_llama3_8b_class2_cls2.png}
            \caption{$y_2$ (biased sampling)}\label{fig:image3}
        \end{subfigure}
        \hfill
        \begin{subfigure}{0.49\linewidth}
            \centering
            \includegraphics[width=\textwidth]{latex/figures/lineplots/examples/qnli_llama3_8b_random_cls1.png}
            \caption{$y_2$ (random sampling)}\label{fig:image4}
        \end{subfigure}%
    \end{minipage}% 
    \hfill
    \begin{minipage}[c]{\linewidth}
        \centering
        \vspace{2mm}
            \begin{subfigure}{0.7 \linewidth}
            \centering
            \includegraphics[width=\textwidth]{latex/figures/lineplots/legend.pdf}
        \end{subfigure}%
    \end{minipage}
    \hfill
    \begin{minipage}[c]{\linewidth}
        \caption{\label{fig:num_ex} Validation performance of Llama3 (8B) across different numbers of demonstrations on the QNLI dataset. In this figure, in-context examples from $y_2$ were sampled from long instances, and in-context examples from $y_1$ were sampled from short instances. Each subfigure shows results on the validation instances in the respective class. The conditions that introduce length bias in the context window (a and c subfigures) demonstrate a larger spread between performance on short and long validation instances, indicating greater potential to learn bias with longer contexts.}
    \end{minipage}
\end{figure}

\begin{figure}[t!]
    \centering
    \begin{minipage}[t]{\linewidth}
        \begin{subfigure}{0.49\linewidth}
            \centering
            \includegraphics[width=\textwidth]{latex/figures/lineplots/examples/mrpc_llama3_8b_class2_cls1.png}
            \caption{$y_1$ (biased sampling)}\label{fig:image1}
        \end{subfigure}% 
        \hfill
        \begin{subfigure}{0.49\linewidth}
            \centering
            \includegraphics[width=\textwidth]{latex/figures/lineplots/examples/mrpc_llama3_8b_random_cls1.png}
            \caption{$y_1$ (random sampling)}\label{fig:image2}
        \end{subfigure}
        \hfill
        \begin{subfigure}{0.49\linewidth}
            \centering
            \includegraphics[width=\textwidth]{latex/figures/lineplots/examples/mrpc_llama3_8b_class2_cls2.png}
            \caption{$y_2$ (biased sampling)}\label{fig:image3}
        \end{subfigure}
        \hfill
        \begin{subfigure}{0.49\linewidth}
            \centering
            \includegraphics[width=\textwidth]{latex/figures/lineplots/examples/mrpc_llama3_8b_random_cls1.png}
            \caption{$y_2$ (random sampling)}\label{fig:image4}
        \end{subfigure}%
    \end{minipage}% 
    \hfill
    \begin{minipage}[c]{\linewidth}
        \centering
        \vspace{2mm}
            \begin{subfigure}{0.7 \linewidth}
            \centering
            \includegraphics[width=\textwidth]{latex/figures/lineplots/legend.pdf}
        \end{subfigure}%
    \end{minipage}
    \hfill
    \begin{minipage}[c]{\linewidth}
        \caption{\label{fig:num_ex} Validation performance of Llama3 (8B) across different numbers of demonstrations on the MPRC dataset. In this figure, in-context examples from $y_2$ were sampled from long instances, and in-context examples from $y_1$ were sampled from short instances. Each subfigure shows results on the validation instances in the respective class. The conditions that introduce length bias in the context window (a and c subfigures) demonstrate a larger spread between performance on short and long validation instances, indicating greater potential to learn bias with longer contexts.}
    \end{minipage}
\end{figure}

\begin{figure}[t!]
    \centering
    \begin{minipage}[t]{\linewidth}
        \begin{subfigure}{0.49\linewidth}
            \centering
            \includegraphics[width=\textwidth]{latex/figures/lineplots/examples/sst2_llama3_8b_class2_cls1.png}
            \caption{$y_1$ (biased sampling)}\label{fig:image1}
        \end{subfigure}% 
        \hfill
        \begin{subfigure}{0.49\linewidth}
            \centering
            \includegraphics[width=\textwidth]{latex/figures/lineplots/examples/sst2_llama3_8b_random_cls1.png}
            \caption{$y_1$ (random sampling)}\label{fig:image2}
        \end{subfigure}
        \hfill
        \begin{subfigure}{0.49\linewidth}
            \centering
            \includegraphics[width=\textwidth]{latex/figures/lineplots/examples/sst2_llama3_8b_class2_cls2.png}
            \caption{$y_2$ (biased sampling)}\label{fig:image3}
        \end{subfigure}
        \hfill
        \begin{subfigure}{0.49\linewidth}
            \centering
            \includegraphics[width=\textwidth]{latex/figures/lineplots/examples/sst2_llama3_8b_random_cls1.png}
            \caption{$y_2$ (random sampling)}\label{fig:image4}
        \end{subfigure}%
    \end{minipage}% 
    \hfill
    \begin{minipage}[c]{\linewidth}
        \centering
        \vspace{2mm}
            \begin{subfigure}{0.7 \linewidth}
            \centering
            \includegraphics[width=\textwidth]{latex/figures/lineplots/legend.pdf}
        \end{subfigure}%
    \end{minipage}
    \hfill
    \begin{minipage}[c]{\linewidth}
        \caption{\label{fig:num_ex} Validation performance of Llama3 (8B) across different numbers of demonstrations on the SST2 dataset. In this figure, in-context examples from $y_2$ were sampled from long instances, and in-context examples from $y_1$ were sampled from short instances. Each subfigure shows results on the validation instances in the respective class. The conditions that introduce length bias in the context window (a and c subfigures) demonstrate a larger spread between performance on short and long validation instances, indicating greater potential to learn bias with longer contexts.}
    \end{minipage}
\end{figure}