\section{Previous Work}
\label{sec:headings}

\subsection{Open science and research data}
Interest in open science has been growing steadily, with a noticeable increase in the adoption and enforcement of open science practices across disciplines. For instance, funding organizations such as the European Commission require grant recipients to comply with open-access publishing policies under frameworks like Horizon Europe, aiming to enhance the accessibility and dissemination of research outputs to broader audiences~\citep{eu_openscience}. Similarly, numerous academic journals and institutions now mandate practices such as data sharing and methodological transparency as part of their publication and evaluation processes~\citep{robson2021promoting,gorgolewski2016practical}. Moreover, open science communities play a pivotal role in facilitating the large-scale transition of researchers toward open science practices~\citep{armeni2021towards}.

Open science practices extend beyond open-access publishing and include the early sharing of research outputs. For example, platforms like arXiv and bioRxiv enable the dissemination of preprints, fostering early access to findings. Furthermore, open science encourages the public sharing of data and code, often hosted on online repositories such as Zenodo and GitHub, thereby improving research reproducibility and scalability. Open science also promotes rigorous and transparent research design, exemplified by practices like study preregistration~\citep{gopal2018adherence}. 

Substantial evidence indicates that open science practices offer significant advantages over traditional closed practices~\citep{mckiernan_how_2016}. Open-access articles, for example, not only garner broader academic attention and higher citation rates~\citep{huang2024open} but also attract greater engagement from the general public and news media compared to paywalled articles~\citep{schultz2021all,yang2024open}. Furthermore, open science has been shown to accelerate scientific discovery in specific fields~\citep{woelfle2011open}, enhance research transparency, and improve reproducibility~\citep{besanccon2021open}. These benefits play a critical role in addressing challenges associated with the reproducibility crisis~\citep{open2015estimating}.

In today’s data-driven research landscape, the collection, analysis, and interpretation of large datasets are critical to scientific discovery. Among the pillars of open science, research data is particularly vital for promoting transparency and reproducibility. Access to well-documented research data facilitates independent verification of results, supports secondary analyses, and fosters interdisciplinary collaboration, thereby amplifying the impact of scientific inquiry~\citep{hossain2016state,milham_assessment_2018}. There is evidence that integrated data sets have been instrumental in driving biomedical discoveries and drug development~\citep{shahin2020open}.

The advantages of sharing research data are far-reaching, enhancing both the visibility and reuse of research outputs while maximizing the impact of funding agencies' investments~\citep{los_riding_2010}. Recognizing these benefits, governments and funding bodies worldwide have implemented policies to incentivize open data practices. The United States pioneered such efforts as early as 1991~\citep{bromley_policy_1991}, with countries like China, the United Kingdom, and Australia subsequently strengthening their data management frameworks~\citep{china____policy,uk_policy,australia_policy}. In Europe, the Horizon 2020 initiative introduced the Open Research Data Pilot (ODP) to improve data accessibility and establish credibility in data-sharing practices. Leading funding agencies, including the NSF, NIH, and the UK’s Economic and Social Research Council, now require grant applicants to submit data management plans as part of their application process~\citep{smith2012institutional,spengler2012data}. Publishers such as Elsevier, PLOS, Springer, and Nature have also adopted policies that encourage or mandate data citation within reference lists, promoting transparency and accountability in scientific research~\citep{cousijn_data_2018,walton_data_2010,plos_policy,Springer_policy}.

For researchers, open data practices offer additional benefits: they facilitate the development of scientific software~\citep{niemeyer_challenge_2016}, increase research productivity~\citep{mcnaught_changing_2015}, and promote a collaborative data-sharing culture within the scientific community~\citep{belter_measuring_2014}. By aligning incentives for researchers, funders, and publishers, these policies collectively strengthen the foundation for transparent, reproducible, and impactful research.


However, significant barriers continue to hinder the widespread adoption of open data. These include limited incentives, inconsistent citation practices, concerns about data quality, and researchers' reluctance to relinquish control over their data. Additionally, a lack of awareness and insufficient support mechanisms exacerbate these challenges~\citep{chawinga_global_2019,gajbe_evaluation_2021}. Practical issues such as time constraints, inadequate funding, and insufficient institutional support further impede progress~\citep{tenopir_data_2011,tenopir_data_2020}. Deficiencies in archival standards and infrastructure also contribute to low rates of data sharing~\citep{markiewicz_openneuro_2021}. For example, studies that sought to obtain data directly from authors reported low success rates—ranging from 27\% to 59\%, depending on the discipline and geographical context~\citep{tedersoo_data_2021}. Even among papers with data availability statements claiming “data available upon request,” compliance remains low. A 2018 study revealed that only 44\% of authors shared their data when requested~\citep{stodden_empirical_2018}, a finding corroborated by subsequent research~\citep{strcic_open_2022,danchev_evaluation_2021}. 

\subsection{Sharing and reuse of research data}
Research on data sharing and reuse remains in an exploratory stage, with scholars using various data sources and quantitative methods to analyze and discuss data reuse and sharing behaviors in publications.

Disciplinary differences in data citation practices have been a focal point. For instance, ~\cite{park_informal_2018} examined samples from biological and biomedical sciences in the Data Citation Index (DCI), revealing that informal citations within article text are more prevalent than formal citations in reference sections. Similarly, ~\cite{robinson-garcia_analyzing_2016} also utilized DCI data to examine the varying uses of datasets and data studies across disciplines. Their analysis found that datasets were most frequently cited in the fields of science and engineering \& technology, whereas data studies played a more prominent role in the social sciences and arts \& humanities. ~\cite{park_examination_2017} analyzed 148 articles from the Web of Science Data Citation Index to identify factors influencing data sharing and reuse. They found that formal data citation remains relatively uncommon, while informal references in the main text are more typical.

Certain factors have been found to influence researchers' willingness and ability to share and reuse datasets. Studies suggest a correlation between dataset sharing and higher citation rates~\citep{piwowar_data_2013,piwowar_sharing_2007}. Authors also tend to reuse their own shared data, resulting in higher self-citation rates~\citep{robinson-garcia_analyzing_2016}. Data-sharing practices vary notably by discipline, suggesting a need for tailored approaches for each field~\citep{helbig_supporting_2015,torres-salinas_how_2014}. Furthermore, data-sharing rates vary by scientific field~\citep{tenopir_data_2011}, and researchers’ data-sharing behaviors and perceptions differ across age groups and geographical locations~\citep{tenopir_changes_2015}. Certain data types, such as survey, aggregated, and sequence data, receive more frequent citations and higher altmetric scores~\citep{peters_research_2015}.

Studies of data-sharing behavior highlight the impact of shared data on research practices. For instance, an analysis of 600 articles across PLOS journals showed that 74\% of studies rely on datasets created by authors, with fewer reusing prior datasets~\citep{zhao_data_2018}. In biodiversity research, studies using Global Biodiversity Information Facility (GBIF) data demonstrate a rise in open data use, though best practices for data citation remain underutilized~\citep{khan_measuring_2021}.

In addition, journal compliance policies for data sharing have improved, with an increase in the use of repositories instead of supplementary materials for data storage~\citep{jiao_data_2024}. However, data availability statements (DAS) remain inconsistent, especially in COVID-19 research, where only a quarter of preprints provide explicit data-sharing statements~\citep{strcic_open_2022}.

Despite these findings, certain gaps in data-sharing and reuse research remain, particularly in the context of cross-disciplinary data-sharing practices. Most studies are based on samples or case studies from specific fields or repositories, lacking comprehensive cross-disciplinary insights~\citep{kafkas_database_2015, piwowar_beginning_2011, zhao_data_2018,khan_measuring_2021,cao_rise_2023}. Furthermore, the use of data availability statements to accurately identify datasets in academic publications remains limited~\citep{jiao_data_2024,strcic_open_2022}. Some studies suggest that datasets are more commonly cited informally within the text, as opposed to formal citations in references~\citep{belter_measuring_2014,kafkas_database_2015}. While some researchers use the Data Citation Index (DCI) to examine dataset usage, the DCI's focus on natural sciences results in limited coverage across disciplines~\citep{silvello_theory_2017,park_informal_2018,park_examination_2017}, with citation patterns that remain incomplete~\citep{robinson-garcia_analyzing_2016}.

One related study, \cite{cao_rise_2023} investigated the adoption of data and method-sharing practices by analyzing a dataset of 1.1 million arXiv papers, concentrating on physics, mathematics, and computer science. They utilized regular expression matching to extract URLs from the LaTeX-formatted full text of these papers, classifying the URLs as ``data URLs'' or ``method URLs'' using manual annotation and a fine-tuned SciBERT model. Their findings highlighted a growing trend in link-sharing for methods and data, with an increasing number of papers incorporating such URLs over time. They also noted a rise in the reuse of the same links across papers, particularly in computer science, indicating a possible expansion of reproducibility efforts. Furthermore, the analysis revealed a consolidation of links within fewer web domains, such as GitHub, over time. Importantly, papers featuring shared links tended to have a higher citation impact, especially when the links remained active, underscoring the practical benefits of data-sharing practices.

While this study represents a valuable contribution by leveraging full-text analysis on a large dataset, it has notable limitations. Its focus on preprint articles and specific disciplines (physics, mathematics, and computer science) may restrict the generalizability of its findings. Preprints are not universally utilized across all academic disciplines, meaning the dataset may not adequately capture fields where preprint culture is less established. Moreover, the exclusion of formally published articles leaves unanswered questions about potential differences in data-sharing practices between preprints and peer-reviewed publications. Considering the diverse adoption rates of data-sharing practices across scientific disciplines, expanding this research to include formally published articles and additional fields would offer a more comprehensive understanding of how data-sharing practices vary and evolve.


To deepen our understanding of data-sharing and reuse practices, further work across disciplines is essential. Our study seeks to provide a more comprehensive perspective on data use across scientific fields, filling gaps left by previous research that focused on specific disciplines or datasets. By broadening the scope of analysis, the study aims to offer practical insights into the factors influencing data-sharing practices and the variability observed across disciplines. These insights can help foster the adoption of standardized practices and promote a more widespread culture of data-sharing within the research community, ultimately enhancing collaboration, reproducibility, and the overall impact of scientific research.