%File: formatting-instructions-latex-2025.tex
%release 2025.0
\documentclass[letterpaper]{article} % DO NOT CHANGE THIS
\usepackage{aaai25}  % DO NOT CHANGE THIS
\usepackage{times}  % DO NOT CHANGE THIS
\usepackage{helvet}  % DO NOT CHANGE THIS
\usepackage{courier}  % DO NOT CHANGE THIS
\usepackage[hyphens]{url}  % DO NOT CHANGE THIS
\usepackage{graphicx} % DO NOT CHANGE THIS
\urlstyle{rm} % DO NOT CHANGE THIS
\def\UrlFont{\rm}  % DO NOT CHANGE THIS
\usepackage{natbib}  % DO NOT CHANGE THIS AND DO NOT ADD ANY OPTIONS TO IT
\usepackage{caption} % DO NOT CHANGE THIS AND DO NOT ADD ANY OPTIONS TO IT
\frenchspacing  % DO NOT CHANGE THIS
\setlength{\pdfpagewidth}{8.5in}  % DO NOT CHANGE THIS
\setlength{\pdfpageheight}{11in}  % DO NOT CHANGE THIS
%
% These are recommended to typeset algorithms but not required. See the subsubsection on algorithms. Remove them if you don't have algorithms in your paper.
\usepackage{algorithm}
\usepackage{algorithmic}

%
% These are are recommended to typeset listings but not required. See the subsubsection on listing. Remove this block if you don't have listings in your paper.
\usepackage{newfloat}
\usepackage{listings}
\DeclareCaptionStyle{ruled}{labelfont=normalfont,labelsep=colon,strut=off} % DO NOT CHANGE THIS
\lstset{%
	basicstyle={\footnotesize\ttfamily},% footnotesize acceptable for monospace
	numbers=left,numberstyle=\footnotesize,xleftmargin=2em,% show line numbers, remove this entire line if you don't want the numbers.
	aboveskip=0pt,belowskip=0pt,%
	showstringspaces=false,tabsize=2,breaklines=true}
\floatstyle{ruled}
\newfloat{listing}{tb}{lst}{}
\floatname{listing}{Listing}

% ------------------my package----------------------
\usepackage[switch]{lineno} 
\usepackage{xcolor}
\usepackage{cite}
\usepackage{multirow}
\usepackage{subfigure}
\usepackage{xspace}
\usepackage{dsfont}
\usepackage{epstopdf}
\usepackage{booktabs}
% \usepackage{indentfirst}
\usepackage{amsmath}
\usepackage{amsthm}
\usepackage{amsfonts}
\usepackage{bm}
\usepackage{adjustbox}
\usepackage{appendix}


%------------------my command----------------------
\newtheorem{mydef}{Definition}
\newcommand{\bitem}[1]{\noindent$\bullet$ \textbf{#1}}
\newcommand{\paratitle}[1]{\noindent\textbf{#1}}
\newcommand{\ie}{\emph{i.e.,}\xspace}
\newcommand{\aka}{\emph{a.k.a.,}\xspace}
\newcommand{\eg}{\emph{e.g.,}\xspace}
\newcommand{\etal}{\emph{et al.}\xspace}
\newcommand{\etc}{\emph{etc.}\xspace}
\newcommand{\wrt}{\emph{w.r.t.}\xspace}
\newcommand{\uem}[1]{\emph{\underline{#1}}}
\newcommand{\tabincell}[2]{\begin{tabular}{@{}#1@{}}#2\end{tabular}}

\newcommand{\name}{POI-Enhancer\xspace}
\newcommand{\figureautorefname}{Fig.}
\newcommand{\tableautorefname}{Tab.}
\DeclareMathOperator{\udiv}{div}
\DeclareMathOperator{\st}{s.t.}
\DeclareMathOperator{\wh}{where}


%
% Keep the \pdfinfo as shown here. There's no need
% for you to add the /Title and /Author tags.
\pdfinfo{
/Title (POI-Enhancer: An LLM-based Semantic Enhancement Framework \\ for POI Representation Learning)
/Author (Paper ID: 11002)
/TemplateVersion (2025.1)
}
% DISALLOWED PACKAGES
% \usepackage{authblk} -- This package is specifically forbidden
% \usepackage{balance} -- This package is specifically forbidden
% \usepackage{color (if used in text)
% \usepackage{CJK} -- This package is specifically forbidden
% \usepackage{float} -- This package is specifically forbidden
% \usepackage{flushend} -- This package is specifically forbidden
% \usepackage{fontenc} -- This package is specifically forbidden
% \usepackage{fullpage} -- This package is specifically forbidden
% \usepackage{geometry} -- This package is specifically forbidden
% \usepackage{grffile} -- This package is specifically forbidden
% \usepackage{hyperref} -- This package is specifically forbidden
% \usepackage{navigator} -- This package is specifically forbidden
% (or any other package that embeds links such as navigator or hyperref)
% \indentfirst} -- This package is specifically forbidden
% \layout} -- This package is specifically forbidden
% \multicol} -- This package is specifically forbidden
% \nameref} -- This package is specifically forbidden
% \usepackage{savetrees} -- This package is specifically forbidden
% \usepackage{setspace} -- This package is specifically forbidden
% \usepackage{stfloats} -- This package is specifically forbidden
% \usepackage{tabu} -- This package is specifically forbidden
% \usepackage{titlesec} -- This package is specifically forbidden
% \usepackage{tocbibind} -- This package is specifically forbidden
% \usepackage{ulem} -- This package is specifically forbidden
% \usepackage{wrapfig} -- This package is specifically forbidden
% DISALLOWED COMMANDS
% \nocopyright -- Your paper will not be published if you use this command
% \addtolength -- This command may not be used
% \balance -- This command may not be used
% \baselinestretch -- Your paper will not be published if you use this command
% \clearpage -- No page breaks of any kind may be used for the final version of your paper
% \columnsep -- This command may not be used
% \newpage -- No page breaks of any kind may be used for the final version of your paper
% \pagebreak -- No page breaks of any kind may be used for the final version of your paperr
% \pagestyle -- This command may not be used
% \tiny -- This is not an acceptable font size.
% \vspace{- -- No negative value may be used in proximity of a caption, figure, table, section, subsection, subsubsection, or reference
% \vskip{- -- No negative value may be used to alter spacing above or below a caption, figure, table, section, subsection, subsubsection, or reference

\setcounter{secnumdepth}{1} %May be changed to 1 or 2 if section numbers are desired.

% The file aaai25.sty is the style file for AAAI Press
% proceedings, working notes, and technical reports.
%

% Title

% Your title must be in mixed case, not sentence case.
% That means all verbs (including short verbs like be, is, using,and go),
% nouns, adverbs, adjectives should be capitalized, including both words in hyphenated terms, while
% articles, conjunctions, and prepositions are lower case unless they
% directly follow a colon or long dash
\title{POI-Enhancer: An LLM-based Semantic Enhancement Framework \\ for POI Representation Learning}
\author{
    %Authors
    % All authors must be in the same font size and format.
    Jiawei Cheng\textsuperscript{\rm 1,2},
    Jingyuan Wang\textsuperscript{\rm 1,3,4,}\footnote{Corresponding authors.},
    Yichuan Zhang\textsuperscript{\rm 1},\\
    Jiahao Ji\textsuperscript{\rm 1},
    Yuanshao Zhu\textsuperscript{\rm 2},
    Zhibo Zhang\textsuperscript{\rm 1},
    Xiangyu Zhao\textsuperscript{\rm 2,}\footnotemark[1]
}
\affiliations{
    %Afiliations
    % If you have multiple authors and multiple affiliations
    % use superscripts in text and roman font to identify them.
    % For example,

    % Sunil Issar\textsuperscript{\rm 2}, 
    % J. Scott Penberthy\textsuperscript{\rm 3}, 
    % George Ferguson\textsuperscript{\rm 4},
    % Hans Guesgen\textsuperscript{\rm 5}
    % Note that the comma should be placed after the superscript
    % email address must be in roman text type, not monospace or sans serif

    \textsuperscript{\rm 1}SKLCCSE, School of Computer Science and Engineering, Beihang University, Beijing, China\\
    \textsuperscript{\rm 2}Department of Data Science, City University of Hong Kong, Hong Kong, China\\
    \textsuperscript{\rm 3}MIIT Key Laboratory of Data Intelligence and Management, Beihang University, Beijing, China\\
    \textsuperscript{\rm 4}School of Economics and Management, Beihang University, Beijing, China\\
    \{JarvisC, jywang, tadshi, jiahaoji, hljzhangzhibo\}@buaa.edu.cn, yaso.zhu@my.cityu.edu.hk, xy.zhao@cityu.edu.hk
%
% See more examples next
}

\begin{document}

\maketitle

\begin{abstract}
POI representation learning plays a crucial role in handling tasks related to user mobility data. Recent studies have shown that enriching POI representations with multimodal information can significantly enhance their task performance. 
Previously, the textual information incorporated into POI representations typically involved only POI categories or check-in content, leading to relatively weak textual features in existing methods. 
In contrast, large language models (LLMs) trained on extensive text data have been found to possess rich textual knowledge.
However leveraging such knowledge to enhance POI representation learning presents two key challenges: first, how to extract POI-related knowledge from LLMs effectively, and second, how to integrate the extracted information to enhance POI representations.
To address these challenges, we propose POI-Enhancer, a portable framework that leverages LLMs to improve POI representations produced by classic POI learning models. We first design three specialized prompts to extract semantic information from LLMs efficiently. Then, the Dual Feature Alignment module enhances the quality of the extracted information, while the Semantic Feature Fusion module preserves its integrity. The Cross Attention Fusion module then fully adaptively integrates such high-quality information into POI representations and Multi-View Contrastive Learning further injects human-understandable semantic information into these representations. Extensive experiments on three real-world datasets demonstrate the effectiveness of our framework, showing significant improvements across all baseline representations.
\end{abstract}

% Uncomment the following to link to your code, datasets, an extended version or similar.
%
\begin{links}
    \link{Code}{https://github.com/Applied-Machine-Learning-Lab/POI-Enhancer}
    \link{Extended version}{https://arxiv.org/abs/2502.10038}
\end{links}

%%%1111111111111111_introduction_1111111111111111111%%%
\section{Introduction}\label{sec:intro}

With the advancement of smart city technology~\cite{cityshield, DGeye,infer} and the widespread adoption of smart devices, the volume of location-based mobile data, such as POI (Points of Interest) check-in data and user trajectory data, has surged~\cite{ding2018ultraman,nipsdataset}. Predicting user destinations~\cite{zhao2020go}, forecasting visit flow~\cite{song2020spatial}, and similar tasks~\cite{hx} have become key research focuses. In tackling these difficulties, POI representation learning, which can be trained via self-supervised methods and utilized across various tasks like traffic forecasting~\cite{fullbayesian, STDEN,energyfield} and trajectory predction~\cite{trajrepr, Astar, CD-CNN}, stands as a particularly meaningful and promising direction.

To enhance the diversity of information within POI representation vectors and achieve superior performance in complex downstream tasks, researchers are exploring the integration of various information beyond basic geographic data. For example, they incorporated user preference data~\cite{ppr} and visual information~\cite{cityfm} into POI representations. Although related textual information, such as POI categories (e.g. restaurants and hotels) and check-in content on social media like Twitter, provides some insights into the social functions and other aspects of POIs, the semantic richness and depth of these data are limited. When compared to the vast amount of descriptive information available on the internet regarding POIs, these data sources fall short in both content richness and coverage. 
In recent years, large language models (LLMs) trained on extensive volumes of internet data have been applied across numerous fields, demonstrating remarkable capabilities, particularly in the domain of spatial-temporal data~\cite{urbangpt}. Although LLMs have proven beneficial in addressing challenges in this area, leveraging LLMs to enhance POI representation presents two specific challenges.


The first challenge lies in effectively \textbf{extracting the geographical knowledge within LLMs}. A common idea~\cite{xval} is to provide LLMs with prompts related to geographic information and then obtain text output. However, LLMs have limitations in handling numerical input, and for representation learning, we need vectors that are versatile across tasks, which makes this method not suitable. Some studies~\cite{gatgpt,stllm} have also experimented with feeding extracted spatial-temporal features to a partially or fully frozen LLM, using the LLM as the backbone to solve specific problems. But these works are typically tailored to a single spatial-temporal task and extract information specific to that task only. However, POI representation learning aims to capture versatile information to address diverse tasks. Clearly, task-specific extraction is insufficient for this requirement.

The second challenge is how to effectively \textbf{integrate the extracted textual information into POI representation} for enhancement. Since the information extracted by LLMs is versatile, combining these diverse aspects information with the POI representation is difficult. Most researchers ~\cite{ppr} adopted the approach of one-hot encoding the corresponding POI category features and then concatenating them with the representation vectors, which overlooks the interactions between features. For example, the POI category and time are related: a restaurant's lunch hours and lunch break times exhibit different visitor flow patterns. This limits the ability to exploit the richness of semantic information to enhance POI representations.

To address the challenges, we propose a POI representation enhancement framework, called \name, which is designed to leverage textual information in LLM to strengthen embedding vectors. Specifically, to better utilize LLMs for extracting textual features of POIs, we develop unique prompts to separately extract features related to POI addresses, visit patterns, and surrounding environments. Following this, we design the Dual Feature Alignment module to leverage the relationships between textual features, enabling the acquisition of higher-quality semantic information. The Semantic Feature Fusion module is specifically designed to ensure the preservation of high-quality semantic information. Then, to fully integrate the extracted information with the representation vectors, we introduce the Cross Attention Fusion module based on the attention mechanism. Finally,  we incorporate Multi-View Contrastive Learning to further inject human-understandable semantic information into POI representations to enhance its capability of capturing real-world patterns. 

We summarize our main contributions as follows:
\begin{itemize}
    \item To the best of our knowledge, \name is the first to introduce LLM-based textual knowledge to enhance POI representations of POI learning models. We demonstrate that leveraging knowledge from LLMs is crucial for boosting the performance of POI embedding models.
    
    \item We design three kinds of specialized prompts to thoroughly extract textual information from LLMs, and employ a cross-attention mechanism to adaptively integrate these information into POI representations. We also introduce temporal, spatial, and functional contrastive learning to refine the POI representations.

    \item We conducted extensive experiments on three public real-world datasets across various downstream tasks. The results demonstrate that our approach significantly enhances the performance of POI embedding methods.
\end{itemize}



%%%22222222222_preliminaries_22222222222222%%%
\section{Preliminaries}\label{sec:pre}
\begin{mydef}[\textbf{Point of Interest (POI)}]
A POI is a specific geographic location with some basic attributes $p=(id, pn, c, lon, lat)$, where $id$ indicates index, $pn$ means name of POI, $c$ denotes category, $lon$ and $lat$ represent longitude and latitude coordinates respectively. Besides, each POI has some extra attributes such as visit pattern, address, and surrounding environment. 
An example of the attributes of a POI in New York City is provided in \tableautorefname~\ref{tab:poi_example}.
\end{mydef}

\begin{mydef}[\textbf{Check-in Record}]
A check-in record is a triplet $r=(u,p,t)$ which means a user $u$ visits POI $p$ at time $t$. A user's movement behavior over a period of time can be modeled by a sequence of check-in records, which we define as a Check-in Record Sequence. It can be represented by $R = \{r_1,r_2,...,r_L\}$, where the check-in records are arranged in the order of time sequence and $L$ is the length of the Check-in Record Sequence. 
We also denote the set of all users' check-in record sequences as $S$.
\end{mydef}

\begin{mydef}[\textbf{POI Representation}]
Given the set of all POIs $P=\{p_1,~p_2,~\ldots,~p_N\}$, where n is the number of the set, a mapping function $f$ is used to generate a fixed vector representation $E_{p_i} = f(p_i)$ for each POI. 
\end{mydef}

\noindent \textbf{Problem Statement}. Given a POI Representation function $\mathcal{F}$, POIs set $P=\{p_1,~p_2,~\ldots,~p_N\}$ and other related data \eg check-in record sequences $S$, with the aid of LLM, the aim of our framework is to learn a function $g$ that enhance the capability of the function $F$, \ie $E_{p_i} = g(\mathcal{F} (p_i)), \ E_{p_i} \in \mathbb{R}^d$, where $d$ is a uniform dimension. 

\begin{table}[ht]
\centering
    \resizebox{0.7\columnwidth}{!}{%
        \begin{tabular}{rccc}
            \toprule
            Attribute & Value \\
            \midrule
            POI ID &  22337 \\ 
            Name &  New York Stock Exchange \\ 
            Longitude & 74.011154 \\ 
            Latitude & 40.706806 \\
            Category & Stock Exchange \\
            Street Name  & Wall Street\\
            House Number & 11 \\
            Postal Code & 10005 \\
            Surrounding & Office, Building and Road \\
            Visit Pattern & Between 6 am and 9 am, Weekday \\
            \bottomrule
        \end{tabular}
        }
        \caption{An Example of the POI attributes.} 
    \label{tab:poi_example}
\end{table}



%%%3333333333333_Methoddology_3333333333333%%%
\section{Methodology}\label{sec:method}

\begin{figure*}[t]
    \centering
    \includegraphics[width=\textwidth]{framework_big.pdf} 
    \caption{(a): Prompt Generation and Feature Extraction are used to obtain prompts and get textual features from the LLM. (b): Embedding Enhancement is designed to enhance POI embeddings by leveraging textual features. (c): Multi-View Contrastive Learning enables the sampling of more diverse positive and negative examples during training.}
    \label{fig:framework and prompt}
\end{figure*}

This section provides a comprehensive demonstration of the technical details of \name framework and  \figureautorefname~\ref{fig:framework and prompt} presents the overall architecture. In \figureautorefname~\ref{fig:framework and prompt}, part (a) is the Prompt Generation and Feature Extraction phase, where specialized prompts are generated and used to extract relevant semantic information from the LLM. The second phase, Embedding Enhancement, corresponds to part (b), where the extracted information is further refined and integrated with the POI representations to be enhanced. Finally, part (c) represents Multi-View Contrastive Learning, where we designed three sampling strategies to select positive and negative samples for contrastive learning. Besides, to assist LLMs in more accurately capturing POI-related knowledge, we additionally processed and derived three kinds of extra attributes mentioned above. A detailed description of this procedure can be found in the Supplementary Material.



%-------------------------------------------------------------------%

\subsection{Prompt Generation and Feature Extraction}
    \paratitle{Generate prompt}
    Due to the LLM's low sensitivity to numbers, we need to bundle basic attributes like latitude, longitude, and name with extra attributes when inputting them, to help the LLM accurately target the desired POI. Besides, simply stacking various features into a prompt can make it difficult for the LLM to focus on key points and effectively extract information. Hence, the proposed prompt pattern consists of three parts: (1) Role-Playing, (2) POI Information, and (3) the Question. The POI Information part encompasses basic information and extra information, corresponding to the basic and extra attributes, respectively. Firstly, the design purpose of role-playing at the beginning of the prompt is to allow the LLM to fully unleash its knowledge, enabling the LLM to embody a role familiar with geographical knowledge. An attribute header is added in front of the POI information to help the LLM accurately capture the information of input attributes. Next, we generate multiple sentences based on combinations of the basic attributes and three extra attributes. Lastly, inspired by the ~\cite{gurnee2024worldmodel}, we design the question at the end of each prompt about the content to trigger the relevant knowledge.
    Consequently, we generate three types of prompts for each POI $p_i$:POI Visit Pattern Prompt, POI Address Prompt, and POI Surrounding Prompt, denoted as $T^V_{p_{i}}, T^A_{p_{i}}, T^S_{p_{i}}$. An Example of the prompt we generated is shown in \figureautorefname~\ref{fig: prompt}, which is in Supplementary Material. 
    
    
    \paratitle{Extract from LLM}
    In \name, we input the prompts into the LLM and take the final hidden layer state from the LLM as the semantic feature. It is worth noting that the LLM serves as a frozen encoder when training. So, for a POI $p_i$, the feature extraction process can be denoted as:
    \begin{equation} \small
        \begin{aligned}
             \boldsymbol{E}^V_{p_i} = \mathcal{H}(T^V_{p_{i}}) , \
             \boldsymbol{E}^A_{p_i} = \mathcal{H}(T^A_{p_{i}}) , \
             \boldsymbol{E}^S_{p_i} = \mathcal{H}(T^S_{p_{i}}) ,
        \end{aligned}
    \end{equation}

    \noindent where $\boldsymbol{E}^V_{p_i},\boldsymbol{E}^A_{p_i},\boldsymbol{E}^S_{p_i}\in \mathbb{R}^{D}$ are the corresponding semantic feature of three kinds of prompts, $\mathcal{H}$ is the process of extracting the last hidden state from the LLM, and $D$ is the dimension size of the hidden state vector.
    
    
    

%-------------------------------------------------------------------%
\subsection{Embedding Enhancement}
    \textit{Dual Feature Alignment} leverages the intricate connections between address and visit patterns, as well as between address and surrounding environment to obtain higher-quality semantic features. \textit{Semantic Feature Fusion} uses attention score-weighted merging to ensure the quality of the features when fusing the semantic features into a single semantic vector. Afterward, \textit{Cross Attention Enhancement}, based on the cross-attention method, employs the semantic vector obtained earlier to fully integrate and enhance the POI representations, resulting in the final output vector.
    
    \paratitle{Dual Feature Alignment}
    A POI's address, a key factor of geography information, is closely linked to its visit patterns and surrounding environment. For example, as shown in \tableautorefname ~\ref{tab:poi_example}, the New York Stock Exchange is on Wall Street, a well-known hub of financial firms. People often visit there during daytime working hours, and the surrounding environment mainly consists of office spaces. If we align the address textual feature with the visit pattern and surrounding environment textual features, we can obtain higher-quality textual information.
    Thus, we designed Dual Feature Alignment. First, Given a batch of textual information of $n$ POIs $\{E^V, E^A, E^S\}$, they will be fed into a linear layer to transform them into a hidden space with the same dimension as the POI embedding to be enhanced, denoted as:
     \begin{equation}\small
        \boldsymbol{\tilde{E}}^V=\boldsymbol{W}^{V'} \boldsymbol{E}^V , \
        \boldsymbol{\tilde{E}}^A=\boldsymbol{W}^A \boldsymbol{E}^A , \
        \boldsymbol{\tilde{E}}^S=\boldsymbol{W}^S \boldsymbol{E}^S  ,
    \end{equation}
    \noindent where $\boldsymbol{\tilde{E}}^{V},\boldsymbol{\tilde{E}}^A,\boldsymbol{\tilde{E}}^S \in \mathbb{R}^{n \times d}$, $d$ is the dimension of the hidden space and $\boldsymbol{W}^{V'},\boldsymbol{W}^A,\boldsymbol{W}^S$ are all learnable matrices.

    Next, to leverage the relationships between textual features and obtain higher-quality information, multiple layers of the Transformer encoder are introduced. Each layer consists of multi-head attention ($\mathrm{MHA}$), residual connections, and layer normalization operations ($\mathrm{LN}$) and the number of layers is $L_1$.
    Formally, take the relation between address and visit patterns as an example, given  the vectors $\{\tilde{E}^V,\tilde{E}^A\}$, we computed a $\mathrm{MHA}$ as follows:
    \begin{equation}\small \label{eq:qkv} 
        \begin{aligned}
            \boldsymbol{Q} = \boldsymbol{\tilde{E}}^V \boldsymbol{W}^Q  ,\
            \boldsymbol{K} = \boldsymbol{\tilde{E}}^A \boldsymbol{W}^K  ,\
            \boldsymbol{V} = \boldsymbol{\tilde{E}}^A \boldsymbol{W}^V  ,
        \end{aligned}
    \end{equation}
    \begin{equation}\small
        \begin{aligned}
            head_h = \phi(\frac{\boldsymbol{Q}\boldsymbol{K}^T}{\sqrt{d}})\boldsymbol{V}  ,\\
        \end{aligned}
    \end{equation}
    \begin{equation}
        \begin{aligned}\small \label{eq:mha}
            \mathrm{MHA}(\boldsymbol{\tilde{E}}^V,\boldsymbol{\tilde{E}}^A)=(\Vert^H_{h=1}head_h)\boldsymbol{W}^O  ,
        \end{aligned}
    \end{equation}
    \noindent where $\boldsymbol{W}^Q,\boldsymbol{W}^K,\boldsymbol{W}^V \in \mathbb{R}^{d \times d_h}$ are learnable parameters, $\phi$ is softmax activation function,  $d_h$ is the dimension of a single head. And $\Vert$ is the concatenation operation, $\boldsymbol{W}^O \in \mathbb{R}^{ (d_h \cdot H) \times d}$ is a trainable parameter and $H$ denotes the number of heads.
    The output of the first layer $Z_1'$ is denoted as:
    \begin{equation}\small \label{eq:addnorm}
        \begin{aligned}
           \boldsymbol{Z} = \mathrm{LN}(\boldsymbol{\tilde{E}}^A+\mathrm{MHA} (\boldsymbol{\tilde{E}}^V,\boldsymbol{\tilde{E}}^A)) ,
        \end{aligned}
    \end{equation}
    \begin{equation}\small
        \begin{aligned}
           \boldsymbol{Z}_1' = \mathrm{LN}(\boldsymbol{Z}+\mathrm{FFN}(\boldsymbol{Z})) ,
        \end{aligned}
    \end{equation}
    \noindent where $\mathrm{FFN}$ is a feed-forward neural network. Then, vector $\boldsymbol{Z}_1'$, along with $\boldsymbol{E}^{A}$, will be fed back into the next layer as input, and after repeating this process $L_1-1$ times, the final layer result $\boldsymbol{Z}_{L_1}'$ is the vector $\boldsymbol{E}^{A-V} \in \mathbb{R}^{n \times d}$. It should be noted that $\{Z_k'| k \in [1,L_1]\}$ is transformed into $K$ and $V$, while  $\tilde{E}^A$ is converted into $\boldsymbol{Q}$ in the subsequent repetition process.
    Similarly, to deal with the connection between address and surrounding by replacing $\boldsymbol{\tilde{E}}^V$ with $\boldsymbol{\tilde{E}}^S$ in Formula ~(\ref{eq:qkv}), (\ref{eq:mha}) and (\ref{eq:addnorm}). We can get the output $E^{A-S}$ accordingly.

    \paratitle{Semantic Feature Fusion}
    Considering that visit patterns are related to the surrounding environment, for example, POIs near entertainment venues are mostly accessed on weekends. We got a comprehensive semantic feature by combining the two feature vectors from the previous module into one.
    To integrate two vectors into one while maintaining the quality of the vector, we designed the Semantic Feature Fusion based on a weighted sum method. 
    Accordingly, the computation process can be represented as follows:
    \begin{equation}\small
        \begin{aligned}
            \theta^{A-V} = \boldsymbol{W}_2 \cdot LeakyReLU([\boldsymbol{W}_1 \boldsymbol{E}^{A-V} || \boldsymbol{W}_1  \boldsymbol{E}^{A-S}]), \\ 
            \theta^{A-S} = W_2 \cdot LeakyReLU([\boldsymbol{W}_1 \boldsymbol{E}^{A-S} || \boldsymbol{W}_1  \boldsymbol{E}^{A-V}]),
        \end{aligned}
    \end{equation}
    \noindent where $\theta^{A-V}$ and $\theta^{A-S}$ are the attention scores for $\boldsymbol{E}^{A-V}$ and $\boldsymbol{E}^{A-S}$. $\boldsymbol{W}_1 \in \mathbb{R}^{d \times d'}$ and $\boldsymbol{W}_2 \in \mathbb{R}^{2d' \times 1} $ are used to project the features into the same hidden space and to transform them into attention weights, respectively.  $LeakyReLU$ is an activation function, and $d'$ is the dimension of the latent space.
    After that, a softmax activation is employed to get the normalized weight, and a weighted sum fusion of two semantic features is applied to get the output  $E^{LLM} \in \mathbb{R}^{n \times d}$, which can be represented as: 
    \begin{equation}\small
        \begin{aligned}
        [\omega^{A-V},\omega^{A-S}] = \phi([\theta^{A-V},\theta^{A-S}]) ,
        \end{aligned}
    \end{equation}
    \begin{equation}\small
        \begin{aligned}
            \boldsymbol{E}^{LLM} = \omega^{A-V} \cdot \boldsymbol{E}^{A-V} +  \omega^{A-S} \cdot \boldsymbol{E}^{A-S}.
        \end{aligned}
    \end{equation}
    
    
    \paratitle{Cross Attention Fusion}
    Cross-attention techniques have been employed to fully fuse features from different views ~\cite{dafusion}. Hence, inspired by ~\cite{urbanclip}, to enhance other embedding methods by making use of the vector $E^{LLM}$, a Cross Attention Fusion is developed.  

    Here, we also employ a multi-layer transformer encoder architecture but in each layer we use the multi-query attention ~\cite{MultiQueryAttention} plus parallel attention and feed-forward net (PAF) ~\cite{parallelAttention} to combine $E^{LLM}$ and $E^{POI} \in \mathbb{R}^{n \times d}$. The multi-query attention (MQA) is almost the same as the multi-head attention except all heads share the same set of $K$ and $V$, which is proved to be faster with minor quality degradation in the calculation. Additionally,  PAF can be effective in improving the performance of transformer-based models. As shown in \figureautorefname~\ref{fig:framework and prompt}, the first layer of the Cross Attention Fusion can be presented formally as:
    \begin{equation}\small
        \begin{aligned}
            \boldsymbol{X} = \mathrm{LN}(\boldsymbol{E}^{POI}+MQA(\boldsymbol{E}^{LLM},\boldsymbol{E}^{POI})) ,
        \end{aligned}
    \end{equation} 
    \begin{equation}
        \begin{aligned}
            \boldsymbol{X}_1' = \mathrm{LN}(\boldsymbol{X}+\mathrm{FFN}(\boldsymbol{X})) ,
        \end{aligned}
    \end{equation}
    
    Then, the vector $\boldsymbol{X}_1'$ and $\boldsymbol{E}^{POI}$
    will be fed into the next layer, and after repeating this process in $\boldsymbol{L}_2-1$ times, the outcome of the last layer $\boldsymbol{E}_{FUSE}$ is obtained. It is worth to noticed that $\{\boldsymbol{X}_k'|k\in [1,L_2]\}$ is transformed into $\boldsymbol{K}$ and $\boldsymbol{V}$, and $\boldsymbol{E}^{LLM}$ is converted into $\boldsymbol{Q}$ in the following repetition.
%-------------------------------------------------------------------%

\subsection{Multi-View Contrastive Learning}
    Our Multi-View Contrastive Learning approach is designed to enhance the similarity between the anchor POI and positive samples, while simultaneously reducing the similarity with negative samples. This strategy aims to strengthen the robustness and effectiveness of the embedding vector.
    However, Unlike previous works that only use distance as the sampling criterion~\cite{urbanfootnote},  we incorporated temporal, spatial, and functional views into our considerations and designed three sampling strategies. Besides, the formal definitions of the following three sampling strategies are presented in the Supplementary Material.
    
    \paratitle{Sequence-Time Contrastive Learning}
    The visit context of a POI \ie the neighboring check-in records in the check-in record sequence is often considered an important factor. However, if the duration of a check-in record sequence is very long, two adjacent consecutive check-in records may be separated by several days. Considering such neighbors as positive samples will reduce the effectiveness of contrastive learning. Therefore, to avoid this situation, we propose a Sequence-Time sampling strategy. The positive samples are required not only to be neighbors of the check-in record but also to have the same visit date as the anchor sample.
    
    
    \paratitle{Geography Contrastive Learning}
    From a spatial perspective, our strategy incorporates both local spatial similarity and category similarity as criteria. Specifically, for a given POI, we define a square area centered around it and consider POIs of the same category in that area as positive samples.


    \paratitle{Functional Contrastive Learning}
    Apart from the two types of contrastive learning mentioned above, we aim to identify groups of POIs that are similar in social function. Therefore, based on the category and visit patterns of POIs, We propose the following principle for selecting positive samples: only POIs that share the same category and visit pattern as the anchor sample are regarded as positive samples.
    \newline
    In summary, based on the above three criteria, we sampled more high-quality positive samples for subsequent contrastive learning training. This approach helps enhance the comprehensive capability of the representation vectors.


    
    \subsection{Model Training}
    
    Given a POI $p_i$ and a set of all the POIs $P$, we derive its positive set $P_i^+$ through the above strategies. And for each pair in $\{(p_i,p_i^+)|p_i^+\in P_i^+\}$, we will randomly choose $m-2$ negative samples from the negative set $\{p_i^-|p_i^- \in P_i^-, P_i^-= P-p_i-P_i^+\}$,  to form a training batch.
    
    Then we use InfoNCE as the loss for contrastive learning:
    \begin{equation}\small
        \mathcal{L}_{Cont} = -log\frac{e^{\frac{1}{\gamma}sim(p_i, p_i^{+})}}{\sum_{i=0}^{m}e^{\frac{1}{\gamma}sim(p_i, p_i^{-})}}
    \end{equation}
    where $sim(\cdot, \cdot)$ is a similarity measure function,  $\gamma$ is a temperature parameter and $m$ is number of POIs in the batch.

    Apart from this, in order to maintain the similarity between the origin vectors and enhanced vectors, a loss based on cosine similarity is constructed, which can be defined as:
    \begin{equation}\small
        \begin{aligned}
            &\mathcal{L}_{Sim} =  \\
        &\frac{1}{m^2}\sum^m_{i=1}\sum^m_{j=1}|cos(E^{FUSE}_i,E^{FUSE}_j)-cos(E^{POI}_i,E^{POI}_j)|,
        \end{aligned}
    \end{equation}

    \noindent where $cos$ is the cosine similarity function.

    Ultimately, the loss of \name  can be denoted:
    \begin{equation}\small
        \mathcal{L} = \mathcal{L}_{Cont} + \mathcal{L}_{Sim}.
    \end{equation}

    
    

%%%444444444444_Experiments_44444444444444%%%
\section{Experiments}\label{sec:expt}
\subsection{Experiment Setup}
\paratitle{Datasets}
We conducted experiments on three check-in datasets provided by~\cite{4square}: Foursquare-NY, Foursq-SG, and Foursquare-TKY, sampled from New York, Singapore, and Tokyo, respectively. We remove all POIs with less than 5 check-ins in the dataset and check-in sequences with less than 10 POIs. The statistics of the processed dataset are in Supplementary Material.
Then we shuffled the dataset and split it into a ratio of 2:1:7 for the test set, validation set, and training set. It should be noted that the training set for the POI Recommendation task will also be used as the dataset for sampling in contrastive learning. 

\paratitle{Baselines}
We introduced six baselines in our experiment including Skip-Gram~\cite{mikolov2013efficient}, POI2Vec~\cite{feng2017poi2vec}, Geo-Teaser~\cite{zhao2017geo}, TALE~\cite{wan2021pre}, Hier~\cite{shimizu2020enabling}, and CTLE~\cite{lin2021pre}. The details of the baselines are in the Supplementary Material. LLM-based baselines are also included like Llama2~\cite{llama2}, ChatGLM2~\cite{chatglm}, and GPT-2~\cite{gpt2}.

\paratitle{Downstream Tasks \& Metrics}
To evaluate \name and make a comprehensive comparison, we set up three downstream tasks based on LibCity~\cite{libcity}.

\bitem{POI Recommendation},
Based on a user's historical check-in sequence, the POI Recommendation task aims to predict the next POI the user would visit.      

\bitem{Check-in Sequence Classification},
Given an arbitrary check-in sequence, this task requires the downstream model to detect which user this sequence belongs to.  


\bitem{POI Visitor Flow Prediction},
POI visitor flow prediction requires the downstream model to forecast the future volume of visitor flow based on historical visitor data.

In the POI Recommendation task, we use Hit@$k$ as the evaluation metric (value equals  1 if the ground truth is among the top k in the recommendation list, otherwise 0, \ $k=1,~5$). The Check-in Sequence Classification task is evaluated using Accuracy (ACC) and Macro-F1 while the Visitor Flow Prediction task is assessed with Mean Absolute Error (MAE) and Root Mean Square Error (RMSE).


\paratitle{Implementation}
% In our framework, we use the Llama-2-7B as the LLM backbone. The dimension $d$ is uniformly set to 256.  Moreover, the number of encoder layers $L_1$ in Dual Feature Alignment and the number of encoder layers $L_2$ in Cross Attention Fusion is 4 and 2. The temperature parameter $\gamma$ in InfoNCE is 0.1. The complete implementation details are in the Supplementary Material.
In our framework, we use the Llama-2-7B as the LLM backbone. The complete implementation details are provided in the Supplementary Material.

\begin{table*}[t]
\belowrulesep=0pt
\aboverulesep=0pt
\setlength{\tabcolsep}{1mm}
\centering
\resizebox{\linewidth}{!}{
\begin{tabular}{l|cc|cc|cc|cc|cc|cc|cc|cc|cc}
    \toprule
    \textbf{Task} & \multicolumn{6}{c|}{\textbf{POI Recommendation}} & \multicolumn{6}{c|}{\textbf{Check-in Sequence Classification}} & \multicolumn{6}{c}{\textbf{POI Visitor Flow Prediction}} \\
    \midrule
    Dataset & \multicolumn{2}{c|}{NY} & \multicolumn{2}{c|}{TKY} & \multicolumn{2}{c|}{SG} & \multicolumn{2}{c|}{NY} & \multicolumn{2}{c|}{TKY} & \multicolumn{2}{c|}{SG} & \multicolumn{2}{c|}{NY} & \multicolumn{2}{c|}{TKY} & \multicolumn{2}{c}{SG} \\
    \midrule
    Metric & Hit@1 & Hit@5 & Hit@1 & Hit@5 & Hit@1 & Hit@5 & Acc   & F1    & Acc   & F1    & Acc   & F1    & MAE   & RMSE  & MAE   & RMSE  & MAE   &  RMSE  \\
    \midrule
    Skip-Gram & 6.984  & 17.356  & 15.037  & 33.305  & 9.194  & 23.652  & 48.967  & 0.224  & 59.982  & 0.413  & 43.768  & 0.229  & 0.371  & 0.747  & 0.494  & 0.691  & 0.665  & 0.994  \\
    Skip-Gram+ & 7.610  & 18.032  & \textcolor[rgb]{ .122,  .137,  .161}{15.557 } & \textcolor[rgb]{ .122,  .137,  .161}{34.197 } & \textcolor[rgb]{ .122,  .137,  .161}{10.747 } & \textcolor[rgb]{ .122,  .137,  .161}{24.468 } & 52.151  & 0.251  & 62.936  & 0.438  & \textcolor[rgb]{ .122,  .137,  .161}{47.285 } & 0.255  & 0.336  & 0.514  & 0.492  & 0.668  & 0.621  & 0.890  \\
    \textbf{Impr.} & \textbf{8.96\%} & \textbf{3.89\%} & \textbf{3.46\%} & \textbf{2.68\%} & \textbf{16.89\%} & \textbf{3.45\%} & \textbf{6.5\%} & \textbf{12.07\%} & \textbf{4.92\%} & \textbf{5.97\%} & \textbf{8.04\%} & \textbf{11.37\%} & \textbf{9.43\%} & \textbf{31.14\%} & \textbf{0.47\%} & \textbf{3.31\%} & \textbf{6.66\%} & \textbf{10.45\%} \\
    \midrule
    POI2Vec & 6.417  & 14.684  & 15.195  & 33.214  & 8.828  & 21.729  & 32.057  & 0.113  & 51.499  & 0.331  & 31.736  & 0.139  & 0.343  & 0.574  & 0.531  & 0.764  & 0.625  & 0.918  \\
     POI2Vec+ & 7.851  & 18.353  & 15.800  & 34.768  & 10.630  & 24.030  & 52.151  & 0.245  & 62.358  & 0.438  & 46.521  & 0.264  & 0.326  & 0.492  & 0.490  & 0.696  & 0.602  & 0.868  \\
    \textbf{Impr.} & \textbf{22.35\%} & \textbf{24.99\%} & \textbf{3.98\%} & \textbf{4.68\%} & \textbf{20.41\%} & \textbf{10.59\%} & \textbf{62.68\%} & \textbf{117.35\%} & \textbf{21.09\%} & \textbf{32.39\%} & \textbf{46.59\%} & \textbf{89.32\%} & \textbf{4.78\%} & \textbf{14.27\%} & \textbf{7.8\%} & \textbf{8.99\%} & \textbf{3.66\%} & \textbf{5.36\%} \\
    \midrule
    Geo-Teaser & 6.174  & 15.355  & 14.956  & 33.814  & 8.768  & 22.851  & 38.296  & 0.149  & 54.852  & 0.355  & 39.511  & 0.182  & 0.394  & 0.778  & 0.498  & 0.696  & 0.623  & 0.913  \\
    Geo-Teaser+ & \textcolor[rgb]{ .216,  .235,  .263}{7.116 } & \textcolor[rgb]{ .216,  .235,  .263}{16.657 } & 15.500  & 34.475  & \textcolor[rgb]{ .122,  .137,  .161}{10.122 } & \textcolor[rgb]{ .122,  .137,  .161}{23.532 } & 49.910  & 0.233  & 62.647  & 0.437  & 50.064  & 0.279  & 0.341  & 0.524  & \textcolor[rgb]{ .122,  .137,  .161}{0.483 } & \textcolor[rgb]{ .122,  .137,  .161}{0.669 } & 0.588  & 0.854  \\
    \textbf{Impr.} & \textbf{15.27\%} & \textbf{8.48\%} & \textbf{3.64\%} & \textbf{1.95\%} & \textbf{15.45\%} & \textbf{2.98\%} & \textbf{30.33\%} & \textbf{55.84\%} & \textbf{14.21\%} & \textbf{23.11\%} & \textbf{26.71\%} & \textbf{52.98\%} & \textbf{13.35\%} & \textbf{32.64\%} & \textbf{3.07\%} & \textbf{3.87\%} & \textbf{5.57\%} & \textbf{6.41\%} \\
    \midrule
    TALE  & 6.025  & 13.618  & 13.608  & 30.612  & 7.555  & 19.238  & 33.950  & 0.127  & 51.521  & 0.330  & 33.112  & 0.140  & 0.336  & 0.645  & 0.523  & 0.716  & 0.639  & 0.926  \\
    TALE+ & 6.690  & 15.208  & 14.940  & 33.223  & 8.694  & 20.342  & 50.689  & 0.240  & 63.380  & 0.448  & 47.719  & 0.263  & 0.320  & 0.482  & 0.510  & 0.701  & \textcolor[rgb]{ .122,  .137,  .161}{0.610 } & \textcolor[rgb]{ .122,  .137,  .161}{0.903 } \\
    \textbf{Impr.} & \textbf{11.04\%} & \textbf{11.67\%} & \textbf{9.79\%} & \textbf{8.53\%} & \textbf{15.08\%} & \textbf{5.74\%} & \textbf{49.3\%} & \textbf{88.82\%} & \textbf{23.02\%} & \textbf{35.82\%} & \textbf{44.11\%} & \textbf{87.29\%} & \textbf{4.76\%} & \textbf{25.17\%} & \textbf{2.49\%} & \textbf{2.08\%} & \textbf{4.56\%} & \textbf{2.49\%} \\
    \midrule
    Hier  & 6.982  & 15.631  & 15.120  & 32.091  & 9.181  & 22.174  & 37.436  & 0.143  & 50.189  & 0.316  & 41.269  & 0.196  & 0.361  & 0.584  & 0.536  & 0.733  & 0.634  & 1.000  \\
    Hier+ & 8.009  & 19.197  & 16.187  & 35.715  & \textcolor[rgb]{ .122,  .137,  .161}{10.592 } & \textcolor[rgb]{ .122,  .137,  .161}{24.079 } & 51.893  & 0.254  & 63.380  & 0.441  & 47.795  & 0.258  & 0.313  & 0.483  & \textcolor[rgb]{ .122,  .137,  .161}{0.510 } & \textcolor[rgb]{ .122,  .137,  .161}{0.719 } & 0.574  & 0.804  \\
    \textbf{Impr.} & \textbf{14.72\%} & \textbf{22.81\%} & \textbf{7.06\%} & \textbf{11.29\%} & \textbf{15.37\%} & \textbf{8.59\%} & \textbf{38.62\%} & \textbf{77.33\%} & \textbf{26.28\%} & \textbf{39.59\%} & \textbf{15.81\%} & \textbf{31.5\%} & \textbf{13.09\%} & \textbf{17.33\%} & \textbf{4.88\%} & \textbf{1.91\%} & \textbf{9.39\%} & \textbf{19.58\%} \\
    \midrule
    CTLE  & 6.653  & 14.594  & 14.859  & 31.852  & 8.625  & 20.218  & 40.103  & 0.181  & 55.030  & 0.369  & 41.805  & 0.206  & 0.337  & 0.566  & 0.515  & 0.703  & 0.697  & 1.061  \\
     CTLE+ & \textcolor[rgb]{ .216,  .235,  .263}{7.093 } & \textcolor[rgb]{ .216,  .235,  .263}{17.032 } & \textcolor[rgb]{ .216,  .235,  .263}{15.479 } & 34.138  & \textcolor[rgb]{ .122,  .137,  .161}{10.315 } & \textcolor[rgb]{ .122,  .137,  .161}{24.027 } & 50.430  & 0.234  & 61.848  & 0.434  & 51.440  & 0.287  & 0.291  & 0.456  & \textcolor[rgb]{ .122,  .137,  .161}{0.495 } & \textcolor[rgb]{ .122,  .137,  .161}{0.689 } & 0.610  & 0.892  \\
    \textbf{Impr.} & \textbf{6.61\%} & \textbf{16.71\%} & \textbf{4.18\%} & \textbf{7.18\%} & \textbf{19.59\%} & \textbf{18.84\%} & \textbf{25.75\%} & \textbf{29.59\%} & \textbf{12.39\%} & \textbf{17.36\%} & \textbf{23.05\%} & \textbf{39.62\%} & \textbf{13.65\%} & \textbf{19.44\%} & \textbf{4.01\%} & \textbf{2.02\%} & \textbf{12.4\%} & \textbf{15.92\%} \\
    \bottomrule
    \end{tabular}%
} 
\caption{The overall performance of downstream tasks and (+) means being enhanced by \name. Hit@1, Hit@5 and Acc are in \%, and F1 means macro-F1. For MAE and RMSE, lower is better, while for other metrics, higher is better.}  
\label{tab:main result}%
\end{table*}%    





\subsection{Overall Result Analysis}
The result of downstream tasks is presented in \tableautorefname ~\ref{tab:main result}, demonstrating that \name significantly improved the performance of all baselines across all datasets.
Skip-Gram and POI2Vec incorporate spatial information differently: Skip-Gram uses co-occurrence frequencies, while POI2Vec employs a geographic binary tree, both ignoring temporal features. Geo-Teaser includes spatial and temporal data with coarse granularity, while TALE, Hier, and CTLE integrate finer-grained spatiotemporal data. However, all six methods overlook POI semantic knowledge. Our framework addresses this gap, significantly enhancing performance. 

In three tasks, POI-Enhancer shows the most significant improvement in the Check-in Sequence Classification. This could be because the textual knowledge provided by POI-Enhancer is more beneficial for handling classification tasks. 
For the first task, POI2Vec achieves the greatest improvement on the New York dataset, with both metrics increasing by over 20\%. This is due to its focus on capturing check-in sequence patterns while neglecting other modalities. Our framework compensates for these limitations by enriching textual knowledge.
As for the second task, our findings indicate that Skip-Gram shows the weakest improvement, which is because it focuses on modeling representations from user trajectories and limits the potential for improvement.
In the last task, CTLE shows strong performance after enhancement. CTLE effectively captures contextual neighbors and temporal information in trajectories, and when combined with the textual vector extracted by POI-Enhancer, it greatly improves the performance in this tasks.

Besides, comparison experiments with LLM-based baselines reveal that, with the aid of \name, the POI representation method still holds a considerable advantage. This advantage stems from the fact that the POI representation method captures the fundamental spatial-temporal features, and when further enhanced with textual knowledge, it outperforms the text-centric LLM-based baselines. The results of this experiment are in the Supplementary Material.

\subsection{Further Analysis on \name}
\paratitle{Ablation Experiment} 
In this subsection, we conduct comprehensive experiments with four variant settings to evaluate the effectiveness of the components we design:

\bitem{POI-Enhancer/P} We remove the special prompt design including the role-playing, the attribute headers, and the question.
\bitem{POI-Enhancer/D} We removed the Dual Feature Alignment and Semantic Embedding Fusion. Instead, we generated a single prompt, which accumulates the content of the previous three kinds of prompts while maintaining the same format. The features extracted from this prompt by the LLM will be directly input into the Cross Attention Fusion.
\bitem{POI-Enhancer/F} We remove Cross Attention Fusion and concatenate the $E^{POI}$ and $E^{LLM}$ as the final vector instead.
\bitem{POI-Enhancer/C}  We only consider the spatial perspective. Specifically, given a POI, we define a square area centered around it to collect positive samples, with the parameters consistent with Geography Contrastive Learning.

We tested them on three downstream tasks using the New York dataset, with Hit@1, ACC, and MAE as evaluation metrics. 
 As shown in the \figureautorefname ~\ref{fig:ablation}, POI-Enhancer outperforms all variant settings and we can draw the following conclusions:
(1) The specialized prompts can enhance the framework's performance because they stimulate the LLM to extract spatial-temporal knowledge more efficiently.
(2) The Dual Feature Alignment and the Semantic Feature Fusion help obtain and maintain high-quality semantic vectors and improve the capabilities of the POI representation.
(3) The Cross Attention Fusion enables a more thorough integration, allowing the final vector to capture richer semantic information, resulting in improved performance.
(4) Compared to distanced-based positive samples, Multi-View Contrastive Learning selects richer samples from different perspectives, enhancing the ability of the embedding vectors.


\begin{figure}[t]
    \centering
    \includegraphics[width=\columnwidth]{Ablation.pdf}
    \caption{The result of ablation experiment. (A) is for POI Recommedation, (B) is for Check-in Sequence Classification and (C) is for POI Vistor Flow Prediction.}
    \label{fig:ablation}
\end{figure}



\paratitle{Parameters Analysis}
In this subsection, we study the effect of different $L_1$ and $L_2$ parameter settings in our framework. Specifically, we focus on enhancing the Hier model using the New York dataset, with POI recommendation as the downstream task. When evaluating the impact of one parameter, we keep the other parameters fixed at their optimal values.
As shown in the \figureautorefname ~\ref{fig:paprameter}, we can observe that for both $L_1$ and $L_2$, the performance initially improves with the increasing number of layers, reaches optimal performance, and then deteriorates. So, in other experiments, we set $L_1$ to 4 and $L_2$ to 2.
On the one hand, this indicates that when $L_1$ is too low, our alignment method fails to fully utilize the relational information between features, while an excessively high number of $L_1$ layers tends to cause over-fitting. On the other hand, this suggests that when $L_2$ is below the optimal value, our fusion method fails to effectively incorporate the knowledge from LLM into the original representations. However, when $L_2$ exceeds a certain threshold, the semantic knowledge will overshadow the original vectors.
\begin{figure}[t]
    \centering
    \includegraphics[width=\columnwidth]{parameter.pdf}
    \caption{The effect of $L_1$ and $L_2$.}\label{fig:paprameter} 
\end{figure}

\paratitle{Quality Analysis}
To further evaluate the quality of the enhanced vectors produced by \name, we conducted clustering tasks using the K-means algorithm on three datasets. We applied this algorithm to all types of representation vectors, both before and after enhancement. The number of clusters was set to match the number of POI categories in each dataset. We then assessed the clustering performance using the Normalized Mutual Information (NMI) metric. 
The results depicted in the \figureautorefname ~\ref{fig:cluster} demonstrate the effectiveness of our framework, as all evaluation metrics for the representation vectors across the three datasets have shown significant improvement. This indicates that:
(1) We successfully extracted high-quality textual features, and the rich textual information helps similar representation vectors to cluster more closely together.
(2) We effectively integrated textual information into the initial representations, further enhancing the quality of the original vectors.
(3) The Multi-View Contrastive Learning approach encouraged vectors of the same class to be closer together while pushing vectors of different classes further apart.
\begin{figure}[t]
    \centering
    \includegraphics[width=\columnwidth]{cluster_result.pdf}
    \caption{The result of POI cluster task.}\label{fig:cluster} 
\end{figure}



%%%555555555_relatedwork_555555555555%
\section{Related Work}\label{sec:related}
\paratitle{LLMs in Spatial-temporal Tasks}
Considerable efforts have been dedicated to using LLMs to improve the performance of spatial-temporal tasks~\cite{yu2024bigcity}. For instance, GeoGPT~\cite{geogpt} introduced an LLM-based tool capable of automating the processing of geographic data, but it does not delve into extracting detailed information about locations. GEOLLM~\cite{geollm} designed prompts that include coordinates, address, and surrounding buildings, but it can only address simple questions in a Q\&A format and are unable to handle complex tasks like POI recommendation. Besides, they fail to fully extract the semantic information of POIs.
Some researchers have used LLMs as backbones to tackle complex real-world tasks. For example, GATGPT~\cite{gatgpt} input spatial-temporal features into a frozen LLM to predict traffic speeds, while ST-LLM~\cite{stllm} used a partially frozen LLM to forecast traffic flow. However, these methods are designed for specific individual problems and cannot be applied across multiple tasks.
To solve these limitations, we designed three types of special prompts to extract the semantic information of POIs from LLMs effectively.

\paratitle{POI Representation with Semantic Information}
POI representation aims to turn each POI into a vector that can be utilized in various downstream tasks like traffic forecasting tasks~\cite{MLPST, PDFormer,stssl} and trajectory tasks~\cite{controltraj, START, gan}. Most existing methods like ~\cite{ppr}, leverage textual features typically using one-hot code to encode POI categories and then concatenate them with the embedding vectors. For data types like check-in content,~\cite{ge} model the similarity between POIs by constructing a POI-Word relationship graph, while ~\cite{cape} draws inspiration from Word2Vec method, simultaneously training word vectors and POI vectors. However, these methods often fall short in preserving semantic information and achieving a more comprehensive integration during the fusion process.
To address this issue, we designed three modules in the Embedding Enhancement to improve the preservation and integration of semantic information in the POI embedding.

%%%6666666666_conclusion_66666666666%%%%
\section{Conclusion and Future Work}\label{sec:conclusion}
We propose a framework called POI-Enhancer, which enhances all POI representation methods by leveraging the LLM. 
To introduce textual information into POI embeddings, we designed three special prompts to extract features from the LLM. 
To use the links between address features and other features, we introduced Dual Feature Alignment and Semantic Feature Fusion, which help obtain and preserve high-quality textual features. 
To better integrate the extracted knowledge into POI representations, we further developed the Cross Attention Fusion. 
Lastly, to enhance the representation capabilities of the vectors, we proposed Multi-View Contrastive Learning, using three strategies to sample positive and negative examples. 
The experiment results demonstrate that our framework significantly improves the performance of POI representation vectors across various downstream tasks in three real-world datasets.





%%%7777777777777_acknowledgements_777777777777%%%
\clearpage
\section*{Acknowledgments}
Prof. Jingyuan Wang's work was partially supported by the National Natural Science Foundation of China (No. 72171013, 72222022, 72242101), the Special Fund for Health Development Research of Beijing (2024-2G-30121) and State Key Laboratory of Complex \& Critical Software Environment (SKLSDE-2023ZX-04).
Prof. Xiangyu Zhao's work was partially supported by Research Impact Fund (No.R1015-23), APRC - CityU New Research Initiatives (No.9610565, Start-up Grant for New Faculty of CityU), CityU - HKIDS Early Career Research Grant (No.9360163), Hong Kong ITC Innovation and Technology Fund Midstream Research Programme for Universities Project (No.ITS/034/22MS), Hong Kong Environmental and Conservation Fund (No. 88/2022), and SIRG - CityU Strategic Interdisciplinary Research Grant (No.7020046), Huawei (Huawei Innovation Research Program), Tencent (CCF-Tencent Open Fund, Tencent Rhino-Bird Focused Research Program), Ant Group (CCF-Ant Research Fund, Ant Group Research Fund), Alibaba (CCF-Alimama Tech Kangaroo Fund No. 2024002), CCF-BaiChuan-Ebtech Foundation Model Fund, and Kuaishou.

%%%888888888888_acknowledgements_888888888888%%%
\bibliography{aaai25}

%%%99999999999_apendix_999999999999999%w
\subsection{Lloyd-Max Algorithm}
\label{subsec:Lloyd-Max}
For a given quantization bitwidth $B$ and an operand $\bm{X}$, the Lloyd-Max algorithm finds $2^B$ quantization levels $\{\hat{x}_i\}_{i=1}^{2^B}$ such that quantizing $\bm{X}$ by rounding each scalar in $\bm{X}$ to the nearest quantization level minimizes the quantization MSE. 

The algorithm starts with an initial guess of quantization levels and then iteratively computes quantization thresholds $\{\tau_i\}_{i=1}^{2^B-1}$ and updates quantization levels $\{\hat{x}_i\}_{i=1}^{2^B}$. Specifically, at iteration $n$, thresholds are set to the midpoints of the previous iteration's levels:
\begin{align*}
    \tau_i^{(n)}=\frac{\hat{x}_i^{(n-1)}+\hat{x}_{i+1}^{(n-1)}}2 \text{ for } i=1\ldots 2^B-1
\end{align*}
Subsequently, the quantization levels are re-computed as conditional means of the data regions defined by the new thresholds:
\begin{align*}
    \hat{x}_i^{(n)}=\mathbb{E}\left[ \bm{X} \big| \bm{X}\in [\tau_{i-1}^{(n)},\tau_i^{(n)}] \right] \text{ for } i=1\ldots 2^B
\end{align*}
where to satisfy boundary conditions we have $\tau_0=-\infty$ and $\tau_{2^B}=\infty$. The algorithm iterates the above steps until convergence.

Figure \ref{fig:lm_quant} compares the quantization levels of a $7$-bit floating point (E3M3) quantizer (left) to a $7$-bit Lloyd-Max quantizer (right) when quantizing a layer of weights from the GPT3-126M model at a per-tensor granularity. As shown, the Lloyd-Max quantizer achieves substantially lower quantization MSE. Further, Table \ref{tab:FP7_vs_LM7} shows the superior perplexity achieved by Lloyd-Max quantizers for bitwidths of $7$, $6$ and $5$. The difference between the quantizers is clear at 5 bits, where per-tensor FP quantization incurs a drastic and unacceptable increase in perplexity, while Lloyd-Max quantization incurs a much smaller increase. Nevertheless, we note that even the optimal Lloyd-Max quantizer incurs a notable ($\sim 1.5$) increase in perplexity due to the coarse granularity of quantization. 

\begin{figure}[h]
  \centering
  \includegraphics[width=0.7\linewidth]{sections/figures/LM7_FP7.pdf}
  \caption{\small Quantization levels and the corresponding quantization MSE of Floating Point (left) vs Lloyd-Max (right) Quantizers for a layer of weights in the GPT3-126M model.}
  \label{fig:lm_quant}
\end{figure}

\begin{table}[h]\scriptsize
\begin{center}
\caption{\label{tab:FP7_vs_LM7} \small Comparing perplexity (lower is better) achieved by floating point quantizers and Lloyd-Max quantizers on a GPT3-126M model for the Wikitext-103 dataset.}
\begin{tabular}{c|cc|c}
\hline
 \multirow{2}{*}{\textbf{Bitwidth}} & \multicolumn{2}{|c|}{\textbf{Floating-Point Quantizer}} & \textbf{Lloyd-Max Quantizer} \\
 & Best Format & Wikitext-103 Perplexity & Wikitext-103 Perplexity \\
\hline
7 & E3M3 & 18.32 & 18.27 \\
6 & E3M2 & 19.07 & 18.51 \\
5 & E4M0 & 43.89 & 19.71 \\
\hline
\end{tabular}
\end{center}
\end{table}

\subsection{Proof of Local Optimality of LO-BCQ}
\label{subsec:lobcq_opt_proof}
For a given block $\bm{b}_j$, the quantization MSE during LO-BCQ can be empirically evaluated as $\frac{1}{L_b}\lVert \bm{b}_j- \bm{\hat{b}}_j\rVert^2_2$ where $\bm{\hat{b}}_j$ is computed from equation (\ref{eq:clustered_quantization_definition}) as $C_{f(\bm{b}_j)}(\bm{b}_j)$. Further, for a given block cluster $\mathcal{B}_i$, we compute the quantization MSE as $\frac{1}{|\mathcal{B}_{i}|}\sum_{\bm{b} \in \mathcal{B}_{i}} \frac{1}{L_b}\lVert \bm{b}- C_i^{(n)}(\bm{b})\rVert^2_2$. Therefore, at the end of iteration $n$, we evaluate the overall quantization MSE $J^{(n)}$ for a given operand $\bm{X}$ composed of $N_c$ block clusters as:
\begin{align*}
    \label{eq:mse_iter_n}
    J^{(n)} = \frac{1}{N_c} \sum_{i=1}^{N_c} \frac{1}{|\mathcal{B}_{i}^{(n)}|}\sum_{\bm{v} \in \mathcal{B}_{i}^{(n)}} \frac{1}{L_b}\lVert \bm{b}- B_i^{(n)}(\bm{b})\rVert^2_2
\end{align*}

At the end of iteration $n$, the codebooks are updated from $\mathcal{C}^{(n-1)}$ to $\mathcal{C}^{(n)}$. However, the mapping of a given vector $\bm{b}_j$ to quantizers $\mathcal{C}^{(n)}$ remains as  $f^{(n)}(\bm{b}_j)$. At the next iteration, during the vector clustering step, $f^{(n+1)}(\bm{b}_j)$ finds new mapping of $\bm{b}_j$ to updated codebooks $\mathcal{C}^{(n)}$ such that the quantization MSE over the candidate codebooks is minimized. Therefore, we obtain the following result for $\bm{b}_j$:
\begin{align*}
\frac{1}{L_b}\lVert \bm{b}_j - C_{f^{(n+1)}(\bm{b}_j)}^{(n)}(\bm{b}_j)\rVert^2_2 \le \frac{1}{L_b}\lVert \bm{b}_j - C_{f^{(n)}(\bm{b}_j)}^{(n)}(\bm{b}_j)\rVert^2_2
\end{align*}

That is, quantizing $\bm{b}_j$ at the end of the block clustering step of iteration $n+1$ results in lower quantization MSE compared to quantizing at the end of iteration $n$. Since this is true for all $\bm{b} \in \bm{X}$, we assert the following:
\begin{equation}
\begin{split}
\label{eq:mse_ineq_1}
    \tilde{J}^{(n+1)} &= \frac{1}{N_c} \sum_{i=1}^{N_c} \frac{1}{|\mathcal{B}_{i}^{(n+1)}|}\sum_{\bm{b} \in \mathcal{B}_{i}^{(n+1)}} \frac{1}{L_b}\lVert \bm{b} - C_i^{(n)}(b)\rVert^2_2 \le J^{(n)}
\end{split}
\end{equation}
where $\tilde{J}^{(n+1)}$ is the the quantization MSE after the vector clustering step at iteration $n+1$.

Next, during the codebook update step (\ref{eq:quantizers_update}) at iteration $n+1$, the per-cluster codebooks $\mathcal{C}^{(n)}$ are updated to $\mathcal{C}^{(n+1)}$ by invoking the Lloyd-Max algorithm \citep{Lloyd}. We know that for any given value distribution, the Lloyd-Max algorithm minimizes the quantization MSE. Therefore, for a given vector cluster $\mathcal{B}_i$ we obtain the following result:

\begin{equation}
    \frac{1}{|\mathcal{B}_{i}^{(n+1)}|}\sum_{\bm{b} \in \mathcal{B}_{i}^{(n+1)}} \frac{1}{L_b}\lVert \bm{b}- C_i^{(n+1)}(\bm{b})\rVert^2_2 \le \frac{1}{|\mathcal{B}_{i}^{(n+1)}|}\sum_{\bm{b} \in \mathcal{B}_{i}^{(n+1)}} \frac{1}{L_b}\lVert \bm{b}- C_i^{(n)}(\bm{b})\rVert^2_2
\end{equation}

The above equation states that quantizing the given block cluster $\mathcal{B}_i$ after updating the associated codebook from $C_i^{(n)}$ to $C_i^{(n+1)}$ results in lower quantization MSE. Since this is true for all the block clusters, we derive the following result: 
\begin{equation}
\begin{split}
\label{eq:mse_ineq_2}
     J^{(n+1)} &= \frac{1}{N_c} \sum_{i=1}^{N_c} \frac{1}{|\mathcal{B}_{i}^{(n+1)}|}\sum_{\bm{b} \in \mathcal{B}_{i}^{(n+1)}} \frac{1}{L_b}\lVert \bm{b}- C_i^{(n+1)}(\bm{b})\rVert^2_2  \le \tilde{J}^{(n+1)}   
\end{split}
\end{equation}

Following (\ref{eq:mse_ineq_1}) and (\ref{eq:mse_ineq_2}), we find that the quantization MSE is non-increasing for each iteration, that is, $J^{(1)} \ge J^{(2)} \ge J^{(3)} \ge \ldots \ge J^{(M)}$ where $M$ is the maximum number of iterations. 
%Therefore, we can say that if the algorithm converges, then it must be that it has converged to a local minimum. 
\hfill $\blacksquare$


\begin{figure}
    \begin{center}
    \includegraphics[width=0.5\textwidth]{sections//figures/mse_vs_iter.pdf}
    \end{center}
    \caption{\small NMSE vs iterations during LO-BCQ compared to other block quantization proposals}
    \label{fig:nmse_vs_iter}
\end{figure}

Figure \ref{fig:nmse_vs_iter} shows the empirical convergence of LO-BCQ across several block lengths and number of codebooks. Also, the MSE achieved by LO-BCQ is compared to baselines such as MXFP and VSQ. As shown, LO-BCQ converges to a lower MSE than the baselines. Further, we achieve better convergence for larger number of codebooks ($N_c$) and for a smaller block length ($L_b$), both of which increase the bitwidth of BCQ (see Eq \ref{eq:bitwidth_bcq}).


\subsection{Additional Accuracy Results}
%Table \ref{tab:lobcq_config} lists the various LOBCQ configurations and their corresponding bitwidths.
\begin{table}
\setlength{\tabcolsep}{4.75pt}
\begin{center}
\caption{\label{tab:lobcq_config} Various LO-BCQ configurations and their bitwidths.}
\begin{tabular}{|c||c|c|c|c||c|c||c|} 
\hline
 & \multicolumn{4}{|c||}{$L_b=8$} & \multicolumn{2}{|c||}{$L_b=4$} & $L_b=2$ \\
 \hline
 \backslashbox{$L_A$\kern-1em}{\kern-1em$N_c$} & 2 & 4 & 8 & 16 & 2 & 4 & 2 \\
 \hline
 64 & 4.25 & 4.375 & 4.5 & 4.625 & 4.375 & 4.625 & 4.625\\
 \hline
 32 & 4.375 & 4.5 & 4.625& 4.75 & 4.5 & 4.75 & 4.75 \\
 \hline
 16 & 4.625 & 4.75& 4.875 & 5 & 4.75 & 5 & 5 \\
 \hline
\end{tabular}
\end{center}
\end{table}

%\subsection{Perplexity achieved by various LO-BCQ configurations on Wikitext-103 dataset}

\begin{table} \centering
\begin{tabular}{|c||c|c|c|c||c|c||c|} 
\hline
 $L_b \rightarrow$& \multicolumn{4}{c||}{8} & \multicolumn{2}{c||}{4} & 2\\
 \hline
 \backslashbox{$L_A$\kern-1em}{\kern-1em$N_c$} & 2 & 4 & 8 & 16 & 2 & 4 & 2  \\
 %$N_c \rightarrow$ & 2 & 4 & 8 & 16 & 2 & 4 & 2 \\
 \hline
 \hline
 \multicolumn{8}{c}{GPT3-1.3B (FP32 PPL = 9.98)} \\ 
 \hline
 \hline
 64 & 10.40 & 10.23 & 10.17 & 10.15 &  10.28 & 10.18 & 10.19 \\
 \hline
 32 & 10.25 & 10.20 & 10.15 & 10.12 &  10.23 & 10.17 & 10.17 \\
 \hline
 16 & 10.22 & 10.16 & 10.10 & 10.09 &  10.21 & 10.14 & 10.16 \\
 \hline
  \hline
 \multicolumn{8}{c}{GPT3-8B (FP32 PPL = 7.38)} \\ 
 \hline
 \hline
 64 & 7.61 & 7.52 & 7.48 &  7.47 &  7.55 &  7.49 & 7.50 \\
 \hline
 32 & 7.52 & 7.50 & 7.46 &  7.45 &  7.52 &  7.48 & 7.48  \\
 \hline
 16 & 7.51 & 7.48 & 7.44 &  7.44 &  7.51 &  7.49 & 7.47  \\
 \hline
\end{tabular}
\caption{\label{tab:ppl_gpt3_abalation} Wikitext-103 perplexity across GPT3-1.3B and 8B models.}
\end{table}

\begin{table} \centering
\begin{tabular}{|c||c|c|c|c||} 
\hline
 $L_b \rightarrow$& \multicolumn{4}{c||}{8}\\
 \hline
 \backslashbox{$L_A$\kern-1em}{\kern-1em$N_c$} & 2 & 4 & 8 & 16 \\
 %$N_c \rightarrow$ & 2 & 4 & 8 & 16 & 2 & 4 & 2 \\
 \hline
 \hline
 \multicolumn{5}{|c|}{Llama2-7B (FP32 PPL = 5.06)} \\ 
 \hline
 \hline
 64 & 5.31 & 5.26 & 5.19 & 5.18  \\
 \hline
 32 & 5.23 & 5.25 & 5.18 & 5.15  \\
 \hline
 16 & 5.23 & 5.19 & 5.16 & 5.14  \\
 \hline
 \multicolumn{5}{|c|}{Nemotron4-15B (FP32 PPL = 5.87)} \\ 
 \hline
 \hline
 64  & 6.3 & 6.20 & 6.13 & 6.08  \\
 \hline
 32  & 6.24 & 6.12 & 6.07 & 6.03  \\
 \hline
 16  & 6.12 & 6.14 & 6.04 & 6.02  \\
 \hline
 \multicolumn{5}{|c|}{Nemotron4-340B (FP32 PPL = 3.48)} \\ 
 \hline
 \hline
 64 & 3.67 & 3.62 & 3.60 & 3.59 \\
 \hline
 32 & 3.63 & 3.61 & 3.59 & 3.56 \\
 \hline
 16 & 3.61 & 3.58 & 3.57 & 3.55 \\
 \hline
\end{tabular}
\caption{\label{tab:ppl_llama7B_nemo15B} Wikitext-103 perplexity compared to FP32 baseline in Llama2-7B and Nemotron4-15B, 340B models}
\end{table}

%\subsection{Perplexity achieved by various LO-BCQ configurations on MMLU dataset}


\begin{table} \centering
\begin{tabular}{|c||c|c|c|c||c|c|c|c|} 
\hline
 $L_b \rightarrow$& \multicolumn{4}{c||}{8} & \multicolumn{4}{c||}{8}\\
 \hline
 \backslashbox{$L_A$\kern-1em}{\kern-1em$N_c$} & 2 & 4 & 8 & 16 & 2 & 4 & 8 & 16  \\
 %$N_c \rightarrow$ & 2 & 4 & 8 & 16 & 2 & 4 & 2 \\
 \hline
 \hline
 \multicolumn{5}{|c|}{Llama2-7B (FP32 Accuracy = 45.8\%)} & \multicolumn{4}{|c|}{Llama2-70B (FP32 Accuracy = 69.12\%)} \\ 
 \hline
 \hline
 64 & 43.9 & 43.4 & 43.9 & 44.9 & 68.07 & 68.27 & 68.17 & 68.75 \\
 \hline
 32 & 44.5 & 43.8 & 44.9 & 44.5 & 68.37 & 68.51 & 68.35 & 68.27  \\
 \hline
 16 & 43.9 & 42.7 & 44.9 & 45 & 68.12 & 68.77 & 68.31 & 68.59  \\
 \hline
 \hline
 \multicolumn{5}{|c|}{GPT3-22B (FP32 Accuracy = 38.75\%)} & \multicolumn{4}{|c|}{Nemotron4-15B (FP32 Accuracy = 64.3\%)} \\ 
 \hline
 \hline
 64 & 36.71 & 38.85 & 38.13 & 38.92 & 63.17 & 62.36 & 63.72 & 64.09 \\
 \hline
 32 & 37.95 & 38.69 & 39.45 & 38.34 & 64.05 & 62.30 & 63.8 & 64.33  \\
 \hline
 16 & 38.88 & 38.80 & 38.31 & 38.92 & 63.22 & 63.51 & 63.93 & 64.43  \\
 \hline
\end{tabular}
\caption{\label{tab:mmlu_abalation} Accuracy on MMLU dataset across GPT3-22B, Llama2-7B, 70B and Nemotron4-15B models.}
\end{table}


%\subsection{Perplexity achieved by various LO-BCQ configurations on LM evaluation harness}

\begin{table} \centering
\begin{tabular}{|c||c|c|c|c||c|c|c|c|} 
\hline
 $L_b \rightarrow$& \multicolumn{4}{c||}{8} & \multicolumn{4}{c||}{8}\\
 \hline
 \backslashbox{$L_A$\kern-1em}{\kern-1em$N_c$} & 2 & 4 & 8 & 16 & 2 & 4 & 8 & 16  \\
 %$N_c \rightarrow$ & 2 & 4 & 8 & 16 & 2 & 4 & 2 \\
 \hline
 \hline
 \multicolumn{5}{|c|}{Race (FP32 Accuracy = 37.51\%)} & \multicolumn{4}{|c|}{Boolq (FP32 Accuracy = 64.62\%)} \\ 
 \hline
 \hline
 64 & 36.94 & 37.13 & 36.27 & 37.13 & 63.73 & 62.26 & 63.49 & 63.36 \\
 \hline
 32 & 37.03 & 36.36 & 36.08 & 37.03 & 62.54 & 63.51 & 63.49 & 63.55  \\
 \hline
 16 & 37.03 & 37.03 & 36.46 & 37.03 & 61.1 & 63.79 & 63.58 & 63.33  \\
 \hline
 \hline
 \multicolumn{5}{|c|}{Winogrande (FP32 Accuracy = 58.01\%)} & \multicolumn{4}{|c|}{Piqa (FP32 Accuracy = 74.21\%)} \\ 
 \hline
 \hline
 64 & 58.17 & 57.22 & 57.85 & 58.33 & 73.01 & 73.07 & 73.07 & 72.80 \\
 \hline
 32 & 59.12 & 58.09 & 57.85 & 58.41 & 73.01 & 73.94 & 72.74 & 73.18  \\
 \hline
 16 & 57.93 & 58.88 & 57.93 & 58.56 & 73.94 & 72.80 & 73.01 & 73.94  \\
 \hline
\end{tabular}
\caption{\label{tab:mmlu_abalation} Accuracy on LM evaluation harness tasks on GPT3-1.3B model.}
\end{table}

\begin{table} \centering
\begin{tabular}{|c||c|c|c|c||c|c|c|c|} 
\hline
 $L_b \rightarrow$& \multicolumn{4}{c||}{8} & \multicolumn{4}{c||}{8}\\
 \hline
 \backslashbox{$L_A$\kern-1em}{\kern-1em$N_c$} & 2 & 4 & 8 & 16 & 2 & 4 & 8 & 16  \\
 %$N_c \rightarrow$ & 2 & 4 & 8 & 16 & 2 & 4 & 2 \\
 \hline
 \hline
 \multicolumn{5}{|c|}{Race (FP32 Accuracy = 41.34\%)} & \multicolumn{4}{|c|}{Boolq (FP32 Accuracy = 68.32\%)} \\ 
 \hline
 \hline
 64 & 40.48 & 40.10 & 39.43 & 39.90 & 69.20 & 68.41 & 69.45 & 68.56 \\
 \hline
 32 & 39.52 & 39.52 & 40.77 & 39.62 & 68.32 & 67.43 & 68.17 & 69.30  \\
 \hline
 16 & 39.81 & 39.71 & 39.90 & 40.38 & 68.10 & 66.33 & 69.51 & 69.42  \\
 \hline
 \hline
 \multicolumn{5}{|c|}{Winogrande (FP32 Accuracy = 67.88\%)} & \multicolumn{4}{|c|}{Piqa (FP32 Accuracy = 78.78\%)} \\ 
 \hline
 \hline
 64 & 66.85 & 66.61 & 67.72 & 67.88 & 77.31 & 77.42 & 77.75 & 77.64 \\
 \hline
 32 & 67.25 & 67.72 & 67.72 & 67.00 & 77.31 & 77.04 & 77.80 & 77.37  \\
 \hline
 16 & 68.11 & 68.90 & 67.88 & 67.48 & 77.37 & 78.13 & 78.13 & 77.69  \\
 \hline
\end{tabular}
\caption{\label{tab:mmlu_abalation} Accuracy on LM evaluation harness tasks on GPT3-8B model.}
\end{table}

\begin{table} \centering
\begin{tabular}{|c||c|c|c|c||c|c|c|c|} 
\hline
 $L_b \rightarrow$& \multicolumn{4}{c||}{8} & \multicolumn{4}{c||}{8}\\
 \hline
 \backslashbox{$L_A$\kern-1em}{\kern-1em$N_c$} & 2 & 4 & 8 & 16 & 2 & 4 & 8 & 16  \\
 %$N_c \rightarrow$ & 2 & 4 & 8 & 16 & 2 & 4 & 2 \\
 \hline
 \hline
 \multicolumn{5}{|c|}{Race (FP32 Accuracy = 40.67\%)} & \multicolumn{4}{|c|}{Boolq (FP32 Accuracy = 76.54\%)} \\ 
 \hline
 \hline
 64 & 40.48 & 40.10 & 39.43 & 39.90 & 75.41 & 75.11 & 77.09 & 75.66 \\
 \hline
 32 & 39.52 & 39.52 & 40.77 & 39.62 & 76.02 & 76.02 & 75.96 & 75.35  \\
 \hline
 16 & 39.81 & 39.71 & 39.90 & 40.38 & 75.05 & 73.82 & 75.72 & 76.09  \\
 \hline
 \hline
 \multicolumn{5}{|c|}{Winogrande (FP32 Accuracy = 70.64\%)} & \multicolumn{4}{|c|}{Piqa (FP32 Accuracy = 79.16\%)} \\ 
 \hline
 \hline
 64 & 69.14 & 70.17 & 70.17 & 70.56 & 78.24 & 79.00 & 78.62 & 78.73 \\
 \hline
 32 & 70.96 & 69.69 & 71.27 & 69.30 & 78.56 & 79.49 & 79.16 & 78.89  \\
 \hline
 16 & 71.03 & 69.53 & 69.69 & 70.40 & 78.13 & 79.16 & 79.00 & 79.00  \\
 \hline
\end{tabular}
\caption{\label{tab:mmlu_abalation} Accuracy on LM evaluation harness tasks on GPT3-22B model.}
\end{table}

\begin{table} \centering
\begin{tabular}{|c||c|c|c|c||c|c|c|c|} 
\hline
 $L_b \rightarrow$& \multicolumn{4}{c||}{8} & \multicolumn{4}{c||}{8}\\
 \hline
 \backslashbox{$L_A$\kern-1em}{\kern-1em$N_c$} & 2 & 4 & 8 & 16 & 2 & 4 & 8 & 16  \\
 %$N_c \rightarrow$ & 2 & 4 & 8 & 16 & 2 & 4 & 2 \\
 \hline
 \hline
 \multicolumn{5}{|c|}{Race (FP32 Accuracy = 44.4\%)} & \multicolumn{4}{|c|}{Boolq (FP32 Accuracy = 79.29\%)} \\ 
 \hline
 \hline
 64 & 42.49 & 42.51 & 42.58 & 43.45 & 77.58 & 77.37 & 77.43 & 78.1 \\
 \hline
 32 & 43.35 & 42.49 & 43.64 & 43.73 & 77.86 & 75.32 & 77.28 & 77.86  \\
 \hline
 16 & 44.21 & 44.21 & 43.64 & 42.97 & 78.65 & 77 & 76.94 & 77.98  \\
 \hline
 \hline
 \multicolumn{5}{|c|}{Winogrande (FP32 Accuracy = 69.38\%)} & \multicolumn{4}{|c|}{Piqa (FP32 Accuracy = 78.07\%)} \\ 
 \hline
 \hline
 64 & 68.9 & 68.43 & 69.77 & 68.19 & 77.09 & 76.82 & 77.09 & 77.86 \\
 \hline
 32 & 69.38 & 68.51 & 68.82 & 68.90 & 78.07 & 76.71 & 78.07 & 77.86  \\
 \hline
 16 & 69.53 & 67.09 & 69.38 & 68.90 & 77.37 & 77.8 & 77.91 & 77.69  \\
 \hline
\end{tabular}
\caption{\label{tab:mmlu_abalation} Accuracy on LM evaluation harness tasks on Llama2-7B model.}
\end{table}

\begin{table} \centering
\begin{tabular}{|c||c|c|c|c||c|c|c|c|} 
\hline
 $L_b \rightarrow$& \multicolumn{4}{c||}{8} & \multicolumn{4}{c||}{8}\\
 \hline
 \backslashbox{$L_A$\kern-1em}{\kern-1em$N_c$} & 2 & 4 & 8 & 16 & 2 & 4 & 8 & 16  \\
 %$N_c \rightarrow$ & 2 & 4 & 8 & 16 & 2 & 4 & 2 \\
 \hline
 \hline
 \multicolumn{5}{|c|}{Race (FP32 Accuracy = 48.8\%)} & \multicolumn{4}{|c|}{Boolq (FP32 Accuracy = 85.23\%)} \\ 
 \hline
 \hline
 64 & 49.00 & 49.00 & 49.28 & 48.71 & 82.82 & 84.28 & 84.03 & 84.25 \\
 \hline
 32 & 49.57 & 48.52 & 48.33 & 49.28 & 83.85 & 84.46 & 84.31 & 84.93  \\
 \hline
 16 & 49.85 & 49.09 & 49.28 & 48.99 & 85.11 & 84.46 & 84.61 & 83.94  \\
 \hline
 \hline
 \multicolumn{5}{|c|}{Winogrande (FP32 Accuracy = 79.95\%)} & \multicolumn{4}{|c|}{Piqa (FP32 Accuracy = 81.56\%)} \\ 
 \hline
 \hline
 64 & 78.77 & 78.45 & 78.37 & 79.16 & 81.45 & 80.69 & 81.45 & 81.5 \\
 \hline
 32 & 78.45 & 79.01 & 78.69 & 80.66 & 81.56 & 80.58 & 81.18 & 81.34  \\
 \hline
 16 & 79.95 & 79.56 & 79.79 & 79.72 & 81.28 & 81.66 & 81.28 & 80.96  \\
 \hline
\end{tabular}
\caption{\label{tab:mmlu_abalation} Accuracy on LM evaluation harness tasks on Llama2-70B model.}
\end{table}

%\section{MSE Studies}
%\textcolor{red}{TODO}


\subsection{Number Formats and Quantization Method}
\label{subsec:numFormats_quantMethod}
\subsubsection{Integer Format}
An $n$-bit signed integer (INT) is typically represented with a 2s-complement format \citep{yao2022zeroquant,xiao2023smoothquant,dai2021vsq}, where the most significant bit denotes the sign.

\subsubsection{Floating Point Format}
An $n$-bit signed floating point (FP) number $x$ comprises of a 1-bit sign ($x_{\mathrm{sign}}$), $B_m$-bit mantissa ($x_{\mathrm{mant}}$) and $B_e$-bit exponent ($x_{\mathrm{exp}}$) such that $B_m+B_e=n-1$. The associated constant exponent bias ($E_{\mathrm{bias}}$) is computed as $(2^{{B_e}-1}-1)$. We denote this format as $E_{B_e}M_{B_m}$.  

\subsubsection{Quantization Scheme}
\label{subsec:quant_method}
A quantization scheme dictates how a given unquantized tensor is converted to its quantized representation. We consider FP formats for the purpose of illustration. Given an unquantized tensor $\bm{X}$ and an FP format $E_{B_e}M_{B_m}$, we first, we compute the quantization scale factor $s_X$ that maps the maximum absolute value of $\bm{X}$ to the maximum quantization level of the $E_{B_e}M_{B_m}$ format as follows:
\begin{align}
\label{eq:sf}
    s_X = \frac{\mathrm{max}(|\bm{X}|)}{\mathrm{max}(E_{B_e}M_{B_m})}
\end{align}
In the above equation, $|\cdot|$ denotes the absolute value function.

Next, we scale $\bm{X}$ by $s_X$ and quantize it to $\hat{\bm{X}}$ by rounding it to the nearest quantization level of $E_{B_e}M_{B_m}$ as:

\begin{align}
\label{eq:tensor_quant}
    \hat{\bm{X}} = \text{round-to-nearest}\left(\frac{\bm{X}}{s_X}, E_{B_e}M_{B_m}\right)
\end{align}

We perform dynamic max-scaled quantization \citep{wu2020integer}, where the scale factor $s$ for activations is dynamically computed during runtime.

\subsection{Vector Scaled Quantization}
\begin{wrapfigure}{r}{0.35\linewidth}
  \centering
  \includegraphics[width=\linewidth]{sections/figures/vsquant.jpg}
  \caption{\small Vectorwise decomposition for per-vector scaled quantization (VSQ \citep{dai2021vsq}).}
  \label{fig:vsquant}
\end{wrapfigure}
During VSQ \citep{dai2021vsq}, the operand tensors are decomposed into 1D vectors in a hardware friendly manner as shown in Figure \ref{fig:vsquant}. Since the decomposed tensors are used as operands in matrix multiplications during inference, it is beneficial to perform this decomposition along the reduction dimension of the multiplication. The vectorwise quantization is performed similar to tensorwise quantization described in Equations \ref{eq:sf} and \ref{eq:tensor_quant}, where a scale factor $s_v$ is required for each vector $\bm{v}$ that maps the maximum absolute value of that vector to the maximum quantization level. While smaller vector lengths can lead to larger accuracy gains, the associated memory and computational overheads due to the per-vector scale factors increases. To alleviate these overheads, VSQ \citep{dai2021vsq} proposed a second level quantization of the per-vector scale factors to unsigned integers, while MX \citep{rouhani2023shared} quantizes them to integer powers of 2 (denoted as $2^{INT}$).

\subsubsection{MX Format}
The MX format proposed in \citep{rouhani2023microscaling} introduces the concept of sub-block shifting. For every two scalar elements of $b$-bits each, there is a shared exponent bit. The value of this exponent bit is determined through an empirical analysis that targets minimizing quantization MSE. We note that the FP format $E_{1}M_{b}$ is strictly better than MX from an accuracy perspective since it allocates a dedicated exponent bit to each scalar as opposed to sharing it across two scalars. Therefore, we conservatively bound the accuracy of a $b+2$-bit signed MX format with that of a $E_{1}M_{b}$ format in our comparisons. For instance, we use E1M2 format as a proxy for MX4.

\begin{figure}
    \centering
    \includegraphics[width=1\linewidth]{sections//figures/BlockFormats.pdf}
    \caption{\small Comparing LO-BCQ to MX format.}
    \label{fig:block_formats}
\end{figure}

Figure \ref{fig:block_formats} compares our $4$-bit LO-BCQ block format to MX \citep{rouhani2023microscaling}. As shown, both LO-BCQ and MX decompose a given operand tensor into block arrays and each block array into blocks. Similar to MX, we find that per-block quantization ($L_b < L_A$) leads to better accuracy due to increased flexibility. While MX achieves this through per-block $1$-bit micro-scales, we associate a dedicated codebook to each block through a per-block codebook selector. Further, MX quantizes the per-block array scale-factor to E8M0 format without per-tensor scaling. In contrast during LO-BCQ, we find that per-tensor scaling combined with quantization of per-block array scale-factor to E4M3 format results in superior inference accuracy across models. 


\end{document}
