\section{Related Work}
\label{section2}


%\subsection{3D Image/Video Coding for Human}

%Over several decades, to reconstruct high-fidelity 3D videos for human perceptual with the restriction of bitrates, the traditional 3D video coding standards have been developed. As the first generation of 3D codecs, the multi-view video coding (MVC) \cite{2011-IEEE-MVC} extends multi-view coding technologies based on advanced video coding (AVC) for advanced  stereo- and multi-view video compression capability. To address the unique characteristics of multi-view video plus depth (MVD) video, 3D-HEVC develops a series of inter-view prediction \cite{2022-TIP-Lei} and depth-image coding technologies \cite{2022-TCSVT-Li} based on HEVC. More recently, with the increasing demands for more flexible camera arrangement, the MPEG-I has developed MIV for immersive volumetric content compression, where the inter-view redundancy between multiple views is removed by compositing the atlases of patches.

\subsection{Stereo Image Compression for HVS} 
%During the past decades, stereo image compression has attracted great attentions from both academia and industry \cite{2002-TCSVT-SIC} \cite{2009-TCSVT-SIC} \cite{2012-TCSVT-SVC}. Different from 2D image compression, the stereo image compression generally focuses on removing inter-view redundancy by exploiting the correlation between stereo images, such as disparity-compensated prediction (DCP) \cite{2009-TCSVT-SIC}. Wong et al  \cite{2012-TCSVT-SVC} introduced horizontal scaling and shearing (HSS) deformation-based prediction to improve the DCP accuracy. 

During the past decades, with the aim to reconstruct high-fidelity stereo images for HVS with the restriction of bitrate, \textcolor[rgb]{0,0,0}{SIC has been widely studied for efficient stereo image storage and transmission \cite{2002-TCSVT-SIC}, \cite{2009-TCSVT-SIC}, \cite{2012-TCSVT-SVC}.}
%attracted great attention from both academia and industry.
Different from 2D image compression, the SIC generally focuses on removing inter-view redundancies by exploiting the correlation between stereo images. For instance, on the basis of high efficiency video coding (HEVC), the multiview HEVC (MV-HEVC) \cite{2011-IEEE-MVC} significantly improves the SIC performance by extending multi-view coding technologies, such as disparity-compensated prediction (DCP) \cite{2024-TCSVT-Zhang}. To improve the DCP accuracy,  Wong et al.  \cite{2012-TCSVT-SVC} considered the horizontal scaling and shearing (HSS) deformations in stereo images and introduced HSS-based DCP to improve stereo image compression efficiency. 	

With the growing success of deep learning, several researchers have exploited neural networks to construct the learned stereo image compression framework. Compared to traditional methods, learned stereo image compression methods can boost the compression efficiency \textcolor[rgb]{0,0,0}{by jointly optimizing the whole framework with the rate-distortion cost}. For instance,  Lei et al. \cite{2022-BCSIC} employed bi-directional coding mechanism in stereo image compression and achieved impressive coding performance improvement. Wodlinger et al.  \cite{2022-SASIC} proposed stereo image compression with latent shifts and stereo attention (SASIC), where the encoded left-image latent representation is shifted to the \textcolor[rgb]{0,0,0}{right-image as an inter-view reference prior}.

Despite significant progress on SIC for human perceptual, it is undesirable to directly apply the above methods for SICM, due to the emphasized pixel-level signal fidelity rather than high-level semantic fidelity for visual analysis.

\begin{figure*}[htbp]
	\centering
	\includegraphics[width =18.0cm]{fig1}
	\caption{\textcolor[rgb]{0,0,0}{The architecture of the proposed MVSFC-Net. For the encoding stereo images $ \left\{I_{L}, I_{R}\right\} $, the stereo feature extraction module is firstly applied to obtain the stereo multi-scale features $ \left\{f_{L}^{i}, f_{R}^{i}\|i\in 0,1,2\right\} $. Then, the $ \left\{f_{L}^{i}, f_{R}^{i}\|i\in 0,1,2\right\} $ are efficiently compressed by the proposed stereo multi-scale feature compression module. Finally, the visual analysis module deployed at service-end is utilized to perform vision task based on reconstructed stereo multi-scale features $ \left\{\hat{f}_{L}^{i}, \hat{f}_{R}^{i}\|i\in 0,1,2\right\} $}.}
	
	\label{fig1}
\end{figure*}

\subsection{Image Coding for Machines}

%\textcolor[rgb]{0,0,0}{To promote the high-level machine analysis in terms of task performance and compression efficiency, lots of efforts have been devoted to image coding for machines, which can be  broadly categorized into two paradigms, i.e., analyze-then-compress (ATC) [xx] and  compress-then-analyze (CTA) [xx].}
%In view of the huge demand for high-efficiency visual data storage and transmission, 

\textcolor[rgb]{0,0,0}{In the era of deep learning, to promote image compression efficiency for visual tasks, many efforts have been devoted to ICM based on CTA pipeline and ATC pipeline.}

As for the CTA pipeline, the source image is compressed at the front-end and reconstructed at the decoder side for visual analysis. \textcolor[rgb]{0,0,0}{To reduce bitrate while maintaining visual task performance}, several researchers focused on masking out information that is not required for the visual task. For instance, Fischer et al. \cite{2022-TCSVT-LSM} proposed a latent space masking network (LSMnet) to discriminatively squeeze out the non-salient components in the compact latent representation. Another promising solution is to enhance the image coding framework with the visual task-related loss optimization. Concretely, Fischer et al. \cite{2022-MMSP-FRDO} replaced conventional pixel-level rate-distortion optimization (RDO) with a standard-compliant feature-based RDO (FRDO) for VVC to improve the object detection accuracy. \textcolor[rgb]{0,0,0}{Gao et al. \cite{2023-TMM-Gao} proposed semantics-oriented metrics to encourage the reconstructed images more suitable for various visual tasks.} Patwa et al. \cite{2020-ICIP-SPIC} incorporated a cross-entropy loss term in the end-to-end image compression framework to improve the image classification performance.  Wang et al. \cite{2021-CAS-Wang} proposed to optimize the image compression network with the combination of visual task loss and the rate-distortion loss. \textcolor[rgb]{0,0,0}{Although these methods obviously improve the visual task performance under bitrate constraints for CTA pipeline, redundancy information for machines in the reconstructed color images is inevitably maintained, which prevents further improvements for coding performance.}



In the ATC pipeline, the compact features are firstly extracted at the front-end, and then encoded as well as transmitted to the service end for visual analysis. In particular, MPEG has designed a series of visual feature descriptors \cite{2018-TMM-CDVS-IPS}, \cite{2019-TMM-CDVS-CBF} in CDVS \cite{2016-TIP-CDVS} for visual search and video analysis. Choi et al. \cite{{2018-ICIP-Choi}} developed a collaborative intelligence for mobile applications, where the deep features are near-lossless and lossy compressed for object detection. To enhance the feature generalization for multiple visual tasks, Feng et al. \cite{2022-ECCV-Feng} proposed to obtain omnipotent features by taking advantage of self-supervised learning technologies. \textcolor[rgb]{0,0,0}{Liu et al. \cite{2023-ICASSP-Liu} proposed an efficient feature compression network, which explores the correlation between features of different scales, and utilizes the encoded small-scale features to predict large-scale features.} Sun et al. \cite{2021-TCSVT-Sun} proposed to separately represent the object in a semantically structured bitstream, where visual tasks are performed on the partial bitstream to reduce the decoding computational complexity and the transferred data. To fulfill the needs for machine vision and human vision jointly, several efforts have been devoted to developing a scalable coding framework \cite{2021-TMM-Yang-SFIC}, \cite{2022-TMM-Wang},  \cite{2022-TCSVT-Huang}, where an additional bitstream is designed to ensure signal fidelity for HVS. For instance, Yang et al. \cite{2021-TMM-Yang-SFIC} proposed a face image coding framework, where the compact structure representations are extracted to be the basic layer for machine vision, and the color representation is served as an enhanced layer to generate images for human vision. Wang et al. \cite{2022-TMM-Wang} proposed a collaborative visual information representation to investigate interactions between the visual feature for machine vision and the texture for human vision. Huang et al. \cite{2022-TCSVT-Huang} proposed an efficient human-machine friendly video coding scheme (HMFVC), where the semantic information for machine vision is efficiently exploited to enhance the video reconstruction quality for human vision.

However, as far as we know, no previous research  on SICM has been reported. Since the inter-view correlations between two viewpoints have not been explicitly explored, it is not satisfactory to apply the aforementioned \textcolor[rgb]{0,0,0}{ICM methods} for SICM.