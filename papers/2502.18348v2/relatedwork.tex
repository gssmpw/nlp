\section{RELATED WORK}
Our contribution is an attempt to bridge the gap between the knowledge we have on accessibility for visualization (as a complex space of design and engineering) with research and practice that centers on users with disabilities being able to adust, change, or control the interfaces they interact with. We intend to frame our work towards the benefit of data visualization designers, system engineers, and end-users of data visualizations. We believe that more flexible data visualization systems that enable user preferences will require a careful approach to architecture and thorough consideration for the burdens placed on end-users.

\subsection{Data Visualization and Accessibility}
Data visualization accessibility has come far in recent years. But little work has been done to explore what disability scholars call ``access friction'' - a tension that arises when access must be negotiated \cite{Hsueh}. This friction is often a result of static barriers in shared spaces: one artifact or approach designed to include some people may end up excluding others.

In general, accessibility concerns itself with a broad spectrum of barriers that people with different disabilities face. And while most literature focuses on visual disabilities \cite{Wimer}, there are growing resources on areas such as cognitive/intellectual disabilities \cite{Wu}, neurodivergence \cite{Tran}, and both research and systems exploring epilepsy and vestibular/motion inaccessibility in visualization \cite{South}.

% access for assistive technologies that are used by people with dexterity and upper-body motor impairments \cite{Elavsky2023}, 

Yet despite these resources, making data visualizations more accessible remains a difficult task for practitioners \cite{Joyner}. Some accessibility guidelines even conflict, for example on the topic of patterns and textures used in charts. One side stresses that patterns are harmful to cognitive and visual accessibility while another stresses that redundant encoding strategies are necessary \cite{Elavsky2022}. Understanding how to make the correct design decisions may sometimes be impossible. Either existing guidelines are incorrect or it is possible that access friction becomes inevitable the more we know what different barriers look like for different people with disabilities.

\subsection{Systems that Adapt}

One angle of exploration that has been engaging this issue already focuses on systems that can adapt. Work on adaptive systems for people with disabilities, such as in \textit{ability-based design} \cite{Wobbrock2011}, stresses the importance of design alleviating burdens placed on users. Users who don't fit initial system designs are often expected to adapt to fit the system. This means that they may have to acquire an assistive technology, learn a peripheral skill, hack the system, or wait on a design fix. This places the burden on the user to fit the system. Ability-based design instead stresses that systems should be capable of automatically adapting, in order to reduce these burdens placed on the system's users. 

However, building data visualizations that automatically adapt to users via some form of data collection often do so through means such as monitoring live biometric data and input patterns, collecting a user's self-declared conditions and cognitive ability, parsing a user's history, and sensing a user's environmental or situational context \cite{Yanez}. We argue that these methods for an adaptive system raise questions of end-user agency, trust, privacy, and awareness in regards to the system decision-making. They may not be sufficient for addressing a user's needs while also preserving their privacy and agency.

% \cite{Pallapa}

\subsection{Personalization and Accessibility}
Lastly, we researched broader spaces where users have more design agency and explicit awareness of a system that is built to be adapted. We were interested in literature and projects that explore ways end-users can enact meaningful change on an interface, with special attention paid to accessibility and disability.

One specific project has emerged at the intersection of accessibility, visualization, and customization which focuses on screen reader users adjusting the content of textual tokens when navigating data visualizations \cite{Jones}. While this is excellent work, we still have larger questions about when preferences, options, and customizations are appropriate and in what contexts as well as other ways of conceptualizing end-user agency over a system. It remains unclear when, why, and how customization and personalization can be used effectively when designing a system.

In the field of meta-design, meta-designers consider these end-user manipulations of a system to be one facet of ``end-user design'' and ``continuous co-design'' between a system and a user \cite{Maceli2013}, which helps give us some meaningful language to refer to our system goals.

Recent work on the influential factors for personalization and adoption of accessibility settings \cite{Wood2023} also informs our work in 2 key ways: conceptual mismatching between a system and user can contribute to a system's under-use while features that propose value, are time-saving, or reduce cognitive load for a user can contribute to positive perception and use of personalization of a system.