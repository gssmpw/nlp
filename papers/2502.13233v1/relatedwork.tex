\section{Related Work}
\label{related_work}
While large language models excel at question answering, they struggle with specialized medical knowledge~\cite{shi2023mededit}.
Traditional RAG systems have been proposed to mitigate these limitations by integrating external knowledge sources into the LLMs' input, thereby enhancing their ability to generate accurate and contextually relevant responses \cite{brown2020language, guu2020retrieval}. Nonetheless, these early RAG approaches predominantly relied on retrieving information from structured databases or knowledge graphs, which can be insufficient for handling the complexity and nuanced requirements of specialized medical queries \cite{jin2020disease, xiong2024benchmarking}.

Recent advancements in RAG systems have focused on improving the accuracy and relevance of retrieved medical information. 
Xiong et al.~\cite{xiong2024improving} propose i-MedRAG, which uses iterative follow-up questions to help LLMs better understand complex medical queries through dynamic information seeking.
Das et al.~\cite{das2024two} introduce a two-layer RAG framework and use soical media data to address challenges in low-resource medical question answering scenarios. 

Recent work has also explored uncertainty estimation in RAG systems for various purposes~\cite{ozaki2024understanding, li2024uncertaintyrag}. 
Some approaches use uncertainty to determine when to retrieve external information~\cite{ding2024retrieve, liu2024ctrla}, while others focus on improving robustness by reducing semantic inconsistencies from random chunking~\cite{li2024uncertaintyrag} or enhancing decoding~\cite{qiu2024entropy}. 
Despite these advancements, existing RAG frameworks still struggle with retrieving fine-grained and up-to-date knowledge, particularly for complex medical questions. 
This is because traditional knowledge bases and graphs often lack the granularity and timeliness needed for specialized medical queries.
Our approach addresses these limitations by utilizing search engines for real-time knowledge retrieval.