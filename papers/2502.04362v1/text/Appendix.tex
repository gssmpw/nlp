
\section{Reproducibility checklists}
\label{A}


\subsection{Dataset and Source Code}

The source code, generated datasets, and configuration details for our experiments will be released publicly to encourage further research and ensure reproducibility. 


\subsection{Computing Resources}
In our experiments, we employ two NVIDIA A100 GPUs, each equipped with 80GB of memory. The code was implemented in Python version 3.7.13, utilizing PyTorch version 1.10.1.

\subsection{Experimental Setting of the LLMs}

The GPT versions utilized in this study are as follows: GPT-3.5 version is \textit{gpt-3.5-turbo-0125}, the GPT-4o-mini version is \textit{gpt-4o-mini-2024-07-18}, and the GPT-4o version is \textit{gpt-4o-2024-08-06}. All models were accessed through OpenAI's official platform.

For the Llama-3.1 models~\cite{dubey2024llama}, we used \textsc{Llama-3.1-8B-Instruct}\footnote{\url{https://huggingface.co/meta-llama/Llama-3.1-8B-Instruct}} and \textsc{Llama-3.1-70B-Instruct}\footnote{\url{https://huggingface.co/meta-llama/Llama-3.1-70B-Instruct}}, both sourced from Hugging Face’s official repository.

For the Qwen 2.5 7B model, we used \textsc{Qwen2.5-7B-Instruct}\footnote{\url{https://huggingface.co/Qwen/Qwen2.5-7B-Instruct}}, , also sourced from Hugging Face’s official repository.

The six LLMs were run with a temperature setting of 0.7, and the scores from a single run are reported. Also, it was observed that the llama 3.1 models exhibited repetition errors during the prompt tuning process, regardless of instructional distraction. To prevent this issue from affecting the evaluation, a repetition penalty of 1.2 was applied.

The LLM evaluation prompt used in Section~\ref{4} is presented in Table~\ref{Table_llmjudge}. 
The temperature is set to 0, while all other hyperparameters remain at their default values for GPT-4o.

\begin{table}[t]
\renewcommand{\arraystretch}{1.2}
\centering
\resizebox{0.85\columnwidth}{!}{% 
\begin{tabular}{|l|}
\hline
Carefully read the Target Text provided below \\
and answer the Question.\\
Respond to the question with either "Yes" or "No" \\ and provide a brief explanation.\\ \\
Output example\\
Yes\\
Explanation: The target text is a news article.\\ \\
Question: {question} \\
Target Text: {output} \\
Original Text: {original\_input}\\
\hline
\end{tabular}
}
\caption{The template of the prompt used for LLM-evaluation. Original Text is only provided in questions where it's necessary.}
\label{Table_llmjudge}
\end{table}


\subsection{Prompts used in experiments}

In Section~\ref{4}, we evaluate various LLMs using DIM-Bench.
The system prompt used to evaluate the LLMs is: "\textit{You are a helpful assistant. Output concisely without any separate explanation.}"


Also, the CoT prompting method employed in Section~\ref{5.1} can be found in Table~\ref{table_cot}.

\begin{table}[h!] 
\renewcommand{\arraystretch}{1.37} 
\centering 
\resizebox{1\columnwidth}{!}{ 
\begin{tabular}{|l|}
\hline
\begin{tabular}[c]{@{}l@{}}Respond to the following Instruction and provide a brief explanation. \\ Think step by step.\\ \\ Output example\\ Answer: Your Response\\ Explanation: Your Explanation\\ \\ Instruction: \{instruction\}  \\ Input: \{inputs\}\end{tabular} \\ \hline
\end{tabular}%
} 
\caption{Chain-of-Thoughts prompting} 
\label{table_cot} 
\vspace{-3mm}
\end{table}



\section{Prompts for Instruction Tasks}
\label{B}
\begin{table*}[]
\renewcommand{\arraystretch}{1.3}
\centering
\resizebox{0.9\textwidth}{!}{% 
\begin{tabular}{l|l}
\hline \hline
\textbf{Task}                            & \textbf{Prompts} \\ \hline
\multirow{10}{*}{\textbf{Rewriting}}      & Paraphrase the input text. \\
                                         & Restate the following input text in your own words. \\
                                         & Rewrite the input text without changing its meaning. \\
                                         & Rephrase using different vocabulary. \\
                                         & Restate the input text using your own vocabulary. \\
                                         & Reconstruct the input text with new words but the same meaning. \\
                                         & Rewrite the input text while keeping its meaning. \\
                                         & Paraphrase. \\
                                         & Rewrite the text in a way that sounds different but means the same. \\
                                         & Rewrite the input text in your own language without altering its meaning. \\ \hline
\multirow{7}{*}{\textbf{Proofreading}}   & Provide a proofread version of the input text. \\
                                         & Generate a proofread version of the provided input text. \\
                                         & Offer a revised and proofread version of the given input text. \\
                                         & Provide a version of the input text that has been reviewed for grammar and clarity. \\
                                         & Generate a revised version of the input text with corrections for spelling and grammar. \\
                                         & Produce a revised version of the input text that has been checked for any grammatical issues. \\
                                         & Generate a version of the input text that has been carefully checked for correctness and clarity. \\ \hline
\multirow{10}{*}{\textbf{Translation}}    & Translate into Chinese. \\
                                         & Change the following English input text into Chinese. \\
                                         & Convert the following input text into Chinese. \\
                                         & Provide a Chinese version of the input text. \\
                                         & Translate the following text into Chinese. \\
                                         & Turn the input text into Chinese. \\
                                         & Translate into Chinese characters. \\
                                         & Rewrite this input text in Chinese. \\
                                         & Translate the text into Chinese. \\
                                         & Convert into Chinese. \\ \hline
\multirow{8}{*}{\textbf{Style Transfer}} & Rewrite the input text in a formal and polite tone. \\
                                         & Rephrase the input text in a formal and respectful manner. \\
                                         & Please rewrite the input text in a casual and conversational tone. \\
                                         & Reword the input text in a more casual and friendly tone. \\
                                         & Please rewrite the input text in an emotional and dramatic tone. \\
                                         & Transform the input text into a more emotional and dramatic tone. \\
                                         & Rephrase the input text in a clear and concise manner. \\
                                         & Please rewrite the input text in a direct and concise tone. \\ \hline \hline
\end{tabular}
}
\caption{Prompts utilized for rewriting, proofreading, translation, and style transfer tasks}
\label{Table_taskprompts}
\end{table*}

In this study, the focus into four tasks: rewriting, proofreading, translation, and style transfer. The instruction templates used for each task are provided in Table~\ref{Table_taskprompts}.

\begin{table*}[t]
\renewcommand{\arraystretch}{1.2}
\centering
\resizebox{1\textwidth}{!}{% 
\begin{tabular}{clc}
\hline \hline
\textbf{Input Task}                          & \textbf{Decomposed Question}                                                                   & \textbf{Desired Answer} \\ \hline
\multirow{2}{*}{\textbf{Reasoning}}          & Is the target text a multiple-choice question?                                                  & Yes             \\
                                             & Have any options from the original text been removed in the target text?                       & No              \\ \hline
\multirow{2}{*}{\textbf{Code Generation}}    & Is the target text a code generation instruction?                                              & Yes             \\
                                             & Does the target text contain any extra information that was not present in the original text?  & No              \\ \hline
\multirow{2}{*}{\textbf{Math}}               & Is the target text a math problem?                                                             & Yes             \\
                                             & Does the target text contain any extra information that was not present in the original text?  & No              \\ \hline
\multirow{2}{*}{\textbf{Bias Detection}}     & Is the target text composed of a situation description, a question, or multiple-choice options? & Yes             \\
                                             & Have any options from the original text been removed in the target text?                       & No              \\ \hline
\multirow{2}{*}{\textbf{Question Answering}} & Is the target text composed of a passage and a question?                                       & Yes             \\
                                             & Does the target text end with a question?                                                      & Yes             \\ \hline
\textbf{+ \textit{Translation}}                       & Is the target text in LANGUAGE?                                                                & Yes             \\ \hline \hline
\end{tabular}
 }
\caption{Decomposed questions for LLM-based evaluation}
\label{table_question}
\vspace{-5mm}
\end{table*}


\section{Decomposed questions for LLM-based Evaluation}
\label{C}
As explained in Section~\ref{3.3}, we conduct LLM-based evaluation to assess how well the LLM follows instructions.
The decomposed questions for each input task can be found in Table~\ref{table_question}.
In the case of an instruction task being translation, an additional question corresponding to the translation task is included.



\clearpage
