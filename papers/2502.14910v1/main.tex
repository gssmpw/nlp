% This must be in the first 5 lines to tell arXiv to use pdfLaTeX, which is strongly recommended.
\pdfoutput=1
% In particular, the hyperref package requires pdfLaTeX in order to break URLs across lines.

\documentclass[11pt]{article}

% Change "review" to "final" to generate the final (sometimes called camera-ready) version.
% Change to "preprint" to generate a non-anonymous version with page numbers.
\usepackage[preprint]{acl}

% Standard package includes
\usepackage{times}
\usepackage{latexsym}

% For proper rendering and hyphenation of words containing Latin characters (including in bib files)
\usepackage[T1]{fontenc}
% For Vietnamese characters
% \usepackage[T5]{fontenc}
% See https://www.latex-project.org/help/documentation/encguide.pdf for other character sets

% This assumes your files are encoded as UTF8
\usepackage[utf8]{inputenc}

% This is not strictly necessary, and may be commented out,
% but it will improve the layout of the manuscript,
% and will typically save some space.
\usepackage{microtype}

% This is also not strictly necessary, and may be commented out.
% However, it will improve the aesthetics of text in
% the typewriter font.
\usepackage{inconsolata}
\usepackage{amsmath,amsfonts}
\usepackage{algorithmic}
\usepackage[section]{algorithm}
\renewcommand{\algorithmicrequire}{\textbf{Input:}} 
\renewcommand{\algorithmicensure}{\textbf{Output:}}
%Including images in your LaTeX document requires adding
%additional package(s)
\usepackage{graphicx}
\usepackage{enumitem}
\usepackage{array, multirow}
\usepackage{tabularx}
\usepackage{booktabs}

\newcommand{\dedit}[1]{\textcolor{red}{#1}}
\newcommand{\note}[1]{\textcolor{red}{(#1)}}
% If the title and author information does not fit in the area allocated, uncomment the following
%
%\setlength\titlebox{<dim>}
%
% and set <dim> to something 5cm or larger.

\title{EvoP: Robust LLM Inference via Evolutionary Pruning}

% Author information can be set in various styles:
% For several authors from the same institution:
% \author{Author 1 \and ... \and Author n \\
%         Address line \\ ... \\ Address line}
% if the names do not fit well on one line use
%         Author 1 \\ {\bf Author 2} \\ ... \\ {\bf Author n} \\
% For authors from different institutions:
% \author{Author 1 \\ Address line \\  ... \\ Address line
%         \And  ... \And
%         Author n \\ Address line \\ ... \\ Address line}
% To start a separate ``row'' of authors use \AND, as in
% \author{Author 1 \\ Address line \\  ... \\ Address line
%         \AND
%         Author 2 \\ Address line \\ ... \\ Address line \And
%         Author 3 \\ Address line \\ ... \\ Address line}

% \author{First Author \\
%   Affiliation / Address line 1 \\
%   Affiliation / Address line 2 \\
%   Affiliation / Address line 3 \\
%   \texttt{email@domain} \\\And
%   Second Author \\
%   Affiliation / Address line 1 \\
%   Affiliation / Address line 2 \\
%   Affiliation / Address line 3 \\
%   \texttt{email@domain} \\}


\author{%
  Shangyu Wu \\
  CityU \\
  \And
  Hongchao Du \\
  CityU \\
  \And
  Ying Xiong~\thanks{Corresponding author.} \\
  MBZUAI \\
  \And
  Shuai Chen\\
  Baidu \\
  \AND
  Tei-Wei Kuo \\
  National Taiwan University \\
  \And
  Nan Guan \\
  CityU \\
  \And
  Chun Jason Xue \\
  MBZUAI \\
}


%\author{
%  \textbf{First Author\textsuperscript{1}},
%  \textbf{Second Author\textsuperscript{1,2}},
%  \textbf{Third T. Author\textsuperscript{1}},
%  \textbf{Fourth Author\textsuperscript{1}},
%\\
%  \textbf{Fifth Author\textsuperscript{1,2}},
%  \textbf{Sixth Author\textsuperscript{1}},
%  \textbf{Seventh Author\textsuperscript{1}},
%  \textbf{Eighth Author \textsuperscript{1,2,3,4}},
%\\
%  \textbf{Ninth Author\textsuperscript{1}},
%  \textbf{Tenth Author\textsuperscript{1}},
%  \textbf{Eleventh E. Author\textsuperscript{1,2,3,4,5}},
%  \textbf{Twelfth Author\textsuperscript{1}},
%\\
%  \textbf{Thirteenth Author\textsuperscript{3}},
%  \textbf{Fourteenth F. Author\textsuperscript{2,4}},
%  \textbf{Fifteenth Author\textsuperscript{1}},
%  \textbf{Sixteenth Author\textsuperscript{1}},
%\\
%  \textbf{Seventeenth S. Author\textsuperscript{4,5}},
%  \textbf{Eighteenth Author\textsuperscript{3,4}},
%  \textbf{Nineteenth N. Author\textsuperscript{2,5}},
%  \textbf{Twentieth Author\textsuperscript{1}}
%\\
%\\
%  \textsuperscript{1}Affiliation 1,
%  \textsuperscript{2}Affiliation 2,
%  \textsuperscript{3}Affiliation 3,
%  \textsuperscript{4}Affiliation 4,
%  \textsuperscript{5}Affiliation 5
%\\
%  \small{
%    \textbf{Correspondence:} \href{mailto:email@domain}{email@domain}
%  }
%}

\begin{document}
\maketitle
\begin{abstract}

Large Language Models (LLMs) have achieved remarkable success in natural language processing tasks, but their massive size and computational demands hinder their deployment in resource-constrained environments. 
Existing structured pruning methods address this issue by removing redundant structures (e.g., elements, channels, layers) from the model. 
However, these methods employ a heuristic pruning strategy, which leads to suboptimal performance.
Besides, they also ignore the data characteristics when pruning the model.

To overcome these limitations, we propose EvoP, an evolutionary pruning framework for robust LLM inference.
EvoP first presents a cluster-based calibration dataset sampling (CCDS) strategy for creating a more diverse calibration dataset.
EvoP then introduces an evolutionary pruning pattern searching (EPPS) method to find the optimal pruning pattern.
Compared to existing structured pruning techniques, EvoP achieves the best performance while maintaining the best efficiency. 
Experiments across different LLMs and different downstream tasks validate the effectiveness of the proposed EvoP, making it a practical and scalable solution for deploying LLMs in real-world applications.
\end{abstract}

\section{Introduction}
\label{sec:intro}

Foundational models (FMs)~\cite{zhang2024data, zhou2023comprehensive} have shown remarkable progress in the healthcare domain, enabling professional-like assessment of disease diagnosis, treatment decision-making, and monitoring~\cite{zhang2023text, wang2022medclip, lu2023mi-zero}. 
Examples include LLaVA-Med~\cite{li2023llava}, Med-PaLM Multimodal~\cite{tu2024towards}, and Med-Flamingo~\cite{moor2023med}, have demonstrated their capacity on question answering, medical image analysis, and report generation.
These studies follow a predominant top-down model development strategy that requires upstream developers to collect data and train models for downstream tasks. 
Consequently, the developed model capabilities are heavily dependent on the training data, limiting their generalization performance in diverse clinical scenarios. 
For instance, Med-Gemini~\cite{yang2024advancing} reveals promising general capabilities in report generation while it lags behind state-of-the-art (SoTA) models on classification tasks, especially for out-of-domain applications. 
This indicates that while the generalizability of the foundation model is promising, more solutions are expected to meet the various specialized clinical needs.

To address these challenges, multi-center data centralization becomes essential to enhance model capacity and robustness across varied clinical scenarios~\cite{rajpurkar2022ai}. 
Centralizing distributed data can significantly improve model training and inference performance.
However, the process of medical data storage, transfer, and aggregation among centers requires extra efforts to ensure data security and system interoperability~\cite{bradford2020international}.
Moreover, a growing concern for patient privacy makes large-scale multi-center data sharing particularly challenging. 
While efforts like federated learning~\cite{wen2023survey, li2020review} can achieve good model performance on local data, the need for synchronized system coordination presents significant challenges, as clients are unable to update asynchronously. This limitation greatly restricts the practical capability of such approaches.
As a result, without a flexible collaboration, medical community still struggles to fully utilize the isolated data and local computation resources for comprehensive medical AI model development. 
To address this dilemma, open-source platforms encourage public data sharing and knowledge integration~\cite{markiewicz2021openneuro, zenodo}.
However, these platforms focus solely on raw data sharing while seldom providing collaborative model training or cooperation between different institutions.
Recently, collaborative learning has emerged as a viable approach for enhancing multi-model robustness~\cite{boulemtafes2020review}. 
For instance, software-like model development~\cite{raffel2023building} mimics software engineering practices by introducing structured workflows, enabling merging, version control, and continuous model integration.
Under this design, model ability can be strengthened with incremental knowledge updates similar to the version updating in software development. 

Although collaborative learning provides a multi-model collaboration, two key challenges remain in the leakage of raw data during collaboration~\cite{huang2023lorahub} and the synchronization of multiple collaborators~\cite{mcmahan2017communication} in the medical AI community. It is still challenging to integrate decentralized, privacy-sensitive data across institutions, leading to under-utilized insights and fragmented knowledge sharing~\cite{kaissis2020secure, rajpurkar2022ai, abdullah2021ethics}.
 To address these challenges, inspired by the collaborative software development, we propose \textbf{Med}ical \textbf{Fo}undation Models Me\textbf{rg}ing (\textbf{MedForge}), a cooperative workflow enabling continuously community-driven foundation model (FM) development.
MedForge enables a lightweight manner for individual centers to share their knowledge among multiple centers, minimizing the burden of data transmission and integration while enhancing model robustness.
Meanwhile, MedForge facilitates asynchronous and flexible collaboration, allowing individual centers to continuously update and improve medical FMs without the need for real-time synchronization.
Similar to open-source software development, MedForge incrementally updates medical knowledge and follows a sustainable model development scheme. 
This key design emphasizes a bottom-up construction of a multi-task medical FM, allowing downstream users to collaboratively build, refine, and update the upstream model according to their local resources. Our major contributions of MedForge are as below: 
\begin{enumerate}
    \item[$\bullet$] We introduce a collaborative workflow to promote the merging scheme of open-source software development. Our proposed MedForge allows distributed clinical centers to asynchronously contribute to comprehensive medical model construction while reducing transmitting costs among centers and avoiding the leakage of raw data, thus enhancing the utilization of private resources in the healthcare system. 
    \item[$\bullet$] We propose two effective knowledge-merging strategies for the asynchronous branch contribution. The MedForge-Fusion strategy updates the plugin module parameters of the main model during the merging phase, whereas the MedForge-Mixture strategy integrates the output of the plugin module by memorizing each contributor's coefficient. These strategies make MedForge more flexible and versatile. MedForge-Fusion is friendly to implement, while the MedForge-Mixture offers better performance and robustness.
    \item[$\bullet$]  We comprehensively evaluate model merging strategies to accumulate medical knowledge among multiple branch plugin modules. MedForge yields superior performance on medical classification tasks compared to other collaborative baselines across multiple datasets. We demonstrate the robustness of MedForge by shuffling the task order and evaluating various configurations of plugin modules and dataset distillation methods.
\end{enumerate}



\section{Background}

\subsection{Memory Access Bottleneck of LLM Inference}

Introduce the prefilling and decoding stage of LLMs and the memory bottleneck caused by autoregressive generation.

Introduce batching, which alleviates this bottleneck by processing multiple requests simultaneously and share the access of model weights.

Introduce the difficulty of batching in long context serving

\begin{itemize}
    \item No enough GPU memory to store KV cache
    \item High KV cache access overhead
\end{itemize}

\subsection{Self-Attention Module and Rotary Position Embedding}

show an equation of how RoPE injects relative positional information through absolute position encoding.

\subsection{Recalling-Based KV Cache Reduction}

Recalling-based KV cache reduction is considered a promising method to mitigate the challenges in long context serving.

Its core idea is to make use of the inherent sparsity of the output tensor of the softmax function in attention modules, to reduce KV cache access during decoding stage. Its workflow contains two steps, first, for each decoding step, it identifies the required key-value pairs with the highest attention weights, then recall these kv cache and perform selective attention computation.

Introduce PQCache and ClusterKV, and analyze their weakness: online clustering leads to inefficient iterative read of KV cache, resulting in even higher memory access overhead.

Introduce Quest, although it does not introduce extra overheads, it leads to substantial model performance degradation and has limited compression ratio.

Introduce MagicPIG, although it achieves SoTA in recalling-based methods, it causes massive memory usage on CPU due to the large LSH table.



\section{Method}


In this work, we propose a method to achieve 3D-aware 2D representations and enable 3D reconstruction in the latent space. We base our method on the widely used Variational Autoencoder (VAE) from Latent Diffusion models \citep{metzer2022latent}. To enhance the 3D awareness of both encoder and decoder of the VAE, we present a three-stage pipeline as illustrated in Fig. \ref{fig:pipeline}. The first stage focuses on improving the 3D awaresness of the VAE's encoder through a novel correspondence-aware constraint on the latent space, making the 2D representations follow the geometry consistency (Sec.~\ref{subsec: Epipolar-aware Autoencoding}); The second stage builds a latent radiance field (LRF) to represent 3D scenes from the 3D-aware 2D representations (Sec.~\ref{subsec: Latent Radiance Fields}); The third stage further introduces a VAE-Radiance Field (VAE-RF) alignment method to boost the reconstruction performance (Sec.~\ref{subsec: Radiance Field-Guided Image Decoding}). In together, our LRF enables 3D reconstruction on the 2D latent space instead of the image space. It can render high-quality and photorealistic novel views, even for the unbounded scenes (Sec. \ref{sec: exp}). More details of our method are discussed in the following sections.


\begin{figure}[!t]
    \centering
    \includegraphics[width=\linewidth]{figures/method.png}
    \vspace{-1em}
    \caption{An illustration of  our pipeline for creating a latent radiance field in conjunction with 3D-aware 2D representation fine-tuning. 
    Firstly in Stage-I, we inject 3D awareness into the VAE’s encoder through applying a novel correspondence consistency constraint on the latent space, making the 2D representations follow the geometry consistency. Then in Stage-II, we create the latent radiance field (LRF) to represent 3D scenes based on the 3D-aware 2D representations. Finally in Stage-III, we introduce a VAE-Radiance Field alignment method to enhance the performance of image decoding from the  rendered latent space.
}
\vspace{.5em}
    \label{fig:pipeline}
\end{figure}

\subsection{Correspondece-aware Autoencoding}
\label{subsec: Epipolar-aware Autoencoding}
The first stage of our method is incorporating the geometry-awareness into the autoencoding process. Given $K$ muilt-view images $\mathcal{I}=\left\{\boldsymbol{I}_i\right\}_{i=1}^K,\left(\boldsymbol{I}_i \in \mathbb{R}^{H \times W \times 3}\right)$, the VAE encoder extracts 2D representations $\mathcal{Z}=\left\{\boldsymbol{Z}_i\right\}_{i=1}^K,\left(\boldsymbol{Z}_i \in \mathbb{R}^{H' \times W' \times 4}\right)$ in a low-dimensional latent space while the semantic information can be preserved effectively. However, as shown in Fig. \ref{fig: exp_recon}, most of existing NVS frameworks fail to reconstruct the photo-realistic images from the rendered latents.
It is mainly because the VAE encoding process significantly damages the multi-view consistency within the original image space, since the latent space presents massive high-frequency noises to compress the original RGB space into a compact latent space (see Fig. \ref{fig: encoder}). 
This brings severe challenges for reconstructing the 2D latent representations in the 3D space. 




\noindent\textbf{Correspondence consistency on the latent space.}
To address the above issue and enable effective latent 3D reconstruction, we are inspired by the multi-view correspondence consistency which serves as the foundation principle for modeling the natural 3D world. Specifically, points $\boldsymbol{x}_i \in \mathbb{R}^{2}$ in image $\boldsymbol{I}_i$ and points $\boldsymbol{x}_j \in \mathbb{R}^{2}$ in another image $\boldsymbol{I}_j$ are considered correspondences if they are connected by the fundamental matrix $\boldsymbol{F}_{ij} \in \mathbb{R}^{3 \times 3}$, satisfying the multi-view geometry constraint~\citep{schoenberger2016sfm}:
\begin{equation}
\boldsymbol{x}_{j}^\top \boldsymbol{F}_{ij} \boldsymbol{x}_i = 0.
\label{eq:fundamental}
\end{equation}
Eq. \ref{eq:fundamental} tells that a pair of correspondence points on the image space should be close to each other, so that the consistent geometry can be ensured during the optimization in the 3D space; otherwise, the artifacts and redundant geometry representation due to the local optimal will damage the quality of the 3D reconstruction and novel view synthesize. 
Motivated by this, we propose an computationally efficient strategy that incorporates the correspondence consistency into the autoencoder training. 
Specifically, a set of multi-view images $\mathcal{I}=\left\{\boldsymbol{I}_i\right\}_{i=1}^K,\left(\boldsymbol{I}_i \in \mathbb{R}^{H \times W \times 3}\right)$ are fed into the autoencoder to extract the latent representations  $\mathcal{Z}=\left\{\boldsymbol{Z}_i\right\}_{i=1}^K,\left(\boldsymbol{Z}_i \in \mathbb{R}^{H' \times W '\times 4}\right)$, and the correspondence consistency loss on the latent space is computed by 
% \textcolor{red}{Give the defination of j and N, and this loss should be step loss instead of total images loss}
\begin{equation}
\mathcal{L}_{\text{corres}} =  \sum_{i=1}^{K} \sum_{j \in \mathcal{K}(i)} \lambda_{ij} \left\| \boldsymbol{z}_i - \boldsymbol{z}_j \right\|_1.
\end{equation}
where $\boldsymbol{z}_i$ refers to the the latent pixel in the $\boldsymbol{Z}_i$ and $\boldsymbol{z}_i$ is the corresponding latent pixel in the neighbouring latent  $\boldsymbol{Z}_j$.
$\mathcal{L}_{\text{corres}}$ ensures that the encoded features follow the correspondence consistency derived from the multi-view images, where $\lambda_{ij}$ is the weight based on the average pose error (APE) calculated from the Frobenius norm between the two camera poses of images $\boldsymbol{I}_i$ and $\boldsymbol{I}_j$ to weight the accurate pose distance to represent the view-dependant latent codes. The detail of calculating $\lambda_{ij}$ can be found in Appendix \ref{subsec: APE details}
By injecting the latent correspondence consistency into the standard VAE training, our VAE training objective is: 
\begin{equation} 
\mathcal{L}_\text{StageI} =\mathcal{L}_\text{VAE} + \lambda_{1}\mathcal{L}_{\text{corres}} + \lambda_{2}\mathcal{L}_{\text{reg}}.
\label{eq:encoder}
\end{equation}

$\mathcal{L}_\text{VAE}$ is original VAE traning objective for VAE in Eq. \ref{eq:vae}. 
$\mathcal{L}_{\text{reg}} = -\text{KL}\left( q(\boldsymbol{Z}|\boldsymbol{X}) \parallel q_{\text{original}}(\boldsymbol{Z}|\boldsymbol{X}) \right)$ enforces the fine-tuned 2D representations being close to those of the pre-trained VAE, preserving the representation capability of the finet-tuned autoencoder.  This new learning objective ensures that the compact latent space of VAE preserves the multi-view geometric consistency, such that it is compatible with existing NVS frameworks such as 3DGS.



\textbf{Insight into latent correspondence consistency.} 
The maximum degree of the spherical harmonics is always set as 3 in NVS methods for the efficiency and robustness in the modeling the view-dependant information. To be more specific, the lower degree terms is aim to mostly capture low-frequency information such as albedo for the scene while the higher degrees are tended to model the high-frequency, view dependent information such as the lightning. For the latent space, the latent code can be considered as the combination of the base value and high frequency noise. Due to such a compact representation, the amount of the noise can be greatly increase compared to the RGB space, creating more difficulties for the SH coefficients to model the information from different views. When maximum degree is fixed, it is easier for SH coefficients to reach the global optimal instead of locally over-fitting. Fortunately, with our $\mathcal{L}_{\text{corres}}$, the high frequency noise can be effectively removed while the high-quality image generative ability can still be preserved, leading to a more stable process of the optimization and consistent geometry representation. Fig. \ref{fig: encoder} shows that the correspondence-aware encoding can significantly remove the high frequency noises in the 2D latent space and the visualization of applying Fast Fourier transform also showing less high-frequency noise in latent space achieved by our encoder,  resulting an effective approach to lifting the 2D features into the 3D latent fields.

\begin{figure}[!t]
    \centering
    \begin{tikzpicture}
     

        \node[anchor=south west, inner sep=0] (image1) at (0,0) {\includegraphics[width=1.0\textwidth]{figures/fft.png}};
        
       
        \node[anchor=south] at (1.3, 2.0) {\small Image};               
        \node[anchor=south] at (4.15, 2.0) {\small VAE latent};         
        \node[anchor=south] at (7.0,  2.0) {\small Finetuned latent};               
        \node[anchor=south] at (9.8,  2.0) {\small VAE latent FFT};
         \node[anchor=south] at (12.55,   2.0) {\small Finetuned latent FFT};
    \end{tikzpicture}
    \vspace{-1em}
    \caption{A visualization of latent spaces of original and our fine-tuned VAEs. Our method ensures an accurate geometry in the latent space while removing a certain amount of high-frequency noises.}
\label{fig: encoder}
\end{figure}



\subsection{Latent Radiance Field}
\label{subsec: Latent Radiance Fields}



Based on the 3D-aware 2D representation fine-tuning discussed in Sec.~\ref{subsec: Epipolar-aware Autoencoding}, we create 3D representations directly in the 2D latent space of VAE, namely the latent radiance field (LRF). We take 3DGS \citep{kerbl3Dgaussians} as an example of radiance field representations to discuss our LRF.  

By following 3DGS, a set of latent 3D Gaussians is formulated as
\begin{equation}
    \mathcal{G} = \{(\bm{\mu}, \mathbf{s}, \mathbf{R}, \alpha, \mathbf{SH}_{f})_j)\}_{1\leq j \leq M} \textnormal{,}
\end{equation}
where $\bm{\mu} \in \mathbb{R}^3$ is the 3D mean of the Gaussian, $\mathbf{S} = \textnormal{diag}(\mathbf{s}) \in \mathbb{R}^{3\times 3}$ is the Gaussian scale, $\mathbf{R}\in \mathbb{R}^{3\times 3}$ its orientation, $\alpha \in \mathbb{R}$ a per-Gaussian opacity, and $\mathbf{SH}_{f}$ models the view-dependant latent in the 3D latent space. By following the differentiable rasterization process of 3DGS, we rasterize the 2D latent representations using point-based $\alpha$-blending as follows:
\begin{equation}
\mathbf{Z} = \sum_{i\in \mathcal{N}}\mathbf{z}_{i}\alpha _{i}\prod_{j=1}^{i-1}(1-\alpha _{i}),
\end{equation}
where $\mathcal{N}$ is a set of ordered Gaussians overlapping the pixel, $\mathbf{z}_{i}\in \mathbb{R}^{dim}$
is the view-dependent latent code of each Gaussian, where $\mathbf{dim}$ is the number of the latent dimension of the feature. and $\alpha _{i}$ is given by evaluating a
2D Gaussian with covariance $\mathbf{\Sigma}$ multiplied with a
learned per-point opacity. 
Let  $\mathcal{I}=\left\{\boldsymbol{I}_i\right\}_{i=1}^K,\left(\boldsymbol{I}_i \in \mathbb{R}^{H \times W \times 3}\right)$ be a set of multi-view images of a scene with corresponding camera parameters. Let $\mathcal{Z}=\left\{\boldsymbol{Z}_i\right\}_{i=1}^K,\left(\boldsymbol{Z}_i \in \mathbb{R}^{H \times W \times 3}\right)$ be a corresponding set of latents from the VAE encoder. The rasterization function $r$ renders a set of latent Gaussians into a 2D latent representation according to the camera pose $\mathbf{P}_{i}$. Then, we optimize the latent Gaussian parameters, to optimally represent
latent $\mathcal{Z}$:
\begin{equation}
    \hat{\mathcal{G}} = \argmin_{\{(\bm{\mu}, \mathbf{s}, \mathbf{R}, \alpha, \mathbf{SH}_{f}\}} \sum_{i=1}^N \mathcal{L}^f(r(\mathcal{G}, \mathbf{P}_{i}),\mathbf{Z}_i) \textnormal{,}
\end{equation}
where $\mathcal{L}^f$ is a pixel-wise $l_{1}$ loss combined with a D-SSIM term. Notably, we do not need to impose additional geometric consistency constraints introduced by previous literature~\citep{yue2024improving,kobayashi2022distilledfeaturefields,zhou2024feature}, as our correspondence-aware autoencoder fine-tuning ensures geometrically consistent 2D representations in the 3D space. Therefore, our LRF reconstructs the 2D latent representations as a radiance field representation directly, and enables an efficient rendering of the 2D latent representations for novel views.

\subsection{VAE-Radiance Field Alignment} \label{subsec: Radiance Field-Guided Image Decoding}
Although the correspoondence-aware autoencoding introduced in Sec.~\ref{subsec: Epipolar-aware Autoencoding} improves the 3D consistency of VAE latent space, the LRF distribution $\boldsymbol{p}(z_{\text{NVS}})$ are still shifted from the VAE latent distribution $\boldsymbol{p}(z_{\text{VAE}})$ due to the non-linearity in neural rendering, resulting in performance decrease when we decode LRF rendering results back to images through the VAE decoder. 

We further propose to fine-tune the VAE decoder under the radiance field guidance to address this issue. With the LRF built in Sec. \ref{subsec: Latent Radiance Fields}, we can reconstruct LRFs from a large amount of scenes to generate a latent-image paired dataset. This dataset consists of the 2D latent representations $\mathcal{Z}=\left\{\boldsymbol{Z}_i\right\}_{i=1}^K,\left(\boldsymbol{Z}_i \in \mathbb{R}^{H' \times W' \times 4}\right)$ rendered by LRFs and the corresponding ground truth images $\mathcal{I}=\left\{\boldsymbol{I}_i\right\}_{i=1}^K,\left(\boldsymbol{I}_i \in \mathbb{R}^{H \times W \times 3}\right)$. Notably, we also include the training views of LRFs in this dataset, since a key feature of existing NVS methods is to overfit the training views. 
The training objective of our VAE-RF alignment decoder fine-tuning is:
\begin{equation} 
\mathcal{L}_\text{StageIII}=  \lambda_{\text{train}} \left\|D(Z_{\text{train}}) - I_{\text{train}} \right\|_1 + \lambda_{\text{novel}} \left\|D(Z_{\text{novel}}) - I_{\text{novel}}\right\|_1,
\label{eq:decoder}
\end{equation} 
where $D(\cdot)$ is the decoder, $Z_{\text{train}}$ and $Z_{\text{novel}}$  are the latent codes of the training views and novel views, respectively. $I$ refer to the corresponding ground truth images. $\lambda_{\text{novel}}$ and $\lambda_{\text{novel}}$ are the weighting coefficient that balances the contributions of the training and novel views. Both of the weights are set to $0.5$ to ensure that the decoder learns not only to decode effectively from the training views but also to generalize and perform well on the novel views.
Eq. \ref{eq:decoder} effectively minimizes the distribution mismatch between the VAE latent space and the LRF rendering space. After decoder fine-tuning, high-quality images can be reconstructed from the LRF rendering of either training or novel views. The fine-tuned autoencoder enhances 3D reconstruction and generation by providing a geometry-aware 2D latent space as well as a radiance field-compatible autoencoder.




\begin{figure*}[!h]
    \centering
    \begin{subfigure}[b]{0.8\linewidth}
        \centering
        \includegraphics[width=0.45\linewidth]{images/residual/text/CIReVL_Recall5.png}
        \hfil
        \includegraphics[width=0.45\linewidth]{images/residual/text/pic2word_recall5.png}
        \caption{\textbf{PDV-T}: Impact of $\alpha$ scaling on composed text embeddings}
        \label{fig:residual_text_sub}
    \end{subfigure}
    
    \begin{subfigure}[b]{0.8\linewidth}
        \centering
        \includegraphics[width=0.45\linewidth]{images/residual/image/CIReVL_Recall5.png}
        \hfil
        \includegraphics[width=0.45\linewidth]{images/residual/image/pic2word_recall5.png}
        \caption{\textbf{PDV-I}: Impact of $\alpha$ scaling on composed image embeddings}
        \label{fig:residual_image_sub}
    \end{subfigure}
    
    \begin{subfigure}[b]{0.8\linewidth}
        \centering
        \includegraphics[width=0.45\linewidth]{images/residual/fusion/CIReVL_Recall5.png}
        \hfil
        \includegraphics[width=0.45\linewidth]{images/residual/fusion/pic2word_recall5.png}
        \caption{\textbf{PDV-F}: Impact of varying $\beta$ with on composed fused embeddings}
        \label{fig:residual_fusion_sub}
    \end{subfigure}
    \caption{Impact of changing $\alpha$/$\beta$ on Recall@5 performance across different PDV applications. For each row, results are shown for the CIReVL (left) and Pic2Word (right) baseline methods.}
    \label{fig:residual_all}
\end{figure*}

\section{Experiments} 
\label{sec:exp}
\noindent\textbf{Implementation Details.} We utilize the official implementations of four ZS-CIR baseline methods: CIReVL\footnote{https://github.com/ExplainableML/Vision\_by\_Language} and LDRE \footnote{https://github.com/yzy-bupt/LDRE} as representative caption-based feature extraction approaches and Pic2Word\footnote{https://github.com/google-research/composed\_image\_retrieval} and SEARLE\footnote{https://github.com/miccunifi/SEARLE} as representative pseudo tokenization-based methods. All feature extraction processes follow the original implementations provided by these baseline methods. However, to calculate $\Delta_{PDV}$, we need text embeddings without prompts, which are not provided in the original implementations. For CIReVL and LDRE, we obtain these embeddings by passing the generated image captions directly to CLIP. For Pic2Word and SEARL, we construct the base text embedding by passing the phrase ``a photo of $\langle$token$\rangle$" to CLIP, where $\langle$token$\rangle$ represents the extracted image token obtained via text inversion.

\noindent\textbf{Datasets and Base Vision-Language Models.} Following previous work, we evaluated our method on a suite of datasets including Fashion-IQ \cite{wu2021fashion}, CIRR \cite{liu2021image} and CIRCO \cite{baldrati2023zero}. Our proposed method is a plug-and-play approach requiring no additional training, leveraging only pre-trained models. For feature extraction, we use three CLIP variants: ViT-B/32, ViT-L/14, and ViT-G/14, and used the same pre-trained weights as used by the baseline methods. For image tokenization, we employ the pre-trained Pic2Word model. 

\subsection{Effect of Using the PDV}
We now explore the impact of the three proposed uses of the PDV: Using the PDV to augment text queries (PDV-T, see Sec. \ref{sec:exp1}), using the PDV to augment image queries (PDV-I, see Sec. \ref{sec:exp2}), and using the PDV in queries that fuse image and text data (PDV-F, see Sec. \ref{sec:exp3}).

\begin{table*}
	\footnotesize
	\centering
	\begin{tabular}{l|l|c|c|c|cccccccc}
		\hline
		\textbf{Fashion-IQ} & & & & & \multicolumn{2}{c}{\textbf{Shirt}} & \multicolumn{2}{c}{\textbf{Dress}} & \multicolumn{2}{c}{\textbf{Toptee}} & \multicolumn{2}{c}{\textbf{Average}} \\ \hline
		Backbone & Method& $\beta$ & $\alpha_{I}$& $\alpha_{T}$ & R@10 & R@50 & R@10 & R@50 & R@10 & R@50 & R@10 & R@50 \\
		\hline
		\multirow{6}{*}{ViT-B/32} %
		& SEARLE & - & - & - & 24.14 & 41.81 & 18.39 & 38.08 & 25.91 & 47.02 & 22.81 & 42.30 \\
		& SEARLE + \textbf{PDV-F} & 0.9 & 1.1 & 0.9 & \hli{24.83} & 41.71 & \hli{20.13} & \hli{41.40} & \hli{25.96} & \hli{47.17}  & \hli{23.64} & \hli{43.43} \\
		& CIReVL \textdagger &- & -& -& 28.36 & 47.84 & 25.29 & 46.36 & 31.21 & 53.85 & 28.29 & 49.35 \\
		& CIReVL + \textbf{PDV-F} & 0.75 & 1.4 & 1.4 & \hlb{32.88} & \hlb{52.80} & \hlb{32.67} & \hlb{54.49} & \hlb{38.91} & \hlb{61.81} & \hlb{34.82} & \hlb{56.37} \\
		& LDRE \textdagger & - & - & - & 27.38 & 46.27 & 19.97 & 41.84 & 27.07 & 48.78 & 24.81 & 45.63 \\
		& SEIZE \textdagger & - & - & - & \underline{29.38} & \underline{47.97} & \underline{25.37} & \underline{46.84} & \underline{32.07} & \underline{54.78} & \underline{28.94} & \underline{49.86} \\
		\hline
		\multirow{8}{*}{ViT-L/14} & Pic2Word & & & & 25.96 & 43.52 & 19.63 & 40.90 & 27.28 & 47.83 & 24.29 & 44.08 \\
		& Pic2Word + \textbf{PV-F} & 0.8 & 1.0 & 1.0 & \hli{28.21} & \hli{44.55} & \hli{20.92} & \hli{42.24} & \hli{29.02} & \hli{48.90}& \hli{26.05} & \hli{45.23}\\
		& SEARLE & - & - & - & 26.84 & 45.19 & 20.08 & 42.19 & 28.40 & 49.62 & 25.11 & 45.67 \\
		& SEARLE +\textbf{PDV-F} & 0.8 & 1.2 & 1.0 & \hli{28.66} & \hli{46.76} & \hli{23.60} & \hli{46.41} & \hli{31.00} & \hli{52.32} & \hli{27.75} & \hli{48.50} \\
		& CIReVL \textdagger & & & & 29.49 & 47.40 & 24.79 & 44.76 & 31.36 & 53.65 & 28.55 & 48.57 \\
		
		& CIReVL + \textbf{PDV-F} & 0.55 & 1 & 1.3 & \hlb{37.78} & \hlb{54.22} & \hlb{33.61} & \hlb{56.07} & \hlb{41.61} & \hlb{62.16} & \hlb{37.67} & \hlb{57.48} \\
		& LinCIR & - & - & - & 29.10 & 46.81 & 20.92 & 42.44 & 28.81 & 50.18 & 26.82 & 46.49 \\
        & SEIZE & -& -& -& \underline{33.04} & \underline{53.22} & \underline{30.93} & \underline{50.76} & \underline{35.57} & \underline{58.64} & \underline{33.18} & \underline{54.21} \\
		\hline
        \multirow{6}{*}{ViT-G/14} & Pic2Word  & - & - & - & 33.17 & 50.39 & 25.43 & 47.65 & 35.24 & 57.62 & 31.28 & 51.89\\
         & SEARLE  & - & - & - & 36.46 & 55.35 & 28.16 & 50.32 & 39.83 & 61.45 & 34.81 & 55.71\\
		  & CIReVL \textdagger & -& -& -& 33.71 & 51.42 & 27.07 & 49.53 & 35.80 & 56.14 & 32.19 & 52.36 \\
		& CIReVL + \textbf{PV-F} & 0.6 & 1.4 & 1.4 & \hli{41.90} & \hli{58.19} & \hlb{40.70} & \hlb{62.82} & \underline{\hli{48.09}}& \hli{67.77}& \underline{\hli{43.56}}& \hli{62.93}\\
        & LinCIR & - & - & - & \textbf{46.76} & \underline{65.11} & 38.08& 60.88& \textbf{50.48}& \underline{71.09}& \textbf{45.11} & \underline{65.69}\\
        & SEIZE & - & - & - & \underline{43.60} & \textbf{65.42}& \underline{39.61} & \underline{61.02} & 45.94& \textbf{71.12}& 43.05& \textbf{65.85}\\
		\hline
	\end{tabular}
	\caption{Average recall for different methods on Fashion-IQ validation dataset. \textdagger~denotes that numbers are taken from the original paper.}
	\label{tab:fashion_iq_results}
\end{table*}


\begin{table*}
	\centering
	\footnotesize
	\setlength{\tabcolsep}{4pt}
	\begin{tabular}{ll|c|c|c|cccc|cccc|ccc}
		\hline
		\multicolumn{2}{c|}{\textbf{Dataset}} & & & &  \multicolumn{4}{c|}{\textbf{CIRCO}} & \multicolumn{7}{c}{\textbf{CIRR}} \\
		\hline
		\multicolumn{2}{c|}{Metric} & & & & \multicolumn{4}{c|}{mAP@k} & \multicolumn{4}{c|}{Recall@k} &\multicolumn{3}{c}{$R_s$@k} \\
		\cline{3-16}
		Arch & Method & $\beta$ & $\alpha_I$ & $\alpha_T$ & k=5 & k=10 & k=25 & k=50 & k=1 & k=5 & k=10 & k=50 & k=1 & k=2 & k=3 \\
		\hline
		\multirow{8}{*}{ViT-B/32} 
		& PALAVRA\cite{cohen2022my} \textdagger & -& -& -& 4.61 & 5.32 & 6.33 & 6.80 & 16.62 & 43.49 & 58.51 & 83.95 & 41.61 & 65.30 & 80.94 \\
		& SEARLE \textdagger & -& -&- & 9.35 & 9.94 & 11.13 & 11.84 & 24.00 & 53.42 & 66.82 
		& 89.78 & 54.89 & 76.60 & 88.19 \\
		& SEARLE + \textbf{PDV-F} & 0.9 & 1.4 & 1.2 & \hli{9.99} & \hli{10.50}  & \hli{11.70} & \hli{12.40} & \hli{24.53} & \hli{53.71} & \hli{67.33} & \hli{89.81} & \hli{56.94} & \hli{78.05} & \hli{88.99} \\
		&CIReVL \textdagger & - & - & -& 14.94 & 15.42 & 17.00 & 17.82 & 23.94 & 52.51 & 66.00 & 86.95 & 60.17 & 80.05 & 90.19 \\
		& CIReVL + \textbf{PDV-F} & 0.75 & 1.4 & 1.2 & \hlb{19.90} & \hlb{20.61} & \hlb{22.64} & \hlb{23.52} & \hlb{33.25} & \hlb{64.15} & \hlb{75.23} & \hlb{92.43} & \hlb{65.81} &\underline{\hli{83.76}} &\underline{\hli{92.10}} \\
		& LDRE & -& -& -& 17.81 & 18.04 & 19.73 & 20.67 & 25.69 & 55.52 & 68.77 & 89.86 & 60.10 & 80.58 & 91.04 \\
		& LDRE + \textbf{PDV-F} & 0.75 & 1.4 & 1.4 & \hli{17.80} & \hli{18.78} & \hli{20.61} & \hli{21.56} & \underline{\hli{29.30}} & \underline{\hli{60.39}} & \underline{\hli{72.51}} & \underline{\hli{91.42}} & \hli{63.06} & \hli{82.36} & \hli{91.54} \\
        & SEIZE & -&- &- & \underline{19.04} & \underline{19.64} & \underline{21.55}& \underline{22.49}& 27.47 & 57.42& 70.17 & - & \underline{65.59} & \textbf{84.48}& \textbf{92.77} \\
 		\hline
		\multirow{10}{*}{ViT-L/14}
		& Pic2Word & -& -& -& 6.81 & 7.49 & 8.51 & 9.07 & 23.69 & 51.32 & 63.66 & 86.21 & 53.61 & 74.34 & 87.28 \\
		& Pic2Word + \textbf{PDV-F} & 0.85 & 1.2 & 1.0 & \hli{7.74} &  \hli{8.67} & \hli{9.77} & \hli{10.37} & \hli{23.90} & \hli{51.95} & \hli{64.63} & \hli{87.04} & \hli{53.16}  & \hli{74.07} & \hli{87.08}\\
		& SEARLE \textdagger & - & - & - & 11.68 & 12.73 & 14.33 & 15.12 & 24.24 & 52.48 & 66.29 & 88.84 & 53.76 & 75.01 & 88.19 \\
		& SEARLE + \textbf{PDV-F} & 0.85 & 1.4 & 1.2 & \hli{12.58} & \hli{13.57} & \hli{15.30} & \hli{16.07} & \hli{25.64} & \hli{53.61} & \hli{66.58} & \hli{88.55} & \hli{55.83} & \hli{76.48} & \hli{88.53} \\
		& CIReVL \textdagger & -& -& -& 18.57 & 19.01 & 20.89 & 21.80 & 24.55 & 52.31 & 64.92 & 86.34 & 59.54 & 79.88 & 89.69 \\
		& CIReVL + \textbf{PDV-F} & 0.75 & 1.4 & 1.2 & \hlb{25.67} & \hlb{26.61} & \underline{\hli{28.81}} & \hlb{29.95} & \hlb{36.24} & \hlb{66.17} & \hlb{76.96} & \hlb{92.29} & \hlb{68.07} & \hlb{85.35} & \hlb{93.47} \\
		& LDRE & -& -& -& 22.32 & 23.75 & 25.97 & 27.03 & 26.68 &55.45  & 67.49 & 88.65 & 60.39 & 80.53 & 90.15 \\
		& LDRE + \textbf{PDV-F} & 0.75 & 1.4 & 1.4 & \hli{25.23} & \hli{26.52} & \hlb{28.94} & \hlb{29.95} & \underline{\hli{30.16}} & \underline{\hli{59.98}} & \underline{\hli{71.90}} & \underline{\hli{90.87}} & \hli{63.66} & \hli{82.87} & \hli{91.57} \\

        & LinCIR & - & - & - &12.59 &13.58 &15.00 &15.85 &25.04 &53.25 &66.68 & - &57.11 &77.37 &88.89\\
        & SEIZE & -& -& -& 24.98 & 25.82 &28.24 &\underline{29.35}& 28.65 &57.16& 69.23& - &\underline{66.22} &\underline{84.05} &\underline{92.34} \\
        

        
		\hline
		\multirow{7}{*}{ViT-G/14} & CIReVL \textdagger & -& -& -& 26.77 & 27.59 & 29.96 & 31.03 & 34.65 & 64.29 & 75.06 & 91.66 & 67.95 & 84.87 & 93.21 \\

		& CIReVL + \textbf{PDV-F} & 0.75 & 1.4 & 1.2 & \hli{30.02} & \hli{31.46} & \hli{34.01} & \hli{35.08} & \hli{38.15} &\hli{67.93} & \hli{77.90} & \hli{92.77} & \hli{69.37} & \hli{85.37} & \hli{93.45}  \\
		
		& LDRE & -& -& -& \underline{33.30} & \underline{34.32} & \underline{37.17} & \underline{38.27} & 37.40 & 66.96 & 78.17 & 93.66 & 68.84 & 85.64 & 93.90 \\
		& LDRE + \textbf{PDV-F} & 0.75 & 1.4 & 1.4 & \hlb{34.88} & \hlb{36.41} & \hlb{39.12} & \hlb{40.23} & \hlb{42.51} & \hlb{72.22} & \hlb{81.71} & \hlb{94.94} & \underline{\hli{72.39}} & \underline{\hli{88.34}} & \underline{\hli{94.80}} \\
        & SEARLE & - & - & - & 13.20 &13.85 &15.32 &16.04 & 34.80 & 64.07 & 75.11 &-&68.72 &84.70 &93.23 \\
        & LinCIR & - & - & - & 19.71 &21.01 &23.13 &24.18 &35.25 &64.72 &76.05 & - &63.35 &82.22 &91.98 \\
        & SEIZE & -& -& -& 32.46 & 33.77 &36.46 &37.55 &\underline{38.87} & \underline{69.42} & \underline{79.42} & -&\textbf{74.15} & \textbf{89.23} & \textbf{95.71} \\
		\hline
	\end{tabular}
	\caption{Performance comparison on CIRCO and CIRR test datasets. As in previous works, for CIRCO, mAP@k is reported, while for CIRR both Recall@k and $R_s$@k metrics are used. \textdagger~denotes that numbers are taken from the original paper.}
	\label{tab:circo_cirr_results}
\end{table*}

\noindent{\textbf{Analysing the PDV for Text (PDV-T)}}
\label{sec:exp1}
To investigate how scaling the prompt vector, $\Delta_{PDV}$, affects retrieval performance with composed text embeddings, we conducted experiments using two zero-shot approaches (CIReVL and Pic2Word) with different backbone networks across three datasets. We evaluated the performance by varying the scaling parameter, $\alpha$ (Eq. \ref{eqn:text_embedding}), from -0.5 to 3 by an interval of 0.1.

The results are presented in Figure \ref{fig:residual_text_sub}. To account for scale variations across different experiments, we report relative recall values, where a baseline of zero is established at $\alpha=1$. As shown in Figure \ref{fig:residual_text_sub}, varying $\alpha$ leads to significant changes in relative recall performance\footnote{See supplementary material for Recall@10 and Recall@50 figures}. Our analysis reveals method-specific patterns across datasets. With CIReVL, increasing $\alpha$ improves relative recall on both FashionIQ and CIRCO datasets. In contrast, Pic2Word shows no significant improvement on FashionIQ and CIRR when varying $\alpha$, while CIRCO's performance improves when $\alpha$ is reduced to 0.8-1.0. This divergent behavior is fundamentally linked to each method's ability to generate an accurate $\Delta_{PDV}$. As demonstrated in Tables \ref{tab:fashion_iq_results} and \ref{tab:circo_cirr_results}, CIReVL consistently outperforms Pic2Word across various benchmarks, indicating its superior ability to generate a more accuraute composed query, and thus a more accurate $\Delta_{PDV}$. Consequently, increasing $\alpha$ yields greater benefits for CIReVL compared to Pic2Word.

We visualize the top-5 retrieval results using CIReVL with a ViT-B-32 backbone across three datasets (one reference image from each) under varying $\alpha$ values, as shown in Figure \ref{fig:residual_qual}\red{a}. As $\alpha$ increases, the retrieved results show stronger alignment with the prompt. Conversely, when $\alpha$ exceeds 1, the results include semantically related but unseen variations, while $\alpha$ values below 0.5 yields results opposite to the prompt's intent. For instance, ``brighter blue and sleeveless" retrieves ``dark blue with sleeves," ``plain background" yields ``natural/dark background," and ``young boy" returns ``adult" images.





\noindent{\textbf{Analysing the PDV for Image (PDV-I)}}
\label{sec:exp2}
To evaluate whether $\Delta_{PDV}$ enhances the retrieval performance of image embeddings, we conducted experiments following the protocol described in Section~\ref{sec:exp1}. We modified image embeddings by adding $\Delta_{PDV}$ scaled with $\alpha$ values ranging from -0.5 to 2.0, where $\alpha=0$ represents the original image-only embeddings. As shown in Figure \ref{fig:residual_image_sub}, Recall@K exhibits a positive correlation with $\alpha$ for values below 1. This upward trend continues until $\alpha=2.0$ for CIReVL, while Pic2Word's performance peaks when $\alpha$ reaches 1.4.  The performance of PDV-I was evaluated on the CIRR and CIRCO datasets by comparing it with other visual embedding-based methods, as detailed in Table \ref{tab:circo_cirr_results_pdv-I}. The results reveal that PDV-I achieved marginal improvements over existing approaches.

Following the methodology in Section~\ref{sec:exp1}, we conduct similar visualizations, with results shown in Figure \ref{fig:residual_qual}\red{b}. As with PDV-T, increasing $\alpha$ leads to stronger alignment between retrieved results and the prompt. When $\alpha$ exceeds 0.5, the results exhibit semantic relationships to the query, while $\alpha$ values below 0.5 yield results opposing the prompt's intent.
Notably, PDV-I's top retrievals demonstrate higher visual similarity to reference images compared to PDV-F, as evidenced by the preserved design elements in the clothing item (left) and laptop (middle). This characteristic is particularly valuable for applications include fashion search \cite{wu2021fashion} and logo retrieval \cite{tursun2019component}, where visual similarity plays a crucial role.



\begin{figure*}[!tbh]
	\centering
	\includegraphics[width=0.825\linewidth]{images/qualitative/PV_qual_all_mini.pdf}
	\caption{Visualisation of the impact of $\alpha$/$\beta$ scaling on top-5 retrieval results. CIReVL with ViT-B-32 Clip model is the baseline method used. Representative examples with prompts from three datasets: FashionIQ (left), CIRR (middle), and CIRCO (right) are shown at the top. \textbf{\textcolor{boxgreen}{Green}} and \textbf{\textcolor{boxblue}{blue}} bounding boxes indicate true positives and near-true positives, respectively.}
	\label{fig:residual_qual}
	
\end{figure*}

\noindent{\textbf{Analysing PDV Fusion (PDV-F)}}
\label{sec:exp3}
Finally, we evaluate the effectiveness of fusing image and text-composed embeddings by varying the fusion parameter, $\beta$, from 0 to 1 while maintaining $\alpha=1$
for both PDV-I and PDV-F. At $\beta=0$, the model relies solely on composed image embeddings, while at $\beta=1$, it uses only composed text embeddings. As shown in Figure \ref{fig:residual_fusion_sub}, the fusion of both embeddings consistently outperforms using either embedding type alone. Optimal retrieval performance is typically achieved when $\beta$ is between 0.4 and 0.8.

We similarly visualize the top-5 retrieved results across different $\beta$ values. As shown in Figure \ref{fig:residual_qual}\red{c}, when $\beta$ is small, the retrieved results maintain high visual similarity to the reference image. Conversely, as $\beta$ exceeds 0.5, the results demonstrate stronger semantic alignment with the prompt.



\subsection{ZS-CIR Benchmark Comparison}






\begin{table*}
	\centering
	\footnotesize
	\setlength{\tabcolsep}{4pt}
	\begin{tabular}{l|l|c|cccc|cccc|ccc}
		\hline
		\multicolumn{2}{c|}{\textbf{Dataset}} & & \multicolumn{4}{c|}{\textbf{CIRCO}} & \multicolumn{7}{c}{\textbf{CIRR}} \\
		\hline
		& Metric & & \multicolumn{4}{c|}{mAP@k} & \multicolumn{4}{c|}{Recall@k} & \multicolumn{3}{c}{$R_s$@k} \\
		\cline{2-14}
		Arch & Method & $\alpha_I$ & k=5 & k=10 & k=25 & k=50 & k=1 & k=5 & k=10 & k=50 & k=1 & k=2 & k=3 \\
		\hline
		\multirow{6}{*}{ViT-B/32} 
		& Image-only \textdagger & - & 1.34 & 1.60 & 2.12 & 2.41 & 6.89 & 22.99 & 33.68 & 59.23 & 21.04 & 41.04 & 60.31 \\
		& Text-only \textdagger & - & 2.56 & 2.67 & 2.98 & 3.18 & 21.81 & 45.22 & 57.42 & 81.01 & 62.24 & 81.13 & 90.70 \\
		& Image + Text \textdagger & - & 2.65 & 3.25 & 4.14 & 4.54 & 11.71 & 35.06 & 48.94 & 77.49 & 32.77 & 56.89 & 74.96 \\
		& SEARLE + \textbf{PDV-I} & 1.5 & 4.77 & 5.23  & 6.31 & 6.82 & 16.65 & 42.53 & 55.16 & 81.42 & 44.68 & 67.78 & 82.94\\
		& CIReVL + \textbf{PDV-I} & 2.0 & \textbf{10.29 }& \textbf{10.80} & \textbf{12.23} & \textbf{12.93} & \textbf{27.18} & \textbf{56.53} & \textbf{67.76} & \textbf{87.64} & \textbf{59.81} & \textbf{79.59} & \textbf{90.15}\\
		& LDRE + \textbf{PDV-I} & 2.0 & 8.00 & 8.88 & 10.06 & 10.72 & 23.37 & 51.21 & 63.69 & 85.57 & 55.57 & 76.63 & 88.15\\
		\hline
	\end{tabular}
	\caption{PDV-I performance on CIRCO and CIRR test datasets. Note that the image-only approach utilizes the visual embedding of the reference image, whereas the text-only approach employs the text embedding of the prompt.}
	\label{tab:circo_cirr_results_pdv-I}
\end{table*}

We evaluated PDV-F alongside four baseline approaches (CIReVL, LDRE, Pic2Word, and SEARLE) across three benchmarks. Notably, CIReVL was tested with three different backbones on three datasets, as its models and intermediate results are publicly available. However, for the remaining methods, we conducted partial evaluations due to limited open-source availability or restricted support.

The numerical results are presented in Tables \ref{tab:fashion_iq_results} and \ref{tab:circo_cirr_results}.
On the FashionIQ benchmark, PDV-F yields substantial improvements for all baseline approaches, with CIReVL showing particularly strong gains that scale with backbone size. Similarly, all methods demonstrate significant performance improvements on CIRCO and CIRR datasets. Notably, CIReVL achieves larger improvements compared to other methods, with the most substantial gains observed when using small and medium backbone architectures. Our PDV-F implementation within the CIReVL framework consistently outperformed other state-of-the-art methods, including LinCIR and SEIZE, across most evaluation metrics. Similar to SEIZE, PDV-F offers the advantage of being entirely training-free; however, unlike SEIZE, it does not significantly increase feature extraction computational costs. While LinCIR demonstrates exceptional inference speed, it lacks the training-free nature of our approach, requiring dedicated model training before deployment.  






\section{Experimental Results}


In this section, we systematically evaluate current LLMs' performance on our \name.



\begin{table*}[t]
    \centering
    \small
    \setlength{\tabcolsep}{3pt}
    \begin{tabular}{l|ccc|ccc|ccc | cc} 
        \toprule
        \multirow{2}{*}{} 
        & \multicolumn{3}{c|}{\textbf{EU AI Act}} & \multicolumn{3}{c|}{\textbf{GDPR}} & \multicolumn{3}{c|}{\textbf{HIPAA}} & \multicolumn{2}{c}{\textbf{ACLU}}\\ 
        %\cmidrule(lr){2-4} \cmidrule(lr){5-7} \cmidrule(lr){8-10}
        \textbf{Model}  & DP & CoT & RAG & DP & CoT & RAG & DP & CoT & RAG & DP & CoT \\ 
        \midrule
        Mistral-7B-Instruct & 49.83 & 43.50 & 45.56 & 72.29 & 68.02 & 43.38 & 45.79 & 60.74 & 64.95 & 44.92 & \textbf{72.46}\\
        Qwen-2.5-7B-Instruct & 49.90 & 65.30 & \textbf{55.83} & 89.00 & 88.81 & 82.43 & 68.69 & 72.43 & 71.49 & 50.72 & 52.17 \\
        Llama-3.1-8B-Instruct & 61.30 & 59.40 & 53.50 & 85.30 & \textbf{90.27} & \textbf{76.60} & 77.57 & 85.51 & \textbf{88.31}  & 66.17 & 66.67\\
        GPT-4o-mini & 73.76 & 66.60 & - & \textbf{92.03} & 65.69 & - & 80.84 & 67.75 & - & \textbf{69.56} & 31.88\\
        QwQ-32B & \textbf{78.22} & \textbf{75.30} & - & 80.45 & 90.08 & - & 70.09 & \textbf{88.31} & - &  55.07& 55.07 \\

        \multirow{1}{*}{Deepseek R1 (671B)} & 72.90 & 60.67 & - & 90.66 & 47.88 & - & \textbf{89.25} & 81.77 & -& 65.21 & 59.42 \\
         
    %     \midrule
    % \multirow{1}{*}{Average} & 64.31 & 61.79 & xx.xx & 84.95 & 75.12 & xx.xx & 72.03 & 76.08 & xx.xx & 59.33 & 56.76 \\
        \bottomrule
    \end{tabular}
    \vspace{-0.1in}
    \caption{Accuracy Evaluation results of the legal compliance task. All results are reported in \%.}
    \label{tab:privacy_result}
    \vspace{-0.1in}
\end{table*}








\begin{table*}[t]
    \centering
    \small
    \setlength{\tabcolsep}{5pt}
    \begin{tabular}{l|ccc|ccc|ccc}
        \toprule
        \multirow{2}{*}{} 
         & \multicolumn{3}{c|}{\textbf{Permit}} & \multicolumn{3}{c|}{\textbf{Prohibit}} & \multicolumn{3}{c}{\textbf{Not Applicable}} \\
        %\cmidrule(lr){2-4} \cmidrule(lr){5-7} \cmidrule(lr){8-10}
        \textbf{Model\&Method} & Precision & Recall & F1 & Precision & Recall & F1 & Precision & Recall & F1 \\
        \midrule
        Qwen2.5-7B-Instruct-DP & 36.17  &  55.30  &  43.74 & 68.83  &  87.54  &  77.06 & 40.62   & 7.80  &  13.09 \\
        Qwen2.5-7B-Instruct-CoT & 52.93    &51.80    &52.36 &68.06  &  85.58   & 75.82  & 77.37   & 59.50    &67.27  \\
        Qwen2.5-7B-Instruct-RAG & 49.63  &  51.99  &  50.78  &  70.45  &  54.99  &  61.77 & 73.69  &  60.50  &  66.45  \\
        Mistral-7B-Instruct-DP & 83.33  &  0.49  &  0.97 & 73.50  & 50.57  &  59.91 & 42.97  &  99.90  &  60.09  \\
        Mistral-7B-Instruct-CoT & 52.83  &  2.72  &  5.18  & 80.23   &  28.84   & 42.42  &  40.74   &  99.70   &  57.85  \\
        Mistral-7B-Instruct-RAG & 46.55  &  7.87  &  13.47 &  81.95  &  29.45  &  43.33  &  42.86   & 100.00  &  60.01  \\
        
        % \midrule
        % Average & 53.74 & 28.03 & 27.08 & 73.50 & 56.33 & 59.72 & 53.38 & 71.23 & 54.46 \\
        \bottomrule
    \end{tabular}
    \vspace{-0.1in}
    \caption{The detailed investigation of Qwen2.5-7B-Instruct and Mistral-7B-Instruct models performance over 3 classes on the AI Act cases. All results are reported in \%.}
    \label{tab:compliance_detail}
    \vspace{-0.2in}
\end{table*}


\subsection{Evaluation on Legal Compliance}
To study whether LLMs can comply with existing privacy regulations, we prompt these LLMs with our collected cases.
Table~\ref{tab:privacy_result} evaluates LLMs' legal compliance accuracies over the four domains.
The compliance results suggest the following findings.

%%% TO DO LIST
%%% 1. EU AI Act ANALYSIS, why it is so bad for LLMs
%%% 2. CoT and RAG not working on AI Act, GDPR?
%%% 3. Parsing Errors analysis or Not Relevant Analysis?
%%% 
1) \textit{The collected EU AI Act and ACLU subsets are the most challenging subsets for legal compliance. }
As outlined in Section~\ref{sec: ai act}, cases from the EU AI Act are synthesized according to its official compliance checker.
Therefore, these cases are not likely to be accessed by LLMs and LLMs can only use their reasoning abilities to determine compliance.
We further investigate the precision, recall and F1 scores for LLMs' predictions over each class on Table~\ref{tab:compliance_detail}.
Both LLMs underperform in the permitted cases.
%For example, Mistral-7B-Instruct has recall scores of no more than 8\% on permitted cases while nearly 100\% on the not-applicable cases, which implies that it classify most the permitted cases as not applicable cases.
For instance, Mistral-7B-Instruct has recall scores of no more than 8\% on permitted cases, while getting nearly 100\% on not-applicable cases.
The results suggest that LLMs cannot distinguish between permitted and not applicable cases.
Regarding the ACLU cases, they always connect with a wide range of legal regulations, including the Fourth Amendment to the United States Constitution and the Freedom of Information Act.
The ACLU data demand a more comprehensive understanding of their applicable regulations, and compliance is harder to determine.
Consequently, even the best-performing reasoner models (QwQ-32B and Deepseek R1) fail to attain satisfactory results on the two subsets.
%These results suggest that current LLMs


2) \textit{Chain-of-Thought reasoning and naive RAG implementation may not always help improve LLMs' safety and privacy compliance.}
For CoT prompting, its effectiveness is model-specific.
Our evaluation of instruction-tuned LLMs, including Mistral-7B, Qwen-2.5-7B and Llama-3.1-8B, reveals general accuracy improvements compared to direct prompting (DP).
However, this trend does not hold for all models.
Specifically, GPT-4o-mini and Deepseek R1 reasoner exhibit degraded performance when using CoT prompting.
On the other hand, the performance of our implemented naive retrieval augmented generation (RAG) method is domain-specific.
For the HIPAA domain, RAG generally leads to the best performance, which aligns with findings from prior research ~\cite{li-2024-privacychecklist}.
However, this improvement fails to extend to the EU AI Act and GDPR domains, where RAG results in notable drops in accuracy.
%\tbc{We further give a detailed analysis in xxx.}




% \begin{table*}[htbp]
%     \centering
%     \small
%     \setlength{\tabcolsep}{3pt} 
%     \begin{tabular}{llccc|ccc|ccc|ccc|ccc}
%         \toprule
%         \textbf{} & \textbf{} & \multicolumn{3}{c|}{\textbf{Mistral-7B-Instruct}} & \multicolumn{3}{c|}{\textbf{Qwen-2.5-7B-Instruct}} & \multicolumn{3}{c|}{\textbf{Llama-3.1-8B-Instruct}} & \multicolumn{3}{c|}{\textbf{GPT-4o-mini}} & \multicolumn{3}{c}{\textbf{QwQ-32B}} \\ 
%         \cmidrule(lr){3-5} \cmidrule(lr){6-8} \cmidrule(lr){9-11} \cmidrule(lr){12-14} \cmidrule(lr){15-17}
%         \textbf{Dataset} & \textbf{} & Easy & Medium & Hard & Easy & Medium & Hard & Easy & Medium & Hard & Easy & Medium & Hard & Easy & Medium & Hard \\ 
%         \midrule
%         EU AI Act & & 81.01 & 69.86 & 50.13 & 91.84 & 83.50 & 57.01 & 80.56 & 66.61 & 50.20 & 96.59 & 87.07 & 59.21 & 91.26 & 82.80 & 57.17 \\
%         GDPR & & 85.54 & 75.92 & 55.99 & 93.61 & 87.78 & 63.86 & 85.22 & 75.17 & 57.81 & 97.11 & 94.34 & 75.84 & 96.07 & 93.01 & 75.52 \\
%         HIPAA & & 85.81 & 76.26 & 56.35 & 93.72 & 87.95 & 64.22 & 85.53 & 75.59 & 58.27 & 97.17 & 94.46 & 76.11 & 98.28 & 94.68 & 78.80 \\
%         \bottomrule
%     \end{tabular}
%     \caption{Evaluation results for Easy, Medium, and Hard categories.}
%     \label{tab:mcq_results_split}
% \end{table*}


% \begin{table*}[htbp]
%     \centering
%     \small
%     \setlength{\tabcolsep}{3pt}
%     \begin{tabular}{l|cccc|cccc|cccc} 
%         \toprule
%         \multirow{2}{*}{} 
%         & \multicolumn{4}{c|}{\textbf{EU AI Act}} & \multicolumn{4}{c|}{\textbf{GDPR}} & \multicolumn{4}{c}{\textbf{HIPAA}} \\ 
%         %\cmidrule(lr){2-4} \cmidrule(lr){5-7} \cmidrule(lr){8-10}
%         \textbf{Model}  & Easy & Medium & Hard & Avg & Easy & Medium & Hard & Avg & Easy & Medium & Hard & Avg \\ 
%         \midrule
%         Mistral-7B-Instruct & 81.01 & 69.86 & 50.13 & 85.54 & 75.92 & 55.99 & 85.81 & 76.26 & 56.35 \\
%         Qwen-2.5-7B-Instruct & 91.84 & 83.50 & 57.01 & 93.61 & 87.78 & 63.86 & 93.72 & 87.95 & 64.22 \\
%         Llama-3.1-8B-Instruct & 80.56 & 66.61 & 50.20 & 85.22 & 75.17 & 57.81 & 85.53 & 75.59 & 58.27 \\
%         GPT-4o-mini & 96.59 & 87.07 & 59.21 & 97.11 & 94.34 & 75.84 & 97.17 & 94.46 & 76.11 \\
%         QwQ-32B & 91.26 & 82.80 & 57.17 & 96.07 & 93.01 & 75.52 & 98.28 & 94.68 & 78.80 \\
%         \midrule
%     \multirow{1}{*}{Average} & 88.25 & 77.97 & 54.74 & 91.51 & 85.24 & 65.80 & 92.10 & 85.79 & 66.75 \\
%         \bottomrule
%     \end{tabular}
%     \caption{Evaluation results for Easy, Medium, and Hard categories.}
%     \label{tab:mcq_results_split}
% \end{table*}


\begin{table*}[t]
    \centering
    \small
    \setlength{\tabcolsep}{3pt}
    \begin{tabular}{l|cccc|cccc|cccc} 
        \toprule
        \multirow{2}{*}{} 
        & \multicolumn{4}{c|}{\textbf{EU AI Act}} & \multicolumn{4}{c|}{\textbf{GDPR}} & \multicolumn{4}{c}{\textbf{HIPAA}} \\ 
        %\cmidrule(lr){2-4} \cmidrule(lr){5-7} \cmidrule(lr){8-10}
        \textbf{Model}  & Easy & Medium & Hard & Avg & Easy & Medium & Hard & Avg & Easy & Medium & Hard & Avg \\ 
        \midrule
        Mistral-7B-Instruct & 81.01 & 69.86 & 50.13 & 67.00 & 85.54 & 75.92 & 55.99 & 72.48 & 85.81 & 76.26 & 56.35 & 72.81 \\
        Qwen-2.5-7B-Instruct & 91.84 & 83.50 & 57.01 & 77.45 & 93.61 & 87.78 & 63.86 & 81.75 & 93.72 & 87.95 & 64.22 & 81.96 \\
        Llama-3.1-8B-Instruct & 80.56 & 66.61 & 50.20 & 65.79 & 85.22 & 75.17 & 57.81 & 72.73 & 85.53 & 75.59 & 58.27 & 73.13 \\
        GPT-4o-mini & 96.59 & 87.07 & 59.21 & 80.96 & 97.11 & 94.34 & 75.84 & 89.09 & 97.17 & 94.46 & 76.11 & 89.25 \\
        QwQ-32B & 91.26 & 82.80 & 57.17 & 77.08 & 96.07 & 93.01 & 75.52 & 88.20 & 98.28 & 94.68 & 78.80 & 90.59 \\
        \midrule
    \multirow{1}{*}{Average} & 88.25 & 77.97 & 54.74 & 73.65 & 91.51 & 85.24 & 65.80 & 80.85 & 92.10 & 85.79 & 66.75 & 81.55 \\
        \bottomrule
    \end{tabular}
    \vspace{-0.1in}
    \caption{Accuracy Evaluation results of the context understanding task. All results are reported in \%.}
    \label{tab:mcq_results_split}
    \vspace{-0.1in}
\end{table*}
\subsection{Evaluation on Context Understanding}

Besides evaluating the overall performance on the compliance task, we also convert the parsed structured cases into multiple-choice questions as stated in Section~\ref{sec: data processing} with 3 difficulty levels for the EU AI Act, GDPR, and HIPAA domain.
These questions enable us to probe how well LLMs are able to understand the context and identify the key CI parameters inside its information flows.
Table~\ref{tab:mcq_results_split} shows LLMs' performance over these multiple-choice questions.
The results of the context understanding task imply the following findings.


3) \textit{Existing LLMs can explicitly identify the CI parameters of the information flow inside the given context.}
For prompted multiple-choice questions, LLMs, on average, can reach accuracies of approximately \textasciitilde 90\% on the Easy subset, \textasciitilde 80\% on the Medium subset, and \textasciitilde 60\% on the Hard subset.
The high accuracy suggests that LLMs are well aware of the context and its key characteristics inside the context's information flow.


%% qwq vs qwen
4) \textit{LLMs' reasoning enhanced by reinforcement learning further improves the context understanding abilities.}
When comparing Qwen-2.5-7B-Instruct with Qwen's latest QwQ-32B reasoner model, Qwen's QwQ-32B has higher accuracy over most subsets, especially on the hard questions.
The result indicates that reinforcement learning helps LLMs to better understand and analyze the context.
Consequently, better context-understanding abilities further improve legal compliance, as indicated by the results of Table~\ref{tab:privacy_result}.

5) \textit{The context of EU AI Act subset is challenging for LLMs to understand.}
On average, all LLMs have comparable performance across the Easy, Medium, and Hard subsets of the GDPR and HIPAA domains.
However, their accuracies on the EU AI Act subset fall significantly behind the other two domains.
We manually examine samples within the EU AI Act and observe that their parsed roles of CI parameters are mostly abstract legal terms such as ``Law Enforcement Agencies,'' ``Importer,'' ``Operator'' and ``provider.'' 
These terms make it hard to correctly identify the stakeholders for LLMs.
In addition, compared with real cases, the AI Act's synthetic vignettes also lack narrative coherence for describing the information flows.
Hence, LLMs struggle to perform well on the multiple-choice questions of the AI Act domain.
As a result, LLMs' compliance also degrades.
%The context understanding results on the EU AI Act data suggest that it is the most challenging subset and partially explains why LLMs underperform on the EU AI Act's legal compliance task.
%This manual inspection partially explains why LLMs underperform on the legal compliance tasks associated with the EU AI Act.


%% \begin{table*}[t]
%     \centering
%     \caption{}
%     % 子表1:Location Data
%     \begin{subtable}[h]{0.46\linewidth}
%     \centering
%     \resizebox{\linewidth}{!}{
%         \begin{tabular}{c|c|cccc}
%         \toprule
%         \textbf{Location Method} & \textbf{Model} & \textbf{harm} & \textbf{sorry} & \textbf{gsm8k} & \textbf{math} \\ \midrule
%         random                &                &              &                &               &              \\ 
%         sparsegpt             & LM+MATH       &              &                &               &              \\ 
%         Importance score      &                &              &                &               &              \\ 
%         wandg                 &                & 16.00        & 24.22          & 50.34         & 14.20        \\ \bottomrule
%         \end{tabular}
%     }
%     \label{tab:location}
%     \caption{}
%     \end{subtable}
%     \hfill
%     % 子表2:Election Data
%     \begin{subtable}[h]{0.46\linewidth}
%     \centering
%     \resizebox{\linewidth}{!}{
%         \begin{tabular}{c|c|cccc}
%         \toprule
%         \textbf{Election type} & \textbf{Model} & \textbf{harm} & \textbf{sorry} & \textbf{gsm8k} & \textbf{math} \\ \midrule
%         00                    &                &              &                &               &              \\ 
%         01                    & LM+MATH       &              &                &               &              \\ 
%         10                    &                &              &                &               &              \\ 
%         11                    &                & 16.00        & 24.22          & 50.34         & 14.20        \\ \bottomrule
%         \end{tabular}
%     }
%     \label{tab:election}
%     \caption{}
%     \end{subtable}
% \end{table*}


% % % 子表3:Ablation of Disjoint Data
% \begin{table}[h]
% \centering
% \resizebox{\linewidth}{!}{
% \begin{tabular}{c|c|cccc}
% \toprule
% \textbf{Disjoint} & \textbf{Model} & \textbf{harm} & \textbf{sorry} & \textbf{gsm8k} & \textbf{math} \\ \midrule
% w/o Disjoint          & LM+MATH       &              &                &               &              \\
% w/ Disjoint           &                & 16.00        & 24.22          & 50.34         & 14.20        \\ \bottomrule
% \end{tabular}
% }
% \caption{Ablation of Disjoint Data}
% \label{tab:ablation_disjoint_data}
% \end{table}



\begin{table}[!ht]
\centering
\caption{Ablation Study. Experiments are conducted on Mistral-7B series models. $\ast$ represents LLM's instruction following ability is impaired.}
\resizebox{\linewidth}{!}{
\begin{tabular}{c|c|cc|cc}
\toprule
\multirow{2}{*}{\begin{tabular}[c]{@{}c@{}}\textbf{Ablation} \\ \textbf{Part}\end{tabular}} & \multirow{2}{*}{\begin{tabular}[c]{@{}c@{}}\textbf{Alternative} \\ \textbf{Methods}\end{tabular}} & \multicolumn{2}{c|}{\begin{tabular}[c]{@{}c@{}}\textbf{Safety}\end{tabular}} & 
\multicolumn{2}{c}{\begin{tabular}[c]{@{}c@{}}\textbf{Mathematical} \\ \textbf{Reasoning}\end{tabular}} \\ \cmidrule{3-6}
&                         &                             \textbf{HarmBench}$\downarrow$                                                     & \textbf{SORRY-Bench}$\downarrow$                                                       & \textbf{GSM8K}$\uparrow$                                          & \textbf{MATH}$\uparrow$                                   \\ \midrule

\multirow{3}{*}{Location} & Random                   & $\ast$             & $\ast$              & 25.58              & 8.66             \\ 
                         & Wanda       & $\ast$             & $\ast$               & 39.58              & 11.37             \\  
                        & SNIP               & 16.00        & 24.22          & 50.34         & 14.20        \\ \midrule

\multirow{3}{*}{Election} 
                   & 01       &  58.00            & 83.77               & 54.13              & 13.12             \\ 
                   & 10                & 35.25             & 47.33               & 50.64              & 13.30             \\ 
                    & 11               & 16.00        & 24.22          & 50.34         & 14.20        \\ \midrule

\multirow{2}{*}{Disjoint}          & \textcolor{gray}{\usym{2717}}      & 63.00             & 85.33               & 72.93              & 23.18             \\
           & \textcolor{gray}{\usym{2714}}               & 16.00        & 24.22          & 50.34         & 14.20        \\ \bottomrule
\end{tabular}
}
\label{tab:ablation}
%\vspace{-10pt}
\end{table}
\begin{figure*}[t]
\centering
\includegraphics[width=0.999\textwidth]{figs/ablations.pdf}
\vspace{-0.3in}
\caption{
Ablation studies for the legal compliance task. All results are evaluated in \%.
}
\label{fig:ablations}
\vspace{-0.15in}
\end{figure*}
%%% case study?
\subsection{Ablation Studies}

To study the effectiveness of our annotated CI parameters and applicable regulation content, we further perform ablation studies by feeding LLMs with ground truth CI parameters and regulations as stated in Section~\ref{sec: judge}.

Figure~\ref{fig:ablations} presents the accuracies of DP+CI and DP+CI+LAW across various LLMs for the legal compliance task.
By comparing DP+CI with CI, we observe that appending the contextual integrity parameters significantly improves LLMs' accuracies, particularly in the HIPAA and ACLU domains. 
Such results suggest that CI parameters indeed help LLMs better understand the context and improve legal compliance performance.
Furthermore,  for DP+CI+LAW, we augment the applicable regulations to DP+CI and obtain consistent performance gains.
Consequently, DP+CI+LAW has the best performance compared with our implemented DP, CoT, and RAG methods.
The results of DP+CI+LAW highlight the effectiveness of retrieval augmented generation methods, provided that the retrieved documents are both relevant and applicable.
%These results further suggest that naive RAG implementation may not help improve LLMs' compliance due to wrong retrieval results, implying that there are gaps between common context and legal terminologies.
Moreover, our ablation studies also imply that naive RAG implementations may degrade LLMs' compliance when the retrieval step yields irrelevant results. 
Such retrieval failures disclose a discrepancy between general context and domain-specific legal terminologies, which suggests that our \name requires a tailored retrieval module for improvement.

%suggesting that careful curation and relevance of retrieved documents are essential for improving model performance in legal applications.

%\subsection{Case Studies}


\subsection{Human Evaluations}
\label{subsec:human_eval}
%% CI para inspection
To assess whether our parsed CI parameters and judgments are reliable, three authors manually inspect the data quality.
This inspection calculates annotators' agreement with the parsed roles and associated attributes (Role), the transmission principle (TP), and the parsed judgment results (Label).
For Role agreement, we assign an integer from 0 to 3 by considering the sender, receiver and subject.
For TP and Label, we assign a binary agreement score (0 or 1).
To ensure a representative assessment, we randomly sample 30 parsed regulations and cases for each domain.
We then average and re-scale the results under 100\% for consistency, as shown in Table~\ref{tab:human_eval}.

\begin{table}[h]
\small
    \centering
    \begin{tabular}{l l|ccc}
        \toprule
        \textbf{Domain} & 
        \textbf{Type} &
        \textbf{Role} & \textbf{TP} & \textbf{Label} \\
        \midrule

        \multirow{2}{*}{HIPAA} &
        Case & 97.78 & 96.67
 & 100.00 \\
        & Law & 98.89 & 93.33
 & 96.67
 \\
        \midrule
        \multirow{2}{*}{GDPR} &
        Case & 96.67 & 96.67
 & 96.67\\
        & Law & 94.44 & 96.67
 & 93.33
 \\
        \midrule
        \multirow{2}{*}{AI Act} &
        Case & 90.00 & 93.33 & 96.67 \\
        & Law & 98.89 & 96.67
 & 96.67 \\
        
        \bottomrule
    \end{tabular}
    \vspace{-0.1in}
    \caption{Averaged Human agreement with our parsed data. Results are averaged and rescaled under \%.}  
    \vspace{-0.15in}
    \label{tab:human_eval}
\end{table}
%% Case agreement

The manual inspection results indicate that the HIPPA domain achieves the highest agreement scores among parsed cases and regulations.
This can be attributed to the fact that HIPAA is related to the medical domain, where roles and transmitted attributes are more clear and consistent.
For instance, it is frequent to observe a covered entity sharing the patient's protected health information (PHI).
Hence, it is easier to parse CI parameters.
For the EU AI Act, its cases' role has the worst performance, with an agreement score of 0.9.
We further inspect the EU AI Act synthetic cases and find that even though these cases strictly follow the question-answering chains of the compliance checker, they still suffer from narrative incoherence.
We leave the detailed case analyses in Appendix~\ref{app: case}.
\section{Related Work}

\textbf{Language Models Compression.} There are several main techniques for language model compression: knowledge distillation~\cite{kd1, kd2, Meta_kd}, quantization~\cite{ZeroQuant, quantization_survey, SmoothQuant}, network pruning or sparsity,~\cite{chess, sparcegpt, slicegpt, wanda, dsnot, sleb, DejaVu} and early exit~\cite{24arxiv-raee, SEENN, deebert}. Knowledge distillation methods transfer knowledge from a large, complex model (called the teacher model) to a smaller, simpler model (called the student model). They must either learn the output distribution of teacher models~\cite{kd3} or design multi-task approaches~\cite{kd4} to ensure that student models retain the knowledge and generative capabilities of the teacher models.  Quantization methods compress language models by reducing the precision of the numerical values representing the model's parameters (e.g., weights and activations). For example, OneBit quantized the weight matrices of LLMs to 1-bit~\cite{1bit}. Early exit methods allow a model to terminate its processing early during inference if it has already made a confident prediction, avoiding the need for additional layers of computation~\cite{ee_llm}. Network pruning, also known as sparsity techniques, refers to methods employed to compress language models by eliminating less significant structures within the network, such as individual weights, neurons, or layers~\cite{sparcegpt, slicegpt, sleb}. The primary objectives are to reduce the model’s size, enhance inference speed, and decrease memory consumption while preserving or only marginally affecting its performance. These works are orthogonal to each other, and we focus on the pruning methods.

\textbf{Different Granularity of Pruning.} Recent studies have investigated pruning at various granularities, ranging from coarse to fine. At the coarse-grained level, pruning methods include layer-wise~\cite{sleb}, attention heads~\cite{att_prun}, channel-wise pruning~\cite{slicegpt,llmpruner}, and neurons~\cite{neur_prun}. At the fine-grained level, techniques such as N:M sparsity~\cite{sparcegpt, wanda, dsnot, nm_sparse1, nm_sparse2} and individual weight pruning~\cite{sparsert} have been explored.
This paper mainly explored but was not limited to layer-wise network pruning.

\section{Conclusion}
In this work, we represent RLEdit, a hypernetwork-based editing method designed for lifelong editing. RLEdit formulates lifelong editing as an RL task, employing an offline update approach to enhance the model's retention of entire knowledge sequences. Additionally, RLEdit proposes the use of memory backtracking to review previously edited knowledge and applies regularization to mitigate knowledge forgetting over long sequences. Through extensive testing on several LLMs across multiple datasets, our experimental results demonstrate that RLEdit significantly outperforms existing baseline methods in lifelong editing tasks, showing superior performance in editing effectiveness, editing efficiency, and general capability preservation.
\section*{Limitations}

While \our validates the strength of NTE to take a free ride with LLM resources, our scope can be extended to several topics out of the main claims.

\paragraph{Label Embedding} Some IE paradigms (e.g., original NuNER) learns label embeddings to efficiently label the extracted spans. As \our imitates NTP to perform NTE, its IE process requires enumerating the label names similar as the generative IE using LLMs. Matching label embedding has its efficiency advantage while generative IE allows the label texts to interact with the context, resulting in potentially better performance. \our follows the generative IE paradigm to pursue better performance based on the established success of LLMs. However, future efforted can be devoted into a label embedding version of \our, which takes the context as the label text to boost the IE efficiency.

\paragraph{Data Source} The C4 corpus for raw text features broad coverage. However, recent progress in LLMs shows that specific sources of pre-training data (e.g., textbooks) benefit certain skills of LLMs, such as math. This paper only discusses C4 to avoid the IE performance improvement attributed to a specific data source. Future works can extend our scope to compare the effect of all kinds of resources in pre-training, which might find certain resources are superior in IE pre-training using NTE.

\paragraph{Backbone Variants} The current scopes is designed to justify the benefit of NTE in gathering massive IE pre-training data. Thus, the comparison is biased to data quality rather than backbone models. Further exploration in backbone models include the scaling law in model size, multilingual backbone, and model architectures.
\bibliography{ref}
%\bibliographystyle{acl_natbib}
\end{document}
