% This must be in the first 5 lines to tell arXiv to use pdfLaTeX, which is strongly recommended.
\pdfoutput=1
% In particular, the hyperref package requires pdfLaTeX in order to break URLs across lines.

\documentclass[11pt]{article}

% Change "review" to "final" to generate the final (sometimes called camera-ready) version.
% Change to "preprint" to generate a non-anonymous version with page numbers.
\usepackage[preprint]{acl}

% Standard package includes
\usepackage{times}
\usepackage{latexsym}

% For proper rendering and hyphenation of words containing Latin characters (including in bib files)
\usepackage[T1]{fontenc}
% For Vietnamese characters
% \usepackage[T5]{fontenc}
% See https://www.latex-project.org/help/documentation/encguide.pdf for other character sets

% This assumes your files are encoded as UTF8
\usepackage[utf8]{inputenc}

% This is not strictly necessary, and may be commented out,
% but it will improve the layout of the manuscript,
% and will typically save some space.
\usepackage{microtype}

% This is also not strictly necessary, and may be commented out.
% However, it will improve the aesthetics of text in
% the typewriter font.
\usepackage{inconsolata}
\usepackage{amsmath,amsfonts}
\usepackage{algorithmic}
\usepackage[section]{algorithm}
\renewcommand{\algorithmicrequire}{\textbf{Input:}} 
\renewcommand{\algorithmicensure}{\textbf{Output:}}
%Including images in your LaTeX document requires adding
%additional package(s)
\usepackage{graphicx}
\usepackage{enumitem}
\usepackage{array, multirow}
\usepackage{tabularx}
\usepackage{booktabs}

\newcommand{\dedit}[1]{\textcolor{red}{#1}}
\newcommand{\note}[1]{\textcolor{red}{(#1)}}
% If the title and author information does not fit in the area allocated, uncomment the following
%
%\setlength\titlebox{<dim>}
%
% and set <dim> to something 5cm or larger.

\title{EvoP: Robust LLM Inference via Evolutionary Pruning}

% Author information can be set in various styles:
% For several authors from the same institution:
% \author{Author 1 \and ... \and Author n \\
%         Address line \\ ... \\ Address line}
% if the names do not fit well on one line use
%         Author 1 \\ {\bf Author 2} \\ ... \\ {\bf Author n} \\
% For authors from different institutions:
% \author{Author 1 \\ Address line \\  ... \\ Address line
%         \And  ... \And
%         Author n \\ Address line \\ ... \\ Address line}
% To start a separate ``row'' of authors use \AND, as in
% \author{Author 1 \\ Address line \\  ... \\ Address line
%         \AND
%         Author 2 \\ Address line \\ ... \\ Address line \And
%         Author 3 \\ Address line \\ ... \\ Address line}

% \author{First Author \\
%   Affiliation / Address line 1 \\
%   Affiliation / Address line 2 \\
%   Affiliation / Address line 3 \\
%   \texttt{email@domain} \\\And
%   Second Author \\
%   Affiliation / Address line 1 \\
%   Affiliation / Address line 2 \\
%   Affiliation / Address line 3 \\
%   \texttt{email@domain} \\}


\author{%
  Shangyu Wu \\
  CityU \\
  \And
  Hongchao Du \\
  CityU \\
  \And
  Ying Xiong~\thanks{Corresponding author.} \\
  MBZUAI \\
  \And
  Shuai Chen\\
  Baidu \\
  \AND
  Tei-Wei Kuo \\
  National Taiwan University \\
  \And
  Nan Guan \\
  CityU \\
  \And
  Chun Jason Xue \\
  MBZUAI \\
}


%\author{
%  \textbf{First Author\textsuperscript{1}},
%  \textbf{Second Author\textsuperscript{1,2}},
%  \textbf{Third T. Author\textsuperscript{1}},
%  \textbf{Fourth Author\textsuperscript{1}},
%\\
%  \textbf{Fifth Author\textsuperscript{1,2}},
%  \textbf{Sixth Author\textsuperscript{1}},
%  \textbf{Seventh Author\textsuperscript{1}},
%  \textbf{Eighth Author \textsuperscript{1,2,3,4}},
%\\
%  \textbf{Ninth Author\textsuperscript{1}},
%  \textbf{Tenth Author\textsuperscript{1}},
%  \textbf{Eleventh E. Author\textsuperscript{1,2,3,4,5}},
%  \textbf{Twelfth Author\textsuperscript{1}},
%\\
%  \textbf{Thirteenth Author\textsuperscript{3}},
%  \textbf{Fourteenth F. Author\textsuperscript{2,4}},
%  \textbf{Fifteenth Author\textsuperscript{1}},
%  \textbf{Sixteenth Author\textsuperscript{1}},
%\\
%  \textbf{Seventeenth S. Author\textsuperscript{4,5}},
%  \textbf{Eighteenth Author\textsuperscript{3,4}},
%  \textbf{Nineteenth N. Author\textsuperscript{2,5}},
%  \textbf{Twentieth Author\textsuperscript{1}}
%\\
%\\
%  \textsuperscript{1}Affiliation 1,
%  \textsuperscript{2}Affiliation 2,
%  \textsuperscript{3}Affiliation 3,
%  \textsuperscript{4}Affiliation 4,
%  \textsuperscript{5}Affiliation 5
%\\
%  \small{
%    \textbf{Correspondence:} \href{mailto:email@domain}{email@domain}
%  }
%}

\begin{document}
\maketitle
\begin{abstract}

Large Language Models (LLMs) have achieved remarkable success in natural language processing tasks, but their massive size and computational demands hinder their deployment in resource-constrained environments. 
Existing structured pruning methods address this issue by removing redundant structures (e.g., elements, channels, layers) from the model. 
However, these methods employ a heuristic pruning strategy, which leads to suboptimal performance.
Besides, they also ignore the data characteristics when pruning the model.

To overcome these limitations, we propose EvoP, an evolutionary pruning framework for robust LLM inference.
EvoP first presents a cluster-based calibration dataset sampling (CCDS) strategy for creating a more diverse calibration dataset.
EvoP then introduces an evolutionary pruning pattern searching (EPPS) method to find the optimal pruning pattern.
Compared to existing structured pruning techniques, EvoP achieves the best performance while maintaining the best efficiency. 
Experiments across different LLMs and different downstream tasks validate the effectiveness of the proposed EvoP, making it a practical and scalable solution for deploying LLMs in real-world applications.
\end{abstract}

%!TEX root = gcn.tex
\section{Introduction}
Graphs, representing structural data and topology, are widely used across various domains, such as social networks and merchandising transactions.
Graph convolutional networks (GCN)~\cite{iclr/KipfW17} have significantly enhanced model training on these interconnected nodes.
However, these graphs often contain sensitive information that should not be leaked to untrusted parties.
For example, companies may analyze sensitive demographic and behavioral data about users for applications ranging from targeted advertising to personalized medicine.
Given the data-centric nature and analytical power of GCN training, addressing these privacy concerns is imperative.

Secure multi-party computation (MPC)~\cite{crypto/ChaumDG87,crypto/ChenC06,eurocrypt/CiampiRSW22} is a critical tool for privacy-preserving machine learning, enabling mutually distrustful parties to collaboratively train models with privacy protection over inputs and (intermediate) computations.
While research advances (\eg,~\cite{ccs/RatheeRKCGRS20,uss/NgC21,sp21/TanKTW,uss/WatsonWP22,icml/Keller022,ccs/ABY318,folkerts2023redsec}) support secure training on convolutional neural networks (CNNs) efficiently, private GCN training with MPC over graphs remains challenging.

Graph convolutional layers in GCNs involve multiplications with a (normalized) adjacency matrix containing $\numedge$ non-zero values in a $\numnode \times \numnode$ matrix for a graph with $\numnode$ nodes and $\numedge$ edges.
The graphs are typically sparse but large.
One could use the standard Beaver-triple-based protocol to securely perform these sparse matrix multiplications by treating graph convolution as ordinary dense matrix multiplication.
However, this approach incurs $O(\numnode^2)$ communication and memory costs due to computations on irrelevant nodes.
%
Integrating existing cryptographic advances, the initial effort of SecGNN~\cite{tsc/WangZJ23,nips/RanXLWQW23} requires heavy communication or computational overhead.
Recently, CoGNN~\cite{ccs/ZouLSLXX24} optimizes the overhead in terms of  horizontal data partitioning, proposing a semi-honest secure framework.
Research for secure GCN over vertical data  remains nascent.

Current MPC studies, for GCN or not, have primarily targeted settings where participants own different data samples, \ie, horizontally partitioned data~\cite{ccs/ZouLSLXX24}.
MPC specialized for scenarios where parties hold different types of features~\cite{tkde/LiuKZPHYOZY24,icml/CastigliaZ0KBP23,nips/Wang0ZLWL23} is rare.
This paper studies $2$-party secure GCN training for these vertical partition cases, where one party holds private graph topology (\eg, edges) while the other owns private node features.
For instance, LinkedIn holds private social relationships between users, while banks own users' private bank statements.
Such real-world graph structures underpin the relevance of our focus.
To our knowledge, no prior work tackles secure GCN training in this context, which is crucial for cross-silo collaboration.


To realize secure GCN over vertically split data, we tailor MPC protocols for sparse graph convolution, which fundamentally involves sparse (adjacency) matrix multiplication.
Recent studies have begun exploring MPC protocols for sparse matrix multiplication (SMM).
ROOM~\cite{ccs/SchoppmannG0P19}, a seminal work on SMM, requires foreknowledge of sparsity types: whether the input matrices are row-sparse or column-sparse.
Unfortunately, GCN typically trains on graphs with arbitrary sparsity, where nodes have varying degrees and no specific sparsity constraints.
Moreover, the adjacency matrix in GCN often contains a self-loop operation represented by adding the identity matrix, which is neither row- nor column-sparse.
Araki~\etal~\cite{ccs/Araki0OPRT21} avoid this limitation in their scalable, secure graph analysis work, yet it does not cover vertical partition.

% and related primitives
To bridge this gap, we propose a secure sparse matrix multiplication protocol, \osmm, achieving \emph{accurate, efficient, and secure GCN training over vertical data} for the first time.

\subsection{New Techniques for Sparse Matrices}
The cost of evaluating a GCN layer is dominated by SMM in the form of $\adjmat\feamat$, where $\adjmat$ is a sparse adjacency matrix of a (directed) graph $\graph$ and $\feamat$ is a dense matrix of node features.
For unrelated nodes, which often constitute a substantial portion, the element-wise products $0\cdot x$ are always zero.
Our efficient MPC design 
avoids unnecessary secure computation over unrelated nodes by focusing on computing non-zero results while concealing the sparse topology.
We achieve this~by:
1) decomposing the sparse matrix $\adjmat$ into a product of matrices (\S\ref{sec::sgc}), including permutation and binary diagonal matrices, that can \emph{faithfully} represent the original graph topology;
2) devising specialized protocols (\S\ref{sec::smm_protocol}) for efficiently multiplying the structured matrices while hiding sparsity topology.


 
\subsubsection{Sparse Matrix Decomposition}
We decompose adjacency matrix $\adjmat$ of $\graph$ into two bipartite graphs: one represented by sparse matrix $\adjout$, linking the out-degree nodes to edges, the other 
by sparse matrix $\adjin$,
linking edges to in-degree nodes.

%\ie, we decompose $\adjmat$ into $\adjout \adjin$, where $\adjout$ and $\adjin$ are sparse matrices representing these connections.
%linking out-degree nodes to edges and edges to in-degree nodes of $\graph$, respectively.

We then permute the columns of $\adjout$ and the rows of $\adjin$ so that the permuted matrices $\adjout'$ and $\adjin'$ have non-zero positions with \emph{monotonically non-decreasing} row and column indices.
A permutation $\sigma$ is used to preserve the edge topology, leading to an initial decomposition of $\adjmat = \adjout'\sigma \adjin'$.
This is further refined into a sequence of \emph{linear transformations}, 
which can be efficiently computed by our MPC protocols for 
\emph{oblivious permutation}
%($\Pi_{\ssp}$) 
and \emph{oblivious selection-multiplication}.
% ($\Pi_\SM$)
\iffalse
Our approach leverages bipartite graph representation and the monotonicity of non-zero positions to decompose a general sparse matrix into linear transformations, enhancing the efficiency of our MPC protocols.
\fi
Our decomposition approach is not limited to GCNs but also general~SMM 
by 
%simply 
treating them 
as adjacency matrices.
%of a graph.
%Since any sparse matrix can be viewed 

%allowing the same technique to be applied.

 
\subsubsection{New Protocols for Linear Transformations}
\emph{Oblivious permutation} (OP) is a two-party protocol taking a private permutation $\sigma$ and a private vector $\xvec$ from the two parties, respectively, and generating a secret share $\l\sigma \xvec\r$ between them.
Our OP protocol employs correlated randomnesses generated in an input-independent offline phase to mask $\sigma$ and $\xvec$ for secure computations on intermediate results, requiring only $1$ round in the online phase (\cf, $\ge 2$ in previous works~\cite{ccs/AsharovHIKNPTT22, ccs/Araki0OPRT21}).

Another crucial two-party protocol in our work is \emph{oblivious selection-multiplication} (OSM).
It takes a private bit~$s$ from a party and secret share $\l x\r$ of an arithmetic number~$x$ owned by the two parties as input and generates secret share $\l sx\r$.
%between them.
%Like our OP protocol, o
Our $1$-round OSM protocol also uses pre-computed randomnesses to mask $s$ and $x$.
%for secure computations.
Compared to the Beaver-triple-based~\cite{crypto/Beaver91a} and oblivious-transfer (OT)-based approaches~\cite{pkc/Tzeng02}, our protocol saves ${\sim}50\%$ of online communication while having the same offline communication and round complexities.

By decomposing the sparse matrix into linear transformations and applying our specialized protocols, our \osmm protocol
%($\prosmm$) 
reduces the complexity of evaluating $\numnode \times \numnode$ sparse matrices with $\numedge$ non-zero values from $O(\numnode^2)$ to $O(\numedge)$.

%(\S\ref{sec::secgcn})
\subsection{\cgnn: Secure GCN made Efficient}
Supported by our new sparsity techniques, we build \cgnn, 
a two-party computation (2PC) framework for GCN inference and training over vertical
%ly split
data.
Our contributions include:

1) We are the first to explore sparsity over vertically split, secret-shared data in MPC, enabling decompositions of sparse matrices with arbitrary sparsity and isolating computations that can be performed in plaintext without sacrificing privacy.

2) We propose two efficient $2$PC primitives for OP and OSM, both optimally single-round.
Combined with our sparse matrix decomposition approach, our \osmm protocol ($\prosmm$) achieves constant-round communication costs of $O(\numedge)$, reducing memory requirements and avoiding out-of-memory errors for large matrices.
In practice, it saves $99\%+$ communication
%(Table~\ref{table:comm_smm}) 
and reduces ${\sim}72\%$ memory usage over large $(5000\times5000)$ matrices compared with using Beaver triples.
%(Table~\ref{table:mem_smm_sparse}) ${\sim}16\%$-

3) We build an end-to-end secure GCN framework for inference and training over vertically split data, maintaining accuracy on par with plaintext computations.
We will open-source our evaluation code for research and deployment.

To evaluate the performance of $\cgnn$, we conducted extensive experiments over three standard graph datasets (Cora~\cite{aim/SenNBGGE08}, Citeseer~\cite{dl/GilesBL98}, and Pubmed~\cite{ijcnlp/DernoncourtL17}),
reporting communication, memory usage, accuracy, and running time under varying network conditions, along with an ablation study with or without \osmm.
Below, we highlight our key achievements.

\textit{Communication (\S\ref{sec::comm_compare_gcn}).}
$\cgnn$ saves communication by $50$-$80\%$.
(\cf,~CoGNN~\cite{ccs/KotiKPG24}, OblivGNN~\cite{uss/XuL0AYY24}).

\textit{Memory usage (\S\ref{sec::smmmemory}).}
\cgnn alleviates out-of-memory problems of using %the standard 
Beaver-triples~\cite{crypto/Beaver91a} for large datasets.

\textit{Accuracy (\S\ref{sec::acc_compare_gcn}).}
$\cgnn$ achieves inference and training accuracy comparable to plaintext counterparts.
%training accuracy $\{76\%$, $65.1\%$, $75.2\%\}$ comparable to $\{75.7\%$, $65.4\%$, $74.5\%\}$ in plaintext.

{\textit{Computational efficiency (\S\ref{sec::time_net}).}} 
%If the network is worse in bandwidth and better in latency, $\cgnn$ shows more benefits.
$\cgnn$ is faster by $6$-$45\%$ in inference and $28$-$95\%$ in training across various networks and excels in narrow-bandwidth and low-latency~ones.

{\textit{Impact of \osmm (\S\ref{sec:ablation}).}}
Our \osmm protocol shows a $10$-$42\times$ speed-up for $5000\times 5000$ matrices and saves $10$-2$1\%$ memory for ``small'' datasets and up to $90\%$+ for larger ones.

\section{Background}

\subsection{Memory Access Bottleneck of LLM Inference}

Introduce the prefilling and decoding stage of LLMs and the memory bottleneck caused by autoregressive generation.

Introduce batching, which alleviates this bottleneck by processing multiple requests simultaneously and share the access of model weights.

Introduce the difficulty of batching in long context serving

\begin{itemize}
    \item No enough GPU memory to store KV cache
    \item High KV cache access overhead
\end{itemize}

\subsection{Self-Attention Module and Rotary Position Embedding}

show an equation of how RoPE injects relative positional information through absolute position encoding.

\subsection{Recalling-Based KV Cache Reduction}

Recalling-based KV cache reduction is considered a promising method to mitigate the challenges in long context serving.

Its core idea is to make use of the inherent sparsity of the output tensor of the softmax function in attention modules, to reduce KV cache access during decoding stage. Its workflow contains two steps, first, for each decoding step, it identifies the required key-value pairs with the highest attention weights, then recall these kv cache and perform selective attention computation.

Introduce PQCache and ClusterKV, and analyze their weakness: online clustering leads to inefficient iterative read of KV cache, resulting in even higher memory access overhead.

Introduce Quest, although it does not introduce extra overheads, it leads to substantial model performance degradation and has limited compression ratio.

Introduce MagicPIG, although it achieves SoTA in recalling-based methods, it causes massive memory usage on CPU due to the large LSH table.


\section{Method}

In Fig. \ref{fig:overview}, we illustrate two major stages of MedForge for collaborative model development, including feature branch development (Sec~\ref{branch}) and model merging (Sec~\ref{forging}). In the feature branch development, individual contributors (i.e., medical centers) could make individual knowledge contributions asynchronously. Our MedForge allows each contributor to develop their own plugin module and distilled data locally without the need to share any private data. In the model merging stage, MedForge enables multi-task knowledge integration by merging the well-prepared plugin module asynchronously. This key integration process is guided by the distilled dataset produced by individual branch contributors, resulting in a generalizable model that performs strongly among multiple tasks.


\subsection{Preliminary}
\label{pre}
In MedForge, the development of a multi-capability model relies on the multi-center and multi-task knowledge introduced by branch plugin modules and the distilled datasets.
The relationship between the main base model and branch plugin modules in our proposed MedForge is conceptually similar to the relationship between the main repository and its branches in collaborative software version control platforms (e.g., GitHub~\cite{github}). 
To facilitate plugin module training on branches and model merging, we use the parameter-efficient finetuning (PEFT) technique~\cite{hu2021lora} for integrating knowledge from individual contributors into the branch plugin modules. 

\subsubsection{Parameter-efficient Finetuning}
Compared to resource-intensive full-parameter finetuning, parameter-efficient finetuning (PEFT) only updates a small fraction of the pretrained model parameters to reduce computational costs and accelerate training on specific tasks. These benefits are particularly crucial in medical scenarios where computational resources are often limited.
As the representative PEFT technique, LoRA (Low-Rank Adaptation)~\cite{hu2021lora} is widely utilized in resource-constrained downstream finetuning scenarios. In our MedForge, each contributor trains a lightweight LoRA on a specific task as the branch plugin module. LoRA decomposes the weight matrices of the target layer into two low-rank matrices to represent the update made to the main model when adapting to downstream tasks. If the target weight matrix is $W_0 \in R^{d \times k}$, during the adaptation, the updated weight matrix can be represented as $W_0+\Delta W=W_0+B A$, where $B \in \mathbb{R}^{d \times r}, A \in \mathbb{R}^{r \times k}$ are the low-rank matrices with rank $r \ll  \min (d, k)$ and $AB$ constitute the LoRA module. 



\subsubsection{Dataset Distillation}
Dataset distillation~\cite{wang2018dataset, yu2023dataset, lei2023comprehensive} is particularly valuable for medicine scenarios that have limited storage capabilities, restricted transmitting bandwidth, and high concerns for data privacy~\cite{li2024dataset}. 
We leverage the power of dataset distillation to synthesize a small-scale distilled dataset from the original data.

The distilled datasets serve as the training set in the subsequent merging stage to allow multi-center knowledge integration. Models trained on this distilled dataset maintain comparable performance to those trained on the original dataset (\ref{tab:main_res}). Moreover, the distinctive visual characteristics among images of the raw dataset are blurred (see \ref{fig:overview}(a)), which alleviates the potential patient information leakage. 

To perform dataset distillation, we define the original dataset as $\mathcal{T}=\{x_i,y_i\}^N_{i=1}$ and the model parameters as $\theta$. The dataset distillation aims to synthesize a distilled dataset ${\mathcal{S}=\{{s_i},\tilde{y_i}\}^M_{i=1}}$ with a much smaller scale (${M \ll N}$), while models trained on $\mathcal{S}$ can show similar performance as models trained on $\mathcal{T}$. 
This process is achieved by narrowing the performance gap between the real dataset $\mathcal{T}$ and the synthesized dataset $\mathcal{S}$. In MedForge, we utilize the distribution matching (DM)~\cite{zhao2023dataset}, which increases data distribution similarity between the synthesized distilled data and the real dataset
The distribution similarity between the real and synthesized dataset is evaluated through the empirical estimate of the Maximum Mean Discrepancy (MMD)~\cite{gretton2012kernel}:
\begin{equation}
\mathbb{E}_{\boldsymbol{\vartheta} \sim P_{\vartheta}}\left\|\frac{1}{|\mathcal{T}|} \sum_{i=1}^{|\mathcal{T}|} \psi_{\boldsymbol{\vartheta}}\left(\boldsymbol{x}_i\right)-\frac{1}{|\mathcal{S}|} \sum_{j=1}^{|\mathcal{S}|} \psi_{\boldsymbol{\vartheta}}\left(\boldsymbol{s}_j\right)\right\|^2
\end{equation}

where $P_\vartheta$ is the distribution of network parameters, $\psi_{\boldsymbol{\vartheta}}$ is a feature extractor. Then the distillation loss $\mathcal{L}_{DM}$ is:
\begin{equation}\scalebox{0.9}{$
\mathcal{L}_{\mathrm{DM}}(\mathcal{T},\mathcal{S},\psi_{\boldsymbol{\vartheta}})=\sum_{c=0}^{C-1}\left\|\frac{1}{\left|\mathcal{T}_c\right|} \sum_{\mathbf{x} \in \mathcal{T}_c} \psi(\mathbf{x})-\frac{1}{\left|\mathcal{S}_c\right|} \sum_{\mathbf{s} \in \mathcal{S}_c} \psi(\mathbf{s})\right\|^2$}
\end{equation}

We also applied the Differentiable Siamese Augmentation (DSA) strategy~\cite{zhao2021dataset} in the training process of distilled data to enhance the quality of the distilled data. DSA could ensure the distilled dataset is representative of the original data by exploiting information in real data with various transformations. The distilled images extract invariant and critical features from these augmented real images to ensure the distilled dataset remains representative.
\begin{figure}[t]
    \centering
    \includegraphics[width=\linewidth]{assets/img/model_arch.png}
    \caption{\textbf{Main model architecture.} We adopt CLIP as the base module and attach LoRA modules to the visual encoder and visual projection as the plugin module. During all the procedures, only the plugin modules are tuned while the rest are frozen. We get the classification result by comparing the cosine similarity of the visual and text embeddings.}
    \label{fig:model_arch}
\end{figure}

\subsection{Feature Branch Development}
\label{branch}
In the feature branch development stage, the branch contributors are responsible for providing the locally trained branch plugin modules and the distilled data to the MedForge platform, as shown in Fig~\ref{fig:overview} (a).
In collaborative software development, contributors work on individual feature branches, push their changes to the main platform, and later merge the changes into the main branch to update the repository with new features. Inspired by such collaborative workflow, branch contributors in MedForge follow similar preparations before the merging stage, enabling the integration of diverse branch knowledge into the main branch while effectively utilizing local resources.

MedForge consistently keeps a base module and a forge item as the main branch. The base module preserves generative knowledge of the foundation model pretrained on natural image datasets (i.e., ImageNet~\cite{deng2009imagenet}), while the forge item contains model merging information that guides the integration of feature branch knowledge (i.e., a merged plugin module or the merging coefficients assigned to plugin modules). 
Similar to individual software developers working in their own branches, each branch contributor (e.g., individual medical centers) trains a task-specific plugin module using their private data to introduce feature branch knowledge into the main branch. These branch plugin modules are then committed and pushed to update the forge items of the main branch in the merging stage, thus enhancing the model's multi-task capabilities.


\begin{figure*}
    \centering
    \includegraphics[width=\textwidth]{assets/img/fusion.png}
    \caption{\textbf{The detailed methodology of the proposed Fusion.} Branch contributors can asynchronously commit and push their branch plugin modules and the distilled datasets. the plugin modules will then be weighted fused to the current main plugin module.}

    \label{fig:merge}
\end{figure*}


Regarding model architecture, MedForge contains a base module and a plugin module (Fig ~\ref{fig:model_arch}). The base module is pretrained on general datasets (e.g., ImageNet) and remains the model parameters frozen in all processes and branches (main and feature branches) to avoid catastrophic forgetting of foundational knowledge acquired from pretraining. Meanwhile, the plugin module is adaptable for knowledge integration and can be flexibly added or removed from the base module, allowing updates without affecting the base model. In our study, we use the pretrained CLIP~\cite{radford2021learning} model as the base module. For the language encoder and projection layer of the CLIP model, all the parameters are frozen, which enables us to directly leverage the language capability of the original CLIP model. For the visual encoder, we apply LoRA on weight matrices of query ($W_q$) and value ($W_v$), following the previous study~\cite{hu2021lora}. To better adapt the model to downstream visual tasks, we apply the LoRA technique to both the visual encoder and the visual projection, and these LoRA modules perform as the plugin module. During the training, only the plugin module (LoRA modules) participates in parameter updates, while the base module (the original CLIP model) remains unchanged. 

In addition to the plugin modules, the feature branch contributors also develop a distilled dataset based on their private local data, which encapsulates essential patterns and features, serving as the foundation for training the merging coefficients in the subsequent merging stage~\ref{forging}. Compared to previous model merging approaches that rely on whole datasets or few-shot sampling, distilled data is lightweight and representative, mitigating the privacy risks associated with sharing raw data. 
We illustrate our distillation procedure in Algorithm~\ref{algorithm:alg1}. In each distillation step, the synthesized data $\mathcal{S}$ will be updated by minimizing $\mathcal{L}_{DM}$.
\begin{algorithmic}[1]
    \STATE \textbf{Input:} A list of clauses $C$
    \STATE \textbf{Output:} List of primary outputs $PO$, primary inputs $PI$, intermediate variables $IV$, and Boolearn expressions $BE$
    \STATE $SC$ = [] \COMMENT{List of sub-clauses}
    \FOR{$i = 1$ to length($C$)}
        % \IF{$C[i] \cap SC = \emptyset$}
        %     \STATE Append \text{Simplify}(\text{FindBooleanExpression}([], $SC$)) to $BE$
        %     %\COMMENT{Simplify Boolean expression}
        %     \FOR{each item $w$ in $SC$}
        %         \IF{$w \notin IV$ and $w \neq v$}
        %             \STATE Append $w$ to $PI$
        %         \ENDIF
        %     \ENDFOR
        %     \STATE $SC$ = []
        % \ELSE
            \STATE Append $C[i]$ to $SC$
            \FOR{each item $v$ in $SC$}
                \IF{$v \notin PI$ and $v \notin IV$}
                    \STATE $f \gets \text{FindBooleanExpression}(v, SC)$ %\COMMENT{Find Boolean expression for $v$}
                    \STATE $g \gets \text{FindBooleanExpression}(\neg v, SC)$ %\COMMENT{Find Boolean expression for $\neg v$}
                    \IF{$f = \neg g$}
                        \STATE Append \text{Simplify}($f$) to $BE$ %\COMMENT{Simplify Boolean expression}
                        \IF{$f = True$ or $f = False$}
                            \STATE Append $v$ to $PO$
                        \ELSE
                            \STATE Append $v$ to $IV$
                        \ENDIF
                        \FOR{each item $w$ in $SC$}
                            \IF{$w \notin IV$ and $w \neq v$}
                                \STATE Append $w$ to $PI$
                            \ENDIF
                        \ENDFOR
                        \STATE SC = []
                        \STATE \textbf{break}
                    \ENDIF
                \ENDIF    
            \ENDFOR
        % \ENDIF
    \ENDFOR
    \STATE \textbf{Return} $PO, PI, IV, BE$
    \vspace{-0.65cm}
\end{algorithmic}



\subsection{MedForge Merging Stage}
\label{forging}
Following the feature branch development stage illustrated in Fig~\ref{fig:overview} (a), branch contributors push and merge their branch plugin modules along with the corresponding distilled dataset into the main branch, as shown in Fig~\ref{fig:overview} (b). Our MedForge allows an incremental capability accumulation from branches to construct a comprehensive medical model that can handle multiple tasks.

In the merging stage, the $i^{th}$ branch contributor is assigned a coefficient $w'_i$ for the contribution of merging, while the coefficient for the current main branch is $w_i$. By adaptively adjusting the value of coefficients, the main branch can balance and coordinate updates from different contributors, ultimately enhancing the overall performance of the model across multiple tasks.
The optimization of the coefficients is done by minimizing the cross-entropy loss for classification based on the distilled datasets. We also add $L1$ regularization to the loss to regulate the weights to avoid outlier coefficient values (e.g., extremely large or small coefficient values)~\cite{huang2023lorahub}. During optimization, following~\cite{huang2023lorahub}, we utilize Shiwa algorithm~\cite{liu2020versatile} to enable model merging under gradient-free conditions, with lower computational and time costs. The optimizer selector~\cite{liu2020versatile} automatically chooses the most suitable optimization method for coefficient optimization. 

In the following sections, we introduce the two merging methods used in our MedForge: Fusion and Mixture. In MedForge-Fusion, the parameters of the branch plugin modules are fused into the main branch after each round of the merging stage. For MedForge-Mixture, the outputs of the branch modules are weighted and summed based on their respective coefficients rather than directly applying the weighted sum to the model parameters. This largely preserves the internal parameter structure of each branch module.

\paragraph{MedForge-Fusion}
In MedForge, forge items are utilized to facilitate the integration of branch knowledge into the main branch.
For MedForge-Fusion, the forge item refers to adaptable main plugin modules. When the $i^{th}$ branch contributor pushes its branch plugin module $\theta'_i=A'_iB'_i$ to the main branch, the current main plugin module $\theta_{i-1}=A_{i-1}B_{i-1}$ will be updated to $\theta_{i}=A_{i}B_{i}$. The parameters of the branch and the current main plugin modules are weighted with coefficients and added to fuse a new version. The $A_i$, $B_i$ are the low-rank matrices composing the LoRA module $\theta_i$. The detailed fusion process can be represented as:
\begin{equation}
\theta_{i}=(w_i A_{i-1}+w'_i A'_i)(w_i B_{i-1}+ w'_i B'_i)
\end{equation}
Where $w_i$ is the coefficient assigned to the current main branch, while $w'_i$ is the coefficient assigned to the branch contributor. After this round of merging, the resulting plugin module $\theta_{i}$ is the updated version of main forging item, thus the main model is able to obtain new capacity introduced by the current branch contributor. When new contributors push their plugin modules and distilled datasets, the main branch can be incrementally updated through merging stages, and the optimization of the coefficients is guided by distilled data.
As shown in Fig.~\ref{fig:merge}, though multiple contributors commit their branch plugin modules and distilled datasets at different times, they can flexibly merge their plugin modules with the current main branch. After each merging round, the plugin module of the main branch will be updated, and thus the version iteration has been achieved.
\begin{figure*}[t]
    \centering
    \includegraphics[width=\textwidth]{assets/img/mixture.png}
    \caption{\textbf{The detailed methodology of the proposed Mixture.} Branch contributors can asynchronously commit and push their branch plugin modules and the distilled datasets. the outputs of different plugin modules will be weighted aggregated. The weights of each merging step will be saved.}

    \label{fig:mixmerge}
\end{figure*}


\paragraph{MedForge-Mixture}
To further improve the model merging performance, inspired by~\cite{zhao2024loraretriever}, we also propose medForge-mixture. For MedForge-Mixture, the forge items refer to the optimized coefficients.
As shown in Fig.~\ref{fig:mixmerge}, for MedForge-Mixture, the coefficient of each branch contributor is acquired and optimized based on distilled datasets. Then the outputs of plugin modules will be weighted combined with these coefficients to get the merged output. 

For each merging round, with branch contributor $i$, the branch coefficient is $w'_i$, the main coefficient is $w_i$, the branch plugin module is $\theta'_i=A'_iB'_i$, and the current main plugin module is $\theta_i=A_iB_i$. With the input $x$, the resulted MedForge-Mixture output can be represented as:
\begin{equation}
y_{i}=w_i A_{i-1} B_{i-1} x+w'_i A'_i B'_i x
\end{equation}

In this way, MedForge encourages additional contributors as the workflow supports continuous incremental knowledge updates.

Overall, both MedForge merging strategies greatly improve the communication efficiency among contributors. We use this design to build a multi-task medical foundation model that enhances the full utilization of resources in the medical community. For the MedForge-Fusion strategy, the main plugin module is updated after each merging round, thus avoiding storing the previous plugin modules and saving space. Meanwhile, the MedForge-Mixture strategy avoids directly updating the parameters of each plugin module, thus preserving their original structure and preventing the introduction of additional noise, which enhances the robustness and stability of the models.

\section{Experiments}
\label{sec:exp}
\noindent\textbf{Datasets.} {\YuiR Following existing studies~\cite{liu2023hybrid,zhoustrengthened,liu2020learning,mccreesh2017partitioning,solnon2015complexity,hoffmann2017between},} we use four benchmark graph collections, namely biochemicalReactions (\textsf{BI}), images-CVIU11 (\textsf{CV}), images-PR15 (\textsf{PR}) and LV (\textsf{LV}), in the experiments. All datasets are collected from http://liris.cnrs.fr/csolnon/SIP.html and come from real-world applications in various domains, {\Yui as shown in Table~\ref{tab:my_label}}. Specifically, \textsf{BI} contains 136 unlabeled bipartite graphs, each of which corresponds to a biochemical reaction network. \textsf{CV} contains 44 pattern graphs and 146 target graphs, which are generated from segmented images. \textsf{PR} contains 24 pattern graphs and 1 target graph, which are also from segmented images. \textsf{LV} contains 112 graphs generated from biological networks. 
%
{\YuiR All graphs have up to thousands of vertices. We note that (1) solving our problem on two graphs with beyond 10K vertices is challenging based on the worst-case time complexity of $O^*((|V_G|+1)^{|V_Q|})$, (2) the largest graph used in previous studies~\cite{liu2023hybrid,zhoustrengthened,liu2020learning,mccreesh2017partitioning} has 6,771 vertices, which is also covered (in LV) by our experiments, and (3) finding the largest common subgraph between two graphs with thousands of vertices has found many real applications~\cite{ehrlich2011maximum}.}
%
{\Yui Following existing studies~\cite{liu2023hybrid,zhoustrengthened,liu2020learning,mccreesh2017partitioning,solnon2015complexity,hoffmann2017between}}, for \textsf{BI} and \textsf{LV}, we generate and test the problem instances (i.e., $Q$ and $G$) by pairing any two distinct graphs; and for \textsf{CV} and \textsf{PR} {\revision which consist of two types of graphs, namely pattern graphs and target graphs}, we test all those problem instances with one graph $Q$ from pattern graphs and the other $G$ from target graphs.

\begin{table*}[]
    \centering
    \caption{\Yui Datasets used in the experiments (``\# of solved instances'' refers to the number of instances solved by algorithms within 1,800 seconds and ``Achieved speedups'' refers to the percentage of the solved instances that \texttt{RRSplit} runs at least 5$\times$/10$\times$/100$\times$ faster than \texttt{McSplitDAL})}
    \vspace{-0.15in}
    \begin{tabular}{|c|c|c|c|c|c|c|c|c|c|c|}
        \hline
        \multirow{2}{*}{Dataset} & \multirow{2}{*}{Domain} & \multirow{2}{*}{\# of graphs} & \multirow{2}{*}{\# of instances} & \multirow{2}{*}{\# of vertices} & \multicolumn{2}{c|}{\# of solved instances} & \multicolumn{3}{c|}{Achieved speedups} \\
        \cline{6-10}
        & & & & & \texttt{RRSplit} & \texttt{McSplitDAL} & 5$\times$ & 10$\times$ & 100$\times$\\
        \hline
        \textsf{BI} & Biochemical & 136 & 9,180 & 9$\sim$ 386 & 7,730 & 4,696 & 91.3\% & 84.4\% & 69.7\% \\
        \textsf{CV} & Segmented images & 190 & 6,424 & 22$\sim$ 5,972 & 1,351 & 1,291& 76.5\% & 48.6\% & 0.2\% \\
        \textsf{PR} & Segmented images & 25& 24& 4$\sim$ 4,838  & 24 & 24 & 91.7\% & 91.7\% & 58.3\% \\
        \textsf{LV} & Synthetic & 112 & 6,216& 10$\sim$ 6,671 & 1,059 & 883 & 68.0\% & 54.7\% & 38.3\%\\
        \hline
    \end{tabular}
    
    \label{tab:my_label}
\end{table*}

\begin{table*}[]
    \centering
    \caption{\YuiR Comparison of running time on all datasets (statistics of achieved speedups in Figure~\ref{fig:all_datasets_T})}
    \vspace{-0.15in}
    \begin{tabular}{|c|c|c|c|c|c|c|}
        \hline
        \multirow{2}{*}{Dataset} & \multicolumn{3}{c|}{\texttt{RRSplit} runs faster} & \multicolumn{3}{c|}{\texttt{McSplitDAL} runs faster} \\
        \cline{2-7}
        & \% of instances & Avg. speedup & Max. speedup & \% of instances & Avg. speedup & Max. speedup\\
        \hline
        BI& 99.43\% & 3.3$\times 10^4$ & $10^6$ & 0.5\% & 24.81 & 872.37 \\
        CV& 92.15\% & 10.92 & 161 & 7.84\% & 4.96 & 38.97 \\
        PR& 95.83\% & 139.39 & 234 & 4.17\% & 1.23 & 1.23 \\
        LV& 93.48\% & 1.2$\times 10^4$ & $10^6$ & 6.51\% & 24.23 & 652.13 \\
        \hline
    \end{tabular}
    
    \label{tab:results}
\end{table*}

\smallskip
\noindent\textbf{Algorithms.} We compare the newly proposed algorithm \texttt{RRSplit} with \texttt{McSplitDAL}~\cite{liu2023hybrid}. To be specific, \texttt{McSplitDAL} is one variant of \texttt{McSplit} as introduced in Section~\ref{sec:sota}, which follows the framework of \texttt{McSplit} (i.e., Algorithm~\ref{alg:mcsplit}) and introduces some learning-based techniques for optimizing the policies of selecting vertices at line 6, line 8 and line 10 of Algorithm~\ref{alg:mcsplit}. To our best knowledge, \texttt{McSplitDAL} is the state-of-the-art algorithm and runs significantly faster than previous solutions, including \texttt{McSplitLL}~\cite{zhoustrengthened} and \texttt{McSplitRL}~\cite{liu2023hybrid}. Besides these, in order to study the effectiveness of different reductions employed in our algorithm \texttt{RRSplit}, we evaluate three variants of  \texttt{RRSplit} --  
{\YuiR \texttt{RRSplit-VE}, \texttt{RRSplit-MB}, and \texttt{RRSplit-UB}, respectively obtained by turning off vertex-equivalence based reductions, maximality based reductions,  and  vertex-equivalence based upper bound}. 
%namely \texttt{RRSplit-MR} and \texttt{RRSplit-VER},


\smallskip
\noindent\textbf{Implementation and metrics.} All algorithms are implemented in C++ and compiled with -O3 optimization. All experiments run on a Linux machine with a 2.10GHz Intel CPU and 128GB memory. Note that, for the implementation of \texttt{McSplitDAL}, we directly use the source code from the authors of~\cite{liu2023hybrid}. We record and compare the total running times of the algorithms on different problem instances (note that the measured running time excludes the I/O time of reading graphs from the disk). We set the running time limit (INF) as 1,800 seconds by default. Our data and code are available at https://github.com/KaiqiangYu/SIGMOD25-MCSS. 

\subsection{Comparison among algorithms}

\begin{figure}[]
		\subfigure[\textsf{BI}]{
			\includegraphics[width=4.0cm]{figure/BI_TDS.pdf}
		}
		\subfigure[\textsf{CV}]{
			\includegraphics[width=4.0cm]{figure/ICVIU11_TDS.pdf}
		}
		\subfigure[\textsf{PR}]{
			\includegraphics[width=4.0cm]{figure/PR15_TDS.pdf}
		}	
		\subfigure[\textsf{LV}]{
			\includegraphics[width=4.0cm]{figure/LV_TDS.pdf}
		}
        \vspace{-0.15in}
	\caption{Running time on all datasets. {\Yui For those problem instances locating at the right side of dash line `- .' with orange color (resp. `- -' with green color),  \texttt{RRSplit} achieves at least 100$\times$ (resp. 10$\times$) speedup compared with \texttt{McSplitDAL}.}}
	\label{fig:all_datasets_T}
\end{figure}

\noindent\textbf{All datasets (running time)}. We compare our algorithm \texttt{RRSplit} with the baseline \texttt{McSplitDAL} on all graph collections. {\YuiR Following some existing works~\cite{mccreesh2016clique}}, we report the running times of the algorithms on various problem instances in Figure~\ref{fig:all_datasets_T}. 
%
%
Specifically, each dot in the scatter figures represents a problem instance, with the $x$-axis (resp. $y$-axis) corresponding to the running time of \texttt{RRSplit} (resp. \texttt{McSplitDAL}) {\chengC on the instance}. Hence, for those problem instances with small values on $x$-axis and large values on $y$-axis (which thus locate on the top left region of the figures), \texttt{RRSplit} performs better than \texttt{McSplitDAL}.
%In particular, for those problem instances locating at the right side of dash line `- .' {\Yui with orange color} (resp. `- -' with green color),  \texttt{RRSplit} achieves at least 100$\times$ (resp. 10$\times$) speedup compared with \texttt{McSplitDAL}. 
We mark the running time as INF if the problem instance cannot be solved within the default time limit.
%
{\YuiR Besides, we also provide some statistics in Table~\ref{tab:my_label} and Table~\ref{tab:results}.}
%
We observe that (1) \texttt{RRSplit} outperforms \texttt{McSplitDAL} by achieving around one to {\Yui four} orders of magnitude speedup {\YuiR (in average)} on the majority {\YuiR (above 92\%)} of the tested problem instances and (2) \texttt{McSplitDAL} cannot handle all problem instances within the time limit. 
% This fact demonstrates the efficiency of our algorithm \texttt{RRSplit}. 
We do note that \texttt{McSplitDAL} runs slightly faster on a few {\YuiR (below 8\%)} problem instances in \textsf{CV} and \textsf{LV}. {\YuiR Some possible reasons are as follows. 
First, our \texttt{RRSplit} introduces some extra time costs for conducting the proposed reductions as well as computing the upper bound. Second, the heuristic polices adopted in \texttt{RRSplit} and \texttt{McSplitDAL} for branching may have different behaviors. 
%
In specific, on these problem instances, the heuristic policies may help \texttt{McSplitDAL} to find a large common subgraph quickly so as to prune more unpromising branches {\revision via the upper-bound based reduction} (note that they are based on reinforcement learning {\cheng and the behaviors of the learned policy is} based on the explored branches during the running time). } 

%{\cheng One possible reason could be that} their learned heuristic policies can help to find a large common subgraph quickly so as to prune more unpromising branches {\cheng on these datasets}. 


\begin{figure}[]
		\subfigure[\textsf{BI}]{
			\includegraphics[width=4.0cm]{figure/BI_Branch_TDS.pdf}
		}
		\subfigure[\textsf{CV}]{
			\includegraphics[width=4.0cm]{figure/ICVIU11_Branch_TDS.pdf}
		}
		\subfigure[\textsf{PR}]{
			\includegraphics[width=4.0cm]{figure/PR15_Branch_TDS.pdf}
		}	
		\subfigure[\textsf{LV}]{
			\includegraphics[width=4.0cm]{figure/LV_Branch_TDS.pdf}
		}
        \vspace{-0.15in}
	\caption{Number of formed branches on all datasets}
	\label{fig:all_datasets_BT}
\end{figure}

\smallskip
\noindent\textbf{All datasets (number of formed branches)}. We report the number of branches formed by the algorithms on different problem instances in Figure~\ref{fig:all_datasets_BT}. Similarly, each dot in the scatter figures represents a problem instance, with the $x$-axis (resp. $y$-axis) corresponding to the number of branches formed by \texttt{RRSplit} (resp. \texttt{McSplitDAL}) {\chengC on the instance}. We have the following observations. First, the number of branches formed by \texttt{RRSplit} is significantly {\chengC smaller} than that formed by \texttt{McSplitDAL}, e.g., the former is around 10\% - 0.01\% of the latter on the most of problem instances. This shows the effectiveness of our proposed maximality-based reductions and vertex-equivalence-based reductions.
%, and is also compatible with the theoretical results.
Second, the distribution of the number of formed branches in Figure~\ref{fig:all_datasets_BT} is consistent with that of the running time in Figure~\ref{fig:all_datasets_T}. This indicates the achieved speedups on the running time \laks{can be traced} to our newly-designed reductions.

\begin{figure}[]
		\subfigure[\textsf{BI}]{
			\includegraphics[width=4.0cm]{figure/BI_CDF.pdf}
		}
		\subfigure[\textsf{CV}]{
			\includegraphics[width=4.0cm]{figure/ICVIU11_CDF.pdf}
		}
		\subfigure[\textsf{PR}]{
			\includegraphics[width=4.0cm]{figure/PR15_CDF.pdf}
		}	
		\subfigure[\textsf{LV}]{
			\includegraphics[width=4.0cm]{figure/LV_CDF.pdf}
		}
        \vspace{-0.2in}
	\caption{Comparison by varying time limits}
	\label{fig:all_vary_T}
\end{figure}

\begin{figure}[]
		\subfigure[\textsf{BI}]{
			\includegraphics[width=4.0cm]{figure/BI_Branch_CDF.pdf}
		}
		\subfigure[\textsf{CV}]{
			\includegraphics[width=4.0cm]{figure/ICVIU11_Branch_CDF.pdf}
		}
		\subfigure[\textsf{PR}]{
			\includegraphics[width=4.0cm]{figure/PR15_Branch_CDF.pdf}
		}	
		\subfigure[\textsf{LV}]{
			\includegraphics[width=4.0cm]{figure/LV_Branch_CDF.pdf}
		}
        \vspace{-0.2in}
	\caption{Comparison by varying the limit of number of formed branches}
	\label{fig:all_vary_B}
\end{figure}

\smallskip
\noindent\textbf{Varying time limits}. We report the number of solved problem instances in Figure~\ref{fig:all_vary_T} as the time limit is varied. Clearly, all algorithms solve more problem instances as the time limit increases. We observe that \texttt{RRSplit} solves more problem instances than \texttt{McSplitDAL} within the same time limit. In particular, \texttt{RRSplit} with a time limit of 1 second even solves more problem instances than \texttt{McSplitDAL} with a time limit of 10 seconds in all graph collections {\cheng except for} \textsf{CV}; and on \texttt{PR}, \texttt{RRSplit} solves all problem instances within the time limit of 10 seconds. This further demonstrates the superiority of our algorithm \texttt{RRSplit} over the baseline \texttt{McSplitDAL}. 

\smallskip
\noindent\textbf{Varying the limits of number of formed branches}. We report the number of solved problem instances in Figure~\ref{fig:all_vary_B} as the limit on  number of formed branches is varied. We note that the more branches are allowed to be formed, the more instances will be solved. We observe that (1) \texttt{RRSplit} solves more problem instances than \texttt{McSplitDAL} within the same limit of the number of formed branches and (2) the results in Figure~\ref{fig:all_vary_B} show  similar tendencies as those in Figure~\ref{fig:all_vary_T} in general. This further {\cheng explains} the practical superiority of the newly proposed reductions.


\begin{figure}[]
		\subfigure[\textsf{Running time (BI)}]{
			\includegraphics[width=4.0cm]{figure/BI_SIM.pdf}
		}	
		\subfigure[\textsf{Running time (LV)}]{
			\includegraphics[width=4.0cm]{figure/LV_SIM.pdf}
		}
        \subfigure[\textsf{\# of branches (BI)}]{
			\includegraphics[width=4.0cm]{figure/BI_SIMB.pdf}
		}	
		\subfigure[\textsf{\# of branches (LV)}]{
			\includegraphics[width=4.0cm]{figure/LV_SIMB.pdf}
		}
        \vspace{-0.2in}
	\caption{Comparison by varying similarities}
	\label{fig:all_vary_S}
\end{figure}

\smallskip
\noindent\textbf{Varying the similarities of  input graphs}. We define the similarity of  input graphs $Q$ and $G$, $Sim(Q,G)$, as follows.
\begin{equation}
\label{eq:sim}
    Sim(Q,G)=\frac{|S^*|}{\min\{|V_Q|,|V_G|\}},
\end{equation}
where $S^*$ is the maximum common subgraph between $Q$ and $G$. Clearly, $Sim(Q,G)$ varies from 0 to 1, and the larger the value of $Sim(Q,G)$, the higher the similarity between $Q$ and $G$. We test different problem instances as the similarity varies from 0.5 to 1 on \textsf{BI} and \textsf{LV}, and report the average running time in Figures~\ref{fig:all_vary_S}(a)-(b) and the average number of formed branches in Figures~\ref{fig:all_vary_S}(c)-(d). {\Yui The results on \textsf{CV} and \textsf{PR} show similar trends, complete details of which appear in the 
\ifx \CR\undefined
Appendix. 
\else
technical report~\cite{TR}. 
\fi
}  We can see that \texttt{RRSplit} consistently outperforms \texttt{McSplitDAL} {\chengC in} various settings, e.g., \texttt{RRSplit} runs several orders of magnitude faster and forms fewer branches than \texttt{McSplitDAL}. This demonstrates that our designed reductions are effective for pruning the redundant branches on problem instances with various similarities. Besides, we observe that both \texttt{RRSplit} and \texttt{McSplitDAL} have the running time and the number of formed branches first increase and then decrease as the similarity grows. {\revision The possible reasons are as follows. (1) The maximum common subgraphs become larger as the similarity increases according to Equation~(\ref{eq:sim}) and typically more common subgraphs will be explored for finding a large maximum common subgraph. Therefore, the running time firstly increases; (2) the upper-bound based reduction {\chengE performs} better as the similarity grows. {\chengE For example, in the setting of} $Sim(Q,G)=1$, the algorithm can be terminated directly once a common subgraph with $\min\{|V_Q|,|V_G|\}$ vertices is found. Therefore, the running time then decreases.}

%Possible reasons include (1) the number of common subgraphs (i.e., search space) first increases and then decreases as the similarity grows and/or (2) the proposed reductions performs better on those problem instances with the similarity {\cheng close to} 0.5 or 1.   


\begin{figure}[]
		\subfigure[\textsf{\Yui Varying time limits (BI)}]{
			\includegraphics[width=4.0cm]{figure/BI_RT.pdf}
		}	
		\subfigure[\textsf{\Yui Varying time limits (LV)}]{
			\includegraphics[width=4.0cm]{figure/LV_RT.pdf}
		}
        \subfigure[\textsf{\Yui Varying limit of \#branches (BI)}]{
			\includegraphics[width=4.0cm]{figure/BI_BT.pdf}
		}	
		\subfigure[\textsf{\Yui Varying limit of \#branches (LV)}]{
			\includegraphics[width=4.0cm]{figure/LV_BT.pdf}
		}
        \vspace{-0.15in}
	\caption{Comparison among various reductions}
	\label{fig:all_vary_R}
\end{figure}

{\revision
\smallskip
\noindent\textbf{Scalability test.} We test the scalability of our \texttt{RRSplit} on two large datasets, i.e., \textsf{Twitter} and \textsf{DBLP}, which are collected from different domains (http://konect.cc/). Here, \textsf{Twitter} is a social network with 465,017 vertices and 833,540 edges, and \textsf{DBLP} is a collaboration network with 317,080 vertices and 1,049,866 edges. Following existing studies~\cite{arai2023gup,jin2023circinus,sun2023efficient}, we generate the problem instances (i.e., $Q$ and $G$) as follows. Let \textsf{Twitter} or \textsf{DBLP} be the graph $G$. We first extract a set of graphs $Q$ from $G$. Specifically, we conduct a random walk on $G$ and extract a subgraph induced by the visited vertices. By varying the size of the extracted graph $Q$ (among $\{20,30,40,50,60\}$), we extract 5 sets and each of them contains 100 different graphs $Q$.
%
Then, we generate different problems by pairing the graph $G$ (i.e., \textsf{Twitter} or \textsf{DBLP}) with different graphs $Q$ in the set. In summary, for each dataset, we have 500 different problem instances. 

We compare our \texttt{RRSplit} with \texttt{McSplitDAL} by varying the size of $Q$, and report the average running time in Figure~\ref{fig:scalability_test}. We observe that our \texttt{RRSplit} outperforms \texttt{McSplitDAL} significantly.
% , which demonstrates the scalability of the proposed method. 
Besides, \texttt{McSplitDAL} cannot handle almost all the problem instances within the time and/or space limit (INF/OOM). This is because the implementation of \texttt{McSplitDAL} highly relies on the adjacent matrix of $Q$ and $G$, which introduces huge space and time costs when $G$ is very large. Finally, we observe that \texttt{RRSplit} has the running time increase as the size of $Q$ grows. This is also consistent with the theoretical analysis.
}

\begin{figure}[]
		\subfigure[\textsf{Running time (Twitter)}]{
			\includegraphics[width=4.0cm]{figure/TW_S.pdf}
		}	
		\subfigure[\textsf{Running time (DBLP)}]{
			\includegraphics[width=4.0cm]{figure/DBLP_S.pdf}
		}
        \vspace{-0.2in}
	\caption{\revision Scalability test on large datasets}
	\label{fig:scalability_test}
\end{figure}

\subsection{Ablation studies}

We study the effects of various reductions on reducing the redundant computations. In specific, we compare \texttt{RRSplit} with three variants, namely \texttt{RRSplit-VE}: the full version without vertex-equivalence based reductions, \texttt{RRSplit-MB}: the full version without maximality based reductions and \texttt{RRSplit-UB}: the full version without the vertex-equivalence based upper bound, on \textsf{BI} and \textsf{LV}. We report the number of solved problem instances in Figure~\ref{fig:all_vary_R}(a,b) for varying the time limit and in Figure~\ref{fig:all_vary_R}(c,d) for varying the limit of number of formed branches. {\Yui The results on \textsf{CV} and \textsf{PR} show similar clues, which we put in the 
\ifx \CR\undefined
Appendix. 
\else
technical report~\cite{TR}. 
\fi
First, we can see that all four algorithms perform better than the baseline \texttt{McSplitDAL}, among which \texttt{RRSplit} performs the best. This demonstrates the effectiveness of vertex-equivalence-based reductions, maximality-based reductions and vertex-equivalence-based upper bound. Second, \texttt{RRSplit-VE} and \texttt{RRSplit-MB} {\chengB achieve} comparable performance and {\chengB both} contribute to the improvements. Specifically, we note that \texttt{RRSplit-VE} runs slightly faster than \texttt{RRSplit-MB} on \textsf{BI} while \texttt{RRSplit-MB} runs slightly faster than \texttt{RRSplit-VE} on \textsf{LV}. }  {\revision This is possibly because graphs in \textsf{BI} are relatively small biochemical networks where two vertices are more likely to be structural equivalent and thus the vertex-equivalence based reductions outperform other reductions, while graphs in \texttt{LV} are synthetic networks. }
\section{Related Work}

\textbf{Language Models Compression.} There are several main techniques for language model compression: knowledge distillation~\cite{kd1, kd2, Meta_kd}, quantization~\cite{ZeroQuant, quantization_survey, SmoothQuant}, network pruning or sparsity,~\cite{chess, sparcegpt, slicegpt, wanda, dsnot, sleb, DejaVu} and early exit~\cite{24arxiv-raee, SEENN, deebert}. Knowledge distillation methods transfer knowledge from a large, complex model (called the teacher model) to a smaller, simpler model (called the student model). They must either learn the output distribution of teacher models~\cite{kd3} or design multi-task approaches~\cite{kd4} to ensure that student models retain the knowledge and generative capabilities of the teacher models.  Quantization methods compress language models by reducing the precision of the numerical values representing the model's parameters (e.g., weights and activations). For example, OneBit quantized the weight matrices of LLMs to 1-bit~\cite{1bit}. Early exit methods allow a model to terminate its processing early during inference if it has already made a confident prediction, avoiding the need for additional layers of computation~\cite{ee_llm}. Network pruning, also known as sparsity techniques, refers to methods employed to compress language models by eliminating less significant structures within the network, such as individual weights, neurons, or layers~\cite{sparcegpt, slicegpt, sleb}. The primary objectives are to reduce the model’s size, enhance inference speed, and decrease memory consumption while preserving or only marginally affecting its performance. These works are orthogonal to each other, and we focus on the pruning methods.

\textbf{Different Granularity of Pruning.} Recent studies have investigated pruning at various granularities, ranging from coarse to fine. At the coarse-grained level, pruning methods include layer-wise~\cite{sleb}, attention heads~\cite{att_prun}, channel-wise pruning~\cite{slicegpt,llmpruner}, and neurons~\cite{neur_prun}. At the fine-grained level, techniques such as N:M sparsity~\cite{sparcegpt, wanda, dsnot, nm_sparse1, nm_sparse2} and individual weight pruning~\cite{sparsert} have been explored.
This paper mainly explored but was not limited to layer-wise network pruning.

\section{Conclusion}
In this work, we represent RLEdit, a hypernetwork-based editing method designed for lifelong editing. RLEdit formulates lifelong editing as an RL task, employing an offline update approach to enhance the model's retention of entire knowledge sequences. Additionally, RLEdit proposes the use of memory backtracking to review previously edited knowledge and applies regularization to mitigate knowledge forgetting over long sequences. Through extensive testing on several LLMs across multiple datasets, our experimental results demonstrate that RLEdit significantly outperforms existing baseline methods in lifelong editing tasks, showing superior performance in editing effectiveness, editing efficiency, and general capability preservation.
\section{Limitations}

The search time is the main limitation of the proposed EvoP, which is longer than existing pruning methods and usually takes about hours to run on one NVIDIA A100 GPU.
However, this search time should be accepted.
The reasons are: 
1) This is an offline process and only requires running once for multiple deployments; 
2) With the searched pruning pattern, the pruned model can consistently improve its performance, making it worthwhile to spend the time cost;
3) Furthermore, parallel or distributed computing can accelerate the evolutionary pruning pattern searching process, while existing implementations lack parallelism.
\bibliography{ref}
%\bibliographystyle{acl_natbib}
\end{document}
