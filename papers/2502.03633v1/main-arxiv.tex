\documentclass[a4paper,UKenglish,cleveref, linesno,autoref,thm-restate]{lipics-v2021}
%\documentclass[a4paper,UKenglish,anonymous,cleveref, linesno,autoref,thm-restate]{socg-lipics-v2021}
\usepackage[utf8]{inputenc}
\usepackage{xspace}
\usepackage{graphicx}
\usepackage{amsthm,amssymb}
\usepackage[dvipsnames]{xcolor}
\usepackage{enumitem}
\usepackage{url}
\usepackage{hyperref}

%\newtheorem{theorem}{Theorem}
%\newtheorem{lemma}[theorem]{Lemma}
%\newtheorem{observation}[theorem]{Observation}
%\newtheorem{conjecture}[theorem]{Conjecture}
%\newtheorem{corollary}[theorem]{Corollary}
%\newtheorem{definition}[theorem]{Definition}
%\newtheorem{claim}[theorem]{Claim}
\newtheorem{problem}[theorem]{Open Problem}

%\newenvironment{remark}{\noindent{\bf Remark.}}{}

\newcommand{\etal}{{\it{et al.}}\xspace}
\newcommand{\R}{\ensuremath{\mathbb R}}
\renewcommand{\S}{\ensuremath{\mathbb S}}
\newcommand{\nice}{{smooth}\xspace}

\newcommand{\parag}[1]{\\~\newline{\bf #1.}\newline}

\newcommand{\RR}{\ensuremath{\mathcal R}\xspace}
\def\from{\mathrel\subset\mkern-10mu\joinrel\sim}
%\newcommand{\mod}{\ensuremath{\mathbf{mod}}\:}
\newcommand{\eps}{\varepsilon}

\newcommand{\CH}{\ensuremath{\mathrm {CH}}}

\newcommand{\myremark}[4]{\textcolor{blue}{\textsc{#1 #2:}} \textcolor{#4}{\textsf{#3}}}

\newcommand{\ben}    [2][says]{\myremark{Ben}    {#1}{#2}{JungleGreen}}
\newcommand{\maarten}[2][says]{\myremark{Maarten}{#1}{#2}{WildStrawberry}}
% %%%%%%%%%%%Use instead to remove comments%%%%%%%%%
\renewcommand{\ben}    [2][says]{}
\renewcommand{\maarten}[2][says]{}

\newcommand{\remove}[1]{}




\title{Preprocessing Disks for Convex Hulls, Revisited}
%\title{Supersequences as Auxiliary Structures in the Preprocessing Framework for Uncertain Points}

%\titlerunning{Dummy short title} %TODO optional, please use if title is longer than one line

\author{Maarten L\"offler}%use full name
    {Department of Information and Computing Sciences; Utrecht University, the Netherlands}%mandatory
    {m.loffler@uu.nl}%
    {https://orcid.org/0009-0001-9403-8856}%
    {}

   \author{Benjamin Raichel}{Department of Computer Science;
      University of Texas at Dallas, USA \and
      \url{http://utdallas.edu/\string~benjamin.raichel} }
   {benjamin.raichel@utdallas.edu}%
   {{https://orcid.org/0000-0001-6584-4843}}%
   {Work on this paper was partially supported by NSF CAREER Award
      1750780 and NSF Award 2311179.}




\authorrunning{M. L\"offler and B. Raichel} %TODO mandatory. First: Use abbreviated first/middle names. Second (only in severe cases): Use first author plus 'et al.'

\Copyright{Maarten L\"offler and Ben Raichel} %TODO mandatory, please use full first names. LIPIcs license is "CC-BY";  http://creativecommons.org/licenses/by/3.0/

\ccsdesc[500]{Theory of computation~Computational geometry}%

\keywords{Convex hull, uncertainty, preprocessing} %TODO mandatory; please add comma-separated list of keywords

\category{} %optional, e.g. invited paper

\relatedversion{} %optional, e.g. full version hosted on arXiv, HAL, or other respository/website
%\relatedversiondetails[linktext={opt. text shown instead of the URL}, cite=DBLP:books/mk/GrayR93]{Classification (e.g. Full Version, Extended Version, Previous Version}{URL to related version} %linktext and cite are optional

%\supplement{}%optional, e.g. related research data, source code, ... hosted on a repository like zenodo, figshare, GitHub, ...
%\supplementdetails[linktext={opt. text shown instead of the URL}, cite=DBLP:books/mk/GrayR93, subcategory={Description, Subcategory}, swhid={Software Heritage Identifier}]{General Classification (e.g. Software, Dataset, Model, ...)}{URL to related version} %linktext, cite, and subcategory are optional

%\funding{(Optional) general funding statement \dots}%optional, to capture a funding statement, which applies to all authors. Please enter author specific funding statements as fifth argument of the \author macro.

\acknowledgements{
The authors would like to thank Ivor van der Hoog, Vahideh Keikha, and Frank Staals for inspiring discussions on the topic of this paper.
}%optional

\nolinenumbers %uncomment to disable line numbering



%Editor-only macros:: begin (do not touch as author)%%%%%%%%%%%%%%%%%%%%%%%%%%%%%%%%%%
\EventEditors{John Q. Open and Joan R. Access}
\EventNoEds{2}
\EventLongTitle{42nd Conference on Very Important Topics (CVIT 2016)}
\EventShortTitle{CVIT 2016}
\EventAcronym{CVIT}
\EventYear{2016}
\EventDate{December 24--27, 2016}
\EventLocation{Little Whinging, United Kingdom}
\EventLogo{}
\SeriesVolume{42}
\ArticleNo{23}
%%%%%%%%%%%%%%%%%%%%%%%%%%%%%%%%%%%%%%%%%%%%%%%%%%%%%%





\begin{document}

\remove{
{\bf TO DO FOR SOCG}:

\begin {itemize}
  \item rework introduction (halfway done)
  \item write new abstract
  \item look at SOSA reviews and see if anything needs to be done with those
  \item write down details for overlapping disks in 2D section
  \item write down details for sublinear reconstructio in 2D section
  \item merge sublinear sections of 1D and 2D part into a single new sectino?
  \item look at related work section again and maybe add more references
  \item think about what to put in the appendix
\end {itemize}

\maarten {and for completeness, this was our old todo list which was commented out, but may still contain relevant items:}

\begin {itemize}
  \item expand on related work section
  \item write the "our results" section (probably last thing to do)
  \item go through theorem statements and their proofs and check that they have consistent bounds; especially their dependence on parameters $\alpha$, $\beta$, or $\Delta$ (alternatively, remove all these explicit dependencies and say in theorem statements that they are considered to be constants?)
  \item check whether any remaining inline comments need action
  \item optionally, add more figures / unify figure style
  \item include a candid assessment of prior work in terms of simplicity and elegance
\end {itemize}
}


\maketitle

\begin{abstract}
In the preprocessing framework one is given a set of regions that one is allowed to preprocess to create some auxiliary structure such that when a realization of these regions is given, consisting of  one point per region, this auxiliary structure can be used to reconstruct some desired output geometric structure more efficiently than would have been possible without preprocessing. Prior work showed that a set of $n$ unit disks of constant ply can be preprocessed in $O(n\log n)$ time such that the convex hull of any realization can be reconstructed in $O(n)$ time. (This prior work focused on triangulations and the convex hull was a byproduct.) In this work we show for the first time that we can reconstruct the convex hull in time proportional to the number of \emph{unstable} disks, which may be sublinear, and that such a running time is the best possible. Here a disk is called \emph{stable} if the combinatorial structure of the convex hull does not depend on the location of its realized point.

The main tool by which we achieve our results is by using a supersequence as the auxiliary structure constructed in the preprocessing phase, that is we output a supersequence of the disks such that the convex hull of any realization is a subsequence. One advantage of using a supersequence as the auxiliary structure is that it allows us to decouple the preprocessing phase from the reconstruction phase in a stronger sense than was possible in previous work, resulting in two separate algorithmic problems which may be independent interest. Finally, in the process of obtaining our results for convex hulls, we solve the corresponding problem of creating such supersequences for intervals in one dimension, yielding corresponding results for that case.
\end{abstract}

\clearpage\tableofcontents





\clearpage
\section{Introduction}


\subparagraph*{Preprocessing framework.} 
The {\em preprocessing framework} for dealing with data uncertainty in geometric algorithms was initially proposed by Held and Mitchell \cite{held2008triangulating}. In this framework, we have a set $\RR = \{R_1, R_2, \ldots, R_n\}$ of {\em regions}, often in $\R^2$, and a point set $P = \{p_1, p_2, \ldots, p_n\}$ with $p_i \in R_i$
(we also write $P \from \RR$).
This model has 2 consecutive phases: a preprocessing phase, followed by a reconstruction phase. In the preprocessing phase we have access only to $\RR$ and we typically want to preprocess $\RR$ in $O(n \log n)$ time to create some linear-size auxiliary data structure which we will denote by $\Xi$. In the reconstruction phase, we have access to $P$ and we want to construct a desired output structure $S(P)$ on $P$ using $\Xi$ faster than would be possible otherwise (see Figure~\ref {fig:intro-prep}).
L{\"o}ffler and Snoeyink~\cite{loffler2010delaunay} were the first to use this model as a way to deal with data uncertainty: one may interpret the regions $\RR$ as {\em imprecise} points, and the points in $P$ as their true (initially unknown) locations. This interpretation of the preprocessing framework
has since been successfully applied to various problems in computational geometry~\cite {buchin2011delaunay,devillers2011delaunay,ezra2013convex, loffler2013unions,van2010preprocessing,hkls-pippf-22}.


\begin {figure}
 \begin{center}
  \includegraphics[width=\textwidth]{figures/intro-prep.pdf}
 \end{center}
 \caption {A set of disjoint unit disks $\RR$ can be preprocessed into an auxiliery structure $\Xi(\RR)$ in $O(n\log n)$ time, such that the convex hull of a set of points $P$ that respects $\RR$ can be computed in linear time using $\Xi(\RR)$ (compared to $\Theta(n \log n)$ time without preprocessing)~\cite {held2008triangulating,van2010preprocessing,BLMM11}.}
 \label {fig:intro-prep}
\end {figure}

\subparagraph*{Convex hull.} 
The {\em convex hull} of a set of points in $\R^2$ is the smallest convex set that contains all points.
The algorithmic problem of computing the convex hull is arguably one of the most fundamental problems in computational geometry~\cite {ch-survey}, and has been considered in an impressive range of different models for geometric uncertainty~\cite {DBLP:journals/cvgip/GoodrichS90,gss-cscah-93,nagai1999convex,edalat2001,hermann2007robust,loffler2010largest,evans2011possible,ezra2013convex,joskowicztopological,devillers2016smoothed,DBLP:journals/algorithmica/AgarwalHSYZ17,DBLP:conf/cccg/HuangR20}.
The first result for convex hulls within the preprocessing framework can in fact be derived from the work by Held and Mitchell~\cite {held2008triangulating}, who show that a set of disjoint unit disks can be preprocessed in $O(n \log n)$ time such that a triangulation can be reconstructed in linear time: since any triangulation contains at least the edges of the convex hull the convex hull can be extracted in the same amount of time (see Figure~\ref {fig:intro-prep}).
Similarly, the work by van Kreveld~\etal~\cite {van2010preprocessing} implies that a set of disjoint disks of arbitrary radii (or a set of disjoint polygons of constant complexity) can also be preprocessed in $O(n \log n)$ time for the same purpose, and the work by Buchin~\etal~\cite {BLMM11} implies that the same is true for a set of moderately overlapping unit disks. While these results focus on different problems (triangulation or Delaunay triangulation) and the convex hull is ``only a by-product'', Ezra and Mulzer~\cite {ezra2013convex} explicitly study the convex hull and show that any set of {\em lines} in the plane can also be preprocessed to speed up the reconstruction of the convex hull, although not to linear time: they achieve a reconstruction time of $O(n \alpha(n) \log^* n)$ in expectation.

\subparagraph* {Sublinear reconstruction.}
Traditionally, the aim in the preprocessing model has been to achieve reconstruction times that are faster than computing a solution from scratch, with the understanding that there is a natural lower bound of $\Omega(n)$ time to reconstruct $S(P)$, since even replacing each region of $\RR$ with the corresponding point in $P$ will take linear time.
However, for problems in which the output is an ordered sequence of points, this reasoning is not entirely satisfactory, for two reasons:
\begin {enumerate}
  \item it is possible that not all elements of $P$ appear in $S(P)$; in this case it is not a priori clear that we need to spend time retrieving points that do not need to be output;
  \item even for those points that do appear in $S(P)$, in some applications one might be happy knowing just the order and not the exact points; if this order is already clear from $\RR$ we also do not necessarily need to spend even $|S(P)|$ time in the reconstruction phase.
\end {enumerate}
Motivated by these observations, van der Hoog \etal~\cite {hkls-paip-19, hkls-pippf-22} have recently explored how the preprocessing model may be modified to allow for such sublinear reconstruction; 
for instance, for sorting they achieve reconstruction time proportional to the entropy of the interval graph, which can be less than $n$ if many intervals are isolated. 
In order to support sublinear reconstruction, van der Hoog \etal introduce an additional phase to the preprocessing framework, see Section~\ref{sec:1d-sub} for more details.

\begin {figure}
 \begin{center}
  \includegraphics{figures/intro-types.pdf}
 \end{center}
 \caption {(a) A set of disjoint unit disks $\RR$. (b) All possible combinatorial convex hulls. We identify five types of disks:
(i) {\em stable impossible interior} disks that never contribute to the convex hull (red);
(ii) {\em unstable potential interior} disks that may or may not contribute (orange);
(ii) {\em unstable potential boundary} disks that may or may not contribute (yellow);
(iii) {\em unstable guaranteed boundary} disks that are guaranteed to contribute, but for which the location may influence the structure of the hull (blue); and
(iv) {\em stable guaranteed boundary} disks that are guaranteed to contribute, and the structure is independent of their location (green).
 (Refer to Section~\ref {sec:classifications} for precise definitions).}
 \label {fig:intro-types}
\end {figure}

\subsection {Our contribution}

\subparagraph*{Preprocessing disks for convex hulls.} 
In this work, we investigate the problem of preprocessing a set of disks in the plane to reconstruct the convex hull directly for the first time. Doing so allows us to achieve features that are not possible when viewing the convex hull as a by-product of a triangulation, such as potential sub-linear reconstruction times. 
For some sets of disks we can already identify a subset of disks that is guaranteed to not take part in the convex hull, and a subset of disks that is guaranteed to take part in the convex hull (see Figure~\ref {fig:intro-types}), and thus we need not spend time on them during the reconstruction phase.

\begin {figure}
 \begin{center}
  \includegraphics{figures/intro-supseq.pdf}
 \end{center}
 \caption {(a) A set of disjoint unit disks $\RR$ and (b) a sequence of disks that is guaranteed to contain the vertices of the convex hull in the correct order. (c) A possible true convex hull.}
 \label {fig:intro-supseq}
\end {figure}


\subparagraph*{Supersequences.} 
As the main tool in our solution, we explore a new generic auxiliary structure: the {\em supersequence}.
In principle, it is applicable to any computational problem where the output structure $S(P)$ is an ordered subset of $P$; that is, $S(P) = \langle p_{i_1}, p_{i_2}, \ldots, p_{i_s} \rangle$ for some indices $\{i_1, i_2, \ldots, i_s\} \subseteq [n]$.
For such problems, an attractive option to use for the auxiliary structure $\Xi$ is a \emph {supersequence} of the regions corresponding to $S$; that is, $\Xi$ is a sequence of (possibly reccuring) elements of $\RR$ with the guarantee that, no matter where the true points $P$ lie in their regions, the sequence of elements of $P$ which we would obtain by replacing the regions in $\RR$ by their points will always contain $S(P)$ as a subsequence.
In the case of the convex hull $S(P)$ is the vertices of the convex hull in counter-clockwise order (the fact that this is a cyclic order slightly complicates things; in our solution we instead compute four {\em quarter hulls} separately) and $\Xi(P)$ is a sequence of disks (see Figure~\ref {fig:intro-supseq}). 
\ben{Figure 3 sequence is in clockwise not counterclockwise order}

An advantage of using a supersequence as auxiliary structure is that it allows us to decouple the preprocessing phase from the reconstruction phase into two separate algorithmic problems of independent value, in a stronger sense than was possible in previous work.
%It is related to the concept of 
In {\em heriditary} algorithms, introduced by Chazelle and Mulzer~\cite {hereditary09}, %in which 
one is given a structure % (such as the convex hull) 
on a superset, and is interested in computing the structure on a subset.
% (this problem is fairly easy for 2-dimensional convex hulls, Chazelle and Mulzer focus instead of 3-dimensional convex hulls in their work). 
Heriditary algorithms have been used in the context of the preprocessing framework, for instance, van Kreveld~\etal~\cite {van2010preprocessing} in their reconstruction phase first produce  a triangulation of red and blue points, where the red points are the points of interest, and then describe a linear-time hereditary algorithm for triangulations to obtain the final result.
Our approach of using supersequences is different in that we do not require explicit construction of a structure on a superset; instead, in the reconstruction phase we can trivially replace the disks by their points and are left with a pure computational problem (Open Problem~\ref {prob:2d}, in the case of convex hulls).

\subparagraph*{Preprocessing intervals for sorting.} 
As a subroutine in our solution, we require to preprocess certain sets of moderately overlapping unit intervals into a supersequence such that when we get a single value per interval, the sorted order can be reconstructed quickly. We believe this result is also of independent interest;
refer to Section~\ref {sec:1d-prep} for more background on this problem.
\ben{Added discussion in section 3}
%\ben{It appears [39], i.e. your paper mentioned above, is the optimal 1d result that the reviewer wanted us to discuss. So please discuss either here, above in the sublinear section, or later in the actual 1d section. }

%\subsection {Related work}


%In contrast, a supersequence as an auxiliary structure fairly naturally allows for sublinear reconstruction, by either simply omitting elements of $\RR$ that are certainly not in the ouput, or marking those (sequences of) elements of $\RR$ that are guaranteed to be in the correct place in the sequence and need not be inspected during reconstruction.

% \paragraph {Sorting under partial information.}

% Still a very active topic of research: \cite {haeupler2024fastsimplesortingusing, vanderhoog2024tightboundssortingpartial}.
% \ben{INCOMPLETE}

%\paragraph* {Uncertainty in geometric algorithms.}
%The study of geometric algorithms in uncertain settings is an active and by now well established area of research. 
%Similar to our setting, often papers in this area consider geometric regions as possible locations of points, and then go on to compute various quantities, such as the largest or smallest possible bounding box or convex hull (e.g.\ \cite{l-dicg-09} considers many variants). However, the objective in these papers often differs from that in our preprocessing model, as the true points are never revealed.
%Other papers consider models where points can be revealed but at a cost, and the goal is to compute some quantity while minimizing the cost of such queries. In one dimension this closely relates to sorting under partial information, which is an active topic of research \cite {haeupler2024fastsimplesortingusing, vanderhoog2024tightboundssortingpartial}. 

%In this paper we bound the complexity of our algorithms and output sequences in terms of the \emph{ply} of $\RR$ (i.e.\ the maximum number of regions overlapping any single point). 
%
%This is quite common in prior works in this area, many of which assume the set of regions $\RR$ to be pairwise disjoint, while others consider the ply to a constant or parameterize results in terms of the ply \cite{evans2013competitive, khramtcova2017dynamic}.
%Other papers working with geometric imprecision~\cite{evans2013competitive, khramtcova2017dynamic} 
%parameterize the degree of overlap in $\RR$ by the \emph{ply} (the maximum number of regions overlapping in a single point). 
%Their results are upper bound algorithms whose running times depends on the ply.
%\maarten {Expand (this paragraph was cut from somewhere else)}

%\ben{Please read the two paragraphs above, and suggest more specific citations for first paragraph. }

%\ben{reread rewritten our results section below}


% \subsection {Results \& organization}

% %Our main goal is to show that the problem of preprocessing unit disks of bounded ply for computing the convex hull can be reformulated using a supersequence as an auxiliary structure.
% %
% %We first consider the problem of preprocessing unit intervals of bounded ply in $\R^1$ for sorting, which can be seen as the $1$-dimensional equivalent of this question and is in fact needed as a subroutine. 



% % Our results are:
% % \ben{Either we need to state a simplified version of our resutls, or define $(\alpha,\beta)$-\nice.}
% % \begin {itemize}
% %   \item Preprocessing in 1D (Section~\ref {sec:1d-prep})
% %   \item Reconstruction in 1D (Section~\ref {sec:1d-rec})
% %   \item Preprocessing in 2D (Section~\ref {sec:2d-prep})
% %   \item Reconstruction in 2D (Section~\ref {sec:2d-rec})
% % \end {itemize}

% Recall that for a set of regions $\RR = \{R_1, R_2, \ldots, R_n\}$ and a point set $P = \{p_1, p_2, \ldots, p_n\}$, we write $P \from \RR$ to denote that $p_i \in R_i$ for all $i$. 
% Let $\Xi$ be a supersequence of $\RR$ and let $S$ be some sequence consisting of points from $P$.
% We say that $S$ is a subsequence of $\Xi$ if it is a subsequence of the result of replacing each region in $\Xi$ with its corresponding point from $P$.
% We have the following main results.

% % \begin {theorem} 
% %   Let $\RR$ be a set of $n$ unit intervals of ply $\Delta$.
% %   Then in $O(n (\Delta+\log n))$ time one can compute a supersequence $\Xi$ of $\RR$ of size $O(n\Delta)$ such that for any $P\from \RR$, the sorted order of $P$ is a subsequence of $\Xi$.
% %   (See Section~\ref {sec:1d-prep}.)
% %   Moreover, given $P$, the sorted order of $P$ can be computed from $\Xi$ in $O(n \Delta\log(\Delta))$ time. 
% %   (See Section~\ref {sec:1d-rec}.)
% % \end {theorem}

% % \begin{theorem}
% %   Let $\RR$ be a set of $n$ unit radius disks in the plane of ply $\Delta$.
% %   Then in $O(\Delta n \log n)$ time one can compute a supersequence $\Xi$ of $\RR$ of size $O(\Delta n)$ such that for any $P\from \RR$, the convex hull of $P$ is a subsequence of $\Xi$.
% %   (See Section~\ref {sec:2d-prep}.)
% %   Moreover, given $P$, the convex hull of $P$ can be computed from $\Xi$ in $O(\Delta^2 n)$ time.
% %   (See Section~\ref {sec:2d-rec}.)
% % \end{theorem}

% \begin{itemize}
%     \item Let $\RR$ be a set of $n$ unit intervals in $\R^1$ of ply $\Delta$.
%     In Section~\ref {sec:1d-prep}, we show that in $O(n (\Delta+\log n))$ time one can compute a supersequence $\Xi$ of $\RR$ of size $O(n\Delta)$ such that for any $P\from \RR$, the sorted order of $P$ is a subsequence of $\Xi$.
%     \item 
%     In Section~\ref {sec:1d-rec}, we show that given $P$, the sorted order of $P$ can be computed from $\Xi$ (as given in Section~\ref {sec:1d-prep}) in $O(n \Delta\log(\Delta))$ time. 
% \item
%   Let $\RR$ be a set of $n$ unit radius disks in the plane of ply $\Delta$.
%   In Section~\ref {sec:2d-prep}, we show that in $O(\Delta n \log n)$ time one can compute a supersequence $\Xi$ of $\RR$ of size $O(\Delta n)$ such that for any $P\from \RR$, the convex hull of $P$ is a subsequence of $\Xi$.
% \item
%   In Section~\ref {sec:2d-rec}, we show that given $P$, the convex hull of $P$ can be computed from $\Xi$ (as given in Section~\ref {sec:2d-prep}) in $O(\Delta^2 n)$ time.
% \end{itemize}
% \maarten {But these sections are written for unit disks, right?}

% %\ben{Added paragraph below on sublinear reconstruction.}

% %Additionally, we show that because our auxiliary structure is simply a supersequence, our results can easily be extended to the sublinear reconstruction case (when such sublinear times are possible). Specifically, once we have identified the regions that are guaranteed to occur in a given place in the ordering then we mark these regions, denoting that they can be skipped over during reconstruction. (In the case of convex hulls in $\R^2$, regions guaranteed to never occur will not even be in $\Xi$.) This is both simpler than what was required in prior work on sublinear reconstruction, and moreover, our results for sublinear reconstruction of the convex hull appears to be new. 


\subsection {Organization}

The remainder of this paper is organized as follows.
\begin {itemize}
  \item In Section~\ref {sec:2d}, we first establish the necessary vocabulary to be able to state our main results, which includes a classification of the input disks that is interesting in its own right. We then state our results for preprocessing unit disks for the convex hull, and discuss some of their limitations.
  \item Before proceeding with the proofs of our main results, in Section~\ref {sec:1d-prep}, we first switch to the one-dimensional problem of preprocessing a set of intervals for sorting, which is required as a subroutine later, but is also of independent interest. Our main result here is an algorithm to produce a supersequence of intervals that is guaranteed to contain any instance in sorted order.
  \item In Section~\ref {sec:1d-rec}, we then consider the corresponding problem of reconstructing the sorted order from such a sequence. In particular, we show that it is always possible to recover the sorted order from a {\em \nice} supersequence in linear time, but for general sequences this depends on the model of computation.
  \item Then, in Section~\ref {sec:2d-prep-disjoint} we return to the two-dimensional problem and discuss how to preprocess a set of disjoint unit disks as a supersequence of the convex hull. In Section~\ref {sec:2d-prep-overlapping} we extend this to overlapping unit disks using the one-dimensional result.
  \item In Section~\ref {sec:2d-rec}, we consider the corresponding reconstruction problem. Again, we show that it is always possible to recover the convex hull from a {\em \nice} supersequence. Whether it is possible for general supersequences is left as an open problem.
  \item Finally, in Section~\ref {sec:conclusion}, we summarize our findings and discuss directions for future research.
\end {itemize}










%\section{Preprocessing Disks for the Convex Hull in 2D} 
\section{Preliminaries, Results, and Limitations}
\label {sec:2d}

In this section we formally define the problem and state our result for preprocessing unit disks of bounded ply in $\R^2$ for reconstructing the {\em convex hull}.
Before we can state our results, we need to first discuss a {\em classification} of the input disks into five types, a notion of {\em quarter hull} to decompose the problem into smaller parts, and a notion of {\em smoothness}.


\subsection{Disk Characterization} \label {sec:classifications}


\begin {figure}
 \begin{center}
  \includegraphics{figures/EI.pdf}
 \end{center}
 \caption {The same example as Figure~\ref {fig:intro-types} with the regions $I$ and $E$ drawn in.}
 \label {fig:EI}
\end {figure}


We may classify disks\footnote {Although we mostly consider unit disks in this work, the classification in this section applies to (potentially overlapping) disks of any radii.} in the plane in multiple ways with respect to the convex hull in the preprocessing framework. First, we classify the disks on whether their corresponding points will appear on the convex hull or not.
We will define {\em impossible}, {\em potential}, and {\em guaranteed} disks.
%and make some basic observations about these classifications.
%, which we later use to mark disks for sublinear reconstruction.

\begin {definition}
  For a set of disks $\RR$ in the plane, we classify each disk $D\in \RR $ as one of the following three types: $D$ is either
  \begin{itemize}
      \item \emph{impossible} when its realization is never a vertex of the convex hull of any realization of $\RR$
      \item \emph{guaranteed}\! when its\! realization is always a vertex of the\! convex hull of any realization of $\RR$
      \item \emph{potential} when it is neither impossible nor guaranteed.
  \end{itemize}
 \end {definition}

\ben{Reviewer also suggested citing "Efficient Update Strategies for Geometric Computing with Uncertainty" which give some similar definitions/observations.  We should probably mention this, and more importantly the two papers below, somewhere in the intro.}

Prior work has considered the union or intersection of all possible realizations of $\RR$ under different names (see for example \cite{nyt-chpii-98,Sember2011}), and these structures are  closely related to our above classification of disks. In particular, the above definitions can be equivalently phrased in terms of inclusion or exclusion in certain regions; the following observations have been made before.
%\begin{definition}\label{def:polygons}
Let $I$ be the intersection of all halfplanes that intersect all disks. 
Let $E$ be the intersection of all halfplanes that fully contain at least $n-1$ disks. See Figure~\ref{fig:EI}. 
%\end{definition}
\begin{observation}\label{obs:polygons}
%First, observe that the above three types partition the disks (i.e.\ each disk has exactly one type).  
A disk is impossible if and only if it is contained in $I$, and a disk is guaranteed if and only if it lies outside $E$.  
%Note that equivalently a disk $D$ is guaranteed if $D$ lies entirely outside the convex hull of $\RR\setminus \{D\}$.
\end{observation}

Second, we may also classify the disks according to whether the location of their corresponding point influences the combinatorial structure of the convex hull or not.

\begin {definition}
  A disk $D \in \RR$ is called {\em stable}, if for any fixed realization $P'$ of $\RR'=\RR \setminus \{D\}$, the combinatorial structure\footnote{The ordered list of disks corresponding to the clockwise sorted order of the vertices of the convex hull.} of the convex hull of  $P'\cup\{p\}$ is the same for any realization $p$ of $D$. 
%
%  A {\em stable disk} is a disk $R \in \RR$ such that such that, for all realizations of $\RR \setminus \{R\}$, it is the case that the combinatorial structure\footnote{The ordered list of disks corresponding to the clockwise sorted order of the vertices of the convex hull.} of the convex hull of everything does not depend on the location of the point in $R$. 
  A disk that is not stable is \emph{unstable}.
\end {definition}

Note that stable disks are candidates for saving on reconstruction time, as they need not necessarily be inspected during the reconstruction phase to recover the convex hull. So, the best we may hope for is a reconstruction time proportional to the number of unstable disks.

Third, we classify disks depending on whether they appear on the convex hull of the disks themselves.

\begin {definition}
  A disk $D \in \RR$ is called a {\em boundary} disk if the boundary of $\CH(\RR)$ intersects $D$; otherwise $D$ is called an {\em interior} disk.
\end {definition}

These 3 different classifiers for each disk, in principle give $3 \times 2 \times 2 = 12$ different types of disks. However, due to certain dependencies many of these types are not possible.

\begin{lemma} \label {lem:implications}
We observe the following implications:
\begin {itemize}
  \item every guaranteed disk is a boundary disk;
  \item every potential disk is unstable; and
  \item every impossible disk is both interior and stable.
\end {itemize}
\end{lemma}

\begin {proof}
First, observe that a guaranteed disk $D\in \RR$ must be a boundary disk, as by Observation~\ref{obs:polygons}, $D$ must lie outside $E$, or equivalently outside the convex hull of $\RR\setminus \{D\}$. 

Second, a potential disk $D\in\RR$ must be unstable, as $D$ being potential means there is both a realization of $\RR$ where $D$ is realized as a vertex of the convex hull and a realization of $\RR$ where it is not a vertex of the convex hull, and thus the combinatorial structure depends on the realization of $D$.%
\footnote{The definition of stable does not directly prohibit a scenario where for some fixed realization of $\RR\setminus \{D\}$, any realization of $D$ is always a vertex of the convex hull, while for some other fixed realization of $\RR\setminus\{D\}$, any realization of $D$ is never a vertex of the convex hull. However, such a scenario is impossible due to the continuity of our disk based uncertain regions.}

Finally, an impossible disk must be both interior and stable. Clearly it is stable as its realization is never a vertex of the convex hull and thus does not affect its combinatorial structure. It is interior as by Observation~\ref{obs:polygons} it lies inside $I$ which lies strictly inside the conex hull of $\RR$.
\end{proof}

The above observation implies there are only 5 possible types of disks are (see 
Figure~\ref{fig:EI}).
%
%Due to space, the proof of the following important lemma was moved to Appendix~\ref{subproofs}.

\begin{lemma}\label{lem:stable}
Given a set $\RR$ of $n$ disks, we can classify them in $O(n \log n)$ time.
\end{lemma}

%\newcommand{\classifyproof}{
\begin{proof}[Proof of Lemma~\ref{lem:stable}]
First, the boundary disks can be easily detected in $O(n \log n)$ time by computing the convex hull of the disks~\cite {Devillers1995IncrementalAF}.
Second, Observation~\ref {obs:polygons} allows us to classify the disks into guaranteed, potential, and impossible disks using a simple $O(n \log n)$ algorithm~\cite {nyt-chpii-98}.
By Lemma~\ref {lem:implications} all impossible interior disks are stable. 
It remains to distinguish the stable from the unstable guaranteed boundary disks.

Let $D\in \RR$ be a stable guaranteed boundary disk. Let $B(\RR)=\{B_1,\ldots,B_k\}$ be the ordered set of boundary disks. As $D$ is a boundary disks, $D=B_i$ for some $i$, where $B_i$ has two adjacent strips, namely $\CH(B_{i-1},B_i)$ and $\CH(B_{i},B_{i+1})$. \maarten {strips are not defined yet}

First, observe that no disk in $\RR\setminus \{B_{i-1},B_i\}$ can properly intersect the strip  $\CH(B_{i-1},B_i)$ (and the same holds for $\CH(B_{i},B_{i+1})$). To see this, suppose there is at least one such disk, and that its realization in inside this strip. 
Now if $B_{i-1}$ and $B_i$ are realized at the points in the disks that are extreme in the direction orthogonal to the spine of this strip and exterior to the hull, then any realization of any other disk intersecting this strip cannot be a vertex of the convex hull. Conversely, if $B_{i-1}$ and $B_i$ are realized at the opposite extremes, then at least one of the other disks whose realization is in this strip will be a vertex of the convex hull.

By the same reasoning, $B_{i-1}$ must be a guaranteed disk (and similarly for $B_{i+1}$). Specifically, if $B_{i-1}$ is not guaranteed it intersects the convex hull of $\RR\setminus \{B_{i-1}\}$, and because $B_{i-1}$ is a boundary disk, it in fact intersects a strip of $\RR\setminus \{B_{i-1}\}$. Moreover, this strip must have $B_i$ as one of its defining disks as the strip $\CH(B_{i-1},B_i)$ had no intersecting disks. Thus by the same reasoning used to argue the strip $\CH(B_{i-1},B_i)$ had no intersecting disks, $B_{i-1}$ cannot intersect this strip, and so must be guaranteed. 

So in summary, in order for a guaranteed boundary disk to be stable, its adjacent strips must be empty and its neighbors on the strip hull must be guaranteed. \maarten {maybe this sentence could be moved out of this proof and made into its own lemma, put above this one?}
Furthermore, it is clear that any guaranteed boundary disk with these properties is stable as then the realizations of $B_{i-1},B_i,B_{i+1}$ all appear on the convex hull as they are all guaranteed, and no vertex can occur between the realization of $B_{i-1}$ and $B_i$ (nor $B_{i-1}$ and $B_{i+1}$), effectively blocking $D=B_i$ from affecting the combinatorial structure of the rest of the convex hull.

As for the running time for finding the stable guaranteed boundary disk, as discussed above we can already determine all boundary disks, and for each such disks whether it and its neighbors are guaranteed, in $O(n\log n)$ time. Thus we only determine whether the adajacent strips are empty, though this can be done for all stips in $O(n\log n)$ time using  Lemma~\ref{lem:comprri}.
%
%computing the sorted order of boundary disks can be done in $O(n\log n)$, and Lemma~\ref{lem:comprri} tells us we can also determine the sets of disks intersecting every strip in $O(n\log n)$ time. Thus after these computations, for each boundary disk in constant time we can determine its neighboring boundary disks and whether the corresponding strips are empty. 
\end{proof}
%}

%For a set of disks $\RR$ in the plane, let $T$ be the subset of stable guaranteed boundary disks. By Lemma~\ref {lem:stable}, $T$ can be computed in $O(n\log n)$ time. 

%There are two types of stable disks: stable guaranteed boundary disks and stable impossible interior disks.  
%Stable impossible interior disks do not occur in our output supersequence, but stable guaranteed boundary disks do. In order to avoid spending too much time on them during reconstruction, we need to make a minor modification to the preprocessing phase: we should mark which disks in $\Xi$ are stable and which are not.


\subsection{Quarter Hulls}

Note that the convex hull is cyclic in nature: it has no well-defined start or end point, 
and in the presence of uncertainty, it may be difficult even to designate an arbitrary disk as start point, as there may not be any guaranteed disks.

To make our lives easier, we opt to focus on only one quarter of the convex hull: the part from the rightmost point on the hull to the topmost point on the hull (i.e. the positive quadrant).
Indeed, the idea of splitting the computation of the convex hull into separate pieces is well established~\cite {ANDREW1979216}, and choosing quarters will have the added benefit that within such a quarter, no vertex can have an internal angle smaller than $90^\circ$, which will aid our geometric arguments later.

Formally, the {\em quarter hull} of a point set $P$ is the sequence of points on the convex hull starting from the rightmost point in $P$ and following the hull in  counterclockwise order until ending with the topmost point in $P$.


We will define a supersequence for one quadrant; clearly, to obtain a supersequence for the entire hull, one can simply repeat this process four times for appropriately rotated copies of $\RR$.
%
Note that in subsequent sections, depending on the context, the term \emph{convex hull} may refer to either the quarter hull or the full convex hull. \maarten {ideally, remove this sentence, assuming we have fixed all occurences}

\maarten {Maybe also say here:}
In the remainder of this work, we also adapt the notation established in Section~\ref {sec:classifications} to the notion of quarter hulls.
That is, a disk will be called {\em guaranteed} or {\em impossible} when it is guaranteed or impossible to contribute to the quarter hull,
and it will be {\em stable} when it does not influence the combinatorial structure of the quarter hull.

Notably, observe that the total number of unstable disks for the entire convex hull is at most a constant number more than the sum of the number of unstable disks in the four quarter hull problems; hence, dividing the problem into quarters does not jeopardize our aim to achieve sublinear reconstruction times.

\subsection {Supersequences}

We are now ready to formally define supersequences.

\begin {definition}
  Let $\RR$ be a set of $n$ regions in the plane.
  A {\em quarter-hull-supersequence} of $\RR$ is a sequence $\Xi$ of (possibly reoccurring) disks of $\RR$
  such that,
  for any point set $P \from \RR$, 
  %the sequence of points on the (counterclockwise) top right quarter of the convex hull of $P$ 
  the quarter hull of $P$
  is a subsequence of $\Xi$.
  
\end {definition}

%Similar to the 1D case, 
We will focus on the case where $\RR$ is a set of unit disks of bounded ply $\Delta$ in this work, where the {\em ply} of a set of regions is the maximum number of regions that any point in the plane is contained in.
%
Replacing the disks in a quarter-hull-supersequence by their points, we now obtain a sequence of points that contains all points of the quarter hull of $P$ in the correct order. Since it is not clear if the convex hull of any such sequence can be computed in linear time (Open Problem~\ref {prob:2d}), we additionally introduce {\em \nice} sequences. 

%\subsection{Smoothness}
%, we will use the signed distance between points in the $(-1,1)$ direction.
%$y-x$ direction. \ben{$(1,-1)$ direction?}
For two points $p, q \in \R^2$, let $\nwarrow(p,q)$ be the signed distance from $p$ to $q$ projected on a line with slope $-1$; that is, 
\[\nwarrow(p,q) = (p.x - p.y)-(q.x - q.y).\]

\begin {definition}
  A sequence $X$ of points in $\R^2$ is {\em $(\alpha, \beta)$-\nice} if:
  \begin {itemize}
    \item for any two elements $p_i$ and $p_j$ in $X$ with $i < j$ we have
      that if $p_i$ is on the quarter hull of $X$, then $\nwarrow(p_j,p_i) \le \alpha$, and symmetrically if $p_j$ is on the quarter hull of $X$, then $\nwarrow(p_i,p_j) \ge -\alpha$.
    \hfill (distance property)
    \item for any disk $D$ in the plane of diameter $d>1$, there are only $\beta d$ distinct points from the sequence in $D$.
    \hfill (packing property)
  \end {itemize}
\end {definition}

Essentially, a \nice supersequence may have points that are out of order, but the extent in which they can be out of order is limited by the parametesr $\alpha$ and $\beta$, which makes the reconstruction problem significantly easier.

We will also say a quarter-hull-supersequence $\Xi$ is {\em $(\alpha, \beta)$-\nice} if the corresponding sequence of points is {\em $(\alpha, \beta)$-\nice} for any $P \from \RR$.


\subsection{Statement of results}

We now state our main results for preprocessing unit disks in the plane for the convex hull. Our first three theorems imply that the classical result of $O(n \log n)$ preprocessing for $O(n)$ reconstruction (as in Figure~\ref {fig:intro-prep}) can indeed be replicated for unit disks of bounded ply using supersequences.

%The goal of this section is to prove the following theorems. 
\begin {theorem} \label{thm:mainunitdisks-existence}
  Let $\RR$ be a set of $n$ unit disks in the plane of ply $\Delta$.
  There exists a $(3\sqrt2,12\Delta)$-\nice quarter-hull-supersequence $\Xi$ of $\RR$ of size $O(\Delta^2 n)$.
\end {theorem}

\begin {theorem} \label{thm:mainunitdisks-preprocessing}
  Let $\RR$ be a set of $n$ unit disks in the plane of ply $\Delta$.
  A $(3\sqrt2,12\Delta)$-\nice quarter-hull-supersequence $\Xi$ of $\RR$ of size $O(\Delta^2 n)$ can be computed in $O(\Delta n(\Delta + \log n))$ time.
\end {theorem}

\begin {theorem} \label {thm:mainunitdisks-reconstruction}
  Let $\RR$ be a set of $n$ unit disks.
  Given an $(\alpha,\beta)$-\nice quarter-hull-supersequence $\Xi$ of $\RR$,
  and given a point set $P \from \RR$, the quarter hull of $P$ can be computed in $O(\alpha\beta|\Xi|)$ time. 
\end {theorem}

The next theorem implies that we can also obtain sublinear reconstruction using this method.

\begin{theorem}\label {thm:sublinear-reconstruction}
 Let $\RR$ be a set of $n$ unit disks. We are given an $(\alpha,\beta)$-\nice quarter-hull-supersequence $\Xi$ of $\RR$ in which all its stable disks have been marked, and let $\mu$ denote the number of unstable disks in $\Xi$. Then given a point set $P \from \RR$, the quarter hull of $P$ can be computed in $O(\alpha\beta \mu)$ time. 
\end{theorem}

When $\RR$ has ply $\Delta$,  Theorem~\ref{thm:mainunitdisks-preprocessing} gives a $(3\sqrt{2},12\Delta)$-\nice quarter-hull-supersequence $\Xi$ of size $O(\Delta^2 n)$. Thus we immediately have the following corollary of the theorems above. %Theorem~\ref{thm:mainunitdisks-reconstruction}.

\begin{corollary}
%If $\RR$ has ply $\Delta$, then for the $(3\sqrt{2},12\Delta)$-\nice quarter-hull-supersequence of Theorem~\ref{thm:mainunitdisks-preprocessing}, the quarter hull of $P$ can be computed in $O(\Delta^2 n)$ time.
%
If $\RR$ has ply $\Delta$, then for the $(3\sqrt{2},12\Delta)$-\nice quarter-hull-supersequence of Theorem~\ref{thm:mainunitdisks-preprocessing} with its stable disks marked, then the quarter hull of $P$ can be computed in $O(\Delta^3 m)$ time, where $m$ is the number of unstable disks in $\RR$.
\end {corollary}


For ease of exposition, in Section~\ref{sec:2d-prep-disjoint} we first prove Theorems \ref{thm:mainunitdisks-existence} and \ref{thm:mainunitdisks-preprocessing} for the case $\Delta = 1$; i.e., disjoint unit disks.

In Section~\ref{sec:1d-prep} we solve the problem of preprocessing a set of unit intervals of bounded ply in one dimension, which is needed to prove Theorems~\ref{thm:mainunitdisks-existence} and \ref{thm:mainunitdisks-preprocessing} in full generality for disks of bounded ply (i.e.\ $\Delta > 1$) in Section~\ref {sec:2d-prep-overlapping}.

In Section~\ref{sec:2d-rec} we prove Theorems~\ref{thm:mainunitdisks-reconstruction} and~\ref {thm:sublinear-reconstruction}, which are decoupled from the preprocessing phase and hold for any set of disks.

\subsection {Limitations \& open problems}


\subparagraph* {Restrictions on the supersequences.}
Our reconstruction algorithms require not only that the auxiliary structure is a supersequence of the desired output, but we also require that these sequences are {\em \nice}. 
This property is naturally fulfilled by our preprocessing algorithms and simplifies the reconstruction task.
However, it is not clear whether this restriction is necessary for the reconstruction problem, and being able to reconstruct the convex hull from {\em any} supersequence would even further decouple the preprocessing and reconstruction phases.

%\begin {problem} \label {prob:1d}
%  Given a sequence of values $P$ in $\R$ (possibly with duplicates), and the guarantee that there exists a sorted subsequence $Q$ of $P$ such every distinct element in $P$ appears at least once in $Q$.
%  Is it possible to sort $P$ in $o(n \log n)$ time?
%\end {problem}

\begin {problem} \label {prob:2d}
  We are given a sequence of points $P$ in $\R^2$ (possibly with duplicates), and the guarantee that there exists a subsequence $Q$ of $P$ such that the convex hull of $P$ equals the convex hull of $Q$, and the points in $Q$ are sorted in counterclockwise order.
  Is it possible to compute the convex hull of $P$ in $o(n \log n)$ time?
\end {problem}

Note that, although we require handling duplicates in our application, Problem~\ref {prob:2d} is also open even when all points are unique.
We believe this question is of independent interest.


\subparagraph* {Model of computation.}
We assume we are working in the real RAM model of computation, which is consistent with earlier work in our framework. (For a formal discussion of the Real RAM, refer to \cite{ehm22}.) 
In order to achieve linear reconstruction time, we require our supersequences to be \nice, however, in Section~\ref{sec:1d-rec} we do show that reconstruction in $\R^1$ can be done in linear time, without this \nice restriction, in the word RAM model, where universal hashing is allowed. 

\subparagraph* {General position.}
For ease of exposition, throughout we assume {\em general position} of the disk centers in the preprocessing phase and general position of the true points in the reconstruction phase; in both cases for us general position means no three points (or centers) are collinear.
However, because we actively duplicate regions and points in our sequences, we will not be
able to assume full general position in all subroutines (e.g.\ the discussion in Section~\ref{sec:1d-rec} hinges critically on the potential existence of duplicates).
Our general position assumptions may be completely lifted by using
symbolic perturbation \cite{em-ss-90}.











%\newcommand{\onedsection}{
\section{Preprocessing Intervals for Sorting in One Dimension} \label {sec:1d-prep}

In this section we consider the problem of preprocessing unit intervals of bounded ply in $\R^1$ such that the {\em sorted order} of a corresponding set of points (values) can be recovered quickly.
This will be later used as a subroutine in Section~\ref{sec:2d-prep-overlapping}.
We remark that the problem of optimally preprocessing intervals for sorting was considered in prior work (see for example \cite{hkls-paip-19} and reference therein). Here though we require that our output structure is a supersequence, as this is what will be required to use it as a subroutine for our results in $\R^2$. 


\begin {definition}
  Let $\RR$ be a set of $n$ intervals.
  A {\em sorting-supersequence} of $\RR$ is a sequence $\Xi$ of (possibly reoccurring) intervals of $\RR$
  such that,
  for any point set $P \from \RR$, the sorted order of $P$ is a subsequence of $\Xi$.
\end {definition}


While this definition works for any set of intervals, we will focus on unit intervals of ply $\Delta$ in this work.
Furthermore, we will consider an additional restriction on the sequences, and we will show that such a restricted sequence can both be computed on a set of  unit intervals of ply $\Delta$, and used to recover the sorted order; whether this restriction is necessary is left as an open problem (and may depend on the model of computation; refer to Section~\ref {sec:1d-rec}).


\begin {definition}
  A sequence of values $X=x_1,\ldots, x_n$ is {\em $(\alpha, \beta)$-\nice} if: 
  \begin {itemize}
    \item for any two elements $x_i, x_j \in X$ such that $i < j$ but $x_i > x_j$, we have $x_i - x_j \le \alpha$, and
    \hfill (distance property)
    \item for any interval $J \subset \R$ with $|J| > 1$, there are $\leq \beta |J|$ distinct elements from $X$ in $J$.
    \hfill (packing property)
  \end {itemize}
\end {definition}

We will also say a sorting-supersequence $\Xi$ is {\em $(\alpha, \beta)$-\nice} if the corresponding sequence of points (values) is {\em $(\alpha, \beta)$-\nice} for any $P \from \RR$.

The goal of this section is to prove the following theorems.
\begin {theorem} \label{thm:mainunitintervals-preprocessing}
  Let $\RR$ be a set of $n$ unit intervals of ply $\Delta$.
  There exists a $(3, 2\Delta)$-\nice sorting-supersequence $\Xi$ of $\RR$ of size $O(n\Delta)$.
  Furthermore, such a sequence can be computed in $O(n (\Delta+\log n))$ time.
\end {theorem}

\begin {theorem} \label{thm:mainunitintervals-reconstruction}
  % Let $\RR$ be a set of $n$ unit intervals of ply $\Delta$.
  % Given a $(\alpha, \beta)$-\nice sorting-supersequence $\Xi$ of $\RR$,
  % and given a point set $P \from \RR$, the sorted order of $P$ can be computed in $O(\alpha\beta|\Xi|)$ time. 
Let $\RR$ be a set of $n$ unit intervals. Given an $(\alpha, \beta)$-\nice sorting-supersequence $\Xi$ of $\RR$, and given a point set $P \from \RR$, the sorted order of $P$ can be computed in $O(|\Xi|\log(\alpha \beta))$ time. 
\end{theorem}

When $\RR$ has ply $\Delta$,  Theorem~\ref{thm:mainunitintervals-preprocessing} gives a $(3,2\Delta)$-\nice sorting-supersequence $\Xi$ of size $O(n\Delta)$. Thus we immediately have the following corollary of Theorem~\ref{thm:mainunitintervals-reconstruction}.

\begin{corollary}
If $\RR$ has ply $\Delta$, then for the $(3,2\Delta)$-\nice sorting-supersequence of Theorem~\ref{thm:mainunitintervals-preprocessing}, the sorted order of $P$ can be computed in $O(n \Delta\log(\Delta))$ time.
\end {corollary}

%Due to space constraints the remainder of this section has been moved to Appendix~\ref{apnd:1d}. The main workhorse to prove Theorem~\ref{thm:mainunitintervals-preprocessing} is the following lemma, which we will need in Section~\ref{sec:2d}, and whose proof is in Appendix~\ref{apnd:1d}.

% \begin{lemma} \label{lem:interval-windows}
%   Let $\cal I$ be a set of $n$ unit intervals in $\R$ of ply $\Delta$.
%   There is a $(3, 2\Delta)$-\nice sorting-supersequence $\Xi$ of $\cal I$ with at most $4\Delta n$ intervals.
% \end{lemma}

\subsection{Preliminaries}

%\ben{Add references from reviewer that complained. (Add forward ref in intro that we are covering prior work here.) Remark that the purpose here is because it is needed later, and that here we will output a supersequence, which was not done in prior work (I believe).}

\paragraph* {Supersequences and sorting.}
%\maarten {more on supersequences?}
A {\em universal word} is a string that contains as subsequences all permutations of its distinct characters.
The question of constructing the shortest universal word was first posed by Knuth and attributed to Karp~\cite {10.5555/891957}.
The length of known shortest universal words are in \cite {oeis/A062714}.
The best known upper bound is due to Tan~\cite {tan2022skiplettersshortsupersequence}.
An earlier bound of $n^2 - 7/3n + 19/3$ was given by Radomirovic~\cite {Radomirovic2012ACO}.
His proof is constructive and also gives an $O(n^2)$ algorithm to construct such a word.
  Uzna\'{n}ski shows that testing whether a given string is a universal word is coNP-complete~\cite {u2015permutationssupersequenceconpcomplete}.

The problem of finding a sorted sequence in a supersequence is closely related to the {\em longest increasing subsequence} problem. 
%\maarten {more literature on that here?}
Fredman shows that the problem has a $\Omega (n \log n)$ lower bound~\cite{FREDMAN197529}. There is also a matching $O(n\log n)$ dynamic programming algorithm, which is commonly taught in advanced algorithms courses. 
\maarten {would be good if we can replace the last part of this sentence by an actual citation}

\subsection {Preprocessing algorithm} \label {sec:1d-prep-algo}

We will use the following result.
\begin {lemma} [\cite{csscap-r-12}] \label {lem:shortestsuperseqeunce}
  The length of the shortest sequence on $m$ symbols which contains all permutations of $[1,m]$ as a subsequence is upper-bounded by $m^2$. Moreover, such a sequence of length $m^2$ can be computed in $O(m^2)$ time.
\end {lemma}
%\maarten {check http://oeis.org/A062714 and references mentioned there}


%We remark that the above sequence can be computed in linear time in its length, i.e.\ $O(m^2)$ time. 
While Radomiroci\'c~\cite{csscap-r-12} does not discuss the running time, it is implicit in their construction. 
Ultimately we will apply it when $m$ is the ply, which we assume is bounded.
%
Using this result, we prove the following technical lemma, which we will also reuse later in Section~\ref {sec:2d}.

\begin{lemma}\label{lem:interval-windows}
  Let $\cal I$ be a set of $n$ unit intervals in $\R$ of ply $\Delta$.
  There is a $(3, 2\Delta)$-\nice sorting-supersequence $\Xi$ of $\cal I$ with at most $4\Delta n$ intervals.
%  There is a sequence $\Xi$ of at most $4\Delta n$ intervals such that:
%    for any point set $P \from {\cal I}$, the order of the points is a subsequence of $\Xi$.
%  Furthermore, any such sequence is $(2, 2\Delta)$-\nice.
\end{lemma}

\begin {figure} \label {fig:interval-windows}
%  \includegraphics [width=\textwidth] {figures/interval-windows}
%  \maarten {Maybe needs to be like this (see below):}
  \includegraphics [width=\textwidth] {figures/interval-windows-alternative}
  \caption {An example of a sequence of unit intervals of ply $\Delta=3$.}
\end {figure}

Note, to get a sequence of length $n \Delta^2$, we could simply apply Lemma~\ref {lem:shortestsuperseqeunce} to every cell of the subdivision induced by $I$. However, Lemma~\ref{lem:interval-windows} gives a more refined bound.

\begin {proof}
  Assume the intervals ${\cal I} = \{I_1, \ldots, I_n\}$ are ordered and indexed from left to right (since they are unit intervals, this order is well-defined).
  Now, let $S_1 = \{I_i \mid I_i \cap I_1\neq \emptyset\}$ be the set of all intervals that intersect the leftmost interval (including the leftmost interval itself) and let $\eta_1 = \min \{i\mid I_i \cap I_1 = \emptyset\}$ be the index of the first interval not in $S_1$.
  We recursively define $S_{i+1}$ to be the set of all intervals that intersect $I_{\eta_{i}}$ and are not in $S_{i}$.
  Then, let $W_i = \bigcup_{I \in S_i} I$ be the window from the left endpoint of the leftmost interval in $S_i$ to the right endpoint of the rightmost interval in $S_i$.
  
  Now, define $S'_i$ to be the set of all intervals from $S_i$ unioned with all intervals from $S_{i+1}$ which intersect $W_i$.
  %, hence containing all intervals that intersect any interval in $S_i$. 
  We claim that $|S'_i| \le 2 |S_i| - 1 \le 2\Delta - 1$. This is because any interval from $S_{i+1}$ intersecting $W_i$, that is not already in $S_i$, must contain the right endpoint of $W_i$, and as $\Delta$ is the ply, there can therefore be at most $\Delta - 1$ such intervals.
  We further claim that every pair of intervals that intersect each other is present in some set $S'_i$: indeed, an interval from $S_i$ cannot intersect an interval from $S_{i+2}$ since then the interval in $S_{i+2}$ would also intersect the leftmost interval of $S_{i+1}$ (but then it would be part of $S_{i+1}$).
%  \ben{How is the second half of the above paragraph used? In arguing for $\alpha$?}
%  \maarten {No, it is used to argue why $\Xi$ contains every possible order for $P$ (if there would be a pair of intersecting intervals that is not in one of these groups, then they would not necessarily appear in both possible orders in $\Xi$).} 
%  \ben{Agreed, but where is it actually used/referenced below? i.e.\ it is a necessary property, but seems to be implied by the argument a couple paragraph below.}

  Let $k$ be the resulting number of $S_i'$ sets.
  We process these sets of intervals from left to right, and apply Lemma~\ref {lem:shortestsuperseqeunce} to $S'_i$ for $i = 1, \ldots, k$. Let the resulting $i$th sequence be $\Xi_i$. Then we simply concatenate the resulting sequences, to get a sequence $\Xi = \Xi_1 \cup \Xi_2 \cup \cdots \cup \Xi_{k}$.

 As argued above, $|S_i'|\leq 2\Delta$ and moreover $\sum_{i=1}^k |S_i'| \leq 2\sum_{i=1}^k |S_i|=2n$. By Lemma~\ref {lem:shortestsuperseqeunce}, $|\Xi_i|=|S_i'|^2$. Thus $|\Xi|$  is bounded by a sum of squares of values, i.e.\ the $|S_i'|$, which are each bounded by $2\Delta$ and sum to at most $2n$. It is not hard to see that such a sum of squares is maximized when $S_i'=2\Delta$ for all $i$ and $k= n/\Delta$. Thus the length of $\Xi$ is at most $n/\Delta \cdot 4\Delta^2 = 4n\Delta$.
%  \ben{Fixed, but still would be nice to cite an inequality.}

  Now, let $P \from {\cal I}$ be any realization, listed in sorted left to right order on the real line.  
  We divide the points in $P$ into $k$ groups: let $P_i \subset P$ be the set of all points that are contained in $W_i$ but not in $W_{i-1}$.
%  
  Let $\Pi=\Pi(P)$ be the ordered string of intervals corresponding to the ordered points of $P$, and let $\Pi_i$ be the part of $\Pi$ corresponding to $P_i$.
 The string $\Pi_i$ is a substring of $\Xi_i$ (by Lemma~\ref {lem:shortestsuperseqeunce}).
%
 Therefore, the points in $P_i$ can be charged to the subsequence $\Xi_i$, thus proving $\Xi$ is a sorting-supersequence, as the $\Xi_i$ are concatenated in order.

  
  %The realized points of $\Pi_i$ must fall within the window $W_i$ on the real line, and since as argued above only intervals from $S'_i$ can intersect $W_i$, it follows that $\Pi_i$ can only contain intervals from $S'_i$. As we explicitly added a subsequence containing all permutation of $S'_i$, our full sequence must contain $\Pi_i$.


%  To get $n \Delta$, ... be more smart?
%  \maarten {how to be smart:}
%  - scan from left to right
%  - consider the "profile" of how many intervals overlap at any point in time
%  - when at a local maximum \maarten {<- unimportant?}, we have at most $\Delta$ intervals
%  - consider the set of at most $3\Delta$ (?) intervals, including those nearby on both sides.
%  - take the sequence of length $9\Delta^2$ including all possible permutations of those $3\Delta$ intervals
%  - now move forward until all $\Delta$ intervals are no longer containing our pointer
%  - this should lead to $n/\Delta \cdot 9\Delta^2$.

%  Alternatively, we can imagine an "obvlivous" algorithm, which simply orders all intervals left-to-right, and takes the first $3\Delta$, then skips the first $\Delta$ ones and adds the next $\Delta$, etc. This might always work independent from the geometry of the intervals.

  Finally, we need to argue that the resulting sequence $\Xi$ is $(3, 2\Delta)$-\nice.
  To argue that $\alpha = 3$, note that all intervals in $S_i$ intersect each other, and thus all intervals in $S'_i$ either intersect each other or intersect a common interval. In the first case, the distance between any two points in them is at most $2$; in the latter case they could have points at distance $3$ from each other.
  Moreover, the $S_i'$ are sorted on the real line (say by the left ends of their leftmost interval), and since we concatenate the $\Xi_i$ in this order, this bound for $\alpha$ on each $\Xi_i$ holds for $\Xi$ as well.
  To argue that $\beta = 2\Delta$, note that the largest number of unit intervals that can intersect an interval of length $J$ is $(|J| + 1) \cdot \Delta$, 
  where $(|J| + 1) \cdot \Delta\leq 2\Delta |J|$ when $|J|>1$.
  %which is at most $2\Delta$ when $|J| \le 1$.
\end {proof}

We are now ready to finish the preprocessing algorithm.

\begin {proof} [Proof of Theorem~\ref{thm:mainunitintervals-preprocessing}]
Observe that Lemma~\ref{lem:interval-windows} establishes the existence of the desired sorting-supersequence. Moreover, the proof of Lemma~\ref{lem:interval-windows} is constructive. First, sort the $n$ unit intervals in $O(n\log n)$ time, yielding the $S_i$ sets. $S_i'$ can then be computed (using this sorted ordering) from $S_i$ in $O(\Delta)$ time, and thus $O(\Delta n)$ time over all $i$. Finally, Lemma~\ref{lem:shortestsuperseqeunce} states that the time to compute $\Xi_i$ from $S_i'$ is proportional to $|\Xi_i|$ (which itself is $|S_i'|^2$). 
Thus $\Xi$ is computed from the $S_i'$ in $O(\Delta n)$ time overall, as it is the concatenation of the $\Xi_i$, and Lemma~\ref{lem:interval-windows} already argued $|\Xi|\leq 4\Delta n$.
%
%$\Xi_i$ is obtained from $S_i'$ in $O(\Delta^2)$ time by applying Lemma~\ref{lem:shortestsuperseqeunce}, and thus $O(n/\Delta \cdot \Delta^2)=O(\Delta n)$ time overall.
\end {proof}

\section {Reconstruction in One Dimension} \label {sec:1d-rec}


We now consider the problem of recovering the sorted order of a set of values from a supersequence as computed in Section~\ref {sec:1d-prep}. 
We will first consider the problem in the classical preprocessing framework, where the goal is to achieve linear-time reconstruction, in Section~\ref {sec:1d-rec-standard}, and show that for \nice sequences this is achievable in the Real RAM, while on the Word RAM we do not require the sequences to be \nice.
Then, in Section~\ref {sec:1d-sub}, we discuss how to combine the idea of supersequences with the aim of achieving sublinear reconstruction.

\subsection {Standard Reconstruction} \label {sec:1d-rec-standard}

For the reconstruction phase, we are now given a sorting-supersequence $\Xi$ of $\RR$, and additionally a set of points (values) $P \from \RR$, and the goal is to efficiently compute the sorted order of $P$ using $\Xi$. (In particular, in linear time when $\Xi$ is $(O(1),O(1))$-\nice.)

We will assume that the regions in $\Xi$ come as pointers to the original regions in $\RR$, and that our sets $\RR$ and $P$ are equipped with bidirectional pointers.

The natural approach to the reconstruction problem is to replace each region in $\Xi$ by its corresponding point in $P$. This results now in a sequence $X$ of numbers, which comes with a guarantee that the sorted order is a subsequence of $X$. 
%The problem we need to solve is thus Open Problem~\ref {prob:1d}.
%This reconstruction problem for sorting is interesting in its own right.

\begin {definition}
Let $X=x_1, \ldots, x_n$ be a sequence of $n$ numbers (possibly with duplicate numbers), which contains $k\leq n$ unique numbers. Then we call $X$ a \emph{hidden sorted} sequence if it contains a subsequence of length $k$ with all $k$ distinct numbers in sorted order.  
\end {definition}

First, we show that given such a sequence $X$, we can find the sorted subsequence in linear time when universal hashing is allowed. Subsequently, we show even without universal hashing linear time is possible when the sequence is appropriately smooth. It is an open problem whether linear time is possible without universal hashing nor a smoothness bound.


\subsubsection* {Sorting hidden sorted sequences in linear time in the word RAM model.}

We first show that there is a natural greedy algorithm to sort hidden sorted sequences, which runs in expected linear time assuming universal hashing. 

\begin {lemma}\label{lem:hash}
Given a hidden sorted sequence $X=x_1, \ldots, x_n$ with $k$ unique numbers, there is a greedy algorithm which outputs a sorted subsequence of length $k$ containing all $k$ distinct numbers. This algorithm runs in $O(n)$ expected time if universal hashing is allowed, and otherwise it runs in $O(n \log k)$ time.
\end {lemma}

\begin {proof}

Create a stack which initially contains just $x_1$. Now process the items in the sequence in order (starting at $x_2$). Let $x_i$ be the current item being processed.
If $x_i$ exists in the stack already then discard it. Otherwise, pop items off the stack until the top item is less than $x_i$, at which point we push $x_i$ onto the stack.%
\footnote{Note that the above algorithm is oblivious of the value $k$.}

%If $x_i$ is strictly greater then the item on top of the stack, push $x_i$ onto the stack.  If $x_i$ is equal to the item on top of the stack, then skip $x_i$ and move onto $x_{i+1}$. If $x_i$ is strictly less than the item on top of the stack, remove the item on top of the stack, and repeat this process comparing $x_i$ to the next item on the stack. 

%\ben{Added proof of correctness.}
Let $x_{i_1},\ldots,x_{i_k}$ be any subsequence of $X$ containing all $k$ unique items in sorted order. To argue correctness, we prove by induction that after $x_{i_j}$ has been processed by our algorithm, the bottom of the stack contains the $j$ smallest (unique) items in sorted order. (For $j<k$ there may be additional items above on the stack, but for $j=k$ this is not possible as by construction the stack contains sorted unique items.) For the base case when $j=1$, the claim trivially holds as then the smallest item has been processed (either when $x_{i_j}$ is processed or earlier), at which point the algorithm will place it at the bottom of the stack, and the algorithm will never remove it. Now consider some $j>1$. By induction, after $x_{i_{j-1}}$ was processed, the stack contains the $j-1$ smallest (unique) items in sorted order at the bottom, and these can never be removed by the algorithm. Thus as $x_{i_j}$ is the next smallest item, the same logic as in the base case implies that after $x_{i_j}$ is processed, it will be on the stack immediately after the first $j-1$ items, and will never be removed.

Assuming we store the current stack items in a hash table, it takes expected constant time to check if $x_i$ is already in the stack. 
Thus we get a linear time algorithm, since in each time step we are either inserting or deleting from the stack, and each item is inserted at most once. If hashing is not allowed, we can instead use a binary tree for the items in the stack (whose size is at most $k$), and thus the running time is $O(n\log k)$.
\end {proof}

Note that the algorithm runs in linear time when hashing is available, but as we are interested in working in the Real RAM model, hashing is not available to us. We note that the problem of finding the sorted subsequence in a hidden sorted sequence is a special case of the longest increasing subseqeunce problem, which has a $\Omega (n \log n)$ lower bound in the decision tree model; however, in our problem if their are no duplicates the sequence is already sorted, whereas this is not the case for LIS, and thus it is unclear whether a lower bound for LIS implies a lowerbound for our problem.
%the lower bound does not apply when duplicate elements may be present, which is crucial in the hidden sorted sequence problem.
Whether a hidden sorted sequence can be recovered in linear time in the Real RAM model is left as an open problem.


\subsubsection* {Sorting \nice sequences}

For our purposes, it is sufficient to recover the sorted subsequence from a hidden sorted sequence that additionally is {\em $(\alpha, \beta$)-\nice}. We assume $\alpha$ and $\beta$ are constants, so the goal is to sort the sequence in time of the form $O(f(\alpha, \beta) \cdot n)$.


Note that if we convert the supersequence $\Xi$ as constructed in Section~\ref {sec:1d-prep} to their corresponding values in $P$, we get a sequence that is $(3, 2\Delta)$-\nice.

\begin {lemma} \label {lem:linearrecovery}
  Given a $(\alpha, \beta)$-\nice hidden sorted sequence $X=x_1, \ldots, x_n$ with $k$ unique numbers
  we can output a sorted subsequence of length $k$ containing all $k$ distinct numbers in $O(n \log(\alpha \beta))$ time in the Real RAM model.
\end {lemma}

\begin{proof}%\ben{Replaced old proof.}
Recall the algorithm of Lemma~\ref{lem:hash}, which greedily processes the items in $X$ in order, discarding $x_i$ if it has already occurred, and otherwise popping items off the stack until the top item is less than $x_i$, at which point we push $x_i$ onto the stack.  
%
We already argued this algorithm is correct, though its running time depends on how long it takes to determine whether $x_i$ has already occurred. 

The claim is that if $x_i$ has already occurred, then is must be one of the largest $\alpha \beta$ items seen so far. Assuming this claim is true, we can implement this algorithm in $O(n\log(\alpha\beta))$ time, as maintaining a balanced binary tree with the largest $\alpha\beta$ items seen, takes $O(\log(\alpha\beta))$ time per round (i.e.\ $O(n\log(\alpha\beta))$ overall), and the remaining stack operations are linear time overall. 

To prove the claim, fix any $x_i$, and let $x_j$ be the largest value item such that $j<i$. If $x_i>x_j$, the $x_i$ has not been seen before, and the claim trivially holds. So suppose that $x_i\leq x_j$. Since $X$ is $(\alpha,\beta)$-\nice, we have that $x_j-x_i\leq \alpha$. Thus if $x_i$ has occurred already, it must be within an interval of length $\alpha$ (ending at $x_j$), and there can be at most $\alpha \beta$ distinct points in such an interval, by the packing property of $(\alpha,\beta)$-\nice sequences, thus implying the claim.
\end{proof}

% \begin {proof}
%   \maarten {Essentially a scan; should be similar to the algorithm above.}

%   Let $x_i, x_j \in X$ such that $i < j$ but $x_i > x_j$.
%   Since $X$ is $(\alpha, \beta)$-\nice, we have that $x_i - x_j \le \alpha$ by definition. This means that between $x_i$ and $x_j$, we can only encounter points that lie within some interval of length $\alpha$, and there can be at most $\alpha \cdot \beta$ distinct points.

%   This means that, instead of the hash table in the algorithm above, we can maintain explicitly a list of distinct values that we encounter that are smaller than the largest value seen so far. This list will never have size larger than $\alpha\beta$; hence, we can inspect it in $\alpha\beta$ time per point, making the final algorithm run in $O(\alpha\beta |X|)$ time.
%   \maarten {I guess we should be able to do it in $O(n \log (\alpha\beta))$ time, no? Not that it matters in our application, but it would be nice to have the sharpest possible bound here.}
% \end {proof}

% Let $\RR$ be a set of $n$ unit intervals of ply $\Delta$.
% Given a sorting-supersequence $\Xi$ of $\RR$,
% and given a point set $P \from \RR$, the sorted order of
% $P$ can be computed in $O(|\Xi|)$ time.

\begin {proof} [Proof of Theorem~\ref{thm:mainunitintervals-reconstruction}]
If $\Xi$ is an {\em $(\alpha, \beta)$-\nice} sorting-supersequence of $\RR$, then by definition for any $P\from\RR$, if we replace the regions from $\Xi$ by their corresponding points in $P$, then it results in an $(\alpha,\beta)$-\nice hidden sorted sequence. Thus Lemma~\ref{lem:linearrecovery}, directly states we can recover the sorted order of $P$ in $O(|\Xi| \log(\alpha\beta))$ time. 
%Finally, the latter part of the theorem statement for when $\RR$ has ply $\Delta$ holds by plugging in $\alpha=3$ and $\beta=2\Delta$, and recalling that the size of $|\Xi|$ from Theorem~\ref{thm:mainunitintervals-preprocessing} was $O(n\Delta)$.
\end {proof}

\subsection {Sublinear reconstruction}
\label {sec:1d-sub}

The preprocessing framework was originally developed with linear-time reconstruction as the ultimate goal: since, during the reconstruction phase, as we need to spend linear time to replace each region in $\RR$ by the corresponding point in $P$, we cannot hope to be faster.
Van der Hoog~\etal~\cite{hkls-paip-19,hkls-pippf-22} argue that this viewpoint may be limiting, and that in some applications we may be content with an output structure that still contains some regions, as long as we are guaranteed that the true points in those regions are combinatorially in the correct location. They argue that the preprocessing framework may be adapted to allow for this behaviour, at the cost of introducing another phase: the reconstruction phase is formally separated into a first subphase (that can take sublinear time), in which the auxiliary structure $\Xi$ is transformed into another structure $\Xi'$ which is combinatorially equivalent to the desired output $S(P)$, and a second subphase in which $\Xi'$ is actually transformed into $S(P)$ in linear time, if so desired.

We show that the same behavior can be quite naturally obtained when the auxiliary structure $\Xi$ is a supersequence. 

\begin {definition}
A {\em marked} sorting-supersequence $\Xi$ of $\RR$ is a sequence of (possibly recurring) intervals of $\RR$, some of which may be marked, and such that, for any point set $P \from \RR$, the sorted order of $P$ is a subsequence of $\Xi$, that contains all marked items of $\Xi$.
\end {definition}

A normal sorting-supersequence is a marked supersequence with no items marked. To achieve maximum benefit, we want to mark as many items as possible, since these may be skipped during reconstruction.

In the case of sorting, intervals are marked if the position in the sorted order does not depend on the location of the point in the interval. This happens exactly when the interval is disjoint from all other intervals.

We adapt the preprocessing algorithm as follows. In $O(n \log n)$ time, we sort the intervals (by their left endpoint) as before. Now, in linear time, we scan through the intervals and detect any intervals that do not overlap with any other interval. We mark these intervals, and collect clusters of remaining intervals that are separated by the marked intervals. We apply Lemma~\ref {lem:interval-windows} to each cluster separately, and concatenate the resulting sequences. 

In order to be able to quickly skip over marked intervals during the reconstruction phase, we can equip the resulting sorting-supersequence with pointers from each item to the next non-marked item. During the reconstruction phase we then also apply Lemma~\ref {lem:linearrecovery} separately to each subsequence of consecutive non-marked items, in time propertional to the total number of non-marked items in the list. This results in a mixed sequence of sorted points and marked intervals, which is guaranteed to be in the correct sorted order (this list would be called $\Xi'$ in~\cite{hkls-paip-19}).
%}















\section{Preprocessing Disjoint Unit Disks for the convex hull}
\label{sec:2d-prep-disjoint}

Our goal in this section is to prove Theorems~\ref{thm:mainunitdisks-existence} and \ref{thm:mainunitdisks-preprocessing} for the case of disjoint unit disks. 

\subsection {Definitions and notation}
\label{sec:defnot}

Throughout this section, let $\RR=\{R_1,\ldots,R_n\}$ be a set of $n$ disjoint unit disks in the plane, arbitrarily indexed. \maarten {refer to Section~\ref {sec:classifications} to recall definitions here}
Let $\mathcal{B}(\RR)=\{B_1,\ldots,B_k\}$ be the subset of disks from $\RR$ on the boundary of the convex hull of $\RR$, indexed in counterclockwise order of appearance on the convex hull. We refer to $\mathcal{B}(\RR)$ as the {boundary} disks, and disks in the set $\mathcal{I}(\RR)= \RR\setminus \mathcal{B}(\RR)$ as {interior} disks. Refer to Figure~\ref{fig:definitions}.
When $\RR$ is clear from the context, we may write simply $\mathcal B$ and $\mathcal I$ instead of $\mathcal B(\RR)$ and $\mathcal I(\RR)$.

\begin {figure}[t]\centering
\includegraphics {figures/definitions.pdf}
\caption {The convex hull of $\RR = \{R_1, \ldots, R_{11}\}$, with {\em yellow} and {\em blue} disks on the boundary (depending on whether potential or guaranteed, see Figure~\ref{fig:intro-types}),  {\em orange} internal disks, and {\em strips} consisting of two consecutive boundary disks. In this case $\mathcal{B}(\RR) = \{B_1, \ldots, B_6\} = \{R_7, R_2, R_5, R_9, R_{11}, R_4\}$.}
 \label{fig:definitions}
\end {figure}

For any $i \in [1,k]$, define the $i$th \emph{strip} as $s_i=\CH(B_i,B_{i+1})$, where $\CH$ denotes the convex hull (and $s_k=\CH(B_k,B_1)$). We refer to the chain of strips, ordered by increasing $i$, as the \emph{strip hull} of $\RR$. (Indeed the convex hull of all strips is equal to that of all regions, where straight segments on the boundary follow individual strips and turns occur at the intersection of the common disk with the next strip.)
Furthermore, for any $i$, we call the line segment connecting the centers of $B_i$ and $B_{i+1}$ the {\em spine} of the corresponding strip. Observe that the spines themselves define an inner convex polygon, which we call the {\em spine hull}.

% The (boundary of the) convex hull of $\RR$ can viewed as being defined by a sequence of \emph{strips}. Specifically, for any $i \in [1,k]$, the $i$th strip is given by $s_i=\CH(B_i,B_{i+1})$, where $\CH$ denotes the convex hull (and $s_k=\CH(B_k,B_1)$). 
%  For any $i$, we call the line segment connecting the centers of $B_i$ and $B_{i+1}$ the {\em spine} of the corresponding strip. Observe that the spines themselves define an inner convex polygon, which we call the {\em spine hull}.

For a point set $P \from \RR$, we refer to the vertices  $Q=\{q_1,\ldots q_k\}$ of $P$'s convex hull, as the {\em $Q$-hull}.
\begin {observation} \label {obs:Q-vertices-in-stips}
  Every vertex of the $Q$-hull is contained in at least one strip.
\end {observation}

%\begin{definition}
%We take the convex hull of $\RR$, and say the disks that are on the convex hull boundary are {\em blue} and the rest are {\em green}. We say a {\em strip} is the convex hull of two blue disks that are consecutive on the convex hull boundary. We call the line segment connecting the centers of the two defining disks of a strip, the {\em spine} of the strip. Observe that the spines themselves define an inner convex polygon, which we call the {\em spine hull}.
%\end{definition}

Since our goal is to recover the {\em quarter hull}, we extend the above hull definitions to quarter hulls. 
The {\em quarter $Q$-hull} is simply the quarter hull of $Q$. The {\em quarter spine hull} is the quarter hull of the centers of the disks in $\mathcal{B}(\RR)$. 
%
Finally, the {\em quarter strip hull} is the subchain of the strip hull defined by the disks whose centers are on the quarter spine hull, with the addition of two new unbounded {\em half-strips}, one which goes downward from the rightmost disk, and the other which goes leftward from the topmost disk.%
\footnote {Note that the rightmost and topmost extremal disks do not necessarily contribute the rightmost and topmost points in $P$.} 
See Figure~\ref {fig:quarter-hull-definitions}.

% Since our goal is to recover the {\em quarter hull}, it is convenient to define the corresponding {\em quarter $Q$-hull} and {\em quarter spine hull} to be the portions of the $Q$-hull and spine hull from the rightmost extremal disk in $\RR$ to the topmost extremal disk in $\RR$.\footnote {Note that the rightmost and topmost extremal disks do not necessarily contribute the rightmost and topmost points in $P$.} 
% For the quarter $Q$-hull, we additionally consider two new {\em half-strips} that downward from the rightmost disk, and leftward from the topmost disk. Refer to Figure~\ref {fig:quarter-hull-definitions}.

%\ben{Above, "portions of the $Q$-hull and spine hull from the rightmost extremal disk in $\RR$ to the topmost extremal disk in $\RR$" doesn't make sense. The confusion appears to be due to thinking the Q-hull refers to the strips, which is used below in the section "Proof that $\Xi$ contains all possible quarter hulls in order." Maybe define the "strip hull"?}
%\ben{Corresponding, the figure caption is wrong.}

%\maarten {I'm not entirely sure how much terminology of this section is actually used - it does look like we need both the $Q$-hull and the quarter $Q$-hull though. Maybe at the end we could make a pass and see if anything is superfluous.}

\begin {figure}[t] 
\centering
\includegraphics {figures/quarter-hull-definitions.pdf}
\caption {The quarter strip hull of $\RR$, with {\em yellow} disks on the boundary,  {\em orange} and {\em red} internal disks (depending on whether unstable potential or stable impossible, see Figure~\ref{fig:intro-types}), and {\em strips} consisting of two consecutive boundary disks; there are two infinite {\em half-strips} from the rightmost disk and the topmost disk.}
\label {fig:quarter-hull-definitions}
\end {figure}



\subsection {Geometric preliminaries}\label{sec:geomprelim}

In this section we prove various geometric properties that will be required for the proof of Theorem~\ref{thm:mainunitdisks-existence}.
%Missing proofs are in Appendix~\ref{apnd:proofs}



\begin {lemma} \label {lem:intersection_bound}
  Let $P$ be a convex polygon, with edges of length at least $x>0$.
  Let $D$ be a unit radius disk.
  Then $D$ can intersect at most % $f(x)= 
  $\lfloor 2 \pi / x \rfloor + 1$ edges of $P$.
\end {lemma}

%\newcommand{\polygonproof}{
\begin {proof}[Proof of Lemma~\ref{lem:intersection_bound}]
  Let $D'$ be a disk of radius $1+x$ concentric with $D$.
  Observe that for every edge $e$ of $P$ that intersects $D$, the intersection of $e$ with $D'$ must have length at least $x$.
 
We now argue that the total length of $P$ that can be contained in $D'$ is at most $2 \pi (1 + x)$. More generally, let $\gamma$ be any convex closed curve, bounding a closed convex body $\Gamma$. As $D'$ is convex, the intersection $\Gamma\cap D'$ is also convex, and thus the maximum length of the boundary of $\Gamma\cap D'$ is the length of the circumference of $D'$, namely $2 \pi (1 + x)$. Thus as $\gamma \cap D'$ is a subset of the boundary of $\Gamma\cap D'$, its maximum possible length is $2 \pi (1 + x)$.
  
Combining our upper bound on the length of $P$ contained in $D'$ with our lower bound on the length of edges in $D'$ that intersect $D$, we can conclude that the number of edges of $P$ that $D$ intersects is at most $\lfloor 2 \pi (1 + x) / x \rfloor \leq \lfloor 2 \pi / x \rfloor + 1$.
\end {proof}
%}

\begin {corollary} \label {cor:diskstrips}
  Every unit disk intersects at most 7 strips.
\end {corollary}
%\newcommand{\sevenstripsproof}{
\begin {proof}[Proof of Corollary~\ref{cor:diskstrips}]
 A unit radius disk $D$ intersects a strip if and only if a radius 2 disk $D'$ with the same center point intersects the spine of the strip. Now consider the convex polygon $P$ which is the spine hull of the strips. Observe that the edges of $P$ have length at least 2, as the defining disks of a strip are disjoint. Now scale both $P$ and $D'$ by a factor $1/2$, and now the edges of $P$ have length at least 1 and $D'$ is a unit radius disk. So now we can apply Lemma~\ref {lem:intersection_bound} where $x=1$, implying that at most 7 edges of $P$ are intersected by $D'$, which by the above implies that $D$ intersected at most 7 strips.  
\end {proof}
%}

\begin {corollary} \label {cor:ply}
  The arrangement of strips has ply at most 4.
\end {corollary}

%\newcommand{\plyfourproof}{
\begin {proof}[Proof of Corollary~\ref{cor:ply}]
  Consider any point $z$ in the plane. $z$ is contained in a strip a strip if and only if a unit disk $D$ centered at $z$ intersects the spine of that strip. So consider the convex polygon $P$ which is the spine hull of the strips. Observe that the edges of $P$ have length at least 2, as the defining disks of a strip are disjoint. Thus applying Lemma~\ref {lem:intersection_bound} with $x=2$, we know that $D$ intersects at most 4 edges of $P$, and therefore $z$ intersects at most 4 strips. The ply of the arrangement is the maximum of all points in the plane, and since $z$ was an arbitrary point, the ply is at most 4.
\end {proof}

\begin {figure} 
\centering
\includegraphics {figures/ply-4.pdf}
\caption {An example of four disjoint unit disks where the arrangement of strips has ply 4.}
\label {fig:ply}
\end {figure}

We observe that Corollary~\ref {cor:ply} is in fact tight, as seen by Figure~\ref {fig:ply}.
%}

%\newcommand{\fhsection}{
\subsubsection{Fr\'echet and Hausdorff}

Here we define and discuss basic properties of the Fr\'echet and Hausdorff distance measures. 

Roughly speaking, the Hausdorff distance between two point sets is the distance of the furthest point from its nearest neighbor in the other set.

\begin{definition}
 Given two point sets $P,Q\subseteq \R^2$, their Hausdorff distance is 
 \[\delta_H(P,Q) = \max\{\sup_{p\in P} d(p,Q), \sup_{q\in Q} d(q,P) \}\]
\end{definition}

We use the following definition of Fr\'echet distance from \cite [Definition 3.10]{k-acgp-03}, which applies both to the unit interval $\S^0$, and well as the unit circle $\S^1$.

\begin {definition}[Fr\'echet distance]\label{def:Frechet}
Let $P : \S^k \to \R^2$ and
$Q : \S^k \to \R^2$ 
with $k \in \{0, 1\}$ be (closed) curves. 
Then $\delta_F (P, Q)$ denotes the Fr\'echet
distance between $P$ and $Q$, defined as
\[
  \delta_F (P, Q) := 
  \inf_{\alpha : \S^k \to \S^k, \beta : \S^k \to \S^k}
  \max_{t \in \S^k}
  ||P(\alpha(t)) - Q(\beta(t))||,
\]
where $\alpha$ and $\beta$ range over all continuous and increasing bijections on $\S^k$.
\end {definition}

\cite[Theorem 5.1]{k-acgp-03} discusses the equivalence between Hausdorff distance and Fr\'echet distance for closed convex curves, originally argued in 
\cite {10.1007/BFb0032068}. 


\begin {theorem} [\cite {10.1007/BFb0032068}] \label {thm:Hausdorff=Fréchet}
  For two convex closed curves in the plane, the Hausdorff distance is the same as the Fréchet distance.
\end {theorem}
%\maarten {Cite these sources (in comments):}
%Book:
%http://webdoc.sub.gwdg.de/ebook/diss/2003/fu-berlin/2002/110/part2chap5.pdf
%http://webdoc.sub.gwdg.de/ebook/diss/2003/fu-berlin/2002/110/part2.pdf
%\cite [Chapter 5]{k-acgp-03}

%Paper: Approximations of Convex Polygons
%https://link.springer.com/content/pdf/10.1007/BFb0032068.pdf

%\ben {Find a reference for the following statement:}

We will require the following lemma, which is a standard observation, though for completeness we include a short proof.
Note that in the following lemma, a convex polygon refers only to the boundary and not the interior of the enclosed region. 

%\ben{Added prior sentences. Lemma statement was confusing, as it said boundary in one place but not others, but I believe we mean it everywhere.}

\begin {lemma} \label {lem:discrete_Hausdorff}
Given two convex polygons A and B, where every vertex of A is within distance $\varepsilon$ of B and every vertex of B is within distance $\varepsilon$ of A, then the Hausdorff distance between A and B is at most $\varepsilon$.
\end {lemma}
\begin{proof}

Let $p$ be any point on $A$. We wish to argue that $p$ is within distance $\varepsilon$ of some point on $B$. (A symmetric argument can then be applied for any point on $B$). If $p$ is a vertex of $A$ then the property holds by the lemma statement. Otherwise $p$ is an interior point of some segment with endpoints $s$ and $t$, and let $\ell$ be the supporting line of this segment. 

Let $D$ be the $\varepsilon$ radius ball centered at $p$. Assume for contradiction that $D$ does not intersect $B$. Then there are two cases. In the first case $B$ contains $D$ in its interior. This implies $B$ has at least one vertex farther than $\varepsilon$ from $\ell$ on both sides of $\ell$. By convexity $A$ lies entirely on one side of $\ell$, and thus this in turn implies one of these vertices from $B$ is farther than $\varepsilon$ from $A$, which is a contradiction with the lemma statement.

In the second case $B$ is separated from $D$ by some line $m$. Thus $p$ is farther than $\varepsilon$ from $m$, and as $p$ lies on the segment $st$, at least one of the vertices $s$ or $t$ is also separated by $m$ from $B$ and lies farther than $\varepsilon$ from $m$. However, in this case this vertex is also farther than $\varepsilon$ away from $B$, which again is a contradiction with the lemma statement.     
\end{proof}
% \begin{proof}
% %    \maarten {Here $s'$ and $t'$ are ``the wrong'' points:}\\
% %   \includegraphics{figures/AB-example.pdf}

%         Let $p$ be any point on $A$. We wish to argue that $p$ is within distance $\varepsilon$ of some point on $B$. (A symmetric argument can then be applied for any point on $B$). If $p$ is a vertex of $A$ then the property holds by the lemma statement. Otherwise $p$ is an interior point of some segment with endpoints $s$ and $t$.

%     Now, assume without loss of generality that the supporting line $\ell$ of $st$ is horizontal, and that $A$ lies completely below the $\ell$.
%     Then consider the leftmost and rightmost points on $B$, which subdivide $B$ into two chains $B^{upper}$ and $B^{lower}$.
%     Let $s'$ be the closest point to $s$ on $B^{upper}$, and
%     let $t'$ be the closest point to $t$ on $B^{upper}$. and
%     let $p'$ be the closest point to $p$ on $B^{upper}$.
%     As $s$ and $t$ are vertices we can argue that $||s-s'||\leq \varepsilon$, and similarly $||t-t'||\leq \varepsilon$.
%     To argue this, we consider two cases. 
%     First, suppose that $s'$ is blow $\ell$. In this case $s'$ is the closest point to $s$ on $B$, and thus the lemma directly states that $||s-s'||\leq \varepsilon$.
%     Now suppose that $s'$ is above $\ell$. In this case observe that since $A$ lies entirely below $\ell$, by the lemma statement all vertices of $B$, cannot be more than $\varepsilon$ above from $\ell$. 
%     %This implies all point on $B^{upper}$ are within 
% %
%     \ben{Removed old figure. The second case above may have an issue, see figure below.}
    
%      \includegraphics{figures/counter.pdf}

%     Now let $X$ be the part of $B^{upper}$ between $s'$ and $t'$.
% %   
%     We claim that $X$ lies within distance $\eps$ of $\ell$
%     This is true because $s'$ and $t'$ are within distance $\eps$ from $\ell$, $B$ is convex, and every internal vertex of $B^{upper}$ must be within distance $\eps$ of $A$ (which is below $\ell$).

%     Since $p'$ is on $X$, the distance between $p$ and $p'$ is at most $\eps$.
% \end{proof}
%}


\subsection{Proof of Theorem~\ref{thm:mainunitdisks-existence} (Existence) for Disjoint Disks}
\label {sec:2d-exist}
In this section, we argue that a sequence $\Xi$, as described in Theorem~\ref{thm:mainunitdisks-existence}, exists.
We first argue that we can collect the disks that intersect a single strip.

Consider any strip $s$ spanned by two disks $L$ and $R$, and assume w.l.o.g. that the strip is horizontal and is at the top of the strip hull; let $\beta$ be the bottom edge of the strip (the segment connecting the bottommost point of $L$ to the bottommost point of $R$). (See Figure~\ref {fig:order-matters}.) 
Now consider the set $\RR_s \subset \RR \setminus \{L,R\}$ of other disks that intersect $s$. Clearly, all disks in $\RR_s$ must intersect $\beta$. But since they are disjoint, they intersect $\beta$ in a well-defined order.
%\ben{Lemma statement needs fixing}
\begin{lemma} \label {lem:line}
 A portion of the convex hull that intersects $L$ and $R$ may only contain vertices from $\RR_s$ in the order in which they intersect $\beta$.
\end{lemma}
\begin{proof}
Consider any two disks $a$ and $b$ in $\RR_s$ with $a$ left of $b$ in their intersection order with $\beta$. Let $\lambda$ be the bisector of $a$ and $b$. Now assume that $a$ is higher than $b$ (the other case is symmetric). Then $\lambda$ has positive slope, and it intersects $\beta$ (since it passes between $a$ and $b$), therefore $L$ lies completely above $\lambda$. But then we can never have a convex curve that passes through $L$ hit first $b$ and then $a$. So, if $a$ and $b$ both appear on the hull, $a$ comes before $b$.
\end{proof}

Lemma~\ref {lem:line} implies that for a single strip $s$, we can simply put the disks in $\RR_s$ in this order to form a subsequence $\Xi_s$.

\begin {figure}
 \begin{center}
  \includegraphics{figures/order-matters.pdf}
 \end{center}
 \caption {$a$ and $b$ can only both appear on the upper hull in one order.}
 \label {fig:order-matters}
\end {figure}

Next, we argue that we can simply concatenate the sequences per strip to obtain a single global sequence.
%
%The proof of the following lemma and the requisite background material for the proof can be found in Appendix~\ref{sec:backlemma}.

\begin{lemma} \label {lem:order}
  Let $P \from \RR$ be a point set, and let $Q = \{q_1, \ldots, q_m\}$ be its convex hull.
  Let $k$ be the number of vertices on the spine hull of $\RR$.
  There exist a mapping $f : [1:m] \to [1:k]$ such that:
  \begin {itemize}
  \item $q_i$ is contained in $s_{f(i)}$; and
  \item if $a,b,c$ are increasing $(\operatorname{mod}  m)$ then $f(a), f(b), f(c)$ are non-decreasing $(\operatorname{mod} k)$
  \end {itemize}
\end{lemma}

%\newcommand{\mapproof}{
\begin{proof}[Proof of Lemma~\ref{lem:order}]
We start by applying Lemma~\ref {lem:discrete_Hausdorff} in order to conclude that the Hausdorff distance between the spine hull and the $Q$-hull is at most 1.
% 
Specifically, Lemma~\ref {lem:discrete_Hausdorff} can be applied since:
\begin{enumerate}[label=(\alph*)]
    \item every vertex of the spine hull is in distance 1 from the $Q$-hull (if not, the corresponding disk would be either completely inside or completely outside the $Q$-hull, both of which are impossible), and
    \item every vertex of the $Q$-hull is in distance 1 from the spine hull, or, in other words, every vertex of the $Q$-hull is contained in one of the strips, 
%\maarten {The statement "every point on the Q-hull is contained in one of the strips" doesn't seem to be made explicit anywhere; this should probabaly be an observation near to the beginning of the paper which can refer to here.}
which follows from Observation~\ref {obs:Q-vertices-in-stips}.
\end{enumerate}
Now given this bound on the Hausdorrf distance, by Theorem~\ref{thm:Hausdorff=Fréchet}, the Fréchet distance between the $Q$-hull and the spine hull is at most 1.

Let this Fréchet mapping be denoted $f$. As it maps the $Q$-hull to the spine-hull, with Fréchet distance $\le 1$, $f$ trivially satisfies the first condition of the lemma. As for the second condition, we consider two cases. First suppose that of three values $f(a)$, $f(b)$, and $f(c)$, that at least two are equal. Then since we are working $\operatorname{mod} k$ this implies the $f(a)$, $f(b)$, $f(c)$ are non-decreasing, regardless of the third value, and so the second condition in the lemma trivially holds. On the other hand, if $f(a)$, $f(b)$, and $f(c)$ are all distinct then the second condition holds due to the monotonicity of the Fréchet distance in Definition~\ref{def:Frechet}.
%
%Finally, we argue that a Fréchet mapping (for FD $\le$ 1) implies a map $f$ as needed
%\maarten {Once we have a proper definition of (cyclic?) Fr\'echet, this should be straightforward.}
\end{proof}
%}

%We will now prove Theorem \ref{thm:mainunitdisks-existence} for the case of disjoint unit disks; that is, there exists a $(3\sqrt2,12)$-\nice quarter-hull-supersequence $\Xi$ for $\RR$.

%\ben{Change above sentence: 
We will now prove Theorem \ref{thm:mainunitdisks-existence} for the case of disjoint unit disks; that is, there exists a $(3\sqrt2,12)$-\nice quarter-hull-supersequence $\Xi$ for $\RR$. 
In fact, the sequence in the case of disjoint disks is even $(1,3)$-\nice;
we will argue the precise smoothness constants only for the general bounded ply case in Section~\ref {sec:2d-prep-overlapping}.

\begin {proof} [Proof of Theorem \ref{thm:mainunitdisks-existence} for disjoint disks]~
%\subsubsection*{Construction of the sequence $\Xi$.}
%Consider the strips $s_1,\ldots,s_k$ of the disks on the quarter hull of $\RR$, such that $s_1$ and $s_k$ are half strips, see Figure~\ref {fig:quarter-hull-definitions}
Consider the quarter strip hull, as shown in  Figure~\ref {fig:quarter-hull-definitions}.
For each strip (including the two unbounded half-strips), let $\RR_i \subseteq \RR$ be  the set of disks that intersect the strip $s_i$.
By %Lemma~\ref{lem:intersection_bound},
Corollary~\ref{cor:diskstrips},
$\sum |\RR_i| \in O(n)$.


%\ben{Move above paragraph to appendix and replaced with:
For each strip $s_i$ separately, we define a sequence $\Xi_i$ consisting of the disks of $\RR_i$, ordered according to Lemma~\ref {lem:line}. As the disks in $\RR_i$ are disjoint and all intersect the bottom edge of the strip in order, the corresponding points can never be our of order by more than $1$, and no more than $3$ points can be within a region of radius $1$, so $\Xi_i$ is $(1,3)$-\nice.
%}

Finally, to produce the required sequence of the theorem statement, we concatenate the $\Xi_i$ producing $\Xi = \langle \Xi_1, \Xi_2, \ldots, \Xi_k \rangle$. 
To see that $\Xi$ has the desired size, by the above discussion we have 
$|\Xi| = \sum_i |\Xi_i| = O(\sum_i |\RR_i|) = O(n)$.

%\subsubsection*{Proof that $\Xi$ contains all possible quarter hulls in order.}
Let $P \from \RR$ be an arbitrary realization. What remains is to argue that the sequence of points on the quarter hull of $P$ (i.e.\ the quarter $Q$-hull) is a subsequence of $\Xi$.
By Lemma~\ref {lem:order}, there exists a mapping $f$ that maps the points in the $Q$-hull to the strips in the strip hull, such that their cyclic orderings match.
%such that the cyclic ordering of the convex hull ordering of $P$ matches that of the strips. 

%Consider the image of the quarter $Q$-hull under $f$; this is a sequence of strips of the strip hull. 
%Note that not all of these strips need be strips of the quarter strip hull; however, all points of the $Q$-hull that are mapped to strips not on the quarter strip hull must necessarily lie in the vertical halfstrip under the rightmost disk or in the horizontal halfstrip left of the topmost disk.
Recall that when defining the quarter strip hull, we included the two unbounded half strips extending to the left of the topmost disk and below the rightmost disk. Let the \emph{bounded quarter strip hull} refer to the portion of the quarter strip hull excluding these two half strips. 
Now consider the image of the quarter $Q$-hull under $f$; this is a sequence of strips of the strip hull. 
Note that not all of these strips need to be strips of the bounded quarter strip hull; however, all points of the quarter $Q$-hull that are mapped to strips not on the bounded quarter strip hull must necessarily lie in the two half strips in the (unbounded) quarter strip hull.

Thus, as $\Xi$ appends the $\Xi_i$ in the ordering of the strips of the quarter strip hull (including the unbounded ones), it suffices to argue that for all $i$, the order of the subset of $Q$-hull vertices mapped to $s_i$ under $f$ appears as a subsequence of $\Xi_i$; however this is guaranteed
by Lemma~\ref {lem:line}.
%the point of $P$ assigned to $s_i$ appear in the hull order as a subsequence of $|\Xi_i|$ and hence as a subsequence of $\Xi$.
%Consider the resulting partion of the vertices of the convex hull of $P$ into sets that are mapped to the same strip.
%We will construct a sequence $\Xi$ that respects this partition.
\end {proof}

%The proof that such sequences can be computed in $O(n\log n)$ time, namely the proof of Theorem~\ref{thm:mainunitdisks-preprocessing} for disjoint unit disks, is in Appendix~\ref{sec:2d-prep}. 
%The extension of these results to disks of bounded ply is in Appendix~\ref{sec:2d-prep-overlapping}

%\newcommand{\preprocsection}{
\subsection{Proof of Theorem~\ref{thm:mainunitdisks-preprocessing} (Preprocessing) for Disjoint Disks}
\label {sec:2d-prep}

In this section, we argue that we can also compute such a sequence $\Xi$ in $O(n \log n)$ time.
%
First, we want to argue that we can compute the arrangement of all disks and strips in $O(n \log n)$ time.

\begin {lemma}
  The arrangement of all strips has compleixty $O(n)$
\end {lemma}

\begin {proof}
  By Corollary \ref{cor:ply}, the arrangements of strips has constant ply. 
  Since the strips are in convex position, they form a system of pseudodisks.
  Finally, any system of pseudodisks of bounded ply has complexity $O(n)$ \cite{s-ksacs-91}.
\end {proof}

By the above lemma and Corollary~\ref {cor:diskstrips}, we have that the arrangement of all strips and all disks has complexity $O(n)$.
Furthermore, such an arrangement can be computed in time proportional to its complexity multiplied by an additional log factor~\cite {4567905}.

\begin{lemma} \label {lem:comprri}
  We can compute the sets $\RR_i$, for all $i$ simultaneously, in $O(n\log n)$ time.
\end{lemma}
\begin{proof}
First, we compute the arrangement of all disks and all strips, which can be done in $O(n\log n)$ time as the arrangement has complexity $O(n)$.
Then, for each strip, we identify the set of disks $\RR_i$ that intersect the strip $s_i$.
 Since we constructed the arrangement explicitly, we can just walk through it to collect the disks intersecting each strip.
\end{proof}

\begin {proof} [Proof of Theorem \ref{thm:mainunitdisks-preprocessing} for disjoint disks]
We follow the proof of Theorem~\ref{thm:mainunitdisks-existence}, which is constructive. First, we compute the $\RR_i$ sets for all $i$ using Lemma~\ref {lem:comprri}.
Once we have the sets $\RR_i$, we need to construct the $\Xi_i$, after which we simply concatenate the $\Xi_i$ in order.
Each $\Xi_i$ is constructed using Lemma~\ref{lem:line}, which in turn projects $\RR_i$ onto a line and then applies Lemma~\ref {lem:interval-windows}, whose run time is proportional to the output sequence length. As Theorem~\ref{thm:mainunitdisks-existence} states that $|\Xi|=O(n)$, the overall running time is dominated by the $O(n\log n)$ time to construct all the $\RR_i$ sets. 
\end {proof}
%}









\section{Preprocessing overlapping unit disks for convex hulls}
\label {sec:2d-prep-overlapping}

We will now prove Theorems~\ref{thm:mainunitdisks-existence} and \ref{thm:mainunitdisks-preprocessing} in full generality for overlapping disks, by utilizing our one dimensional results, namely Lemma~\ref {lem:interval-windows}.  
Generally, the main ideas and structure of the arguments remain the same; hence we only describe the main differences.

Most importantly, Lemma~\ref {lem:line} from Section~\ref {sec:2d-prep-disjoint} fails for overlapping disks, which implies that we need to construct our sequences differently.
Instead we will reuse Lemma~\ref {lem:interval-windows} from Section~\ref {sec:1d-prep-algo} as a subroutine.

\begin{lemma} \label {lem:line-overlapping}
  Let $\ell$ be a line, and $\RR$ a set of unit disks of ply $\Delta$, which all have their centers within distance $d$ to the line  $\ell$.
  There is a sequence $\Xi$ of at most $(8\Delta
  \lceil d \rceil +8\Delta)|\RR|$ disks such that:
    for any point set $P \from \RR$,
    the order of the points in $P$ projected on $\ell$
    is a subsequence of $\Xi$.
Moreover, when considering the intervals resulting from the projection of $\RR$ onto $\ell$,
$\Xi$ is a $(3, (4\Delta\lceil d \rceil+4\Delta))$-\nice sorting-supersequence.
\end{lemma}


To prove Lemma~\ref{lem:line-overlapping} we will require a bound on the number of disks that can locally project onto a line.

\begin {lemma} \label {lem:project}
  Let $\ell$ be a line, and $\RR$ a set of unit disks of ply $\Delta$, which all have their centers within distance $d$ to the line  $\ell$.
  The ply of the projected intervals on $\ell$ is at most $2\Delta \lceil d \rceil + 2\Delta$.
\end {lemma}
\begin {proof}
  We need to find a bound on the ply of the projected disks in $\cal I$.
  The maximum ply is given by $\Delta$ times the number of disjoint disks that fit inside a rectangle of dimensions $4 \times (2 d + 2)$. Refer to Figure~\ref {fig:fitrectangle}. Specht~\etal~\cite {SPECHT201358} show that this number is at most $2 \lceil d \rceil + 2$.
\end {proof}

\begin{proof} [Proof of Lemma~\ref {lem:line-overlapping}]
  W.l.o.g. assume $\ell$ is horizontal.
  We will project all disks to $\ell$ and consider the resulting set of unit intervals $\cal I$.
  By Lemma~\ref {lem:project}, we can bound the ply of the projected disks in $\cal I$ by $\Delta' = 2 \lceil d \rceil + 2$.  
  Then, we apply Lemma~\ref {lem:interval-windows} to obtain a sequence $\Xi$ of at most $(8\Delta\lceil d \rceil+8\Delta)|\RR|$ intervals, which translates directly to the required sequence of disks.
  Moreover, Lemma~\ref {lem:interval-windows} also tells us that $\Xi$ is 
  $(3, (4\Delta\lceil d \rceil+4\Delta))$-\nice.
\end{proof}

\begin {figure}[t] 
\centering
\includegraphics{figures/fitrectangle.pdf}
\caption {The maximum ply of a set of projected disjoint unit disks is given by exactly the number of disjoint unit disks that fit inside a rectangle of dimensions $4 \times (2 d + 2)$.}
\label {fig:fitrectangle}
\end {figure}



Using Lemma~\ref{lem:line-overlapping} in place of Lemma~\ref {lem:line}, all our algorithms extend straightforwardly to overlapping disks of bounded ply. In particular, the definitions of strips, inner and outer disks, \nice sequences all remain the same.

However, we can no longer derive
{Corollary} \ref {cor:diskstrips}
and
{Corollary} \ref {cor:ply}
from
Lemma~\ref {lem:intersection_bound}.
Instead, we have the following more general bounds:

\begin {lemma} \label {lem:diskstrips}
  Consider the strip hull of a set $\RR$ of unit disks of ply $\Delta$.
  Every unit disk intersects at most $14\Delta-7$ strips.
\end {lemma}  

\begin {proof}
  Let $\cal B$ be the set of boundary disks of $\RR$,
  and let $\cal I$ be a maximum independent set (in the intersection graph) of $\cal B$.
  By {Corollary} \ref {cor:diskstrips},
  every unit disk intersects at most $7$ strips of the strip hull of $\cal I$. 
  Clearly every vertex of the strip hull of $\cal I$ is also a vertex of the strip hull of $\cal B$; furthermore, since every disk in $\cal B$ intersects at least one disk in $\cal I$, there can be at most $2\Delta-2$ additional vertices between any pair of adjacent vertices on the strip hull of $\cal B$; that is, each strip in of the strip hull of $\cal I$ corresponds to a sequence of at most $2\Delta-1$ strips in the strip hull of $\cal B$.
  The lemma follows.
\end {proof}

\begin {lemma} \label {lem:ply}
  Consider the strip hull of a set $\RR$ of unit disks of ply $\Delta$.
  The arrangement of strips has ply at most $8\Delta-4$.
\end {lemma}  
\begin {proof}
  Let $\cal B$ and $\cal I$ be as in the proof of Lemma~\ref {lem:diskstrips}.
  By {Corollary} \ref {cor:ply},
  Any point in the plane is contained in at most $4$ strips of the strip hull of $\cal I$, and thus in at most $8\Delta-4$ strips of the strip hull of $\cal B$.
\end {proof}

\subparagraph{Proof of  Theorem~\ref{thm:mainunitdisks-existence}.}
The proof remains the same as the disjoint disk case, except that our bounds on $\sum |\RR_i|$ and $|\Xi_i|$ increase by $\Delta$, and the smoothness constants change. Specifically, 
using Lemma~\ref{lem:diskstrips} in place of 
Corollary~\ref{cor:diskstrips} in the proof of Theorem~\ref{thm:mainunitdisks-existence}, we now obtain the bound
$\sum |\RR_i| \in O(\Delta n)$. %, which matches the theorem statement. 
%The rest of the proof remains the same.

For each strip $s_i$ separately, we define a sequence $\Xi_i$ consisting of possibly reoccurring disks of $\RR_i$, according to Lemma~\ref {lem:line-overlapping}. As the disks of $\RR_i$ have their centers within distance 2 of the line supporting the strip $s_i$, by Lemma~\ref {lem:line-overlapping} we have $|\Xi_i|\leq (8\cdot 2+8)\Delta|\RR_i| = O(\Delta|\RR_i|)$.
Since we concatenate the $\Xi_i$ to obtain $\Xi$, we have $ |\Xi|  \in O(\Delta^2 n)$ as required.

Furthermore, Lemma~\ref {lem:line-overlapping} states that when considering the intervals resulting from the projection of $\RR$ on $\ell$, $\Xi_i$ is a $(3, (4\cdot 2+4)\Delta) = (3,12\Delta)$-\nice
sorting-supersequence. 
%by Lemma~\ref {lem:interval-windows}, each resulting sequence $\Xi_i$ is $(2, 6\sqrt2)$-\nice with respect to $\ell$.
Note that the angle between $\ell$ and a line of slope $-1$ is at most $45^\circ$; therefore, the distance $\alpha$ increases by at most a factor $\sqrt 2$. Moreover, $\beta$ will not increase when going from the projection on the line back to the plane.
We conclude that each $\Xi_i$ is $(3\sqrt2, 12\Delta)$-\nice.


%\ben{Change previous sentence to: The rest of the proof remains the same, though we now must determine the smoothness constants.}
% \ben{
% %For each strip $s_i$ separately, we define a sequence $\Xi_i$ consisting of possibly reoccurring disks of $\RR_i$, according to Lemma~\ref {lem:line}. As the disks of $\RR_i$ have their centers within distance 2 of the line supporting the strip $s_i$, by Lemma~\ref {lem:line} we have $|\Xi_i|\leq (8\cdot 2+8)|\RR_i| = O(|\RR_i|)$.
% Lemma~\ref {lem:line-overlapping} states that when considering the intervals resulting from the projection of $\RR$ on $\ell$, $\Xi_i$ is a $(3, (4\cdot 2+4)\Delta) = (3,12\Delta)$-\nice
% sorting-supersequence. 
% Note that the angle between $\ell$ and a line of slope $-1$ is at most $45^\circ$; therefore, the distance $\alpha$ increases by at most a factor $\sqrt 2$. Moreover, $\beta$ will not increase when going from the projection on the line back to the plane.
% We conclude that each $\Xi_i$ is $(3\sqrt2, 12\Delta)$-\nice.
% }

\subparagraph{Proof of  Theorem~\ref{thm:mainunitdisks-preprocessing}.}
Using Lemma~\ref {lem:ply} in place of Corollary \ref{cor:ply}
in the proof of Theorem~\ref{thm:mainunitdisks-preprocessing},
we now have an arrangement of pseudodisks of ply $O(\Delta)$, which, using again the bound by Sharir~\cite{s-ksacs-91}, has complexity at most $O(\Delta n)$.
The rest of the proof remains the same.
%While we do not focus on optimizing the depence on $\Delta$, we nonetheless note that in this case a sharper bound of $O(\Delta n)$ can also be easily obtained.

%\begin {lemma} \label {lem:arr-comp-over}
%  The complexity of the arrangement of strips is at most $O(\Delta n)$.
%\end {lemma}

%With Lemma~\ref {lem:arr-comp-over} now in place, we can compute the arrangement in $O(\Delta n \log n)$ time. The remainder of the proof of Theorem~\ref{thm:mainunitdisks-preprocessing}, remains the same.









\section{Reconstruction in Two Dimensions} \label{sec:2d-rec}

In this section we present our reconstruction algorithm and then discuss how to extend it to achieve potential sublinear reconstruction by marking disks. Note the results in the section work directly for disks of bounded ply.

\subsection{Standard Reconstruction}


In this section, we consider the problem of recovering the true quarter hull from $\Xi$, given a realization $P \from \RR$. 
We again start by replacing the disks in $\Xi$ by their corresponding points; now we have a sequence $X$ of points that contains the points on the quarter hull in counterclockwise order as a subsequence. Whether we can recover the quarter hull of such a sequence in linear time in general is equivalent to Open Problem~\ref {prob:2d}.
%
Here, we additionally assume the sequence $X$ is $(\alpha, \beta)$-\nice for constants $\alpha$ and $\beta$.

%\maarten {Not necessary anymore, since we only look at quarter hulls to begin with}
%We will first cut our sequence into four pieces, and recover four quadrants of the convex hull separately.

%\begin {claim}
%  We can assume that $X$ starts with the leftmost point.
%\end {claim}

%\begin {proof}
%  We can find the leftmost point in our sequence in linear time.

%  However, there might be multiple copies of that point, and not all of them might work.

%  But this is not a problem: we can simply duplicate our sequence, and still be sure that the vertices of the convex hull all appear in the correct order after the first occurence of the leftmost point.
%\end {proof}

% We similarly construct sequence that start with the topmost, rightmost, and bottommost points.

\subsubsection{Algorithm for a Single Quadrant}

We start with the following observation.
%First, note that we may assume our sequence starts with the rightmost point and ends with the topmost point.

\begin {observation}\label{obs:firstlast}
  We can assume that $X$ starts with the rightmost point and ends with the topmost point.
\end {observation}

\begin {proof}
  We can find the rightmost and topmost points in $X$ in linear time by scanning $X$ once.
  Note that there might be multiple copies of these points.
  We delete the prefix of $X$ before the first occurrence of the rightmost point, and we delete the suffix of $X$ after the last occurrence of the topmost point.
  The resulting sequence must still contain the quarter hull of $X$ as a subsequence; moreover, deleting points cannot violate that $X$ is $(\alpha,\beta)$-\nice.
\end {proof}

We now describe our algorithm for extracting the quarter hull.
It is an adaptation of the algorithm of Andrew~\cite {ANDREW1979216} (which is in turn an adaptation of Graham's scan), but we don't have a guarantee that the points come completely sorted. However, whenever we encounter a point that is not in the correct sorted order, we argue that we can determine whether it should still be inserted or can be safely discarded in $O(\alpha \cdot \beta)$ time be checking only the last $\alpha\cdot\beta$ edges of the hull so far, similar to the algorithm in Section~\ref {sec:1d-rec}.

\begin {lemma}\label{lem:quartersub}
Let $X$ be a $(\alpha, \beta)$-\nice sequence where the first point is the rightmost, the last is the topmost, and the quarter hull is a subsequence of $X$.
%and the hull is a $(\alpha, \beta)$-\nice subsequence; that is, the distance and packing properties hold property.
We can compute the quarter hull of $X$ in $O(\alpha\beta |X|)$ time.
\end {lemma}
\begin {proof}
We process the points in order, maintaining  a convex chain, $C$, of a subset the points processed so far, where initially $C$ is just the rightmost point. So let $q$ be the next point to be processed, and let $p$ be the last (i.e.\ topmost) point on $C$. If $q=p$, then we skip $q$ and move on to processing the next point. Otherwise, 
If the $y$-coordinate of $q$ is greater than or equal to that of $p$, then we apply a standard Graham scan insert (i.e. pop from $C$ until it is a left turn to $q$, and then insert $q$ at the end of the chain). Finally, if the $y$-coordinate of $q$ is less than that of $p$ then there are two cases. Let $\ell$ be the line of slope $-1$ through $p$, and let $proj(q)$ be the orthogonal projection of $q$ onto $\ell$. If $||proj(q)-p|| > \alpha$ then we pop $p$ from $C$, and we restart the process of trying to insert $q$. Otherwise, we check the last $\alpha \cdot \beta$ edges of $C$ in order from left to right, where if for any edge $q$ lies above the supporting line of that edge then we delete all vertices of $C$ starting from the left endpoint of that edge and going to the left, after which we then perform a Graham scan insert of $q$ into the remaining chain. If $q$ does not lie above any such edge, then we skip $q$ and move on to processing the next point.

For the running time, recall that in the last case when processing a point we might perform a constant time check on each of the last $\alpha \cdot \beta$ edges of $C$. Thus over all points this cost is $O(\alpha\beta |X|)$. This dominates the running time, as by the same analysis as in standard Graham scan, the remaining operations when processing a point take time proportional the number of deletions, and points are deleted at most once. 

As for correctness, first observe that in the case that $||proj(q)-p|| > \alpha$, then as $X$ is an $(\alpha,\beta)$-\nice sequence, by definition $p$ cannot be a vertex of the quarter hull. Thus the algorithm correctly deletes it, and so we will ignore this case in the remainder of the proof. 

Let  $C=c_1,\ldots c_m$ be the current convex chain at some point during the algorithm. Suppose that the prefix $c_1,\ldots, c_k$ is a prefix of the actual quarter hull. Then observe that this prefix can never be removed from $C$. Specifically, there are two cases when points are removed from $C$. The first case is by a Graham scan insertion, which by the correctness of Graham scan will not remove a vertex of the actual quarter hull. The second case is when if checking the last $\alpha\cdot \beta$ edges of $C$, the algorithm finds an edge where $q$ lies above its supporting line, however, such an edge cannot be found in the $c_1,\ldots, c_k$ portion of $C$ as it contains edges of the actual quarter hull.

Again let $C=c_1,\ldots,c_m$ be the convex chain at some iteration of the algorithm. Suppose that the $i-1$ prefix of $C$, namely $C_{i-1} = c_1,\ldots,c_{i-1}$, is the $i-1$ prefix of the quarter hull, and that the $i$th vertex of the quarter hull, call it $q$, is not in $c_i,\ldots, c_m$. In this case, we claim that if $q$ is processed next by the algorithm then afterwards $C_i$ with be the $i$th prefix of the quarter hull. 
As prefixes of the quarter hull are never removed as argued above, proving this claim inductively implies we correctly compute the quarter hull, as we know the quarter hull is a subsequence of $X$ and thus we must eventually process the next vertex of the quarter hull. (The base case holds as initially $C$ is the first point in $X$, which by the lemma statement is the first vertex of the quarter hull.)

To prove the claim we consider the different cases handled by the algorithm, in order. First, $q=c_m$ ($c_m$ is called $p$ above) is not possible since $q$ by assumption of the claim statement is not in $C$.  If the $y$-coordinate of $q$ is greater or equal to $c_m$ then we perform a Graham scan insert of $q$, which is guaranteed to place $q$ in $C$ after $c_{i-1}$ (since the fact that $q$ is above $c_m$ implies that $c_1,\ldots, c_{i-1}$ are the only vertices of the quarter hull in $C$). So suppose that the $y$-coordinate of $q$ is less than that of $c_m$.
Next, in the case that $||proj(q)-c_m|| > \alpha$, we already remarked that $c_m$ is correctly deleted. So suppose that $||proj(q)-c_m|| \leq \alpha$.

Now because $q$ is on the actual quarter hull, though it lies below $c_m$, it must lie to the right of $c_m$. In particular, it must lie above the chain $C$, i.e.\ in the region right of and below $c_m$, above and left of $c_1$, and above $C$. Thus $q$ lies above some edge of $C$, and since $||proj(q)-c_m|| \leq \alpha$, as $X$ is an $(\alpha,\beta)$-\nice sequence, the packing property implies it must be one of the last $\alpha \cdot \beta$ edges of $C$. Finally, after identifying and removing up to the right end of this edge, $q$ will make a right turn with the remainder of $C$ until $c_{i-1}$. Thus after applying a Graham scan update, $q$ will be placed after $c_{i-1}$ as claimed.
%
%\ben{The last part of the above paragraph is hand wavy. Once we have identified an edge that $q$ is above, how do we now Graham scan will work correctly on what remains, as $q$ may actually have a smaller $y$ (larger $x$) coordinate, which would not occur in Graham scan? (Actually in the algorithm it seems once we have identified the first edge it is above we should continue checking if it above edges and removing, until we reach an edge it is not. Naturally this is related to Graham scan checks, though I am getting lost in the distinction between the two.)}
\end {proof}

Note in the above algorithm it is possible that $q$ is a duplicate of one of the vertices on the current chain (other than the last point $p$). This case is correctly handled, assuming in the later step we only check if $q$ is strictly above the supporting line of the edges, as this will discard $q$ which is the desired behavior. 

\subsubsection{Combining the Pieces}

\begin {proof} [Proof of Theorem~\ref{thm:mainunitdisks-reconstruction}]
If $\Xi$ is an $(\alpha,\beta)$-\nice quarter-hull-supersequence of $\RR$, then by definition for any $P\from\RR$, if we replace the regions from $\Xi$ by their corresponding points in $P$, then it results in an $(\alpha,\beta)$-\nice sequence such that the quarter hull of $P$ is a subsequence. 
Moreover, by Observation~\ref{obs:firstlast} we can assume the first point in the sequence is the rightmost and the last is the topmost.
Thus Lemma~\ref{lem:quartersub} directly states we can recover the quarter-hull of $P$ in $O(\alpha\beta|\Xi|)$ time.
\end {proof}




\subsection {Sublinear Reconstruction}
\label {sec:2d-sub}

If the reconstruction phase requires replacing each region in $\RR$ by the corresponding point in $P$, then this phase has an immediate linear time lower bound. However, if the location of the realization of a certain region does not affect the combinatorial structure of the output, then it may not be necessary to replace it with its point.
Following the approach of Van der Hoog~\etal~\cite{hkls-paip-19,hkls-pippf-22}, we thus modify the preprocessing framework by introducing another phase: the reconstruction phase is formally separated into a first subphase (that can take sublinear time), in which the auxiliary structure $\Xi$ is transformed into another structure $\Xi'$ which is combinatorially equivalent to the desired output $S(P)$, and a second subphase in which $\Xi'$ is actually transformed into $S(P)$ in linear time, if so desired.

An advantage of using supersequences as an auxiliary structures is that $\Xi'$ for our problem is simply a sequence of points and disks, which is the desired output convex hull when replacing the disks in the sequence with their corresponding realizations. As this second subphase is now trivial, we focus on the first subphase in the remainder of this section. 
%
This requires identifying the disks whose realizations do not affect the combinatorial structure of the convex hull, which are precisely the \emph{stable} disks defined in Section~\ref{sec:classifications}.

%This requires identifying and utilizing \emph{stable} disks, as defined in Section~\ref{sec:classifications}, 




%For intervals in one dimension, Appendix~\ref{sec:1d-sub} discusses how $\Xi'$ can easily be obtained when $\Xi$ is a supersequence, and can be read as warmup. In this section we discuss how to achieve this for disks in the plane, though doing so requires significantly more care, and we start by carefully classifying disks.

%In the section we discuss how to mark appropriately classified disks, in turn allowing for the potential of sublinear reconstruction times depending on how the disks are arranged. 

%\maarten {Disk characterization subsection that used to be here is now moving to Section 2, see if anything should be done here to make it still read well.}

% \subsubsection {TO DO}

% \begin {definition}
%   A {\em candidate disk} is a disk $R \in \RR$ such that $R \cap hull(\RR \setminus \{R\}) = \emptyset$; that is, it is separated from the other disks by a line.
% \end {definition}
% \maarten{I don't like the name "candidate", since that sounds like they are candidates to be on the hull while in fact they are certain to be on the hull}
% \ben{Perhaps "separated" for the above, and "stable" for the below.}

% claim: all candidate disks are vertices of the strip hull

% \begin {definition}
%   A {\em safe disk} is a disk $R \in \RR$ such that such that, for all realizations of $\RR \setminus \{R\}$, it is the case that the combinatorial structure of the convex hull of everything does not depend on the location of the point in $R$.
% \end {definition}

% \ben{Something confuses me a bit. Consider two realizations of $\RR \setminus \{R\}$, call them $X$ and $Y$. The above definition does not exclude the possibility that say any realization of $R$ must always occur on the hull of $X$ but never on $Y$. 
% This may not be possible geometrically, though unless proven, the above definition should be clarified to say $R$ must be on the hull.}

%Refer to Figure~\ref {fig:growing-disks} for an illustration.

%claim: all safe disks are candidate disks

%claim: safe is the right requirement



%\subsubsection{Marking Disks for Sublinear Reconstruction}
%As discussed at the start of Section~\ref{sec:2d-sub}, in order to achieve sublinear reconstruction, we need to identify the disks whose realizations do not affect the combinatorial structure of the convex hull, which by definition are the \emph{stable} disks. As argued above, there are two types of stable disks, namely stable guaranteed boundary disks and stable impossible interior disks. 



As argued in Section~\ref{sec:classifications}, there are two types of stable disks, namely stable guaranteed boundary disks and stable impossible interior disks.
First observe that stable impossible interior disks do not properly intersect any strip on the boundary (that is $I$ from Observations~\ref{obs:polygons} lies inside the chain of strips). Thus our constructed supersequence $\Xi$ already omits such disks, as it is constructed only from disks which intersect strips.

The remaining stable disks are all guaranteed boundary disks, which Lemma~\ref{lem:stable} allows us to efficiently identify, and can be utilized as follows.

%\subsubsection{Old}

%In Section~\ref {sec:1d-sub}, we argued that using a supersequence as an auxiliary structure allows us to naturally achieve sublinear reconstruction times if desired, following the adapted preprocessing framework of Van der Hoog~\etal~\cite{hkls-paip-19,hkls-pippf-22}.

%In this section, we show that the same is true for the convex hull problem in $\R^2$. This is in fact a new result: no previous work that we are aware of has studied the convex hull problem in the preprocessing framework with the intention of achieving sublinear reconstruction time.

% Contrary to the problem of sorting, but similar to the problem of computing the {\em Pareto front}~\cite {hkls-pippf-22}, the convex hull problem allows two different ways to obtain sublinear reconstruction times:
% \begin {itemize}
%   \item some disks may be guaranteed to never contribute to the output; such disks need not be inspected at all during the reconstruction phase, or
%   \item some disks may be present in the output in the same position in the ordered sequene, independent of the locations of the points in $P$.
% \end {itemize}

%Note that disks of the first type are already omitted by our preprocessing algorithm; no additional work is required. For the disks of the second type, we proceed similar to Section~\ref {sec:1d-sub}.

\begin {definition}
A {\em marked} convex-hull-supersequence $\Xi$ of $\RR$ is a sequence of (possibly recurring) disks of $\RR$, some of which may be marked, and such that, for any point set $P \from \RR$, the points on the convex hull of $P$ appear as a subsequence of $\Xi$, that contains all marked items of $\Xi$.
\end {definition}

A normal convex-hull-supersequence is a marked supersequence with no items marked. Our goal is to mark as many items as possible, since these may be skipped during reconstruction. By the discussion above, the largest set possible to mark is precisely the stable guaranteed boundary disks, which we will  mark using Lemma~\ref{lem:stable}.


We now adapt the preprocessing algorithm to build a marked supersequence $\Xi$ as follows. Let the counterclockwise sorted order of boundary disks be $\mathcal{B}(\mathcal{R})=\{B_1,\ldots,B_k\}$. The marked disks in this ordering intuitively partition the disks into maximal length substrings of unmarked disks, which in turn partitions the strip hull into subchains. In particular, if $B_i$ is a marked disk then both of its adjacent strips ($B_{i-1}B_i$ and $B_{i}B_{i+1}$) are no longer considered, which is the desired behavior as recall the proof of Lemma~\ref{lem:stable} argued no other disks intersected these strips, and because $B_i$ is stable the location of its realization does not affect the result on the substring ending at $B_{i-1}$ (and analogously for $B_{i+1}$).
Thus we can safely apply Theorem~\ref{thm:mainunitdisks-preprocessing} independently to each resulting substring and concatenate the results in order, with the respective marked disks in between.

In order to be able to quickly skip over marked disks during the reconstruction phase, we can equip the resulting supersequence with pointers from each item to the next non-marked item. During the reconstruction phase we then also apply Lemma~\ref{lem:quartersub} separately to each subsequence of consecutive non-marked items, in time proportional to the total number of non-marked items in the list. 
This results in a mixed sequence of points and marked disks, which is guaranteed to be the convex hull of $P$.

\maarten {May need to be slightly precise here on how we combine splitting the problem by marked disks with splitting the problem on the leftmost, rightmost, topmost, or bottommost global points (which we might not know if they are in marked disks). I think it is clear enough that this will work out to be handwavy for now though - but for a full version we should probably work out the details.}





 \section {Conclusion \& Future Work} \label {sec:conclusion}



\begin {figure}
 \begin{center}
  \includegraphics[width=\textwidth]{figures/safe-cadidate-disks.pdf}
 \end{center}
 \caption 
 { When we grow the uncertainty radii from $0$ to $\infty$, all disks eventually turn from stable (red or green) to unstable potential (orange or yellow).
 }
 \label {fig:growing-disks}
\end {figure}


\ben{This section needs updating}

We introduced the idea of using supersequences as auxiliary structures in the preprocessing model for dealing with uncertain geometric data.
We were able to reproduce several known results with an arguably simpler auxiliary structure; moreover, we obtained at least one new result.

To follow up these first results using supersequences, we see three main lines of future work. 

First, our work exposed an open problems that is of independent interest (Open Problem~\ref {prob:2d}); settling whether this problem allows solutions in $o(n \log n)$ time or has a $\Omega(n \log n)$ lower bound would be very interesting.

Second, our results are restricted to unit intervals or disks of constant ply. Previous work in the preprocessing model has focused on allowing more general sets of regions, that may overlap more, be less fat, or have different sizes - which combinations of such generalizations are possible and for which problems is the topic of a whole line of papers. It would be very interesting to see whether our results can be generalized to less restricted regions as well; likely, such generalizations would require a relaxation of the \nice property, thus linking this line of future work to solving the open problems as well.
Relatedly, when the uncertainty radius is not fixed in advance, understanding the relation between the radius, the resulting ply, and the efficiency of reconstruction would be of interest (see Figure \ref{fig:growing-disks}).

Finally, we only looked at two problems that produce an output that is a subsequence of the input: sorting and the convex hull. There are other natural problems that might fit in this framework, such as the Pareto Front, the traveling salesperson problem, shortest paths in polygons, and more. It would be interesting to study such problems in the same model.


\maarten{Do we want to write something about Rev 2's idea to use other kinds of sequences?}


\section* {Acknowledgements}
%Omitted for anonymity. 
The authors would like to thank Ivor van der Hoog, Vahideh Keikha, and Frank Staals for inspiring discussions on the topic of this paper.

\bibliographystyle{plainurl}
\bibliography{refs}




\appendix

%\section{Preprocessing Intervals for Sorting in 1D Continued}
%\label{apnd:1d}

%Here we continue the discussion from Section~\ref{sec:1d} about preprocessing intervals for sorting in 1D.

%\onedsection

%\section{Content from Section~\ref{sec:2d-prep-disjoint}}

%\subsection{Proofs from Section~\ref{sec:geomprelim}}
%\label{apnd:proofs}
%\polygonproof
%\sevenstripsproof
%\plyfourproof

%\subsection{Background and proof of Lemma~\ref {lem:order}}
%\label{sec:backlemma}

%Here we prove Lemma~\ref{lem:order} from Section~\ref{sec:2d-exist}, though first we require the following background material relating to the Fr\'echet and Hausdorff distances. 

%\fhsection

%\subsubsection{The Proof}
%\mapproof

%\preprocsection

%%%%%%%%%%%%%%%%%%%%%%%%%%%%%%%%%%%%%
%%%%%%%%%%%%%%%%%%%%%%%%%%%%%%%%%%%%%


%\section{Proofs from Section~\ref{sec:2d-sub}}
%\label{subproofs}

%\classifyproof


\end{document}
