\section{Literature Review}
It has been seen that open-source LLMs such as LLaMA \cite{touvron2023llama}, Mistral \cite{yu2024breakingceilingllmcommunity}, and TinyLLama among others have risen as a relevant and needful alternative to closed-source models because of the scalability, flexibility, and creativity in which they offer. These open-source models have demonstrated imminent relevance in various NLP and machine learning tasks such as natural language generation (NLG) and natural language understanding (NLU). While praising these models for their remarkable ability to capture semantic and syntactic nuances in various tasks, they still struggle with capturing pragmatic nuances such as speech acts \cite{Dresner2003-DREROB-3}. Addressing this limitation opens up a less trivial opportunity for additional linguistic and contextual analysis interfaces to improve conversational applications. 

To give a background of speech acts in Text Analysis, speech acts are often referred to as the communicative intent behind a statement or locution, as defined in speech act theory \cite{austin1962speech} \cite{searle1969speech}. Thus far, there has been a lot of visible significance of the concept of speech acts theoretically, but very limited practical application to NLP. It is seen that more focused text analysis for sentiment, emotion, and intent classification has been enabled by recent improvements in transformer-based architecture models, such as BERT and its derivatives \cite{devlin2018bert}. However, speech act classification remains underexplored and overlooked in computational studies. The core of this study is to build on existing foundations of the speech act theory, leveraging the AIF dataset to model speech acts in dialogue structures, thereby filling in a critical research gap in text analysis for conversational systems.