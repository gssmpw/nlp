\section{Determination of the parametric resonance regions}\label{Appendix:PR}
In section \ref{SysDesc} we discussed how detuning of the parametrically pumped Ising mode leads to conditional forgetting. Here, we provide an in-depth analysis of parametric resonance using a combination of Floquet theory and perturbation methods. Our main objectives are to derive the resonance curve shown in Fig. \ref{fig:flip}a and to show that detuning has the effect of shifting it. The relevant equation is that of the Ising mode (Eq. \ref{eqn:Ising}) without interactions and noise, which can be cast in the following form:
\begin{equation}
\ddot{x} + \mu\dot{x} + \left(\delta + \epsilon\sin(t)\right)x + \gamma x^3 = 0,\label{eqn:math_PO}
\end{equation}
where $\delta$ represents the squared ratio between the mode's natural frequency and the pumping frequency and $\epsilon$ represents the driving amplitude. Resonance occurs because the pumping effectively serves as a negative damping. Growing oscillations are detuned by the Duffing term, restricting the steady state amplitude. So, for fixed damping $\mu$, the relevant parameters that control whether or not resonance occurs are $\delta$ and $\epsilon$. To derive the resonance curve in the $\delta-\epsilon$ plane we treat the damping and Duffing terms as small perturbations to the well-known and extensively studied \emph{Mathieu equation}:
\begin{equation}
    \ddot{x} + \left(\delta + \epsilon\sin(t)\right)x  = 0,
\end{equation}
for which the resonance regions can be found using Floquet theory. They form tongues emanating from the points $\delta = n^2/4$ along the $\delta-$axis in what is known as the Ince-Strutt diagram~\cite{Mathieu_eq}. For the purposes of this paper, we consider the resonance region around the so-called \emph{parametric resonance condition}, which holds when the driving frequency is twice the mode's natural frequency, or equivalently when $\delta = 1/4$. We can use perturbation theory to find an approximate expression for this resonance curve in the case where $\mu, \gamma > 0$, following an approach adapted from~\cite{Mathieu_eq} known as the \emph{the method of multiple scales}. Firstly, we treat the pumping amplitude $\epsilon$ as a small perturbation, and rescale $\mu$ and $\gamma$ to be of the same order, $\mu =: \epsilon \tilde{\mu}$ and $\gamma =: \epsilon \tilde{\gamma}$. Next, we expand the resonance curve in powers of $\epsilon$, $\delta = 1/4 + \delta_1\epsilon + \dots$ and define two independent timescales; a fast one $\xi := t$ and a slow one $\eta := \epsilon t$. Because the timescales are assumed to be independent, the full time derivative with respect to $t$ splits into partial derivatives with respect to $\xi$ and $\eta$ as follows:
\[\dv{}{t} = \pdv{}{\xi} + \epsilon \pdv{}{\eta}\]
Substituting everything into Eq.~\ref{eqn:math_PO} and ignoring terms of second order or higher in $\epsilon$, leaves us with: 
\begin{equation}
\frac{\partial^2x}{\partial \xi^2} + 2\epsilon\frac{\partial^2x}{\partial \eta\partial \xi} + \epsilon\tilde{\mu} \frac{\partial x}{\partial \xi} + \left(\frac{1}{4} + \epsilon \delta_1 + \epsilon \sin(\xi)\right)x + \epsilon \tilde{\gamma} x^3 = O(\epsilon^2). \label{eqn:pertubed_math_PO}
\end{equation}
Expanding the solution $x$ in powers of $\epsilon$, $x = x_0 + x_1\epsilon + \dots$ and collecting terms of like order yields the following two equations (one for the zeroth and one for the first order):
\begin{align}
   \frac{\partial^2x_0}{\partial \xi^2} + \frac{1}{4}x_0 &= 0  \\ 
   \frac{\partial^2x_1}{\partial \xi^2} + \frac{1}{4}x_1 &= -2\frac{\partial^2x_0}{\partial \eta\partial \xi} - \tilde{\mu} \frac{\partial x_0}{\partial \xi} - \delta_1x_0 - \sin(\xi)x_0 - \tilde{\gamma} x_0^3. 
\end{align}
From which we find the general solution for the zeroth order: 
\begin{equation}
x_0 = A(\eta)\cos\left(\frac{\xi}{2}\right) + B(\eta)\sin\left(\frac{\xi}{2}\right).  
\end{equation}
Substituting this in the equation for the first order, and collecting resonance terms on the right hand side we find:
\begin{align*}
 \frac{\partial^2x_1}{\partial \xi^2} + \frac{1}{4}x_1 &=  \sin\left(\frac{\xi}{2}\right)\left(A' + A\frac{\tilde{\mu}}{2} - B\delta_1\right)\\
 &- \cos\left(\frac{\xi}{2}\right)\left(B' + B\frac{\tilde{\mu}}{2} + A\delta_1\right)\\
    & -\sin\left(\frac{\xi}{2}\right)\left(\frac{3}{4}\tilde{\gamma} B^3 + \frac{3}{4}\tilde{\gamma} A^2 B + \frac{1}{2}A\right) \\
    &- \cos\left(\frac{\xi}{2}\right)\left(\frac{3}{4}\tilde{\gamma} A^3 + \frac{3}{4}\tilde{\gamma} AB^2 + \frac{1}{2}B\right)\\
& + \textit{off-resonance terms}.
\end{align*}
The terms on the left hand side that are on resonance are known as \emph{secular terms}~\cite{pertubation_methods}, and they cause the first order $x_1$ to grow without bound. So, in order for the perturbation assumption to be valid, these terms need to vanish. Setting the coefficients of the secular terms to zero leaves us with the following equations describing the (slow) dynamics of the amplitude and phase modulating the (fast) zeroth order oscillations $x_0$:
\begin{align*}
\frac{dA}{d\eta} &= \frac{1}{2}A\left(1 -\tilde{\mu}\right) + \delta_1B + \frac{3}{4}\tilde{\gamma} B\left(A^2 + B^2\right)\\ 
\frac{dB}{d\eta} &= -\frac{1}{2}B\left(1 +\tilde{\mu}\right) - \delta_1A - \frac{3}{4}\tilde{\gamma} A\left(A^2 + B^2\right). 
\end{align*}
Note that the amplitude of the zeroth order oscillations is given by the length of the vector $(A, B)$. So, when the origin of the dynamical system above is stable, small oscillations of the parametrically driven mode die out, and when it is unstable, small oscillations are amplified. Therefore, the parametric resonance curve is defined by the parameter values at which the linearized system transitions from having no eigenvalues with positive real part, to having at least one. The eigenvalues of the linearized system are given by:
\[\lambda_\pm = -\frac{\tilde{\mu}}{2} \pm \sqrt{\frac{1}{4}-\delta_1^2}.\]
From this we conclude that the origin transitions from unstable to asymptotically stable when $\lambda_- = 0$. With some rearrangements, substituting back in $\delta$ and $\mu = \tilde{\mu}/\epsilon$, we find the following equation for the resonance curve in the $\delta-\epsilon$ plane:
\begin{equation*}
\left(\delta - \frac{1}{4}\right)^2= \frac{1}{4}(\epsilon^2 - \mu^2).
\end{equation*}
Finally, we can re-dimensionalize the expression by introducing the driving frequency $\omega_d$, which in Eq. \ref{eqn:Ising} is set to $2\omega_l$ in accordance with the parametric resonance condition: 
\begin{equation*}
\alpha^2 = \frac{\omega_d^2}{\omega_l^2Q_l^2} + \left(\frac{\omega_d^2}{2\omega_l^2} - 2\right)^2. \label{stability_curve_phys}   
\end{equation*}
This is the expression used to determine the region of self-resonance in Fig.~\ref{fig:flip}.

\section{Numerical methods}\label{Appendix:NumMeth}
We solve the stochastic differential equations (SDEs) using a Splitting Path Runge-Kutta solver (SPaRK), while for the deterministic simulations (four dots in Fig.~\ref{fig:lattice}b), we use a Dormand-Prince algorithm with order 7-8. Both solvers are implemented in Python using the Diffrax library. We choose a time step for a minimum resolution of 50 points per period of oscillation.

\subsection{Simulation of the learning process}
We simulate this process by computing the equilibrium amplitudes of $\vec{\psi}$ using the linearized model, and then determining the flip probability using the single-site flip probability (Eq.~\ref{eqn:switchprobability})---which provides an exact result as we switch off the inter-site potential during coupling.

\section{Realization of the interaction potential with nonlinear springs}\label{Appendix:NonlinearSprings}
The spin-dependent part of potential from Eq.~\ref{eqn:interactionpotential} can be approximated by a set of cubic springs, governed by the Hamiltonian:
\begin{equation}
H_{I,ij}=\lambda'(t)\sum{\alpha_kd^4_{v_{k,i}-u_{k,j}}},
\label{eqn:interactionpotentialSpring}
\end{equation}
where $d_{v_{k,i}-u_{k,j}}=v_{k,i}-u_{k,j}$ is a difference  between the field value $v_{k,i}$ at drum $i$ and the field value $u_{k,i}$ at drum $j$. Field values are defined as a linear combination of modes, $v_{k,i}=\vec{\kappa}^T_k \cdot (\psi_{x, i}, \psi_{y, i}, \sigma_{i})$ and $u_{k,j}=\vec{\kappa}^T_k{}' \cdot (\psi_{x, j}, \psi_{y, j}, \sigma_{j})$ respectively.  The coefficients $\vec{\kappa}_k$, $\vec{\kappa}_k{}'$ and value $\alpha_k$ have been calculated to induce a spin-dependent spring between sites, while compensating for spin-dependent local frequency shifts in the computational and long-term degrees of freedom. These coefficients are provided in Table \ref{tab:coefficients}. Although our work concerns a tight-binding model, a potential experimental realization might involve a mechanical resonator network, where each of the coupling terms in Eq. \ref{eqn:interactionpotential} corresponds to a cubic spring connecting a point $\vec{r}_{1,k}$ in resonator $i$ where the tight-binding basis functions $\phi_{\sigma/\psi_x/\psi_y}(\vec{r}_{1,k})$ take the values $\kappa_{1/2/3,k}$ with a point $\vec{r}_{2,k}$ of resonator $j$ where the tight-binding eigenfunctions $\phi_{\sigma/\psi_x/\psi_y}(\vec{r}_{1,k})$ take the value $\kappa_{1/2/3,k}'$. 

\begin{table}[ht]
    \centering
    \begin{tabular}{|c|c|c|}
    \hline
    $\alpha_k$ & $\kappa_k$ & $\kappa_k'$ \\
    \hline
    1 & $(1,0,1)$ & $(1,0,1)$ \\
    1 & $(0,1,1)$ & $(0,1,1)$ \\
    \hline
    -1 & $(1,0,1)$ & $(0,0,1)$ \\
    -1 & $(0,0,1)$ & $(1,0,1)$ \\
    -1  & $(0,1,1)$ & $(0,0,1)$ \\
    -1  & $(0,0,1)$ & $(0,1,1)$ \\
    \hline
    2  & $(0,0,1)$ & $(0,0,1)$ \\
    \hline
    \end{tabular}
    \caption{Coefficients for the inter-site interaction potential. }
    \label{tab:coefficients}
\end{table}
Of the seven interaction potential terms in Table~\ref{tab:coefficients}, the first two introduce the spin-dependent inter-site interactions between modes $\psi_x-\psi_x'$ and $\psi_y-\psi_y'$ respectively; the next four terms compensate for spin-dependent changes in the natural frequency of the computational degrees of freedom, while the last term compensates for spin-dependent changes in the Ising degrees of freedom. These extra terms are requierd to cancel the contribution of cross-products in Eq.~\ref{eqn:interactionpotentialSpring}, that do not exist in Eq.~\ref{Appendix:NonlinearSprings}. Only the first two terms determine the flow of information, as they couple computational degrees of freedom between sites. In absence of the compensating terms, the system may still be capable of learning. However, in this case, the output phase becomes dependent on the spin texture, which complicates the learning process.

The values in Table~\ref{tab:coefficients} have been determined by expanding Eq.~\ref{eqn:interactionpotentialSpring} and choosing the coefficients to cancel on-resonance terms other than those in Eq.~\ref{eqn:interactionpotential}.

 We expect that these results will motivate experimental realizations of the model, for example in metamaterial or optomechanical systems, in the same way that the emergence of topological tight-binding Hamiltonians~\cite{benalcazar2017quantized} motivated the realization of metamaterial topological insulators~\cite{serra2018observation, peterson2018quantized, matlack2018designing}. In addition, this work poses a broad spectrum of novel questions exploring the realization of alternative learning potentials, learning rules and network architectures; the study of systems with more than two memory scales; as well as of the kinetics of the learning process.