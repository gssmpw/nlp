\section{Related Work}
\vspace{-0.1cm}
\label{related_work}
\subsection{Vision Foundation Models}
Vision-Language Models (VLMs) \cite{ALIGN} are trained using massive datasets that pair images with text. These models utilize distinct encoders for image modality and text modality to generate embeddings, respectively. During training, they employ a contrastive learning objective to enhance the alignment of embeddings from positively correlated image-text pairs. A primary use of these models is in tasks such as zero-shot image-text retrieval or zero-shot classification through textual prompts \cite{CLIP}. Additionally, models like ViLT~\cite{VILT}, VLMo~\cite{VLMO}, and BLIP~\cite{BLIP} have been designed to enhance zero-shot capabilities of visual question answering and image captioning. Methods such as LiT~\cite{LIT}, and BLIP-2~\cite{BLIP2} have been developed to minimize the training expenses for CLIP-like architectures by utilizing pre-trained unimodal models. SAM \cite{segment-anything} can produce segmentation masks effectively, while it is hindered in widely practical use owing to prompt necessity, coarser segmentation, and slow segmentation speed. To tackle SAM's such weaknesses, this paper proposes an interpretable, high-fidelity and prompt-free annotator LAM by leveraging an unrolling optimization mechanism and a single pre-annotated RGB seed image. 

\subsection{Prompt Engineering}
\vspace{-0.1cm}
Prompt engineering \cite{tonmoy2024comprehensive} has become a pivotal strategy for boosting the functionality of pre-trained large language models (LLMs) \cite{10265134} and VLMs \cite{10611726}. It entails the deliberate creation of task-specific directives (known as prompts) to direct the output of models without modifying their parameters. Prompt engineering is particularly prominent in enhancing the flexibility of LLMs and VLMs \cite{10610948}, which allows these models to perform excellently across a variety of tasks and fields. This flexibility marks a departure from conventional methods that typically require retraining or extensive fine-tuning for specific task. As prompt engineering continues to evolve, ongoing research continually uncovers new methods \cite{zhang2024extracting} and applications \cite{xiao2024efficient}. Current researches on prompt engineering include a range of techniques, such as zero-shot prompts \cite{allingham2023simple}, few-shot prompts \cite{lu2021fantastically}, etc. Despite such achievements of prompt engineering, crafting effective prompts often requires deep expertise not only in the model's workings but also in the specific domain knowledge \cite{liu2022design}. This can limit the application of prompt engineering. To surmount such prompt's limitations, this paper proposes to provide a pre-annotated RGB seed image instead of image-specific prompts to enhance the quality of annotations.