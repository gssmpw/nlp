\section{Related Work}
Sound has long been used as a communication medium, with conventional applications in technologies such as line modems and fax machines, which rely on acoustic signals for data transmission ____. These systems encode information into audible or inaudible frequencies, demonstrating the foundational principles of sound-based communication. In more modern contexts, acoustic communication has expanded to include techniques leveraging the omnipresence of microphones and speakers in everyday environments, facilitating innovative applications like device pairing and covert data channels ____. 

Audio-based communication has been extensively studied in computer science, particularly in the contexts of communication channels, covert channels, data exfiltration, and human-computer interaction. Techniques utilizing sound as a communication medium often exploit the inherent advantages of acoustic signals, such as their ability to traverse through the air and their compatibility with wide range of existing hardware. For instance, ultrasonic communication has been explored for device pairing (e.g., smartphones), leveraging inaudible frequencies to enable short-range, high-frequency data exchanges ____. Other approaches have demonstrated the feasibility of both audible and inaudible audio channels for covert communication channels ____. Note that covert channels, in general, have long been recognized as a significant threat due to their ability to circumvent traditional security measures. Recent research has explored their exploitation in Internet protocols, highlighting their persistent impact on security systems ____.


\subsection{Active Acoustic Side-Channels}
Researchers have explored innovative ways to generate sound from systems that lack dedicated audio hardware. Guri et al presented Fansmitter, a malware that manipulates CPU and GPU fan speeds to produce acoustic signals detectable by nearby devices such as smartphones ________. Similarly, DiskFiltration encodes data into the operational noise of hard disk drives (HDDs) actuator arm during read and write operations, allowing for acoustic data transmission ____. Another method, CD-LEAK, leverages the mechanical noise of optical drives, such as CD/DVD players, for data modulation ____. Additionally, power-supply-based attacks utilize the switching frequencies of power supply units (PSUs) to emit sound or ultrasonic waves, providing a novel channel for data exfiltration without relying on conventional audio hardware ____. More recently, EL-GRILLO demonstrated how ultrasonic data can be leaked from air-gapped PCs using the tiny motherboard buzzer, further expanding the field of acoustic side-channel research ____. 
Unconventional sound sources have further expanded the field of audio-based communication. For instance, InkFiltration demonstrates the use of mechanical noise from inkjet printers to transmit data over short distances ____. The MOSQUITO method employs ultrasonic communication to transfer data between air-gapped systems, enabling speaker-to-speaker communication using inaudible frequencies ____. More recently, PIXHELL has explored the potential of generating sound through LCD displays by modulating pixel transitions ____. While PIXHELL introduces the concept of LCD-based audio generation, it primarily emphasizes the broader possibilities rather than delving into the precise algorithms and methods required for effective sound production and data transmission.

\subsection{Passive Acoustic Side-Channels}
Passive Side-channel attacks exploit unintended information leakage from devices, such as timing, power consumption, or acoustic emanations. Among these, acoustic side-channels focus on analyzing sound emitted by devices to infer operations or extract sensitive information. These attacks are often passive, requiring no direct interaction with the device, and are particularly effective when other side-channels are inaccessible. Backes et al. demonstrated an attack leveraging the acoustic emanations of dot matrix printers to reconstruct printed text, highlighting vulnerabilities in printer operations ____. Deepa et al. provided a comprehensive survey on acoustic side-channel attacks, detailing various techniques and methodologies to exploit sound leakage for data retrieval ____. Faruque et al. introduced an acoustic side-channel attack on 3D printing systems, where sound emanations were used to infer the geometry of printed objects, bridging the physical and cyber domains ____. Halevi and Saxena analyzed keyboard acoustic emanations to eavesdrop on password entry, employing statistical and machine learning models to reconstruct typed input ____. Taheritajar et al. reviewed advancements in keyboard acoustic side-channel attacks, emphasizing improvements in sound recording techniques and their implications for security ____. Harrison et al. proposed a deep learning-based method for acoustic side-channel attacks on keyboards, achieving significant improvements in accuracy over traditional approaches ____. Toreini et al. explored the feasibility of acoustic side-channel attacks on the Enigma machine, showing that its operations could be inferred through sound leakage ____. Finally, Shumailov et al. introduced an acoustic side-channel attack targeting virtual keyboards on smartphones, reconstructing typed input by analyzing the sound of touchscreen taps ____.