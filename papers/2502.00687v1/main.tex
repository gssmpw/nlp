\documentclass[conference]{IEEEtran}
\IEEEoverridecommandlockouts
% The preceding line is only needed to identify funding in the first footnote. If that is unneeded, please comment it out.
\usepackage{cite}
\usepackage{amsmath,amssymb,amsfonts}
\usepackage{algorithmic}
\usepackage{graphicx}
\usepackage{subfigure}
\usepackage{multirow}
\usepackage{array}
\usepackage{threeparttable} % 引入 threeparttable 宏包
\usepackage{tablefootnote} % 引入 tablefootnote 宏包
\renewcommand\arraystretch{1.2}
\setlength{\abovecaptionskip}{0.cm} %%% 3pt 个人觉得稍妥,可自行设置
\setlength{\belowcaptionskip}{0.cm}
% \usepackage{hyperref}
% \hypersetup{
%     colorlinks=true,
%     linkcolor=black,  % 内部链接颜色为黑色
%     citecolor=black,  % 引用链接颜色为黑色
%     urlcolor=black,   % URL 链接颜色为黑色
%     linkbordercolor=0 0 0, % 边框颜色设置为透明,即不显示边框
% }
\setlength{\arrayrulewidth}{0.15mm} % 设置线条粗细

\usepackage{textcomp}
\usepackage{xcolor}
\def\BibTeX{{\rm B\kern-.05em{\sc i\kern-.025em b}\kern-.08em
    T\kern-.1667em\lower.7ex\hbox{E}\kern-.125emX}}
\begin{document}

% \title{A Flexible Precision Scaling Deep Neural Network Accelerator with Systolic Dataflow}
\title{A Flexible Precision Scaling Deep Neural Network Accelerator with Efficient Weight Combination}


 \author{\IEEEauthorblockN{Liang Zhao\textsuperscript{1,*}, Kunming Shao\textsuperscript{2,3,*}, Fengshi Tian\textsuperscript{2,3}, Tim Kwang-Ting Cheng\textsuperscript{2,3}, Chi-Ying Tsui\textsuperscript{2,3}, Yi Zou\textsuperscript{1,†}}

 \thanks{*Both authors contributed equally.}

 \thanks{†The corresponding author is Yi Zou (e-mail: zouyi@scut.edu.cn)}

 \IEEEauthorblockA{\textsuperscript{1}South China University of Technology, Guangzhou, China\\
 \textsuperscript{2}The Hong Kong University of Science and Technology, Hong Kong SAR, China\\
 \textsuperscript{3}AI Chip Center for Emerging Smart Systems (ACCESS), Hong Kong SAR, China
 }
 
\vspace{-10mm}
}

\maketitle



\begin{abstract}
Deploying mixed-precision neural networks on edge devices is friendly to hardware resources and power consumption. To support fully mixed-precision neural network inference, it is necessary to design flexible hardware accelerators for continuous varying precision operations. However, the previous works have issues on hardware utilization and overhead of reconfigurable logic. In this paper, we propose an efficient accelerator for 2$\sim$8-bit precision scaling with serial activation input and parallel weight preloaded. First, we set two loading modes for the weight operands and decompose the weight into the corresponding bitwidths, which extends the weight precision support efficiently. Then, to improve hardware utilization of low-precision operations, we design the architecture that performs bit-serial MAC operation with systolic dataflow, and the partial sums are combined spatially. Furthermore, we designed an efficient carry save adder tree supporting both signed and unsigned number summation across rows. The experiment result shows that the proposed accelerator, synthesized with TSMC 28nm CMOS technology, achieves peak throughput of 4.09TOPS and peak energy efficiency of 68.94TOPS/W at 2/2-bit operations.
% Deploying mixed-precision neural networks on edge devices is hardware-friendly for higher efficiency while maintaining the accuracy. Continuous mixed-precision quantization further improves the layer-wise quantization flexibility without significant accuracy degradation. However, it puts a challenge on the reconfigurability overhead of the hardware design. In this paper, we proposed an efficient accelerator for 2$\sim$8-bit precision scaling with serial activation input and parallel weight. First, we set two weight loading modes and decompose the weight by forming weight groups, supporting the weights' continuous precision efficiently. Then, to avoid hardware under-utilization of low-precision operation, we designed the architecture that performs bit-serial MAC operation with systolic dataflow based on the two weight loading modes, and the partial sums are combined spatially. Furthermore, we designed an efficient carry save adder tree supporting both signed and unsigned product summation across rows. The experiment result shows that the proposed accelerator, synthesized with under TSMC 28nm CMOS technology, achieves a peak throughput of xxxTOPS and a peak energy efficiency of xxxTOPS/W at 2bit.

\end{abstract}


% (1) Background for mixed-precision

% (2) Background for hardware under-utilization

% (1) Continuous precision
% (2) Systolic input
% (3) Reconfigurable adder tree 

% Based on the bit-serial design, the accumulation and shift logic can be shared in the same MAC operation, which reduces the overhead of shift registers and multipliers.

\begin{IEEEkeywords}
Deep neural network, mixed-precision accelerator, hardware utilization, systolic dataflow.
\end{IEEEkeywords}


\section{Introduction}
% With the development of deep neural networks (DNNs), their applications in edge devices have become increasingly widespread. However, the growing number of model parameters brings challenges to memory access and power consumption for hardware computing. % In these quantization, a suitable search algorithm is designed to find the optimal bit setting and each layer is quantized with different precision. 
The increase in model parameters and size of deep neural network (DNN) poses challenges for hardware computing. On edge devices with constraints on hardware resources and power, quantizing the model has become an effective method to address this issue. To reduce parameter bitwidths without significantly impacting accuracy, some studies \cite{dong2019hawq,wang2019haq} have explored methods for mixed-precision quantization of models. Using fixed-bitwidths accelerators for mixed-precision neural network accelerators will waste hardware resources for low-precision computations. To efficiently support mixed-precision neural network inference, many accelerators are designed with reconfigurable computing units, enabling optimal utilization of hardware resources under different precision computations. 

For the DNN accelerator, the primary computation involves multiply-accumulate (MAC) operations of weights and activations. Depending on the input format of weights/activations and the scaling method of MAC units, mixed-precision accelerators can be classified into bit-parallel and bit-serial architectures, representing precision scaling in spatial and temporal dimensions\cite{camus2019review}. Based on bit-parallel architecture, \cite{moons201714} scales precision by gating larger computing units. However, only hardware resources on the diagonal of the array are used for operations so the hardware utilization is low. Some other works \cite{BitFusion,ryu2022bitblade,li2023precision} achieve precision scaling by combining computing units, which improves hardware utilization, but case more overhead of the reconfigurable logic. 

The accelerators based on bit-parallel architecture mentioned above only support precision with power-of-2 bitwidths, lacking precision flexibility. In contrast, bit-serial accelerators can efficiently achieve continuous precision scaling for serial input operands. \cite{judd2016stripes,albericio2017bit,sharify2018loom} support 1$\sim$16-bit precision with serial input for both weights and activations but they require many cycles for MAC operations and have complex control logic. Some works \cite{lee2018unpu,yang2020bitsystolic,chi202116} are designed with only one of the operands are serial-input. However, their parallel-input operands cannot efficiently support flexible precision scaling. They have problems with hardware utilization when performing low-precision(e.g.,2$\sim$5-bit) operations.
% the fixed-size registers used to store parallel operands lead to lower hardware utilization when performing lower precision computation. 

In order to support flexible precision scaling and improve hardware utilization in low-precision computation, we propose an precision-scalable architecture with an efficiency weight combination method. Our contributions are as follows:
\begin{itemize}
\item We set two loading modes for weight operands, 2-bit mode and 3-bit mode. Based on these two modes, we decomposed the weight and the architecture can efficiently support flexible scaling of weight precision.
\item We use systolic dataflow to process activations fed in serial and weights preloaded in parallel. Partial sums computed by the decomposed weights from each column are combined spatially, which improves the hardware utilization in low-precision computations. 
\item We use carry save adder (CSA) tree to sum the multiplication results, with two independent adder tree paths to support both signed and unsigned operations. Compared with the binary adder tree (BAT), the CSA tree in this paper has lower power consumption, especially when processing unsigned numbers.
\end{itemize}
\section{Motivation and Related Work}
For parallel operands, \cite{BitFusion} achieved precision scaling by combining 2bit$\times$2bit computing units. The advantage of this kind of design is that when computing low-precision values, the operation scale increases and the hardware utilization is high. However, the bit-serial design has better precision scaling flexibility and simpler basic unit with shared shift operations than the bit-parallel design.

In the previous bit-serial work, some work \cite{judd2016stripes,sharify2018loom} used serial input for both weights and activations, while others\cite{lee2018unpu,yang2020bitsystolic} used serial input for only one of them. The latter has fewer operation cycles and simpler control logic but the precision scaling of its parallel operands will lead to hardware utilization issues. The accelerator can have high hardware utilization when the parallel operands are input with the bitwidths matching their fixed-size registers because the low-precision operations can only be achieved by gating unnecessary bits in the registers. For example, \cite{yang2020bitsystolic} uses 8-bit register files to store the bit-vector of weight, which means when the weight precision is 2-bit, 6-bit registers are not used and a part of the bitwidths of the adder and accumulator are wasted, as illustrated in Fig. \ref{relatedwork}(a). In addition, \cite{yang2020bitsystolic} does not involve the processing of signed bit. Fig. \ref{relatedwork}(b) shows that \cite{chi202116} achieves  4/8/16-bit weight bitwidths by combining 4-bit units, which is similar to the methods of combining parallel operands in \cite{BitFusion}. We could use lower bitwidths units for combination to increase the precision scaling flexibility and the hardware utilization under the low-precision operations, but the redundancy caused by the combination logic and the appropriate method for scaling precision should be taken into consideration.
\begin{figure}[t]
\centering
\includegraphics[width=\columnwidth]{figure/background_comb.png}
 \caption{Architecture of (a) \cite{yang2020bitsystolic} and (b) \cite{chi202116}.} 
 \label{relatedwork}
\end{figure}
\section{Proposed Precision-Scalable Architecture}
To support the bitwidths flexibility of mixed-precision DNNs and improve the hardware utilization under low-precision computations, we propose an efficient precision-scalable architecture, as shown in Fig. \ref{fig1}, with off-chip memory, peripheral buffer and PE array that has 64 rows and 64 columns. We use systolic dataflow so the activations are iterated to each group through registers, which results in a smaller input fanout and allows the accelerator to operate at a higher frequency than multicast dataflow.
\begin{figure}[t]
     \centering
     \includegraphics[width =\columnwidth]{figure/overall_new.png}%
     \caption{The overall architecture.}
     \label{fig1}
\end{figure}
\begin{figure}[h]
\centering
\includegraphics[width=0.9\columnwidth]{figure/loadw_comb.png}
 \caption{(a) Schematic of the multiplier. (b) Load weight in 2-bit mode. (c) Decompose 5-bit weight with 3-bit and 2-bit mode.} 
 \label{loadw}
\end{figure}
% \vspace{-5mm}
\subsection{Weight Precision Scaling Method}
We design two weight loading modes for each column, 2-bit mode and 3-bit mode. In 2-bit mode, the decomposed weights are input to the multipliers with 2-bit and the extended sign bit is obtained through the signal S which indicates whether the inputs are signed or unsigned. If the input is a signed number, S$=$1, otherwise S$=$0. As shown in Fig. \ref{loadw}(a), the highest bit of the multiplier input is connected to the extended sign bit and the multipliers in each column share the same S signal. Fig. \ref{loadw}(b) illustrates how the weights are loaded in 2-bit mode. Fig. \ref{loadw}(c) shows the way of decomposing and loading weight with 3-bit mode. The highest 3 bits of the weight, including the original sign bit, are directly input to the multipliers and signal S can be ignored. Since the products of the multiplier in both modes are 3-bit signed numbers, the computation logic can be fully reused.

We set four columns as a group for weight combination to minimize the overhead of shifters and the logic for reconfigurability. When all columns in the group are in 2-bit mode, it can support four 2-bit, two 4-bit, one 6-bit and one 8-bit weight operations. Based on this condition, we can set the specific column to 3-bit mode to support four 3-bit, two 5-bit, and one 7-bit weight operations. The bitwidths of decomposed weight are shown in Table. \ref{table1}. 

When the weight precision is 6/7-bit, one column in each group is idle, which can be gated to reduce energy consumption. However, this method reduces array utilization. To fully utilize array resources, we design independent shift-add paths to accumulate the results of these columns. Fig. \ref{fig3} illustrates the shift-add path design based on systolic dataflow and it has the same number of register stages as the original path. Five shift-add paths need to be added, which can be gated in other precision modes, and the whole array only has one column that remains idle. The weights need to be rearranged and routed to the corresponding columns.
% to ensure they are combined correctly through the independent paths.

Compared to gating large registers or combining weight with the extended sign bit in the previous work, our weight scaling method has higher hardware utilization for low-precision operation and uses the bitwidths occupied by the extended sign bit to achieve higher precision flexibility.
% \vspace{-2mm}
\begin{table}[t]
\centering
\caption{The configure information of decomposed weight and shift-add logic}
\begin{tabular}{|c|ccccccc|}
\hline
\multirow{2}{*}{\textbf{Config Info}} & \multicolumn{7}{c|}{\textbf{Weight Precision}}                                                     \\ \cline{2-8} 
 &
  \multicolumn{1}{c|}{\textbf{8}} &
  \multicolumn{1}{c|}{\textbf{7}} &
  \multicolumn{1}{c|}{\textbf{6}} &
  \multicolumn{1}{c|}{\textbf{5}} &
  \multicolumn{1}{c|}{\textbf{4}} &
  \multicolumn{1}{c|}{\textbf{3}} &
  \textbf{2} \\ \hline
\textbf{Decomposed bit} &
  \multicolumn{1}{c|}{2-2-2-2} &
  \multicolumn{1}{c|}{3-2-2} &
  \multicolumn{1}{c|}{2-2-2} &
  \multicolumn{1}{c|}{3-2} &
  \multicolumn{1}{c|}{2-2} &
  \multicolumn{1}{c|}{3} &
  2 \\ \hline
\textbf{Shifter \#0 (bit)}                   & \multicolumn{1}{c|}{2}  & \multicolumn{2}{c|}{0} & \multicolumn{2}{c|}{2} & \multicolumn{2}{c|}{0} \\ \hline
\textbf{Shifter \#1 (bit)}                   & \multicolumn{5}{c|}{2}                                                    & \multicolumn{2}{c|}{0} \\ \hline
\textbf{Shifter \#2 (bit)}                   & \multicolumn{3}{c|}{4}                           & \multicolumn{4}{c|}{0}                          \\ \hline
\end{tabular}
\label{table1}
\end{table}
% \vspace{-6mm}
\begin{figure}[t]
     \centering
     \includegraphics[width =0.8\columnwidth]{figure/bit7sum.png}%
     \caption{Independent shift-add path for summation in 6/7-bit weight operation.}
     \label{fig3}
\end{figure}
% \vspace{-6mm}
\begin{figure}[t]
\centering
\includegraphics[width=0.9\columnwidth]{figure/shift_sum.png}
 % \caption{Schematic of (a)accumulator and (b)configurable shift-add logic.} \label{fig4}
 \caption{Schematic of configurable shift-add logic.} \label{fig4}
\end{figure}
% \vspace{-4mm}
\subsection{Bit-seial MAC Operation}
The proposed architecture uses weight-stationary systolic dataflow. Weights are preloaded into the array from top to bottom, and activations are fed to each groups by 1-bit iterations. The operands in our work are all represented in 2’s complement format. For M-bit weights and N-bit activations, the principle of MAC operation based on the proposed architecture is as follows:
\begin{equation}
\begin{aligned}
MAC = \sum_{c=0}^{\left [ \frac{M}{2} \right ]- 1} \left (  \sum_{t=0}^{N} \sum_{r=0}^{63}A^{r}\left [ t \right ] \cdot W^{r}_{dcp} \left [ c \right ] \right. \\
\cdot \left ( -1 \right )^{SF}  \cdot 2^{t} \left. \right)\cdot 2^{2c}  
 \end{aligned}
\label{eq1}
\end{equation}
Where $W_{dcp}$ represents the value of decomposed weight assigned to each column, $SF$ is the signal which indicates the sign bit of the activation, $t$ represents different cycles in time dimension, $c$ is column index and $r$ is row index. $W_{dcp}$ is considered positive if it is unsigned. From Eq. (\ref{eq1}), it can be noticed that the weight value of the sign bit of activation is $-2^{N-1}$.

In each column of the array, the decomposed weights are bit-wise multiplied with the 1-bit activation values, and the products from each row are summed by the adder tree. The results of the adder tree from N cycles are input to the accumulator to obtain the MAC results of N-bit activation and decomposed weights. From Eq. (\ref{eq1}), we can know that when the 1-bit activation input is the sign bit, the output value of the adder tree needs to be the inverse value before accumulation.
% As shown in Fig. \ref{fig4}(a), 
When the 1-bit activation input is the sign bit and $SF$=1, the output of the adder tree needs to be bit-wise inverted and plus 1. When $SF$=0, the output of the adder tree is directly input to the shift-accumulator. In our design, the activations are input starting from the least significant bit (LSB). Finally, the results of each column are shifted and added to obtain the final result, as shown in Fig. \ref{fig4}. We use another clock domain with lower frequency for the shifter and adder logic to reduce the power consumption and the frequency is set according to activation precision. For example, if activation precision is 8-bit, the frequency of $clk\_SA$ is 1/8 of that of $clk$. Table \ref{table1} illustrates the configuration of the corresponding shifters. For each shifter, it only has two configuration cases, which saves the overhead of reconfigurable logic. 
% Some network layers, such as depthwise convolution layer whose weights are mapped to a one-dimensional vector, have different characteristics from convolution layer. Based on the data flow of the activations propagating to each group of the array, the weights need to be preloaded multiple times, which will lead to clock cycle overhead and low array utilization. Therefore, we hope that the different sets of activations can be directly input to the corresponding group for MAC operation. In this case, different sets of weights can be preloaded at once, which improves computation efficiency. Based on this architecture, we can increase the support of one-dimensional weight operations by connecting the activation register of each group directly to the input buffer and using a multiplexer to switch the path. However, this data flow requires higher bandwidth for the buffer and is not well adapted to the weight decomposition method mentioned above, so we only conducted a feasibility analysis.
% \subsection{Adder Tree Path Optimization}
% In this work, we use adder tree consisting of carry save adder (CSA) to perform summation operations and optimize the energy efficiency for precision scaling. \cite{sridharan2024ps} has discussed that CSA tree which separates partial sums and carries has better performance and fewer full adders than binary adder tree. To further improve the efficiency of adder tree, we consider the characteristics of summation operations. For 4$\sim$8-bit weight operations, the products of the bit-wise multipliers in some columns are unsigned numbers, which means the most significant bit (MSB) of them are zero and perform invalid summation operation. 
% Therefore, we use two independent sub-adder trees to sum the MSB and the lower 2 bits of the products separately, as shown in Fig \ref{fig5}. When the adder tree in a column performs unsigned number summation, the result of lower 2-bit sub-adder tree is directly output and the MSB sub-adder tree is gated. In this way, the number of shift-adds that are required to obtain the final result reduces, which saves the power consumption. If the column performs signed number summation, the sum of the results from two sub-adder trees is selected to be output through the multiplexer.
\begin{figure}[b]
     \centering
     \includegraphics[width =0.75\columnwidth]{figure/adder_tree_new.png}%
     \caption{CSA tree for both signed and unsigned number summation.}
     \label{fig5}
\end{figure}
% \vspace{-5mm}
\subsection{Adder Tree Design}
In this work, we use adder tree consisting of CSA to perform summation operations and optimize the energy efficiency for precision scaling. \cite{sridharan2024ps,shao2024syndcim} has used CSA to design the adder tree which has fewer full adders than BAT. In the proposed work, the adder tree needs to sum 64 3-bit signed numbers. However, as the CSA tree separates the carries and partial sums during the operation, it is hard to take a 3-bit signed number as a whole and process the sign bit as BAT. Thus, we use two independent CSA tree paths to sum the MSB and the lower 2 bits of the products separately, as shown in Fig \ref{fig5}. The result of the MSB adder tree can be considered as the number of 1s among the highest bit of the 64 signed numbers. As mentioned in the last section, the weight value of the sign bit is negative, so the result should be inverse. The lowest 2 bits of the result of lower 2-bit adder tree are directly output, while the highest 7 bits of that are added to the result of the MSB adder tree and then output. When the weights loaded in this column are unsigned numbers, all the inputs of the MSB addition tree are 0. As the two adder tree paths are independent, the number of invalid carries is reduced compared to the BAT.

\section{Experimental Results}
We use Design Compiler and PrimeTime PX to evaluate the proposed design and our design is synthesized with TSMC 28nm CMOS technology. We evaluate the area and power breakdown of the PE array, as shown in Fig. \ref{fig7}. The area occupation of the independent shift-add path for 6/7-bit weight operation is only 0.97$\%$, which is considered acceptable. 
% \begin{figure}[h]
%      \centering
%      \subfigure[]{
% \begin{minipage}[t]{0.49\columnwidth}
% \includegraphics[width=1\textwidth]{figure/Area_breakdown.png}
% \label{Area_breakdown}
% \end{minipage}
% }
% \subfigure[]{
% \begin{minipage}[t]{0.44\columnwidth}
% \includegraphics[width=1\textwidth]{figure/Power_breakdown.png}
% \label{Power_breakdown}
% \end{minipage}
% }
% \caption{(a)Area and (b)power breakdown of PE array(8/8-bit).}
% \label{fig7}
% \end{figure}
 \vspace{-3mm}
\begin{figure}[h]
     \centering
\includegraphics[width=1\columnwidth]{figure/breakdown_comb.png}
\vspace{-6mm}
\caption{(a)Area and (b)power breakdown of PE array(8/8-bit).}
\label{fig7}
\end{figure}
 \vspace{-3mm}
 
 We compare the area and power consumption of the adder tree proposed in this paper and the binary adder trees. For the sake of fairness, two types of adder tree are designed and tested with the same input bitwidths and data. Table. \ref{table2} shows the area and power of proposed CSA tree which is normalized according to the BAT. Compared with BAT, our CSA adder tree achieves 15.14$\%$ reduction in area. It has 31.03$\%$ and 22.28$\%$ power reduction when computing unsigned number and signed number respectively.
 \vspace{-4mm}
\begin{table}[h]
\centering
\caption{Area and power consumption of proposed adder tree}
\resizebox{0.8\columnwidth}{!}{
\begin{tabular}{|c|c|cc|}
\hline
\multirow{2}{*}{Normalized value} & \multirow{2}{*}{\textbf{Area}} & \multicolumn{2}{c|}{\textbf{Power}}     \\ \cline{3-4} 
                                  &                                & \multicolumn{1}{c|}{Unsigned} & Signed \\ \hline
\textbf{Proposed CSA Tree}        & 0.8486                         & \multicolumn{1}{c|}{0.6897}    & 0.7772 \\ \hline
\end{tabular}
}
\label{table2}
\end{table}
 \vspace{-2mm}
% \begin{figure}[h]
%      \centering
% \includegraphics[width=1\columnwidth]{figure/toggle_comb.png}
% \vspace{-6mm}
% \caption[t]{Energy efficiency of PE array under different input sparsity.}
% \label{fig8}
% \end{figure} 

We measure the energy efficiency of the PE array under different input toggle rates and supply voltages, aiming to demonstrate the general processing efficiency of the PE array. To facilitate testing, we set the bitwidths of weights and activations to be the same, and weight sparsity is set to 50$\%$. The result are shown in Fig. \ref{fig7}, the PE array achieves a peak energy efficiency with a voltage supply of 0.72V under a clock frequency of 500MHz. In this condition, the design has 14TOPS/W, 52.1TOPS/W, 139.8TOPS/W and 205.8TOPS/W energy efficiency at 8/8-bit, 4/4-bit, 3/3-bit and 2/2-bit operations respectively.

\begin{figure}[h]
     \centering
\includegraphics[width=1\columnwidth]{figure/toggle_comb.png}
\vspace{-6mm}
\caption[t]{Energy efficiency of PE array under different input toggle rates.}
\label{fig8}
\end{figure}

% \cite{yang2020bitsystolic} based on bit-level systolic dataflow and \cite{PSEA} with multiplier optimization directly. The result is given in \ref{table3}.
The whole accelerator is designed with buffer size of 144KB and tested the inference performance on MobileNetV2\cite{sandler2018mobilenetv2}. Compared to fixed 8-bit model, the accelerator has power reduction of 35.2$\%$ when performing mixed-precision model. We compare the whole flexible precision scaling accelerator with some previous works, as shown in Table \ref{table3}. Compared with the early bit-serial design \cite{lee2018unpu}(UNPU), the proposed work has the advantage of precision scaling flexibility. Our accelerator has higher peak throughput peak energy efficiency in three precision modes than \cite{yang2020bitsystolic}(BitSystolic) which also support flexible precision scaling. The energy efficiency is increased by 18.7$\%$, 10.5$\%$, 11.2$\%$ at 8/8-bit, 4/4-bit, 2/2-bit operations respectively. Compared with the energy-efficiency optimized work based on bit-parallel design\cite{9920733}, the proposed accelerator also performs higher peak energy efficiency, especially at 2/2-bit operations.
\vspace{-4mm}
\begin{table}[h]
\caption{Comparison with other precision-scalable accelerators}
\begin{threeparttable}
% \begin{tabular}{|c|c|ccc|}
% \hline
% Type                                                                   & Bit-parallel                                                                    & \multicolumn{3}{c|}{Bit-serial}                                                                                                                                                                                                                                             \\ \hline
% Work                                                                   & \begin{tabular}[c]{@{}c@{}}TCAS-I 20'\\ \cite{PSEA}\end{tabular}                    & \multicolumn{1}{c|}{\begin{tabular}[c]{@{}c@{}}JSSC 18'\\ \cite{lee2018unpu}$^{\dagger}$\end{tabular}}          & \multicolumn{1}{c|}{\begin{tabular}[c]{@{}c@{}}TCAS-I 20'\\ \cite{yang2020bitsystolic}$^{\dagger}$\end{tabular}}                      & \begin{tabular}[c]{@{}c@{}}Proposed\\ work\end{tabular}                          \\ \hline
% Tech(nm)                                                                   & 90                                                                            & \multicolumn{1}{c|}{65}                                                                & \multicolumn{1}{c|}{65}                                                                              & 28                                                                             \\ \hline
% \begin{tabular}[c]{@{}c@{}}Supply\\ voltage(V)\end{tabular}               & 1.0                                                                             & \multicolumn{1}{c|}{0.63$\sim$1.1}                                                       & \multicolumn{1}{c|}{0.54$\sim$0.66}                                                                    & 0.72$\sim$1.05                                                                   \\ \hline
% Freq(MHz)                                                              & 500                                                                          & \multicolumn{1}{c|}{200}                                                              & \multicolumn{1}{c|}{200}                                                                            & 500                                                                           \\ \hline
% Area$^{\ast}$(mm$^{2}$)                                                                   & 0.71                                                                            & \multicolumn{1}{c|}{2.969}                                                               & \multicolumn{1}{c|}{0.7422}                                                                            &                                                                             \\ \hline
% Precision                                                              & \begin{tabular}[c]{@{}c@{}}W:2/4/8-bit\\ A:2/4/8-bit\end{tabular}               & \multicolumn{1}{c|}{\begin{tabular}[c]{@{}c@{}}W:1$\sim$16-bit\\ A:16-bit\end{tabular}}  & \multicolumn{1}{c|}{\begin{tabular}[c]{@{}c@{}}W:2$\sim$8-bit\\ A:2$\sim$8-bit\end{tabular}}           & \begin{tabular}[c]{@{}c@{}}W:2$\sim$8-bit\\ A:2$\sim$8-bit\end{tabular}          \\ \hline
% \begin{tabular}[c]{@{}c@{}}Peak \\ throughtput\\ (TOPS)\end{tabular}   & /                                                                               & \multicolumn{1}{c|}{7.372}                                                               & \multicolumn{1}{c|}{0.403}                                                                             & 2.048                                                                             \\ \hline
% \begin{tabular}[c]{@{}c@{}}Energy\\ efficiency$^{\ast}$\\ (TOPS/W)\end{tabular} & \begin{tabular}[c]{@{}c@{}}3.87(8-bit)\\ 15.3(4-bit)\\ 49.8(2-bit)\end{tabular} & \multicolumn{1}{c|}{\begin{tabular}[c]{@{}c@{}}7.15(16-bit)\\ 26.92(4-bit)\end{tabular}} & \multicolumn{1}{c|}{\begin{tabular}[c]{@{}c@{}}3.94(8-bit)\\ 15.79(4-bit)\\ 61.98(2-bit)\end{tabular}} & \begin{tabular}[c]{@{}c@{}}4.52(8-bit)\\ 16.28(4-bit)\\ 62.4(2-bit)\end{tabular} \\ \hline
% % \begin{tabular}[c]{@{}c@{}}Peak area \\ density$^{\ast}$\\(TOPS/mm$^{2}$)\end{tabular}   & /                                                                               & \multicolumn{1}{c|}{2.484}                                                               & \multicolumn{1}{c|}{0.544}                                                                             &                                                                                  \\ \hline
% \end{tabular}
\resizebox{\columnwidth}{!}{ % 0.8\textwidth 表示将表格宽度缩小为页面宽度的80%
\begin{tabular}{|c|c|ccc|}
\hline
Type                                                                   & Bit-parallel                                                                    & \multicolumn{3}{c|}{Bit-serial}                                                                                                                                                                                                                                             \\ \hline
Work                                                                   & \begin{tabular}[c]{@{}c@{}}TVLSI'22\\ \cite{9920733}\end{tabular}                    & \multicolumn{1}{c|}{\begin{tabular}[c]{@{}c@{}}JSSC'18\\ \cite{lee2018unpu}$^{\dagger}$\end{tabular}}          & \multicolumn{1}{c|}{\begin{tabular}[c]{@{}c@{}}TCAS-I'20\\ \cite{yang2020bitsystolic}$^{\dagger}$\end{tabular}}                      & \begin{tabular}[c]{@{}c@{}}\textbf{Proposed}\\ \textbf{work}\end{tabular}                          \\ \hline
Tech(nm)                                                                   & 28                                                                            & \multicolumn{1}{c|}{65}                                                                & \multicolumn{1}{c|}{65}                                                                              & 28                                                                             \\ \hline
\begin{tabular}[c]{@{}c@{}}Supply\\ voltage(V)\end{tabular}               & 0.6$\sim$1                                                                             & \multicolumn{1}{c|}{0.63$\sim$1.1}                                                       & \multicolumn{1}{c|}{0.54$\sim$0.66}                                                                    & 0.72$\sim$1.05                                                                   \\ \hline
Freq(MHz)                                                              & 500                                                                          & \multicolumn{1}{c|}{200}                                                              & \multicolumn{1}{c|}{200}                                                                            & 1000$^{1}$                                                                           \\ \hline
Area$^{\ast}$(mm$^{2}$)                                                                   & 1.43                                                                            & \multicolumn{1}{c|}{2.97}                                                               & \multicolumn{1}{c|}{0.74}                                                                            & 0.75                                                                             \\ \hline
Precision                                                              & \begin{tabular}[c]{@{}c@{}}W:2/4/8-bit\\ A:2/4/8-bit\end{tabular}               & \multicolumn{1}{c|}{\begin{tabular}[c]{@{}c@{}}W:1$\sim$16-bit\\ A:16-bit\end{tabular}}  & \multicolumn{1}{c|}{\begin{tabular}[c]{@{}c@{}}W:2$\sim$8-bit\\ A:2$\sim$8-bit\end{tabular}}           & \begin{tabular}[c]{@{}c@{}}W:2$\sim$8-bit\\ A:2$\sim$8-bit\end{tabular}          \\ \hline
\begin{tabular}[c]{@{}c@{}}Peak \\ throughput\\ (TOPS)\end{tabular}   & 4.12                                                                              & \multicolumn{1}{c|}{7.372}                                                               & \multicolumn{1}{c|}{0.403}                                                                             & 4.09                                                                             \\ \hline
\begin{tabular}[c]{@{}c@{}}Peak energy\\ efficiency$^{\ast}$\\ (TOPS/W)\end{tabular} & \begin{tabular}[c]{@{}c@{}}3.62(8-bit)\\ 12.13(4-bit)\\ 22.89(2-bit)\end{tabular} & \multicolumn{1}{c|}{\begin{tabular}[c]{@{}c@{}}7.15(16-bit)\\ 26.93(4-bit)\end{tabular}} & \multicolumn{1}{c|}{\begin{tabular}[c]{@{}c@{}}3.95(8-bit)\\ 15.79(4-bit)\\ 61.98(2-bit)\end{tabular}} & \begin{tabular}[c]{@{}c@{}}4.69(8-bit)$^{2}$\\ 17.45(4-bit)$^{2}$\\ 68.94(2-bit)$^{2}$\end{tabular} \\ \hline
% \begin{tabular}[c]{@{}c@{}}Peak area \\ density$^{\ast}$\\(TOPS/mm$^{2}$)\end{tabular}   & /                                                                               & \multicolumn{1}{c|}{2.484}                                                               & \multicolumn{1}{c|}{0.544}                                                                             &                                                                                  \\ \hline
\end{tabular}}
\begin{tablenotes}
\raggedright % 使脚注居左
\item[$\ast$]The results are scaled to 28nm.  $^{\dagger}$The works use post-layout estimation. 
% : $s={65}/{28}$, $ Power_{28nm}=Power_{65nm}/{s}$, $ Area_{28nm}=Area_{65nm}/{s^{2} }$.
\item[1]@1.05V.  $^{2}$@0.72V, 500MHz.
\end{tablenotes}
\end{threeparttable}
\label{table3}
\end{table}
\vspace{-5mm}
\section{Conclusion}
\vspace{-1mm}
This paper proposes a flexible precision scaling accelerator with efficient method of weight combination, supporting 2$\sim$8-bit operations of weights and activations. We first design an efficient weight decomposed method based on two weight loading modes to address the hardware utilization issue under low-precision operation and achieve flexible scaling of weight precision. Then, We designed the architecture with bit-serial MAC operation and configurable shifter and adder logic. Moreover, we design the CSA tree for both signed and unsigned number summation. Experiment results show that proposed accelerator has better peak energy efficiency performance compared to the previous work.



\newpage
\bibliographystyle{IEEEtran} % 这里加载IEEEtran.bst
\bibliography{IEEEabrv,reference}

\end{document}

