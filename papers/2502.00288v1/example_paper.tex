%%%%%%%% ICML 2025 EXAMPLE LATEX SUBMISSION FILE %%%%%%%%%%%%%%%%%

\documentclass{article}

% Recommended, but optional, packages for figures and better typesetting:
\usepackage{microtype}
\usepackage{graphicx}
% \usepackage{subfigure}
\usepackage{subcaption}
\usepackage{booktabs} % for professional tables

% hyperref makes hyperlinks in the resulting PDF.
% If your build breaks (sometimes temporarily if a hyperlink spans a page)
% please comment out the following usepackage line and replace
% \usepackage{icml2025} with \usepackage[nohyperref]{icml2025} above.
\usepackage{hyperref}


% Attempt to make hyperref and algorithmic work together better:
\newcommand{\theHalgorithm}{\arabic{algorithm}}

% Use the following line for the initial blind version submitted for review:
% \usepackage{icml2025}

% If accepted, instead use the following line for the camera-ready submission:
\usepackage[accepted]{icml2025}

% For theorems and such
\usepackage{amsmath}
\usepackage{amssymb}
\usepackage{mathtools}
\usepackage{amsthm}

% if you use cleveref..
\usepackage[capitalize,noabbrev]{cleveref}

%%%%%%%%%%%%%%%%%%%%%%%%%%%%%%%%
% THEOREMS
%%%%%%%%%%%%%%%%%%%%%%%%%%%%%%%%
\theoremstyle{plain}
\newtheorem{theorem}{Theorem}[section]
\newtheorem{proposition}[theorem]{Proposition}
\newtheorem{lemma}[theorem]{Lemma}
\newtheorem{corollary}[theorem]{Corollary}
\theoremstyle{definition}
\newtheorem{definition}[theorem]{Definition}
\newtheorem{assumption}[theorem]{Assumption}
\theoremstyle{remark}
\newtheorem{remark}[theorem]{Remark}

% Todonotes is useful during development; simply uncomment the next line
%    and comment out the line below the next line to turn off comments
%\usepackage[disable,textsize=tiny]{todonotes}
\usepackage[textsize=tiny]{todonotes}


% The \icmltitle you define below is probably too long as a header.
% Therefore, a short form for the running title is supplied here:
\icmltitlerunning{Learning from Suboptimal Data in Continuous Control via Auto-Regressive Soft Q-Network}

\begin{document}

\twocolumn[
\icmltitle{Learning from Suboptimal Data in Continuous Control via \\Auto-Regressive Soft Q-Network}

% It is OKAY to include author information, even for blind
% submissions: the style file will automatically remove it for you
% unless you've provided the [accepted] option to the icml2025
% package.

% List of affiliations: The first argument should be a (short)
% identifier you will use later to specify author affiliations
% Academic affiliations should list Department, University, City, Region, Country
% Industry affiliations should list Company, City, Region, Country

% You can specify symbols, otherwise they are numbered in order.
% Ideally, you should not use this facility. Affiliations will be numbered
% in order of appearance and this is the preferred way.

% \icmlsetsymbol{equal}{*}
\icmlsetsymbol{advising}{†}

% hand made
% \newcommand{\icmlEqualAdvising}{\textsuperscript{†}Equal advising }
% \icmlsetsymbol{advising}{†}

\begin{icmlauthorlist}
\icmlauthor{Jijia Liu}{yyy}
\icmlauthor{Feng Gao}{yyy}
\icmlauthor{Qingmin Liao}{yyy}
\icmlauthor{Chao Yu}{yyy}
\icmlauthor{Yu Wang}{yyy}
%\icmlauthor{}{sch}
%\icmlauthor{}{sch}
\end{icmlauthorlist}

\icmlaffiliation{yyy}{Tsinghua University, Beijing, China}

\icmlcorrespondingauthor{Chao Yu}{zoeyuchao@gmail.com}
\icmlcorrespondingauthor{Yu Wang}{yu-wang@tsinghua.edu.cn}

% You may provide any keywords that you
% find helpful for describing your paper; these are used to populate
% the "keywords" metadata in the PDF but will not be shown in the document
\icmlkeywords{Reinforcement Learning}

\vskip 0.3in
]

% this must go after the closing bracket ] following \twocolumn[ ...

% This command actually creates the footnote in the first column
% listing the affiliations and the copyright notice.
% The command takes one argument, which is text to display at the start of the footnote.
% The \icmlEqualContribution command is standard text for equal contribution.
% Remove it (just {}) if you do not need this facility.

\printAffiliationsAndNotice{}  % leave blank if no need to mention equal contribution
% \printAffiliationsAndNotice{\textsuperscript{†}Equal advising}
% \printAffiliationsAndNotice{\icmlEqualContribution} % otherwise use the standard text.

\newcommand{\ljj}[1]{\textcolor{red}{#1}}
\newcommand{\gf}[1]{\textcolor{blue}{#1}}
\newcommand{\yc}[1]{\textcolor{purple}{#1}}
\newcommand{\arsq}{\textit{Auto-Regressive Soft Q-learning }}

% \newtheorem{definition}{Definition}


\begin{abstract}

In this work, we tackle the challenge of disambiguating queries in retrieval-augmented generation (RAG) to diverse yet answerable interpretations.
State-of-the-arts follow a Diversify-then-Verify (DtV) pipeline, where diverse interpretations are generated by an LLM,
later used as search queries to retrieve supporting passages.
Such a process
may introduce noise in either interpretations or retrieval,
particularly in enterprise settings, where LLMs---trained on static data---may struggle with domain-specific disambiguations.
Thus, a post-hoc verification phase is introduced to prune noises.
Our distinction is \textbf{to unify diversification with verification} by incorporating feedback from retriever and generator early on.
This joint approach improves both efficiency and robustness by reducing reliance on multiple retrieval and inference steps, which are susceptible to cascading errors.
We validate the efficiency and effectiveness of our method, \ourslong (\ours), on the widely adopted ASQA benchmark
to achieve diverse yet verifiable interpretations.
Empirical results show that \ours improves grounding-aware $\textrm{F}_1$ score by an average of 23\% over the strongest baseline across different backbone LLMs.
\end{abstract}

\section{Introduction}
\label{sec:intro}
% Image editing methods in diffusion models depend on user-defined control directions - users can unlock their creativity using these methods by specifying the desired manipulation through prompts~\cite{gandikota2023concept}, reference images~\cite{ruiz2022dreambooth, kumari2022customdiffusion, gal2022image, chen2024trainingfreeregionalpromptingdiffusion}, or attribute vectors~\cite{parmar2023zero,hertz2022prompt}. In this work, we ask a fundamentally different question: \emph{Can we automatically discover the underlying visual structure of a concept within diffusion model's knowledge?} %Rather than requiring user-specified controls, we aim to decompose the model's internal knowledge into meaningful directions.

% This question touches on a fundamental limitation in how we interact with diffusion models. Current control methods ~\cite{zhang2023addingconditionalcontroltexttoimage, gandikota2023concept, ye2023ipadaptertextcompatibleimage,ye2023ipadaptertextcompatibleimage, hertz2024stylealignedimagegeneration, li2023photomaker, shi2024instantbooth, chen2024trainingfreeregionalpromptingdiffusion} require users to specify their desired manipulations in advance, limiting interactive creativity. This contrasts with natural human artistic workflows, where creators dynamically explore creative ideas while jointly refining them toward meaningful artistic outcomes~\cite{hoffmann2016modeling}. This synergy between specification and exploration is not new to generative models. Early GAN architectures naturally developed disentangled latent spaces that enabled continuous\cite{harkonen2020ganspace,radford2015unsupervised, wu2021stylespace, shen2020interfacegan}, compositional control over generated images. Users could explore these spaces to discover interesting variations that would be difficult to describe in words~\cite{wu2021stylespace}, then combine them to achieve their creative goals~\cite{grabe2022towards}. 


% While diffusion models have largely superseded GANs in conditional image synthesis~\cite{dhariwal2021diffusion},  their underlying structure remains less understood. Diffusion models achieve remarkable diversity through high-dimensional latents, unlike GANs' compact latent spaces.  With a single prompt, diffusion models can generate radically different variations through different random initializations of input noise. We ask - Is it possible to discover interpretable structure within this vast space of variations?

Text-to-image diffusion models are capable of generating remarkable visual variations from a single prompt through different random initializations. However, this vast creative potential remains largely opaque to users---while we can generate diverse images, we lack understanding of the underlying structure of these variations. This presents a fundamental challenge: how can we discover and expose the latent visual capabilities encoded within these models?

\let\thefootnote\relax \footnote{$^{*}$Correspondence to \texttt{gandikota.ro@northeastern.edu}}

The challenge touches on a key limitation in how we interact with diffusion models today. Current control methods require users to explicitly specify their desired edits in advance through prompts~\cite{gandikota2023concept}, reference images~\cite{zhang2023addingconditionalcontroltexttoimage, chen2024trainingfreeregionalpromptingdiffusion, ruiz2022dreambooth,kumari2022customdiffusion, Ryu_lora, hu2021lora}, or attribute vectors~\cite{ye2023ipadaptertextcompatibleimage, hertz2024stylealignedimagegeneration, li2023photomaker, shi2024instantbooth,parmar2023zero,hertz2022prompt}. That contrasts sharply with natural human creative workflows, where artists dynamically explore creative ideas and jointly refine them toward meaningful artistic outcomes~\cite{hoffmann2016modeling}. The need for pre-specified controls creates a barrier between users and the full creative potential of these models.

Interestingly, earlier generative models like GANs~\cite{gans,karras2019style,brock2018large} naturally developed more interpretable internal structures. Their compact latent spaces often exhibited emergent disentanglement~\cite{harkonen2020ganspace,radford2015unsupervised, wu2021stylespace, shen2020interfacegan}, enabling continuous and compositional control over generated images. Users could explore these spaces to discover interesting variations that would be difficult to describe in words~\cite{wu2021stylespace}, then combine them to achieve their creative goals~\cite{grabe2022towards}.

Diffusion models have largely superseded GANs in conditional image synthesis~\cite{dhariwal2021diffusion}, achieving greater diversity through much higher-dimensional latents. And yet an understanding of the underlying structure of these larger latent spaces has remained elusive. In this work, we ask a fundamental question: \emph{Can we automatically discover the visual structure within a diffusion model's knowledge of a concept?} Rather than requiring user-specified controls, we aim to decompose the model's internal representations into expressive directions that users can explore and combine.

To address these needs, we present \textbf{SliderSpace}, a framework that brings systematic explorability to diffusion models. Given just a text prompt, SliderSpace discovers a canonical set of meaningful, diverse, and controllable directions within the model's knowledge of that concept. Each direction is implemented as a low-rank adapter~\cite{hu2021lora} that can be scaled and composed with others, allowing users to explore and smoothly combine different aspects of variation, as shown in Figure~\ref{fig:intro}.

We ground SliderSpace discovery in three key requirements for meaningful decomposition of a diffusion model's visual manifold: 
\begin{enumerate}
    \item \textbf{Unsupervised Discovery:} The decomposition process should emerge from the intrinsic structure of the model's learned representation, rather than being guided by predefined attributes. This ensures we capture the true topology of the model's knowledge space rather than projecting our assumptions onto it.
    
    \item \textbf{Semantic Orthogonality:} Each discovered control must represent a distinct semantic direction. This is enforced in a semantic feature space, like CLIP, where every slider has an orthogonal effect in embeddings. This prevents discovering multiple controls that create similar semantic effects, making the system more efficient and easier.
    
    \item \textbf{Distribution Consistency:} Directions must induce consistent transformations across both random seeds and prompt variations. 
\end{enumerate}

These requirements naturally lead to our proposed framework, which we formalize in Section~\ref{sec:method}. As we show in our experiments, SliderSpace is architecture-agnostic, working with both conventional U-Net based models like Stable Diffusion~\cite{rombach2022high, rombach2022sd20, podell2023sdxl, turbo, dmd} and recent transformer-based architectures like Flux~\cite{flux}.

We demonstrate the expressiveness of SliderSpace through three applications: First, we show how SliderSpace can decompose high-level concepts into diverse and expressive components, revealing the natural axes of variation in the model's understanding. Second, we explore artistic style variation, where SliderSpace discovers directions that match or exceed the diversity of manually curated artist lists while being judged more useful by human evaluators. Finally, we show how SliderSpace can help reverse the mode collapse commonly observed in distilled diffusion models, restoring diversity while maintaining generation speed.

Beyond providing practical creative control, SliderSpace opens new avenues for understanding and utilizing the latent capabilities of diffusion models. By mapping these models' visual potential into intuitive, composable directions, we take a step toward making their creative possibilities more accessible and interpretable to users.

% Image editing methods in diffusion models unlock the creativity of users. In this work we ask an alternate question: \emph{Can we organize and expose what of the diffusion model is already capable of?}.
% Existing methods for controlling image generation typically require users to manually specify edit directions for desired changes. This process is time-consuming, requires technical expertise, and limits the spontaneity of the creative process. For instance, if a user wants to adjust the smile of a generated person, they must explicitly request this edit, often through imprecise prompt engineering or model fine-tuning. This approach of predefined controls or manual specifications restricts users from fully exploring the latent capabilities of the model. There may be interesting stylistic variations or attributes that the model can generate, but users have no easy way to discover or utilize these.

% Natural visual disentanglement was an emergent property in the latent space of Generative Adversarial Models (GANs) \cite{harkonen2020ganspace,radford2015unsupervised, wu2021stylespace, shen2020interfacegan}. In particular, it has been observed that StyleGAN~\cite{karras2019style} stylespace neurons offer detailed control over many meaningful aspects of images that would be difficult to describe in words~\cite{wu2021stylespace}. However, diffusion models do not share such a compact latent space~\cite{park2023unsupervised}; and efforts to uncover such a space in the semantic embeddings of the text conditioning have met with limited success \nik{Nick - is there a specific citation you were thinking about?}.

% In this work we introduce \textbf{SliderSpace}, which takes a step towards uncovering an analogous low dimensional representation of diffusion models' visual breadth; in essence treating the diffusion model as many generators sharing parameters, where a particular generator is defined by a specific prompt. For a given prompt we sample many random seeds (and optionally prompt expansions using an LLM), generate the corresponding images, and apply an off the shelf feature extractor (in this work CLIP, but our method can be applied to any differentiable feature extractor). We use PCA to analyze these features, and for each of the leading $k$ principal components we train a LoRA \cite{} which causes the diffusion model to produces images which increase the feature magnitude along that component when passed back through the same feature extractor. This leads to a 'Slider' for each principal component, because each LoRA can be scaled and applied to the original diffusion model, continuously varying those visual features in the generated results (as measured, in our case, by CLIP).

% There are many other works that enhance the controllability of diffusion models. One common approach is enabling users to add spatial constraints to a generation either manually, or via a reference image \cite{zhang2023addingconditionalcontroltexttoimage, chen2024trainingfreeregionalpromptingdiffusion}, a second is leveraging more abstract embeddings (e.g. identity, style) extracted from a reference image \cite{ye2023ipadaptertextcompatibleimage, hertz2024stylealignedimagegeneration, li2023photomaker, shi2024instantbooth}, a third is finetuning a foundation model to better generate a concept important to the user \cite{ruiz2022dreambooth, kumari2022customdiffusion, Ryu_lora, hu2021lora}, and a fourth (most relevant to this work) is finding low-rank adaptors of the model based on a prompt or small training set which can be scaled to provide continous control over one aspect of generated image (e.g. night vs day, basic vs luxury, etc.) \cite{gandikota2023concept}. SliderSpace is complementary to all of these methods and offers something distinct. All of the other methods we are aware require the user (and / or model designer) to know in advance what type of control they want. In contrast SliderSpace assists users in discovering and controlling hidden capabilities present in the diffusion model's distribution of possible generations.

%We propose that truly intuitive creative control in a text-to-image model should meet three key criteria: \emph{discoverability}, \emph{intuitiveness}, and \emph{specificity}. The model should reveal controllable attributes that may not be immediately obvious, offer controls that are easy to understand and manipulate, and ensure each control affects a distinct attribute of the generated image.

% We demonstrate the utility and power of SliderSpace using three applications built on top of SDXL-DMD \cite{dmd}, because its fast generation speed lends itself well to the continuous control offered by SliderSpace.

% First, we study concept decomposition (Section \ref{sec:concept_exp}), where we learn sliders for a specific concept (e.g. 'monster', 'waterfall', 'car'). Through quantitative metrics of diversity and text alignment we demonstrate that the learned sliders dramatically boost the diversity of generations when randomly applied without harming text alignment; we also ask humans to qualitatively judge these results in a user study where they find the SliderSpace results to be more 'Diverse', 'Useful', and 'Creative' than our baselines.

% Second, we attempt to compare the automatic discoveries of SliderSpace to a large scale manual study of artistic styles (Section \ref{sec:art_exp}), open-sourced by ParrotZone \cite{parrotzone}. In this study SDXL was prompted with over 4300 artist names,  and based on visual inspection the cases of successful stylistic mimicry recorded. Quantitatively SliderSpace more closely matches the distribution of artistic variation discovered by ParrotZone than other baselines, and in our user studies was judged to be significantly more 'Diverse' and 'Useful' than the baselines. To our surprise humans even judged SliderSpace results to be slightly more 'Diverse' than the results generated by the manually discovered artist names of \cite{parrotzone}.

% Third, we attempt to use SliderSpace to reverse the mode collapse commonly observed in distilled few-step diffusion models relative to the original teacher model (Section \ref{sec:diverse_exp}). We quantitatively demonstrate that applying SliderSpace to SDXL-DMD leads to more closely matching the distribution of images by the original teacher, SDXL.

%Through extensive experiments on various state-of-the-art text-to-image models, we demonstrate that SliderSpace significantly enhances user control and creative expression in AI-assisted image generation tasks. Our method enables a range of applications, including concept decomposition and control, diversity improvement in generated images, customization dissection and edits, and the exploration of artistic styles inherent in the model.

% SliderSpace goes beyond providing a practical tool for enhanced creative control. By mapping the visual potential of diffusion models it can open new avenues for generative creativity and deepens our understanding of each model's hidden potential.
\section{Related Work}

\paragraph{LLMs for Agent tasks.}

Our research is related to deploying large language models (LLMs) as agents for decision-making tasks in interactive environments~\citep{liu2023agentbench,zhou2023webarena,shridhar2020alfred,toyama2021androidenv}. Earlier works, such as~\citep{yao2023webshopscalablerealworldweb}, fine-tuned models like BERT~\citep{devlin2019bertpretrainingdeepbidirectional} for decision-making in simplified environments, such as online shopping or mobile phone manipulation. With the advent of large language models~\citep{brown2020languagemodelsfewshotlearners,openai2024gpt4technicalreport}, it became feasible to perform decision-making tasks through zero-shot or few-shot in-context learning. To better assess the capabilities of LLMs as agents, several models have been developed~\citep{deng2024mind2web,xiong2024watch,hong2023cogagent,yan2023gpt}. Most approaches~\citep{zheng2024seeact,deng2024mind2web} provide the agent with observation and action history, and the language model predicts the next action via in-context learning. Additionally, some methods~\citep{zhang2023building,li2023camel,song2024trial} attempt to distill trajectories from state-of-the-art language models to train more effective policy models. In contrast, our paper introduces a novel framework that automatically learns a reward model from LLM agent navigation, using it to guide the agents in making more effective plans.

\textbf{LLM Planning.} Our paper is also related to planning with large language models. Early researchers~\citep{brown2020languagemodelsfewshotlearners} often prompted large language models to directly perform agent tasks. Later, \citet{yao2022react} proposed ReAct, which combined LLMs for action prediction with chain-of-thought prompting~\citep{wei2022chain}. Several other works~\citep{yao2023treethoughtsdeliberateproblem,hao2023reasoning,zhao2023large,qiao2024agentplanningworldknowledge} have focused on enhancing multi-step reasoning capabilities by integrating LLMs with tree search methods. Our model differs from these previous studies in several significant ways. First, rather than solely focusing on text generation tasks, our pipeline addresses multi-step action planning tasks in interactive environments, where we must consider not only historical input but also multimodal feedback from the environment. Additionally, our pipeline involves automatic learning of the reward model from the environment without relying on human-annotated data, whereas previous works rely on prompting-based frameworks that require large commercial LLMs like GPT-4~\citep{openai2024gpt4technicalreport} to learn action prediction. Furthermore, \Model supports a variety of planning algorithms beyond tree search.

\textbf{Learning from AI Feedback.} In contrast to prior work on LLM planning, our approach also draws on recent advances in learning from AI feedback~\citep{bai2022constitutional,lee2023rlaif,yuan2024self,sharma2024critical,pan2024autonomous,koh2024tree}. These studies initially prompt state-of-the-art large language models to generate text responses that adhere to predefined principles and then potentially fine-tune the LLMs with reinforcement learning. Like previous studies, we also prompt large language models to generate synthetic data. However, unlike them, we focus not on fine-tuning a better generative model but on developing a classification model that evaluates how well action trajectories fulfill the intended instructions. This approach is simpler, requires no reliance on state-of-the-art LLMs, and is more efficient. We also demonstrate that our learned reward model can integrate with various LLMs and planning algorithms, consistently improving their performance.

\textbf{Inference-Time Scaling.} ~\citet{snell2024scaling} validates the efficacy of inference-time scaling for language models. Based on inference-time scaling, various methods have been proposed, such as random sampling~\citep{wang2022self} and tree-search methods~\citep{hao2023reasoning, zhang2024accessing, guan2025rstar}. Concurrently, several works have also leveraged inference-time scaling to improve the performance of agentic tasks. ~\citet{koh2024tree} adopts a training-free approach, employing MCTS to enhance policy model performance during inference and prompting the LLM to return the reward. ~\citet{gu2024your} introduces a novel speculative reasoning approach to bypass irreversible actions by leveraging LLMs or VLMs. It also employs tree search to improve performance and prompts an LLM to output rewards. ~\citet{yu2024exact} proposes Reflective-MCTS to perform tree search and fine-tune the GPT model, leading to improvements in ~\citet{koh2024visualwebarena}. ~\citet{putta2024agent} also utilizes MCTS to enhance performance on web-based tasks such as ~\citet{yao2023webshopscalablerealworldweb} and real-world booking environments. ~\cite{lin2025qlass} utilizes the stepwise reward to give effective intermediate guidance across different agentic tasks. Our work differs from previous efforts in two key aspects: (1) Broader Application Domain. Unlike prior studies that primarily focus on tasks from a single domain, our method demonstrates strong generalizability across web agents, mathematical reasoning, and scientific discovery domains, further proving its effectiveness. (2) Flexible and Effective Reward Modeling. Instead of simply prompting an LLM as a reward model, we finetune a small scale VLM~\citep{lin2023vila} to evaluate input trajectories. %Our reward scores range continuously between 0 and 1, in contrast to existing methods that rely on discrete scoring (e.g., 0 and 1, or 0, 0.5, and 1) through direct LLM prompting.

% Concurrently, several works have also leveraged inference-time scaling to improve the performance of agentic tasks. ~\citet{pan2024autonomous} demonstrates that LLMs and VLMs, such as the GPT series, can function as evaluators or reward models to provide guidance for fine-tuning or reflection, thereby enhancing digital agents. This lays the groundwork for subsequent studies that directly prompt LLMs as reward models. ~\citet{koh2024tree} adopts a training-free approach, employing MCTS to enhance policy model performance during inference. However, it is limited to web environments~\citep{koh2024visualwebarena}. Moreover, its value function relies on prompting an LLM, which is less effective than our proposed method. We validate our approach through ablation studies, demonstrating that our fine-tuned reward model is more effective. ~\citet{gu2024your} introduces a novel speculative reasoning approach to bypass irreversible actions, such as purchasing a product, by leveraging LLMs or VLMs. It also employs tree search to improve performance, but it remains restricted to the web domain~\citep{koh2024visualwebarena, deng2024mind2web}. Additionally, it lacks reward modeling and instead prompts an LLM to output rewards. ~\citet{yu2024exact} proposes Reflective-MCTS to perform tree search and fine-tune the GPT model, leading to improvements in ~\citep{koh2024visualwebarena}. However, this work focuses solely on a single web agent task, and its reward modeling is derived from multi-agent debate, differing from our more effective and efficient reward modeling approach. ~\citet{putta2024agent} also utilizes MCTS to enhance performance, but it is limited to web-based tasks such as ~\citep{yao2023webshopscalablerealworldweb} and real-world booking environments.
% \input{sec/3_toy}
\section{Preliminaries}
\subsection{Problem Formulation}

In this paper, we consider the standard RL setting with the addition of a pre-collected dataset \(\mathcal{D}\) for continuous control. 
The problem can be represented as MDP, defined by the tuple \((\mathcal{S}, \mathcal{A}, \gamma, p, r, d_0)\). 
Here, \(\mathcal{S}\) is the continuous state space, \(\mathcal{A}\) is the continuous action space, \(\gamma \in (0,1)\) is the discount factor, \(p(s' \mid s,a)\) is the transition dynamics, \(r(s,a)\) is the reward function, and \(d_0(s)\) is the distribution of the initial state.
In addition to interacting with the environment online, we assume access to a pre-collected dataset \(\mathcal{D}=\{(s_i,a_i,r_i,s_i')\}\), which can substantially reduce sample complexity and provide broader state-action coverage.

\subsection{Soft Q Learning}

To improve policy exploration, maximum entropy RL enhances the reward by adding an entropy term \cite{MaxEntIRL, SoftQ, SAC}, so the optimal policy seeks to maximize entropy at every state it visits. The objective is defined as
\begin{equation}
    J(\pi) =  \sum_{t=0}^{T} \mathbb{E}_{(\mathbf{s}_t, \mathbf{a}_t) \sim \rho_\pi} \left[ r(\mathbf{s}_t, \mathbf{a}_t) + \alpha \mathcal{H}(\pi(\cdot | \mathbf{s}_t)) \right]
    \label{eq:pre-soft-obj}
\end{equation}
where $\mathcal{H}$ is entropy, $T$ is the episode length and $\rho_\pi$ is the trajectory distribution induced by policy $\pi$.
The temperature parameter $\alpha$ dictates how much importance is placed on the entropy term in comparison to the reward. 
Let the soft Q-function defined as:
\begin{equation}
\begin{split}
    & Q^*_{\text{soft}}(\mathbf{s}_t, \mathbf{a}_t) = r_t + \\
    & \mathbb{E}_{(\mathbf{s}_{t+1}, \dots) \sim \rho_\pi} 
    \left[ \sum_{l=1}^\infty \gamma^l \left( r_{t+l} + \alpha H\left(\pi^*(\cdot | \mathbf{s}_{t+l})\right) \right) \right]
\end{split}
\end{equation}
Then the optimal policy for Eq.~(\ref{eq:pre-soft-obj}) is given by
\begin{equation}
\label{eq:pre-soft-policy}
    \pi^*(\mathbf{a}_t | \mathbf{s}_t) = \exp \left( \frac{1}{\alpha} \left( Q^*_{\text{soft}}(\mathbf{s}_t, \mathbf{a}_t) - V^*_{\text{soft}}(\mathbf{s}_t) \right) \right)
\end{equation}
\begin{equation}
\label{eq:pre-soft-v}
    V^*_{\text{soft}}(\mathbf{s}_t) = \alpha \log \int_{\mathcal{A}} \exp \left( \frac{1}{\alpha} Q^*_{\text{soft}}(\mathbf{s}_t, \mathbf{a}') \right) d\mathbf{a}'
\end{equation}
Similar to the standard Q-function and value function, the Q-function can be connected to the value function at a future state using a soft Bellman equation.
\begin{equation}
\label{eq:pre-soft-update}
    Q^*_{\text{soft}}(\mathbf{s}_t, \mathbf{a}_t) = r_t + \gamma \mathbb{E}_{\mathbf{s}_{t+1} \sim p(\mathbf{s}_t, \mathbf{a}_t)} \left[ V^*_{\text{soft}}(\mathbf{s}_{t+1}) \right]
\end{equation}
The proof can be found in \cite{MaxEntIRL, SoftQ}.




\section{Method}

In this section, we begin by discussing the process of discretizing multi-dimensional actions in a coarse-to-fine manner. 
Building on this, we extend the soft Q-learning theory with a focus on the dimensional soft advantage. 
Subsequently, we introduce our \arsq (ARSQ) algorithm, which is overviewed in Fig.~\ref{fig:med-main}.

\begin{figure*}[ht]
    \centering
    \includegraphics[width=\linewidth]{fig/5_med-main.png}
    \vspace{-2em}
    \caption{The ARSQ algorithm. The action space is discretized using a coarse-to-fine approach. By predicting dimensional soft advantages, ARSQ generates actions in an auto-regressive manner within a single decision-making step.}
    \label{fig:med-main}
\end{figure*}

\subsection{Coarse-to-fine Action Discretization}
\label{sec:med-coarse}

To apply Q-learning \cite{NatureDQN} in a continuous domain, a straightforward approach is to discretize the action space \cite{DiscretePPO, CQN}. 
For a continuous action of $d$ dimensions $\mathbf{a}_c = (a_c^1, a_c^2, \ldots, a_c^d) \in \mathbb{R}^D$, the discretized action $\mathbf{a} = (a^1, a^2, \ldots, a^D)$ can be represented by

\begin{equation}
    a^d = \arg \max_i | a_c^d - b_i |
\end{equation}

where $\mathbf{b} = (b_1, \ldots, b_B)$ are the centers of $B$ discretization intervals, or bins, which typically provide a uniform separation of the given action space.
However, obtaining a finer separation of the action space necessitates a greater number of bins, thereby increasing the computational load when assessing the Q function for each discrete action bin.

To address this issue, we can apply a coarse-to-fine action discretization approach \cite{CQN}, similar to the method used in \cite{HD-CNN} for computer vision, as illustrated in Fig.~\ref{fig:med-main}. 
With $L$ levels and $B$ uniform separation bins at each level, the discrete action for dimension $d$ at level $l$ is expressed as:

\begin{equation}
    a^{d, l} = \lfloor \frac{a^{d} - \sum_{i=1}^{l-1} B^{L-i} a^{d, i}}{B^{L-l}} \rfloor
\end{equation}

Here, $\lfloor \cdot \rfloor$ represents the floor function.

During inference, the policy generates discrete actions progressively through each level $(\mathbf{a}^{\langle \cdot \rangle, 1}, \mathbf{a}^{\langle \cdot \rangle, 2}, \cdots, \mathbf{a}^{\langle \cdot \rangle, L})$. 
These are then combined to produce the final discrete action.

\subsection{Dimensional Soft Advantage for Policy Representation}
\label{sec:med-dsa}

Building on action discretization, we initially extend soft Q-learning to discrete spaces. 
The soft value function is expressed as
\begin{equation}
    V^*_{\text{soft}}(\mathbf{s}) = \alpha \log \sum_{\mathbf{a}' \in \mathcal{A}} \exp \left( \frac{1}{\alpha} Q^*_{\text{soft}}(\mathbf{s}, \mathbf{a}') \right)
\end{equation}
And we omit the subscript $t$ for $\mathbf{s}_t$ and $\mathbf{a}_t$
To further streamline the expression of the policy, we define the soft advantage.

\begin{definition}[Soft Advantage]
    The soft advantage of $\mathbf{a}$ at $\mathbf{s}$ is given by
    \begin{equation}
        A^*(\mathbf{s}, \mathbf{a}) = Q^*_{\text{soft}}(\mathbf{s}, \mathbf{a}) - V^*_{\text{soft}}(\mathbf{s})
    \end{equation}
\end{definition}

Similar to the advantage in policy gradient-based RL algorithms, the soft advantage assesses how much taking action $\mathbf{a}$ at state $\mathbf{s}$ is beneficial. 
Thus, the optimal policy in Eq.~(\ref{eq:pre-soft-policy}) can be expressed as
\begin{equation}
\label{eq:med-sa-pi}
    \pi^*(\mathbf{a} | \mathbf{s}) = \exp \left( \frac{1}{\alpha} A^*(\mathbf{s}, \mathbf{a}) \right)
\end{equation}
Considering the multi-dimensional action space, it still remains necessary to use a neural network to output $B^{L \times D}$ Q values in the final layer, as per the DQN \cite{NatureDQN}.

However, outputting such a large number of Q values imposes a significant computational burden on the neural network.
Inspired by auto-regression \cite{GPT3}, we address this problem by generating the Q function for a state-action pair in an auto-regressive manner. 

For clarity, we treats discrete action discussed in Sec.~\ref{sec:med-coarse} in one level. 
The multi-level coarse-to-fine discrete action can be considered as additional action dimensions, without compromising generalization.
We first define the dimensional soft advantage for policy representation.

\begin{definition}[Dimensional Soft Advantage]
    The dimensional soft advantage of the action $a^d$ at state $\mathbf{s}$, considering the previous dimensional actions $\mathbf{a}^{-d} = (a^1, \cdots, a^{d-1})$, is expressed by
    \begin{equation}
    \label{eq:med-dsa}
        \pi (a^d | \mathbf{s}, \mathbf{a}^{-d}) = 
        \frac
        {exp \left( \frac{1}{\alpha} A^d(\mathbf{s}, \mathbf{a}^{-d}, a^d)) \right)}
        {Z(\mathbf{s}, \mathbf{a}^{-d})}
    \end{equation}
\end{definition}

where $Z^d(\mathbf{s}, \mathbf{a}^{-d})$ represents the partition function.

However, the dimensional soft advantage is not related to other equations and remains intractable. 
To address this, we propose the following theorem to establish a connection between the dimensional soft advantage and the soft advantage.

\begin{theorem}
\label{the:med-sum}
    If dimensional soft advantage $m^d(\mathbf{s}, \mathbf{a}^{-d}, a^d))$ satisfies
    \begin{equation}
    \label{eq:med-dsa-cond}
        log \sum_{a^{d'}} \exp{\left( \frac{1}{\alpha} A^d(\mathbf{s}, \mathbf{a}^{-d}, a^{d'}) \right)} = 0
    \end{equation}
    then the soft advantage can then be expressed as the summation of the dimensional soft advantages
    \begin{equation}
        \sum_{d=1}^D A^d(\mathbf{s}, \mathbf{a}^{-d}, a^d) = A(\mathbf{s}, \mathbf{a})
    \end{equation}
\end{theorem}
\begin{proof}
    See Appendix~\ref{sec:app-proof}.
\end{proof}

Through Eq.~(\ref{eq:med-dsa}) and Theorem~\ref{the:med-sum}, we extend soft Q-learning to a auto-regressive policy along action dimension.

Since we do not introduce additional elements in policy optimization, the Q-iteration follows the same update rule as soft Q-learning.
Based on Eq.~(\ref{eq:pre-soft-update}), we have
\begin{equation}
\label{eq:med-dsa-update}
    V_{\text{soft}}(\mathbf{s}_t) + A(\mathbf{s}_t, \mathbf{a}_t) 
    \leftarrow r_t + \gamma \mathbb{E}_{\mathbf{s}_{t+1} \sim p(s)} \left[ V_{\text{soft}}(\mathbf{s}_{t+1}) \right]
\end{equation}
The maximum entropy policy described in Eq.~(\ref{eq:pre-soft-policy}) can be obtained by repeatedly applying Eq.~(\ref{eq:med-dsa-update}) until it converges.

\subsection{Auto-Regressive Soft Q-learning}
\label{sec:med-alg}
\begin{algorithm}[tb]
    \caption{Auto-Regressive Soft Q Algorithm (ARSQ) }
    \label{pse:med-alg}
\begin{algorithmic}
    \STATE Initialize $\theta_{1, 2}, \phi_{1, 2}$ for $A^{\theta_i}$ and $V_{\text{soft}}^{\phi_i}$
    \STATE Assign target parameters $\overline{\theta}_i,  \overline{\phi}_i \leftarrow \theta_i, \phi_i$.
    \STATE Offline dataset $\mathcal{D}$, replay buffer $\mathcal{R} \leftarrow \mathcal{D}$.
    \FOR{each epoch}
        \FOR{each environment step}
            \STATE select $\mathbf{a}_t$ with $A_{\theta_1}$ and $A_{\theta_2}$ (\ref{eq:med-sa-pi}, \ref{eq:med-alg-cons})
            \STATE $\mathbf{s}_{t+1} \sim p(\mathbf{s}_{t+1} | \mathbf{s}_t, \mathbf{a}_t)$
            \STATE $\mathcal{R} \leftarrow \mathcal{R} \cup \{ \mathbf{s}_t, \mathbf{a}_t, r_t, \mathbf{s}_{t+1} \}$
        \ENDFOR
        \FOR{each gradient step}
            \STATE Sample mini-batch $b_D$, $b_R$ from $\mathcal{D}$, $\mathcal{R}$
            \STATE Calculate $\mathcal{L}_D = \mathcal{L}_{RL} + \beta \mathcal{L}_{BC}$ with $b_D$ (\ref{eq:med-alg-bc}, \ref{eq:med-alg-rl})
            \STATE Calculate $\mathcal{L}_R = \mathcal{L}_{RL}$ with $b_R$ (\ref{eq:med-alg-rl})
            \STATE Update $m_{\theta_i}$ according to $\hat{\nabla}_{\theta_i}(\mathcal{L}_D + \mathcal{L}_R)$
            \STATE Update $V_{s, \phi_i}$ according to $\hat{\nabla}_{\phi_i}(\mathcal{L}_D + \mathcal{L}_R)$
            % \STATE Update target networks $\overline{\theta}_i \leftarrow \rho \overline{\theta}_i + (1 - \rho) \theta_i$
            % \STATE Update target networks $\overline{\phi}_i \leftarrow \rho \overline{\phi}_i + (1 - \rho) \phi_i$
            \STATE Update target networks $\overline{\theta}_i \leftarrow \rho \overline{\theta}_i + (1 - \rho) \theta_i$ and $\overline{\phi}_i \leftarrow \rho \overline{\phi}_i + (1 - \rho) \phi_i$.
        \ENDFOR
    \ENDFOR
\end{algorithmic}
\end{algorithm}


Building on the theory outlined in Sec.~\ref{sec:med-dsa}, we introduce the \arsq (ARSQ) algorithm. 
The pseudo code for the ARSQ algorithm is presented in Algorithm~\ref{pse:med-alg}. 
We will discuss the various design choices of ARSQ.

\paragraph{Behavior cloning objective.}
To leverage offline demonstration data during online training, we introduce an additional behavior cloning loss term. 
Following previous works \cite{DQfD,CQN}, we encourage actions present in the offline dataset to be preferred over other actions. 
Specifically, we define the loss as
\begin{equation}
\label{eq:med-alg-bc}
\begin{aligned}
    \mathcal{L}_{BC}^{d} = \sum_{a^{d}} \max ( 
    & A^{d, \theta_i}(\mathbf{s}, \mathbf{a}^{-d}_e, a^{d}) \\ 
    & - A^{d, \theta_i}(\mathbf{s}, \mathbf{a}^{-d}_e, a_e^{d}), C_m )
\end{aligned}
\end{equation}
where $\mathbf{a}_e$ denotes the expert action observed in the offline dataset, and $C_m$ is a hyper-parameter controlling the margin. 
This objective encourages the soft advantages of expert actions to be at least $C_m$ higher than those of other actions.

\paragraph{Policy representation.}
As discussed in Sec. \ref{sec:med-dsa}, ARSQ predicts dimensional soft advantages, which function as both components of the Q function and policy representation. 
The network architecture is illustrated in Fig. \ref{fig:med-nn}.
In practical design, the soft value $V_{\text{soft}}$ and the dimensional soft advantage $A^d$ are predicted using two separate neural networks. 
The advantage prediction network estimates the dimensional soft advantage for each action dimension, based on the partially generated action from previous dimensions, creating an auto-regressive sequence. 
In practical design, we use a globally-shared MLP in the advantage network, with separate heads to predict the dimensional soft advantages.

\begin{figure*}[ht]
    \centering
    \includegraphics[width=0.9\linewidth]{fig/5_med-nn.png}
    \vspace{-2em}
    \caption{Network architecture of ARSQ. The soft value $V_{\text{soft}}$ and the dimensional soft advantage $A^d$ are predicted by two separate networks. The advantage network utilizes a shared backbone, and advantage constraints are applied to its output.}
    \label{fig:med-nn}
\end{figure*}

Another challenge is applying the constraint of the dimensional soft advantage as per Eq.~(\ref{eq:med-dsa-cond}). 
Here, we enforce a hard constraint by normalizing each output head through log-sum-exp subtraction, ensuring consistency across outputs.
\begin{equation}
\label{eq:med-alg-cons}
\begin{aligned}
    A^d(\mathbf{s}_t, \mathbf{a}^{-d} &, a^d) 
    = u^d(\mathbf{s}_t, \mathbf{a}^{-d}, a^d) \\
    & - log \sum_{a^{d'}} \exp \left(
    \frac{1}{\alpha} u^d(\mathbf{s}_t, \mathbf{a}^{-d}, a^{d'})
    \right)
\end{aligned}
\end{equation}
where $u^d$ is the output of the $d$-th output head.

Furthermore, to stabilize training and address the over-estimation problem \cite{fujimoto2018addressing, DoubleQ}, we implemented a double Q network alongside a target network in our practical application.
Therefore, the optimization objective in Eq.~(\ref{eq:med-dsa-update}) is modified to
\begin{equation}
    \mathbf{y}_t =  \gamma \mathbb{E}_{\mathbf{s}_{t+1} \sim p(s)} \left[ \text{min} \left( V^{\overline{\phi}_1}_{\text{soft}1}(\mathbf{s}_{t+1}), V^{\overline{\phi}_2}_{\text{soft}2} (\mathbf{s}_{t+1}) \right) \right]
\end{equation}
\begin{equation}
\label{eq:med-alg-rl}
    \mathcal{L}_{RL} = \frac{1}{2} \left( V^{\phi_i}_{\text{soft}}(\mathbf{s}_t) + A^{\theta_i}(\mathbf{s}_t, \mathbf{a}_t) - \mathbf{y}_t  \right)
\end{equation}
where $V^{\overline{\phi}_i}_{\text{soft}}$ represents the soft value predicted by the target network.

\paragraph{Auto-regressive conditioning.} 
\label{sec:med-alg-ar}
In Sec.~\ref{sec:med-dsa}, we explained the process of handling discrete action in one coarse-to-fine level.
With multi-level coarse-to-fine action discretization, the auto-regressive conditioning encompasses two aspects. 
\emph{Dimensional conditioning} refers to generating actions for each dimension in an auto-regressive sequence, while \emph{coarse-to-fine conditioning} involves generating actions for each dimension from coarse to fine. 
In practice, we implement coarse-to-fine conditioning prior to dimensional conditioning. 
Specifically, dimensional conditioning serves as the inner conditioning, while coarse-to-fine conditioning acts as the outer conditioning across levels. 
We explore swapping the order of conditioning in Sec.~\ref{sec:exp-abl}, and the results indicate that the current design better captures interdependencies between action dimensions.




\section{Experiments}\label{sec:experiments}


\begin{figure*}[!ht]
\centering
\includegraphics[width=0.32\textwidth]{eps_figs/eps_eps_diff_m_Random.pdf}
\includegraphics[width=0.32\textwidth]{eps_figs/eps_eps_diff_m_False_Negative.pdf}
\includegraphics[width=0.32\textwidth]{eps_figs/eps_eps_diff_m_False_Positive.pdf}
\caption{
Three kinds of error rates with different bit-array lengths $m$. We fix the number of inserted elements $|A|=10^5$, the number of hash functions $k = 3$, and $\delta = 0.01$ in $(\epsilon, \delta)$-DP. 
In the figure, $\log$ denotes $\log_2$. 
{\bf Left:} Total error denotes the case when we randomly choose queries from the universe $[n]$; 
{\bf Middle:} False negative denotes the case when we randomly choose queries from the set $S$, which represents the set of elements inserted into the DP Bloom filter; 
{\bf Right:} False positive denotes the case when we randomly choose queries from the set $\ov{S} = [n] \backslash S$.  
As $m$ increases, the total error rate and false positive error rate decrease accordingly, while false negative error rate remains constant. 
As $\epsilon$ approaches $0$, the DP Bloom filter gets closer to random guessing. In this case, the false positive error rate converges to $\frac{1}{2^k}$, and the false negative error rate converges to $1 - \frac{1}{2^k}$. This is consistent with our result in Lemma~\ref{lem:random_guess}
Our \textsc{DPBloomFilter} achieves practical utility when $\epsilon$ is small(e.g. $\epsilon < 10$).
}
\label{fig:eps_diff_m}
\end{figure*}


\begin{figure*}[!ht]
\centering
\includegraphics[width=0.32\textwidth]{eps_figs/eps_eps_diff_na_Random.pdf}
\includegraphics[width=0.32\textwidth]{eps_figs/eps_eps_diff_na_False_Negative.pdf}
\includegraphics[width=0.32\textwidth]{eps_figs/eps_eps_diff_na_False_Positive.pdf}
\caption{
Three kinds of error rates with different numbers of inserted elements $|A|$. We fix the length of bit-array $m=2^{19}$, the number of hash functions $k = 3$, and $\delta = 0.01$ in $(\epsilon, \delta)$-DP.
As $|A|$ increases, the Total Error Rate and false positive error rate increase accordingly, while the false negative error rate remains constant. 
}
\label{fig:eps_diff_na}
\end{figure*}

\begin{figure*}[!ht]
\centering
\includegraphics[width=0.32\textwidth]{eps_figs/eps_eps_diff_k_Random.pdf}
\includegraphics[width=0.32\textwidth]{eps_figs/eps_eps_diff_k_False_Negative.pdf}
\includegraphics[width=0.32\textwidth]{eps_figs/eps_eps_diff_k_False_Positive.pdf}
\caption{
Three kinds of error rates with different numbers of hash function $k$.  
We fix the length of bit-array $m=2^{19}$, the number of inserted elements $|A| = 10^5$, and $\delta = 0.01$ in $(\epsilon, \delta)$-DP.
As $k$ increases, the Total Error Rate and false positive error rate decrease accordingly, while the false negative error rate increases accordingly. 
}
\label{fig:eps_diff_k}
\end{figure*}

In this section, we introduce the simulation experiments conducted on the DPBloomfilter.
In Section~\ref{sec:exp:setup}, we introduce the basic setup of our experiments and restate basic definitions of three kinds of error.
In Section~\ref{sec:exp:main_result}, we discuss the results of our experiments, which align with our theoretical analysis. 

\subsection{Experiments Setup and Basic Notations} \label{sec:exp:setup}


Recall that we have the following notations. 
Let $m$ denote the length of the bit array in the DPBloomfilter.
Let $|A|$ denote the number of elements inserted into the DPBloomfilter. 
Let $k$ denote the number of hash functions used in the DPBloomfilter.
Let $\epsilon, \delta$ denote the differential privacy parameters of the DPBloomfilter. 
Let $N$ denotes the $1 - \delta$ quantile of $W$ (see Definition~\ref{def:W}), and the close-form of the distribution of $W$ is shown in Lemma~\ref{lem:distribution_of_W}. 
Let $\epsilon_0 = \epsilon / N$. By Theorem~\ref{thm:query_privacy:informal}, we choose $\epsilon_0$ in this way can guarantee to $(\epsilon, \delta)$-DP in the whole algorithm. 
Unless specified, we adopt $m = 2^{19}, |A| = 10^5, k=8, n = 2^{63} \approx 10^{19}$ in the following experiments. 
We choose this $n$ because this $n$ is the biggest integer that can be represented on our server.

Recall that $[n]$ denotes the universe. 
Let $S$ denote the elements inserted into the DPBloomfilter. 
Let $\ov{S} = [n] \backslash S$ denote the elements not inserted into the DPBloomfilter. Let $\wt{z} \in \{ 0, 1 \}$ denote the answer output by DPBloomfilter. 

We report three kinds of error rates in our experiments. They are the following: 
(1) {\bf total error}, where we randomly choose queries from the universe $[n]$ and report the error rate of our DPBloomfilter;
(2) {\bf false positive error}, where we random choose queries from $\ov{S}$. When the DPBloomfilter outputs $\wt{z} = 1$, this will cause a false positive error; 
(3) {\bf false negative error}, where we random choose queries from $S$. When the DPBloomfilter outputs $\wt{z} = 0$, this will cause a false negative error. 

\subsection{Experiment Results} \label{sec:exp:main_result}

In this section, we conduct experiments based on the setting mentioned in the previous section. Specifically, we run simulation experiments on different $m$, $|A|$, and $k$ to demonstrate the utility of our algorithm under differential privacy guarantees. 

In Figure~\ref{fig:eps_diff_m}, we conduct experiments on different $m$, whereas $m$ increases, the total error rate and false positive error rate decrease accordingly, while the false negative error rate remains constant. 

In Figure~\ref{fig:eps_diff_na}, we also conduct experiments on different $|A|$, whereas $|A|$ increases, the total error rate and false positive error rate increase accordingly. At the same time, the false negative error rate remains constant.
This phenomenon is consistent with our theoretical analysis of the utility of DPBloomfilter (Theorem~\ref{thm:dpbloom_true_accuracy:informal}). Recall that we have $\alpha = \Pr[z=0]$, denoting the probability of an arbitrary query $q \notin A$. 
Since $|A|$ increases, $\alpha$ decreases, the utility guarantee in Theorem~\ref{thm:dpbloom_true_accuracy:informal}, which is consistent with higher error rate in our experiment results. 


In Figure~\ref{fig:eps_diff_k}, we conduct experiments on different $k$ as well, whereas $k$ increases, the total error rate, and false positive error rate decrease, while the false negative error rate increases accordingly. 

Note that in Figure~\ref{fig:eps_diff_m}, Figure~\ref{fig:eps_diff_na}, and Figure~\ref{fig:eps_diff_k}, as $\epsilon$ approaches $0$, the DPBloomfilter gets closer to random guessing. In this case, the false positive error rate converges to $\frac{1}{2^k}$, and the false negative error rate converges to $1 - \frac{1}{2^k}$. This is consistent with our result in Lemma~\ref{lem:random_guess}. 
Also, as $\epsilon$ increases, the three types of error rates in the Bloom filter with differential privacy (DP) approach the error rates observed when DP is not applied. This is consistent with the intuition that when $\epsilon$ increases, there is less privacy. Therefore, the performance approaches the performance of a Bloom filter without any privacy guarantees. 


\section{Conclusion and future directions} \label{sec:conclusion}

In this paper we proposed a nested MLMC framework that offers important computational savings by performing most calculations in low precision and exploiting approximate random normal variables for the low precision path calculations. The low precision calculations could be performed in fixed precision on an FPGA for greater efficiency, and we suggested a procedure to optimise the bit-widths of every variable at each Monte Carlo level. This is an important improvement over previous mixed precision MLMC frameworks which held the lower precision fixed \cite{Rounding_error_oliver} or defined uniform bit-width at every level heuristically \cite{brugger2014mixed}. Our numerical results suggest that for the first levels our procedure reduces the cost at these levels by a factor 5 or 7. Hence the overall savings are significant since most paths are calculated on the first levels. Our approach would be even more efficient for the Milstein scheme because its higher order strong convergence leads to a greater proportion of the computational costs being on the coarsest levels.

The next stage of the research project will be to implement the RNG methods and the nested framework on FPGAs to determine the hardware requirements and confirm the extent of the computational savings. It would also be good to compare the performance benefits to using half-precision floating point arithmetic on GPUs or CPUs for the low-accuracy computations.





% Acknowledgements should only appear in the accepted version.
% \section*{Acknowledgements}
% \textbf{Do not} include acknowledgements in the initial version of
% the paper submitted for blind review.

\section*{Impact Statement}

This paper presents work whose goal is to advance the field of Machine Learning. There are many potential societal consequences of our work, none which we feel must be specifically highlighted here.


% In the unusual situation where you want a paper to appear in the
% references without citing it in the main text, use \nocite
\nocite{langley00}

\bibliography{example_paper}
\bibliographystyle{icml2025}


%%%%%%%%%%%%%%%%%%%%%%%%%%%%%%%%%%%%%%%%%%%%%%%%%%%%%%%%%%%%%%%%%%%%%%%%%%%%%%%
%%%%%%%%%%%%%%%%%%%%%%%%%%%%%%%%%%%%%%%%%%%%%%%%%%%%%%%%%%%%%%%%%%%%%%%%%%%%%%%
% APPENDIX
%%%%%%%%%%%%%%%%%%%%%%%%%%%%%%%%%%%%%%%%%%%%%%%%%%%%%%%%%%%%%%%%%%%%%%%%%%%%%%%
%%%%%%%%%%%%%%%%%%%%%%%%%%%%%%%%%%%%%%%%%%%%%%%%%%%%%%%%%%%%%%%%%%%%%%%%%%%%%%%
\newpage
\appendix
\onecolumn

\section{Proof of Theorem~\ref{the:med-sum}}
\label{sec:app-proof}

First, we express the policy using conditional probability, and then replace it with Eq.~(\ref{eq:med-dsa}).
\begin{equation}
\begin{aligned}
    \pi (\mathbf{a} | \mathbf{s})
    &= \prod_{d=1}^D \pi (a^d | \mathbf{s}, \mathbf{a}^{-d} ) \\
    &= \prod_{d=1}^D \frac
    {exp \left( \frac{1}{\alpha} A^d(\mathbf{s}, \mathbf{a}^{-d}, a^d) \right)}
    {Z(\mathbf{s}, \mathbf{a}^{-d})} \\
    &= \frac
    {\prod_{d=1}^D exp \left(\frac{1}{\alpha} A^d(\mathbf{s}, \mathbf{a}^{-d}, a^d) \right) }
    {\prod_{d=1}^D Z^d(\mathbf{s}, \mathbf{a}^{-d})} \\
    &= \frac
    {exp \left( \frac{1}{\alpha} \sum_{d=1}^D A^d(\mathbf{s}, \mathbf{a}^{-d}, a^d)\right)}
    {\prod_{d=1}^D Z^d(\mathbf{s}, \mathbf{a}^{-d})} \\
\end{aligned}
\end{equation}
We can then apply Eq.~(\ref{eq:med-dsa-cond}), resulting in
\begin{equation}
    \pi (\mathbf{a} | \mathbf{s})
    = exp \left( \frac{1}{\alpha} \sum_{d=1}^D A^d(\mathbf{s}, \mathbf{a}^{-d}, a^d)\right)
\end{equation}
Recall that the policy $\pi (\mathbf{a} | \mathbf{s})$ can be represented using the soft advantage as shown in Eq.~(\ref{eq:med-sa-pi}). Therefore, we have
\begin{equation}
    \sum_{d=1}^D A^d(\mathbf{s}, \mathbf{a}^{-d}, a^d)
     = A(\mathbf{s}, \mathbf{a})
\end{equation}

\section{Implementation Details}

\subsection{Action Selection}
As illustrated in Algorithm~\ref{pse:med-alg}, the action selection process receives inputs from $A_{\theta_1}$ and $A_{\theta_2}$ and produces $\mathbf{a}_t$. 
Eq.~(\ref{eq:med-sa-pi}) and Eq.~(\ref{eq:med-alg-cons}) describe the action selection process utilizing a single soft advantage network. 
To leverage the benefits of a double network, we employ two advantage networks to generate more precise actions. 
This process is detailed in Algorithm~\ref{pse:med-alg-act}.

\begin{algorithm}[h]
    \caption{ARSQ Action Selection with Double Q Network}
    \label{pse:med-alg-act}
\begin{algorithmic}
    \STATE \textbf{Input:} parameter $\theta_{1, 2}$ for $A^{\theta_i}$, state $\mathbf{s}_t$
    \STATE \textbf{Output:} action $\mathbf{a}_t$
    \STATE Initialize output action $\mathbf{a}_t = \emptyset$
    \FOR{each action dimension $d$}
        \STATE Compute $A^{d, \theta_i}(\mathbf{s}_t, \mathbf{a}_t, a^d )$ for each $a^d$ (\ref{eq:med-alg-cons})
        \STATE Compute $A^{d}(a^d ) = \min_i A^{d, \theta_i}(\mathbf{s}_t, \mathbf{a}_t, a^d )$
        \STATE Compute $\tilde{\pi}^d(a^d)= \text{exp} \left( \frac{1}{\alpha} A^{d}(a^d ) \right)$ (\ref{eq:med-sa-pi})
        \STATE Normalize $\tilde{\pi}^d$ by $\pi^d(a^d)=\frac{\tilde{\pi}^d (a^d)}{\sum_{a^{d'}} \tilde{\pi}^d(a^{d'})} $
        \STATE Sample discrete action at dimension $d$ with $\pi^d(a^d)$
        \STATE Append action $\mathbf{a}_t = \mathbf{a}_t \cup \{ a^d \}$
    \ENDFOR
\end{algorithmic}
\end{algorithm}

\subsection{Variant of Behavior Cloning Objective}
As discussed in Sec.~\ref{sec:med-alg}, we incorporate an behavior cloning objective to effectively utilize offline demonstration data during online training, as defined in Eq.~(\ref{eq:med-alg-bc}). 

Following prior works \cite{CQL}, we also employ a variant of this objective, expressed as:  
\begin{equation}
    \mathcal{L}_{BC-v}^{d} = \max \left( 
    \text{log} \sum_{a^d \neq a^d_e}
    \text{exp} \left( A^{d, \theta_i}(\mathbf{s}, \mathbf{a}^{-d}_e, a^{d}) \right) - A^{d, \theta_i}(\mathbf{s}, \mathbf{a}^{-d}_e, a^{d}_e), C_m \right)
\end{equation}

where $a^d_e$ is the expert action and $C_m$ is a predefined margin constant.

We observe that this variant objective achieves better performance in scenarios where action modes are concentrated, such as in the \textit{medium} and \textit{medium-expert} series of datasets in D4RL.
Consequently, we adopt this variant objective when working with such datasets.

\subsection{Network Architecture}

In RLBench tasks, observations consist of a combination of RGB images and low-dimensional states. 
To compute the dimensional soft advantage for a given dimension, we first input the RGB images and low-dimensional states into a Convolutional Neural Network (CNN) \cite{CNN} encoder and a Multi-Layer Perceptron (MLP) \cite{MLP} encoder, respectively, to extract feature representations. 
These representations are then used to predict the soft value. 
Concurrently, the feature representations are combined with actions from previous dimensions and coarse-to-fine levels to create auto-regressive conditioning. 
An MLP-based shared backbone and output head are then utilized to determine the dimensional soft advantage for the given dimension.

In D4RL tasks, observations consist solely of low-dimensional states, and feature representations are derived directly from these states.


\subsection{Hyper-parameters}
\begin{table}[ht]
\centering
\begin{tabular}{lll}
    \toprule
    Hyper-parameter & D4RL & RLBench \\
    \midrule
    Image resolution & / & $84 \times 84 \times 3$ \\
    Image augmentation & / & RandomShift \\
    Frame stack & 1 & 8 \\
    \midrule
    CNN - Encoder & / & Conv (c=[32, 64, 128, 256], s=2, p=1) \\
    Backbone & Linear (512, 512, 512) & Linear (512, 512, 512, bias=False) \\
    Output Head Layers & 1 & 1 \\
    Activation & Tanh & SiLU \& LayerNorm \\
    \midrule
    Coarse-to-fine Levels & 2 & 3 \\
    Coarse-to-fine Bins & 7 & 5 \\
    \midrule
    Batch Size & 512 & 512 \\
    Optimizer & Adam & AdamW (weight decay = 0.1) \\
    Learning Rate & 3e-4 & 5e-5 \\
    Temperature Coefficient $\alpha$ & 0.01 & 0.001 \\
    Target Critic Update Ratio ($\tau$) & 0.005 & 0.02 \\
    BC Margin $C_m$ & -1 & -0.01 \\
    Action Roll-out Network & Current & Target \\
    \bottomrule
\end{tabular}
\caption{Typical hyper-parameters of ARSQ in D4RL and RLBench.}
\label{tab:alg-hyperparam}
\end{table}

The hyperparameters of ARSQ are presented in Table~\ref{tab:alg-hyperparam}. 
We provide the typical hyperparameters for ARSQ in D4RL (\textit{hopper-medium}) and RLBench (\textit{Open Oven}). 
In RLBench, ARSQ employs RandomShift \cite{DrQv2} for image augmentation. 
Additionally, ARSQ utilizes SiLU \cite{SiLU} and LayerNorm \cite{LayerNorm} as activation functions in RLBench.


\section{Experiment Setup}

\subsection{Motivating Example Setup}
\label{sec:app-example}

As introduced in Sec.~\ref{sec:intro} and illustrated in Fig.~\ref{fig:intro-case}, we consider a motivating example to demonstrate the impact of Q decomposition on policy training. 
The dataset is depicted in Fig.~\ref{fig:intro-toy-env}, with each point to be a data point in the dataset.
The color of the data points indicates the reward of the data point.
To illustrate the Q function of value-based RL algorithms, we first discretize the action space with $2$ bins in each action dimension. 

\begin{itemize}
    \item Q function given by independent action decomposition is an example of DecQN \cite{DecQN}, as well as in CQN \cite{CQN}, which features just a single coarse-to-fine level. 
    In this setting, we employ separate tabular Q functions, $Q(s, a_1)$ and $Q(s, a_2)$, for action dimension 1 and action dimension 2. 
    The Q function is learned by gradient descent.
    \item For the Q function obtained through auto-regressive action decomposition, we employ both tabular soft advantage functions, $A^1(s, a_1)$ and $A^2(s, a_1, a_2)$ for action dimension 1 and action dimension 2, and a tabular soft value function $V_{\text{soft}}(s)$. 
    The Q value reported in Fig.~\ref{fig:intro-toy-ar} is a sum of the soft value and the dimensional soft advantage of the corresponding dimensions, i.e., $Q(s, a_1, a_2) = V_{\text{soft}}(s) + A^1(s, a_1) + A^2(s, a_1, a_2)$. The soft advantage functions and the soft value function are simultaneously learned through gradient descent.
\end{itemize}



\subsection{Environment and Dataset}
\label{sec:app-env}

\paragraph{D4RL Gym Environment}
D4RL \cite{D4RL} provides datasets for various tasks to evaluate the performance of reinforcement learning. 
In this context, we use 3 Gym Locomotion tasks and datasets from D4RL to assess the performance of ARSQ and other baselines. 
These tasks are illustrated in Fig.~\ref{fig:app-env-d4rl}. The agent's observations include its states, such as the angle and velocity of each rotor. 
The agent's actions consist of torques applied between the robot's links, constrained within the range of $(-1, 1)$. 
The reward is dense, offering incentives for task completion and survival, while penalizing excessive energy-consuming actions.

\begin{figure}[h]
    \centering
    \includegraphics[width=0.8\linewidth]{fig/app-env-d4rl.png}
    \caption{D4RL Gym tasks used in experiment.}
    \label{fig:app-env-d4rl}
\end{figure}

\paragraph{D4RL Dataset}
In D4RL, we use the \textit{medium-replay}, \textit{medium}, and \textit{medium-expert} datasets for tasks involving \textit{half-cheetah}, \textit{hopper}, and \textit{walker2d}. 
In Section~\ref{sec:exp-d4rl}, to examine the impact of dataset quality, we rank trajectories based on episode returns within these nine datasets. 
Specifically, we compute the total reward for each data chunk within each dataset. 
We then rank these data chunks and select the top, middle, and bottom $30\%$ accordingly.
This is akin to rank trajectories but is easier to handle.

To better demonstrate the suboptimal nature of the datasets, we have plotted a histogram of the data chunk rewards, as shown in Fig.~\ref{fig:app-env-data-d4rl}.

\begin{figure}[h]
    \centering
    \includegraphics[width=0.8\linewidth]{fig/app-env-data-d4rl.png}
    \caption{Histogram of reward in D4RL datasets.}
    \label{fig:app-env-data-d4rl}
\end{figure}


\paragraph{RLBench Environment}
RLBench \cite{RLBench} serves as a benchmark and learning environment for robot control. 
We have selected 20 tasks from RLBench and present results for 6 of them in Sec.~\ref{sec:exp}. 
An illustration of the environment can be seen in Fig.~\ref{fig:app-env-rlb}. 
The input consists of RGB images with a resolution of 84 × 84, captured from four camera angles: front, wrist, left-shoulder, and right-shoulder, along with a history of the past seven observations. 
The output specifies the change in joint angles at each time step, utilizing the delta JointPosition mode provided by RLBench. 
In our experiments, we use a binary sparse reward system (0 or 1), which is awarded only at the final timestamp of an episode to indicate task success.

\begin{figure}[h]
    \centering
    \includegraphics[width=0.8\linewidth]{fig/app-env-rlb.png}
    \caption{Example of RLBench tasks used in experiment.}
    \label{fig:app-env-rlb}
\end{figure}



\subsection{Baselines and Evaluation Details}
\label{sec:app-baseline}

\paragraph{Main results baselines.}
As mentioned in Sec.~\ref{sec:exp-d4rl}, within D4RL, we utilize the implementation from \cite{CQN} and modify its CNN-based encoder to an MLP-based encoder as the CQN baseline. 
The BC baseline originates from CQN but operates with the RL learning objective turned off and without any online environment interaction.

In RLBench, we use the baselines and results of CQN, DrQ-v2+, DrQ-v2, ACT, and CBC as reported in the original CQN paper \cite{CQN}.

\paragraph{Ablation study baselines.}
As mentioned in Sec.~\ref{sec:exp-abl}, we utilize the \textit{Separate} and \textit{Level Shared} backbone baselines for an ablation study to explore the effectiveness of the shared backbone in the advantage network. 
The network architectures of these two baselines are illustrated in Fig.~\ref{fig:app-baseline-abl-nn-s} and Fig.~\ref{fig:app-baseline-abl-nn-ls}.

\begin{figure}[h]
    \centering
    \includegraphics[width=0.7\linewidth]{fig/5_med-nn-abl-s.png}
    \caption{Network architecture of \textit{Separate} backbone baseline in ablation study.}
    \label{fig:app-baseline-abl-nn-s}
\end{figure}

\begin{figure}[h]
    \centering
    \includegraphics[width=0.7\linewidth]{fig/5_med-nn-abl-ls.png}
    \caption{Network architecture of \textit{Level Shared} backbone baseline in ablation study.}
    \label{fig:app-baseline-abl-nn-ls}
\end{figure}



\section{Additional Results}
\label{sec:app-exp}

\paragraph{Sensitivity of temperature coefficient $\alpha$.}
Our methods are derived from Soft Q-learning, which aims to achieve a maximum-entropy policy.
The temperature coefficient $\alpha$ in Eq.~(\ref{eq:pre-soft-obj}) affects the balance between maximizing policy entropy and the reward from the environment. 
We conducted experiments to examine how varying $\alpha$ impacts policy learning. 

\begin{figure}[h]
    \centering
    \begin{subfigure}[t]{0.30\textwidth}
        \centering
        \includegraphics[width=\textwidth]{fig/6_exp-abl-alpha-d4rl.png}
        \label{fig:exp-abl-alpha-d4rl}
    \end{subfigure}
    \hspace{0.01\textwidth}
    \begin{subfigure}[t]{0.30\textwidth}
        \centering
        \includegraphics[width=\textwidth]{fig/6_exp-abl-alpha-rlb.png}
        \label{fig:exp-abl-alpha-rlb}
    \end{subfigure}
    \caption{Sensitivity of temperature coefficient $\alpha$ in D4RL and RLBench.}
    \label{fig:exp-abl-alpha}
\end{figure}

As shown in Fig.~\ref{fig:exp-abl-alpha}, a very high $\alpha$ results in reduced performance and unstable training, whereas a very low $\alpha$ also hampers policy improvement by restricting exploration.

\paragraph{D4RL results per task for different demonstration quality.}
In Sec.~\ref{sec:exp-d4rl}, we present the D4RL results, averaged over all 9 datasets, based on varying demonstration quality. 
The results for each task are illustrated in Fig.~\ref{fig:exp-d4rl-quality}. 
ARSQ consistently outperforms the CQN and BC baselines in nearly every task, demonstrating its ability to maintain stable performance across datasets of varying quality.

\begin{figure}[h]
    \centering
    \includegraphics[width=0.9\linewidth]{fig/6_exp-d4rl-quality.png}
    \caption{D4RL results per task on different demonstration quality.}
    \label{fig:exp-d4rl-quality}
\end{figure}

\paragraph{RLBench results in all 20 tasks.}
In Sec.~\ref{sec:exp-rlb}, we present results for six selected tasks from RLBench. 
The complete results for all 20 tasks are displayed in Fig.~\ref{fig:exp-rlb-task-all}. 
These results indicate that ARSQ performs comparably or better across these tasks, showcasing its ability to learn effectively even when the data collected online is not optimal.

\begin{figure}[h]
    \centering
    \includegraphics[width=0.95\linewidth]{fig/6_exp-rlb-full.png}
    \caption{RLBench results in all 20 tasks.}
    \label{fig:exp-rlb-task-all}
\end{figure}



\section{Computational Cost Analysis}
As discussed in Sec.~\ref{sec:med-alg}, ARSQ generates actions in each dimension in an auto-regressive manner. 
To analyze the overhead, we conducted experiments on both D4RL (\textit{hopper-medium}) and RLBench (\textit{Open Oven}) tasks. The training and inference times for ARSQ and CQN were evaluated 1,000 times and averaged. 
These experiments were conducted on a single Nvidia RTX 3090 graphics card.

The results are shown in Fig.~\ref{fig:exp-abl-compute}. 
ARSQ exhibits similar training times to CQN, due to the parallel optimization implemented and the batch training nature of the auto-regressive model. 
However, ARSQ experiences higher inference latency compared to CQN. 
We aim to address this issue by grouping the action dimensions and outputting the grouped dimensional actions auto-regressively, a solution we plan to explore in future work.

\begin{table}[h]
    \centering
    \begin{tabular}{c|cc}
        \toprule
         & D4RL & RLBench \\
        \midrule
        ARSQ Inference & 4.1 & 32.1 \\
        ARSQ Training & 12.2 & 290.5  \\
        CQN Inference & 2.6 & 6.9 \\
        CQN Training & 11.6 & 260.5 \\
        \bottomrule
    \end{tabular}
    \caption{Computational time in D4RL and RLBench (ms). }
    \label{fig:exp-abl-compute}
\end{table}

%%%%%%%%%%%%%%%%%%%%%%%%%%%%%%%%%%%%%%%%%%%%%%%%%%%%%%%%%%%%%%%%%%%%%%%%%%%%%%%
%%%%%%%%%%%%%%%%%%%%%%%%%%%%%%%%%%%%%%%%%%%%%%%%%%%%%%%%%%%%%%%%%%%%%%%%%%%%%%%


\end{document}


% This document was modified from the file originally made available by
% Pat Langley and Andrea Danyluk for ICML-2K. This version was created
% by Iain Murray in 2018, and modified by Alexandre Bouchard in
% 2019 and 2021 and by Csaba Szepesvari, Gang Niu and Sivan Sabato in 2022.
% Modified again in 2023 and 2024 by Sivan Sabato and Jonathan Scarlett.
% Previous contributors include Dan Roy, Lise Getoor and Tobias
% Scheffer, which was slightly modified from the 2010 version by
% Thorsten Joachims & Johannes Fuernkranz, slightly modified from the
% 2009 version by Kiri Wagstaff and Sam Roweis's 2008 version, which is
% slightly modified from Prasad Tadepalli's 2007 version which is a
% lightly changed version of the previous year's version by Andrew
% Moore, which was in turn edited from those of Kristian Kersting and
% Codrina Lauth. Alex Smola contributed to the algorithmic style files.
