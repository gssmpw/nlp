%%%%%%%% ICML 2025 EXAMPLE LATEX SUBMISSION FILE %%%%%%%%%%%%%%%%%

\documentclass{article}

% Recommended, but optional, packages for figures and better typesetting:
\usepackage{microtype}
\usepackage{graphicx}
% \usepackage{subfigure}
\usepackage{subcaption}
\usepackage{booktabs} % for professional tables

% hyperref makes hyperlinks in the resulting PDF.
% If your build breaks (sometimes temporarily if a hyperlink spans a page)
% please comment out the following usepackage line and replace
% \usepackage{icml2025} with \usepackage[nohyperref]{icml2025} above.
\usepackage{hyperref}


% Attempt to make hyperref and algorithmic work together better:
\newcommand{\theHalgorithm}{\arabic{algorithm}}

% Use the following line for the initial blind version submitted for review:
% \usepackage{icml2025}

% If accepted, instead use the following line for the camera-ready submission:
\usepackage[accepted]{icml2025}

% For theorems and such
\usepackage{amsmath}
\usepackage{amssymb}
\usepackage{mathtools}
\usepackage{amsthm}

% if you use cleveref..
\usepackage[capitalize,noabbrev]{cleveref}

%%%%%%%%%%%%%%%%%%%%%%%%%%%%%%%%
% THEOREMS
%%%%%%%%%%%%%%%%%%%%%%%%%%%%%%%%
\theoremstyle{plain}
\newtheorem{theorem}{Theorem}[section]
\newtheorem{proposition}[theorem]{Proposition}
\newtheorem{lemma}[theorem]{Lemma}
\newtheorem{corollary}[theorem]{Corollary}
\theoremstyle{definition}
\newtheorem{definition}[theorem]{Definition}
\newtheorem{assumption}[theorem]{Assumption}
\theoremstyle{remark}
\newtheorem{remark}[theorem]{Remark}

% Todonotes is useful during development; simply uncomment the next line
%    and comment out the line below the next line to turn off comments
%\usepackage[disable,textsize=tiny]{todonotes}
\usepackage[textsize=tiny]{todonotes}


% The \icmltitle you define below is probably too long as a header.
% Therefore, a short form for the running title is supplied here:
\icmltitlerunning{Learning from Suboptimal Data in Continuous Control via Auto-Regressive Soft Q-Network}

\begin{document}

\twocolumn[
\icmltitle{Learning from Suboptimal Data in Continuous Control via \\Auto-Regressive Soft Q-Network}

% It is OKAY to include author information, even for blind
% submissions: the style file will automatically remove it for you
% unless you've provided the [accepted] option to the icml2025
% package.

% List of affiliations: The first argument should be a (short)
% identifier you will use later to specify author affiliations
% Academic affiliations should list Department, University, City, Region, Country
% Industry affiliations should list Company, City, Region, Country

% You can specify symbols, otherwise they are numbered in order.
% Ideally, you should not use this facility. Affiliations will be numbered
% in order of appearance and this is the preferred way.

% \icmlsetsymbol{equal}{*}
\icmlsetsymbol{advising}{†}

% hand made
% \newcommand{\icmlEqualAdvising}{\textsuperscript{†}Equal advising }
% \icmlsetsymbol{advising}{†}

\begin{icmlauthorlist}
\icmlauthor{Jijia Liu}{yyy}
\icmlauthor{Feng Gao}{yyy}
\icmlauthor{Qingmin Liao}{yyy}
\icmlauthor{Chao Yu}{yyy}
\icmlauthor{Yu Wang}{yyy}
%\icmlauthor{}{sch}
%\icmlauthor{}{sch}
\end{icmlauthorlist}

\icmlaffiliation{yyy}{Tsinghua University, Beijing, China}

\icmlcorrespondingauthor{Chao Yu}{zoeyuchao@gmail.com}
\icmlcorrespondingauthor{Yu Wang}{yu-wang@tsinghua.edu.cn}

% You may provide any keywords that you
% find helpful for describing your paper; these are used to populate
% the "keywords" metadata in the PDF but will not be shown in the document
\icmlkeywords{Reinforcement Learning}

\vskip 0.3in
]

% this must go after the closing bracket ] following \twocolumn[ ...

% This command actually creates the footnote in the first column
% listing the affiliations and the copyright notice.
% The command takes one argument, which is text to display at the start of the footnote.
% The \icmlEqualContribution command is standard text for equal contribution.
% Remove it (just {}) if you do not need this facility.

\printAffiliationsAndNotice{}  % leave blank if no need to mention equal contribution
% \printAffiliationsAndNotice{\textsuperscript{†}Equal advising}
% \printAffiliationsAndNotice{\icmlEqualContribution} % otherwise use the standard text.

\newcommand{\ljj}[1]{\textcolor{red}{#1}}
\newcommand{\gf}[1]{\textcolor{blue}{#1}}
\newcommand{\yc}[1]{\textcolor{purple}{#1}}
\newcommand{\arsq}{\textit{Auto-Regressive Soft Q-learning }}

% \newtheorem{definition}{Definition}


\begin{abstract}
Building a virtual cell capable of accurately simulating cellular behaviors in silico has long been a dream in computational biology. We introduce \emph{CellFlow}, an image-generative model that simulates cellular morphology changes induced by chemical and genetic perturbations using flow matching. Unlike prior methods, \emph{CellFlow} models distribution-wise transformations from unperturbed to perturbed cell states, effectively distinguishing actual perturbation effects from experimental artifacts such as batch effects—a major challenge in biological data. Evaluated on chemical (BBBC021), genetic (RxRx1), and combined perturbation (JUMP) datasets, \emph{CellFlow} generates biologically meaningful cell images that faithfully capture perturbation-specific morphological changes, achieving a 35\% improvement in FID scores and a 12\% increase in mode-of-action prediction accuracy over existing methods. Additionally, \emph{CellFlow} enables continuous interpolation between cellular states, providing a potential tool for studying perturbation dynamics. These capabilities mark a significant step toward realizing virtual cell modeling for biomedical research.
\end{abstract}

\section{Introduction}
Backdoor attacks pose a concealed yet profound security risk to machine learning (ML) models, for which the adversaries can inject a stealth backdoor into the model during training, enabling them to illicitly control the model's output upon encountering predefined inputs. These attacks can even occur without the knowledge of developers or end-users, thereby undermining the trust in ML systems. As ML becomes more deeply embedded in critical sectors like finance, healthcare, and autonomous driving \citep{he2016deep, liu2020computing, tournier2019mrtrix3, adjabi2020past}, the potential damage from backdoor attacks grows, underscoring the emergency for developing robust defense mechanisms against backdoor attacks.

To address the threat of backdoor attacks, researchers have developed a variety of strategies \cite{liu2018fine,wu2021adversarial,wang2019neural,zeng2022adversarial,zhu2023neural,Zhu_2023_ICCV, wei2024shared,wei2024d3}, aimed at purifying backdoors within victim models. These methods are designed to integrate with current deployment workflows seamlessly and have demonstrated significant success in mitigating the effects of backdoor triggers \cite{wubackdoorbench, wu2023defenses, wu2024backdoorbench,dunnett2024countering}.  However, most state-of-the-art (SOTA) backdoor purification methods operate under the assumption that a small clean dataset, often referred to as \textbf{auxiliary dataset}, is available for purification. Such an assumption poses practical challenges, especially in scenarios where data is scarce. To tackle this challenge, efforts have been made to reduce the size of the required auxiliary dataset~\cite{chai2022oneshot,li2023reconstructive, Zhu_2023_ICCV} and even explore dataset-free purification techniques~\cite{zheng2022data,hong2023revisiting,lin2024fusing}. Although these approaches offer some improvements, recent evaluations \cite{dunnett2024countering, wu2024backdoorbench} continue to highlight the importance of sufficient auxiliary data for achieving robust defenses against backdoor attacks.

While significant progress has been made in reducing the size of auxiliary datasets, an equally critical yet underexplored question remains: \emph{how does the nature of the auxiliary dataset affect purification effectiveness?} In  real-world  applications, auxiliary datasets can vary widely, encompassing in-distribution data, synthetic data, or external data from different sources. Understanding how each type of auxiliary dataset influences the purification effectiveness is vital for selecting or constructing the most suitable auxiliary dataset and the corresponding technique. For instance, when multiple datasets are available, understanding how different datasets contribute to purification can guide defenders in selecting or crafting the most appropriate dataset. Conversely, when only limited auxiliary data is accessible, knowing which purification technique works best under those constraints is critical. Therefore, there is an urgent need for a thorough investigation into the impact of auxiliary datasets on purification effectiveness to guide defenders in  enhancing the security of ML systems. 

In this paper, we systematically investigate the critical role of auxiliary datasets in backdoor purification, aiming to bridge the gap between idealized and practical purification scenarios.  Specifically, we first construct a diverse set of auxiliary datasets to emulate real-world conditions, as summarized in Table~\ref{overall}. These datasets include in-distribution data, synthetic data, and external data from other sources. Through an evaluation of SOTA backdoor purification methods across these datasets, we uncover several critical insights: \textbf{1)} In-distribution datasets, particularly those carefully filtered from the original training data of the victim model, effectively preserve the model’s utility for its intended tasks but may fall short in eliminating backdoors. \textbf{2)} Incorporating OOD datasets can help the model forget backdoors but also bring the risk of forgetting critical learned knowledge, significantly degrading its overall performance. Building on these findings, we propose Guided Input Calibration (GIC), a novel technique that enhances backdoor purification by adaptively transforming auxiliary data to better align with the victim model’s learned representations. By leveraging the victim model itself to guide this transformation, GIC optimizes the purification process, striking a balance between preserving model utility and mitigating backdoor threats. Extensive experiments demonstrate that GIC significantly improves the effectiveness of backdoor purification across diverse auxiliary datasets, providing a practical and robust defense solution.

Our main contributions are threefold:
\textbf{1) Impact analysis of auxiliary datasets:} We take the \textbf{first step}  in systematically investigating how different types of auxiliary datasets influence backdoor purification effectiveness. Our findings provide novel insights and serve as a foundation for future research on optimizing dataset selection and construction for enhanced backdoor defense.
%
\textbf{2) Compilation and evaluation of diverse auxiliary datasets:}  We have compiled and rigorously evaluated a diverse set of auxiliary datasets using SOTA purification methods, making our datasets and code publicly available to facilitate and support future research on practical backdoor defense strategies.
%
\textbf{3) Introduction of GIC:} We introduce GIC, the \textbf{first} dedicated solution designed to align auxiliary datasets with the model’s learned representations, significantly enhancing backdoor mitigation across various dataset types. Our approach sets a new benchmark for practical and effective backdoor defense.



\section{Related Work}
\label{sec:related-works}
\subsection{Novel View Synthesis}
Novel view synthesis is a foundational task in the computer vision and graphics, which aims to generate unseen views of a scene from a given set of images.
% Many methods have been designed to solve this problem by posing it as 3D geometry based rendering, where point clouds~\cite{point_differentiable,point_nfs}, mesh~\cite{worldsheet,FVS,SVS}, planes~\cite{automatci_photo_pop_up,tour_into_the_picture} and multi-plane images~\cite{MINE,single_view_mpi,stereo_magnification}, \etal
Numerous methods have been developed to address this problem by approaching it as 3D geometry-based rendering, such as using meshes~\cite{worldsheet,FVS,SVS}, MPI~\cite{MINE,single_view_mpi,stereo_magnification}, point clouds~\cite{point_differentiable,point_nfs}, etc.
% planes~\cite{automatci_photo_pop_up,tour_into_the_picture}, 


\begin{figure*}[!t]
    \centering
    \includegraphics[width=1.0\linewidth]{figures/overview-v7.png}
    %\caption{\textbf{Overview.} Given a set of images, our method obtains both camera intrinsics and extrinsics, as well as a 3DGS model. First, we obtain the initial camera parameters, global track points from image correspondences and monodepth with reprojection loss. Then we incorporate the global track information and select Gaussian kernels associated with track points. We jointly optimize the parameters $K$, $T_{cw}$, 3DGS through multi-view geometric consistency $L_{t2d}$, $L_{t3d}$, $L_{scale}$ and photometric consistency $L_1$, $L_{D-SSIM}$.}
    \caption{\textbf{Overview.} Given a set of images, our method obtains both camera intrinsics and extrinsics, as well as a 3DGS model. During the initialization, we extract the global tracks, and initialize camera parameters and Gaussians from image correspondences and monodepth with reprojection loss. We determine Gaussian kernels with recovered 3D track points, and then jointly optimize the parameters $K$, $T_{cw}$, 3DGS through the proposed global track constraints (i.e., $L_{t2d}$, $L_{t3d}$, and $L_{scale}$) and original photometric losses (i.e., $L_1$ and $L_{D-SSIM}$).}
    \label{fig:overview}
\end{figure*}

Recently, Neural Radiance Fields (NeRF)~\cite{2020NeRF} provide a novel solution to this problem by representing scenes as implicit radiance fields using neural networks, achieving photo-realistic rendering quality. Although having some works in improving efficiency~\cite{instant_nerf2022, lin2022enerf}, the time-consuming training and rendering still limit its practicality.
Alternatively, 3D Gaussian Splatting (3DGS)~\cite{3DGS2023} models the scene as explicit Gaussian kernels, with differentiable splatting for rendering. Its improved real-time rendering performance, lower storage and efficiency, quickly attract more attentions.
% Different from NeRF-based methods which need MLPs to model the scene and huge computational cost for rendering, 3DGS has stronger real-time performance, higher storage and computational efficiency, benefits from its explicit representation and gradient backpropagation.

\subsection{Optimizing Camera Poses in NeRFs and 3DGS}
Although NeRF and 3DGS can provide impressive scene representation, these methods all need accurate camera parameters (both intrinsic and extrinsic) as additional inputs, which are mostly obtained by COLMAP~\cite{colmap2016}.
% This strong reliance on COLMAP significantly limits their use in real-world applications, so optimizing the camera parameters during the scene training becomes crucial.
When the prior is inaccurate or unknown, accurately estimating camera parameters and scene representations becomes crucial.

% In early works, only photometric constraints are used for scene training and camera pose estimation. 
% iNeRF~\cite{iNerf2021} optimizes the camera poses based on a pre-trained NeRF model.
% NeRFmm~\cite{wang2021nerfmm} introduce a joint optimization process, which estimates the camera poses and trains NeRF model jointly.
% BARF~\cite{barf2021} and GARF~\cite{2022GARF} provide new positional encoding strategy to handle with the gradient inconsistency issue of positional embedding and yield promising results.
% However, they achieve satisfactory optimization results when only the pose initialization is quite closed to the ground-truth, as the photometric constrains can only improve the quality of camera estimation within a small range.
% Later, more prior information of geometry and correspondence, \ie monocular depth and feature matching, are introduced into joint optimisation to enhance the capability of camera poses estimation.
% SC-NeRF~\cite{SCNeRF2021} minimizes a projected ray distance loss based on correspondence of adjacent frames.
% NoPe-NeRF~\cite{bian2022nopenerf} chooses monocular depth maps as geometric priors, and defines undistorted depth loss and relative pose constraints for joint optimization.
In earlier studies, scene training and camera pose estimation relied solely on photometric constraints. iNeRF~\cite{iNerf2021} refines the camera poses using a pre-trained NeRF model. NeRFmm~\cite{wang2021nerfmm} introduces a joint optimization approach that simultaneously estimates camera poses and trains the NeRF model. BARF~\cite{barf2021} and GARF~\cite{2022GARF} propose a new positional encoding strategy to address the gradient inconsistency issues in positional embedding, achieving promising results. However, these methods only yield satisfactory optimization when the initial pose is very close to the ground truth, as photometric constraints alone can only enhance camera estimation quality within a limited range. Subsequently, 
% additional prior information on geometry and correspondence, such as monocular depth and feature matching, has been incorporated into joint optimization to improve the accuracy of camera pose estimation. 
SC-NeRF~\cite{SCNeRF2021} minimizes a projected ray distance loss based on correspondence between adjacent frames. NoPe-NeRF~\cite{bian2022nopenerf} utilizes monocular depth maps as geometric priors and defines undistorted depth loss and relative pose constraints.

% With regard to 3D Gaussian Splatting, CF-3DGS~\cite{CF-3DGS-2024} also leverages mono-depth information to constrain the optimization of local 3DGS for relative pose estimation and later learn a global 3DGS progressively in a sequential manner.
% InstantSplat~\cite{fan2024instantsplat} focus on sparse view scenes, first use DUSt3R~\cite{dust3r2024cvpr} to generate a set of densely covered and pixel-aligned points for 3D Gaussian initialization, then introduce a parallel grid partitioning strategy in joint optimization to speed up.
% % Jiang et al.~\cite{Jiang_2024sig} proposed to build the scene continuously and progressively, to next unregistered frame, they use registration and adjustment to adjust the previous registered camera poses and align unregistered monocular depths, later refine the joint model by matching detected correspondences in screen-space coordinates.
% \gjh{Jiang et al.~\cite{Jiang_2024sig} also implemented an incremental approach for reconstructing camera poses and scenes. Initially, they perform feature matching between the current image and the image rendered by a differentiable surface renderer. They then construct matching point errors, depth errors, and photometric errors to achieve the registration and adjustment of the current image. Finally, based on the depth map, the pixels of the current image are projected as new 3D Gaussians. However, this method still exhibits limitations when dealing with complex scenes and unordered images.}
% % CG-3DGS~\cite{sun2024correspondenceguidedsfmfree3dgaussian} follows CF-3DGS, first construct a coarse point cloud from mono-depth maps to train a 3DGS model, then progressively estimate camera poses based on this pre-trained model by constraining the correspondences between rendering view and ground-truth.
% \gjh{Similarly, CG-3DGS~\cite{sun2024correspondenceguidedsfmfree3dgaussian} first utilizes monocular depth estimation and the camera parameters from the first frame to initialize a set of 3D Gaussians. It then progressively estimates camera poses based on this pre-trained model by constraining the correspondences between the rendered views and the ground truth.}
% % Free-SurGS~\cite{freesurgs2024} matches the projection flow derived from 3D Gaussians with optical flow to estimate the poses, to compensate for the limitations of photometric loss.
% \gjh{Free-SurGS~\cite{freesurgs2024} introduces the first SfM-free 3DGS approach for surgical scene reconstruction. Due to the challenges posed by weak textures and photometric inconsistencies in surgical scenes, Free-SurGS achieves pose estimation by minimizing the flow loss between the projection flow and the optical flow. Subsequently, it keeps the camera pose fixed and optimizes the scene representation by minimizing the photometric loss, depth loss and flow loss.}
% \gjh{However, most current works assume camera intrinsics are known and primarily focus on optimizing camera poses. Additionally, these methods typically rely on sequentially ordered image inputs and incrementally optimize camera parameters and scene representation. This inevitably leads to drift errors, preventing the achievement of globally consistent results. Our work aims to address these issues.}

Regarding 3D Gaussian Splatting, CF-3DGS~\cite{CF-3DGS-2024} utilizes mono-depth information to refine the optimization of local 3DGS for relative pose estimation and subsequently learns a global 3DGS in a sequential manner. InstantSplat~\cite{fan2024instantsplat} targets sparse view scenes, initially employing DUSt3R~\cite{dust3r2024cvpr} to create a densely covered, pixel-aligned point set for initializing 3D Gaussian models, and then implements a parallel grid partitioning strategy to accelerate joint optimization. Jiang \etal~\cite{Jiang_2024sig} develops an incremental method for reconstructing camera poses and scenes, but it struggles with complex scenes and unordered images. 
% Similarly, CG-3DGS~\cite{sun2024correspondenceguidedsfmfree3dgaussian} progressively estimates camera poses using a pre-trained model by aligning the correspondences between rendered views and actual scenes. Free-SurGS~\cite{freesurgs2024} pioneers an SfM-free 3DGS method for reconstructing surgical scenes, overcoming challenges such as weak textures and photometric inconsistencies by minimizing the discrepancy between projection flow and optical flow.
%\pb{SF-3DGS-HT~\cite{ji2024sfmfree3dgaussiansplatting} introduced VFI into training as additional photometric constraints. They separated the whole scene into several local 3DGS models and then merged them hierarchically, which leads to a significant improvement on simple and dense view scenes.}
HT-3DGS~\cite{ji2024sfmfree3dgaussiansplatting} interpolates frames for training and splits the scene into local clips, using a hierarchical strategy to build 3DGS model. It works well for simple scenes, but fails with dramatic motions due to unstable interpolation and low efficiency.
% {While effective for simple scenes, it struggles with dramatic motion due to unstable view interpolation and suffers from low computational efficiency.}

However, most existing methods generally depend on sequentially ordered image inputs and incrementally optimize camera parameters and 3DGS, which often leads to drift errors and hinders achieving globally consistent results. Our work seeks to overcome these limitations.

% \input{sec/3_toy}
\section{Preliminaries}
\subsection{Problem Formulation}

In this paper, we consider the standard RL setting with the addition of a pre-collected dataset \(\mathcal{D}\) for continuous control. 
The problem can be represented as MDP, defined by the tuple \((\mathcal{S}, \mathcal{A}, \gamma, p, r, d_0)\). 
Here, \(\mathcal{S}\) is the continuous state space, \(\mathcal{A}\) is the continuous action space, \(\gamma \in (0,1)\) is the discount factor, \(p(s' \mid s,a)\) is the transition dynamics, \(r(s,a)\) is the reward function, and \(d_0(s)\) is the distribution of the initial state.
In addition to interacting with the environment online, we assume access to a pre-collected dataset \(\mathcal{D}=\{(s_i,a_i,r_i,s_i')\}\), which can substantially reduce sample complexity and provide broader state-action coverage.

\subsection{Soft Q Learning}

To improve policy exploration, maximum entropy RL enhances the reward by adding an entropy term \cite{MaxEntIRL, SoftQ, SAC}, so the optimal policy seeks to maximize entropy at every state it visits. The objective is defined as
\begin{equation}
    J(\pi) =  \sum_{t=0}^{T} \mathbb{E}_{(\mathbf{s}_t, \mathbf{a}_t) \sim \rho_\pi} \left[ r(\mathbf{s}_t, \mathbf{a}_t) + \alpha \mathcal{H}(\pi(\cdot | \mathbf{s}_t)) \right]
    \label{eq:pre-soft-obj}
\end{equation}
where $\mathcal{H}$ is entropy, $T$ is the episode length and $\rho_\pi$ is the trajectory distribution induced by policy $\pi$.
The temperature parameter $\alpha$ dictates how much importance is placed on the entropy term in comparison to the reward. 
Let the soft Q-function defined as:
\begin{equation}
\begin{split}
    & Q^*_{\text{soft}}(\mathbf{s}_t, \mathbf{a}_t) = r_t + \\
    & \mathbb{E}_{(\mathbf{s}_{t+1}, \dots) \sim \rho_\pi} 
    \left[ \sum_{l=1}^\infty \gamma^l \left( r_{t+l} + \alpha H\left(\pi^*(\cdot | \mathbf{s}_{t+l})\right) \right) \right]
\end{split}
\end{equation}
Then the optimal policy for Eq.~(\ref{eq:pre-soft-obj}) is given by
\begin{equation}
\label{eq:pre-soft-policy}
    \pi^*(\mathbf{a}_t | \mathbf{s}_t) = \exp \left( \frac{1}{\alpha} \left( Q^*_{\text{soft}}(\mathbf{s}_t, \mathbf{a}_t) - V^*_{\text{soft}}(\mathbf{s}_t) \right) \right)
\end{equation}
\begin{equation}
\label{eq:pre-soft-v}
    V^*_{\text{soft}}(\mathbf{s}_t) = \alpha \log \int_{\mathcal{A}} \exp \left( \frac{1}{\alpha} Q^*_{\text{soft}}(\mathbf{s}_t, \mathbf{a}') \right) d\mathbf{a}'
\end{equation}
Similar to the standard Q-function and value function, the Q-function can be connected to the value function at a future state using a soft Bellman equation.
\begin{equation}
\label{eq:pre-soft-update}
    Q^*_{\text{soft}}(\mathbf{s}_t, \mathbf{a}_t) = r_t + \gamma \mathbb{E}_{\mathbf{s}_{t+1} \sim p(\mathbf{s}_t, \mathbf{a}_t)} \left[ V^*_{\text{soft}}(\mathbf{s}_{t+1}) \right]
\end{equation}
The proof can be found in \cite{MaxEntIRL, SoftQ}.




\section{Method}

In this section, we begin by discussing the process of discretizing multi-dimensional actions in a coarse-to-fine manner. 
Building on this, we extend the soft Q-learning theory with a focus on the dimensional soft advantage. 
Subsequently, we introduce our \arsq (ARSQ) algorithm, which is overviewed in Fig.~\ref{fig:med-main}.

\begin{figure*}[ht]
    \centering
    \includegraphics[width=\linewidth]{fig/5_med-main.png}
    \vspace{-2em}
    \caption{The ARSQ algorithm. The action space is discretized using a coarse-to-fine approach. By predicting dimensional soft advantages, ARSQ generates actions in an auto-regressive manner within a single decision-making step.}
    \label{fig:med-main}
\end{figure*}

\subsection{Coarse-to-fine Action Discretization}
\label{sec:med-coarse}

To apply Q-learning \cite{NatureDQN} in a continuous domain, a straightforward approach is to discretize the action space \cite{DiscretePPO, CQN}. 
For a continuous action of $d$ dimensions $\mathbf{a}_c = (a_c^1, a_c^2, \ldots, a_c^d) \in \mathbb{R}^D$, the discretized action $\mathbf{a} = (a^1, a^2, \ldots, a^D)$ can be represented by

\begin{equation}
    a^d = \arg \max_i | a_c^d - b_i |
\end{equation}

where $\mathbf{b} = (b_1, \ldots, b_B)$ are the centers of $B$ discretization intervals, or bins, which typically provide a uniform separation of the given action space.
However, obtaining a finer separation of the action space necessitates a greater number of bins, thereby increasing the computational load when assessing the Q function for each discrete action bin.

To address this issue, we can apply a coarse-to-fine action discretization approach \cite{CQN}, similar to the method used in \cite{HD-CNN} for computer vision, as illustrated in Fig.~\ref{fig:med-main}. 
With $L$ levels and $B$ uniform separation bins at each level, the discrete action for dimension $d$ at level $l$ is expressed as:

\begin{equation}
    a^{d, l} = \lfloor \frac{a^{d} - \sum_{i=1}^{l-1} B^{L-i} a^{d, i}}{B^{L-l}} \rfloor
\end{equation}

Here, $\lfloor \cdot \rfloor$ represents the floor function.

During inference, the policy generates discrete actions progressively through each level $(\mathbf{a}^{\langle \cdot \rangle, 1}, \mathbf{a}^{\langle \cdot \rangle, 2}, \cdots, \mathbf{a}^{\langle \cdot \rangle, L})$. 
These are then combined to produce the final discrete action.

\subsection{Dimensional Soft Advantage for Policy Representation}
\label{sec:med-dsa}

Building on action discretization, we initially extend soft Q-learning to discrete spaces. 
The soft value function is expressed as
\begin{equation}
    V^*_{\text{soft}}(\mathbf{s}) = \alpha \log \sum_{\mathbf{a}' \in \mathcal{A}} \exp \left( \frac{1}{\alpha} Q^*_{\text{soft}}(\mathbf{s}, \mathbf{a}') \right)
\end{equation}
And we omit the subscript $t$ for $\mathbf{s}_t$ and $\mathbf{a}_t$
To further streamline the expression of the policy, we define the soft advantage.

\begin{definition}[Soft Advantage]
    The soft advantage of $\mathbf{a}$ at $\mathbf{s}$ is given by
    \begin{equation}
        A^*(\mathbf{s}, \mathbf{a}) = Q^*_{\text{soft}}(\mathbf{s}, \mathbf{a}) - V^*_{\text{soft}}(\mathbf{s})
    \end{equation}
\end{definition}

Similar to the advantage in policy gradient-based RL algorithms, the soft advantage assesses how much taking action $\mathbf{a}$ at state $\mathbf{s}$ is beneficial. 
Thus, the optimal policy in Eq.~(\ref{eq:pre-soft-policy}) can be expressed as
\begin{equation}
\label{eq:med-sa-pi}
    \pi^*(\mathbf{a} | \mathbf{s}) = \exp \left( \frac{1}{\alpha} A^*(\mathbf{s}, \mathbf{a}) \right)
\end{equation}
Considering the multi-dimensional action space, it still remains necessary to use a neural network to output $B^{L \times D}$ Q values in the final layer, as per the DQN \cite{NatureDQN}.

However, outputting such a large number of Q values imposes a significant computational burden on the neural network.
Inspired by auto-regression \cite{GPT3}, we address this problem by generating the Q function for a state-action pair in an auto-regressive manner. 

For clarity, we treats discrete action discussed in Sec.~\ref{sec:med-coarse} in one level. 
The multi-level coarse-to-fine discrete action can be considered as additional action dimensions, without compromising generalization.
We first define the dimensional soft advantage for policy representation.

\begin{definition}[Dimensional Soft Advantage]
    The dimensional soft advantage of the action $a^d$ at state $\mathbf{s}$, considering the previous dimensional actions $\mathbf{a}^{-d} = (a^1, \cdots, a^{d-1})$, is expressed by
    \begin{equation}
    \label{eq:med-dsa}
        \pi (a^d | \mathbf{s}, \mathbf{a}^{-d}) = 
        \frac
        {exp \left( \frac{1}{\alpha} A^d(\mathbf{s}, \mathbf{a}^{-d}, a^d)) \right)}
        {Z(\mathbf{s}, \mathbf{a}^{-d})}
    \end{equation}
\end{definition}

where $Z^d(\mathbf{s}, \mathbf{a}^{-d})$ represents the partition function.

However, the dimensional soft advantage is not related to other equations and remains intractable. 
To address this, we propose the following theorem to establish a connection between the dimensional soft advantage and the soft advantage.

\begin{theorem}
\label{the:med-sum}
    If dimensional soft advantage $m^d(\mathbf{s}, \mathbf{a}^{-d}, a^d))$ satisfies
    \begin{equation}
    \label{eq:med-dsa-cond}
        log \sum_{a^{d'}} \exp{\left( \frac{1}{\alpha} A^d(\mathbf{s}, \mathbf{a}^{-d}, a^{d'}) \right)} = 0
    \end{equation}
    then the soft advantage can then be expressed as the summation of the dimensional soft advantages
    \begin{equation}
        \sum_{d=1}^D A^d(\mathbf{s}, \mathbf{a}^{-d}, a^d) = A(\mathbf{s}, \mathbf{a})
    \end{equation}
\end{theorem}
\begin{proof}
    See Appendix~\ref{sec:app-proof}.
\end{proof}

Through Eq.~(\ref{eq:med-dsa}) and Theorem~\ref{the:med-sum}, we extend soft Q-learning to a auto-regressive policy along action dimension.

Since we do not introduce additional elements in policy optimization, the Q-iteration follows the same update rule as soft Q-learning.
Based on Eq.~(\ref{eq:pre-soft-update}), we have
\begin{equation}
\label{eq:med-dsa-update}
    V_{\text{soft}}(\mathbf{s}_t) + A(\mathbf{s}_t, \mathbf{a}_t) 
    \leftarrow r_t + \gamma \mathbb{E}_{\mathbf{s}_{t+1} \sim p(s)} \left[ V_{\text{soft}}(\mathbf{s}_{t+1}) \right]
\end{equation}
The maximum entropy policy described in Eq.~(\ref{eq:pre-soft-policy}) can be obtained by repeatedly applying Eq.~(\ref{eq:med-dsa-update}) until it converges.

\subsection{Auto-Regressive Soft Q-learning}
\label{sec:med-alg}
\begin{algorithm}[tb]
    \caption{Auto-Regressive Soft Q Algorithm (ARSQ) }
    \label{pse:med-alg}
\begin{algorithmic}
    \STATE Initialize $\theta_{1, 2}, \phi_{1, 2}$ for $A^{\theta_i}$ and $V_{\text{soft}}^{\phi_i}$
    \STATE Assign target parameters $\overline{\theta}_i,  \overline{\phi}_i \leftarrow \theta_i, \phi_i$.
    \STATE Offline dataset $\mathcal{D}$, replay buffer $\mathcal{R} \leftarrow \mathcal{D}$.
    \FOR{each epoch}
        \FOR{each environment step}
            \STATE select $\mathbf{a}_t$ with $A_{\theta_1}$ and $A_{\theta_2}$ (\ref{eq:med-sa-pi}, \ref{eq:med-alg-cons})
            \STATE $\mathbf{s}_{t+1} \sim p(\mathbf{s}_{t+1} | \mathbf{s}_t, \mathbf{a}_t)$
            \STATE $\mathcal{R} \leftarrow \mathcal{R} \cup \{ \mathbf{s}_t, \mathbf{a}_t, r_t, \mathbf{s}_{t+1} \}$
        \ENDFOR
        \FOR{each gradient step}
            \STATE Sample mini-batch $b_D$, $b_R$ from $\mathcal{D}$, $\mathcal{R}$
            \STATE Calculate $\mathcal{L}_D = \mathcal{L}_{RL} + \beta \mathcal{L}_{BC}$ with $b_D$ (\ref{eq:med-alg-bc}, \ref{eq:med-alg-rl})
            \STATE Calculate $\mathcal{L}_R = \mathcal{L}_{RL}$ with $b_R$ (\ref{eq:med-alg-rl})
            \STATE Update $m_{\theta_i}$ according to $\hat{\nabla}_{\theta_i}(\mathcal{L}_D + \mathcal{L}_R)$
            \STATE Update $V_{s, \phi_i}$ according to $\hat{\nabla}_{\phi_i}(\mathcal{L}_D + \mathcal{L}_R)$
            % \STATE Update target networks $\overline{\theta}_i \leftarrow \rho \overline{\theta}_i + (1 - \rho) \theta_i$
            % \STATE Update target networks $\overline{\phi}_i \leftarrow \rho \overline{\phi}_i + (1 - \rho) \phi_i$
            \STATE Update target networks $\overline{\theta}_i \leftarrow \rho \overline{\theta}_i + (1 - \rho) \theta_i$ and $\overline{\phi}_i \leftarrow \rho \overline{\phi}_i + (1 - \rho) \phi_i$.
        \ENDFOR
    \ENDFOR
\end{algorithmic}
\end{algorithm}


Building on the theory outlined in Sec.~\ref{sec:med-dsa}, we introduce the \arsq (ARSQ) algorithm. 
The pseudo code for the ARSQ algorithm is presented in Algorithm~\ref{pse:med-alg}. 
We will discuss the various design choices of ARSQ.

\paragraph{Behavior cloning objective.}
To leverage offline demonstration data during online training, we introduce an additional behavior cloning loss term. 
Following previous works \cite{DQfD,CQN}, we encourage actions present in the offline dataset to be preferred over other actions. 
Specifically, we define the loss as
\begin{equation}
\label{eq:med-alg-bc}
\begin{aligned}
    \mathcal{L}_{BC}^{d} = \sum_{a^{d}} \max ( 
    & A^{d, \theta_i}(\mathbf{s}, \mathbf{a}^{-d}_e, a^{d}) \\ 
    & - A^{d, \theta_i}(\mathbf{s}, \mathbf{a}^{-d}_e, a_e^{d}), C_m )
\end{aligned}
\end{equation}
where $\mathbf{a}_e$ denotes the expert action observed in the offline dataset, and $C_m$ is a hyper-parameter controlling the margin. 
This objective encourages the soft advantages of expert actions to be at least $C_m$ higher than those of other actions.

\paragraph{Policy representation.}
As discussed in Sec. \ref{sec:med-dsa}, ARSQ predicts dimensional soft advantages, which function as both components of the Q function and policy representation. 
The network architecture is illustrated in Fig. \ref{fig:med-nn}.
In practical design, the soft value $V_{\text{soft}}$ and the dimensional soft advantage $A^d$ are predicted using two separate neural networks. 
The advantage prediction network estimates the dimensional soft advantage for each action dimension, based on the partially generated action from previous dimensions, creating an auto-regressive sequence. 
In practical design, we use a globally-shared MLP in the advantage network, with separate heads to predict the dimensional soft advantages.

\begin{figure*}[ht]
    \centering
    \includegraphics[width=0.9\linewidth]{fig/5_med-nn.png}
    \vspace{-2em}
    \caption{Network architecture of ARSQ. The soft value $V_{\text{soft}}$ and the dimensional soft advantage $A^d$ are predicted by two separate networks. The advantage network utilizes a shared backbone, and advantage constraints are applied to its output.}
    \label{fig:med-nn}
\end{figure*}

Another challenge is applying the constraint of the dimensional soft advantage as per Eq.~(\ref{eq:med-dsa-cond}). 
Here, we enforce a hard constraint by normalizing each output head through log-sum-exp subtraction, ensuring consistency across outputs.
\begin{equation}
\label{eq:med-alg-cons}
\begin{aligned}
    A^d(\mathbf{s}_t, \mathbf{a}^{-d} &, a^d) 
    = u^d(\mathbf{s}_t, \mathbf{a}^{-d}, a^d) \\
    & - log \sum_{a^{d'}} \exp \left(
    \frac{1}{\alpha} u^d(\mathbf{s}_t, \mathbf{a}^{-d}, a^{d'})
    \right)
\end{aligned}
\end{equation}
where $u^d$ is the output of the $d$-th output head.

Furthermore, to stabilize training and address the over-estimation problem \cite{fujimoto2018addressing, DoubleQ}, we implemented a double Q network alongside a target network in our practical application.
Therefore, the optimization objective in Eq.~(\ref{eq:med-dsa-update}) is modified to
\begin{equation}
    \mathbf{y}_t =  \gamma \mathbb{E}_{\mathbf{s}_{t+1} \sim p(s)} \left[ \text{min} \left( V^{\overline{\phi}_1}_{\text{soft}1}(\mathbf{s}_{t+1}), V^{\overline{\phi}_2}_{\text{soft}2} (\mathbf{s}_{t+1}) \right) \right]
\end{equation}
\begin{equation}
\label{eq:med-alg-rl}
    \mathcal{L}_{RL} = \frac{1}{2} \left( V^{\phi_i}_{\text{soft}}(\mathbf{s}_t) + A^{\theta_i}(\mathbf{s}_t, \mathbf{a}_t) - \mathbf{y}_t  \right)
\end{equation}
where $V^{\overline{\phi}_i}_{\text{soft}}$ represents the soft value predicted by the target network.

\paragraph{Auto-regressive conditioning.} 
\label{sec:med-alg-ar}
In Sec.~\ref{sec:med-dsa}, we explained the process of handling discrete action in one coarse-to-fine level.
With multi-level coarse-to-fine action discretization, the auto-regressive conditioning encompasses two aspects. 
\emph{Dimensional conditioning} refers to generating actions for each dimension in an auto-regressive sequence, while \emph{coarse-to-fine conditioning} involves generating actions for each dimension from coarse to fine. 
In practice, we implement coarse-to-fine conditioning prior to dimensional conditioning. 
Specifically, dimensional conditioning serves as the inner conditioning, while coarse-to-fine conditioning acts as the outer conditioning across levels. 
We explore swapping the order of conditioning in Sec.~\ref{sec:exp-abl}, and the results indicate that the current design better captures interdependencies between action dimensions.




\section{Experiments}\label{sec:experiments}


\begin{figure*}[!ht]
\centering
\includegraphics[width=0.32\textwidth]{eps_figs/eps_eps_diff_m_Random.pdf}
\includegraphics[width=0.32\textwidth]{eps_figs/eps_eps_diff_m_False_Negative.pdf}
\includegraphics[width=0.32\textwidth]{eps_figs/eps_eps_diff_m_False_Positive.pdf}
\caption{
Three kinds of error rates with different bit-array lengths $m$. We fix the number of inserted elements $|A|=10^5$, the number of hash functions $k = 3$, and $\delta = 0.01$ in $(\epsilon, \delta)$-DP. 
In the figure, $\log$ denotes $\log_2$. 
{\bf Left:} Total error denotes the case when we randomly choose queries from the universe $[n]$; 
{\bf Middle:} False negative denotes the case when we randomly choose queries from the set $S$, which represents the set of elements inserted into the DP Bloom filter; 
{\bf Right:} False positive denotes the case when we randomly choose queries from the set $\ov{S} = [n] \backslash S$.  
As $m$ increases, the total error rate and false positive error rate decrease accordingly, while false negative error rate remains constant. 
As $\epsilon$ approaches $0$, the DP Bloom filter gets closer to random guessing. In this case, the false positive error rate converges to $\frac{1}{2^k}$, and the false negative error rate converges to $1 - \frac{1}{2^k}$. This is consistent with our result in Lemma~\ref{lem:random_guess}
Our \textsc{DPBloomFilter} achieves practical utility when $\epsilon$ is small(e.g. $\epsilon < 10$).
}
\label{fig:eps_diff_m}
\end{figure*}


\begin{figure*}[!ht]
\centering
\includegraphics[width=0.32\textwidth]{eps_figs/eps_eps_diff_na_Random.pdf}
\includegraphics[width=0.32\textwidth]{eps_figs/eps_eps_diff_na_False_Negative.pdf}
\includegraphics[width=0.32\textwidth]{eps_figs/eps_eps_diff_na_False_Positive.pdf}
\caption{
Three kinds of error rates with different numbers of inserted elements $|A|$. We fix the length of bit-array $m=2^{19}$, the number of hash functions $k = 3$, and $\delta = 0.01$ in $(\epsilon, \delta)$-DP.
As $|A|$ increases, the Total Error Rate and false positive error rate increase accordingly, while the false negative error rate remains constant. 
}
\label{fig:eps_diff_na}
\end{figure*}

\begin{figure*}[!ht]
\centering
\includegraphics[width=0.32\textwidth]{eps_figs/eps_eps_diff_k_Random.pdf}
\includegraphics[width=0.32\textwidth]{eps_figs/eps_eps_diff_k_False_Negative.pdf}
\includegraphics[width=0.32\textwidth]{eps_figs/eps_eps_diff_k_False_Positive.pdf}
\caption{
Three kinds of error rates with different numbers of hash function $k$.  
We fix the length of bit-array $m=2^{19}$, the number of inserted elements $|A| = 10^5$, and $\delta = 0.01$ in $(\epsilon, \delta)$-DP.
As $k$ increases, the Total Error Rate and false positive error rate decrease accordingly, while the false negative error rate increases accordingly. 
}
\label{fig:eps_diff_k}
\end{figure*}

In this section, we introduce the simulation experiments conducted on the DPBloomfilter.
In Section~\ref{sec:exp:setup}, we introduce the basic setup of our experiments and restate basic definitions of three kinds of error.
In Section~\ref{sec:exp:main_result}, we discuss the results of our experiments, which align with our theoretical analysis. 

\subsection{Experiments Setup and Basic Notations} \label{sec:exp:setup}


Recall that we have the following notations. 
Let $m$ denote the length of the bit array in the DPBloomfilter.
Let $|A|$ denote the number of elements inserted into the DPBloomfilter. 
Let $k$ denote the number of hash functions used in the DPBloomfilter.
Let $\epsilon, \delta$ denote the differential privacy parameters of the DPBloomfilter. 
Let $N$ denotes the $1 - \delta$ quantile of $W$ (see Definition~\ref{def:W}), and the close-form of the distribution of $W$ is shown in Lemma~\ref{lem:distribution_of_W}. 
Let $\epsilon_0 = \epsilon / N$. By Theorem~\ref{thm:query_privacy:informal}, we choose $\epsilon_0$ in this way can guarantee to $(\epsilon, \delta)$-DP in the whole algorithm. 
Unless specified, we adopt $m = 2^{19}, |A| = 10^5, k=8, n = 2^{63} \approx 10^{19}$ in the following experiments. 
We choose this $n$ because this $n$ is the biggest integer that can be represented on our server.

Recall that $[n]$ denotes the universe. 
Let $S$ denote the elements inserted into the DPBloomfilter. 
Let $\ov{S} = [n] \backslash S$ denote the elements not inserted into the DPBloomfilter. Let $\wt{z} \in \{ 0, 1 \}$ denote the answer output by DPBloomfilter. 

We report three kinds of error rates in our experiments. They are the following: 
(1) {\bf total error}, where we randomly choose queries from the universe $[n]$ and report the error rate of our DPBloomfilter;
(2) {\bf false positive error}, where we random choose queries from $\ov{S}$. When the DPBloomfilter outputs $\wt{z} = 1$, this will cause a false positive error; 
(3) {\bf false negative error}, where we random choose queries from $S$. When the DPBloomfilter outputs $\wt{z} = 0$, this will cause a false negative error. 

\subsection{Experiment Results} \label{sec:exp:main_result}

In this section, we conduct experiments based on the setting mentioned in the previous section. Specifically, we run simulation experiments on different $m$, $|A|$, and $k$ to demonstrate the utility of our algorithm under differential privacy guarantees. 

In Figure~\ref{fig:eps_diff_m}, we conduct experiments on different $m$, whereas $m$ increases, the total error rate and false positive error rate decrease accordingly, while the false negative error rate remains constant. 

In Figure~\ref{fig:eps_diff_na}, we also conduct experiments on different $|A|$, whereas $|A|$ increases, the total error rate and false positive error rate increase accordingly. At the same time, the false negative error rate remains constant.
This phenomenon is consistent with our theoretical analysis of the utility of DPBloomfilter (Theorem~\ref{thm:dpbloom_true_accuracy:informal}). Recall that we have $\alpha = \Pr[z=0]$, denoting the probability of an arbitrary query $q \notin A$. 
Since $|A|$ increases, $\alpha$ decreases, the utility guarantee in Theorem~\ref{thm:dpbloom_true_accuracy:informal}, which is consistent with higher error rate in our experiment results. 


In Figure~\ref{fig:eps_diff_k}, we conduct experiments on different $k$ as well, whereas $k$ increases, the total error rate, and false positive error rate decrease, while the false negative error rate increases accordingly. 

Note that in Figure~\ref{fig:eps_diff_m}, Figure~\ref{fig:eps_diff_na}, and Figure~\ref{fig:eps_diff_k}, as $\epsilon$ approaches $0$, the DPBloomfilter gets closer to random guessing. In this case, the false positive error rate converges to $\frac{1}{2^k}$, and the false negative error rate converges to $1 - \frac{1}{2^k}$. This is consistent with our result in Lemma~\ref{lem:random_guess}. 
Also, as $\epsilon$ increases, the three types of error rates in the Bloom filter with differential privacy (DP) approach the error rates observed when DP is not applied. This is consistent with the intuition that when $\epsilon$ increases, there is less privacy. Therefore, the performance approaches the performance of a Bloom filter without any privacy guarantees. 


\section{Discussion}\label{sec:discussion}



\subsection{From Interactive Prompting to Interactive Multi-modal Prompting}
The rapid advancements of large pre-trained generative models including large language models and text-to-image generation models, have inspired many HCI researchers to develop interactive tools to support users in crafting appropriate prompts.
% Studies on this topic in last two years' HCI conferences are predominantly focused on helping users refine single-modality textual prompts.
Many previous studies are focused on helping users refine single-modality textual prompts.
However, for many real-world applications concerning data beyond text modality, such as multi-modal AI and embodied intelligence, information from other modalities is essential in constructing sophisticated multi-modal prompts that fully convey users' instruction.
This demand inspires some researchers to develop multimodal prompting interactions to facilitate generation tasks ranging from visual modality image generation~\cite{wang2024promptcharm, promptpaint} to textual modality story generation~\cite{chung2022tale}.
% Some previous studies contributed relevant findings on this topic. 
Specifically, for the image generation task, recent studies have contributed some relevant findings on multi-modal prompting.
For example, PromptCharm~\cite{wang2024promptcharm} discovers the importance of multimodal feedback in refining initial text-based prompting in diffusion models.
However, the multi-modal interactions in PromptCharm are mainly focused on the feedback empowered the inpainting function, instead of supporting initial multimodal sketch-prompt control. 

\begin{figure*}[t]
    \centering
    \includegraphics[width=0.9\textwidth]{src/img/novice_expert.pdf}
    \vspace{-2mm}
    \caption{The comparison between novice and expert participants in painting reveals that experts produce more accurate and fine-grained sketches, resulting in closer alignment with reference images in close-ended tasks. Conversely, in open-ended tasks, expert fine-grained strokes fail to generate precise results due to \tool's lack of control at the thin stroke level.}
    \Description{The comparison between novice and expert participants in painting reveals that experts produce more accurate and fine-grained sketches, resulting in closer alignment with reference images in close-ended tasks. Novice users create rougher sketches with less accuracy in shape. Conversely, in open-ended tasks, expert fine-grained strokes fail to generate precise results due to \tool's lack of control at the thin stroke level, while novice users' broader strokes yield results more aligned with their sketches.}
    \label{fig:novice_expert}
    % \vspace{-3mm}
\end{figure*}


% In particular, in the initial control input, users are unable to explicitly specify multi-modal generation intents.
In another example, PromptPaint~\cite{promptpaint} stresses the importance of paint-medium-like interactions and introduces Prompt stencil functions that allow users to perform fine-grained controls with localized image generation. 
However, insufficient spatial control (\eg, PromptPaint only allows for single-object prompt stencil at a time) and unstable models can still leave some users feeling the uncertainty of AI and a varying degree of ownership of the generated artwork~\cite{promptpaint}.
% As a result, the gap between intuitive multi-modal or paint-medium-like control and the current prompting interface still exists, which requires further research on multi-modal prompting interactions.
From this perspective, our work seeks to further enhance multi-object spatial-semantic prompting control by users' natural sketching.
However, there are still some challenges to be resolved, such as consistent multi-object generation in multiple rounds to increase stability and improved understanding of user sketches.   


% \new{
% From this perspective, our work is a step forward in this direction by allowing multi-object spatial-semantic prompting control by users' natural sketching, which considers the interplay between multiple sketch regions.
% % To further advance the multi-modal prompting experience, there are some aspects we identify to be important.
% % One of the important aspects is enhancing the consistency and stability of multiple rounds of generation to reduce the uncertainty and loss of control on users' part.
% % For this purpose, we need to develop techniques to incorporate consistent generation~\cite{tewel2024training} into multi-modal prompting framework.}
% % Another important aspect is improving generative models' understanding of the implicit user intents \new{implied by the paint-medium-like or sketch-based input (\eg, sketch of two people with their hands slightly overlapping indicates holding hand without needing explicit prompt).
% % This can facilitate more natural control and alleviate users' effort in tuning the textual prompt.
% % In addition, it can increase users' sense of ownership as the generated results can be more aligned with their sketching intents.
% }
% For example, when users draw sketches of two people with their hands slightly overlapping, current region-based models cannot automatically infer users' implicit intention that the two people are holding hands.
% Instead, they still require users to explicitly specify in the prompt such relationship.
% \tool addresses this through sketch-aware prompt recommendation to fill in the necessary semantic information, alleviating users' workload.
% However, some users want the generative AI in the future to be able to directly infer this natural implicit intentions from the sketches without additional prompting since prompt recommendation can still be unstable sometimes.


% \new{
% Besides visual generation, 
% }
% For example, one of the important aspect is referring~\cite{he2024multi}, linking specific text semantics with specific spatial object, which is partly what we do in our sketch-aware prompt recommendation.
% Analogously, in natural communication between humans, text or audio alone often cannot suffice in expressing the speakers' intentions, and speakers often need to refer to an existing spatial object or draw out an illustration of her ideas for better explanation.
% Philosophically, we HCI researchers are mostly concerned about the human-end experience in human-AI communications.
% However, studies on prompting is unique in that we should not just care about the human-end interaction, but also make sure that AI can really get what the human means and produce intention-aligned output.
% Such consideration can drastically impact the design of prompting interactions in human-AI collaboration applications.
% On this note, although studies on multi-modal interactions is a well-established topic in HCI community, it remains a challenging problem what kind of multi-modal information is really effective in helping humans convey their ideas to current and next generation large AI models.




\subsection{Novice Performance vs. Expert Performance}\label{sec:nVe}
In this section we discuss the performance difference between novice and expert regarding experience in painting and prompting.
First, regarding painting skills, some participants with experience (4/12) preferred to draw accurate and fine-grained shapes at the beginning. 
All novice users (5/12) draw rough and less accurate shapes, while some participants with basic painting skills (3/12) also favored sketching rough areas of objects, as exemplified in Figure~\ref{fig:novice_expert}.
The experienced participants using fine-grained strokes (4/12, none of whom were experienced in prompting) achieved higher IoU scores (0.557) in the close-ended task (0.535) when using \tool. 
This is because their sketches were closer in shape and location to the reference, making the single object decomposition result more accurate.
Also, experienced participants are better at arranging spatial location and size of objects than novice participants.
However, some experienced participants (3/12) have mentioned that the fine-grained stroke sometimes makes them frustrated.
As P1's comment for his result in open-ended task: "\emph{It seems it cannot understand thin strokes; even if the shape is accurate, it can only generate content roughly around the area, especially when there is overlapping.}" 
This suggests that while \tool\ provides rough control to produce reasonably fine results from less accurate sketches for novice users, it may disappoint experienced users seeking more precise control through finer strokes. 
As shown in the last column in Figure~\ref{fig:novice_expert}, the dragon hovering in the sky was wrongly turned into a standing large dragon by \tool.

Second, regarding prompting skills, 3 out of 12 participants had one or more years of experience in T2I prompting. These participants used more modifiers than others during both T2I and R2I tasks.
Their performance in the T2I (0.335) and R2I (0.469) tasks showed higher scores than the average T2I (0.314) and R2I (0.418), but there was no performance improvement with \tool\ between their results (0.508) and the overall average score (0.528). 
This indicates that \tool\ can assist novice users in prompting, enabling them to produce satisfactory images similar to those created by users with prompting expertise.



\subsection{Applicability of \tool}
The feedback from user study highlighted several potential applications for our system. 
Three participants (P2, P6, P8) mentioned its possible use in commercial advertising design, emphasizing the importance of controllability for such work. 
They noted that the system's flexibility allows designers to quickly experiment with different settings.
Some participants (N = 3) also mentioned its potential for digital asset creation, particularly for game asset design. 
P7, a game mod developer, found the system highly useful for mod development. 
He explained: "\emph{Mods often require a series of images with a consistent theme and specific spatial requirements. 
For example, in a sacrifice scene, how the objects are arranged is closely tied to the mod's background. It would be difficult for a developer without professional skills, but with this system, it is possible to quickly construct such images}."
A few participants expressed similar thoughts regarding its use in scene construction, such as in film production. 
An interesting suggestion came from participant P4, who proposed its application in crime scene description. 
She pointed out that witnesses are often not skilled artists, and typically describe crime scenes verbally while someone else illustrates their account. 
With this system, witnesses could more easily express what they saw themselves, potentially producing depictions closer to the real events. "\emph{Details like object locations and distances from buildings can be easily conveyed using the system}," she added.

% \subsection{Model Understanding of Users' Implicit Intents}
% In region-sketch-based control of generative models, a significant gap between interaction design and actual implementation is the model's failure in understanding users' naturally expressed intentions.
% For example, when users draw sketches of two people with their hands slightly overlapping, current region-based models cannot automatically infer users' implicit intention that the two people are holding hands.
% Instead, they still require users to explicitly specify in the prompt such relationship.
% \tool addresses this through sketch-aware prompt recommendation to fill in the necessary semantic information, alleviating users' workload.
% However, some users want the generative AI in the future to be able to directly infer this natural implicit intentions from the sketches without additional prompting since prompt recommendation can still be unstable sometimes.
% This problem reflects a more general dilemma, which ubiquitously exists in all forms of conditioned control for generative models such as canny or scribble control.
% This is because all the control models are trained on pairs of explicit control signal and target image, which is lacking further interpretation or customization of the user intentions behind the seemingly straightforward input.
% For another example, the generative models cannot understand what abstraction level the user has in mind for her personal scribbles.
% Such problems leave more challenges to be addressed by future human-AI co-creation research.
% One possible direction is fine-tuning the conditioned models on individual user's conditioned control data to provide more customized interpretation. 

% \subsection{Balance between recommendation and autonomy}
% AIGC tools are a typical example of 
\subsection{Progressive Sketching}
Currently \tool is mainly aimed at novice users who are only capable of creating very rough sketches by themselves.
However, more accomplished painters or even professional artists typically have a coarse-to-fine creative process. 
Such a process is most evident in painting styles like traditional oil painting or digital impasto painting, where artists first quickly lay down large color patches to outline the most primitive proportion and structure of visual elements.
After that, the artists will progressively add layers of finer color strokes to the canvas to gradually refine the painting to an exquisite piece of artwork.
One participant in our user study (P1) , as a professional painter, has mentioned a similar point "\emph{
I think it is useful for laying out the big picture, give some inspirations for the initial drawing stage}."
Therefore, rough sketch also plays a part in the professional artists' creation process, yet it is more challenging to integrate AI into this more complex coarse-to-fine procedure.
Particularly, artists would like to preserve some of their finer strokes in later progression, not just the shape of the initial sketch.
In addition, instead of requiring the tool to generate a finished piece of artwork, some artists may prefer a model that can generate another more accurate sketch based on the initial one, and leave the final coloring and refining to the artists themselves.
To accommodate these diverse progressive sketching requirements, a more advanced sketch-based AI-assisted creation tool should be developed that can seamlessly enable artist intervention at any stage of the sketch and maximally preserve their creative intents to the finest level. 

\subsection{Ethical Issues}
Intellectual property and unethical misuse are two potential ethical concerns of AI-assisted creative tools, particularly those targeting novice users.
In terms of intellectual property, \tool hands over to novice users more control, giving them a higher sense of ownership of the creation.
However, the question still remains: how much contribution from the user's part constitutes full authorship of the artwork?
As \tool still relies on backbone generative models which may be trained on uncopyrighted data largely responsible for turning the sketch into finished artwork, we should design some mechanisms to circumvent this risk.
For example, we can allow artists to upload backbone models trained on their own artworks to integrate with our sketch control.
Regarding unethical misuse, \tool makes fine-grained spatial control more accessible to novice users, who may maliciously generate inappropriate content such as more realistic deepfake with specific postures they want or other explicit content.
To address this issue, we plan to incorporate a more sophisticated filtering mechanism that can detect and screen unethical content with more complex spatial-semantic conditions. 
% In the future, we plan to enable artists to upload their own style model

% \subsection{From interactive prompting to interactive spatial prompting}


\subsection{Limitations and Future work}

    \textbf{User Study Design}. Our open-ended task assesses the usability of \tool's system features in general use cases. To further examine aspects such as creativity and controllability across different methods, the open-ended task could be improved by incorporating baselines to provide more insightful comparative analysis. 
    Besides, in close-ended tasks, while the fixing order of tool usage prevents prior knowledge leakage, it might introduce learning effects. In our study, we include practice sessions for the three systems before the formal task to mitigate these effects. In the future, utilizing parallel tests (\textit{e.g.} different content with the same difficulty) or adding a control group could further reduce the learning effects.

    \textbf{Failure Cases}. There are certain failure cases with \tool that can limit its usability. 
    Firstly, when there are three or more objects with similar semantics, objects may still be missing despite prompt recommendations. 
    Secondly, if an object's stroke is thin, \tool may incorrectly interpret it as a full area, as demonstrated in the expert results of the open-ended task in Figure~\ref{fig:novice_expert}. 
    Finally, sometimes inclusion relationships (\textit{e.g.} inside) between objects cannot be generated correctly, partially due to biases in the base model that lack training samples with such relationship. 

    \textbf{More support for single object adjustment}.
    Participants (N=4) suggested that additional control features should be introduced, beyond just adjusting size and location. They noted that when objects overlap, they cannot freely control which object appears on top or which should be covered, and overlapping areas are currently not allowed.
    They proposed adding features such as layer control and depth control within the single-object mask manipulation. Currently, the system assigns layers based on color order, but future versions should allow users to adjust the layer of each object freely, while considering weighted prompts for overlapping areas.

    \textbf{More customized generation ability}.
    Our current system is built around a single model $ColorfulXL-Lightning$, which limits its ability to fully support the diverse creative needs of users. Feedback from participants has indicated a strong desire for more flexibility in style and personalization, such as integrating fine-tuned models that cater to specific artistic styles or individual preferences. 
    This limitation restricts the ability to adapt to varied creative intents across different users and contexts.
    In future iterations, we plan to address this by embedding a model selection feature, allowing users to choose from a variety of pre-trained or custom fine-tuned models that better align with their stylistic preferences. 
    
    \textbf{Integrate other model functions}.
    Our current system is compatible with many existing tools, such as Promptist~\cite{hao2024optimizing} and Magic Prompt, allowing users to iteratively generate prompts for single objects. However, the integration of these functions is somewhat limited in scope, and users may benefit from a broader range of interactive options, especially for more complex generation tasks. Additionally, for multimodal large models, users can currently explore using affordable or open-source models like Qwen2-VL~\cite{qwen} and InternVL2-Llama3~\cite{llama}, which have demonstrated solid inference performance in our tests. While GPT-4o remains a leading choice, alternative models also offer competitive results.
    Moving forward, we aim to integrate more multimodal large models into the system, giving users the flexibility to choose the models that best fit their needs. 
    


\section{Conclusion}\label{sec:conclusion}
In this paper, we present \tool, an interactive system designed to help novice users create high-quality, fine-grained images that align with their intentions based on rough sketches. 
The system first refines the user's initial prompt into a complete and coherent one that matches the rough sketch, ensuring the generated results are both stable, coherent and high quality.
To further support users in achieving fine-grained alignment between the generated image and their creative intent without requiring professional skills, we introduce a decompose-and-recompose strategy. 
This allows users to select desired, refined object shapes for individual decomposed objects and then recombine them, providing flexible mask manipulation for precise spatial control.
The framework operates through a coarse-to-fine process, enabling iterative and fine-grained control that is not possible with traditional end-to-end generation methods. 
Our user study demonstrates that \tool offers novice users enhanced flexibility in control and fine-grained alignment between their intentions and the generated images.



% Acknowledgements should only appear in the accepted version.
% \section*{Acknowledgements}
% \textbf{Do not} include acknowledgements in the initial version of
% the paper submitted for blind review.

\section*{Impact Statement}

This paper presents work whose goal is to advance the field of Machine Learning. There are many potential societal consequences of our work, none which we feel must be specifically highlighted here.


% In the unusual situation where you want a paper to appear in the
% references without citing it in the main text, use \nocite
\nocite{langley00}

\bibliography{example_paper}
\bibliographystyle{icml2025}


%%%%%%%%%%%%%%%%%%%%%%%%%%%%%%%%%%%%%%%%%%%%%%%%%%%%%%%%%%%%%%%%%%%%%%%%%%%%%%%
%%%%%%%%%%%%%%%%%%%%%%%%%%%%%%%%%%%%%%%%%%%%%%%%%%%%%%%%%%%%%%%%%%%%%%%%%%%%%%%
% APPENDIX
%%%%%%%%%%%%%%%%%%%%%%%%%%%%%%%%%%%%%%%%%%%%%%%%%%%%%%%%%%%%%%%%%%%%%%%%%%%%%%%
%%%%%%%%%%%%%%%%%%%%%%%%%%%%%%%%%%%%%%%%%%%%%%%%%%%%%%%%%%%%%%%%%%%%%%%%%%%%%%%
\newpage
\appendix
\onecolumn

\section{Proof of Theorem~\ref{the:med-sum}}
\label{sec:app-proof}

First, we express the policy using conditional probability, and then replace it with Eq.~(\ref{eq:med-dsa}).
\begin{equation}
\begin{aligned}
    \pi (\mathbf{a} | \mathbf{s})
    &= \prod_{d=1}^D \pi (a^d | \mathbf{s}, \mathbf{a}^{-d} ) \\
    &= \prod_{d=1}^D \frac
    {exp \left( \frac{1}{\alpha} A^d(\mathbf{s}, \mathbf{a}^{-d}, a^d) \right)}
    {Z(\mathbf{s}, \mathbf{a}^{-d})} \\
    &= \frac
    {\prod_{d=1}^D exp \left(\frac{1}{\alpha} A^d(\mathbf{s}, \mathbf{a}^{-d}, a^d) \right) }
    {\prod_{d=1}^D Z^d(\mathbf{s}, \mathbf{a}^{-d})} \\
    &= \frac
    {exp \left( \frac{1}{\alpha} \sum_{d=1}^D A^d(\mathbf{s}, \mathbf{a}^{-d}, a^d)\right)}
    {\prod_{d=1}^D Z^d(\mathbf{s}, \mathbf{a}^{-d})} \\
\end{aligned}
\end{equation}
We can then apply Eq.~(\ref{eq:med-dsa-cond}), resulting in
\begin{equation}
    \pi (\mathbf{a} | \mathbf{s})
    = exp \left( \frac{1}{\alpha} \sum_{d=1}^D A^d(\mathbf{s}, \mathbf{a}^{-d}, a^d)\right)
\end{equation}
Recall that the policy $\pi (\mathbf{a} | \mathbf{s})$ can be represented using the soft advantage as shown in Eq.~(\ref{eq:med-sa-pi}). Therefore, we have
\begin{equation}
    \sum_{d=1}^D A^d(\mathbf{s}, \mathbf{a}^{-d}, a^d)
     = A(\mathbf{s}, \mathbf{a})
\end{equation}

\section{Implementation Details}

\subsection{Action Selection}
As illustrated in Algorithm~\ref{pse:med-alg}, the action selection process receives inputs from $A_{\theta_1}$ and $A_{\theta_2}$ and produces $\mathbf{a}_t$. 
Eq.~(\ref{eq:med-sa-pi}) and Eq.~(\ref{eq:med-alg-cons}) describe the action selection process utilizing a single soft advantage network. 
To leverage the benefits of a double network, we employ two advantage networks to generate more precise actions. 
This process is detailed in Algorithm~\ref{pse:med-alg-act}.

\begin{algorithm}[h]
    \caption{ARSQ Action Selection with Double Q Network}
    \label{pse:med-alg-act}
\begin{algorithmic}
    \STATE \textbf{Input:} parameter $\theta_{1, 2}$ for $A^{\theta_i}$, state $\mathbf{s}_t$
    \STATE \textbf{Output:} action $\mathbf{a}_t$
    \STATE Initialize output action $\mathbf{a}_t = \emptyset$
    \FOR{each action dimension $d$}
        \STATE Compute $A^{d, \theta_i}(\mathbf{s}_t, \mathbf{a}_t, a^d )$ for each $a^d$ (\ref{eq:med-alg-cons})
        \STATE Compute $A^{d}(a^d ) = \min_i A^{d, \theta_i}(\mathbf{s}_t, \mathbf{a}_t, a^d )$
        \STATE Compute $\tilde{\pi}^d(a^d)= \text{exp} \left( \frac{1}{\alpha} A^{d}(a^d ) \right)$ (\ref{eq:med-sa-pi})
        \STATE Normalize $\tilde{\pi}^d$ by $\pi^d(a^d)=\frac{\tilde{\pi}^d (a^d)}{\sum_{a^{d'}} \tilde{\pi}^d(a^{d'})} $
        \STATE Sample discrete action at dimension $d$ with $\pi^d(a^d)$
        \STATE Append action $\mathbf{a}_t = \mathbf{a}_t \cup \{ a^d \}$
    \ENDFOR
\end{algorithmic}
\end{algorithm}

\subsection{Variant of Behavior Cloning Objective}
As discussed in Sec.~\ref{sec:med-alg}, we incorporate an behavior cloning objective to effectively utilize offline demonstration data during online training, as defined in Eq.~(\ref{eq:med-alg-bc}). 

Following prior works \cite{CQL}, we also employ a variant of this objective, expressed as:  
\begin{equation}
    \mathcal{L}_{BC-v}^{d} = \max \left( 
    \text{log} \sum_{a^d \neq a^d_e}
    \text{exp} \left( A^{d, \theta_i}(\mathbf{s}, \mathbf{a}^{-d}_e, a^{d}) \right) - A^{d, \theta_i}(\mathbf{s}, \mathbf{a}^{-d}_e, a^{d}_e), C_m \right)
\end{equation}

where $a^d_e$ is the expert action and $C_m$ is a predefined margin constant.

We observe that this variant objective achieves better performance in scenarios where action modes are concentrated, such as in the \textit{medium} and \textit{medium-expert} series of datasets in D4RL.
Consequently, we adopt this variant objective when working with such datasets.

\subsection{Network Architecture}

In RLBench tasks, observations consist of a combination of RGB images and low-dimensional states. 
To compute the dimensional soft advantage for a given dimension, we first input the RGB images and low-dimensional states into a Convolutional Neural Network (CNN) \cite{CNN} encoder and a Multi-Layer Perceptron (MLP) \cite{MLP} encoder, respectively, to extract feature representations. 
These representations are then used to predict the soft value. 
Concurrently, the feature representations are combined with actions from previous dimensions and coarse-to-fine levels to create auto-regressive conditioning. 
An MLP-based shared backbone and output head are then utilized to determine the dimensional soft advantage for the given dimension.

In D4RL tasks, observations consist solely of low-dimensional states, and feature representations are derived directly from these states.


\subsection{Hyper-parameters}
\begin{table}[ht]
\centering
\begin{tabular}{lll}
    \toprule
    Hyper-parameter & D4RL & RLBench \\
    \midrule
    Image resolution & / & $84 \times 84 \times 3$ \\
    Image augmentation & / & RandomShift \\
    Frame stack & 1 & 8 \\
    \midrule
    CNN - Encoder & / & Conv (c=[32, 64, 128, 256], s=2, p=1) \\
    Backbone & Linear (512, 512, 512) & Linear (512, 512, 512, bias=False) \\
    Output Head Layers & 1 & 1 \\
    Activation & Tanh & SiLU \& LayerNorm \\
    \midrule
    Coarse-to-fine Levels & 2 & 3 \\
    Coarse-to-fine Bins & 7 & 5 \\
    \midrule
    Batch Size & 512 & 512 \\
    Optimizer & Adam & AdamW (weight decay = 0.1) \\
    Learning Rate & 3e-4 & 5e-5 \\
    Temperature Coefficient $\alpha$ & 0.01 & 0.001 \\
    Target Critic Update Ratio ($\tau$) & 0.005 & 0.02 \\
    BC Margin $C_m$ & -1 & -0.01 \\
    Action Roll-out Network & Current & Target \\
    \bottomrule
\end{tabular}
\caption{Typical hyper-parameters of ARSQ in D4RL and RLBench.}
\label{tab:alg-hyperparam}
\end{table}

The hyperparameters of ARSQ are presented in Table~\ref{tab:alg-hyperparam}. 
We provide the typical hyperparameters for ARSQ in D4RL (\textit{hopper-medium}) and RLBench (\textit{Open Oven}). 
In RLBench, ARSQ employs RandomShift \cite{DrQv2} for image augmentation. 
Additionally, ARSQ utilizes SiLU \cite{SiLU} and LayerNorm \cite{LayerNorm} as activation functions in RLBench.


\section{Experiment Setup}

\subsection{Motivating Example Setup}
\label{sec:app-example}

As introduced in Sec.~\ref{sec:intro} and illustrated in Fig.~\ref{fig:intro-case}, we consider a motivating example to demonstrate the impact of Q decomposition on policy training. 
The dataset is depicted in Fig.~\ref{fig:intro-toy-env}, with each point to be a data point in the dataset.
The color of the data points indicates the reward of the data point.
To illustrate the Q function of value-based RL algorithms, we first discretize the action space with $2$ bins in each action dimension. 

\begin{itemize}
    \item Q function given by independent action decomposition is an example of DecQN \cite{DecQN}, as well as in CQN \cite{CQN}, which features just a single coarse-to-fine level. 
    In this setting, we employ separate tabular Q functions, $Q(s, a_1)$ and $Q(s, a_2)$, for action dimension 1 and action dimension 2. 
    The Q function is learned by gradient descent.
    \item For the Q function obtained through auto-regressive action decomposition, we employ both tabular soft advantage functions, $A^1(s, a_1)$ and $A^2(s, a_1, a_2)$ for action dimension 1 and action dimension 2, and a tabular soft value function $V_{\text{soft}}(s)$. 
    The Q value reported in Fig.~\ref{fig:intro-toy-ar} is a sum of the soft value and the dimensional soft advantage of the corresponding dimensions, i.e., $Q(s, a_1, a_2) = V_{\text{soft}}(s) + A^1(s, a_1) + A^2(s, a_1, a_2)$. The soft advantage functions and the soft value function are simultaneously learned through gradient descent.
\end{itemize}



\subsection{Environment and Dataset}
\label{sec:app-env}

\paragraph{D4RL Gym Environment}
D4RL \cite{D4RL} provides datasets for various tasks to evaluate the performance of reinforcement learning. 
In this context, we use 3 Gym Locomotion tasks and datasets from D4RL to assess the performance of ARSQ and other baselines. 
These tasks are illustrated in Fig.~\ref{fig:app-env-d4rl}. The agent's observations include its states, such as the angle and velocity of each rotor. 
The agent's actions consist of torques applied between the robot's links, constrained within the range of $(-1, 1)$. 
The reward is dense, offering incentives for task completion and survival, while penalizing excessive energy-consuming actions.

\begin{figure}[h]
    \centering
    \includegraphics[width=0.8\linewidth]{fig/app-env-d4rl.png}
    \caption{D4RL Gym tasks used in experiment.}
    \label{fig:app-env-d4rl}
\end{figure}

\paragraph{D4RL Dataset}
In D4RL, we use the \textit{medium-replay}, \textit{medium}, and \textit{medium-expert} datasets for tasks involving \textit{half-cheetah}, \textit{hopper}, and \textit{walker2d}. 
In Section~\ref{sec:exp-d4rl}, to examine the impact of dataset quality, we rank trajectories based on episode returns within these nine datasets. 
Specifically, we compute the total reward for each data chunk within each dataset. 
We then rank these data chunks and select the top, middle, and bottom $30\%$ accordingly.
This is akin to rank trajectories but is easier to handle.

To better demonstrate the suboptimal nature of the datasets, we have plotted a histogram of the data chunk rewards, as shown in Fig.~\ref{fig:app-env-data-d4rl}.

\begin{figure}[h]
    \centering
    \includegraphics[width=0.8\linewidth]{fig/app-env-data-d4rl.png}
    \caption{Histogram of reward in D4RL datasets.}
    \label{fig:app-env-data-d4rl}
\end{figure}


\paragraph{RLBench Environment}
RLBench \cite{RLBench} serves as a benchmark and learning environment for robot control. 
We have selected 20 tasks from RLBench and present results for 6 of them in Sec.~\ref{sec:exp}. 
An illustration of the environment can be seen in Fig.~\ref{fig:app-env-rlb}. 
The input consists of RGB images with a resolution of 84 × 84, captured from four camera angles: front, wrist, left-shoulder, and right-shoulder, along with a history of the past seven observations. 
The output specifies the change in joint angles at each time step, utilizing the delta JointPosition mode provided by RLBench. 
In our experiments, we use a binary sparse reward system (0 or 1), which is awarded only at the final timestamp of an episode to indicate task success.

\begin{figure}[h]
    \centering
    \includegraphics[width=0.8\linewidth]{fig/app-env-rlb.png}
    \caption{Example of RLBench tasks used in experiment.}
    \label{fig:app-env-rlb}
\end{figure}



\subsection{Baselines and Evaluation Details}
\label{sec:app-baseline}

\paragraph{Main results baselines.}
As mentioned in Sec.~\ref{sec:exp-d4rl}, within D4RL, we utilize the implementation from \cite{CQN} and modify its CNN-based encoder to an MLP-based encoder as the CQN baseline. 
The BC baseline originates from CQN but operates with the RL learning objective turned off and without any online environment interaction.

In RLBench, we use the baselines and results of CQN, DrQ-v2+, DrQ-v2, ACT, and CBC as reported in the original CQN paper \cite{CQN}.

\paragraph{Ablation study baselines.}
As mentioned in Sec.~\ref{sec:exp-abl}, we utilize the \textit{Separate} and \textit{Level Shared} backbone baselines for an ablation study to explore the effectiveness of the shared backbone in the advantage network. 
The network architectures of these two baselines are illustrated in Fig.~\ref{fig:app-baseline-abl-nn-s} and Fig.~\ref{fig:app-baseline-abl-nn-ls}.

\begin{figure}[h]
    \centering
    \includegraphics[width=0.7\linewidth]{fig/5_med-nn-abl-s.png}
    \caption{Network architecture of \textit{Separate} backbone baseline in ablation study.}
    \label{fig:app-baseline-abl-nn-s}
\end{figure}

\begin{figure}[h]
    \centering
    \includegraphics[width=0.7\linewidth]{fig/5_med-nn-abl-ls.png}
    \caption{Network architecture of \textit{Level Shared} backbone baseline in ablation study.}
    \label{fig:app-baseline-abl-nn-ls}
\end{figure}



\section{Additional Results}
\label{sec:app-exp}

\paragraph{Sensitivity of temperature coefficient $\alpha$.}
Our methods are derived from Soft Q-learning, which aims to achieve a maximum-entropy policy.
The temperature coefficient $\alpha$ in Eq.~(\ref{eq:pre-soft-obj}) affects the balance between maximizing policy entropy and the reward from the environment. 
We conducted experiments to examine how varying $\alpha$ impacts policy learning. 

\begin{figure}[h]
    \centering
    \begin{subfigure}[t]{0.30\textwidth}
        \centering
        \includegraphics[width=\textwidth]{fig/6_exp-abl-alpha-d4rl.png}
        \label{fig:exp-abl-alpha-d4rl}
    \end{subfigure}
    \hspace{0.01\textwidth}
    \begin{subfigure}[t]{0.30\textwidth}
        \centering
        \includegraphics[width=\textwidth]{fig/6_exp-abl-alpha-rlb.png}
        \label{fig:exp-abl-alpha-rlb}
    \end{subfigure}
    \caption{Sensitivity of temperature coefficient $\alpha$ in D4RL and RLBench.}
    \label{fig:exp-abl-alpha}
\end{figure}

As shown in Fig.~\ref{fig:exp-abl-alpha}, a very high $\alpha$ results in reduced performance and unstable training, whereas a very low $\alpha$ also hampers policy improvement by restricting exploration.

\paragraph{D4RL results per task for different demonstration quality.}
In Sec.~\ref{sec:exp-d4rl}, we present the D4RL results, averaged over all 9 datasets, based on varying demonstration quality. 
The results for each task are illustrated in Fig.~\ref{fig:exp-d4rl-quality}. 
ARSQ consistently outperforms the CQN and BC baselines in nearly every task, demonstrating its ability to maintain stable performance across datasets of varying quality.

\begin{figure}[h]
    \centering
    \includegraphics[width=0.9\linewidth]{fig/6_exp-d4rl-quality.png}
    \caption{D4RL results per task on different demonstration quality.}
    \label{fig:exp-d4rl-quality}
\end{figure}

\paragraph{RLBench results in all 20 tasks.}
In Sec.~\ref{sec:exp-rlb}, we present results for six selected tasks from RLBench. 
The complete results for all 20 tasks are displayed in Fig.~\ref{fig:exp-rlb-task-all}. 
These results indicate that ARSQ performs comparably or better across these tasks, showcasing its ability to learn effectively even when the data collected online is not optimal.

\begin{figure}[h]
    \centering
    \includegraphics[width=0.95\linewidth]{fig/6_exp-rlb-full.png}
    \caption{RLBench results in all 20 tasks.}
    \label{fig:exp-rlb-task-all}
\end{figure}



\section{Computational Cost Analysis}
As discussed in Sec.~\ref{sec:med-alg}, ARSQ generates actions in each dimension in an auto-regressive manner. 
To analyze the overhead, we conducted experiments on both D4RL (\textit{hopper-medium}) and RLBench (\textit{Open Oven}) tasks. The training and inference times for ARSQ and CQN were evaluated 1,000 times and averaged. 
These experiments were conducted on a single Nvidia RTX 3090 graphics card.

The results are shown in Fig.~\ref{fig:exp-abl-compute}. 
ARSQ exhibits similar training times to CQN, due to the parallel optimization implemented and the batch training nature of the auto-regressive model. 
However, ARSQ experiences higher inference latency compared to CQN. 
We aim to address this issue by grouping the action dimensions and outputting the grouped dimensional actions auto-regressively, a solution we plan to explore in future work.

\begin{table}[h]
    \centering
    \begin{tabular}{c|cc}
        \toprule
         & D4RL & RLBench \\
        \midrule
        ARSQ Inference & 4.1 & 32.1 \\
        ARSQ Training & 12.2 & 290.5  \\
        CQN Inference & 2.6 & 6.9 \\
        CQN Training & 11.6 & 260.5 \\
        \bottomrule
    \end{tabular}
    \caption{Computational time in D4RL and RLBench (ms). }
    \label{fig:exp-abl-compute}
\end{table}

%%%%%%%%%%%%%%%%%%%%%%%%%%%%%%%%%%%%%%%%%%%%%%%%%%%%%%%%%%%%%%%%%%%%%%%%%%%%%%%
%%%%%%%%%%%%%%%%%%%%%%%%%%%%%%%%%%%%%%%%%%%%%%%%%%%%%%%%%%%%%%%%%%%%%%%%%%%%%%%


\end{document}


% This document was modified from the file originally made available by
% Pat Langley and Andrea Danyluk for ICML-2K. This version was created
% by Iain Murray in 2018, and modified by Alexandre Bouchard in
% 2019 and 2021 and by Csaba Szepesvari, Gang Niu and Sivan Sabato in 2022.
% Modified again in 2023 and 2024 by Sivan Sabato and Jonathan Scarlett.
% Previous contributors include Dan Roy, Lise Getoor and Tobias
% Scheffer, which was slightly modified from the 2010 version by
% Thorsten Joachims & Johannes Fuernkranz, slightly modified from the
% 2009 version by Kiri Wagstaff and Sam Roweis's 2008 version, which is
% slightly modified from Prasad Tadepalli's 2007 version which is a
% lightly changed version of the previous year's version by Andrew
% Moore, which was in turn edited from those of Kristian Kersting and
% Codrina Lauth. Alex Smola contributed to the algorithmic style files.
