\subsection{Phase I: Initial Die Assignment}
\label{sec:initial_assig}

This stage aims to evenly assign the given blocks $\mathcal{B}$ to the dies $\mathcal{D}$, such that the subsequent floorplanning may converge faster than a random initial solution.
Suppose there are $n$ blocks, $m$ dies with 
$k$ technologies, where $k \le m$. The area
of a block $b_i$ is denoted as $A(b_i, t_i, \rho_i)$, where $t_i$ is the technology assigned to $b_i$ and $\rho_i$ is its aspect ratio. 
The area $A(b_i, t_i, \rho_i)$ can be estimated using the machine learning (ML) model described in
Section~\ref{sec:ppa_models}. Note that the aspect ratio is a layout tool parameter and an input feature to the ML model. In the initial assignment, the aspect ratios of all blocks are temporarily set be 1.
The initial die assignment involves three steps, which are elaborated as follows.

\noindent{\bf Step 1: Average die area estimation.}
Assuming the $k$ technologies $g_1, g_2, ..., g_k$ are ordered from the oldest (largest) to the newest (smallest). The first step is to estimate the average die area for 
all dies in $\mathcal{D}$. In order to do so, we temporarily assign all blocks to the oldest technology $g_1$ and scale all newer technology dies to the area of $g_1$. For example, if $g_1=14$nm, then we scale a $7$nm die by a factor of $4$. This ensures that all blocks and all dies are normalized to the oldest technology. 
Let the area of the die in the oldest technology be a variable $z$. To ensure that all dies have approximately the same area in their own technologies, we enforce
\[
\sum_{i=1}^m s_i \cdot z = \sum_{i=1}^n A(b_i, g_1, 1)
\]
where $s_i$ indicates the scaling factor between a newer technology and $g_1$.
By solving this linear equation, we can determine the value of $z$, which represents the approximate equal area for all dies. The value of $z$ approximately can accommodate all blocks.

\noindent{\bf Step 2: Assigning blocks to dies.}
In step 2, all blocks are sorted in non-increasing order of their areas in the oldest technology with an aspect ratio if 1, i.e., $A(b_i, g_1, 1) \forall b_i \in \mathcal{B}$. Following this order, the blocks are assigned to dies one by one until the total block area of each die is approximately $z$. Note that when a block $b_i$ is assigned to a die with a newer technology $g_j$, its area becomes $A(b_i, g_j, 1)$.

%to inference the area of each block in the largest technology node and assuming that the aspect ratio is $1$. The blocks are ordered based on the area and the total area per die $A_{die}$ is estimated as the summation of each block divided by the number of dies $n_d$. Then, the blocks are assigned into $n_d$ subsets such that the estimated area within a die is close to $A_{die}$. We use a greedy-approach to find the $n_d$ subsets to determine the minimum number of blocks per subset. Once the subsets are determined, the technology assignment is performed by setting the small technology node to the subset with largest total area, and larger technology node with smaller total area. Thus, we ensure that very large blocks are assigned to small technology, reducing the area, and cost.

\noindent{\bf Step 3: Block aspect ratio refinement.}
In step 3, we determine the aspect ratio of each block to minimize the relevant part of the objective function, which is $\beta \cdot P + \tau \cdot T$. This aims to achieve good aspect ratios in terms of TNS and power. Although the floorplanning has not yet been performed, this provides a good starting point for both TNS and power.

%Next, we refine the subset by reassigning the aspect ratio of each block to improve TNS and Power. We query the PPA models to compute the objective $\beta_2 \times T+ \beta_4\times P$ for aspect ratios of $0.8$, $1$ and $1.2$ of each block in the corresponding technology node. The aspect ratio that minimize the objective is chosen for a given block.