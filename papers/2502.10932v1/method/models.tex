\subsection{Area, Cost and Wirelength Models}
\label{sec:models}

% \item We used machine learning models to predict \textbf{TNS, dynamic power and area}, which are key components in our workflow because of the rapid inference time. In~\cite{sengupta2022ppa}, feature extraction and vector-based machine learning models are introduced to predict TNS and dynamic power for Verilog RTL code and logic synthesis parameters. The feature extraction is based on the Abstract Syntax Tree (AST) which is a tree like representation of Verilog code hierarchy. Then, the register assignments are identified in the AST to compute features such as total input/output bits, total register bits, etc, retrieving a two dimensional feature vector. Next, we choose the XGBoost due to the better performance in terms of R2 coefficient and root mean squared error (RSME). Furthermore, the work of~\cite{sengupta2022ppa} is extended to area prediction using similar features set.

\subsubsection{Die Area}
The area $A(d_i)$ of a die $d_i \in \mathcal{D}$ is the area of the minimum bounding box enclosing its block floorplan, along with its margin area.

\subsubsection{Cost Model}
\label{sec:cost_model}
The manufacturing cost of a die is closely related to its manufacturing yield. According to~\cite{feng2022costModel,cunningham1990costModel}, a yield model for a single die is described by
\begin{equation}
  Y(d) = \left(1+ \frac{\delta\cdot A(d)}{\alpha} \right)^{-\alpha},
\end{equation}
where $d$ indicates a die, $A(d)$ is the die area, $\delta$ is the defect density, and $\alpha$ is a parameter in the underlying statistical model. For example, in~\cite{feng2022costModel}, $d$ has a value of $0.09$~cm$^{-2}$, and $\alpha$ of $10$ for the $7$nm technology. In our method, we consider only the manufacturing cost per yield area $C(d){=}\Phi/Y(d)$ as the cost model, where $\Phi$ is a technology-dependent constant. As shown in~\cite{feng2022costModel}, an old technology has a lower cost per area than a relatively new technology. However, a circuit implemented with a new technology requires a much smaller area and, hence, significantly lower cost.
In this work, we focus on the die cost, while packaging cost
and inter-die interconnect cost are not explicitly included in the objective function. However, they are partially addressed, as packaging cost is approximately proportional to total die cost, and the inter-die interconnect cost is partly captured by the inter-die wirelength in our objective function. Please note that MMFP framework can easily accommodate different cost models. For example, a circuit in an old technology may have pre-designed hard IPs and therefore its design cost can be lower in the old technology.


\subsubsection{Wirelength Model}
A net $e$ is a subset of blocks $e\subseteq \mathcal{B}$.
The HPWL (Half-Perimeter Wire-Length) of net $e$ is defined as
\begin{equation}
  W_e = \max_{b_i \in e}{x_i}-\min_{b_i \in e}{x_i} + \max_{b_i \in e}{y_i}-\min_{b_i \in e}{y_i},
\end{equation}
where $x_i$ and $y_i$ are the horizontal and vertical coordinates of the center of block $b_i$, respectively. The total HPWL includes both intra-die nets $E_\text{intra-die}$ and inter-die nets $E_\text{inter-die}$ as
\begin{equation*}
  W =\sum_{e \in E_\text{intra-die}} W_e + \sum_{e \in  E_\text{inter-die}} W_e.
\end{equation*}
