\subsection{Phase III: Inter-Die Refinement}
\label{sec:refinement}


After every $K$ moves of SA (or $K$ actions of RL), an inter-die refinement is performed. An inter-die refinement randomly selects a block $b$ and moves it from its current
die $d_i$ to another die $d_j$ if such a move does not violate the die area constraints.
Block $b$ becomes a new node to be inserted into the B*-tree of die $d_j$, and the insertion is performed in a way such that $b$ is near die $d_i$. 

There is a sweet spot in choosing the interval $K$. When $K$ is too small, intra-die floorplanning may be frequently interrupted before a high quality solution has been reached. If the value of $K$ is too large, the objective function improvement may have been saturated for a long time before the next inter-die refinement, and therefore, significant computation is wasted. 

An alternative approach is to treat the inter-die refinement as an SA move or RL action. This is equivalent to interleaving the inter-die refinement with intra-die floorplanning after a random number of moves/actions. Since the random number can often deviate from the sweet spot, it is conceivably better to have inter-die refinement separate, as in our current scheme. 

%The refinement of the floorplan solution uses a variation of Kernighan-Lin (KL) algorithm and allows to further improve TNS and power while ensuring the constraint. In the refinement, blocks that have and inter-die connection are swapped between die $i$ and $j$. In the KL formulation, the vertices are blocks that have inter-die connection $B_i$ in die $i$ and $B_j$ in die $j$ and the cost function of $\beta_2 \times T + \beta_4 \times P$ for the blocks in $B_i$ and $B_j$. If $|B_i|$ and $|B_j|$ sets are unbalanced, dummy blocks are inserted to balance the size of sets. Dummy blocks are set with no connections and cost function of $0$. We choose the solution that minimized the cost function and meet the maximum net constraint. The refinement is performed every $K$ steps during the intra-die floorplanning to not get stuck in a local minima solution.

%Once the SA/RL approach is completed, the heterogeneous integration is performed with the die floorplan solutions. The margin space between dies requires a minimum distance $t_{min}$ and is limited to the maximum net constraint $net_{max}$, setting the maximum distance $t_{max}$. We model the margin space as a linear function of the number of nets, where the $net_{max}$ is a parameter that is chosen.