\subsection{Experiment Setup}
\vspace{-5mm}

\begin{table}[H]
  \centering
  \caption{Designs for Optimization}
  \begin{tabular}{ crr }
    \hline
    Design & \# Cells & \# Circuit blocks \\
    \hline
    vga\_lcd & 56,031 & 15 \\
    OpenPiton & 435,987 & 28 \\
    leon3mp & 374,583 & 50 \\
    netcard & 346,592 & 60 \\
    leon2 & 513,894 & 80 \\
    leon3-avnet & 636,509 & 100 \\
    \hline
  \end{tabular}
  \label{tab:designs}
\end{table}
\vspace{-3mm}


\begin{table*}[!ht]
\centering
\caption{PPAC optimization results in 2 silicon dies (one with 7nm, the other with 45nm).}
\begin{tabular}{clrrrrrrr}
\hline
\multirow{2}{*}{Design} & \multirow{2}{*}{Method} & Area & HPWL & Cost & \multicolumn{2}{c}{Timing (ns)} & Power & CPU \\
\cline{6-7}
& & ($\times10^3 \mu$m$^2$) & ($\mu$m) & ($\times10^{-3}$) & TNS & WNS & (mW) & (sec) \\
\hline
\hline
\multirow{3}{*}{vga\_lcd} & Baseline & 92.47 & 230 & 2120 & -55.16 & -0.441 & 153.0 & 119 \\
& MMFP-SA & 87.19 & 211 & 2024 & -20.96 & -0.151 & 141.2 & 143 \\
& MMFP-RL & 85.40 & 205 & 2005 & -19.01 & -0.147 & 138.5 & 372 \\
\hline
\multirow{3}{*}{OpenPiton} & Baseline & 371.92 & 5701 & 2551 & -280.51 & -0.878 & 596.5 & 460 \\
& MMFP-SA & 355.19 & 5418 & 2518 & -259.28 & -0.867 & 580.2 & 529 \\
& MMFP-RL & 335.08 & 5208 & 2447 & -232.91 & -0.845 & 557.1 & 758 \\
\hline
\multirow{3}{*}{leon3mp} & Baseline & 201.49 & 3893 & 2481 & -212.02 & -1.794 & 583.3 & 904 \\
& MMFP-SA & 188.18 & 3610 & 2360 & -182.51 & -1.491 & 543.7 & 1058 \\
& MMFP-RL & 182.40 & 3402 & 2276 & -175.09 & -1.351 & 540.5 & 1102 \\
\hline
\multirow{3}{*}{netcard} & Baseline & 207.72 & 4610 & 2505 & -241.95 & -1.628 & 598.2 & 935 \\
& MMFP-SA & 195.09 & 4182 & 2419 & -225.98 & -1.493 & 559.3 & 961 \\
& MMFP-RL & 190.46 & 4016 & 2380 & -213.59 & -1.460 & 541.1 & 1018 \\
\hline
\multirow{3}{*}{leon2} & Baseline & 519.08 & 10583 & 2620 & -495.31 & -1.156 & 790.7 & 1473 \\
& MMFP-SA & 484.19 & 9682 & 2499 & -451.94 & -1.104 & 726.6 & 1508 \\
& MMFP-RL & 472.96 & 9309 & 2429 & -439.31 & -1.092 & 723.8 & 1305 \\
\hline
\multirow{3}{*}{leon3-avnet} & Baseline & 802.88 & 12986 & 2801 & -792.20 & -1.197 & 1106.0 & 1759 \\
& MMFP-SA & 750.81 & 11782 & 2718 & -755.72 & -1.133 & 1035.9 & 1891 \\
& MMFP-RL & 736.02 & 11307 & 2684 & -732.53 & -1.129 & 1023.6 & 1498 \\
\hline
\hline
\multirow{3}{*}{Norm. Average} & Baseline & 1 & 1 & 1 & 1 & 1 & 1 & 1 \\
& MMFP-SA & 0.940 & 0.920 & 0.964 & 0.828 & 0.830 & 0.936 & 0.884 \\
& MMFP-RL & 0.912 & 0.883 & 0.943 & 0.783 & 0.806 & 0.919 & 0.800 \\
\hline
\end{tabular}
\label{tab:2die_optimization}
\end{table*}



The testcases are synthesizable HDL code for 5 circuit designs from the IWLS 2005 benchmarks~\cite{albrecht2005iwls} and a RISC-V-based multi-core system OpenPiton~\cite{balkind2016Openpiton}. The OpenPiton system is configured as a 2$\times$2 processor with default parameters according to the user manual. 
Each design is divided into a number of circuit blocks based on its design hierarchy using Synopsys Design Compiler. 
The number of cells and blocks for all designs are summarized in Table~\ref{tab:designs}. Two public-domain technologies are used in the experiments: 45nm~\cite{Stine2007freepdk45} and 7nm~\cite{Vashishtha2017asap7}.

%For each design, the HDL hierarchy is divided into the number of blocks described in Table~\ref{tab:designs} through Synopsys Design Compiler, obtaining the interconnected blocks $\mathcal{B}$. 

%The average number of cells per HDL block is no more than $7,500$, making these blocks suitable for the machine learning models in Section~\ref{sec:ppa_models}.

In the experiments, %the set of HDL blocks for the designs in Table~\ref{tab:designs} are employed to evaluate 
the following three floorplanning techniques are compared.
\begin{itemize}
\item \textbf{Baseline} is a na\"ive approach that employs hMetis partitioning~\cite{harypis1997HMetis} to divide the interconnected blocks $\mathcal{B}$ into $|\mathcal{D}|$ subsets. Each subset is then randomly assigned to a silicon die in $\mathcal{D}$. 
Simulated annealing-based floorplanning is performed using the same 
objective functions as our MMFP. 
Please note that hMetis is formulated to minimize the cut sizes among partitions. 
%The aspect ratio of all blocks is set to $1$.

\item \textbf{MMFP-SA} is our MMFP method that uses SA optimization during intra-die floorplanning. In the SA implementation, the initial temperature is set to $400$, the cooling factor to $0.85$, and convergence stopping criteria to $10^{-4}$.

\item \textbf{MMFP-RL} is our MMFP method that uses RL optimization during intra-die floorplanning. In the PPO implementation, the hyperparameter $\lambda$ is set to $0.2$, the discount factor $\eta$ to $0.95$, the height $h$ for feature selection is $6$ and the stopping criteria is $10^{-4}$.
\end{itemize}

In the experiments, the weights $\omega, \beta, \gamma, \tau$ in the objective function are set to $1, 1, 0.5$ and $2$ by default respectively.
The upper bound of inter-die nets $N_{max}$ is $30$, the bounds $A_{min}$ and $A_{max}$ are $0.8z$ and $1.2z$, respectively, where $z$ is the average die area. The number of SA moves (or RL actions) between inter-die refinements $K$ is $20$.
The total number of SA moves (or RL actions) plus inter-die refinements is constrained to no greater than $2,500$. 

The MMFP methods are implemented in the Python programming language. The PPO algorithm uses Spinning Up~\cite{achiam2018spinningup} and Gymnasium~\cite{towers2024gymnasium} libraries to set and interact with the RL environment. Once an optimized MMFP solution is obtained, the block floorplan, die assignment and aspect ratio of blocks are used to perform logic synthesis and placement for each block at a frequency of $400$ MHz using Synopsys Design Compiler and Cadence Innovus respectively. 
% Then, the 2.5D integration is achieved using Cadence Integrity.
The reported timing and power results are obtained from Cadence Innovus post-placement analysis.
The experiments were conducted on a Intel Core i7-1065 CPU 1.3GHz with 16GB RAM.