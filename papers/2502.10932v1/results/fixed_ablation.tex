\subsection{MMFP-RL with Hard IPs}

Table~\ref{tab:fixed_ips} shows the results for the \emph{netcard} design when the number of dies $\mathcal{D}$ is 2, with some circuit blocks are set as hard IPs in the 45nm node (fixed technology and aspect ratio). The results show that MMFP-RL with hard IPs still optimizes the floorplan solution compared to the baseline. MMFP-RL achieves average reductions of 11.8\%, 14.4\% and 7.3\% in area, HPWL and cost, respectively. In terms of performance and power saving, MMFP-RL improves TNS by 18.4\% and power by 12.1\%.

% with a slight overhead compared to MMFP-RL. The averaged overhead is 3.5\%, 2.4\% and 1.8\% in area, cost and power respectively. Moreover, MMFP-RL with hard IPs has an HPWL overhead of 1.6\% compared to MMFP-RL, while power overhead of 1.1\%.

\vspace{-4mm}
\begin{table}[!ht]
\centering
\caption{PPAC optimization results with hard IPs for netcard design in 2 silicon dies (one with 7nm, the other with 45nm).}
\begin{tabular}{clrrrrr}
\hline
Hard & \multirow{2}{*}{Method} & \multirow{2}{*}{Area} & \multirow{2}{*}{HPWL} & \multirow{2}{*}{Cost} & \multirow{2}{*}{TNS} & \multirow{2}{*}{Power} \\
IPs & & & & & & \\
\hline
\hline
\multirow{2}{*}{5} & Baseline & 214.04 & 4692 & 2581 & -250.10 & 609.2 \\
& MMFP-RL & 192.98 & 4036 & 2405 & -214.28 & 545.7 \\
\hline
\multirow{2}{*}{10} & Baseline & 223.17 & 4758 & 2624 & -262.33 & 618.9 \\
& MMFP-RL & 197.04 & 4069 & 2437 & -216.02 & 549.3 \\
\hline
\multirow{2}{*}{15} & Baseline & 234.31 & 4841 & 2682 & -283.34 & 653.0 \\
& MMFP-RL & 201.57 & 4131 & 2469 & -217.50 & 557.2 \\
\hline
\hline
\multicolumn{2}{c}{Norm. Avg.} & 0.882 & 0.856 & 0.927 & 0.816 & 0.879 \\
\hline
\end{tabular}
\label{tab:fixed_ips}
\end{table}
