\subsection{Results on PPAC Optimization with 2 Silicon Dies}
Table~\ref{tab:2die_optimization} shows the results for the three techniques: Baseline, MMFP-SA and MMFP-RL when the number of dies $\mathcal{D}$ is 2 (one die with 7nm, and the other with 45nm). The results indicate that MMFP-SA achieves average reductions of 6\%, 8\% and 3.6\%, while MMFP-RL achieves reductions of 8.8\%, 11.7\% and 5.7\% in area, HPWL and cost, respectively, compared to the baseline. In terms of post-placement TNS, MMFP-SA and MMFP-RL show average improvements of 17.2\% and 21.7\% respectively. Moreover, MMFP-SA and MMFP-RL achieve average savings in dynamic power of 6.4\% and 8.1\% respectively. In terms of CPU runtime, MMFP-RL is on average 0.8$\times$ slower, and MMFP-SA is 0.9$\times$ slower than the baseline. However, MMFP-RL is faster than MMFP-SA and the baseline as the number of interconnected blocks $\mathcal{B}$ increases. Figure~\ref{fig:objective_iteration} shows the objective function $f$ across iterations during optimization on the \emph{netcard} design. MMFP-RL requires $293$ fewer iterations than MMFP-SA and achieves a better objective value.

\begin{figure}[ht] 
\centering
\includegraphics[width=.95\linewidth]{figures/objective_iterations1.png}
\caption{Objective function $f$ value for MMFP-SA/RL across iterations.}
\label{fig:objective_iteration}
\end{figure}



\subsection{Results on PPAC Optimization with 4 Silicon Dies}
%In heterogeneous integration, the number of dies $\mathcal{D}$ is typically more than 2. 
Table~\ref{tab:4die_optimization} shows the results for the \mbox{\emph{leon3-avnet}} design when the number of dies $\mathcal{D}$ is 4. The results show that MMFP-SA achieves average reductions of 7.2\%, 6.5\% and 3.2\% in area, HPWL and cost, respectively, compared to the baseline. MMFP-RL further improves the reductions achieving 9.7\%, 9\% and 4.3\%. Post-placement TNS is improved by 7.3\% for MMFP-SA, and 11.1\% for MMFP-RL. Furthermore, MMFP-SA achieves 7\% power savings, while MMFP-RL achieves 8.6\%. In terms of CPU runtime, MMFP-RL is 1.19$\times$ faster than the baseline, while MMFP-SA is 0.95$\times$ slower.

\begin{table*}[!ht]
\centering
\caption{PPAC optimization results for leon3-avnet design in 4 silicon dies.}
\begin{tabular}{cclrrrrrrr}
\hline
\multicolumn{2}{c}{\# dies} & \multirow{2}{*}{Method} & Area & HPWL & Cost & \multicolumn{2}{c}{Timing (ns)} & Power & CPU \\
\cline{1-2} \cline{7-8}
7nm & 45nm & & ($\times10^3 \mu$m$^2$) & ($\mu$m) & ($\times10^{-3}$) & TNS & WNS & (mW) & (sec) \\
\hline
\hline
\multirow{3}{*}{1} & \multirow{3}{*}{3} & Baseline & 1409.29 & 17495.01 & 3173 & -1253.06 & -1.948 & 1397.8 & 2104 \\
& & MMFP-SA & 1320.57 & 16029.53 & 3086 & -1171.95 & -1.872 & 1320.6 & 2308 \\
& & MMFP-RL & 1296.30 & 15702.68 & 3069 & -1128.43 & -1.830 & 1315.1 & 1963 \\
\hline
\multirow{3}{*}{2} & \multirow{3}{*}{2} & Baseline & 831.06 & 13972.70 & 2880 & -810.57 & -1.304 & 1150.3 & 2376 \\
& & MMFP-SA & 759.25 & 13295.18 & 2755 & -752.86 & -1.258 & 1051.3 & 2450 \\
& & MMFP-RL & 738.14 & 12960.91 & 2716 & -719.25 & -1.230 & 1016.3 & 2003 \\
\hline
\multirow{3}{*}{3} & \multirow{3}{*}{1} & Baseline & 590.71 & 9713.55 & 2674 & -603.74 & -1.071 & 674.9 & 2502 \\
& & MMFP-SA & 552.05 & 9101.84 & 2603 & -552.83 & -0.994 & 628.1 & 2581 \\
& & MMFP-RL & 531.92 & 8782.73 & 2568 & -530.06 & -0.976 & 619.3 & 1924 \\
\hline
\hline
& & Baseline & 1 & 1 & 1 & 1 & 1 & 1 & 1 \\
\multicolumn{2}{c}{Norm. Avg.}  & MMFP-SA & 0.928 & 0.935 & 0.968 & 0.927 & 0.951 & 0.930 & 0.950 \\
& & MMFP-RL & 0.903 & 0.910 & 0.957 & 0.889 & 0.931 & 0.914 & 1.186 \\
\hline
\end{tabular}
\label{tab:4die_optimization}
\end{table*}