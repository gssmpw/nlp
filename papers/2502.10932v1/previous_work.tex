\section{Previous Related Works}
\label{sec:previousWorks}
The physical design challenges in heterogeneous integration are discussed in~\cite{chang2024multidieChallenges}, that also presents the problem of multi-die and multi-technology floorplanning. 
A previous work~\cite{ho2013btreeSA} focuses on interposer-based chiplet floorplanning that performs a simulated annealing (SA) optimization to minimize wirelength and area. However, a drawback is that the number of chiplets is limited, and the approach requires a large runtime.
In~\cite{lee2023multiDieFloorplan}, a SA-based methodology is proposed to also optimize wirelength and area, while considering the multi-die interconnect bridge. Despite improvements in the objectives and runtime, there is a lack of internal die floorplanning in the formulation.
In~\cite{hsu2022warpageFloorplan}, a heterogeneous floorplanning method, that performs SA optimization, is proposed to address warpage as a potential issue during the packing process.
The work of~\cite{ma2021thermalFloorplan} introduces a thermal-aware chiplet floorplanning approach that employs thermal simulators and perform SA optimization to minimize operating temperature and wirelength. Nevertheless, the experiments are limited to 8 chiplets and rely on the premise of safe scalability.
In~\cite{osmolovskyi2018diePlacement}, a wirelength-driven chiplet placement is proposed that utilizes a constraint-satisfaction problem formulation and performs branch-and-bound-based optimization. The optimization is conducted by exploring the solution space and smartly pruning unpromising solutions.
% In~\cite{kim2019cowos25d}, a interposer-based chiplet place-and-route flow is automated using only commercial tools in the EDA. Even though the integration of tools allows to perform interposed-based design, the flow achieves 2.1x PPA overhead compared to regular 2D ICs.
Later, GoodFloorplan formulates the floorplanning problem as a Markov Decision Process (MDP), enabling the use of reinforcement learning (RL) frameworks integrated with graph convolutional networks~\cite{xu2022goodfloorplan}. GoodFloorplan outperforms SA-based methods in terms of area and wirelength optimization. %they use sequence pair (SP) as floorplan representation that increases time complexity.
In~\cite{amin2024RLFloorplan3D}, a RL-based framework with a decision transformer is introduced to optimize wirelength, congestion, and heat using 3D Manhattan distance and density kernels. The decision transformer allows to prompt desired objective values.
The work of~\cite{duan2024RLPlanner} also presents an RL framework, RLPlanner, that optimizes wirelength and temperature. RLPlanner is inspired in~\cite{ma2021thermalFloorplan} and incorporates a fast thermal evaluation module, which is a physics-informed model, that provides speedup during optimization.
In~\cite{chen2024floorplet}, Floorplet is presented as a performance-aware floorplanning method that uses yield, warpage and bump stress models to optimize wirelength, warpage and packing cost. Floorplet also integrates simulation tools and the yield model introduced in~\cite{feng2022costModel,cunningham1990costModel} to perform optimization, which is formulated as mathematical programming (MP) problem. The experiments show improvements on packing cost, wirelength and latency.
Similarly, the work of~\cite{zhuang2022multipackage} also uses an MP formulation for multi-package co-design integration that optimizes wirelength, warpage, bump stress, and interconnection cost while maintaining non-overlapping and bump margin constraints. Nevertheless, the MP formulation requires large runtime to achieve optimized solutions.
Overall, the floorplanning techniques lack to explicitly address power consumption and timing performance. While the SA and RL approaches have proven to be feasible for the floorplanning problem.
