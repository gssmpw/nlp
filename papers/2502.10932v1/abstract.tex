\begin{abstract}
%The floorplanning is an early stage in the electronic design automation (EDA), and very important as the floorplan solution heavily influence the timing performance, power consumption and cost (PPAC) of a chip in the downstream flow. However, the floorplanning becomes exponentially complex as the number of functional blocks increases, affecting the PPAC metrics. The 2.5D integration became a solution to improve PPAC and reuse of IP blocks by using dies of multiple technology nodes. In this work, we propose a multi-die assignment and die floorplanning workflow driven by PPAC models and powered by simulated annealing or reinforcement learning engines.The results shows that our framework outperforms the baseline and provides a good quality 2.5D floorplan solution.
In heterogeneous integration, where different dies may utilize distinct technologies, floorplanning across multiple dies inherently requires simultaneous technology selection. This work presents the first systematic study of multi-die and multi-technology floorplanning. Unlike many conventional approaches, which are primarily driven by area and wirelength, this study additionally considers performance, power, and cost, highlighting the impact of technology selection. A simulated annealing method and a reinforcement learning techniques are developed. Experimental results show that the proposed techniques significantly outperform a na\"ive baseline approach.
\end{abstract}