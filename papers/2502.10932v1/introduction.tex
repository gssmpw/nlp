\section{Introduction}
\label{sec:introduction}


% what is the problem you want to solve, in big picture from application point of view (not detailed problem formulation)
Heterogeneous integration (2.5D and 3D-IC) opened up opportunities for building complex IC designs into a single chip with applications in high-performance computing, 5G technology, artificial intelligence, etc. One key aspect of heterogeneous integration compared to monolithic IC designs is that multiple dies of different manufacturing process technologies are integrated into a single system, allowing the re-usability of already designed IPs that are otherwise difficult to redesign in a smaller technology. Heterogeneous integrations also achieve enhanced functionality, compact area and design flexibility. 
%However, thermal reliability, interconnection, and yield manufacture come to be challenging when combining multiple technologies. 

Multi-die floorplanning plays a critical role in determining die area, global interconnect, thermal and warpage, which have been the main focus of existing methods. %~\cite{ho2013btreeSA,hsu2022warpageFloorplan,ma2021thermalFloorplan,osmolovskyi2018diePlacement,amin2024RLFloorplan3D,duan2024RLPlanner,chen2024floorplet}.
In~\cite{ho2013btreeSA}, a simulated annealing-based multi-die floorplanning technique is proposed for minimizing area and wirelength with consideration of IO assignment. A floorplanning that considers multi-die interconnect bridge assignments is proposed in~\cite{lee2023multiDieFloorplan} based on simulated annealing, targeting bounding area and wirelength. A die placement work is introduced in~\cite{osmolovskyi2018diePlacement} for minimizing wirelength through a branch-and-bound approach. A thermal-driven die placement technique based on simulated annealing is introduced in~\cite{ma2021thermalFloorplan}.
Another die placement work~\cite{hsu2022warpageFloorplan} is mainly targeted to addressing the warpage issue. A thermal-driven chiplet floorplan using reinforcement learning is reported in~\cite{duan2024RLPlanner}.
A reinforcement learning approach to 3D floorplanning
is proposed in~\cite{amin2024RLFloorplan3D} for wirelength, routability and thermal optimization. The work of~\cite{chen2024floorplet} considers warpage, cost and performance as objectives during the floorplanning by proposing a more elaborated methodology based on a mathematical programming formulation. Even so, power consumption and technology selection is left aside. 


In heterogeneous integration, assigning a circuit block to different dies often implies simultaneous selection of different technologies. As a result, a circuit block may exhibit significantly different performance, power, and area characteristics depending on the die it is placed on. While the challenges of multi-die and multi-technology floorplanning have been acknowledged in~\cite{chang2024multidieChallenges}, to the best of our knowledge, little to no prior research addressing this problem comprehensively.

%More importantly, cross-phisical floorplanning, and multi-technology assignment of IPs/components are issues in the physical domain of the integration. The floorplanning heavily determine the performance, power and area (PPA) objectives of a monolithic chip regardless downstream optimizations. For instance, a poor floorplan solution would lead to complex routing and increased manufacturing costs as often large dead space. Similarly, heterogeneous floorplanning requires adequate multi-die and multi-technology assignment while meeting the performance, power, area and cost (PPAC) constraints.



%Thus, a multi-die and multi-technology floorplanning is important to further leverage the heterogeneous integration.


%For heterogeneous floorplanning, there are several works~\cite{ho2013btreeSA,hsu2022warpageFloorplan,ma2021thermalFloorplan,osmolovskyi2018diePlacement,xu2022goodfloorplan,amin2024RLFloorplan3D,duan2024RLPlanner,chen2024floorplet} that address distinct issues while optimizing metrics such as die area, wirelength, thermal stress and performance. The main approach to the floorplan problem is the simulated annealing algorithm that applies a move in the floorplan representation to eventually find an optimal solution~\cite{ho2013btreeSA,hsu2022warpageFloorplan,osmolovskyi2018diePlacement}. Nevertheless, these works only handle an objective separately, i.e. wirelength/warpage. Reinforcement learning approaches improved the quality of floorplan solutions, while handling more objectives~\cite{xu2022goodfloorplan,amin2024RLFloorplan3D,duan2024RLPlanner} and usually performs better than SA-based methods. However, these works are limited to a single technology node or fixed multi technology selection. On the other hand, 

% highlight your main contributions A multi objective optimization is frequently complex, while also performing the die and technology assignments of IPs/components. 
%For given HDL code of interconnect components, silicon dies and technologies nodes for assignment, our MMFP performs an initial assignment, a intra-die floorplan, and refinement of solution, while explicitly targeting PPAC objectives. Moreover, SA and RL approaches are proposed for the intra-die floorplanning, and integrates ML-based models for PPA estimation from previous work. 


In this work, we present a methodology for multi-die and multi-technology floorplanning (MMFP). Our approach optimizes multiple objectives, including performance (measured by total negative slack), power, area, die cost, and total wirelength, accounting for both intra-die and inter-die connections. The input to our MMFP can accommodate both soft IPs in synthesizable HDL code and hard IPs with layouts.
A notable feature of our MMFP is the use of recent machine learning techniques~\cite{sengupta2022ppa} for technology-specific PPA (Performance, Power, Area) estimation of circuit blocks. Two optimization techniques are studied: simulated annealing and reinforcement learning. Experimental results demonstrate that our MMFP consistently outperforms a na\"ive method across all objectives.

The key contributions of this work are summarized as follows:
\begin{itemize}
  \item To the best of our knowledge, this is the first study on multi-die and multi-technology floorplanning.
  \item Two optimization techniques are studied, simulated annealing and reinforcement learning.
  %Our proposed methods integrates: 1) ML-based models for rapid total negative slack, power consumption and area estimation. 2) Analytical model for die cost. 3) SA/RL approaches for intra-die floorplanning.
  \item We demonstrate that the concurrent technology selection and its impact on circuit PPA can be effectively addressed by leveraging a recent ML technique. 
  %Our multi-die and multi-technology explicitly optimize PPAC that is not consider in previous works.
  \item Experimental results based on post-placement analysis using commercial tool show that our RL method outperforms a na\"ive method by 21.7\% in TNS, 8.1\% in power, 12\% in wirelength, 8.8\% in area, and 5.7\% in cost. 
  \item Ablation study results confirm that our MMFP method achieves different PPAC tradeoffs and accommodate both soft and hard IPs. 
  %While, our SA methods reach slightly less improvements.
\end{itemize}
Our future studies will additionally consider thermal and warpage issues. 
The rest of this paper is organized as follows. Previous related works are briefly reviewed in Section~\ref{sec:previousWorks}. The background knowledge relevant to our work is presented in Section~\ref{sec:background}. Section \ref{sec:formulation} provides the problem formulation. Our MMFP techniques are described in Section~\ref{sec:method}. Experimental results are covered in Section~\ref{sec:experiments}. Finally, Section~\ref{sec:conclusions} presents the conclusions. 

% highlight the key results, better numerically.
