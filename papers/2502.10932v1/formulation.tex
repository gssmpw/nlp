\section{Problem Formulation}
\label{sec:formulation}

% Given a set of blocks, our method find the partitioning and floorplan such that PPAC is minimized. Post-placement PPA is calculated using ML models for 45nm and 7nm~\cite{sengupta2022ppa}. In the floorplan, each block is assigned to a die of either 7nm or 45nm to satisfy the PPA goal. We use the btree representation~\cite{chang2000btree} and HMetis partitioning~\cite{harypis1997HMetis} and the cost model~\cite{cunningham1990costModel}

\noindent 
{\bf MMFP} (multi-die and multi-technology floorplanning): 
Given a circuit system composed of a set of interconnected blocks $\mathcal{B}$ described by synthesizable HDL code, aspect ratio options for each block, a set of technologies 
$g_1, g_2, ...,g_k$,
and a number of silicon dies $\mathcal{D}$, each having a distinct technology, MMFP assigns blocks to the given dies and decides their locations on the dies along with their aspect ratios to 
\begin{align}
 \text{minimize}~~ & f = \omega \cdot W + \beta \cdot P+\gamma \cdot \sum_{d_i \in \mathcal{D}} C(d_i) + \tau \cdot T \label{eq:objective}\\
 \text{subject to}~~ & N_{i,j} \le N_{max},~~ d_i, d_j \in \mathcal{D}, i\ne j  \\
  & A_{min} \le A(d_i) \le A_{max},~~ d_i \in \mathcal{D},  \label{eq:area_constraint}
\end{align}
where $W$ is the total inter-block Half-Perimeter Wire-Length (HPWL), $P$ is the total dynamic power of all blocks, $C(d_i)$ is the cost of die $d_i$ based on its yield model, $T$ is the total negative slack of all the blocks, $N_{i,j}$ is the number of nets between dies $d_i$ and $d_j$, $A(d_i)$ is the area of die $d_i$, and $\omega$,
$\beta$, $\gamma$, $\tau$, $N_{max}$, $A_{min}$ and $A_{max}$ are constant parameters. The HPWL $W$ assumes that the pins of a block are at its center, as in many previous works on floorplanning, and it includes inter-die wirelength. Area is considered in the constraints instead of the objective function, as it is correlated with $W$, $C$ and $P$. Please note that power $P$ and total negative slack $T$ are primarily determined by the technology selection. This correlation is confirmed in our experimental study.
This formulation is targeted to 2.5D-based heterogeneous integration using interposers. However, it is applicable to multi-die InFO packaging~\cite{feng2022costModel} and can be easily extended to 3D integration. 

\noindent
{\bf MMFP with hard IPs.} A special case of MMFP occurs when some circuit blocks are hard IPs, and therefore their technologies and aspect ratios are fixed throughout the optimization. This scenario is common in practice, for example, the complete design (including layout) of a circuit block exists for an old technology and can be reused in heterogeneous integration. 
