\section{Challenges and Motivation}
\label{sec:app}

Despite its speed and fidelity advantages for completing a single query, BB QRAM is not capable of processing multiple queries in parallel. This limitation is intrinsic to the binary tree structure of the QRAM architecture. For example, the $0^{\text{th}}$ address qubit is routed into the tree and occupies the root node for the entire duration of the query. In a binary tree structure, the root node serves as the sole escape route (i.e., external interface) through which every address qubit must pass. Consequently, all queries must be queued and executed sequentially.

In a shared memory system, as illustrated in Fig.~\ref{fig:motivation}, a BB QRAM inevitably leads to \emph{resource contention}. When $p$ parallel processes attempt to query the shared memory, BB QRAM must to execute them sequentially. This lack of query parallelism leads to a total query latency of $O(p \log(N))$, potentially causing a slowdown in quantum algorithms. Motivated by advancements in parallel computing and networking in classical literature, we propose an alternative router-based QRAM architecture based on a Fat-Tree structure, similar to the Fat-Tree network initially proposed by Charles Leiserson in 1985 \cite{leiserson1985fat}. 
Indeed, Fat-Tree QRAM routes differently than a classical Fat-Tree network~\cite{leiserson1985fat}, despite their similarity in geometry. A useful conceptual picture is that qubits are routed from root to leaf in QRAM, as opposed to communicating among leaf memory cells. With only a moderate (i.e., small constant factor) increase in the number of qubits in quantum routers at the higher levels of the tree, we can pipeline multiple queries simultaneously, offering immense parallelism benefits to a shared memory system.
