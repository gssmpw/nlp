\section{Introduction}

Many quantum algorithms for solving classically intractable problems assume that a large classical or quantum memory can be queried in superposition. Bucket-Brigade Quantum Random Access Memory (BB QRAM)~\cite{giovannetti2008quantum} is a promising candidate for realizing such queries efficiently, achieving desirable (poly-)logarithmic scalings in query latency and infidelity relative to the memory size~\cite{hann2021practicality}.  Recent resource estimates have revealed that QRAM's utility in quantum algorithms varies depending on the input data size and algorithmic speedup. For instance, a quadratic speedup in Grover's algorithm~\cite{grover1996fast} for database search is insufficient to realize a practical quantum advantage~\cite{jaques2023qram, hoefler2023disentangling}. However, QRAM remains central to enabling quantum advantages in many algorithms like the qubitization algorithm for chemistry simulation~\cite{lee2021even, berry2019qubitization, van2020convex}, Harrow-Hassidim-Lloyd algorithm for solving systems of equations and machine learning~\cite{biamonte2017quantum, harrow2009quantum}, and variants of Shor's algorithm for prime factorization~\cite{gidney2021factor, shor1994algorithms}. 

Running these quantum algorithms is challenging due to their demanding resource requirements, including large numbers of qubits with long coherence times. Specifically, these algorithms are inherently sequential---they make serial QRAM queries and consequently require deep circuits. These challenges can be alleviated through a parallel processing approach. Motivated by the ubiquitous use of parallelism in classical computation, numerous parallel quantum algorithms have recently emerged. Examples include distributed variational quantum eigensolver (VQE) \cite{niu2023parameter}, distributed Shor's algorithm \cite{meter2006architecture}, distributed quantum phase estimation (QPE) \cite{liu2021distributed, ang2022architectures}, parallel quantum walk \cite{zhang2024parallel}, and parallel quantum signal processing (QSP) \cite{martyn2024parallel}. The success of many parallel algorithms critically depends on high-bandwidth QRAM capable of supporting simultaneous queries. 


\begin{figure}
    \centering
    \includegraphics[width=\linewidth]{Figures/motivation.pdf} % Placeholder for now
    \caption{(a) Architectural schematics of a shared QRAM that is concurrently accessed by multiple QPUs. (b) Cost comparison between Fat-Tree and Bucket-Brigade (BB) QRAMs for executing $O(\log(N))$ independent queries. The proposed Fat-Tree QRAM allows $O(\log(N))$ queries to be executed in parallel while maintaining desirable asymptotic scalings, including $O(N)$ qubit count, $O(\log(N))$ total latency (i.e., circuit depth), and $O(\log^2(N) \varepsilon)$ infidelity.}
    \label{fig:motivation}
\end{figure}


In tandem with algorithmic advances, tremendous hardware progress has been made towards realizing QRAM. Multiple platforms have successfully demonstrated fast and high-fidelity controlled-SWAP (\texttt{CSWAP}) gates, a critical native operation in QRAM~\cite{xue2023hybrid,leger2024implementation,gao2019entanglement}. Experimental QRAM prototypes have been proposed based on quantum optics~\cite{jiang2019experimental}, Rydberg atoms \cite{patton2013ultrafast}, photonics \cite{chen2021teleqram}, and circuit quantum acoustodynamics \cite{hann2019hardware}, and superconducting cavities \cite{weiss2024quantum}. Yet, one of the most substantial limitations of QRAM is the large number of qubits required for practically relevant problems, typically $O(N)$ qubits for a size-$N$ memory. This issue can be mitigated by a \emph{shared memory} approach, where multiple quantum processing units (QPUs) share QRAM resources to improve utilization, as illustrated schematically in Fig.~\ref{fig:motivation}(a). For example, recent proposals for quantum data centers (QDC) \cite{liu2022quantum,liu2023data,liu2024quantum} have highlighted the utility of a shared QRAM system for quantum applications including multi-party private communication and quantum sensing \cite{liu2022quantum}. This shared QRAM model~\cite{alexeev2021quantum} also aligns well with the technology trends towards distributed or multi-core quantum computing, where multiple users can access shared quantum systems via cloud. Meanwhile, the emerging modular approach of building complex quantum systems from smaller modules also provides hardware support for such large-scale quantum computing architectures \cite{monroe2014large,bombin2021interleaving,krutyanskiy2023entanglement}. However, existing QRAM architectures, such as the BB QRAM, have extremely poor performances under contention. That is, a single query occupies all $O(N)$ quantum routers for the entire duration of the query. Consequently, queries must be queued and executed sequentially. 


In this work, we introduce a novel shared QRAM architecture that pipelines multiple independent queries simultaneously while preserving the qubit number and query fidelity scalings of a BB QRAM. We term this design ``Fat-Tree QRAM,'' as the organization of the quantum routers resembles a Fat-Tree \cite{leiserson1985fat} that is commonly seen in classical computing and networking systems. 

\begin{itemize}
    \item Fat-Tree QRAM architecture pipelines $O(\log(N))$ independent queries to a size-$N$ memory in $O(\log(N))$ time using $O(N)$ qubits (Fig. ~\ref{fig:motivation}(b)). This approach provides a scalable path towards building a hardware-efficient, high-bandwidth quantum shared memory system. 
    \item We consider both modular and on-chip implementations of the Fat-Tree QRAM architecture using superconducting cavities. While our QRAM design can be generalized to any technology platform that supports native \texttt{CSWAP} operations, we demonstrate that Fat-Tree QRAM can be efficiently implemented despite restrictive connectivity constraints in superconducting platforms.
    \item We analyze the optimal query scheduling/pipelining protocol that resolves resource contention and maximizes utilization and throughput for parallel queries. We discuss the benefits of such parallelism in the context of parallel quantum algorithms and parallel execution of multiple quantum algorithms. 
    \item Of great interests from an experimental standpoint are the new metrics we introduced to benchmark shared QRAM architectures, including QRAM bandwidth, space-time volume per query, hardware utilization, and memory access rate. 
\end{itemize}

Our paper is organized as follows. Sec.~\ref{sec:background} reviews current noisy intermediate-scale quantum (NISQ) machines and state-of-the-art quantum random access memory architectures. In Sec.~\ref{sec:app} and Sec.~\ref{sec:arch}, we explore quantum shared memory systems by introducing the hardware architecture of Fat-Tree QRAM with detailed implementations based on superconducting circuits. In Sec.~\ref{sec:schedule}, we provide a scheduling protocol to maximize the utilization of Fat-Tree QRAM. In Sec.~\ref{sec:eval} and Sec.~\ref{sec:results}, we evaluate the performance of the Fat-Tree QRAM for both real-world parallel quantum algorithms and synthetic algorithms. We conclude with a brief discussion on the implication of these results for large-scale quantum computing.