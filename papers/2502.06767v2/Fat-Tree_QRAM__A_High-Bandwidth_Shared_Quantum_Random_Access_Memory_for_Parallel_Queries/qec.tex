\section{Error Robustness of Fat-Tree QRAM}\label{sec:qec}
In this section, we show that Fat-Tree QRAM maintains BB QRAM's error resilience \cite{hann, xu2023systems} and requires fewer quantum resources for error correction to reach a desired circuit fidelity. Moreover, its capacity for additional queries can enhance query fidelity through virtual distillation \cite{huggins2021virtual} and support error correction. Thus, Fat-Tree QRAM achieves high query fidelity by combining error robustness and mitigation strategies, while remaining compatible with low-overhead QEC techniques for fault-tolerant quantum computing.

\subsection{Noise Resilience of Fat-Tree QRAM}
\label{subsec:noise}

We show that Fat-Tree QRAM maintains the same intrinsic error-resilience and logarithmic infidelity scaling as BB QRAM \cite{hann,xu2023systems}. Following~\cite{hann}, we define \textit{query fidelity} for a single query $\ket{\psi_{in}}$ as $F = \braket{\psi_{out}|\rho_{out}|\psi_{out}}$
where $\rho_{out}$ is the actual output density matrix under noisy channel, and $\psi_{out}$ is the ideal expected output state. For the noise model, we assume that each qubit is subjected to a generic error channel when a gate is applied:
$\mathcal{E}(\rho) = (1-\epsilon) \rho + \epsilon K\rho K^\dag$, where $\epsilon$ is the error probability and  $K$ denotes the error Kraus operator. We show that in the presence of this error channel, the Fat-Tree QRAM has the asymptotic lower bound in query fidelity $F \ge 1 - 2 \cdot \log^2{N}\cdot (\epsilon_0+\epsilon_1+\epsilon_2)$ where $\epsilon_0,\epsilon_1, \epsilon_2$ correspond to the three types of gates introduced in Sec.~\ref{sec:arch}: a total of $\log^2(N)$ number of (intra-node) \texttt{CSWAP} gates each with error rate $\epsilon_0$,  $\log^2(N)$ (inter-node) \texttt{SWAP} gates with error rate $\epsilon_1$, and $\log^2(N)$ local (intra-node) \texttt{SWAP} gates with error rate $\epsilon_2$. Thus, the total fidelity is $F \ge (2 \cdot \prod_i(1-\epsilon_i)^{G_i}-1)^2 \ge 1 - 2 \cdot \log^2{N}\cdot (\epsilon_0+\epsilon_1+\epsilon_2)$.
Compared to BB QRAM's infidelity lower bound of $F \ge 1 -  2\cdot\log^2{N}\cdot(\epsilon_0+\epsilon_1)$ \cite{hann}, Fat-Tree QRAM achieves parallelism with only a moderate decrease in fidelity. Furthermore, Fat-Tree QRAM is compatible with the error robust analysis in \cite{mehta2024analysis}, where this error resilience in QRAM is extended to more generic error models, including initialization errors, spatially correlated errors, and coherent errors.

Fig.~\ref{fig:qecinf} provides a query infidelity comparison between Fat-Tree and BB, where the error rates for the three types of gates are set to experimentally realistic values: $\epsilon_0=0.002, \epsilon_1=0.002, \epsilon_2=0.001$~\cite{weiss2024quantum,zhong2021deterministic,niu2023low}. The infidelity scaling of Fat-Tree QRAM is only a constant factor (0.25x) worse compared to BB, due to intra-node \texttt{SWAP} gates implemented using beam-splitters, which is fast and high fidelity compared to the other two types of gates described above \cite{weiss2024quantum, chapman2023high}. As hardware continues to improve, QRAM of larger capacity will become practical. This is illustrated in Table \ref{tab:error_rates}, for different baseline error rates $\epsilon_0$ with realistic parameters from \cite{weiss2024quantum}.

\begin{table}[t]
\small
\centering
\begin{tabular}{ p{1.5cm}||p{1.5cm}|p{1.5cm}|p{2.2cm}}
\hline
Capacity $N$ & $\epsilon_0=10^{-3}$ & $\epsilon_0=10^{-4}$ & $\epsilon_0=10^{-5}$ (with post-selection) \\ \hline
8 & 0.045 & 0.0045 & 0.00045 \\ \hline
16 & 0.08 & 0.008 & 0.0008 \\ \hline
32 & 0.125 & 0.0125 & 0.00125 \\ \hline
64 & 0.18 & 0.018 & 0.0018 \\ \hline
\end{tabular}
\caption{Query infidelity of QRAM with capacity $N$, given different input gate error rates $\epsilon_0$ from \cite{weiss2024quantum}. The desirable scaling comes from QRAM's intrinsic noise resilience.}
\label{tab:error_rates}
\end{table}

\subsection{Virtual Distillation using Fat-Tree QRAM}
\label{subsec:vd}

In this section, we show how to leverage parallel queries provided by Fat-Tree QRAM to boost query fidelity. Virtual distillation is a quantum error mitigation technique that "distills" a higher-fidelity state from multiple noisy copies~\cite{huggins2021virtual}. Let an $n$-qubit noisy quantum state be $\rho = (1 - \epsilon)\rho_0 + \epsilon \rho_{\text{error}}$ where $\rho_0$ is the ideal (error-free) state, $\rho_{\text{error}}$ is the error component, and $\epsilon$ is the error rate. As for Fat-Tree QRAM, we prepare $k$ identical copies of the noisy QRAM query 
state $\rho$ in parallel, leading to the combined state $\rho^{\otimes k}$. The objective is to approximate a ``distilled'' state, defined as
$\rho_{\text{distilled}} = \frac{\rho^k}{\operatorname{Tr}(\rho^k)}$. Here, $\rho^k$ represents the $k$-th power of $\rho$.
To measure an observable \( O \) with reduced error, we calculate the expectation value using the distilled state: $\langle O \rangle_{\text{distilled}} = \operatorname{Tr}\left( \frac{\rho^k}{\operatorname{Tr}(\rho^k)} O \right)$. This approach effectively amplifies the ideal component $\rho_0$'s contribution to $\rho$ while suppressing the erroneous content in noisy queries $\rho_{\text{error}}$.

Both using 256 qubits, one Fat-Tree (N=16) and two BB QRAMs (N=16) can perform four parallel queries and two parallel queries, respectively. Under independent stochastic errors, Fat-Tree achieves an exponentially higher fidelity after distillation, shown in table \ref{tab:distill}.

\begin{table}[t]
\small
\centering
\begin{tabular}{ p{5.2cm}||p{1.2cm}|p{0.8cm}}
\hline
 & Fat-Tree & 2 BB \\ \hline
Resource state prepared for distillation & 4 & 2 \\ \hline
Fidelity of single query before distillation & 0.84 & 0.872 \\ \hline
Fidelity after distillation & 0.9994 & 0.984 \\ \hline
\end{tabular}
\caption{Fidelity comparison of capacity-4 Fat-Tree and two BB QRAMs (same-qubit-count baseline) before and after virtual distillation. Details are included in Sec.~\ref{subsec:vd}.}
\label{tab:distill}
\end{table}

In general, Fat-Tree QRAM can group $k$ copies for distillation and still provide $\log(N)/k$ parallel queries, thus presenting a trade-off between query parallelism and fidelity.

\subsection{Error Correction in Fat-Tree QRAM}
\label{subsec:qec}
\subsubsection{Error-corrected query using encoded QRAM}
The intrinsic noise resilience of Fat-Tree QRAM mentioned in Section \ref{subsec:noise} also helps reduce quantum error correction (QEC) overhead in the fault-tolerant era. We assume each qubit is encoded in an $[[m, 1, d]]$ code ($m$ for the number of physical qubits, $d$ for the code distance), along with fault-tolerant  \texttt{SWAP} and \texttt{CSWAP} gates. Notably, although this gate set includes a non-Clifford gate, it can still be implemented using transversal non-Clifford gates while circumventing the constraints of the Eastin–Knill theorem~\cite{eastin2009restrictions}. This is because the limited gates in QRAM circuits do not form a universal set.
For instance, the color code supports a transversal CCZ gate~\cite{kubica2015unfolding}. Beyond transversal gates, alternative methods exist that are more efficient than magic state distillation for implementing non-Clifford operations. One such approach is pieceable fault-tolerant gates~\cite{yoder2016universal}, which perform intermediate error checks during gate execution. For example, [5,1,3] code implements a fault-tolerant Toffoli gate this way. This technique could serve as a strong candidate for implementing fault-tolerant CSWAP gates.

Notably, using different error correction architectures for QRAM and QPU may introduce extra code-switching overhead.  However, only a moderate amount of code-switching is necessary, because only the $n$ input address qubits (and 1 bus qubit) from the QPU need to be converted to the QEC code for QRAM. This is a relatively small fraction of all qubits involved, e.g., there are $N = 2^n$ qubits within the QRAM. For instance, a well-studied code teleportation approach converts between two QEC codes of distance $d_1$ and $d_2$, respectively. The procedure requires the $d_1 * d_2$ ancilla qubits per query and $O(1)$ circuit depth to transfer each of the $n$ address qubits and the bus qubit sequentially~\cite{xu2024constant}. Because Fat-Tree QRAM implements queries in a pipeline fashion, the ancilla qubits can be reused for the parallel queries. A similar claim also applies to state-of-the-art quantum low density parity check (LDPC) codes, with a linear number of ancilla qubits and a linear-depth circuit~\cite{cross2024improved}.

As a result, we can correct any $(d-1)/2$ bit-flip or phase-flip errors at the expense of $O(m\cdot N)$ total qubits. Here we compare with a generic circuit (GC) whose worst-case infidelity scales \emph{linearly} with its circuit size, which is a standard assumption in formal fault tolerance analyses. As shown in Figure \ref{fig:qecinf}, the intrinsic noise resilience protects BB/Fat-Tree QRAM from exponentially decaying fidelity for circuit of growing QRAM tree depth, when compared to a generic circuit. For example, to maintain same infidelity below $5\times 10^{-4}$, QEC with distance 3 allows us to run a GC of tree-depth $n\approx 6$ or a QRAM of tree-depth $n=10$. At the same QEC cost, we can execute a QRAM circuit of larger size than a GC. Conversely, QRAM circuit (of the same size compared to GC) requires a lower QEC code distance to achieve the same fidelity.

When compared to BB QRAM, the infidelity of Fat-Tree QRAM differs by a small constant factor, due to its additional (Clifford) gates. The efficiency of QEC resources for QRAM circuits indicates that Fat-Tree QRAM's robustness could also benefit future fault-tolerant architectures.

\subsubsection{Error-corrected query using noisy QRAM}
Experimental implementations of encoded QRAM can be challenging due to the $O(m\cdot N)$ qubit cost. We further propose a novel scheme that leverages parallel queries on encoded addresses for error protection, without replacing every physical qubit in the QRAM with an encoded logical qubit. Specifically, we assume QRAM qubits are noisy but address/bus qubits are encoded using a QEC code. Finding a QRAM-tailored code is beyond the scope of this work, but we present a resource estimate (in Table~\ref{tab:qecestimation}) for QEC code with parameters $[[m, 1, d]]$ and syndrome extraction circuit of depth $D$.

Fat-Tree enables the encoded address qubits to be routed into the QRAM in parallel. Specifically, due to the fault-tolerant implementation of \texttt{CSWAP}, we can route each of the $m$ physical qubits in an encoded logical address qubit as $m$ pipelined queries. If $m\leq \log(N)$, then $\lfloor \log(N)/m\rfloor$ logical queries can be pipelined. Within each logical query, we can interleave syndrome extraction circuit on qubits from different physical queries reaching the same positions in the QRAM, resulting in a total logical query of depth $O(D\log(N) + m)$. Table~\ref{tab:qecestimation} provides a comparison of this pipelined QEC scheme with an encoded BB QRAM.

\begin{figure}[t]
         \centering
         \includegraphics[width=\linewidth]{Figures/qecinf.pdf}
         \caption{Infidelity of a Fat-Tree QRAM, a BB QRAM, and a generic quantum circuit (GC) as a function of circuit size $N$ (or QRAM tree depth $n=\log(N)$) and QEC code distance $d$, assuming physical gate error rate $\epsilon_0=10^{-3}$. Fat-Tree and BB differ only slightly by a small constant factor, while GC performs exponentially worse than QRAM circuits.}
         \label{fig:qecinf}
\end{figure}

\begin{table}[t]
\small
\centering
\begin{tabular}{ p{3.5cm}||p{2cm}|p{1.5cm}}
\hline
 & Fat-Tree &  BB \\ \hline
Physical Qubits & $N$ & $m\cdot N$ \\ \hline
Logical Query Parallelism & $\lfloor \log(N)/m \rfloor$ & 1 \\ \hline
Logical Query Latency & $D\cdot \log(N) + m$ & $D\cdot \log(N)$ \\ \hline
\end{tabular}
\caption{Cost of the error-corrected query (in Big $O$) using (noisy) Fat-Tree QRAM and (encoded) BB QRAM.  We assume an [[$m,1,d$]] ($m\leq\log(N))$ QEC code with transversal \texttt{CSWAP} gate and $D$ syndrome extraction circuit depth. Detailed analysis is included in Sec.~\ref{subsec:qec}. }
\label{tab:qecestimation}
\end{table}

