\section{Discussion}\label{sec:discussion}

\subsection{Beyond Superconducting Platforms}
In Section \ref{sec:arch}, we proposed a hybrid cavity-transmon implementation of Fat-tree QRAM, demonstrating that Fat-tree QRAM can be realized even under the stringent connectivity constraints inherent to superconducting architectures. Another promising candidate for this implementation is the trapped-ion platform, which benefits from all-to-all connectivity. By substituting each module in our design with a trapped-ion chip and linking chips through quantum charge-coupled devices (QCCDs), we achieve a scalable Fat-tree QRAM architecture \cite{pino2021demonstration}. Recent advancements in QCCD technology enable multiple operational zones within a single trapped-ion chip, further enhancing the feasibility and practicality of Fat-Tree QRAM deployment at scale \cite{mordini2024multi}.

\subsection{Related Work}
In \cite{paler2020parallelizing}, Paler et al. introduced a "parallel query Bucket Brigade QRAM" based on different query definitions. Their parallel queries refer to classical queries to classical memory, reducing the depth of a single query (quantum query defined in this paper) from $O(N)$ to $O(\log(N))$ by parallelizing Clifford + T gates in the data retrieval stage. This improvement is accounted for and further enhanced by the $O(1)$ data retrieval in Sec.~\ref{sec:background}. However, serving multiple \emph{quantum} queries in a single QRAM remains a highly non-trivial problem and is resolved by our work.
