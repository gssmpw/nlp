\section{Evaluation Methodology}
\label{sec:eval}

\subsection{Baseline Architectures}

In this paper, we analyze the performance of Fat-Tree QRAM, Distributed Fat-Tree QRAM (D-Fat-Tree), BB QRAM\cite{giovannetti2008architectures}, Distributed Bucket-Brigade QRAM (D-BB), and Virtual QRAM (Virtual) \cite{xu2023systems}. For Fat-Tree and BB, we assume a single QRAM of capacity $N$ is used as a shared memory.  For D-Fat-Tree, we assume $\log(N)$ distributed BB QRAMs of capacity $N$ are used. For D-BB, we assume $\log(N)$ distributed BB QRAMs of capacity $N$ are used. For Virtual QRAM, we create $\log(N)$ virtual QRAMs, each using $O(N/\log(N))$ qubits to access a large address space ($N$) at the expense of increased latency. More concretely, it divides the capacity $N$ into $K$ pages with size $M=N/K$ for each page and constructs a QRAM with $O(K\log{(M)})$ query latency and $O(M+\log{(K)})$ qubit counts. For a fair comparison, the Virtual baseline uses the same total number of qubits as Fat-Tree. In the subsequent results, the baselines are organized into two groups: the first group, comprising Fat-tree, BB, and Virtual, utilizes $(O(N))$ qubits, while the second group, including D-Fat-Tree and D-BB, requires $(O(N\log(N)))$ qubits—an asymptotically greater quantity than the first group.

\subsection{Quantitative Performance Metrics for QRAM}
To quantify the performance of QRAM, especially under a parallel query setting, we define several important metrics: (i) query parallelism, (ii) max query rate, (iii) QRAM bandwidth. \emph{Query parallelism} is the maximum number of parallel queries a QRAM can execute simultaneously. We say a QRAM is under utilized if it is executing queries fewer than the allowed query parallelism. \emph{Max query rate} (in units of queries per sec) is defined as the maximum number of queries completed per unit time. In our pipelined Fat-Tree QRAM, this is calculated by inverting the amortized time of a single query. Finally, we define QRAM \emph{bandwidth} (in units of qubits per second) as the rate at which data are queried and written into bus qubits, which can be calculated by the product of max query rate and bus width (i.e., number of data qubits returned per query). We focus on results with bus width $= 1$ in Sec.~\ref{sec:results}.

To assess QRAM performance, it is useful to isolate hardware capabilities by using device-independent metrics like \emph{circuit layers}. A circuit layer is one logical step where all quantum gates within the same layer are executed in parallel. It can be further combined with the device clock speed, e.g., Circuit Layer Operations Per Second (CLOPS) \cite{amico2023defining}, to quantify the actual hardware performance.

\subsection{Benchmarks and Applications}
A shared QRAM can facilitate running multiple algorithms in parallel or running a parallel quantum algorithm. We benchmark the benefits of Fat-Tree QRAM for both synthetic and real-world algorithms:

\emph{Synthetic algorithms:} We define a family of circuits, each with alternating processing (for time $d$) and query (for time $t_1$). We test ratios of $d/t_0$ ranging from 0 to 2. Section~\ref{sec:utibench} presents benchmarking results of synthetic algorithms, each repeating querying and processing 10 times with capacity $N = 1024$. These algorithms enable the comparison of QRAM utilization and overall algorithm depth between different QRAM architectures.

\emph{Parallel Grover’s search:} Grover's algorithm can be  applied in parallel to $p$ segments of the database~\cite{zalka1999grover}, where each segment is queried $O\left(\sqrt{N/p}\right)$ times.

    
\emph{Parallel $k$-Sum:} By implementing $p$-parallelized queries to create modified states for the quantum walk, the parallel $k$-Sum algorithm improves the query complexity from $O(N^{k/k+1})$ to $O({(N/p)}^{k/k+1})$.

\emph{Parallel Hamiltonian Simulation}: Some structured Hamiltonian simulations can be implemented by parallel quantum walks \cite{zhang2024parallel}.

\emph{Parallel Quantum Signal Processing (QSP)}: By factoring degree-$d$ polynomials to the product of $p$ smaller polynomials of degree $O(d/p)$ \cite{martyn2024parallel}, the parallel QSP improves the query complexity from $O(d)$ to $O(d/p)$.
