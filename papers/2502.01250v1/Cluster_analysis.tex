% This is samplepaper.tex, a sample chapter demonstrating the
% LLNCS macro package for Springer Computer Science proceedings;
% Version 2.21 of 2022/01/12
%
\documentclass[runningheads]{llncs}
%
\usepackage[T1]{fontenc}
% T1 fonts will be used to generate the final print and online PDFs,
% so please use T1 fonts in your manuscript whenever possible.
% Other font encondings may result in incorrect characters.
%
\usepackage{graphicx}
% Used for displaying a sample figure. If possible, figure files should
% be included in EPS format.
%
% If you use the hyperref package, please uncomment the following two lines
% to display URLs in blue roman font according to Springer's eBook style:
%\usepackage{color}
%\renewcommand\UrlFont{\color{blue}\rmfamily}
%\urlstyle{rm}
%
\usepackage{amsmath}
%
\begin{document}
%
\title{Beyond Win Rates: A Clustering-Based Approach to Character Balance Analysis in Team-Based Games}
%
\titlerunning{Team Game Balance: A Clustering Approach}
% If the paper title is too long for the running head, you can set
% an abbreviated paper title here
%
\author{Haokun Zhou}
%
\authorrunning{H. Zhou}
% First names are abbreviated in the running head.
% If there are more than two authors, 'et al.' is used.
%
\institute{%
  University of Nottingham, Ningbo, China\\
  \email{hnyhz13@nottingham.edu.cn}
}
%
\maketitle              % typeset the header of the contribution
%
\begin{abstract}
Character diversity in competitive games, while enriching gameplay, often introduces balance challenges that can negatively impact player experience and strategic depth. Traditional balance assessments rely on aggregate metrics like win rates and pick rates, which offer limited insight into the intricate dynamics of team-based games and nuanced character roles. This paper proposes a novel clustering-based methodology to analyze character balance, leveraging in-game data from Valorant to account for team composition influences and reveal latent character roles. By applying hierarchical agglomerative clustering with Jensen-Shannon Divergence to professional match data from the Valorant Champions Tour 2022, our approach identifies distinct clusters of agents exhibiting similar co-occurrence patterns within team compositions. This method not only complements existing quantitative metrics but also provides a more holistic and interpretable perspective on character synergies and potential imbalances, offering game developers a valuable tool for informed and context-aware balance adjustments.

\keywords{Clustering Analysis  \and Competitive Games \and Jensen-Shannon Divergence \and Game Design.}
\end{abstract}
%
%
%
\section{Introduction}
In competitive games with selectable characters, the diversity in characters' skill sets allows players to influence the course of a battle significantly, as seen in popular titles like Overwatch, Rainbow Six Siege, and League of Legends. However, this diversity can also lead to imbalances in character strength, resulting in skewed selection rates, diminished gameplay variety, and overpowered characters that undermine the enemy's experience. These imbalances may also cause certain characters to have a disproportionately large impact on the outcome of a game, further exacerbating issues of fairness and enjoyment.

Traditionally, developers have relied on quantifying player data using metrics such as non-mirror win rates, win percentages, and pick rates to assess and adjust character strength. For instance, balance patches in Rainbow Six and Overwatch frequently reference these statistics (Overwatch Developers, 2022; Rainbow Six Siege Developers, 2024). While these metrics provide some insight into character performance, they fall short of capturing the complex dynamics of team-based gameplay and the nuanced influence of individual playstyles and abilities.

In this paper, we propose a clustering-based analysis method that offers a more holistic approach to character balance assessment. By analyzing in-game data from Valorant, we demonstrate how this method can account for team influences and reveal deeper insights into character roles and player behavior. This approach not only complements traditional metrics but also provides developers with a new perspective for making more informed and balanced adjustments to game characters.

\section{Related Work}
At present, the use of clustering for the analysis of in-game data has been more commonly used in the game industry. Past research has typically focused on the analysis of player behavior using clustered data (Bauckhage et al., 2015). These uses include the identification and classification of player behavior as well as player gaming styles (Sifa et al., 2013; Ide \& Hosobe, 2023). There are also papers that use more novel clustering approaches to make a more fine-grained analysis of player behavior data in games, such as Bayesian clustering (Normoyle \& Jensen, 2021). However, there are still no relevant studies that use clustering algorithms to evaluate in-game character design.


\section{Data Collection and Preprocessing}

This study use Valorant as a target game, utilizes data from the Valorant Champions Tour (VCT) 2022, focusing on professional matches played on the map Haven. The dataset, sourced from the ``Valorant Champion Tour 2021-2024 Data'' collection on Kaggle (Luong, 2024), was originally compiled through web scraping of vlr.gg, a prominent Valorant esports statistics website.

Our data selection base on several justifications. First, Valorant is a highly strategic, team-based first-person shooter in which player skill sets and team compositions significantly impact game outcomes, aligning with our research objectives. Given the VCT is the highest level of professional tournaments in Valorant, the data collected from VCT 2022 provides a rich and representative sample of professional play, where character choices and strategies are most refined. These matches are carefully curated and observed, offering a robust dataset that is less prone to the noise of casual play, ensuring that the analysis is grounded in the true potential of characters' skills.

Second, focusing on a single map—Haven—allows us to isolate and examine agent synergies and team compositions in a controlled environment. In Valorant, each map introduces its own strategic elements, such as specific chokepoints, sightlines, and objectives. By focusing on Haven, we minimize variability that might arise from differing map dynamics, thereby improving the precision of our analysis. The decision to work with a single map also simplifies the computational complexity, as we do not need to account for potentially diverse strategies across multiple map environments.

\subsection{Dataset Description}

The primary data source is the ``teams\_picked\_agents'' file from the VCT 2022 agent data folder. This file contains detailed information about agent selections in professional Valorant matches, including variables such as Tournament, Stage, Match Type, Map, Team, Agent, Total Wins By Map, Total Loss By Map, and Total Maps Played.

\subsection{Data Preprocessing}

The pre-processing phase involved several key steps to prepare the data for analysis. First, we addressed the ``Total Maps Played'' variable, which indicated instances where identical team compositions were employed multiple times on the same map within a single tournament stage. To ensure accurate representation of each match, we expanded these entries into individual records. This process was crucial to maintain integrity of the composition frequency in our analysis.

To control for map-specific strategies and compositions, we filtered the dataset to include only matches played on Haven. This decision allows for a focused analysis of agent relationships within a consistent environmental context.

We implemented rigorous checks to ensure data integrity. These included verification of consistency in Tournament, Stage, Match Type, Map, and Team information across expanded records; confirmation that each team composition comprised exactly five unique agents; and examination of the Agent variable for missing or inconsistent data.

A comprehensive list of all unique agents present in the dataset was compiled. This list serves as the foundation for constructing characteristic vectors in our subsequent analysis.

\subsection{Final Dataset}

The resulting preprocessed dataset consists of $N$ total records, where $N$ represents the number of individual agent selections on Haven after data expansion, and $M$ unique agents, where $M$ denotes the total number of distinct agents in the dataset. This refined dataset forms the basis for our clustering analysis, enabling a detailed examination of agent relationships and team composition strategies specific to Haven in professional Valorant play during the 2022 VCT season.

\section{Method}

This study employs hierarchical agglomerative clustering combined with Jensen-Shannon Divergence (JSD) to analyze agent relationships in Valorant based on team composition tendencies. The analysis focuses on the Haven map, using data from professional matches.

\subsection{Data Representation}

In this study, we represent each agent by constructing a characteristic probability vector 
\(\mathbf{v} = (v_1, \dots, v_n)\), where \(n\) is the total number of agents available in the game. Each element \(v_i\) of the vector corresponds to the probability that agent \(i\) is selected as a teammate when the agent corresponding to the vector is picked. This approach is motivated by the underlying assumption that the team composition context in which an agent appears reflects its functional role in a match.

Our central hypothesis is as follows: if, in different matches, the set of teammates accompanying a given agent remains largely the same, then variations in the agent itself—when compared across different team compositions—can be interpreted as indicative of substitutability. In other words, if two agents are consistently paired with nearly identical sets of teammates, they can be considered perfect substitutes in terms of their roles within the team. Consequently, the similarity between the probability vectors of two agents serves as an index of how similar (or substitutable) their roles are.

For example, if agent A’s probability vector assigns high probabilities to the same subset of teammates as agent B’s vector, then even if A and B do not co-occur together frequently, the similarity in their teammate profiles implies that they fulfill analogous roles in team compositions. Conversely, significant differences between the vectors suggest that the agents are employed in distinct tactical contexts.

By normalizing the co-occurrence counts to form these probability vectors (using the \(L_1\) norm), we ensure that each vector accurately reflects the distribution of teammate selections independent of the absolute frequency of picks. This representation thus provides a robust and interpretable measure of role similarity that is used as the foundation for the subsequent distance-based clustering analysis.


\subsection{Co-occurrence Matrix Construction}

To build the co-occurrence matrix, we aggregate the occurrences of agents being selected together in teams across multiple matches. For each match, the composition of agents is noted, and we count how often each pair of agents appears in the same lineup. This co-occurrence matrix is symmetric, with each element $C_{ij}$ representing the frequency of agents $i$ and $j$ appearing together in the same team.

The co-occurrence count for each pair of agents is used to capture the relationship between agents based on their joint presence in team compositions.

\subsection{Probability Vector Normalization}

Once the co-occurrence counts are computed, we convert these counts into probability vectors for each agent. Specifically, the probability vector for each agent is obtained by normalizing the co-occurrence counts:

\[
v_{\text{norm}} = \frac{v}{\|v\|_1}
\]

where $\|v\|_1$ is the L1 norm of the vector, ensuring that each agent’s vector sums to 1. This normalization step allows for a meaningful comparison of the agents' co-occurrence patterns, accounting for varying agent pick frequencies and mitigating the impact of any agent's self-position.

\subsection{Distance Metric: Jensen-Shannon Divergence}

To measure the dissimilarity between agents, we use Jensen-Shannon Divergence (JSD) as the distance metric. JSD is a symmetric measure that quantifies the divergence between two probability distributions. Given two probability vectors $\mathbf{v}_i$ and $\mathbf{v}_j$ for agents $i$ and $j$, the JSD is computed as:

\[
d(\mathbf{v}_i, \mathbf{v}_j) = \frac{1}{2} \left( D_{\text{KL}}(\mathbf{v}_i \parallel m) + D_{\text{KL}}(\mathbf{v}_j \parallel m) \right)
\]

where $m = \frac{\mathbf{v}_i + \mathbf{v}_j}{2}$ is the mixture distribution, and $D_{\text{KL}}$ is the Kullback-Leibler divergence. JSD ensures that the distance between agents is not just based on raw co-occurrence but also on the distribution of their co-occurrences across other agents. This metric helps us capture more nuanced differences between agents' roles in team compositions.

\subsection{Clustering Algorithm}

We apply hierarchical agglomerative clustering (HAC) to the pairwise distances between agents. This method is chosen for its ability to uncover the hierarchical structure of relationships between agents. Initially, each agent is considered its own cluster, and the most similar clusters are merged iteratively based on the Jensen-Shannon Divergence distances.

The resulting hierarchical structure is visualized as a dendrogram, which provides insight into the clustering of agents based on their co-occurrence patterns and reveals potential roles within the agent pool.

\subsection{Linkage Method}

In hierarchical agglomerative clustering, the average linkage method (UPGMA) is used to determine the distance between clusters. Given two clusters $C_1$ and $C_2$, the distance between them is defined as the average of the distances between all pairs of agents $x \in C_1$ and $y \in C_2$:

\[
d(C_1, C_2) = \frac{1}{|C_1| |C_2|} \sum_{x \in C_1} \sum_{y \in C_2} d(x, y)
\]

where $|C_1|$ and $|C_2|$ are the sizes of the clusters. Average linkage is selected for its robustness to outliers and its tendency to produce more compact, well-defined clusters.

\subsection*{Justification for Not Using Stochastic Block Model (SBM)}

While the Stochastic Block Model (SBM) is a widely-used tool for identifying community structures in networks, we opted not to use it for this analysis due to the following key reasons:

\subsubsection*{Co-occurrence as a Direct Measure}
The SBM is particularly useful in situations where the data structure is more complex or when trying to uncover hidden community structures in large, sparse networks. However, in our case, the co-occurrence matrix provides a direct and interpretable measure of how agents are grouped together within the same team compositions. SBM, though powerful, requires more intricate probabilistic inference, which we found unnecessary given the clarity and simplicity of using distance-based measures for this analysis.

\subsubsection*{Simplicity and Interpretability}
The hierarchical agglomerative clustering method, when paired with Jensen-Shannon Divergence, provides a clearer and more interpretable solution for identifying agent roles. The dendrogram produced by this clustering method is easy to visualize and interpret, offering direct insights into the relationships between agents. SBM, while effective in identifying latent blocks, adds a layer of complexity that may not provide additional actionable insights in the context of understanding agent relationships in team compositions.

\subsubsection*{Measuring the Effect of Character Changes}
One significant reason for opting against the SBM is the need for flexibility in measuring the impact of character changes. Using distance-based clustering methods like Jensen-Shannon Divergence enables us to track the effect of changes in agents on the roles. Specifically, when a new agent is introduced, we can measure how the distances between agents change, which allows us to quantify the impact of introducing a new character into the team composition. This ability to track shifts in cluster assignments (i.e., roles) is more challenging in SBM, as SBM tends to focus on assigning agents to predefined blocks without a straightforward mechanism to measure how changes affect existing role structures. By using hierarchical clustering, we can directly analyze the shift in the clustering of agents and observe how the introduction of a new agent alters the existing roles.


\subsection{Interpretation}

The resulting clusters are interpreted as groups of agents with similar roles or synergies within team compositions on the Haven map. Agents clustered together are understood to have similar roles or synergies, contributing similarly to the team's composition strategies. This approach allows us to uncover implicit agent relationships based on actual gameplay data, potentially revealing nuanced roles and synergies that may not be apparent from predefined agent categories.

By focusing on team composition tendencies, we gain deeper insights into how professional players structure their teams around different agents, especially in the specific context of the Haven map. This clustering methodology reveals roles that emerge dynamically from the data, capturing patterns of synergy and substitution between agents.

\section{Result}

\begin{figure}
\includegraphics[width=\textwidth]{Charater_cluster.png}
\caption{Hierarchical Clustering of Valorant Agents on the Haven Map. The dendrogram displays the hierarchical clustering of agents based on their team composition tendencies, measured using Jensen-Shannon Divergence. Agents with similar co-occurrence patterns (i.e., frequently paired with the same teammates) are grouped together in the same clusters. Different colors represent distinct clusters of agents with similar roles or synergies within team compositions.} \label{fig1}
\end{figure}

Figure 1 shows the hierarchical clustering analysis of Valorant agents on the Haven map, based on their co-occurrence patterns in team compositions, revealed distinct clusters that align with functional roles within the game. As figure 1 shown, agents such as Astra, Omen, and Brimstone clustered together, reflecting their shared role as Controllers, who focus on map control and utility. Similarly, Chamber, Killjoy, and Cypher formed a separate cluster, identifying them as Sentinels, agents specializing in defense, trap-setting, and intelligence gathering. The proximity of these agents suggests that, while each brings unique tactical elements to the team, they fulfill similar defensive functions. In another cluster, agents like Phoenix and Reyna were grouped together due to their common traits as Duelists—with overlapping skills in flash utility and recovery. Likewise, Kayo and Breach were clustered together as Initiators, both utilizing flash abilities for team engagements. Viper and Sage also clustered closely, highlighting their shared strategy of controlling space and splitting the battlefield. The final distinct cluster, consisting of Jett, Neon, and Raze, was formed around agents with high-mobility, fast-paced, and aggressive playstyles, categorized as Duelists. Their separation from other clusters underscores their unique role in disrupting enemy positions and prioritizing offensive actions. Overall, the dendrogram demonstrates that agents with similar roles, such as Controllers, Sentinels, and Duelists with displacement skills, tend to cluster together. Additionally, the analysis identified characters like Yoru, Sova, and Fade, whose unique skill sets bring a distinctive impact to the Haven map, setting them apart from the other agents, which align with our original objective. This clustering methodology uncovers the nuanced synergies within team compositions, revealing that agents' roles are not strictly confined to predefined categories, but are instead shaped by the interplay of their co-occurrence patterns. 
Overall, the clustering results provide insights that complement traditional balance metrics, such as win rates and pick rates, by uncovering the synergies and subtleties of how agents interact within teams. The clustering approach reveals not only the similarity between agents based on their co-occurrence patterns but also how these agents collectively contribute to the dynamics of the team. This method presents a more holistic view of agent balance, helping developers better understand the relationships between agents and providing a deeper perspective on the impact of individual characters on the outcome of a game. The analysis offers a valuable tool for making more informed, context-driven adjustments to character balance that reflect the complexity of team-based gameplay.

\section{Measuring the Impact of Agent Adjustments on Game Balance}

Beyond clustering agents into role-based groups, the Jensen-Shannon Divergence (JSD) distances calculated between agent probability vectors offer a valuable tool for quantitatively assessing the impact of developer-initiated character adjustments on game balance. In the dynamic environment of competitive games, developers frequently release balance patches that modify agent abilities, statistics, or other attributes to refine gameplay and address balance concerns. Our methodology provides a framework to evaluate the effects of these adjustments by observing changes in agent relationships and team composition dynamics.

To measure the impact of a specific agent adjustment, this approach advocates for a comparative analysis of agent co-occurrence patterns before and after the implementation of a balance patch. Firstly, the clustering analysis, as described in Section 4, should be performed using gameplay data collected prior to the patch. This establishes a baseline understanding of agent roles and relationships within team compositions. Subsequently, after the balance patch has been deployed and sufficient gameplay data has been accumulated, the clustering analysis is repeated using the post-patch data.

The core of the impact assessment lies in comparing the JSD distances and cluster memberships of agents between the pre-patch and post-patch analyses. Several key metrics, including absolute distance changes, can be examined:

\begin{itemize}
    \item \textbf{Absolute Change in Agent Distance to Cluster Centroid ($\Delta D_{centroid}$)}: For each agent, particularly those directly affected by the balance patch, we can measure the change in its distance to the centroid of its assigned cluster. Let $C_{pre}$ be the cluster to which agent $A$ belongs in the pre-patch analysis, and let $\mu_{pre}$ be the centroid of $C_{pre}$, calculated as the average of the probability vectors of all agents in $C_{pre}$:
    \begin{equation}
        \boldsymbol{\mu}_{pre} = \frac{1}{|C_{pre}|} \sum_{\mathbf{v}_i \in C_{pre}} \mathbf{v}_i^{pre}
    \end{equation}
    where $\mathbf{v}_i^{pre}$ is the probability vector of agent $i$ pre-patch. Similarly, let $\mu_{post}$ be the centroid of the corresponding cluster $C_{post}$ in the post-patch analysis:
    \begin{equation}
        \boldsymbol{\mu}_{post} = \frac{1}{|C_{post}|} \sum_{\mathbf{v}_j \in C_{post}} \mathbf{v}_j^{post}
    \end{equation}
    Then, the absolute change in distance of agent $A$ to its cluster centroid is given by:
    \begin{equation}
        \Delta D_{centroid}(A) = |JSD(\mathbf{v}_A^{post}, \boldsymbol{\mu}_{post}) - JSD(\mathbf{v}_A^{pre}, \boldsymbol{\mu}_{pre})|
    \end{equation}
    A significant positive $\Delta D_{centroid}(A)$ might indicate a shift in agent $A$'s role away from its previous cluster archetype, potentially indicating an imbalance or role disruption. A negative or negligible change could suggest the agent's role remains consistent, or the balance adjustment has reinforced its intended role.

    \item \textbf{Absolute Change in Average Inter-Agent Distance within Clusters ($\Delta D_{inter}$)}: We can also measure how the average pairwise JSD distance between agents within a cluster changes after a balance patch. For a cluster $C$, the average inter-agent distance can be defined as:
    \begin{equation}
        \bar{D}_{inter}(C) = \frac{2}{|C|(|C|-1)} \sum_{\mathbf{v}_i \in C} \sum_{\mathbf{v}_j \in C, i<j} JSD(\mathbf{v}_i, \mathbf{v}_j)
    \end{equation}
    The absolute change in average inter-agent distance for a cluster $C$ between pre-patch and post-patch is:
    \begin{equation}
        \Delta D_{inter}(C) = |\bar{D}_{inter}(C_{post}) - \bar{D}_{inter}(C_{pre})|
    \end{equation}
    An increase in $\Delta D_{inter}(C)$ might suggest that the agents within cluster $C$ have become more differentiated in their roles or synergies post-patch, potentially indicating a destabilization within that role archetype. Conversely, a decrease could suggest increased homogeneity or clearer role definitions within the cluster.

    \item \textbf{Shift in Cluster Membership}: A more qualitative but crucial observation is whether agents change cluster membership entirely between pre-patch and post-patch analyses. This indicates a fundamental shift in an agent's role within team compositions, which can be directly observed by comparing cluster assignments in the dendrograms.

    \item \textbf{Overall Dendrogram Structure Comparison}: Visually comparing the dendrograms from pre-patch and post-patch data provides a holistic view of how balance changes affect the broader agent ecosystem. Changes in branching patterns and cluster formations can reveal systemic shifts in agent roles and team composition strategies.
\end{itemize}


\section{Limitation and future works}
The primary assumption of this study—that agents with similar team compositions are perfect substitutes—presents several limitations. While this assumption provides a useful starting point for analyzing agent relationships, it oversimplifies the complexities of team compositions and gameplay dynamics. In reality, agent roles are not strictly interchangeable even when agents share similar co-occurrence patterns in team compositions. The strategic context in which agents are chosen can vary significantly, and players may select agents based not only on their abilities but also on personal playstyle preferences, team synergy, and broader game strategy. For instance, two agents with similar co-occurrence data may still play distinct roles depending on the player's preferred tactics, such as an agent focusing on aggressive plays versus one providing support and utility. Therefore, assuming perfect substitutability risks overlooking these strategic nuances, which can have a significant impact on gameplay balance.

Moreover, while clustering agents based on team composition provides valuable insights, the exact definition of "balance" remains elusive. The distance metric used in this study captures how similar agents are in terms of team composition, but it does not inherently provide a clear boundary between balanced and imbalanced agents. In other words, the clustering approach does not establish a direct link between the distance measure and game balance—meaning we cannot definitively conclude whether an agent's distance from others indicates that it is "too strong" or "too weak" in the context of game balance. This absence of a clear threshold for balance presents a challenge for developers seeking to fine-tune agent strength or adjust their roles within team compositions.

Future work could explore incorporating additional factors, such as side-specific win rates, into the clustering process. Including this metric would allow for a deeper understanding of how agent compositions perform under different conditions and could provide a more comprehensive view of balance. Additionally, incorporating more granular factors like player behavior or contextual gameplay data could refine this approach. Developing robust metrics for determining "balance" based on distance and adding subjective elements, such as strategy and player preference, would further enhance the utility of this method in real-world game balancing, enabling a more nuanced and flexible assessment of agent roles and team compositions.




\begin{thebibliography}{8}
\bibitem{clustergame1}
Bauckhage, C., Drachen, A. and Sifa, R. (2015) ‘Clustering game behavior data’, IEEE Transactions on Computational Intelligence and AI in Games, 7(3), pp. 266–278. doi:10.1109/tciaig.2014.2376982. 

\bibitem{onlinegame}
Ide, T. and Hosobe, H. (2023) ‘Supporting online game players by the visualization of personalities and skills based on in-game statistics’, Proceedings of the 18th International Joint Conference on Computer Vision, Imaging and Computer Graphics Theory and Applications [Preprint]. doi:10.5220/0011784000003417. 

\bibitem{ref_vct_data}
Luong, R. (2024) Valorant Champion Tour 2021-2024 data, Kaggle. Available at: https://www.kaggle.com/datasets/ryanluong1/valorant-champion-tour-2021-2023-data (Accessed: 03 February 2025).

\bibitem{bayesian}
Normoyle, A. and Jensen, S. (2021) ‘Bayesian clustering of player styles for multiplayer games’, Proceedings of the AAAI Conference on Artificial Intelligence and Interactive Digital Entertainment, 11(1), pp. 163–169. doi:10.1609/aiide.v11i1.12805. 

\bibitem{overwatch}
Overwatch 2 developer blog: Post-launch updates on gameplay, maps, and competitive (2022) Overwatch. Available at: https://overwatch.blizzard.com/en-us/news/23865965/overwatch-2-developer-blog-post-launch-updates-on-gameplay-maps-and-competitive/ (Accessed: 03 February 2025). 

\bibitem{tomb}
Sifa, R. et al. (2013) ‘Behavior evolution in tomb raider underworld’, 2013 IEEE Conference on Computational Inteligence in Games (CIG), pp. 1–8. doi:10.1109/cig.2013.6633637. 

\bibitem{ubisoft}
Y9S1 Designer’s notes (2024) Ubisoft Rainbow Six Siege. Available at: https://www.ubisoft.com/en-us/game/rainbow-six/siege/news-updates/3jBlCdtRBQx2sCjmY2umNu/y9s1-designers-notes (Accessed: 03 February 2025).

 


\end{thebibliography}
\end{document}
