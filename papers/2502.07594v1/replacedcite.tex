\section{Related Work}
Our work continues a long line of research on distributed certification, where a prover assigns certificates for some graph property to the nodes, who then verify the property in a deterministic manner (proof-labeling schemes (PLFs)____, locally checkable proofs____, and the class $\mathsf{NLD}$____).
More relevant to our work are RPLS____, where the nodes   exchange randomized messages instead of deterministic ones.
Our work is also related to distributed interactive proofs____, where the units can exchange messages with the prover instead of merely receiving certificates. 
Specifically, we elaborate upon a previous dMA protocol for triangle-freeness____.

As previously discussed, the most relevant related work to our paper is ____, who first defined the notion of distributed zero-knowledge. 
%
%
Their most general definition is of \emph{distributed knowledge}, denoted dK[$r,\ell,\mathcal{A}_v,\mathcal{A}_s,k$],
where the zero-knowledge property is w.r.t. a distributed protocol from some family.
Here, $r$ is the number of communication rounds between the prover and the nodes, and $\ell$ is the number of bits exchanged in each such  round. 
$\mathcal{A}_v$ is the type of distributed algorithm that the nodes run after communicating with the prover, in order to decide whether to accept or reject, while
$\mathcal{A}_s$ is the type of distributed algorithm used by the simulator. 
Finally, $k$ is the size of an adversarial coalition of nodes for which the protocol is still zero-knowledge.
%

%
Let $\cgst(\mu)$ be the the standard $\cgst$ model with $O(\mu)$-bit massages,
and $\bot$ be the class of zero-rounds, no-communication protocols.
%
%
%
%
Then, $\dzma(\ell,\mu)=$dK[$1,\ell,\cgst(\mu),\bot,1$], which is also denoted dSZK[$1,\ell,\cgst(\mu),1$] there.

Finally, we notice that____ use the classiacl Shamir secret sharing____ in their zero-knowledge protocol for spanning tree verification. 
Our polynomial sharing technique scheme extends Shamir secret sharing, exploiting more structured polynomials to enable efficient verification without revealing further information.

%


%
%
%