% !TEX root = ./main.tex
We consider the general problem of blocking all solutions of some given combinatorial problem with only few elements. 
For example, the problem of destroying all Hamiltonian cycles of a given graph by forbidding only few edges; or the problem of destroying all maximum cliques of a given graph by forbidding only few vertices.
Problems of this kind are so fundamental that they have been studied under many different names in many different disjoint research communities already since the 90s.
Depending on the context, they have been called the interdiction, most vital vertex, most vital edge, blocker, or vertex deletion problem.

Despite their apparent popularity, surprisingly little is known about the computational complexity of interdiction problems in the case where the original problem is already NP-complete.
In this paper, we fill that gap of knowledge by showing that a large amount of interdiction problems are even harder than NP-hard. 
Namely, they are complete for the second stage of Stockmeyer's polynomial hierarchy, the complexity class $\Sigma^p_2$.
Such complexity insights are important because they imply that all these problems can not be modelled by a compact integer program (unless the unlikely conjecture NP $= \Sigma_2^p$ holds).
Concretely, we prove $\Sigma^p_2$-completeness of the following interdiction problems:
    satisfiability,
    3satisfiability,
    dominating set,
    set cover,
    hitting set,
    feedback vertex set,
    feedback arc set,
    uncapacitated facility location,
    $p$-center,
    $p$-median,
    independent set,
    clique,
    subset sum,
    knapsack,
    Hamiltonian path/cycle (directed/undirected),
    TSP,
    $k$ directed vertex disjoint path ($k \geq 2$),
    Steiner tree.
We show that all of these problems share an abstract property which implies that their interdiction counterpart is $\Sigma_2^p$-complete.
Thus, all of these problems are $\Sigma_2^p$-complete \enquote{for the same reason}.
Our result extends a recent framework by Grüne and Wulf.
