% !TEX root = main.tex
\subsection{An Invulnerability Reduction for Hamiltonian Cycle}
The invulnerability gadget for Hamiltonian cycle is the most involved of all our constructions, 
hence we devote a subsection to it. 
The main result in this section is that the minimum cardinality interdiction problem is $\Sigma^p_2$-complete for the nominal problems of both directed and undirected Hamiltonian cycle and path, as well as the TSP.

We present our reduction for the case of undirected Hamiltonian cycle and then argue how it can be adapted to the other cases. The main idea is to consider as an intermediate step only 3-regular graphs $G = (V, E)$, and then for a subset $C \subseteq E$ show how $E \setminus C$ can be made invulnerable. To this end, consider the SSP problem

\begin{description}
    \item[]\textsc{3Reg Ham}\hfill\\
    \textbf{Instances:} Undirected, 3-regular Graph $G = (V, E)$\\
    \textbf{Universe:} $\U := E$.\\
    \textbf{Solution set:} The set of all Hamiltonian cycles in $G$.
\end{description}

Recall that it is shown in \cite{gruene2024completeness} that \textsc{Hamiltonian Cycle} is SSP-NP-complete. We now require the stronger statement

\begin{lemma}
\label{lem:3-reg-ham-ssp}
    \textsc{3Reg Ham} is SSP-NP-complete.
\end{lemma}
\begin{proof}
    Garey, Johnson \& Tarjan \cite{DBLP:journals/siamcomp/GareyJT76} give a reduction from \textsc{3Sat} to  \textsc{3Reg Ham}, such that for every variable $x_i$ in the \textsc{3Sat} instance the graph $G$ has two distinct edges $e(x_i)$ and $e(\overline x_i)$ (compare Figure 7 in \cite{DBLP:journals/siamcomp/GareyJT76}). Let $E' := \bigcup_i \set{e(x_i), e(\overline x_i)}$ be the set of all these edges. For some assignment $\alpha$ of the \textsc{3Sat} variables, we say that $\alpha$ corresponds to the edge set $E_\alpha$ defined by $\set{e(x_i) : \alpha(x_i) = 1} \cup \set{e(\overline x_i) : \alpha(x_i) = 0}$.
    Garey, Johnson \& Tarjan show that there is a bijection between satisfying assignments and edge sets $E'' \subseteq E'$ that can be subset of a Hamiltonian cycle. More formally: 1.) For every satisfying assignment $\alpha$, 
    if one considers the set $E_\alpha \subseteq E'$ of edges corresponding to that assignment, 
    there exists a Hamiltonian cycle $H$ extending $E_\alpha$, i.e.\ $H \cap E' = E_\alpha$. 
    2.) For every Hamiltonian cycle $H$, we have that $H \cap E'$ equals $E_\alpha$ for some satisfying assignment $\alpha$.
    In total, 1.) and 2.) together show that the reduction in \cite{DBLP:journals/siamcomp/GareyJT76} is an SSP-reduction. (By defining $f(x_i) := e(x_i), f(\overline x_i) := e(\overline x_i)$.)
\end{proof}

We remark that it follows from \cite{akiyama1980np,DBLP:journals/siamcomp/GareyJT76} by the same argument that the problem is even SSP-NP-complete if restricted to 3-regular, bipartite, planar, 2-connected graphs. However, for our arguments it suffices to consider 3-regular graphs.

Consider now an instance of \textsc{3Reg Ham}, i.e. a 3-regular undirected graph  $G = (V, E)$. Let $C \subseteq E$ be a subset of the edges and $k \in \N_0$ the attacker's budget. We call $C$ the vulnerable edges. Let $D := E(G) \setminus C$.
In the remainder of this section we describe and prove a construction how to make the edges in $D$ invulnerable.
We quickly sketch the main idea: To make an edge $e = ab$ invulnerable, we enlarge it by replacing it with a large clique $W'_{ab}$ making sure that $e$ can be traversed no matter which $k$ edges inside $W'_{ab}$  are attacked. 
We also blow up each vertex $a$ of the original graph into a clique $W_a$.
However, this introduces new vertices into the instance, and we need to make sure that a Hamiltonian cycle can always trivially visit all the new vertices.
At the same time however, it should still hold that a Hamiltonian cycle in the new graph should be able to enter and exit these new objects $W_a$ and $W'_{ab}$ at most once, since otherwise a corresponding cycle in the old graph $G$ would visit edges or vertices twice, which is of course forbidden.
We achieve this by associating to each edge $e = ab$ a star of edges $F_{ab}$ and argue that a Hamiltonian cycle can use (essentially) at most one edge of each star $F_{ab}$. 
Furthermore, we will show that the fact that $G$ is 3-regular implies that each clique $W_a$ can be traversed (essentially) only once.

We are ready to begin with the construction.
First, let the directed graph $\overrightarrow{G}$ result from $G$ by orienting its edges arbitrarily and $k$ be the budget of the attacker.
We construct an undirected graph $G' = (V', E')$ from $\overrightarrow{G}$ as follows: 
Let $n := |V(G)|$.
For each vertex $a \in V(\overrightarrow{G})$, let $d_a$ be the out-degree of $a$, and let $W_a$ be a set of $2d_a + 4k + 1$ vertices.
For each invulnerable edge $ab \in D$ in the old graph, let  $W'_{ab}$ be a set of $4k$ vertices.
The vertex set $V(G')$ of the new graph $G'$ is then defined by
\[ V(G') = \bigcup_{a \in V} W_a \cup \bigcup_{ab \in D} W'_{ab}.\]
\begin{figure}
    \centering
    \includegraphics[scale=1.0]{src/img/ham-cycle-invulnerability.pdf}
    \caption{Invulnerability gagdet for Hamiltonian cycle which makes the edge $ab$ invulnerable while the edge $ac$ remains vulnerable.}
    \label{fig:ham-cycle-invulnerability}
\end{figure}
We further partition $W_a$ into three disjoint parts $W_a = X_a \cup Y_a \cup \set{z_a}$ of size $|X_a| = 2d_a$ and $|Y_a| = 4k$ and $|\set{z_a}| = 1$.
We denote the vertices of $X_a$ by $x^{(a)}_1, \dots, x^{(a)}_{2d_a}$.
The edges of $G'$ are defined as follows:
First, we let $W_a$ be a clique for all $v \in V$.
Second, for each vertex $a \in V$ in $\overrightarrow{G}$, let $e_1, \dots, e_{d_a}$ be its outgoing edges.
For each $i = 1, \dots, d_a$, consider the $i$-th outgoing edge $e_i = (a, b)$ of $a$, where $b$ is the corresponding neighbor. 
If $e_i \in C$, i.e. $e_i$ is vulnerable, then $G'$ contains simply the single edge $x^{(a)}_{2i-1}z_b$. 
In the other case, i.e.\ $e_i \in D$ is invulnerable, then $G'$ contains an invulnerability gadget as depicted in \cref{fig:ham-cycle-invulnerability} induced on the vertices $\set{x^{(a)}_{2i-1}, x^{(a)}_{2i}} \cup W'_{ab} \cup \set{z_b}$.
The invulnerability gadget consists out of a clique on the vertex set $\set{x^{(a)}_{2i-1}, x^{(a)}_{2i}} \cup W'_{ab}$, together with all edges from the set $W'_{ab}$ to the vertex $z_b$, i.e.\ a star centered at $z_b$ that has $W'_{ab}$ as its leaves.
Let $F_{ab}$ denote this star. 
Finally, for all vulnerable edges $ab \in D$, we also define $F_{ab}$ to be the single edge $x^{(a)}_{2i-1}z_b$ that connects $W_a$ to $W_b$.
This can be interpreted as a trivial star centered at $z_b$ with only one leaf.
This completes the description of $G'$.

The overall idea of this construction is that the cliques of $W_a$ cannot be attacked because they have at least $k$ vertices.
Thus it is always possible to find a path visiting all vertices of $W_a$.
Additionally, a star $F_{ab}$ of size larger than $k$ makes the edge $ab \in E$ invulnerable because at most $k$ many of the edges can be attacked.
Thus there is always the possibility to travel over one edge of $F_{ab}$ which corresponds to using edge $ab$ in the original graph.
On the other hand, since every edge of the star is connected to the same vertex $z_b$, we have that the star $F_{ab}$ can be used (essentially) exactly once.
Thus only the stars of size one (which correspond to the vulnerable edges) are attackable.
We now have everything that we need to prove our main result of this section. 

\begin{theorem}
\label{thm:ham-cycle-interdiction}
Minimum cardinality interdiction for \textsc{Undirected Hamiltonian Cycle} is $\Sigma^p_2$-complete.
\end{theorem}
\begin{proof}
Due to \cite{gruene2024completeness}, and \cref{lem:3-reg-ham-ssp}, we have that \textsc{Comb. Interdiction-3Reg Ham} is $\Sigma^p_2$-complete.
We claim that the construction of $G'$ yields a correct reduction from \textsc{Comb. Interdiction-3Reg Ham} to \textsc{Min Cardinality Interdiction-HamCycle}.
Indeed, the following two \cref{lem:hamcycle-if,lem:hamcycle-only-if} show that yes-instances of one problem get transformed into yes-instances of the other problem. 
\end{proof}
We remark that the 3-regularity of the graph is not maintained by the reduction.
(Indeed, an argument similar to the arguments given later in \cref{sec:noMeta} shows that the interdiction problem for Hamiltonian cycle restriced to only 3-regular graphs is likely not $\Sigma^p_2$-complete).

\begin{lemma}
\label{lem:hamcycle-if}
    If there exists $B' \subseteq E'$ of size $|B'| \leq k$, such that $G' - B'$ has no Hamiltonian cycle, then there is $B \subseteq C$ of size $|B| \leq k$ such that $G - B$ has no Hamiltonian cycle.
\end{lemma}
\begin{proof}
    Proof by contraposition. Assume that for all $B \subseteq C$ with $|B| \leq k$ the graph $G - B$ has a Hamiltonian cycle $H$.
    Given some $B' \subseteq E'$ with $|B'| \leq k$, we have to show that the graph $G' - B'$ has a Hamiltonian cycle. Let $B \subseteq C$ be the set of vulnerable edges in $G$ whose copies in $G'$ are attacked by $B'$ (i.e. $B = \set{ab \in C : F_{ab} \in B'}$). 
    Since $B \subseteq C$ and $|B| \leq k$, by assumption $G - B$ has a Hamiltonian cycle $H$. We want to modify $H$ to a Hamiltonian cycle of $H'$ of $G' - B'$. 
    The basic idea is to follow globally the same route as $H$. However, we have to pay attention, because we are not allowed to use edges from $B'$.
    For each vertex in $G'$ call it \emph{attacked}, if at least one of its incident edges are attacked by $B'$, and call it \emph{free} otherwise. 
    Note that since $|B'| \leq k$ and $|Y_a| = 4k$ and $|W'_{ab}| = 4k$ for $a \in V, ab \in E$, the vertex sets $Y_a$ and $W'_{ab}$ have at least $2k$ free vertices. Free vertices are good for the following reason: 
    Whenever we plan to go from some vertex $u$ to $v$ in $G'$, but we cannot because $uv \in B'$ was attacked, then we can instead choose any free vertex $f$ and go the route $u,f,v$ instead.
    Now the plan is that $H'$ will roughly employ the following strategy: Follow globally the same path in $G'$ like $H$ does in $G$. 
    Whenever $H'$ enters some new set $W_a$ for the first time, then we visit all the sets $W'_{ab}$ for all out-neighbors $b$ of $a$ in $\overrightarrow{G}$.
    Note that for such $b$, the set $W'_{ab}$ has two adjacent vertices with $W_a$ (we use these two vertices to enter and leave), and we collect all the vertices of $W'_{ab}$. 
    Here, we prioritize to visit first the attacked vertices of $W'_{ab}$ and then the remaining vertices of $W'_{ab}$. 
    After that, we collect all remaining vertices of $W_a$ (again prioritizing the attacked vertices first) before leaving $W_a$. (If the path on which we are leaving $W_a$ corresponds to an invulnerable edge $ab$ in $G$, we also collect all of $W'_{ab}$ in the process of leaving $W_a$.)

    Note that this plan might at first not be feasible, because it requires going over some edge $e' \in B'$. However note that, since $H$ does not use any edge of $B$, for every such edge $e'$ there are always at least $2k$ free vertices that are adjacent to both endpoints of $e'$.
    Hence it is possible to \enquote{repair} such an edge $e'$ by rerouting over some free vertex instead (and later skip over this free vertex). 
    Since there are at most $k$ defects, and there are at least $2k$ free vertices available at the end of traversing every set $W_a$ or $W'_{ab}$, all defects can be repaired. 
    Hence we can modify $H'$ to be a Hamiltonian cycle of $G' - B'$, which was to show.
\end{proof}

\begin{lemma}
\label{lem:hamcycle-only-if}
    If there exists $B \subseteq C$ of size $|B| \leq k$, such that $G - B$ has no Hamiltonian cycle, then there is $B' \subseteq E'$ of size $|B'| \leq k$ such that $G' - B'$ has no Hamiltonian cycle.
\end{lemma}
\begin{proof}
    Proof by contraposition. Assume that for all $B' \subseteq E'$ of size $|B'| \leq k$ the graph $G' - B'$ has a Hamiltonian cycle. 
    Given some $B \subseteq C$ with $|B| \leq k$, we have to show that the graph $G - B$ has a Hamiltonian cycle. 
    Let $B'$ be the trivial stars in $G'$ corresponding to the edges in $B$ (i.e.\ $B' = \set{F_{ab} : ab \in B}$). 
    Since $|B'| \leq k$, by assumption there is a Hamiltonian cycle $H'$ in $G' - B'$.  
    Consider the set $F := \bigcup_{ab \in E}E(F_{ab})$, i.e.\ the union of the edge sets of all the stars, trivial or not.
    We claim that w.l.o.g.\ we can assume that $|H' \cap F_{ab}| \leq 1$ for all $ab \in E$.
    Indeed, the graph $G' - F$ consists out of multiple connected components. 
    Each of these components contains exactly one set of the form $W_a$, and is incident to exactly three sets of the form $F_e$ in $G'$ (where $e$ is an edge that is either incoming to or outgoing from $a$ in $\overrightarrow{G}$).
    Suppose for some $F_{ab}$ we have $|H' \cap F_{ab}| \geq 2$.
    Since $F_{ab}$ is a star connected to a single vertex $z_b$, we have $|H' \cap F_{ab}| = 2$.
    Consider the edge $ab$ such that $F_{ab}$ connects the vertex $z_b$ with $W'_{ab}$. 
    By the observation about $G' - F$, the following is true about $H'$: 
    It enters $W'_{ab}$ in one of the two vertices attached to $X_a$, then traverses exactly all of $W'_{ab} \cup \set{z_b}$, 
    then leaves through the other of the two vertices attached to $X_a$, and at a later point returns to collect all vertices of $X_b \setminus \set{z_b}$.
    However, by the same observation as in \cref{lem:hamcycle-if}, if we define a free vertex to be a vertex not adjacent to any edge in $B'$, then both $W'_{ab}$ and $W_b$ have $2k$ free vertices.
    Hence we can modify $H'$ such that $H' \cap F_{ab} = \emptyset$.
    We thus assume that $|H' \cap F_{ab}| \leq 1$ for all $ab \in E$.
    Consider again the graph $G' - F$.
    Since each of its component is adjacent to three sets $F_e$ and $|H' \cap F_e| \leq 1$, we conclude that $H'$ uses exactly two of these three sets $F_e$.
    But this implies that $H'$ enters and exits each of the components of $G'- F$ only once and collects all of its vertices in the process.
    This implies that $H'$ globally follows the same path as some Hamiltonian cycle $H$ of $G$.
    Since $H' \subseteq G' - B'$, we conclude $H \subseteq G - B$.
    This was to show.
\end{proof}


These two lemmas together prove \cref{thm:ham-cycle-interdiction}.
We would now like to prove $\Sigma^p_2$-completeness also for Hamiltonian cycle interdiction of directed graphs.
Note that this does not follow from a trivial argument: 
Even though one can transform an undirected graph into a directed one, by substituting every undirected edge $uv$ by two directed edges $(u,v), (v,u)$, there is a problem: In the new setting the interdictor needs two attacks to separate $u,v$, while in the old setting the attacker only needs one. 

Still, the above proof can be adapted to the case of directed Hamiltonian cycle the following way: 
We start with \cite{plesn1979np}, which provides a SSP reduction to prove that the Hamiltonian cycle problem is NP-complete even in directed graphs $G$ such that $\text{indegree}(v) + \text{outdegree}(v) \leq 3$ for every vertex $v$, and such that for all pairs $u,v$ of $G$ at most one of the two edges $(u,v)$ and $(v,u)$ is present. 
Given a directed graph $G$, we then repeat the same construction as before, 
with the difference that we can start directly with the directed graph $G$ instead of obtaining an orientation $\overrightarrow{G}$ first.
This way, we can obtain an undirected graph $G'$ in the same way as before. 
In a final step, we turn $G'$ into a directed graph by substituting every undirected edge $uv$ by a pair of two edges $(u,v), (v,u)$. We perform this substitution for every edge of $G'$ with the exception of the edges that are part of some star $F_{ab}$.
Instead, for each star $F_{ab}$, we orient the edges of $F_{ab}$ the same way as the original directed edge of $G$ between $a,b$.
It can be shown that all the arguments from the above construction still hold. 
Hence the minimum cardinality interdiction problem is $\Sigma^p_2$-complete also for directed graphs.


If one is interested in Hamiltonian paths instead of cycles, a similar modification is possible.
Inspecting the proof of \cite{DBLP:journals/siamcomp/GareyJT76} (of \cite{plesn1979np}, respectively) more closely, we find that in both constructions the graph $G$ contains some edge $e = uv$ (some edge $e = (u,v)$, respectively) such that every Hamiltonian cycle uses $e$. 
We can delete $e$ and identify the vertices $s,t$ with the endpoints of $e$.
Then a Hamiltonian $s$-$t$-path in the new graph corresponds to a Hamiltonian cycle in the old graph and vice versa.
Note that this does not increase the degree of the graph.
The rest of the proof proceeds in the same manner, both in the undirected and directed case.
Finally, the proof can also easily be adapted to the TSP by a standard reduction of undirected Hamiltonian cycle to the TSP 
(a graph $G$ is transformed into a TSP instance on the complete graph where the costs obey $c(uv) = 1$ if $ev \in E(G)$ and $c(uv) = n+1$ if $uv \not\in E(G)$).
In conclusion, we have proven that the minimum cardinality interdiction problem is $\Sigma^p_2$-complete for the directed/undirected Hamiltonian path/cycle problem and the TSP. 
