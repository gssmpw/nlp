\section{Introduction}

This paper is concerned with the \emph{minimum cardinality interdiction problem}, by which we understand the following task: Given some base problem (the so-called \emph{nominal problem}) we wish to find a small subset of elements such that this subset has a non-empty intersection with every optimal solution of the base problem.
The concept of interdiction is so natural that is has re-appeared under many different names in different research communities. 
Depending on the context, the interdiction problem (or slight variants of it) has been called the \emph{most vital node/most vital edge} problem, the \emph{blocker} problem, and \emph{node deletion/edge deletion} problem. (More details on the sometimes subtle differences between these variants is provided further below.)
As an example for the type of problems that this paper is concerned with, consider the following two problems:
\begin{quote}
    \textbf{Problem}: {\sc Min Cardinality Clique Interdiction}

    \textbf{Input}: Graph $G = (V, E)$

    \textbf{Task}: Find a minimum-size subset $V' \subseteq V$ such that every maximum clique shares at least one vertex with $V'$. 
\end{quote}

\begin{quote}
    \textbf{Problem}: {\sc Min Cardinality Hamiltonian Cycle Interdiction}

    \textbf{Input}: Graph $G = (V, E)$

    \textbf{Task}: Find a minimum-size subset $E' \subseteq E$ such that every Hamiltonian cycle shares at least one edge with $E'$.
\end{quote}

In particular, if in the above examples the set $V'$ (respectively the set $E'$) is deleted from the graph, 
the maximum clique size decreases (respectively the graph becomes Hamiltonian-cycle-free). 
Hence the interdiction problem can be interpreted as the minimal effort required to destroy all optimal solutions.
Clearly, analogous problems can be defined and analyzed for a wealth of different nominal problems. Indeed, this has been done extensively by past researchers. 
The following is a non-exhaustive list:
Interdiction-like problems have been considered already since the 90's for a large amount of problems, among others for
shortest path \cite{bar1998complexity,DBLP:journals/mst/KhachiyanBBEGRZ08,malik1989k},
matching \cite{DBLP:journals/dam/Zenklusen10a},
minimum spanning tree \cite{DBLP:journals/ipl/LinC93},
or maximum flow \cite{WOOD19931}.
Note that in all these cases the nominal problem can be solved in polynomial time. Interdiction for nominal problems that are NP-complete has also been extensively considered, for example for
vertex covers \cite{DBLP:conf/iwoca/BazganTT10, DBLP:journals/dam/BazganTT11},
independent sets \cite{DBLP:journals/gc/BazganBPR15,DBLP:conf/iwoca/BazganTT10, DBLP:journals/dam/BazganTT11,DBLP:journals/dam/HoangLW23, DBLP:conf/tamc/PaulusmaPR17},
colorings \cite{DBLP:journals/gc/BazganBPR15, DBLP:conf/iscopt/PaulusmaPR16, DBLP:conf/tamc/PaulusmaPR17},
cliques \cite{DBLP:journals/eor/FuriniLMS19, DBLP:journals/anor/Pajouh20, DBLP:journals/networks/PajouhBP14,DBLP:conf/iscopt/PaulusmaPR16},
knapsack \cite{weninger2024fast,DBLP:conf/ipco/CapraraCLW13},
dominating sets \cite{DBLP:journals/eor/PajouhWBP15},
facility location \cite{DBLP:journals/tcs/FrohlichR21},
1- and $p$-center \cite{DBLP:conf/cocoa/BazganTV10, DBLP:journals/jco/BazganTV13}, and
1- and $p$-median \cite{DBLP:conf/cocoa/BazganTV10, DBLP:journals/jco/BazganTV13}.
A general survey is provided by Smith, Prince and Geunes \cite{smith2013modern}.

This large interest is due to the fact that interdiction problems are well-motivated from many different directions. 
In the area of robust optimization, interdiction is studied because it concerns robust network design, defense against (terrorist) attacks, and sensitivity analysis \cite{DBLP:journals/corr/abs-2406-01756}. 
In particular, we want to find the most vital nodes/edges of a given network in order to identify its most vulnerable points, and understand where small changes have the largest impact. 
Interdiction in these contexts is often interpreted as a min-max optimization problem, or alternatively as a game between a network interdictor (attacker) and a network owner (defender) with competing goals.
In the area of bilevel optimization, interdiction-like problems arise naturally from the dynamic between two independent hierarchical agents \cite{DBLP:conf/ipco/CapraraCLW13}.
In the area of pure graph theory, interdiction problems are usually called vertex and edge blocker problems. 
They relate to the important concepts of maximum induced subgraphs, critical vertices and edges, cores, and transversals (with respect to some fixed property) \cite{DBLP:conf/tamc/PaulusmaPR17}.
In the area of (parameterized) complexity, interdiction-like problems are usually called vertex deletion problems.
They arise from the desire to delete a constant number of vertices until the resulting graph has some desirable property, for example so that it can be handled by an efficient algorithm.
For instance, Lewis and Yannakakis showed that the vertex deletion problem for hereditary graph properties is NP-complete \cite{DBLP:journals/jcss/LewisY80} and Bannach, Chudigiewitsch and Tantau analyzed the parameterized complexity for properties formulatable by first order formulas \cite{DBLP:conf/mfcs/BannachCT24}.


\textbf{The natural complexity of minimum cardinality interdiction.}
In this paper, we are mainly concerned with interdiction problems where the nominal problem is already NP-complete. From a complexity-theoretic point of view, such interdiction problems are often times even harder than NP-complete, namely they are complete for the second stage in the so-called \emph{polynomial hierarchy} \cite{DBLP:journals/tcs/Stockmeyer76}. A problem complete for the second stage of the hierarchy is called $\Sigma^p_2$-complete. 
The theoretical study of $\Sigma^p_2$-complete problems is important:
If a problem is found to be $\Sigma^p_2$-complete, it means that, under some basic complexity-theoretic assumptions\footnote{More specifically, we assume here that $\text{NP} \neq \Sigma^p_2$, i.e.\ we assume the polynpmial hierarchy does not collapse to the first level. Similar to the famous $\text{P} \neq \text{NP}$ conjecture, this is believed to be unlikely by experts.
However, the true status of the conjecture is not known
(see e.g.\ \cite{DBLP:journals/4or/Woeginger21}).}, it is not possible to find a mixed-integer programming formulation of the problem of polynomial size \cite{DBLP:journals/4or/Woeginger21} (also called a \emph{compact} model).
This means that no matter how cleverly a decision maker tries to design their mixed integer programming model, it must inherently have a huge number of constraints and/or variables, and may be much harder to solve than even NP-complete problems. 
Furthermore, for the type of interdiction problems discussed here, where the nominal problem is NP-complete, 
under the same assumption $\text{NP} \neq \Sigma^p_2$, one can show that (the decision variant of) the interdiction problem is often times actually not contained in the complexity class NP, only in the class $\Sigma^p_2$.
Hence the class $\Sigma^p_2$ is the natural class for this type of problem.

Even though this fact makes the study of $\Sigma^p_2$-complete problems compelling, and even though interdiction-like problems have received a large amount of attention in recent years, 
surprisingly few $\Sigma^p_2$-completeness results relevant to the area of interdiction were known until recently. 
While the usual approach to prove $\Sigma^p_2$-completeness (or NP-completeness) is to formulate a new proof for each single problem,
a recent paper by Grüne \& Wulf \cite{gruene2024completeness}, extending earlier ideas by Johannes \cite{johannes2011new} breaks with this approach. 
Instead, it is shown that there exists a large 'base list' of problems (called \emph{SSP-NP-complete} problems in \cite{gruene2024completeness}), including many classic problems like satisfiability, vertex cover, clique, knapsack, subset sum, Hamiltonian cycle, etc.
Grüne and Wulf show that for each problem from the base list, some corresponding min-max version is $\Sigma^p_2$-complete, and some corresponding min-max-min version is $\Sigma^p_3$-complete. 
This approach has three main advantages: 
1.) It uncovers a large number of previously unknown $\Sigma^p_2$-complete problems. 
2.) It reveals the theoretically interesting insight, that for all these problems the $\Sigma^p_2$-completeness follows from essentially the same argument. 
3.) It can simplify future proofs, since heuristically it seems to be true that for a new problem it is often easier to show that the nominal problem belongs to the list of SSP-NP-complete problems, than to find a $\Sigma^p_2$-completeness proof from scratch.

\textbf{Our results.}
In this paper, we extend the framework of Grüne \& Wulf \cite{gruene2024completeness} to include the case of minimum cardinality interdiction problems. 
We remark that the original framework of Grüne \& Wulf already shows such a result in the case where the action of interdicting an element is associated with so-called interdiction costs, which may be different for each element. 
Hence our work can be understood as an extension to the unit-cost case, which is arguably the most natural variant of interdiction.
Concretely, in this paper we consider minimum cardinality interdiction simultaneously for all of the following nominal problems:
\begin{quote}
    satisfiability,
    3satisfiability,
    dominating set,
    set cover,
    hitting set,
    feedback vertex set,
    feedback arc set,
    uncapacitated facility location,
    $p$-center,
    $p$-median,
    independent set,
    clique,
    subset sum,
    knapsack,
    Hamiltonian path/cycle (directed/undirected),
    TSP,
    $k$ directed vertex disjoint path ($k \geq 2$),
    Steiner tree.
\end{quote}
We show that for all these problems, the minimum cardinality interdiction problem is $\Sigma^p_2$-complete.

More abstractly, we introduce a meta-theorem from which our concrete results follows.
This means we introduce a set of sufficient conditions for some nominal problem, 
which imply that the minimum cardinality interdiction problem becomes $\Sigma^p_2$-complete.
It turns out that compared to the original framework of Grüne and Wulf, additional assumptions are necessary in the unit-cost case. 

We remark that $\Sigma^p_2$-completeness was already known in the case of clique/independent set, and knapsack \cite{DBLP:conf/ipco/CapraraCLW13,DBLP:journals/amai/Rutenburg93,DBLP:journals/corr/abs-2406-01756}. Hence our work is an extension of these results.

\textbf{Related Work}. Usually in the literature, the complexity of interdiction problems is not discussed beyond NP-hardness. 
However, there are the following exceptions: Rutenburg \cite{DBLP:journals/amai/Rutenburg93} proves $\Sigma^p_2$-completeness for clique interdiction. Caprara, Carvalho, Lodi \& Woeginger \cite{DBLP:conf/ipco/CapraraCLW13} consider different bilevel knapsack formulations and prove $\Sigma^p_2$-completeness of the DeNegre \cite{10.5555/2231641} knapsack variant, which can be interpreted as an interdiction knapsack variant.
Tomasaz, Carvalho, Cordone \& Hosteins \cite{DBLP:journals/corr/abs-2406-01756} consider interdiction-fortification games and prove $\Sigma^p_2$-completeness of another knapsack interdiction variant. 
Fröhlich and Ruzika prove $\Sigma^p_2$-completeness of a facility location interdiction problem on graphs
(in contrast to our work, the interdictor attacks edges instead of vertices) \cite[Section 4]{DBLP:journals/tcs/FrohlichR21}.
Our work extends these results to more problem classes.
Finally, in a seminal paper, Lewis \& Yannakakis prove the very general result that the most vital vertex problem is NP-hard for every nontrivial hereditary graph property \cite{DBLP:journals/jcss/LewisY80}.
Our work adds to these results by showing that in many cases, interdiction is even harder than NP-hard.
As already mentioned, our work is based on the framework by Grüne and Wulf \cite{gruene2024completeness}, which itself is based on earlier ideas by Johannes \cite{johannes2011new}.
