% !TEX root = ./main.tex
\section{Conclusion}
We have shown that for a large class of NP-complete problems, the corresponding minimum cardinality interdiction problem is $\Sigma^2_p$-complete.
With that we have also shown the hardness of several different variants of interdiction that can be found in the literature including minimum blocker and most vital elements problems.
For this, we introduced a new type of reduction, namely invulnerability reductions.
This reduction uses the corresponding minimum cost interdiction problem as basis and ensures that non-blockable elements are effectively not attackable.
The hardness of the minimum cost interdiction problem is provable via an SSP reduction.
Additionally, we show that for some problems (e.g. vertex cover, satisfiability), the $\Sigma^p_2$-completeness cannot be derived despite the fact that the minimum cost interdiction problem is $\Sigma^p_2$-complete.
Overall, we show for 23 minimum cardinality interdiction problems that they are either $\Sigma^p_2$-complete or coNP-complete.
with the ability to apply the framework to further problems.

The following natural questions arise.
First, it is of interest to find more problems for which this framework is applicable.
Furthermore, it is relevant whether this framework is also extendable to problems that are in NP but not NP-complete.
One might lose the $\Sigma^2_p$-completeness for these problems, however, a meta-theorem that proves NP-completeness for such problems and generalizes the existing results in the literature is important to obtain a deeper understanding on the structure of such problems.
At last, the results of this paper are not always applicable (albeit sometimes) to the vertex deletion or in general element deletion interdiction problem.
Thus, it is of interest to show a similar meta-theorem for element deletion problems.
