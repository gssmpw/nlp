% !TEX root = ./main.tex
\section{Framework}
\label{sec:framework} 

Since this work is an extension of the framework of Grüne \& Wulf, it becomes necessary to re-introduce the most important concepts of the framework. 
A more in-depth explanation of these concepts and their motivation in provided in the original paper \cite{gruene2024completeness}.
Grüne \& Wulf start by giving a precise definition of the objects they are interested in, linear optimization problems (LOP).
An example of an LOP problem is the vertex cover problem.

\begin{definition}[Linear Optimization Problem, from \cite{gruene2024completeness}]
\label{def:LOSPP}
    A linear optimization problem (or in short LOP)  $\Pi$ is a tuple $(\I, \U, \F, d, t)$, such that
    \begin{itemize}
        \item $\I \subseteq \{0,1\}^*$ is a language. We call $\I$ the set of instances of $\Pi$.
        \item To each instance $I \in \I$, there is some
        \begin{itemize}
            \item set $\U(I)$ which we call the universe associated to the instance $I$.
            \item set $\F(I) \subseteq \powerset{\U(I)}$ that we call the feasible solution set associated to the instance $I$. 
            \item function $d^{(I)}: \U(I) \rightarrow \Z$ mapping each universe element $e$ to its costs $d^{(I)}(e)$.
            \item threshold $t^{(I)} \in \Z$. 
        \end{itemize}
    \end{itemize}
    For $I \in \I$, we define the solution set $\sol(I) := \set{S \in \F(I) : d^{(I)}(S) \leq t^{(I)}}$ as the set of feasible solutions below the cost threshold. 
    The instance $I$ is a Yes-instance, if and only if $\sol(I) \neq \emptyset$.
    We assume (for LOP problems in NP) that it can be checked in polynomial time in $|I|$ whether some proposed set $F \subseteq \U(I)$ is feasible.
\end{definition}

\begin{description}
    \item[]\textsc{Vertex Cover}\hfill\\
    \textbf{Instances:} Graph $G = (V, E)$, number $k \in \N$.\\
    \textbf{Universe:} Vertex set $V =: \U$.\\
    \textbf{Feasible solution set:} The set of all vertex covers of $G$.\\
    \textbf{Solution set:} The set of all vertex covers of $G$ of size at most $k$.
\end{description}

It turns out that often times the mathematical discussion is a lot clearer, when one omits the concepts $\F, d^{(I)}$, and $t^{(I)}$, since for the abstract proof of the theorems only $\I$, $\U$, $\sol$ are important.
This leads to the following abstraction from the concept of an LOP problem:

\begin{definition}[Subset Search Problem (SSP), from \cite{gruene2024completeness}]
\label{def:SSP}
A subset search problem (or short SSP problem) $\Pi$ is a tuple $(\I, \U, \sol)$, such that
\begin{itemize}
    \item $\I \subseteq \set{0,1}^*$ is a language. We call $\I$ the set of instances of $\Pi$. 
    \item To each instance $I \in \I$, there is some set $\U(I)$ which we call the universe associated to the instance $I$. 
    \item To each instance $I \in \I$, there is some (potentially empty) set $\sol(I)\subseteq \powerset{\U(I)}$ which we call the solution set associated to the instance $I$.
\end{itemize}
\end{definition}

An instance of an SSP problem is a called yes-instance, if $\sol(I) \neq \emptyset$. 
Every LOP problem becomes an SSP problem with the definition $\sol(I) := \set{S \in \F(I) : d^{(I)}(S) \leq t^{(I)}}$.
We call this the \emph{SSP problem derived from an LOP problem}. Some problems are more naturally modelled as an SSP problem to begin with, rather than as an LOP problem.
For example, the satisfiability problem becomes an SSP problem with the following definition.

\begin{description}
    \item[]\textsc{Satisfiability}\hfill\\
    \textbf{Instances:} Literal set $L = \fromto{\ell_1}{\ell_n} \cup \fromto{\overline \ell_1}{\overline \ell_n}$, clause set $C = \fromto{C_1}{C_m}$ such that $C_j \subseteq L$ for all $j \in \fromto{1}{m}.$\\
    \textbf{Universe:} $L =: \U$.\\
    \textbf{Solution set:} The set of all subsets $L' \subseteq \U$ of the literals such that for all $i \in \fromto{1}{n}$ we have $|L' \cap \set{\ell_i, \overline \ell_i}| = 1$, and such that $|L' \cap C_j| \geq 1$ for all clauses $C_j \in C$.
\end{description}

Grüne \& Wulf introduce a new type of reduction, called \emph{SSP reduction}. 
Roughly speaking, a usual polynomial-time reduction from some problem $\Pi$ to another problem $\Pi'$ has the SSP property, 
if it comes with an additional injective map $f$ which embeds the universe of $\Pi$ into the universe of $\Pi'$, 
in such a way that $\Pi$ can be interpreted as a \enquote{subinstance} of $\Pi'$ and the topology of solutions is maintained in the subset that is induced by the image of $f$. 
More formally, let $W$ denote the image of $f$. We interpret $W$ as the subinstance of $\Pi$ contained in the instance of $\Pi'$ and we want the following two conditions to hold: 
1.) For every solution $S'$ of $\Pi'$, the set $f^{-1}(S' \cap W)$ is a solution of $\Pi$. 
2.) For, every solution $S$ of $\Pi$, the set $f(S)$ is a partial solution of $\Pi'$ and can be completed to a solution by using elements not in $W$.
These two conditions together are summarized in the single equation (\ref{eq:SSP}). We write $\Pi \leq_\text{SSP} \Pi'$ to denote that such a reduction exists.
We refer the reader to \cite{gruene2024completeness} for a more intuitive explanation of these properties and an example 3\textsc{Sat} $\leq_\text{SSP}$ \textsc{vertex cover}.

\begin{definition}[SSP Reduction, from \cite{gruene2024completeness}]
\label{def:ssp-reduction}
    Let $\Pi = (\I,\U,\sol)$ and $\Pi' = (\I',\U',\sol')$ be two SSP problems. We say that there is an SSP reduction from $\Pi$ to $\Pi'$, and write $\Pi \leqSSP \Pi'$, if
    \begin{itemize}
        \item There exists a function $g : \I \to \I'$ computable in polynomial time in the input size $|I|$, such that $I$ is a Yes-instance iff $g(I)$ is a Yes-instance (i.e. $\sol(I) \neq \emptyset$ iff $\sol'(g(I)) \neq \emptyset$).
        \item There exist functions $(f_I)_{I \in \I}$ computable in polynomial time in $|I|$ such that for all instances $I \in \I$, we have that $f_I : \U(I) \to \U'(g(I))$ is an injective function mapping from the universe of the instance $I$ to the universe of the instance $g(I)$ such that 
        \begin{equation}
            \set{f_I(S) : S \in \sol(I) } = \set{S' \cap f_I(\U(I)) : S' \in  \sol'(g(I))}. \label{eq:SSP}
        \end{equation}

    \end{itemize}
\end{definition}

It is shown in \cite{gruene2024completeness} that SSP reductions are transitive, i.e.\ $\Pi_1 \leqSSP \Pi_2$ and $\Pi_2 \leqSSP \Pi_3$ implies $\Pi_1 \leqSSP \Pi_3$.
The class of SSP-NP-complete problems is denoted by SSP-NPc and consists out of all SSP problems $\Pi$ that are polynomially-time verifiable and such that $\textsc{Satisfiability} \leq_\text{SSP} \Pi$. 
The main observation in \cite{gruene2024completeness} is that many classic problems are contained in the class SSP-NPc, 
and that this fact can be used to prove that their corresponding min-max versions are $\Sigma^p_2$-complete. 
