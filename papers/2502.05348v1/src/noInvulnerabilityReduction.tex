% !TEX root = main.tex
\section{Cases where the meta-theorem does not apply}\label{sec:noMeta}

It would be nice to establish a meta-theorem providing $\Sigma^p_2$-completeness of the minimum cardinality interdiction version of all nominal problems, which are SSP-NP-complete, instead of only those problems that admit an additional function $g$ with properties as stated in \Cref{thm:meta-theorem}.
However, we show in this section that this is not possible.
More precisely, we provide a lemma that guarantees that the minimum cardinality version of a problem in SSP-NP is in coNP.
Therefore, under the usual complexity-theoretic assumption NP $\neq \Sigma^p_2$, the interdiction problem is not $\Sigma^p_2$-complete.

In order to provide an intuition under which circumstances a minimum cardinality interdiction problem resides in the class coNP, we examine the vertex cover problem.
In a vertex cover, every edge $uv$ needs to be covered by at least one of the two incident vertices $u$ and $v$.
This, however, gives the attacker the opportunity to attack both $u$ and $v$ such that the edge $uv$ can never be covered.
Therefore, an attacker budget of at least $2$ results in a clear Yes-instance.
On the other hand, if the attacker budget if at most $1$, we can provide a certificate for No-instances.
We can summarize this observation in the following lemma.

\begin{lemma}\label{lem:minCardInCoNP}
    Let $\Pi = (\I, \U, \sol)$ be an SSP problem.
    If in each instance $I \in \I$ there is a subset $U' \subseteq \U(I)$ of constant size, i.e. $|U'| = O(1)$, such that for $U' \cap S \neq \emptyset$ for all $S \in \sol(I)$, then \textsc{Min Cardinality Interdiction-$\Pi$} is contained in coNP.
\end{lemma}
\begin{proof}
    Let $k$ be the interdiction budget.
    If $|U'| \leq k$, then the interdictor is able to block the whole set $U'$.
    By definition of $U'$, there is no solution $S \in \sol(I)$ such that $U' \cap S \neq \emptyset$ and thus the interdictor has a winning strategy.
    If on the other hand $k < |U'| = O(1)$, then there is a polynomially sized certificate encoding a winning strategy of the defender, i.e. a certificate for a No-instance of the problem.
    For this, we first encode the ${|\U(I)| \choose k} = |\U(I)|^{O(1)}$ possible blockers $B' \subseteq \U(I)$ and then the solution $S \in \mathcal S(I)$ such that $S \cap B' \neq \emptyset$ for all $B' \subseteq \U(I)$.
    It is possible to efficiently verify the solution by checking whether there is a solution $S \in \mathcal S(I)$ such that $S \cap B' \neq \emptyset$ for all $B' \subseteq \U(I)$ holds because the nominal problem $\Pi$ is in NP.
    It follows that the problem lies in coNP.
\end{proof}

Consider the different variants of interdiction problems introduced in \cref{sec:different-variants-of-interdiction}.
Since they are more general, \cref{lem:minCardInCoNP} does not immediately imply that those variants are contained in coNP.
However, if for each instance the stronger condition $U' \cap F \neq \emptyset$ for all feasible solutions $F \in \F(I)$ and for some constant size set $U' \subseteq \U(I)$ holds, then the \emph{full decision variant of interdiction} and the \emph{most vital element problem} are contained in coNP.
Besides the containment in coNP, we can also derive the following corollary pinpointing the complexity of minimum cardinality interdiction problems whose nominal problem is in SSP-NP.

\begin{corollary}
    Let $\Pi = (\I, \U, \sol)$ be an SSP-NP-complete problem.
    If in each instance $I \in \I$ there is a subset $U' \subseteq \U(I)$ of constant size, i.e. $|U'| = O(1)$, such that for $U' \cap S \neq \emptyset$ for all $S \in \sol(I)$, then \textsc{Min Cardinality Interdiction-$\Pi$} is coNP-complete.
\end{corollary}
\begin{proof}
    There is a reduction by restriction:
    Setting the interdiction budget $k = 0$ results in the corresponding co-problem co$\Pi$ of the nominal problem $\Pi$.
\end{proof}

\subsection{Applying the Lemma to Various Problems}

In this section, we apply \cref{lem:minCardInCoNP} to the problems mentioned earlier in this paper.
Some of the problems are affected in their original general form, e.g. vertex cover or satisfiability, while for others the lemma can be applied on a restricted version such as independent set on graphs with bounded minimum degree.
For this, we shortly describe the problem and then give the argument on how the lemma is applicable.

\textbf{Vertex Cover.}
An instance of the vertex cover interdiction problem consists of a graph $G$ and numbers $t,k \in \N_0$.
The question is if the attacker can find a set $B \subseteq V(G)$ with $|B| \leq k$ such that $B \cap S \neq \emptyset$ for every vertex cover $S$ of size at most $t$.
Now, observe that if $k \geq 2$ (and the graph is non-empty), the attacker can easily find such a set $B$ by selecting two adjacent vertices.
Thus, \cref{lem:minCardInCoNP} applies by defining $U' = \{u, v\}$ for some edge $uv \in E(G)$.
Observe that this not only destroys the solutions $S \in \sol(I)$ but also all feasible solutions $F \in \F(I)$.
Thus the minimum cardinality interdiction version, the full decision variant of interdiction and the most vital elements problem of vertex cover are coNP-complete.

\textbf{Satisfiability.}
An instance of the satisfiability interdiction problem consists out of a formula in CNF over the variables $X = \fromto{x_1}{x_n}$, with the literal set as universe, i.e. $\U = X \cup \overline X$, and interdiction budget $k$.
A similar issue as in vertex cover interdiction arises here:
If $k \geq 2$, the interdictor can just choose for some $i \in \fromto{1}{n}$ to attack both literals $x_i, \overline x_i$.
Every satisfying assignment (of non-trivial instances) contains either $x_i$ or $\overline{x_i}$, hence this is a successful attack.
Thus, \cref{lem:minCardInCoNP} applies by defining $U' = \{x, \overline x\}$ for some literal pair $x, \overline x \in \U$.
Again, this also destroys all feasible solutions $F \in \F(I)$.
Thus the minimum cardinality interdiction version, the full decision variant of interdiction and the most vital elements problem of satisfiability are coNP-complete.

\textbf{Independent Set on graphs with bounded minimum degree.}
An instance of the independent set interdiction problem consists of a graph $G =(V,E)$ with universe $U = V$, a threshold $t$ and an interdiction budget $k$.
The question of the independent set  problem is if there is a set $I \subseteq V$ such that all vertices in $I$ do not share an edge.
We now take the vertex $d$ of bounded degree into consideration.
If the attacker attacks the closed neighborhood $N[d]$ of $d$, all (optimal) solutions $S \in \mathcal S$ can be interdicted and thus \Cref{lem:minCardInCoNP} is applicable.
Thus minimum cardinality interdiction independent set on graphs with bounded minimum degree is coNP-complete.
In contrast to the other problems, this statement is not true for general feasible solutions $F \in \mathcal F(I)$. Hence we do not obtain a result for the variants from \Cref{sec:different-variants-of-interdiction}.

\textbf{Dominating Set on graphs with bounded minimum degree.}
An instance of the dominating set interdiction problem consists of a graph $G =(V,E)$ with universe $U = V$, a threshold $t$ and an interdiction budget $k$.
The question of the dominating set problem is if there is a set $D \subseteq V$ of size at most $t$ such that $D$ dominates all vertices of vertex set $V$. In other words, the union of the neighborhoods of the vertices in $D$ is the vertex set $V$, i.e. $\bigcup_{v \in D} N[v] = V$.
Again we consider a vertex $d$ of bounded degree.
Then, we can define the set of constant size to be $U' = N[d]$.
All feasible solutions $F \in \mathcal F$ have to include some vertex from $U'$ (otherwise $d$ would not be dominated).
Thus \Cref{lem:minCardInCoNP} is applicable to dominating set.
Accordingly, the minimum cardinality interdiction version, the full decision variant of interdiction and the most vital elements problem of dominating set on graphs with bounded minimum degree are coNP-complete.

\textbf{Hitting Set with bounded minimum set size.}
An instance of hitting set interdiction consists of a ground set $\{1, \ldots, n\}$ and $m$ sets $S_j \subseteq \{1,\ldots,n\}$ as well as a threshold $t$ and an interdiction budget $k$.
The universe is defined by $\U = \fromto{1}{n}$.
The question of the hitting set problem is whether there is a hitting set $H \subseteq \fromto{1}{n}$ of size at most $t$ for the sets $S_j$, that is, $H \cap S_j \neq \emptyset$ for $1 \leq j \leq m$.
We can apply \Cref{lem:minCardInCoNP} by defining $U'$ to be the set of constant size $|S_c| = O(1)$.
Then, the attacker is able to block the entire set $S_c$ such that it is not hittable, which interdicts all feasible solutions $F \in \mathcal F$.
Therefore the minimum cardinality interdiction version, the full decision variant of interdiction and the most vital elements problem of hitting set with bounded minimum set size are coNP-complete.

\textbf{Set Cover with bounded minimum coverage.}
An instance of the set cover interdiction problem consists of sets $S_i \subseteq \{1, \ldots, m\}$ for $1 \leq i \leq n$, a threshold $t$ and the an interdiction budget $k$.
The universe is defined as the sets $S_i$, $1 \leq i \leq n$.
The question of the set cover problem is whether there is selection $S \subseteq \{S_1, \ldots, S_n\}$ of size at most $k$ such that $\bigcup_{s \in S} s = \{1, \ldots, m\}$.
If there is an element $e \in \{1, \ldots, m\}$ of bounded coverage, i.e. there is a constant number of $S_i$, $1 \leq i \leq n$, with $e \in S_i$, then the attacker can attack all of these sets $S_i$.
Thus, we can apply \Cref{lem:minCardInCoNP} by choosing $U' = \{S_i \mid e \in S_i\}$ and all feasible solutions $F \in \mathcal F$ are blockable.
Accordingly, the minimum cardinality interdiction version, the full decision variant of interdiction and the most vital elements problem of set cover with bounded minimum coverage are coNP-complete.

\textbf{Steiner Tree on graphs with bounded minimum degree of terminal vertices.}
An instance of the Steiner tree interdiction problem consists of a graph $G= (S \cup T, E)$ of Steiner vertices $S$ and terminals $T$, edge weights $c: E \rightarrow \mathbb N$, a threshold $t$ and a interdiction budget $k$.
The universe is the edge set $\U = E$.
The question of the Steiner tree problem is if there is a tree $E' \subseteq E$ of weight $c(E') \leq t$ such that all terminal vertices $T$ are connected by $E'$.
If there is a terminal vertex $d \in T$ of bounded degree, then all incident edges build up a set $U' = \{dv \in E\}$ on which we can apply \Cref{lem:minCardInCoNP}.
This blocks all feasible solutions $F \in \mathcal F$.
Therefore, the minimum cardinality interdiction version, the full decision variant of interdiction and the most vital elements problem of Steiner tree on graphs with bounded minimum degree of terminal vertices are coNP-complete.

\textbf{Two Vertex-Disjoint Path on graphs with bounded degree.}
An instance of the two vertex-disjoint path interdiction problem consists of a directed graph $G=(V,A)$, vertices $s_1, s_2, t_1, t_2 \in V$ and interdiction budget $k$.
The universe is the arc set $\U = A$.
The question of the two vertex-disjoint path is if there are two paths $P_1, P_2 \subseteq A$ such that $P_i$ starts at $s_i$ and ends at $t_i$ and both paths $P_1$ and $P_2$ do not share a vertex.
If the the graph has bounded degree, we can choose any of the vertices that have to be included in on of the paths, e.g. $s_1$, and include all the incident arcs in $U' = \{(s_1, v) \in A\}$ such that we can apply \Cref{lem:minCardInCoNP}.
This blocks all feasible solutions $F \in \mathcal F$.
Accordingly, the minimum cardinality interdiction version, the full decision variant of interdiction and the most vital elements problem of two vertex-disjoint path on graphs with bounded degree are coNP-complete.

\textbf{Feedback Vertex Set on graphs with bounded girth.}
An instance of the feedback vertex set interdiction problem consists of a directed graph $G=(V,A)$, a threshold $t$ and interdiction budget $k$.
The universe is the vertex set $\U = V$.
The question of feedback vertex set is if there is a set $V' \subseteq V$ such that the graph is cycle free.
Accordingly, if the graph has bounded girth, there is a cycle of bounded length, which the attacker can attack or in other words, the cycle cannot be deleted by the defender by choosing a corresponding vertex to be in the feedback vertex set.
Thus all feasible solutions $F \in \mathcal F$ are blockable by applying \Cref{lem:minCardInCoNP} with $U' = \{v \in V \mid v \text{ is part of the smallest cycle in } G\}$.
Therefore, the minimum cardinality interdiction version, the full decision variant of interdiction and the most vital elements problem of feedback vertex set on graphs with bounded girth are coNP-complete.

\textbf{Feedback Arc Set on graphs with bounded girth.}
An instance of the feedback arc set interdiction problem consists of a directed graph $G=(V,A)$, a threshold $t$ and interdiction budget $k$.
The universe is the arc set $\U = A$.
The question of feedback arc set is if there is an arc set $A' \subseteq A$ such that the graph is acyclic.
We can use the same argument as in feedback vertex set.
That is, the attacker can choose the arcs of the smallest cycle in $G$.
Thus all feasible solutions $F \in \mathcal F$ are blockable by applying \Cref{lem:minCardInCoNP} with $U' = \{a \in A \mid a \text{ is part of the smallest cycle in } G\}$.
Therefore, the minimum cardinality interdiction version, the full decision variant of interdiction and the most vital elements problem of feedback arc set on graphs with bounded girth are coNP-complete.


\textbf{Uncapacitated Facility Location, p-Center, p-Median with bounded minimum customer coverage.}
An instance of the minimum cardinality interdiction version of these three problems consists of a set of potential facilities $F$ and a set of clients $C$ together with a cost function on the facilities $f: F \rightarrow \mathbb N$ and a service cost function $c: F \times C \rightarrow \mathbb N$ as well as a threshold $t$ and an interdiction budget $k$.
The universe is the facility set $\U = F$ and it is asked for a set of facilities $F' \subseteq F$ not exceeding the cost threshold $t$.
If the coverage of one customer is bounded, i.e. there is a bounded number of potential facilities that are able to serve the customer, the attacker is able to block all of these.
Thus we can define $U'$ as the set of facilities that are able to serve the customer of bounded coverage such that all feasible solutions $F \in \mathcal F$ can be interdicted.
Therefore, we can apply \Cref{lem:minCardInCoNP} and the minimum cardinality interdiction version, the full decision variant of interdiction and the most vital elements problem of these three facility locations problems with bounded minimum customer coverage are coNP-complete.

\textbf{Hamiltonian path/cycle (directed/undirected), TSP on graphs with bounded minimum degree.}
An instance of the minimum cardinality interdiction version of these problems consists of a graph $G=(V,E)$ (respectively $G=(V,A)$ in the directed case) and an interdiction budget $k$.
The universe is the set of edges $\U = E$ (respectively the set of arcs $\U = A$).
The question is whether there is a Hamiltonian path or cycle in $G$, i.e. a path/cycle that visits every vertex exactly once.
Because there is a vertex $d$ of bounded degree which has to be visited, we can define the set of constant size $U' = \{dv \in E\}$ (respectively $U' = \{(d,v),(v,d) \in A\}$).
If the set $U'$ is blocked it is not possible to visit the vertex, thus all feasible solutions $F \in \mathcal F$ can be interdicted.
Therefore, we can apply \Cref{lem:minCardInCoNP} and the minimum cardinality interdiction version, the full decision variant of interdiction and the most vital elements problem of these five Hamiltonian problems on graphs with bounded minimum degree are coNP-complete.

\subsection{Satisfiability with Universe over the Variables}

In the previous subsection we explained why minimum cardinality interdiction-\textsc{Sat} is contained in coNP, hence likely not $\Sigma^p_2$-complete.
Note that this is a consequence of our choice of definition of \textsc{Satisfiability}, where we explicitly defined the universe to be the literal set $L = X \cup \overline X$.
As a consequence, the interdictor may attack $X \cup \overline X$. 
\begin{samepage}
    \begin{mdframed}
    	\begin{description}
        \item[]\textsc{Satisfiability ($\U = L$)}\hfill\\
        \textbf{Instances:} Literal Set $L = \fromto{x_1}{x_n} \cup \fromto{\overline x_1}{\overline x_n}$, Clauses $C \subseteq \powerset{L}$\\
        \textbf{Universe:} $L =: \U$.\\
        \textbf{Solution set:} The set of all sets $L' \subseteq \U$ such that for all $i \in \fromto{1}{n}$ we have $|L' \cap \set{\ell_i, \overline \ell_i}| = 1$, and such that $|L' \cap c_j| \geq 1$ for all $c_j \in C$.
    	\end{description}
    \end{mdframed}
\end{samepage}

An interesting behavior occurs, when we consider the following alternative version \textsc{Satisfiability ($\U = X$)}. 
\begin{samepage}
    \begin{mdframed}
    	\begin{description}
        \item[]\textsc{Satisfiability  ($\U = X$)}\hfill\\
        \textbf{Instances:} Variable Set $X = \fromto{x_1}{x_n}$, Clauses $C \subseteq 2^{X \cup \overline X}$ \\
        \textbf{Universe:} $X =: \U$.\\
        \textbf{Solution set:} The set of all sets $X' \subseteq \U$ such that the assignment $\alpha: X \rightarrow \{0,1\}$ with $\alpha(x) = 1 \leftrightarrow x \in X'$ satisfies all clauses in $C$.
        \end{description}
    \end{mdframed}
\end{samepage}
Here the universe is only the variable set $X$, so in the interdiction version, the interdictor may only attack $X$, i.e.\ the interdictor may target individual variables and enforce that they must be set to \emph{false}. 
We show now that in contrast to the variant, where the universe is the literal set, in this new variant the interdiction problem is $\Sigma^p_2$-complete again. 
Since the problem \textsc{Satisfiability ($\U = X$)} is not part of the original problem set of \cite{gruene2024completeness}, we perform this proof in two steps.
\begin{lemma}
    \textsc{Satisfiability ($\U = X$)} is SSP-NP-complete, even when all clauses are restricted to length at most three.
\end{lemma}
\begin{proof}
    We provide an SSP reduction from the SSP-NP-complete problem \textsc{Satisfiability ($\U = L$)} to \textsc{Satisfiability ($\U = X$)}. 
    Consider an instance of \textsc{Satisfiability ($\U = L$)} given by a formula $\varphi$ with $n$ variables $X = \fromto{x_1}{x_n}$ and universe/literal set $\U = L = X \cup \overline{X}$.
    \textsc{Satisfiability ($\U = L$)} is SSP-NP-complete even when all clauses are restricted to length three, so let us w.l.o.g.\ assume that property.
    We have to show how to embed this universe into the universe $\U'$ of some corresponding \textsc{Satisfiability ($\U = X$)} instance $\varphi'$, where only positive literals are allowed in $\U'$.
    This can be done the following way:
    We introduce $2n$ new variables $X' := \fromto{x^t_1}{x^t_n} \cup \fromto{x^f_1}{x^f_n}$.
    The universe $\U' := X'$ consists out of the $2n$ corresponding positive literals $X'$.
    The new formula $\varphi'$ is defined from $\varphi$ in two steps.
    First a substitution process takes place: 
    For each $i=1,\dots,n$, the positive literal $x_i$ is replaced by the positive literal $x^t_i$ and each negative literal $\overline x_i$ is replaced by the positive literal $x_i^f$.
    In a second step, the clauses $(x^t_i \lor \overline x^f_i) \land (\overline x^t_i \lor x^f_i)$ (note that these are equivalent to $x_i^t \oplus x_i^f$) are added to $\varphi'$.
    Formally,
    \[
        \varphi' = \text{substitute}(\varphi) \land \bigwedge_{i=1}^n (x_i^t \lor x_i^f)\land (\overline{x}_i^t \lor \overline{x}^f_i).
    \]
    The SSP reduction is completed by specifying the embedding function $f : \U \to \U'$ via $f(x_i) := x_i^t$ and $f(\overline x_i) := x_i^f$.
    Clearly all clauses of $\varphi'$ have length at most three.
    Note that this reduction is a correct reduction, i.e.\ it transforms yes-instances into yes-instances and no-instances into no-instances, because the added constraints make sure that exactly one of $x_i^t$ and $x_i^f$ is true.
    Furthermore, it has the SSP property:
    For every solution $S \subseteq \U$ of \textsc{Satisfiability ($\U = L$)}, the \enquote{translated} set $f(S) \subseteq \U'$ is a solution of \textsc{Satisfiability ($\U = X$)}.
    Furthermore, for every solution $S \subseteq \U'$ of \textsc{Satisfiability ($\U = X$)}, the set $f^{-1}(S)\subseteq \U$ is a solution of \textsc{Satisfiability ($\U = L$)}.
    Accordingly, we have a correct SSP reduction (where the SSP mapping $f$ is even bijective due to $f(\U) = \U'$). 
\end{proof}

\begin{theorem}
    \textsc{Min Cardinality Interdiction-Satisfiability ($\U = X$)} is $\Sigma^p_2$-complete.
\end{theorem}
\begin{proof}
    By the previous lemma, \textsc{Satisfiability ($\U = X$)} is SSP-NP-complete, even if all clauses are restricted to length three.
    Due to \cite{gruene2024completeness}, the problem \textsc{Comb. Interdiction-Satisfiability ($\U = X$)} is $\Sigma^p_2$-complete, even if all clauses are restricted to length three.
    We provide a reduction from the latter problem in terms of an invulnerability gadget analogous to the gadgets presented in \cref{sec:invulnerability-gadgets}. 
    For this, consider an instance of \textsc{Satisfiability ($\U = X$)} with formula $\varphi$ in CNF and every clause of length three, together with the universe $\U = \fromto{x_1}{x_n}$, a set $C \subseteq \U$ of vulnerable literals, and interdiction budget $k \in \N_0$.
    For every variable $x_i \in \U \setminus C$, we explain how to make $x_i$ invulnerable.
    We introduce $k+1$ new variables $x^{(1)}_i, \dots x^{(k+1)}_i$.
    Our goal is to establish the equivalence
    \[
        x_i \equiv x^{(1)}_i \lor \dots \lor x^{(k+1)}_i.
    \]
    We can achieve this through means of the following substitution process starting from formula $\varphi$: 
    Every occurrence of $x_i$ in the formula gets substituted by $x^{(1)}_i \lor \dots \lor x^{(k+1)}_i$. 
    Every occurrence of $\overline x_i$ gets substituted (by De Morgan's law) by $(\overline x^{(1)}_i \land \dots \land \overline x^{(k+1)}_i)$.
    Note that this has two effects: 
    First, the length of a clause may now exceed 3.
    Secondly, the formula is not in CNF anymore. 
    Note however that we can use the distributive law to expand every clause that is not in CNF. 
    Since before each clause before had a length of at most three, this results in a blow-up of the instance size of a factor at most $(k+1)^3$, i.e.\ at most a polynomial factor.
    Let $\varphi'$ be the resulting formula. 
    We can see that there is an equivalence of the satisfying assignments of $\varphi$ and $\varphi'$, in the sense that $x_i$ is true in $\varphi$ if and only if $x^{(1)}_i \lor \dots \lor x^{(k+1)}_i$ is true in $\varphi'$ (for all invulnerable $x_i$). 
    However, since the interdiction budget is only $k$, the interdictor can never enforce $x^{(1)}_i \lor \dots \lor x^{(k+1)}_i$ to be false for all invulnerable variables.
    This shows that \textsc{Comb. Interdiction-Satisfiability ($\U = X$)} reduces to \textsc{Min. Cardinality Interdiction-Satisfiability ($\U = X$)}, hence proving its $\Sigma^p_2$-completeness.
\end{proof}

Note that the reasoning presented in this proof was slightly different from \cref{thm:meta-theorem}, since we start with a formula where every clause has length three, but do not preserve this property during the proof.
Hence $\Sigma^p_2$-completeness is only shown in the case where clauses can have unrestricted length.

We can use an argument similar to \Cref{lem:minCardInCoNP} to show the coNP-completeness of the minimum cardinality interdiction version, the full decision variant of interdiction and the most vital elements problem of {\sc $b$-Satisfiability ($\U = X$)}, i.e. with clauses of length bounded by $b$.
Indeed, it is easy to see that the interdiction problem of \textsc{Satisfiability ($\U = X$)} where every clause has length three is coNP-complete:
If $k \geq 3$ holds for the interdiction budget, the attacker distinguishes two cases:
If there is a clause with three positive literals, the attacker blocks all of them and immediately wins. 
In the other case, every clause has at least one negative literal.
Then the attacker can never win, since the defender can set every variable to false, which is a satisfying assignment that can never be blocked.
By an analogous argument, we can see that for any $t = O(1)$, the interdiction problem of \textsc{Satisfiability ($\U = X$)} with clauses restricted to length $t$ is coNP-complete.

Finally, we remark that slightly different variants of interdiction-3-Sat have been shown to be $\Sigma^p_2$-complete. In these variants, the interdictor does not have access to all variables (see \cite[Sec. 4.2]{gruene2024completeness} or \cite[Thm. 1]{jackiewicz2024computational}).




















% It would be nice to have our main theorem, \cref{thm:meta-theorem}, for all problems from the class SSP-NPc, instead of only those who admit an additional function $g$ with properties as stated. However, we show in this section that this is not possible.
% Concretely, we show in this section that for the following SSP-NP-complete problems, their interdiction version $\textsc{Min. Card. Interdiction-$\Pi$}$ is contained in the class coNP: Vertex cover, satisfiability, Partition/two-machine scheduling. 
% Therefore, under the usual complexity-theoretic assumption $NP \neq \Sigma^p_2$, the interdiction problem is not $\Sigma^p_2$-complete.

% In the case of the satisfiability problem, it turns out that certain variants of it remain $\Sigma^p_2$-complete, while others do not. We discuss these details below. 
% In the following, it is helpful to interpret interdiction as a game between attacker and defender.

% \textbf{Vertex cover.} An instance of the vertex cover interdiction problem consists out of a graph $G$ and numbers $t,k \in \N_0$.
% The question is if the attacker can find a set $B \subseteq V(G)$ with $|B| \leq k$ such that $B \cap C \neq \emptyset$ for every vertex cover $C$ of size at most $t$.
% Now, observe that if $k \geq 2$ (and the graph is non-empty), the attacker can easily find such a set $B$ by selecting two adjacent vertices. 
% Every vertex cover has a non-trivial intersection with $B$. Hence every instance with $k \geq 2$ is a yes-instance.
% On the other hand, if $k \leq 1$, if the defender has a winning strategy, then this strategy can be encoded in polynomial space: 
% For each choice of the attacker, some corresponding vertex cover that avoids this attack exists. A list of these at most $O(n)$ vertex covers correctly encodes the strategy.
% Since each no-instance can be certified in polynomial space, the problem is contained in coNP. It is not hard to see, that it is coNP-hard as well, hence it is coNP-complete. 

% \textbf{Satisfiability.} An instance of the satisfiability interdiction problem consists out of a formula in CNF over the variables $X = \fromto{x_1}{x_n}$, with universe $\U = X \cup \overline X$, and interdiction budget $k$.
% A similar issue as in vertex cover interdiction arises here: If $k \geq 2$, the interdictor can just choose for some $i \in \fromto{1}{n}$ to attack both literals $x_i, \overline x_i$.
% Every satisfying assignment contains either $x_i$ or $\overline{x_i}$, hence this is a successful attack.
% By a reasoning analogous to the case of vertex cover, this shows that interdiction-SAT is coNP-complete.



% \textbf{Partition/two-machine scheduling.}% \lasse{Partition is currently missing from the list in the appendix. Do we want to add it?}
% An instance of the partition interdiction problem consists out of a set of numbers $A = \fromto{a_1}{a_n}$ and an interdiction budget $k$. The universe is $\U = A$.
% Note that the partition problem has an interesting property: 
% For every set $S \subseteq \fromto{a_1}{a_n}$ which has the property that the sum of $S$ is exactly $1/2$ of the total sum of $A$, 
% the complement of $S$ has the same property. 
% Since the partition problem has this symmetric property, but the satisfiability problem does not have it, 
% a SSP reduction to partition can only exist if this symmetry is broken. Therefore in \cite{grüne2024completeness} in order to break the symmetry the partition problem is defined the following way: 
% A solution of the partition problem is a set $S \subseteq \fromto{a_1}{a_n}$ such that $S$ sums up to $1/2$ the total sum of $A$ and $a_1 \in S$.
% Under this interpretation, an attack of the interdictor on some item $a_i$ means that the interdictor enforces that $a_i$ is not in the same part of the partition as $a_1$.
% Note however, that with this definition the minimum cardinality partition interdiction the interdictor can always attack $a_1$ itself. Hence, analogous to the cases above, the problem is coNP-complete.
% Finally, we remark that even if one would choose another natural definition, 
% where the universe has two items 
% $(a_i, 1)$ and $(a_i, 2)$ for each number $a_i$, indicating whether this number is packed into part 1 or part 2, a similar pattern occurs:
% As soon as $k \geq 2$, the interdictor can attack both $(a_i, 1)$ and $(a_i, 2)$ for some $i$.
% Hence even under this alternate definition, the partition interdiction problem is coNP-complete.


% \subsection{The case of satisfiability}
% In the previous subsection we explained why interdiction-SAT is contained in coNP, hence likely not $\Sigma^p_2$-complete. Note that this is a consequence of our choice of definition of the \textsc{Satisfiability} problem, where we explicitly defined the universe to be $X \cup \overline X$. As a consequence, the interdictor may attack $X \cup \overline X$. 
% \begin{samepage}
%     \begin{mdframed}
%     	\begin{description}
%         \item[]\textsc{Satisfiability}\hfill\\
%         \textbf{Instances:} Literal Set $L = \fromto{x_1}{x_n} \cup \fromto{\overline x_1}{\overline x_n}$, Clauses $C \subseteq \powerset{L}$\\
%         \textbf{Universe:} $L =: \U$.\\
%         \textbf{Solution set:} The set of all sets $L' \subseteq \U$ such that for all $i \in \fromto{1}{n}$ we have $|L' \cap \set{\ell_i, \overline \ell_i}| = 1$, and such that $|L' \cap c_j| \geq 1$ for all $c_j \in C$.
%     	\end{description}
%     \end{mdframed}
% \end{samepage}

% Interesting behavior occurs, when we consider the following alternative version \textsc{Satisfiability'}. 
% \begin{samepage}
%     \begin{mdframed}
%     	\begin{description}
%         \item[]\textsc{Satisfiability'}\hfill\\
%         \textbf{Instances:} Variable Set $X = \fromto{x_1}{x_n}$, Clauses $C \subseteq 2^{X \cup \overline X}$ \\
%         \textbf{Universe:} $X =: \U$.\\
%         \textbf{Solution set:} The set of all sets $X' \subseteq \U$ such that the assignment $\alpha: X \rightarrow \{0,1\}$ with $\alpha(x) = 1 \leftrightarrow x \in X'$ satisfies all clauses in $C$.
%         \end{description}
%     \end{mdframed}
% \end{samepage}
% Here the universe is only $X$, so in the interdiction version, the interdictor may only attack $X$, i.e.\ the interdictor may target individual variables and enforce that they must be set to '0'. 
% We show now that in contrast to the old variant, 
% in this new variant the interdiction problem is $\Sigma^p_2$-complete again. 
% Since the problem \textsc{Satisfiability'} is not part of the original problem set of \cite{grüne2024completeness},
% we perform this proof in two steps.
% \begin{lemma}
% Problem \textsc{Satisfiability'} is SSP-NP-complete, even when all clauses are restricted to length at most three.
% \end{lemma}
% \begin{proof}
% We provide a SSP-reduction from the (old) SSP-NP-complete problem \textsc{Satisfiability} to the (new) problem \textsc{Satisfiability'}. 
% Consider an instance of \textsc{Satisfiability} given by a formula $\varphi$ with $n$ variables $X = \fromto{x_1}{x_n}$ and universe/literal set $\U = L = X \cup \overline{X}$. \textsc{Satisfiability} is SSP-NP-complete even when all clauses are restricted to length three, so let us w.l.o.g.\ assume that property.
% We have to show how to embed this universe into the  universe $\U'$ of some corresponding \textsc{Satisfiability'} instance $\varphi'$, where only positive literals are allowed in $\U'$.
% This can be done the following way: We introduce $2n$ new variables $X' := \fromto{x^t_1}{x^t_n} \cup \fromto{x^f_1}{x^f_n}$. The universe $\U' := X'$ consists out of the $2n$ positive literals $X'$.
% The new formula $\varphi$ is defined from $\varphi$ in two steps. First a substitution process takes place: 
% For each $i=1,\dots,n$ each positive literal $x_i$ gets replaced by the positive literal $x^t_i$. 
% Each negative literal $\overline x_i$  gets replaced by the positive literal $x_i^f$.
% In a second step, all the conditions $x_i^t \not \leftrightarrow x_i^f$ are added to $\varphi'$.
% (Note that this is equivalent to $(x_i^t \lor x_i^f)\land (\overline{x}_i^t \lor \overline{x}^f_i)$). Formally,
% \[
% \varphi' = \text{substitute}(\varphi) \bigwedge_{i=1}^n (x_i^t \lor x_i^f)\land (\overline{x}_i^t \lor \overline{x}^f_i).
% \]
% The SSP reduction is completed by specifying the embedding function $f : \U \to \U'$ via $f(x_i) := x_i^t$ and $f(\overline x_i) := x_i^f$.
% Clearly all clauses of $\varphi'$ have length at most three. Note that this reduction is a correct reduction, i.e.\ it transforms yes-instances into yes-instances and no-instances into no-instances, because the added constraints make sure that exactly one of $x_i^t, x_i^f$ is true.
% Furthermore, it has the SSP property: For every solution $S \subseteq \U$ of \textsc{Satisfiability}, 
% the \enquote{translated} set $f(S) \subseteq \U'$ is a solution of \textsc{Satisfiability'}.
% Furthermore, for every solution $S \subseteq \U'$ of \textsc{Satisfiability'}, the set $f^{-1}(S)\subseteq \U$ is a solution of \textsc{Satisfiability}.
% These two facts together imply that we have a correct SSP reduction (in this case the SSP property is simpler because not only $f(\U) \subseteq \U'$, but even $f(\U) = \U'$). 
% \end{proof}

% \begin{theorem}
%     Problem \textsc{Min. Card. interdiction-Satisfiability'} is $\Sigma^p_2$-complete.
% \end{theorem}
% \begin{proof}
%     By the previous lemma, \textsc{Satisfiability'} is SSP-NP-complete, even if all clauses are restricted to length three.
%     Due to \cite{grüne2024completeness}, the problem $\textsc{Comb. Interdiction-Satisfiability'}$ is $\Sigma^p_2$-complete, even if all clauses are restrcited to length three.
%     We provide a reduction from the latter problem in terms of an invulnerability gadget analogous to the gadgets presented in \cref{sec:invulnerability-gadgets}. 
%     Indeed, consider an instance of \textsc{Satisfiability'} with formula $\varphi$ in CNF and every clause of length three, together with the universe $\U = \fromto{x_1}{x_n}$, a set $C \subseteq \U$ 
%     of vulnerable literals, and interdiction budget $k \in \N_0$.
%     For every variable $x_i \in \U \setminus C$, we explain how to make $x_i$ invulnerable.
%     We introduce $k+1$ new variables $x^{(1)}_i, \dots x^{(k+1)}_i$. Our goal is to have the equivalence
%     \[
%         x_i \equiv x^{(1)}_i \lor \dots \lor x^{(k+1)}_i.
%     \]
%     We can achieve this equivalence through means of the following substitution process starting from formula $\varphi$: 
%     Every occurence of $x_i$ in the formula gets substituted by $x^{(1)}_i \lor \dots \lor x^{(k+1)}_i$. 
%     Every occurence of $\overline x_i$ gets substituted (by De Morgan's law) by $(\overline x^{(1)}_i \land \dots \land \overline x^{(k+1)}_i)$.
%     Note that this has two effects: 
%     First, the length of a clause may now exceed 3. Secondly, the formula is not in CNF anymore. 
%     Note however that we can use the distributive law to expand every clause that is not in CNF. 
%     Since before each clause before had a length of at most three, this results in a blow-up of the instance size of a factor at most $(k+1)^3$, i.e.\ at most a polynomial factor.
%     Let $\varphi'$ be the resulting formula. 
%     We can see that there is an equivalence of satisfying assignment of $\varphi$ and $\varphi'$, in the sense that $x_i$ is true in $\varphi$ if and only if $x^{(1)}_i \lor \dots \lor x^{(k+1)}_i$ is true in $\varphi'$ (for all invulnerable $x_i$). 
%     However, since the interdiciton budget is only $k$, for all invulnerable variables we see that the interdictor can never enforce $x^{(1)}_i \lor \dots \lor x^{(k+1)}_i$ to be false.
%     This shows that \textsc{Comb. Interdiction-Satisfiability'} reduces to \textsc{Min. Card. interdiction-Satisfiability'}, hence proving $\Sigma^p_2$-completeness.
    
% \end{proof}

% Note that the reasoning presented in this proof was slightly different from \cref{thm:meta-theorem}, 
% since we start with a formula where every clause has length three, but do not preserve this property during the proof. Hence $\Sigma^p_2$-completeness is only shown in the case where clauses can have unrestricted length.

% Indeed, it is easy to see that the interdiction problem of \textsc{Satisfiability'} 
% where every clause has length three is coNP-complete: If $k \geq 3$ holds for the interdiction budget, the attacker distinguishes two cases:
% If there is a clause with three positive literals, the attacker blocks all of them and immediately wins. 
% In the other case, every clause has at least one negative literal.
% Then the attacker can never win, since the defender can set every variable to false, which is a satisfying assignment that can never be blocked.
% By an analogous argument, we can see that for any $t = O(1)$, the interdiction problem of \textsc{Satisfiability'} 
% with clauses restricted to length $t$ is coNP-complete.

% Finally, we remark that slightly different variants of interdiction-3-Sat have been shown to be $\Sigma^p_2$-complete. In these variants, the interdictor does not have access to all variables (see \cite[Sec. 4.2]{grüne2024completeness} or \cite[Thm. 1]{jackiewicz2024computational}).


% \christoph{Ich glaube, wir sollten noch einiges an Überlegung in diese Section investieren. Wenigstens ein meta-theorem für eine hinreichende Bedingung sollte relativ schnell aufstellbar sein. Besser wäre natürlich eine richtige Dichotomie.}

% \christoph{Unter diese Probleme fallen auch so ziemlich alle Graphprobleme auf Bounded Degree Graphen. Wenn Du Dir die Gadgets anschuast, die wir gebastelt haben, dann verlangen diese, dass wir einen nicht konstanten Grad haben nämlich in O(Größe des Budgets des Angreifers). Grundlegend sind alle Probleme betroffen, die eine kostant lange Klausel mit rein-positiven Literalen erzeugen. Dann kann nämlich genau diese angegriffen werden. Dann kann immer das k-Sat-Argument angewendet werden.
% D.h. wenn das Problem durch per Reduktion in eine k-CNF-Sat-Instanz transformieren kann (hier muss wohl noch die SSP Eigenschaft gelten), dann ist dies auch einfach angreifbar und lediglich coNP-vollständig.
% Für Graph-Probleme auf Bounded Degree Graphen könnte man vielleicht den zugehörigen Bounded Search tree Algorithmus (aus der FPT-Theorie) als k-CNF-Sat-Instanz kodieren und damit die coNP-Vollständigkeit zeigen.
% Das wäre wenigstens ein hinreichendes Kriterium, aber noch kein notwendiges...

% Wenn wir so etwas zeigen könnten, dann haben wir eine coole Verbindung dieser Probleme mit der parametrisierten Welt und dann auch Model Checking/Datenbanktheorie, das könnte für ein größeres Publikum interessant sein.
% }
% \lasse{Das klingt sehr interessant und auch plausibel. Allerdings weiß ich momentan nicht, wie genau man dieses meta-theorem mit der Sprache unseres frameworks beschreiben könnte. Würde es nicht auch ausreichen zu sagen: Wir beobachten das Muster, dass ein ähnliches Argument wie die coNP-completeness von interdiction-kSAT anscheinend oft gemacht werden kann, insbesondere wenn das problem als eine Vereinigung von lauter constraints begrenzter Länge interpretiert werden kann.}

% \begin{itemize}
% \item dominating set in graphs with minimum degree constant, e.g. in planar graphs, bounded genus graphs.
% \item $k$-sat for $k = O(1)$
% \item vertex cover, oder allgemeiner hitting set mit $\min|S_i| = O(1)$, wo $S_i$ die sets die gehittet werden müssen
% \item oder bzw. set cover wenn $\min_e \text{coverage}(e) = O(1)$, wobei für ein element $e$ im ground set $\Omega$ $\text{coverage}(e) = |\set{S \in \U : e \in S}|$.
% \item Steiner tree, two-vertex disjoint path in $O(1)$ degree graphs
% \item feedback arc set, feedback vertex set in graphs of constant girth
% \item facility location, $p$-center, $p$-median, if there exists some customer for who only $O(1)$ possible locations exist which could serve the customer
% \item clique, independent set, in $O(1)$-degree graphs, but only the interdiction version, where optimal solutions must be interdicted $(t = t^\star)$. For the more general interdiciton problem this is not true
% \end{itemize}

% \begin{lemma}
%     If $\Pi$ is a problem in SSP-NP such that there exists in each instance $I$ of $\Pi$ a small subset $U' \subseteq \U(I)$ of size $|U'| = O(1)$ the universe of $\Pi$, such that every solution intersects $U'$, i.e. $U' \cap S \neq \emptyset$ for all $S \in \sol(I)$, then $\textsc{Min. Card-Interdiction-$\Pi$}$ is contained in coNP.
% \end{lemma}
% \begin{proof}
%     Let $t := |U'|$.
%     If the budget $k$ of the interdictor is at least $t$, then the interdictor trivially wins by interdicting all of $U'$. 
%     If $k < t$, and if the defender has a winning strategy, 
%     we can encode a winning strategy of the defender, by enumerating for each potential attack $B \subseteq \U(I)$ with $|B| \leq k$ a solution $S \in \sol(I)$ that avoids $B$.
%     Since $t = O(1)$, these $O(|U(I)|^t)$ possibilities can be encoded in polynomial space.
% \end{proof}
%  % Falls man beweisen kann: Es existiert in jeder Instanz eine Teilmenge $U' \subseteq \U$ mit $|U'| = O(1)$, und jede Lösung $S \in \sol$ des nominalen Problems mindestens ein Element mit $U'$ gemeinsam, dann gilt: $\textsc{Min. Card-Interdiction-$\Pi$} \in$ co-NP. 

% We remark that one could also ask whether the above statement about coNP-containment is also true for the more general interdiction problems introduced in \cref{sec:different-variants-of-interdiction}.
% Note that the problems from \cref{sec:different-variants-of-interdiction} are more general than $\textsc{Min. Card-Interdiction-$\Pi$}$, hence the coNP-containment of the latter does not imply coNP-containment of the first.
% One can show that if an equivalent statement holds about the feasible solutions $F \in \F$ instead of the optimal solutions $S \in \sol$, then also the problems from \cref{sec:different-variants-of-interdiction} are contained in coNP (i.e.\ there exists $U' \subseteq \U$ of size $|U'| =  O(1)$ such that $\forall F \in \F: F \cap U' \neq \emptyset$).