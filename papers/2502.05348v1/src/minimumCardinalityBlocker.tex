% !TEX root = main.tex
\section{Minimum Cardinality Interdiction Problems}

In this section, we prove our $\Sigma^p_2$-completeness results regarding the minimum cardinality interdiction problem. 
Since we want to prove the theorem simultaneously for multiple problems at once, we require an abstract definition of the interdiction problem.
For this, consider the following definition.

\begin{definition}[Minimum Cardinality Interdiction Problem]
\label{def:min-card-interdiction-pi}
    Let an SSP problem $\Pi = (\I, \U, \sol)$ be given.
    The minimum cardinality interdiction problem associated to $\Pi$ is denoted by \textsc{Min Cardinality Interdiction-$\Pi$} and defined as follows:
    The input is an instance $I \in \I$ together with a number $k \in \N_0$.
    The question is whether
    \[
        \exists B \subseteq \U(I), \ |B| \leq k : \forall S \in \sol(I) : B \cap S \neq \emptyset.
    \]
\end{definition}

For the remainder of the paper, it is helpful to imagine this problem as a game between two players: the \emph{attacker} and the \emph{defender}.
That is, interdiction is an action performed by an attacker (or interdictor), who wishes to select a blocker of few elements to destroy all solutions.
On the other hand, the defender wants to find a solution to the problem after the attacker selected a blocker.
This leads to the following interpretation:
\begin{itemize}
    \item The set $\U(I)$ contains all the elements the attacker is allowed to attack. 
    \item The set $\sol(I)$ contains all the solutions the attacker wants to destroy such that the defender is not able to find any solution.
    For example, this could be the set of all Hamiltonian cycles, the set of all cliques of a certain size, etc.
\end{itemize}
Therefore, the formulation of the base problem as SSP problem $(\I, \U, \sol)$ determines which elements the attacker can attack, which he cannot attack (e.g. edges/vertices of a graph), and what the attacker's goal is.
We note that different formulations $(\I, \U, \sol)$ of the same problem are formally different SSP problems. They might be both SSP-NP-complete independent of each other, but require their own SSP-NP-completeness proof each.
For all the concrete problems studied in this paper, our complexity results hold for the natural choices of $(\I, \U, \sol)$ formally given in \cref{app:sec:problemDefinitions}.
Finally, note that if the base problem is an LOP problem, then by definition $\sol(I)$ is the set of feasible solutions below some threshold specified in the input. 
For example, applying \cref{def:min-card-interdiction-pi} to $\Pi = \textsc{Clique}$ yields the following decision problem: 
\begin{quote}
    \textbf{Problem}: $\textsc{Min Cardinality Interdiction-Clique}$

    \textbf{Input}: Graph $G = (V, E)$, numbers $k, t \in\N_0$

    \textbf{Question}: Does there exist a subset $B \subseteq V$ of size $|B| \leq k$ such that every clique of size at least $t$ shares at least one vertex with $B$? 
\end{quote}

Some more technical details, concerning the subtle differences between different variants of interdiction referenced in the literature as well as concerning the question whether $t$ can be chosen to be optimal are discussed in \cref{sec:different-variants-of-interdiction}.
We now proceed with the main result.
For the complexity analysis of minimum cardinality blocker, we first show the containment in the class $\Sigma^p_2$, if the nominal problem is in NP.

\begin{lemma}
\label{lem:containment}
    Let $\Pi = (\I, \U, \sol)$ be an SSP problem in $NP$, then \textsc{Min Cardinality Interdiction}-$\Pi$ is in $\Sigma^p_2$.
\end{lemma}
\begin{proof}
    We provide a polynomial time algorithm $V$ that verifies a specific solution $y_1, y_2$ of polynomial size for instance $I$ such that
    $$
        I \in L \ \Leftrightarrow \ \exists y_1 \in \{0,1\}^{m_1} \ \forall y_2 \in \{0,1\}^{m_2} : V(I, y_1, y_2) = 1.
    $$
    With the $\exists$-quantified $y_1$, we encode the blocker $B \subseteq \U(I)$.
    The encoding size of $y_1$ is polynomially bounded in the input size of $\Pi$ because $|\U(I)| \leq poly(|x|)$.
    Next, we encode the solution $S \in \sol(I)$ to the nominal problem $\Pi$ using the $\forall$-quantified $y_2$ within polynomial space.
    This is doable because the problem $\Pi$ is in NP (and thus $co\Pi$ is in coNP).
    At last, the verifier $V$ has to verify the correctness of the given solution provided by the $\exists$-quantified $y_1$ and $\forall$-quantified $y_2$.
    Checking whether $|B| \leq t$ and $B \cap S \neq \emptyset$ is trivial and checking whether $S \in \sol(I)$ is clearly in polynomial time because $\Pi$ is in SSP-NP.
    It follows that $\textsc{Min Cardinality Interdiction-}\Pi$ is in $\Sigma^p_2$.
\end{proof}

Next, we show the hardness of minimum cardinality interdiction problems as long as the nominal problem is NP-complete.
For this, we introduce the concept of invulnerability reductions that helps us to grasp the problems in a unified approach.
We describe this concept in the following subsection with the goal to obtain the following main theorem of the paper.

\begin{theorem}
\label{thm:min-card-interdiction}
    The problem \textsc{Min Cardinality interdiction-$\Pi$} is $\Sigma^p_2$-complete for all the following problems:
    independent set,
    clique,
    subset sum,
    knapsack,
    Hamiltonian path/cycle (directed/undirected),
    TSP,
    $k$-directed vertex disjoint paths ($k \geq 2$),
    Steiner tree,
    dominating set,
    set cover,
    hitting set, feedback vertex set,
    feedback arc set,
    uncapacitated facility location,
    $p$-center,
    $p$-median.
\end{theorem}

We remark that the case of satisfiability deserves special attention, which is discussed more thoroughly in \cref{sec:noMeta}.

\subsection{Invulnerability Reduction}

Our proof strategy for each of the problems listed in \Cref{thm:min-card-interdiction} is essentially the same.
In fact, we show that \cref{thm:min-card-interdiction} is actually a consequence of the following, more powerful \emph{meta-theorem}.
This meta-theorem catches the essence of an invulnarability reduction.

\begin{theorem}
\label{thm:meta-theorem}
    Consider an SSP-NP-complete problem $\Pi$.
    If there exists a polynomial-time reduction $g$ which receives as input a tuple $(I, C, k)$ of an instance $I$ of $\Pi$, some set $C \subseteq \U(I)$ and some $k \in \N_0$, and returns instances $I' := g(I, C, k)$ of $\Pi$, such that the following holds: 
     \begin{align*}
         \exists B \subseteq C : |B| \leq k \text{ and } B \cap S \neq \emptyset \ \forall S \in \sol(I) \qquad\qquad\qquad\qquad\qquad\qquad\qquad\qquad\quad\\
         \Leftrightarrow \ \ \exists B' \subseteq \U(I') : |B'| \leq k \text{ and } B' \cap S' \neq \emptyset \ \forall S' \in \sol(I').
     \end{align*}
     Then \textsc{Min Cardinality Interdiction-$\Pi$} is $\Sigma^p_2$-complete. 
\end{theorem}

It would be nice to have \cref{thm:meta-theorem} for all problems in the class SSP-NPc, not only those who admit a funciton $g$ with the properties as described above. However, we give a reasoning in \Cref{sec:noMeta} why such a generalization is not possible.
The rest of this section is devoted to the proof of \cref{thm:min-card-interdiction}.
In \cite{gruene2024completeness} the following more general version of interdiction was considered, where there is a set $C \subseteq \U(I)$ of so-called vulnerable elements.
One can also interpret the set of vulnerable elements $C$ as the elements that have cost of interdiction of $1$ while all other elements $\U(I) \setminus C$ have a cost of interdiction of $\infty$ and a blocker of small costs is sought.
This problem is called the \emph{combinatorial interdiction problem}.

\begin{definition}[Comb. Interdiction Problem, from \cite{gruene2024completeness}.]
    Let an SSP problem $\Pi = (\I, \U, \sol)$ be given.
    We define \textsc{Comb. Interdiction-$\Pi$} as follows:
    The input is an instance $I \in \I$, a number $k \in \N_0$, and a set $C \subseteq \U(I)$. The set $C$ is called the set of vulnerable elements.
    The question is whether
    \[
        \exists B \subseteq C, \ |B| \leq k : \forall S \in \sol(I) : B \cap S \neq \emptyset.
    \]
\end{definition}

It is proven in \cite{gruene2024completeness} that for every problem in SSP-NPc, the combinatorial interdiction problem is $\Sigma^p_2$-complete.
Now, let $\Pi$ be in SSP-NPc and $g$ be a reduction such that
    \begin{align*}
         \exists B \subseteq C : |B| \leq k \text{ and } B \cap S \neq \emptyset \ \forall S \in \sol(I) \qquad\qquad\qquad\qquad\qquad\qquad\qquad\qquad\quad\\
         \Leftrightarrow \ \ \exists B' \subseteq \U(I') : |B'| \leq k \text{ and } B' \cap S' \neq \emptyset \ \forall S' \in \sol(I'),
     \end{align*}
then $g$ is a reduction from \textsc{Comb. Interdiction-$\Pi$} to \textsc{Min Cardinality interdiction-$\Pi$}. This is because the first line is equivalent to the statement that instance $I$ is a yes-instance of \textsc{Comb. Interdiction-$\Pi$}, and the second line is equivalent to the statement that $I'$ is a yes-instance of \textsc{Min Cardinality interdiction-$\Pi$}.
It directly follows that \textsc{Min Cardinality interdiction-$\Pi$} is $\Sigma^p_2$-complete. This completes the proof of \cref{thm:meta-theorem}. 

We remark that while in some sense the proof is rather trivial, we still see a lot of value in explicitly stating a set of easy-to-check sufficient conditions that render some minimum-cardinality interdiction problem $\Sigma^p_2$-complete.

How can one find a function $g$ with the properties as described above? Often times it is possible by employing the following natural idea:
Given an instance of the comb. interdiction problem, let the set $D := \U(I) \setminus C$ be called the \emph{invulnerable} elements. 
For each problem separately we explain that a gadget for the invulnerable elements in $D$ exists, which
intuitively speaking guarantees that an attacker, no matter which $k$ elements of the universe they attack, can never render the elements of $D$ unusable.
On the other hand, we make sure that the \emph{invulnerability gadgets} do not meaningfully change the set of solutions.
The next section gives many examples of such gadgets.
We remark that we are not the first to come up with this natural idea.
For example, Zenklusen \cite{DBLP:journals/dam/Zenklusen10a} used the same idea in the context of matching interdiction.

\subsection{Different Variants of Interdiction}
\label{sec:different-variants-of-interdiction}
In this section, we discuss variants of interdiction problems that can be found in the literature.
For this, we study the relation of our definition of a minimum cardinality interdiction problems and the existing variants.
Additionally, we argue what the implications of the hardness of our minimum cardinality interdiction problems on the other variants are.

\begin{description}
    \item[1. Minimal Blocker Problem.]\hfill
        \begin{description}
            \item[Input] Instance $I$ with universe $U$, blocker cost function $c$, solution cost function $d$, and solution threshold $\tau$
            \item[Task] Find the minimum-cost set $\min_{B \subseteq U} c(B)$ such that for all solutions $S$ with $S \cap B = \emptyset$, we have $d(S) \leq \tau$.
        \end{description}
    \item[2. Full Decision Variant of Interdiction.]\hfill
        \begin{description}
            \item[Input] Instance $I$ with universe $U$, blocker cost function $c$, blocker budget $k$, solution cost function $d$, and solution threshold $\tau$
            \item[Task] Is there a set $B \subseteq U$ with $c(B) \leq k$ such that for all solutions $S$ with $S \cap B = \emptyset$, we have $d(S) \leq \tau$?
        \end{description}
    \item[3. Most Vital Elements Problem.]\hfill
        \begin{description}
            \item[Input] Instance $I$ with universe $U$, blocker cost function $c$, and solution cost function $d$
            \item[Task] Find a set $B \subseteq U$ with $c(B) \leq k$ such that the costs of all solutions $S \cap B = \emptyset$ are maximized, i.e. $\max_{B} \min_{S, S \cap B = \emptyset} d(S)$.
        \end{description}
\end{description}

Our goal is to show that all of the variants from above are at least as hard as our formulation of \emph{minimum cardinality interdiction} (\Cref{def:min-card-interdiction-pi}).
This results in the following theorem.

\begin{theorem}
    Let $\Pi = (\I, \U, \sol)$ be an SSP problem.
    Then the
    Most Vital Elements Problem of $\Pi$ (for all problems $\Pi$ in \Cref{thm:min-card-interdiction}),
    the Minimal Blocker Problem of $\Pi$, and
    the Full Decision Variant of Interdiction of $\Pi$
    are at least as hard to compute as \textsc{Min Cardinality Interdiction}-$\Pi$.
\end{theorem}

The rest of this section is devoted to the proof of this theorem.
In our formulation of minimum cardinality interdiction, a set $B$ is sought, which intersects every solution in the set $\sol$ as given by the corresponding SSP problem.
We now have to distinguish between problems, which are naturally formulated as SSP problems (e.g. Hamiltonian cycle), and SSP problems, which are derived from an LOP problem (e.g. clique).
For natural SSP problems, the solution set $\sol$ consists of all solutions, i.e. there are no feasible solutions outside of $\sol$ due to the missing cost function $d$ on the solution elements.
Thus all of the three variants from above are generalizations of minimum cardinality interdiction:
\begin{enumerate}
    \item The \emph{minimal blocker problem} is the optimization version of the corresponding minimum cardinality interdiction problem.
    \item The \emph{full decision version of interdiction} is a generalization of the corresponding minimum cardinality interdiction problem because the latter assumes to have unit costs in the cost function $c$ for all elements from $U$.
    \item The \emph{most vital element problem} behaves analogous to (2).
\end{enumerate}
For SSP problems that are derived from an LOP problem, basically the same holds, however, with a modified and a technically more intricate argumentation.
Here the solution set is defined by $\sol = \{F \in \F : d(F) \leq t\}$ and we can find a reduction by generalization as follows:
\begin{enumerate}
    \item For \emph{minimal blocker problems}, we can set $\tau := t-1$.
    Then, we again have that the minimal blocker problem is the optimization version of the corresponding minimum cardinality interdiction problem.
    \item For \emph{minimal blocker problems}, we can also set $\tau := t-1$.
    Then, the full decision version is again a generalization of the corresponding minimum cardinality interdiction problem due to the fact that the latter has a unit cost function $c$.
    \item For \emph{most vital element problem}, the situation is more complicated.
    We first observe that the blocker part of $B \subseteq U$ with $c(B) \leq k$ is a generalization of the blocker part in minimum cardinality interdiction.
    The inner part on the nominal problem deserves special attention, though, due to the fact that the most vital element problem maximizes the objective while minimum cardinality interdiction blocks all solutions from the solution set $\sol$.
    We focus on this in the next paragraph.
\end{enumerate}

\par{\bf Reducing Minimum Cardinality Interdiction to Most Vital Elements.}
The concepts of minimum cardinality interdiction and most vital elements coincide if and only of the set $\sol$ contains exactly the optimal solutions, i.e. $\sol = \{F \in \F : d(F) \leq t^\star\}$, where $t^\star$ is optimal (i.e. minimal).
In order to assure that $\sol$ captures exactly the optimal solutions, we need to include this condition into the reduction.
In particular, the SSP reduction $(g, f)$ needs to guarantee that all instances $I$ are mapped to instances $g(I)$ such that all possible solutions are necessarily optimal.
In other words, $t$ is the optimal objective value of the LOP instance $g(I)$, since there are no feasible solutions, whose cost is even smaller than $t$.
We call SSP reductions that fulfill this criterion \emph{tight} and formally define them as follows.

\begin{definition}[Tight SSP reduction]
    Let $\Pi_1$ be an SSP problem and $\Pi_2 = (\I, \U, \F, d, t)$ be an LOP problem.
    Consider an SSP reduction $(g, (f_I)_{I \in \I})$ from $\Pi_1$ to (the SSP problem derived from) $\Pi_2$. 
    The reduction is called tight if for all yes-instances $I_1$ of $\Pi_1$, the corresponding instance $I_2 = g(I_1)$ of $\Pi_2$ with the associated parameter $t := t^{(I_2)}$ and associated cost function $d := d^{(I_2)}$, the following holds:
    \begin{align}
        \set{ F \in \F(I_2) : d(F) \leq t } \neq \emptyset \text{ and } \set{ F \in \F(I_2) : d(F) \leq t - 1} = \emptyset
    \end{align}
\end{definition}

All SSP reductions (to SSP problems derived form LOP problems) that can be found in \cite{gruene2024completeness} fulfill this definition and are thus tight.
Therefore, for all LOP problems (independent set, clique, knapsack, TSP, Steiner tree, dominating set, set cover, hitting set, feedback vertex set, feedback arc set), we obtain that the most vital element problem is at least as hard to compute as the minimum cardinality problem.

\paragraph*{Vertex/Edge Deletion Problems}
In this paper, we are concerned with finding a set $B$ such that $B \cap S \neq \emptyset$ for every solution $S$.
Note that this definition is meaningful even if the nominal problem is not graph-based.
However, in the special case where the nominal problem is graph-based, one could also consider a very related notion which is usually called \emph{vertex deletion problem} or \emph{edge deletion problem}.
Here, the question is how many vertices (edges) need to be deleted from the graph until some desired property is met.
Element deletion problems are well-studied in classical complexity theory for hereditary graph properties \cite{DBLP:journals/jcss/LewisY80} and in parameterized complexity theory for properties expressible by first order formulas \cite{DBLP:conf/mfcs/BannachCT24}.
In the general case, element deletion problems are not the same problem as our problem \textsc{Interdiction-$\Pi$}.
This is because for every set of deleted elements, the underlying instance is changed (vertices/edges are removed, which changes the graph). This is not the case for minimum cardinality interdiction problems as defined in this paper.
Thus, it is not possible to transfer the results of minimum cardinality interdiction directly to element deletion problems.
Albeit for the problems of clique and independent set, the $\Sigma^p_2$-completeness results hold for both minimum cardinality interdiction as well as for vertex deletion interdiction 
because for these problems the deletion of a vertex coincides with not taking this vertex into the solution. 
An analogous statement holds for edge deletions for the problems of directed/undirected Hamiltonian cycle/path, $k$-vertex-disjoint path, and Steiner tree.
