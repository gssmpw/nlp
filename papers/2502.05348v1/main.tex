

\documentclass[a4paper,UKenglish,cleveref, autoref, thm-restate]{lipics-v2019}
%This is a template for producing LIPIcs articles. 
%See lipics-manual.pdf for further information.
%for A4 paper format use option "a4paper", for US-letter use option "letterpaper"
%for british hyphenation rules use option "UKenglish", for american hyphenation rules use option "USenglish"
%for section-numbered lemmas etc., use "numberwithinsect"
%for enabling cleveref support, use "cleveref"
%for enabling autoref support, use "autoref"
%for anonymousing the authors (e.g. for double-blind review), add "anonymous"
%for enabling thm-restate support, use "thm-restate"

%\graphicspath{{./graphics/}}%helpful if your graphic files are in another directory

\newif\iflongversion
%\longversiontrue
\longversionfalse

\usepackage{tikz}
\usetikzlibrary{calc}
\usetikzlibrary{patterns}
\usepackage{csquotes}
%\usepackage[sort]{natbib}

\usepackage{mdframed}

\crefname{observation}{observation}{observations}
\Crefname{observation}{Observation}{Observations}
\newtheorem{observation}[theorem]{Observation}

\bibliographystyle{plainurl}


\title{The Complexity of Blocking All Solutions}

%\titlerunning{The Complexity of Blocking All Solutions} %TODO optional, please use if title is longer than one line

\author{Christoph Grüne}{Department of Computer Science, RWTH Aachen University, Germany}{gruene@algo.rwth-aachen.de}{https://orcid.org/0000-0002-7789-8870}{Funded by the German Research Foundation (DFG) – GRK 2236/2.}
\author{Lasse Wulf}{Section of Algorithms, Logic and Graphs, Technical University of Denmark, Kongens Lyngby, Denmark}{lawu@dtu.dk}{https://orcid.org/0000-0001-7139-4092}{Funded by the Carlsberg Foundation CF21-0302 ``Graph Algorithms with Geometric Applications''.}


\authorrunning{C. Grüne and L. Wulf} %TODO mandatory. First: Use abbreviated first/middle names. Second (only in severe cases): Use first author plus 'et al.'

\Copyright{Christoph Grüne and Lasse Wulf} %TODO mandatory, please use full first names. LIPIcs license is "CC-BY";  http://creativecommons.org/licenses/by/3.0/

\ccsdesc[500]{Theory of computation~Problems, reductions and completeness} %TODO mandatory: Please choose ACM 2012 classifications from https://dl.acm.org/ccs/ccs_flat.cfm 

\keywords{
Computational Complexity,
Robust Optimization,
Most Vital Elements,
Most Vital Nodes,
Most Vital Vertex,
Most Vital Edges,
Blocker Problems,
Vertex Blocker,
Node Blocker,
Edge Blocker,
Interdiction Problems,
Polynomial Hierarchy,
Sigma-2} %TODO mandatory; please add comma-separated list of keywords

\category{} %optional, e.g. invited paper

\relatedversion{} %optional, e.g. full version hosted on arXiv, HAL, or other respository/website
%\relatedversion{A full version of the paper is available at \url{...}.}

\supplement{}%optional, e.g. related research data, source code, ... hosted on a repository like zenodo, figshare, GitHub, ...

%\funding{(Optional) general funding statement \dots}%optional, to capture a funding statement, which applies to all authors. Please enter author specific funding statements as fifth argument of the \author macro.

%\acknowledgements{I want to thank \dots}%optional

\nolinenumbers %uncomment to disable line numbering

\hideLIPIcs  %uncomment to remove references to LIPIcs series (logo, DOI, ...), e.g. when preparing a pre-final version to be uploaded to arXiv or another public repository

%Editor-only macros:: begin (do not touch as author)%%%%%%%%%%%%%%%%%%%%%%%%%%%%%%%%%%
\EventEditors{John Q. Open and Joan R. Access}
\EventNoEds{2}
\EventLongTitle{42nd Conference on Very Important Topics (CVIT 2016)}
\EventShortTitle{CVIT 2016}
\EventAcronym{CVIT}
\EventYear{2016}
\EventDate{December 24--27, 2016}
\EventLocation{Little Whinging, United Kingdom}
\EventLogo{}
\SeriesVolume{42}
\ArticleNo{23}
%%%%%%%%%%%%%%%%%%%%%%%%%%%%%%%%%%%%%%%%%%%%%%%%%%%%%%

\definecolor{darkgreen}{RGB}{0,128,0}
\definecolor{darkred}{RGB}{128,0,0}

\newcommand{\todo}[1]{\textcolor{red}{\textbf{TODO: #1}}\\}
\newcommand{\important}[1]{\textcolor{purple}{\textsc{IMPORTANT: #1}}\\}
\newcommand{\lasse}[1]{\textcolor{orange}{Lasse: #1}}
\newcommand{\christoph}[1]{{\color{darkgreen} Christoph: #1}}
\newcommand{\tocheck}[1]{\textcolor{darkred}{#1}\\}

\newcommand{\boxxx}[1]
 {\fbox{\begin{minipage}{11.80cm}\begin{center}\bigskip\begin{minipage}{11.30cm}
  #1\end{minipage}\end{center}~\end{minipage}}}
\newcommand{\R}{\mathbb{R}}
\newcommand{\N}{\mathbb{N}}
\newcommand{\Z}{\mathbb{Z}}
\newcommand{\I}{\mathcal{I}}
\newcommand{\U}{\mathcal{U}}
\newcommand{\F}{\mathcal{F}}
\newcommand{\sol}{\mathcal{S}}
\newcommand{\powerset}[1]{2^{#1}}
\newcommand{\set}[1]{\{ #1 \}}
\newcommand{\fromto}[2]{\set{#1, \ldots, #2}}

\DeclareMathOperator{\poly}{poly}
\DeclareMathOperator{\reg}{reg}
\DeclareMathOperator*{\argmax}{arg\,max}
\DeclareMathOperator*{\argmin}{arg\,min}
\DeclareMathOperator{\dist}{dist}
\newcommand{\leqSSP}{\leq_\text{SSP}}
\newcommand{\bin}{\set{0,1}}

\newcommand{\NP}{\textsl{NP}}
\newcommand{\coNP}{\textsl{coNP}}
\newcommand{\PSPACE}{\textsl{PSPACE}}
\newcommand{\PTIME}{\textsl{PTIME}}

\newcommand{\citationneeded}[1][None]{\textsuperscript{\color{red} [Citation needed: #1]}}

\begin{document}

\maketitle
\begin{abstract}
    \begin{abstract}
Retrieval-Augmented Generation (RAG) is often used with Large Language Models (LLMs) to infuse domain knowledge or user-specific information. In RAG, given a user query, a retriever extracts chunks of relevant text from a knowledge base. These chunks are sent to an LLM as part of the input prompt. Typically, any given chunk is repeatedly retrieved across user questions. However, currently, for every question, attention-layers in LLMs fully compute the key values (KVs) repeatedly for the input chunks, as state-of-the-art methods cannot reuse KV-caches when chunks appear at arbitrary locations with arbitrary contexts. Naive reuse leads to output quality degradation.  This leads to potentially redundant computations on expensive GPUs and increases latency. In this work, we propose \sys, a system for managing and reusing precomputed KVs corresponding to the text chunks (we call \textit{chunk-caches}) in RAG-based systems. We present how to identify \hl{\textit{chunk-caches} that are reusable}, how to efficiently perform a small fraction of recomputation to \textit{fix} the cache to maintain output quality, and how to efficiently store and evict \textit{chunk-caches} in the hardware for maximizing reuse while masking any overheads. With real production workloads as well as synthetic datasets, we show that \sys reduces redundant computation by \textbf{51\%} over SOTA prefix-caching and \textbf{75\%} over full recomputation.
\hl{Additionally, with continuous batching on a real production workload, we get a \textbf{1.6$\times$} speedup in throughput and a \textbf{2$\times$} reduction in end-to-end response latency over prefix-caching while maintaining quality, for both the \llama-3-8B and \llama-3-70B models. 
}
\end{abstract}




    
\end{abstract}

\documentclass[../main.tex]{subfiles}
\graphicspath{{../images/}}
\makeatletter
\def\input@path{{../images/}}
\makeatother
\begin{document}
\section{Introduction}
\begin{figure}
\centering
\begin{tikzpicture}
\node[inner sep=0pt] (ws) at (0, 0) {
\includegraphics[height=.4\textwidth, trim={10cm 0 10cm 0},clip]{world_space.png}};
\node[inner sep=0pt] (cs) at (6,0) {\includegraphics[height=.4\textwidth, trim={10cm 1cm 10cm 4cm},clip]{conf_space.png}};
\end{tikzpicture}
\vspace{-5pt}
\label{fig:pbrm_intro}
\caption{\textbf{Left}: Shows world space obstacles as grey spheres. Robots start and goal configuration is colored red and green, respectively. Configurations along the computed path are colored transparent blue. \textbf{Right:} Mapped world space scenario to configuration space. Obstacle region is the grey mesh. Red spheres are collision-free regions computed by the neural SCDF. The optimized shortest path in the convex corridor is the blue curve.}
\vspace{-25pt}
\end{figure}
Motion planning is the problem of finding a collision-free trajectory that connects a given start and goal configuration. The planning takes place in the configuration space of the robot. For single body robots, like mobile robots or drones, the configuration space and the world space are usually the same. This simplifies the planning, since explicit obstacle representations are available which enables geometrical tools like separating hyperplanes, smallest distance to obstacles etc., to be used when designing motion planning algorithms. For multi-body robots like manipulators, the situation is completely different. The world space obstacles are usually mapped to non-convex regions, and to make the problem even harder, the mapping is usually not known. Forming explicit representations of the obstacle region in the configuration space is usually too expensive or intractable. Despite all of this, sampling based planners are used with great success, which mainly is due to their use of implicit representations of the obstacle region. The basic idea is to construct a graph in the configuration space that covers and connects the collision-free region. From this graph, a path can be extracted that connects a given start and goal configuration. The approach is computationally expensive, since the graph is constructed with the smallest geometrical building block available, points, which represents a collision-check. Furthermore, the extracted paths from the graph are non-smooth and jagged due to the stochastic nature of the approach. This adds an additional post-processing step to the process, where the paths are shortcutted and smoothened, before the path can be used for tracking. Clearly a lot of time is invested to form this graph and produce smooth paths. Thus, if the obstacles start to move, then all of this work is done in no use, since all points that make up this graph need to be re-verified, which is simply too time consuming to be done in real time.
\\\\
In this work, we want to address the existing drawbacks of the sampling based planners. Our main contribution is an improved motion planner where each vertex in the graph covers a collision-free region in the form of a sphere instead of a point and where the edges are formed with neighboring intersecting spheres. This representation has the advantage of instead of returning piecewise linear paths, returning a sequence of overlapping spheres, i.e. a convex corridor, that connects a given start and goal configuration, illustrated in Figure \ref{fig:pbrm_intro}. This convex corridor allows us to use convex optimization to produce smooth trajectories, instead of computationally expensive post-processing methods. The representation further allows us to estimate the coverage of the collision-free space, which gives us awareness and feedback in the offline roadmap construction phase. Finally, our representation is simple to adapt to moving obstacles, simply requery for the new radii and recheck for intersections. 
\\\\
The spherical collision-free regions are formed using a signed distance function (SDF), which is a function that returns the smallest distance from an arbitrary point to the boundary of an obstacle. As the name implies, the distance is signed, thus if the point is inside the obstacle it is negative otherwise positive. If the distance is positive, a sphere with radius equal to the distance is guaranteed to cover a collision-free region. Using an SDF in motion planning is not new, but what is novel about our approach is that we express the distance in the configuration space instead of the world space and by doing so allows us to form these convex collision-free regions. We refer to the resulting SDF as a signed configuration distance function (SCDF). Computing an SCDF analytically is non-trivial, our approach is therefore to parameterize the SCDF with a deep neural network and learn the mapping by supervised learning. Our resulting neural SCDF can compute distances for different parameter values of obstacle shapes and we also show how multiple distances can be combined, thus making our approach flexible.
\section{Related work}
Motion planning algorithms can roughly be divided into three families, grid-based, sampling based and optimization based methods. Grid-based methods (GBM) discretize the planning space from which a graph is then compiled. A standard search method is A$^\star$ \citep{a_star}, which is classified as an \textit{informed} search method, since it employs a heuristic function to speed up the search. A$^\star$ guarantees to return an optimal path at the level of discretization used. GBMs usually discretize the planning space by a regular lattice and this limits the GBMs to problems with low dimensionality due to the curse of dimensionality. Thus, GBMs are usually limited to single-body robots where the degrees of freedom (DOF) are low. To overcome the inherent scaling problem with the GBMs, stochastic methods are usually used for multi-body robots. These methods are termed as sampling-based methods (SBM) and core members within this family are the rapidly-exploring random trees (RRT) \citep{rrt} and the probabilistic roadmap (PRM) \citep{prm}. RRT grows a tree from the start configuration and explores the collision-free region in a rapid way until it is able to connect to the goal region. RRT is usually improved by bi-directional planning \citep{rrt_connect}, i.e. an additional tree is grown from the goal configuration and the trees are tested for connection after any tree has been expanded. RRT is a single-query method, thus it searches for a path from scratch each time it is queried. Contrary to this, PRM is a multi-query method, which solves for multiple queries without starting from scratch. PRM does this by creating a roadmap (graph) that covers the collision-free space as an offline step. The graph is then used to solve for multiple queries. PRMs are used in cases where the environment does not change since the extra offline step is too computationally costly and needs to be re-done if the environment is changed. In our work, we address this inherent issue by using a different roadmap representation. Our vertices in the graph cover a collision-free region in the form of spheres and we form the edges by checking for intersecting spheres. If something in the environment changes, we recompute the spheres radii and recheck the intersections, without relying on collision detection. We use a trained neural network to compute the sphere radius, therefore querying for the radius can be done fast, hence our representation enables the PRM for dynamic environments.
\\\\
In the recent decades, optimization based methods (OBM) \citep{chomp, schulman, itomp, stomp} have been introduced as an alternative to SBM for multi-body robots. Like the SBM, the OBMs scale well to higher dimensional problems and produce smoother motion. It is common to use a SDF in the optimization since it is a smooth function, thus enabling gradient-based methods. However, the standard way of expressing the SDF is in world space. The distance therefore needs to be mapped to the configuration space by the forward kinematics. This mapping makes the optimization problem a non-linear program (NLP), which is computationally expensive to solve. Recently, a different approach has been proposed. In \cite{mp_gcs} motion planning is formulated as a convex optimization problem by using the graph of convex sets framework \citep{gcs}. The underlying idea is to decompose the collision-free space into intersecting convex sets from which a convex optimization problem is formulated. In cases where an explicit representation of the obstacles in the configuration space exists, like for single-body robots, creating collision-free convex regions can be done fast \citep{iris}. For multi-body robots, this is non-trivial. Existing work does this successfully \citep{iris_nlp, iris_c} by an optimization based approach, but the methods are still too time consuming to be used in the presence of moving obstacles. Our approach is instead to use deep learning to learn an SDF expressed in the configuration space. With this, we can query for shortest distances to the collision boundary, which allows us to expand spherical regions which are collision-free. Our approach is fast and therefore enables our suggested roadmap planner to be used in dynamic environments.
\\\\
Recent research has focused on learning collision detection \citep{fk_kernel_distance, diffco, graphdistnet} by predicting the signed distance between the robot links and the surrounding obstacles in the world space. The learned SDF is used in trajectory optimization but since the distance is expressed in the world space, the problem becomes an NLP and therefore takes a long time to solve. We take a novel approach and suggest to instead express the signed distance in the configuration space. This allows us to improve the PRM at the same time as it enables convex optimization for trajectory optimization, which runs faster and is more reliable than NLP solvers. In \cite{cspf} a learned signed distance function in the configuration space is proposed similar to our approach. However, their approach is restricted to point cloud representations, while we propose to represent the obstacles as parameterized geometric shapes, e.g. spheres. Furthermore, we also show how to use our learned SCDF to improve an existing roadmap planner.
\section{Problem formulation}
A robot is located in the world space, $\W \subset \R^3 $. The unique location of the robot is given by its configuration $\q \in \C$, where $\C$ is the configuration space. The set of points covered by the robots bodies at a certain configuration is expressed as $\B(\q) \subset \W$. The robot is surrounded by $\NrObst$ obstacles $\O = \bigcup_{i=1}^{\NrObst} \O_i$, where  $\O_i \subset \W$. The representation of the obstacle in the configuration space is the set $\C\O_i = \{\q \in \C \: |\: \B(\q) \cap \O_i \neq \emptyset \}$. The obstacle space is formed as $\Co = \bigcup_{i=1}^{\NrObst} \C \O_i$. The complement is referred to as the free space, $\Cf = \C \setminus \Co$. The path planning problem is a tuple, ($\Cf$, $\qStart$, $\qGoal$), where we want to connect a query pair, consisting of a start, $\qStart$, and goal configuration, $\qGoal$, with a geometric path, $\q(s): [0, 1] \mapsto \Cf$, such that $\q(0)=\qStart$ and $\q(1)=\qGoal$, or report correctly when such a path does not exist.
\end{document}


\section{Preliminaries}\label{sec:preliminaries}



%We denote by $(\Ac(x_\Ac),\Bc(x_\Bc))(z)$ a random execution of $\pi$ with private inputs $(x_\Ac,y_\Ac)$, and common input $z$.

%\Jnote{Move to DP}
% At the end of such an execution, the protocol outputs a public transcript denoted by the random variable $\trans_\pi(x_\Ac,x_\Ac,z)$ we denotes the common as $\out(\trans_\pi(x_\Ac,x_\Ac,z)$, and each party $\Pc \in \set{\Ac,\Bc}$ obtains his view denoted $\view^\Pc_\pi(x_\Ac,x_\Bc,z)$, which may also contain a ``local output'' \Jnote{Local} $\out^\Pc(x_\Ac,x_\Bc,z)$ (if the protocol specifies such an output). \Jnote{Common output, and parties output}


\subsection{Distributions and Random Variables}\label{sec:prelim:dist}
The support of a distribution $P$ over a finite set $\cS$ is defined by $\Supp(P) \eqdef \set{x\in \cS: P(x)>0}$. For a distribution or a random variable $D$, let $d\from D$ denote that $d$ was sampled according to $D$. Similarly,  for a set $\cS$, let $x \from \cS$ denote that $x$ is drawn uniformly from $\cS$, and denote by $\cU_{\cS}$ the uniform distribution over $\cS$. For a finite set $\cX$ and a distribution $C_X$ over $\cX$, we use the capital letter $X$ to denote the random variable that takes values in $\cX$ and is sampled according to $C_X$. The {\sf statistical distance} (\aka {\sf~variation distance}) of two distributions $P$ and $Q$ over a discrete domain $\cX$ is defined by $\sdist{P}{Q} \eqdef \max_{\cS\subseteq \cX} \size{P(\cS)-Q(\cS)} = \frac{1}{2} \sum_{x \in \cS}\size{P(x)-Q(x)}$. 
For a vector $x = (x_1,\ldots,x_n)$ and index $i\in [n]$, we let $x_{-i} = (x_1,\ldots,x_{i-1},x_{i+1},\ldots,x_n)$ and $x^{(i)} = (x_1,\ldots,x_{i-1}, -x_i, x_{i+1},\ldots,x_n)$, for a set $\cS \subseteq [n]$ we let $x_{\cS} = (x_i)_{i \in \cS}$ and $x_{-\cS} = (x_i)_{i \in [n]\setminus \cS}$, and for a vector $r \in \zo^n$ we let $x_r = (x_i)_{\set{i \colon r_i = 1}}$ and $x_{-r} = (x_i)_{\set{i \colon r_i = 0}}$.

%For $n \in \N$ we let $U_n$ be the uniform distribution over $\oo^n$, and let $S_n$ be the distribution induces by the sum of $n$ i.i.d.\ random variables, each is distributed according to $U_1$. Let $\cN(0,1)$ be the standard normal distribution.
%For a distribution $\cD$ and a function $f$, we define by $f(\cD)$ the distribution that is induced by the output of $f(x)$ for $x \from \cD$. 





% \begin{theorem}[\cite{McGregorMPRTV10}]\label{thm:sv-extracotr}
% 	\Enote{Remove if not needed}
% 	There is a constant $c$ to make the following holds. Let $X$ be an $\alpha$-SV source on $\{0,1\}^n$, let $Y$ be a source on $\{0,1\}^n$ with min-entropy at least $\beta n$ (independent from $X$), and let $Z=\ip{X,Y}\mbox{mod m}$ for some $m\in\mathbb{N}$. Then for every $\delta\in[0,1]$, the random variable $(Y,Z)$ is $\delta$-close to $(Y,U)$ where $U$ is uniform on $\mathbb{Z}_m$ and independent of $Y$, provided that
% 	$$
% 	n\geq c\cdot\frac{m^2}{\alpha\beta}\cdot\log(\frac{m}{\beta})\cdot\log(\frac{m}{\delta}).
% 	$$
% \end{theorem}



\Enote{I removed the definition of DP since it already appears in the intro}
\remove{
\subsection{Differential Privacy}\label{sec:prelim:DP}
We use the following standard definition of (information theoretic) differential privacy, due to \citet{DMNS06}. For notational convenience, we focus on databases over $\oo$.
\begin{definition}[Differentially private mechanisms]\label{def:mech}
	A randomized function $f\colon\oo^n\mapsto \zs$ is an {\sf $n$-size, $(\eps,\delta)$-differentially private mechanism} (denoted $(\eps,\delta)$-\DP) if for every neighboring $w,w'\in \oo^n$ and every function $g\colon \zs\mapsto \zo$, it holds that 
	$$
	\pr{g(f(w))=1}\leq e^{\eps}\cdot \pr{g(f(w'))=1} +\delta.
	$$ 	
	If $\delta=0$, we omit it from the notation.
\end{definition}
}


\subsubsection{Computational Differential Privacy}
There are several ways for defining computational differential privacy (see \cref{sec:related-works}). We use the most relaxed version due to \cite{BNO08}. For notational convenience, we focus on databases over $\oo$.
\begin{definition}[Computational differentially private mechanisms]\label{def:ComMech}
	A randomized function ensemble $f=\set{f_\pk\colon\oo^{n(\pk)}\mapsto \zs}$ is an {\sf $n$-size, $(\eps,\delta)$-computationally differentially private} (denoted $(\eps,\delta)$-$\CDP$) if for every poly-size circuit family $\set{\Ac_\pk}_{\pk\in \N}$, the following holds for every large enough $\pk$ and every neighboring $w,w'\in\oo^{n(\pk)}$:
	$$
	\pr{\Ac_\pk(f_\pk(w))=1}\leq e^{\eps(\pk)}\cdot \pr{\Ac_\pk(f_\pk(w'))=1} +\delta(\pk).
	$$ 
	If $\delta(\pk) = \negl(\pk)$, we omit it from the notation. 
\end{definition}



\subsubsection{Two-Party Differential Privacy}\label{sec:DP}
In this section we formally define distributed differential privacy mechanism (\ie protocols). %For the ease of notation, we consider protocol with no common input.

\begin{definition}\label{def:DP}%\Nnote{fix security parameter}
	A two-party protocol $\Pi=(\Ac,\Bc)$ is {\sf $(\eps,\delta)$-differentially private}, denoted $(\eps,\delta)$-$\DP$, if the following holds for every algorithm $\Dc$: let $\V^\Pc(x,y)(\pk)$ be the view of party $\Pc$ in a random execution of $\Pi(x,y)(1^\pk)$. Then for every $\pk,n \in \N$, $x\in \oo^n$ and neighboring $y,y'\in\oo^n$:
	\begin{align*}
	\pr{\Dc(V^\Ac(x,y)(\pk))=1}\le e^{\eps(\pk)}\cdot \pr{\Dc(V^\Ac (x,y')(\pk))=1}+\delta(\pk),
	\end{align*} 
	and for every $y\in \oo^n$ and neighboring $x,x'\in\oo^{n}$:
	\begin{align*}
	\pr{\Dc(V^\Bc(x,y)(\pk))=1}\le e^{\eps(\pk)}\cdot \pr{\Dc(V^\Bc (x',y)(\pk))=1}+\delta(\pk).
	\end{align*} 	
	Protocol $\Pi$ is {\sf $(\eps,\delta)$-computational differentially private}, denoted $(\eps,\delta)$-$\CDP$, if the above inequalities only hold for a non-uniform \ppt $\Dc$ and large enough $\pk$. We omit $\delta = \negl(\pk)$ from the notation. \footnote{Note that define we give for two-party differentially private protocols is a semi-honest definition, in which we ask for the security to hold when the parties interact in an honest execution of the protocol. Since we are proving a lower bound, starting from this weaker guarantee (as opposed to security against malicious players), yields a stronger result.}
\end{definition}
%We omit $\delta$ from the notation if $\delta$ is a negligible function of $n$.

%\Enote{simulation-based}
\begin{remark}[The definition for computational differential privacy we use]\label{rem:comDPChannel} 
	An alternative, stronger definition of computational differential privacy, known as simulation-based computational differential privacy, requires that the distribution of each party’s view be computationally indistinguishable from a distribution that ensures privacy in an information-theoretic sense. \cref{def:DP} is a weaker notion in comparison. Consequently, establishing a lower bound for a protocol that satisfies this weaker guarantee (as we do in this work) yields a stronger result.%Actually, our lower bound only requires the privacy to hold against \emph{uniform} external observer.
	%\Nnote{Maybe add: When only interesting in \Dp against external observer, the two definitions can be achieve using key-agreement and (single-party) \Dp mechanism. }
\end{remark}




\subsection{Useful Claims}
\remove{
In this section, we state generic lemmas and propositions that we will use later in our proofs.

The following lemma which we prove in \cref{sec:missing-proofs:distance-I}, measures the distance between two uniform stings conditioned one a random index $i$ either being fixed to $0$ or to $1$.

\def\distanceILemma{
    Let $R \la \zo^n$. For any (randomized) function $f:\{0,1\}^n\rightarrow \{0,1\}$ and $\alpha > 0$, it holds that
    \begin{align}\label{eq:f-alpha}
        \ppr{i \la [n]}{\size{\:\ex{f(R) \mid R_i = 0}-\ex{f(R) \mid R_i = 1}\:}\geq \alpha} \leq \frac{2}{n \alpha^2},
    \end{align}
    where the expectations are taken over $R$ and the randomness of $f$.
}

\begin{lemma}\label{lem:distance-I}
    \distanceILemma
\end{lemma}
}

The following two propositions state that given the output of a differentially private function, it is not possible to predict well even a random index (even if all other indexes are leaked). The first proposition handles the information-theoretic case and the second handles the computation case. Both propositions are proven in \cref{sec:missing-proofs:hard-to-guess}. 

\def\propHardToGuessInf{
    Let $f\colon \oo^n \rightarrow \cY$ be an $(\eps,\delta)$-\DP function, let $g \colon [n] \times \oo^{n-1} \times \cY \rightarrow \set{-1,1,\bot}$ be a (randomized) function, and let $X = (X_1,\ldots,X_n) \la \oo^n$. Then the following holds for every $i \in [n]$ where $X_i^* = g(i,X_{-i},f(X_1,\ldots,X_n))$:
    \begin{align*}
        \pr{X_i^* = X_i} \leq e^{\eps}\cdot \pr{X_i^* = -X_i} + \delta.
    \end{align*}
}

\begin{proposition}\label{prop:hard-to-guess-inf}
    \propHardToGuessInf
\end{proposition}


\def\propHardToGuessComp{
    Let $f = \set{f_{\pk} \colon \oo^{n(\pk)} \rightarrow \zo^{m(\pk)}}_{\pk \in \bbN}$ be an $(\eps,\delta)$-\CDP function ensemble, and let $\set{g_{\pk}}_{\pk \in \bbN}$ be a poly-size circuit family. Then, for large enough $\pk$ and $X = (X_1,\ldots,X_{n(\pk)}) \la \oo^{n(\pk)}$, the following holds for every $i \in [n(\pk)]$ where $X_i^* = g_{\pk}(i,X_{-i},f_{\pk}(X_1,\ldots,X_n))$:
    \begin{align*}
        \pr{X_i^* = X_i} \leq e^{\eps(\pk)}\cdot \pr{X_i^* = -X_i} + \delta(\pk).
    \end{align*}
}

\begin{proposition}\label{prop:hard-to-guess-comp}
    \propHardToGuessComp
\end{proposition}





\remove{
\Enote{Chao's old statement:}
\begin{lemma}\label{lem:distance-I-old}
        Let $R \la \zo^n$. 
	For any function $f:\{0,1\}^n\rightarrow \{0,1\}$ and $\alpha<0.01$, it holds that
	$$
	\Pr_{i\la[n]}\left[\: \size{\:\mathbb{E}[f(R) \mid R_i = 0]-\mathbb{E}[f(R) \mid R_i = 1]\:}\geq \alpha\right]\leq \frac{2+2\log(\frac{1}{\alpha})}{n\alpha^2}.
	$$
\end{lemma}
\begin{proof}
	Define $S_1=\{r \in \zo^n \colon f(r)=1\}$. Then for any $i\in[n]$, we have
	$$
	\begin{array}{rl}
		\size{\mathbb{E}[f(R) \mid R_i = 0]-\mathbb{E}[f(R) \mid R_i = 1]}
		&=\size{\Pr[R\in S_1|R_i=0]-\Pr[R\in S_1|R_i=1]}\\
		&=\size{\frac{\Pr[R_i=0|R\in S_1]\cdot\Pr[R\in S_1]}{\Pr[R_i=0]}-\frac{\Pr[R_i=1|R\in S_1]\cdot\Pr[R\in S_1]}{\Pr[R_i=1]}}\\
		&=\frac{2\size{S_1}}{2^n}\size{\Pr[R_i=0|R\in S_1]-\Pr[R_i=1|R\in S_1]}
	\end{array}
	$$
	When $|S_1|\leq \alpha\cdot 2^{n-1}$, we have $\size{\mathbb{E}[f(R) \mid R_i = 0]-\mathbb{E}[f(R) \mid R_i = 1]}\leq\frac{2\size{S_1}}{2^n}\leq \alpha$ for any $i\in[n]$. Hence, in the following, we assume $|S_1|> \alpha\cdot 2^{n-1}$.

	%Define $I_{bad}=\{i|\size{\Pr[R_i=0|R\in S_1]-\Pr[R_i=1|R\in S_1]}>2\alpha\}$ and $k=\size{I_{bad}}$, then for any $i\notin I_{bad}$, we have 
    %$$
    %\begin{array}{rl}
    %    2\alpha&\geq \size{\Pr[R_i=0|R\in S_1]-\Pr[R_i=1|R\in S_1]}\\
    %    &=\size{\frac{\Pr[R\in S_1|R_i=0]\cdot\Pr[R_i=0]}{\Pr[R\in S_1]}-\frac{\Pr[R\in S_1|R_i=1]\cdot\Pr[R_i=1]}{\Pr[R\in S_1]}}\\
    %    &=\size{\Pr[R\in S_1|R_i=0]-\Pr[R\in S_1|R_i=1]}\cdot\frac{1}{2\Pr[R\in S_1]}\\
    %    &\geq \size{\mathbb{E}[f(R) \mid R_i = 0]-\mathbb{E}[f(R) \mid R_i = 1]}\cdot \frac{1}{2},
    %\end{array}
    %$$ 
    %where the last inequality is because $\Pr[R\in S_1]\leq 1$. So that $\size{\mathbb{E}}[f(R) \mid R_i = 0]-\mathbb{E}[f(R) \mid R_i = 1]\leq %4\alpha$.
    Define $I_{bad}=\{i \colon \size{\Pr[R_i=0|R\in S_1]-\Pr[R_i=1|R\in S_1]} \geq 2\alpha\}$ and $k=\size{I_{bad}}$, and denote $I_{bad}=\{i_1,\dots,i_k\}$. Define $(X_{i_1}, \ldots X_{i_k}) = (R_{i_1},\dots,R_{i_k})\mid_{R \in S_1}$. 
    Consider the min-entropy
	$$
	\begin{array}{rl}
		H_{min}(X_{i_1},\dots,X_{i_k})&\leq H(X_{i_1},\dots,X_{i_k})\\
		&\leq \sum_{j=1}^k H(X_{i_j})\\
		&\leq k\cdot \left(-(\frac{1}{2}+2\alpha)\cdot\log(\frac{1}{2}+2\alpha)-(\frac{1}{2}-2\alpha)\cdot\log(\frac{1}{2}-2\alpha)\right)\\
            &=k\cdot \left(-(\frac{1}{2}+2\alpha)\cdot(\log(1+4\alpha)-1)-(\frac{1}{2}-2\alpha)\cdot(\log(1-4\alpha)-1)\right)\\
            &=k\cdot \left(1-(\frac{1}{2}+2\alpha)\cdot\log(1+4\alpha)-(\frac{1}{2}-2\alpha)\cdot\log(1-4\alpha)\right),
		
	\end{array}
	$$
	where $H_{min}(Y)$ is the minimum entropy of $Y$ and $H(Y)$ is the Shannon entropy of $Y$.\Enote{add to preliminaries.}
        The third inequality holds since by the definition of $I_{bad}$, for every $j \in [k]$ it holds that $\size{\pr{X_{i_j} = 1}-\pr{X_{i_j} = 0}} > 2\alpha$, and therefore $H(X_{i_j}) \leq H(1/2 + 2\alpha)$\Enote{define}.
	
	Therefore, there exists $b_1,\dots,b_k\in\{0,1\}$, such that 
	
	\begin{align}\label{eq:min-entropy-result}
		\Pr\left[(R_{i_1},\ldots,R_{i_k}) = (b_1,\ldots,b_k) \mid R\in S_1\right]
		&= \pr{(X_{i_1},\ldots,X_{i_k}) = (b_1,\ldots,b_k)}\\
		&= 2^{-H_{min}(X_{i_1},\dots,X_{i_k})}\nonumber\\
		&\geq 2^{k\cdot \left(-1+(\frac{1}{2}+2\alpha)\cdot\log(1+4\alpha)+(\frac{1}{2}-2\alpha)\cdot\log(1-4\alpha)\right)}.\nonumber
	\end{align}
	
	Let $S_{bad}=\{r \in \zo^n  \colon \set{(r_{i_1},\ldots,r_{i_k}) = (b_1,\ldots,b_k)} \land \set{r\in S_1}\}$.
	It holds that
	\begin{align*}
		|S_{bad}|
		&= \size{S_1} \cdot \Pr\left[(R_{i_1},\ldots,R_{i_k}) = (b_1,\ldots,b_k) \mid R\in S_1\right]\\
		&\geq \alpha\cdot 2^{n-1}\cdot2^{k\cdot \left(-1+(\frac{1}{2}+2\alpha)\cdot\log(1+4\alpha)+(\frac{1}{2}-2\alpha)\cdot\log(1-4\alpha)\right)},
	\end{align*} 
	where the inequality holds by \cref{eq:min-entropy-result} and since $\size{S_1} \geq \alpha\cdot 2^{n-1}$.
	Notice that any string in $S_{bad}$ depends on at most $n-k$ bits. It implies that $|S_{bad}|\leq 2^{n-k}$. Therefore, we have
	$$
	\begin{array}{rl}
		&2^{n-k}\geq \alpha\cdot 2^{n-1}\cdot2^{k\cdot \left(-1+(\frac{1}{2}+2\alpha)\cdot\log(1+4\alpha)+(\frac{1}{2}-2\alpha)\cdot\log(1-4\alpha)\right)} \\
		\Rightarrow& n-k \geq \log \alpha+n-1+k\cdot \left(-1+(\frac{1}{2}+2\alpha)\cdot\log(1+4\alpha)+(\frac{1}{2}-2\alpha)\cdot\log(1-4\alpha)\right)\\
		\Rightarrow& 1-\log \alpha \geq k\cdot((\frac{1}{2}+2\alpha)\cdot\log(1+4\alpha)+(\frac{1}{2}-2\alpha)\cdot\log(1-4\alpha))\\
		\Rightarrow& 1-\log \alpha \geq k\cdot(4\alpha\cdot\log(1+4\alpha)+(\frac{1}{2}-2\alpha)\cdot\log(1-16\alpha^2))\\
        \Rightarrow& 1-\log\alpha \geq k\cdot(15.9\alpha^2-8\alpha^2+32\alpha^3)=k\cdot(7.9\alpha^2+32\alpha^3)>0.5k\alpha^2\\
		\Rightarrow& k\leq \frac{2-2\log \alpha}{\alpha^2} = \frac{2+2\log (1/\alpha)}{\alpha^2},
	\end{array}
	$$
	Where the third transition holds since 
	\begin{align*}
		\lefteqn{(\frac{1}{2}+2\alpha)\cdot\log(1+4\alpha)+(\frac{1}{2}-2\alpha)\cdot\log(1-4\alpha)}\\
		&= 4\alpha\cdot\log(1+4\alpha) + (\frac{1}{2}-2\alpha)\paren{\log(1+4\alpha)+\log(1-4\alpha)}\\
		&= 4\alpha\cdot\log(1+4\alpha)+(\frac{1}{2}-2\alpha)\cdot\log(1-16\alpha^2),
	\end{align*}
	and the forth transition holds since $4\alpha\cdot\log(1+4\alpha)+(\frac{1}{2}-2\alpha)\cdot\log(1-16\alpha^2) > 15.9\alpha^2-8\alpha^2+32\alpha^3$ for $\alpha < 0.01$.
	Thus, we conclude that 
	$$
	\Pr_{i\la[n]}\left[\size{\mathbb{E}[f(R) \mid R_i=0]-\mathbb{E}[f(R) \mid R_i = 1]}\geq \alpha\right]\leq \frac{k}{n}\leq \frac{2+2\log (1/\alpha)}{n\alpha^2}.
	$$
\end{proof}
}


\subsection{Channels and Two-Party Protocols}\label{sec:protocol}

\paragraph{Channels.}A channel is simply a distribution of a pair of tuples defined as follows. 
\begin{definition}[Channels]\label{def:channel} A {\sf channel} $C_{(X,U)(Y,V)}$ of size $\isize$ over alphabet $\Sigma$ is a probability distribution over $(\Sigma^\isize \times\zo^\ast) \times(\Sigma^\isize \times\zo^\ast)$. The ensemble $C_{(X,U)(Y,V)}= \set{C_{(X_\pk,U_\pk)(Y_\pk,V_\pk)}}_{\pk\in \N}$ is an $\isize$-size channel ensemble, if for every $\pk\in \N$, $C_{(X_\pk,U_\pk)(Y_\pk,V_\pk)}$ is an $\isize(\pk)$-size channel. %We denote a channel of size one by a \emph{single-bit} channel. 
We refer to $X$ and $Y$ as the {\sf local outputs}, and to $U$ and $V$ as the {\sf views}.	
\end{definition}

We view a  channel as the experiment in which there are two parties $\Ac$ and $\Bc$.  Party $\Ac$ receives ``output'' $X$ and ``view'' $U$, and party $\Bc$ receives ``output'' $Y$ and ``view'' $V$. Unless stated otherwise, the channels we consider are over the alphabet $\Sigma = \oo$. We naturally identify channels with the distribution that characterizes their output.








\subsubsection{Two-Party Protocols}

A two-party protocol $\Pi=(\Ac,\Bc)$ is \ppt if the running time of both parties is polynomial in their input length. We let $\Pi(x,y)(z)$ or $(\Ac(x),\Bc(y))(z)$ denote a random execution of $\Pi$ on a common input $z$, and private inputs $x,y$.%We assume \wlg that a protocol has a common output (part of its transcript).\Jnote{This is not really the case we consider in this paper..}

\begin{definition}[Oracle-aided protocols]\label{def:ChannelAidedProtocol}
	In a two-party protocol $\Pi$ with oracle access to a {\sf protocol} $\Psi$, denoted $\Pi^\Psi$, the parties make use of the \textit{next-message function} of $\Psi$.\footnote{The function that on a partial view of one of the parties, returns its next message.} In a two-party protocol $\Pi$ with oracle access to a {\sf channel} $C_{Z W}$, denoted $\Pi^C$, the parties can jointly invoke $C$ for several times. In each call, an independent pair $(z,w)$ is sampled according to $C_{Z W}$, one party gets $z$, the other gets $w$.
\end{definition}


\begin{definition}[The channel of a protocol]\label{def:ChannlOfProtocol}
	For a no-input two-party protocol $\Pi= (\Ac,\Bc)$, we associate the channel $C_\Pi$, defined by $\C_\Pi= C_{(X, U),(Y, V)}$, where $X$ and $Y$ are the local outputs of $\Ac$ and $\Bc$ (respectively) and
	$U$ and $V$ are the local views of $\Ac$ and $\Bc$ (respectively).
    
	For a two-party protocol $\Pi$ that gets a security parameter $1^\pk$ as its (only, common) input, we associate the channel ensemble $ \set{C_{\Pi(1^\pk)}}_{\pk\in \N}$. 
\end{definition}

\begin{definition}[$(\alpha,\gamma)$-Accurate channel]\label{def:accurate-func}
	A channel $C = C_{(X, U),(Y, V)}$ is {\sf $(\alpha,\gamma)$-accurate for the function $f$}, if $\ppr{C}{\size{\out(V)-f(X,Y)}\leq \alpha}\ge \gamma$, where $\out(V)$ is the designated output.
    A channel ensemble $C_{(X, U),(Y, V)}= \set{C_{(X_\pk, U_\pk),(Y_\pk, V_\pk)}}_{\pk\in \N}$ is  $(\alpha,\gamma)$-accurate for  $f$ if $C_{(X_\pk, U_\pk),(Y_\pk, V_\pk)}$ is $(\alpha(\pk),\gamma(\pk))$-accurate for $f$, for every $\pk \in \N$.
\end{definition}

\subsubsection{Differentially Private Channels}\label{sec:DPChannel}
Differentially private channels are naturally defined as follows:
\begin{definition}[Differentially private channels]\label{def:DPChannel}
	An $n$-size channel $C = C_{(X, U),(Y, V)}$ with $X, Y$ over $\oo^n$ 
	is {\sf$(\eps,\delta)$-differentially private} (denoted $(\eps,\delta)$-$\DP$) if for every $x \in \Supp(X)$ there exists an $n$-size $(\eps,\delta)$-$\DP$ mechanisms $\Mc_x$ such that $(X,Y,U) \equiv (X,Y,\Mc_X(Y))$, and for every $y \in \Supp(Y)$ there exists an $n$-size $(\eps,\delta)$-$\DP$ mechanisms $\Mc_y'$ such that $(X,Y,V) \equiv (X,Y,\Mc_Y'(X))$. In addition, we say that the channel is \emph{uniform} if $X$ and $Y$ are independent random variables uniformly distributed in $\oo^n$. 
\end{definition}

\begin{definition}[Computational differentially private channels]\label{def:CDPChannel}
	An $n$-size channel ensemble $C = \set{C_{(X_\pk, U_\pk),(Y_\pk, V_\pk)}}_{\pk\in\N}$ with $X_\pk, Y_\pk$ over $\oo^n$ 
	is {\sf$(\eps,\delta)$-computationally differentially private} (denoted $(\eps,\delta)$-$\CDP$) if for every ensemble $\set{x_\pk \in \Supp(X_\pk)}_{\pk\in\N}$ there exists an $n$-size $(\eps,\delta)$-\CDP mechanisms ensemble $\set{\Mc_{x_\pk}}_{\pk\in\N}$ such that $(X_\pk,Y_\pk,U_\pk) \equiv (X_\pk,Y_\pk,\Mc_{X_\pk}(Y_\pk))$, for every $\pk\in\N$, and for every ensemble $\set{y_\pk \in \Supp(Y_\pk)}_{\pk\in\N}$ there exists an $n$-size $(\eps,\delta)$-$\CDP$ mechanisms ensemble $\set{\Mc'_{y_\pk}}_{\pk\in\N}$ such that $(X_\pk,Y_\pk,V_\pk) \equiv (X_\pk,Y_\pk,\Mc_{Y_\pk}'(X_\pk))$ for every $\pk\in \N$. In addition, we say that the channel is \emph{uniform} if $X_\pk$ and $Y_\pk$ are independent random variables uniformly distributed in $\{\pm 1\}^n$ for all $\pk\in\N$.
\end{definition}




% \begin{lemma}~\label{lem:dp-sv-source}
% 	Let $P$ be an $\varepsilon$-DP randomized protocol. Let $X$ and $Y$ be independent random variables uniformly distributed in $\{\pm 1\}^n$ and let random variable $\Pi(X,Y)$ denote the transcript of running $P(X,y)$. Then for every $\pi\in Supp(\Pi)$, the random variables corresponding to the inputs conditioned on transcript $\pi$, $X_\pi$ and $Y_\pi$, are independent $e^{-\varepsilon}$-strong SV source.
% \end{lemma}





\subsubsection{Weak Erasure Channel (\WEC)}

\begin{definition}[\WEC]\label{def:WEC}
	A channel $((O_A,V_A), (O_B,V_B))$ with $O_A \in \set{0,1}$ and $O_B \in \set{0,1,\bot}$ is a {\sf weak erasure channel}, denoted $(\alpha,p,q)$-$\WEC$, if:
	\begin{itemize}
		%\item $O_A\in \set{-1,1}$ and $O_B\in \set{-1,1,\bot}$.
		\item Random erasure: $\pr{O_B = \perp} = 1/2$.
		
		\item Agreement: $\pr{O_A\ne O_B\mid O_B\ne \bot}\le \alpha$.
		
		\item Secrecy:
		
		\begin{enumerate}
			\item For every algorithm $\Dc$ it holds that\label{WEC:item:A}
			\begin{align*}
				%\size{\pr{\Ac(O_A,V_A) = 1 \mid O_B \neq \perp} - \pr{\Ac(O_A,V_A) = 1 \mid O_B = \perp}} \le p
				\size{\pr{\Dc(V_A) = 1 \mid O_B \neq \perp} - \pr{\Dc(V_A) = 1 \mid O_B = \perp}} \le p
			\end{align*}
			(Alice doesn't know if $O_B = \perp$.)
			
			\item For every algorithm $\Dc$ it holds that\label{WEC:item:B}
			\begin{align*}
				\pr{\Dc(V_B) = O_A \mid O_B=\bot} \leq \frac{1+q}{2}.
			\end{align*}
			(i.e., if $O_B=\bot$, Bob don't know what is the value of $O_A$).
			
			%\item $SD((O_A U|O_B=\bot),(O_A U|O_B\ne \bot))\le p$ (The sender don't know if $O_B=\bot$).
			
			%\item $SD(V O_A|O_B=\bot,V(-O_A)|O_B=\bot)\le q$ (If $O_B=\bot$, Bob don't know what the value of $O_A$).
		\end{enumerate}
	\end{itemize}
   We say that a channel ensemble $C=\set{C_\pk}_{\pk\in N}$ is a {\sf computational weak erasure channel}, denoted $(\alpha,p,q)$-\CompWEC, if for every \ppt algorithm $\Dc$ and every sufficiently large $\pk\in\N$, $C_\pk$ satisfies the properties stated in the items above, where the secrecy property holds with respect to a \ppt algorithm $\Dc$. A protocol $\Lambda$ is said to be $(\alpha,p,q)$-$\CompWEC$, if the ensemble induces by the protocol (that is, $C=\set{C_{\Lambda(\pk)}}_{\pk\in\N}$) is $(\alpha,p,q)$-$\CompWEC$.  
\end{definition}



\subsubsection{Approximate Weak Erasure Channel (\AWEC)}\label{sec:AWEC}

\begin{definition}[\AWEC]\label{def:AWEC}
	A channel $C = ((O_A,V_A), (O_B,V_B))$ over $([-n,n] \times \zo^*) \times (([-n,n] \cup \bot)  \times \zo^*)$ is an {\sf approximate weak erasure channel}, denoted $(\ell,\alpha,p,q)$-\AWEC if:
	\begin{itemize}
		
		\item Random erasure: $\pr{O_B = \perp} = 1/2$.
		
		\item Accuracy: $\pr{\size{O_A - O_B} > \ell \mid O_B \ne \bot}\le \alpha$.
		
		\item Secrecy:
		
		\begin{enumerate}
			\item For every algorithm $\Dc$ it holds that\label{AWEC:item:A}
			\begin{align*}
				%\size{\pr{\Ac(O_A,V_A) = 1 \mid O_B \neq \perp} - \pr{\Ac(O_A,V_A) = 1 \mid O_B = \perp}} \le p
				\size{\pr{\Dc(V_A) = 1 \mid O_B \neq \perp} - \pr{\Dc(V_A) = 1 \mid O_B = \perp}} \le p
			\end{align*}
			(Alice doesn't know if $O_B=\bot$).
			
			\item For every algorithm $\Dc$ it holds that\label{AWEC:item:B}
			\begin{align*}
				\pr{\size{\Dc(V_B) - O_A} \leq 1000 \ell \mid O_B=\bot} \leq q.
			\end{align*}
			(i.e., if $O_B=\bot$, Bob can't estimate the value of $O_A$ with error $\leq 1000 \ell$).
		\end{enumerate}
	\end{itemize}
     We say that a channel ensemble $C=\set{C_\pk}_{\pk\in N}$ is a {\sf computational approximate weak erasure channel}, denoted $(\ell,\alpha,p,q)$-\CompAWEC, if for every \ppt algorithm $\Dc$ and every sufficiently large $\pk\in\N$, $C_\pk$ satisfies the properties stated in the items above. A protocol $\Gamma$ is said to be $(\ell,\alpha,p,q)$-$\CompAWEC$, if the ensemble induced by the protocol (that is, $C=\set{C_{\Gamma(\pk)}}_{\pk\in\N}$) is $(\ell,\alpha,p,q)$-$\CompAWEC$.  
\end{definition}

We will make use of the following lemma, which shows that for some choices of the parameters, \AWEC implies \WEC. The lemma is proven in \cref{sec:AWEC-to-WEC}.

\begin{lemma}\label{lemma:AWEC-to-WEC}
	For every $\ell> 0$, there exists a \ppt protocol $\Lambda = (\Pc_1,\Pc_2)$ such that given an oracle access to an $(\ell,\alpha,p,q)$-\AWEC $C$, the channel $\tilde{C}$ induced by $\Lambda^C$ is $(\alpha'=\alpha+0.001,\: p' = p ,\:  q' = 1/2 + 2(q+0.01))$-\WEC.
	Furthermore, the proof is constructive in a black-box manner:
	\begin{enumerate}
		\item There exists an oracle-aided \ppt algorithm $\Ec_1$ such that for every channel $C = ((\OA,\VA), (\OB,\VB))$ and algorithm $\Dc$ violating the \WEC secrecy property~\ref{WEC:item:A} of $\tilde{C}$, algorithm $\Ec_1^{\Dc}$ violates the \AWEC secrecy property~\ref{AWEC:item:A} of $C$.
		
		\item There exists an oracle-aided \ppt algorithm $\Ec_2$ such that for every channel $C = ((\OA,\VA), (\OB,\VB))$ and algorithm $\Dc$ violating the \WEC secrecy property~\ref{WEC:item:B} of $\tilde{C}$, algorithm $\Ec_2^{\Dc}$ violates the \AWEC secrecy property~\ref{AWEC:item:B} of $C$.
	\end{enumerate}
\end{lemma}

Since \cref{lemma:AWEC-to-WEC} is constructive, the following is an immediate corollary.
\begin{corollary}\label{cor:CompAWEC to CompWEC}
There exists an oracle aided \ppt protocol $\Lambda$, such that given a protocol $\Gamma$ that induces $(\ell,\alpha,p,q)$-\CompAWEC, it holds that $\Lambda^\Gamma$ is $(\alpha'=\alpha+0.001,\: p' = p ,\:  q' = 1/2 + 2(q+0.01))$-\CompWEC.  
\end{corollary}
\begin{proof}[Proof of \ref{cor:CompAWEC to CompWEC}]
Let $\Lambda$ be the \ppt algorithm guaranteed  by Lemma \ref{lemma:AWEC-to-WEC}. Given an $(\ell,\alpha,p,q)$-\CompAWEC protocol $\Gamma$, we define $\Lambda(\pk)={\Lambda^{\Gamma(\pk)}(\pk)}$. Assume towards a contradiction that $\Lambda$ is not a $(\alpha',p',q')$-\CompWEC. It follows that there exists a \ppt $\Dc$ that for infinity many $\pk\in\N$ contradicts one of the \WEC secrecy properties of channel ensemble $\set{C_{\Lambda(\pk)}}_{\pk\in\N}$. Fix $\pk\in\N$ for which this holds. By Lemma \ref{lemma:AWEC-to-WEC}, there exists a \ppt $\Ec^\Dc$ that for every such $\pk$  contradicts one of the secrecy properties of the channel $C_{\Gamma(\pk)}$. This implies that for infinity many $\pk\in\N$, $\Ec^\Dc$  contradict the secrecy of the channel ensemble $\set{C_{\Gamma(\pk)}}_{\pk\in\N}$, which is a contradiction since this would means that $\Gamma$ is not a $(\ell,\alpha,p,q)$-\CompAWEC.       
\end{proof}



\subsection{Oblivious Transfer (\OT)}

\paragraph{Secure Computation.}
We use the standard notion of securely computing a functionality, \cf  \cite{Goldreich04}.
\begin{definition}[Secure computation]\label{def:SFE}
	A two-party protocol {\sf securely computes a functionality $f$}, if it does so according to the real/ideal paradigm.   We add the term perfectly/statistically/computationally/non-uniform computationally, if the simulator's output is  perfect/statistical/computationally indistinguishable/  non-uniformly indistinguishable from  the real distribution.  The protocol have the above notions of security {\sf against semi-honest  adversaries}, if its security only  guaranteed to holds against an adversary that follows the prescribed protocol.   Finally, for the case of perfectly secure computation, we naturally apply the above notion also to the non-asymptotic case: the protocol with no security parameter perfectly  compute a functionality $f$.
	
	A two-party protocol {\sf securely computes a functionality ensemble $f$ with oracle to a channel $C$}, if it does so according to the above definition when the parties have access to a trusted party computing $C$. All the above adjectives naturally extend to this setting.
\end{definition}

\paragraph{Oblivious Transfer.}
The (one-out-of-two) oblivious transfer functionality is defined as follows.
\begin{definition}[oblivious transfer functionality $f_{\OT}$]\label{def:OTfunc}
	The oblivious transfer functionality over $\zo \times (\zs)^2$ is defined by  $f_{\OT} (i,(\sigma_0,\sigma_1)) = (\perp,\sigma_i)$.
\end{definition}
A protocol is $\ast$ secure OT,   for \\$\ast\in \set{\text{semi-honest statistically/computationally/computationally non-uniform}}$, if it  compute the $f_{\OT}$  functionality with $\ast$ security.





% \begin{definition}[Computational oblivious transfer, semi-honest model]
% A protocol $\Pi=(\Ac,\Bc)$ is a semi-honest 1-out-of-2 computational oblivious transfer (comp-OT) protocol if the following holds. Given a common input $1^{\pk}$, the parties $\Ac$ and $\Bc$ run the protocol $\Pi(1^\pk)$ (in an honest manner) and    
% $\Ac$ outputs $X=(m_1,m_2)\in \zo\times\zo$ and has a view $U$ and $\Bc$ outputs $Y=(i,\hat{m})\in\zo\times\zo$ and has a view $V$, and the following properties are satisfied:
% \begin{enumerate}
%     \item \textbf{Correctness:} 
%     $\pr{\hat{m}\neq m_i}<\negl(\pk).$ 
    
%     \item \textbf{A's Privacy:} For every \ppt $\Dc$ and every sufficiently large $\pk$:
%     $\pr{\Dc(V)=m_{i-1}}<(1+\negl(\pk))/2$
    
%     \item \textbf{B's Privacy:} For every \ppt $\Dc$ and every sufficiently large $\pk$:
%     $\pr{\Dc(U)=i}<(1+\negl(\pk))/2$  
% \end{enumerate}
% \end{definition}

We make use of the following useful results by Wullschleger on oblivious transfer amplification from weak channels.
\begin{theorem}[\cite{Wullschleger09}, from \WEC to statistically secure \OT]\label{thm:WEC TO OT IT}
    There exists an oracle aided protocol $\Pi$ such that the following holds: Given a $(\alpha,p,q)$-\WEC $C$, if $44(\alpha+p)\le 1-q$ then $\Pi^{C}(1^\pk)$ is a semi-honest statistically secure \OT.
\end{theorem}

The following computational version of \cref{thm:WEC TO OT IT} is implicit in \cite{Wullschleger09} and is based on the computational proof explicitly stated in \cite{Wul07} (see Section 6 in \cite{Wullschleger09} for discussion).   

\begin{theorem}[\cite{Wullschleger09,   Wul07}, from \CompWEC to computinally secure \OT]\label{thm:WEC TO OT Comp}
    There exists an oracle aided protocol $\Pi$ such that the following holds: Given a $(\alpha,p,q)$-\CompWEC protocol $\Lambda$, if $44(\alpha+p)\le 1-q$ then $\Pi^{\Lambda}$ is a semi-honest computational secure \OT.
\end{theorem}



% \begin{definition}[Computational 1-out-of-2 Oblivious Transfer, semi-honest model]
% A protocol $\Pi=(\Ac,\Bc)$ is a semi-honest 1-out-of-2 $(\eps,\alpha,\beta)$-oblivious transfer (OT) protocol if the following holds. 

% The parties $\Ac$ and $\Bc$ run the protocol (in an honest manner) and    
% $\Ac$ outputs $X=(m_1,m_2)\in \zo\times\zo$ and has a view $U$ and $\Bc$ outputs $Y=(i,\hat{m})\in\zo\times\zo$ and has a view $V$, and following properties are satisfied:
% \begin{enumerate}
%     \item \textbf{Correctness:} 
%     $\pr{\hat{m}\neq m_i}<\eps.$ 
    
%     \item \textbf{A's Privacy:} For every adversary $\Dc$:
%     $\pr{\Dc(V)=m_{i-1}}<(1+\alpha)/2$
    
%     \item \textbf{B's Privacy:} For every adversary $\Dc$: $\pr{\Dc(U)=i}<(1+\beta)/2$  
% \end{enumerate}
% \end{definition}
% TODO more details in methodology and data processing
% merge methodology and 
\section{Framework for Analyzing Emotion}
In this section, we present our framework for analyzing emotion. We first establish a basic understanding of emotion polarity by determining the sentiment valence of each root tweet and comment. We then use multi-label emotion detection to predict the emotion categories associated with each post. Based on this data, we explore the interactive nature of emotions, by identifying common patterns in emotion transition pairs between temporally-adjacent posts. Finally we investigate the emotional trajectory within threads to understand how emotional intensity and type shift over time, by aggregating the predicted labels for posts at each time stamp in a given thread. As part of this, we contrast rumour with non-rumour threads, to gain a holistic understanding of emotional expression in rumours and non-rumours on Twitter.

% elaborate a bit on why we choose EmoLLM, compared with other automatic emotion detection methods
\paragraph{Affective Computing: Automatic Emotion Detection}
Manually annotating emotions is both costly and time-consuming, so we use an LLM-based emotion detection model, EmoLLM~\citep{liu2024emollms}, which is specifically designed for sentiment analysis and emotion detection. The model was instruction-tuned on SemEval 2018 Task1 using a comprehensive emotion labeling scheme grounded in established theoretical frameworks. We prompt the model to perform Valence Ordinal Classification (V-oc), Emotion Classification (E-c), and Emotion Intensity regression (E-i). Detailed prompts are shown in \Cref{tab:emollm_ins}.

\paragraph{Categorical Emotion Labeling Scheme} \label{para:emotion_label}
Numerous emotion label sets  have been proposed~\citep{Ekman1992AnAF, Plutchik1980AGP, Russell1980ACM}. According to \citet{Ekman1992AnAF, Plutchik1980AGP}, certain emotions, such as joy, fear, and sadness, are considered more fundamental than others, both physiologically and cognitively. The Valence-Arousal-Dominance (VAD) model \citep{Russell1980ACM} categorizes emotions within a three-dimensional space of valence (positivity-negativity), arousal (active-passive), and dominance (dominant-submissive). Inspired by \citet{mohammad-etal-2018-semeval}, we incorporate elements from both basic emotion theories and the VAD model, and further ground EmoLLM emotion predictions to develop the following emotion label schemes: (1) \textit{neutral or no emotion}; (2) \textit{negative emotions}: anger (also includes annoyance and rage),  disgust (also includes disinterest, dislike, and loathing), fear (also includes apprehension, anxiety, and terror), pessimism (also includes cynicism, and no confidence), sadness (also includes pensiveness and grief); 3) \textit{positive emotions}: joy (also includes serenity and ecstasy), love (also includes affection), optimism (also includes hopefulness and confidence), anticipation (also includes interest and vigilance), surprise (also includes distraction and amazement) and trust (also includes acceptance, liking, and admiration). 


\paragraph{Emotion Polarity: Sentiment Valence} 
To understand the basic emotion polarity expressed in rumour and non-rumour content, we begin with sentiment valence analysis. Sentiment valence aims to capture the overall emotional tone conveyed by a post, in terms of how positive or negative it is~\citep{liu2024emosurvey}. We frame the sentiment valence task as ordinal regression~\citep{mohammad-etal-2018-semeval}. As shown in \Cref{tab:emollm_ins}, for a given tweet post, we classify it into one of seven ordinal levels of sentiment intensity, spanning varying degrees of positive and negative valence, that best represents the tweeter's mental state. The tweet posts within a thread can be divided into two categories: root tweets, which are posted by the publisher, and follow posts, which include all subsequent replies under the root post. We begin by conducting sentiment valence analysis on each post within the thread conversation. 
% TJB: confused by how comments can include all subsequent replies; we seem to be overloading the terminology, for comments to be both individual posts and series of posts
% RX: yes, I am unifiying all terms.
For each category, we compute the mean sentiment valence to enable further investigation into the specific emotions associated with different sentiment valences over a thread.
% TJB: clarify for comments whether the classification is done over the combined meta-document (i.e. the root + all comments to that point) or individually over the separate documents and then combined ... or over individual documents, in which case the statement about "all replies" needs clarification
% RX: we separate root and comments for each tweet conversation, the former is the root tweet posted by the publisher while the rest are comments. "all replies" mean all comments under root tweet, we aggregate them by computing the mean sentiment, and then average over each part.

\paragraph{Emotion Distribution} 
Following sentiment valence analysis, we then examine specific emotions and their distribution in rumour and non-rumour tweet posts.
Motivated by the fact that a certain tweet might exhibit more than one emotion, we frame the task as multi-label emotion detection problem. As shown as V-oc in \Cref{tab:emollm_ins}, given a tweet, we classify it into one of seven ordinal classes, corresponding to various levels of positive and negative sentiment intensity. To reduce noise from automatic emotion detectors, we take the top-three predicted emotions for each tweet. We then aggregate and plot the emotion distribution to provide an overview of dominant emotional trends across the rumour and non-rumour posts. Given that the follow posts make up the majority of the data compared to the root posts, we will focus on using follow posts in our next analysis.
% TJB: what is the basis of saying that the signal is richer? simply that there are more reply posts than root posts? clarify
% RX: yes, and we are more interested in interaction in comments.

\paragraph{Emotion Transitions} 
Emotions are contagious and highly interactive~\citep{Ferrara_2015}. When publishers write tweets that convey their emotions, readers are likely to respond with emotional reactions of their own~\citep{Ferrara_2015,emotion_dynamics}. In this part, we model this interactive nature of emotions in the form of emotion transition pairs, which are built from two chronologically-adjacent tweets. In each pair, the first element represents the emotion inferred from the initial content published at a given time, and the second element represents the emotion inferred from the reply content published immediately after. For example, if the first tweet exhibits \textit{joy} \textit{trust} and \textit{anticipation}, and the second tweet shows \textit{anger}, \textit{disgust} and \textit{surprise}, we form the pairs (\textit{joy}, \textit{anger}), (\textit{joy}, \textit{disgust}), (\textit{joy}, \textit{surprise}), (\textit{trust}, \textit{surprise}), (\textit{trust}, \textit{surprise}), (\textit{trust}, \textit{disgust}), (\textit{anticipation}, \textit{anger}), (\textit{anticipation}, \textit{surprise}) and (\textit{anticipation}, \textit{disgust}). We create transitions for all combinations of emotion pairs and explore the likelihood of emotion transition pairs occurring in rumour and non-rumour content. Exploring emotion transitions allows us to understand the emotional flow in social media conversations and uncover typical patterns of rumour and non-rumour content, and any differences between the two.

\paragraph{Emotion Trajectories} 
We explore the cumulative trajectory of emotion over time to observe how emotions evolve during the conversational thread. We collect all detected emotion labels for each tweet from both rumour and non-rumour content, then track cumulative emotion counts at each chronological step. Finally, we visualize these trends and apply regression models to analyze the growth of emotions over time. This temporal analysis reveals how emotions accumulate or intensify across time, offering insight into the trajectory of emotions in rumour and non-rumour content.

\begin{table*}[!h]
    \centering
    \small
    \begin{tabular}{cccccccccccc}
        \toprule
        \textbf{Setting} & \textbf{Ru} & \textbf{Non} & \textbf{p} & \textbf{\#Ru/Non} & \textbf{T} & \textbf{F} & \textbf{U} & \textbf{$p$ (U vs T)} & \textbf{$p$ (U vs F)} & \textbf{\#T/\#F/\#U} \\
        \midrule
        \textbf{PHEME root} & \textbf{$-$0.25} & $-$0.17 & 0.00 & 2602/2602 & $-$0.21 & $-$0.11 & \textbf{$-$0.39} & 7.75e-11 & 4.41e-11 & 629/629/629 \\
        \textbf{PHEME follow} & \textbf{$-$0.33} & $-$0.26 & 6.47e-09 & & $-$0.35 & $-$0.20 & \textbf{$-$0.39} & 0.03 & 8.38e-15 & \\
        \textbf{Twitter15 root} & \textbf{$-$0.26} & $-$0.01 & 3.51e-05 & 372/372 & $-$0.21 & $-$0.20 & \textbf{$-$0.34} & 0.01 & 0.01 & 359/359/359 \\
        \textbf{Twitter15 follow} & \textbf{$-$0.27} & $-$0.06 & 1.65e-09 & & $-$0.24 & $-$0.25 & \textbf{$-$0.30} & 0.16 & 0.21 & \\
        \textbf{Twitter16 root} & \textbf{$-$0.18} & \z0.07 & 0.00 & 205/205 & \z0.11 & $-$0.22 & \textbf{$-$0.30} & 1.35e-06 & 0.18 & 63/63/63 \\
        \textbf{Twitter16 follow} & \textbf{$-$0.31} & $-$0.12 & 9.19e-06 & & $-$0.30 & \textbf{$-$0.36} & $-$0.27 & 0.67 & 0.90 & \\
        % \textbf{CoAID root} & \textbf{$-$0.34} & $-$0.16 & 0.01 & 167/167 & - & - & - & - & - & - \\
        % \textbf{CoAID follow} & \textbf{$-$0.24} & $-$0.13 & 0.01 & & - & - & - & - & - & \\
        \bottomrule
    \end{tabular}
    \caption{Valence Ordinal Regression results for all datasets. root = root posts, follow = follow posts, Ru = rumour, Non = Non-rumour, T = True rumour, F = False rumour, U = Unverified rumour; $p$ values indicates significance of a one-tailed t-test.}
\label{tab:voc_results}
\end{table*}

\begin{algorithm}[ht!] 
\caption{PC Algorithm}
\label{pc}
\begin{algorithmic}[1] 
\State \textbf{Input:} Data $\mathbf{X}$, significance level $\alpha$
\State \textbf{Output:} Completed Partially Directed Acyclic Graph (CPDAG)

\State Initialize a complete undirected graph $G$ with all variables as nodes.

\State \textbf{Step 1: Skeleton Identification}
\For{each pair of variables $(X, Y)$ in $G$}
    \State Find the subset $S \subseteq \text{Adj}(X, G) \setminus \{Y\}$ such that 
    $X \indep Y \mid S$ with significance $\alpha$.
    \If{such a subset $S$ exists}
        \State Remove the edge $X - Y$ from $G$.
    \EndIf
\EndFor

\State \textbf{Step 2: Edge Orientation}
\For{each triple of variables $(X, Y, Z)$ in $G$ where $X - Z - Y$ and $X, Y$ are not adjacent}
    \If{$Z \notin S$ for all separating sets $S$ for $X$ and $Y$}
        \State Orient as $X \to Z \leftarrow Y$ (identify a collider).
    \EndIf
\EndFor

\While{possible}
    \For{each edge $(X - Y)$ in $G$}
        \If{there exists a directed path $X \to \dots \to Z$ such that $Z - Y$}
            \State Orient as $X \to Y$ (acyclicity rule).
        \ElsIf{orienting $X - Y$ as $X \to Y$ creates a new v-structure}
            \State Orient as $X \to Y$ (v-structure rule).
        \EndIf
    \EndFor
\EndWhile

\State \textbf{return} the CPDAG representing the equivalence class of causal graphs.

% how we frame the task, compute the emotion intensity, how to aggregate on conversation level

\end{algorithmic}
\end{algorithm}


\paragraph{Causal Relationship of Emotions in Rumour \& Non-Rumour Threads}
To gain a deeper insight into the relationship between rumours and the emotions underlying them, we extend our analysis beyond statistical correlation by conducting a causal analysis. Specifically, we apply the Peter-Clark (PC) algorithm \cite{Spirtes2000}, a classical constraint-based causal discovery algorithm on the three merged datasets. 

Uncovering causal relations between variables of interest is never an easy problem. Under the fundamental assumption of \textit{causal Markov condition} that a variable is conditionally independent of all its non-effects given its direct cause, \textit{faithfulness} ensures that the casual graph exactly encodes the independence and conditional independence relations among variables. These two assumptions allow us to infer causal relationships from observed statistical independencies, forming the cornerstone of constraint-based causal discovery methods. 

The PC algorithm identifies causal relationships among the variables of interest, represented as a directed acyclic graph (DAG), by numerating the independence and conditional independence relationships. The algorithm consists of two main steps: 
\begin{enumerate}
    \item \textbf{Skeleton Identification}: Starting with a complete undirected graph where all variables are connected, edges are iteratively removed based on conditional independence and independence relationships among variables, inferred by a conditional independence test. This step returns an undirected graph, which we call a skeleton. 
    \item \textbf{Edge Orientation}: After constructing the skeleton, edges are oriented by a set of predefined rules (Meek's Rule \cite{meek1997graphical}) to avoid cycles and orient collider structures.
\end{enumerate}

The complete PC algorithm is provided in algorithm \ref{pc}. It returns a  completed partially directed acyclic graph (CPDAG), which represents an equivalence class of causal graphs that are consistent with the observed data’s independence and conditional independence relations. In our implementation, we adopt the  Fisher-z test \cite{fisher_probable_1921} to infer the conditional independence relations.

%%% Local Variables:
%%% mode: latex
%%% TeX-master: "../main_anonymous"
%%% End:

% !TEX root = main.tex
\section{Minimum Cardinality Interdiction Problems}

In this section, we prove our $\Sigma^p_2$-completeness results regarding the minimum cardinality interdiction problem. 
Since we want to prove the theorem simultaneously for multiple problems at once, we require an abstract definition of the interdiction problem.
For this, consider the following definition.

\begin{definition}[Minimum Cardinality Interdiction Problem]
\label{def:min-card-interdiction-pi}
    Let an SSP problem $\Pi = (\I, \U, \sol)$ be given.
    The minimum cardinality interdiction problem associated to $\Pi$ is denoted by \textsc{Min Cardinality Interdiction-$\Pi$} and defined as follows:
    The input is an instance $I \in \I$ together with a number $k \in \N_0$.
    The question is whether
    \[
        \exists B \subseteq \U(I), \ |B| \leq k : \forall S \in \sol(I) : B \cap S \neq \emptyset.
    \]
\end{definition}

For the remainder of the paper, it is helpful to imagine this problem as a game between two players: the \emph{attacker} and the \emph{defender}.
That is, interdiction is an action performed by an attacker (or interdictor), who wishes to select a blocker of few elements to destroy all solutions.
On the other hand, the defender wants to find a solution to the problem after the attacker selected a blocker.
This leads to the following interpretation:
\begin{itemize}
    \item The set $\U(I)$ contains all the elements the attacker is allowed to attack. 
    \item The set $\sol(I)$ contains all the solutions the attacker wants to destroy such that the defender is not able to find any solution.
    For example, this could be the set of all Hamiltonian cycles, the set of all cliques of a certain size, etc.
\end{itemize}
Therefore, the formulation of the base problem as SSP problem $(\I, \U, \sol)$ determines which elements the attacker can attack, which he cannot attack (e.g. edges/vertices of a graph), and what the attacker's goal is.
We note that different formulations $(\I, \U, \sol)$ of the same problem are formally different SSP problems. They might be both SSP-NP-complete independent of each other, but require their own SSP-NP-completeness proof each.
For all the concrete problems studied in this paper, our complexity results hold for the natural choices of $(\I, \U, \sol)$ formally given in \cref{app:sec:problemDefinitions}.
Finally, note that if the base problem is an LOP problem, then by definition $\sol(I)$ is the set of feasible solutions below some threshold specified in the input. 
For example, applying \cref{def:min-card-interdiction-pi} to $\Pi = \textsc{Clique}$ yields the following decision problem: 
\begin{quote}
    \textbf{Problem}: $\textsc{Min Cardinality Interdiction-Clique}$

    \textbf{Input}: Graph $G = (V, E)$, numbers $k, t \in\N_0$

    \textbf{Question}: Does there exist a subset $B \subseteq V$ of size $|B| \leq k$ such that every clique of size at least $t$ shares at least one vertex with $B$? 
\end{quote}

Some more technical details, concerning the subtle differences between different variants of interdiction referenced in the literature as well as concerning the question whether $t$ can be chosen to be optimal are discussed in \cref{sec:different-variants-of-interdiction}.
We now proceed with the main result.
For the complexity analysis of minimum cardinality blocker, we first show the containment in the class $\Sigma^p_2$, if the nominal problem is in NP.

\begin{lemma}
\label{lem:containment}
    Let $\Pi = (\I, \U, \sol)$ be an SSP problem in $NP$, then \textsc{Min Cardinality Interdiction}-$\Pi$ is in $\Sigma^p_2$.
\end{lemma}
\begin{proof}
    We provide a polynomial time algorithm $V$ that verifies a specific solution $y_1, y_2$ of polynomial size for instance $I$ such that
    $$
        I \in L \ \Leftrightarrow \ \exists y_1 \in \{0,1\}^{m_1} \ \forall y_2 \in \{0,1\}^{m_2} : V(I, y_1, y_2) = 1.
    $$
    With the $\exists$-quantified $y_1$, we encode the blocker $B \subseteq \U(I)$.
    The encoding size of $y_1$ is polynomially bounded in the input size of $\Pi$ because $|\U(I)| \leq poly(|x|)$.
    Next, we encode the solution $S \in \sol(I)$ to the nominal problem $\Pi$ using the $\forall$-quantified $y_2$ within polynomial space.
    This is doable because the problem $\Pi$ is in NP (and thus $co\Pi$ is in coNP).
    At last, the verifier $V$ has to verify the correctness of the given solution provided by the $\exists$-quantified $y_1$ and $\forall$-quantified $y_2$.
    Checking whether $|B| \leq t$ and $B \cap S \neq \emptyset$ is trivial and checking whether $S \in \sol(I)$ is clearly in polynomial time because $\Pi$ is in SSP-NP.
    It follows that $\textsc{Min Cardinality Interdiction-}\Pi$ is in $\Sigma^p_2$.
\end{proof}

Next, we show the hardness of minimum cardinality interdiction problems as long as the nominal problem is NP-complete.
For this, we introduce the concept of invulnerability reductions that helps us to grasp the problems in a unified approach.
We describe this concept in the following subsection with the goal to obtain the following main theorem of the paper.

\begin{theorem}
\label{thm:min-card-interdiction}
    The problem \textsc{Min Cardinality interdiction-$\Pi$} is $\Sigma^p_2$-complete for all the following problems:
    independent set,
    clique,
    subset sum,
    knapsack,
    Hamiltonian path/cycle (directed/undirected),
    TSP,
    $k$-directed vertex disjoint paths ($k \geq 2$),
    Steiner tree,
    dominating set,
    set cover,
    hitting set, feedback vertex set,
    feedback arc set,
    uncapacitated facility location,
    $p$-center,
    $p$-median.
\end{theorem}

We remark that the case of satisfiability deserves special attention, which is discussed more thoroughly in \cref{sec:noMeta}.

\subsection{Invulnerability Reduction}

Our proof strategy for each of the problems listed in \Cref{thm:min-card-interdiction} is essentially the same.
In fact, we show that \cref{thm:min-card-interdiction} is actually a consequence of the following, more powerful \emph{meta-theorem}.
This meta-theorem catches the essence of an invulnarability reduction.

\begin{theorem}
\label{thm:meta-theorem}
    Consider an SSP-NP-complete problem $\Pi$.
    If there exists a polynomial-time reduction $g$ which receives as input a tuple $(I, C, k)$ of an instance $I$ of $\Pi$, some set $C \subseteq \U(I)$ and some $k \in \N_0$, and returns instances $I' := g(I, C, k)$ of $\Pi$, such that the following holds: 
     \begin{align*}
         \exists B \subseteq C : |B| \leq k \text{ and } B \cap S \neq \emptyset \ \forall S \in \sol(I) \qquad\qquad\qquad\qquad\qquad\qquad\qquad\qquad\quad\\
         \Leftrightarrow \ \ \exists B' \subseteq \U(I') : |B'| \leq k \text{ and } B' \cap S' \neq \emptyset \ \forall S' \in \sol(I').
     \end{align*}
     Then \textsc{Min Cardinality Interdiction-$\Pi$} is $\Sigma^p_2$-complete. 
\end{theorem}

It would be nice to have \cref{thm:meta-theorem} for all problems in the class SSP-NPc, not only those who admit a funciton $g$ with the properties as described above. However, we give a reasoning in \Cref{sec:noMeta} why such a generalization is not possible.
The rest of this section is devoted to the proof of \cref{thm:min-card-interdiction}.
In \cite{gruene2024completeness} the following more general version of interdiction was considered, where there is a set $C \subseteq \U(I)$ of so-called vulnerable elements.
One can also interpret the set of vulnerable elements $C$ as the elements that have cost of interdiction of $1$ while all other elements $\U(I) \setminus C$ have a cost of interdiction of $\infty$ and a blocker of small costs is sought.
This problem is called the \emph{combinatorial interdiction problem}.

\begin{definition}[Comb. Interdiction Problem, from \cite{gruene2024completeness}.]
    Let an SSP problem $\Pi = (\I, \U, \sol)$ be given.
    We define \textsc{Comb. Interdiction-$\Pi$} as follows:
    The input is an instance $I \in \I$, a number $k \in \N_0$, and a set $C \subseteq \U(I)$. The set $C$ is called the set of vulnerable elements.
    The question is whether
    \[
        \exists B \subseteq C, \ |B| \leq k : \forall S \in \sol(I) : B \cap S \neq \emptyset.
    \]
\end{definition}

It is proven in \cite{gruene2024completeness} that for every problem in SSP-NPc, the combinatorial interdiction problem is $\Sigma^p_2$-complete.
Now, let $\Pi$ be in SSP-NPc and $g$ be a reduction such that
    \begin{align*}
         \exists B \subseteq C : |B| \leq k \text{ and } B \cap S \neq \emptyset \ \forall S \in \sol(I) \qquad\qquad\qquad\qquad\qquad\qquad\qquad\qquad\quad\\
         \Leftrightarrow \ \ \exists B' \subseteq \U(I') : |B'| \leq k \text{ and } B' \cap S' \neq \emptyset \ \forall S' \in \sol(I'),
     \end{align*}
then $g$ is a reduction from \textsc{Comb. Interdiction-$\Pi$} to \textsc{Min Cardinality interdiction-$\Pi$}. This is because the first line is equivalent to the statement that instance $I$ is a yes-instance of \textsc{Comb. Interdiction-$\Pi$}, and the second line is equivalent to the statement that $I'$ is a yes-instance of \textsc{Min Cardinality interdiction-$\Pi$}.
It directly follows that \textsc{Min Cardinality interdiction-$\Pi$} is $\Sigma^p_2$-complete. This completes the proof of \cref{thm:meta-theorem}. 

We remark that while in some sense the proof is rather trivial, we still see a lot of value in explicitly stating a set of easy-to-check sufficient conditions that render some minimum-cardinality interdiction problem $\Sigma^p_2$-complete.

How can one find a function $g$ with the properties as described above? Often times it is possible by employing the following natural idea:
Given an instance of the comb. interdiction problem, let the set $D := \U(I) \setminus C$ be called the \emph{invulnerable} elements. 
For each problem separately we explain that a gadget for the invulnerable elements in $D$ exists, which
intuitively speaking guarantees that an attacker, no matter which $k$ elements of the universe they attack, can never render the elements of $D$ unusable.
On the other hand, we make sure that the \emph{invulnerability gadgets} do not meaningfully change the set of solutions.
The next section gives many examples of such gadgets.
We remark that we are not the first to come up with this natural idea.
For example, Zenklusen \cite{DBLP:journals/dam/Zenklusen10a} used the same idea in the context of matching interdiction.

\subsection{Different Variants of Interdiction}
\label{sec:different-variants-of-interdiction}
In this section, we discuss variants of interdiction problems that can be found in the literature.
For this, we study the relation of our definition of a minimum cardinality interdiction problems and the existing variants.
Additionally, we argue what the implications of the hardness of our minimum cardinality interdiction problems on the other variants are.

\begin{description}
    \item[1. Minimal Blocker Problem.]\hfill
        \begin{description}
            \item[Input] Instance $I$ with universe $U$, blocker cost function $c$, solution cost function $d$, and solution threshold $\tau$
            \item[Task] Find the minimum-cost set $\min_{B \subseteq U} c(B)$ such that for all solutions $S$ with $S \cap B = \emptyset$, we have $d(S) \leq \tau$.
        \end{description}
    \item[2. Full Decision Variant of Interdiction.]\hfill
        \begin{description}
            \item[Input] Instance $I$ with universe $U$, blocker cost function $c$, blocker budget $k$, solution cost function $d$, and solution threshold $\tau$
            \item[Task] Is there a set $B \subseteq U$ with $c(B) \leq k$ such that for all solutions $S$ with $S \cap B = \emptyset$, we have $d(S) \leq \tau$?
        \end{description}
    \item[3. Most Vital Elements Problem.]\hfill
        \begin{description}
            \item[Input] Instance $I$ with universe $U$, blocker cost function $c$, and solution cost function $d$
            \item[Task] Find a set $B \subseteq U$ with $c(B) \leq k$ such that the costs of all solutions $S \cap B = \emptyset$ are maximized, i.e. $\max_{B} \min_{S, S \cap B = \emptyset} d(S)$.
        \end{description}
\end{description}

Our goal is to show that all of the variants from above are at least as hard as our formulation of \emph{minimum cardinality interdiction} (\Cref{def:min-card-interdiction-pi}).
This results in the following theorem.

\begin{theorem}
    Let $\Pi = (\I, \U, \sol)$ be an SSP problem.
    Then the
    Most Vital Elements Problem of $\Pi$ (for all problems $\Pi$ in \Cref{thm:min-card-interdiction}),
    the Minimal Blocker Problem of $\Pi$, and
    the Full Decision Variant of Interdiction of $\Pi$
    are at least as hard to compute as \textsc{Min Cardinality Interdiction}-$\Pi$.
\end{theorem}

The rest of this section is devoted to the proof of this theorem.
In our formulation of minimum cardinality interdiction, a set $B$ is sought, which intersects every solution in the set $\sol$ as given by the corresponding SSP problem.
We now have to distinguish between problems, which are naturally formulated as SSP problems (e.g. Hamiltonian cycle), and SSP problems, which are derived from an LOP problem (e.g. clique).
For natural SSP problems, the solution set $\sol$ consists of all solutions, i.e. there are no feasible solutions outside of $\sol$ due to the missing cost function $d$ on the solution elements.
Thus all of the three variants from above are generalizations of minimum cardinality interdiction:
\begin{enumerate}
    \item The \emph{minimal blocker problem} is the optimization version of the corresponding minimum cardinality interdiction problem.
    \item The \emph{full decision version of interdiction} is a generalization of the corresponding minimum cardinality interdiction problem because the latter assumes to have unit costs in the cost function $c$ for all elements from $U$.
    \item The \emph{most vital element problem} behaves analogous to (2).
\end{enumerate}
For SSP problems that are derived from an LOP problem, basically the same holds, however, with a modified and a technically more intricate argumentation.
Here the solution set is defined by $\sol = \{F \in \F : d(F) \leq t\}$ and we can find a reduction by generalization as follows:
\begin{enumerate}
    \item For \emph{minimal blocker problems}, we can set $\tau := t-1$.
    Then, we again have that the minimal blocker problem is the optimization version of the corresponding minimum cardinality interdiction problem.
    \item For \emph{minimal blocker problems}, we can also set $\tau := t-1$.
    Then, the full decision version is again a generalization of the corresponding minimum cardinality interdiction problem due to the fact that the latter has a unit cost function $c$.
    \item For \emph{most vital element problem}, the situation is more complicated.
    We first observe that the blocker part of $B \subseteq U$ with $c(B) \leq k$ is a generalization of the blocker part in minimum cardinality interdiction.
    The inner part on the nominal problem deserves special attention, though, due to the fact that the most vital element problem maximizes the objective while minimum cardinality interdiction blocks all solutions from the solution set $\sol$.
    We focus on this in the next paragraph.
\end{enumerate}

\par{\bf Reducing Minimum Cardinality Interdiction to Most Vital Elements.}
The concepts of minimum cardinality interdiction and most vital elements coincide if and only of the set $\sol$ contains exactly the optimal solutions, i.e. $\sol = \{F \in \F : d(F) \leq t^\star\}$, where $t^\star$ is optimal (i.e. minimal).
In order to assure that $\sol$ captures exactly the optimal solutions, we need to include this condition into the reduction.
In particular, the SSP reduction $(g, f)$ needs to guarantee that all instances $I$ are mapped to instances $g(I)$ such that all possible solutions are necessarily optimal.
In other words, $t$ is the optimal objective value of the LOP instance $g(I)$, since there are no feasible solutions, whose cost is even smaller than $t$.
We call SSP reductions that fulfill this criterion \emph{tight} and formally define them as follows.

\begin{definition}[Tight SSP reduction]
    Let $\Pi_1$ be an SSP problem and $\Pi_2 = (\I, \U, \F, d, t)$ be an LOP problem.
    Consider an SSP reduction $(g, (f_I)_{I \in \I})$ from $\Pi_1$ to (the SSP problem derived from) $\Pi_2$. 
    The reduction is called tight if for all yes-instances $I_1$ of $\Pi_1$, the corresponding instance $I_2 = g(I_1)$ of $\Pi_2$ with the associated parameter $t := t^{(I_2)}$ and associated cost function $d := d^{(I_2)}$, the following holds:
    \begin{align}
        \set{ F \in \F(I_2) : d(F) \leq t } \neq \emptyset \text{ and } \set{ F \in \F(I_2) : d(F) \leq t - 1} = \emptyset
    \end{align}
\end{definition}

All SSP reductions (to SSP problems derived form LOP problems) that can be found in \cite{gruene2024completeness} fulfill this definition and are thus tight.
Therefore, for all LOP problems (independent set, clique, knapsack, TSP, Steiner tree, dominating set, set cover, hitting set, feedback vertex set, feedback arc set), we obtain that the most vital element problem is at least as hard to compute as the minimum cardinality problem.

\paragraph*{Vertex/Edge Deletion Problems}
In this paper, we are concerned with finding a set $B$ such that $B \cap S \neq \emptyset$ for every solution $S$.
Note that this definition is meaningful even if the nominal problem is not graph-based.
However, in the special case where the nominal problem is graph-based, one could also consider a very related notion which is usually called \emph{vertex deletion problem} or \emph{edge deletion problem}.
Here, the question is how many vertices (edges) need to be deleted from the graph until some desired property is met.
Element deletion problems are well-studied in classical complexity theory for hereditary graph properties \cite{DBLP:journals/jcss/LewisY80} and in parameterized complexity theory for properties expressible by first order formulas \cite{DBLP:conf/mfcs/BannachCT24}.
In the general case, element deletion problems are not the same problem as our problem \textsc{Interdiction-$\Pi$}.
This is because for every set of deleted elements, the underlying instance is changed (vertices/edges are removed, which changes the graph). This is not the case for minimum cardinality interdiction problems as defined in this paper.
Thus, it is not possible to transfer the results of minimum cardinality interdiction directly to element deletion problems.
Albeit for the problems of clique and independent set, the $\Sigma^p_2$-completeness results hold for both minimum cardinality interdiction as well as for vertex deletion interdiction 
because for these problems the deletion of a vertex coincides with not taking this vertex into the solution. 
An analogous statement holds for edge deletions for the problems of directed/undirected Hamiltonian cycle/path, $k$-vertex-disjoint path, and Steiner tree.

% !TEX root = main.tex
\section{Invulnerability Reductions for Various Problems}
\label{sec:invulnerability-gadgets}
In this section, we show that a lot of well-known problems satisfy the assumptions of \cref{thm:meta-theorem}, i.e.\ it is possible to construct so-called invulnerability gadgets for them.
Note that this proves \cref{thm:min-card-interdiction}.
(More precisely, it proves the hardness part and the containment part is analogous to \cite{gruene2024completeness}).
Let in the following always $C \subseteq \U(I)$ denote the set of vulnerable elements, let $\U(I) \setminus C$ denote the set of invulnerable elements, and $k$ denote the budget of the attacker.

\textbf{Clique.}
We have $\U = V$ in this case.
For a given graph $G = (V,E)$, and a set $C \subseteq V$, we explain how to make $V \setminus C$ invulnerable.
We obtain a graph $G'$ from $G$ by replacing every vertex $v \in V \setminus C$ with an independent set $X_v$ of size $|X_v| = k+1$. 
For a vertex $v \in C$, we define $X_v := \set{v}$.
For all edges $uv$ in $G$, the new graph $G'$ contains the complete bipartite graph between $X_u$ and $X_v$.
Note that every clique of $G'$ contains at most one vertex from every set $X_v$. Hence the size of a maximum clique is the same in $G$ and $G'$. 
Since for $v \in V \setminus C$, we have $|X_v| = k+1$ and all vertices in $X_v$ have the same neighborhood, the attacker is not able to attack all vertices of $X_v$ at once because its budget of $k$ is too small.
Hence $v$ has been made \enquote{invulnerable}.
Furthermore, for every clique in $G$, we find a corresponding clique in $G'$ that contains at most one vertex from each set $X_v$. 
Together, this implies that an attacker can find a set $B' \subseteq V(G')$ of size $|B'| \leq k$ interdicting all maximum cliques in $G'$ if and only the attacker can find a set $B \subseteq C$ of size $|B| \leq k$ interdicting all maximum cliques of $G$, i.e.\ the assumptions of \cref{thm:meta-theorem} are met.

\textbf{Independent Set.} Analogous to clique in the complement graph.

\textbf{Dominating Set.} We have $\U = V$ in this case. 
To make a vertex $v \in V \setminus C$ invulnerable, we use the same construction as for the clique problem, with the only difference that $X_v$ is a clique instead of an independent set. 
Every optimal dominating set takes at most one vertex from each set $X_v$, but all $k+1$ vertices inside $X_v$ are equivalent. More precisely, they have the same (closed) neighborhood. 
This means for an invulnerable $v \in V \setminus C$, an attacker can not attack all $k+1$ vertices of $X_v$ simultaneously. 
Furthermore, it is easily seen that on the vulnerable vertices, the attacker interdicts all optimal dominating sets in the old graph if and only if the analogous attack interdicts all optimal dominating sets in the new graph.

\textbf{Hitting Set.} In this case, we have some universe $\U$, sets $Y_1,\dots,Y_t \subseteq \U$, and the problem is to find a minimal hitting set $X \subseteq \U$ hitting all the sets $Y_j$, $j =1,\dots,t$. 
To make an element $e \in \U$ invulnerable, simply delete it and replace it by $k + 1$ copies.
We modify the sets such that every set $Y_j$ that contained $e$ now contains the $k+1$ copies of $e$ instead. 
It is clear that all the copies of $e$ hit the same sets as $e$ (i.e.\ taking multiple copies into the hitting set does not offer any advantage).
Furthermore, it is not possible for the attacker to attack all $k+1$ copies simultaneously.
By an argument analogous to the above paragraphs, we are done.

\textbf{Set cover.} We have a ground set $E$, and a family $\mathcal{F}$ of sets $S_1, \dots, S_n \subseteq E$ over the ground set. 
We let $\U := \fromto{1}{n}$ and the goal is to pick a subset $I \subseteq \U$ of the indices such that $\bigcup_{i \in I} S_i = E$.
The attacker can attack up to $k$ of the indices $i \in I$ to forbid the corresponding sets from being picked.
We can make some index $i \in \U$ invulnerable, by simply duplicating the set $S_i$ a total amount of $k+1$ times.

Note that this satisfies the assumptions of \cref{thm:meta-theorem}, but modifies the family $\mathcal{F}$ such that the same set could appear multiple times in the family.
Alternatively, our construction can be adjusted such that this is avoided.
For this, we introduce $k+1$ new elements $e_1, \dots, e_{k+1}$ and $k+2$ new elements $f_1,\dots, f_{k+2}$ to the ground set $E$.
For each invulnerable index $i \in \fromto{1}{n} \setminus C$, we substitute $S_i$ by the $k+1$ sets $S_i^{(j)} = S_i \cup \{e_j\}$ for $j=1,\dots,k+1$.
Furthermore, we introduce $k+2$ new sets $S'_j := \fromto{e_1}{e_{k+1}} \cup \fromto{f_1}{f_{k+2}} \setminus \set{f_j}$ for $j =1, \dots, k+2$. This completes the description of the instance.
Note that the following holds: The elements $\fromto{f_1}{f_{k+2}}$ are covered by a set cover, if and only if it contains at least two sets of the form $S'_j$. 
Assuming this condition is true, all the elements $\fromto{e_1}{e_{k+1}}$ are already covered.
Hence all the different copies $S_i^{(j)}$ for $j=1,\dots,k+1$ are essentially equivalent.
Thus the attacker can not meaningfully attack all these copies simultaneously.
Note that the attacker can also not meaningfully attack the sets $S'_j$, since no matter which $k$ of them are attacked, 2 of them always remain.


\textbf{Steiner tree.} We have $\U = E$ in this case.
To make an edge $uv \in E \setminus C$ invulnerable, we replace it with $k+1$ parallel subdivided edges, i.e.\ we introduce vertices $w_1, \dots, w_{k+1}$ and edges $uw_i$ and $w_iv$ for $i =1,\dots, k+1$.
Every vulnerable edge $uv$ is replaced with only a single subdivided edge, i.e.\ a vertex $w$ and edges $uw, wv$.
It is clear that the number of edges of a minimum Steiner tree in the new instance is exactly two times as big as before, and the edge $uv$ has become effectively invulnerable.  

\textbf{Two vertex-disjoint path.}
We have $\U = A$ in this case.
The gadget is the same as for Steiner tree, except that the construction is directed, i.e. the arc $(u,v)$ is replaced either by the arcs $(u,w_i), (w_i, v)$ for $i=1,\dots,k+1$ (invulnerable case) or by the two arcs $(u,w), (w,v)$ (vulnerable case).
Since the paths in this problem have to be vertex disjoint, adding additional subdivided arcs between two existing vertices does not produce additional solutions because traveling from $u$ to $v$ renders all other paths from $u$ to $v$ unusable.

\textbf{Feedback arc set.} We have $\U = A$ in this case. 
Note that making some arc $a = (u,v) \in A \setminus C$ invulnerable means to ensure that it can be used in a minimal feedback arc set, no matter which $k$ arcs the attacker chooses.
This can be achieved the following way: Subdivide $a$ into $k + 1$ arcs. 
Clearly, the set of cycles in the new graph stays essentially the same. 
Furthermore, the attacker cannot block all $k + 1$ arcs from being chosen for the solution.
Choosing one of the subdivided pieces of $a$ in the new instance has the same effect as choosing $e$ in the old instance.

\textbf{Feedback vertex set.}
We have $\U = V$ in this case.
To make a vertex $v \in V \setminus C$ invulnerable, we split it into two vertices $v_\text{in}$ and $v_\text{out}$, 
put all incoming edges of the old vertex $v$ to $v_\text{in}$, 
put all outgoing edges of the old vertex $v$ to $v_\text{out}$,
and connect $v_\text{in}$ to $v_\text{out}$ with a directed path $P_v$ on $k+1$ vertices.
Note that in the new instance, a directed cycle uses one vertex of $P_v$ if and only if the cycle uses all vertices of $P_v$ if and only if a corresponding cycle in the old instance uses $v$.
By an analogous to argument to the feedback arc set case, we are done.

\textbf{Uncapacitated facility location.} We have $\U = J$ in this case, where $J$ is the set of sites for potential facilities. The attacker selects facility sites and forbids the decision maker to build a facility there. 
To make a facility site $j \in J \setminus C$ invulnerable, we can simply delete the site and replace it with $k+1$ identical sites, i.e.\ sites which have the same facility opening cost and service cost functions as the original facility $j$. 
Clearly, this way the attacker can not stop one of the equivalent facilities to be opened. On the other hand, since the facilities are identical (and uncapacitated), 
the decision maker has no advantage from opening two identical copies of the same facility.
Hence the new instance is identical to the old instance, with the only difference that facility site $j$ is invulnerable.

\textbf{$p$-median, $p$-center.} The difference between the facility location problem and the $p$-center and $p$-median problem is that in the latter two, there are no facility opening costs, at most $p$ facilities are allowed to be opened, 
and the service costs in the $p$-center problem are calculated using a minimum, and in the $p$-median problem they are calculated using the sum. 
All of these differences do not affect the argument from above, i.e.\ one can still make a facility site invulnerable by creating $k+1$ identical facilities. Hence the same argument holds.

\textbf{Subset Sum.}
We have $\U = \fromto{1}{n}$ and are given numbers $a_1, \dots, a_n \in \N$ and a target value $T$. The question is whether there exists $S \subseteq U$ with $\sum_{i \in S} a_i = T$. 
Consider some index $i \in \U \setminus C$. In order to make the index $i$ invulnerable, the first idea is to copy the number $a_i$ a total amount of $k+1$ times. 
But there is a problem with this construction -- if we do this, then the same number $a_i$ could be picked multiple times, which is not allowed in the original instance.
We need an additional gadget to make sure that $a_i$ gets used at most once for each $i$. This can be done the following way: 
The new instance contains the following numbers: Choose some number $B > 2k+2$ as a basis. 
For each $i \in C$, it contains the single number $B^{n(k+1)}a_i$. 
For each $i \in \fromto{1}{n} \setminus C$, it contains the $k+1$ distinct numbers  $c_i^{(j)} := B^{n(k+1)}a_i + B^{(i-1)(k+1) + j}$ for $j = 0,\dots, k$ as well as the $k+1$ distinct numbers $d_i^{(j)} := \sum_{\ell = 0,\ell \neq j}^k B^{(i-1)(k+1) + \ell}$ for $j = 0,\dots, k$ and the $k+1$ distinct numbers $e_i^{(j)} := B^{(i-1)(k+1) + j}$ for $j = 0,\dots, k$. We call $d_i^{(j)}$ and $e_i^{(j)}$ the helper numbers.
The new instance contains a total of $|C| + 3(k+1)(n - |C|)$ numbers. The new target value is 
$$
T' := B^{n(k+1)}T \ + \sum_{i \in \fromto{1}{n} \setminus C} \quad \sum_{\ell = 0}^k B^{(i-1)(k+1) + \ell}.
$$
Note that this has the following effect: 
Consider the representation of all involved numbers in base $B$. Let us call the digits $0$ up to $n(k+1) - 1$ the lower positions. 
Note that in the lower positions there can never be any carry, since for every lower position, all involved numbers have either a zero or one in that position and less than $B$ numbers have a one in the same place.
Due to that fact, in the lower positions the target $T'$ is reached if and only if for every $i \in \fromto{1}{n} \setminus C$, the corresponding \enquote{bitmask} is filled out (by this, we mean the positions $(i-1)(k+1)$ up to $i(k+1) - 1$).
This is achieved if and only if for some $j \in \fromto{0}{k}$ both the values $c^{(j)}_i$ and $d^{(j)}_i$ or both the values $d^{(j)}_i$ and $e^{(j)}_i$ are picked. In particular, at most one of the $k+1$ values $c^{(j)}_i$ for $j=0,\dots,k$ are picked.
In the upper positions, the target $T'$ is reached if and only if the corresponding choice in the old instance meets the target $T$.

Consider an attack of $k+1$ numbers by the attacker. For each $i \in \fromto{1}{n} \setminus C$ it holds that there exists a $j$ such that both $c_i^{(j)}$ and $d_i^{(j)}$ are not attacked. Likewise there exists a $j'$ such that both $d_i^{(j')}$ and $e_i^{(j')}$ are not attacked. 
That means that if $i$ is an invulnerable index, then no matter which $k+1$ values of  $c_i^{(j)}$, $d_i^{(j)}$ and  $e_i^{(j)}$ are attacked, 
a correct solution of subset sum will take for some $j$ either both $c_i^{(j)}$ and $d_i^{(j)}$ (which corresponds to taking $a_i$ in the original instance) 
or take both $d_i^{(j)}$ and $e_i^{(j)}$ (which corresponds to not taking $a_i$ in the original instance).
It follows that it is possible to block the new instance by attacking $k+1$ values if and only if it is possible to block the old instance by attacking $k+1$ of the vulnerable values. 
This was to show.
Finally, if the old numbers $a_1, \dots, a_n$ are pairwise distinct, the new numbers are as well. Hence the interdiction problem for subset sum is $\Sigma^p_2$-complete, even if all involved numbers are distinct.

\textbf{Knapsack.} The knapsack problem can be seen as a more general version of the subset sum problem, by creating for each $i$ 
from the subset sum instance a knapsack item with both profit $p_i = a_i$ and weight $w_i = a_i$, and setting both the weight and profit threshold to $T$.
Hence the $\Sigma^p_2$-completeness of \textsc{Min Cardinality Interdiction-Knapsack} follows as a consequence of the $\Sigma^p_2$-completeness of \textsc{Min Cardinality Interdiction-Subset Sum}. This holds even if all the involved knapsack items are distinct.

% !TEX root = main.tex
\subsection{An Invulnerability Reduction for Hamiltonian Cycle}
The invulnerability gadget for Hamiltonian cycle is the most involved of all our constructions, 
hence we devote a subsection to it. 
The main result in this section is that the minimum cardinality interdiction problem is $\Sigma^p_2$-complete for the nominal problems of both directed and undirected Hamiltonian cycle and path, as well as the TSP.

We present our reduction for the case of undirected Hamiltonian cycle and then argue how it can be adapted to the other cases. The main idea is to consider as an intermediate step only 3-regular graphs $G = (V, E)$, and then for a subset $C \subseteq E$ show how $E \setminus C$ can be made invulnerable. To this end, consider the SSP problem

\begin{description}
    \item[]\textsc{3Reg Ham}\hfill\\
    \textbf{Instances:} Undirected, 3-regular Graph $G = (V, E)$\\
    \textbf{Universe:} $\U := E$.\\
    \textbf{Solution set:} The set of all Hamiltonian cycles in $G$.
\end{description}

Recall that it is shown in \cite{gruene2024completeness} that \textsc{Hamiltonian Cycle} is SSP-NP-complete. We now require the stronger statement

\begin{lemma}
\label{lem:3-reg-ham-ssp}
    \textsc{3Reg Ham} is SSP-NP-complete.
\end{lemma}
\begin{proof}
    Garey, Johnson \& Tarjan \cite{DBLP:journals/siamcomp/GareyJT76} give a reduction from \textsc{3Sat} to  \textsc{3Reg Ham}, such that for every variable $x_i$ in the \textsc{3Sat} instance the graph $G$ has two distinct edges $e(x_i)$ and $e(\overline x_i)$ (compare Figure 7 in \cite{DBLP:journals/siamcomp/GareyJT76}). Let $E' := \bigcup_i \set{e(x_i), e(\overline x_i)}$ be the set of all these edges. For some assignment $\alpha$ of the \textsc{3Sat} variables, we say that $\alpha$ corresponds to the edge set $E_\alpha$ defined by $\set{e(x_i) : \alpha(x_i) = 1} \cup \set{e(\overline x_i) : \alpha(x_i) = 0}$.
    Garey, Johnson \& Tarjan show that there is a bijection between satisfying assignments and edge sets $E'' \subseteq E'$ that can be subset of a Hamiltonian cycle. More formally: 1.) For every satisfying assignment $\alpha$, 
    if one considers the set $E_\alpha \subseteq E'$ of edges corresponding to that assignment, 
    there exists a Hamiltonian cycle $H$ extending $E_\alpha$, i.e.\ $H \cap E' = E_\alpha$. 
    2.) For every Hamiltonian cycle $H$, we have that $H \cap E'$ equals $E_\alpha$ for some satisfying assignment $\alpha$.
    In total, 1.) and 2.) together show that the reduction in \cite{DBLP:journals/siamcomp/GareyJT76} is an SSP-reduction. (By defining $f(x_i) := e(x_i), f(\overline x_i) := e(\overline x_i)$.)
\end{proof}

We remark that it follows from \cite{akiyama1980np,DBLP:journals/siamcomp/GareyJT76} by the same argument that the problem is even SSP-NP-complete if restricted to 3-regular, bipartite, planar, 2-connected graphs. However, for our arguments it suffices to consider 3-regular graphs.

Consider now an instance of \textsc{3Reg Ham}, i.e. a 3-regular undirected graph  $G = (V, E)$. Let $C \subseteq E$ be a subset of the edges and $k \in \N_0$ the attacker's budget. We call $C$ the vulnerable edges. Let $D := E(G) \setminus C$.
In the remainder of this section we describe and prove a construction how to make the edges in $D$ invulnerable.
We quickly sketch the main idea: To make an edge $e = ab$ invulnerable, we enlarge it by replacing it with a large clique $W'_{ab}$ making sure that $e$ can be traversed no matter which $k$ edges inside $W'_{ab}$  are attacked. 
We also blow up each vertex $a$ of the original graph into a clique $W_a$.
However, this introduces new vertices into the instance, and we need to make sure that a Hamiltonian cycle can always trivially visit all the new vertices.
At the same time however, it should still hold that a Hamiltonian cycle in the new graph should be able to enter and exit these new objects $W_a$ and $W'_{ab}$ at most once, since otherwise a corresponding cycle in the old graph $G$ would visit edges or vertices twice, which is of course forbidden.
We achieve this by associating to each edge $e = ab$ a star of edges $F_{ab}$ and argue that a Hamiltonian cycle can use (essentially) at most one edge of each star $F_{ab}$. 
Furthermore, we will show that the fact that $G$ is 3-regular implies that each clique $W_a$ can be traversed (essentially) only once.

We are ready to begin with the construction.
First, let the directed graph $\overrightarrow{G}$ result from $G$ by orienting its edges arbitrarily and $k$ be the budget of the attacker.
We construct an undirected graph $G' = (V', E')$ from $\overrightarrow{G}$ as follows: 
Let $n := |V(G)|$.
For each vertex $a \in V(\overrightarrow{G})$, let $d_a$ be the out-degree of $a$, and let $W_a$ be a set of $2d_a + 4k + 1$ vertices.
For each invulnerable edge $ab \in D$ in the old graph, let  $W'_{ab}$ be a set of $4k$ vertices.
The vertex set $V(G')$ of the new graph $G'$ is then defined by
\[ V(G') = \bigcup_{a \in V} W_a \cup \bigcup_{ab \in D} W'_{ab}.\]
\begin{figure}
    \centering
    \includegraphics[scale=1.0]{src/img/ham-cycle-invulnerability.pdf}
    \caption{Invulnerability gagdet for Hamiltonian cycle which makes the edge $ab$ invulnerable while the edge $ac$ remains vulnerable.}
    \label{fig:ham-cycle-invulnerability}
\end{figure}
We further partition $W_a$ into three disjoint parts $W_a = X_a \cup Y_a \cup \set{z_a}$ of size $|X_a| = 2d_a$ and $|Y_a| = 4k$ and $|\set{z_a}| = 1$.
We denote the vertices of $X_a$ by $x^{(a)}_1, \dots, x^{(a)}_{2d_a}$.
The edges of $G'$ are defined as follows:
First, we let $W_a$ be a clique for all $v \in V$.
Second, for each vertex $a \in V$ in $\overrightarrow{G}$, let $e_1, \dots, e_{d_a}$ be its outgoing edges.
For each $i = 1, \dots, d_a$, consider the $i$-th outgoing edge $e_i = (a, b)$ of $a$, where $b$ is the corresponding neighbor. 
If $e_i \in C$, i.e. $e_i$ is vulnerable, then $G'$ contains simply the single edge $x^{(a)}_{2i-1}z_b$. 
In the other case, i.e.\ $e_i \in D$ is invulnerable, then $G'$ contains an invulnerability gadget as depicted in \cref{fig:ham-cycle-invulnerability} induced on the vertices $\set{x^{(a)}_{2i-1}, x^{(a)}_{2i}} \cup W'_{ab} \cup \set{z_b}$.
The invulnerability gadget consists out of a clique on the vertex set $\set{x^{(a)}_{2i-1}, x^{(a)}_{2i}} \cup W'_{ab}$, together with all edges from the set $W'_{ab}$ to the vertex $z_b$, i.e.\ a star centered at $z_b$ that has $W'_{ab}$ as its leaves.
Let $F_{ab}$ denote this star. 
Finally, for all vulnerable edges $ab \in D$, we also define $F_{ab}$ to be the single edge $x^{(a)}_{2i-1}z_b$ that connects $W_a$ to $W_b$.
This can be interpreted as a trivial star centered at $z_b$ with only one leaf.
This completes the description of $G'$.

The overall idea of this construction is that the cliques of $W_a$ cannot be attacked because they have at least $k$ vertices.
Thus it is always possible to find a path visiting all vertices of $W_a$.
Additionally, a star $F_{ab}$ of size larger than $k$ makes the edge $ab \in E$ invulnerable because at most $k$ many of the edges can be attacked.
Thus there is always the possibility to travel over one edge of $F_{ab}$ which corresponds to using edge $ab$ in the original graph.
On the other hand, since every edge of the star is connected to the same vertex $z_b$, we have that the star $F_{ab}$ can be used (essentially) exactly once.
Thus only the stars of size one (which correspond to the vulnerable edges) are attackable.
We now have everything that we need to prove our main result of this section. 

\begin{theorem}
\label{thm:ham-cycle-interdiction}
Minimum cardinality interdiction for \textsc{Undirected Hamiltonian Cycle} is $\Sigma^p_2$-complete.
\end{theorem}
\begin{proof}
Due to \cite{gruene2024completeness}, and \cref{lem:3-reg-ham-ssp}, we have that \textsc{Comb. Interdiction-3Reg Ham} is $\Sigma^p_2$-complete.
We claim that the construction of $G'$ yields a correct reduction from \textsc{Comb. Interdiction-3Reg Ham} to \textsc{Min Cardinality Interdiction-HamCycle}.
Indeed, the following two \cref{lem:hamcycle-if,lem:hamcycle-only-if} show that yes-instances of one problem get transformed into yes-instances of the other problem. 
\end{proof}
We remark that the 3-regularity of the graph is not maintained by the reduction.
(Indeed, an argument similar to the arguments given later in \cref{sec:noMeta} shows that the interdiction problem for Hamiltonian cycle restriced to only 3-regular graphs is likely not $\Sigma^p_2$-complete).

\begin{lemma}
\label{lem:hamcycle-if}
    If there exists $B' \subseteq E'$ of size $|B'| \leq k$, such that $G' - B'$ has no Hamiltonian cycle, then there is $B \subseteq C$ of size $|B| \leq k$ such that $G - B$ has no Hamiltonian cycle.
\end{lemma}
\begin{proof}
    Proof by contraposition. Assume that for all $B \subseteq C$ with $|B| \leq k$ the graph $G - B$ has a Hamiltonian cycle $H$.
    Given some $B' \subseteq E'$ with $|B'| \leq k$, we have to show that the graph $G' - B'$ has a Hamiltonian cycle. Let $B \subseteq C$ be the set of vulnerable edges in $G$ whose copies in $G'$ are attacked by $B'$ (i.e. $B = \set{ab \in C : F_{ab} \in B'}$). 
    Since $B \subseteq C$ and $|B| \leq k$, by assumption $G - B$ has a Hamiltonian cycle $H$. We want to modify $H$ to a Hamiltonian cycle of $H'$ of $G' - B'$. 
    The basic idea is to follow globally the same route as $H$. However, we have to pay attention, because we are not allowed to use edges from $B'$.
    For each vertex in $G'$ call it \emph{attacked}, if at least one of its incident edges are attacked by $B'$, and call it \emph{free} otherwise. 
    Note that since $|B'| \leq k$ and $|Y_a| = 4k$ and $|W'_{ab}| = 4k$ for $a \in V, ab \in E$, the vertex sets $Y_a$ and $W'_{ab}$ have at least $2k$ free vertices. Free vertices are good for the following reason: 
    Whenever we plan to go from some vertex $u$ to $v$ in $G'$, but we cannot because $uv \in B'$ was attacked, then we can instead choose any free vertex $f$ and go the route $u,f,v$ instead.
    Now the plan is that $H'$ will roughly employ the following strategy: Follow globally the same path in $G'$ like $H$ does in $G$. 
    Whenever $H'$ enters some new set $W_a$ for the first time, then we visit all the sets $W'_{ab}$ for all out-neighbors $b$ of $a$ in $\overrightarrow{G}$.
    Note that for such $b$, the set $W'_{ab}$ has two adjacent vertices with $W_a$ (we use these two vertices to enter and leave), and we collect all the vertices of $W'_{ab}$. 
    Here, we prioritize to visit first the attacked vertices of $W'_{ab}$ and then the remaining vertices of $W'_{ab}$. 
    After that, we collect all remaining vertices of $W_a$ (again prioritizing the attacked vertices first) before leaving $W_a$. (If the path on which we are leaving $W_a$ corresponds to an invulnerable edge $ab$ in $G$, we also collect all of $W'_{ab}$ in the process of leaving $W_a$.)

    Note that this plan might at first not be feasible, because it requires going over some edge $e' \in B'$. However note that, since $H$ does not use any edge of $B$, for every such edge $e'$ there are always at least $2k$ free vertices that are adjacent to both endpoints of $e'$.
    Hence it is possible to \enquote{repair} such an edge $e'$ by rerouting over some free vertex instead (and later skip over this free vertex). 
    Since there are at most $k$ defects, and there are at least $2k$ free vertices available at the end of traversing every set $W_a$ or $W'_{ab}$, all defects can be repaired. 
    Hence we can modify $H'$ to be a Hamiltonian cycle of $G' - B'$, which was to show.
\end{proof}

\begin{lemma}
\label{lem:hamcycle-only-if}
    If there exists $B \subseteq C$ of size $|B| \leq k$, such that $G - B$ has no Hamiltonian cycle, then there is $B' \subseteq E'$ of size $|B'| \leq k$ such that $G' - B'$ has no Hamiltonian cycle.
\end{lemma}
\begin{proof}
    Proof by contraposition. Assume that for all $B' \subseteq E'$ of size $|B'| \leq k$ the graph $G' - B'$ has a Hamiltonian cycle. 
    Given some $B \subseteq C$ with $|B| \leq k$, we have to show that the graph $G - B$ has a Hamiltonian cycle. 
    Let $B'$ be the trivial stars in $G'$ corresponding to the edges in $B$ (i.e.\ $B' = \set{F_{ab} : ab \in B}$). 
    Since $|B'| \leq k$, by assumption there is a Hamiltonian cycle $H'$ in $G' - B'$.  
    Consider the set $F := \bigcup_{ab \in E}E(F_{ab})$, i.e.\ the union of the edge sets of all the stars, trivial or not.
    We claim that w.l.o.g.\ we can assume that $|H' \cap F_{ab}| \leq 1$ for all $ab \in E$.
    Indeed, the graph $G' - F$ consists out of multiple connected components. 
    Each of these components contains exactly one set of the form $W_a$, and is incident to exactly three sets of the form $F_e$ in $G'$ (where $e$ is an edge that is either incoming to or outgoing from $a$ in $\overrightarrow{G}$).
    Suppose for some $F_{ab}$ we have $|H' \cap F_{ab}| \geq 2$.
    Since $F_{ab}$ is a star connected to a single vertex $z_b$, we have $|H' \cap F_{ab}| = 2$.
    Consider the edge $ab$ such that $F_{ab}$ connects the vertex $z_b$ with $W'_{ab}$. 
    By the observation about $G' - F$, the following is true about $H'$: 
    It enters $W'_{ab}$ in one of the two vertices attached to $X_a$, then traverses exactly all of $W'_{ab} \cup \set{z_b}$, 
    then leaves through the other of the two vertices attached to $X_a$, and at a later point returns to collect all vertices of $X_b \setminus \set{z_b}$.
    However, by the same observation as in \cref{lem:hamcycle-if}, if we define a free vertex to be a vertex not adjacent to any edge in $B'$, then both $W'_{ab}$ and $W_b$ have $2k$ free vertices.
    Hence we can modify $H'$ such that $H' \cap F_{ab} = \emptyset$.
    We thus assume that $|H' \cap F_{ab}| \leq 1$ for all $ab \in E$.
    Consider again the graph $G' - F$.
    Since each of its component is adjacent to three sets $F_e$ and $|H' \cap F_e| \leq 1$, we conclude that $H'$ uses exactly two of these three sets $F_e$.
    But this implies that $H'$ enters and exits each of the components of $G'- F$ only once and collects all of its vertices in the process.
    This implies that $H'$ globally follows the same path as some Hamiltonian cycle $H$ of $G$.
    Since $H' \subseteq G' - B'$, we conclude $H \subseteq G - B$.
    This was to show.
\end{proof}


These two lemmas together prove \cref{thm:ham-cycle-interdiction}.
We would now like to prove $\Sigma^p_2$-completeness also for Hamiltonian cycle interdiction of directed graphs.
Note that this does not follow from a trivial argument: 
Even though one can transform an undirected graph into a directed one, by substituting every undirected edge $uv$ by two directed edges $(u,v), (v,u)$, there is a problem: In the new setting the interdictor needs two attacks to separate $u,v$, while in the old setting the attacker only needs one. 

Still, the above proof can be adapted to the case of directed Hamiltonian cycle the following way: 
We start with \cite{plesn1979np}, which provides a SSP reduction to prove that the Hamiltonian cycle problem is NP-complete even in directed graphs $G$ such that $\text{indegree}(v) + \text{outdegree}(v) \leq 3$ for every vertex $v$, and such that for all pairs $u,v$ of $G$ at most one of the two edges $(u,v)$ and $(v,u)$ is present. 
Given a directed graph $G$, we then repeat the same construction as before, 
with the difference that we can start directly with the directed graph $G$ instead of obtaining an orientation $\overrightarrow{G}$ first.
This way, we can obtain an undirected graph $G'$ in the same way as before. 
In a final step, we turn $G'$ into a directed graph by substituting every undirected edge $uv$ by a pair of two edges $(u,v), (v,u)$. We perform this substitution for every edge of $G'$ with the exception of the edges that are part of some star $F_{ab}$.
Instead, for each star $F_{ab}$, we orient the edges of $F_{ab}$ the same way as the original directed edge of $G$ between $a,b$.
It can be shown that all the arguments from the above construction still hold. 
Hence the minimum cardinality interdiction problem is $\Sigma^p_2$-complete also for directed graphs.


If one is interested in Hamiltonian paths instead of cycles, a similar modification is possible.
Inspecting the proof of \cite{DBLP:journals/siamcomp/GareyJT76} (of \cite{plesn1979np}, respectively) more closely, we find that in both constructions the graph $G$ contains some edge $e = uv$ (some edge $e = (u,v)$, respectively) such that every Hamiltonian cycle uses $e$. 
We can delete $e$ and identify the vertices $s,t$ with the endpoints of $e$.
Then a Hamiltonian $s$-$t$-path in the new graph corresponds to a Hamiltonian cycle in the old graph and vice versa.
Note that this does not increase the degree of the graph.
The rest of the proof proceeds in the same manner, both in the undirected and directed case.
Finally, the proof can also easily be adapted to the TSP by a standard reduction of undirected Hamiltonian cycle to the TSP 
(a graph $G$ is transformed into a TSP instance on the complete graph where the costs obey $c(uv) = 1$ if $ev \in E(G)$ and $c(uv) = n+1$ if $uv \not\in E(G)$).
In conclusion, we have proven that the minimum cardinality interdiction problem is $\Sigma^p_2$-complete for the directed/undirected Hamiltonian path/cycle problem and the TSP. 

% !TEX root = main.tex
\section{Cases where the meta-theorem does not apply}\label{sec:noMeta}

It would be nice to establish a meta-theorem providing $\Sigma^p_2$-completeness of the minimum cardinality interdiction version of all nominal problems, which are SSP-NP-complete, instead of only those problems that admit an additional function $g$ with properties as stated in \Cref{thm:meta-theorem}.
However, we show in this section that this is not possible.
More precisely, we provide a lemma that guarantees that the minimum cardinality version of a problem in SSP-NP is in coNP.
Therefore, under the usual complexity-theoretic assumption NP $\neq \Sigma^p_2$, the interdiction problem is not $\Sigma^p_2$-complete.

In order to provide an intuition under which circumstances a minimum cardinality interdiction problem resides in the class coNP, we examine the vertex cover problem.
In a vertex cover, every edge $uv$ needs to be covered by at least one of the two incident vertices $u$ and $v$.
This, however, gives the attacker the opportunity to attack both $u$ and $v$ such that the edge $uv$ can never be covered.
Therefore, an attacker budget of at least $2$ results in a clear Yes-instance.
On the other hand, if the attacker budget if at most $1$, we can provide a certificate for No-instances.
We can summarize this observation in the following lemma.

\begin{lemma}\label{lem:minCardInCoNP}
    Let $\Pi = (\I, \U, \sol)$ be an SSP problem.
    If in each instance $I \in \I$ there is a subset $U' \subseteq \U(I)$ of constant size, i.e. $|U'| = O(1)$, such that for $U' \cap S \neq \emptyset$ for all $S \in \sol(I)$, then \textsc{Min Cardinality Interdiction-$\Pi$} is contained in coNP.
\end{lemma}
\begin{proof}
    Let $k$ be the interdiction budget.
    If $|U'| \leq k$, then the interdictor is able to block the whole set $U'$.
    By definition of $U'$, there is no solution $S \in \sol(I)$ such that $U' \cap S \neq \emptyset$ and thus the interdictor has a winning strategy.
    If on the other hand $k < |U'| = O(1)$, then there is a polynomially sized certificate encoding a winning strategy of the defender, i.e. a certificate for a No-instance of the problem.
    For this, we first encode the ${|\U(I)| \choose k} = |\U(I)|^{O(1)}$ possible blockers $B' \subseteq \U(I)$ and then the solution $S \in \mathcal S(I)$ such that $S \cap B' \neq \emptyset$ for all $B' \subseteq \U(I)$.
    It is possible to efficiently verify the solution by checking whether there is a solution $S \in \mathcal S(I)$ such that $S \cap B' \neq \emptyset$ for all $B' \subseteq \U(I)$ holds because the nominal problem $\Pi$ is in NP.
    It follows that the problem lies in coNP.
\end{proof}

Consider the different variants of interdiction problems introduced in \cref{sec:different-variants-of-interdiction}.
Since they are more general, \cref{lem:minCardInCoNP} does not immediately imply that those variants are contained in coNP.
However, if for each instance the stronger condition $U' \cap F \neq \emptyset$ for all feasible solutions $F \in \F(I)$ and for some constant size set $U' \subseteq \U(I)$ holds, then the \emph{full decision variant of interdiction} and the \emph{most vital element problem} are contained in coNP.
Besides the containment in coNP, we can also derive the following corollary pinpointing the complexity of minimum cardinality interdiction problems whose nominal problem is in SSP-NP.

\begin{corollary}
    Let $\Pi = (\I, \U, \sol)$ be an SSP-NP-complete problem.
    If in each instance $I \in \I$ there is a subset $U' \subseteq \U(I)$ of constant size, i.e. $|U'| = O(1)$, such that for $U' \cap S \neq \emptyset$ for all $S \in \sol(I)$, then \textsc{Min Cardinality Interdiction-$\Pi$} is coNP-complete.
\end{corollary}
\begin{proof}
    There is a reduction by restriction:
    Setting the interdiction budget $k = 0$ results in the corresponding co-problem co$\Pi$ of the nominal problem $\Pi$.
\end{proof}

\subsection{Applying the Lemma to Various Problems}

In this section, we apply \cref{lem:minCardInCoNP} to the problems mentioned earlier in this paper.
Some of the problems are affected in their original general form, e.g. vertex cover or satisfiability, while for others the lemma can be applied on a restricted version such as independent set on graphs with bounded minimum degree.
For this, we shortly describe the problem and then give the argument on how the lemma is applicable.

\textbf{Vertex Cover.}
An instance of the vertex cover interdiction problem consists of a graph $G$ and numbers $t,k \in \N_0$.
The question is if the attacker can find a set $B \subseteq V(G)$ with $|B| \leq k$ such that $B \cap S \neq \emptyset$ for every vertex cover $S$ of size at most $t$.
Now, observe that if $k \geq 2$ (and the graph is non-empty), the attacker can easily find such a set $B$ by selecting two adjacent vertices.
Thus, \cref{lem:minCardInCoNP} applies by defining $U' = \{u, v\}$ for some edge $uv \in E(G)$.
Observe that this not only destroys the solutions $S \in \sol(I)$ but also all feasible solutions $F \in \F(I)$.
Thus the minimum cardinality interdiction version, the full decision variant of interdiction and the most vital elements problem of vertex cover are coNP-complete.

\textbf{Satisfiability.}
An instance of the satisfiability interdiction problem consists out of a formula in CNF over the variables $X = \fromto{x_1}{x_n}$, with the literal set as universe, i.e. $\U = X \cup \overline X$, and interdiction budget $k$.
A similar issue as in vertex cover interdiction arises here:
If $k \geq 2$, the interdictor can just choose for some $i \in \fromto{1}{n}$ to attack both literals $x_i, \overline x_i$.
Every satisfying assignment (of non-trivial instances) contains either $x_i$ or $\overline{x_i}$, hence this is a successful attack.
Thus, \cref{lem:minCardInCoNP} applies by defining $U' = \{x, \overline x\}$ for some literal pair $x, \overline x \in \U$.
Again, this also destroys all feasible solutions $F \in \F(I)$.
Thus the minimum cardinality interdiction version, the full decision variant of interdiction and the most vital elements problem of satisfiability are coNP-complete.

\textbf{Independent Set on graphs with bounded minimum degree.}
An instance of the independent set interdiction problem consists of a graph $G =(V,E)$ with universe $U = V$, a threshold $t$ and an interdiction budget $k$.
The question of the independent set  problem is if there is a set $I \subseteq V$ such that all vertices in $I$ do not share an edge.
We now take the vertex $d$ of bounded degree into consideration.
If the attacker attacks the closed neighborhood $N[d]$ of $d$, all (optimal) solutions $S \in \mathcal S$ can be interdicted and thus \Cref{lem:minCardInCoNP} is applicable.
Thus minimum cardinality interdiction independent set on graphs with bounded minimum degree is coNP-complete.
In contrast to the other problems, this statement is not true for general feasible solutions $F \in \mathcal F(I)$. Hence we do not obtain a result for the variants from \Cref{sec:different-variants-of-interdiction}.

\textbf{Dominating Set on graphs with bounded minimum degree.}
An instance of the dominating set interdiction problem consists of a graph $G =(V,E)$ with universe $U = V$, a threshold $t$ and an interdiction budget $k$.
The question of the dominating set problem is if there is a set $D \subseteq V$ of size at most $t$ such that $D$ dominates all vertices of vertex set $V$. In other words, the union of the neighborhoods of the vertices in $D$ is the vertex set $V$, i.e. $\bigcup_{v \in D} N[v] = V$.
Again we consider a vertex $d$ of bounded degree.
Then, we can define the set of constant size to be $U' = N[d]$.
All feasible solutions $F \in \mathcal F$ have to include some vertex from $U'$ (otherwise $d$ would not be dominated).
Thus \Cref{lem:minCardInCoNP} is applicable to dominating set.
Accordingly, the minimum cardinality interdiction version, the full decision variant of interdiction and the most vital elements problem of dominating set on graphs with bounded minimum degree are coNP-complete.

\textbf{Hitting Set with bounded minimum set size.}
An instance of hitting set interdiction consists of a ground set $\{1, \ldots, n\}$ and $m$ sets $S_j \subseteq \{1,\ldots,n\}$ as well as a threshold $t$ and an interdiction budget $k$.
The universe is defined by $\U = \fromto{1}{n}$.
The question of the hitting set problem is whether there is a hitting set $H \subseteq \fromto{1}{n}$ of size at most $t$ for the sets $S_j$, that is, $H \cap S_j \neq \emptyset$ for $1 \leq j \leq m$.
We can apply \Cref{lem:minCardInCoNP} by defining $U'$ to be the set of constant size $|S_c| = O(1)$.
Then, the attacker is able to block the entire set $S_c$ such that it is not hittable, which interdicts all feasible solutions $F \in \mathcal F$.
Therefore the minimum cardinality interdiction version, the full decision variant of interdiction and the most vital elements problem of hitting set with bounded minimum set size are coNP-complete.

\textbf{Set Cover with bounded minimum coverage.}
An instance of the set cover interdiction problem consists of sets $S_i \subseteq \{1, \ldots, m\}$ for $1 \leq i \leq n$, a threshold $t$ and the an interdiction budget $k$.
The universe is defined as the sets $S_i$, $1 \leq i \leq n$.
The question of the set cover problem is whether there is selection $S \subseteq \{S_1, \ldots, S_n\}$ of size at most $k$ such that $\bigcup_{s \in S} s = \{1, \ldots, m\}$.
If there is an element $e \in \{1, \ldots, m\}$ of bounded coverage, i.e. there is a constant number of $S_i$, $1 \leq i \leq n$, with $e \in S_i$, then the attacker can attack all of these sets $S_i$.
Thus, we can apply \Cref{lem:minCardInCoNP} by choosing $U' = \{S_i \mid e \in S_i\}$ and all feasible solutions $F \in \mathcal F$ are blockable.
Accordingly, the minimum cardinality interdiction version, the full decision variant of interdiction and the most vital elements problem of set cover with bounded minimum coverage are coNP-complete.

\textbf{Steiner Tree on graphs with bounded minimum degree of terminal vertices.}
An instance of the Steiner tree interdiction problem consists of a graph $G= (S \cup T, E)$ of Steiner vertices $S$ and terminals $T$, edge weights $c: E \rightarrow \mathbb N$, a threshold $t$ and a interdiction budget $k$.
The universe is the edge set $\U = E$.
The question of the Steiner tree problem is if there is a tree $E' \subseteq E$ of weight $c(E') \leq t$ such that all terminal vertices $T$ are connected by $E'$.
If there is a terminal vertex $d \in T$ of bounded degree, then all incident edges build up a set $U' = \{dv \in E\}$ on which we can apply \Cref{lem:minCardInCoNP}.
This blocks all feasible solutions $F \in \mathcal F$.
Therefore, the minimum cardinality interdiction version, the full decision variant of interdiction and the most vital elements problem of Steiner tree on graphs with bounded minimum degree of terminal vertices are coNP-complete.

\textbf{Two Vertex-Disjoint Path on graphs with bounded degree.}
An instance of the two vertex-disjoint path interdiction problem consists of a directed graph $G=(V,A)$, vertices $s_1, s_2, t_1, t_2 \in V$ and interdiction budget $k$.
The universe is the arc set $\U = A$.
The question of the two vertex-disjoint path is if there are two paths $P_1, P_2 \subseteq A$ such that $P_i$ starts at $s_i$ and ends at $t_i$ and both paths $P_1$ and $P_2$ do not share a vertex.
If the the graph has bounded degree, we can choose any of the vertices that have to be included in on of the paths, e.g. $s_1$, and include all the incident arcs in $U' = \{(s_1, v) \in A\}$ such that we can apply \Cref{lem:minCardInCoNP}.
This blocks all feasible solutions $F \in \mathcal F$.
Accordingly, the minimum cardinality interdiction version, the full decision variant of interdiction and the most vital elements problem of two vertex-disjoint path on graphs with bounded degree are coNP-complete.

\textbf{Feedback Vertex Set on graphs with bounded girth.}
An instance of the feedback vertex set interdiction problem consists of a directed graph $G=(V,A)$, a threshold $t$ and interdiction budget $k$.
The universe is the vertex set $\U = V$.
The question of feedback vertex set is if there is a set $V' \subseteq V$ such that the graph is cycle free.
Accordingly, if the graph has bounded girth, there is a cycle of bounded length, which the attacker can attack or in other words, the cycle cannot be deleted by the defender by choosing a corresponding vertex to be in the feedback vertex set.
Thus all feasible solutions $F \in \mathcal F$ are blockable by applying \Cref{lem:minCardInCoNP} with $U' = \{v \in V \mid v \text{ is part of the smallest cycle in } G\}$.
Therefore, the minimum cardinality interdiction version, the full decision variant of interdiction and the most vital elements problem of feedback vertex set on graphs with bounded girth are coNP-complete.

\textbf{Feedback Arc Set on graphs with bounded girth.}
An instance of the feedback arc set interdiction problem consists of a directed graph $G=(V,A)$, a threshold $t$ and interdiction budget $k$.
The universe is the arc set $\U = A$.
The question of feedback arc set is if there is an arc set $A' \subseteq A$ such that the graph is acyclic.
We can use the same argument as in feedback vertex set.
That is, the attacker can choose the arcs of the smallest cycle in $G$.
Thus all feasible solutions $F \in \mathcal F$ are blockable by applying \Cref{lem:minCardInCoNP} with $U' = \{a \in A \mid a \text{ is part of the smallest cycle in } G\}$.
Therefore, the minimum cardinality interdiction version, the full decision variant of interdiction and the most vital elements problem of feedback arc set on graphs with bounded girth are coNP-complete.


\textbf{Uncapacitated Facility Location, p-Center, p-Median with bounded minimum customer coverage.}
An instance of the minimum cardinality interdiction version of these three problems consists of a set of potential facilities $F$ and a set of clients $C$ together with a cost function on the facilities $f: F \rightarrow \mathbb N$ and a service cost function $c: F \times C \rightarrow \mathbb N$ as well as a threshold $t$ and an interdiction budget $k$.
The universe is the facility set $\U = F$ and it is asked for a set of facilities $F' \subseteq F$ not exceeding the cost threshold $t$.
If the coverage of one customer is bounded, i.e. there is a bounded number of potential facilities that are able to serve the customer, the attacker is able to block all of these.
Thus we can define $U'$ as the set of facilities that are able to serve the customer of bounded coverage such that all feasible solutions $F \in \mathcal F$ can be interdicted.
Therefore, we can apply \Cref{lem:minCardInCoNP} and the minimum cardinality interdiction version, the full decision variant of interdiction and the most vital elements problem of these three facility locations problems with bounded minimum customer coverage are coNP-complete.

\textbf{Hamiltonian path/cycle (directed/undirected), TSP on graphs with bounded minimum degree.}
An instance of the minimum cardinality interdiction version of these problems consists of a graph $G=(V,E)$ (respectively $G=(V,A)$ in the directed case) and an interdiction budget $k$.
The universe is the set of edges $\U = E$ (respectively the set of arcs $\U = A$).
The question is whether there is a Hamiltonian path or cycle in $G$, i.e. a path/cycle that visits every vertex exactly once.
Because there is a vertex $d$ of bounded degree which has to be visited, we can define the set of constant size $U' = \{dv \in E\}$ (respectively $U' = \{(d,v),(v,d) \in A\}$).
If the set $U'$ is blocked it is not possible to visit the vertex, thus all feasible solutions $F \in \mathcal F$ can be interdicted.
Therefore, we can apply \Cref{lem:minCardInCoNP} and the minimum cardinality interdiction version, the full decision variant of interdiction and the most vital elements problem of these five Hamiltonian problems on graphs with bounded minimum degree are coNP-complete.

\subsection{Satisfiability with Universe over the Variables}

In the previous subsection we explained why minimum cardinality interdiction-\textsc{Sat} is contained in coNP, hence likely not $\Sigma^p_2$-complete.
Note that this is a consequence of our choice of definition of \textsc{Satisfiability}, where we explicitly defined the universe to be the literal set $L = X \cup \overline X$.
As a consequence, the interdictor may attack $X \cup \overline X$. 
\begin{samepage}
    \begin{mdframed}
    	\begin{description}
        \item[]\textsc{Satisfiability ($\U = L$)}\hfill\\
        \textbf{Instances:} Literal Set $L = \fromto{x_1}{x_n} \cup \fromto{\overline x_1}{\overline x_n}$, Clauses $C \subseteq \powerset{L}$\\
        \textbf{Universe:} $L =: \U$.\\
        \textbf{Solution set:} The set of all sets $L' \subseteq \U$ such that for all $i \in \fromto{1}{n}$ we have $|L' \cap \set{\ell_i, \overline \ell_i}| = 1$, and such that $|L' \cap c_j| \geq 1$ for all $c_j \in C$.
    	\end{description}
    \end{mdframed}
\end{samepage}

An interesting behavior occurs, when we consider the following alternative version \textsc{Satisfiability ($\U = X$)}. 
\begin{samepage}
    \begin{mdframed}
    	\begin{description}
        \item[]\textsc{Satisfiability  ($\U = X$)}\hfill\\
        \textbf{Instances:} Variable Set $X = \fromto{x_1}{x_n}$, Clauses $C \subseteq 2^{X \cup \overline X}$ \\
        \textbf{Universe:} $X =: \U$.\\
        \textbf{Solution set:} The set of all sets $X' \subseteq \U$ such that the assignment $\alpha: X \rightarrow \{0,1\}$ with $\alpha(x) = 1 \leftrightarrow x \in X'$ satisfies all clauses in $C$.
        \end{description}
    \end{mdframed}
\end{samepage}
Here the universe is only the variable set $X$, so in the interdiction version, the interdictor may only attack $X$, i.e.\ the interdictor may target individual variables and enforce that they must be set to \emph{false}. 
We show now that in contrast to the variant, where the universe is the literal set, in this new variant the interdiction problem is $\Sigma^p_2$-complete again. 
Since the problem \textsc{Satisfiability ($\U = X$)} is not part of the original problem set of \cite{gruene2024completeness}, we perform this proof in two steps.
\begin{lemma}
    \textsc{Satisfiability ($\U = X$)} is SSP-NP-complete, even when all clauses are restricted to length at most three.
\end{lemma}
\begin{proof}
    We provide an SSP reduction from the SSP-NP-complete problem \textsc{Satisfiability ($\U = L$)} to \textsc{Satisfiability ($\U = X$)}. 
    Consider an instance of \textsc{Satisfiability ($\U = L$)} given by a formula $\varphi$ with $n$ variables $X = \fromto{x_1}{x_n}$ and universe/literal set $\U = L = X \cup \overline{X}$.
    \textsc{Satisfiability ($\U = L$)} is SSP-NP-complete even when all clauses are restricted to length three, so let us w.l.o.g.\ assume that property.
    We have to show how to embed this universe into the universe $\U'$ of some corresponding \textsc{Satisfiability ($\U = X$)} instance $\varphi'$, where only positive literals are allowed in $\U'$.
    This can be done the following way:
    We introduce $2n$ new variables $X' := \fromto{x^t_1}{x^t_n} \cup \fromto{x^f_1}{x^f_n}$.
    The universe $\U' := X'$ consists out of the $2n$ corresponding positive literals $X'$.
    The new formula $\varphi'$ is defined from $\varphi$ in two steps.
    First a substitution process takes place: 
    For each $i=1,\dots,n$, the positive literal $x_i$ is replaced by the positive literal $x^t_i$ and each negative literal $\overline x_i$ is replaced by the positive literal $x_i^f$.
    In a second step, the clauses $(x^t_i \lor \overline x^f_i) \land (\overline x^t_i \lor x^f_i)$ (note that these are equivalent to $x_i^t \oplus x_i^f$) are added to $\varphi'$.
    Formally,
    \[
        \varphi' = \text{substitute}(\varphi) \land \bigwedge_{i=1}^n (x_i^t \lor x_i^f)\land (\overline{x}_i^t \lor \overline{x}^f_i).
    \]
    The SSP reduction is completed by specifying the embedding function $f : \U \to \U'$ via $f(x_i) := x_i^t$ and $f(\overline x_i) := x_i^f$.
    Clearly all clauses of $\varphi'$ have length at most three.
    Note that this reduction is a correct reduction, i.e.\ it transforms yes-instances into yes-instances and no-instances into no-instances, because the added constraints make sure that exactly one of $x_i^t$ and $x_i^f$ is true.
    Furthermore, it has the SSP property:
    For every solution $S \subseteq \U$ of \textsc{Satisfiability ($\U = L$)}, the \enquote{translated} set $f(S) \subseteq \U'$ is a solution of \textsc{Satisfiability ($\U = X$)}.
    Furthermore, for every solution $S \subseteq \U'$ of \textsc{Satisfiability ($\U = X$)}, the set $f^{-1}(S)\subseteq \U$ is a solution of \textsc{Satisfiability ($\U = L$)}.
    Accordingly, we have a correct SSP reduction (where the SSP mapping $f$ is even bijective due to $f(\U) = \U'$). 
\end{proof}

\begin{theorem}
    \textsc{Min Cardinality Interdiction-Satisfiability ($\U = X$)} is $\Sigma^p_2$-complete.
\end{theorem}
\begin{proof}
    By the previous lemma, \textsc{Satisfiability ($\U = X$)} is SSP-NP-complete, even if all clauses are restricted to length three.
    Due to \cite{gruene2024completeness}, the problem \textsc{Comb. Interdiction-Satisfiability ($\U = X$)} is $\Sigma^p_2$-complete, even if all clauses are restricted to length three.
    We provide a reduction from the latter problem in terms of an invulnerability gadget analogous to the gadgets presented in \cref{sec:invulnerability-gadgets}. 
    For this, consider an instance of \textsc{Satisfiability ($\U = X$)} with formula $\varphi$ in CNF and every clause of length three, together with the universe $\U = \fromto{x_1}{x_n}$, a set $C \subseteq \U$ of vulnerable literals, and interdiction budget $k \in \N_0$.
    For every variable $x_i \in \U \setminus C$, we explain how to make $x_i$ invulnerable.
    We introduce $k+1$ new variables $x^{(1)}_i, \dots x^{(k+1)}_i$.
    Our goal is to establish the equivalence
    \[
        x_i \equiv x^{(1)}_i \lor \dots \lor x^{(k+1)}_i.
    \]
    We can achieve this through means of the following substitution process starting from formula $\varphi$: 
    Every occurrence of $x_i$ in the formula gets substituted by $x^{(1)}_i \lor \dots \lor x^{(k+1)}_i$. 
    Every occurrence of $\overline x_i$ gets substituted (by De Morgan's law) by $(\overline x^{(1)}_i \land \dots \land \overline x^{(k+1)}_i)$.
    Note that this has two effects: 
    First, the length of a clause may now exceed 3.
    Secondly, the formula is not in CNF anymore. 
    Note however that we can use the distributive law to expand every clause that is not in CNF. 
    Since before each clause before had a length of at most three, this results in a blow-up of the instance size of a factor at most $(k+1)^3$, i.e.\ at most a polynomial factor.
    Let $\varphi'$ be the resulting formula. 
    We can see that there is an equivalence of the satisfying assignments of $\varphi$ and $\varphi'$, in the sense that $x_i$ is true in $\varphi$ if and only if $x^{(1)}_i \lor \dots \lor x^{(k+1)}_i$ is true in $\varphi'$ (for all invulnerable $x_i$). 
    However, since the interdiction budget is only $k$, the interdictor can never enforce $x^{(1)}_i \lor \dots \lor x^{(k+1)}_i$ to be false for all invulnerable variables.
    This shows that \textsc{Comb. Interdiction-Satisfiability ($\U = X$)} reduces to \textsc{Min. Cardinality Interdiction-Satisfiability ($\U = X$)}, hence proving its $\Sigma^p_2$-completeness.
\end{proof}

Note that the reasoning presented in this proof was slightly different from \cref{thm:meta-theorem}, since we start with a formula where every clause has length three, but do not preserve this property during the proof.
Hence $\Sigma^p_2$-completeness is only shown in the case where clauses can have unrestricted length.

We can use an argument similar to \Cref{lem:minCardInCoNP} to show the coNP-completeness of the minimum cardinality interdiction version, the full decision variant of interdiction and the most vital elements problem of {\sc $b$-Satisfiability ($\U = X$)}, i.e. with clauses of length bounded by $b$.
Indeed, it is easy to see that the interdiction problem of \textsc{Satisfiability ($\U = X$)} where every clause has length three is coNP-complete:
If $k \geq 3$ holds for the interdiction budget, the attacker distinguishes two cases:
If there is a clause with three positive literals, the attacker blocks all of them and immediately wins. 
In the other case, every clause has at least one negative literal.
Then the attacker can never win, since the defender can set every variable to false, which is a satisfying assignment that can never be blocked.
By an analogous argument, we can see that for any $t = O(1)$, the interdiction problem of \textsc{Satisfiability ($\U = X$)} with clauses restricted to length $t$ is coNP-complete.

Finally, we remark that slightly different variants of interdiction-3-Sat have been shown to be $\Sigma^p_2$-complete. In these variants, the interdictor does not have access to all variables (see \cite[Sec. 4.2]{gruene2024completeness} or \cite[Thm. 1]{jackiewicz2024computational}).




















% It would be nice to have our main theorem, \cref{thm:meta-theorem}, for all problems from the class SSP-NPc, instead of only those who admit an additional function $g$ with properties as stated. However, we show in this section that this is not possible.
% Concretely, we show in this section that for the following SSP-NP-complete problems, their interdiction version $\textsc{Min. Card. Interdiction-$\Pi$}$ is contained in the class coNP: Vertex cover, satisfiability, Partition/two-machine scheduling. 
% Therefore, under the usual complexity-theoretic assumption $NP \neq \Sigma^p_2$, the interdiction problem is not $\Sigma^p_2$-complete.

% In the case of the satisfiability problem, it turns out that certain variants of it remain $\Sigma^p_2$-complete, while others do not. We discuss these details below. 
% In the following, it is helpful to interpret interdiction as a game between attacker and defender.

% \textbf{Vertex cover.} An instance of the vertex cover interdiction problem consists out of a graph $G$ and numbers $t,k \in \N_0$.
% The question is if the attacker can find a set $B \subseteq V(G)$ with $|B| \leq k$ such that $B \cap C \neq \emptyset$ for every vertex cover $C$ of size at most $t$.
% Now, observe that if $k \geq 2$ (and the graph is non-empty), the attacker can easily find such a set $B$ by selecting two adjacent vertices. 
% Every vertex cover has a non-trivial intersection with $B$. Hence every instance with $k \geq 2$ is a yes-instance.
% On the other hand, if $k \leq 1$, if the defender has a winning strategy, then this strategy can be encoded in polynomial space: 
% For each choice of the attacker, some corresponding vertex cover that avoids this attack exists. A list of these at most $O(n)$ vertex covers correctly encodes the strategy.
% Since each no-instance can be certified in polynomial space, the problem is contained in coNP. It is not hard to see, that it is coNP-hard as well, hence it is coNP-complete. 

% \textbf{Satisfiability.} An instance of the satisfiability interdiction problem consists out of a formula in CNF over the variables $X = \fromto{x_1}{x_n}$, with universe $\U = X \cup \overline X$, and interdiction budget $k$.
% A similar issue as in vertex cover interdiction arises here: If $k \geq 2$, the interdictor can just choose for some $i \in \fromto{1}{n}$ to attack both literals $x_i, \overline x_i$.
% Every satisfying assignment contains either $x_i$ or $\overline{x_i}$, hence this is a successful attack.
% By a reasoning analogous to the case of vertex cover, this shows that interdiction-SAT is coNP-complete.



% \textbf{Partition/two-machine scheduling.}% \lasse{Partition is currently missing from the list in the appendix. Do we want to add it?}
% An instance of the partition interdiction problem consists out of a set of numbers $A = \fromto{a_1}{a_n}$ and an interdiction budget $k$. The universe is $\U = A$.
% Note that the partition problem has an interesting property: 
% For every set $S \subseteq \fromto{a_1}{a_n}$ which has the property that the sum of $S$ is exactly $1/2$ of the total sum of $A$, 
% the complement of $S$ has the same property. 
% Since the partition problem has this symmetric property, but the satisfiability problem does not have it, 
% a SSP reduction to partition can only exist if this symmetry is broken. Therefore in \cite{grüne2024completeness} in order to break the symmetry the partition problem is defined the following way: 
% A solution of the partition problem is a set $S \subseteq \fromto{a_1}{a_n}$ such that $S$ sums up to $1/2$ the total sum of $A$ and $a_1 \in S$.
% Under this interpretation, an attack of the interdictor on some item $a_i$ means that the interdictor enforces that $a_i$ is not in the same part of the partition as $a_1$.
% Note however, that with this definition the minimum cardinality partition interdiction the interdictor can always attack $a_1$ itself. Hence, analogous to the cases above, the problem is coNP-complete.
% Finally, we remark that even if one would choose another natural definition, 
% where the universe has two items 
% $(a_i, 1)$ and $(a_i, 2)$ for each number $a_i$, indicating whether this number is packed into part 1 or part 2, a similar pattern occurs:
% As soon as $k \geq 2$, the interdictor can attack both $(a_i, 1)$ and $(a_i, 2)$ for some $i$.
% Hence even under this alternate definition, the partition interdiction problem is coNP-complete.


% \subsection{The case of satisfiability}
% In the previous subsection we explained why interdiction-SAT is contained in coNP, hence likely not $\Sigma^p_2$-complete. Note that this is a consequence of our choice of definition of the \textsc{Satisfiability} problem, where we explicitly defined the universe to be $X \cup \overline X$. As a consequence, the interdictor may attack $X \cup \overline X$. 
% \begin{samepage}
%     \begin{mdframed}
%     	\begin{description}
%         \item[]\textsc{Satisfiability}\hfill\\
%         \textbf{Instances:} Literal Set $L = \fromto{x_1}{x_n} \cup \fromto{\overline x_1}{\overline x_n}$, Clauses $C \subseteq \powerset{L}$\\
%         \textbf{Universe:} $L =: \U$.\\
%         \textbf{Solution set:} The set of all sets $L' \subseteq \U$ such that for all $i \in \fromto{1}{n}$ we have $|L' \cap \set{\ell_i, \overline \ell_i}| = 1$, and such that $|L' \cap c_j| \geq 1$ for all $c_j \in C$.
%     	\end{description}
%     \end{mdframed}
% \end{samepage}

% Interesting behavior occurs, when we consider the following alternative version \textsc{Satisfiability'}. 
% \begin{samepage}
%     \begin{mdframed}
%     	\begin{description}
%         \item[]\textsc{Satisfiability'}\hfill\\
%         \textbf{Instances:} Variable Set $X = \fromto{x_1}{x_n}$, Clauses $C \subseteq 2^{X \cup \overline X}$ \\
%         \textbf{Universe:} $X =: \U$.\\
%         \textbf{Solution set:} The set of all sets $X' \subseteq \U$ such that the assignment $\alpha: X \rightarrow \{0,1\}$ with $\alpha(x) = 1 \leftrightarrow x \in X'$ satisfies all clauses in $C$.
%         \end{description}
%     \end{mdframed}
% \end{samepage}
% Here the universe is only $X$, so in the interdiction version, the interdictor may only attack $X$, i.e.\ the interdictor may target individual variables and enforce that they must be set to '0'. 
% We show now that in contrast to the old variant, 
% in this new variant the interdiction problem is $\Sigma^p_2$-complete again. 
% Since the problem \textsc{Satisfiability'} is not part of the original problem set of \cite{grüne2024completeness},
% we perform this proof in two steps.
% \begin{lemma}
% Problem \textsc{Satisfiability'} is SSP-NP-complete, even when all clauses are restricted to length at most three.
% \end{lemma}
% \begin{proof}
% We provide a SSP-reduction from the (old) SSP-NP-complete problem \textsc{Satisfiability} to the (new) problem \textsc{Satisfiability'}. 
% Consider an instance of \textsc{Satisfiability} given by a formula $\varphi$ with $n$ variables $X = \fromto{x_1}{x_n}$ and universe/literal set $\U = L = X \cup \overline{X}$. \textsc{Satisfiability} is SSP-NP-complete even when all clauses are restricted to length three, so let us w.l.o.g.\ assume that property.
% We have to show how to embed this universe into the  universe $\U'$ of some corresponding \textsc{Satisfiability'} instance $\varphi'$, where only positive literals are allowed in $\U'$.
% This can be done the following way: We introduce $2n$ new variables $X' := \fromto{x^t_1}{x^t_n} \cup \fromto{x^f_1}{x^f_n}$. The universe $\U' := X'$ consists out of the $2n$ positive literals $X'$.
% The new formula $\varphi$ is defined from $\varphi$ in two steps. First a substitution process takes place: 
% For each $i=1,\dots,n$ each positive literal $x_i$ gets replaced by the positive literal $x^t_i$. 
% Each negative literal $\overline x_i$  gets replaced by the positive literal $x_i^f$.
% In a second step, all the conditions $x_i^t \not \leftrightarrow x_i^f$ are added to $\varphi'$.
% (Note that this is equivalent to $(x_i^t \lor x_i^f)\land (\overline{x}_i^t \lor \overline{x}^f_i)$). Formally,
% \[
% \varphi' = \text{substitute}(\varphi) \bigwedge_{i=1}^n (x_i^t \lor x_i^f)\land (\overline{x}_i^t \lor \overline{x}^f_i).
% \]
% The SSP reduction is completed by specifying the embedding function $f : \U \to \U'$ via $f(x_i) := x_i^t$ and $f(\overline x_i) := x_i^f$.
% Clearly all clauses of $\varphi'$ have length at most three. Note that this reduction is a correct reduction, i.e.\ it transforms yes-instances into yes-instances and no-instances into no-instances, because the added constraints make sure that exactly one of $x_i^t, x_i^f$ is true.
% Furthermore, it has the SSP property: For every solution $S \subseteq \U$ of \textsc{Satisfiability}, 
% the \enquote{translated} set $f(S) \subseteq \U'$ is a solution of \textsc{Satisfiability'}.
% Furthermore, for every solution $S \subseteq \U'$ of \textsc{Satisfiability'}, the set $f^{-1}(S)\subseteq \U$ is a solution of \textsc{Satisfiability}.
% These two facts together imply that we have a correct SSP reduction (in this case the SSP property is simpler because not only $f(\U) \subseteq \U'$, but even $f(\U) = \U'$). 
% \end{proof}

% \begin{theorem}
%     Problem \textsc{Min. Card. interdiction-Satisfiability'} is $\Sigma^p_2$-complete.
% \end{theorem}
% \begin{proof}
%     By the previous lemma, \textsc{Satisfiability'} is SSP-NP-complete, even if all clauses are restricted to length three.
%     Due to \cite{grüne2024completeness}, the problem $\textsc{Comb. Interdiction-Satisfiability'}$ is $\Sigma^p_2$-complete, even if all clauses are restrcited to length three.
%     We provide a reduction from the latter problem in terms of an invulnerability gadget analogous to the gadgets presented in \cref{sec:invulnerability-gadgets}. 
%     Indeed, consider an instance of \textsc{Satisfiability'} with formula $\varphi$ in CNF and every clause of length three, together with the universe $\U = \fromto{x_1}{x_n}$, a set $C \subseteq \U$ 
%     of vulnerable literals, and interdiction budget $k \in \N_0$.
%     For every variable $x_i \in \U \setminus C$, we explain how to make $x_i$ invulnerable.
%     We introduce $k+1$ new variables $x^{(1)}_i, \dots x^{(k+1)}_i$. Our goal is to have the equivalence
%     \[
%         x_i \equiv x^{(1)}_i \lor \dots \lor x^{(k+1)}_i.
%     \]
%     We can achieve this equivalence through means of the following substitution process starting from formula $\varphi$: 
%     Every occurence of $x_i$ in the formula gets substituted by $x^{(1)}_i \lor \dots \lor x^{(k+1)}_i$. 
%     Every occurence of $\overline x_i$ gets substituted (by De Morgan's law) by $(\overline x^{(1)}_i \land \dots \land \overline x^{(k+1)}_i)$.
%     Note that this has two effects: 
%     First, the length of a clause may now exceed 3. Secondly, the formula is not in CNF anymore. 
%     Note however that we can use the distributive law to expand every clause that is not in CNF. 
%     Since before each clause before had a length of at most three, this results in a blow-up of the instance size of a factor at most $(k+1)^3$, i.e.\ at most a polynomial factor.
%     Let $\varphi'$ be the resulting formula. 
%     We can see that there is an equivalence of satisfying assignment of $\varphi$ and $\varphi'$, in the sense that $x_i$ is true in $\varphi$ if and only if $x^{(1)}_i \lor \dots \lor x^{(k+1)}_i$ is true in $\varphi'$ (for all invulnerable $x_i$). 
%     However, since the interdiciton budget is only $k$, for all invulnerable variables we see that the interdictor can never enforce $x^{(1)}_i \lor \dots \lor x^{(k+1)}_i$ to be false.
%     This shows that \textsc{Comb. Interdiction-Satisfiability'} reduces to \textsc{Min. Card. interdiction-Satisfiability'}, hence proving $\Sigma^p_2$-completeness.
    
% \end{proof}

% Note that the reasoning presented in this proof was slightly different from \cref{thm:meta-theorem}, 
% since we start with a formula where every clause has length three, but do not preserve this property during the proof. Hence $\Sigma^p_2$-completeness is only shown in the case where clauses can have unrestricted length.

% Indeed, it is easy to see that the interdiction problem of \textsc{Satisfiability'} 
% where every clause has length three is coNP-complete: If $k \geq 3$ holds for the interdiction budget, the attacker distinguishes two cases:
% If there is a clause with three positive literals, the attacker blocks all of them and immediately wins. 
% In the other case, every clause has at least one negative literal.
% Then the attacker can never win, since the defender can set every variable to false, which is a satisfying assignment that can never be blocked.
% By an analogous argument, we can see that for any $t = O(1)$, the interdiction problem of \textsc{Satisfiability'} 
% with clauses restricted to length $t$ is coNP-complete.

% Finally, we remark that slightly different variants of interdiction-3-Sat have been shown to be $\Sigma^p_2$-complete. In these variants, the interdictor does not have access to all variables (see \cite[Sec. 4.2]{grüne2024completeness} or \cite[Thm. 1]{jackiewicz2024computational}).


% \christoph{Ich glaube, wir sollten noch einiges an Überlegung in diese Section investieren. Wenigstens ein meta-theorem für eine hinreichende Bedingung sollte relativ schnell aufstellbar sein. Besser wäre natürlich eine richtige Dichotomie.}

% \christoph{Unter diese Probleme fallen auch so ziemlich alle Graphprobleme auf Bounded Degree Graphen. Wenn Du Dir die Gadgets anschuast, die wir gebastelt haben, dann verlangen diese, dass wir einen nicht konstanten Grad haben nämlich in O(Größe des Budgets des Angreifers). Grundlegend sind alle Probleme betroffen, die eine kostant lange Klausel mit rein-positiven Literalen erzeugen. Dann kann nämlich genau diese angegriffen werden. Dann kann immer das k-Sat-Argument angewendet werden.
% D.h. wenn das Problem durch per Reduktion in eine k-CNF-Sat-Instanz transformieren kann (hier muss wohl noch die SSP Eigenschaft gelten), dann ist dies auch einfach angreifbar und lediglich coNP-vollständig.
% Für Graph-Probleme auf Bounded Degree Graphen könnte man vielleicht den zugehörigen Bounded Search tree Algorithmus (aus der FPT-Theorie) als k-CNF-Sat-Instanz kodieren und damit die coNP-Vollständigkeit zeigen.
% Das wäre wenigstens ein hinreichendes Kriterium, aber noch kein notwendiges...

% Wenn wir so etwas zeigen könnten, dann haben wir eine coole Verbindung dieser Probleme mit der parametrisierten Welt und dann auch Model Checking/Datenbanktheorie, das könnte für ein größeres Publikum interessant sein.
% }
% \lasse{Das klingt sehr interessant und auch plausibel. Allerdings weiß ich momentan nicht, wie genau man dieses meta-theorem mit der Sprache unseres frameworks beschreiben könnte. Würde es nicht auch ausreichen zu sagen: Wir beobachten das Muster, dass ein ähnliches Argument wie die coNP-completeness von interdiction-kSAT anscheinend oft gemacht werden kann, insbesondere wenn das problem als eine Vereinigung von lauter constraints begrenzter Länge interpretiert werden kann.}

% \begin{itemize}
% \item dominating set in graphs with minimum degree constant, e.g. in planar graphs, bounded genus graphs.
% \item $k$-sat for $k = O(1)$
% \item vertex cover, oder allgemeiner hitting set mit $\min|S_i| = O(1)$, wo $S_i$ die sets die gehittet werden müssen
% \item oder bzw. set cover wenn $\min_e \text{coverage}(e) = O(1)$, wobei für ein element $e$ im ground set $\Omega$ $\text{coverage}(e) = |\set{S \in \U : e \in S}|$.
% \item Steiner tree, two-vertex disjoint path in $O(1)$ degree graphs
% \item feedback arc set, feedback vertex set in graphs of constant girth
% \item facility location, $p$-center, $p$-median, if there exists some customer for who only $O(1)$ possible locations exist which could serve the customer
% \item clique, independent set, in $O(1)$-degree graphs, but only the interdiction version, where optimal solutions must be interdicted $(t = t^\star)$. For the more general interdiciton problem this is not true
% \end{itemize}

% \begin{lemma}
%     If $\Pi$ is a problem in SSP-NP such that there exists in each instance $I$ of $\Pi$ a small subset $U' \subseteq \U(I)$ of size $|U'| = O(1)$ the universe of $\Pi$, such that every solution intersects $U'$, i.e. $U' \cap S \neq \emptyset$ for all $S \in \sol(I)$, then $\textsc{Min. Card-Interdiction-$\Pi$}$ is contained in coNP.
% \end{lemma}
% \begin{proof}
%     Let $t := |U'|$.
%     If the budget $k$ of the interdictor is at least $t$, then the interdictor trivially wins by interdicting all of $U'$. 
%     If $k < t$, and if the defender has a winning strategy, 
%     we can encode a winning strategy of the defender, by enumerating for each potential attack $B \subseteq \U(I)$ with $|B| \leq k$ a solution $S \in \sol(I)$ that avoids $B$.
%     Since $t = O(1)$, these $O(|U(I)|^t)$ possibilities can be encoded in polynomial space.
% \end{proof}
%  % Falls man beweisen kann: Es existiert in jeder Instanz eine Teilmenge $U' \subseteq \U$ mit $|U'| = O(1)$, und jede Lösung $S \in \sol$ des nominalen Problems mindestens ein Element mit $U'$ gemeinsam, dann gilt: $\textsc{Min. Card-Interdiction-$\Pi$} \in$ co-NP. 

% We remark that one could also ask whether the above statement about coNP-containment is also true for the more general interdiction problems introduced in \cref{sec:different-variants-of-interdiction}.
% Note that the problems from \cref{sec:different-variants-of-interdiction} are more general than $\textsc{Min. Card-Interdiction-$\Pi$}$, hence the coNP-containment of the latter does not imply coNP-containment of the first.
% One can show that if an equivalent statement holds about the feasible solutions $F \in \F$ instead of the optimal solutions $S \in \sol$, then also the problems from \cref{sec:different-variants-of-interdiction} are contained in coNP (i.e.\ there exists $U' \subseteq \U$ of size $|U'| =  O(1)$ such that $\forall F \in \F: F \cap U' \neq \emptyset$).
\section*{Conclusion}
This paper aims to enhance our understanding of the computational complexity of computing various Shapley value variants. We found that for various ML models --- including decision trees, regression tree ensembles, weighted automata, and linear regression --- both local and global interventional and baseline SHAP can be computed in polynomial time under HMM modeled distributions. This extends popular algorithms, such as TreeSHAP, beyond their empirical distributional scope. We also establish strict complexity gaps between the various SHAP variants (baseline, interventional, and conditional) and prove the intractability of computing SHAP for tree ensembles and neural networks in simplified scenarios. Overall, we present SHAP as a versatile framework whose complexity depends on four key factors: \begin{inparaenum}[(i)] \item model type, \item SHAP variant, \item distribution modeling approach, \item and local vs. global explanations\end{inparaenum}. We believe this perspective provides deeper insight into the computational complexity of SHAP, paving the way for future work.




%We believe that our framework provides a more intricate understanding of SHAP computation complexity across different models, distributions, and variants, paving the way for further research.

Our work opens promising directions for future research. First, expanding our computational analysis to other SHAP-related metrics, such as asymmetric SHAP~\citep{frye20} and SAGE~\citep{covert2020understanding}, would be valuable. Additionally, we aim to explore more expressive distribution classes and relaxed assumptions beyond those in Section \ref{sec:tractable} while maintaining tractable SHAP computation. Finally, when exact computation is intractable (Section \ref{sec:intractable}), investigating the approximability of SHAP metrics through approximation and parameterized complexity theory~\citep{downey2012parameterized} is an important direction.

%Our work opens several promising avenues for future research on the computational properties of explainable AI methods, with a particular focus on SHAP. First, it would be interesting to broaden the computational analysis conducted in this work to include other popular SHAP-related metrics in the literature, such as asymmetric SHAP \cite{frye20} and SAGE \cite{covert2020understanding}. Also, in the future, we aim to explore more expressive distribution classes and relaxed distributional assumptions—extending beyond those examined in Section \ref{sec:tractable} —that still yield tractable SHAP computation. Finally, when exact computation proves intractable (Section \ref{sec:intractable}), it is worthwhile to theoretically investigate the question of the approximability of computing the SHAP metrics across various configurations, through the lens of approximation and parametrized complexity theory \cite{arora2009computational}.

%This paper aims to deepen our understanding of the computational complexity involved in obtaining different Shapley value variants. We found that for a variety of ML models, including decision trees, tree ensembles for regression, weighted automata, and linear regression models — computing both local and global interventional and baseline SHAP can be done in polynomial time when distributions are modeled by HMMs. This extends the distributional scope of popular algorithms like TreeSHAP, which is limited to empirical distributions. Additionally, we demonstrate a strict complexity gap between SHAP variants, showing that interventional and baseline SHAP can be strictly easier to compute than conditional SHAP. Despite these positive results, we uncovered intractability for various SHAP variants in neural networks and tree ensembles. Finally, we provided generalized complexity relations across SHAP variants. We believe that our framework offers a deeper understanding of the complexity involved in computing SHAP across various variants, models, distributions, as well as in both local and global computations, laying the groundwork for future research.

\newpage

\bibliography{bib_general,bib_interdiction,bib_reductions}

\newpage

\appendix
% !TEX root = main.tex
\section{Problems Definitions}
\label{app:sec:problemDefinitions}

\begin{samepage}
    \begin{mdframed}
    	\begin{description}
        \item[]\textsc{Satisfiability}\hfill\\
        \textbf{Instances:} Literal Set $L = \fromto{\ell_1}{\ell_n} \cup \fromto{\overline \ell_1}{\overline \ell_n}$, Clauses $C \subseteq \powerset{L}$.\\
        \textbf{Universe:} $L =: \U$.\\
        \textbf{Solution set:} The set of all sets $L' \subseteq \U$ such that for all $i \in \fromto{1}{n}$ we have $|L' \cap \set{\ell_i, \overline \ell_i}| = 1$, and such that $|L' \cap c_j| \geq 1$ for all $c_j \in C$, $j \in \fromto{1}{|C|}$.
    	\end{description}
    \end{mdframed}
\end{samepage}

\begin{samepage}
    \begin{mdframed}
    	\begin{description}
        \item[]\textsc{3-Satisfiability}\hfill\\
        \textbf{Instances:} Literal Set $L = \fromto{\ell_1}{\ell_n} \cup \fromto{\overline \ell_1}{\overline \ell_n}$, Clauses $C \subseteq \powerset{L}$ s.t. $\forall c_j \in C : |c_j| = 3$.\\
        \textbf{Universe:} $L =: \U$.\\
        \textbf{Solution set:} The set of all sets $L' \subseteq \U$ such that for all $i \in \fromto{1}{n}$ we have $|L' \cap \set{\ell_i, \overline \ell_i}| = 1$, and such that $|L' \cap c_j| \geq 1$ for all $c_j \in C$.
    	\end{description}
    \end{mdframed}
\end{samepage}

\begin{samepage}
    \begin{mdframed}
    	\begin{description}
        \item[]\textsc{Dominating Set}\hfill\\
        \textbf{Instances:} Graph $G = (V, E)$, number $k \in \N$.\\
        \textbf{Universe:} Vertex set $V =: \U$.\\
        \textbf{Feasible solution set:} The set of all dominating sets.\\
        \textbf{Solution set:} The set of all dominating sets of size at most $k$.
    	\end{description}
    \end{mdframed}
\end{samepage}

\begin{samepage}
    \begin{mdframed}
    	\begin{description}
        \item[]\textsc{Set Cover}\hfill\\
        \textbf{Instances:} Sets $S_i \subseteq \fromto{1}{m}$ for $i \in \fromto{1}{n}$, number $k \in \N$.\\
        \textbf{Universe:} $\{S_1 \dots, S_n\} =: \U$.\\
        \textbf{Feasible solution set:} The set of all $S \subseteq \{S_1, \dots, S_n\}$ s.t. $\bigcup_{s \in S} s = \fromto{1}{m}$.\\
        \textbf{Solution set:} Set of all feasible solutions with $|S| \leq k$.
    	\end{description}
    \end{mdframed}
\end{samepage}

\begin{samepage}
    \begin{mdframed}
    	\begin{description}   
        \item[]\textsc{Hitting Set}\hfill\\
        \textbf{Instances:} Sets $S_j \subseteq \fromto{1}{n}$ for $j \in \fromto{1}{m}$, number $k \in \N$.\\
        \textbf{Universe:} $\fromto{1}{n} =: \U$.\\
        \textbf{Feasible solution set:} The set of all $H \subseteq \fromto{1}{n}$ such that $H \cap S_j \neq \emptyset$ for all $j \in \fromto{1}{m}$.\\
        \textbf{Solution set:} Set of all feasible solutions with $|H| \leq k$.
    	\end{description}
    \end{mdframed}
\end{samepage}

\begin{samepage}
    \begin{mdframed}
    	\begin{description}
        \item[]\textsc{Feedback Vertex Set}\hfill\\
        \textbf{Instances:} Directed Graph $G = (V, A)$, number $k \in \N$.\\
        \textbf{Universe:} Vertex set $V =: \U$.\\
        \textbf{Feasible solution set:} The set of all vertex sets $V' \subseteq V$ such that after deleting $V'$ from $G$, the resulting graph is cycle-free (i.e. a forest).\\
        \textbf{Solution set:} The set of all feasible solutions $V'$ of size at most $k$.
    	\end{description}
    \end{mdframed}
\end{samepage}

\begin{samepage}
    \begin{mdframed}
    	\begin{description}
        \item[]\textsc{Feedback Arc Set}\hfill\\
        \textbf{Instances:} Directed Graph $G = (V, A)$, number $k \in \N$.\\
        \textbf{Universe:} Arc set $A =: \U$.\\
        \textbf{Feasible solution set:} The set of all arc sets $A' \subseteq A$ such that after deleting $A'$ from $G$, the resulting graph is cycle-free (i.e. a forest).\\
        \textbf{Solution set:} The set of all feasible solutions $A'$ of size at most $k$.
    	\end{description}
    \end{mdframed}
\end{samepage}

\begin{samepage}
    \begin{mdframed}
    	\begin{description} 
        \item[]\textsc{Uncapacitated Facility Location}\hfill\\
        \textbf{Instances:} Set of potential facilities $F = \fromto{1}{n}$, set of clients $C = \fromto{1}{m}$, fixed cost of opening facility function $f: F \rightarrow \Z$, service cost function $c: F \times C \rightarrow \Z$, cost threshold $k \in \Z$\\
        \textbf{Universe:} Facility set $F =: \U$.\\
        \textbf{Solution set:} The set of sets $F' \subseteq F$ s.t. $\sum_{i \in F'} f(i) + \sum_{j \in C} \min_{i \in F'} c(i, j) \leq k$.
    	\end{description}
    \end{mdframed}
\end{samepage}

\begin{samepage}
    \begin{mdframed}
    	\begin{description} 
        \item[]\textsc{p-Center}\hfill\\
        \textbf{Instances:} Set of potential facilities $F = \fromto{1}{n}$, set of clients $C = \fromto{1}{m}$, service cost function $c: F \times C \rightarrow \Z$, facility threshold $p \in \N$, cost threshold $k \in \Z$\\
        \textbf{Universe:} Facility set $F =: \U$.\\
        \textbf{Solution set:} The set of sets $F' \subseteq F$ s.t. $|F'| \leq p$ and $\max_{j \in C} \min_{i \in F'} c(i, j) \leq k$.
    	\end{description}
    \end{mdframed}
\end{samepage}

\begin{samepage}
    \begin{mdframed}
    	\begin{description} 
        \item[]\textsc{p-Median}\hfill\\
        \textbf{Instances:} Set of potential facilities $F = \fromto{1}{n}$, set of clients $C = \fromto{1}{m}$, service cost function $c: F \times C \rightarrow \Z$, facility threshold $p \in \N$, cost threshold $k \in \Z$\\
        \textbf{Universe:} Facility set $F =: \U$.\\
        \textbf{Solution set:} The set of sets $F' \subseteq F$ s.t. $|F'| \leq p$ and $\sum_{j \in C} \min_{i \in F'} c(i, j) \leq k$.
    	\end{description}
    \end{mdframed}
\end{samepage}

\begin{samepage}
    \begin{mdframed}
    	\begin{description} 
        \item[]\textsc{Independent Set}\hfill\\
        \textbf{Instances:} Graph $G = (V,E)$, number $k \in \N$.\\
        \textbf{Universe:} Vertex set $V =: \U$.\\
        \textbf{Feasible solution set:} The set of all independent sets.\\
        \textbf{Solution set:} The set of all independent sets of size at least $k$.
    	\end{description}
    \end{mdframed}
\end{samepage}

\begin{samepage}
    \begin{mdframed}
    	\begin{description}   
        \item[]\textsc{Clique}\hfill\\
        \textbf{Instances:} Graph $G = (V, E)$, number $k \in \N$.\\
        \textbf{Universe:} Vertex set $V =: \U$.\\
        \textbf{Feasible solution set:} The set of all cliques.\\
        \textbf{Solution set:} The set of all cliques of size at least $k$.
    	\end{description}
    \end{mdframed}
\end{samepage}

\begin{samepage}
    \begin{mdframed}
    	\begin{description}   
        \item[]\textsc{Subset Sum}\hfill\\
        \textbf{Instances:} Numbers $\fromto{a_1}{a_n} \subseteq \N$, and target value $M \in \N$.\\
        \textbf{Universe:} $\fromto{a_1}{a_n} =: \U$.\\
        \textbf{Solution set:} The set of all sets $S \subseteq \U$ with $\sum_{a_i \in S}a_i = M$.
    	\end{description}
    \end{mdframed}
\end{samepage}

\begin{samepage}
    \begin{mdframed}
    	\begin{description}   
        \item[]\textsc{Knapsack}\hfill\\
        \textbf{Instances:} Objects with prices and weights $\fromto{(p_1, w_1)}{(p_n, w_n)} \subseteq \N^2$, and $W, P \in \N$.\\
        \textbf{Universe:} $\fromto{(p_1, w_1)}{(p_n, w_n)} =: \U$.\\
        \textbf{Feasible solution set:} The set of all $S \subseteq \U$ with $\sum_{(p_i, w_i) \in S}w_i \leq W$.\\
        \textbf{Solution set:} The set of feasible $S$ with $\sum_{(p_i, w_i) \in S} p_i \geq P$.
    	\end{description}
    \end{mdframed}
\end{samepage}

\begin{samepage}
    \begin{mdframed}
    	\begin{description}
        \item[]\textsc{Directed Hamiltonian Path}\hfill\\
        \textbf{Instances:} Directed Graph $G = (V, A)$, Vertices $s, t \in V$.\\
        \textbf{Universe:} Arc set $A =: \U$.\\
        \textbf{Solution set:} The set of all sets $C \subseteq A$ forming a Hamiltonian path going from $s$ to $t$.
    	\end{description}
    \end{mdframed}
\end{samepage}

\begin{samepage}
    \begin{mdframed}
    	\begin{description}
        \item[]\textsc{Directed Hamiltonian Cycle}\hfill\\
        \textbf{Instances:} Directed Graph $G = (V, A)$.\\
        \textbf{Universe:} Arc set $A =: \U$.\\
        \textbf{Solution set:} The set of all sets $C \subseteq A$ forming a Hamiltonian cycle.
    	\end{description}
    \end{mdframed}
\end{samepage}

\begin{samepage}
    \begin{mdframed}
    	\begin{description}  
        \item[]\textsc{Undirected Hamiltonian Cycle}\hfill\\
        \textbf{Instances:} Graph $G = (V, E)$.\\
        \textbf{Universe:} Edge set $E =: \U$.\\
        \textbf{Solution set:} The set of all sets $C \subseteq E$ forming a Hamiltonian cycle.
    	\end{description}
    \end{mdframed}
\end{samepage}

\begin{samepage}
    \begin{mdframed}
    	\begin{description}
        \item[]\textsc{Traveling Salesman Problem}\hfill\\
        \textbf{Instances:} Complete Graph $G = (V, E)$, weight function $w: E \rightarrow \Z $, number $k \in \N$.\\
        \textbf{Universe:} Edge set $E =: \U$.\\
        \textbf{Feasible solution set:} The set of all TSP tours $T\subseteq E$.\\
        \textbf{Solution set:} The set of feasible $T$ with $w(T) \leq k$.
    	\end{description}
    \end{mdframed}
\end{samepage}

\begin{samepage}
    \begin{mdframed}
    	\begin{description}
        \item[]\textsc{Directed} $k$-\textsc{Vertex Disjoint Path}\hfill\\
        \textbf{Instances:} Directed graph $G = (V, A)$, $s_i, t_i \in V$ for $i \in \fromto{1}{k}$.\\
        \textbf{Universe:} Arc set $A =: \U$.\\
        \textbf{Solution set:} The sets of all sets $A' \subseteq A$ such that $A' = \bigcup^k_{i = 1} A(P_i)$, where all $P_i$ are pairwise vertex-disjoint paths from $s_i$ to $t_i$ for $1 \leq i \leq k$.
    	\end{description}
    \end{mdframed}
\end{samepage}

\begin{samepage}
    \begin{mdframed}
    	\begin{description}
        \item[]\textsc{Steiner Tree}\hfill\\
        \textbf{Instances:} Undirected graph $G = (S \cup T, E)$, set of Steiner vertices $S$, set of terminal vertices $T$, edge weights $c: E \rightarrow \N$, number $k \in \N$.\\
        \textbf{Universe:} Edge set $E =: \U$.\\
        \textbf{Feasible solution set:} The set of all sets $E' \subseteq E$ such that $E'$ is a tree connecting all terminal vertices from $T$.\\
        \textbf{Solution set:} The set of feasible solutions $E'$ with $\sum_{e' \in E'} c(e') \leq k$.
    	\end{description}
    \end{mdframed}
\end{samepage}



\end{document}
