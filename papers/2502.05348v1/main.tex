

\documentclass[a4paper,UKenglish,cleveref, autoref, thm-restate]{lipics-v2019}
%This is a template for producing LIPIcs articles. 
%See lipics-manual.pdf for further information.
%for A4 paper format use option "a4paper", for US-letter use option "letterpaper"
%for british hyphenation rules use option "UKenglish", for american hyphenation rules use option "USenglish"
%for section-numbered lemmas etc., use "numberwithinsect"
%for enabling cleveref support, use "cleveref"
%for enabling autoref support, use "autoref"
%for anonymousing the authors (e.g. for double-blind review), add "anonymous"
%for enabling thm-restate support, use "thm-restate"

%\graphicspath{{./graphics/}}%helpful if your graphic files are in another directory

\newif\iflongversion
%\longversiontrue
\longversionfalse

\usepackage{tikz}
\usetikzlibrary{calc}
\usetikzlibrary{patterns}
\usepackage{csquotes}
%\usepackage[sort]{natbib}

\usepackage{mdframed}

\crefname{observation}{observation}{observations}
\Crefname{observation}{Observation}{Observations}
\newtheorem{observation}[theorem]{Observation}

\bibliographystyle{plainurl}


\title{The Complexity of Blocking All Solutions}

%\titlerunning{The Complexity of Blocking All Solutions} %TODO optional, please use if title is longer than one line

\author{Christoph Grüne}{Department of Computer Science, RWTH Aachen University, Germany}{gruene@algo.rwth-aachen.de}{https://orcid.org/0000-0002-7789-8870}{Funded by the German Research Foundation (DFG) – GRK 2236/2.}
\author{Lasse Wulf}{Section of Algorithms, Logic and Graphs, Technical University of Denmark, Kongens Lyngby, Denmark}{lawu@dtu.dk}{https://orcid.org/0000-0001-7139-4092}{Funded by the Carlsberg Foundation CF21-0302 ``Graph Algorithms with Geometric Applications''.}


\authorrunning{C. Grüne and L. Wulf} %TODO mandatory. First: Use abbreviated first/middle names. Second (only in severe cases): Use first author plus 'et al.'

\Copyright{Christoph Grüne and Lasse Wulf} %TODO mandatory, please use full first names. LIPIcs license is "CC-BY";  http://creativecommons.org/licenses/by/3.0/

\ccsdesc[500]{Theory of computation~Problems, reductions and completeness} %TODO mandatory: Please choose ACM 2012 classifications from https://dl.acm.org/ccs/ccs_flat.cfm 

\keywords{
Computational Complexity,
Robust Optimization,
Most Vital Elements,
Most Vital Nodes,
Most Vital Vertex,
Most Vital Edges,
Blocker Problems,
Vertex Blocker,
Node Blocker,
Edge Blocker,
Interdiction Problems,
Polynomial Hierarchy,
Sigma-2} %TODO mandatory; please add comma-separated list of keywords

\category{} %optional, e.g. invited paper

\relatedversion{} %optional, e.g. full version hosted on arXiv, HAL, or other respository/website
%\relatedversion{A full version of the paper is available at \url{...}.}

\supplement{}%optional, e.g. related research data, source code, ... hosted on a repository like zenodo, figshare, GitHub, ...

%\funding{(Optional) general funding statement \dots}%optional, to capture a funding statement, which applies to all authors. Please enter author specific funding statements as fifth argument of the \author macro.

%\acknowledgements{I want to thank \dots}%optional

\nolinenumbers %uncomment to disable line numbering

\hideLIPIcs  %uncomment to remove references to LIPIcs series (logo, DOI, ...), e.g. when preparing a pre-final version to be uploaded to arXiv or another public repository

%Editor-only macros:: begin (do not touch as author)%%%%%%%%%%%%%%%%%%%%%%%%%%%%%%%%%%
\EventEditors{John Q. Open and Joan R. Access}
\EventNoEds{2}
\EventLongTitle{42nd Conference on Very Important Topics (CVIT 2016)}
\EventShortTitle{CVIT 2016}
\EventAcronym{CVIT}
\EventYear{2016}
\EventDate{December 24--27, 2016}
\EventLocation{Little Whinging, United Kingdom}
\EventLogo{}
\SeriesVolume{42}
\ArticleNo{23}
%%%%%%%%%%%%%%%%%%%%%%%%%%%%%%%%%%%%%%%%%%%%%%%%%%%%%%

\definecolor{darkgreen}{RGB}{0,128,0}
\definecolor{darkred}{RGB}{128,0,0}

\newcommand{\todo}[1]{\textcolor{red}{\textbf{TODO: #1}}\\}
\newcommand{\important}[1]{\textcolor{purple}{\textsc{IMPORTANT: #1}}\\}
\newcommand{\lasse}[1]{\textcolor{orange}{Lasse: #1}}
\newcommand{\christoph}[1]{{\color{darkgreen} Christoph: #1}}
\newcommand{\tocheck}[1]{\textcolor{darkred}{#1}\\}

\newcommand{\boxxx}[1]
 {\fbox{\begin{minipage}{11.80cm}\begin{center}\bigskip\begin{minipage}{11.30cm}
  #1\end{minipage}\end{center}~\end{minipage}}}
\newcommand{\R}{\mathbb{R}}
\newcommand{\N}{\mathbb{N}}
\newcommand{\Z}{\mathbb{Z}}
\newcommand{\I}{\mathcal{I}}
\newcommand{\U}{\mathcal{U}}
\newcommand{\F}{\mathcal{F}}
\newcommand{\sol}{\mathcal{S}}
\newcommand{\powerset}[1]{2^{#1}}
\newcommand{\set}[1]{\{ #1 \}}
\newcommand{\fromto}[2]{\set{#1, \ldots, #2}}

\DeclareMathOperator{\poly}{poly}
\DeclareMathOperator{\reg}{reg}
\DeclareMathOperator*{\argmax}{arg\,max}
\DeclareMathOperator*{\argmin}{arg\,min}
\DeclareMathOperator{\dist}{dist}
\newcommand{\leqSSP}{\leq_\text{SSP}}
\newcommand{\bin}{\set{0,1}}

\newcommand{\NP}{\textsl{NP}}
\newcommand{\coNP}{\textsl{coNP}}
\newcommand{\PSPACE}{\textsl{PSPACE}}
\newcommand{\PTIME}{\textsl{PTIME}}

\newcommand{\citationneeded}[1][None]{\textsuperscript{\color{red} [Citation needed: #1]}}

\begin{document}

\maketitle
\begin{abstract}
    \begin{abstract}  
Test time scaling is currently one of the most active research areas that shows promise after training time scaling has reached its limits.
Deep-thinking (DT) models are a class of recurrent models that can perform easy-to-hard generalization by assigning more compute to harder test samples.
However, due to their inability to determine the complexity of a test sample, DT models have to use a large amount of computation for both easy and hard test samples.
Excessive test time computation is wasteful and can cause the ``overthinking'' problem where more test time computation leads to worse results.
In this paper, we introduce a test time training method for determining the optimal amount of computation needed for each sample during test time.
We also propose Conv-LiGRU, a novel recurrent architecture for efficient and robust visual reasoning. 
Extensive experiments demonstrate that Conv-LiGRU is more stable than DT, effectively mitigates the ``overthinking'' phenomenon, and achieves superior accuracy.
\end{abstract}      
\end{abstract}

\section{Introduction}
\label{sec:introduction}
The business processes of organizations are experiencing ever-increasing complexity due to the large amount of data, high number of users, and high-tech devices involved \cite{martin2021pmopportunitieschallenges, beerepoot2023biggestbpmproblems}. This complexity may cause business processes to deviate from normal control flow due to unforeseen and disruptive anomalies \cite{adams2023proceddsriftdetection}. These control-flow anomalies manifest as unknown, skipped, and wrongly-ordered activities in the traces of event logs monitored from the execution of business processes \cite{ko2023adsystematicreview}. For the sake of clarity, let us consider an illustrative example of such anomalies. Figure \ref{FP_ANOMALIES} shows a so-called event log footprint, which captures the control flow relations of four activities of a hypothetical event log. In particular, this footprint captures the control-flow relations between activities \texttt{a}, \texttt{b}, \texttt{c} and \texttt{d}. These are the causal ($\rightarrow$) relation, concurrent ($\parallel$) relation, and other ($\#$) relations such as exclusivity or non-local dependency \cite{aalst2022pmhandbook}. In addition, on the right are six traces, of which five exhibit skipped, wrongly-ordered and unknown control-flow anomalies. For example, $\langle$\texttt{a b d}$\rangle$ has a skipped activity, which is \texttt{c}. Because of this skipped activity, the control-flow relation \texttt{b}$\,\#\,$\texttt{d} is violated, since \texttt{d} directly follows \texttt{b} in the anomalous trace.
\begin{figure}[!t]
\centering
\includegraphics[width=0.9\columnwidth]{images/FP_ANOMALIES.png}
\caption{An example event log footprint with six traces, of which five exhibit control-flow anomalies.}
\label{FP_ANOMALIES}
\end{figure}

\subsection{Control-flow anomaly detection}
Control-flow anomaly detection techniques aim to characterize the normal control flow from event logs and verify whether these deviations occur in new event logs \cite{ko2023adsystematicreview}. To develop control-flow anomaly detection techniques, \revision{process mining} has seen widespread adoption owing to process discovery and \revision{conformance checking}. On the one hand, process discovery is a set of algorithms that encode control-flow relations as a set of model elements and constraints according to a given modeling formalism \cite{aalst2022pmhandbook}; hereafter, we refer to the Petri net, a widespread modeling formalism. On the other hand, \revision{conformance checking} is an explainable set of algorithms that allows linking any deviations with the reference Petri net and providing the fitness measure, namely a measure of how much the Petri net fits the new event log \cite{aalst2022pmhandbook}. Many control-flow anomaly detection techniques based on \revision{conformance checking} (hereafter, \revision{conformance checking}-based techniques) use the fitness measure to determine whether an event log is anomalous \cite{bezerra2009pmad, bezerra2013adlogspais, myers2018icsadpm, pecchia2020applicationfailuresanalysispm}. 

The scientific literature also includes many \revision{conformance checking}-independent techniques for control-flow anomaly detection that combine specific types of trace encodings with machine/deep learning \cite{ko2023adsystematicreview, tavares2023pmtraceencoding}. Whereas these techniques are very effective, their explainability is challenging due to both the type of trace encoding employed and the machine/deep learning model used \cite{rawal2022trustworthyaiadvances,li2023explainablead}. Hence, in the following, we focus on the shortcomings of \revision{conformance checking}-based techniques to investigate whether it is possible to support the development of competitive control-flow anomaly detection techniques while maintaining the explainable nature of \revision{conformance checking}.
\begin{figure}[!t]
\centering
\includegraphics[width=\columnwidth]{images/HIGH_LEVEL_VIEW.png}
\caption{A high-level view of the proposed framework for combining \revision{process mining}-based feature extraction with dimensionality reduction for control-flow anomaly detection.}
\label{HIGH_LEVEL_VIEW}
\end{figure}

\subsection{Shortcomings of \revision{conformance checking}-based techniques}
Unfortunately, the detection effectiveness of \revision{conformance checking}-based techniques is affected by noisy data and low-quality Petri nets, which may be due to human errors in the modeling process or representational bias of process discovery algorithms \cite{bezerra2013adlogspais, pecchia2020applicationfailuresanalysispm, aalst2016pm}. Specifically, on the one hand, noisy data may introduce infrequent and deceptive control-flow relations that may result in inconsistent fitness measures, whereas, on the other hand, checking event logs against a low-quality Petri net could lead to an unreliable distribution of fitness measures. Nonetheless, such Petri nets can still be used as references to obtain insightful information for \revision{process mining}-based feature extraction, supporting the development of competitive and explainable \revision{conformance checking}-based techniques for control-flow anomaly detection despite the problems above. For example, a few works outline that token-based \revision{conformance checking} can be used for \revision{process mining}-based feature extraction to build tabular data and develop effective \revision{conformance checking}-based techniques for control-flow anomaly detection \cite{singh2022lapmsh, debenedictis2023dtadiiot}. However, to the best of our knowledge, the scientific literature lacks a structured proposal for \revision{process mining}-based feature extraction using the state-of-the-art \revision{conformance checking} variant, namely alignment-based \revision{conformance checking}.

\subsection{Contributions}
We propose a novel \revision{process mining}-based feature extraction approach with alignment-based \revision{conformance checking}. This variant aligns the deviating control flow with a reference Petri net; the resulting alignment can be inspected to extract additional statistics such as the number of times a given activity caused mismatches \cite{aalst2022pmhandbook}. We integrate this approach into a flexible and explainable framework for developing techniques for control-flow anomaly detection. The framework combines \revision{process mining}-based feature extraction and dimensionality reduction to handle high-dimensional feature sets, achieve detection effectiveness, and support explainability. Notably, in addition to our proposed \revision{process mining}-based feature extraction approach, the framework allows employing other approaches, enabling a fair comparison of multiple \revision{conformance checking}-based and \revision{conformance checking}-independent techniques for control-flow anomaly detection. Figure \ref{HIGH_LEVEL_VIEW} shows a high-level view of the framework. Business processes are monitored, and event logs obtained from the database of information systems. Subsequently, \revision{process mining}-based feature extraction is applied to these event logs and tabular data input to dimensionality reduction to identify control-flow anomalies. We apply several \revision{conformance checking}-based and \revision{conformance checking}-independent framework techniques to publicly available datasets, simulated data of a case study from railways, and real-world data of a case study from healthcare. We show that the framework techniques implementing our approach outperform the baseline \revision{conformance checking}-based techniques while maintaining the explainable nature of \revision{conformance checking}.

In summary, the contributions of this paper are as follows.
\begin{itemize}
    \item{
        A novel \revision{process mining}-based feature extraction approach to support the development of competitive and explainable \revision{conformance checking}-based techniques for control-flow anomaly detection.
    }
    \item{
        A flexible and explainable framework for developing techniques for control-flow anomaly detection using \revision{process mining}-based feature extraction and dimensionality reduction.
    }
    \item{
        Application to synthetic and real-world datasets of several \revision{conformance checking}-based and \revision{conformance checking}-independent framework techniques, evaluating their detection effectiveness and explainability.
    }
\end{itemize}

The rest of the paper is organized as follows.
\begin{itemize}
    \item Section \ref{sec:related_work} reviews the existing techniques for control-flow anomaly detection, categorizing them into \revision{conformance checking}-based and \revision{conformance checking}-independent techniques.
    \item Section \ref{sec:abccfe} provides the preliminaries of \revision{process mining} to establish the notation used throughout the paper, and delves into the details of the proposed \revision{process mining}-based feature extraction approach with alignment-based \revision{conformance checking}.
    \item Section \ref{sec:framework} describes the framework for developing \revision{conformance checking}-based and \revision{conformance checking}-independent techniques for control-flow anomaly detection that combine \revision{process mining}-based feature extraction and dimensionality reduction.
    \item Section \ref{sec:evaluation} presents the experiments conducted with multiple framework and baseline techniques using data from publicly available datasets and case studies.
    \item Section \ref{sec:conclusions} draws the conclusions and presents future work.
\end{itemize}
% !TEX root =  ../main.tex
\section{Background on causality and abstraction}\label{sec:preliminaries}

This section provides the notation and key concepts related to causal modeling and abstraction theory.

\spara{Notation.} The set of integers from $1$ to $n$ is $[n]$.
The vectors of zeros and ones of size $n$ are $\zeros_n$ and $\ones_n$.
The identity matrix of size $n \times n$ is $\identity_n$. The Frobenius norm is $\frob{\mathbf{A}}$.
The set of positive definite matrices over $\reall^{n\times n}$ is $\pd^n$. The Hadamard product is $\odot$.
Function composition is $\circ$.
The domain of a function is $\dom{\cdot}$ and its kernel $\ker$.
Let $\mathcal{M}(\mathcal{X}^n)$ be the set of Borel measures over $\mathcal{X}^n \subseteq \reall^n$. Given a measure $\mu^n \in \mathcal{M}(\mathcal{X}^n)$ and a measurable map $\varphi^{\V}$, $\mathcal{X}^n \ni \mathbf{x} \overset{\varphi^{\V}}{\longmapsto} \V^\top \mathbf{x} \in \mathcal{X}^m$, we denote by $\varphi^{\V}_{\#}(\mu^n) \coloneqq \mu^n(\varphi^{\V^{-1}}(\mathbf{x}))$ the pushforward measure $\mu^m \in \mathcal{M}(\mathcal{X}^m)$. 


We now present the standard definition of SCM.

\begin{definition}[SCM, \citealp{pearl2009causality}]\label{def:SCM}
A (Markovian) structural causal model (SCM) $\scm^n$ is a tuple $\langle \myendogenous, \myexogenous, \myfunctional, \zeta^\myexogenous \rangle$, where \emph{(i)} $\myendogenous = \{X_1, \ldots, X_n\}$ is a set of $n$ endogenous random variables; \emph{(ii)} $\myexogenous =\{Z_1,\ldots,Z_n\}$ is a set of $n$ exogenous variables; \emph{(iii)} $\myfunctional$ is a set of $n$ functional assignments such that $X_i=f_i(\parents_i, Z_i)$, $\forall \; i \in [n]$, with $ \parents_i \subseteq \myendogenous \setminus \{ X_i\}$; \emph{(iv)} $\zeta^\myexogenous$ is a product probability measure over independent exogenous variables $\zeta^\myexogenous=\prod_{i \in [n]} \zeta^i$, where $\zeta^i=P(Z_i)$. 
\end{definition}
A Markovian SCM induces a directed acyclic graph (DAG) $\mathcal{G}_{\scm^n}$ where the nodes represent the variables $\myendogenous$ and the edges are determined by the structural functions $\myfunctional$; $ \parents_i$ constitutes then the parent set for $X_i$. Furthermore, we can recursively rewrite the set of structural function $\myfunctional$ as a set of mixing functions $\mymixing$ dependent only on the exogenous variables (cf. \cref{app:CA}). A key feature for studying causality is the possibility of defining interventions on the model:
\begin{definition}[Hard intervention, \citealp{pearl2009causality}]\label{def:intervention}
Given SCM $\scm^n = \langle \myendogenous, \myexogenous, \myfunctional, \zeta^\myexogenous \rangle$, a (hard) intervention $\iota = \operatorname{do}(\myendogenous^{\iota} = \mathbf{x}^{\iota})$, $\myendogenous^{\iota}\subseteq \myendogenous$,
is an operator that generates a new post-intervention SCM $\scm^n_\iota = \langle \myendogenous, \myexogenous, \myfunctional_\iota, \zeta^\myexogenous \rangle$ by replacing each function $f_i$ for $X_i\in\myendogenous^{\iota}$ with the constant $x_i^\iota\in \mathbf{x}^\iota$. 
Graphically, an intervention mutilates $\mathcal{G}_{\mathsf{M}^n}$ by removing all the incoming edges of the variables in $\myendogenous^{\iota}$.
\end{definition}

Given multiple SCMs describing the same system at different levels of granularity, CA provides the definition of an $\alpha$-abstraction map to relate these SCMs:
\begin{definition}[$\abst$-abstraction, \citealp{rischel2020category}]\label{def:abstraction}
Given low-level $\mathsf{M}^\ell$ and high-level $\mathsf{M}^h$ SCMs, an $\abst$-abstraction is a triple $\abst = \langle \Rset, \amap, \alphamap{} \rangle$, where \emph{(i)} $\Rset \subseteq \datalow$ is a subset of relevant variables in $\mathsf{M}^\ell$; \emph{(ii)} $\amap: \Rset \rightarrow \datahigh$ is a surjective function between the relevant variables of $\mathsf{M}^\ell$ and the endogenous variables of $\mathsf{M}^h$; \emph{(iii)} $\alphamap{}: \dom{\Rset} \rightarrow \dom{\datahigh}$ is a modular function $\alphamap{} = \bigotimes_{i\in[n]} \alphamap{X^h_i}$ made up by surjective functions $\alphamap{X^h_i}: \dom{\amap^{-1}(X^h_i)} \rightarrow \dom{X^h_i}$ from the outcome of low-level variables $\amap^{-1}(X^h_i) \in \datalow$ onto outcomes of the high-level variables $X^h_i \in \datahigh$.
\end{definition}
Notice that an $\abst$-abstraction simultaneously maps variables via the function $\amap$ and values through the function $\alphamap{}$. The definition itself does not place any constraint on these functions, although a common requirement in the literature is for the abstraction to satisfy \emph{interventional consistency} \cite{rubenstein2017causal,rischel2020category,beckers2019abstracting}. An important class of such well-behaved abstractions is \emph{constructive linear abstraction}, for which the following properties hold. By constructivity, \emph{(i)} $\abst$ is interventionally consistent; \emph{(ii)} all low-level variables are relevant $\Rset=\datalow$; \emph{(iii)} in addition to the map $\alphamap{}$ between endogenous variables, there exists a map ${\alphamap{}}_U$ between exogenous variables satisfying interventional consistency \cite{beckers2019abstracting,schooltink2024aligning}. By linearity, $\alphamap{} = \V^\top \in \reall^{h \times \ell}$ \cite{massidda2024learningcausalabstractionslinear}. \cref{app:CA} provides formal definitions for interventional consistency, linear and constructive abstraction.
\begin{figure*}[ht]
    \centering
    \includegraphics[width=\textwidth, trim=79 280 93 123, clip]{figures/framework_img.pdf}
    \caption{The pipeline of the \ENDow{} framework 
    %where each component is specified in a given configuration. 
    which yields a downstream task score and a WER score of the transcript set input to the task. The pipeline is executed for several severeties of noising and types of cleaning techniques. %Acoustic noising is applied at $k$ intensities, providing $k+1$ audio versions (including the non-noised version), eventually producing $k+2$ transcript versions (including the source transcript). Applying transcript cleaning reveals the effect of \textit{types} of noise. 
    Resulting scores are plotted on a graph for the analyses, as in, e.g., \autoref{fig_cleaning_graphs}.}
    %The pipeline is executed on $k+1$ intensities of acoustic noising (including the non-noised version), producing $k+2$ scores for the downstream task (including execution on the source transcripts). This process eventually describes the effect of the \textit{intensity} of transcript noise on the downstream task. The process is repeated for $m$ cleaning techniques ($m+1$ when including no cleaning), to analyze the benefit of a cleaning approach and the effect of the \textit{types} of transcript noise.}
    \label{fig_framework}
\end{figure*}
% !TEX root = main.tex
\section{Minimum Cardinality Interdiction Problems}

In this section, we prove our $\Sigma^p_2$-completeness results regarding the minimum cardinality interdiction problem. 
Since we want to prove the theorem simultaneously for multiple problems at once, we require an abstract definition of the interdiction problem.
For this, consider the following definition.

\begin{definition}[Minimum Cardinality Interdiction Problem]
\label{def:min-card-interdiction-pi}
    Let an SSP problem $\Pi = (\I, \U, \sol)$ be given.
    The minimum cardinality interdiction problem associated to $\Pi$ is denoted by \textsc{Min Cardinality Interdiction-$\Pi$} and defined as follows:
    The input is an instance $I \in \I$ together with a number $k \in \N_0$.
    The question is whether
    \[
        \exists B \subseteq \U(I), \ |B| \leq k : \forall S \in \sol(I) : B \cap S \neq \emptyset.
    \]
\end{definition}

For the remainder of the paper, it is helpful to imagine this problem as a game between two players: the \emph{attacker} and the \emph{defender}.
That is, interdiction is an action performed by an attacker (or interdictor), who wishes to select a blocker of few elements to destroy all solutions.
On the other hand, the defender wants to find a solution to the problem after the attacker selected a blocker.
This leads to the following interpretation:
\begin{itemize}
    \item The set $\U(I)$ contains all the elements the attacker is allowed to attack. 
    \item The set $\sol(I)$ contains all the solutions the attacker wants to destroy such that the defender is not able to find any solution.
    For example, this could be the set of all Hamiltonian cycles, the set of all cliques of a certain size, etc.
\end{itemize}
Therefore, the formulation of the base problem as SSP problem $(\I, \U, \sol)$ determines which elements the attacker can attack, which he cannot attack (e.g. edges/vertices of a graph), and what the attacker's goal is.
We note that different formulations $(\I, \U, \sol)$ of the same problem are formally different SSP problems. They might be both SSP-NP-complete independent of each other, but require their own SSP-NP-completeness proof each.
For all the concrete problems studied in this paper, our complexity results hold for the natural choices of $(\I, \U, \sol)$ formally given in \cref{app:sec:problemDefinitions}.
Finally, note that if the base problem is an LOP problem, then by definition $\sol(I)$ is the set of feasible solutions below some threshold specified in the input. 
For example, applying \cref{def:min-card-interdiction-pi} to $\Pi = \textsc{Clique}$ yields the following decision problem: 
\begin{quote}
    \textbf{Problem}: $\textsc{Min Cardinality Interdiction-Clique}$

    \textbf{Input}: Graph $G = (V, E)$, numbers $k, t \in\N_0$

    \textbf{Question}: Does there exist a subset $B \subseteq V$ of size $|B| \leq k$ such that every clique of size at least $t$ shares at least one vertex with $B$? 
\end{quote}

Some more technical details, concerning the subtle differences between different variants of interdiction referenced in the literature as well as concerning the question whether $t$ can be chosen to be optimal are discussed in \cref{sec:different-variants-of-interdiction}.
We now proceed with the main result.
For the complexity analysis of minimum cardinality blocker, we first show the containment in the class $\Sigma^p_2$, if the nominal problem is in NP.

\begin{lemma}
\label{lem:containment}
    Let $\Pi = (\I, \U, \sol)$ be an SSP problem in $NP$, then \textsc{Min Cardinality Interdiction}-$\Pi$ is in $\Sigma^p_2$.
\end{lemma}
\begin{proof}
    We provide a polynomial time algorithm $V$ that verifies a specific solution $y_1, y_2$ of polynomial size for instance $I$ such that
    $$
        I \in L \ \Leftrightarrow \ \exists y_1 \in \{0,1\}^{m_1} \ \forall y_2 \in \{0,1\}^{m_2} : V(I, y_1, y_2) = 1.
    $$
    With the $\exists$-quantified $y_1$, we encode the blocker $B \subseteq \U(I)$.
    The encoding size of $y_1$ is polynomially bounded in the input size of $\Pi$ because $|\U(I)| \leq poly(|x|)$.
    Next, we encode the solution $S \in \sol(I)$ to the nominal problem $\Pi$ using the $\forall$-quantified $y_2$ within polynomial space.
    This is doable because the problem $\Pi$ is in NP (and thus $co\Pi$ is in coNP).
    At last, the verifier $V$ has to verify the correctness of the given solution provided by the $\exists$-quantified $y_1$ and $\forall$-quantified $y_2$.
    Checking whether $|B| \leq t$ and $B \cap S \neq \emptyset$ is trivial and checking whether $S \in \sol(I)$ is clearly in polynomial time because $\Pi$ is in SSP-NP.
    It follows that $\textsc{Min Cardinality Interdiction-}\Pi$ is in $\Sigma^p_2$.
\end{proof}

Next, we show the hardness of minimum cardinality interdiction problems as long as the nominal problem is NP-complete.
For this, we introduce the concept of invulnerability reductions that helps us to grasp the problems in a unified approach.
We describe this concept in the following subsection with the goal to obtain the following main theorem of the paper.

\begin{theorem}
\label{thm:min-card-interdiction}
    The problem \textsc{Min Cardinality interdiction-$\Pi$} is $\Sigma^p_2$-complete for all the following problems:
    independent set,
    clique,
    subset sum,
    knapsack,
    Hamiltonian path/cycle (directed/undirected),
    TSP,
    $k$-directed vertex disjoint paths ($k \geq 2$),
    Steiner tree,
    dominating set,
    set cover,
    hitting set, feedback vertex set,
    feedback arc set,
    uncapacitated facility location,
    $p$-center,
    $p$-median.
\end{theorem}

We remark that the case of satisfiability deserves special attention, which is discussed more thoroughly in \cref{sec:noMeta}.

\subsection{Invulnerability Reduction}

Our proof strategy for each of the problems listed in \Cref{thm:min-card-interdiction} is essentially the same.
In fact, we show that \cref{thm:min-card-interdiction} is actually a consequence of the following, more powerful \emph{meta-theorem}.
This meta-theorem catches the essence of an invulnarability reduction.

\begin{theorem}
\label{thm:meta-theorem}
    Consider an SSP-NP-complete problem $\Pi$.
    If there exists a polynomial-time reduction $g$ which receives as input a tuple $(I, C, k)$ of an instance $I$ of $\Pi$, some set $C \subseteq \U(I)$ and some $k \in \N_0$, and returns instances $I' := g(I, C, k)$ of $\Pi$, such that the following holds: 
     \begin{align*}
         \exists B \subseteq C : |B| \leq k \text{ and } B \cap S \neq \emptyset \ \forall S \in \sol(I) \qquad\qquad\qquad\qquad\qquad\qquad\qquad\qquad\quad\\
         \Leftrightarrow \ \ \exists B' \subseteq \U(I') : |B'| \leq k \text{ and } B' \cap S' \neq \emptyset \ \forall S' \in \sol(I').
     \end{align*}
     Then \textsc{Min Cardinality Interdiction-$\Pi$} is $\Sigma^p_2$-complete. 
\end{theorem}

It would be nice to have \cref{thm:meta-theorem} for all problems in the class SSP-NPc, not only those who admit a funciton $g$ with the properties as described above. However, we give a reasoning in \Cref{sec:noMeta} why such a generalization is not possible.
The rest of this section is devoted to the proof of \cref{thm:min-card-interdiction}.
In \cite{gruene2024completeness} the following more general version of interdiction was considered, where there is a set $C \subseteq \U(I)$ of so-called vulnerable elements.
One can also interpret the set of vulnerable elements $C$ as the elements that have cost of interdiction of $1$ while all other elements $\U(I) \setminus C$ have a cost of interdiction of $\infty$ and a blocker of small costs is sought.
This problem is called the \emph{combinatorial interdiction problem}.

\begin{definition}[Comb. Interdiction Problem, from \cite{gruene2024completeness}.]
    Let an SSP problem $\Pi = (\I, \U, \sol)$ be given.
    We define \textsc{Comb. Interdiction-$\Pi$} as follows:
    The input is an instance $I \in \I$, a number $k \in \N_0$, and a set $C \subseteq \U(I)$. The set $C$ is called the set of vulnerable elements.
    The question is whether
    \[
        \exists B \subseteq C, \ |B| \leq k : \forall S \in \sol(I) : B \cap S \neq \emptyset.
    \]
\end{definition}

It is proven in \cite{gruene2024completeness} that for every problem in SSP-NPc, the combinatorial interdiction problem is $\Sigma^p_2$-complete.
Now, let $\Pi$ be in SSP-NPc and $g$ be a reduction such that
    \begin{align*}
         \exists B \subseteq C : |B| \leq k \text{ and } B \cap S \neq \emptyset \ \forall S \in \sol(I) \qquad\qquad\qquad\qquad\qquad\qquad\qquad\qquad\quad\\
         \Leftrightarrow \ \ \exists B' \subseteq \U(I') : |B'| \leq k \text{ and } B' \cap S' \neq \emptyset \ \forall S' \in \sol(I'),
     \end{align*}
then $g$ is a reduction from \textsc{Comb. Interdiction-$\Pi$} to \textsc{Min Cardinality interdiction-$\Pi$}. This is because the first line is equivalent to the statement that instance $I$ is a yes-instance of \textsc{Comb. Interdiction-$\Pi$}, and the second line is equivalent to the statement that $I'$ is a yes-instance of \textsc{Min Cardinality interdiction-$\Pi$}.
It directly follows that \textsc{Min Cardinality interdiction-$\Pi$} is $\Sigma^p_2$-complete. This completes the proof of \cref{thm:meta-theorem}. 

We remark that while in some sense the proof is rather trivial, we still see a lot of value in explicitly stating a set of easy-to-check sufficient conditions that render some minimum-cardinality interdiction problem $\Sigma^p_2$-complete.

How can one find a function $g$ with the properties as described above? Often times it is possible by employing the following natural idea:
Given an instance of the comb. interdiction problem, let the set $D := \U(I) \setminus C$ be called the \emph{invulnerable} elements. 
For each problem separately we explain that a gadget for the invulnerable elements in $D$ exists, which
intuitively speaking guarantees that an attacker, no matter which $k$ elements of the universe they attack, can never render the elements of $D$ unusable.
On the other hand, we make sure that the \emph{invulnerability gadgets} do not meaningfully change the set of solutions.
The next section gives many examples of such gadgets.
We remark that we are not the first to come up with this natural idea.
For example, Zenklusen \cite{DBLP:journals/dam/Zenklusen10a} used the same idea in the context of matching interdiction.

\subsection{Different Variants of Interdiction}
\label{sec:different-variants-of-interdiction}
In this section, we discuss variants of interdiction problems that can be found in the literature.
For this, we study the relation of our definition of a minimum cardinality interdiction problems and the existing variants.
Additionally, we argue what the implications of the hardness of our minimum cardinality interdiction problems on the other variants are.

\begin{description}
    \item[1. Minimal Blocker Problem.]\hfill
        \begin{description}
            \item[Input] Instance $I$ with universe $U$, blocker cost function $c$, solution cost function $d$, and solution threshold $\tau$
            \item[Task] Find the minimum-cost set $\min_{B \subseteq U} c(B)$ such that for all solutions $S$ with $S \cap B = \emptyset$, we have $d(S) \leq \tau$.
        \end{description}
    \item[2. Full Decision Variant of Interdiction.]\hfill
        \begin{description}
            \item[Input] Instance $I$ with universe $U$, blocker cost function $c$, blocker budget $k$, solution cost function $d$, and solution threshold $\tau$
            \item[Task] Is there a set $B \subseteq U$ with $c(B) \leq k$ such that for all solutions $S$ with $S \cap B = \emptyset$, we have $d(S) \leq \tau$?
        \end{description}
    \item[3. Most Vital Elements Problem.]\hfill
        \begin{description}
            \item[Input] Instance $I$ with universe $U$, blocker cost function $c$, and solution cost function $d$
            \item[Task] Find a set $B \subseteq U$ with $c(B) \leq k$ such that the costs of all solutions $S \cap B = \emptyset$ are maximized, i.e. $\max_{B} \min_{S, S \cap B = \emptyset} d(S)$.
        \end{description}
\end{description}

Our goal is to show that all of the variants from above are at least as hard as our formulation of \emph{minimum cardinality interdiction} (\Cref{def:min-card-interdiction-pi}).
This results in the following theorem.

\begin{theorem}
    Let $\Pi = (\I, \U, \sol)$ be an SSP problem.
    Then the
    Most Vital Elements Problem of $\Pi$ (for all problems $\Pi$ in \Cref{thm:min-card-interdiction}),
    the Minimal Blocker Problem of $\Pi$, and
    the Full Decision Variant of Interdiction of $\Pi$
    are at least as hard to compute as \textsc{Min Cardinality Interdiction}-$\Pi$.
\end{theorem}

The rest of this section is devoted to the proof of this theorem.
In our formulation of minimum cardinality interdiction, a set $B$ is sought, which intersects every solution in the set $\sol$ as given by the corresponding SSP problem.
We now have to distinguish between problems, which are naturally formulated as SSP problems (e.g. Hamiltonian cycle), and SSP problems, which are derived from an LOP problem (e.g. clique).
For natural SSP problems, the solution set $\sol$ consists of all solutions, i.e. there are no feasible solutions outside of $\sol$ due to the missing cost function $d$ on the solution elements.
Thus all of the three variants from above are generalizations of minimum cardinality interdiction:
\begin{enumerate}
    \item The \emph{minimal blocker problem} is the optimization version of the corresponding minimum cardinality interdiction problem.
    \item The \emph{full decision version of interdiction} is a generalization of the corresponding minimum cardinality interdiction problem because the latter assumes to have unit costs in the cost function $c$ for all elements from $U$.
    \item The \emph{most vital element problem} behaves analogous to (2).
\end{enumerate}
For SSP problems that are derived from an LOP problem, basically the same holds, however, with a modified and a technically more intricate argumentation.
Here the solution set is defined by $\sol = \{F \in \F : d(F) \leq t\}$ and we can find a reduction by generalization as follows:
\begin{enumerate}
    \item For \emph{minimal blocker problems}, we can set $\tau := t-1$.
    Then, we again have that the minimal blocker problem is the optimization version of the corresponding minimum cardinality interdiction problem.
    \item For \emph{minimal blocker problems}, we can also set $\tau := t-1$.
    Then, the full decision version is again a generalization of the corresponding minimum cardinality interdiction problem due to the fact that the latter has a unit cost function $c$.
    \item For \emph{most vital element problem}, the situation is more complicated.
    We first observe that the blocker part of $B \subseteq U$ with $c(B) \leq k$ is a generalization of the blocker part in minimum cardinality interdiction.
    The inner part on the nominal problem deserves special attention, though, due to the fact that the most vital element problem maximizes the objective while minimum cardinality interdiction blocks all solutions from the solution set $\sol$.
    We focus on this in the next paragraph.
\end{enumerate}

\par{\bf Reducing Minimum Cardinality Interdiction to Most Vital Elements.}
The concepts of minimum cardinality interdiction and most vital elements coincide if and only of the set $\sol$ contains exactly the optimal solutions, i.e. $\sol = \{F \in \F : d(F) \leq t^\star\}$, where $t^\star$ is optimal (i.e. minimal).
In order to assure that $\sol$ captures exactly the optimal solutions, we need to include this condition into the reduction.
In particular, the SSP reduction $(g, f)$ needs to guarantee that all instances $I$ are mapped to instances $g(I)$ such that all possible solutions are necessarily optimal.
In other words, $t$ is the optimal objective value of the LOP instance $g(I)$, since there are no feasible solutions, whose cost is even smaller than $t$.
We call SSP reductions that fulfill this criterion \emph{tight} and formally define them as follows.

\begin{definition}[Tight SSP reduction]
    Let $\Pi_1$ be an SSP problem and $\Pi_2 = (\I, \U, \F, d, t)$ be an LOP problem.
    Consider an SSP reduction $(g, (f_I)_{I \in \I})$ from $\Pi_1$ to (the SSP problem derived from) $\Pi_2$. 
    The reduction is called tight if for all yes-instances $I_1$ of $\Pi_1$, the corresponding instance $I_2 = g(I_1)$ of $\Pi_2$ with the associated parameter $t := t^{(I_2)}$ and associated cost function $d := d^{(I_2)}$, the following holds:
    \begin{align}
        \set{ F \in \F(I_2) : d(F) \leq t } \neq \emptyset \text{ and } \set{ F \in \F(I_2) : d(F) \leq t - 1} = \emptyset
    \end{align}
\end{definition}

All SSP reductions (to SSP problems derived form LOP problems) that can be found in \cite{gruene2024completeness} fulfill this definition and are thus tight.
Therefore, for all LOP problems (independent set, clique, knapsack, TSP, Steiner tree, dominating set, set cover, hitting set, feedback vertex set, feedback arc set), we obtain that the most vital element problem is at least as hard to compute as the minimum cardinality problem.

\paragraph*{Vertex/Edge Deletion Problems}
In this paper, we are concerned with finding a set $B$ such that $B \cap S \neq \emptyset$ for every solution $S$.
Note that this definition is meaningful even if the nominal problem is not graph-based.
However, in the special case where the nominal problem is graph-based, one could also consider a very related notion which is usually called \emph{vertex deletion problem} or \emph{edge deletion problem}.
Here, the question is how many vertices (edges) need to be deleted from the graph until some desired property is met.
Element deletion problems are well-studied in classical complexity theory for hereditary graph properties \cite{DBLP:journals/jcss/LewisY80} and in parameterized complexity theory for properties expressible by first order formulas \cite{DBLP:conf/mfcs/BannachCT24}.
In the general case, element deletion problems are not the same problem as our problem \textsc{Interdiction-$\Pi$}.
This is because for every set of deleted elements, the underlying instance is changed (vertices/edges are removed, which changes the graph). This is not the case for minimum cardinality interdiction problems as defined in this paper.
Thus, it is not possible to transfer the results of minimum cardinality interdiction directly to element deletion problems.
Albeit for the problems of clique and independent set, the $\Sigma^p_2$-completeness results hold for both minimum cardinality interdiction as well as for vertex deletion interdiction 
because for these problems the deletion of a vertex coincides with not taking this vertex into the solution. 
An analogous statement holds for edge deletions for the problems of directed/undirected Hamiltonian cycle/path, $k$-vertex-disjoint path, and Steiner tree.

% !TEX root = main.tex
\section{Invulnerability Reductions for Various Problems}
\label{sec:invulnerability-gadgets}
In this section, we show that a lot of well-known problems satisfy the assumptions of \cref{thm:meta-theorem}, i.e.\ it is possible to construct so-called invulnerability gadgets for them.
Note that this proves \cref{thm:min-card-interdiction}.
(More precisely, it proves the hardness part and the containment part is analogous to \cite{gruene2024completeness}).
Let in the following always $C \subseteq \U(I)$ denote the set of vulnerable elements, let $\U(I) \setminus C$ denote the set of invulnerable elements, and $k$ denote the budget of the attacker.

\textbf{Clique.}
We have $\U = V$ in this case.
For a given graph $G = (V,E)$, and a set $C \subseteq V$, we explain how to make $V \setminus C$ invulnerable.
We obtain a graph $G'$ from $G$ by replacing every vertex $v \in V \setminus C$ with an independent set $X_v$ of size $|X_v| = k+1$. 
For a vertex $v \in C$, we define $X_v := \set{v}$.
For all edges $uv$ in $G$, the new graph $G'$ contains the complete bipartite graph between $X_u$ and $X_v$.
Note that every clique of $G'$ contains at most one vertex from every set $X_v$. Hence the size of a maximum clique is the same in $G$ and $G'$. 
Since for $v \in V \setminus C$, we have $|X_v| = k+1$ and all vertices in $X_v$ have the same neighborhood, the attacker is not able to attack all vertices of $X_v$ at once because its budget of $k$ is too small.
Hence $v$ has been made \enquote{invulnerable}.
Furthermore, for every clique in $G$, we find a corresponding clique in $G'$ that contains at most one vertex from each set $X_v$. 
Together, this implies that an attacker can find a set $B' \subseteq V(G')$ of size $|B'| \leq k$ interdicting all maximum cliques in $G'$ if and only the attacker can find a set $B \subseteq C$ of size $|B| \leq k$ interdicting all maximum cliques of $G$, i.e.\ the assumptions of \cref{thm:meta-theorem} are met.

\textbf{Independent Set.} Analogous to clique in the complement graph.

\textbf{Dominating Set.} We have $\U = V$ in this case. 
To make a vertex $v \in V \setminus C$ invulnerable, we use the same construction as for the clique problem, with the only difference that $X_v$ is a clique instead of an independent set. 
Every optimal dominating set takes at most one vertex from each set $X_v$, but all $k+1$ vertices inside $X_v$ are equivalent. More precisely, they have the same (closed) neighborhood. 
This means for an invulnerable $v \in V \setminus C$, an attacker can not attack all $k+1$ vertices of $X_v$ simultaneously. 
Furthermore, it is easily seen that on the vulnerable vertices, the attacker interdicts all optimal dominating sets in the old graph if and only if the analogous attack interdicts all optimal dominating sets in the new graph.

\textbf{Hitting Set.} In this case, we have some universe $\U$, sets $Y_1,\dots,Y_t \subseteq \U$, and the problem is to find a minimal hitting set $X \subseteq \U$ hitting all the sets $Y_j$, $j =1,\dots,t$. 
To make an element $e \in \U$ invulnerable, simply delete it and replace it by $k + 1$ copies.
We modify the sets such that every set $Y_j$ that contained $e$ now contains the $k+1$ copies of $e$ instead. 
It is clear that all the copies of $e$ hit the same sets as $e$ (i.e.\ taking multiple copies into the hitting set does not offer any advantage).
Furthermore, it is not possible for the attacker to attack all $k+1$ copies simultaneously.
By an argument analogous to the above paragraphs, we are done.

\textbf{Set cover.} We have a ground set $E$, and a family $\mathcal{F}$ of sets $S_1, \dots, S_n \subseteq E$ over the ground set. 
We let $\U := \fromto{1}{n}$ and the goal is to pick a subset $I \subseteq \U$ of the indices such that $\bigcup_{i \in I} S_i = E$.
The attacker can attack up to $k$ of the indices $i \in I$ to forbid the corresponding sets from being picked.
We can make some index $i \in \U$ invulnerable, by simply duplicating the set $S_i$ a total amount of $k+1$ times.

Note that this satisfies the assumptions of \cref{thm:meta-theorem}, but modifies the family $\mathcal{F}$ such that the same set could appear multiple times in the family.
Alternatively, our construction can be adjusted such that this is avoided.
For this, we introduce $k+1$ new elements $e_1, \dots, e_{k+1}$ and $k+2$ new elements $f_1,\dots, f_{k+2}$ to the ground set $E$.
For each invulnerable index $i \in \fromto{1}{n} \setminus C$, we substitute $S_i$ by the $k+1$ sets $S_i^{(j)} = S_i \cup \{e_j\}$ for $j=1,\dots,k+1$.
Furthermore, we introduce $k+2$ new sets $S'_j := \fromto{e_1}{e_{k+1}} \cup \fromto{f_1}{f_{k+2}} \setminus \set{f_j}$ for $j =1, \dots, k+2$. This completes the description of the instance.
Note that the following holds: The elements $\fromto{f_1}{f_{k+2}}$ are covered by a set cover, if and only if it contains at least two sets of the form $S'_j$. 
Assuming this condition is true, all the elements $\fromto{e_1}{e_{k+1}}$ are already covered.
Hence all the different copies $S_i^{(j)}$ for $j=1,\dots,k+1$ are essentially equivalent.
Thus the attacker can not meaningfully attack all these copies simultaneously.
Note that the attacker can also not meaningfully attack the sets $S'_j$, since no matter which $k$ of them are attacked, 2 of them always remain.


\textbf{Steiner tree.} We have $\U = E$ in this case.
To make an edge $uv \in E \setminus C$ invulnerable, we replace it with $k+1$ parallel subdivided edges, i.e.\ we introduce vertices $w_1, \dots, w_{k+1}$ and edges $uw_i$ and $w_iv$ for $i =1,\dots, k+1$.
Every vulnerable edge $uv$ is replaced with only a single subdivided edge, i.e.\ a vertex $w$ and edges $uw, wv$.
It is clear that the number of edges of a minimum Steiner tree in the new instance is exactly two times as big as before, and the edge $uv$ has become effectively invulnerable.  

\textbf{Two vertex-disjoint path.}
We have $\U = A$ in this case.
The gadget is the same as for Steiner tree, except that the construction is directed, i.e. the arc $(u,v)$ is replaced either by the arcs $(u,w_i), (w_i, v)$ for $i=1,\dots,k+1$ (invulnerable case) or by the two arcs $(u,w), (w,v)$ (vulnerable case).
Since the paths in this problem have to be vertex disjoint, adding additional subdivided arcs between two existing vertices does not produce additional solutions because traveling from $u$ to $v$ renders all other paths from $u$ to $v$ unusable.

\textbf{Feedback arc set.} We have $\U = A$ in this case. 
Note that making some arc $a = (u,v) \in A \setminus C$ invulnerable means to ensure that it can be used in a minimal feedback arc set, no matter which $k$ arcs the attacker chooses.
This can be achieved the following way: Subdivide $a$ into $k + 1$ arcs. 
Clearly, the set of cycles in the new graph stays essentially the same. 
Furthermore, the attacker cannot block all $k + 1$ arcs from being chosen for the solution.
Choosing one of the subdivided pieces of $a$ in the new instance has the same effect as choosing $e$ in the old instance.

\textbf{Feedback vertex set.}
We have $\U = V$ in this case.
To make a vertex $v \in V \setminus C$ invulnerable, we split it into two vertices $v_\text{in}$ and $v_\text{out}$, 
put all incoming edges of the old vertex $v$ to $v_\text{in}$, 
put all outgoing edges of the old vertex $v$ to $v_\text{out}$,
and connect $v_\text{in}$ to $v_\text{out}$ with a directed path $P_v$ on $k+1$ vertices.
Note that in the new instance, a directed cycle uses one vertex of $P_v$ if and only if the cycle uses all vertices of $P_v$ if and only if a corresponding cycle in the old instance uses $v$.
By an analogous to argument to the feedback arc set case, we are done.

\textbf{Uncapacitated facility location.} We have $\U = J$ in this case, where $J$ is the set of sites for potential facilities. The attacker selects facility sites and forbids the decision maker to build a facility there. 
To make a facility site $j \in J \setminus C$ invulnerable, we can simply delete the site and replace it with $k+1$ identical sites, i.e.\ sites which have the same facility opening cost and service cost functions as the original facility $j$. 
Clearly, this way the attacker can not stop one of the equivalent facilities to be opened. On the other hand, since the facilities are identical (and uncapacitated), 
the decision maker has no advantage from opening two identical copies of the same facility.
Hence the new instance is identical to the old instance, with the only difference that facility site $j$ is invulnerable.

\textbf{$p$-median, $p$-center.} The difference between the facility location problem and the $p$-center and $p$-median problem is that in the latter two, there are no facility opening costs, at most $p$ facilities are allowed to be opened, 
and the service costs in the $p$-center problem are calculated using a minimum, and in the $p$-median problem they are calculated using the sum. 
All of these differences do not affect the argument from above, i.e.\ one can still make a facility site invulnerable by creating $k+1$ identical facilities. Hence the same argument holds.

\textbf{Subset Sum.}
We have $\U = \fromto{1}{n}$ and are given numbers $a_1, \dots, a_n \in \N$ and a target value $T$. The question is whether there exists $S \subseteq U$ with $\sum_{i \in S} a_i = T$. 
Consider some index $i \in \U \setminus C$. In order to make the index $i$ invulnerable, the first idea is to copy the number $a_i$ a total amount of $k+1$ times. 
But there is a problem with this construction -- if we do this, then the same number $a_i$ could be picked multiple times, which is not allowed in the original instance.
We need an additional gadget to make sure that $a_i$ gets used at most once for each $i$. This can be done the following way: 
The new instance contains the following numbers: Choose some number $B > 2k+2$ as a basis. 
For each $i \in C$, it contains the single number $B^{n(k+1)}a_i$. 
For each $i \in \fromto{1}{n} \setminus C$, it contains the $k+1$ distinct numbers  $c_i^{(j)} := B^{n(k+1)}a_i + B^{(i-1)(k+1) + j}$ for $j = 0,\dots, k$ as well as the $k+1$ distinct numbers $d_i^{(j)} := \sum_{\ell = 0,\ell \neq j}^k B^{(i-1)(k+1) + \ell}$ for $j = 0,\dots, k$ and the $k+1$ distinct numbers $e_i^{(j)} := B^{(i-1)(k+1) + j}$ for $j = 0,\dots, k$. We call $d_i^{(j)}$ and $e_i^{(j)}$ the helper numbers.
The new instance contains a total of $|C| + 3(k+1)(n - |C|)$ numbers. The new target value is 
$$
T' := B^{n(k+1)}T \ + \sum_{i \in \fromto{1}{n} \setminus C} \quad \sum_{\ell = 0}^k B^{(i-1)(k+1) + \ell}.
$$
Note that this has the following effect: 
Consider the representation of all involved numbers in base $B$. Let us call the digits $0$ up to $n(k+1) - 1$ the lower positions. 
Note that in the lower positions there can never be any carry, since for every lower position, all involved numbers have either a zero or one in that position and less than $B$ numbers have a one in the same place.
Due to that fact, in the lower positions the target $T'$ is reached if and only if for every $i \in \fromto{1}{n} \setminus C$, the corresponding \enquote{bitmask} is filled out (by this, we mean the positions $(i-1)(k+1)$ up to $i(k+1) - 1$).
This is achieved if and only if for some $j \in \fromto{0}{k}$ both the values $c^{(j)}_i$ and $d^{(j)}_i$ or both the values $d^{(j)}_i$ and $e^{(j)}_i$ are picked. In particular, at most one of the $k+1$ values $c^{(j)}_i$ for $j=0,\dots,k$ are picked.
In the upper positions, the target $T'$ is reached if and only if the corresponding choice in the old instance meets the target $T$.

Consider an attack of $k+1$ numbers by the attacker. For each $i \in \fromto{1}{n} \setminus C$ it holds that there exists a $j$ such that both $c_i^{(j)}$ and $d_i^{(j)}$ are not attacked. Likewise there exists a $j'$ such that both $d_i^{(j')}$ and $e_i^{(j')}$ are not attacked. 
That means that if $i$ is an invulnerable index, then no matter which $k+1$ values of  $c_i^{(j)}$, $d_i^{(j)}$ and  $e_i^{(j)}$ are attacked, 
a correct solution of subset sum will take for some $j$ either both $c_i^{(j)}$ and $d_i^{(j)}$ (which corresponds to taking $a_i$ in the original instance) 
or take both $d_i^{(j)}$ and $e_i^{(j)}$ (which corresponds to not taking $a_i$ in the original instance).
It follows that it is possible to block the new instance by attacking $k+1$ values if and only if it is possible to block the old instance by attacking $k+1$ of the vulnerable values. 
This was to show.
Finally, if the old numbers $a_1, \dots, a_n$ are pairwise distinct, the new numbers are as well. Hence the interdiction problem for subset sum is $\Sigma^p_2$-complete, even if all involved numbers are distinct.

\textbf{Knapsack.} The knapsack problem can be seen as a more general version of the subset sum problem, by creating for each $i$ 
from the subset sum instance a knapsack item with both profit $p_i = a_i$ and weight $w_i = a_i$, and setting both the weight and profit threshold to $T$.
Hence the $\Sigma^p_2$-completeness of \textsc{Min Cardinality Interdiction-Knapsack} follows as a consequence of the $\Sigma^p_2$-completeness of \textsc{Min Cardinality Interdiction-Subset Sum}. This holds even if all the involved knapsack items are distinct.

% !TEX root = main.tex
\subsection{An Invulnerability Reduction for Hamiltonian Cycle}
The invulnerability gadget for Hamiltonian cycle is the most involved of all our constructions, 
hence we devote a subsection to it. 
The main result in this section is that the minimum cardinality interdiction problem is $\Sigma^p_2$-complete for the nominal problems of both directed and undirected Hamiltonian cycle and path, as well as the TSP.

We present our reduction for the case of undirected Hamiltonian cycle and then argue how it can be adapted to the other cases. The main idea is to consider as an intermediate step only 3-regular graphs $G = (V, E)$, and then for a subset $C \subseteq E$ show how $E \setminus C$ can be made invulnerable. To this end, consider the SSP problem

\begin{description}
    \item[]\textsc{3Reg Ham}\hfill\\
    \textbf{Instances:} Undirected, 3-regular Graph $G = (V, E)$\\
    \textbf{Universe:} $\U := E$.\\
    \textbf{Solution set:} The set of all Hamiltonian cycles in $G$.
\end{description}

Recall that it is shown in \cite{gruene2024completeness} that \textsc{Hamiltonian Cycle} is SSP-NP-complete. We now require the stronger statement

\begin{lemma}
\label{lem:3-reg-ham-ssp}
    \textsc{3Reg Ham} is SSP-NP-complete.
\end{lemma}
\begin{proof}
    Garey, Johnson \& Tarjan \cite{DBLP:journals/siamcomp/GareyJT76} give a reduction from \textsc{3Sat} to  \textsc{3Reg Ham}, such that for every variable $x_i$ in the \textsc{3Sat} instance the graph $G$ has two distinct edges $e(x_i)$ and $e(\overline x_i)$ (compare Figure 7 in \cite{DBLP:journals/siamcomp/GareyJT76}). Let $E' := \bigcup_i \set{e(x_i), e(\overline x_i)}$ be the set of all these edges. For some assignment $\alpha$ of the \textsc{3Sat} variables, we say that $\alpha$ corresponds to the edge set $E_\alpha$ defined by $\set{e(x_i) : \alpha(x_i) = 1} \cup \set{e(\overline x_i) : \alpha(x_i) = 0}$.
    Garey, Johnson \& Tarjan show that there is a bijection between satisfying assignments and edge sets $E'' \subseteq E'$ that can be subset of a Hamiltonian cycle. More formally: 1.) For every satisfying assignment $\alpha$, 
    if one considers the set $E_\alpha \subseteq E'$ of edges corresponding to that assignment, 
    there exists a Hamiltonian cycle $H$ extending $E_\alpha$, i.e.\ $H \cap E' = E_\alpha$. 
    2.) For every Hamiltonian cycle $H$, we have that $H \cap E'$ equals $E_\alpha$ for some satisfying assignment $\alpha$.
    In total, 1.) and 2.) together show that the reduction in \cite{DBLP:journals/siamcomp/GareyJT76} is an SSP-reduction. (By defining $f(x_i) := e(x_i), f(\overline x_i) := e(\overline x_i)$.)
\end{proof}

We remark that it follows from \cite{akiyama1980np,DBLP:journals/siamcomp/GareyJT76} by the same argument that the problem is even SSP-NP-complete if restricted to 3-regular, bipartite, planar, 2-connected graphs. However, for our arguments it suffices to consider 3-regular graphs.

Consider now an instance of \textsc{3Reg Ham}, i.e. a 3-regular undirected graph  $G = (V, E)$. Let $C \subseteq E$ be a subset of the edges and $k \in \N_0$ the attacker's budget. We call $C$ the vulnerable edges. Let $D := E(G) \setminus C$.
In the remainder of this section we describe and prove a construction how to make the edges in $D$ invulnerable.
We quickly sketch the main idea: To make an edge $e = ab$ invulnerable, we enlarge it by replacing it with a large clique $W'_{ab}$ making sure that $e$ can be traversed no matter which $k$ edges inside $W'_{ab}$  are attacked. 
We also blow up each vertex $a$ of the original graph into a clique $W_a$.
However, this introduces new vertices into the instance, and we need to make sure that a Hamiltonian cycle can always trivially visit all the new vertices.
At the same time however, it should still hold that a Hamiltonian cycle in the new graph should be able to enter and exit these new objects $W_a$ and $W'_{ab}$ at most once, since otherwise a corresponding cycle in the old graph $G$ would visit edges or vertices twice, which is of course forbidden.
We achieve this by associating to each edge $e = ab$ a star of edges $F_{ab}$ and argue that a Hamiltonian cycle can use (essentially) at most one edge of each star $F_{ab}$. 
Furthermore, we will show that the fact that $G$ is 3-regular implies that each clique $W_a$ can be traversed (essentially) only once.

We are ready to begin with the construction.
First, let the directed graph $\overrightarrow{G}$ result from $G$ by orienting its edges arbitrarily and $k$ be the budget of the attacker.
We construct an undirected graph $G' = (V', E')$ from $\overrightarrow{G}$ as follows: 
Let $n := |V(G)|$.
For each vertex $a \in V(\overrightarrow{G})$, let $d_a$ be the out-degree of $a$, and let $W_a$ be a set of $2d_a + 4k + 1$ vertices.
For each invulnerable edge $ab \in D$ in the old graph, let  $W'_{ab}$ be a set of $4k$ vertices.
The vertex set $V(G')$ of the new graph $G'$ is then defined by
\[ V(G') = \bigcup_{a \in V} W_a \cup \bigcup_{ab \in D} W'_{ab}.\]
\begin{figure}
    \centering
    \includegraphics[scale=1.0]{src/img/ham-cycle-invulnerability.pdf}
    \caption{Invulnerability gagdet for Hamiltonian cycle which makes the edge $ab$ invulnerable while the edge $ac$ remains vulnerable.}
    \label{fig:ham-cycle-invulnerability}
\end{figure}
We further partition $W_a$ into three disjoint parts $W_a = X_a \cup Y_a \cup \set{z_a}$ of size $|X_a| = 2d_a$ and $|Y_a| = 4k$ and $|\set{z_a}| = 1$.
We denote the vertices of $X_a$ by $x^{(a)}_1, \dots, x^{(a)}_{2d_a}$.
The edges of $G'$ are defined as follows:
First, we let $W_a$ be a clique for all $v \in V$.
Second, for each vertex $a \in V$ in $\overrightarrow{G}$, let $e_1, \dots, e_{d_a}$ be its outgoing edges.
For each $i = 1, \dots, d_a$, consider the $i$-th outgoing edge $e_i = (a, b)$ of $a$, where $b$ is the corresponding neighbor. 
If $e_i \in C$, i.e. $e_i$ is vulnerable, then $G'$ contains simply the single edge $x^{(a)}_{2i-1}z_b$. 
In the other case, i.e.\ $e_i \in D$ is invulnerable, then $G'$ contains an invulnerability gadget as depicted in \cref{fig:ham-cycle-invulnerability} induced on the vertices $\set{x^{(a)}_{2i-1}, x^{(a)}_{2i}} \cup W'_{ab} \cup \set{z_b}$.
The invulnerability gadget consists out of a clique on the vertex set $\set{x^{(a)}_{2i-1}, x^{(a)}_{2i}} \cup W'_{ab}$, together with all edges from the set $W'_{ab}$ to the vertex $z_b$, i.e.\ a star centered at $z_b$ that has $W'_{ab}$ as its leaves.
Let $F_{ab}$ denote this star. 
Finally, for all vulnerable edges $ab \in D$, we also define $F_{ab}$ to be the single edge $x^{(a)}_{2i-1}z_b$ that connects $W_a$ to $W_b$.
This can be interpreted as a trivial star centered at $z_b$ with only one leaf.
This completes the description of $G'$.

The overall idea of this construction is that the cliques of $W_a$ cannot be attacked because they have at least $k$ vertices.
Thus it is always possible to find a path visiting all vertices of $W_a$.
Additionally, a star $F_{ab}$ of size larger than $k$ makes the edge $ab \in E$ invulnerable because at most $k$ many of the edges can be attacked.
Thus there is always the possibility to travel over one edge of $F_{ab}$ which corresponds to using edge $ab$ in the original graph.
On the other hand, since every edge of the star is connected to the same vertex $z_b$, we have that the star $F_{ab}$ can be used (essentially) exactly once.
Thus only the stars of size one (which correspond to the vulnerable edges) are attackable.
We now have everything that we need to prove our main result of this section. 

\begin{theorem}
\label{thm:ham-cycle-interdiction}
Minimum cardinality interdiction for \textsc{Undirected Hamiltonian Cycle} is $\Sigma^p_2$-complete.
\end{theorem}
\begin{proof}
Due to \cite{gruene2024completeness}, and \cref{lem:3-reg-ham-ssp}, we have that \textsc{Comb. Interdiction-3Reg Ham} is $\Sigma^p_2$-complete.
We claim that the construction of $G'$ yields a correct reduction from \textsc{Comb. Interdiction-3Reg Ham} to \textsc{Min Cardinality Interdiction-HamCycle}.
Indeed, the following two \cref{lem:hamcycle-if,lem:hamcycle-only-if} show that yes-instances of one problem get transformed into yes-instances of the other problem. 
\end{proof}
We remark that the 3-regularity of the graph is not maintained by the reduction.
(Indeed, an argument similar to the arguments given later in \cref{sec:noMeta} shows that the interdiction problem for Hamiltonian cycle restriced to only 3-regular graphs is likely not $\Sigma^p_2$-complete).

\begin{lemma}
\label{lem:hamcycle-if}
    If there exists $B' \subseteq E'$ of size $|B'| \leq k$, such that $G' - B'$ has no Hamiltonian cycle, then there is $B \subseteq C$ of size $|B| \leq k$ such that $G - B$ has no Hamiltonian cycle.
\end{lemma}
\begin{proof}
    Proof by contraposition. Assume that for all $B \subseteq C$ with $|B| \leq k$ the graph $G - B$ has a Hamiltonian cycle $H$.
    Given some $B' \subseteq E'$ with $|B'| \leq k$, we have to show that the graph $G' - B'$ has a Hamiltonian cycle. Let $B \subseteq C$ be the set of vulnerable edges in $G$ whose copies in $G'$ are attacked by $B'$ (i.e. $B = \set{ab \in C : F_{ab} \in B'}$). 
    Since $B \subseteq C$ and $|B| \leq k$, by assumption $G - B$ has a Hamiltonian cycle $H$. We want to modify $H$ to a Hamiltonian cycle of $H'$ of $G' - B'$. 
    The basic idea is to follow globally the same route as $H$. However, we have to pay attention, because we are not allowed to use edges from $B'$.
    For each vertex in $G'$ call it \emph{attacked}, if at least one of its incident edges are attacked by $B'$, and call it \emph{free} otherwise. 
    Note that since $|B'| \leq k$ and $|Y_a| = 4k$ and $|W'_{ab}| = 4k$ for $a \in V, ab \in E$, the vertex sets $Y_a$ and $W'_{ab}$ have at least $2k$ free vertices. Free vertices are good for the following reason: 
    Whenever we plan to go from some vertex $u$ to $v$ in $G'$, but we cannot because $uv \in B'$ was attacked, then we can instead choose any free vertex $f$ and go the route $u,f,v$ instead.
    Now the plan is that $H'$ will roughly employ the following strategy: Follow globally the same path in $G'$ like $H$ does in $G$. 
    Whenever $H'$ enters some new set $W_a$ for the first time, then we visit all the sets $W'_{ab}$ for all out-neighbors $b$ of $a$ in $\overrightarrow{G}$.
    Note that for such $b$, the set $W'_{ab}$ has two adjacent vertices with $W_a$ (we use these two vertices to enter and leave), and we collect all the vertices of $W'_{ab}$. 
    Here, we prioritize to visit first the attacked vertices of $W'_{ab}$ and then the remaining vertices of $W'_{ab}$. 
    After that, we collect all remaining vertices of $W_a$ (again prioritizing the attacked vertices first) before leaving $W_a$. (If the path on which we are leaving $W_a$ corresponds to an invulnerable edge $ab$ in $G$, we also collect all of $W'_{ab}$ in the process of leaving $W_a$.)

    Note that this plan might at first not be feasible, because it requires going over some edge $e' \in B'$. However note that, since $H$ does not use any edge of $B$, for every such edge $e'$ there are always at least $2k$ free vertices that are adjacent to both endpoints of $e'$.
    Hence it is possible to \enquote{repair} such an edge $e'$ by rerouting over some free vertex instead (and later skip over this free vertex). 
    Since there are at most $k$ defects, and there are at least $2k$ free vertices available at the end of traversing every set $W_a$ or $W'_{ab}$, all defects can be repaired. 
    Hence we can modify $H'$ to be a Hamiltonian cycle of $G' - B'$, which was to show.
\end{proof}

\begin{lemma}
\label{lem:hamcycle-only-if}
    If there exists $B \subseteq C$ of size $|B| \leq k$, such that $G - B$ has no Hamiltonian cycle, then there is $B' \subseteq E'$ of size $|B'| \leq k$ such that $G' - B'$ has no Hamiltonian cycle.
\end{lemma}
\begin{proof}
    Proof by contraposition. Assume that for all $B' \subseteq E'$ of size $|B'| \leq k$ the graph $G' - B'$ has a Hamiltonian cycle. 
    Given some $B \subseteq C$ with $|B| \leq k$, we have to show that the graph $G - B$ has a Hamiltonian cycle. 
    Let $B'$ be the trivial stars in $G'$ corresponding to the edges in $B$ (i.e.\ $B' = \set{F_{ab} : ab \in B}$). 
    Since $|B'| \leq k$, by assumption there is a Hamiltonian cycle $H'$ in $G' - B'$.  
    Consider the set $F := \bigcup_{ab \in E}E(F_{ab})$, i.e.\ the union of the edge sets of all the stars, trivial or not.
    We claim that w.l.o.g.\ we can assume that $|H' \cap F_{ab}| \leq 1$ for all $ab \in E$.
    Indeed, the graph $G' - F$ consists out of multiple connected components. 
    Each of these components contains exactly one set of the form $W_a$, and is incident to exactly three sets of the form $F_e$ in $G'$ (where $e$ is an edge that is either incoming to or outgoing from $a$ in $\overrightarrow{G}$).
    Suppose for some $F_{ab}$ we have $|H' \cap F_{ab}| \geq 2$.
    Since $F_{ab}$ is a star connected to a single vertex $z_b$, we have $|H' \cap F_{ab}| = 2$.
    Consider the edge $ab$ such that $F_{ab}$ connects the vertex $z_b$ with $W'_{ab}$. 
    By the observation about $G' - F$, the following is true about $H'$: 
    It enters $W'_{ab}$ in one of the two vertices attached to $X_a$, then traverses exactly all of $W'_{ab} \cup \set{z_b}$, 
    then leaves through the other of the two vertices attached to $X_a$, and at a later point returns to collect all vertices of $X_b \setminus \set{z_b}$.
    However, by the same observation as in \cref{lem:hamcycle-if}, if we define a free vertex to be a vertex not adjacent to any edge in $B'$, then both $W'_{ab}$ and $W_b$ have $2k$ free vertices.
    Hence we can modify $H'$ such that $H' \cap F_{ab} = \emptyset$.
    We thus assume that $|H' \cap F_{ab}| \leq 1$ for all $ab \in E$.
    Consider again the graph $G' - F$.
    Since each of its component is adjacent to three sets $F_e$ and $|H' \cap F_e| \leq 1$, we conclude that $H'$ uses exactly two of these three sets $F_e$.
    But this implies that $H'$ enters and exits each of the components of $G'- F$ only once and collects all of its vertices in the process.
    This implies that $H'$ globally follows the same path as some Hamiltonian cycle $H$ of $G$.
    Since $H' \subseteq G' - B'$, we conclude $H \subseteq G - B$.
    This was to show.
\end{proof}


These two lemmas together prove \cref{thm:ham-cycle-interdiction}.
We would now like to prove $\Sigma^p_2$-completeness also for Hamiltonian cycle interdiction of directed graphs.
Note that this does not follow from a trivial argument: 
Even though one can transform an undirected graph into a directed one, by substituting every undirected edge $uv$ by two directed edges $(u,v), (v,u)$, there is a problem: In the new setting the interdictor needs two attacks to separate $u,v$, while in the old setting the attacker only needs one. 

Still, the above proof can be adapted to the case of directed Hamiltonian cycle the following way: 
We start with \cite{plesn1979np}, which provides a SSP reduction to prove that the Hamiltonian cycle problem is NP-complete even in directed graphs $G$ such that $\text{indegree}(v) + \text{outdegree}(v) \leq 3$ for every vertex $v$, and such that for all pairs $u,v$ of $G$ at most one of the two edges $(u,v)$ and $(v,u)$ is present. 
Given a directed graph $G$, we then repeat the same construction as before, 
with the difference that we can start directly with the directed graph $G$ instead of obtaining an orientation $\overrightarrow{G}$ first.
This way, we can obtain an undirected graph $G'$ in the same way as before. 
In a final step, we turn $G'$ into a directed graph by substituting every undirected edge $uv$ by a pair of two edges $(u,v), (v,u)$. We perform this substitution for every edge of $G'$ with the exception of the edges that are part of some star $F_{ab}$.
Instead, for each star $F_{ab}$, we orient the edges of $F_{ab}$ the same way as the original directed edge of $G$ between $a,b$.
It can be shown that all the arguments from the above construction still hold. 
Hence the minimum cardinality interdiction problem is $\Sigma^p_2$-complete also for directed graphs.


If one is interested in Hamiltonian paths instead of cycles, a similar modification is possible.
Inspecting the proof of \cite{DBLP:journals/siamcomp/GareyJT76} (of \cite{plesn1979np}, respectively) more closely, we find that in both constructions the graph $G$ contains some edge $e = uv$ (some edge $e = (u,v)$, respectively) such that every Hamiltonian cycle uses $e$. 
We can delete $e$ and identify the vertices $s,t$ with the endpoints of $e$.
Then a Hamiltonian $s$-$t$-path in the new graph corresponds to a Hamiltonian cycle in the old graph and vice versa.
Note that this does not increase the degree of the graph.
The rest of the proof proceeds in the same manner, both in the undirected and directed case.
Finally, the proof can also easily be adapted to the TSP by a standard reduction of undirected Hamiltonian cycle to the TSP 
(a graph $G$ is transformed into a TSP instance on the complete graph where the costs obey $c(uv) = 1$ if $ev \in E(G)$ and $c(uv) = n+1$ if $uv \not\in E(G)$).
In conclusion, we have proven that the minimum cardinality interdiction problem is $\Sigma^p_2$-complete for the directed/undirected Hamiltonian path/cycle problem and the TSP. 

% !TEX root = main.tex
\section{Cases where the meta-theorem does not apply}\label{sec:noMeta}

It would be nice to establish a meta-theorem providing $\Sigma^p_2$-completeness of the minimum cardinality interdiction version of all nominal problems, which are SSP-NP-complete, instead of only those problems that admit an additional function $g$ with properties as stated in \Cref{thm:meta-theorem}.
However, we show in this section that this is not possible.
More precisely, we provide a lemma that guarantees that the minimum cardinality version of a problem in SSP-NP is in coNP.
Therefore, under the usual complexity-theoretic assumption NP $\neq \Sigma^p_2$, the interdiction problem is not $\Sigma^p_2$-complete.

In order to provide an intuition under which circumstances a minimum cardinality interdiction problem resides in the class coNP, we examine the vertex cover problem.
In a vertex cover, every edge $uv$ needs to be covered by at least one of the two incident vertices $u$ and $v$.
This, however, gives the attacker the opportunity to attack both $u$ and $v$ such that the edge $uv$ can never be covered.
Therefore, an attacker budget of at least $2$ results in a clear Yes-instance.
On the other hand, if the attacker budget if at most $1$, we can provide a certificate for No-instances.
We can summarize this observation in the following lemma.

\begin{lemma}\label{lem:minCardInCoNP}
    Let $\Pi = (\I, \U, \sol)$ be an SSP problem.
    If in each instance $I \in \I$ there is a subset $U' \subseteq \U(I)$ of constant size, i.e. $|U'| = O(1)$, such that for $U' \cap S \neq \emptyset$ for all $S \in \sol(I)$, then \textsc{Min Cardinality Interdiction-$\Pi$} is contained in coNP.
\end{lemma}
\begin{proof}
    Let $k$ be the interdiction budget.
    If $|U'| \leq k$, then the interdictor is able to block the whole set $U'$.
    By definition of $U'$, there is no solution $S \in \sol(I)$ such that $U' \cap S \neq \emptyset$ and thus the interdictor has a winning strategy.
    If on the other hand $k < |U'| = O(1)$, then there is a polynomially sized certificate encoding a winning strategy of the defender, i.e. a certificate for a No-instance of the problem.
    For this, we first encode the ${|\U(I)| \choose k} = |\U(I)|^{O(1)}$ possible blockers $B' \subseteq \U(I)$ and then the solution $S \in \mathcal S(I)$ such that $S \cap B' \neq \emptyset$ for all $B' \subseteq \U(I)$.
    It is possible to efficiently verify the solution by checking whether there is a solution $S \in \mathcal S(I)$ such that $S \cap B' \neq \emptyset$ for all $B' \subseteq \U(I)$ holds because the nominal problem $\Pi$ is in NP.
    It follows that the problem lies in coNP.
\end{proof}

Consider the different variants of interdiction problems introduced in \cref{sec:different-variants-of-interdiction}.
Since they are more general, \cref{lem:minCardInCoNP} does not immediately imply that those variants are contained in coNP.
However, if for each instance the stronger condition $U' \cap F \neq \emptyset$ for all feasible solutions $F \in \F(I)$ and for some constant size set $U' \subseteq \U(I)$ holds, then the \emph{full decision variant of interdiction} and the \emph{most vital element problem} are contained in coNP.
Besides the containment in coNP, we can also derive the following corollary pinpointing the complexity of minimum cardinality interdiction problems whose nominal problem is in SSP-NP.

\begin{corollary}
    Let $\Pi = (\I, \U, \sol)$ be an SSP-NP-complete problem.
    If in each instance $I \in \I$ there is a subset $U' \subseteq \U(I)$ of constant size, i.e. $|U'| = O(1)$, such that for $U' \cap S \neq \emptyset$ for all $S \in \sol(I)$, then \textsc{Min Cardinality Interdiction-$\Pi$} is coNP-complete.
\end{corollary}
\begin{proof}
    There is a reduction by restriction:
    Setting the interdiction budget $k = 0$ results in the corresponding co-problem co$\Pi$ of the nominal problem $\Pi$.
\end{proof}

\subsection{Applying the Lemma to Various Problems}

In this section, we apply \cref{lem:minCardInCoNP} to the problems mentioned earlier in this paper.
Some of the problems are affected in their original general form, e.g. vertex cover or satisfiability, while for others the lemma can be applied on a restricted version such as independent set on graphs with bounded minimum degree.
For this, we shortly describe the problem and then give the argument on how the lemma is applicable.

\textbf{Vertex Cover.}
An instance of the vertex cover interdiction problem consists of a graph $G$ and numbers $t,k \in \N_0$.
The question is if the attacker can find a set $B \subseteq V(G)$ with $|B| \leq k$ such that $B \cap S \neq \emptyset$ for every vertex cover $S$ of size at most $t$.
Now, observe that if $k \geq 2$ (and the graph is non-empty), the attacker can easily find such a set $B$ by selecting two adjacent vertices.
Thus, \cref{lem:minCardInCoNP} applies by defining $U' = \{u, v\}$ for some edge $uv \in E(G)$.
Observe that this not only destroys the solutions $S \in \sol(I)$ but also all feasible solutions $F \in \F(I)$.
Thus the minimum cardinality interdiction version, the full decision variant of interdiction and the most vital elements problem of vertex cover are coNP-complete.

\textbf{Satisfiability.}
An instance of the satisfiability interdiction problem consists out of a formula in CNF over the variables $X = \fromto{x_1}{x_n}$, with the literal set as universe, i.e. $\U = X \cup \overline X$, and interdiction budget $k$.
A similar issue as in vertex cover interdiction arises here:
If $k \geq 2$, the interdictor can just choose for some $i \in \fromto{1}{n}$ to attack both literals $x_i, \overline x_i$.
Every satisfying assignment (of non-trivial instances) contains either $x_i$ or $\overline{x_i}$, hence this is a successful attack.
Thus, \cref{lem:minCardInCoNP} applies by defining $U' = \{x, \overline x\}$ for some literal pair $x, \overline x \in \U$.
Again, this also destroys all feasible solutions $F \in \F(I)$.
Thus the minimum cardinality interdiction version, the full decision variant of interdiction and the most vital elements problem of satisfiability are coNP-complete.

\textbf{Independent Set on graphs with bounded minimum degree.}
An instance of the independent set interdiction problem consists of a graph $G =(V,E)$ with universe $U = V$, a threshold $t$ and an interdiction budget $k$.
The question of the independent set  problem is if there is a set $I \subseteq V$ such that all vertices in $I$ do not share an edge.
We now take the vertex $d$ of bounded degree into consideration.
If the attacker attacks the closed neighborhood $N[d]$ of $d$, all (optimal) solutions $S \in \mathcal S$ can be interdicted and thus \Cref{lem:minCardInCoNP} is applicable.
Thus minimum cardinality interdiction independent set on graphs with bounded minimum degree is coNP-complete.
In contrast to the other problems, this statement is not true for general feasible solutions $F \in \mathcal F(I)$. Hence we do not obtain a result for the variants from \Cref{sec:different-variants-of-interdiction}.

\textbf{Dominating Set on graphs with bounded minimum degree.}
An instance of the dominating set interdiction problem consists of a graph $G =(V,E)$ with universe $U = V$, a threshold $t$ and an interdiction budget $k$.
The question of the dominating set problem is if there is a set $D \subseteq V$ of size at most $t$ such that $D$ dominates all vertices of vertex set $V$. In other words, the union of the neighborhoods of the vertices in $D$ is the vertex set $V$, i.e. $\bigcup_{v \in D} N[v] = V$.
Again we consider a vertex $d$ of bounded degree.
Then, we can define the set of constant size to be $U' = N[d]$.
All feasible solutions $F \in \mathcal F$ have to include some vertex from $U'$ (otherwise $d$ would not be dominated).
Thus \Cref{lem:minCardInCoNP} is applicable to dominating set.
Accordingly, the minimum cardinality interdiction version, the full decision variant of interdiction and the most vital elements problem of dominating set on graphs with bounded minimum degree are coNP-complete.

\textbf{Hitting Set with bounded minimum set size.}
An instance of hitting set interdiction consists of a ground set $\{1, \ldots, n\}$ and $m$ sets $S_j \subseteq \{1,\ldots,n\}$ as well as a threshold $t$ and an interdiction budget $k$.
The universe is defined by $\U = \fromto{1}{n}$.
The question of the hitting set problem is whether there is a hitting set $H \subseteq \fromto{1}{n}$ of size at most $t$ for the sets $S_j$, that is, $H \cap S_j \neq \emptyset$ for $1 \leq j \leq m$.
We can apply \Cref{lem:minCardInCoNP} by defining $U'$ to be the set of constant size $|S_c| = O(1)$.
Then, the attacker is able to block the entire set $S_c$ such that it is not hittable, which interdicts all feasible solutions $F \in \mathcal F$.
Therefore the minimum cardinality interdiction version, the full decision variant of interdiction and the most vital elements problem of hitting set with bounded minimum set size are coNP-complete.

\textbf{Set Cover with bounded minimum coverage.}
An instance of the set cover interdiction problem consists of sets $S_i \subseteq \{1, \ldots, m\}$ for $1 \leq i \leq n$, a threshold $t$ and the an interdiction budget $k$.
The universe is defined as the sets $S_i$, $1 \leq i \leq n$.
The question of the set cover problem is whether there is selection $S \subseteq \{S_1, \ldots, S_n\}$ of size at most $k$ such that $\bigcup_{s \in S} s = \{1, \ldots, m\}$.
If there is an element $e \in \{1, \ldots, m\}$ of bounded coverage, i.e. there is a constant number of $S_i$, $1 \leq i \leq n$, with $e \in S_i$, then the attacker can attack all of these sets $S_i$.
Thus, we can apply \Cref{lem:minCardInCoNP} by choosing $U' = \{S_i \mid e \in S_i\}$ and all feasible solutions $F \in \mathcal F$ are blockable.
Accordingly, the minimum cardinality interdiction version, the full decision variant of interdiction and the most vital elements problem of set cover with bounded minimum coverage are coNP-complete.

\textbf{Steiner Tree on graphs with bounded minimum degree of terminal vertices.}
An instance of the Steiner tree interdiction problem consists of a graph $G= (S \cup T, E)$ of Steiner vertices $S$ and terminals $T$, edge weights $c: E \rightarrow \mathbb N$, a threshold $t$ and a interdiction budget $k$.
The universe is the edge set $\U = E$.
The question of the Steiner tree problem is if there is a tree $E' \subseteq E$ of weight $c(E') \leq t$ such that all terminal vertices $T$ are connected by $E'$.
If there is a terminal vertex $d \in T$ of bounded degree, then all incident edges build up a set $U' = \{dv \in E\}$ on which we can apply \Cref{lem:minCardInCoNP}.
This blocks all feasible solutions $F \in \mathcal F$.
Therefore, the minimum cardinality interdiction version, the full decision variant of interdiction and the most vital elements problem of Steiner tree on graphs with bounded minimum degree of terminal vertices are coNP-complete.

\textbf{Two Vertex-Disjoint Path on graphs with bounded degree.}
An instance of the two vertex-disjoint path interdiction problem consists of a directed graph $G=(V,A)$, vertices $s_1, s_2, t_1, t_2 \in V$ and interdiction budget $k$.
The universe is the arc set $\U = A$.
The question of the two vertex-disjoint path is if there are two paths $P_1, P_2 \subseteq A$ such that $P_i$ starts at $s_i$ and ends at $t_i$ and both paths $P_1$ and $P_2$ do not share a vertex.
If the the graph has bounded degree, we can choose any of the vertices that have to be included in on of the paths, e.g. $s_1$, and include all the incident arcs in $U' = \{(s_1, v) \in A\}$ such that we can apply \Cref{lem:minCardInCoNP}.
This blocks all feasible solutions $F \in \mathcal F$.
Accordingly, the minimum cardinality interdiction version, the full decision variant of interdiction and the most vital elements problem of two vertex-disjoint path on graphs with bounded degree are coNP-complete.

\textbf{Feedback Vertex Set on graphs with bounded girth.}
An instance of the feedback vertex set interdiction problem consists of a directed graph $G=(V,A)$, a threshold $t$ and interdiction budget $k$.
The universe is the vertex set $\U = V$.
The question of feedback vertex set is if there is a set $V' \subseteq V$ such that the graph is cycle free.
Accordingly, if the graph has bounded girth, there is a cycle of bounded length, which the attacker can attack or in other words, the cycle cannot be deleted by the defender by choosing a corresponding vertex to be in the feedback vertex set.
Thus all feasible solutions $F \in \mathcal F$ are blockable by applying \Cref{lem:minCardInCoNP} with $U' = \{v \in V \mid v \text{ is part of the smallest cycle in } G\}$.
Therefore, the minimum cardinality interdiction version, the full decision variant of interdiction and the most vital elements problem of feedback vertex set on graphs with bounded girth are coNP-complete.

\textbf{Feedback Arc Set on graphs with bounded girth.}
An instance of the feedback arc set interdiction problem consists of a directed graph $G=(V,A)$, a threshold $t$ and interdiction budget $k$.
The universe is the arc set $\U = A$.
The question of feedback arc set is if there is an arc set $A' \subseteq A$ such that the graph is acyclic.
We can use the same argument as in feedback vertex set.
That is, the attacker can choose the arcs of the smallest cycle in $G$.
Thus all feasible solutions $F \in \mathcal F$ are blockable by applying \Cref{lem:minCardInCoNP} with $U' = \{a \in A \mid a \text{ is part of the smallest cycle in } G\}$.
Therefore, the minimum cardinality interdiction version, the full decision variant of interdiction and the most vital elements problem of feedback arc set on graphs with bounded girth are coNP-complete.


\textbf{Uncapacitated Facility Location, p-Center, p-Median with bounded minimum customer coverage.}
An instance of the minimum cardinality interdiction version of these three problems consists of a set of potential facilities $F$ and a set of clients $C$ together with a cost function on the facilities $f: F \rightarrow \mathbb N$ and a service cost function $c: F \times C \rightarrow \mathbb N$ as well as a threshold $t$ and an interdiction budget $k$.
The universe is the facility set $\U = F$ and it is asked for a set of facilities $F' \subseteq F$ not exceeding the cost threshold $t$.
If the coverage of one customer is bounded, i.e. there is a bounded number of potential facilities that are able to serve the customer, the attacker is able to block all of these.
Thus we can define $U'$ as the set of facilities that are able to serve the customer of bounded coverage such that all feasible solutions $F \in \mathcal F$ can be interdicted.
Therefore, we can apply \Cref{lem:minCardInCoNP} and the minimum cardinality interdiction version, the full decision variant of interdiction and the most vital elements problem of these three facility locations problems with bounded minimum customer coverage are coNP-complete.

\textbf{Hamiltonian path/cycle (directed/undirected), TSP on graphs with bounded minimum degree.}
An instance of the minimum cardinality interdiction version of these problems consists of a graph $G=(V,E)$ (respectively $G=(V,A)$ in the directed case) and an interdiction budget $k$.
The universe is the set of edges $\U = E$ (respectively the set of arcs $\U = A$).
The question is whether there is a Hamiltonian path or cycle in $G$, i.e. a path/cycle that visits every vertex exactly once.
Because there is a vertex $d$ of bounded degree which has to be visited, we can define the set of constant size $U' = \{dv \in E\}$ (respectively $U' = \{(d,v),(v,d) \in A\}$).
If the set $U'$ is blocked it is not possible to visit the vertex, thus all feasible solutions $F \in \mathcal F$ can be interdicted.
Therefore, we can apply \Cref{lem:minCardInCoNP} and the minimum cardinality interdiction version, the full decision variant of interdiction and the most vital elements problem of these five Hamiltonian problems on graphs with bounded minimum degree are coNP-complete.

\subsection{Satisfiability with Universe over the Variables}

In the previous subsection we explained why minimum cardinality interdiction-\textsc{Sat} is contained in coNP, hence likely not $\Sigma^p_2$-complete.
Note that this is a consequence of our choice of definition of \textsc{Satisfiability}, where we explicitly defined the universe to be the literal set $L = X \cup \overline X$.
As a consequence, the interdictor may attack $X \cup \overline X$. 
\begin{samepage}
    \begin{mdframed}
    	\begin{description}
        \item[]\textsc{Satisfiability ($\U = L$)}\hfill\\
        \textbf{Instances:} Literal Set $L = \fromto{x_1}{x_n} \cup \fromto{\overline x_1}{\overline x_n}$, Clauses $C \subseteq \powerset{L}$\\
        \textbf{Universe:} $L =: \U$.\\
        \textbf{Solution set:} The set of all sets $L' \subseteq \U$ such that for all $i \in \fromto{1}{n}$ we have $|L' \cap \set{\ell_i, \overline \ell_i}| = 1$, and such that $|L' \cap c_j| \geq 1$ for all $c_j \in C$.
    	\end{description}
    \end{mdframed}
\end{samepage}

An interesting behavior occurs, when we consider the following alternative version \textsc{Satisfiability ($\U = X$)}. 
\begin{samepage}
    \begin{mdframed}
    	\begin{description}
        \item[]\textsc{Satisfiability  ($\U = X$)}\hfill\\
        \textbf{Instances:} Variable Set $X = \fromto{x_1}{x_n}$, Clauses $C \subseteq 2^{X \cup \overline X}$ \\
        \textbf{Universe:} $X =: \U$.\\
        \textbf{Solution set:} The set of all sets $X' \subseteq \U$ such that the assignment $\alpha: X \rightarrow \{0,1\}$ with $\alpha(x) = 1 \leftrightarrow x \in X'$ satisfies all clauses in $C$.
        \end{description}
    \end{mdframed}
\end{samepage}
Here the universe is only the variable set $X$, so in the interdiction version, the interdictor may only attack $X$, i.e.\ the interdictor may target individual variables and enforce that they must be set to \emph{false}. 
We show now that in contrast to the variant, where the universe is the literal set, in this new variant the interdiction problem is $\Sigma^p_2$-complete again. 
Since the problem \textsc{Satisfiability ($\U = X$)} is not part of the original problem set of \cite{gruene2024completeness}, we perform this proof in two steps.
\begin{lemma}
    \textsc{Satisfiability ($\U = X$)} is SSP-NP-complete, even when all clauses are restricted to length at most three.
\end{lemma}
\begin{proof}
    We provide an SSP reduction from the SSP-NP-complete problem \textsc{Satisfiability ($\U = L$)} to \textsc{Satisfiability ($\U = X$)}. 
    Consider an instance of \textsc{Satisfiability ($\U = L$)} given by a formula $\varphi$ with $n$ variables $X = \fromto{x_1}{x_n}$ and universe/literal set $\U = L = X \cup \overline{X}$.
    \textsc{Satisfiability ($\U = L$)} is SSP-NP-complete even when all clauses are restricted to length three, so let us w.l.o.g.\ assume that property.
    We have to show how to embed this universe into the universe $\U'$ of some corresponding \textsc{Satisfiability ($\U = X$)} instance $\varphi'$, where only positive literals are allowed in $\U'$.
    This can be done the following way:
    We introduce $2n$ new variables $X' := \fromto{x^t_1}{x^t_n} \cup \fromto{x^f_1}{x^f_n}$.
    The universe $\U' := X'$ consists out of the $2n$ corresponding positive literals $X'$.
    The new formula $\varphi'$ is defined from $\varphi$ in two steps.
    First a substitution process takes place: 
    For each $i=1,\dots,n$, the positive literal $x_i$ is replaced by the positive literal $x^t_i$ and each negative literal $\overline x_i$ is replaced by the positive literal $x_i^f$.
    In a second step, the clauses $(x^t_i \lor \overline x^f_i) \land (\overline x^t_i \lor x^f_i)$ (note that these are equivalent to $x_i^t \oplus x_i^f$) are added to $\varphi'$.
    Formally,
    \[
        \varphi' = \text{substitute}(\varphi) \land \bigwedge_{i=1}^n (x_i^t \lor x_i^f)\land (\overline{x}_i^t \lor \overline{x}^f_i).
    \]
    The SSP reduction is completed by specifying the embedding function $f : \U \to \U'$ via $f(x_i) := x_i^t$ and $f(\overline x_i) := x_i^f$.
    Clearly all clauses of $\varphi'$ have length at most three.
    Note that this reduction is a correct reduction, i.e.\ it transforms yes-instances into yes-instances and no-instances into no-instances, because the added constraints make sure that exactly one of $x_i^t$ and $x_i^f$ is true.
    Furthermore, it has the SSP property:
    For every solution $S \subseteq \U$ of \textsc{Satisfiability ($\U = L$)}, the \enquote{translated} set $f(S) \subseteq \U'$ is a solution of \textsc{Satisfiability ($\U = X$)}.
    Furthermore, for every solution $S \subseteq \U'$ of \textsc{Satisfiability ($\U = X$)}, the set $f^{-1}(S)\subseteq \U$ is a solution of \textsc{Satisfiability ($\U = L$)}.
    Accordingly, we have a correct SSP reduction (where the SSP mapping $f$ is even bijective due to $f(\U) = \U'$). 
\end{proof}

\begin{theorem}
    \textsc{Min Cardinality Interdiction-Satisfiability ($\U = X$)} is $\Sigma^p_2$-complete.
\end{theorem}
\begin{proof}
    By the previous lemma, \textsc{Satisfiability ($\U = X$)} is SSP-NP-complete, even if all clauses are restricted to length three.
    Due to \cite{gruene2024completeness}, the problem \textsc{Comb. Interdiction-Satisfiability ($\U = X$)} is $\Sigma^p_2$-complete, even if all clauses are restricted to length three.
    We provide a reduction from the latter problem in terms of an invulnerability gadget analogous to the gadgets presented in \cref{sec:invulnerability-gadgets}. 
    For this, consider an instance of \textsc{Satisfiability ($\U = X$)} with formula $\varphi$ in CNF and every clause of length three, together with the universe $\U = \fromto{x_1}{x_n}$, a set $C \subseteq \U$ of vulnerable literals, and interdiction budget $k \in \N_0$.
    For every variable $x_i \in \U \setminus C$, we explain how to make $x_i$ invulnerable.
    We introduce $k+1$ new variables $x^{(1)}_i, \dots x^{(k+1)}_i$.
    Our goal is to establish the equivalence
    \[
        x_i \equiv x^{(1)}_i \lor \dots \lor x^{(k+1)}_i.
    \]
    We can achieve this through means of the following substitution process starting from formula $\varphi$: 
    Every occurrence of $x_i$ in the formula gets substituted by $x^{(1)}_i \lor \dots \lor x^{(k+1)}_i$. 
    Every occurrence of $\overline x_i$ gets substituted (by De Morgan's law) by $(\overline x^{(1)}_i \land \dots \land \overline x^{(k+1)}_i)$.
    Note that this has two effects: 
    First, the length of a clause may now exceed 3.
    Secondly, the formula is not in CNF anymore. 
    Note however that we can use the distributive law to expand every clause that is not in CNF. 
    Since before each clause before had a length of at most three, this results in a blow-up of the instance size of a factor at most $(k+1)^3$, i.e.\ at most a polynomial factor.
    Let $\varphi'$ be the resulting formula. 
    We can see that there is an equivalence of the satisfying assignments of $\varphi$ and $\varphi'$, in the sense that $x_i$ is true in $\varphi$ if and only if $x^{(1)}_i \lor \dots \lor x^{(k+1)}_i$ is true in $\varphi'$ (for all invulnerable $x_i$). 
    However, since the interdiction budget is only $k$, the interdictor can never enforce $x^{(1)}_i \lor \dots \lor x^{(k+1)}_i$ to be false for all invulnerable variables.
    This shows that \textsc{Comb. Interdiction-Satisfiability ($\U = X$)} reduces to \textsc{Min. Cardinality Interdiction-Satisfiability ($\U = X$)}, hence proving its $\Sigma^p_2$-completeness.
\end{proof}

Note that the reasoning presented in this proof was slightly different from \cref{thm:meta-theorem}, since we start with a formula where every clause has length three, but do not preserve this property during the proof.
Hence $\Sigma^p_2$-completeness is only shown in the case where clauses can have unrestricted length.

We can use an argument similar to \Cref{lem:minCardInCoNP} to show the coNP-completeness of the minimum cardinality interdiction version, the full decision variant of interdiction and the most vital elements problem of {\sc $b$-Satisfiability ($\U = X$)}, i.e. with clauses of length bounded by $b$.
Indeed, it is easy to see that the interdiction problem of \textsc{Satisfiability ($\U = X$)} where every clause has length three is coNP-complete:
If $k \geq 3$ holds for the interdiction budget, the attacker distinguishes two cases:
If there is a clause with three positive literals, the attacker blocks all of them and immediately wins. 
In the other case, every clause has at least one negative literal.
Then the attacker can never win, since the defender can set every variable to false, which is a satisfying assignment that can never be blocked.
By an analogous argument, we can see that for any $t = O(1)$, the interdiction problem of \textsc{Satisfiability ($\U = X$)} with clauses restricted to length $t$ is coNP-complete.

Finally, we remark that slightly different variants of interdiction-3-Sat have been shown to be $\Sigma^p_2$-complete. In these variants, the interdictor does not have access to all variables (see \cite[Sec. 4.2]{gruene2024completeness} or \cite[Thm. 1]{jackiewicz2024computational}).




















% It would be nice to have our main theorem, \cref{thm:meta-theorem}, for all problems from the class SSP-NPc, instead of only those who admit an additional function $g$ with properties as stated. However, we show in this section that this is not possible.
% Concretely, we show in this section that for the following SSP-NP-complete problems, their interdiction version $\textsc{Min. Card. Interdiction-$\Pi$}$ is contained in the class coNP: Vertex cover, satisfiability, Partition/two-machine scheduling. 
% Therefore, under the usual complexity-theoretic assumption $NP \neq \Sigma^p_2$, the interdiction problem is not $\Sigma^p_2$-complete.

% In the case of the satisfiability problem, it turns out that certain variants of it remain $\Sigma^p_2$-complete, while others do not. We discuss these details below. 
% In the following, it is helpful to interpret interdiction as a game between attacker and defender.

% \textbf{Vertex cover.} An instance of the vertex cover interdiction problem consists out of a graph $G$ and numbers $t,k \in \N_0$.
% The question is if the attacker can find a set $B \subseteq V(G)$ with $|B| \leq k$ such that $B \cap C \neq \emptyset$ for every vertex cover $C$ of size at most $t$.
% Now, observe that if $k \geq 2$ (and the graph is non-empty), the attacker can easily find such a set $B$ by selecting two adjacent vertices. 
% Every vertex cover has a non-trivial intersection with $B$. Hence every instance with $k \geq 2$ is a yes-instance.
% On the other hand, if $k \leq 1$, if the defender has a winning strategy, then this strategy can be encoded in polynomial space: 
% For each choice of the attacker, some corresponding vertex cover that avoids this attack exists. A list of these at most $O(n)$ vertex covers correctly encodes the strategy.
% Since each no-instance can be certified in polynomial space, the problem is contained in coNP. It is not hard to see, that it is coNP-hard as well, hence it is coNP-complete. 

% \textbf{Satisfiability.} An instance of the satisfiability interdiction problem consists out of a formula in CNF over the variables $X = \fromto{x_1}{x_n}$, with universe $\U = X \cup \overline X$, and interdiction budget $k$.
% A similar issue as in vertex cover interdiction arises here: If $k \geq 2$, the interdictor can just choose for some $i \in \fromto{1}{n}$ to attack both literals $x_i, \overline x_i$.
% Every satisfying assignment contains either $x_i$ or $\overline{x_i}$, hence this is a successful attack.
% By a reasoning analogous to the case of vertex cover, this shows that interdiction-SAT is coNP-complete.



% \textbf{Partition/two-machine scheduling.}% \lasse{Partition is currently missing from the list in the appendix. Do we want to add it?}
% An instance of the partition interdiction problem consists out of a set of numbers $A = \fromto{a_1}{a_n}$ and an interdiction budget $k$. The universe is $\U = A$.
% Note that the partition problem has an interesting property: 
% For every set $S \subseteq \fromto{a_1}{a_n}$ which has the property that the sum of $S$ is exactly $1/2$ of the total sum of $A$, 
% the complement of $S$ has the same property. 
% Since the partition problem has this symmetric property, but the satisfiability problem does not have it, 
% a SSP reduction to partition can only exist if this symmetry is broken. Therefore in \cite{grüne2024completeness} in order to break the symmetry the partition problem is defined the following way: 
% A solution of the partition problem is a set $S \subseteq \fromto{a_1}{a_n}$ such that $S$ sums up to $1/2$ the total sum of $A$ and $a_1 \in S$.
% Under this interpretation, an attack of the interdictor on some item $a_i$ means that the interdictor enforces that $a_i$ is not in the same part of the partition as $a_1$.
% Note however, that with this definition the minimum cardinality partition interdiction the interdictor can always attack $a_1$ itself. Hence, analogous to the cases above, the problem is coNP-complete.
% Finally, we remark that even if one would choose another natural definition, 
% where the universe has two items 
% $(a_i, 1)$ and $(a_i, 2)$ for each number $a_i$, indicating whether this number is packed into part 1 or part 2, a similar pattern occurs:
% As soon as $k \geq 2$, the interdictor can attack both $(a_i, 1)$ and $(a_i, 2)$ for some $i$.
% Hence even under this alternate definition, the partition interdiction problem is coNP-complete.


% \subsection{The case of satisfiability}
% In the previous subsection we explained why interdiction-SAT is contained in coNP, hence likely not $\Sigma^p_2$-complete. Note that this is a consequence of our choice of definition of the \textsc{Satisfiability} problem, where we explicitly defined the universe to be $X \cup \overline X$. As a consequence, the interdictor may attack $X \cup \overline X$. 
% \begin{samepage}
%     \begin{mdframed}
%     	\begin{description}
%         \item[]\textsc{Satisfiability}\hfill\\
%         \textbf{Instances:} Literal Set $L = \fromto{x_1}{x_n} \cup \fromto{\overline x_1}{\overline x_n}$, Clauses $C \subseteq \powerset{L}$\\
%         \textbf{Universe:} $L =: \U$.\\
%         \textbf{Solution set:} The set of all sets $L' \subseteq \U$ such that for all $i \in \fromto{1}{n}$ we have $|L' \cap \set{\ell_i, \overline \ell_i}| = 1$, and such that $|L' \cap c_j| \geq 1$ for all $c_j \in C$.
%     	\end{description}
%     \end{mdframed}
% \end{samepage}

% Interesting behavior occurs, when we consider the following alternative version \textsc{Satisfiability'}. 
% \begin{samepage}
%     \begin{mdframed}
%     	\begin{description}
%         \item[]\textsc{Satisfiability'}\hfill\\
%         \textbf{Instances:} Variable Set $X = \fromto{x_1}{x_n}$, Clauses $C \subseteq 2^{X \cup \overline X}$ \\
%         \textbf{Universe:} $X =: \U$.\\
%         \textbf{Solution set:} The set of all sets $X' \subseteq \U$ such that the assignment $\alpha: X \rightarrow \{0,1\}$ with $\alpha(x) = 1 \leftrightarrow x \in X'$ satisfies all clauses in $C$.
%         \end{description}
%     \end{mdframed}
% \end{samepage}
% Here the universe is only $X$, so in the interdiction version, the interdictor may only attack $X$, i.e.\ the interdictor may target individual variables and enforce that they must be set to '0'. 
% We show now that in contrast to the old variant, 
% in this new variant the interdiction problem is $\Sigma^p_2$-complete again. 
% Since the problem \textsc{Satisfiability'} is not part of the original problem set of \cite{grüne2024completeness},
% we perform this proof in two steps.
% \begin{lemma}
% Problem \textsc{Satisfiability'} is SSP-NP-complete, even when all clauses are restricted to length at most three.
% \end{lemma}
% \begin{proof}
% We provide a SSP-reduction from the (old) SSP-NP-complete problem \textsc{Satisfiability} to the (new) problem \textsc{Satisfiability'}. 
% Consider an instance of \textsc{Satisfiability} given by a formula $\varphi$ with $n$ variables $X = \fromto{x_1}{x_n}$ and universe/literal set $\U = L = X \cup \overline{X}$. \textsc{Satisfiability} is SSP-NP-complete even when all clauses are restricted to length three, so let us w.l.o.g.\ assume that property.
% We have to show how to embed this universe into the  universe $\U'$ of some corresponding \textsc{Satisfiability'} instance $\varphi'$, where only positive literals are allowed in $\U'$.
% This can be done the following way: We introduce $2n$ new variables $X' := \fromto{x^t_1}{x^t_n} \cup \fromto{x^f_1}{x^f_n}$. The universe $\U' := X'$ consists out of the $2n$ positive literals $X'$.
% The new formula $\varphi$ is defined from $\varphi$ in two steps. First a substitution process takes place: 
% For each $i=1,\dots,n$ each positive literal $x_i$ gets replaced by the positive literal $x^t_i$. 
% Each negative literal $\overline x_i$  gets replaced by the positive literal $x_i^f$.
% In a second step, all the conditions $x_i^t \not \leftrightarrow x_i^f$ are added to $\varphi'$.
% (Note that this is equivalent to $(x_i^t \lor x_i^f)\land (\overline{x}_i^t \lor \overline{x}^f_i)$). Formally,
% \[
% \varphi' = \text{substitute}(\varphi) \bigwedge_{i=1}^n (x_i^t \lor x_i^f)\land (\overline{x}_i^t \lor \overline{x}^f_i).
% \]
% The SSP reduction is completed by specifying the embedding function $f : \U \to \U'$ via $f(x_i) := x_i^t$ and $f(\overline x_i) := x_i^f$.
% Clearly all clauses of $\varphi'$ have length at most three. Note that this reduction is a correct reduction, i.e.\ it transforms yes-instances into yes-instances and no-instances into no-instances, because the added constraints make sure that exactly one of $x_i^t, x_i^f$ is true.
% Furthermore, it has the SSP property: For every solution $S \subseteq \U$ of \textsc{Satisfiability}, 
% the \enquote{translated} set $f(S) \subseteq \U'$ is a solution of \textsc{Satisfiability'}.
% Furthermore, for every solution $S \subseteq \U'$ of \textsc{Satisfiability'}, the set $f^{-1}(S)\subseteq \U$ is a solution of \textsc{Satisfiability}.
% These two facts together imply that we have a correct SSP reduction (in this case the SSP property is simpler because not only $f(\U) \subseteq \U'$, but even $f(\U) = \U'$). 
% \end{proof}

% \begin{theorem}
%     Problem \textsc{Min. Card. interdiction-Satisfiability'} is $\Sigma^p_2$-complete.
% \end{theorem}
% \begin{proof}
%     By the previous lemma, \textsc{Satisfiability'} is SSP-NP-complete, even if all clauses are restricted to length three.
%     Due to \cite{grüne2024completeness}, the problem $\textsc{Comb. Interdiction-Satisfiability'}$ is $\Sigma^p_2$-complete, even if all clauses are restrcited to length three.
%     We provide a reduction from the latter problem in terms of an invulnerability gadget analogous to the gadgets presented in \cref{sec:invulnerability-gadgets}. 
%     Indeed, consider an instance of \textsc{Satisfiability'} with formula $\varphi$ in CNF and every clause of length three, together with the universe $\U = \fromto{x_1}{x_n}$, a set $C \subseteq \U$ 
%     of vulnerable literals, and interdiction budget $k \in \N_0$.
%     For every variable $x_i \in \U \setminus C$, we explain how to make $x_i$ invulnerable.
%     We introduce $k+1$ new variables $x^{(1)}_i, \dots x^{(k+1)}_i$. Our goal is to have the equivalence
%     \[
%         x_i \equiv x^{(1)}_i \lor \dots \lor x^{(k+1)}_i.
%     \]
%     We can achieve this equivalence through means of the following substitution process starting from formula $\varphi$: 
%     Every occurence of $x_i$ in the formula gets substituted by $x^{(1)}_i \lor \dots \lor x^{(k+1)}_i$. 
%     Every occurence of $\overline x_i$ gets substituted (by De Morgan's law) by $(\overline x^{(1)}_i \land \dots \land \overline x^{(k+1)}_i)$.
%     Note that this has two effects: 
%     First, the length of a clause may now exceed 3. Secondly, the formula is not in CNF anymore. 
%     Note however that we can use the distributive law to expand every clause that is not in CNF. 
%     Since before each clause before had a length of at most three, this results in a blow-up of the instance size of a factor at most $(k+1)^3$, i.e.\ at most a polynomial factor.
%     Let $\varphi'$ be the resulting formula. 
%     We can see that there is an equivalence of satisfying assignment of $\varphi$ and $\varphi'$, in the sense that $x_i$ is true in $\varphi$ if and only if $x^{(1)}_i \lor \dots \lor x^{(k+1)}_i$ is true in $\varphi'$ (for all invulnerable $x_i$). 
%     However, since the interdiciton budget is only $k$, for all invulnerable variables we see that the interdictor can never enforce $x^{(1)}_i \lor \dots \lor x^{(k+1)}_i$ to be false.
%     This shows that \textsc{Comb. Interdiction-Satisfiability'} reduces to \textsc{Min. Card. interdiction-Satisfiability'}, hence proving $\Sigma^p_2$-completeness.
    
% \end{proof}

% Note that the reasoning presented in this proof was slightly different from \cref{thm:meta-theorem}, 
% since we start with a formula where every clause has length three, but do not preserve this property during the proof. Hence $\Sigma^p_2$-completeness is only shown in the case where clauses can have unrestricted length.

% Indeed, it is easy to see that the interdiction problem of \textsc{Satisfiability'} 
% where every clause has length three is coNP-complete: If $k \geq 3$ holds for the interdiction budget, the attacker distinguishes two cases:
% If there is a clause with three positive literals, the attacker blocks all of them and immediately wins. 
% In the other case, every clause has at least one negative literal.
% Then the attacker can never win, since the defender can set every variable to false, which is a satisfying assignment that can never be blocked.
% By an analogous argument, we can see that for any $t = O(1)$, the interdiction problem of \textsc{Satisfiability'} 
% with clauses restricted to length $t$ is coNP-complete.

% Finally, we remark that slightly different variants of interdiction-3-Sat have been shown to be $\Sigma^p_2$-complete. In these variants, the interdictor does not have access to all variables (see \cite[Sec. 4.2]{grüne2024completeness} or \cite[Thm. 1]{jackiewicz2024computational}).


% \christoph{Ich glaube, wir sollten noch einiges an Überlegung in diese Section investieren. Wenigstens ein meta-theorem für eine hinreichende Bedingung sollte relativ schnell aufstellbar sein. Besser wäre natürlich eine richtige Dichotomie.}

% \christoph{Unter diese Probleme fallen auch so ziemlich alle Graphprobleme auf Bounded Degree Graphen. Wenn Du Dir die Gadgets anschuast, die wir gebastelt haben, dann verlangen diese, dass wir einen nicht konstanten Grad haben nämlich in O(Größe des Budgets des Angreifers). Grundlegend sind alle Probleme betroffen, die eine kostant lange Klausel mit rein-positiven Literalen erzeugen. Dann kann nämlich genau diese angegriffen werden. Dann kann immer das k-Sat-Argument angewendet werden.
% D.h. wenn das Problem durch per Reduktion in eine k-CNF-Sat-Instanz transformieren kann (hier muss wohl noch die SSP Eigenschaft gelten), dann ist dies auch einfach angreifbar und lediglich coNP-vollständig.
% Für Graph-Probleme auf Bounded Degree Graphen könnte man vielleicht den zugehörigen Bounded Search tree Algorithmus (aus der FPT-Theorie) als k-CNF-Sat-Instanz kodieren und damit die coNP-Vollständigkeit zeigen.
% Das wäre wenigstens ein hinreichendes Kriterium, aber noch kein notwendiges...

% Wenn wir so etwas zeigen könnten, dann haben wir eine coole Verbindung dieser Probleme mit der parametrisierten Welt und dann auch Model Checking/Datenbanktheorie, das könnte für ein größeres Publikum interessant sein.
% }
% \lasse{Das klingt sehr interessant und auch plausibel. Allerdings weiß ich momentan nicht, wie genau man dieses meta-theorem mit der Sprache unseres frameworks beschreiben könnte. Würde es nicht auch ausreichen zu sagen: Wir beobachten das Muster, dass ein ähnliches Argument wie die coNP-completeness von interdiction-kSAT anscheinend oft gemacht werden kann, insbesondere wenn das problem als eine Vereinigung von lauter constraints begrenzter Länge interpretiert werden kann.}

% \begin{itemize}
% \item dominating set in graphs with minimum degree constant, e.g. in planar graphs, bounded genus graphs.
% \item $k$-sat for $k = O(1)$
% \item vertex cover, oder allgemeiner hitting set mit $\min|S_i| = O(1)$, wo $S_i$ die sets die gehittet werden müssen
% \item oder bzw. set cover wenn $\min_e \text{coverage}(e) = O(1)$, wobei für ein element $e$ im ground set $\Omega$ $\text{coverage}(e) = |\set{S \in \U : e \in S}|$.
% \item Steiner tree, two-vertex disjoint path in $O(1)$ degree graphs
% \item feedback arc set, feedback vertex set in graphs of constant girth
% \item facility location, $p$-center, $p$-median, if there exists some customer for who only $O(1)$ possible locations exist which could serve the customer
% \item clique, independent set, in $O(1)$-degree graphs, but only the interdiction version, where optimal solutions must be interdicted $(t = t^\star)$. For the more general interdiciton problem this is not true
% \end{itemize}

% \begin{lemma}
%     If $\Pi$ is a problem in SSP-NP such that there exists in each instance $I$ of $\Pi$ a small subset $U' \subseteq \U(I)$ of size $|U'| = O(1)$ the universe of $\Pi$, such that every solution intersects $U'$, i.e. $U' \cap S \neq \emptyset$ for all $S \in \sol(I)$, then $\textsc{Min. Card-Interdiction-$\Pi$}$ is contained in coNP.
% \end{lemma}
% \begin{proof}
%     Let $t := |U'|$.
%     If the budget $k$ of the interdictor is at least $t$, then the interdictor trivially wins by interdicting all of $U'$. 
%     If $k < t$, and if the defender has a winning strategy, 
%     we can encode a winning strategy of the defender, by enumerating for each potential attack $B \subseteq \U(I)$ with $|B| \leq k$ a solution $S \in \sol(I)$ that avoids $B$.
%     Since $t = O(1)$, these $O(|U(I)|^t)$ possibilities can be encoded in polynomial space.
% \end{proof}
%  % Falls man beweisen kann: Es existiert in jeder Instanz eine Teilmenge $U' \subseteq \U$ mit $|U'| = O(1)$, und jede Lösung $S \in \sol$ des nominalen Problems mindestens ein Element mit $U'$ gemeinsam, dann gilt: $\textsc{Min. Card-Interdiction-$\Pi$} \in$ co-NP. 

% We remark that one could also ask whether the above statement about coNP-containment is also true for the more general interdiction problems introduced in \cref{sec:different-variants-of-interdiction}.
% Note that the problems from \cref{sec:different-variants-of-interdiction} are more general than $\textsc{Min. Card-Interdiction-$\Pi$}$, hence the coNP-containment of the latter does not imply coNP-containment of the first.
% One can show that if an equivalent statement holds about the feasible solutions $F \in \F$ instead of the optimal solutions $S \in \sol$, then also the problems from \cref{sec:different-variants-of-interdiction} are contained in coNP (i.e.\ there exists $U' \subseteq \U$ of size $|U'| =  O(1)$ such that $\forall F \in \F: F \cap U' \neq \emptyset$).
\section{Conclusion}
In this work, we propose a simple yet effective approach, called SMILE, for graph few-shot learning with fewer tasks. Specifically, we introduce a novel dual-level mixup strategy, including within-task and across-task mixup, for enriching the diversity of nodes within each task and the diversity of tasks. Also, we incorporate the degree-based prior information to learn expressive node embeddings. Theoretically, we prove that SMILE effectively enhances the model's generalization performance. Empirically, we conduct extensive experiments on multiple benchmarks and the results suggest that SMILE significantly outperforms other baselines, including both in-domain and cross-domain few-shot settings.

\newpage

\bibliography{bib_general,bib_interdiction,bib_reductions}

\newpage

\appendix
% !TEX root = main.tex
\section{Problems Definitions}
\label{app:sec:problemDefinitions}

\begin{samepage}
    \begin{mdframed}
    	\begin{description}
        \item[]\textsc{Satisfiability}\hfill\\
        \textbf{Instances:} Literal Set $L = \fromto{\ell_1}{\ell_n} \cup \fromto{\overline \ell_1}{\overline \ell_n}$, Clauses $C \subseteq \powerset{L}$.\\
        \textbf{Universe:} $L =: \U$.\\
        \textbf{Solution set:} The set of all sets $L' \subseteq \U$ such that for all $i \in \fromto{1}{n}$ we have $|L' \cap \set{\ell_i, \overline \ell_i}| = 1$, and such that $|L' \cap c_j| \geq 1$ for all $c_j \in C$, $j \in \fromto{1}{|C|}$.
    	\end{description}
    \end{mdframed}
\end{samepage}

\begin{samepage}
    \begin{mdframed}
    	\begin{description}
        \item[]\textsc{3-Satisfiability}\hfill\\
        \textbf{Instances:} Literal Set $L = \fromto{\ell_1}{\ell_n} \cup \fromto{\overline \ell_1}{\overline \ell_n}$, Clauses $C \subseteq \powerset{L}$ s.t. $\forall c_j \in C : |c_j| = 3$.\\
        \textbf{Universe:} $L =: \U$.\\
        \textbf{Solution set:} The set of all sets $L' \subseteq \U$ such that for all $i \in \fromto{1}{n}$ we have $|L' \cap \set{\ell_i, \overline \ell_i}| = 1$, and such that $|L' \cap c_j| \geq 1$ for all $c_j \in C$.
    	\end{description}
    \end{mdframed}
\end{samepage}

\begin{samepage}
    \begin{mdframed}
    	\begin{description}
        \item[]\textsc{Dominating Set}\hfill\\
        \textbf{Instances:} Graph $G = (V, E)$, number $k \in \N$.\\
        \textbf{Universe:} Vertex set $V =: \U$.\\
        \textbf{Feasible solution set:} The set of all dominating sets.\\
        \textbf{Solution set:} The set of all dominating sets of size at most $k$.
    	\end{description}
    \end{mdframed}
\end{samepage}

\begin{samepage}
    \begin{mdframed}
    	\begin{description}
        \item[]\textsc{Set Cover}\hfill\\
        \textbf{Instances:} Sets $S_i \subseteq \fromto{1}{m}$ for $i \in \fromto{1}{n}$, number $k \in \N$.\\
        \textbf{Universe:} $\{S_1 \dots, S_n\} =: \U$.\\
        \textbf{Feasible solution set:} The set of all $S \subseteq \{S_1, \dots, S_n\}$ s.t. $\bigcup_{s \in S} s = \fromto{1}{m}$.\\
        \textbf{Solution set:} Set of all feasible solutions with $|S| \leq k$.
    	\end{description}
    \end{mdframed}
\end{samepage}

\begin{samepage}
    \begin{mdframed}
    	\begin{description}   
        \item[]\textsc{Hitting Set}\hfill\\
        \textbf{Instances:} Sets $S_j \subseteq \fromto{1}{n}$ for $j \in \fromto{1}{m}$, number $k \in \N$.\\
        \textbf{Universe:} $\fromto{1}{n} =: \U$.\\
        \textbf{Feasible solution set:} The set of all $H \subseteq \fromto{1}{n}$ such that $H \cap S_j \neq \emptyset$ for all $j \in \fromto{1}{m}$.\\
        \textbf{Solution set:} Set of all feasible solutions with $|H| \leq k$.
    	\end{description}
    \end{mdframed}
\end{samepage}

\begin{samepage}
    \begin{mdframed}
    	\begin{description}
        \item[]\textsc{Feedback Vertex Set}\hfill\\
        \textbf{Instances:} Directed Graph $G = (V, A)$, number $k \in \N$.\\
        \textbf{Universe:} Vertex set $V =: \U$.\\
        \textbf{Feasible solution set:} The set of all vertex sets $V' \subseteq V$ such that after deleting $V'$ from $G$, the resulting graph is cycle-free (i.e. a forest).\\
        \textbf{Solution set:} The set of all feasible solutions $V'$ of size at most $k$.
    	\end{description}
    \end{mdframed}
\end{samepage}

\begin{samepage}
    \begin{mdframed}
    	\begin{description}
        \item[]\textsc{Feedback Arc Set}\hfill\\
        \textbf{Instances:} Directed Graph $G = (V, A)$, number $k \in \N$.\\
        \textbf{Universe:} Arc set $A =: \U$.\\
        \textbf{Feasible solution set:} The set of all arc sets $A' \subseteq A$ such that after deleting $A'$ from $G$, the resulting graph is cycle-free (i.e. a forest).\\
        \textbf{Solution set:} The set of all feasible solutions $A'$ of size at most $k$.
    	\end{description}
    \end{mdframed}
\end{samepage}

\begin{samepage}
    \begin{mdframed}
    	\begin{description} 
        \item[]\textsc{Uncapacitated Facility Location}\hfill\\
        \textbf{Instances:} Set of potential facilities $F = \fromto{1}{n}$, set of clients $C = \fromto{1}{m}$, fixed cost of opening facility function $f: F \rightarrow \Z$, service cost function $c: F \times C \rightarrow \Z$, cost threshold $k \in \Z$\\
        \textbf{Universe:} Facility set $F =: \U$.\\
        \textbf{Solution set:} The set of sets $F' \subseteq F$ s.t. $\sum_{i \in F'} f(i) + \sum_{j \in C} \min_{i \in F'} c(i, j) \leq k$.
    	\end{description}
    \end{mdframed}
\end{samepage}

\begin{samepage}
    \begin{mdframed}
    	\begin{description} 
        \item[]\textsc{p-Center}\hfill\\
        \textbf{Instances:} Set of potential facilities $F = \fromto{1}{n}$, set of clients $C = \fromto{1}{m}$, service cost function $c: F \times C \rightarrow \Z$, facility threshold $p \in \N$, cost threshold $k \in \Z$\\
        \textbf{Universe:} Facility set $F =: \U$.\\
        \textbf{Solution set:} The set of sets $F' \subseteq F$ s.t. $|F'| \leq p$ and $\max_{j \in C} \min_{i \in F'} c(i, j) \leq k$.
    	\end{description}
    \end{mdframed}
\end{samepage}

\begin{samepage}
    \begin{mdframed}
    	\begin{description} 
        \item[]\textsc{p-Median}\hfill\\
        \textbf{Instances:} Set of potential facilities $F = \fromto{1}{n}$, set of clients $C = \fromto{1}{m}$, service cost function $c: F \times C \rightarrow \Z$, facility threshold $p \in \N$, cost threshold $k \in \Z$\\
        \textbf{Universe:} Facility set $F =: \U$.\\
        \textbf{Solution set:} The set of sets $F' \subseteq F$ s.t. $|F'| \leq p$ and $\sum_{j \in C} \min_{i \in F'} c(i, j) \leq k$.
    	\end{description}
    \end{mdframed}
\end{samepage}

\begin{samepage}
    \begin{mdframed}
    	\begin{description} 
        \item[]\textsc{Independent Set}\hfill\\
        \textbf{Instances:} Graph $G = (V,E)$, number $k \in \N$.\\
        \textbf{Universe:} Vertex set $V =: \U$.\\
        \textbf{Feasible solution set:} The set of all independent sets.\\
        \textbf{Solution set:} The set of all independent sets of size at least $k$.
    	\end{description}
    \end{mdframed}
\end{samepage}

\begin{samepage}
    \begin{mdframed}
    	\begin{description}   
        \item[]\textsc{Clique}\hfill\\
        \textbf{Instances:} Graph $G = (V, E)$, number $k \in \N$.\\
        \textbf{Universe:} Vertex set $V =: \U$.\\
        \textbf{Feasible solution set:} The set of all cliques.\\
        \textbf{Solution set:} The set of all cliques of size at least $k$.
    	\end{description}
    \end{mdframed}
\end{samepage}

\begin{samepage}
    \begin{mdframed}
    	\begin{description}   
        \item[]\textsc{Subset Sum}\hfill\\
        \textbf{Instances:} Numbers $\fromto{a_1}{a_n} \subseteq \N$, and target value $M \in \N$.\\
        \textbf{Universe:} $\fromto{a_1}{a_n} =: \U$.\\
        \textbf{Solution set:} The set of all sets $S \subseteq \U$ with $\sum_{a_i \in S}a_i = M$.
    	\end{description}
    \end{mdframed}
\end{samepage}

\begin{samepage}
    \begin{mdframed}
    	\begin{description}   
        \item[]\textsc{Knapsack}\hfill\\
        \textbf{Instances:} Objects with prices and weights $\fromto{(p_1, w_1)}{(p_n, w_n)} \subseteq \N^2$, and $W, P \in \N$.\\
        \textbf{Universe:} $\fromto{(p_1, w_1)}{(p_n, w_n)} =: \U$.\\
        \textbf{Feasible solution set:} The set of all $S \subseteq \U$ with $\sum_{(p_i, w_i) \in S}w_i \leq W$.\\
        \textbf{Solution set:} The set of feasible $S$ with $\sum_{(p_i, w_i) \in S} p_i \geq P$.
    	\end{description}
    \end{mdframed}
\end{samepage}

\begin{samepage}
    \begin{mdframed}
    	\begin{description}
        \item[]\textsc{Directed Hamiltonian Path}\hfill\\
        \textbf{Instances:} Directed Graph $G = (V, A)$, Vertices $s, t \in V$.\\
        \textbf{Universe:} Arc set $A =: \U$.\\
        \textbf{Solution set:} The set of all sets $C \subseteq A$ forming a Hamiltonian path going from $s$ to $t$.
    	\end{description}
    \end{mdframed}
\end{samepage}

\begin{samepage}
    \begin{mdframed}
    	\begin{description}
        \item[]\textsc{Directed Hamiltonian Cycle}\hfill\\
        \textbf{Instances:} Directed Graph $G = (V, A)$.\\
        \textbf{Universe:} Arc set $A =: \U$.\\
        \textbf{Solution set:} The set of all sets $C \subseteq A$ forming a Hamiltonian cycle.
    	\end{description}
    \end{mdframed}
\end{samepage}

\begin{samepage}
    \begin{mdframed}
    	\begin{description}  
        \item[]\textsc{Undirected Hamiltonian Cycle}\hfill\\
        \textbf{Instances:} Graph $G = (V, E)$.\\
        \textbf{Universe:} Edge set $E =: \U$.\\
        \textbf{Solution set:} The set of all sets $C \subseteq E$ forming a Hamiltonian cycle.
    	\end{description}
    \end{mdframed}
\end{samepage}

\begin{samepage}
    \begin{mdframed}
    	\begin{description}
        \item[]\textsc{Traveling Salesman Problem}\hfill\\
        \textbf{Instances:} Complete Graph $G = (V, E)$, weight function $w: E \rightarrow \Z $, number $k \in \N$.\\
        \textbf{Universe:} Edge set $E =: \U$.\\
        \textbf{Feasible solution set:} The set of all TSP tours $T\subseteq E$.\\
        \textbf{Solution set:} The set of feasible $T$ with $w(T) \leq k$.
    	\end{description}
    \end{mdframed}
\end{samepage}

\begin{samepage}
    \begin{mdframed}
    	\begin{description}
        \item[]\textsc{Directed} $k$-\textsc{Vertex Disjoint Path}\hfill\\
        \textbf{Instances:} Directed graph $G = (V, A)$, $s_i, t_i \in V$ for $i \in \fromto{1}{k}$.\\
        \textbf{Universe:} Arc set $A =: \U$.\\
        \textbf{Solution set:} The sets of all sets $A' \subseteq A$ such that $A' = \bigcup^k_{i = 1} A(P_i)$, where all $P_i$ are pairwise vertex-disjoint paths from $s_i$ to $t_i$ for $1 \leq i \leq k$.
    	\end{description}
    \end{mdframed}
\end{samepage}

\begin{samepage}
    \begin{mdframed}
    	\begin{description}
        \item[]\textsc{Steiner Tree}\hfill\\
        \textbf{Instances:} Undirected graph $G = (S \cup T, E)$, set of Steiner vertices $S$, set of terminal vertices $T$, edge weights $c: E \rightarrow \N$, number $k \in \N$.\\
        \textbf{Universe:} Edge set $E =: \U$.\\
        \textbf{Feasible solution set:} The set of all sets $E' \subseteq E$ such that $E'$ is a tree connecting all terminal vertices from $T$.\\
        \textbf{Solution set:} The set of feasible solutions $E'$ with $\sum_{e' \in E'} c(e') \leq k$.
    	\end{description}
    \end{mdframed}
\end{samepage}



\end{document}
