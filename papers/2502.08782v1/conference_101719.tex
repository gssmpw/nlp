\documentclass[conference]{IEEEtran}
\IEEEoverridecommandlockouts
% The preceding line is only needed to identify funding in the first footnote. If that is unneeded, please comment it out.
\usepackage{cite}
\usepackage{amsmath,amssymb,amsfonts}

\usepackage{graphicx}
\usepackage{textcomp}
\usepackage{xcolor}

\usepackage{algorithm}
\usepackage{algorithmic}
% \usepackage{algcompatible}
% \usepackage{algpseudocode}
\usepackage{array}
\usepackage{nomencl}
\usepackage{eurosym}
\usepackage{multirow}
\usepackage{eurosym}
\bibliographystyle{unsrt} 

\usepackage{amsmath} % Required for the align environment
\usepackage{hyperref} % Optional for better referencing


\def\BibTeX{{\rm B\kern-.05em{\sc i\kern-.025em b}\kern-.08em
    T\kern-.1667em\lower.7ex\hbox{E}\kern-.125emX}}
\begin{document}

\title{A comparative study of different TSO-DSO coordination in the reserve market\\
% A comparative study of different TSO-DSO coordinations in using balancing services from EV aggregators
% Participation of Electric vehicle Aggregator in different TSO-DSO coordination schemes\\
%{\footnotesize \textsuperscript{*}Note: Sub-titles are not captured in Xplore and
%should not be used}
%\thanks{Identify applicable funding agency here. If none, delete this.}

\thanks{The authors would like to acknowledge the financial support for this work from the Netherlands Organization for Scientific Research (NWO) funded DEMOSES project.}
}
\author{\IEEEauthorblockN{1\textsuperscript{st} Hang Nguyen}
\IEEEauthorblockA{\textit{Electrical Energy Systems Group} \\
\textit{Eindhoven University of Technology}\\
Eindhoven, The Netherlands \\
t.h.nguyen@tue.nl}
\and
\IEEEauthorblockN{2\textsuperscript{rd} Phuong Nguyen}
\IEEEauthorblockA{\textit{Electrical Energy Systems Group} \\
\textit{Eindhoven University of Technology}\\
Eindhoven, The Netherlands \\
P.Nguyen.Hong@tue.nl}
\and
\IEEEauthorblockN{3\textsuperscript{nd} Koen Kok}
\IEEEauthorblockA{\textit{Electrical Energy Systems Group} \\
\textit{Eindhoven University of Technology}\\
Eindhoven, The Netherlands \\
j.k.kok@tue.nl}
\and
}

\maketitle

\begin{abstract}
The increasing penetration of Distributed Energy Resources (DERs) in the distribution system has led to the emergence of a new market actor - the aggregator. 
The aggregator serves as a facilitator, enabling flexibility asset owners to get access to different markets. In which, EVs aggregators are gaining more attention due to their expanding use and potential to provide services in various types of markets, particularly in the reserve market.
% the transmission system operator (TSO) uses the DER flexibility under conditions of location and specific device information scarcity. 
Currently, TSO indirectly utilizes these resources under the management of the distribution system operators (DSO), which can negatively impact the distribution grid. 
Conversely, adjustments from DSOs can impact service provision to TSO due to the shortage of TSO usage information. These factors highlight the importance of evaluating the service provision from aggregators under different TSO-DSO coordination schemes.
% Some studies focus on the DSO-managed coordination scheme while the TSO-DSO hybrid-managed offers distinct advantages and aligns well with the prevailing situation in many countries. Therefore, 
This paper focuses on the provision of flexibility from electric vehicles (EVs) aggregators for balancing service in the TSO-DSO hybrid-managed and compares it with the DSO-managed coordination schemes.
The behavior of aggregators reacting to price fluctuations and TSO requests under different coordination schemes and simulation scenarios is thoroughly evaluated. Additionally, their impact on the grid is analyzed through the DSO’s congestion management process and validated using data from a real part of the Dutch distribution network. 
Results find that the hybrid-managed coordination scheme gives more benefit to the aggregator than the DSO-managed scheme and the EVs aggregator will gain more profit in winter than summer due to more upward regulation service is needed.
% The results demonstrate that effective coordination between TSO-DSO in leveraging DER flexibility is essential. 

\end{abstract}

\begin{IEEEkeywords}
TSO-DSO coordination, DERs, aggregator, congestion, flexibility.
\end{IEEEkeywords}

\section{Introduction}

% Nowadays, researching the provision of flexibility from aggregators is no longer a novel topic because it has already been implemented in some European countries like France, Finland, Hungary, Estonia, Denmark, Belgium, Romania, etc \cite{17}. 
Along with exploiting the potential of distributed energy resources, providing flexibility services with these sources attracts the attention of many organizations and research groups and has already been implemented in some European countries like France, Finland, Hungary, Estonia, Denmark, Belgium, Romania, etc \cite{17}.
The role of the aggregator - a market participant responsible for bundling flexibility from multiple small customers and producers into a portfolio and offering the combined capacity in other markets\cite{18} - is becoming increasingly important.
Many research studies focus on optimal biding strategies\cite{3,4,5,6,7,8}, providing service to multiple markets (e.g., spot market, reserve,\cite{3,9,10,14,15} congestion management services\cite{7} or balancing portfolios\cite{3}) and leveraging the potential of diverse appliances including electric vehicles (EVs), battery energy storage system (BESS), heat pumps, thermal-controlled load or their combination.

% According to \cite{3},  aggregators have 3 main functions: flexibility developer, flexibility operator and flexibility trader, all of which require interaction with both the prosumer and market side. Based on their roles, aggregators are classified either as independent aggregators or as entities assuming the role of an existing market participant. To effectively aggregate small-scale resources, various aggregation methods have been proposed including coordinated, bottom-up, bus-split, and clustering-based approaches. Additionally, customer rewarding mechanisms such as real-time pricing, critical peak pricing, time of use, or time of export have been introduced.

% The authors in \cite{4} introduce an optimal decision-making strategy for electric vehicles (EVs) aggregators to provide service to both day-ahead (DA) and balancing market, considering the competitive and risk-aversion when buying prosumers's electricity. Results indicate that aggregators can adopt a more reliable approach by taking less risk, which leads to lower profit. Conversely, when aggregators take more risk and attempt to increase charge prices and decrease discharge prices to maximize profit, that way ultimately has a negative impact on their profitability. Therefore, in a competitive environment, risk evaluation plays a critical role in the decision-making process of the aggregator.  
% José et al \cite{5} introduce a 2-stages stochastic optimization to support aggregators in the definition of the bid in DA and reserve market while considering the uncertainty of external factors like temperature, renewable generation, and customer preferences. The optimal bidding strategy helps reduce the cost of both aggregator and prosumer by 40\% compared to typical bidding.
% His research group \cite{6} proposes a hierarchical model predictive control to ensure reliable delivery of multiple market products traded by aggregators through real-time control of flexible resources. The results show that the aggregator can deliver several combinations of energy and secondary reserves without compromising the conformity and preferences of their clients.
% Via risk evaluation with a bottom-up approach, the authors in \cite{14} emphasize the importance of modeling the customer-driven constraint when estimating the expected outcome for demand respond aggregator when providing three different types of demand respond services to the DA market, balancing market and forward contract. 

% The above research works consider the aggregator business model without evaluating its impact on the distribution grid. 
% Therefore, Junjie Hu et al \cite{15} propose a framework that allows optimal provision of prosumer balancing service within the network constraints. Results prove the potential conflict of using flexibility between TSO and DSO can be resolved in a transactive manner while retaining user privacy.

Recently, EV aggregators are receiving increasing attention due to their expanding use, driven by environmental factors and government incentives. 
In 2023, the Netherlands was the sixth-largest EV market in Europe with more than 300 thousand battery electric vehicles registered and leads in EV infrastructure \cite{EV_Dutch_1}. 
% In September of 2024, a new subsidy scheme to help fund private charging infrastructure for commercial vehicles was announced by the Netherlands Enterprise Agency prove their focus on EV development in the upcoming years \cite{EV_Dutch_2}.
Research on the Dutch situation shows the potential of EV aggregators to provide service in the reserve market\cite{10}.
EV fleets also have the potential to provide flexibility to the internal balancing portfolio\cite{3}, and congestion management \cite{7}. 
% Results from \cite{5} show that EV is the main source of flexibility and revenue. 
% Via a case study in Austria, Christoph \cite{7} provides important insights that using flexible resources from EVs fleets as a redispatch measure leads to reduced curtailment of renewable energy while less additional thermal power plant usage is needed.
% Miniti et al \cite{8} propose hybrid stochastic programming for optimizing the bidding strategy of EV aggregator, which applied a robust approach in the first stage instead of scenario generation, leading to reduced computation time while bringing a better estimate of the actual daily energy cost than the two-stage stochastic approach. 
The above studies show that EV aggregators have great prospective to provide flexibility services and bring benefits not only for aggregators but also for prosumers and system operators. 
% However, most of them consider the EVs under the virtual EV while ignoring the state of charge limit of each EV. 
% using resources under the management of the distribution system operator (DSO) to provide service to the transmission system operator (TSO).
Therefore, the detailed EV aggregator is investigated in this research to consider its behaviors on different market price fluctuations.

% interaction with different transmission system operator (TSO) and distribution system operator (DSO) coordination.

Moreover, the above research works focus on the aggregator business model when providing service to the transmission system operator (TSO) without a comprehensive analysis impact on the grid under the management of the distribution system operator (DSO). While an effective approach to leveraging DER flexibility is enhancing TSO-DSO coordination \cite{16}. In recent years, TSO-DSO coordination has 
% become a research trend, attracting the attention of many countries, organizations and research groups. 
grown into a research trend of great interest to many researchers.
% This coordination primarily involves the establishment of a common data platform, the sharing of metering data, and network planning to enhance five key aspects:
% (1) Voltage regulation, (2) Reactive power management, (3) Operational cost optimization, (4) Operational planning and (5) Congestion management \cite{2}. 
% Various coordination schemes have been proposed to identify optimal solutions for these purposes via multiple European projects like SmartNet \cite{smartnet}, CoordiNet, Interrface \cite{coordinet}, InteGrid \cite{intergrid} and national projects such as Gopacs \cite{gopacs}, PicoFlex \cite{picoflex} or Soteria \cite{soteria}. 
Arthur et al \cite{1} classify the existing coordination into three main schemes (1) TSO-managed; (2) DSO-managed; and (2) TSO-DSO Hybrid. The first one prioritizes the TSO in using flexibility, restricting the DSO from resources connected to their network. Conversely, the DSO-managed model assigns a greater role along with responsibility to DSO in optimizing the utility of flexibility. 
DSOs are prioritized to use the DER flexibility to manage congestion in their grids, then send the remainder to the central market.
However, this scheme has the potential to create conflicts of benefit among DSOs. The Hybrid TSO-DSO managed model appears to be the most balanced option as TSO and DSO manage their own market and it allows TSO to use service with DSO's consent. This coordination aligns well with the prevailing circumstances in many countries and offers distinct advantages over others by empowering DSOs role and fostering greater social welfare. Therefore, the DSO-TSO hybrid-managed model is one of the schemes selected for investigation in this research.

% why agg vs TSO-DSO
Considering the provision of DERs flexibility from the aggregator within TSO-DSO coordination offers a broader perspective by highlighting the risk of violating grid constraints when procuring flexibility for the TSO.  Therefore, this research constructs a comprehensive model for the EV aggregator to provide multiple services in different TSO-DSO coordination schemes.
% However, most of studies focus on 
To the best of the authors's knowledge, this topic has not been widely explored in the literature. This paper finds some similarities with the 3 papers \cite{11,12,13} in using flexibility from the distribution grid via aggregator while considering the TSO-DSO coordination. However, unlike \cite{11}, this paper models the detail of EV aggregator, improves the DSO model with linearized ACOPF and compares the flexibility provision in 2 different coordination schemes. In \cite{12}, the authors show that 16\% and nearly 52\% of the operational cost is reduced for TSO and DSO, respectively. While aggregator lost about 54\% profit in the DSO-managed model. They didn't show how it would be different if TSO used flexibility from the aggregator in the balancing market.
Compared with \cite{13}, this paper models the EV aggregator that considers individual EV to ensure EV energy within the limit range while providing vehicle to grid service and considering the TSO-DSO hybrid-managed model that advances over the DSO-managed coordination scheme. 
% The authors in \cite{15} consider the provision of DER flexibility with grid constraints by adopting the concept of transactive energy. However, this framework considers the interaction between aggregator and DSO while not considering the interaction between TSO and DSO.

The contributions of this work are as follows:
\begin{itemize}
\item Modeling the interaction between TSO, DSO and EV aggregator in the provision of balancing service (aFRR), while considering the constraint in the distribution grid.
\item Considering the service provision in different TSO-DSO coordination schemes and validating the mechanism to one part of the distribution network of the Netherlands.
\item Analysing the behavior of multiple EV aggregators in price fluctuations and seasonal changes.
% response to day-ahead price and imbalance price fluctuation and BRP contracted price.
\end{itemize}
The remainder of the paper is organized into four main parts. Firstly, Section II introduces the interaction of aggregator, TSO and DSO in different coordination schemes. The mathematical model of EV aggregator, TSO and DSO is presented in Section III. Section IV assesses and discusses the aggregator behavior and cost analysis via case study and simulation results.  Finally, the conclusions are presented in Section V.

\section{TSO-DSO coordination with aggregator participation}
% Introdude the TSO, DSO, aggregator (in physical layer) (thể hiện sự tương tác theo số. trong text giải thích số đó nghĩa là gì.
Coordination between TSOs and DSOs can occur in various processes including pre-qualification, offering, and real-time activation of services. 
% Traditionally, TSOs and DSOs share measured power flows at the point of connection, forecast data, and relevant data in emergency situations, but this sharing is not a regular occurrence. With the advent of flexible services, they should also share pricing signals, scheduling, and activation controls of their services to ensure transparency, prevent harmful conflicts in service activation, and avoid double counting \cite{19,20,21,22,23}.
In the case of the Dutch balancing service, they coordinate pre-qualification of the equipment specifications before granting permission to connect to the grid. However, there is no interaction during the bidding and activation process, which takes place in real-time and near real-time.
Balancing service is normally provided to TSO from the large generator and demand via a balancing service provider (BSP) \cite{21}. A BSP offers a balancing service to TSO to balance out any unpredicted imbalances in the electricity grid. Two balancing services are used in the Netherlands, automatic frequency restoration reserve (aFRR) and manual frequency restoration reserve (mFRR). aFRR is frequently used to correct the imbalance within 15 minutes, while mFRR is used for larger and longer disturbances. 
% aFRR has 2 types, "contracted bid" and "free-bid". "Contracted-bid" have to be submitted before 14:45 pm of the prior day to ensure that there are always sufficient balancing bids available, and "free-bids" from non-contracted BSPs can be submitted in 2 imbalance settlement period (IPS) before real-time. 
Currently, BSP also takes the role of aggregator to provide these services by combining resources from scattered customers.
In this study, coordination in the offering process for aFRR provision is proposed with two different schemes: the DSO-managed model and the TSO-DSO hybrid-managed model.
% Depend on the service they provide, aggregator can be balancing service provider or congestion service provider (CSP).
% (but CSP and market close in the day-before, a long duration with uncertain of price or generation and demand error.
% According to \cite{23}, aggregators must be able to undertake value stacking, it means they can provide multiple services or to multiple flexibility responsible parties.
% From the above discussion,


\subsection{TSO-DSO hybrid-managed}
\begin{figure}[htbp]
\centerline{\includegraphics[scale=0.13]{images/TSO_DSO_aggregator_1.png}}
\caption{TSO-DSO hybrid-managed coordination scheme.}
\label{fig:fig0_0}
\end{figure}

The first diagram is presented in Fig. \ref{fig:fig0_0}, with 5 main processes. In which, the aggregator is responsible for aggregating all the resources from their prosumers, determining the optimal bidding strategy and sending it to the balancing responsible party (BRP) before the previous market gate closing date and determining the volume they will dedicate to service balancing. The TSO will make the economic dispatch offers and define the merit order list (MOL) (a list of offers sorted by price from low to high) for the next two imbalance settlement period (ISP), which is 30 minutes in the Netherlands. After receiving the MOL from the TSO, the DSO proceeds to validation, eliminate invalid bids/offers, and notify the TSO of the valid results. The processes are as follows:
% or the aggregator 
\begin{enumerate}
 \item Aggregator optimizes their bidding strategy and defines the volume they can bid for regulation service.
 \renewcommand{\labelenumi}{2a)}
 \item Aggregator sends the volume they can bid for upward and downward regulation to TSO.
 \renewcommand{\labelenumi}{3a)}
 \item TSO sends the MOL to DSO per ISP.
 \renewcommand{\labelenumi}{4a)}
 \item DSO updates the boundary from the MOL and sends the new boundary to TSO.
 \renewcommand{\labelenumi}{5)}
 \item TSO activates service in real-time.
\end{enumerate}

The coordination between TSO and DSO starts from step 3a), 
% above is based on \cite{11} and is summarized in Algorithm 1. 
after receiving the upper and lower boundary from the aggregator, TSO performs economic dispatch of flexibility to define the MOL and requests DSO for validation. Then, DSOs perform power flow analysis and inform TSO congestion state of the grid according to the traffic light concept mentioned in \cite{soteria}. The green state means no congestion occurs. The yellow state informs the congestion happening in their network, DSO updates the upper and lower boundary of the flexibility by using linearized ACOPF and sends the new value to the TSO. Finally, the new boundary is sent to TSO.  

% \begin{algorithm}[t]
% \caption{TSO-DSO hybrid-managed}
% \begin{algorithmic}
% % \small{
% % \Procedure
%     \STATE{\bfseries Input}: flexibility data, activated volume
%     \STATE{\bfseries Output}: $New\_MOL$
%     \STATE $DSO \gets MOL$
%     \STATE $MOL \gets $ TSO performs economic dispatch of flexibility
%     \STATE $state \gets $ DSO check congestion
%     \STATE{\bfseries While}{ $state == Yellow$}
%     \STATE \hspace{10pt} $new\_MOL \gets $ DSO update boundary
%     \STATE \hspace{10pt} $new\_state \gets$ DSO check congestion
%     \STATE \hspace{10pt} $state \gets new\_state$
%     % \EndWhile\label{euclidendwhile}
%     \STATE $TSO \gets New\_MOL$
% % \STATE \textbf{return} $New\_MOL$
% % \EndProcedure
% % }
% \end{algorithmic}
% \end{algorithm}

\subsection{DSO-managed}

\begin{figure}[htbp]
\centerline{\includegraphics[scale=0.13]{images/TSO_DSO_aggregator_2.png}}
\caption{DSO-managed coordination scheme.}
\label{fig:fig0_1}
\end{figure}
Figure \ref{fig:fig0_1} illustrates the schemes where the DSO is prioritized in utilizing flexibility. The sequence of information exchange differs and is described below: 
\begin{enumerate}
 \item Aggregator optimal bidding strategy and define the volume they can bid for upward and downward regulation.
 \renewcommand{\labelenumi}{2b)}
 \item Aggregators send the volume they can bid for upward and downward regulation to \textbf{DSO}.
 \renewcommand{\labelenumi}{3b)}
 \item DSO sends the new boundary of the upward/downward volume that the aggregator can provide to TSO.
 \renewcommand{\labelenumi}{4)}
 \item TSO defines the MOL and sends activation requests to the aggregator in real-time.
\end{enumerate}
% \begin{algorithm}
% \caption{DSO-managed}\label{euclid}
% \begin{algorithmic}
% \small{
% % \Procedure
% \STATE{\bfseries Input}: flexibility data, activated volume
% \STATE{\bfseries Output}: $MOL$
% \STATE $DSO \gets $ Aggregator sends the boundaries of their flexibility 
% \STATE $state \gets $ DSO check congestion with maximum flexibility value
% \STATE{\bfseries While}{ $state == Yellow$}
%     \STATE \hspace{10pt} $New\_boundary \gets $ DSO update boundary
%     \STATE \hspace{10pt} $New\_state \gets$ DSO check congestion
%     \STATE \hspace{10pt} $state \gets New\_state$
% % \EndWhile\label{euclidendwhile}
% \STATE $TSO \gets $ DSO send new boundaries
% \STATE $MOL \gets $ TSO performs economic dispatch with new boundaries
% % \STATE \textbf{return} $New\_MOL$
% % \EndProcedure
% }
% \end{algorithmic}
% \end{algorithm}
% Different from Algorithm 1, in Algorithm 2, 
The difference with the previous coordination schemes lies in step 2b,
the aggregator will send their flexibility boundaries to DSO for validation instead of to the TSO. DSO updates the generation and demand of the grid with all flexibility and performs power flow analysis to check the congestion state in their network. Then DSO applies linearized ACOPF to determine the new boundaries of flexibility. Until no congestion occurs, DSO sends new boundaries to TSO. Finally, TSO defines the MOL based on the new boundaries.

In the above processes, the aggregator performs its functions the day before, while the TSO and DSO exchange data during the offering of aFRR services. The service cost and benefit are determined when the market closes. 
Moreover, when providing services to the TSO, the aggregator deviates from the day-ahead schedule submitted to the BRP the day before \cite{usef_doc}.
% Broker model, step 4 (page 31 in USEF document)
% Depending on the type of aggregator implementation model, the amount of the deviation cost is compensated in different ways. In the broker model, the aggregator has a contract with the BRP to compensate for the deviations they caused, or in the corrected model, the Meta Data Company will correct the meter information from the connection point with the increase/decrease in energy triggered by the aggregator and inform the TSO of the adjustment. Besides, in the central settlement model, an Allocation Responsible Party will based on a pre-defined price formula to define the amount to be paid by the party to which the energy is transferred. In the case of the Netherlands, TenneT will be the party that informs the BRP about the provision of the BSP service to avoid counteraction and the imbalance cost will not be charged to the BRP anymore.\cite{mFRR_doc}.
% This study focuses on the broker model, where 
As a result, the aggregator incurs a deviation fee payable to the BRP to compensate for the imbalance. This fee influences the aggregator's decisions regarding how much flexibility to reserve. In addition to the deviation fee, the TSO's minimum bid size presents another challenge for aggregators. To facilitate greater DER flexibility, the TSO should consider lowering the minimum bid size and increasing procurement times to reduce the uncertainty associated with forecast errors \cite{smarten}.
Therefore, this study assumes that 
% (1) the aggregator has perfect knowledge of the EV session and market price, (2) 
TSO will use all resources provided by the aggregator.
% , and (3) Customers agree with the aggregator to control the charging/discharging of their EV. 

\section{Mathematical models}
The mathematical representation of TSO, DSO and aggregator model will be discussed in the following subsections.
\subsection{Aggregator model}
The aggregator links prosumers and markets by determining the optimal bidding schedule to purchase electricity from the day-ahead market to sell to prosumers and to purchase services from prosumers to sell to other markets. Their objective is to maximize total benefit when providing services. The aggregator objective function and the relevant constraints for the whole day (96 steps) is present in Eq. \ref{eq:agg_obj} as below:
\begin{equation}
F^{Agg} = \sum_{t}^{T}\sum_{i}^{N} E_{i,t}^{\uparrow} (\lambda_t^{\uparrow} - \lambda^{\text{brp}}) 
+ E_{i,t}^{\downarrow}  (\lambda_t^{\downarrow} + \lambda^{\text{brp}}) + E_{i,t}^{\text{da}} (\lambda_t^{\text{da}}-\lambda^{\text{p}})\label{eq:agg_obj}
\end{equation}
% \end{equation}
The following constraints need to be satisfied:
\begin{align}
% \small{
\sum_{t\in T} E_{i,t}^{\uparrow} = 0  \quad\forall i, t &\in [T_{\text{d},i},T_{\text{a},i}]) \label{eq:agg_sum_up_trip}\\
\sum_{t\in T} E_{i,t}^{\downarrow} = 0 \quad\forall i,t & \in [T_{\text{d},i},T_{\text{a},i}]) \label{eq:agg_sum_down_trip}\\
\sum_{t\in T} E_{i,t}^{\text{da}} = 0 \quad\forall i,t & \in [T_{\text{d},i},T_{\text{a},i}]) \label{eq:agg_sum_da_trip}\\
P^{\uparrow}_{i,t} = E^{\uparrow}_{i,t}/\Delta T &\quad\forall {t,i} \label{eq:P2E_up}\\ 
P^{\downarrow}_{i,t} = E^{\downarrow}_{i,t}/\Delta T &\quad\forall {t,i} \label{eq:P2E_down}\\ 
P^{da}_{i,t} = E^{da}_{i,t}/\Delta T &\quad\forall {t,i} \label{eq:P2E_da}\\
\begin{cases} 
    P^{\uparrow}_{i,t} = 0 &\text{if } \lambda^{\uparrow}_t = 0,\\
    \underline{P_i^{\uparrow}}u_{i,t}\leq P^{\uparrow}_{i,t} \leq \overline{P^{\uparrow}_i}u_{i,t}  &\text{if } \lambda^{\uparrow}_t \neq 0
\end{cases} &\quad \forall t, i \label{eq:Plim_up}\\
\begin{cases}
    P^{\downarrow}_{i,t} = 0 & \text{if } \lambda^{\downarrow}_t = 0,\\
    \underline{P_i^{\downarrow}}v_{i,t} \leq P^{\downarrow}_{i,t} \leq 
    \overline{P^{\downarrow}_i}v_{i,t} & \text{if } \lambda^{\downarrow}_t \neq 0 
\end{cases} &\quad \forall t, i \label{eq:Plim_down}\\
\underline{P^{\downarrow}_i}w_{i,t} \leq P^{da}_{i,t} \leq \overline{P^{\downarrow}_i}w_{i,t} &\quad \forall t, i \label{eq:Plim_da}\\
% \underline{E_i^{\uparrow}} \leq E_{i,t}^{\uparrow} \leq \overline{E_i^{\uparrow}} &\quad \forall i,t \label{eq:agg_up_limit}\\
% \underline{E^{\downarrow}} \leq E_{i,t}^{\downarrow} \leq \overline{E^{\downarrow}} &\quad \forall i,t \label{eq:agg_down_limit}\\
% \underline{E_i^{\downarrow}} \leq E_{i,t}^{\text{da}} \leq \overline{E_i^{\downarrow}} &\quad \forall i,t \label{eq:agg_da_limit}\\
% E^{\uparrow}(t) - E^{\downarrow}(t) - E^{\text{da}}(t) &= E^{\text{trip}} \quad (t \in [T_{\text{arrive}}, T_{\text{max}}]) \label{eq:agg_trip}\\
% no charge, discharge at the same time
u_{i,t} + v_{i,t} + v_{i,t} \leq 0 &\quad\forall {t,i} \label{eq:cond_up}\\
% E^{\uparrow}_{i,t}\times E^{\downarrow}_{i,t} = 0 &\quad\forall {t,i} \label{eq:cond_up}\\
% E^{\uparrow}_{i,t}\times E^{\text{da}}_{i,t} = 0 &\quad\forall {t,i} \label{eq:cond_da}\\
% E^{\downarrow}_{i,t}\times E^{\text{da}}_{i,t} = 0 &\quad\forall {t,i} \label{eq:cond_down}\\
% EV soc
0.2 \leq E^{\text{EV}}_{i,t}/{E^{\text{EV,Size}}_i} \leq 1 &\quad\forall {t,i} \label{eq:agg_soc_limit}\\
% energy by t
E^{\text{EV}}_{i,t} = E^{\text{EV,Size}}_{i} &\quad \forall i,t=0 \label{eq:agg_init}\\
% t depart
E^{\text{EV}}_{i,t} = E^{\text{EV,Size}}_i \quad\forall i, t=T_{d,i}&(\neq 0) \label{eq:agg_dep}\\
% t mid night
E^{\text{EV}}_{i,t} = E^{\text{EV,Size}}_i \quad\forall i, t=95&(\neq T_{\text{a}}) \label{eq:agg_mid}\\
% t at home
E^{\uparrow}_{i,t} + E^{\downarrow}_{i,t} + E^{\text{da}}_{i,t} =E^{\text{EV}}_{i,t-1}-E^{\text{EV}}_{i,t} \quad \forall i, t &\notin [T_{\text{d},i},T_{\text{a},i}] \label{eq:agg_at_home} \\
% t leave home
E^{\text{EV}}_{i,t} = E^{\text{EV}}_{i,t-1} - E^{EV,trip}_i/TL \quad \forall i, t &\in [T_{\text{d},i},T_{\text{a},i}] \label{eq:agg_leave_home}
% \quad (t \in (T_{\text{init}}, & T_{\text{leave}}]) \& (T_{\text{arrive}}, T_{\text{max}}]) \label{eq:agg_at_home}
% } 
\end{align}

% %%%
% 0.2 \leq E^{\text{EV}}_{i,t}/{E^{\text{EV,Size}}_i} \leq 1 &\quad\forall {t,i} \label{eq:agg_soc_limit}\\
% % energy by t
% E^{\text{EV}}_{i,t} = E^{\text{EV,init}}_{i} &\quad \forall i,t=0 \label{eq:agg_init}\\ 
% % t leave
% % E^{\text{EV}}_{i,t} = E^{\text{EV,size}}_{i} &\quad\forall i,t=T_{\text{dep} and t \neq 0 \label{eq:agg_max}\\
% % t at home
% E^{\uparrow}_{i,t} + E^{\downarrow}_{i,t} + E^{\text{da}}_{i,t} =E^{\text{EV}}_{i,t-1}-E^{\text{EV}}_{i,t} &\quad \forall i, t \notin [T_{\text{dep}},T_{\text{arr}}] \label{eq:agg_at_home} \\
% %%%

The volumes of flexibility used by each EV for upward and downward and the volume of energy bought from the day-ahead market are $E_i^{\uparrow}$, $E_i^{\downarrow}$, and 
$E_i^{\text{da}}$ accordingly. The set of $\lambda^{\uparrow}$, $\lambda^{\downarrow}$ and $\lambda^{\text{da}}$ are balancing price of flexibility for regulating upward, downward and the day-ahead price. $\lambda_{\text{brp}}$ is the contracted price between aggregator with BRP to pay for their deviation with the day-ahead schedule. $\lambda_{\text{p}}$ is the price aggregator sell electricity to the EV's owner for charging their EV. And $N$ is the number of EVs considered.

Aggregator considers Eq. \ref{eq:agg_sum_up_trip}, Eq. \ref{eq:agg_sum_down_trip} and Eq. \ref{eq:agg_sum_da_trip} to make sure the EV will not interact with the electricity market when EVs are away from home, it means the total of energy provided for upward or downward regulation or buy from the DA market is zeros after the departure time $(T_{d})$ and before arrival time $(T_{a})$. In this study, EV is assumed to have only one leave session per day. 
Equations \ref{eq:P2E_up}, \ref{eq:P2E_down} and \ref{eq:P2E_da} present the relationship between power $(P_i)$ and energy $(E_i)$ within one ISP, $\Delta T$ is 0.25 hour. Besides, power charge/discharge is limited to the min/max charging rate $(\underline{P_i^{\downarrow}},\overline{P_i^{\downarrow}})$ or min/max discharging rate $(\underline{P_i^{\uparrow}},\overline{P_i^{\uparrow}})$  by Eq. \ref{eq:Plim_up}, \ref{eq:Plim_down} and \ref{eq:Plim_da}. 
Constraint \ref{eq:cond_up} ensures that the aggregator will provide multiple services with the "In time" stacking type. This means that the EV will not provide up or downward service or buy from the DA market at the same time thanks to the use of the binary variables $u_{i,t}, v_{i,t}$ and $w_{i,t}$.
Moreover, the state of charge (SOC) of each EV is constrained by Eq. \ref{eq:agg_soc_limit} means that EV will not charge over maximum capacity (100\%) and will not discharge lower than 20\% because deeper discharges can lead to accelerated wear and reduced lifespan of the battery.

At the beginning of the day, before leaving home and before midnight EV energy $(E^{\text{EV}}_{i,t})$ is initialized to be fully charged in Eq. \ref{eq:agg_init}, Eq. \ref{eq:agg_dep}, Eq. \ref{eq:agg_mid}. 
Equation \ref{eq:agg_at_home} updates the EV's energy by step when EV is at home, EV energy at step $t$ is equal to the energy from the previous step with the change from total energy charge and discharge. When EV leaves home, \ref{eq:agg_leave_home} energy is assumed to decrease gradually by the total energy for the trip ($E_i^{EV,trip}$) divided by the trip length ($TL$).

\subsection{TSO model}
TSO leverages the DER flexibility to minimize the cost they pay to balance the system per each ISP. 
TSO objective function is described in Eq. \ref{eq:TSO_obj} and the constraints following need to be satisfied:

\begin{align}
% \small{
% Objective function
F_{t}^{TSO} = \sum_{m\in M} \
E_{m,t}^{a,\uparrow}\lambda_{m}^{a,\uparrow} +\sum_{k\in K} E_{k,t}^{a,\downarrow}\lambda_{k}^{a,\downarrow} &+ E_t^{r,\uparrow}\lambda_{t}^{\uparrow} - E_t^{r,\downarrow}\lambda_{t}^{\downarrow}\label{eq:TSO_obj}\\
% upward regulation
\sum_{m\in M} E_{m,t}^{a,\uparrow} + E^{r,\uparrow}_t = E_{t}^{reg,\uparrow} &\quad \forall {m,t} \label{eq:sum_equal_activated_volume_up}\\
% downward regulation
\sum_{k\in K} E_{k,t}^{a,\downarrow} + E^{r,\downarrow}_t = E_{t}^{reg,\downarrow} &\quad\forall {k,t} \label{eq:sum_equal_activated_volume_dn}\\
% up, down limit
\underline{E_{m,t}^{a,\uparrow}} \leq E_{m,t}^{a,\uparrow} \leq \overline{E_{m,t}^{a,\uparrow}} &\quad\forall {m,t} \label{eq:up_agg_lim} \\
\underline{E_{k,t}^{a,\downarrow}} \leq E_{k,t}^{a,\downarrow} \leq \overline{E_{k,t}^{a,\downarrow}} &\quad\forall {k,t} \label{eq:dn_agg_lim} \\
E_{k,t}^{a,\downarrow} \leq 0 \leq E_{m,t}^{a,\uparrow} &\quad\forall {m,k,t} \label{eq:agg_sign}\\
% F_{i}^{\uparrow} &> 0 \\
E_{t}^{r,\downarrow} \leq 0 \leq E_{t}^{r,\uparrow} &\quad\forall {t} \label{eq:reserve_sign}\\
\overline{E_{m,t}^{a,\uparrow}} = \sum_{n\in N} E_{n,t}^{\uparrow} &\quad\forall {m,t} \label{eq:agg_upper_bound}\\ 
\underline{E_{k,t}^{a,\downarrow}} = \sum_{n\in N} E_{n,t}^{\downarrow} &\quad\forall {k,t} \label{eq:agg_lower_bound}\\
\underline{E_{m,t}^{a,\uparrow}} = \overline{E_{k,t}^{a,\downarrow}} = 0 &\quad\forall {m,k,t} \label{eq:agg_middle_bound}
% }
\end{align}

In the offering process, TSO defines a list of selected bids, with prices ranging from low to high until the required quantity is reached to ensure balance. Therefore, the bid prices ($\lambda_{m}^{a,\uparrow}$, $\lambda_{k}^{a,\downarrow}$) for regulating upward and for regulating downward by aggregator $m$ and $k$ are factored into the cost function of TSO.
Besides, $\lambda_{t}^{\uparrow}$ and $\lambda_{t}^{\downarrow}$ are prices of reserve capacity for regulating upward and for regulating downward from reserve units, which is assumed equal the balancing prices (imbalance price is calculated by the balancing price and most of the time when only upward or downward regulation in an ISP, these prices are identical).
The variables $E_{m,t}^{a,\uparrow}$,$E_{k,t}^{a,\downarrow}$ are the volume of flexibility provided by aggregator $m$ and aggregator $k$ at step $t$. 
$E_{t}^{reg,\uparrow}$ and $E_{t}^{reg,\downarrow}$ are the total volume activated for regulation and $E_t^{r}$ is the volume of reserve capacity uses. 
$M$ and $K$  are the number of aggregator flexibility units.

Equations \ref{eq:sum_equal_activated_volume_up} and \ref{eq:sum_equal_activated_volume_dn} ensure that the total volume provided by the DER flexibility and reserve units is equal to the volume required for upward or downward regulation by the TSO. Equations \ref{eq:up_agg_lim} and \ref{eq:dn_agg_lim} limit the volume activated by the TSO within the boundary that the aggregator can provide.
Equations \ref{eq:agg_sign} and \ref{eq:reserve_sign} represent the sign that the volume regulation for upward regulation is greater than zero and the volume for downward regulation less than zero.
While equations \ref{eq:agg_upper_bound}, \ref{eq:agg_lower_bound}, and \ref{eq:agg_middle_bound} are the boundaries calculated from the output of aggregator models.

% The output of this model is the merit order list of the flexibility volume that will be used from each aggregator and ordered by price from low to high. 
% Figure \ref{fig:fig1} demonstrates a merit order in two different steps where TSO activates upward and downward regulation.

% \begin{figure}[htbp]
% \centerline{\includegraphics[scale=0.52]{images/MOL_example.png}}
% \caption{Example of a figure caption.}
% \label{fig:fig1}
% \end{figure}
\subsection{DSO model}
The DSO will receive the MOL from the TSO or the boundaries from the aggregator. Along with their demand forecast, DSO will perform the power flow analysis to check if any congestion occurs in their network. The DSO's objective function is to minimize the potential cost they have to pay to resolve congestion per ISP which is shown in Eq. \ref{eq:dso_obj}:
\begin{align}
% \small{
F^{DSO}_t = \sum_{x\in \Omega_x} (V_{x,t}^\uparrow C_t^\uparrow) -  (V_{x,t}^\downarrow C_t^\downarrow) \label{eq:dso_obj}
% }
\end{align}

In this case, only congestion caused by the activation of downward or upward regulation from the EVs aggregators within the management of DSO is considered. Therefore, flexibility used to solve congestion is from the same aggregation unit as TSO. 
$V_{x,t}^\uparrow, V_{x,t}^\downarrow$ are the volume of flexibility used for managing congestion at node $x$, which is in a set of buses $\Omega_x$. $M$ and $K$ is a subset of $\Omega_x$.
$C_t^\uparrow$ and $C_t^\downarrow$ are congestion prices of flexibility for adjusting upward and downward at step $t$. 
% DSO constraints for each time step in the TSO-DSO hybrid-managed are as follows. 

In the TSO-DSO hybrid-managed model, the DSO uses Eq. \ref{eq:P_up_updated} and Eq. \ref{eq:P_down_updated} to update the demand $P_{i,t}^{d'}$ and generation $P_{i,t}^{g'}$ in the next two ISPs with the volumes used by the TSO from the MOL $E_{x,t}^{a,\uparrow}, E_{x,t}^{a,\downarrow}$. Then, DSO performs the optimal power flow calculation based on Eq. \ref{eq:P_balance}-\ref{eq:line_sus} to determine the flexibility volume $V_{x,t}^\uparrow, V_{x,t}^\downarrow$ to resolve congestion. 
If the optimization calculation gives a feasible solution, meaning that the amount of flexible resources from the aggregators is sufficient to resolve the congestion caused by the use of the TSO's service.
Otherwise, the boundary for the TSO is divided by $step$-th in \ref{eq:updated_bound_1a} and \ref{eq:updated_bound_1b}. TSO can only use resources within this new limitation. The process is then performed until the DSO finds the optimal solution or reaches the stopping condition after five divisions, the $E_{m,t}^{a,\uparrow '}, E_{k,t}^{a,\downarrow '}$ are reset to 0.
\begin{align}
% \small{
P_{x,t}^{g'} = P_{x,t}^{g} + E_{x,t}^{a,\uparrow'}  &\quad\forall{x,t} \label{eq:P_up_updated}\\
P_{x,t}^{d'} = P_{x,t}^{d} - E_{x,t}^{a,\downarrow'} &\quad\forall{x,t} \label{eq:P_down_updated}\\
V_{x,t}^\uparrow + V_{x,t}^\downarrow + P_{x,t}^{g'} - P_{x,t}^{d'} = \sum_{y\in J_x} Flow_{xy,t} &\quad\forall{x,y,t} \label{eq:P_balance}\\
\underline{E_{x,t}^{a,\uparrow}} \leq V_{x,t}^{\uparrow} \leq \overline{E_{x,t}^{a,\uparrow}} &\quad\forall {x,t} \label{eq:flex_lim1} \\
\underline{E_{x,t}^{a,\downarrow}} \leq V_{x,t}^{\downarrow} \leq \overline{E_{x,t}^{a,\downarrow}} &\quad\forall {x,t} \label{eq:flex_lim2} \\
V_{x,t}^{\downarrow} \leq 0 \leq V_{x,t}^{\uparrow} &\quad\forall {x,t} \label{eq:flex_lim3}\\
% A^\uparrow_\text{min} \leq A^\uparrow \leq A^\uparrow_\text{max}\label{eq:flex_lim1}\\
% -A^\downarrow_\text{max} \leq A^\downarrow \leq -A^\downarrow_\text{min}\label{eq:flex_lim2}\\
% A^\downarrow < 0, \quad A^\uparrow > 0\label{eq:flex_lim3}\\
Flow_{xy,t} = -Flow_{yx,t} &\quad\forall {x,y,t} \label{eq:p_sym}\\
Flow_{xy,t} < Rated_{xy} PF &\quad\forall {x,y,t} \label{eq:thermal_rating}\\
% \end{align}
% \begin{align}
Flow_{xy,t} = Base_{MVA} B_{xy} (\theta_{x,t} - \theta_{y,t}) &\quad\forall {x,y,t} \label{eq:v_angle}\\
B_{xy} \approx X_{xy} / (R_{xy}^2 + X_{xy}^2) &\quad\forall {x,y} \label{eq:line_sus}\\
% divide by step size
E_{x,t}^{a,\uparrow'} = E_{x,t}^{a,\uparrow}/step - V_{x,t}^\uparrow  &\quad\forall{x,t} \label{eq:updated_bound_1a}\\
E_{x,t}^{a,\downarrow'} = E_{x,t}^{a,\downarrow}/step - V_{x,t}^\downarrow  &\quad\forall{x,t} \label{eq:updated_bound_1b}
% new bound
% E_{x,t}^{a,\uparrow} = E_{x,t}^{a,\uparrow} - V_{x,t}^\uparrow  &\quad\forall{x,t} \label{eq:new_bound_1a}\\
% E_{x,t}^{a,\downarrow} = E_{x,t}^{a,\downarrow} - V_{x,t}^\downarrow    &\quad\forall{x,t} \label{eq:new_bound_1b}
% }
\end{align}
% In the DSO-managed coordination scheme, the DSO constraints for each time step are as follows:
% \begin{align}
% % \small{
% % update P d,g in scheme 2
% P_{x,t}^g = P_{x,t}^g + \overline{E_{x,t}^{a,\uparrow'}}   &\quad\forall{x,t} \label{eq:P_up_updated_2}\\
% P_{x,t}^d = P_{x,t}^d - \underline{E_{x,t}^{a,\downarrow'}} &\quad\forall{x,t} \label{eq:P_down_updated_2}\\
% % \overline{E_{x,t}^{a,\uparrow}} = \overline{E_{x,t}^{a,\uparrow}}/step  &\quad\forall{x,t} \label{eq:updated_bound_2a}\\
% % \underline{E_{x,t}^{a,\downarrow}} = \underline{E_{x,t}^{a,\downarrow}}/step  &\quad\forall{x,t} \label{eq:updated_bound_2b}\\
% \overline{E_{x,t}^{a,\uparrow'}} = \overline{E_{x,t}^{a,\uparrow}}/step - V_{x,t}^\uparrow  &\quad\forall{x,t} \label{eq:new_bound_2a}\\
% \underline{E_{x,t}^{a,\downarrow'}} = \underline{E_{x,t}^{a,\downarrow}}/step - V_{x,t}^\downarrow    &\quad\forall{x,t} \label{eq:new_bound_2b}
% % }
% \end{align}

Differently, in the DSO-managed coordination scheme, DSO updates the demand and generation at each node with $\overline{E_{x,t}^{a,\uparrow}},\underline{E_{x,t}^{a,\downarrow}}$ from the aggregator.
% which is presented in Eq. \ref{eq:P_up_updated_2} and Eq. \ref{eq:P_down_updated_2}. 
% The new boundary is obtained by reducing the aggregator boundary with the feasible solution from DSO optimal power flow calculation $V_{x,t}^\uparrow$ and $V_{x,t}^\downarrow$ in Eq. \ref{eq:new_bound_2a} and Eq. \ref{eq:new_bound_2b}, otherwise, this value is gradually reduced with $step^{th}$ iteration.% as in the previous case.

Active power balance at node $x$ at step $t$ is presented in Eq. \ref{eq:P_balance}, in which, $J_x$ is the subset of nodes that are connected to node $x$. This constraint needs to be satisfied for all nodes in the network.
Similar to TSO model, the constraints on flexibility boundaries are presented in Eq. \ref{eq:flex_lim1}, Eq.  \ref{eq:flex_lim2} and Eq. \ref{eq:flex_lim3}.
Power flow symmetry condition in Eq. \ref{eq:p_sym} means that the flow of power from node $x$ to node $y$ ($Flow_{xy}$) is opposite to the value of flow from node $y$ to node $x$.
This value is determined by the base power for the network (in MVA) $Base_{MVA}$, the susceptance of line in per unit $B_{xy}$ and the voltage phase angle difference  $\theta_{x,t}$ and $\theta_{y,t}$. 
$B_{xy}$ is calculated by resistance and reactance of the line in Eq. \ref{eq:line_sus}.
Moreover, the rated value of each branch ($Rated_{xy}$) in Eq. \ref{eq:thermal_rating} multiplied with the power factor $PF$ forms the thermal rating threshold for each branch.
The power factor $PF$ is considered 98\% in this study. The
above problems are mixed-integer linear programming (MILP)
problems and are solved by the Gurobi solver in Pyomo.

% \begin{figure}[htbp]
% \centerline{\includegraphics[scale=0.8]{images/DSO_priority_flowchart.png}}
% \caption{TSO priority.}
% \label{fig}
% \end{figure}

\section{Case studies}
% - Describe the network, data use (price)
% - Aggregator data
% - Analyse cost
% - Simulation in different days
% - Analyze aggregator behavior
The two proposed coordinations with DERs' flexibility provided by the aggregator are validated using the real distribution network in the Bleiswijk station in the northern part of the Netherlands. This network includes 4 step-down two-winding transformers from 25kV to 10.5kV, 182 busbars, 867 terminals, 224 lines, 25 synchronous machines, 170 substations, and 166 loads.
The total installed capacity is 66.06 MW  with a total spinning reserve is 11.36 MW. The PV penetration rate is 12.85\% (~8.5MW) from 16 installation points.
The aggregator for upward is connected to nodes 12, 42, 145, 146, and 147 and for downward is connected to nodes 15, 18, 41, 179, and 183 in the network.
% as illustrated in orange and yellow blocks in fig.\ref{fig:fig4}.
Each aggregator manages 100 EVs and the aggregated flexibilities are identical. The only difference is their bid prices, which are presented in Tables I.
One week in summer and one week in winter, different BRP prices are simulated to consider the aggregator behavior in price fluctuations and seasonal changes.

\begin{table}[htbp]
\centering
\caption{The DER flexibility for upward regulation}
%\fontsize{6.3}{7.2}\selectfont
% \begin{center}
\begin{tabular}{|m{0.5cm}|m{1.2cm}|m{0.6cm}|m{1.2cm}|m{1cm}|}
\hline
\multirow{2}{*}

\textbf{Unit} & 
\textbf{Name} &
\textbf{Node} & 
\textbf{Type} & 
\textbf{Price}
\\
& & & & (\euro/MWh)
\\
\hline
        1 & EV\_Agg1 & 12 & upward & 25 \\ 
\hline
        2 & EV\_Agg2 & 42 & upward & 30 \\ 
\hline
        3 & EV\_Agg3 & 145 & upward & 20 \\  
\hline
        4 & EV\_Agg4 & 146 & upward & 40 \\ 
\hline
        5 & EV\_Agg5 & 147 & upward & 35 \\
\hline
        6 & EV\_Agg6 & 18 & downward & 5 \\ 
\hline
        7 & EV\_Agg7 & 15 & downward & 10 \\ 
\hline
        8 & EV\_Agg8 & 179 & downward & 15 \\  
\hline
        9 & EV\_Agg9 & 41 & downward & -5 \\ 
\hline
        10 & EV\_Agg10 & 183 & downward & -10 \\  
\hline
\end{tabular}
\label{tab1}
% \end{center}
\end{table}

% \begin{table}[htbp]
% \centering
% \caption{The DER flexibility for downward regulation}
% %\fontsize{6.3}{7.2}\selectfont
% % \begin{center}
% \begin{tabular}{|m{0.5cm}|m{1.2cm}|m{0.6cm}|m{1.2cm}|m{1cm}|}
% \hline
% \multirow{2}{*}
% \textbf{Unit} & 
% \textbf{Name} &
% \textbf{Node} & 
% \textbf{Type} & 
% \textbf{Price}
% \\
% & & & & (\euro/MWh)
% \\
% \hline
%         1 & EV\_Agg6 & 18 & downward & 5 \\ 
% \hline
%         2 & EV\_Agg7 & 15 & downward & 10 \\ 
% \hline
%         3 & EV\_Agg8 & 179 & downward & 15 \\  
% \hline
%         4 & EV\_Agg9 & 41 & downward & -5 \\ 
% \hline
%         5 & EV\_Agg10 & 183 & downward & -10 \\  
% \hline
% \end{tabular}
% \label{tab2}
% % \end{center}
% \end{table}
% \begin{figure}[htbp]
% \centerline{\includegraphics[width=1.\columnwidth]{images/single_line_diagram.png}}
% \caption{{Location of aggregator units in the grid.}}
% \label{fig:fig4}
% \end{figure}
\subsection{Aggregator behavior by market price and season}
One week in summer, 2023 from June 26th to July 2nd and one week in winter, 2024 from February 19th to February 25th is selected. With a BRP price of 30 \euro/MW and a consumer price of 85 \euro/MW, respectively, which is lower and higher than the average day-ahead price this week.
Both in winter and summer, the volume activated for upward and downward is concentrated when the DA price is high, the DA volume is concentrated when the DA price is low, which is presented in Fig. \ref{fig:scatter_winter} and Fig. \ref{fig:scatter_summer}, respectively. The occurrence of downward service is very low and has the lowest total volume, followed by upward service, and the highest is the volume purchased from the DA market. Besides, the downward volume is densely distributed in the region where the upward price is zero and vice versa sparsely distributed in the high upward price region. The same behavior is for the distribution of upward volume versus downward price.
In most cases in reality, when the system is in deficit, the upward regulation price will increase to attract more volume from service providers, the downward regulation at that time will aggravate the situation of the system. Conversely, there is no need for increased volume when the system is surplus, and the downward regulation price goes very low. 
% \begin{figure}[htbp]
% \centerline{\includegraphics[width=1.\columnwidth]{images/distribution of all volume and price.png}}
% \caption{Distribution of volume activated by prices.}
% \label{fig:dis_vol_all}
% \end{figure}
% The results in Fig. \ref{fig:scatter_summer} and Fig. \ref{fig:scatter_winter} show that in general, the distribution is more concentrated in the occurrence of upward and buy from DA market and less in the downward regulation part. Besides, the fluctuation of downward and day-ahead prices in summer is greater than in winter. 
The distribution of buy volume by DA price and downward volume by downward price in summer is wider than in winter due to energy surplus from solar power plants with zero-marginal prices leading to many negative price cases for DA and downward price. On the opposite, it's wider for upward price range caused by energy shortage in winter. 
% The results show that upward volume concentrates more on a high day-ahead price (100 to 150) region and downward volume is distributed most in the negative price region.
\begin{figure}[htbp]
\centerline{\includegraphics[width=1\columnwidth]{images/scatter_plot_vol_by_price_winter.png}}
\caption{Distribution of volume by price in winter.}
\label{fig:scatter_winter}
\end{figure}
\begin{figure}[htbp]
\centerline{\includegraphics[width=0.9\columnwidth]{images/scatter_plot_vol_by_price_summer.png}}
\caption{Distribution of volume by price in summer.}
\label{fig:scatter_summer}
\end{figure}
% However, in summer, more upward volume is distributed in high-price regions to get more benefit, while in winter more upward volume is activated but relatively evenly distributed by price due to the system balance being more priority than profit.

From the pie chart in Fig. \ref{fig:pie_chart} show that the total volume provided for upward service (blue part) in summer is lower than in winter due to the need for more energy in winter. Downward volume (green part) drops sharply in winter when BRP prices are high. The total volume for providing service is higher when the BRP price is small, especially for upward service in winter.
Moreover, aggregator profit is inversely proportional to BRP price. The majority of the aggregator's profits come from providing upward services leading to the aggregator's profit in winter being higher than in summer.
% Besides, the average DA price lower DA price in summer (54.5 \euro/MWh) and winter (57.6 \euro/MWh) explains why the aggregator prefers to buy from the DA market instead of providing downward service. From the objective function of the aggregator, we can see that only when the downward price plus the BRP price is smaller than the DA price, they can make a profit, otherwise, they have to pay the deviation cost to the BRP. BRP price is 30 \euro/MWh in this simulation, therefore only when the downward regulation price drops below 24.5 in summer and 27.6 in winter, the aggregator will prioritize providing the downward service to charge their EV instead of buying from the DA market.
% \begin{figure}[htbp]
% \centerline{\includegraphics[scale=0.5]{images/distribution of volume activate in summmer.png}}
% \caption{Distribution of volume activate in Summer.}
% \label{fig:dis_vol_summer}
% \end{figure}
% \begin{figure}[htbp]
% \centerline{\includegraphics[scale=0.5]{images/distribution of volume activate in winter.png}}
% \caption{Distribution of volume activate in Winter.}
% \label{fig:dis_vol_winter}
% \end{figure}
% \subsection{Aggregator behavior by BRP price}
% When providing balancing service in real-time, aggregator deviates from their portfolio. This deviation leads to a penalty cost for BRP. Therefore, the aggregator has a contract with BRP to pay for the deviation from their side in the BRP e-program. This study assumes that they consensus with a fixed price by time. This subsection analyses the impact of this price on the decision of the aggregator.
% Fig. \ref{fig:brp_price_summer} 
% % and Fig. \ref{fig:brp_price_winter} 
% illustrates the profit of the aggregator according to the change of different BRP prices from low to high in one week in summer 2023. The results show that the smaller the BRP price, the larger the profit that the aggregator can earn. Besides, the average profit in winter is higher than in summer due to providing more upward services, which can be seen in Fig. \ref{fig:pie_chart}. Therefore, the profit is obviously more affected by BRP prices in winter than in summer.
% \begin{figure}[t]
% \centerline{\includegraphics[width=1.\columnwidth]{images/profit_change_by_BRP_price.png}}
% \caption{Profit change by BRP price in summer.}
% \label{fig:brp_price_summer}
% \end{figure}
% \begin{figure}[t]
% \centerline{\includegraphics[width=1.\columnwidth]{images/profit_change_by_BRP_price_in_winter.png}}
% \caption{Profit change by BRP price in winter.}
% \label{fig:brp_price_winter}
% \end{figure}
\begin{figure}[b]
\centerline{\includegraphics[width=0.64\columnwidth]{images/pie_charge_2.png}}
\caption{Distribution of volume by season with different BRP price.}
\label{fig:pie_chart}
\end{figure}

BRP price impact is clearly illustrated in the total volume upward, downward regulation, and buy from DA market of each EV. Fig. \ref{fig:1ev_30} and Fig. \ref{fig:1ev_200} show the difference in the total volume downward when the BRP price is 200  \euro/MWh and 30 \euro/MWh. The orange bar presents the downward service almost disappears. Besides, the volume for upward (blue bar) is also decreased, leading to the volume bought from the DA market (green bar) being reduced, because they don't have to buy more to compensate for the volume discharge when providing upward service.

\begin{figure}[b]
\centerline{\includegraphics[scale=0.41]{images/ev_1_30_volume.png}}
\caption{Volume by type of EV1 with BRP price = 30\euro/MWh.}
\label{fig:1ev_30}
\end{figure}
\begin{figure}[t]
\centerline{\includegraphics[scale=0.41]{images/ev_1_200_volume.png}}
\caption{Volume by type of EV1 with BRP price = 200\euro/MWh.}
\label{fig:1ev_200}
\end{figure}

% The total energy of 100 EVs during 1 week in summer and winter is presented in Fig. \ref{fig:ev1w_s} and Fig. \ref{fig:ev1w_w}.
% In summer, EVs energy is high at the beginning and end of the week, late at night after returning home, and before leaving in the morning. On Friday and Saturday, EV energy is relatively low. Particularly, on Friday, EV aggregators notice that in the afternoon the price will go very low, so they wait until the afternoon to charge/provide downward service to get more benefits. Similar to Sunday in Winter week, EV is charging more when solar energy is high if the EV is at home. In winter, EV is used more so the remaining energy in the EV is less than summer in most of the time. High energy levels occur mainly when returning home at midnight and before leaving home in the morning. 
% Furthermore, energy levels fluctuate more in summer than in winter.
% \begin{figure}[htbp]
% \centerline{\includegraphics[width=1.\columnwidth]{images/EV_1w_summer.png}}
% \caption{Total energy of 100 EVs in 1 week in summer.}
% \label{fig:ev1w_s}
% \end{figure}
% \begin{figure}[htbp]
% \centerline{\includegraphics[width=1.\columnwidth]{images/EV_1w_winter.png}}
% \caption{Total energy of 100 EVs in 1 week in winter.}
% \label{fig:ev1w_w}
% \end{figure}
% correllation
% Figures \ref{fig:corr_s} 
% and \ref{fig:corr_w}  
% shows the correlation between aggregator profits and EV energy with price and volume by type. In general, aggregator profits are highly correlated with upward regulation prices and activated volumes from TSO. The correlation between upward or downward volumes versus upward or downward regulation prices is obvious. However, it is interesting to see that energy purchased from the DA market is more correlated with upward volumes than with DA prices.
% \begin{figure}[ht]
% \centerline{\includegraphics[scale=0.55]{images/correlation in summer.png}}
% \caption{Correlation of features in summer.}
% \label{fig:corr_s}
% \end{figure}
% \begin{figure}[ht]
% \centerline{\includegraphics[scale=0.55]{images/correlation in winter.png}}
% \caption{Correlation of features in winter.}
% \label{fig:corr_w}
% \end{figure}
\subsection{Grid constraint evaluation}
Providing services brings more benefits to aggregators and prosumers.
However, a problem arises when the TSO activates the service without considering the distribution network constraints, and multiple aggregators react at the same time, resulting in congestion within the distribution network. The green line in Fig. \ref{fig:tso_priority} shows that activating the TSO balancing service at ISP 44 causes overloading in the main line straight to the connection point with the external grid. The congestion above is due to the regulation upward from TSO, the line loading is nearly 100\% because of the upward regulation when the distribution grid is in a low-demand situation, causing a reversed flow from the distribution grid to the transmission grid. This will cause rapid deterioration of the device or damage if the overload condition is prolonged. The dashed red line presents the limitation of the line loading, which is 95\% in this study.
With the coordination between TSO and DSO, after the DSO updates the MOL sent by TSO, they limit the volume of flexibility that TSO can use. Loading after TSO use flexibility with the new boundary is now below the limit constraints, which are presented by the orange line.
\begin{figure}[htbp]
\centerline{\includegraphics[width=0.92\columnwidth]{images/TSO_priority_0.95.png}}
\caption{TSO-DSO hybrid-managed.}
\label{fig:tso_priority}
\end{figure}

In the DSO-managed model, DSO based on the demand forecast for the next 2 ISP, performs power flow analysis with all the flexibility volume sent by the aggregator. The result is presented in the green line in Fig. \ref{fig:dso_priority}. Similar to the previous case, there is overloading occurs in the ISP 44. Then DSO has to update the boundary and send the new boundary to TSO. The loading of the measuring point is now the orange line, which shows that the limit constraint is satisfied.

However, the different between two cases is that the volume of flexibility that the aggregator can provide in the second case after DSO updates the boundary is less than the previous one, due to DSO not knowing exactly where TSO will call service, they will gradually reduce the boundary for all flexibility units until the congestion can be resolved by the available flexibility. This leads to the loss of profit for aggregators if they reserve energy for providing service in real-time, but this volume is not valid to provide anymore.

\begin{figure}[htbp]
\centerline{\includegraphics[width=0.92\columnwidth]{images/DSO_priority_0.95.png}}
\caption{DSO-managed.}
\label{fig:dso_priority}
\end{figure}

\section{Conclusion}
To effectively harness the potential of the DERs services in the current context, coordination between TSOs and DSOs is necessary to maximize social welfare as well as minimize the risks of conflicting use cases. Notably, DSOs play a more important role in validating services before being utilized by TSOs. This validation ensures that grid constraints are taken into account when providing services from aggregators under the DSO's management. Thereby, mitigating the potential risks of unexpected overload caused by the simultaneous reaction of aggregators with prices and requests from TSOs.
% correct this sentence
% Analyzing the service provision from multiple aggregators shows that, in the case where aggregators have complete information on prices and EV sessions, they will try to optimize the bidding strategies to maximize their profits. Similar behavior from multiple aggregators in close region in response to attractive very high upward regulation price or very low downward regulation price will lead to peak revert flow from their connection point to the grid with higher voltage level or vise versa. 
Aggregator's objective is to maximize their profits by optimizing bidding strategies. Analyzing the service provision from aggregator shows that similar behavior from multiple aggregators in neighboring regions in response to attractive very high upward regulation price or very low downward regulation price will lead to excessive reversed power flow from their connection points to the higher voltage grid or vice versa.
%%%
Besides, the price the aggregator contracts with BRP for their real-time deviation has a great impact on the decision to provide services and the final profit. 
The EVs aggregator will gain more profit from providing the upward regulation service than downward and therefore it is clear that winter will be more profitable due to the higher energy demand and frequency of upward service calls. The coordination between TSO and DSO in the offering process allows the provision of services within the grid constraints. 
% correct this sentence
% When TSO has the priority of receiving information, they create the MOL and DSO will update the boundary from the selected bid, this lead to more volume from EVs aggregator is used. On the contrary, if DSO receive information of the flexibility first, they gradually decrease the boundary of all units until no violation occurs, this lead to the greater amount of service is excluded. 
When the TSO has the priority in receiving information in the hybrid-managed model, the DSO updates the boundaries based on the TSO's selected bids. Conversely, in the DSO-managed model, unused bids are considered when DSOs resolve congestion resulting in a larger portion of services being excluded in this scenario compared with the first one.
Therefore, the hybrid-managed gives an advantage with more volume provided and more benefit to the aggregator than the DSO-managed coordination scheme. 
%%%
The risks associated with inaccurate forecasts of demand, price, and EV sessions significantly influence the decisions of each participant. These challenges will be addressed in future studies.

% \section*{Acknowledgment}
% The authors would like to acknowledge the financial support for this work from the Netherlands Organization for Scientific Research (NWO) funded DEMOSES project.
% \bibliographystyle{plain}
\bibliography{references}
\end{document}
