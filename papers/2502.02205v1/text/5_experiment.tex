We design three safe PDE control scenarios, including the 1D Burgers’ equation, 2D incompressible fluid, and controlled nuclear fusion problem. These experiments aim to address the following three questions: (1) Does introducing the uncertainty quantile enable \proj to meet safety requirements? (2) Can \proj achieve superior control performance under safety constraints? (3) Do all components of our proposed algorithm contribute effectively to its performance?

For comparison, we choose imitation learning method Behavior Cloning (\textit{BC}) \citep{pomerleau1988alvinn}, and safe reinforcement learning and imitation learning methods involving \textit{BC} with safe data filtering (\textit{BC-Safe}), Constrained Decision Transformer (\textit{CDT}) \citep{liu2023constrained}
and diffusion-based method \textit{TREBI} \citep{lin2023safe}. Note that \textit{CDT} shows the best performance in the offline Safe RL benchmark OSRL \citep{liu2023datasets}. In addition, we combine the physical system control method Supervised Learning \citep{hwang2022solving} and Model Predictive Control \citep{schwenzer2021review} with the Lagrangian approach \citep{chow2018risk} (\textit{SL-Lag, MPC-Lag}) to enforce safety constraints. We also apply the classical control method \textit{PID} \citep{1580152}. We provide the code \href{https://github.com/AI4Science-WestlakeU/safediffcon}{here}.

\begin{table}[ht]
\centering
\caption{\textbf{Results of 1D Burgers' equation.} {\color{gray} Gray}: there are unsafe trajectories. Black: all trajectories are safe. \textbf{Bold}: safe trajectories with \emph{lowest} $\J$.}
\vspace{-5pt}  %
\begin{tabular}{@{}l|c|ccc@{}}
\toprule
Methods    & $\J$ $\downarrow$ & $\R_{\text{sample}}$ $\downarrow$ & $\R_{\text{time}}$ $\downarrow$ & $\R_{\text{point}}$ $\downarrow$ \\ \midrule
BC & {\color{gray} 0.0001} & {\color{gray} 38\%} & {\color{gray} 13\%}   & {\color{gray} 1.2\%}  \\
BC-Safe                                               & {\color{gray} 0.0002}                    & {\color{gray} 14\%} & {\color{gray} 3\%}    & {\color{gray} 0.2\%} \\
PID                                                   & 0.0968       & 0\%                           & 0\%                             & 0.0\%                             \\
SL-Lag                  & 0.0115                                 & 0\%                           & 0\%                           & 0.0\%                             \\ 
MPC-Lag                  & 0.0092  & 0\%                           & 0\%                           & 0.0\%                             \\ 
CDT         & {\color{gray} 0.0012}     & {\color{gray} 16\%}   & {\color{gray} 3\%}      & {\color{gray} 0.2\%}                             \\
TREBI   & 0.0074     & 0\%  & 0\%  & 0.0\%  \\
\midrule
\textbf{\proj} & {\textbf{0.0011}} & 0\%                           & 0\%                             & 0.0\%                             \\ \bottomrule
\end{tabular}
\label{table:1d}
\end{table}

\begin{figure}[t]
\vspace{-2pt}  %
\begin{center}
    \includegraphics[width=\columnwidth]{fig/burgers_sample11.pdf}
\end{center}
\vspace{-6pt}
\caption{\textbf{Visualizations of the 1D Burgers' equation.} The top row shows the original trajectory corresponding to the control target, and the bottom row is the trajectory controlled by SafeConPhy.}
\vspace{-2pt}  %
\label{fig:1d}
\end{figure}


\subsection{1D Burgers' Equation}
\label{sec:1d}
\textbf{Experiment settings.}
1D Burgers' equation is a fundamental equation that governs various physical systems including fluid dynamics and gas dynamics. Here we follow previous works \citep{hwang2022solving,mowlavi2023optimal} and consider the Dirichlet boundary condition along with an external force \(\w (t,x) \). 

Given a target state $\u_d(x)$, the control objective $\J$ is to minimize the control error between the final state $\u_T$ and the target state $\u_d$.
\begin{equation}
\label{eq:burgers_obj_J_actual}
\J\coloneqq\int_{\Omega}|\u(T,x) - \u_d(x)|^2\mathrm{d}x.
\end{equation}
The safety score is defined as:
\begin{equation}
    s(\u) \coloneqq \sup_{(t,x) \in [0,T] \times \Omega}\{\u(t,x)^2\},
\end{equation}\label{eq:burgers_safety_score}
with the safety constraint $s_0$. If $s(\u)>s_0$, the state trajectory $\u$ is unsafe, otherwise, it is safe. We compute three unsafe rates to assess the safety levels of different methods' control results. $\R_{\text{sample}}$ denotes the proportion of unsafe trajectories among total trajectories\footnote{If any point in the full trajectory is unsafe, this trajectory is unsafe. So $\R_{\text{sample}}$ is the most stringent metric.}; $\R_{\text{time}}$ denotes the proportion of unsafe timesteps among all timesteps; $\R_{\text{point}}$ denotes the proportion of unsafe spatial lattice points in all spatial lattice points across all time steps. More details can be found in Appendix \ref{app:1dexp}. 


\textbf{Results.} In Table \ref{table:1d}, We report the results of the control objective $\J$ and safety metrics of different methods. \proj can meet the safety constraint and achieve the best control objective at the same time. As shown in Figure \ref{fig:1d}, given the initial condition and the final state (control target), \proj can control a state trajectory that satisfies the safety constraint and control target. Other methods either suffer from constraint violations or suboptimal objectives. \textit{BC} and \textit{BC-Safe} trained from expert trajectories fail to meet the safety constraints, showing that simple behavior cloning is not feasible under the safety constraints. \textit{SL-Lag} and \textit{MPC-Lag} attempt to use the Lagrangian method to balance the control objective and safety, but this coupled training program makes it difficult to find the right balance, with poor control performance. \textit{CDT} uses the complex Transformer architecture, which can achieve low control error, but it can not
meet safety constraints. The diffusion-based method \textit{TREBI} sacrifices too much control performance to satisfy the safety constraints, because its error bound is soft, and the safety constraints can be easily violated.  


\subsection{2D Incompressible Fluid}
\begin{figure*}[t]
\begin{center}
    \includegraphics[scale=0.35]{fig/2d_fluid_field11.pdf}
\end{center}
\vspace{-10pt}
\caption{\textbf{Visualization of the 2D incompressible fluid control by our \proj.} By controlling the fluid on the outside margin, the yellow smoke is successfully maneuvered to the center top exit while avoiding the red unsafe region.}
\vspace{-10pt}
\label{fig:2d}
\end{figure*}


\textbf{Experiment settings.} We then consider the control problems of the 2D incompressible fluid, which follows the Navier-Stokes equation. Following previous works \citep{holl2020learning, wei2024generative, hu2024wavelet}, the control task we consider is to maximize the amount of smoke that passes through the target bucket in the fluid flow with obstacles and openings, while constraining the amount of smoke passing through the dangerous region under the safety bound. Specifically, referring to Figure \ref{fig:2d}, the control objective \( \J \) is defined as the negative ratio of smoke passing through the target bucket located at the center top, while the safety score \( \score \) corresponds to the ratio of smoke entering the hazardous red region. It is important to note that there is a trade-off between controlling the flow through the hazardous region and achieving a more optimal control objective, which imposes higher demands on the algorithm. We set the safety score bound as \( \score_0 = 0.1 \).

Moreover, this control task is particularly challenging due to its specific setup: not only does it require indirect control, which means that control can only be applied to the peripheral region, but the spatial control parameters also reach as many as 1,792. As for safety, among all the training data, 53.1\% of the samples are unsafe, meaning their safety score \(\score\) exceeds the bound \(\score_0 = 0.1\). The average safety score of the dataset is 0.3215. Other details can be found in Appendix \ref{app:2dexp}.

\begin{table}[ht]
\centering
\vspace{-2pt}  %
\caption{\textbf{2D incompressible fluid control results.} {\color{gray} Gray}: there are unsafe trajectories. Black: all trajectories are safe. \textbf{Bold}: safe trajectories with \emph{lowest} $\J$.}
\vspace{-5pt}  %
\begin{tabular}{@{}l|c|cc@{}}
    \toprule
    Methods    & $\J$ $\downarrow$ & $SVM$ $\downarrow$ & $\R$ $\downarrow$ \\ \midrule
    BC  &  \color{gray}{-0.7104}  & \color{gray}0.7156 &  \color{gray}88\%  \\
    BC-Safe &  \color{gray}{-0.2520}  &   \color{gray}{0.0330} &  \color{gray}{8\%}   \\
    CDT     &  \color{gray}-0.7025  &   \color{gray}0.2519 &  \color{gray}30\%  \\
    TREBI   &  \color{gray}-0.7019  & \color{gray}0.0808  &  \color{gray}18\%  \\\midrule
    ${\textbf{\proj}}$   & {\textbf{-0.3548}}  & {{0.0000}}   & {{0\%}}   \\ \bottomrule
\end{tabular}
\vspace{-2pt}  %
\label{tab:2d}
\end{table}

\textbf{Results.} We report results of \proj and baselines in Table \ref{tab:2d}. Here \textit{SL-Lag} and \textit{MPC-Lag} fail to achieve reasonable control results because of the complex dynamics, and we define $\R$ as the rate of unsafe samples. Additionally, we introduce another metric called Safety Violation Magnitude $SVM=\max[\score - \score_0, 0]$. When \(\score\) does not exceed the bound \(\score_0\), this metric is 0. If \(\score\) exceeds \(\score_0\), the metric reflects the amount by which it is surpassed. From the results, it is clear that our method is the only one to successfully satisfy the safety constraint, even surpassing methods that only consider safety like BC-Safe, which demonstrates the effectiveness of post-training and fine-tuning with the uncertainty quantile.

\subsection{Tokamak Fusion Reactor}
\textbf{Experiment settings.}
A tokamak is a device designed to achieve controlled nuclear fusion by confining high-temperature plasma within a toroidal magnetic field, enabling conditions suitable for fusion reactions \cite{federici2001plasma}. The realization of controlled fusion could provide humanity with a clean, safe, and nearly limitless energy source, free from greenhouse gases, nuclear accident risks, and fuel supply imbalances \cite{schuster2006role}. Safety in a tokamak is crucial to prevent plasma instabilities and disruptions \cite{kates2019predicting}, which can damage the device worth billions of dollars.

Controlling the plasma state in a tokamak is critical for maintaining stability and optimizing fusion performance, which often requires precise management of parameters like beta ($\beta_p$), safety factor ($q_{95}$), and internal inductance ($l_i$). Based on the previous works \citep{seo2021feedforward, seo2022development}, we set the control objective to make $\beta_p$ and $l_i$ reach specific target values $\beta_p^*$ and $l_i^*$ at each time step, while the safety constraint requires $q_{95}$ to remain above the defined safety bound at all times. In this case, the objective $\J$ and safety score $\score$ are formulated as follows:
\begin{align}
    &\J\coloneqq\int_{\Omega\times[0,T]}|\beta_p(t,x) - \beta_p^*(x)|^2\mathrm{d}x\mathrm{d}t, \\
    &\score\coloneqq-\inf_{(t,x) \in [0,T] \times \Omega}\{q_{95}(t,x)\}.
\end{align}
And the safety score bound is $\score_0=-4.98$. We use the same initial state and tokamak simulation environment as in the previous work \citep{seo2022development}, setting time-varying random targets within the reasonable ranges proposed in that study. The control sequences for training are collected using the reinforcement learning model trained in the previous work \citep{seo2022development}, where 71.18\% of the samples in the training dataset are unsafe.

\begin{table}[ht]
\centering
\vspace{-2pt}  %
\caption{\textbf{Results of the tokamak fusion reactor.} {\color{gray} Gray}: there are unsafe trajectories. Black: all trajectories are safe. \textbf{Bold}: safe trajectories with \emph{lowest} $\J$.}
\vspace{-3pt}  %
\begin{tabular}{@{}l|c|cc@{}}
\toprule
Methods    & $\J$ $\downarrow$   & $\R_{\text{sample}}$ $\downarrow$ & $\R_{\text{time}}$ $\downarrow$ \\ \midrule
BC                  & \color{gray}0.0610   &  \color{gray}42\%  &  \color{gray}1.34\%  \\
BC-Safe             & \color{gray}0.0811   &   \color{gray}4\%  &  \color{gray}0.03\%  \\
SL-Lag              & 0.8812   &   0\%  &  0.00\%  \\ 
MPC-Lag             & 0.8659   &   0\%  &  0.00\%  \\ 
CDT                 & \color{gray}0.0071   &   \color{gray}8\%  &  \color{gray}0.54\%  \\
TREBI               & 0.0261   &   0\%  &  0.00\%  \\
\midrule
\textbf{\proj}  & {\color[HTML]{000000} \textbf{0.0094}} & {0\%} & {0.00\%} \\ \bottomrule
\end{tabular}
\label{table:tokamak}
\end{table}
\vspace{-2pt}  %

\textbf{Results.}
We provide the results of control and safety metrics in Table \ref{table:tokamak}, where the definition of $\R_{\text{sample}}$ and $\R_{\text{time}}$ are the same as Section \ref{sec:1d}. Due to the differences in the input and output dimensions of the controller and the lack of clarity regarding the device's specific details, PID cannot be directly applied here. Results show that \proj successfully achieves full compliance with safety constraints. Moreover, among all algorithms that satisfy safety constraints, our algorithm achieves the lowest $\J$, indicating that it also achieves the best control performance. These results verify our statement that our algorithm is not only safe but also effective.

\subsection{Ablation Study}
To demonstrate the necessity and effectiveness of each component of our algorithm, we conduct three ablation experiments on the 2D incompressible fluid control problem by removing post-training, inference-time fine-tuning, and the uncertainty quantile $Q$, respectively.

It can be observed that post-training and fine-tuning, as two methods that influence the model’s output distribution, are both essential for meeting safety constraints. Removing either one results in the algorithm failing to achieve safety, highlighting the indispensable roles of these two components. Furthermore, when the uncertainty quantile is removed, which means $Q$ is set to 0, the algorithm also fails to meet safety requirements. This demonstrates that without considering the model’s uncertainty, its safety predictions are inaccurate, leading to the occurrence of unsafe events.


\begin{table}[ht]
\vspace{-2pt}  %
\centering
\caption{\textbf{Results of the ablation study.} We compare \proj with \proj w/o post-training, inference-time fine-tuning and the uncertainty quantile $Q$.}
\vspace{-3pt}  %
\begin{tabular}{@{}l|c|cc@{}}
    \toprule
    Methods    & $\J$ $\downarrow$ & $SVM$ $\downarrow$ & $\R$ $\downarrow$ \\ \midrule
    \proj             &  -0.3548  &  0.0000  &  0\%  \\ 
    w/o post-training &  -0.4911  &  0.0849  &  12\%  \\
    w/o fine-tuning   &  -0.4692  &  0.0247  &  8\%  \\ 
    w/o $Q$           &  -0.6364  &  0.0623  &  12\%  \\ \bottomrule
\end{tabular}
\vskip -2pt
\label{tab:ablation}
\end{table}

