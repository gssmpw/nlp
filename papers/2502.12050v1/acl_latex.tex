% This must be in the first 5 lines to tell arXiv to use pdfLaTeX, which is strongly recommended.
\pdfoutput=1
% In particular, the hyperref package requires pdfLaTeX in order to break URLs across lines.

\documentclass[twocolumn,11pt]{article}

% Change "review" to "final" to generate the final (sometimes called camera-ready) version.
% Change to "preprint" to generate a non-anonymous version with page numbers.
\usepackage[preprint]{acl}
\usepackage{comment}
% Standard package includes
\usepackage{times}
\usepackage{latexsym}
% \usepackage{todonotes}

% For proper rendering and hyphenation of words containing Latin characters (including in bib files)
\usepackage[T1]{fontenc}

% This assumes your files are encoded as UTF8
\usepackage[utf8]{inputenc}

% This is not strictly necessary, and may be commented out,
% but it will improve the layout of the manuscript,
% and will typically save some space.
\usepackage{microtype}

% This is also not strictly necessary, and may be commented out.
% However, it will improve the aesthetics of text in
% the typewriter font.
\usepackage{inconsolata}

%Including images in your LaTeX document requires adding
%additional package(s)
\usepackage{graphicx}

% use \url{}
\usepackage{url}

% use citep
\usepackage{natbib}

\usepackage{hyperref}

% Hyperlink color
\hypersetup{
    colorlinks=true,
    linkcolor=black,
    filecolor=black,
    urlcolor=black,
    citecolor=black,
}

% Tables
\usepackage{booktabs}
\usepackage{multirow}
\usepackage{float}
\usepackage{caption}
\captionsetup{font=small,labelfont=small}
\usepackage[normalem]{ulem}
\useunder{\uline}{\ul}{}

\usepackage{hyphenat}

\usepackage{fontawesome5}

% Increase space of footnotes
\interfootnotelinepenalty=10000



\title{
    \raisebox{-0.3\height}{\includegraphics[width=0.08\textwidth]{img/signal.png}}
    SpeechT: Findings of the First Mentorship in Speech Translation}

\author{\textbf{\normalsize Yasmin Moslem\textsuperscript{\tiny\faStar[regular]}} \hspace{0.5em}
  \textbf{\normalsize Juan Julián Cea Morán*} \hspace{0.5em}
  \textbf{\normalsize Mariano Gonzalez-Gomez*} \hspace{0.5em}
  \textbf{\normalsize Muhammad Hazim Al Farouq*} \\ \\ 
  \textbf{\normalsize Farah Abdou*} \hspace{0.5em}
  \textbf{\normalsize Satarupa Deb*}
  }

\date{}

\begin{document}
\maketitle
\def\thefootnote{{\scalebox{0.5}{\faStar[regular]}}}
\footnotetext{Correspondence: \url{yasmin [at] machinetranslation.io}}\def\thefootnote{\arabic{footnote}}
\def\thefootnote{*}\footnotetext{Participant in the mentorship}\def\thefootnote{\arabic{footnote}}

\begin{abstract}
\nohyphens{
This work presents the details and findings of the first mentorship in speech translation (SpeechT), which took place in December 2024 and January 2025. To fulfil the requirements of the mentorship, the participants engaged in key activities, including data preparation, modelling, and advanced research.
}

\end{abstract}

\section{Introduction}

At the beginning of the mentorship on speech translation, the participants were provided with the following descriptions and guidelines for each task:

\paragraph{Data:} Define, collect, and process bilingual speech data in a chosen language. Your dataset should consist of “train”, “dev/validation”, and “test” splits. By the end of the task, each participant should share a Hugging Face link to their datasets. The dataset page metadata should include sections for data sources, processing steps you applied in detail, and credits/citations of the original datasets. 

\paragraph{Modelling:} Choose one of the popular models, e.g. Whisper \citep{Radford2022-Whisper} or Wav2Vec \citep{Baevski2020-Wav2Vec}, and fine-tune it on the data prepared in the first task. Experimenting with different fine-tuning approaches and hyperparameters is encouraged. By the end of the task, the participants should share their fine-tuned models, and evaluation scores on the test dataset.

\paragraph{Advanced Research:} Enhance the quality of your model through experimenting with advanced approaches, including creating synthetic data \citep{Lam2022-Speech-Synthetic,Moslem2024-IWSLT}, comparing end-to-end systems to cascaded systems \citep{Agarwal2023-IWSLT}, using language models (e.g. n-grams) \citep{Baevski2020-Wav2Vec}, domain adaptation \citep{Samarakoon2018-Speech-Domain-Adaptation}, or any other valid approach. By the end of the task, the participants should share their advanced models. They should also clarify how the advanced approach improved the speech translation quality compared to the original fine-tuned model.

\paragraph{Release \& Publication:} Write the project details to publish as a paper. Moreover, the outcomes of all the projects are publicly accessible.\footnote{\url{https://huggingface.co/SpeechT}}



\begin{figure}
    \centering
    \includegraphics[width=1.0\linewidth]{img/cascaded-speech2text-system.png}
    \caption{Cascaded Speech-to-Text System: Two models are trained, one for ASR, and one for MT of the transcriptions.}
    \label{fig:cascaded-speech}
\end{figure}

\begin{figure}
    \centering
    \includegraphics[width=0.95\linewidth]{img/end-to-end-speech-to-text-system.png}
    \caption{End-to-End Speech-to-Text System: One model is trained to generate the translation directly.}
    \label{fig:e2e-speech}
\end{figure}

\section{End-to-End vs. Cascaded systems}
\label{sec:e2e-vs-cascaded}

Speech translation systems can be (a) “cascaded” systems, or (b) “end-to-end” systems \citep{Agarwal2023-IWSLT,Ahmad2024-IWSLT}. Cascaded speech translation systems use two models, one for automatic speech recognition (ASR) and one for textual machine translation (MT) (cf. Figure~\ref{fig:cascaded-speech}). End-to-end speech translation systems use one model for the whole process (cf. Figure~\ref{fig:e2e-speech}).

\subsection{Cascaded Speech Translation}

Cascaded speech systems involve sequential modules for Automatic Speech Recognition (ASR), Machine Translation (MT), and optionally Text-to-Speech (TTS), simultaneously combined to deliver the output to the end user. The ASR system generates transcriptions from the input audio, and then the MT model translates the transcriptions into the target language. Among the advantages of building “cascaded” systems are:

\begin{itemize}
    \item Better quality in production.
    \item Each component (ASR, MT, TTS) can be individually optimized.
    \item Domain-specific (e.g. legal or medical) MT can be easily integrated.
\end{itemize}

\subsection{End-to-End (E2E) Speech Translation}

In end-to-end (E2E) speech systems, one model produces the whole process. E2E systems can also be extended with “cascaded” components. Among the advantages of building E2E systems are:

\begin{itemize}
    \item Simpler deployment
    \item Better performance (lower latency)
\end{itemize}



\section{Approaches to synthetic data}

When the data is limited for the language or domain, synthetic data can be used to augment the authentic data. Synthetic data for speech translation systems can be generated in diverse methods, including: 

\begin{itemize}
    \item Using TTS models to generate synthetic source audio for authentic translations \citep{Moslem2024-IWSLT}
    \item Using MT models to generate translations of audio transcriptions
    \item Sampling, translating, recombining: \citet{Lam2022-Speech-Synthetic} used an advanced approach to create synthetic data, by first chunking segments and transcriptions, creating a memory of prefix-suffix chunks based on part-of-speech tagging. Then they retrieve chunks from the memory to augment prefix chunks with similar suffix chunks. Finally, they translate the new transcription with MT. Tools such as WhisperX \citep{Bain2023-WhisperX} (based on Whisper) can be used for creating alignments based on word-level timestamps.
\end{itemize}


% TODO
% more detailed introduction of dataset creation
% the rumour label in such datasets
\section{Data} \label{sec:data}
We use three rumour datasets in this work, namely: PHEME~\citep{pheme2015,kochkina-etal-2018-one}, Twitter15, and Twitter16~\citep{ma-etal-2017-detect}:

% TJB: how can the number of threads be greater than the number of tweets? these numbers don't make sense
% RX: fixed, the numbers were incorrect
\paragraph{PHEME}~\citet{pheme2015} contains 6,425 tweet posts of rumours and non-rumours related to 9 events. To avoid using specific a priori keywords to search for tweet posts, PHEME used the Twitter (now X) steaming API to identify newsworthy events from breaking news and then selected from candidate rumours that met rumour criteria, finally they collected associated conversations and annotate them. They engaged journalists to annotate the threads. The data were collected between 2014 and 2015. The 9 events are split into two groups, the first being breaking news that contains rumours, including Ferguson unrest, Ottawa shooting, Sydney siege, Charlie Hebdo shooting, and Germanwings plane crash. The rest are specific rumours, namely Prince to play in Toronto, Gurlitt collection, Putin missing, and Michael Essien contracting Ebola.
% TJB: say something about the time period when this data was collected
% RX: added

\paragraph{Twitter 15}~\citet{twitter15} was constructed by collecting rumour and non-rumour posts from the tracking websites snopes.com and emergent.info. They then used the Twitter API to gather corresponding posts, resulting in 94 true and 446 false posts. This dataset further includes 1,490 root posts and their follow posts, comprising 1,116 rumours and 374 non-rumours.
% TJB: the "tweet" vs. "comment" terminology is potentially confusing and needs to be clarified
% RX: unified, used root and follow posts to refer to root posts and the comment posts, posts are used to describe tweets in general.

\paragraph{Twitter 16}
Similarly to Twitter 15, \citet{twitter16} collected rumours and non-rumours from snopes.com, resulting in 778 reported events, 64\% of which are rumours. For each event, keywords were extracted from the final part of the Snopes URL and refined manually---adding, deleting, or replacing words iteratively---until the composed queries yielded precise Twitter search results. The final dataset includes 1,490 root tweet posts and their follow posts, comprising 613 rumours and 205 non-rumours.

\begin{table*}[!t]
    \centering
    \small
    \begin{tabular}{p{0.05\linewidth}p{0.9\linewidth}}
    \toprule
    Task & Prompt \\
    \midrule
    V-oc & Categorize the text into an ordinal class that best characterizes the writer's mental state, considering various degrees of positive and negative sentiment intensity. 3: very positive mental state can be inferred. 2: moderately positive mental state can be inferred. 1: slightly positive mental state can be inferred. 0: neutral or mixed mental state can be inferred. -1: slightly negative mental state can be inferred. -2: moderately negative mental state can be inferred. -3: very negative mental state can be inferred.\\
    \midrule
    E-c & Categorize the text's emotional tone as either `neutral or no emotion' or identify the presence of one or more of the given emotions (anger, anticipation, disgust, fear, joy, love, optimism, pessimism, sadness, surprise, trust).\\
    \midrule
    E-i & Assign a numerical value between 0 (least E) and 1 (most E) to represent the intensity of emotion E expressed in the text.\\
    \bottomrule
    \end{tabular}
    \caption{Prompts used for EmoLLM to detect emotion information in tweets. V-oc = Valence Ordinal Classification, E-c = Emotion Classification, and E-i = Emotion Intensity Regression.}
    \label{tab:emollm_ins}
\end{table*}


  
%%% Local Variables:
%%% mode: latex
%%% TeX-master: "../main_anonymous"
%%% End:


\section{Projects}

Most of the projects used a mix of data augmentation of authentic data with synthetic data, fine-tuning models, and comparing the performance of “end-to-end” speech systems to “cascaded” systems (cf. Section \ref{sec:e2e-vs-cascaded}). 

Participants used the Hugging Face Transformers library to fine-tune pretrained models. They fine-tuned Whisper \citep{Radford2022-Whisper} for “end-to-end” speech translation, and for ASR in the “cascaded” system. Moreover, they fine-tuned NLLB-200 \citep{NLLB2022} for text-to-text translation as part of “cascaded” speech translation systems. For evaluation, they used the sacreBLEU library \citep{Post2018-sacreBLEU} to obtain BLEU \citep{Papineni2002-BLEU} and ChrF++ \citep{Popovic2017-chrF++} scores. In addition, one of the participants calculated COMET scores \citep{Rei2020-COMET}. For inference, they either used the Transformers library or Faster-Whisper (based on CTranslate2 \citep{Klein2020-Efficient}) for audio translation and transcription with Whisper. For text-to-text translation with OPUS and NLLB-200 models, some of them used the Transformer library directly while others used CTranslate2 with \textit{float16} quantization, which is more efficient. For synthetic data generation, they used ChatGPT \citep{OpenAI2023-GPT-4} and OPUS \citep{Tiedemann2020-OPUS-MT} models.

Given that each participant chose a language pair, we dedicate a section for each project based on the language pair, including Galician-to-English, Indonesian-to-English, Spanish-to-Japanese, Arabic-to-English, Bengali-to-English, while we mix the last  projects due to their similarity. Each language section describes data, modelling, and evaluation of each project.


\subsection{Galician-to-English}

\subsubsection{Data [GL-EN]}
\label{sec:data-galician}

In this project, two different Galician-to-English Speech Translation datasets have been curated. First, the participant compiled the dataset \textit{OpenHQ-SpeechT-GL-EN } from the \textit{crowdsourced high-quality Galician speech data set} by \citet{Kjartansson2020-Galician-Speech-Dataset}. After deduplicating the Galician audio-transcription pairs, we have applied a machine translation step to generate the corresponding English translations. More specifically, we have used GPT-4o \citep{Brown2020-GPT-3,OpenAI2023-GPT-4} with the following prompt:
\newpage
\begin{itemize}
    \item [] [\{"role":"system", "content": "You are a helpful assistant that translates Galician (gl-ES) to English (en-XX).", \},
    \item [] \{"role": "user", "content": \{source\_text\}\}]
\end{itemize}

Given the absence of reference translation, we assessed the translation quality using CometKiwi (\textit{wmt23-cometkiwi-da-xl}) \citep{Rei2023-CometKiwi}, measuring an average score of 0.75. In total, this dataset contains approximately ten hours and twenty minutes of audio.

The second dataset is \textit{FLEURS-SpeechT-GL-EN}. This is a subset of the \textit{FLEURS} \citep{Conneau2023-FLEURS} dataset, which contains two thousand parallel audio-transcription pairs in a hundred and two languages. For assembling our dataset, each Galician audio-transcription pair has been aligned with the corresponding English text. For this dataset, we used the same method for measuring translation quality, achieving an average score of 0.76. After cleaning and deduplication, this dataset contains around ten hours of audio. Table \ref{tab:data} shows more details about the data.

\subsubsection{Modelling [GL-EN]}

We first employed Whisper to train an “end-to-end” speech translation system.
Whisper is a set of strong automatic speech recognition (ASR) architectures, trained on multilingual and multitask audio data. They can be further fine-tuned for speech translation. It supports Galician audio and text, making it a good choice for our data. Given our compute limitations, we experimented with two different backbones: \textit{whisper-small} and \textit{whisper-large-v3-turbo}, a simplified architecture of \textit{whisper-large} with fewer parameters in the decoder section. We fine-tuned both models over our two datasets (cf. Section \ref{sec:data-galician}).

To further improve our “end-to-end” results, we trained a “cascaded” system which splits the speech translation task into two consecutive steps (cf. Section~\ref{sec:e2e-vs-cascaded}). Intuitively, this separation allows each model to specialise in a specific step of the pipeline, while adding one extra level of explainability to the whole process. The first module consists of a \textit{whisper-large-v3-turbo}, this time in transcription mode, for generating Galician text given the input audio. Thereafter, on the same train split, we fine-tuned the MT model \textit{NLLB-200-distilled-600M} on Galician-to-English text translation.

Inference was performed using the Transformers library. More specifically, we used its pipeline functionality to encapsulate pre-processing and post-processing steps. Training and inference were run on one RTX 4090 GPU.

\subsubsection{Evaluation [GL-EN]}

For the \textit{FLEURS-SpeechT-GL-EN} dataset, the most performant “end-to-end” system was based on \textit{whisper-small}, achieving a BLEU score of 22.62 and a ChrF++ score of 46.11. For the \textit{OpenHQ-SpeechT-GL-EN} dataset, \textit{whisper-large-v3-turbo} was better, with a BLEU score of 55.65 and a ChrF++ score of 72.19. Regarding our cascaded system for \textit{FLEURS-SpeechT-GL-EN}, after using the MT model to translate the transcription generated by the STT model, we obtained a BLEU score of 37.19 and a ChrF++ score of 61.33. For \textit{OpenHQ-SpeechT-GL-EN}, the cascaded approach resulted in a BLEU score of 66.05 and a ChrF++ score of 79.58. The cascaded approach, despite being more computationally demanding, allows for a better specialization for each part of the system, hence generating significantly better results.


\subsection{Indonesian-to-English}

\subsubsection{Data [ID-EN]}

The dataset was compiled by extracting the English and Indonesian datasets from CoVoST2 \citep{Wang2021-CoVoST2}, a speech dataset in 21 languages, including Indonesian. Columns besides the index, Indonesian audio with its transcription, and English transcription were removed. The next preprocessing step was checking duplicate indices within each split and identifying overlapping indices across the splits. This dataset was first used to train an “end-to-end” speech-translation system. For speech translation using a “cascaded” system, two models were trained: an automatic speech recognition (ASR) model and a machine translation (MT) model. Hence, the audio and transcription columns were used to train the ASR model, while textual source and target columns were used to train the text-to-text MT model.

\subsubsection{Modelling [ID-EN]}

We employed different approaches for the speech-translation tasks, an “end-to-end” system and a “cascaded” system (cf. Section \ref{sec:e2e-vs-cascaded}). The pretrained model \textit{whisper-small} was used for training the “end-to-end” system. We fine-tuned the model with the Indonesian audio and English transcription directly. Meanwhile, in the “cascaded” system, the model was fine-tuned to predict the audio transcription in the same language, which is Indonesian. As a “cascaded” system requires an MT model for translating Indonesian transcription into English, we fine-tuned \textit{nllb-200-distilled-600M}, with batch size of 2 and gradient accumulation steps of 8 to simulate the effect of larger batch sizes. The model was trained for 10 epochs, saving the best epoch in the end. 

For inference, we used Faster-Whisper for both translation and transcription with Whisper after converting the model into the CTranslate2 formate with float16 quantization. Similarly, for textual translation with NLLB-200, we used CTranslate2 with float16 quantization. Training was run on the T4 GPU from Google Colab, while inference used an RTX 2000 Ada GPU.


\subsubsection{Evaluation [ID-EN]}

The evaluation result of the “cascaded” system outperforms the “end-to-end” system on the \textit{CoVoST2} test set. The “end-to-end” system achieved a BLUE score of 37.02 and ChrF++ score of 56.04 after fine-tuning Whisper Small, considerably improving the baseline (whose scores were BLUE 26.00 and ChrF++ 44.00). The “cascaded” system, which fine-tuned both Whisper for transcription and NLLB-200 for translation achieved 48.60 BLEU score and 65.10 ChrF++ score, which outperforms the fine-tuned end-to-end model.


\subsection{Spanish-to-Japanese}

\subsubsection{Data [ES-JA]}
\label{sec:data-es-ja}

The foundational dataset is \textit{VoxPopuli} \citep{Wang2021-VoxPopuli}, from which we extracted audio and Spanish transcriptions. We generated Japanese translations using OPUS models \citep{Tiedemann2020-OPUS-MT}, initially translating from Spanish to English and then from English to Japanese. While multilingual options existed, this two-step approach was chosen due to the strong performance of high-resource language pairs.
Post-processing was necessary to refine the dataset. First, we removed blank spaces, which are not typical in Japanese writing, ensuring proper formatting and consistency. Then, we eliminated empty texts and employed quality estimation with a threshold of 0.7 to filter out low-quality translations, using the CometKiwi (\textit{wmt23-cometkiwi-da-xl}) model. This process helped maintain alignment between the audio, transcriptions, and translations, resulting in a final dataset of approximately 12.7k rows.
Regarding content, the dataset consists of European Parliament event recordings featuring various Spanish accents. As a result, models trained on this data are likely to perform better in similar parliamentary or formal discourse scenarios.

\subsubsection{Modelling [ES-JA]}

We built two systems for the Spanish-to-Japanese (ES-JA) speech translation task, an “end-to-end” system and a “cascaded” system (cf. Section \ref{sec:e2e-vs-cascaded}). The backbone of the “end-to-end” model is \textit{whisper-small}, which has been trained on the ES-JA VoxPopuli dataset \ref{sec:data-es-ja}. This \textit{whisper-small} model has been fine-tuned specifically for direct speech-to-text translation, meaning that the Spanish audio is encoded and directly decoded into Japanese, without requiring any intermediate transcription step. This approach offers a simpler architecture and a lower computational cost, since only one model is used, training and inference are more efficient.

On the contrary, the “cascaded” approach involves two separate models, (i) the \textit{whisper-small} for transcribing Spanish audio into text, and (ii) the \textit{nllb-200-distilled-600M} for translating the transcribed Spanish text into Japanese. While this method is more resource-intensive, it allows independent optimization of each component.

For inference, both approaches process Spanish audio inputs into Japanese text output. In the “end-to-end” approach, the model directly translates Spanish speech into Japanese in a single step (only one model is executed, taking less time and resources). However, in the “cascaded” approach there is a sequential process: The output of the model that transcribes Spanish into text is the input to the model that translates Spanish into Japanese (Two models are used, making it possible to optimize each of them but using more resources), providing a higher quality in terms of translation quality metrics. For this, we used the Hugging Face Transformers library pipelines: “automatic-speech-recognition” and “translation”. As for infrastructure, we conducted both training and inference of the models on one RTX~4090 GPU.


\subsubsection{Evaluation [ES-JA]}

The evaluation of the Spanish-to-Japanese translation models reveals a performance gap between the “end-to-end” and “cascaded” approaches. The “end-to-end” model scores on the test split indicate room for improvement, achieving a BLEU score of 20.86, a ChrF++ score of 23.36, and a COMET score of 77.7. This suggests that while the translations maintain some coherence, they lack the precision and fluency. In contrast, the “cascaded” approach outperforms the “end-to-end” model across all metrics. This system reaches a BLEU score of 35.32, a ChrF++ score of 32.82, and a COMET score of 89.86, demonstrating superior lexical and syntactic alignment with reference translations.



\begin{table*}[t]
\centering
\fontsize{11pt}{11pt}\selectfont
\begin{tabular}{lllllllllllll}
\toprule
\multicolumn{1}{c}{\textbf{task}} & \multicolumn{2}{c}{\textbf{Mir}} & \multicolumn{2}{c}{\textbf{Lai}} & \multicolumn{2}{c}{\textbf{Ziegen.}} & \multicolumn{2}{c}{\textbf{Cao}} & \multicolumn{2}{c}{\textbf{Alva-Man.}} & \multicolumn{1}{c}{\textbf{avg.}} & \textbf{\begin{tabular}[c]{@{}l@{}}avg.\\ rank\end{tabular}} \\
\multicolumn{1}{c}{\textbf{metrics}} & \multicolumn{1}{c}{\textbf{cor.}} & \multicolumn{1}{c}{\textbf{p-v.}} & \multicolumn{1}{c}{\textbf{cor.}} & \multicolumn{1}{c}{\textbf{p-v.}} & \multicolumn{1}{c}{\textbf{cor.}} & \multicolumn{1}{c}{\textbf{p-v.}} & \multicolumn{1}{c}{\textbf{cor.}} & \multicolumn{1}{c}{\textbf{p-v.}} & \multicolumn{1}{c}{\textbf{cor.}} & \multicolumn{1}{c}{\textbf{p-v.}} &  &  \\ \midrule
\textbf{S-Bleu} & 0.50 & 0.0 & 0.47 & 0.0 & 0.59 & 0.0 & 0.58 & 0.0 & 0.68 & 0.0 & 0.57 & 5.8 \\
\textbf{R-Bleu} & -- & -- & 0.27 & 0.0 & 0.30 & 0.0 & -- & -- & -- & -- & - &  \\
\textbf{S-Meteor} & 0.49 & 0.0 & 0.48 & 0.0 & 0.61 & 0.0 & 0.57 & 0.0 & 0.64 & 0.0 & 0.56 & 6.1 \\
\textbf{R-Meteor} & -- & -- & 0.34 & 0.0 & 0.26 & 0.0 & -- & -- & -- & -- & - &  \\
\textbf{S-Bertscore} & \textbf{0.53} & 0.0 & {\ul 0.80} & 0.0 & \textbf{0.70} & 0.0 & {\ul 0.66} & 0.0 & {\ul0.78} & 0.0 & \textbf{0.69} & \textbf{1.7} \\
\textbf{R-Bertscore} & -- & -- & 0.51 & 0.0 & 0.38 & 0.0 & -- & -- & -- & -- & - &  \\
\textbf{S-Bleurt} & {\ul 0.52} & 0.0 & {\ul 0.80} & 0.0 & 0.60 & 0.0 & \textbf{0.70} & 0.0 & \textbf{0.80} & 0.0 & {\ul 0.68} & {\ul 2.3} \\
\textbf{R-Bleurt} & -- & -- & 0.59 & 0.0 & -0.05 & 0.13 & -- & -- & -- & -- & - &  \\
\textbf{S-Cosine} & 0.51 & 0.0 & 0.69 & 0.0 & {\ul 0.62} & 0.0 & 0.61 & 0.0 & 0.65 & 0.0 & 0.62 & 4.4 \\
\textbf{R-Cosine} & -- & -- & 0.40 & 0.0 & 0.29 & 0.0 & -- & -- & -- & -- & - & \\ \midrule
\textbf{QuestEval} & 0.23 & 0.0 & 0.25 & 0.0 & 0.49 & 0.0 & 0.47 & 0.0 & 0.62 & 0.0 & 0.41 & 9.0 \\
\textbf{LLaMa3} & 0.36 & 0.0 & \textbf{0.84} & 0.0 & {\ul{0.62}} & 0.0 & 0.61 & 0.0 &  0.76 & 0.0 & 0.64 & 3.6 \\
\textbf{our (3b)} & 0.49 & 0.0 & 0.73 & 0.0 & 0.54 & 0.0 & 0.53 & 0.0 & 0.7 & 0.0 & 0.60 & 5.8 \\
\textbf{our (8b)} & 0.48 & 0.0 & 0.73 & 0.0 & 0.52 & 0.0 & 0.53 & 0.0 & 0.7 & 0.0 & 0.59 & 6.3 \\  \bottomrule
\end{tabular}
\caption{Pearson correlation on human evaluation on system output. `R-': reference-based. `S-': source-based.}
\label{tab:sys}
\end{table*}



\begin{table}%[]
\centering
\fontsize{11pt}{11pt}\selectfont
\begin{tabular}{llllll}
\toprule
\multicolumn{1}{c}{\textbf{task}} & \multicolumn{1}{c}{\textbf{Lai}} & \multicolumn{1}{c}{\textbf{Zei.}} & \multicolumn{1}{c}{\textbf{Scia.}} & \textbf{} & \textbf{} \\ 
\multicolumn{1}{c}{\textbf{metrics}} & \multicolumn{1}{c}{\textbf{cor.}} & \multicolumn{1}{c}{\textbf{cor.}} & \multicolumn{1}{c}{\textbf{cor.}} & \textbf{avg.} & \textbf{\begin{tabular}[c]{@{}l@{}}avg.\\ rank\end{tabular}} \\ \midrule
\textbf{S-Bleu} & 0.40 & 0.40 & 0.19* & 0.33 & 7.67 \\
\textbf{S-Meteor} & 0.41 & 0.42 & 0.16* & 0.33 & 7.33 \\
\textbf{S-BertS.} & {\ul0.58} & 0.47 & 0.31 & 0.45 & 3.67 \\
\textbf{S-Bleurt} & 0.45 & {\ul 0.54} & {\ul 0.37} & 0.45 & {\ul 3.33} \\
\textbf{S-Cosine} & 0.56 & 0.52 & 0.3 & {\ul 0.46} & {\ul 3.33} \\ \midrule
\textbf{QuestE.} & 0.27 & 0.35 & 0.06* & 0.23 & 9.00 \\
\textbf{LlaMA3} & \textbf{0.6} & \textbf{0.67} & \textbf{0.51} & \textbf{0.59} & \textbf{1.0} \\
\textbf{Our (3b)} & 0.51 & 0.49 & 0.23* & 0.39 & 4.83 \\
\textbf{Our (8b)} & 0.52 & 0.49 & 0.22* & 0.43 & 4.83 \\ \bottomrule
\end{tabular}
\caption{Pearson correlation on human ratings on reference output. *not significant; we cannot reject the null hypothesis of zero correlation}
\label{tab:ref}
\end{table}


\begin{table*}%[]
\centering
\fontsize{11pt}{11pt}\selectfont
\begin{tabular}{lllllllll}
\toprule
\textbf{task} & \multicolumn{1}{c}{\textbf{ALL}} & \multicolumn{1}{c}{\textbf{sentiment}} & \multicolumn{1}{c}{\textbf{detoxify}} & \multicolumn{1}{c}{\textbf{catchy}} & \multicolumn{1}{c}{\textbf{polite}} & \multicolumn{1}{c}{\textbf{persuasive}} & \multicolumn{1}{c}{\textbf{formal}} & \textbf{\begin{tabular}[c]{@{}l@{}}avg. \\ rank\end{tabular}} \\
\textbf{metrics} & \multicolumn{1}{c}{\textbf{cor.}} & \multicolumn{1}{c}{\textbf{cor.}} & \multicolumn{1}{c}{\textbf{cor.}} & \multicolumn{1}{c}{\textbf{cor.}} & \multicolumn{1}{c}{\textbf{cor.}} & \multicolumn{1}{c}{\textbf{cor.}} & \multicolumn{1}{c}{\textbf{cor.}} &  \\ \midrule
\textbf{S-Bleu} & -0.17 & -0.82 & -0.45 & -0.12* & -0.1* & -0.05 & -0.21 & 8.42 \\
\textbf{R-Bleu} & - & -0.5 & -0.45 &  &  &  &  &  \\
\textbf{S-Meteor} & -0.07* & -0.55 & -0.4 & -0.01* & 0.1* & -0.16 & -0.04* & 7.67 \\
\textbf{R-Meteor} & - & -0.17* & -0.39 & - & - & - & - & - \\
\textbf{S-BertScore} & 0.11 & -0.38 & -0.07* & -0.17* & 0.28 & 0.12 & 0.25 & 6.0 \\
\textbf{R-BertScore} & - & -0.02* & -0.21* & - & - & - & - & - \\
\textbf{S-Bleurt} & 0.29 & 0.05* & 0.45 & 0.06* & 0.29 & 0.23 & 0.46 & 4.2 \\
\textbf{R-Bleurt} & - &  0.21 & 0.38 & - & - & - & - & - \\
\textbf{S-Cosine} & 0.01* & -0.5 & -0.13* & -0.19* & 0.05* & -0.05* & 0.15* & 7.42 \\
\textbf{R-Cosine} & - & -0.11* & -0.16* & - & - & - & - & - \\ \midrule
\textbf{QuestEval} & 0.21 & {\ul{0.29}} & 0.23 & 0.37 & 0.19* & 0.35 & 0.14* & 4.67 \\
\textbf{LlaMA3} & \textbf{0.82} & \textbf{0.80} & \textbf{0.72} & \textbf{0.84} & \textbf{0.84} & \textbf{0.90} & \textbf{0.88} & \textbf{1.00} \\
\textbf{Our (3b)} & 0.47 & -0.11* & 0.37 & 0.61 & 0.53 & 0.54 & 0.66 & 3.5 \\
\textbf{Our (8b)} & {\ul{0.57}} & 0.09* & {\ul 0.49} & {\ul 0.72} & {\ul 0.64} & {\ul 0.62} & {\ul 0.67} & {\ul 2.17} \\ \bottomrule
\end{tabular}
\caption{Pearson correlation on human ratings on our constructed test set. 'R-': reference-based. 'S-': source-based. *not significant; we cannot reject the null hypothesis of zero correlation}
\label{tab:con}
\end{table*}

\section{Results}
We benchmark the different metrics on the different datasets using correlation to human judgement. For content preservation, we show results split on data with system output, reference output and our constructed test set: we show that the data source for evaluation leads to different conclusions on the metrics. In addition, we examine whether the metrics can rank style transfer systems similar to humans. On style strength, we likewise show correlations between human judgment and zero-shot evaluation approaches. When applicable, we summarize results by reporting the average correlation. And the average ranking of the metric per dataset (by ranking which metric obtains the highest correlation to human judgement per dataset). 

\subsection{Content preservation}
\paragraph{How do data sources affect the conclusion on best metric?}
The conclusions about the metrics' performance change radically depending on whether we use system output data, reference output, or our constructed test set. Ideally, a good metric correlates highly with humans on any data source. Ideally, for meta-evaluation, a metric should correlate consistently across all data sources, but the following shows that the correlations indicate different things, and the conclusion on the best metric should be drawn carefully.

Looking at the metrics correlations with humans on the data source with system output (Table~\ref{tab:sys}), we see a relatively high correlation for many of the metrics on many tasks. The overall best metrics are S-BertScore and S-BLEURT (avg+avg rank). We see no notable difference in our method of using the 3B or 8B model as the backbone.

Examining the average correlations based on data with reference output (Table~\ref{tab:ref}), now the zero-shoot prompting with LlaMA3 70B is the best-performing approach ($0.59$ avg). Tied for second place are source-based cosine embedding ($0.46$ avg), BLEURT ($0.45$ avg) and BertScore ($0.45$ avg). Our method follows on a 5. place: here, the 8b version (($0.43$ avg)) shows a bit stronger results than 3b ($0.39$ avg). The fact that the conclusions change, whether looking at reference or system output, confirms the observations made by \citet{scialom-etal-2021-questeval} on simplicity transfer.   

Now consider the results on our test set (Table~\ref{tab:con}): Several metrics show low or no correlation; we even see a significantly negative correlation for some metrics on ALL (BLEU) and for specific subparts of our test set for BLEU, Meteor, BertScore, Cosine. On the other end, LlaMA3 70B is again performing best, showing strong results ($0.82$ in ALL). The runner-up is now our 8B method, with a gap to the 3B version ($0.57$ vs $0.47$ in ALL). Note our method still shows zero correlation for the sentiment task. After, ranks BLEURT ($0.29$), QuestEval ($0.21$), BertScore ($0.11$), Cosine ($0.01$).  

On our test set, we find that some metrics that correlate relatively well on the other datasets, now exhibit low correlation. Hence, with our test set, we can now support the logical reasoning with data evidence: Evaluation of content preservation for style transfer needs to take the style shift into account. This conclusion could not be drawn using the existing data sources: We hypothesise that for the data with system-based output, successful output happens to be very similar to the source sentence and vice versa, and reference-based output might not contain server mistakes as they are gold references. Thus, none of the existing data sources tests the limits of the metrics.  


\paragraph{How do reference-based metrics compare to source-based ones?} Reference-based metrics show a lower correlation than the source-based counterpart for all metrics on both datasets with ratings on references (Table~\ref{tab:sys}). As discussed previously, reference-based metrics for style transfer have the drawback that many different good solutions on a rewrite might exist and not only one similar to a reference.


\paragraph{How well can the metrics rank the performance of style transfer methods?}
We compare the metrics' ability to judge the best style transfer methods w.r.t. the human annotations: Several of the data sources contain samples from different style transfer systems. In order to use metrics to assess the quality of the style transfer system, metrics should correctly find the best-performing system. Hence, we evaluate whether the metrics for content preservation provide the same system ranking as human evaluators. We take the mean of the score for every output on each system and the mean of the human annotations; we compare the systems using the Kendall's Tau correlation. 

We find only the evaluation using the dataset Mir, Lai, and Ziegen to result in significant correlations, probably because of sparsity in a number of system tests (App.~\ref{app:dataset}). Our method (8b) is the only metric providing a perfect ranking of the style transfer system on the Lai data, and Llama3 70B the only one on the Ziegen data. Results in App.~\ref{app:results}. 


\subsection{Style strength results}
%Evaluating style strengths is a challenging task. 
Llama3 70B shows better overall results than our method. However, our method scores higher than Llama3 70B on 2 out of 6 datasets, but it also exhibits zero correlation on one task (Table~\ref{tab:styleresults}).%More work i s needed on evaluating style strengths. 
 
\begin{table}%[]
\fontsize{11pt}{11pt}\selectfont
\begin{tabular}{lccc}
\toprule
\multicolumn{1}{c}{\textbf{}} & \textbf{LlaMA3} & \textbf{Our (3b)} & \textbf{Our (8b)} \\ \midrule
\textbf{Mir} & 0.46 & 0.54 & \textbf{0.57} \\
\textbf{Lai} & \textbf{0.57} & 0.18 & 0.19 \\
\textbf{Ziegen.} & 0.25 & 0.27 & \textbf{0.32} \\
\textbf{Alva-M.} & \textbf{0.59} & 0.03* & 0.02* \\
\textbf{Scialom} & \textbf{0.62} & 0.45 & 0.44 \\
\textbf{\begin{tabular}[c]{@{}l@{}}Our Test\end{tabular}} & \textbf{0.63} & 0.46 & 0.48 \\ \bottomrule
\end{tabular}
\caption{Style strength: Pearson correlation to human ratings. *not significant; we cannot reject the null hypothesis of zero corelation}
\label{tab:styleresults}
\end{table}

\subsection{Ablation}
We conduct several runs of the methods using LLMs with variations in instructions/prompts (App.~\ref{app:method}). We observe that the lower the correlation on a task, the higher the variation between the different runs. For our method, we only observe low variance between the runs.
None of the variations leads to different conclusions of the meta-evaluation. Results in App.~\ref{app:results}.


\subsection{Arabic-to-English \& Bengali-to-English}

Due to the similarity of the Arabic-to-English and Bengali-to-English projects, we combine them in one section. Moreover, unlike the aforementioned projects, these two project fine-tuned models for the “end-to-end” system only, whilBengali-to-English showe they used baselines directly without fine-tuning for the “cascaded” system.

\subsubsection{Data [AR-EN \& BN-EN]}

The datasets used in the Arabic-to-English and Bengali-to-English projects is the Arabic-to-English are subsets of the FLEURS dataset \citep{Conneau2023-FLEURS}. FLEURS (Fluent and Less-Extensive Universal Recognition of Speech) is a multilingual dataset designed for speech recognition and translation tasks. The dataset includes audio recordings in 102 including Arabic and Bengali, paired with their corresponding English translations. The data is split into training, validation, and test sets to facilitate model training and evaluation. As the dataset includes both the transcriptions and translations, it is useful for “end-to-end” speech translation tasks, as well as “cascaded” approaches that involve separate speech recognition and machine translation models. Table \ref{tab:data} illustrates more details about the used data.

\subsubsection{Modelling [AR-EN \& BN-EN]}

Two approaches were employed for the Arabic-to-English and Bengali-to-English translation tasks:

End-to-End Model: The model utilizes whisper-small model, which is a pre-trained speech-to-text model capable of handling “end-to-end” speech translation. This model directly translates Arabic or Bengali speech into English text without intermediate steps. The model was fine-tuned on the FLEURS dataset.

Cascaded Model: This approach combines two models: (i) Automatic Speech Recognition (ASR) using the Whisper model to transcribe Arabic speech into Arabic text, and (ii) Machine Translation (MT) using NLLB-200 to translate the transcribed Arabic or Bengali text into English. Unlike other projects, the Arabic and Bengali projects did not fine-tune the ASR and MT models; instead, they directly used the pretrained baselines for the “cascaded” system.

For Arabic-to-English inference, the Hugging Face Transformers library was used for both speech-to-text transcription and text translation tasks, as well as “end-to-end” speech translation. For Bengali-to-English “end-to-end” translation, the FasterWhisper library (based on CTranslate2) was used after converting the model with float16 quantization. Training and inference utilized Google Colab, as well as GPU P100 on Kaggle and a multi-GPU setup comprising two NVIDIA T4 GPUs on Kaggle.


\subsubsection{Evaluation [AR-EN \& BN-EN]}

The results of English-to-Arabic speech translation indicate that the “cascaded” model outperforms the “end-to-end” model in terms of translation quality. The “cascaded” model achieved a BLEU score of 24.38 and a chrF++ score of 51.79, compared to the “end-to-end” model's BLEU score of 15.06 and chrF++ score of 39.03. This suggests that separating the tasks of ASR and MT leads to better translation accuracy, as each model can specialize in its respective task. However, the “end-to-end” model offers the advantage of simplicity and faster inference, making it a viable option for scenarios where speed is prioritized over translation quality. On the contrary, the Bengali-to-English models show a different outcome, probably due to use of a baseline transcription model rather than fine-tuning it. Table \ref{tab:results} illustrates the results of all the projects.


\section{Conclusions}

The SpeechT mentorship brought together several practitioners and students from diverse companies and institutions across the world to explore speech translation. The participants have diverse backgrounds, ranging from generic software knowledge to text-to-text MT experience. Ultimately, five participants have made successful submissions and contributed to this work (cf. Section \ref{sec:contributions}).

Successful submissions incorporated a range of techniques. In Particular, participants experimented with synthetic data generation with large language models (e.g. GPT4) and MT models (e.g. OPUS). The focus of most of the experiments was comparing the speech translation performance of “end-to-end” systems with “cascaded” systems (cf. Section \ref{sec:e2e-vs-cascaded}). For this purpose, the participants fine-tuned pretrained models, including Whisper and NLLB-200. While the “end-to-end” systems fine-tuned Whisper for direct speech translation, building the “cascaded” systems involved two steps, namely fine-tuning Whisper for ASR, and then employing an MT model (e.g. OPUS or NLLB) for translation of the generated transcription. As illustrated in Table \ref{tab:results}, “cascaded” systems outperformed “end-to-end” across most language pairs.

In conclusion, this mentorship has enabled the participants to experiment with various system designs and fine-tuning strategies, deepening their understanding of the speech translation area through hands-on practice.






\section{Contributions}
\label{sec:contributions}

\begin{itemize}
    \item \textbf{Yasmin Moslem:} Organizer and mentor of \textit{SpeechT} mentorship in Speech Translation
\end{itemize}

\paragraph{Participants} \hspace{-1em} (alphabetically ordered)
\begin{itemize}
    \item \textbf{Farah Abdou:} Participant, Arabic-to-English Speech Translation
    
    \item \textbf{Juan Julián Cea Morán:} Participant, Galician-to-English Speech Translation
    \item \textbf{Mariano Gonzalez-Gomez:} Participant, Spanish-to-Japanese Speech Translation
    \item \textbf{Muhammad Hazim Al Farouq:} Participant, Indonesian-to-English Speech Translation
    \item \textbf{Satarupa Deb:} Participant, Bengali-to-English Speech Translation
\end{itemize}


\newpage

\bibliography{paperpile}


\end{document}
