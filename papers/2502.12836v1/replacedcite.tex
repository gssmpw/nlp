\section{Related Works}
In this section, we review state-of-the-art methods for LLM-based physiological time-series data analysis. We categorize existing approaches into three main groups: (i) direct integration of raw time-series sensor data into textual prompts, (ii) textual prompting using extracted features from time-series data, and (iii) multimodal prompting, which leverages visual representations of physiological signals. 

Studies utilizing textual prompting with direct integration of time-series data represent the numerical data in a structured text format for the model. One of the earliest studies in this context ____ demonstrated the use of LLMs for tasks such as arrhythmia detection and physical activity recognition by directly integrating interbeat interval sequences and accelerometer sensor data into prompts. Similarly, Kim et al. ____ explored LLMs for health prediction by incorporating various sensor readings into queries, such as step count, calories burned, and resting heart rate. Chan et al. ____ proposed an LLM-based framework for tasks such as semantic segmentation, boundary detection, and anomaly detection, embedding biosignals directly into model inputs. Additionally, Imran et al. ____ employed LLMs for physical activity analysis, where gyroscope and accelerometer sensor data were incorporated into prompts. LLMs have also been applied to cardiac activity monitoring, metabolic health prediction, and sleep detection by leveraging wearable sensor data  ____. However, LLMs are not inherently specialized for processing long raw numerical sequences. Physiological time-series data often consist of thousands or even millions of data points, and representing them as textual input for LLMs can result in poor performance. This approach also exceeds the token limits of most LLMs and leads to high computational costs due to the large number of tokens.

To address these limitations, some studies have instead extracted numerical features from medical time-series data and integrated them into textual prompts. Cosentino et al. ____ fine-tuned the Gemini models ____ for sleep and fitness monitoring by incorporating physiological parameters extracted from wearable data into prompts. Similarly, PhysioLLM ____ was introduced to generate personalized health insights using LLMs that integrate wearable-derived features with contextual information. An LLM-based blood pressure estimation model ____ was also developed, integrating extracted physiological features from ECG and PPG signals. Furthermore, Yu et al. ____ proposed an arrhythmia detection approach that incorporates physiological and morphological features extracted from ECG signals. While these methods address the challenge of excessive token usage in prompting, they still suffer from generic and unreliable outputs. Since LLMs are inherently designed to process and generate human-like text by tokenizing input data and predicting subsequent tokens, their responses—despite advanced prompt engineering or fine-tuning—often lack analytical rigor and reliability, which is required for robust physiological data interpretation.

More recently, multimodal prompting has been explored as an alternative approach by incorporating visual representations of physiological signals ____. Yoon et al. ____ proposed a visual prompting approach, providing the model with images of signals such as accelerometer, ECG, Electromyogram (EMG), and respiration data for tasks including human activity recognition, arrhythmia diagnosis, hand gesture recognition, and stress detection. Similarly, Tang et al. ____ applied visual prompting by providing the model with PPG signal images for heart rate estimation. However, similar to feature-based prompting methods, these multimodal approaches also suffer from generating generic responses rather than analytically grounded insights, as LLMs are not inherently designed to directly interpret waveforms from continuous signals.

We believe the following challenges in the literature need to be addressed:
\begin{enumerate}
    \item Existing LLM-based health time-series analysis methods struggle with handling large numerical sequences, often exceeding LLM token limits, leading to high computational costs and performance degradation.
    \item LLMs are not inherently designed for numerical reasoning, resulting in generic and analytically unreliable outputs when utilized for direct time-series analysis in healthcare applications.
\end{enumerate}




 % To address the problem of generic responses and enhance numerical reasoning in LLMs, Merrill et al. ____ proposed employing code generation and execution to analyze health data from wearables, ensuring factual and reliable arithmetic precision. While promising, this approach lacks efficient mechanisms for essential health data processing tasks, such as handling missing data, mitigating noise, and removing artifacts. Additionally, LLMs are prone to generating faulty, incomplete, or incorrect code, which poses significant risks in health-related applications where accuracy is critical. Finally, the analyses produced by LLM-generated code often lack transparency, while explainability is crucial for trust and adoption by clinicians and researchers.


%