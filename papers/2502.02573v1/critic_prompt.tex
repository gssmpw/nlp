\begin{figure}[t!]
    \centering
    \begin{tikzpicture}
        % Define the background layer
        \pgfdeclarelayer{background}
        \pgfsetlayers{background,main}

        % Title
        \node[align=center, text=black, font=\large\bfseries] (title) {Critic's Initial Base Prompt Template};

        % Prompt 1
        \node[fill=red!30, draw=SlateGray2, rounded corners, inner sep=5pt, text width=\textwidth, below=of title, yshift=0.2cm] (prompt1) {
            \small
            \textit{You are a great expert in the optimization topic and search algorithms and will assist others in solving optimization problems.}
        };
        \node[above=0.25cm of prompt1.north west, anchor=west] {\textbf{Role Assignment}};

        % Prompt 2
        \node[fill=green!30, draw=SlateGray2, rounded corners, inner sep=5pt, text width=\textwidth, below=of prompt1, yshift=0.2cm] (prompt2) {
            \small 
            Your task is to provide guidance, suggestions, and assistance to a very smart AI agent for solving an optimization problem.
            
            The agent is to interact with an unknown function \( f(x,y) \) \((-1000 \leq x, y \leq 1000)\) with the objective of identifying the global maximum value. Here is the procedure the agent will adhere to:
            
            \begin{enumerate}
                \item \textit{Strategy Development}: The agent will begin by devising a comprehensive strategy to explore the space and tackle the problem.
                \item \textit{Point Selection (x, y)}: The agent will choose unique NEW points \([(x_1, y_1), (x_2, y_2), \ldots]\) for function evaluation, based on its strategy.
                \item \textit{Feedback Collection and Strategy Enhancement}: Once the values of \([(x_1, y_1, f_1), (x_2, y_2, f_2), \ldots]\) at the agent's selected points are revealed, the agent can refine its strategy using this feedback.
                \item \textit{Process Persistence}: The agent will continue this procedure for up to \textit{$QueryBudget$} queries (in the form of \((x_i, y_i)\)) or until it is confident that the global maximum has been identified.
            \end{enumerate}
            
            The agent should present its findings in the following way: \{...\}

            The agent will comply with the following rules: \{...\}

            Given this problem statement, your duty is to ensure that the agent identifies the global maximum value.
        };
        \node[above=0.25cm of prompt2.north west, anchor=west] {\textbf{Problem Definition}};

        % Prompt 3
        \node[fill=yellow!30, draw=SlateGray2, rounded corners, inner sep=5pt, text width=\textwidth, below=of prompt2, yshift=0.2cm] (prompt3) {
            \small
            To achieve your goal, please follow these guidelines:
            \begin{enumerate}
                \item After each step, you can critique the agent's chosen coordinates or its strategy. It is essential that you offer constructive criticism to improve its next moves.
                \item It's important to remember that the space is vast, and there may be several LOCAL maximums, so you must help the agent avoid mistaking local maximum values for the answer \{...\}
                \item You can offer suggestions and brainstorming to assist the agent in its task. Remember that the agent is very smart, so do not describe what the agent has already chosen or done! Limit your responses to constructive criticism.
                \item Note: Finding a value in the range of \([0.95 \times (\text{Global Max}), \text{Global Max}]\) is equal to solving the problem, so discourage the agent to spend time on finding values that have small differences.
                \item After every iteration, list the potential issues with the agent's strategy and decision so far.
                \item Ensure that your responses are concise and to the point. Do not provide unnecessarily long responses.
            \end{enumerate}
            \{...\}
        };
        \node[above=0.25cm of prompt3.north west, anchor=west] {\textbf{General Guidelines}};

        % Prompt 4
        \node[fill=cyan!30, draw=SlateGray2, rounded corners, inner sep=5pt, text width=\textwidth, below=of prompt3, yshift=0.2cm] (prompt4) {
            \small
            Here is the Agent's response: $<Thesis_1>$ and the corresponding results: $<Observations_1>$
        };
        \node[above=0.25cm of prompt4.north west, anchor=west] {\textbf{Agent's Response \& the Corresponding Results}};

        % % Prompt 5
        % \node[fill=cyan!30, draw=SlateGray2, rounded corners, inner sep=5pt, text width=\textwidth, below=of prompt4, yshift=0.2cm] (prompt5) {
        %     The space is vast, and there will be several LOCAL maximums, so avoid choosing them as the answer. Make sure that you explore the space enough to ensure that your answer represents the GLOBAL maximum   
        %     ...
        % };
        % \node[above=0.2cm of prompt5.north west, anchor=west] {\textbf{General Hints}};

        % % Prompt 6
        % \node[fill=cyan!30, draw=SlateGray2, rounded corners, inner sep=5pt, text width=\textwidth, below=of prompt4, yshift=0.2cm] (prompt5) {
        %     \small

        % };
        % \node[above=0.2cm of prompt5.north west, anchor=west] {\textbf{Examples}};

        % Prompt 7
        \node[fill=lightgray!10, draw=SlateGray2, rounded corners, inner sep=5pt, text width=\textwidth, below=of prompt4, yshift=0.2cm] (prompt5) {
            \small
            Now, given all the info, review agent's response, make your criticism and suggestions, and detect potential issues ...
        };
        \node[above=0.2cm of prompt5.north west, anchor=west] {\textbf{Start Command}};

        % Background rectangle
        \begin{pgfonlayer}{background}
            \node[fit=(title)(prompt1)(prompt2)(prompt3)(prompt4)(prompt5), fill=blue!10, rounded corners, inner sep=10pt] {};
        \end{pgfonlayer}
    \end{tikzpicture}
    \vspace{-2em}
    \caption{Initial prompt template used for the Critic in ACE}
    \label{fig:prompts:critic}
\end{figure}

\begin{figure}[t!]
    \centering
    \begin{tikzpicture}
        % Define the background layer
        \pgfdeclarelayer{background}
        \pgfsetlayers{background,main}

        % Title
        \node[align=center, text=black, font=\large\bfseries] (title) {Critic's Transitional Prompt Template};

        % Prompt 4
        \node[fill=cyan!30, draw=SlateGray2, rounded corners, inner sep=5pt, text width=\textwidth, below=of title, yshift=0.2cm] (prompt4) {
            \small
            Here is the Agent's response: $<Thesis_i>$ and the corresponding results: $<Observations_i>$
        };
        \node[above=0.25cm of prompt4.north west, anchor=west] {\textbf{Agent's Response \& the Corresponding Results}};

        % % Prompt 5
        % \node[fill=cyan!30, draw=SlateGray2, rounded corners, inner sep=5pt, text width=\textwidth, below=of prompt4, yshift=0.2cm] (prompt5) {
        %     The space is vast, and there will be several LOCAL maximums, so avoid choosing them as the answer. Make sure that you explore the space enough to ensure that your answer represents the GLOBAL maximum   
        %     ...
        % };
        % \node[above=0.2cm of prompt5.north west, anchor=west] {\textbf{General Hints}};

        % % Prompt 6
        % \node[fill=cyan!30, draw=SlateGray2, rounded corners, inner sep=5pt, text width=\textwidth, below=of prompt4, yshift=0.2cm] (prompt5) {
        %     \small

        % };
        % \node[above=0.2cm of prompt5.north west, anchor=west] {\textbf{Examples}};

        % Prompt 7
        \node[fill=lightgray!10, draw=SlateGray2, rounded corners, inner sep=5pt, text width=\textwidth, below=of prompt4, yshift=0.2cm] (prompt5) {
            \small
            Now, given all the info, review agent's response, make your criticism and suggestions, and detect potential issues ...
        };
        \node[above=0.2cm of prompt5.north west, anchor=west] {\textbf{Start Command}};

        % Background rectangle
        \begin{pgfonlayer}{background}
            \node[fit=(title)(prompt4)(prompt5), fill=blue!10, rounded corners, inner sep=10pt] {};
        \end{pgfonlayer}
    \end{tikzpicture}
    \vspace{-2em}
    \caption{Transitional prompt template used for the Critic in ACE}
    \label{fig:prompts:critic:transitional}
\end{figure}