\section{Background: Reasoning and Dialectics}
\label{sec:background}

\textbf{Reasoning:} Despite considerable achievements, LLMs' reasoning capability continues to be a subject of intense debate within the AI research community. A key challenge lies in reaching a consensus on what reasoning entails, how it should be defined, and how it can be reliably measured. Interestingly, the concept of reasoning is not new. The domain of philosophy has a rich tradition of exploring and formalizing reasoning through centuries of discourse
~\cite{aristotle_prior_analytics, aristotle_metaphysics, plato_republic, descartes_meditations, hume_treatise_human_nature, kant_critique_pure_reason, mill_system_logic, hegel_phenomenology_spirit, nietzsche_beyond_good_evil, wittgenstein_tractatus, heidegger_being_time, popper_logic_scientific_discovery, kuhn_structure_scientific_revolutions, adorno_negative_dialectics}.
From ancient philosophers such as Aristotle, who developed formal logic as a foundation for reasoning~\cite{aristotle_prior_analytics, aristotle_metaphysics}, to more recent thinkers like Hegel, who introduced dialectics as a dynamic framework for understanding processes of thought~\cite{hegel_science_logic,hegel_phenomenology_spirit}, the philosophical study of reasoning has produced a wide range of influential theories and formal systems. These works not only define reasoning but also provide structured frameworks for improving and analyzing it.

\textbf{Dialectics:} As a method of reasoning and philosophical argumentation, dialectics involves the resolution of contradictions through a process of development and transformation. Rooted in ancient philosophy, dialectics was first formalized by thinkers like Socrates and Aristotle, who used it as a tool for logical inquiry. Over time, dialectics evolved into a broader philosophical framework, describing the dynamic process through which contradictions are identified, explored, and resolved~\cite{hegel_phenomenology_spirit,engels_dialectics_nature}. At its core, dialectical thinking posits that reality is composed of opposing forces or contradictions, and that these contradictions are not static but dynamic, evolving over time. The resolution of these contradictions leads to the emergence of new, higher forms of understanding or being.


\textbf{Hegelian Dialectics:} Introduced by the German philosopher Georg Wilhelm Friedrich Hegel, Hegelian Dialectics crystallizes the modern notion of dialectics by proposing a structured process of development through three stages: \textit{thesis}, \textit{antithesis}, and \textit{synthesis}~\cite{hegel_science_logic,hegel_phenomenology_spirit}. The thesis represents an initial idea or condition, the antithesis introduces a contradictory or opposing force, and the synthesis resolves the tension by merging elements of both into a higher, more comprehensive understanding. 
Hegel viewed this triadic process as the driving force of intellectual, historical, and societal progress, emphasizing that contradictions, which he calls "negations", are not merely obstacles but necessary components of growth and transformation. His dialectical framework has had profound influence across disciplines, from philosophy to political theory.

\textbf{Dialectics vs. Debate:} From the philosophical point of view, debate is competitive, aiming to persuade an audience of one position's superiority. While effective in contexts like politics or law, it often sacrifices deeper inquiry for rhetoric and winning. Dialectics, however, fosters a cooperative approach, treating opposing perspectives as opportunities for growth. Through structured dialogue, dialectics seeks deeper truths, as seen in the Socratic and Hegelian methods, encouraging intellectual humility and a shared pursuit of wisdom. While debate has been significant in philosophical traditions, figures like Socrates criticized its focus on persuasion over truth. Dialectics, with its emphasis on dialogue and synthesis, is regarded as superior for fostering intellectual growth.

In this work, inspired by Hegel's well-established framework, we demonstrate how the capabilities of LLMs can be enhanced by adapting dialectics.
