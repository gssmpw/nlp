\begin{figure}[t!]
    \centering
    \begin{tikzpicture}
        % Define the background layer
        \pgfdeclarelayer{background}
        \pgfsetlayers{background,main}

        % Title
        \node[align=center, text=black, font=\large\bfseries] (title) {Main Initial Prompt Template for LLM$^+$ and Other Schemes};
 
        % Prompt 1
        \node[fill=red!30, draw=SlateGray2, rounded corners, inner sep=5pt, text width=\textwidth, below=of title, yshift=0.2cm] (prompt1) {
            \small
            \textit{You are a great expert in the optimization topic and search algorithms.}
        };
        \node[above=0.25cm of prompt1.north west, anchor=west] {\textbf{Role Assignment}};

        % Prompt 2
        \node[fill=green!30, draw=SlateGray2, rounded corners, inner sep=5pt, text width=\textwidth, below=of prompt1, yshift=0.2cm] (prompt2) {
            \small
            You are tasked with examining an unknown function \( f(x,y) \) \((-1000 \leq x,y \leq 1000)\). You need to interact with the function \( f(x,y) \) in order to locate the global maximum value. Here's how to do it:
            \begin{enumerate}
                \item \textit{Define Your Strategy}: Start with creating a solid strategy to explore the space and solve the problem.
                \item \textit{Choose a Point (x, y)}: Based on your strategy, select unique NEW points \((x_1, y_1), (x_2, y_2), \ldots\) to evaluate the function.
                \item \textit{Get Feedback and Adjust Your Strategy}: After I reveal the values of \((x_1, y_1, f_1), (x_2, y_2, f_2), \ldots\) at your chosen points, adjust your strategy based on this feedback.
                \item \textit{Repeat the Process}: Continue this process for up to \textit{$QueryBudget$} queries (in the form of \((x_i, y_i)\)) or until you are confident that you have found the global maximum.
            \end{enumerate}
            
            \textit{Note}: Finding a value in the range of \([0.95 \times (\text{Global Max}), \text{Global Max}]\) is equal to solving the problem.
        };
        \node[above=0.25cm of prompt2.north west, anchor=west] {\textbf{Problem Definition}};

        % Prompt 3
        \node[fill=orange!30, draw=SlateGray2, rounded corners, inner sep=5pt, text width=\textwidth, below=of prompt2, yshift=0.2cm] (prompt3) {
            \small
            Here's how you should format your response:
            \begin{itemize}
                \item \textit{MY\_CURRENT\_STRATEGY}: $<$explain your chosen strategy here$>$
                \item \textit{MAX\_SEEN\_SO\_FAR}: $x,y, f(x,y)$ 
                \item \textit{NEXT}: $<$Python code snippet that generates the next coordinates and return a list of tuples [(xi, yi), ...]$>$
            \end{itemize}
            \{...\}          
        };
        \node[above=0.25cm of prompt3.north west, anchor=west] {\textbf{Response Format}};

        % Prompt 4
        \node[fill=cyan!30, draw=SlateGray2, rounded corners, inner sep=5pt, text width=\textwidth, below=of prompt3, yshift=0.2cm] (prompt4) {
            \small
            Here are some rules that you must follow: \{...\}
        };
        \node[above=0.25cm of prompt4.north west, anchor=west] {\textbf{General Rules and Examples of Acceptable/Unacceptable Responses}};

        % Prompt 5
        \node[fill=yellow!30, draw=SlateGray2, rounded corners, inner sep=5pt, text width=\textwidth, below=of prompt4, yshift=0.2cm] (prompt5) {
            \small
            \begin{itemize}
                \item The space is vast, and there will be several LOCAL maximums, so avoid choosing them as the answer. Make sure that you explore the space enough to ensure that your answer represents the GLOBAL maximum  
                \item Asking for a certain coordinates multiple times, consumes your available remaining query budget and reduces your chances of finding the global maximum. So, utilize the responses so far to select only unique coordinates.
                \item A very important point is that YOU SHOULD NOT BE HASTY. You should be patient and explore the space thoroughly. However, remember that you have only a maximum of \textit{$QueryBudget$} queries to solve the problem.
            \end{itemize}
            \{...\}
        };
        \node[above=0.2cm of prompt5.north west, anchor=west] {\textbf{General Hints}};

        % Prompt 6
        \node[fill=magenta!30, draw=SlateGray2, rounded corners, inner sep=5pt, text width=\textwidth, below=of prompt5, yshift=0.2cm] (prompt6) {
            \small
            Here are examples of function $f$ where it has multiple local maxima and one global maximum \{...\}
        };
        \node[above=0.2cm of prompt6.north west, anchor=west] {\textbf{Examples of Function $f$}};

        % Prompt 7
        \node[fill=lightgray!10, draw=SlateGray2, rounded corners, inner sep=5pt, text width=\textwidth, below=of prompt6, yshift=0.2cm] (prompt7) {
            \small 
            Let's start. Create an excellent and efficient strategy and choose your first batch of coordinates accordingly.
        };
        \node[above=0.2cm of prompt7.north west, anchor=west] {\textbf{Start Command}};

        % Background rectangle
        \begin{pgfonlayer}{background}
            \node[fit=(title)(prompt1)(prompt2)(prompt3)(prompt4)(prompt5)(prompt6)(prompt7), fill=blue!10, rounded corners, inner sep=10pt] {};
        \end{pgfonlayer}
    \end{tikzpicture}
    \vspace{-2em}
    \caption{Main initial prompt template used for LLM$^+$ and other schemes}
    \label{fig:prompts:llm+}
\end{figure}