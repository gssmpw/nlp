\begin{figure}[ht!]
    \centering
    \begin{tikzpicture}
        % Define the background layer
        \pgfdeclarelayer{background}
        \pgfsetlayers{background,main}

        % Title
        \node[align=center, text=black, font=\large\bfseries] (title) {Poll Worker Prompt Template};

        % Prompt 1
        \node[fill=red!30, draw=SlateGray2, rounded corners, inner sep=5pt, text width=\textwidth, below=of title, yshift=0.2cm] (prompt1) {
            \small
            \textit{You are a great assistant with a strong background in AI and optimization problems.}
        };
        \node[above=0.25cm of prompt1.north west, anchor=west] {\textbf{Role Assignment}};

        % Prompt 2
        \node[fill=green!30, draw=SlateGray2, rounded corners, inner sep=5pt, text width=\textwidth, below=of prompt1, yshift=0.2cm] (prompt2) {
            \small 
            You are assigned to work as a poll worker to analyze responses from multiple agents to a given problem. Each agent’s response may include long sentences describing their strategy to solve the problem, code snippets, or other relevant information.
            
            Your task is to identify the agent whose response is the most frequently specified among all agents. If there is a tie, you should randomly select one of the tied agents. Your response should only include the integer ID of the selected agent. Ensure that the selection process is fair and unbiased.
        };
        \node[above=0.25cm of prompt2.north west, anchor=west] {\textbf{Problem Definition}};

        % Prompt 3
        \node[fill=yellow!30, draw=SlateGray2, rounded corners, inner sep=5pt, text width=\textwidth, below=of prompt2, yshift=0.2cm] (prompt3) {
            \small
            \begin{itemize}
                \item \textbf{Input Data}
                
                        You will receive a list of responses from multiple agents. Each response is associated with a unique agent ID.           
                        \textit{Example format:}
                        \begin{itemize}
                            \item The response from agent $id_1$: $response_1$
                            \item The response from agent $id_2$: $response_2$
                            \item \ldots
                            \item The response from agent $id_n$: $response_n$
                        \end{itemize}
            
            \item \textbf{Processing}

            Analyze the responses to determine which agent’s response is the most frequently specified. Evaluate the similarity of responses based on the nature of the answer, strategy, and major similarities, rather than exact wording. In case of a tie, randomly select one of the tied agents.
            \end{itemize}
        };
        \node[above=0.25cm of prompt3.north west, anchor=west] {\textbf{General Guidelines}};

        % Prompt 4
        \node[fill=cyan!30, draw=SlateGray2, rounded corners, inner sep=5pt, text width=\textwidth, below=of prompt3, yshift=0.2cm] (prompt4) {
            \small          
            Your output should be a single integer representing the ID of the agent with the most frequently specified response.
            
            \textbf{Example output:} 3
        };
        \node[above=0.25cm of prompt4.north west, anchor=west] {\textbf{Response Format}};

        % % Prompt 5
        % \node[fill=cyan!30, draw=SlateGray2, rounded corners, inner sep=5pt, text width=\textwidth, below=of prompt4, yshift=0.2cm] (prompt5) {
        %     The space is vast, and there will be several LOCAL maximums, so avoid choosing them as the answer. Make sure that you explore the space enough to ensure that your answer represents the GLOBAL maximum   
        %     ...
        % };
        % \node[above=0.2cm of prompt5.north west, anchor=west] {\textbf{General Hints}};

        % Prompt 6
        \node[fill=orange!30, draw=SlateGray2, rounded corners, inner sep=5pt, text width=\textwidth, below=of prompt4, yshift=0.2cm] (prompt5) {
            \footnotesize
            Your response should only include the integer ID of the selected agent. You must avoid apologizing in your answers. Ensure that the selection process is fair and unbiased. \textit{Example:} Given the following input:
            
            \begin{itemize}
                \item The response from agent 1: "Use a divide-and-conquer strategy to break the problem into smaller parts. Start with a few number of smaller parts"
                \item Agent \#2: "Apply a divide-and-conquer approach to split the problem into manageable sections. Start with 10 parts"
                \item agent 3: "Implement a brute-force method to try all possible solutions"
                \item The Agent \#4's response: "Use reinforcement learning to find the optimum solution"
                \item Agent 5: "Divide the problem into 10000 smaller parts and solve each part individually"
            \end{itemize}
            
            The most frequently specified strategy is “using divide-and-conquer with small number of total parts” which is provided by agents 1 and 2. Note that agent 5 specifies the divide-and-conquer part but with a large number of initial small parts. Therefore, you should output one of the IDs 1 or 2. If there is a tie, randomly select one of the tied IDs.
            
            \textbf{Output:} 2
        };
        \node[above=0.2cm of prompt5.north west, anchor=west] {\textbf{Examples}};

        % Prompt 7
        \node[fill=lightgray!10, draw=SlateGray2, rounded corners, inner sep=5pt, text width=\textwidth, below=of prompt5, yshift=0.2cm] (prompt6) {
            \small
            Ok, let's start
        };
        \node[above=0.2cm of prompt6.north west, anchor=west] {\textbf{Start Command}};

        % Background rectangle
        \begin{pgfonlayer}{background}
            \node[fit=(title)(prompt1)(prompt2)(prompt3)(prompt4)(prompt5)(prompt6), fill=blue!10, rounded corners, inner sep=10pt] {};
        \end{pgfonlayer}
    \end{tikzpicture}
    \vspace{-2em} 
    \caption{The prompt template used for the poll worker agent in Majority scheme}
    \label{fig:prompts:pollworker}
\end{figure}