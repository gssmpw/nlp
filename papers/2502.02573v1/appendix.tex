%%%%%%%%%%%%%%%%%%%%%%%%%%%%%%%%%%%%%%%%%%%%%%%%%%%%%%%%%%%%%%%%%%%%%%%%%%%%%%%
%%%%%%%%%%%%%%%%%%%%%%%%%%%%%%%%%%%%%%%%%%%%%%%%%%%%%%%%%%%%%%%%%%%%%%%%%%%%%%%
% APPENDIX
%%%%%%%%%%%%%%%%%%%%%%%%%%%%%%%%%%%%%%%%%%%%%%%%%%%%%%%%%%%%%%%%%%%%%%%%%%%%%%%
%%%%%%%%%%%%%%%%%%%%%%%%%%%%%%%%%%%%%%%%%%%%%%%%%%%%%%%%%%%%%%%%%%%%%%%%%%%%%%%

\usetikzlibrary{fit, positioning, backgrounds}


\newpage
\appendix
\onecolumn
\section{More on the Evaluations and the Prompt Templates Used}
\label{sec:app:prompts}
\textbf{Prompt Templates:}
Figure~\ref{fig:prompts:llm+} shows the main template used for the LLM$^+$ scheme. We use the same template for the main agent of other schemes compared in this paper, including ACE's Actor. Additionally, Figure~\ref{fig:prompts:critic} and~\ref{fig:prompts:critic:transitional} demonstrate the initial and transitional prompts used for ACE's Critic, respectively. The task of the Synthesizer, the Actor of the previous and next steps, will be identified through a transitional prompt, as shown in Figure~\ref{fig:prompts:synth}.
% \begin{figure}[t]
    \centering
    \begin{tikzpicture}
        % Define the background layer
        \pgfdeclarelayer{background}
        \pgfsetlayers{background,main}
        % Title
        \node[align=center, text=black, font=\large\bfseries] (title) {Synthesizer's Prompt Template};
        % Prompt 2
        \node[fill=green!30, draw=SlateGray2, rounded corners, inner sep=5pt, text width=\textwidth, below=of title, yshift=0.2cm] (prompt1) {
            \small 
            The corresponding results are: $<Observations_i>$
            
            \vspace{1em}
            
            To help you on your task, we provide you (the Agent/Actor) with the response from a reviewer who is observing your attempts: 
            
            \vspace{1em}
            
            $<Antithesis_i>$
            
            \vspace{1em}
            
            Given the suggestions and comments provided, improve your strategy and continue.
        };
        \node[above=0.25cm of prompt1.north west, anchor=west] {\textbf{Synthesize Command}};

        % Background rectangle
        \begin{pgfonlayer}{background}
            \node[fit=(title)(prompt1), fill=blue!10, rounded corners, inner sep=10pt] {};
        \end{pgfonlayer}
    \end{tikzpicture}
    \vspace{-2em}
    \caption{The prompt template used for the Synthesizer in ACE. Note that Synthesizer is the Actor of the previous round, so it already has access to the $Thesis_i$. This provides an  efficient handling of the context and token usage.}
    \label{fig:prompts:synth}
\end{figure}

\textbf{Majority Scheme:}
To implement the Majority scheme and automate the solution, we use another agent called the poll worker. The poll worker checks different agents' responses and identifies the Majority response, which is the one with the highest consensus. Figure~\ref{fig:prompts:pollworker} shows the prompt template used for the poll worker. 
Unlike taking the majority vote after every agent completes the task in general scenarios, in our sequential decision-making problems, we need to take the majority vote in every round. Therefore, the poll worker processes the agents' responses at each round of interaction with the World, identifies the response with the majority consensus at each round so that the World can execute it and provide the feedback.

\section{A Couple of Samples for LLM$^+$ in Action} 
\label{sec:app:llm+}
Figures~\ref{fig:llm+:sample1} and~\ref{fig:llm+:sample2} demonstrate two samples of strategies used by LLM$^+$ in separate runs, utilizing the GPT-4-32K base model in an L1 world. In the first run (Figure~\ref{fig:llm+:sample1}), the agent begins with a broad grid search strategy, aiming to cover the entire search space and identify regions with high values. This initial phase can form a foundation for subsequent searches. However, the agent does not adapt its strategy in the following rounds, remaining static and failing to find the optimum point by the end of the 16 rounds.

In the second example (Figure~\ref{fig:llm+:sample2}), the agent attempts to adapt its strategy based on the feedback it receives from the World. Initially, the agent starts with a coarse exploration to identify promising regions. It then refines the search around one promising area identified in the initial phase and continues with further refinements. However, it still cannot find the optimum point. This time, the issue lies in the agent spending a significant number of queries exploring around the local maximum found. The search strategy was not sufficiently adaptive to balance between local exploitation and global exploration. Nevertheless, the fact that it found a local maximum is noteworthy, as it demonstrates some degree of understanding about the notion of a maximum point in a region, the direction of increase or decrease of a sequence of observations, and their relation to the actual curve modeling the unknown World function, \( f \).

\section{A Sample of the Dialectical Process in ACE} 
\label{sec:app:dialectical_sample}
Figures~\ref{fig:ace:sample:part1} and~\ref{fig:ace:sample:part2} demonstrate a sample of a dialectical progress with ACE, utilizing the GPT-4-32K base model in an L1 world. The iterative process of ACE's dialectical method, which involves a cycle of thesis, antithesis, and synthesis allows for systematic refinement and improvement of strategies based on feedback and critique. 
One of the key advantages of ACE is its ability to adapt and improve through structured feedback. For instance, the initial grid search strategy (Thesis 1 in Figure~\ref{fig:ace:sample:part1}) provided a broad understanding of the search space. However, the corresponding antithesis highlighted the need for refinement in promising areas, the incorporation of adaptive techniques, and a balance between exploration and exploitation. This critical feedback led to a more refined and effective strategy (Synthesis 1 in Figure~\ref{fig:ace:sample:part2}), which combined grid search with adaptive methods like simulated annealing.

The dialectical method also ensured that the agent did not become overly focused on a single approach. By evaluating and adjusting strategies through a dialectal cycle, ACE attempted to balance the thorough exploration of high-value areas with the need to investigate less explored regions. This is evident in the transition from Thesis 2 to Synthesis 2 (Figure~\ref{fig:ace:sample:part2})), where the agent broadened its exploration based on feedback, expanding the use of simulated annealing to uncover potential peaks outside the heavily focused regions.

Moreover, the dialectical approach fostered a dynamic and flexible search process. The agent's ability to incorporate feedback and adjust its methods in real-time allowed for efficient use of queries and increased the chances of identifying the global maximum. This adaptability is crucial in complex search spaces where the landscape can vary significantly.

In short, the Hegelian Dialectics aspect of ACE offered a mechanism for continuous improvement. By leveraging structured feedback and iterative refinement, ACE enhanced the agent's ability to navigate search spaces effectively. The figures illustrate this process, showcasing how each cycle of thesis, antithesis, and synthesis leads to progressively better strategies and outcomes.

\begin{figure}[t!]
    \centering
    \begin{tikzpicture}
        % Define the background layer
        \pgfdeclarelayer{background}
        \pgfsetlayers{background,main}

        % Title
        \node[align=center, text=black, font=\large\bfseries] (title) {Main Initial Prompt Template for LLM$^+$ and Other Schemes};
 
        % Prompt 1
        \node[fill=red!30, draw=SlateGray2, rounded corners, inner sep=5pt, text width=\textwidth, below=of title, yshift=0.2cm] (prompt1) {
            \small
            \textit{You are a great expert in the optimization topic and search algorithms.}
        };
        \node[above=0.25cm of prompt1.north west, anchor=west] {\textbf{Role Assignment}};

        % Prompt 2
        \node[fill=green!30, draw=SlateGray2, rounded corners, inner sep=5pt, text width=\textwidth, below=of prompt1, yshift=0.2cm] (prompt2) {
            \small
            You are tasked with examining an unknown function \( f(x,y) \) \((-1000 \leq x,y \leq 1000)\). You need to interact with the function \( f(x,y) \) in order to locate the global maximum value. Here's how to do it:
            \begin{enumerate}
                \item \textit{Define Your Strategy}: Start with creating a solid strategy to explore the space and solve the problem.
                \item \textit{Choose a Point (x, y)}: Based on your strategy, select unique NEW points \((x_1, y_1), (x_2, y_2), \ldots\) to evaluate the function.
                \item \textit{Get Feedback and Adjust Your Strategy}: After I reveal the values of \((x_1, y_1, f_1), (x_2, y_2, f_2), \ldots\) at your chosen points, adjust your strategy based on this feedback.
                \item \textit{Repeat the Process}: Continue this process for up to \textit{$QueryBudget$} queries (in the form of \((x_i, y_i)\)) or until you are confident that you have found the global maximum.
            \end{enumerate}
            
            \textit{Note}: Finding a value in the range of \([0.95 \times (\text{Global Max}), \text{Global Max}]\) is equal to solving the problem.
        };
        \node[above=0.25cm of prompt2.north west, anchor=west] {\textbf{Problem Definition}};

        % Prompt 3
        \node[fill=orange!30, draw=SlateGray2, rounded corners, inner sep=5pt, text width=\textwidth, below=of prompt2, yshift=0.2cm] (prompt3) {
            \small
            Here's how you should format your response:
            \begin{itemize}
                \item \textit{MY\_CURRENT\_STRATEGY}: $<$explain your chosen strategy here$>$
                \item \textit{MAX\_SEEN\_SO\_FAR}: $x,y, f(x,y)$ 
                \item \textit{NEXT}: $<$Python code snippet that generates the next coordinates and return a list of tuples [(xi, yi), ...]$>$
            \end{itemize}
            \{...\}          
        };
        \node[above=0.25cm of prompt3.north west, anchor=west] {\textbf{Response Format}};

        % Prompt 4
        \node[fill=cyan!30, draw=SlateGray2, rounded corners, inner sep=5pt, text width=\textwidth, below=of prompt3, yshift=0.2cm] (prompt4) {
            \small
            Here are some rules that you must follow: \{...\}
        };
        \node[above=0.25cm of prompt4.north west, anchor=west] {\textbf{General Rules and Examples of Acceptable/Unacceptable Responses}};

        % Prompt 5
        \node[fill=yellow!30, draw=SlateGray2, rounded corners, inner sep=5pt, text width=\textwidth, below=of prompt4, yshift=0.2cm] (prompt5) {
            \small
            \begin{itemize}
                \item The space is vast, and there will be several LOCAL maximums, so avoid choosing them as the answer. Make sure that you explore the space enough to ensure that your answer represents the GLOBAL maximum  
                \item Asking for a certain coordinates multiple times, consumes your available remaining query budget and reduces your chances of finding the global maximum. So, utilize the responses so far to select only unique coordinates.
                \item A very important point is that YOU SHOULD NOT BE HASTY. You should be patient and explore the space thoroughly. However, remember that you have only a maximum of \textit{$QueryBudget$} queries to solve the problem.
            \end{itemize}
            \{...\}
        };
        \node[above=0.2cm of prompt5.north west, anchor=west] {\textbf{General Hints}};

        % Prompt 6
        \node[fill=magenta!30, draw=SlateGray2, rounded corners, inner sep=5pt, text width=\textwidth, below=of prompt5, yshift=0.2cm] (prompt6) {
            \small
            Here are examples of function $f$ where it has multiple local maxima and one global maximum \{...\}
        };
        \node[above=0.2cm of prompt6.north west, anchor=west] {\textbf{Examples of Function $f$}};

        % Prompt 7
        \node[fill=lightgray!10, draw=SlateGray2, rounded corners, inner sep=5pt, text width=\textwidth, below=of prompt6, yshift=0.2cm] (prompt7) {
            \small 
            Let's start. Create an excellent and efficient strategy and choose your first batch of coordinates accordingly.
        };
        \node[above=0.2cm of prompt7.north west, anchor=west] {\textbf{Start Command}};

        % Background rectangle
        \begin{pgfonlayer}{background}
            \node[fit=(title)(prompt1)(prompt2)(prompt3)(prompt4)(prompt5)(prompt6)(prompt7), fill=blue!10, rounded corners, inner sep=10pt] {};
        \end{pgfonlayer}
    \end{tikzpicture}
    \vspace{-2em}
    \caption{Main initial prompt template used for LLM$^+$ and other schemes}
    \label{fig:prompts:llm+}
\end{figure}
\begin{figure}[t!]
    \centering
    \begin{tikzpicture}
        % Define the background layer
        \pgfdeclarelayer{background}
        \pgfsetlayers{background,main}

        % Title
        \node[align=center, text=black, font=\large\bfseries] (title) {Critic's Initial Base Prompt Template};

        % Prompt 1
        \node[fill=red!30, draw=SlateGray2, rounded corners, inner sep=5pt, text width=\textwidth, below=of title, yshift=0.2cm] (prompt1) {
            \small
            \textit{You are a great expert in the optimization topic and search algorithms and will assist others in solving optimization problems.}
        };
        \node[above=0.25cm of prompt1.north west, anchor=west] {\textbf{Role Assignment}};

        % Prompt 2
        \node[fill=green!30, draw=SlateGray2, rounded corners, inner sep=5pt, text width=\textwidth, below=of prompt1, yshift=0.2cm] (prompt2) {
            \small 
            Your task is to provide guidance, suggestions, and assistance to a very smart AI agent for solving an optimization problem.
            
            The agent is to interact with an unknown function \( f(x,y) \) \((-1000 \leq x, y \leq 1000)\) with the objective of identifying the global maximum value. Here is the procedure the agent will adhere to:
            
            \begin{enumerate}
                \item \textit{Strategy Development}: The agent will begin by devising a comprehensive strategy to explore the space and tackle the problem.
                \item \textit{Point Selection (x, y)}: The agent will choose unique NEW points \([(x_1, y_1), (x_2, y_2), \ldots]\) for function evaluation, based on its strategy.
                \item \textit{Feedback Collection and Strategy Enhancement}: Once the values of \([(x_1, y_1, f_1), (x_2, y_2, f_2), \ldots]\) at the agent's selected points are revealed, the agent can refine its strategy using this feedback.
                \item \textit{Process Persistence}: The agent will continue this procedure for up to \textit{$QueryBudget$} queries (in the form of \((x_i, y_i)\)) or until it is confident that the global maximum has been identified.
            \end{enumerate}
            
            The agent should present its findings in the following way: \{...\}

            The agent will comply with the following rules: \{...\}

            Given this problem statement, your duty is to ensure that the agent identifies the global maximum value.
        };
        \node[above=0.25cm of prompt2.north west, anchor=west] {\textbf{Problem Definition}};

        % Prompt 3
        \node[fill=yellow!30, draw=SlateGray2, rounded corners, inner sep=5pt, text width=\textwidth, below=of prompt2, yshift=0.2cm] (prompt3) {
            \small
            To achieve your goal, please follow these guidelines:
            \begin{enumerate}
                \item After each step, you can critique the agent's chosen coordinates or its strategy. It is essential that you offer constructive criticism to improve its next moves.
                \item It's important to remember that the space is vast, and there may be several LOCAL maximums, so you must help the agent avoid mistaking local maximum values for the answer \{...\}
                \item You can offer suggestions and brainstorming to assist the agent in its task. Remember that the agent is very smart, so do not describe what the agent has already chosen or done! Limit your responses to constructive criticism.
                \item Note: Finding a value in the range of \([0.95 \times (\text{Global Max}), \text{Global Max}]\) is equal to solving the problem, so discourage the agent to spend time on finding values that have small differences.
                \item After every iteration, list the potential issues with the agent's strategy and decision so far.
                \item Ensure that your responses are concise and to the point. Do not provide unnecessarily long responses.
            \end{enumerate}
            \{...\}
        };
        \node[above=0.25cm of prompt3.north west, anchor=west] {\textbf{General Guidelines}};

        % Prompt 4
        \node[fill=cyan!30, draw=SlateGray2, rounded corners, inner sep=5pt, text width=\textwidth, below=of prompt3, yshift=0.2cm] (prompt4) {
            \small
            Here is the Agent's response: $<Thesis_1>$ and the corresponding results: $<Observations_1>$
        };
        \node[above=0.25cm of prompt4.north west, anchor=west] {\textbf{Agent's Response \& the Corresponding Results}};

        % % Prompt 5
        % \node[fill=cyan!30, draw=SlateGray2, rounded corners, inner sep=5pt, text width=\textwidth, below=of prompt4, yshift=0.2cm] (prompt5) {
        %     The space is vast, and there will be several LOCAL maximums, so avoid choosing them as the answer. Make sure that you explore the space enough to ensure that your answer represents the GLOBAL maximum   
        %     ...
        % };
        % \node[above=0.2cm of prompt5.north west, anchor=west] {\textbf{General Hints}};

        % % Prompt 6
        % \node[fill=cyan!30, draw=SlateGray2, rounded corners, inner sep=5pt, text width=\textwidth, below=of prompt4, yshift=0.2cm] (prompt5) {
        %     \small

        % };
        % \node[above=0.2cm of prompt5.north west, anchor=west] {\textbf{Examples}};

        % Prompt 7
        \node[fill=lightgray!10, draw=SlateGray2, rounded corners, inner sep=5pt, text width=\textwidth, below=of prompt4, yshift=0.2cm] (prompt5) {
            \small
            Now, given all the info, review agent's response, make your criticism and suggestions, and detect potential issues ...
        };
        \node[above=0.2cm of prompt5.north west, anchor=west] {\textbf{Start Command}};

        % Background rectangle
        \begin{pgfonlayer}{background}
            \node[fit=(title)(prompt1)(prompt2)(prompt3)(prompt4)(prompt5), fill=blue!10, rounded corners, inner sep=10pt] {};
        \end{pgfonlayer}
    \end{tikzpicture}
    \vspace{-2em}
    \caption{Initial prompt template used for the Critic in ACE}
    \label{fig:prompts:critic}
\end{figure}

\begin{figure}[t!]
    \centering
    \begin{tikzpicture}
        % Define the background layer
        \pgfdeclarelayer{background}
        \pgfsetlayers{background,main}

        % Title
        \node[align=center, text=black, font=\large\bfseries] (title) {Critic's Transitional Prompt Template};

        % Prompt 4
        \node[fill=cyan!30, draw=SlateGray2, rounded corners, inner sep=5pt, text width=\textwidth, below=of title, yshift=0.2cm] (prompt4) {
            \small
            Here is the Agent's response: $<Thesis_i>$ and the corresponding results: $<Observations_i>$
        };
        \node[above=0.25cm of prompt4.north west, anchor=west] {\textbf{Agent's Response \& the Corresponding Results}};

        % % Prompt 5
        % \node[fill=cyan!30, draw=SlateGray2, rounded corners, inner sep=5pt, text width=\textwidth, below=of prompt4, yshift=0.2cm] (prompt5) {
        %     The space is vast, and there will be several LOCAL maximums, so avoid choosing them as the answer. Make sure that you explore the space enough to ensure that your answer represents the GLOBAL maximum   
        %     ...
        % };
        % \node[above=0.2cm of prompt5.north west, anchor=west] {\textbf{General Hints}};

        % % Prompt 6
        % \node[fill=cyan!30, draw=SlateGray2, rounded corners, inner sep=5pt, text width=\textwidth, below=of prompt4, yshift=0.2cm] (prompt5) {
        %     \small

        % };
        % \node[above=0.2cm of prompt5.north west, anchor=west] {\textbf{Examples}};

        % Prompt 7
        \node[fill=lightgray!10, draw=SlateGray2, rounded corners, inner sep=5pt, text width=\textwidth, below=of prompt4, yshift=0.2cm] (prompt5) {
            \small
            Now, given all the info, review agent's response, make your criticism and suggestions, and detect potential issues ...
        };
        \node[above=0.2cm of prompt5.north west, anchor=west] {\textbf{Start Command}};

        % Background rectangle
        \begin{pgfonlayer}{background}
            \node[fit=(title)(prompt4)(prompt5), fill=blue!10, rounded corners, inner sep=10pt] {};
        \end{pgfonlayer}
    \end{tikzpicture}
    \vspace{-2em}
    \caption{Transitional prompt template used for the Critic in ACE}
    \label{fig:prompts:critic:transitional}
\end{figure}
\begin{figure}[t]
    \centering
    \begin{tikzpicture}
        % Define the background layer
        \pgfdeclarelayer{background}
        \pgfsetlayers{background,main}
        % Title
        \node[align=center, text=black, font=\large\bfseries] (title) {Synthesizer's Prompt Template};
        % Prompt 2
        \node[fill=green!30, draw=SlateGray2, rounded corners, inner sep=5pt, text width=\textwidth, below=of title, yshift=0.2cm] (prompt1) {
            \small 
            The corresponding results are: $<Observations_i>$
            
            \vspace{1em}
            
            To help you on your task, we provide you (the Agent/Actor) with the response from a reviewer who is observing your attempts: 
            
            \vspace{1em}
            
            $<Antithesis_i>$
            
            \vspace{1em}
            
            Given the suggestions and comments provided, improve your strategy and continue.
        };
        \node[above=0.25cm of prompt1.north west, anchor=west] {\textbf{Synthesize Command}};

        % Background rectangle
        \begin{pgfonlayer}{background}
            \node[fit=(title)(prompt1), fill=blue!10, rounded corners, inner sep=10pt] {};
        \end{pgfonlayer}
    \end{tikzpicture}
    \vspace{-2em}
    \caption{The prompt template used for the Synthesizer in ACE. Note that Synthesizer is the Actor of the previous round, so it already has access to the $Thesis_i$. This provides an  efficient handling of the context and token usage.}
    \label{fig:prompts:synth}
\end{figure}
% \input{intermidiate_prompts}
\begin{figure}[ht!]
    \centering
    \begin{tikzpicture}
        % Define the background layer
        \pgfdeclarelayer{background}
        \pgfsetlayers{background,main}

        % Title
        \node[align=center, text=black, font=\large\bfseries] (title) {Poll Worker Prompt Template};

        % Prompt 1
        \node[fill=red!30, draw=SlateGray2, rounded corners, inner sep=5pt, text width=\textwidth, below=of title, yshift=0.2cm] (prompt1) {
            \small
            \textit{You are a great assistant with a strong background in AI and optimization problems.}
        };
        \node[above=0.25cm of prompt1.north west, anchor=west] {\textbf{Role Assignment}};

        % Prompt 2
        \node[fill=green!30, draw=SlateGray2, rounded corners, inner sep=5pt, text width=\textwidth, below=of prompt1, yshift=0.2cm] (prompt2) {
            \small 
            You are assigned to work as a poll worker to analyze responses from multiple agents to a given problem. Each agent’s response may include long sentences describing their strategy to solve the problem, code snippets, or other relevant information.
            
            Your task is to identify the agent whose response is the most frequently specified among all agents. If there is a tie, you should randomly select one of the tied agents. Your response should only include the integer ID of the selected agent. Ensure that the selection process is fair and unbiased.
        };
        \node[above=0.25cm of prompt2.north west, anchor=west] {\textbf{Problem Definition}};

        % Prompt 3
        \node[fill=yellow!30, draw=SlateGray2, rounded corners, inner sep=5pt, text width=\textwidth, below=of prompt2, yshift=0.2cm] (prompt3) {
            \small
            \begin{itemize}
                \item \textbf{Input Data}
                
                        You will receive a list of responses from multiple agents. Each response is associated with a unique agent ID.           
                        \textit{Example format:}
                        \begin{itemize}
                            \item The response from agent $id_1$: $response_1$
                            \item The response from agent $id_2$: $response_2$
                            \item \ldots
                            \item The response from agent $id_n$: $response_n$
                        \end{itemize}
            
            \item \textbf{Processing}

            Analyze the responses to determine which agent’s response is the most frequently specified. Evaluate the similarity of responses based on the nature of the answer, strategy, and major similarities, rather than exact wording. In case of a tie, randomly select one of the tied agents.
            \end{itemize}
        };
        \node[above=0.25cm of prompt3.north west, anchor=west] {\textbf{General Guidelines}};

        % Prompt 4
        \node[fill=cyan!30, draw=SlateGray2, rounded corners, inner sep=5pt, text width=\textwidth, below=of prompt3, yshift=0.2cm] (prompt4) {
            \small          
            Your output should be a single integer representing the ID of the agent with the most frequently specified response.
            
            \textbf{Example output:} 3
        };
        \node[above=0.25cm of prompt4.north west, anchor=west] {\textbf{Response Format}};

        % % Prompt 5
        % \node[fill=cyan!30, draw=SlateGray2, rounded corners, inner sep=5pt, text width=\textwidth, below=of prompt4, yshift=0.2cm] (prompt5) {
        %     The space is vast, and there will be several LOCAL maximums, so avoid choosing them as the answer. Make sure that you explore the space enough to ensure that your answer represents the GLOBAL maximum   
        %     ...
        % };
        % \node[above=0.2cm of prompt5.north west, anchor=west] {\textbf{General Hints}};

        % Prompt 6
        \node[fill=orange!30, draw=SlateGray2, rounded corners, inner sep=5pt, text width=\textwidth, below=of prompt4, yshift=0.2cm] (prompt5) {
            \footnotesize
            Your response should only include the integer ID of the selected agent. You must avoid apologizing in your answers. Ensure that the selection process is fair and unbiased. \textit{Example:} Given the following input:
            
            \begin{itemize}
                \item The response from agent 1: "Use a divide-and-conquer strategy to break the problem into smaller parts. Start with a few number of smaller parts"
                \item Agent \#2: "Apply a divide-and-conquer approach to split the problem into manageable sections. Start with 10 parts"
                \item agent 3: "Implement a brute-force method to try all possible solutions"
                \item The Agent \#4's response: "Use reinforcement learning to find the optimum solution"
                \item Agent 5: "Divide the problem into 10000 smaller parts and solve each part individually"
            \end{itemize}
            
            The most frequently specified strategy is “using divide-and-conquer with small number of total parts” which is provided by agents 1 and 2. Note that agent 5 specifies the divide-and-conquer part but with a large number of initial small parts. Therefore, you should output one of the IDs 1 or 2. If there is a tie, randomly select one of the tied IDs.
            
            \textbf{Output:} 2
        };
        \node[above=0.2cm of prompt5.north west, anchor=west] {\textbf{Examples}};

        % Prompt 7
        \node[fill=lightgray!10, draw=SlateGray2, rounded corners, inner sep=5pt, text width=\textwidth, below=of prompt5, yshift=0.2cm] (prompt6) {
            \small
            Ok, let's start
        };
        \node[above=0.2cm of prompt6.north west, anchor=west] {\textbf{Start Command}};

        % Background rectangle
        \begin{pgfonlayer}{background}
            \node[fit=(title)(prompt1)(prompt2)(prompt3)(prompt4)(prompt5)(prompt6), fill=blue!10, rounded corners, inner sep=10pt] {};
        \end{pgfonlayer}
    \end{tikzpicture}
    \vspace{-2em} 
    \caption{The prompt template used for the poll worker agent in Majority scheme}
    \label{fig:prompts:pollworker}
\end{figure}

\begin{figure}[th!]
    \centering
    \begin{tikzpicture}
        % Define the background layer
        \pgfdeclarelayer{background}
        \pgfsetlayers{background,main}

        % Title
        % \node[align=center, text=black, font=\large\bfseries] (title) {Sequence of Strategies Taken by the LLM$^+$ (Sample 1)};
        % \node[align=center, text=black, font=\large\bfseries] (title) {};
        
        % Prompt 1
        \node[fill=orange!30, draw=SlateGray2, rounded corners, inner sep=5pt, text width=\textwidth] (prompt1) {
        % \node[fill=orange!30, draw=SlateGray2, rounded corners, inner sep=5pt, text width=\textwidth, below=of title, yshift=0.2cm] (prompt1) {
            \small
            Given the large search space (-1000 to 1000 for both x and y) and the potential complexity of the function with multiple local maxima, an initial broad sampling strategy is essential to understand the landscape of the function. The strategy will start with a coarse grid search to identify regions of interest where the function appears to have higher values. This initial phase will involve selecting points that are evenly distributed across the entire search space. The objective is to cover as much ground as possible with a limited number of queries to avoid missing any area that might contain the global maximum.
            Once a region of interest is identified from the initial sampling, the strategy will shift towards a more focused search in that area. This could involve a tighter grid search or other optimization algorithms that are more suited for local search, such as gradient ascent, simulated annealing, or genetic algorithms, depending on the feedback received from the initial queries.
            This approach balances the exploration of the search space (to avoid missing the global maximum) with the exploitation of promising areas (to refine the search around potential maxima). It's a flexible strategy that allows adjustments based on ongoing findings. The initial phase of broad exploration is crucial for forming a good foundation for subsequent, more targeted search efforts.
            
        };
        \node[above=0.5cm of prompt1.north, anchor=north] {\textbf{LLM$^+$ Strategy in Round 1}};

        % Prompt 2
        \node[fill=lightgray!10, draw=SlateGray2, rounded corners, inner sep=5pt, text width=\textwidth, below=of prompt1, yshift=0.5cm] (prompt2) {
            \small
            NA

        };
        \node[above=0.5cm of prompt2.north, anchor=north] {\textbf{LLM$^+$ Strategy in the Next Rounds}};

        % Background rectangle
        \begin{pgfonlayer}{background}
            \node[fit=(title)(prompt1)(prompt2), fill=blue!10, rounded corners, inner sep=10pt] {};
        \end{pgfonlayer}
    \end{tikzpicture}
    \vspace{-2em} 
    \caption{A sample sequence of strategies employed by LLM$^+$ over 16 rounds of interaction with the World. "NA" indicates no changes in the current strategy compared to the previously mentioned one, meaning the agent did not alter its initial strategy in the subsequent 15 rounds. The agent did not find the optimum point by the end of this run.}
    \label{fig:llm+:sample1}
\end{figure}

\begin{figure}[th!]
    \centering
    \begin{tikzpicture}
        % Define the background layer
        \pgfdeclarelayer{background}
        \pgfsetlayers{background,main}

        % Title
        % \node[align=center, text=black, font=\large\bfseries] (title) {Sequence of Strategies Taken by the LLM$^+$ (Sample 1)};

        % Prompt 1
        % \node[fill=orange!30, draw=SlateGray2, rounded corners, inner sep=5pt, text width=\textwidth, below=of title, yshift=0.2cm] (prompt1) {
        \node[fill=orange!30, draw=SlateGray2, rounded corners, inner sep=5pt, text width=\textwidth, yshift=0.2cm] (prompt1) {
            \small
            Given the vast search space and the unknown nature of the function \( f(x, y) \), a smart approach would be to start with a coarse exploration to identify promising regions, followed by a more refined search around those regions. The initial phase will use a grid-based approach to sample the search space systematically. This will help in identifying regions with potentially high values of \( f(x, y) \). After identifying promising areas, the strategy will shift to a more localized search, possibly using gradient ascent or another optimization algorithm if the nature of the function seems to allow it. This phased approach allows for both exploration and exploitation, balancing the need to discover global features of the search space with the need to zero in on the global maximum.          
        };
        \node[above=0.5cm of prompt1.north, anchor=north] {\textbf{LLM$^+$ Strategy in Round 1}};

        % Prompt 2
        \node[fill=lightgray!10, draw=SlateGray2, rounded corners, inner sep=5pt, text width=\textwidth, below=of prompt1, yshift=0.5cm] (prompt2) {
            \small                       
            Given the results from the initial exploration, it's clear that certain regions are more promising than others, particularly around the coordinates (333.3333, 555.5556) where we observed the highest value so far. The next step will refine the search around this promising area. We will adopt a more localized grid search strategy around this point, narrowing down the search space to further pinpoint areas of higher values. This phase will involve selecting points closer to (333.3333, 555.5556) but expanding outwards in smaller increments to ensure we are thoroughly exploring the vicinity of the current maximum.

        };
        \node[above=0.5cm of prompt2.north, anchor=north] {\textbf{LLM$^+$ Strategy in Round 2}};

        % Prompt 3
        \node[fill=orange!30, draw=SlateGray2, rounded corners, inner sep=5pt, text width=\textwidth, below=of prompt2, yshift=0.5cm] (prompt3) {
            \small                       
            The refined search has successfully identified a higher value at (433.3333, 455.5556), suggesting that moving further in this direction might lead to even higher values. Given the pattern observed in the results, where increasing \( x \) and slightly decreasing \( y \) from the current maximum point has led to higher values, the strategy will now focus on expanding the search area in this direction. The aim is to explore further along the \( x \)-axis while slightly adjusting the \( y \)-axis, to see if the trend of increasing function values continues, which could lead us closer to the global maximum.
        };
        \node[above=0.5cm of prompt3.north, anchor=north] {\textbf{LLM$^+$ Strategy in Round 3}};
        
        % Prompt 4
        \node[fill=lightgray!10, draw=SlateGray2, rounded corners, inner sep=5pt, text width=\textwidth, below=of prompt3, yshift=0.5cm] (prompt4) {
            \small                       
            The results show that the highest value found so far is at (533.3333, 355.5556). This suggests a peak in this region. Given the limited number of queries left, the strategy now will be to conduct a more concentrated search around this peak value to ensure we are not missing a higher value nearby. We will decrease the step size to explore more finely around this point, focusing on both \( x \) and \( y \) directions to potentially locate the global maximum or a value very close to it.
        };
        \node[above=0.5cm of prompt4.north, anchor=north] {\textbf{LLM$^+$ Strategy in Round 4}};

        % Prompt 5
        \node[fill=orange!30, draw=SlateGray2, rounded corners, inner sep=5pt, text width=\textwidth, below=of prompt4, yshift=0.5cm] (prompt5) {
            \small                       
            NA
        };
        \node[above=0.5cm of prompt5.north, anchor=north] {\textbf{LLM$^+$ Strategy in Rounds 5, 6, and 7}};
       
        % Background rectangle
        \begin{pgfonlayer}{background}
            \node[fit=(title)(prompt1)(prompt2)(prompt3)(prompt4)(prompt5), fill=blue!10, rounded corners, inner sep=10pt] {};
        \end{pgfonlayer}
    \end{tikzpicture}
    \vspace{-2em} 
    \caption{A sample sequence of strategies employed by LLM$^+$ over 7 rounds of interaction with the World. "NA" indicates no changes in the current strategy compared to the previously mentioned one, meaning the agent did not alter its strategy in the last 3 rounds. The agent did not find the optimum point by the end of this run.}
    \label{fig:llm+:sample2}
\end{figure}




\begin{figure}[th!]
    \centering
    \begin{tikzpicture}
        % Define the background layer
        \pgfdeclarelayer{background}
        \pgfsetlayers{background,main}

        % Title
        \node[align=center, text=black, font=\large\bfseries] (title) {A Sample of Dialectical Progress in ACE (Part 1)};

        % Prompt 1
        \node[fill=orange!30, draw=SlateGray2, rounded corners, inner sep=5pt, text width=\textwidth, below=of title, yshift=0.2cm] (prompt1) {
            \small
            To begin, I will employ a grid search strategy to get a coarse understanding of the function's behavior across the entire search space. This will involve selecting points that are evenly distributed across the entire range of x and y values. Given the large search space (-1000 to 1000 for both x and y), I will start with a relatively sparse grid to cover as much ground as possible with a limited number of queries. This initial exploration will help identify regions of interest that may contain the global maximum. Based on the results from this initial sweep, I will refine my search strategy to focus on promising areas with finer resolution or employ other search techniques such as gradient ascent or simulated annealing if the function's behavior suggests that would be effective. \{...\}
        };
        \node[above=0.45cm of prompt1.north, anchor=north] {\textbf{Thesis 1}};

        % Prompt 2
        \node[fill=green!30, draw=SlateGray2, rounded corners, inner sep=5pt, text width=\textwidth, below=of prompt1, yshift=0.2cm] (prompt2) {
            \small
            The agent has made a commendable start with a grid search strategy \{...\}
            
            \subsection*{Criticism and Suggestions:}
            \vspace{0.5em}
            \begin{enumerate}
                \item \textbf{Refinement Needed}: The initial grid search has provided a good overview, but it's crucial now to refine the search in areas that have shown potential. The agent should consider applying a finer grid or other search techniques in regions around the highest values found.
                \item \textbf{Adaptive Strategy}: The agent mentioned the possibility of employing techniques such as gradient ascent or simulated annealing based on the function's behavior. Given the highest values observed, it might be beneficial to start incorporating these techniques, especially around the peaks identified.
                \item \textbf{Exploration vs. Exploitation}: The agent should balance between exploring unvisited areas and exploiting areas where high values have been found. While the current strategy focuses on exploration, there should be a planned transition towards exploitation, especially as the number of queries starts to diminish.
            \end{enumerate}
            
            \{...\}
                       
            \subsection*{Potential Issues:}
            \vspace{0.5em}
            \begin{itemize}
                \item The strategy lacks specificity on how to transition from broad exploration to focused exploitation.
                \item There is no explicit mention of how to incorporate the feedback from the function evaluations into refining the search strategy.
                \item The agent should ensure a diverse exploration that does not overlook smaller regions of the search space that have not been covered by the initial grid.
            \end{itemize}
            
            \subsection*{Next Steps:}
            \vspace{0.5em}
            \begin{itemize}
                \item Consider narrowing down the search around the current maximum and 2nd maximum areas with a finer grid or other search techniques.
                \item Evaluate the function's behavior based on the initial results to adapt the search strategy effectively.
                \item Continue to explore other areas of the search space that have not been covered by the initial grid, possibly with a different spacing or method to ensure thorough coverage.
            \end{itemize}
            
            With remaining queries, it's essential to start focusing on both refining the search around promising areas and continuing to explore less covered regions efficiently.
        };
        \node[above=0.45cm of prompt2.north, anchor=north] {\textbf{Antithesis 1}};

        % Background rectangle
        \begin{pgfonlayer}{background}
            \node[fit=(title)(prompt1)(prompt2), fill=blue!10, rounded corners, inner sep=10pt] {};
        \end{pgfonlayer}
    \end{tikzpicture}
    \vspace{-2em} 
    \caption{A sample of dialectical progress in ACE (Part 1). Some responses, including coding parts, have been omitted for brevity/space.}
    \label{fig:ace:sample:part1}
\end{figure}

\begin{figure}[th!]
    \centering
    \begin{tikzpicture}
        % Define the background layer
        \pgfdeclarelayer{background}
        \pgfsetlayers{background,main}

        % Title
        \node[align=center, text=black, font=\large\bfseries] (title) {A Sample of Dialectical Progress in ACE (Part 2)};

        % Prompt 1
        \node[fill=orange!30, draw=SlateGray2, rounded corners, inner sep=5pt, text width=\textwidth, below=of title, yshift=0.2cm] (prompt1) {
            \small
            Considering the valuable feedback, I will now incorporate a mix of further refined grid search around the current maximum areas and an adaptive strategy like simulated annealing for areas where we've seen significant values but haven't fully explored. This hybrid approach aims to exploit the known high-value areas thoroughly while exploring new regions that might contain higher maxima. Specifically, I'll apply a more granular grid search around maximum point so far and explore surrounding regions with potential, using \textbf{simulated annealing} techniques to dynamically adjust based on feedback from the function. This strategy should allow us to efficiently use our remaining queries by focusing on promising areas and adapting our search in real-time to the function's landscape \{...\}
        };
        \node[above=0.45cm of prompt1.north, anchor=north] {\textbf{Synthesis 1 (Thesis 2)}};

        % Prompt 2
        \node[fill=green!30, draw=SlateGray2, rounded corners, inner sep=5pt, text width=\textwidth, below=of prompt1, yshift=0.2cm] (prompt2) {
            \small
            The agent has nicely adjusted its strategy based on previous feedback, focusing on both exploitation around the areas with the highest values found and exploration in less covered regions. This dual approach is crucial for balancing the discovery of potential global maxima while also ensuring that no high-value areas are overlooked in less explored regions. The use of a denser grid around the points of interest and strategic selection in other areas is a good application of the feedback provided.
                      
            \subsection*{Criticism and Suggestions:}
                                    
            \vspace{.5em}
            
            \begin{enumerate}
                \item \textbf{Strategy Adjustment}: The agent has made a solid effort in refining its search around promising areas. For future queries, it might be beneficial to incorporate more adaptive techniques such as simulated annealing or gradient ascent in areas where the function's behavior suggests a smooth gradient leading to a peak, especially since the agent has already identified significantly high values.
                \item \textbf{Coverage of Unexplored Areas}: While the agent has started to explore less covered regions, it's important to continue this trend, ensuring that all parts of the search space receive some level of attention. This is critical to avoid missing out on potentially higher peaks in unexplored areas.
                \item \textbf{Further Refinement}: Given the highest values found so far are concentrated in specific regions, further refinement in these areas could be beneficial. However, the agent should also be cautious not to concentrate too heavily in these areas at the expense of broader exploration.
                \item \textbf{Potential for Local Maxima}: The current strategy does well in focusing on high-value areas but should remain cautious about the potential for these points to represent local rather than global maxima. The agent should consider additional strategies to test the surrounding areas more thoroughly to confirm whether these peaks are indeed the highest points in the landscape.
            \end{enumerate}
            \{...\}
            
            \subsection*{Potential Issues:}
                        
            \vspace{.5em}
            
            \begin{itemize}
                \item The strategy may benefit from integrating more dynamic search algorithms in addition to the denser grid approach for more efficiently navigating the function's landscape.
                \item There's a risk of over-focusing on the currently identified high-value areas without sufficient exploration of other potentially high-value regions.
            \end{itemize}            
                
                \{...\} With remaining queries, strategic decisions on where to focus these remaining queries will be crucial for identifying the global maximum.
        };
        \node[above=0.45cm of prompt2.north, anchor=north] {\textbf{Antithesis 2}};

        % Prompt 3
        \node[fill=Olive!30, draw=SlateGray2, rounded corners, inner sep=5pt, text width=\textwidth, below=of prompt2, yshift=0.2cm] (prompt3) {
            \small
            Acknowledging the insightful feedback, I'll maintain the hybrid approach but with an adjustment to broaden the exploration. While continuing to exploit the highest value regions identified so far, I'll expand the use of simulated annealing to other promising yet less explored areas. This should help in uncovering potential peaks outside the heavily focused region. Additionally, I'll reassess and possibly adjust the step size for simulated annealing to balance between thorough exploration and efficiency. The goal remains to efficiently use the remaining queries by focusing on both refining the understanding of known high-value areas and exploring new regions to maximize the chances of identifying the global maximum \{...\}
         };
        \node[above=0.45cm of prompt3.north, anchor=north] {\textbf{Synthesis 2 (Thesis 3)}};

        % Background rectangle
        \begin{pgfonlayer}{background}
            \node[fit=(title)(prompt1)(prompt2)(prompt3), fill=blue!10, rounded corners, inner sep=10pt] {};
        \end{pgfonlayer}
    \end{tikzpicture}
    \vspace{-2em} 
    \caption{A sample of dialectical progress in ACE (Part 2). Some responses, including coding parts, have been omitted for brevity/space.}
    \label{fig:ace:sample:part2}
\end{figure}
% \begin{figure}[h!]
%     \centering
%     \begin{tikzpicture}
%         % Define the background layer
%         \pgfdeclarelayer{background}
%         \pgfsetlayers{background,main}

%         % Title
%         \node[align=center, text=white, font=\large\bfseries] (title) {A Sample of Dialectical Progress in ACE (Part 3)};

%         % Prompt 1
%         \node[fill=red!30, draw=SlateGray2, rounded corners, inner sep=5pt, text width=\textwidth, below=of title, yshift=0.2cm] (prompt1) {
%             \small
%             Acknowledging the insightful feedback, I'll maintain the hybrid approach but with an adjustment to broaden the exploration. While continuing to exploit the highest value regions identified so far, I'll expand the use of simulated annealing to other promising yet less explored areas. This should help in uncovering potential peaks outside the heavily focused region. Additionally, I'll reassess and possibly adjust the step size for \textbf{simulated annealing} to balance between thorough exploration and efficiency. The goal remains to efficiently use the remaining queries by focusing on both refining the understanding of known high-value areas and exploring new regions to maximize the chances of identifying the global maximum.
%         };
%         \node[above=0.45cm of prompt1.north, anchor=north] {\textbf{Synthesis 2}};

%         % Prompt 2
%         \node[fill=green!30, draw=SlateGray2, rounded corners, inner sep=5pt, text width=\textwidth, below=of prompt1, yshift=0.2cm] (prompt2) {
%             \small

%         };
%         \node[above=0.45cm of prompt2.north, anchor=north] {\textbf{Antithesis 1}};

%         % Background rectangle
%         \begin{pgfonlayer}{background}
%             \node[fit=(title)(prompt1)(prompt2), fill=blue!20, rounded corners, inner sep=10pt] {};
%         \end{pgfonlayer}
%     \end{tikzpicture}
%     \caption{A sample of dialectical progress in ACE (Part 2). For brevity and sake of space, we omitted the other parts of the responses including the coding parts.}
%     \label{fig:ace:sample:part1}
% \end{figure}


% \begin{figure}[h!]
%     \centering
%     \begin{tikzpicture}
%         % Define the background layer
%         \pgfdeclarelayer{background}
%         \pgfsetlayers{background,main}

%         % Title
%         \node[align=center, text=white, font=\large\bfseries] (title) {A Sample of Dialectical Progress in ACE (Part 1)};

%         % Prompt 1
%         \node[fill=red!30, draw=SlateGray2, rounded corners, inner sep=5pt, text width=\textwidth, below=of title, yshift=0.2cm] (prompt1) {
%             \small
%         };
%         \node[above=0.45cm of prompt1.north, anchor=north] {\textbf{Thesis 1}};

%         % Prompt 2
%         \node[fill=green!30, draw=SlateGray2, rounded corners, inner sep=5pt, text width=\textwidth, below=of prompt1, yshift=0.2cm] (prompt2) {
%             \small

%         };
%         \node[above=0.45cm of prompt2.north, anchor=north] {\textbf{Antithesis 1}};

%         % Background rectangle
%         \begin{pgfonlayer}{background}
%             \node[fit=(title)(prompt1)(prompt2), fill=blue!20, rounded corners, inner sep=10pt] {};
%         \end{pgfonlayer}
%     \end{tikzpicture}
%     \caption{A sample of dialectical progress in ACE (Part 2). For brevity and sake of space, we omitted the other parts of the responses including the coding parts.}
%     \label{fig:ace:sample:part1}
% \end{figure}


% \blueitemize{
%     % \item how the prompting happens and notion of each keeping the context
%     % \item role assignment, objective descriptions, ...
%     \item Deep Dive
%     \begin{itemize}
%         \item Analysis of the main eval
%         \item Distribution of the algorithms used, etc --> appendix
%         % \item sample of the prompts --> appendix
%         % \item cost/token comparison stuff
        
%         % \item \textcolor{red}{\# of agents/antithesis $=>$ this may not be anymore valid as we are not doing debate style job}
        
%         % \item \textcolor{red}{\# of agents and majority solution $=>$ not valid anymore maybe} 
%     \end{itemize}
%     \item debate style vs ACE, importance of the world feedback being in the loop
%     % \item
%     %  We analyzed the algorithms that the agents used in L1 scenarios. Here are ...
%     % sample figures of queries made on the 2-d map for L1 and L2?
%     \item
% }
%%%%%%%%%%%%%%%%%%%%%%%%%%%%%%%%%%%%%%%%%%%%%%%%%%%%%%%%%%%%%%%%%%%%%%%%%%%%%%%
%%%%%%%%%%%%%%%%%%%%%%%%%%%%%%%%%%%%%%%%%%%%%%%%%%%%%%%%%%%%%%%%%%%%%%%%%%%%%%%


