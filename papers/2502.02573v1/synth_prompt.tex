\begin{figure}[t]
    \centering
    \begin{tikzpicture}
        % Define the background layer
        \pgfdeclarelayer{background}
        \pgfsetlayers{background,main}
        % Title
        \node[align=center, text=black, font=\large\bfseries] (title) {Synthesizer's Prompt Template};
        % Prompt 2
        \node[fill=green!30, draw=SlateGray2, rounded corners, inner sep=5pt, text width=\textwidth, below=of title, yshift=0.2cm] (prompt1) {
            \small 
            The corresponding results are: $<Observations_i>$
            
            \vspace{1em}
            
            To help you on your task, we provide you (the Agent/Actor) with the response from a reviewer who is observing your attempts: 
            
            \vspace{1em}
            
            $<Antithesis_i>$
            
            \vspace{1em}
            
            Given the suggestions and comments provided, improve your strategy and continue.
        };
        \node[above=0.25cm of prompt1.north west, anchor=west] {\textbf{Synthesize Command}};

        % Background rectangle
        \begin{pgfonlayer}{background}
            \node[fit=(title)(prompt1), fill=blue!10, rounded corners, inner sep=10pt] {};
        \end{pgfonlayer}
    \end{tikzpicture}
    \vspace{-2em}
    \caption{The prompt template used for the Synthesizer in ACE. Note that Synthesizer is the Actor of the previous round, so it already has access to the $Thesis_i$. This provides an  efficient handling of the context and token usage.}
    \label{fig:prompts:synth}
\end{figure}