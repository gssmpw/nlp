\begin{figure}[th!]
    \centering
    \begin{tikzpicture}
        % Define the background layer
        \pgfdeclarelayer{background}
        \pgfsetlayers{background,main}

        % Title
        \node[align=center, text=black, font=\large\bfseries] (title) {A Sample of Dialectical Progress in ACE (Part 1)};

        % Prompt 1
        \node[fill=orange!30, draw=SlateGray2, rounded corners, inner sep=5pt, text width=\textwidth, below=of title, yshift=0.2cm] (prompt1) {
            \small
            To begin, I will employ a grid search strategy to get a coarse understanding of the function's behavior across the entire search space. This will involve selecting points that are evenly distributed across the entire range of x and y values. Given the large search space (-1000 to 1000 for both x and y), I will start with a relatively sparse grid to cover as much ground as possible with a limited number of queries. This initial exploration will help identify regions of interest that may contain the global maximum. Based on the results from this initial sweep, I will refine my search strategy to focus on promising areas with finer resolution or employ other search techniques such as gradient ascent or simulated annealing if the function's behavior suggests that would be effective. \{...\}
        };
        \node[above=0.45cm of prompt1.north, anchor=north] {\textbf{Thesis 1}};

        % Prompt 2
        \node[fill=green!30, draw=SlateGray2, rounded corners, inner sep=5pt, text width=\textwidth, below=of prompt1, yshift=0.2cm] (prompt2) {
            \small
            The agent has made a commendable start with a grid search strategy \{...\}
            
            \subsection*{Criticism and Suggestions:}
            \vspace{0.5em}
            \begin{enumerate}
                \item \textbf{Refinement Needed}: The initial grid search has provided a good overview, but it's crucial now to refine the search in areas that have shown potential. The agent should consider applying a finer grid or other search techniques in regions around the highest values found.
                \item \textbf{Adaptive Strategy}: The agent mentioned the possibility of employing techniques such as gradient ascent or simulated annealing based on the function's behavior. Given the highest values observed, it might be beneficial to start incorporating these techniques, especially around the peaks identified.
                \item \textbf{Exploration vs. Exploitation}: The agent should balance between exploring unvisited areas and exploiting areas where high values have been found. While the current strategy focuses on exploration, there should be a planned transition towards exploitation, especially as the number of queries starts to diminish.
            \end{enumerate}
            
            \{...\}
                       
            \subsection*{Potential Issues:}
            \vspace{0.5em}
            \begin{itemize}
                \item The strategy lacks specificity on how to transition from broad exploration to focused exploitation.
                \item There is no explicit mention of how to incorporate the feedback from the function evaluations into refining the search strategy.
                \item The agent should ensure a diverse exploration that does not overlook smaller regions of the search space that have not been covered by the initial grid.
            \end{itemize}
            
            \subsection*{Next Steps:}
            \vspace{0.5em}
            \begin{itemize}
                \item Consider narrowing down the search around the current maximum and 2nd maximum areas with a finer grid or other search techniques.
                \item Evaluate the function's behavior based on the initial results to adapt the search strategy effectively.
                \item Continue to explore other areas of the search space that have not been covered by the initial grid, possibly with a different spacing or method to ensure thorough coverage.
            \end{itemize}
            
            With remaining queries, it's essential to start focusing on both refining the search around promising areas and continuing to explore less covered regions efficiently.
        };
        \node[above=0.45cm of prompt2.north, anchor=north] {\textbf{Antithesis 1}};

        % Background rectangle
        \begin{pgfonlayer}{background}
            \node[fit=(title)(prompt1)(prompt2), fill=blue!10, rounded corners, inner sep=10pt] {};
        \end{pgfonlayer}
    \end{tikzpicture}
    \vspace{-2em} 
    \caption{A sample of dialectical progress in ACE (Part 1). Some responses, including coding parts, have been omitted for brevity/space.}
    \label{fig:ace:sample:part1}
\end{figure}

\begin{figure}[th!]
    \centering
    \begin{tikzpicture}
        % Define the background layer
        \pgfdeclarelayer{background}
        \pgfsetlayers{background,main}

        % Title
        \node[align=center, text=black, font=\large\bfseries] (title) {A Sample of Dialectical Progress in ACE (Part 2)};

        % Prompt 1
        \node[fill=orange!30, draw=SlateGray2, rounded corners, inner sep=5pt, text width=\textwidth, below=of title, yshift=0.2cm] (prompt1) {
            \small
            Considering the valuable feedback, I will now incorporate a mix of further refined grid search around the current maximum areas and an adaptive strategy like simulated annealing for areas where we've seen significant values but haven't fully explored. This hybrid approach aims to exploit the known high-value areas thoroughly while exploring new regions that might contain higher maxima. Specifically, I'll apply a more granular grid search around maximum point so far and explore surrounding regions with potential, using \textbf{simulated annealing} techniques to dynamically adjust based on feedback from the function. This strategy should allow us to efficiently use our remaining queries by focusing on promising areas and adapting our search in real-time to the function's landscape \{...\}
        };
        \node[above=0.45cm of prompt1.north, anchor=north] {\textbf{Synthesis 1 (Thesis 2)}};

        % Prompt 2
        \node[fill=green!30, draw=SlateGray2, rounded corners, inner sep=5pt, text width=\textwidth, below=of prompt1, yshift=0.2cm] (prompt2) {
            \small
            The agent has nicely adjusted its strategy based on previous feedback, focusing on both exploitation around the areas with the highest values found and exploration in less covered regions. This dual approach is crucial for balancing the discovery of potential global maxima while also ensuring that no high-value areas are overlooked in less explored regions. The use of a denser grid around the points of interest and strategic selection in other areas is a good application of the feedback provided.
                      
            \subsection*{Criticism and Suggestions:}
                                    
            \vspace{.5em}
            
            \begin{enumerate}
                \item \textbf{Strategy Adjustment}: The agent has made a solid effort in refining its search around promising areas. For future queries, it might be beneficial to incorporate more adaptive techniques such as simulated annealing or gradient ascent in areas where the function's behavior suggests a smooth gradient leading to a peak, especially since the agent has already identified significantly high values.
                \item \textbf{Coverage of Unexplored Areas}: While the agent has started to explore less covered regions, it's important to continue this trend, ensuring that all parts of the search space receive some level of attention. This is critical to avoid missing out on potentially higher peaks in unexplored areas.
                \item \textbf{Further Refinement}: Given the highest values found so far are concentrated in specific regions, further refinement in these areas could be beneficial. However, the agent should also be cautious not to concentrate too heavily in these areas at the expense of broader exploration.
                \item \textbf{Potential for Local Maxima}: The current strategy does well in focusing on high-value areas but should remain cautious about the potential for these points to represent local rather than global maxima. The agent should consider additional strategies to test the surrounding areas more thoroughly to confirm whether these peaks are indeed the highest points in the landscape.
            \end{enumerate}
            \{...\}
            
            \subsection*{Potential Issues:}
                        
            \vspace{.5em}
            
            \begin{itemize}
                \item The strategy may benefit from integrating more dynamic search algorithms in addition to the denser grid approach for more efficiently navigating the function's landscape.
                \item There's a risk of over-focusing on the currently identified high-value areas without sufficient exploration of other potentially high-value regions.
            \end{itemize}            
                
                \{...\} With remaining queries, strategic decisions on where to focus these remaining queries will be crucial for identifying the global maximum.
        };
        \node[above=0.45cm of prompt2.north, anchor=north] {\textbf{Antithesis 2}};

        % Prompt 3
        \node[fill=Olive!30, draw=SlateGray2, rounded corners, inner sep=5pt, text width=\textwidth, below=of prompt2, yshift=0.2cm] (prompt3) {
            \small
            Acknowledging the insightful feedback, I'll maintain the hybrid approach but with an adjustment to broaden the exploration. While continuing to exploit the highest value regions identified so far, I'll expand the use of simulated annealing to other promising yet less explored areas. This should help in uncovering potential peaks outside the heavily focused region. Additionally, I'll reassess and possibly adjust the step size for simulated annealing to balance between thorough exploration and efficiency. The goal remains to efficiently use the remaining queries by focusing on both refining the understanding of known high-value areas and exploring new regions to maximize the chances of identifying the global maximum \{...\}
         };
        \node[above=0.45cm of prompt3.north, anchor=north] {\textbf{Synthesis 2 (Thesis 3)}};

        % Background rectangle
        \begin{pgfonlayer}{background}
            \node[fit=(title)(prompt1)(prompt2)(prompt3), fill=blue!10, rounded corners, inner sep=10pt] {};
        \end{pgfonlayer}
    \end{tikzpicture}
    \vspace{-2em} 
    \caption{A sample of dialectical progress in ACE (Part 2). Some responses, including coding parts, have been omitted for brevity/space.}
    \label{fig:ace:sample:part2}
\end{figure}
% \begin{figure}[h!]
%     \centering
%     \begin{tikzpicture}
%         % Define the background layer
%         \pgfdeclarelayer{background}
%         \pgfsetlayers{background,main}

%         % Title
%         \node[align=center, text=white, font=\large\bfseries] (title) {A Sample of Dialectical Progress in ACE (Part 3)};

%         % Prompt 1
%         \node[fill=red!30, draw=SlateGray2, rounded corners, inner sep=5pt, text width=\textwidth, below=of title, yshift=0.2cm] (prompt1) {
%             \small
%             Acknowledging the insightful feedback, I'll maintain the hybrid approach but with an adjustment to broaden the exploration. While continuing to exploit the highest value regions identified so far, I'll expand the use of simulated annealing to other promising yet less explored areas. This should help in uncovering potential peaks outside the heavily focused region. Additionally, I'll reassess and possibly adjust the step size for \textbf{simulated annealing} to balance between thorough exploration and efficiency. The goal remains to efficiently use the remaining queries by focusing on both refining the understanding of known high-value areas and exploring new regions to maximize the chances of identifying the global maximum.
%         };
%         \node[above=0.45cm of prompt1.north, anchor=north] {\textbf{Synthesis 2}};

%         % Prompt 2
%         \node[fill=green!30, draw=SlateGray2, rounded corners, inner sep=5pt, text width=\textwidth, below=of prompt1, yshift=0.2cm] (prompt2) {
%             \small

%         };
%         \node[above=0.45cm of prompt2.north, anchor=north] {\textbf{Antithesis 1}};

%         % Background rectangle
%         \begin{pgfonlayer}{background}
%             \node[fit=(title)(prompt1)(prompt2), fill=blue!20, rounded corners, inner sep=10pt] {};
%         \end{pgfonlayer}
%     \end{tikzpicture}
%     \caption{A sample of dialectical progress in ACE (Part 2). For brevity and sake of space, we omitted the other parts of the responses including the coding parts.}
%     \label{fig:ace:sample:part1}
% \end{figure}


% \begin{figure}[h!]
%     \centering
%     \begin{tikzpicture}
%         % Define the background layer
%         \pgfdeclarelayer{background}
%         \pgfsetlayers{background,main}

%         % Title
%         \node[align=center, text=white, font=\large\bfseries] (title) {A Sample of Dialectical Progress in ACE (Part 1)};

%         % Prompt 1
%         \node[fill=red!30, draw=SlateGray2, rounded corners, inner sep=5pt, text width=\textwidth, below=of title, yshift=0.2cm] (prompt1) {
%             \small
%         };
%         \node[above=0.45cm of prompt1.north, anchor=north] {\textbf{Thesis 1}};

%         % Prompt 2
%         \node[fill=green!30, draw=SlateGray2, rounded corners, inner sep=5pt, text width=\textwidth, below=of prompt1, yshift=0.2cm] (prompt2) {
%             \small

%         };
%         \node[above=0.45cm of prompt2.north, anchor=north] {\textbf{Antithesis 1}};

%         % Background rectangle
%         \begin{pgfonlayer}{background}
%             \node[fit=(title)(prompt1)(prompt2), fill=blue!20, rounded corners, inner sep=10pt] {};
%         \end{pgfonlayer}
%     \end{tikzpicture}
%     \caption{A sample of dialectical progress in ACE (Part 2). For brevity and sake of space, we omitted the other parts of the responses including the coding parts.}
%     \label{fig:ace:sample:part1}
% \end{figure}
