\begin{figure}[th!]
    \centering
    \begin{tikzpicture}
        % Define the background layer
        \pgfdeclarelayer{background}
        \pgfsetlayers{background,main}

        % Title
        % \node[align=center, text=black, font=\large\bfseries] (title) {Sequence of Strategies Taken by the LLM$^+$ (Sample 1)};
        % \node[align=center, text=black, font=\large\bfseries] (title) {};
        
        % Prompt 1
        \node[fill=orange!30, draw=SlateGray2, rounded corners, inner sep=5pt, text width=\textwidth] (prompt1) {
        % \node[fill=orange!30, draw=SlateGray2, rounded corners, inner sep=5pt, text width=\textwidth, below=of title, yshift=0.2cm] (prompt1) {
            \small
            Given the large search space (-1000 to 1000 for both x and y) and the potential complexity of the function with multiple local maxima, an initial broad sampling strategy is essential to understand the landscape of the function. The strategy will start with a coarse grid search to identify regions of interest where the function appears to have higher values. This initial phase will involve selecting points that are evenly distributed across the entire search space. The objective is to cover as much ground as possible with a limited number of queries to avoid missing any area that might contain the global maximum.
            Once a region of interest is identified from the initial sampling, the strategy will shift towards a more focused search in that area. This could involve a tighter grid search or other optimization algorithms that are more suited for local search, such as gradient ascent, simulated annealing, or genetic algorithms, depending on the feedback received from the initial queries.
            This approach balances the exploration of the search space (to avoid missing the global maximum) with the exploitation of promising areas (to refine the search around potential maxima). It's a flexible strategy that allows adjustments based on ongoing findings. The initial phase of broad exploration is crucial for forming a good foundation for subsequent, more targeted search efforts.
            
        };
        \node[above=0.5cm of prompt1.north, anchor=north] {\textbf{LLM$^+$ Strategy in Round 1}};

        % Prompt 2
        \node[fill=lightgray!10, draw=SlateGray2, rounded corners, inner sep=5pt, text width=\textwidth, below=of prompt1, yshift=0.5cm] (prompt2) {
            \small
            NA

        };
        \node[above=0.5cm of prompt2.north, anchor=north] {\textbf{LLM$^+$ Strategy in the Next Rounds}};

        % Background rectangle
        \begin{pgfonlayer}{background}
            \node[fit=(title)(prompt1)(prompt2), fill=blue!10, rounded corners, inner sep=10pt] {};
        \end{pgfonlayer}
    \end{tikzpicture}
    \vspace{-2em} 
    \caption{A sample sequence of strategies employed by LLM$^+$ over 16 rounds of interaction with the World. "NA" indicates no changes in the current strategy compared to the previously mentioned one, meaning the agent did not alter its initial strategy in the subsequent 15 rounds. The agent did not find the optimum point by the end of this run.}
    \label{fig:llm+:sample1}
\end{figure}

\begin{figure}[th!]
    \centering
    \begin{tikzpicture}
        % Define the background layer
        \pgfdeclarelayer{background}
        \pgfsetlayers{background,main}

        % Title
        % \node[align=center, text=black, font=\large\bfseries] (title) {Sequence of Strategies Taken by the LLM$^+$ (Sample 1)};

        % Prompt 1
        % \node[fill=orange!30, draw=SlateGray2, rounded corners, inner sep=5pt, text width=\textwidth, below=of title, yshift=0.2cm] (prompt1) {
        \node[fill=orange!30, draw=SlateGray2, rounded corners, inner sep=5pt, text width=\textwidth, yshift=0.2cm] (prompt1) {
            \small
            Given the vast search space and the unknown nature of the function \( f(x, y) \), a smart approach would be to start with a coarse exploration to identify promising regions, followed by a more refined search around those regions. The initial phase will use a grid-based approach to sample the search space systematically. This will help in identifying regions with potentially high values of \( f(x, y) \). After identifying promising areas, the strategy will shift to a more localized search, possibly using gradient ascent or another optimization algorithm if the nature of the function seems to allow it. This phased approach allows for both exploration and exploitation, balancing the need to discover global features of the search space with the need to zero in on the global maximum.          
        };
        \node[above=0.5cm of prompt1.north, anchor=north] {\textbf{LLM$^+$ Strategy in Round 1}};

        % Prompt 2
        \node[fill=lightgray!10, draw=SlateGray2, rounded corners, inner sep=5pt, text width=\textwidth, below=of prompt1, yshift=0.5cm] (prompt2) {
            \small                       
            Given the results from the initial exploration, it's clear that certain regions are more promising than others, particularly around the coordinates (333.3333, 555.5556) where we observed the highest value so far. The next step will refine the search around this promising area. We will adopt a more localized grid search strategy around this point, narrowing down the search space to further pinpoint areas of higher values. This phase will involve selecting points closer to (333.3333, 555.5556) but expanding outwards in smaller increments to ensure we are thoroughly exploring the vicinity of the current maximum.

        };
        \node[above=0.5cm of prompt2.north, anchor=north] {\textbf{LLM$^+$ Strategy in Round 2}};

        % Prompt 3
        \node[fill=orange!30, draw=SlateGray2, rounded corners, inner sep=5pt, text width=\textwidth, below=of prompt2, yshift=0.5cm] (prompt3) {
            \small                       
            The refined search has successfully identified a higher value at (433.3333, 455.5556), suggesting that moving further in this direction might lead to even higher values. Given the pattern observed in the results, where increasing \( x \) and slightly decreasing \( y \) from the current maximum point has led to higher values, the strategy will now focus on expanding the search area in this direction. The aim is to explore further along the \( x \)-axis while slightly adjusting the \( y \)-axis, to see if the trend of increasing function values continues, which could lead us closer to the global maximum.
        };
        \node[above=0.5cm of prompt3.north, anchor=north] {\textbf{LLM$^+$ Strategy in Round 3}};
        
        % Prompt 4
        \node[fill=lightgray!10, draw=SlateGray2, rounded corners, inner sep=5pt, text width=\textwidth, below=of prompt3, yshift=0.5cm] (prompt4) {
            \small                       
            The results show that the highest value found so far is at (533.3333, 355.5556). This suggests a peak in this region. Given the limited number of queries left, the strategy now will be to conduct a more concentrated search around this peak value to ensure we are not missing a higher value nearby. We will decrease the step size to explore more finely around this point, focusing on both \( x \) and \( y \) directions to potentially locate the global maximum or a value very close to it.
        };
        \node[above=0.5cm of prompt4.north, anchor=north] {\textbf{LLM$^+$ Strategy in Round 4}};

        % Prompt 5
        \node[fill=orange!30, draw=SlateGray2, rounded corners, inner sep=5pt, text width=\textwidth, below=of prompt4, yshift=0.5cm] (prompt5) {
            \small                       
            NA
        };
        \node[above=0.5cm of prompt5.north, anchor=north] {\textbf{LLM$^+$ Strategy in Rounds 5, 6, and 7}};
       
        % Background rectangle
        \begin{pgfonlayer}{background}
            \node[fit=(title)(prompt1)(prompt2)(prompt3)(prompt4)(prompt5), fill=blue!10, rounded corners, inner sep=10pt] {};
        \end{pgfonlayer}
    \end{tikzpicture}
    \vspace{-2em} 
    \caption{A sample sequence of strategies employed by LLM$^+$ over 7 rounds of interaction with the World. "NA" indicates no changes in the current strategy compared to the previously mentioned one, meaning the agent did not alter its strategy in the last 3 rounds. The agent did not find the optimum point by the end of this run.}
    \label{fig:llm+:sample2}
\end{figure}


