\section{Limitations \& a Brief Discussion}
% \blueitemize{
%     \item world generator: need for a manual solution as the Expert solution for each world
%     \item ACE: the benefits of ACE in non-iterative tasks that will not provide any actual feedback, ground truth, during the interaction with the problem is not clear yet. Our hypothesis is that having a ground truth is essential to make the triad of thesis, antithesis, and synthesis work in practice for a LLM. To investigate that, we evaluated a modified version of ACE that does not get any feedback signal from the world in MMLU benchmarks.
%     \item our finding shows that ACE will not provide with tangible improvements over the default. We saw that as an evidence for our previous hypothesis, while that will require more investigation    
% }
\textbf{WorldGen's Limitation:}
WorldGen effectively generates worlds with adjustable complexity for testing LLMs in SOPs, but relies on manually designed Expert solutions to solve the SOP. This dependence on human expertise for robust baselines can be time-intensive and limit the automation potential of the approach. We leave addressing the fully automated objective to future work.

\textbf{Limitations of ACE:}
% ACE and its dialectical framework have demonstrated great performance in our main targeted scenarios, SOPs. However, the effectiveness of ACE in other domains, particularly in settings where real-time feedback is unavailable, remains an open question. For instance, in static question-answering tasks or myopic scenarios where answers are final and lack iterative refinement opportunities, the benefits of ACE’s approach might not be as pronounced. To fully understand its potential and limitations in these contexts, extensive future evaluations are necessary. Moreover, by treating LLMs as black boxes, ACE performance is inherently constrained by the capabilities of the underlying base model. Lastly, the dialectical approach of ACE results in a slight natural increase in token consumption, though ACE is more token-efficient than multi-agent schemes like Debate or Majority. This overhead may pose challenges in scenarios where computational resources or cost efficiency are critical considerations, though where finding a robust solution outweighs minor increases in token costs, as in complex SOP tasks, this trade-off is less significant.
ACE's dialectical framework has demonstrated great performance in our main targeted domain, SOPs, but its effectiveness in other domains, particularly those lacking real-time feedback, remains an open question. In static question-answering or static tasks without iterative refinement, the benefits of ACE may be limited, warranting further evaluation. Additionally, by treating LLMs as black boxes, ACE's performance is inherently bound by the capabilities of the underlying model. Moreover, while its token consumption is lower than multi-agent schemes like Debate or Majority, ACE incurs a slight overhead compared to single-agent approaches. This trade-off is minor in complex SOP tasks but could pose challenges in resource-constrained scenarios.

\textbf{On the Potential of ACE:} 
% We think that the potentials of ACE and its Hegelian dialectical roots extend beyond solving SOPs. First, ACE and its dialectical aspect mirroring human-like problem-solving processes encourage the development of solutions that are not only accurate but also deeply contextual and well-reasoned. Moreover, the well-established philosophical frameworks such as Hegelian dialects, have the abilities to elucidate why other prompt engineering techniques, such as self-reflection, are effective. By framing these techniques within a dialectical structure, we can have a foundation to understand existing methods. Additionally, our Hegelian-inspired framework provides a robust structure for \textit{generating synthetic data}. The iterative nature of the dialectical process enables the creation of diverse and high-quality datasets that capture a wide range of perspectives and solutions. These datasets can be instrumental in training and fine-tuning LLMs, ensuring they are better equipped to handle complex and nuanced tasks.
% The potential of ACE, rooted in its Hegelian dialectical framework, extends beyond solving SOPs. Its dialectical approach, mirroring human-like problem-solving processes, fosters solutions that are not only accurate but also deeply contextual and well-reasoned. Furthermore, Hegelian philosophy provides a foundation to explain the effectiveness of other prompt engineering techniques, such as self-reflection, by framing them within a structured dialectical process. This perspective helps deepen our understanding of existing methods and their mechanisms. Additionally, the Hegelian-inspired framework offers a powerful structure for \textit{generating synthetic data}. Its iterative nature facilitates the creation of diverse, high-quality datasets that reflect a broad range of perspectives and solutions, making them invaluable for training and fine-tuning LLMs to tackle complex and nuanced tasks effectively.
The potential of ACE, rooted in its Hegelian dialectical framework, extend beyond solving SOPs. Its dialectical approach, mirroring human-like problem-solving processes, fosters solutions that are not only accurate but also deeply contextual and well-reasoned. Furthermore, Hegelian philosophy provides a foundation to explain the effectiveness of other prompt engineering techniques, such as self-reflection, by framing them within a structured dialectical process. This perspective can deepen our understanding of existing methods and their mechanisms.
Additionally, the Hegelian-inspired framework offers a powerful structure for \textit{generating synthetic data}. Its iterative nature facilitates the creation of diverse, high-quality datasets that reflect a broad range of perspectives and solutions, making them invaluable for training and fine-tuning LLMs to tackle complex and nuanced tasks effectively.

\textbf{LLM$^+$ Could Have Been Better!}
A fair criticism might be that LLMs might perform better in solving SOPs with improved prompt engineering. We are not claiming that LLM$^+$ represents the optimal default scheme; rather, we argue that it serves as a robust baseline. Even with carefully designed prompts, LLMs' performance in this setting remains limited, highlighting the need for approaches like ACE to unlock their full potential and deliver superior performance.

\textbf{What If the Next LLM Becomes Very Capable?} 
% Having a more capable LLM does not risk ACE becoming less useful. On the contrary, as demonstrated in section~\ref{sec:eval}, a better base model provides a stronger foundation for achieving even higher performance. In other words, ACE can make a great LLM even greater!
A more capable LLM makes ACE even more useful, not less. As demonstrated in Section~\ref{sec:eval}, a better base model serves as a stronger foundation, enabling even greater performance improvements. In essence, ACE with its dialectical base is designed to complement and amplify the capabilities of any LLM, regardless of its initial proficiency in SOP context. By leveraging ACE, we can transform an already impressive LLM into an extraordinary one, pushing what is possible and unlocking new levels of performance in this domain.

% While ACE demonstrates great advancements in enhancing LLMs performance across SOPs, it is not without limitations. Firstly, while ACE excels in our targeted scenarios (SOPs) involving sequential decision-making and dynamic feedback, its performance in other tasks such as myopic ones needs to be seen and requires more evaluations. Although ACE's dialectical method benefits in myopic tasks such as answering multiple-choice questions, the benefits might not as pronounced as in SOP setting, as no access to iterative feedback for antithesis and synthesis processes limits sufficient grounding to generate meaningful refinements. Second, ACE operates as a meta-framework, treating LLMs as black boxes without retraining or modifying their internal weights. As a result, its performance is naturally bounded by the inherent capabilities of the underlying base model. Third, the dialectical approach of ACE introduces a slight natural increase in token consumption due to the iterative exchange of Thesis, Antithesis, and Synthesis, along with associated explanations and strategy descriptions. While ACE is more token-efficient than multi-agent schemes such as Debate and Majority, it still incurs additional token cost compared to a single-agent scheme (LLM$^+$). This overhead may present a challenge in scenarios where computational resources or cost-efficiency are critical considerations, though this is less relevant in complex SOP tasks that finding a proper solution overrules the slight increase in the token cost.

% While ACE demonstrates great advancements in enhancing LLMs performance across SOPs, it is not without limitations. These limitations arise primarily from its dependency on the base model's capabilities, the token overhead associated with its dialectical structure, and its reliance on the availability and quality of feedback.

% ACE's architecture heavily relies on feedback from the World to generate effective Antitheses. This feedback acts as a guiding ground truth, allowing ACE to iteratively refine its reasoning and improve decision-making. In settings where real-time feedback is unavailable (e.g., static question-answering tasks), ACE's effectiveness diminishes, as evidenced by the reduced performance of ACE$^\ast$ in the MMLU benchmark. The lack of feedback in such scenarios limits the model's ability to effectively challenge and improve upon its initial hypotheses, thereby narrowing its scope of applicability.

% ACE operates as a meta-framework, treating LLMs as black boxes without retraining or modifying their internal weights. As a result, its performance is bounded by the inherent capabilities of the underlying base model. For example, in cases where the base model lacks foundational knowledge to solve a problem (e.g., achieving a 0\% success rate in certain tasks), ACE cannot overcome this limitation, as its dialectical structure builds upon the model's existing abilities.

% The dialectical approach of ACE introduces a natural increase in token consumption due to the iterative exchange of Thesis, Antithesis, and Synthesis, along with associated explanations and strategy descriptions. While ACE is more token-efficient than multi-agent schemes like Debate$^\star$ and Majority$^\star$, it still incurs approximately $2.27\times$ the token cost of a single-agent scheme (LLM$^+$). This overhead may present a challenge in scenarios where computational resources or cost-efficiency are critical considerations.

% Deploying ACE in real-world applications may involve additional complexity due to its multi-step process. Configuring the interplay between Thesis, Antithesis, and Synthesis, as well as integrating feedback mechanisms, requires careful design to ensure optimal performance.

% Although the WorldGen block is able to generate worlds with flexible complexity to test future LLMs with increased capabilities, we still require the design of an Expert solution to solve the SOP in a generated world. This is essential to establish access to the query budget and to provide a benchmark of efficient solutions for comparison with the performance of LLMs. That said, our approach is not fully automated. The reliance on a manually designed Expert solution introduces a dependency on human expertise for creating robust baselines, which may limit scalability and adaptability for new or unforeseen SOP scenarios. Additionally, designing such Expert solutions can be time-consuming and may not always generalize well to highly dynamic or unconventional world configurations generated by WorldGen. However, WorldGen provides a crucial first step toward automating this process by offering a structured and adaptable environment that can facilitate the eventual development of more autonomous and generalized expert systems.

% In summary, while ACE offers a powerful framework for enhancing LLM capabilities, its reliance on feedback, sensitivity to the base model, and token inefficiency highlight areas for further research and refinement. Future work could explore strategies to mitigate these limitations, such as incorporating lightweight feedback approximations, adapting ACE for tasks without dynamic feedback, and optimizing token usage.