\section{Related Studies}
There are generally three main categories of research on task scheduling. The first category involves analyzing tasks as bit streams, without taking into account any inherent task dependencies/topology. The second category focuses on tasks represented as directed acyclic graphs, which require sequential processing of task components. Finally, the third category considers tasks represented as undirected graphs, allowing for the parallel processing of all task components (which represents our emphasis in this paper). In the following, we provide a summary of the studies regarding each category.

\noindent
\textit{\textbf{Tasks represented as bit streams. }}Bit stream-based task representations have been a common practice, where the tasks of interest lack any internal processing complexities and interdependencies. Considering such tasks, efforts have been made to achieve a trade-off between delay and task execution cost \cite{li2022trade}, improve the task processing time efficiency \cite{alameddine2019dynamic}, and enhance task offloading under delay constraints \cite{yue2021todg,tutuncuouglu2022online}. However, our work migrates from bit-stream task representations and focuses on a different scheduling approach suitable for computation-intensive tasks with internal processing topologies, represented as graphs.

\noindent
\textit{\textbf{Tasks represented as directed acyclic graphs (DAGs). }}{DAG tasks, with their directed edges, facilitate sequential execution between adjacent task components and, consequently, the servers handling them. In DAG task scheduling, the primary focus is to determine the proPerexecution order of task components and the mapping of components to the appropriate servers, respecting the directed edges. This consideration does not impose stringent requirements on keeping inter-server communications, as some components will not be processed in parallel. Instead, ensuring the correct handling of inputs and outputs between task components becomes the primary concern.} Several researchers have proposed innovative scheduling schemes for DAG task scheduling to improve system stability, time efficiency, and resource utilization. For instance, Sun \textit{et al.} \cite{sun2018cooperative} used a genetic algorithm-based approach, Zhang \textit{et al.} \cite{zhang2022dag} leveraged advanced reinforcement learning and graph neural networks to minimize task completion time, and Liu \textit{et al.} \cite{liu2020dependency} proposed a priority-aware scheduling algorithm to speed up the task completion process. To map DAG tasks to virtual computing clusters, Liu \textit{et al.} \cite{liu2023rfid} proposed a dynamic scheduling scheme based on sorting and prediction techniques. Lv \textit{et al.} \cite{lv2022tbtoa} introduced a heuristic algorithm that considers DAG tasks and service cache constraints to predict the impact of offloading decisions on task completion time and energy consumption. In general, DAG task components are processed sequentially according to the directions of edges in their DAG representation (i.e., the scheduling of DAG tasks does not require addressing the subgraph isomorphism problem). However, in this work, we focus on computation-intensive applications with an undirected graph representation, where all the task components can be executed in parallel, and thus our developed methodology is significantly different from the above literature. 
	
\noindent{
\textit{\textbf{Tasks represented as undirected graphs (UGs).}} {In contrast to DAG tasks, the primary focus in scheduling UG tasks is to identify parallel computation opportunities between connected task components, which necessitates continuous communication among computing servers when handling connected components. Generally, this requires solving the isomorphic subgraph problem to optimize task execution. In this paper, we focus on the scheduling of UG tasks in dynamic networks, an area that is not yet well-explored in the literature.} Although various studies have explored task scheduling across both static and dynamic networks, their scope is constrained by the specific conditions of the network or the status of the servers. For instance, Ghaderi \textit{et al.} \cite{ghaderi2016scheduling} proposed a randomized task scheduling algorithm for static networks. Al-Habob \textit{et al.} \cite{al2020task} introduced two scheduling algorithms based on the genetic algorithm and conflict graph model for dynamic networks. Hosseinalipour \textit{et al.} \cite{hosseinalipour2019power} studied the graph task allocation over geo-distributed cloud networks with various scales. Shi \textit{et al.} \cite{shi2016energy} investigated task scheduling aiming to minimize the task completion time while considering energy consumption. Abdisarabshali \textit{et al.} \cite{abdisarabshali2023decomposition} studied redundancy-aware task processing over dynamic vehicular networks.
As compared to the works discussed above, this paPerexplores the scheduling of graph tasks in the presence of uncertainties in resource supply and vehicle mobility, which further complicates the scheduling problem and calls for nuanced solutions. We have made some early efforts towards this direction \cite{liwang2020multi,gao2021truthful,liwang2020energy,liwang2019allocation,liwang2023graph}, which include proposing a randomized task allocation mechanism under task concurrency \cite{liwang2020multi}, auction-promoted task scheduling under resource reutilization \cite{gao2021truthful}, randomized task scheduling under hierarchical tree decomposition \cite{liwang2019allocation}, and simultaneous resource allocation and task scheduling over air-ground integrated networks \cite{liwang2023graph}. Moreover, existing studies generally focus on online decision-making for graph task scheduling according to the current network conditions, which sometimes leads to unacceptable prolonged task scheduling decisions, especially over complicated task/VC topologies. Thus, our proposed hybrid methodology opens a new degree of freedom (i.e., offline/pilot scheduling mode) to the above literature. Moreover, the majority of these studies consider predetermined system information, e.g., a fixed and known computing capability of vehicles. In contrast, this paPeraddresses practical challenges posed by network uncertainties. Table \ref{relate work comparison} presents a summary of related studies, highlighting the key differences of our paper. 
\begin{table}[htb]
	\centering
	\scriptsize
	\vspace{-0.2 cm}
	\caption{A summary of related works}
	\setlength{\tabcolsep}{3.0 pt} % 控制列间距
	\centering
	\vspace{-0.2 cm} 
	\begin{tabular}{|c|c|c|c|c|c|c|c|}
		\hline
		\multirow{2}{*}{\textbf{Reference}} & \multicolumn{2}{c|}{\makecell{\textbf{Environmental}\\ \textbf{attributes}}} & \multicolumn{2}{c|}{\textbf{Decision mode}} & \multicolumn{3}{c|}{\textbf{Task model}} \\ \cline{2-8} 
		& Stable & Dynamic & Online & Offline & Bit streams & DAG & UG \\
		\hline
		[20][21] &  & $\checkmark$ & $\checkmark$ &  & $\checkmark$ &  &  \\
		\hline
		[22][23] & $\checkmark$ &  & $\checkmark$ &  & $\checkmark$ &  &  \\
		\hline
		[2][16] &  & $\checkmark$ & $\checkmark$ &  &  & $\checkmark$ &  \\
		\hline
		[17][24] & $\checkmark$ &  & $\checkmark$ &  &  & $\checkmark$ &  \\
		\hline
		[19] &  & $\checkmark$ & $\checkmark$ &  &  & $\checkmark$ & $\checkmark$ \\
		\hline
		[11][12][18][25] & $\checkmark$ &  & $\checkmark$ &  &  &  & $\checkmark$ \\
		\hline
		[1][8][9][10][26] &  & $\checkmark$ & $\checkmark$ &  &  &  & $\checkmark$ \\
		\hline
		Our paPer&  & $\checkmark$ & $\checkmark$ & $\checkmark$ &  &  & $\checkmark$ \\
		\hline
	\end{tabular}
	\label{relate work comparison}
\end{table}

\vspace{-0.3 cm}