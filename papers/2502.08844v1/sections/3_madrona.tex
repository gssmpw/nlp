\section{Batch Rendering with Madrona}
\label{sec:madrona}

MuJoCo Playground enables vision-based environments through an integration of MJX with Madrona \cite{shacklett23madrona}. Madrona is a GPU-based entity-component-system (ECS), which contains GPU implementations of high throughput rendering \cite{rosenzweig24madronarenderer}. Madrona provides two rendering backends: a software-based batch ray tracer written in CUDA (used for the experiments in this work) and a Vulkan-based rasterizer. The raytracing backend supports features including complex lighting scenarios, shadows, textures, and geometry materials. See \Cref{fig:madrona_highres} for examples of rendered images using the batch ray tracer. Some features such as deformable materials, moving lights, and terrain height fields will be added in the future.

The Madrona Batch Renderer is integrated with MJX through low-level JAX \cite{jax2018github} primitives that connect to the initialization and render functions exposed by Madrona. These JAX primitives allow for Madrona to interact seamlessly with JAX transformations such as \textit{jit} and \textit{vmap}. Mujoco Playground provides two examples: (\texttt{cartpole-balance} and \texttt{PandaPickCubeCartesian}) to showcase the implementation of vision-based environments and training of vision-based policies.

The Madrona MJX integration also supports customization of each environment instance, allowing for domain randomization~\cite{domainrand2017} of visual properties such as geometry size, color, lighting conditions, and camera pose. These randomizations play a crucial role in the sim-to-real transfer of vision-based policies, which we discuss more in \Cref{sec:results_pickcube_pixels}.

