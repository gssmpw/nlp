%%%%%%%% ICML 2025 EXAMPLE LATEX SUBMISSION FILE %%%%%%%%%%%%%%%%%

\documentclass{article}
%%%%% NEW MATH DEFINITIONS %%%%%

\usepackage{amsmath,amsfonts,bm}
\usepackage{derivative}
% Mark sections of captions for referring to divisions of figures
\newcommand{\figleft}{{\em (Left)}}
\newcommand{\figcenter}{{\em (Center)}}
\newcommand{\figright}{{\em (Right)}}
\newcommand{\figtop}{{\em (Top)}}
\newcommand{\figbottom}{{\em (Bottom)}}
\newcommand{\captiona}{{\em (a)}}
\newcommand{\captionb}{{\em (b)}}
\newcommand{\captionc}{{\em (c)}}
\newcommand{\captiond}{{\em (d)}}

% Highlight a newly defined term
\newcommand{\newterm}[1]{{\bf #1}}

% Derivative d 
\newcommand{\deriv}{{\mathrm{d}}}

% Figure reference, lower-case.
\def\figref#1{figure~\ref{#1}}
% Figure reference, capital. For start of sentence
\def\Figref#1{Figure~\ref{#1}}
\def\twofigref#1#2{figures \ref{#1} and \ref{#2}}
\def\quadfigref#1#2#3#4{figures \ref{#1}, \ref{#2}, \ref{#3} and \ref{#4}}
% Section reference, lower-case.
\def\secref#1{section~\ref{#1}}
% Section reference, capital.
\def\Secref#1{Section~\ref{#1}}
% Reference to two sections.
\def\twosecrefs#1#2{sections \ref{#1} and \ref{#2}}
% Reference to three sections.
\def\secrefs#1#2#3{sections \ref{#1}, \ref{#2} and \ref{#3}}
% Reference to an equation, lower-case.
\def\eqref#1{equation~\ref{#1}}
% Reference to an equation, upper case
\def\Eqref#1{Equation~\ref{#1}}
% A raw reference to an equation---avoid using if possible
\def\plaineqref#1{\ref{#1}}
% Reference to a chapter, lower-case.
\def\chapref#1{chapter~\ref{#1}}
% Reference to an equation, upper case.
\def\Chapref#1{Chapter~\ref{#1}}
% Reference to a range of chapters
\def\rangechapref#1#2{chapters\ref{#1}--\ref{#2}}
% Reference to an algorithm, lower-case.
\def\algref#1{algorithm~\ref{#1}}
% Reference to an algorithm, upper case.
\def\Algref#1{Algorithm~\ref{#1}}
\def\twoalgref#1#2{algorithms \ref{#1} and \ref{#2}}
\def\Twoalgref#1#2{Algorithms \ref{#1} and \ref{#2}}
% Reference to a part, lower case
\def\partref#1{part~\ref{#1}}
% Reference to a part, upper case
\def\Partref#1{Part~\ref{#1}}
\def\twopartref#1#2{parts \ref{#1} and \ref{#2}}

\def\ceil#1{\lceil #1 \rceil}
\def\floor#1{\lfloor #1 \rfloor}
\def\1{\bm{1}}
\newcommand{\train}{\mathcal{D}}
\newcommand{\valid}{\mathcal{D_{\mathrm{valid}}}}
\newcommand{\test}{\mathcal{D_{\mathrm{test}}}}

\def\eps{{\epsilon}}


% Random variables
\def\reta{{\textnormal{$\eta$}}}
\def\ra{{\textnormal{a}}}
\def\rb{{\textnormal{b}}}
\def\rc{{\textnormal{c}}}
\def\rd{{\textnormal{d}}}
\def\re{{\textnormal{e}}}
\def\rf{{\textnormal{f}}}
\def\rg{{\textnormal{g}}}
\def\rh{{\textnormal{h}}}
\def\ri{{\textnormal{i}}}
\def\rj{{\textnormal{j}}}
\def\rk{{\textnormal{k}}}
\def\rl{{\textnormal{l}}}
% rm is already a command, just don't name any random variables m
\def\rn{{\textnormal{n}}}
\def\ro{{\textnormal{o}}}
\def\rp{{\textnormal{p}}}
\def\rq{{\textnormal{q}}}
\def\rr{{\textnormal{r}}}
\def\rs{{\textnormal{s}}}
\def\rt{{\textnormal{t}}}
\def\ru{{\textnormal{u}}}
\def\rv{{\textnormal{v}}}
\def\rw{{\textnormal{w}}}
\def\rx{{\textnormal{x}}}
\def\ry{{\textnormal{y}}}
\def\rz{{\textnormal{z}}}

% Random vectors
\def\rvepsilon{{\mathbf{\epsilon}}}
\def\rvphi{{\mathbf{\phi}}}
\def\rvtheta{{\mathbf{\theta}}}
\def\rva{{\mathbf{a}}}
\def\rvb{{\mathbf{b}}}
\def\rvc{{\mathbf{c}}}
\def\rvd{{\mathbf{d}}}
\def\rve{{\mathbf{e}}}
\def\rvf{{\mathbf{f}}}
\def\rvg{{\mathbf{g}}}
\def\rvh{{\mathbf{h}}}
\def\rvu{{\mathbf{i}}}
\def\rvj{{\mathbf{j}}}
\def\rvk{{\mathbf{k}}}
\def\rvl{{\mathbf{l}}}
\def\rvm{{\mathbf{m}}}
\def\rvn{{\mathbf{n}}}
\def\rvo{{\mathbf{o}}}
\def\rvp{{\mathbf{p}}}
\def\rvq{{\mathbf{q}}}
\def\rvr{{\mathbf{r}}}
\def\rvs{{\mathbf{s}}}
\def\rvt{{\mathbf{t}}}
\def\rvu{{\mathbf{u}}}
\def\rvv{{\mathbf{v}}}
\def\rvw{{\mathbf{w}}}
\def\rvx{{\mathbf{x}}}
\def\rvy{{\mathbf{y}}}
\def\rvz{{\mathbf{z}}}

% Elements of random vectors
\def\erva{{\textnormal{a}}}
\def\ervb{{\textnormal{b}}}
\def\ervc{{\textnormal{c}}}
\def\ervd{{\textnormal{d}}}
\def\erve{{\textnormal{e}}}
\def\ervf{{\textnormal{f}}}
\def\ervg{{\textnormal{g}}}
\def\ervh{{\textnormal{h}}}
\def\ervi{{\textnormal{i}}}
\def\ervj{{\textnormal{j}}}
\def\ervk{{\textnormal{k}}}
\def\ervl{{\textnormal{l}}}
\def\ervm{{\textnormal{m}}}
\def\ervn{{\textnormal{n}}}
\def\ervo{{\textnormal{o}}}
\def\ervp{{\textnormal{p}}}
\def\ervq{{\textnormal{q}}}
\def\ervr{{\textnormal{r}}}
\def\ervs{{\textnormal{s}}}
\def\ervt{{\textnormal{t}}}
\def\ervu{{\textnormal{u}}}
\def\ervv{{\textnormal{v}}}
\def\ervw{{\textnormal{w}}}
\def\ervx{{\textnormal{x}}}
\def\ervy{{\textnormal{y}}}
\def\ervz{{\textnormal{z}}}

% Random matrices
\def\rmA{{\mathbf{A}}}
\def\rmB{{\mathbf{B}}}
\def\rmC{{\mathbf{C}}}
\def\rmD{{\mathbf{D}}}
\def\rmE{{\mathbf{E}}}
\def\rmF{{\mathbf{F}}}
\def\rmG{{\mathbf{G}}}
\def\rmH{{\mathbf{H}}}
\def\rmI{{\mathbf{I}}}
\def\rmJ{{\mathbf{J}}}
\def\rmK{{\mathbf{K}}}
\def\rmL{{\mathbf{L}}}
\def\rmM{{\mathbf{M}}}
\def\rmN{{\mathbf{N}}}
\def\rmO{{\mathbf{O}}}
\def\rmP{{\mathbf{P}}}
\def\rmQ{{\mathbf{Q}}}
\def\rmR{{\mathbf{R}}}
\def\rmS{{\mathbf{S}}}
\def\rmT{{\mathbf{T}}}
\def\rmU{{\mathbf{U}}}
\def\rmV{{\mathbf{V}}}
\def\rmW{{\mathbf{W}}}
\def\rmX{{\mathbf{X}}}
\def\rmY{{\mathbf{Y}}}
\def\rmZ{{\mathbf{Z}}}

% Elements of random matrices
\def\ermA{{\textnormal{A}}}
\def\ermB{{\textnormal{B}}}
\def\ermC{{\textnormal{C}}}
\def\ermD{{\textnormal{D}}}
\def\ermE{{\textnormal{E}}}
\def\ermF{{\textnormal{F}}}
\def\ermG{{\textnormal{G}}}
\def\ermH{{\textnormal{H}}}
\def\ermI{{\textnormal{I}}}
\def\ermJ{{\textnormal{J}}}
\def\ermK{{\textnormal{K}}}
\def\ermL{{\textnormal{L}}}
\def\ermM{{\textnormal{M}}}
\def\ermN{{\textnormal{N}}}
\def\ermO{{\textnormal{O}}}
\def\ermP{{\textnormal{P}}}
\def\ermQ{{\textnormal{Q}}}
\def\ermR{{\textnormal{R}}}
\def\ermS{{\textnormal{S}}}
\def\ermT{{\textnormal{T}}}
\def\ermU{{\textnormal{U}}}
\def\ermV{{\textnormal{V}}}
\def\ermW{{\textnormal{W}}}
\def\ermX{{\textnormal{X}}}
\def\ermY{{\textnormal{Y}}}
\def\ermZ{{\textnormal{Z}}}

% Vectors
\def\vzero{{\bm{0}}}
\def\vone{{\bm{1}}}
\def\vmu{{\bm{\mu}}}
\def\vtheta{{\bm{\theta}}}
\def\vphi{{\bm{\phi}}}
\def\va{{\bm{a}}}
\def\vb{{\bm{b}}}
\def\vc{{\bm{c}}}
\def\vd{{\bm{d}}}
\def\ve{{\bm{e}}}
\def\vf{{\bm{f}}}
\def\vg{{\bm{g}}}
\def\vh{{\bm{h}}}
\def\vi{{\bm{i}}}
\def\vj{{\bm{j}}}
\def\vk{{\bm{k}}}
\def\vl{{\bm{l}}}
\def\vm{{\bm{m}}}
\def\vn{{\bm{n}}}
\def\vo{{\bm{o}}}
\def\vp{{\bm{p}}}
\def\vq{{\bm{q}}}
\def\vr{{\bm{r}}}
\def\vs{{\bm{s}}}
\def\vt{{\bm{t}}}
\def\vu{{\bm{u}}}
\def\vv{{\bm{v}}}
\def\vw{{\bm{w}}}
\def\vx{{\bm{x}}}
\def\vy{{\bm{y}}}
\def\vz{{\bm{z}}}

% Elements of vectors
\def\evalpha{{\alpha}}
\def\evbeta{{\beta}}
\def\evepsilon{{\epsilon}}
\def\evlambda{{\lambda}}
\def\evomega{{\omega}}
\def\evmu{{\mu}}
\def\evpsi{{\psi}}
\def\evsigma{{\sigma}}
\def\evtheta{{\theta}}
\def\eva{{a}}
\def\evb{{b}}
\def\evc{{c}}
\def\evd{{d}}
\def\eve{{e}}
\def\evf{{f}}
\def\evg{{g}}
\def\evh{{h}}
\def\evi{{i}}
\def\evj{{j}}
\def\evk{{k}}
\def\evl{{l}}
\def\evm{{m}}
\def\evn{{n}}
\def\evo{{o}}
\def\evp{{p}}
\def\evq{{q}}
\def\evr{{r}}
\def\evs{{s}}
\def\evt{{t}}
\def\evu{{u}}
\def\evv{{v}}
\def\evw{{w}}
\def\evx{{x}}
\def\evy{{y}}
\def\evz{{z}}

% Matrix
\def\mA{{\bm{A}}}
\def\mB{{\bm{B}}}
\def\mC{{\bm{C}}}
\def\mD{{\bm{D}}}
\def\mE{{\bm{E}}}
\def\mF{{\bm{F}}}
\def\mG{{\bm{G}}}
\def\mH{{\bm{H}}}
\def\mI{{\bm{I}}}
\def\mJ{{\bm{J}}}
\def\mK{{\bm{K}}}
\def\mL{{\bm{L}}}
\def\mM{{\bm{M}}}
\def\mN{{\bm{N}}}
\def\mO{{\bm{O}}}
\def\mP{{\bm{P}}}
\def\mQ{{\bm{Q}}}
\def\mR{{\bm{R}}}
\def\mS{{\bm{S}}}
\def\mT{{\bm{T}}}
\def\mU{{\bm{U}}}
\def\mV{{\bm{V}}}
\def\mW{{\bm{W}}}
\def\mX{{\bm{X}}}
\def\mY{{\bm{Y}}}
\def\mZ{{\bm{Z}}}
\def\mBeta{{\bm{\beta}}}
\def\mPhi{{\bm{\Phi}}}
\def\mLambda{{\bm{\Lambda}}}
\def\mSigma{{\bm{\Sigma}}}

% Tensor
\DeclareMathAlphabet{\mathsfit}{\encodingdefault}{\sfdefault}{m}{sl}
\SetMathAlphabet{\mathsfit}{bold}{\encodingdefault}{\sfdefault}{bx}{n}
\newcommand{\tens}[1]{\bm{\mathsfit{#1}}}
\def\tA{{\tens{A}}}
\def\tB{{\tens{B}}}
\def\tC{{\tens{C}}}
\def\tD{{\tens{D}}}
\def\tE{{\tens{E}}}
\def\tF{{\tens{F}}}
\def\tG{{\tens{G}}}
\def\tH{{\tens{H}}}
\def\tI{{\tens{I}}}
\def\tJ{{\tens{J}}}
\def\tK{{\tens{K}}}
\def\tL{{\tens{L}}}
\def\tM{{\tens{M}}}
\def\tN{{\tens{N}}}
\def\tO{{\tens{O}}}
\def\tP{{\tens{P}}}
\def\tQ{{\tens{Q}}}
\def\tR{{\tens{R}}}
\def\tS{{\tens{S}}}
\def\tT{{\tens{T}}}
\def\tU{{\tens{U}}}
\def\tV{{\tens{V}}}
\def\tW{{\tens{W}}}
\def\tX{{\tens{X}}}
\def\tY{{\tens{Y}}}
\def\tZ{{\tens{Z}}}


% Graph
\def\gA{{\mathcal{A}}}
\def\gB{{\mathcal{B}}}
\def\gC{{\mathcal{C}}}
\def\gD{{\mathcal{D}}}
\def\gE{{\mathcal{E}}}
\def\gF{{\mathcal{F}}}
\def\gG{{\mathcal{G}}}
\def\gH{{\mathcal{H}}}
\def\gI{{\mathcal{I}}}
\def\gJ{{\mathcal{J}}}
\def\gK{{\mathcal{K}}}
\def\gL{{\mathcal{L}}}
\def\gM{{\mathcal{M}}}
\def\gN{{\mathcal{N}}}
\def\gO{{\mathcal{O}}}
\def\gP{{\mathcal{P}}}
\def\gQ{{\mathcal{Q}}}
\def\gR{{\mathcal{R}}}
\def\gS{{\mathcal{S}}}
\def\gT{{\mathcal{T}}}
\def\gU{{\mathcal{U}}}
\def\gV{{\mathcal{V}}}
\def\gW{{\mathcal{W}}}
\def\gX{{\mathcal{X}}}
\def\gY{{\mathcal{Y}}}
\def\gZ{{\mathcal{Z}}}

% Sets
\def\sA{{\mathbb{A}}}
\def\sB{{\mathbb{B}}}
\def\sC{{\mathbb{C}}}
\def\sD{{\mathbb{D}}}
% Don't use a set called E, because this would be the same as our symbol
% for expectation.
\def\sF{{\mathbb{F}}}
\def\sG{{\mathbb{G}}}
\def\sH{{\mathbb{H}}}
\def\sI{{\mathbb{I}}}
\def\sJ{{\mathbb{J}}}
\def\sK{{\mathbb{K}}}
\def\sL{{\mathbb{L}}}
\def\sM{{\mathbb{M}}}
\def\sN{{\mathbb{N}}}
\def\sO{{\mathbb{O}}}
\def\sP{{\mathbb{P}}}
\def\sQ{{\mathbb{Q}}}
\def\sR{{\mathbb{R}}}
\def\sS{{\mathbb{S}}}
\def\sT{{\mathbb{T}}}
\def\sU{{\mathbb{U}}}
\def\sV{{\mathbb{V}}}
\def\sW{{\mathbb{W}}}
\def\sX{{\mathbb{X}}}
\def\sY{{\mathbb{Y}}}
\def\sZ{{\mathbb{Z}}}

% Entries of a matrix
\def\emLambda{{\Lambda}}
\def\emA{{A}}
\def\emB{{B}}
\def\emC{{C}}
\def\emD{{D}}
\def\emE{{E}}
\def\emF{{F}}
\def\emG{{G}}
\def\emH{{H}}
\def\emI{{I}}
\def\emJ{{J}}
\def\emK{{K}}
\def\emL{{L}}
\def\emM{{M}}
\def\emN{{N}}
\def\emO{{O}}
\def\emP{{P}}
\def\emQ{{Q}}
\def\emR{{R}}
\def\emS{{S}}
\def\emT{{T}}
\def\emU{{U}}
\def\emV{{V}}
\def\emW{{W}}
\def\emX{{X}}
\def\emY{{Y}}
\def\emZ{{Z}}
\def\emSigma{{\Sigma}}

% entries of a tensor
% Same font as tensor, without \bm wrapper
\newcommand{\etens}[1]{\mathsfit{#1}}
\def\etLambda{{\etens{\Lambda}}}
\def\etA{{\etens{A}}}
\def\etB{{\etens{B}}}
\def\etC{{\etens{C}}}
\def\etD{{\etens{D}}}
\def\etE{{\etens{E}}}
\def\etF{{\etens{F}}}
\def\etG{{\etens{G}}}
\def\etH{{\etens{H}}}
\def\etI{{\etens{I}}}
\def\etJ{{\etens{J}}}
\def\etK{{\etens{K}}}
\def\etL{{\etens{L}}}
\def\etM{{\etens{M}}}
\def\etN{{\etens{N}}}
\def\etO{{\etens{O}}}
\def\etP{{\etens{P}}}
\def\etQ{{\etens{Q}}}
\def\etR{{\etens{R}}}
\def\etS{{\etens{S}}}
\def\etT{{\etens{T}}}
\def\etU{{\etens{U}}}
\def\etV{{\etens{V}}}
\def\etW{{\etens{W}}}
\def\etX{{\etens{X}}}
\def\etY{{\etens{Y}}}
\def\etZ{{\etens{Z}}}

% The true underlying data generating distribution
\newcommand{\pdata}{p_{\rm{data}}}
\newcommand{\ptarget}{p_{\rm{target}}}
\newcommand{\pprior}{p_{\rm{prior}}}
\newcommand{\pbase}{p_{\rm{base}}}
\newcommand{\pref}{p_{\rm{ref}}}

% The empirical distribution defined by the training set
\newcommand{\ptrain}{\hat{p}_{\rm{data}}}
\newcommand{\Ptrain}{\hat{P}_{\rm{data}}}
% The model distribution
\newcommand{\pmodel}{p_{\rm{model}}}
\newcommand{\Pmodel}{P_{\rm{model}}}
\newcommand{\ptildemodel}{\tilde{p}_{\rm{model}}}
% Stochastic autoencoder distributions
\newcommand{\pencode}{p_{\rm{encoder}}}
\newcommand{\pdecode}{p_{\rm{decoder}}}
\newcommand{\precons}{p_{\rm{reconstruct}}}

\newcommand{\laplace}{\mathrm{Laplace}} % Laplace distribution

\newcommand{\E}{\mathbb{E}}
\newcommand{\Ls}{\mathcal{L}}
\newcommand{\R}{\mathbb{R}}
\newcommand{\emp}{\tilde{p}}
\newcommand{\lr}{\alpha}
\newcommand{\reg}{\lambda}
\newcommand{\rect}{\mathrm{rectifier}}
\newcommand{\softmax}{\mathrm{softmax}}
\newcommand{\sigmoid}{\sigma}
\newcommand{\softplus}{\zeta}
\newcommand{\KL}{D_{\mathrm{KL}}}
\newcommand{\Var}{\mathrm{Var}}
\newcommand{\standarderror}{\mathrm{SE}}
\newcommand{\Cov}{\mathrm{Cov}}
% Wolfram Mathworld says $L^2$ is for function spaces and $\ell^2$ is for vectors
% But then they seem to use $L^2$ for vectors throughout the site, and so does
% wikipedia.
\newcommand{\normlzero}{L^0}
\newcommand{\normlone}{L^1}
\newcommand{\normltwo}{L^2}
\newcommand{\normlp}{L^p}
\newcommand{\normmax}{L^\infty}

\newcommand{\parents}{Pa} % See usage in notation.tex. Chosen to match Daphne's book.

\DeclareMathOperator*{\argmax}{arg\,max}
\DeclareMathOperator*{\argmin}{arg\,min}

\DeclareMathOperator{\sign}{sign}
\DeclareMathOperator{\Tr}{Tr}
\let\ab\allowbreak


% Recommended, but optional, packages for figures and better typesetting:
\usepackage{microtype}
\usepackage{graphicx}
\usepackage{subcaption}
% \usepackage{subfigure}
\usepackage{booktabs} % for professional tables
\usepackage{eqparbox}

% hyperref makes hyperlinks in the resulting PDF.
% If your build breaks (sometimes temporarily if a hyperlink spans a page)
% please comment out the following usepackage line and replace
% \usepackage{icml2025} with \usepackage[nohyperref]{icml2025} above.
\usepackage{hyperref}

% Attempt to make hyperref and algorithmic work together better:
\newcommand{\theHalgorithm}{\arabic{algorithm}}


% Use the following line for the initial blind version submitted for review:
%\usepackage{icml2025}

% If accepted, instead use the following line for the camera-ready submission:
% \usepackage[accepted]{icml2025}

\usepackage[arxiv]{icml2025}


% For theorems and such
\usepackage{amsmath}
\usepackage{amssymb}
\usepackage{mathtools}
\usepackage{amsthm}

% if you use cleveref..
\usepackage[capitalize,noabbrev]{cleveref}

%%%%%%%%%%%%%%%%%%%%%%%%%%%%%%%%
% THEOREMS
%%%%%%%%%%%%%%%%%%%%%%%%%%%%%%%%
\theoremstyle{plain}
\newtheorem{theorem}{Theorem}[section]
\newtheorem{proposition}[theorem]{Proposition}
\newtheorem{lemma}[theorem]{Lemma}
\newtheorem{corollary}[theorem]{Corollary}
\theoremstyle{definition}
\newtheorem{definition}[theorem]{Definition}
\newtheorem{assumption}[theorem]{Assumption}
\theoremstyle{remark}
\newtheorem{remark}[theorem]{Remark}


% Todonotes is useful during development; simply uncomment the next line
%    and comment out the line below the next line to turn off comments
%\usepackage[disable,textsize=tiny]{todonotes}
\usepackage[textsize=tiny]{todonotes}


% The \icmltitle you define below is probably too long as a header.
% Therefore, a short form for the running title is supplied here:
\icmltitlerunning{Sharpness-aware Adaptive Second-order Optimization
with Stable Hessian Approximation}

\usepackage{xspace}
\def\sassha{\textsc{Sassha}\xspace}
\def\msassha{\textsc{M-Sassha}\xspace}
\def\gsassha{\textsc{G-Sassha}\xspace}

\usepackage{wrapfig}
%\usepackage[compatibility=false]{caption}
\usepackage{colortbl}
\usepackage{xcolor}
\usepackage{float}
\usepackage{multirow}
\usepackage{subcaption}
\usepackage{framed}
\usepackage{hhline}
\usepackage[T1]{fontenc}

\makeatletter
\DeclareRobustCommand\onedot{\futurelet\@let@token\@onedot}
\def\@onedot{\ifx\@let@token.\else.\null\fi\xspace}
\def\eg{\emph{e.g}\onedot} \def\Eg{\emph{E.g}\onedot}
\def\ie{\emph{i.e}\onedot} \def\Ie{\emph{I.e}\onedot}
\def\cf{\emph{c.f}\onedot} \def\Cf{\emph{C.f}\onedot}
\def\etc{\emph{etc}\onedot} \def\vs{\emph{vs}\onedot}
\def\wrt{w.r.t\onedot} \def\dof{d.o.f\onedot}
\def\etal{\emph{et al}\onedot}
\makeatother

\definecolor{lgray}{rgb}{.906, .902, .902}
\definecolor{pred}{RGB/cmyk}{200,1,80/.1,1,.30,.10}

\def\wip{\textcolor{blue}{(Work in progress)}\xspace}
\def\needcite{\textcolor{red}{(cite)}\xspace}
\newcommand\needfill[1]{\fcolorbox{red}{white}{\makebox(\linewidth, 5cm)[c]{\textcolor{red}{(TBD)} #1}}}
\newcommand\changelater[1]{\textcolor{red}{(change later) #1}}

\newcommand\checkchange[1]{\textcolor{red}{\textbf{(Check changes)} #1}}

\usepackage{eqparbox,array}
% \renewcommand\algorithmiccomment[1]{%
%   \hfill\#\ \eqparbox{COMMENT}{#1}%
% }

\begin{document}

\twocolumn[
\icmltitle{\sassha: Sharpness-aware Adaptive Second-order Optimization\\with Stable Hessian Approximation}

% It is OKAY to include author information, even for blind
% submissions: the style file will automatically remove it for you
% unless you've provided the [accepted] option to the icml2025
% package.

% List of affiliations: The first argument should be a (short)
% identifier you will use later to specify author affiliations
% Academic affiliations should list Department, University, City, Region, Country
% Industry affiliations should list Company, City, Region, Country

% You can specify symbols, otherwise they are numbered in order.
% Ideally, you should not use this facility. Affiliations will be numbered
% in order of appearance and this is the preferred way.
\icmlsetsymbol{equal}{*}

\begin{icmlauthorlist}
\icmlauthor{Dahun Shin}{equal,yyy}
\icmlauthor{Dongyeop Lee}{equal,yyy}
\icmlauthor{Jinseok Chung}{yyy}
\icmlauthor{Namhoon Lee}{yyy}
% \icmlauthor{Firstname5 Lastname5}{yyy}
% \icmlauthor{Firstname6 Lastname6}{sch,yyy,comp}
% \icmlauthor{Firstname7 Lastname7}{comp}
%\icmlauthor{}{sch}
% \icmlauthor{Firstname8 Lastname8}{sch}
% \icmlauthor{Firstname8 Lastname8}{yyy,comp}
%\icmlauthor{}{sch}
%\icmlauthor{}{sch}
\end{icmlauthorlist}

\icmlaffiliation{yyy}{POSTECH}
% \icmlaffiliation{comp}{Company Name, Location, Country}
% \icmlaffiliation{sch}{School of ZZZ, Institute of WWW, Location, Country}

\icmlcorrespondingauthor{Dahun Shin}{dahunshin@postech.ac.kr}
\icmlcorrespondingauthor{Dongyeop Lee}{dylee23@postech.ac.kr}
%\icmlcorrespondingauthor{Dongyeop Lee}{dylee23@postech.ac.kr}

% You may provide any keywords that you
% find helpful for describing your paper; these are used to populate
% the "keywords" metadata in the PDF but will not be shown in the document
\icmlkeywords{Machine Learning, ICML}

\vskip 0.3in
]

% this must go after the closing bracket ] following \twocolumn[ ...

% This command actually creates the footnote in the first column
% listing the affiliations and the copyright notice.
% The command takes one argument, which is text to display at the start of the footnote.
% The \icmlEqualContribution command is standard text for equal contribution.
% Remove it (just {}) if you do not need this facility.

%\printAffiliationsAndNotice{}  % leave blank if no need to mention equal contribution
\printAffiliationsAndNotice{\icmlEqualContribution} % otherwise use the standard text.

\begin{abstract}


The choice of representation for geographic location significantly impacts the accuracy of models for a broad range of geospatial tasks, including fine-grained species classification, population density estimation, and biome classification. Recent works like SatCLIP and GeoCLIP learn such representations by contrastively aligning geolocation with co-located images. While these methods work exceptionally well, in this paper, we posit that the current training strategies fail to fully capture the important visual features. We provide an information theoretic perspective on why the resulting embeddings from these methods discard crucial visual information that is important for many downstream tasks. To solve this problem, we propose a novel retrieval-augmented strategy called RANGE. We build our method on the intuition that the visual features of a location can be estimated by combining the visual features from multiple similar-looking locations. We evaluate our method across a wide variety of tasks. Our results show that RANGE outperforms the existing state-of-the-art models with significant margins in most tasks. We show gains of up to 13.1\% on classification tasks and 0.145 $R^2$ on regression tasks. All our code and models will be made available at: \href{https://github.com/mvrl/RANGE}{https://github.com/mvrl/RANGE}.

\end{abstract}


\section{Introduction}
Backdoor attacks pose a concealed yet profound security risk to machine learning (ML) models, for which the adversaries can inject a stealth backdoor into the model during training, enabling them to illicitly control the model's output upon encountering predefined inputs. These attacks can even occur without the knowledge of developers or end-users, thereby undermining the trust in ML systems. As ML becomes more deeply embedded in critical sectors like finance, healthcare, and autonomous driving \citep{he2016deep, liu2020computing, tournier2019mrtrix3, adjabi2020past}, the potential damage from backdoor attacks grows, underscoring the emergency for developing robust defense mechanisms against backdoor attacks.

To address the threat of backdoor attacks, researchers have developed a variety of strategies \cite{liu2018fine,wu2021adversarial,wang2019neural,zeng2022adversarial,zhu2023neural,Zhu_2023_ICCV, wei2024shared,wei2024d3}, aimed at purifying backdoors within victim models. These methods are designed to integrate with current deployment workflows seamlessly and have demonstrated significant success in mitigating the effects of backdoor triggers \cite{wubackdoorbench, wu2023defenses, wu2024backdoorbench,dunnett2024countering}.  However, most state-of-the-art (SOTA) backdoor purification methods operate under the assumption that a small clean dataset, often referred to as \textbf{auxiliary dataset}, is available for purification. Such an assumption poses practical challenges, especially in scenarios where data is scarce. To tackle this challenge, efforts have been made to reduce the size of the required auxiliary dataset~\cite{chai2022oneshot,li2023reconstructive, Zhu_2023_ICCV} and even explore dataset-free purification techniques~\cite{zheng2022data,hong2023revisiting,lin2024fusing}. Although these approaches offer some improvements, recent evaluations \cite{dunnett2024countering, wu2024backdoorbench} continue to highlight the importance of sufficient auxiliary data for achieving robust defenses against backdoor attacks.

While significant progress has been made in reducing the size of auxiliary datasets, an equally critical yet underexplored question remains: \emph{how does the nature of the auxiliary dataset affect purification effectiveness?} In  real-world  applications, auxiliary datasets can vary widely, encompassing in-distribution data, synthetic data, or external data from different sources. Understanding how each type of auxiliary dataset influences the purification effectiveness is vital for selecting or constructing the most suitable auxiliary dataset and the corresponding technique. For instance, when multiple datasets are available, understanding how different datasets contribute to purification can guide defenders in selecting or crafting the most appropriate dataset. Conversely, when only limited auxiliary data is accessible, knowing which purification technique works best under those constraints is critical. Therefore, there is an urgent need for a thorough investigation into the impact of auxiliary datasets on purification effectiveness to guide defenders in  enhancing the security of ML systems. 

In this paper, we systematically investigate the critical role of auxiliary datasets in backdoor purification, aiming to bridge the gap between idealized and practical purification scenarios.  Specifically, we first construct a diverse set of auxiliary datasets to emulate real-world conditions, as summarized in Table~\ref{overall}. These datasets include in-distribution data, synthetic data, and external data from other sources. Through an evaluation of SOTA backdoor purification methods across these datasets, we uncover several critical insights: \textbf{1)} In-distribution datasets, particularly those carefully filtered from the original training data of the victim model, effectively preserve the model’s utility for its intended tasks but may fall short in eliminating backdoors. \textbf{2)} Incorporating OOD datasets can help the model forget backdoors but also bring the risk of forgetting critical learned knowledge, significantly degrading its overall performance. Building on these findings, we propose Guided Input Calibration (GIC), a novel technique that enhances backdoor purification by adaptively transforming auxiliary data to better align with the victim model’s learned representations. By leveraging the victim model itself to guide this transformation, GIC optimizes the purification process, striking a balance between preserving model utility and mitigating backdoor threats. Extensive experiments demonstrate that GIC significantly improves the effectiveness of backdoor purification across diverse auxiliary datasets, providing a practical and robust defense solution.

Our main contributions are threefold:
\textbf{1) Impact analysis of auxiliary datasets:} We take the \textbf{first step}  in systematically investigating how different types of auxiliary datasets influence backdoor purification effectiveness. Our findings provide novel insights and serve as a foundation for future research on optimizing dataset selection and construction for enhanced backdoor defense.
%
\textbf{2) Compilation and evaluation of diverse auxiliary datasets:}  We have compiled and rigorously evaluated a diverse set of auxiliary datasets using SOTA purification methods, making our datasets and code publicly available to facilitate and support future research on practical backdoor defense strategies.
%
\textbf{3) Introduction of GIC:} We introduce GIC, the \textbf{first} dedicated solution designed to align auxiliary datasets with the model’s learned representations, significantly enhancing backdoor mitigation across various dataset types. Our approach sets a new benchmark for practical and effective backdoor defense.



\section{Related Work}
\label{sec:related-works}
\subsection{Novel View Synthesis}
Novel view synthesis is a foundational task in the computer vision and graphics, which aims to generate unseen views of a scene from a given set of images.
% Many methods have been designed to solve this problem by posing it as 3D geometry based rendering, where point clouds~\cite{point_differentiable,point_nfs}, mesh~\cite{worldsheet,FVS,SVS}, planes~\cite{automatci_photo_pop_up,tour_into_the_picture} and multi-plane images~\cite{MINE,single_view_mpi,stereo_magnification}, \etal
Numerous methods have been developed to address this problem by approaching it as 3D geometry-based rendering, such as using meshes~\cite{worldsheet,FVS,SVS}, MPI~\cite{MINE,single_view_mpi,stereo_magnification}, point clouds~\cite{point_differentiable,point_nfs}, etc.
% planes~\cite{automatci_photo_pop_up,tour_into_the_picture}, 


\begin{figure*}[!t]
    \centering
    \includegraphics[width=1.0\linewidth]{figures/overview-v7.png}
    %\caption{\textbf{Overview.} Given a set of images, our method obtains both camera intrinsics and extrinsics, as well as a 3DGS model. First, we obtain the initial camera parameters, global track points from image correspondences and monodepth with reprojection loss. Then we incorporate the global track information and select Gaussian kernels associated with track points. We jointly optimize the parameters $K$, $T_{cw}$, 3DGS through multi-view geometric consistency $L_{t2d}$, $L_{t3d}$, $L_{scale}$ and photometric consistency $L_1$, $L_{D-SSIM}$.}
    \caption{\textbf{Overview.} Given a set of images, our method obtains both camera intrinsics and extrinsics, as well as a 3DGS model. During the initialization, we extract the global tracks, and initialize camera parameters and Gaussians from image correspondences and monodepth with reprojection loss. We determine Gaussian kernels with recovered 3D track points, and then jointly optimize the parameters $K$, $T_{cw}$, 3DGS through the proposed global track constraints (i.e., $L_{t2d}$, $L_{t3d}$, and $L_{scale}$) and original photometric losses (i.e., $L_1$ and $L_{D-SSIM}$).}
    \label{fig:overview}
\end{figure*}

Recently, Neural Radiance Fields (NeRF)~\cite{2020NeRF} provide a novel solution to this problem by representing scenes as implicit radiance fields using neural networks, achieving photo-realistic rendering quality. Although having some works in improving efficiency~\cite{instant_nerf2022, lin2022enerf}, the time-consuming training and rendering still limit its practicality.
Alternatively, 3D Gaussian Splatting (3DGS)~\cite{3DGS2023} models the scene as explicit Gaussian kernels, with differentiable splatting for rendering. Its improved real-time rendering performance, lower storage and efficiency, quickly attract more attentions.
% Different from NeRF-based methods which need MLPs to model the scene and huge computational cost for rendering, 3DGS has stronger real-time performance, higher storage and computational efficiency, benefits from its explicit representation and gradient backpropagation.

\subsection{Optimizing Camera Poses in NeRFs and 3DGS}
Although NeRF and 3DGS can provide impressive scene representation, these methods all need accurate camera parameters (both intrinsic and extrinsic) as additional inputs, which are mostly obtained by COLMAP~\cite{colmap2016}.
% This strong reliance on COLMAP significantly limits their use in real-world applications, so optimizing the camera parameters during the scene training becomes crucial.
When the prior is inaccurate or unknown, accurately estimating camera parameters and scene representations becomes crucial.

% In early works, only photometric constraints are used for scene training and camera pose estimation. 
% iNeRF~\cite{iNerf2021} optimizes the camera poses based on a pre-trained NeRF model.
% NeRFmm~\cite{wang2021nerfmm} introduce a joint optimization process, which estimates the camera poses and trains NeRF model jointly.
% BARF~\cite{barf2021} and GARF~\cite{2022GARF} provide new positional encoding strategy to handle with the gradient inconsistency issue of positional embedding and yield promising results.
% However, they achieve satisfactory optimization results when only the pose initialization is quite closed to the ground-truth, as the photometric constrains can only improve the quality of camera estimation within a small range.
% Later, more prior information of geometry and correspondence, \ie monocular depth and feature matching, are introduced into joint optimisation to enhance the capability of camera poses estimation.
% SC-NeRF~\cite{SCNeRF2021} minimizes a projected ray distance loss based on correspondence of adjacent frames.
% NoPe-NeRF~\cite{bian2022nopenerf} chooses monocular depth maps as geometric priors, and defines undistorted depth loss and relative pose constraints for joint optimization.
In earlier studies, scene training and camera pose estimation relied solely on photometric constraints. iNeRF~\cite{iNerf2021} refines the camera poses using a pre-trained NeRF model. NeRFmm~\cite{wang2021nerfmm} introduces a joint optimization approach that simultaneously estimates camera poses and trains the NeRF model. BARF~\cite{barf2021} and GARF~\cite{2022GARF} propose a new positional encoding strategy to address the gradient inconsistency issues in positional embedding, achieving promising results. However, these methods only yield satisfactory optimization when the initial pose is very close to the ground truth, as photometric constraints alone can only enhance camera estimation quality within a limited range. Subsequently, 
% additional prior information on geometry and correspondence, such as monocular depth and feature matching, has been incorporated into joint optimization to improve the accuracy of camera pose estimation. 
SC-NeRF~\cite{SCNeRF2021} minimizes a projected ray distance loss based on correspondence between adjacent frames. NoPe-NeRF~\cite{bian2022nopenerf} utilizes monocular depth maps as geometric priors and defines undistorted depth loss and relative pose constraints.

% With regard to 3D Gaussian Splatting, CF-3DGS~\cite{CF-3DGS-2024} also leverages mono-depth information to constrain the optimization of local 3DGS for relative pose estimation and later learn a global 3DGS progressively in a sequential manner.
% InstantSplat~\cite{fan2024instantsplat} focus on sparse view scenes, first use DUSt3R~\cite{dust3r2024cvpr} to generate a set of densely covered and pixel-aligned points for 3D Gaussian initialization, then introduce a parallel grid partitioning strategy in joint optimization to speed up.
% % Jiang et al.~\cite{Jiang_2024sig} proposed to build the scene continuously and progressively, to next unregistered frame, they use registration and adjustment to adjust the previous registered camera poses and align unregistered monocular depths, later refine the joint model by matching detected correspondences in screen-space coordinates.
% \gjh{Jiang et al.~\cite{Jiang_2024sig} also implemented an incremental approach for reconstructing camera poses and scenes. Initially, they perform feature matching between the current image and the image rendered by a differentiable surface renderer. They then construct matching point errors, depth errors, and photometric errors to achieve the registration and adjustment of the current image. Finally, based on the depth map, the pixels of the current image are projected as new 3D Gaussians. However, this method still exhibits limitations when dealing with complex scenes and unordered images.}
% % CG-3DGS~\cite{sun2024correspondenceguidedsfmfree3dgaussian} follows CF-3DGS, first construct a coarse point cloud from mono-depth maps to train a 3DGS model, then progressively estimate camera poses based on this pre-trained model by constraining the correspondences between rendering view and ground-truth.
% \gjh{Similarly, CG-3DGS~\cite{sun2024correspondenceguidedsfmfree3dgaussian} first utilizes monocular depth estimation and the camera parameters from the first frame to initialize a set of 3D Gaussians. It then progressively estimates camera poses based on this pre-trained model by constraining the correspondences between the rendered views and the ground truth.}
% % Free-SurGS~\cite{freesurgs2024} matches the projection flow derived from 3D Gaussians with optical flow to estimate the poses, to compensate for the limitations of photometric loss.
% \gjh{Free-SurGS~\cite{freesurgs2024} introduces the first SfM-free 3DGS approach for surgical scene reconstruction. Due to the challenges posed by weak textures and photometric inconsistencies in surgical scenes, Free-SurGS achieves pose estimation by minimizing the flow loss between the projection flow and the optical flow. Subsequently, it keeps the camera pose fixed and optimizes the scene representation by minimizing the photometric loss, depth loss and flow loss.}
% \gjh{However, most current works assume camera intrinsics are known and primarily focus on optimizing camera poses. Additionally, these methods typically rely on sequentially ordered image inputs and incrementally optimize camera parameters and scene representation. This inevitably leads to drift errors, preventing the achievement of globally consistent results. Our work aims to address these issues.}

Regarding 3D Gaussian Splatting, CF-3DGS~\cite{CF-3DGS-2024} utilizes mono-depth information to refine the optimization of local 3DGS for relative pose estimation and subsequently learns a global 3DGS in a sequential manner. InstantSplat~\cite{fan2024instantsplat} targets sparse view scenes, initially employing DUSt3R~\cite{dust3r2024cvpr} to create a densely covered, pixel-aligned point set for initializing 3D Gaussian models, and then implements a parallel grid partitioning strategy to accelerate joint optimization. Jiang \etal~\cite{Jiang_2024sig} develops an incremental method for reconstructing camera poses and scenes, but it struggles with complex scenes and unordered images. 
% Similarly, CG-3DGS~\cite{sun2024correspondenceguidedsfmfree3dgaussian} progressively estimates camera poses using a pre-trained model by aligning the correspondences between rendered views and actual scenes. Free-SurGS~\cite{freesurgs2024} pioneers an SfM-free 3DGS method for reconstructing surgical scenes, overcoming challenges such as weak textures and photometric inconsistencies by minimizing the discrepancy between projection flow and optical flow.
%\pb{SF-3DGS-HT~\cite{ji2024sfmfree3dgaussiansplatting} introduced VFI into training as additional photometric constraints. They separated the whole scene into several local 3DGS models and then merged them hierarchically, which leads to a significant improvement on simple and dense view scenes.}
HT-3DGS~\cite{ji2024sfmfree3dgaussiansplatting} interpolates frames for training and splits the scene into local clips, using a hierarchical strategy to build 3DGS model. It works well for simple scenes, but fails with dramatic motions due to unstable interpolation and low efficiency.
% {While effective for simple scenes, it struggles with dramatic motion due to unstable view interpolation and suffers from low computational efficiency.}

However, most existing methods generally depend on sequentially ordered image inputs and incrementally optimize camera parameters and 3DGS, which often leads to drift errors and hinders achieving globally consistent results. Our work seeks to overcome these limitations.

\section{Practical Second-order Optimizers Converge to Sharp Minima}

In this section, we investigate the sharpness of minima obtained by approximate second-order methods and their generalization properties.
We posit that poor generalization of second-order methods reported in the literature \citep{amari2021when, wadia2021whitening} can potentially be attributed to sharpness of their solutions.

We employ four metrics frequently used in the literature: maximum eigenvalue of the Hessian, the trace of Hessian, gradient-direction sharpness, and average sharpness \citep{Hochreiter1997, jastrzkebski2018relation, xie2020diffusion, saf, chenvision}.
The first two, denoted as $\lambda_{\max}(H)$ and $\operatorname{tr}(H)$, are often used as standard mathematical measures for the worst-case and the average curvature computed using the power iteration method and the Hutchinson trace estimation, respectively.
The other two measures, $\delta L_\text{grad}$ and $\delta L_\text{avg}$, assess sharpness based on the loss difference under perturbations.
$\delta L_\text{grad}$ evaluates sharpness in the gradient direction and is computed as $L(x^\star+\rho\nabla L(x^\star)/\|\nabla L(x^\star)\|)-L(x^\star)$.
$\delta L_\text{avg}$ computes the average loss difference over Gaussian random perturbations, expressed as
$\E_{z\sim \mathcal{N}(0, 1)}[L(x^\star+\rho z/\|z\|)-L(x^\star)]$.
Here we choose $\rho=0.1$ for the scale of the perturbation.

With these, we measure the sharpness of the minima found by three approximate second-order methods designed for deep learning; Sophia-H \citep{sophia}, AdaHessian \citep{adahessian}, and Shampoo \citep{gupta2018shampoo}, and compare them with \sassha as well as SGD for reference.
We also compute the validation loss and accuracy to see any correlation between sharpness and generalization of these solutions.
The results are presented in \cref{tab:sharp}.

We observe that existing second-order optimizers produce solutions with significantly higher sharpness compared to \sassha in all sharpness metrics, which also correlates well with their generalization.
We also provide a visualization of the loss landscape for the found solutions, where we find that the solutions obtained by second-order methods are indeed much sharper than that of \sassha (\cref{fig:landscape}).

Effective human-robot cooperation in CoNav-Maze hinges on efficient communication. Maximizing the human’s information gain enables more precise guidance, which in turn accelerates task completion. Yet for the robot, the challenge is not only \emph{what} to communicate but also \emph{when}, as it must balance gathering information for the human with pursuing immediate goals when confident in its navigation.

To achieve this, we introduce \emph{Information Gain Monte Carlo Tree Search} (IG-MCTS), which optimizes both task-relevant objectives and the transmission of the most informative communication. IG-MCTS comprises three key components:
\textbf{(1)} A data-driven human perception model that tracks how implicit (movement) and explicit (image) information updates the human’s understanding of the maze layout.
\textbf{(2)} Reward augmentation to integrate multiple objectives effectively leveraging on the learned perception model.
\textbf{(3)} An uncertainty-aware MCTS that accounts for unobserved maze regions and human perception stochasticity.
% \begin{enumerate}[leftmargin=*]
%     \item A data-driven human perception model that tracks how implicit (movement) and explicit (image transmission) information updates the human’s understanding of the maze layout.
%     \item Reward augmentation to integrate multiple objectives effectively leveraging on the learned perception model.
%     \item An uncertainty-aware MCTS that accounts for unobserved maze regions and human perception stochasticity.
% \end{enumerate}

\subsection{Human Perception Dynamics}
% IG-MCTS seeks to optimize the expected novel information gained by the human through the robot’s actions, including both movement and communication. Achieving this requires a model of how the human acquires task-relevant information from the robot.

% \subsubsection{Perception MDP}
\label{sec:perception_mdp}
As the robot navigates the maze and transmits images, humans update their understanding of the environment. Based on the robot's path, they may infer that previously assumed blocked locations are traversable or detect discrepancies between the transmitted image and their map.  

To formally capture this process, we model the evolution of human perception as another Markov Decision Process, referred to as the \emph{Perception MDP}. The state space $\mathcal{X}$ represents all possible maze maps. The action space $\mathcal{S}^+ \times \mathcal{O}$ consists of the robot's trajectory between two image transmissions $\tau \in \mathcal{S}^+$ and an image $o \in \mathcal{O}$. The unknown transition function $F: (x, (\tau, o)) \rightarrow x'$ defines the human perception dynamics, which we aim to learn.

\subsubsection{Crowd-Sourced Transition Dataset}
To collect data, we designed a mapping task in the CoNav-Maze environment. Participants were tasked to edit their maps to match the true environment. A button triggers the robot's autonomous movements, after which it captures an image from a random angle.
In this mapping task, the robot, aware of both the true environment and the human’s map, visits predefined target locations and prioritizes areas with mislabeled grid cells on the human’s map.
% We assume that the robot has full knowledge of both the actual environment and the human’s current map. Leveraging this knowledge, the robot autonomously navigates to all predefined target locations. It then randomly selects subsequent goals to reach, prioritizing grid locations that remain mislabeled on the human’s map. This ensures that the robot’s actions are strategically focused on providing useful information to improve map accuracy.

We then recruited over $50$ annotators through Prolific~\cite{palan2018prolific} for the mapping task. Each annotator labeled three randomly generated mazes. They were allowed to proceed to the next maze once the robot had reached all four goal locations. However, they could spend additional time refining their map before moving on. To incentivize accuracy, annotators receive a performance-based bonus based on the final accuracy of their annotated map.


\subsubsection{Fully-Convolutional Dynamics Model}
\label{sec:nhpm}

We propose a Neural Human Perception Model (NHPM), a fully convolutional neural network (FCNN), to predict the human perception transition probabilities modeled in \Cref{sec:perception_mdp}. We denote the model as $F_\theta$ where $\theta$ represents the trainable weights. Such design echoes recent studies of model-based reinforcement learning~\cite{hansen2022temporal}, where the agent first learns the environment dynamics, potentially from image observations~\cite{hafner2019learning,watter2015embed}.

\begin{figure}[t]
    \centering
    \includegraphics[width=0.9\linewidth]{figures/ICML_25_CNN.pdf}
    \caption{Neural Human Perception Model (NHPM). \textbf{Left:} The human's current perception, the robot's trajectory since the last transmission, and the captured environment grids are individually processed into 2D masks. \textbf{Right:} A fully convolutional neural network predicts two masks: one for the probability of the human adding a wall to their map and another for removing a wall.}
    \label{fig:nhpm}
    \vskip -0.1in
\end{figure}

As illustrated in \Cref{fig:nhpm}, our model takes as input the human’s current perception, the robot’s path, and the image captured by the robot, all of which are transformed into a unified 2D representation. These inputs are concatenated along the channel dimension and fed into the CNN, which outputs a two-channel image: one predicting the probability of human adding a new wall and the other predicting the probability of removing a wall.

% Our approach builds on world model learning, where neural networks predict state transitions or environmental updates based on agent actions and observations. By leveraging the local feature extraction capabilities of CNNs, our model effectively captures spatial relationships and interprets local changes within the grid maze environment. Similar to prior work in localization and mapping, the CNN architecture is well-suited for processing spatially structured data and aligning the robot’s observations with human map updates.

To enhance robustness and generalization, we apply data augmentation techniques, including random rotation and flipping of the 2D inputs during training. These transformations are particularly beneficial in the grid maze environment, which is invariant to orientation changes.

\subsection{Perception-Aware Reward Augmentation}
The robot optimizes its actions over a planning horizon \( H \) by solving the following optimization problem:
\begin{subequations}
    \begin{align}
        \max_{a_{0:H-1}} \;
        & \mathop{\mathbb{E}}_{T, F} \left[ \sum_{t=0}^{H-1} \gamma^t \left(\underbrace{R_{\mathrm{task}}(\tau_{t+1}, \zeta)}_{\text{(1) Task reward}} + \underbrace{\|x_{t+1}-x_t\|_1}_{\text{(2) Info reward}}\right)\right] \label{obj}\\ 
        \subjectto \quad
        &x_{t+1} = F(x_t, (\tau_t, a_t)), \quad a_t\in\Ocal \label{const:perception_update}\\ 
        &\tau_{t+1} = \tau_t \oplus T(s_t, a_t), \quad a_t\in \Ucal\label{const:history_update}
    \end{align}
\end{subequations} 

The objective in~\eqref{obj} maximizes the expected cumulative reward over \( T \) and \( F \), reflecting the uncertainty in both physical transitions and human perception dynamics. The reward function consists of two components: 
(1) The \emph{task reward} incentivizes efficient navigation. The specific formulation for the task in this work is outlined in \Cref{appendix:task_reward}.
(2) The \emph{information reward} quantifies the change in the human’s perception due to robot actions, computed as the \( L_1 \)-norm distance between consecutive perception states.  

The constraint in~\eqref{const:history_update} ensures that for movement actions, the trajectory history \( \tau_t \) expands with new states based on the robot’s chosen actions, where \( s_t \) is the most recent state in \( \tau_t \), and \( \oplus \) represents sequence concatenation. 
In constraint~\eqref{const:perception_update}, the robot leverages the learned human perception dynamics \( F \) to estimate the evolution of the human’s understanding of the environment from perception state $x_t$ to $x_{t+1}$ based on the observed trajectory \( \tau_t \) and transmitted image \( a_t\in\Ocal \). 
% justify from a cognitive science perspective
% Cognitive science research has shown that humans read in a way to maximize the information gained from each word, aligning with the efficient coding principle, which prioritizes minimizing perceptual errors and extracting relevant features under limited processing capacity~\cite{kangassalo2020information}. Drawing on this principle, we hypothesize that humans similarly prioritize task-relevant information in multimodal settings. To accommodate this cognitive pattern, our robot policy selects and communicates high information-gain observations to human operators, akin to summarizing key insights from a lengthy article.
% % While the brain naturally seeks to gain information, the brain employs various strategies to manage information overload, including filtering~\cite{quiroga2004reducing}, limiting/working memory, and prioritizing information~\cite{arnold2023dealing}.
% In this context of our setup, we optimize the selection of camera angles to maximize the human operator's information gain about the environment. 

\subsection{Information Gain Monte Carlo Tree Search (IG-MCTS)}
IG-MCTS follows the four stages of Monte Carlo tree search: \emph{selection}, \emph{expansion}, \emph{rollout}, and \emph{backpropagation}, but extends it by incorporating uncertainty in both environment dynamics and human perception. We introduce uncertainty-aware simulations in the \emph{expansion} and \emph{rollout} phases and adjust \emph{backpropagation} with a value update rule that accounts for transition feasibility.

\subsubsection{Uncertainty-Aware Simulation}
As detailed in \Cref{algo:IG_MCTS}, both the \emph{expansion} and \emph{rollout} phases involve forward simulation of robot actions. Each tree node $v$ contains the state $(\tau, x)$, representing the robot's state history and current human perception. We handle the two action types differently as follows:
\begin{itemize}
    \item A movement action $u$ follows the environment dynamics $T$ as defined in \Cref{sec:problem}. Notably, the maze layout is observable up to distance $r$ from the robot's visited grids, while unexplored areas assume a $50\%$ chance of walls. In \emph{expansion}, the resulting search node $v'$ of this uncertain transition is assigned a feasibility value $\delta = 0.5$. In \emph{rollout}, the transition could fail and the robot remains in the same grid.
    
    \item The state transition for a communication step $o$ is governed by the learned stochastic human perception model $F_\theta$ as defined in \Cref{sec:nhpm}. Since transition probabilities are known, we compute the expected information reward $\bar{R_\mathrm{info}}$ directly:
    \begin{align*}
        \bar{R_\mathrm{info}}(\tau_t, x_t, o_t) &= \mathbb{E}_{x_{t+1}}\|x_{t+1}-x_t\|_1 \\
        &= \|p_\mathrm{add}\|_1 + \|p_\mathrm{remove}\|_1,
    \end{align*}
    where $(p_\mathrm{add}, p_\mathrm{remove}) \gets F_\theta(\tau_t, x_t, o_t)$ are the estimated probabilities of adding or removing walls from the map. 
    Directly computing the expected return at a node avoids the high number of visitations required to obtain an accurate value estimate.
\end{itemize}

% We denote a node in the search tree as $v$, where $s(v)$, $r(v)$, and $\delta(v)$ represent the state, reward, and transition feasibility at $v$, respectively. The visit count of $v$ is denoted as $N(v)$, while $Q(v)$ represents its total accumulated return. The set of child nodes of $v$ is denoted by $\mathbb{C}(v)$.

% The goal of each search is to plan a sequence for the robot until it reaches a goal or transmits a new image to the human. We initialize the search tree with the current human guidance $\zeta$, and the robot's approximation of human perception $x_0$. Each search node consists consists of the state information required by our reward augmentation: $(\tau, x)$. A node is terminal if it is the resulting state of a communication step, or if the robot reaches a goal location. 

% A rollout from the expanded node simulates future transitions until reaching a terminal state or a predefined depth $H$. Actions are selected randomly from the available action set $\mathcal{A}(s)$. If an action's feasibility is uncertain due to the environment's unknown structure, the transition occurs with probability $\delta(s, a)$. When a random number draw deems the transition infeasible, the state remains unchanged. On the other hand, for communication steps, we don't resolve the uncertainty but instead compute the expected information gain reward: \philip{TODO: adjust notation}
% \begin{equation}
%     \mathbb{E}\left[R_\mathrm{info}(\tau, x')\right] = \sum \mathrm{NPM(\tau, o)}.
% \end{equation}

\subsubsection{Feasibility-Adjusted Backpropagation}
During backpropagation, the rewards obtained from the simulation phase are propagated back through the tree, updating the total value $Q(v)$ and the visitation count $N(v)$ for all nodes along the path to the root. Due to uncertainty in unexplored environment dynamics, the rollout return depends on the feasibility of the transition from the child node. Given a sample return \(q'_{\mathrm{sample}}\) at child node \(v'\), the parent node's return is:
\begin{equation}
    q_{\mathrm{sample}} = r + \gamma \left[ \delta' q'_{\mathrm{sample}} + (1 - \delta') \frac{Q(v)}{N(v)} \right],
\end{equation}
where $\delta'$ represents the probability of a successful transition. The term \((1 - \delta')\) accounts for failed transitions, relying instead on the current value estimate.

% By incorporating uncertainty-aware rollouts and backpropagation, our approach enables more robust decision-making in scenarios where the environment dynamics is unknown and avoids simulation of the stochastic human perception dynamics.

% \input{text/5_convergence}
\section{Evaluations}
\label{sec:experiment}

In this section, we demonstrate that \sassha can indeed improve upon existing second-order methods available for standard deep learning tasks.
We also show that \sassha performs competitively to the first-order baseline methods.
Specifically, \sassha is compared to AdaHessian \citep{adahessian}, Sophia-H \citep{sophia}, Shampoo \cite{gupta2018shampoo}, SGD, AdamW \citep{loshchilov2018decoupled}, and SAM \citep{sam} on a diverse set of both vision and language tasks.
We emphasize that we perform an \emph{extensive} hyperparameter search to rigorously tune all optimizers and ensure fair comparisons.
We provide the details of experiment settings to reproduce our results in \cref{app:hypersearch}.
The code to reproduce all results reported in this work is made available for download at \url{https://github.com/LOG-postech/Sassha}.

\subsection{Image Classification}
\begin{table*}[t!]
    \vspace{-0.5em}
    \centering
    \caption{Image classification results of various optimization methods in terms of final validation accuracy (mean$\pm$std).
    \sassha consistently outperforms the other methods for all workloads.
    * means \emph{omitted} due to excessive computational requirements.}
    
    \vskip 0.1in
    \resizebox{0.8\linewidth}{!}{
        \begin{tabular}{clcccccc}
        \toprule
         & 
         & \multicolumn{2}{c}{CIFAR-10} 
         & \multicolumn{2}{c}{CIFAR-100} 
         & \multicolumn{2}{c}{ImageNet} \\
         \cmidrule(l{3pt}r{3pt}){3-4} \cmidrule(l{3pt}r{3pt}){5-6} \cmidrule(l{3pt}r{3pt}){7-8}
         \multicolumn{1}{c}{ Category }
         & \multicolumn{1}{c}{ Method }
         & \multicolumn{1}{c}{ ResNet-20 } 
         & \multicolumn{1}{c}{ ResNet-32 } 
         & \multicolumn{1}{c}{ ResNet-32 }  
         & \multicolumn{1}{c}{ WRN-28-10} 
         & \multicolumn{1}{c}{ ResNet-50 } 
         & \multicolumn{1}{c}{ ViT-s-32} \\ \midrule

        
       \multirow{4}{*}{First-order}  
       & SGD       & 
         $ 92.03 _{ \textcolor{black!60}{\pm 0.32} } $    &
         $ 92.69 _{\textcolor{black!60}{\pm 0.06} }  $    &
         $ 69.32 _{\textcolor{black!60}{\pm 0.19} }  $    &
         $ 80.06 _{\textcolor{black!60}{\pm 0.15} }  $    &
         $ 75.58 _{\textcolor{black!60}{\pm 0.05} }  $    &
         $ 62.90 _{\textcolor{black!60}{\pm 0.36} }  $   \\

        & AdamW      & 
        $ 92.04 _{\textcolor{black!60}{\pm 0.11} }  $     &
        $ 92.42 _{\textcolor{black!60}{\pm 0.13} }  $     &
        $ 68.78 _{\textcolor{black!60}{\pm 0.22} }  $     &
        $ 79.09 _{\textcolor{black!60}{\pm 0.35} }  $     &
        $ 75.38 _{\textcolor{black!60}{\pm 0.08} }  $     &
        $ 66.46 _{\textcolor{black!60}{\pm 0.15} }  $    \\
        
        & SAM $_{\text{SGD}}$  &
        $ 92.85 _{\textcolor{black!60}{\pm 0.07} }  $    &
        $ 93.89 _{\textcolor{black!60}{\pm 0.13} }  $    &
        $ 71.99 _{\textcolor{black!60}{\pm 0.20} }  $    &
        $ 83.14 _{\textcolor{black!60}{\pm 0.13} }  $    &
        $ 76.36 _{\textcolor{black!60}{\pm 0.16} }  $    &
        $ 64.54 _{\textcolor{black!60}{\pm 0.63} }  $    \\
        
        & SAM $_{\text{AdamW}}$  &
        $ 92.77 _{\textcolor{black!60}{\pm 0.29} }  $    &
        $ 93.45 _{\textcolor{black!60}{\pm 0.24} }  $    &
        $ 71.15 _{\textcolor{black!60}{\pm 0.37} }  $    &
        $ 82.88 _{\textcolor{black!60}{\pm 0.31} }  $    &
        $ 76.35 _{\textcolor{black!60}{\pm 0.16} }  $    &
        $ 68.31 _{\textcolor{black!60}{\pm 0.17} }  $    \\

        \midrule
        
        \multirow{4}{*}{Second-order} &
        AdaHessian &
        $ 92.00 _{\textcolor{black!60}{\pm 0.17} } $  &
        $ 92.48 _{\textcolor{black!60}{\pm 0.15} } $  &
        $ 68.06 _{\textcolor{black!60}{\pm 0.22} } $  &
        $ 76.92 _{\textcolor{black!60}{\pm 0.26} } $  &
        $ 73.64 _{\textcolor{black!60}{\pm 0.16} } $  &
        $ 66.42 _{\textcolor{black!60}{\pm 0.23} } $  \\
        
        & Sophia-H   & 
        $ 91.81 _{\textcolor{black!60}{\pm 0.27} } $  &
        $ 91.99 _{\textcolor{black!60}{\pm 0.08} } $  &
        $ 67.76 _{\textcolor{black!60}{\pm 0.37} } $  & 
        $ 79.35 _{\textcolor{black!60}{\pm 0.24} } $  & 
        $ 72.06 _{\textcolor{black!60}{\pm 0.49} } $  &
        $ 62.44 _{\textcolor{black!60}{\pm 0.36} } $  \\
        
        & Shampoo    & 
        $ 88.55 _ {\textcolor{black!60}{\pm 0.83}}$  &
        $ 90.23 _{\textcolor{black!60}{\pm 0.24}} $  &
        $ 64.08 _{\textcolor{black!60}{\pm 0.46}} $  &
        $ 74.06 _{\textcolor{black!60}{\pm 1.28}} $  &
        $*$                                          &
        $*$  \\
        
        \cmidrule(l{3pt}r{3pt}){2-8}
        
        \rowcolor{green!20} &
        \sassha    &
        $ \textbf{92.98} _{\textcolor{black!60}{\pm 0.05} }  $ &
        $ \textbf{94.09} _{\textcolor{black!60}{\pm 0.24} }  $ &
        $ \textbf{72.14} _{\textcolor{black!60}{\pm 0.16} }  $ & 
        $ \textbf{83.54} _{\textcolor{black!60}{\pm 0.08} }  $ &
        $ \textbf{76.43} _{\textcolor{black!60}{\pm 0.18} }  $ &
        $ \textbf{69.20} _{\textcolor{black!60}{\pm 0.30} }  $ \\
        
        \bottomrule
        \end{tabular}
    }
    \vskip 0.1in
    \label{tab:im_cls_results}
\end{table*}

\begin{figure*}[t!]
    \vspace{-0.5em}
    \centering
    \resizebox{0.8\linewidth}{!}{
    \includegraphics[width=0.325\linewidth]{figures/validation/Res32-CIFAR10-Acc.pdf}
    \includegraphics[width=0.325\linewidth]{figures/validation/WRN28-CIFAR100-Acc.pdf}
    \includegraphics[width=0.325\linewidth]{figures/validation/Res50-ImageNet-Acc.pdf}
    }
    \vspace{-0.5em}
    \caption{
    Validation accuracy curves along the training trajectory.
    We also provide loss curves in \cref{app:valloss}.
    }
    \label{fig:im_cls_results}
    \vspace{-0.7em}
\end{figure*}

\begin{table*}[ht!]
    \centering
    \caption{
    Language finetuning and pertraining results for various optimizers. For finetuning, \sassha achieves better results than AdamW and AdaHessian and compares competitively with Sophia-H. For pretraining, \sassha achieves the lowest perplexity among all optimizers.
    }
    \vskip 0.1in
    \resizebox{\linewidth}{!}{
        \begin{tabular}{lc}
            \toprule
             & \multicolumn{1}{c}{$\textbf{Pretrain} / $ GPT1-mini} \\
             \cmidrule(l{3pt}r{3pt}){2-2}
             & Wikitext-2 \\
             & \texttt{Perplexity}\\
            \midrule
            
            AdamW & $ 175.06 $ \\
            SAM $_{\text{AdamW}}$ & $ 158.06 $ \\
            AdaHessian & $ 407.69 $ \\
            Sophia-H & $ 157.60 $ \\
            
            \midrule 
            
            \rowcolor{green!20}
            \sassha &
            $ \textbf{122.40} $ \\
            
            \bottomrule
        \end{tabular}
        
        \begin{tabular}{|ccccccc}
            \toprule
                         \multicolumn{7}{|c}{ \textbf{Finetune} /  SqeezeBERT } \\
                         \cmidrule(l{3pt}r{3pt}){1-7}
                         SST-2 &  MRPC & STS-B & QQP & MNLI & QNLI & RTE \\
             \texttt{Acc} &  \texttt{Acc / F1}  & \texttt{S/P corr.} & \texttt{F1 / Acc} & \texttt{mat/m.mat} &  \texttt{Acc} &  \texttt{Acc} \\
            \midrule
            
            %AdamW         & 
            $ 90.29 _{\textcolor{black!60}{\pm 0.52}} $ 
            & $ 84.56 _{ \textcolor{black!60}{\pm 0.25} } $ / $ 88.99 _{\textcolor{black!60}{\pm 0.11}} $ 
            & $ 88.34 _{\textcolor{black!60}{\pm 0.15}} $ / $ 88.48 _{\textcolor{black!60}{\pm 0.20}} $ 
            & $ 89.92 _{\textcolor{black!60}{\pm 0.05}} $ / $ 86.58 _{\textcolor{black!60}{\pm 0.11}} $ 
            & $ 81.22 _{\textcolor{black!60}{\pm 0.07}} $ / $ 82.26 _{\textcolor{black!60}{\pm 0.05}} $ 
            & $ 89.93 _{\textcolor{black!60}{\pm 0.14}} $ 
            & $ 68.95 _{\textcolor{black!60}{\pm 0.72}} $  \\
    
            %SAM _{\text{AdamW}}   &
            $ \textbf{90.52} _{\textcolor{black!60}{\pm 0.27}} $ 
            & $ 83.25 _{\textcolor{black!60}{\pm 2.79}} $ / $ 87.90 _{\textcolor{black!60}{\pm 2.21}} $ 
            & $ 88.38 _{\textcolor{black!60}{\pm 0.01}} $ / $ 88.79 _{\textcolor{black!60}{\pm 0.99}} $ 
            & $ 90.26 _{\textcolor{black!60}{\pm 0.28}} $ / $ 86.99 _{\textcolor{black!60}{\pm 0.31}} $ 
            & $ 81.56 _{\textcolor{black!60}{\pm 0.18}} $ / $ \textbf{82.46} _{\textcolor{black!60}{\pm 0.19}} $ 
            & $ \textbf{90.38} _{\textcolor{black!60}{\pm 0.05}} $ 
            & $ 68.83 _{\textcolor{black!60}{\pm 1.46}} $  \\
    
            %AdaHessian    & 
            $ 89.64 _{\textcolor{black!60}{\pm 0.13}} $ 
            & $ 79.74 _{\textcolor{black!60}{\pm 4.00}} $ / $ 85.26 _{\textcolor{black!60}{\pm 3.50}} $ 
            & $ 86.08 _{\textcolor{black!60}{\pm 4.04}} $ / $ 86.46 _{\textcolor{black!60}{\pm 4.06}} $ 
            & $ 90.37 _{\textcolor{black!60}{\pm 0.05}} $ / $ 87.07 _{\textcolor{black!60}{\pm 0.05}} $ 
            & $ 81.33 _{\textcolor{black!60}{\pm 0.17}} $ / $ 82.08 _{\textcolor{black!60}{\pm 0.02}} $ 
            & $ 89.94 _{\textcolor{black!60}{\pm 0.12}} $ 
            & $ 71.00 _{\textcolor{black!60}{\pm 1.04}} $ \\
            
            % Sophia-H  &
            $ 90.44 _{\textcolor{black!60}{\pm 0.46}} $ 
            & $ 85.78 _{\textcolor{black!60}{\pm 1.07}} $ / $ 89.90 _{\textcolor{black!60}{\pm 0.82}} $ 
            & $ 88.17 _{\textcolor{black!60}{\pm 1.07}} $ / $ 88.53 _{\textcolor{black!60}{\pm 1.13}} $ 
            & $ 90.70 _{\textcolor{black!60}{\pm 0.04}} $ / $ 87.60 _{\textcolor{black!60}{\pm 0.06}} $ 
            & $ \textbf{81.77} _{\textcolor{black!60}{\pm 0.18}} $ / $ 82.36 _{\textcolor{black!60}{\pm 0.22}} $ 
            & $ 90.12_{\textcolor{black!60}{\pm 0.14}} $ 
            & $ 70.76 _{\textcolor{black!60}{\pm 1.44}} $  \\
            
            \midrule
            
            \rowcolor{green!20} 
            $ 90.44 _{\textcolor{black!60}{\pm 0.98}} $    &
            $ \textbf{86.28} _{\textcolor{black!60}{\pm 0.28}} $ / $ \textbf{90.13} _{\textcolor{black!60}{\pm 0.161}} $     &
            $ \textbf{88.72} _{\textcolor{black!60}{\pm 0.75}} $ / $ \textbf{89.10} _{\textcolor{black!60}{\pm 0.70}}  $     &
            $ \textbf{90.91} _{\textcolor{black!60}{\pm 0.06}} $ / $ \textbf{87.85}  _{\textcolor{black!60}{\pm 0.09}} $     &
            $ 81.61 _{\textcolor{black!60}{\pm 0.25}} $ / $ 81.71 _{\textcolor{black!60}{\pm 0.11}} $     &
            $ 89.85_{\textcolor{black!60}{\pm 0.20}} $    &
            $ \textbf{72.08} _{\textcolor{black!60}{\pm 0.55}} $  \\
            
            \bottomrule
        \end{tabular}
    }
    \vspace{-0.5em}
    \label{tab:language}
\end{table*}

We first evaluate \sassha for image classification on CIFAR-10, CIFAR-100, and ImageNet.
We train various models of the ResNet family \citep{he2016deep,zagoruyko2016wide} and an efficient variant of Vision Transformer \citep{beyer2022better}.
We adhere to standard inception-style data augmentations during training instead of making use of advanced data augmentation techniques \citep{devries2017improved} or regularization methods \citep{gastaldi2017shake}.
Results are presented in \cref{tab:im_cls_results} and \cref{fig:im_cls_results}.

We begin by comparing the generalization performance of adaptive second-order methods to that of first-order methods.
Across all settings, adaptive second-order methods consistently exhibit lower accuracy than their first-order counterparts.
This observation aligns with previous studies indicating that second-order optimization often result in poorer generalization compared to first-order approaches.
In contrast, \sassha, benefiting from sharpness minimization, consistently demonstrates superior generalization performance, outperforming both first-order and second-order methods in every setting.
Particularly, \sassha is up to 4\% more effective than the best-performing adaptive or second-order methods (\eg, WRN-28-10, ViT-s-32).
Moreover, \sassha continually surpasses SGD and AdamW, even when they are trained for twice as many epochs, achieving a performance margin of about 0.3\% to 3\%. 
Further details are provided in \cref{app:comp_fo_fair}.

Interestingly, \sassha also outperforms SAM.
Since first-order methods typically exhibit superior generalization performance compared to second-order methods, it might be intuitive to expect SAM to surpass \sassha if the two are viewed merely as the outcomes of applying sharpness minimization to first-order and second-order methods, respectively.
However, the results conflict with this intuition.
We attribute this to the careful design choices made in \sassha, stabilizing Hessian approximation under sharpness minimization, so as to unleash the potential of the second-order method, leading to its outstanding performance.
As a support, we show that naively incorporating SAM into other second-order methods does not yield these favorable results in \cref{app:samsophia}.
We also make more comparisons with SAM in \cref{sec:sassha_vs_sam}.

\subsection{Language Modeling}

Recent studies have shown the potential of second-order methods for pretraining language models.
Here, we first evaluate how \sassha performs on this task.
Specifically, we train GPT1-mini, a scaled-down variant of GPT1 \citep{radford2019language}, on Wikitext-2 dataset \citep{merity2022pointer} using various methods including \sassha and compare their results (see the left of \cref{tab:language}).
Our results show that \sassha achieves the lowest perplexity among all methods including Sophia-H \citep{sophia}, a recent method that is designed specifically for language modeling tasks and sets state of the art, which highlights generality in addition to the numerical advantage of \sassha.

We also extend our evaluation to finetuning tasks.
Specifically, we finetune SqueezeBERT \citep{iandola2020squeezebert} for diverse tasks in the GLUE benchmark \citep{wang2018glue}.
The results are on the right side of \cref{tab:language}.
It shows that \sassha compares competitively to other second-order methods.
Notably, it also outperforms AdamW---often the method of choice for training language models---on nearly all tasks.

\subsection{Comparison to SAM}\label{sec:sassha_vs_sam}

So far, we have seen that \sassha outperforms second-order methods quite consistently on both vision and language tasks.
Interestingly, we also find that \sassha often improves upon SAM.
In particular, it appears that the gain is larger for the Transformer-based architectures, \ie, ViT results in \cref{tab:im_cls_results} or GPT/BERT results in \cref{tab:language}.

We posit that this is potentially due to the robustness of \sassha to the block heterogeneity inherent in Transformer architectures, where the Hessian spectrum varies significantly across different blocks.
This characteristic is known to make SGD perform worse than adaptive methods like Adam on Transformer-based models \citep{zhang2024why}.
Since \sassha leverages second-order information via preconditioning gradients, it has the potential to address the ill-conditioned nature of Transformers more effectively than SAM with first-order methods.

To push further, we conducted additional experiments.
First, we allocate more training budgets to SAM to see whether it compares to \sassha.
% additionally compare \sassha to SAM with more training budgets.
The results are presented in \cref{tab:sam}.
We find that SAM still underperforms \sassha, even though it is given more budgets of training iterations over data or wall-clock time.
Furthermore, we also compare \sassha to more advanced variants of SAM including ASAM \citep{asam} and GSAM \citep{gsam}, showing that \sassha performs competitively even to these methods (\cref{app:samvariants_vs_sassha}).
Notably, however, these variants of SAM require a lot more hyperparameter tuning to be compared.


\section{Further Analysis}

\label{sec:ablation}

\begin{table}[t!]
    \centering
    %\vspace{-0.5em}
    \bgroup
    \def\arraystretch{1.2}
    \caption{
    Comparison between \sassha and SAM with more training budgets for the ViT-s-32 / ImageNet workload.
    }
    \label{tab:sam}
    \vskip 0.1in
    \resizebox{0.7\linewidth}{!}{
        \centering
        \begin{tabular}{lccc}  % 5 more
        \toprule
        &\multicolumn{1}{c}{ Epoch } & Time (\texttt{s}) &\multicolumn{1}{c}{ Accuracy (\%) } \\ \midrule
        SAM$_\text{ SGD}$      &  180  & 220,852 & $  65.403 _{\textcolor{black!60}{\pm 0.63}}$ \\
        SAM$_\text{ AdamW}$    &  180 & 234,374 & $68.706 _{\textcolor{black!60}{\pm 0.16}}$ \\
        \midrule
        \rowcolor{green!20}\sassha        &  \textbf{90}  & \textbf{123,948} &  $\textbf{69.195} _{\textcolor{black!60}{\pm 0.30} } $  \\
        \bottomrule
        \end{tabular}
    }
    \egroup
    \vspace{-1.5em}
\end{table}

\subsection{Robustness}
\label{sec:robustness}

Noisily labeled training data can critically degrade generalization performance \citep{natarajan2013learning}.
To evaluate how \sassha generalizes under these practical conditions, we randomly corrupt certain fractions of the training data and compare the validation performances between different methods.
The results show that \sassha outperforms other methods across all noise levels with minimal accuracy degradation (\cref{tab:noise_label}).
Additionally, we also observe the same trend on CIFAR-10 (\cref{tab:noise_label_sassha}).

Interestingly, \sassha surpasses SAM \citep{sam}, which is known to be one of the most robust techniques against label noise \citep{baek2024why}. 
We hypothesize that its robustness stems from the complementary benefits of the sharpness-minimization scheme and second-order methods.
Specifically, SAM enhances robustness by adversarially perturbing the parameters and giving more importance to clean data during optimization, making the model more resistant to label noise \citep{sam, baek2024why}.
Also, recent research indicates that second-order methods are robust to label noise due to preconditioning that reduces the variance in the population risk \citep{amari2021when}.

\begin{table}[t!]
    \centering
    %\vspace{-0.5em}
    \caption{
    Validation accuracy measured for ResNet-32/CIFAR-100 at different levels of noise.
    \sassha shows the best robustness.
    }
    \label{tab:noise_label}
    \vskip 0.1in
    \resizebox{0.92\linewidth}{!} & {20\%} & {40\%} & {60\%} \\ 
        \midrule
        SGD                 &
        $69.32_{\textcolor{black!60}{\pm 0.19}}$
        & $62.18_{\textcolor{black!60}{\pm 0.06}}$ 
        & $55.78_{\textcolor{black!60}{\pm 0.55}}$  
        & $45.53_{\textcolor{black!60}{\pm 0.78}}$ \\ 
        
        SAM $_{\text{SGD}}$ & 
        $71.99_{\textcolor{black!60}{\pm 0.20}}$
        & $65.53_{\textcolor{black!60}{\pm 0.11}}$  
        & $ 61.20_{\textcolor{black!60}{\pm 0.17}}$  
        & $ 51.93_{\textcolor{black!60}{\pm 0.47}}$ \\ 
        
        AdaHessian         &
        $68.06_{\textcolor{black!60}{\pm 0.22}}$
        & $63.06_{\textcolor{black!60}{\pm 0.25}}$  
        & $58.37_{\textcolor{black!60}{\pm 0.13}}$  
        & $46.02_{\textcolor{black!60}{\pm 1.96}}$  \\

        Sophia-H           &
        $67.76_{\textcolor{black!60}{\pm 0.37}}$
        & $62.34_{\textcolor{black!60}{\pm 0.47}}$  
        & $56.54_{\textcolor{black!60}{\pm 0.28}}$  
        & $45.37_{\textcolor{black!60}{\pm 0.27}}$  \\
        
        Shampoo           & 
        $64.08_{\textcolor{black!60}{\pm 0.46}}$
        & $58.85_{\textcolor{black!60}{\pm 0.66}}$ & $ 53.82 _{\textcolor{black!60}{\pm 0.71}}$  
        & $ 42.91_{\textcolor{black!60}{\pm 0.99}}$ \\
        
        \midrule
        
        \rowcolor{green!20} \sassha         & 
        $ \textbf{72.14}_{\textcolor{black!60}{\pm 0.16}}    $
        & $\textbf{66.78}_{\textcolor{black!60}{\pm 0.47}}   $  
        & $ \textbf{ 61.97}_{\textcolor{black!60}{\pm 0.27}} $  
        & $\textbf{ 53.98}_{\textcolor{black!60}{\pm 0.57}}  $ \\
        
        % \rowcolor{green!20} \msassha        &  
        % $ 70.93 _{\textcolor{black!60}{\pm 0.26}}     $
        % $ 66.10 _{\textcolor{black!60}{\pm 0.26}}     $  
        % &  $ 61.13 _{\textcolor{black!60}{\pm 0.28}}  $  
        % &  $ 52.45 _{\textcolor{black!60}{\pm 0.34}}  $ \\   
        \bottomrule  
    \end{tabular}}
    %\vspace{-1em}
\end{table}

\begin{figure}[t!]
%\vspace{-2em}
\resizebox{\linewidth}{!}{%
    \centering
    \begin{subfigure}{0.4\linewidth}
        \centering
        \includegraphics[width=\linewidth]{figures/ablation/loss.pdf}
        \caption{Train loss}
        \label{fig:sqrt_ablation_train_loss}
    \end{subfigure}
    \begin{subfigure}{0.4\linewidth}
        \centering
        \includegraphics[width=\linewidth,trim={0.5em 0.5em 1em 0},clip]{figures/ablation/Update_size.pdf}
        \caption{Update size}
        \label{fig:sqrt_ablation_update-size}
    \end{subfigure}
    \hfill
    \begin{subfigure}{0.4\linewidth}
        \centering
        \includegraphics[width=\linewidth,trim={0 1em 0 1.2em},clip]{figures/ablation/sqrt_ridgeline.pdf}\\
        
        \caption{$D$ distribution}
        \label{fig:sqrt_ablation_ridgeline-plot}
    \end{subfigure}
    }
    \caption{
    Effects of square-root measured for ResNet-32/CIFAR-100;
    $D$ is set to be either $|\widehat{H}|^{1/2}$ for \sassha or $|\widehat{H}|$ for \texttt{No-Sqrt}.
    Sharpness minimization drives the diagonal Hessian entries move towards zero, causing divergence.
    The square-root in \sassha helps counteract this effect, stabilizing the training process.
    }
    \label{fig:sqrt_ablation}
    %\vspace{-1em}
\end{figure}

\subsection{Stability} \label{sec:sqrt_ablation}

To show the effect of the square-root function on stabilizing the training process, we run \sassha without the square-root (\texttt{No-Sqrt}), repeatedly for multiple times with different random seeds.
As a result, we find that the training diverges most of the time.
A failure case is depicted in \cref{fig:sqrt_ablation}.

At first, we find that the level of training loss for \texttt{No-Sqrt} is much higher than that of \sassha, and also, it spikes up around step $200$ (\cref{fig:sqrt_ablation_train_loss}).
To look into it further, we also measure the update sizes along the trajectory (\cref{fig:sqrt_ablation_update-size}).
The results show that it matches well with the loss curves, suggesting that the training failure is somehow due to taking too large steps.

It turns out that this problem stems from the preconditioning matrix $D$ being too small;
\ie, the distribution of diagonal entries in the preconditioning matrix gradually shifts toward zero values (\cref{fig:sqrt_ablation_ridgeline-plot});
as a result, $D^{-1}$ becomes too large, creating large steps.
This progressive increase in near-zero diagonal Hessian entries is precisely due to the sharpness minimization scheme that we introduced; it penalizes the Hessian eigenspectrum to yield flat solutions, yet it could also make training unstable if taken naively.
By including square-root, the preconditioner are less situated near zero, effectively suppressing the risk of large updates, thereby stabilizing the training process.
We validate this further by showing its superiority to other alternatives including damping and clipping in \cref{app:sqrt_alternatives}.

We also provide an ablation analysis for the absolute-value function in \cref{sec:abs_ablation}, which demonstrates that it increases the stability of \sassha in tandem with square-root.

\begin{figure}[t!] 
    \centering
    \begin{minipage}{\linewidth}
        \centering
        \hspace{1.2em}
        \includegraphics[width=0.9\linewidth, trim={0 0 0 0},clip]{figures/ablation/legend_only.pdf} % Path to your legend image
        \vspace{-0.6em}
    \end{minipage}

    \resizebox{\linewidth}{!}{
        \begin{subfigure}{0.288\linewidth}
        \centering%
            \includegraphics[width=\textwidth,trim={0 -1em 1.2em 0.5em},clip]{figures/ablation/hessian_update_frequency.pdf}
            \caption{Lazy Hessian}
            \label{fig:lazy_results}
        \end{subfigure}
        % \hspace{2.3em} 
        \begin{subfigure}{0.305\linewidth}
            \centering
            \includegraphics[width=\textwidth,trim={2em 1.4em 0 0.1em},clip]{figures/ablation/hess_diff.pdf}
            \caption{$\widehat{H}$ change}
            \label{fig:lazy_hess_diff}
        \end{subfigure}%
        \begin{subfigure}{0.32\linewidth}
            \centering
            \includegraphics[width=\linewidth,trim={0 -0.7em 0 0.3em},clip]{figures/ablation/perturbed.pdf}
            \caption{Local sensitivity}
            \label{fig:lazy_perturbed}
        \end{subfigure}
    }

    \caption{
    Effect of lazy Hessian for ResNet-32/CIFAR-100.
    \sassha stays within the region where the Hessian varies small.
    }
    \label{fig:diagonal_hessian_comp} %
    %\vspace{-1em}
\end{figure}

\subsection{Efficiency} \label{sec:emp_lazy_hess}

Here we show the effectiveness of lazy Hessian updates in \sassha.
The results are shown in \cref{fig:diagonal_hessian_comp}.
At first, we see that \sassha maintains its performance even at $k=100$, indicating that it is extremely robust to lazy Hessian updates (\cref{fig:lazy_results}).
We also measure the difference between the current and previous Hessians to validate lazy Hessian updates more directly (\cref{fig:lazy_hess_diff}).
The result shows that \sassha keeps the changes in Hessian to be small, and much smaller than other methods, indicating its advantage of robust reuse, and hence, computational efficiency.

We attribute this robustness to the sharpness minimization scheme incorporated in \sassha, which can potentially bias optimization toward the region of low curvature sensitivity.
To verify, we define local Hessian sensitivity as follows:
\begin{equation}\label{eq:diff_hessian(2)}
    \max_{\delta \sim \mathcal{N}(0, 1)}\left\|\widehat{H}\left(x+\rho\frac{\delta}{\|\delta\|_2}\right) - \widehat{H}(x)\right\|_F
\end{equation}
\ie, it measures the maximum change in Hessian induced from normalized random perturbations.
A smaller Hessian sensitivity would suggest reduced variability in the loss curvature, leading to greater relevance of the current Hessian for subsequent optimization steps.
We find that \sassha is far less sensitive compared to other methods (\cref{fig:lazy_perturbed}).

\subsection{Cost}
\label{sec:cost}

Second-order methods can be highly costly.
In this section, we discuss the computational cost of \sassha and reveal its competitiveness to other methods.

\sassha requires one gradient computation (\texttt{GC}) in the sharpness minimization step, one Hessian-vector product (\texttt{HVP}) for diagonal Hessian computation, and an additional \texttt{GC} in the descent step.
That is, a total of $2$\texttt{GC}s and $1$\texttt{HVP} are required.
However, with lazy Hessian updates, the number of \texttt{HVP}s reduces drastically to $ 1 / k $.
With $ k = 10 $ as the default value used in this work, this scales down to $0.1$\texttt{HVP}s.

It turns out that this is critical to the utility of \sassha, because $1$\texttt{HVP} is known to take about $ \times 3 $ the computation time of $1$\texttt{GC} in practice \citep{dagrou2024how}.
Compared to conventional second-order methods ($1$\texttt{GC} $+$ $1$\texttt{HVP} $\simeq$ $4$\texttt{GC}s), the cost of \sassha can roughly be a half of that ($2.3$\texttt{GC}s).
It is also comparable to standard SAM variants ($2$\texttt{GC}s).

Furthermore, we can leverage a momentum of gradients in the perturbation step to reduce the cost.
This variant \msassha requires only $1.3$\texttt{GC}s with minimal decrease in performance.
Notably, \msassha still outperforms standard first-order methods like SGD and AdamW (\cref{app:msassha}).

To verify, we measure the average wall-clock times and present the results in \cref{tab:costs}.
First, one can see that the theoretical cost is reflected well on the actual cost;
\ie, the time measurements scales proportionally roughly well with respect to the total cost.
More importantly, this result indicates the potential of \sassha for performance-critical applications.
Considering its well-balanced cost, and that it has been challenging to employ second-order methods efficiently for large-scale tasks without sacrificing performance, \sassha can be a reasonable addition to the lineup.

\begin{table}[t!]
    \vspace{-0.5em}
    \centering
    \caption{
    Average wall-clock time per epoch (\texttt{s}) and the theoretical cost of different methods.
    \sassha can be an effective alternative to existing methods for its enhanced generalization performance.
    } 
    \vskip 0.1in
    \resizebox{\linewidth}{!}{%
    \begin{tabular}{l|r|r|r|r|r|r|r}
    \toprule
    \multirow{2.5}{*}{Method}
    & \multicolumn{4}{c|}{Cost}
    & \multicolumn{1}{c|}{CIFAR10} 
    & \multicolumn{1}{c|}{CIFAR100} 
    & \multicolumn{1}{c}{ImageNet} \\ 
    \cmidrule(l{3pt}r{3pt}){2-5}
    \cmidrule(l{3pt}r{3pt}){6-8} 
     & \multicolumn{1}{c|}{Descent}
     & \multicolumn{1}{c|}{Sharpness}
     & \multicolumn{1}{c|}{Hessian}
     & \multicolumn{1}{c|}{Total}
     & \multicolumn{1}{c|}{ResNet32} 
     & \multicolumn{1}{c|}{WRN28-10}  
     & \multicolumn{1}{c}{ViT-small} \\ 
    \midrule
    AdamW  & 1 \texttt{GC}
    & 0 \texttt{GC}
    & 0 \texttt{HVP}
    & 1 \texttt{GC} & 5.03 & 59.29 & 976.56 \\
    
    SAM    & 1 \texttt{GC}
    & 1 \texttt{GC}
    & 0 \texttt{HVP}
    & 2 \texttt{GC} & 9.16 & 118.46  &  1302.08\\ 
    
    AdaHessian &  1 \texttt{GC}
    & 0 \texttt{GC}
    & 1 \texttt{HVP}
    & 4 \texttt{GC} & 33.75 & 296.63  & 2489.07 \\ \midrule
    
    \rowcolor{green!20} \sassha & 1 \texttt{GC}
    & 1 \texttt{GC}
    & 0.1 \texttt{HVP}
    & 2.3 \texttt{GC} & 12.00   & 142.06   & 1377.20 \\ 
    
    \rowcolor{green!20} \msassha & 1 \texttt{GC}
    & 0 \texttt{GC}
    & 0.1 \texttt{HVP}
    & 1.3 \texttt{GC}  & 8.91   & 84.12 & 1065.40 \\ 
    \bottomrule
    \end{tabular}
    }
    \vspace{-1em}
    \label{tab:costs}
\end{table}
%\section{Discussions}\label{sec:discussion}
% \paragraph{Strategy space: one-step strategies vs two-step strategies.}
% In a sequential mechanism, depending on whether the agent's choice of first and second attributes coincide, we can distinguish 
% \emph{one-step strategies} and \emph{two-step strategies}.
% Consider an arbitrary fixed order sequential mechanism $(\tilde\classifier_A,\tilde\classifier_B,1)$ where $\tilde\classifier_A$ is offered as the first test.
% Consider an agent with initial attributes $\orifeatures\notin \tilde\classifier_A\cap\tilde\classifier_B$, i.e., such an agent fails both tests initially.
% Knowing that $\tilde\classifier_A$ is offered as the first test, this agent can choose the first attributes $\firstfeatures\in \tilde \classifier_A$ in order to pass the first test.
% Before the second test $\tilde\classifier_B$ takes place, he can again choose some  $\secondfeatures\in \tilde \classifier_B$ to pass the second test. 
% We call such a strategy a \emph{two-step strategy} (See \cref{subfig:two step}).
% Alternatively, the agent can choose some attributes  $\tilde\features^1\in \tilde\classifier_A\cap\tilde\classifier_B$ before both tests.
% Since $\tilde\features^1$ is in the intersection of both tests, the agent can pass both tests and get selected.
% We call such a strategy a \emph{one-step strategy} (See  \cref{subfig:one step}).
% Both \emph{one-step strategies} and \emph{two-step strategies} are feasible in sequential mechanisms.
% However, in simultaneous mechanisms, only one-step strategies are feasible.
% Hence the major difference between simultaneous mechanisms and sequential mechanisms is the strategy space of the agent.



% \begin{figure}[h]  \label{fig: 1-step vs 2-step}
% 	\centering 
% 	\begin{subfigure}[b]{0.45\linewidth}
% 		\begin{tikzpicture}[xscale=7.8,yscale=7.8]
		
% 		\draw [domain=0.68:1.4, thick] plot (\x, {3/4*\x+1/4});
% 		\node [left] at (1.38, 1.3 ) {$\tilde \classifier_B$};%:\feature_2\geq \frac34 \feature_1 +\frac14
% 		\draw [thick] (1,0.68) -- (1,1.3);
% 		\node [right] at (1, 1.3) {$\tilde\classifier_A$};%: \feature_1 \geq 1$
		
% 		\node [left] at (1.33,1.25) {$+$};
% 		\node [right] at (1, 1.25) {$+$};
		
% 		\node[red] at (1-0.1, 1-0.2 ) {$\circ$};
% 		\node [ below] at (1-0.1, 1-0.2 ) {$\orifeatures$};
		
% 		\node [red] at  (1,1) {$\circ$};
% 		\node [right, red] at (1, 1 ) {$\tilde\features^1$};
		
% 		\draw [densely dashdotted, red] (1-0.1, 1-0.2 ) --(1,1);
% 		\end{tikzpicture}
% 		\caption{One-step strategy}  
% 		\label{subfig:one step}
% 	\end{subfigure}
% 	\begin{subfigure}[b]{0.45\linewidth}
% 		\begin{tikzpicture}[xscale=7.8,yscale=7.8]
		
% 		\draw [domain=0.68:1.4, thick] plot (\x, {3/4*\x+1/4});
% 		\node [left] at (1.38, 1.3 ) {$\tilde \classifier_B$};%:\feature_2\geq \frac34 \feature_1 +\frac14
% 		\draw [thick] (1,0.68) -- (1,1.3);
% 		\node [right] at (1, 1.3) {$\tilde\classifier_A$};%: \feature_1 \geq 1$
		
% 		\node [left] at (1.33,1.25) {$+$};
% 		\node [right] at (1, 1.25) {$+$};
		
% 		\node[blue] at (1-0.1, 1-0.2 ) {\textbullet};
% 		\node [ below] at (1-0.1, 1-0.2 ) {$\orifeatures$};
		
% 		\node [blue] at (1, 1-0.072 ) {\textbullet};
% 		\node [blue,right] at (1, 1-0.072 ) {$\firstfeatures$};
		
% 		\node [blue] at  (1+0.1-5.5/24*0.6,1-0.22+5.5/24*0.8) {\textbullet};
% 		\node [blue,above] at  (1+0.1-5.5/24*0.6,1-0.22+5.5/24*0.8) {$\secondfeatures$};
		
		
% 		\draw [densely dashdotted, blue] (1-0.1,1-0.2) -- (1,1-0.072);
% 		\draw [densely dashdotted, blue]  (1,1-0.072) -- (1+0.1-5.5/24*0.6,1-0.22+5.5/24*0.8);
		
% 		\end{tikzpicture}
% 		\caption{Two-step strategy}
% 		\label{subfig:two step}
% 	\end{subfigure}
% 	\caption{Example of two different types of strategies}
% 	\rule{0in}{1.2em}$^\dag$\scriptsize In principle, using a one-step strategy in a simultaneous mechanism could incur a different cost compared to using it in a sequential mechanism.
% 	This is because in the sequential mechanism, there could be a positive cost of maintaining the attributes for extra time.
% \end{figure}  


% Subtly, in a sequential mechanism, even when agent chooses the same attributes for both tests and he does not opt out after the first test, it could incur a positive cost of maintaining the attributes.
% However, this creates unnecessary complication and would not add any new insight to the analysis. 
% \cref{assump: cost function one step} 
% makes sure that once the agent chooses some attributes, there is no cost of maintaining it. 
% Therefore, we can safely write any \emph{one-step strategy} $\strategies=\tilde\features^1$  as 
% $\strategies=(\tilde\features^1,\tilde\features^1)$.

\paragraph{Dimensionality and degenerate sequential mechanisms.}
Our model cannot be reduced to a problem where agent has a one-dimension attribute and the principal offers two selection cutoffs.
Consider a setup where the agent has a one-dimensional attribute $\feature\in \bbR$.
The principal would like to select any agent with attribute $\feature\in [b_1,b_2]$, where $b_1<b_2$.
Suppose now the principal announces that any agent with attribute belonging to $[\tilde b_1,\tilde b_2]$ will be selected, with $\tilde b_1<\tilde b_2$. 
The corresponding tests would be $\tilde\classifier_A: \feature\geq \tilde b_1$ and $\tilde\classifier_B: \feature\leq \tilde b_2$.
Then no matter what the order of the tests is, the agent's best response is fixed: for any $\feature < \tilde b_1$, $\strategies(\feature)=(\tilde b_1,\tilde b_1)$; for any $\feature > \tilde b_2$, $\strategies(\feature)=(\tilde b_2,\tilde b_2)$; for any $\tilde b_1\leq \feature \leq \tilde b_2$, $\strategies(\feature)=(\feature,\feature)$.

Next we introduce the concept of degenerate testing mechanisms. 
\begin{definition}[degenerate mechanism]
	A sequential mechanism $(\tilde\classifier_A,\tilde\classifier_B,\probprincipal,\disclose)$ is said to be \emph{degenerate} if A's best response is the same regardless of the order of the tests.
\end{definition}

We summarize the above observation in the following proposition.
\begin{proposition}
	If the agent's attribute is one-dimension, then any testing mechanism $(\tilde\classifier_A,\tilde\classifier_B,\probprincipal,\disclose)$ is degenerate.
\end{proposition}

\citet{perez2022test} studies a test design problem where the agent's attribute is one dimensional.
The above observation provides a motivation to study the case where agent has multi dimensional attributes and how the order of the tests play a role in such settings. 
As a starting point, we study the simplest case where agent has two dimensional attributes and principal has two tests.




\paragraph{Modeling tests.}
We interpret a test as an evaluation of whether one (or some) aspect(s) of the agent meets the the principal's requirement.
We assume that the agent does not which aspect(s) he is being evaluated in each test.
We argue that this is a reasonable assumption.
For example, in a math test, the students taking the test do not know whether they are being evaluated on their mathematical thinking skill or their mastery of mathematical tools.
This slightly differs from the way we use the word test in English.
When we say there is a math test, the math test is a subject that includes at least the above mentioned two \emph{aspects of ability} being evaluated.
Hence if we interpret a test as an evaluation of aspect(s) of ability, it is unlikely that the students know how each aspect of ability is evaluated (or graded) unless they are told.
% Before we proceed, we point out one subtle but important distinction between the \emph{tests} in our model and the tests we use in day-to-day life.
% In day-to-day life, when we refer to a test or a school test, we usually refer to a specific subject or a topic, for example, math test, English test, etc..
% The testee usually knows which \emph{subject} they are being tested.
% However, in our model, a \emph{test} is an evaluation to one skill or multiple skills.
% Therefore, even if the testee knows which \emph{subject} they are being tested, they may still not know which \emph{aspect(s) of ability} they are being tested. 


\paragraph{Simultaneous mechanisms are not special case of sequential mechanisms.}
Although in simultaneous mechanisms the order of the tests does not matter, they cannot be viewed as special cases of sequential mechanisms, i.e.,  $\simultaneous\not\subset \sequential$.
First of all, in any game specified by a simultaneous mechanism, the agent only has one information node.
Therefore, a simultaneous mechanism cannot be achieved by any fixed order mechanism or random order mechanism with disclosure.
Second, even though in the game specified by a random order mechanism without disclosure, the agent also has one information node, his strategy space is much larger.
This is because in any simultaneous mechanism, the agent is essentially restricted to use \emph{one-step} strategies, while in any random order mechanism without disclosure, the agent can use either \emph{one-step} strategies or \emph{two-step} strategies. 
An alternative way to see this is that in any random order mechanism without disclosure, the agent is evaluated by a \emph{linear} test each time.
However, in any simultaneous mechanism, the agent is evaluated by the intersection of two \emph{linear} tests each time. 
More formally, let $\twolineartest=\{\tilde\classifier_1\cap\tilde\classifier_2:\tilde\classifier_i\in \lineartest, i=1,2\}$ be the space of tests that can be used in any simultaneous mechanism.
Then $\lineartest\subset \twolineartest$.
Therefore, by changing the timing to offer the tests, simultaneous mechanisms are essentially enlarging the space of tests that can be used.

\paragraph{Dynamic nature of sequential mechanisms.}
To further illustrate why the game induced by any sequential mechanism is not static, let's consider the cost function $\cost(\orifeatures,\firstfeatures,\secondfeatures)=\metric(\orifeatures,\firstfeatures)+\metric(\firstfeatures,\secondfeatures)$, where $\metric(\cdot,\cdot)$ is the Euclidean distance.
Consider two fixed order mechanisms that use the same two tests $\tilde\classifier_A$ and $\tilde\classifier_B$ and they only differ in terms of the order of the two tests.
In \cref{subfig:two step A-B}, the mechanism offers $\tilde\classifier_A$ as the first test, while in \cref{subfig:two step B-A}, the mechanism offers $\tilde\classifier_B$ as the first test.
Consider an agent with attributes $\orifeatures$ as in \cref{subfig:two step A-B}.
Notice that $\orifeatures$ pass test $\tilde\classifier_A$ but fail test $\tilde\classifier_B$.
Under the fixed order mechanism $(\tilde\classifier_A,\tilde\classifier_B,1)$, the optimal strategy of such attributes is to choose $\firstfeatures=\orifeatures$ and a different $\secondfeatures$ that are the projection of $\orifeatures$ on the boundary line of $\tilde\classifier_B$ (See \cref{subfig:two step A-B}).
% We call this a \emph{two-step} strategy.
Such a strategy is the least costly for the agent and guarantees that the agent is selected under this fixed order mechanism.
However, under another fixed order mechanism $(\tilde\classifier_B,\tilde\classifier_A,1)$, the optimal strategy of such attributes is to choose $\firstfeatures=\secondfeatures$ that are the intersection of  the boundary line of $\tilde\classifier_A$ and $\tilde\classifier_B$ (See \cref{subfig:two step A-B}).
This is because now $\tilde\classifier_B$ is the first test the agent needs to pass and $\orifeatures$  are so far away from  $\tilde\classifier_B$ that it is less costly to pass the two tests at the same time.
% We call this a \emph{one-step} strategy.
This example shows that a sequential mechanism is naturally dynamic and as a result, the agent's strategy is also dynamic. 



%---------------------------------------
\begin{figure}[h]  \label{fig: sequential is dynamic}
	\centering 
	\begin{subfigure}[b]{0.45\linewidth}
		\begin{tikzpicture}[xscale=7.8,yscale=7.8]
		
		\draw [domain=0.68:1.4, thick] plot (\x, {3/4*\x+1/4});
		\node [left] at (1.38, 1.3 ) {$\tilde \classifier_B$};%:\feature_2\geq \frac34 \feature_1 +\frac14
		\draw [thick, red] (1,0.68) -- (1,1.3);
		\node [right] at (1, 1.3) {$\tilde\classifier_A$};%: \feature_1 \geq 1$
		
		\node [left] at (1.33,1.25) {$+$};
		\node [right] at (1, 1.25) {$+$};
		
		\node[black] at (1, 1-0.2 ) {\textbullet};
		\node [ right] at (1, 1-0.2 ) {$\orifeatures\textcolor{blue}{=\firstfeatures}$};
		
		\node [blue] at  (1-12/125,1-9/125) {\textbullet};
		\node [left, blue] at (1-12/125,1-9/125 ) {$\secondfeatures$};
		
		\draw [densely dashdotted, blue] (1, 1-0.2 ) --(1-12/125,1-9/125);
		\end{tikzpicture}
		\caption{Two-step strategy: $\tilde\classifier_A\rightarrow\tilde\classifier_B$}  
		\label{subfig:two step A-B}
	\end{subfigure}
	\begin{subfigure}[b]{0.45\linewidth}
		\begin{tikzpicture}[xscale=7.8,yscale=7.8]
		
		\draw [domain=0.68:1.4, thick, red] plot (\x, {3/4*\x+1/4});
		\node [left] at (1.38, 1.3 ) {$\tilde \classifier_B$};%:\feature_2\geq \frac34 \feature_1 +\frac14
		\draw [thick] (1,0.68) -- (1,1.3);
		\node [right] at (1, 1.3) {$\tilde\classifier_A$};%: \feature_1 \geq 1$
		
		\node [left] at (1.33,1.25) {$+$};
		\node [right] at (1, 1.25) {$+$};
		
		\node[blue] at (1, 1-0.2 ) {\textbullet};
		\node [ right] at (1, 1-0.2 ) {$\orifeatures$};
		
		% \node [blue] at (1-12/125,1-9/125) {\textbullet};
		% \node [blue,left] at (0.8,1 ) {$\tilde\features$};
		
		\node [blue] at  (1,1) {\textbullet};
		\node [blue,right] at  (1,1) {$\firstfeatures=\secondfeatures$};
		
		
		\draw [densely dashdotted, blue] (1,1-0.2) -- (1,1);
		% \draw [densely dashdotted, blue]  (1,1) -- (0.8,1);
		
		\end{tikzpicture}
		\caption{Two-step strategy: $\tilde\classifier_B\rightarrow\tilde\classifier_A$}
		\label{subfig:two step B-A}
	\end{subfigure}
	\caption{Dynamic nature of sequential mechanisms}
	\rule{0in}{1.2em}$^\dag$\scriptsize Suppose $\cost(\orifeatures,\firstfeatures,\secondfeatures)=\metric(\orifeatures,\firstfeatures)+\metric(\firstfeatures,\secondfeatures)$, where $\metric(\cdot,\cdot)$ is the Euclidean distance. The left panel shows the optimal strategy under the fixed order mechanism that offers test $\tilde\classifier_A$ as the first test, while the right panel shows the optimal strategy under the fixed order mechanism that offers test $\tilde\classifier_B$ as the first test.
\end{figure}  

\paragraph{More tests do not help.}
In the manipulation setting, adding more tests only makes a procedure more stringent and it is counter-productive. 
Suppose the principal offers test $\classifier_1$ and $\classifier_2$ repeatedly. For simplicity, suppose the first test is $\classifier_1$, the second is $\classifier_2$ and the third is $\classifier_1$. For those types that find it profitable to pass both tests together under a fixed-order procedure with two tests, adding an extra test does not change their incentives.
For those types that find it profitable to pass one test at a time, adding a third test only make it more costly to pass one test at a time.
Hence, the benefit of the fixed-order procedure is diminished under more tests.

In the investment setting, under the optimal simultaneous mechanism, adding one more test cannot improve the outcome since using two tests already achieve the first best.


\paragraph{Endogenous agent's technology.}
We assume that the agent's technology, whether he manipulates or invests, is exogenous. 
This usually can be explained by the different monitoring strength of the economic environment.
In the banking application, US banks are tested daily and therefore there is no room for manipulation.
In contrast, European banks are tested monthly or quarterly.
The low monitoring intensity allows banks to temporarily change their balance sheet.

Suppose instead now when the agent changes his type from $\orifeatures$ to $\firstfeatures$, he can choose between  manipulation, which costs $0.2\cdot\onecost(\orifeatures,\firstfeatures)$ and investment, which costs $0.5\cdot\onecost(\orifeatures,\firstfeatures)$.
Then the only equilibrium is such that every type chooses manipulation and the principal chooses the optimal fixed-order sequential mechanism under manipulation, given that the principal's objective is \ref{max qualified}. 
This is because (1) whichever mechanism the principal chooses, the agent prefers manipulation over investment, and (2) no type can credibly commits to investment. 
As a result, although every type would have been better-off by choosing investment, it is not an equilibrium.



\paragraph{Informational robustness.}
When the agent manipulates , the optimal tests are chosen based on the information about the agent's cost.
Whenever there is a small amount of uncertainty in the agent's cost function,  the principal needs to choose even more stringent tests so as not to select any unqualified agent in the worst case where the agent's cost happens to be least costly one.
In contrast, when the agent invests, the optimal tests coincide with the qualified region. 
Hence the optimal mechanism in the investment setting is more robust to the agent's information.

% \paragraph{Stringency of testing procedures}
% Fixing the tests, we use how difficult it is for the agent to to manipulate to measure the stringency of a testing procedure.
% Roughly speaking, a simultaneous procedure is weakly more stringent than a random-order sequential procedure without disclosure, which is  more stringent than a random-order sequential procedure with disclosure,  which is  more stringent than a fixed-order sequential procedure.

% \paragraph{Noisy tests}
% A stochastic \emph{linear} test $\tilde\classifier$ is a half plane and a random variable $\epsilon$ such that the agent passes the test if and only if agent's attributes $\features +\epsilon \in \tilde\classifier$.
% The test result of an agent with attributes $\features$ under test $\tilde\classifier$ is $\test= \indicate{\features+\epsilon\in \tilde\classifier}$.
% Incorporating stochasticity into the test does not change our results.


% \paragraph{Randomizing over multiple tests}




% \paragraph{Connections to persuasion problems.}
% \citet{glazer2004optimal} and \citet{sher2014persuasion} study persuasion problems with evidence where the agent's private information also has two dimensions. 
% Instead of passing tests like in our setting, the agent needs to provide hard evidence to persuade the principal.
% The main difference of our paper compared to these two is that we adopt a slightly more general approach to model what information the agent can use to persuade the principal, which is similar to the modeling approached used in \citet{perez2022test}.
% Evidence usually contains the truth and therefore cannot be fabricated. 
% Or rather, we can view it as the cost of adopting different attributes is infinity.
% In this sense, it is more general to assume that agent can adopt different attributes to present to the principal with some cost. 
% Then persuasion with evidence can be viewed as (1) the agent's cost of changing attributes is infinitely high, and (2) the principal can use \emph{perfect} tests, i.e., principal can observe agent's attributes in each round of communication.

% \vspace{-0.5em}
\section{Conclusion}
In this work, we focus on addressing the issue of poor generalization in approximate second-order methods.
To this end, we propose a new method called \sassha that stably minimizes sharpness within the framework of second-order optimization.
\sassha converges to flat solutions and achieves state-of-the-art performance within this class.
\sassha also performs competitively to widely-used first-order, adaptive, and sharpness-aware methods.
\sassha achieves this in robust, stable, and efficient ways without incurring much cost or requiring extra hyperparameter tuning.
All of these are rigorously assessed with extensive experiments.

% Moreover, stable Hessian approximation in \sassha effectively mitigates instability issues during the sharpness minimization without extra hyperparameters, and allows lazy Hessian updates, significantly reducing the cost of Hessian computation.
% Through this process, we also discover that sharpness minimization can aid lazy Hessian updates.
Nonetheless, there are many limitations to be addressed in this work for more improvements which may include, but are not limited to, 
extending experiments to extreme scales of various models and different data, and consolidating theoretical foundations.
Seeing it as an exciting opportunity, we plan to investigate further in future work.

% evaluating on a more extreme scale or other domains,
% and developing theoretical properties such as convergence rate and implicit bias, all to more rigorously confirm the value of \sassha.
% Seeing it as an exciting opportunity, we are planning to investigate further in future work.

% In this work, we have addressed the poor generalization issue of approximate second-order methods by proposing \sassha, which explicitly minimizes sharpness within approximate second-order optimization, achieving competitive performance for various standard deep learning tasks.
% Nonetheless, there are many remaining possibilities for further improvements which may include, but are not limited to, 
% evaluating on a more extreme scale and other data distributions in different domains,
% and developing theoretical properties such as convergence rate and implicit bias, all to more rigorously confirm the value of \sassha.
% Seeing it as an exciting opportunity, we are planning to investigate further in future work.

% In this work, we have attended to the poor generalization issue of approximate second-order methods.
% Inspired by the concept of sharpness minimization and their potential for generalization improvement, we have made a collection of effective alterations to existing methods and presented a new method called \sassha.
% We emphasize that \sassha has been extensively evaluated and achieved state-of-the-art performance for various standard deep learning tasks.
% Nonetheless, there are many remaining possibilities for further improvements.
% Some examples may include, but not limited to, 
% evaluating on more extreme scale and other data distributions in different domains,
% and developing theoretical properties such as convergence and implicit bias, all to more rigorously confirm the value of \sassha.
% Seeing it as an exciting opportunity, we are planning to investigate further in future work.
\section{Acknowledgement}

This work is supported by the National Science Foundation under grant No. 2144489.

%\input{text/11_impact statment}

% In the unusual situation where you want a paper to appear in the
% references without citing it in the main text, use \nocite
%\nocite{langley00}

\bibliography{example_paper}
\bibliographystyle{icml2025}

%%%%%%%%%%%%%%%%%%%%%%%%%%%%%%%%%%%%%%%%%%%%%%%%%%%%%%%%%%%%%%%%%%%%%%%%%%%%%%%
%%%%%%%%%%%%%%%%%%%%%%%%%%%%%%%%%%%%%%%%%%%%%%%%%%%%%%%%%%%%%%%%%%%%%%%%%%%%%%%
% APPENDIX
%%%%%%%%%%%%%%%%%%%%%%%%%%%%%%%%%%%%%%%%%%%%%%%%%%%%%%%%%%%%%%%%%%%%%%%%%%%%%%%
%%%%%%%%%%%%%%%%%%%%%%%%%%%%%%%%%%%%%%%%%%%%%%%%%%%%%%%%%%%%%%%%%%%%%%%%%%%%%%%

\newpage
\appendix
\onecolumn
\subsection{Lloyd-Max Algorithm}
\label{subsec:Lloyd-Max}
For a given quantization bitwidth $B$ and an operand $\bm{X}$, the Lloyd-Max algorithm finds $2^B$ quantization levels $\{\hat{x}_i\}_{i=1}^{2^B}$ such that quantizing $\bm{X}$ by rounding each scalar in $\bm{X}$ to the nearest quantization level minimizes the quantization MSE. 

The algorithm starts with an initial guess of quantization levels and then iteratively computes quantization thresholds $\{\tau_i\}_{i=1}^{2^B-1}$ and updates quantization levels $\{\hat{x}_i\}_{i=1}^{2^B}$. Specifically, at iteration $n$, thresholds are set to the midpoints of the previous iteration's levels:
\begin{align*}
    \tau_i^{(n)}=\frac{\hat{x}_i^{(n-1)}+\hat{x}_{i+1}^{(n-1)}}2 \text{ for } i=1\ldots 2^B-1
\end{align*}
Subsequently, the quantization levels are re-computed as conditional means of the data regions defined by the new thresholds:
\begin{align*}
    \hat{x}_i^{(n)}=\mathbb{E}\left[ \bm{X} \big| \bm{X}\in [\tau_{i-1}^{(n)},\tau_i^{(n)}] \right] \text{ for } i=1\ldots 2^B
\end{align*}
where to satisfy boundary conditions we have $\tau_0=-\infty$ and $\tau_{2^B}=\infty$. The algorithm iterates the above steps until convergence.

Figure \ref{fig:lm_quant} compares the quantization levels of a $7$-bit floating point (E3M3) quantizer (left) to a $7$-bit Lloyd-Max quantizer (right) when quantizing a layer of weights from the GPT3-126M model at a per-tensor granularity. As shown, the Lloyd-Max quantizer achieves substantially lower quantization MSE. Further, Table \ref{tab:FP7_vs_LM7} shows the superior perplexity achieved by Lloyd-Max quantizers for bitwidths of $7$, $6$ and $5$. The difference between the quantizers is clear at 5 bits, where per-tensor FP quantization incurs a drastic and unacceptable increase in perplexity, while Lloyd-Max quantization incurs a much smaller increase. Nevertheless, we note that even the optimal Lloyd-Max quantizer incurs a notable ($\sim 1.5$) increase in perplexity due to the coarse granularity of quantization. 

\begin{figure}[h]
  \centering
  \includegraphics[width=0.7\linewidth]{sections/figures/LM7_FP7.pdf}
  \caption{\small Quantization levels and the corresponding quantization MSE of Floating Point (left) vs Lloyd-Max (right) Quantizers for a layer of weights in the GPT3-126M model.}
  \label{fig:lm_quant}
\end{figure}

\begin{table}[h]\scriptsize
\begin{center}
\caption{\label{tab:FP7_vs_LM7} \small Comparing perplexity (lower is better) achieved by floating point quantizers and Lloyd-Max quantizers on a GPT3-126M model for the Wikitext-103 dataset.}
\begin{tabular}{c|cc|c}
\hline
 \multirow{2}{*}{\textbf{Bitwidth}} & \multicolumn{2}{|c|}{\textbf{Floating-Point Quantizer}} & \textbf{Lloyd-Max Quantizer} \\
 & Best Format & Wikitext-103 Perplexity & Wikitext-103 Perplexity \\
\hline
7 & E3M3 & 18.32 & 18.27 \\
6 & E3M2 & 19.07 & 18.51 \\
5 & E4M0 & 43.89 & 19.71 \\
\hline
\end{tabular}
\end{center}
\end{table}

\subsection{Proof of Local Optimality of LO-BCQ}
\label{subsec:lobcq_opt_proof}
For a given block $\bm{b}_j$, the quantization MSE during LO-BCQ can be empirically evaluated as $\frac{1}{L_b}\lVert \bm{b}_j- \bm{\hat{b}}_j\rVert^2_2$ where $\bm{\hat{b}}_j$ is computed from equation (\ref{eq:clustered_quantization_definition}) as $C_{f(\bm{b}_j)}(\bm{b}_j)$. Further, for a given block cluster $\mathcal{B}_i$, we compute the quantization MSE as $\frac{1}{|\mathcal{B}_{i}|}\sum_{\bm{b} \in \mathcal{B}_{i}} \frac{1}{L_b}\lVert \bm{b}- C_i^{(n)}(\bm{b})\rVert^2_2$. Therefore, at the end of iteration $n$, we evaluate the overall quantization MSE $J^{(n)}$ for a given operand $\bm{X}$ composed of $N_c$ block clusters as:
\begin{align*}
    \label{eq:mse_iter_n}
    J^{(n)} = \frac{1}{N_c} \sum_{i=1}^{N_c} \frac{1}{|\mathcal{B}_{i}^{(n)}|}\sum_{\bm{v} \in \mathcal{B}_{i}^{(n)}} \frac{1}{L_b}\lVert \bm{b}- B_i^{(n)}(\bm{b})\rVert^2_2
\end{align*}

At the end of iteration $n$, the codebooks are updated from $\mathcal{C}^{(n-1)}$ to $\mathcal{C}^{(n)}$. However, the mapping of a given vector $\bm{b}_j$ to quantizers $\mathcal{C}^{(n)}$ remains as  $f^{(n)}(\bm{b}_j)$. At the next iteration, during the vector clustering step, $f^{(n+1)}(\bm{b}_j)$ finds new mapping of $\bm{b}_j$ to updated codebooks $\mathcal{C}^{(n)}$ such that the quantization MSE over the candidate codebooks is minimized. Therefore, we obtain the following result for $\bm{b}_j$:
\begin{align*}
\frac{1}{L_b}\lVert \bm{b}_j - C_{f^{(n+1)}(\bm{b}_j)}^{(n)}(\bm{b}_j)\rVert^2_2 \le \frac{1}{L_b}\lVert \bm{b}_j - C_{f^{(n)}(\bm{b}_j)}^{(n)}(\bm{b}_j)\rVert^2_2
\end{align*}

That is, quantizing $\bm{b}_j$ at the end of the block clustering step of iteration $n+1$ results in lower quantization MSE compared to quantizing at the end of iteration $n$. Since this is true for all $\bm{b} \in \bm{X}$, we assert the following:
\begin{equation}
\begin{split}
\label{eq:mse_ineq_1}
    \tilde{J}^{(n+1)} &= \frac{1}{N_c} \sum_{i=1}^{N_c} \frac{1}{|\mathcal{B}_{i}^{(n+1)}|}\sum_{\bm{b} \in \mathcal{B}_{i}^{(n+1)}} \frac{1}{L_b}\lVert \bm{b} - C_i^{(n)}(b)\rVert^2_2 \le J^{(n)}
\end{split}
\end{equation}
where $\tilde{J}^{(n+1)}$ is the the quantization MSE after the vector clustering step at iteration $n+1$.

Next, during the codebook update step (\ref{eq:quantizers_update}) at iteration $n+1$, the per-cluster codebooks $\mathcal{C}^{(n)}$ are updated to $\mathcal{C}^{(n+1)}$ by invoking the Lloyd-Max algorithm \citep{Lloyd}. We know that for any given value distribution, the Lloyd-Max algorithm minimizes the quantization MSE. Therefore, for a given vector cluster $\mathcal{B}_i$ we obtain the following result:

\begin{equation}
    \frac{1}{|\mathcal{B}_{i}^{(n+1)}|}\sum_{\bm{b} \in \mathcal{B}_{i}^{(n+1)}} \frac{1}{L_b}\lVert \bm{b}- C_i^{(n+1)}(\bm{b})\rVert^2_2 \le \frac{1}{|\mathcal{B}_{i}^{(n+1)}|}\sum_{\bm{b} \in \mathcal{B}_{i}^{(n+1)}} \frac{1}{L_b}\lVert \bm{b}- C_i^{(n)}(\bm{b})\rVert^2_2
\end{equation}

The above equation states that quantizing the given block cluster $\mathcal{B}_i$ after updating the associated codebook from $C_i^{(n)}$ to $C_i^{(n+1)}$ results in lower quantization MSE. Since this is true for all the block clusters, we derive the following result: 
\begin{equation}
\begin{split}
\label{eq:mse_ineq_2}
     J^{(n+1)} &= \frac{1}{N_c} \sum_{i=1}^{N_c} \frac{1}{|\mathcal{B}_{i}^{(n+1)}|}\sum_{\bm{b} \in \mathcal{B}_{i}^{(n+1)}} \frac{1}{L_b}\lVert \bm{b}- C_i^{(n+1)}(\bm{b})\rVert^2_2  \le \tilde{J}^{(n+1)}   
\end{split}
\end{equation}

Following (\ref{eq:mse_ineq_1}) and (\ref{eq:mse_ineq_2}), we find that the quantization MSE is non-increasing for each iteration, that is, $J^{(1)} \ge J^{(2)} \ge J^{(3)} \ge \ldots \ge J^{(M)}$ where $M$ is the maximum number of iterations. 
%Therefore, we can say that if the algorithm converges, then it must be that it has converged to a local minimum. 
\hfill $\blacksquare$


\begin{figure}
    \begin{center}
    \includegraphics[width=0.5\textwidth]{sections//figures/mse_vs_iter.pdf}
    \end{center}
    \caption{\small NMSE vs iterations during LO-BCQ compared to other block quantization proposals}
    \label{fig:nmse_vs_iter}
\end{figure}

Figure \ref{fig:nmse_vs_iter} shows the empirical convergence of LO-BCQ across several block lengths and number of codebooks. Also, the MSE achieved by LO-BCQ is compared to baselines such as MXFP and VSQ. As shown, LO-BCQ converges to a lower MSE than the baselines. Further, we achieve better convergence for larger number of codebooks ($N_c$) and for a smaller block length ($L_b$), both of which increase the bitwidth of BCQ (see Eq \ref{eq:bitwidth_bcq}).


\subsection{Additional Accuracy Results}
%Table \ref{tab:lobcq_config} lists the various LOBCQ configurations and their corresponding bitwidths.
\begin{table}
\setlength{\tabcolsep}{4.75pt}
\begin{center}
\caption{\label{tab:lobcq_config} Various LO-BCQ configurations and their bitwidths.}
\begin{tabular}{|c||c|c|c|c||c|c||c|} 
\hline
 & \multicolumn{4}{|c||}{$L_b=8$} & \multicolumn{2}{|c||}{$L_b=4$} & $L_b=2$ \\
 \hline
 \backslashbox{$L_A$\kern-1em}{\kern-1em$N_c$} & 2 & 4 & 8 & 16 & 2 & 4 & 2 \\
 \hline
 64 & 4.25 & 4.375 & 4.5 & 4.625 & 4.375 & 4.625 & 4.625\\
 \hline
 32 & 4.375 & 4.5 & 4.625& 4.75 & 4.5 & 4.75 & 4.75 \\
 \hline
 16 & 4.625 & 4.75& 4.875 & 5 & 4.75 & 5 & 5 \\
 \hline
\end{tabular}
\end{center}
\end{table}

%\subsection{Perplexity achieved by various LO-BCQ configurations on Wikitext-103 dataset}

\begin{table} \centering
\begin{tabular}{|c||c|c|c|c||c|c||c|} 
\hline
 $L_b \rightarrow$& \multicolumn{4}{c||}{8} & \multicolumn{2}{c||}{4} & 2\\
 \hline
 \backslashbox{$L_A$\kern-1em}{\kern-1em$N_c$} & 2 & 4 & 8 & 16 & 2 & 4 & 2  \\
 %$N_c \rightarrow$ & 2 & 4 & 8 & 16 & 2 & 4 & 2 \\
 \hline
 \hline
 \multicolumn{8}{c}{GPT3-1.3B (FP32 PPL = 9.98)} \\ 
 \hline
 \hline
 64 & 10.40 & 10.23 & 10.17 & 10.15 &  10.28 & 10.18 & 10.19 \\
 \hline
 32 & 10.25 & 10.20 & 10.15 & 10.12 &  10.23 & 10.17 & 10.17 \\
 \hline
 16 & 10.22 & 10.16 & 10.10 & 10.09 &  10.21 & 10.14 & 10.16 \\
 \hline
  \hline
 \multicolumn{8}{c}{GPT3-8B (FP32 PPL = 7.38)} \\ 
 \hline
 \hline
 64 & 7.61 & 7.52 & 7.48 &  7.47 &  7.55 &  7.49 & 7.50 \\
 \hline
 32 & 7.52 & 7.50 & 7.46 &  7.45 &  7.52 &  7.48 & 7.48  \\
 \hline
 16 & 7.51 & 7.48 & 7.44 &  7.44 &  7.51 &  7.49 & 7.47  \\
 \hline
\end{tabular}
\caption{\label{tab:ppl_gpt3_abalation} Wikitext-103 perplexity across GPT3-1.3B and 8B models.}
\end{table}

\begin{table} \centering
\begin{tabular}{|c||c|c|c|c||} 
\hline
 $L_b \rightarrow$& \multicolumn{4}{c||}{8}\\
 \hline
 \backslashbox{$L_A$\kern-1em}{\kern-1em$N_c$} & 2 & 4 & 8 & 16 \\
 %$N_c \rightarrow$ & 2 & 4 & 8 & 16 & 2 & 4 & 2 \\
 \hline
 \hline
 \multicolumn{5}{|c|}{Llama2-7B (FP32 PPL = 5.06)} \\ 
 \hline
 \hline
 64 & 5.31 & 5.26 & 5.19 & 5.18  \\
 \hline
 32 & 5.23 & 5.25 & 5.18 & 5.15  \\
 \hline
 16 & 5.23 & 5.19 & 5.16 & 5.14  \\
 \hline
 \multicolumn{5}{|c|}{Nemotron4-15B (FP32 PPL = 5.87)} \\ 
 \hline
 \hline
 64  & 6.3 & 6.20 & 6.13 & 6.08  \\
 \hline
 32  & 6.24 & 6.12 & 6.07 & 6.03  \\
 \hline
 16  & 6.12 & 6.14 & 6.04 & 6.02  \\
 \hline
 \multicolumn{5}{|c|}{Nemotron4-340B (FP32 PPL = 3.48)} \\ 
 \hline
 \hline
 64 & 3.67 & 3.62 & 3.60 & 3.59 \\
 \hline
 32 & 3.63 & 3.61 & 3.59 & 3.56 \\
 \hline
 16 & 3.61 & 3.58 & 3.57 & 3.55 \\
 \hline
\end{tabular}
\caption{\label{tab:ppl_llama7B_nemo15B} Wikitext-103 perplexity compared to FP32 baseline in Llama2-7B and Nemotron4-15B, 340B models}
\end{table}

%\subsection{Perplexity achieved by various LO-BCQ configurations on MMLU dataset}


\begin{table} \centering
\begin{tabular}{|c||c|c|c|c||c|c|c|c|} 
\hline
 $L_b \rightarrow$& \multicolumn{4}{c||}{8} & \multicolumn{4}{c||}{8}\\
 \hline
 \backslashbox{$L_A$\kern-1em}{\kern-1em$N_c$} & 2 & 4 & 8 & 16 & 2 & 4 & 8 & 16  \\
 %$N_c \rightarrow$ & 2 & 4 & 8 & 16 & 2 & 4 & 2 \\
 \hline
 \hline
 \multicolumn{5}{|c|}{Llama2-7B (FP32 Accuracy = 45.8\%)} & \multicolumn{4}{|c|}{Llama2-70B (FP32 Accuracy = 69.12\%)} \\ 
 \hline
 \hline
 64 & 43.9 & 43.4 & 43.9 & 44.9 & 68.07 & 68.27 & 68.17 & 68.75 \\
 \hline
 32 & 44.5 & 43.8 & 44.9 & 44.5 & 68.37 & 68.51 & 68.35 & 68.27  \\
 \hline
 16 & 43.9 & 42.7 & 44.9 & 45 & 68.12 & 68.77 & 68.31 & 68.59  \\
 \hline
 \hline
 \multicolumn{5}{|c|}{GPT3-22B (FP32 Accuracy = 38.75\%)} & \multicolumn{4}{|c|}{Nemotron4-15B (FP32 Accuracy = 64.3\%)} \\ 
 \hline
 \hline
 64 & 36.71 & 38.85 & 38.13 & 38.92 & 63.17 & 62.36 & 63.72 & 64.09 \\
 \hline
 32 & 37.95 & 38.69 & 39.45 & 38.34 & 64.05 & 62.30 & 63.8 & 64.33  \\
 \hline
 16 & 38.88 & 38.80 & 38.31 & 38.92 & 63.22 & 63.51 & 63.93 & 64.43  \\
 \hline
\end{tabular}
\caption{\label{tab:mmlu_abalation} Accuracy on MMLU dataset across GPT3-22B, Llama2-7B, 70B and Nemotron4-15B models.}
\end{table}


%\subsection{Perplexity achieved by various LO-BCQ configurations on LM evaluation harness}

\begin{table} \centering
\begin{tabular}{|c||c|c|c|c||c|c|c|c|} 
\hline
 $L_b \rightarrow$& \multicolumn{4}{c||}{8} & \multicolumn{4}{c||}{8}\\
 \hline
 \backslashbox{$L_A$\kern-1em}{\kern-1em$N_c$} & 2 & 4 & 8 & 16 & 2 & 4 & 8 & 16  \\
 %$N_c \rightarrow$ & 2 & 4 & 8 & 16 & 2 & 4 & 2 \\
 \hline
 \hline
 \multicolumn{5}{|c|}{Race (FP32 Accuracy = 37.51\%)} & \multicolumn{4}{|c|}{Boolq (FP32 Accuracy = 64.62\%)} \\ 
 \hline
 \hline
 64 & 36.94 & 37.13 & 36.27 & 37.13 & 63.73 & 62.26 & 63.49 & 63.36 \\
 \hline
 32 & 37.03 & 36.36 & 36.08 & 37.03 & 62.54 & 63.51 & 63.49 & 63.55  \\
 \hline
 16 & 37.03 & 37.03 & 36.46 & 37.03 & 61.1 & 63.79 & 63.58 & 63.33  \\
 \hline
 \hline
 \multicolumn{5}{|c|}{Winogrande (FP32 Accuracy = 58.01\%)} & \multicolumn{4}{|c|}{Piqa (FP32 Accuracy = 74.21\%)} \\ 
 \hline
 \hline
 64 & 58.17 & 57.22 & 57.85 & 58.33 & 73.01 & 73.07 & 73.07 & 72.80 \\
 \hline
 32 & 59.12 & 58.09 & 57.85 & 58.41 & 73.01 & 73.94 & 72.74 & 73.18  \\
 \hline
 16 & 57.93 & 58.88 & 57.93 & 58.56 & 73.94 & 72.80 & 73.01 & 73.94  \\
 \hline
\end{tabular}
\caption{\label{tab:mmlu_abalation} Accuracy on LM evaluation harness tasks on GPT3-1.3B model.}
\end{table}

\begin{table} \centering
\begin{tabular}{|c||c|c|c|c||c|c|c|c|} 
\hline
 $L_b \rightarrow$& \multicolumn{4}{c||}{8} & \multicolumn{4}{c||}{8}\\
 \hline
 \backslashbox{$L_A$\kern-1em}{\kern-1em$N_c$} & 2 & 4 & 8 & 16 & 2 & 4 & 8 & 16  \\
 %$N_c \rightarrow$ & 2 & 4 & 8 & 16 & 2 & 4 & 2 \\
 \hline
 \hline
 \multicolumn{5}{|c|}{Race (FP32 Accuracy = 41.34\%)} & \multicolumn{4}{|c|}{Boolq (FP32 Accuracy = 68.32\%)} \\ 
 \hline
 \hline
 64 & 40.48 & 40.10 & 39.43 & 39.90 & 69.20 & 68.41 & 69.45 & 68.56 \\
 \hline
 32 & 39.52 & 39.52 & 40.77 & 39.62 & 68.32 & 67.43 & 68.17 & 69.30  \\
 \hline
 16 & 39.81 & 39.71 & 39.90 & 40.38 & 68.10 & 66.33 & 69.51 & 69.42  \\
 \hline
 \hline
 \multicolumn{5}{|c|}{Winogrande (FP32 Accuracy = 67.88\%)} & \multicolumn{4}{|c|}{Piqa (FP32 Accuracy = 78.78\%)} \\ 
 \hline
 \hline
 64 & 66.85 & 66.61 & 67.72 & 67.88 & 77.31 & 77.42 & 77.75 & 77.64 \\
 \hline
 32 & 67.25 & 67.72 & 67.72 & 67.00 & 77.31 & 77.04 & 77.80 & 77.37  \\
 \hline
 16 & 68.11 & 68.90 & 67.88 & 67.48 & 77.37 & 78.13 & 78.13 & 77.69  \\
 \hline
\end{tabular}
\caption{\label{tab:mmlu_abalation} Accuracy on LM evaluation harness tasks on GPT3-8B model.}
\end{table}

\begin{table} \centering
\begin{tabular}{|c||c|c|c|c||c|c|c|c|} 
\hline
 $L_b \rightarrow$& \multicolumn{4}{c||}{8} & \multicolumn{4}{c||}{8}\\
 \hline
 \backslashbox{$L_A$\kern-1em}{\kern-1em$N_c$} & 2 & 4 & 8 & 16 & 2 & 4 & 8 & 16  \\
 %$N_c \rightarrow$ & 2 & 4 & 8 & 16 & 2 & 4 & 2 \\
 \hline
 \hline
 \multicolumn{5}{|c|}{Race (FP32 Accuracy = 40.67\%)} & \multicolumn{4}{|c|}{Boolq (FP32 Accuracy = 76.54\%)} \\ 
 \hline
 \hline
 64 & 40.48 & 40.10 & 39.43 & 39.90 & 75.41 & 75.11 & 77.09 & 75.66 \\
 \hline
 32 & 39.52 & 39.52 & 40.77 & 39.62 & 76.02 & 76.02 & 75.96 & 75.35  \\
 \hline
 16 & 39.81 & 39.71 & 39.90 & 40.38 & 75.05 & 73.82 & 75.72 & 76.09  \\
 \hline
 \hline
 \multicolumn{5}{|c|}{Winogrande (FP32 Accuracy = 70.64\%)} & \multicolumn{4}{|c|}{Piqa (FP32 Accuracy = 79.16\%)} \\ 
 \hline
 \hline
 64 & 69.14 & 70.17 & 70.17 & 70.56 & 78.24 & 79.00 & 78.62 & 78.73 \\
 \hline
 32 & 70.96 & 69.69 & 71.27 & 69.30 & 78.56 & 79.49 & 79.16 & 78.89  \\
 \hline
 16 & 71.03 & 69.53 & 69.69 & 70.40 & 78.13 & 79.16 & 79.00 & 79.00  \\
 \hline
\end{tabular}
\caption{\label{tab:mmlu_abalation} Accuracy on LM evaluation harness tasks on GPT3-22B model.}
\end{table}

\begin{table} \centering
\begin{tabular}{|c||c|c|c|c||c|c|c|c|} 
\hline
 $L_b \rightarrow$& \multicolumn{4}{c||}{8} & \multicolumn{4}{c||}{8}\\
 \hline
 \backslashbox{$L_A$\kern-1em}{\kern-1em$N_c$} & 2 & 4 & 8 & 16 & 2 & 4 & 8 & 16  \\
 %$N_c \rightarrow$ & 2 & 4 & 8 & 16 & 2 & 4 & 2 \\
 \hline
 \hline
 \multicolumn{5}{|c|}{Race (FP32 Accuracy = 44.4\%)} & \multicolumn{4}{|c|}{Boolq (FP32 Accuracy = 79.29\%)} \\ 
 \hline
 \hline
 64 & 42.49 & 42.51 & 42.58 & 43.45 & 77.58 & 77.37 & 77.43 & 78.1 \\
 \hline
 32 & 43.35 & 42.49 & 43.64 & 43.73 & 77.86 & 75.32 & 77.28 & 77.86  \\
 \hline
 16 & 44.21 & 44.21 & 43.64 & 42.97 & 78.65 & 77 & 76.94 & 77.98  \\
 \hline
 \hline
 \multicolumn{5}{|c|}{Winogrande (FP32 Accuracy = 69.38\%)} & \multicolumn{4}{|c|}{Piqa (FP32 Accuracy = 78.07\%)} \\ 
 \hline
 \hline
 64 & 68.9 & 68.43 & 69.77 & 68.19 & 77.09 & 76.82 & 77.09 & 77.86 \\
 \hline
 32 & 69.38 & 68.51 & 68.82 & 68.90 & 78.07 & 76.71 & 78.07 & 77.86  \\
 \hline
 16 & 69.53 & 67.09 & 69.38 & 68.90 & 77.37 & 77.8 & 77.91 & 77.69  \\
 \hline
\end{tabular}
\caption{\label{tab:mmlu_abalation} Accuracy on LM evaluation harness tasks on Llama2-7B model.}
\end{table}

\begin{table} \centering
\begin{tabular}{|c||c|c|c|c||c|c|c|c|} 
\hline
 $L_b \rightarrow$& \multicolumn{4}{c||}{8} & \multicolumn{4}{c||}{8}\\
 \hline
 \backslashbox{$L_A$\kern-1em}{\kern-1em$N_c$} & 2 & 4 & 8 & 16 & 2 & 4 & 8 & 16  \\
 %$N_c \rightarrow$ & 2 & 4 & 8 & 16 & 2 & 4 & 2 \\
 \hline
 \hline
 \multicolumn{5}{|c|}{Race (FP32 Accuracy = 48.8\%)} & \multicolumn{4}{|c|}{Boolq (FP32 Accuracy = 85.23\%)} \\ 
 \hline
 \hline
 64 & 49.00 & 49.00 & 49.28 & 48.71 & 82.82 & 84.28 & 84.03 & 84.25 \\
 \hline
 32 & 49.57 & 48.52 & 48.33 & 49.28 & 83.85 & 84.46 & 84.31 & 84.93  \\
 \hline
 16 & 49.85 & 49.09 & 49.28 & 48.99 & 85.11 & 84.46 & 84.61 & 83.94  \\
 \hline
 \hline
 \multicolumn{5}{|c|}{Winogrande (FP32 Accuracy = 79.95\%)} & \multicolumn{4}{|c|}{Piqa (FP32 Accuracy = 81.56\%)} \\ 
 \hline
 \hline
 64 & 78.77 & 78.45 & 78.37 & 79.16 & 81.45 & 80.69 & 81.45 & 81.5 \\
 \hline
 32 & 78.45 & 79.01 & 78.69 & 80.66 & 81.56 & 80.58 & 81.18 & 81.34  \\
 \hline
 16 & 79.95 & 79.56 & 79.79 & 79.72 & 81.28 & 81.66 & 81.28 & 80.96  \\
 \hline
\end{tabular}
\caption{\label{tab:mmlu_abalation} Accuracy on LM evaluation harness tasks on Llama2-70B model.}
\end{table}

%\section{MSE Studies}
%\textcolor{red}{TODO}


\subsection{Number Formats and Quantization Method}
\label{subsec:numFormats_quantMethod}
\subsubsection{Integer Format}
An $n$-bit signed integer (INT) is typically represented with a 2s-complement format \citep{yao2022zeroquant,xiao2023smoothquant,dai2021vsq}, where the most significant bit denotes the sign.

\subsubsection{Floating Point Format}
An $n$-bit signed floating point (FP) number $x$ comprises of a 1-bit sign ($x_{\mathrm{sign}}$), $B_m$-bit mantissa ($x_{\mathrm{mant}}$) and $B_e$-bit exponent ($x_{\mathrm{exp}}$) such that $B_m+B_e=n-1$. The associated constant exponent bias ($E_{\mathrm{bias}}$) is computed as $(2^{{B_e}-1}-1)$. We denote this format as $E_{B_e}M_{B_m}$.  

\subsubsection{Quantization Scheme}
\label{subsec:quant_method}
A quantization scheme dictates how a given unquantized tensor is converted to its quantized representation. We consider FP formats for the purpose of illustration. Given an unquantized tensor $\bm{X}$ and an FP format $E_{B_e}M_{B_m}$, we first, we compute the quantization scale factor $s_X$ that maps the maximum absolute value of $\bm{X}$ to the maximum quantization level of the $E_{B_e}M_{B_m}$ format as follows:
\begin{align}
\label{eq:sf}
    s_X = \frac{\mathrm{max}(|\bm{X}|)}{\mathrm{max}(E_{B_e}M_{B_m})}
\end{align}
In the above equation, $|\cdot|$ denotes the absolute value function.

Next, we scale $\bm{X}$ by $s_X$ and quantize it to $\hat{\bm{X}}$ by rounding it to the nearest quantization level of $E_{B_e}M_{B_m}$ as:

\begin{align}
\label{eq:tensor_quant}
    \hat{\bm{X}} = \text{round-to-nearest}\left(\frac{\bm{X}}{s_X}, E_{B_e}M_{B_m}\right)
\end{align}

We perform dynamic max-scaled quantization \citep{wu2020integer}, where the scale factor $s$ for activations is dynamically computed during runtime.

\subsection{Vector Scaled Quantization}
\begin{wrapfigure}{r}{0.35\linewidth}
  \centering
  \includegraphics[width=\linewidth]{sections/figures/vsquant.jpg}
  \caption{\small Vectorwise decomposition for per-vector scaled quantization (VSQ \citep{dai2021vsq}).}
  \label{fig:vsquant}
\end{wrapfigure}
During VSQ \citep{dai2021vsq}, the operand tensors are decomposed into 1D vectors in a hardware friendly manner as shown in Figure \ref{fig:vsquant}. Since the decomposed tensors are used as operands in matrix multiplications during inference, it is beneficial to perform this decomposition along the reduction dimension of the multiplication. The vectorwise quantization is performed similar to tensorwise quantization described in Equations \ref{eq:sf} and \ref{eq:tensor_quant}, where a scale factor $s_v$ is required for each vector $\bm{v}$ that maps the maximum absolute value of that vector to the maximum quantization level. While smaller vector lengths can lead to larger accuracy gains, the associated memory and computational overheads due to the per-vector scale factors increases. To alleviate these overheads, VSQ \citep{dai2021vsq} proposed a second level quantization of the per-vector scale factors to unsigned integers, while MX \citep{rouhani2023shared} quantizes them to integer powers of 2 (denoted as $2^{INT}$).

\subsubsection{MX Format}
The MX format proposed in \citep{rouhani2023microscaling} introduces the concept of sub-block shifting. For every two scalar elements of $b$-bits each, there is a shared exponent bit. The value of this exponent bit is determined through an empirical analysis that targets minimizing quantization MSE. We note that the FP format $E_{1}M_{b}$ is strictly better than MX from an accuracy perspective since it allocates a dedicated exponent bit to each scalar as opposed to sharing it across two scalars. Therefore, we conservatively bound the accuracy of a $b+2$-bit signed MX format with that of a $E_{1}M_{b}$ format in our comparisons. For instance, we use E1M2 format as a proxy for MX4.

\begin{figure}
    \centering
    \includegraphics[width=1\linewidth]{sections//figures/BlockFormats.pdf}
    \caption{\small Comparing LO-BCQ to MX format.}
    \label{fig:block_formats}
\end{figure}

Figure \ref{fig:block_formats} compares our $4$-bit LO-BCQ block format to MX \citep{rouhani2023microscaling}. As shown, both LO-BCQ and MX decompose a given operand tensor into block arrays and each block array into blocks. Similar to MX, we find that per-block quantization ($L_b < L_A$) leads to better accuracy due to increased flexibility. While MX achieves this through per-block $1$-bit micro-scales, we associate a dedicated codebook to each block through a per-block codebook selector. Further, MX quantizes the per-block array scale-factor to E8M0 format without per-tensor scaling. In contrast during LO-BCQ, we find that per-tensor scaling combined with quantization of per-block array scale-factor to E4M3 format results in superior inference accuracy across models. 


%%%%%%%%%%%%%%%%%%%%%%%%%%%%%%%%%%%%%%%%%%%%%%%%%%%%%%%%%%%%%%%%%%%%%%%%%%%%%%%
%%%%%%%%%%%%%%%%%%%%%%%%%%%%%%%%%%%%%%%%%%%%%%%%%%%%%%%%%%%%%%%%%%%%%%%%%%%%%%%


\end{document}


% This document was modified from the file originally made available by
% Pat Langley and Andrea Danyluk for ICML-2K. This version was created
% by Iain Murray in 2018, and modified by Alexandre Bouchard in
% 2019 and 2021 and by Csaba Szepesvari, Gang Niu and Sivan Sabato in 2022.
% Modified again in 2023 and 2024 by Sivan Sabato and Jonathan Scarlett.
% Previous contributors include Dan Roy, Lise Getoor and Tobias
% Scheffer, which was slightly modified from the 2010 version by
% Thorsten Joachims & Johannes Fuernkranz, slightly modified from the
% 2009 version by Kiri Wagstaff and Sam Roweis's 2008 version, which is
% slightly modified from Prasad Tadepalli's 2007 version which is a
% lightly changed version of the previous year's version by Andrew
% Moore, which was in turn edited from those of Kristian Kersting and
% Codrina Lauth. Alex Smola contributed to the algorithmic style files.
