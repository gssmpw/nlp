\documentclass{article}

% \usepackage{neurips_2024}
\usepackage[utf8]{inputenc}
\usepackage[T1]{fontenc} 

\usepackage{times}
\usepackage{color}
\usepackage{fullpage}
\usepackage{graphicx}
\usepackage{epstopdf}
\usepackage{mathtools}
\usepackage{amsthm, amssymb}
\usepackage[colorlinks,linkcolor=blue,citecolor=blue,urlcolor=DarkBlue]{hyperref}
\usepackage[ruled, vlined, linesnumbered]{algorithm2e}
\usepackage{caption, subcaption}
\usepackage{enumitem}
\usepackage{comment}
\usepackage{placeins}
\usepackage{microtype}
\usepackage{makecell}
\usepackage{multirow}
\usepackage{booktabs}

\newcommand{\abs}[1]{\left| #1 \right|}
\newcommand{\IC}{\mathbf{H}}
\newcommand{\E}{\mathcal{E}}
\newcommand{\N}{\mathbb{N}}

\usepackage{tcolorbox}
\tcbuselibrary{skins,breakable}
\tcbset{enhanced jigsaw}

\definecolor{DarkRed}{rgb}{0.5,0.1,0.1}
\definecolor{DarkBlue}{rgb}{0.1,0.1,0.5}
\definecolor{RURed}{rgb}{0.8,0.1,0.1}
\definecolor{ForestGreen}{rgb}{0.1333,0.5451,0.1333}
%\definecolor{DarkRed}{rgb}{0.8,0,0}
\definecolor{Red}{rgb}{0.9,0,0}

\usepackage{nameref}
\usepackage[noabbrev,nameinlink]{cleveref}
\crefname{property}{property}{Property}
\creflabelformat{property}{(#1)#2#3}
\crefname{equation}{eq}{Eq}
\creflabelformat{equation}{(#1)#2#3}

\newtheorem{theorem}{Theorem}
\newtheorem{lemma}{Lemma}[section]
% \newtheorem{comment}[theorem]{Comment}
\newtheorem{claim}[lemma]{Claim}
\newtheorem{corollary}{Corollary}
\newtheorem{proposition}{Proposition}
\newtheorem{fact}[lemma]{Fact}

\theoremstyle{definition}
\newtheorem{note}{Note}
\newtheorem{observation}[lemma]{Observation}
\newtheorem{problem}[theorem]{Problem}
\newtheorem{conjecture}[theorem]{Conjecture}
\newtheorem{remark}[lemma]{Remark}
\newtheorem{question}[lemma]{Question}
\newtheorem{definition}{Definition}
\newtheorem{property}[lemma]{Property}

\usepackage{mdframed}
\newtheorem{mdresult}{Result}
\newenvironment{result}{\begin{mdframed}[backgroundcolor=lightgray!40,topline=false,rightline=false,leftline=false,bottomline=false,innertopmargin=5pt]\begin{mdresult}}{\end{mdresult}\end{mdframed}}



\definecolor{RURed}{rgb}{0.8,0.1,0.1}
\newcommand{\chen}[1]{\textcolor{RURed}{[\textbf{Chen:} #1]}}

\newcommand{\nicksays}[1]{\textcolor{RURed}{[\textbf{Nick:} #1]}}



%
\setlength\unitlength{1mm}
\newcommand{\twodots}{\mathinner {\ldotp \ldotp}}
% bb font symbols
\newcommand{\Rho}{\mathrm{P}}
\newcommand{\Tau}{\mathrm{T}}

\newfont{\bbb}{msbm10 scaled 700}
\newcommand{\CCC}{\mbox{\bbb C}}

\newfont{\bb}{msbm10 scaled 1100}
\newcommand{\CC}{\mbox{\bb C}}
\newcommand{\PP}{\mbox{\bb P}}
\newcommand{\RR}{\mbox{\bb R}}
\newcommand{\QQ}{\mbox{\bb Q}}
\newcommand{\ZZ}{\mbox{\bb Z}}
\newcommand{\FF}{\mbox{\bb F}}
\newcommand{\GG}{\mbox{\bb G}}
\newcommand{\EE}{\mbox{\bb E}}
\newcommand{\NN}{\mbox{\bb N}}
\newcommand{\KK}{\mbox{\bb K}}
\newcommand{\HH}{\mbox{\bb H}}
\newcommand{\SSS}{\mbox{\bb S}}
\newcommand{\UU}{\mbox{\bb U}}
\newcommand{\VV}{\mbox{\bb V}}


\newcommand{\yy}{\mathbbm{y}}
\newcommand{\xx}{\mathbbm{x}}
\newcommand{\zz}{\mathbbm{z}}
\newcommand{\sss}{\mathbbm{s}}
\newcommand{\rr}{\mathbbm{r}}
\newcommand{\pp}{\mathbbm{p}}
\newcommand{\qq}{\mathbbm{q}}
\newcommand{\ww}{\mathbbm{w}}
\newcommand{\hh}{\mathbbm{h}}
\newcommand{\vvv}{\mathbbm{v}}

% Vectors

\newcommand{\av}{{\bf a}}
\newcommand{\bv}{{\bf b}}
\newcommand{\cv}{{\bf c}}
\newcommand{\dv}{{\bf d}}
\newcommand{\ev}{{\bf e}}
\newcommand{\fv}{{\bf f}}
\newcommand{\gv}{{\bf g}}
\newcommand{\hv}{{\bf h}}
\newcommand{\iv}{{\bf i}}
\newcommand{\jv}{{\bf j}}
\newcommand{\kv}{{\bf k}}
\newcommand{\lv}{{\bf l}}
\newcommand{\mv}{{\bf m}}
\newcommand{\nv}{{\bf n}}
\newcommand{\ov}{{\bf o}}
\newcommand{\pv}{{\bf p}}
\newcommand{\qv}{{\bf q}}
\newcommand{\rv}{{\bf r}}
\newcommand{\sv}{{\bf s}}
\newcommand{\tv}{{\bf t}}
\newcommand{\uv}{{\bf u}}
\newcommand{\wv}{{\bf w}}
\newcommand{\vv}{{\bf v}}
\newcommand{\xv}{{\bf x}}
\newcommand{\yv}{{\bf y}}
\newcommand{\zv}{{\bf z}}
\newcommand{\zerov}{{\bf 0}}
\newcommand{\onev}{{\bf 1}}

% Matrices

\newcommand{\Am}{{\bf A}}
\newcommand{\Bm}{{\bf B}}
\newcommand{\Cm}{{\bf C}}
\newcommand{\Dm}{{\bf D}}
\newcommand{\Em}{{\bf E}}
\newcommand{\Fm}{{\bf F}}
\newcommand{\Gm}{{\bf G}}
\newcommand{\Hm}{{\bf H}}
\newcommand{\Id}{{\bf I}}
\newcommand{\Jm}{{\bf J}}
\newcommand{\Km}{{\bf K}}
\newcommand{\Lm}{{\bf L}}
\newcommand{\Mm}{{\bf M}}
\newcommand{\Nm}{{\bf N}}
\newcommand{\Om}{{\bf O}}
\newcommand{\Pm}{{\bf P}}
\newcommand{\Qm}{{\bf Q}}
\newcommand{\Rm}{{\bf R}}
\newcommand{\Sm}{{\bf S}}
\newcommand{\Tm}{{\bf T}}
\newcommand{\Um}{{\bf U}}
\newcommand{\Wm}{{\bf W}}
\newcommand{\Vm}{{\bf V}}
\newcommand{\Xm}{{\bf X}}
\newcommand{\Ym}{{\bf Y}}
\newcommand{\Zm}{{\bf Z}}

% Calligraphic

\newcommand{\Ac}{{\cal A}}
\newcommand{\Bc}{{\cal B}}
\newcommand{\Cc}{{\cal C}}
\newcommand{\Dc}{{\cal D}}
\newcommand{\Ec}{{\cal E}}
\newcommand{\Fc}{{\cal F}}
\newcommand{\Gc}{{\cal G}}
\newcommand{\Hc}{{\cal H}}
\newcommand{\Ic}{{\cal I}}
\newcommand{\Jc}{{\cal J}}
\newcommand{\Kc}{{\cal K}}
\newcommand{\Lc}{{\cal L}}
\newcommand{\Mc}{{\cal M}}
\newcommand{\Nc}{{\cal N}}
\newcommand{\nc}{{\cal n}}
\newcommand{\Oc}{{\cal O}}
\newcommand{\Pc}{{\cal P}}
\newcommand{\Qc}{{\cal Q}}
\newcommand{\Rc}{{\cal R}}
\newcommand{\Sc}{{\cal S}}
\newcommand{\Tc}{{\cal T}}
\newcommand{\Uc}{{\cal U}}
\newcommand{\Wc}{{\cal W}}
\newcommand{\Vc}{{\cal V}}
\newcommand{\Xc}{{\cal X}}
\newcommand{\Yc}{{\cal Y}}
\newcommand{\Zc}{{\cal Z}}

% Bold greek letters

\newcommand{\alphav}{\hbox{\boldmath$\alpha$}}
\newcommand{\betav}{\hbox{\boldmath$\beta$}}
\newcommand{\gammav}{\hbox{\boldmath$\gamma$}}
\newcommand{\deltav}{\hbox{\boldmath$\delta$}}
\newcommand{\etav}{\hbox{\boldmath$\eta$}}
\newcommand{\lambdav}{\hbox{\boldmath$\lambda$}}
\newcommand{\epsilonv}{\hbox{\boldmath$\epsilon$}}
\newcommand{\nuv}{\hbox{\boldmath$\nu$}}
\newcommand{\muv}{\hbox{\boldmath$\mu$}}
\newcommand{\zetav}{\hbox{\boldmath$\zeta$}}
\newcommand{\phiv}{\hbox{\boldmath$\phi$}}
\newcommand{\psiv}{\hbox{\boldmath$\psi$}}
\newcommand{\thetav}{\hbox{\boldmath$\theta$}}
\newcommand{\tauv}{\hbox{\boldmath$\tau$}}
\newcommand{\omegav}{\hbox{\boldmath$\omega$}}
\newcommand{\xiv}{\hbox{\boldmath$\xi$}}
\newcommand{\sigmav}{\hbox{\boldmath$\sigma$}}
\newcommand{\piv}{\hbox{\boldmath$\pi$}}
\newcommand{\rhov}{\hbox{\boldmath$\rho$}}
\newcommand{\upsilonv}{\hbox{\boldmath$\upsilon$}}

\newcommand{\Gammam}{\hbox{\boldmath$\Gamma$}}
\newcommand{\Lambdam}{\hbox{\boldmath$\Lambda$}}
\newcommand{\Deltam}{\hbox{\boldmath$\Delta$}}
\newcommand{\Sigmam}{\hbox{\boldmath$\Sigma$}}
\newcommand{\Phim}{\hbox{\boldmath$\Phi$}}
\newcommand{\Pim}{\hbox{\boldmath$\Pi$}}
\newcommand{\Psim}{\hbox{\boldmath$\Psi$}}
\newcommand{\Thetam}{\hbox{\boldmath$\Theta$}}
\newcommand{\Omegam}{\hbox{\boldmath$\Omega$}}
\newcommand{\Xim}{\hbox{\boldmath$\Xi$}}


% Sans Serif small case

\newcommand{\Gsf}{{\sf G}}

\newcommand{\asf}{{\sf a}}
\newcommand{\bsf}{{\sf b}}
\newcommand{\csf}{{\sf c}}
\newcommand{\dsf}{{\sf d}}
\newcommand{\esf}{{\sf e}}
\newcommand{\fsf}{{\sf f}}
\newcommand{\gsf}{{\sf g}}
\newcommand{\hsf}{{\sf h}}
\newcommand{\isf}{{\sf i}}
\newcommand{\jsf}{{\sf j}}
\newcommand{\ksf}{{\sf k}}
\newcommand{\lsf}{{\sf l}}
\newcommand{\msf}{{\sf m}}
\newcommand{\nsf}{{\sf n}}
\newcommand{\osf}{{\sf o}}
\newcommand{\psf}{{\sf p}}
\newcommand{\qsf}{{\sf q}}
\newcommand{\rsf}{{\sf r}}
\newcommand{\ssf}{{\sf s}}
\newcommand{\tsf}{{\sf t}}
\newcommand{\usf}{{\sf u}}
\newcommand{\wsf}{{\sf w}}
\newcommand{\vsf}{{\sf v}}
\newcommand{\xsf}{{\sf x}}
\newcommand{\ysf}{{\sf y}}
\newcommand{\zsf}{{\sf z}}


% mixed symbols

\newcommand{\sinc}{{\hbox{sinc}}}
\newcommand{\diag}{{\hbox{diag}}}
\renewcommand{\det}{{\hbox{det}}}
\newcommand{\trace}{{\hbox{tr}}}
\newcommand{\sign}{{\hbox{sign}}}
\renewcommand{\arg}{{\hbox{arg}}}
\newcommand{\var}{{\hbox{var}}}
\newcommand{\cov}{{\hbox{cov}}}
\newcommand{\Ei}{{\rm E}_{\rm i}}
\renewcommand{\Re}{{\rm Re}}
\renewcommand{\Im}{{\rm Im}}
\newcommand{\eqdef}{\stackrel{\Delta}{=}}
\newcommand{\defines}{{\,\,\stackrel{\scriptscriptstyle \bigtriangleup}{=}\,\,}}
\newcommand{\<}{\left\langle}
\renewcommand{\>}{\right\rangle}
\newcommand{\herm}{{\sf H}}
\newcommand{\trasp}{{\sf T}}
\newcommand{\transp}{{\sf T}}
\renewcommand{\vec}{{\rm vec}}
\newcommand{\Psf}{{\sf P}}
\newcommand{\SINR}{{\sf SINR}}
\newcommand{\SNR}{{\sf SNR}}
\newcommand{\MMSE}{{\sf MMSE}}
\newcommand{\REF}{{\RED [REF]}}

% Markov chain
\usepackage{stmaryrd} % for \mkv 
\newcommand{\mkv}{-\!\!\!\!\minuso\!\!\!\!-}

% Colors

\newcommand{\RED}{\color[rgb]{1.00,0.10,0.10}}
\newcommand{\BLUE}{\color[rgb]{0,0,0.90}}
\newcommand{\GREEN}{\color[rgb]{0,0.80,0.20}}

%%%%%%%%%%%%%%%%%%%%%%%%%%%%%%%%%%%%%%%%%%
\usepackage{hyperref}
\hypersetup{
    bookmarks=true,         % show bookmarks bar?
    unicode=false,          % non-Latin characters in AcrobatÕs bookmarks
    pdftoolbar=true,        % show AcrobatÕs toolbar?
    pdfmenubar=true,        % show AcrobatÕs menu?
    pdffitwindow=false,     % window fit to page when opened
    pdfstartview={FitH},    % fits the width of the page to the window
%    pdftitle={My title},    % title
%    pdfauthor={Author},     % author
%    pdfsubject={Subject},   % subject of the document
%    pdfcreator={Creator},   % creator of the document
%    pdfproducer={Producer}, % producer of the document
%    pdfkeywords={keyword1} {key2} {key3}, % list of keywords
    pdfnewwindow=true,      % links in new window
    colorlinks=true,       % false: boxed links; true: colored links
    linkcolor=red,          % color of internal links (change box color with linkbordercolor)
    citecolor=green,        % color of links to bibliography
    filecolor=blue,      % color of file links
    urlcolor=blue           % color of external links
}
%%%%%%%%%%%%%%%%%%%%%%%%%%%%%%%%%%%%%%%%%%%


\title{Nearly Tight Bounds for Exploration in Streaming Multi-armed Bandits with Known Optimality Gap}
%\title{Nearly Tight Bounds for Nearly Instance Optimal Exploration in Streaming Multi-armed Bandits with Known Optimality Gap}
%\title{Memory-pass Bounds for Nearly Instance Optimal Exploration in Streaming Multi-armed Bandits with Known Optimality Gap}
%\title{Memory-pass Bounds in Streaming Multi-armed Bandits with Known Optimality Gap and Nearly Instance-Optimal Sample Complexity}
\author{Nikolai Karpov \thanks{Indiana University. \texttt{email:~kimaska@gmail.com}}
\and 
Chen Wang\thanks{Rice University and Texas A\&M University. \texttt{email:~cwangjhw@tamu.edu}}
}
\date{} 

\begin{document}
\maketitle
% \date{} 

%\begin{abstract}
	We study algorithms for pure exploration in multi-pass streaming Multi-armed Bandits (MABs) with the \emph{a priori} knowledge of the optimality gap $\Delta_{[2]}$. The problem is formlulated as follows: given $n$ arms with unkonwn sub-Gaussian reward distributions, find the best arm, defined as the arm with the highest mean reward, with a sufficiently high probability. Here, the parameter $\Delta_{[i]}$ is defined as the mean gap between the best and the $i$-th best arms. In the multi-pass streaming model, the arms arrive one after another in a stream, and the target of the algorithm is to simulaneously minimize the sample complexity -- the total number of arm pulls -- and the space complexity -- the maximum number of arms ever stored. 
	
	The (nearly-)instance optimal sample complexity bounds of $\tilde{\Theta}(\sum_{i=2}^{n}(1/\Delta^2_{[i]}))$ are known for almost a decade in the offline setting, i.e., the algorithm is allowed to store all arms. The recent results by JHTX [ICML'21] and AW [Arxiv'23] have shown that if no additional information is provided besides the stream, any algorithm to find the best arm with an $o(n)$ arm memory and the $\tilde{O}(\sum_{i=2}^{n}(1/\Delta^2_{[i]}))$ sample complexity requires $\tilde{\Theta}(\log{\frac{1}{\Delta}})$ passes. This is in sharp contrast with the upper bound in AW [STOC'20], which shows that if the value of $\Delta_{[2]}$ is known \emph{a priori}, there exists an algorithm that finds the best arm with a single-arm memory, a single pass over the stream, and a \emph{worst-case optimal} $O(n/\Delta^2_{[2]})$ sample complexity. As such, a natural question araises: in the multi-pass setting with a known $\Delta_{[2]}$, what are the optimal passes and memory bounds for the $\tilde{O}(\sum_{i=2}^{n}(1/\Delta^2_{[i]}))$ sample complexity?
	
	We answer the question in this paper by providing nearly matching pass bounds for streaming algorithms with $o(n)$ arm memory. Concretely, we first show an algorithm that finds the best arm with a high constant probability with a \emph{single-arm} memory, $O(\log{n})$ passes, and $O(\sum_{i=2}^{n}(1/\Delta^2_{[i]}) \cdot \log{n})$ arm pulls. We then present a lower bound, showing that any algorithm that finds the best arm with slightly sublinear memory -- a memory of $o({n}/{\polylog{n}})$ arms -- and $O(\sum_{i=2}^{n}{1}/{\Deltai^{2}}\cdot \log{n})$ arm pulls has to make $\Omega(\frac{\log{n}}{\log\log{n}})$ passes over the stream. The upper and lower bounds form a sharp dichotomy in terms of the memory, and the pass bounds match up an exponentially smalll factor.
\end{abstract}
\begin{abstract}
	We investigate the sample-memory-pass trade-offs for pure exploration in multi-pass streaming multi-armed bandits (MABs) with the \emph{a priori} knowledge of the optimality gap $\Delta_{[2]}$. 
	Here, and throughout, the optimality gap $\Delta_{[i]}$ is defined as the mean reward gap between the best and the $i$-th best arms. A recent line of results by Jin, Huang, Tang, and Xiao [ICML'21] and Assadi and Wang [COLT'24] have shown that if there is no known $\Delta_{[2]}$, a pass complexity of $\Theta(\log(1/\Delta_{[2]}))$ (up to $\log\log(1/\Delta_{[2]})$ terms) is necessary and sufficient to obtain the \emph{worst-case optimal} sample complexity of $O(n/\Delta^{2}_{[2]})$ with a single-arm memory. However, our understanding of multi-pass algorithms with known $\Delta_{[2]}$ is still limited. Here, the key open problem is how many passes are required to achieve the complexity, i.e., $O( \sum_{i=2}^{n}1/\Delta^2_{[i]})$ arm pulls, with a sublinear memory size.
	
	In this work, we show that the ``right answer'' for the question is $\Theta(\log{n})$ passes (up to $\log\log{n}$ terms).  We first present a lower bound, showing that any algorithm that finds the best arm with slightly sublinear memory -- a memory of $o({n}/{\polylog({n})})$ arms -- and $O(\sum_{i=2}^{n}{1}/{\Deltai^{2}}\cdot \log{(n)})$ arm pulls has to make $\Omega(\frac{\log{n}}{\log\log{n}})$ passes over the stream. We then show a nearly-matching algorithm that assuming the knowledge of $\Delta_{[2]}$, finds the best arm with $O( \sum_{i=2}^{n}1/\Delta^2_{[i]} \cdot \log{n})$ arm pulls and a \emph{single arm} memory.
\end{abstract}

% \chen{remember to anonymize for conference submission}
%\chen{Technical roadmap for our lower bound:
%\begin{enumerate}
%	\item We first show that for a $2$-arm instance, either $i).$ both of them are just with mean reward $1/2$ or $ii).$ one arm is with reward $1/2+\alpha$ and the other is with reward $1/2+\alpha+\beta$, there is
%		 \begin{enumerate}
%		 	\item Distinguishing whether the arm is from the yes or no cases takes $\Omega(\eps^2/(\alpha+\beta)^2)$ arm pulls.
%		 	\item To ``learn'' with advantage $\eps$ from the original distribution, it takes $\Omega(\eps^3/(\alpha+\beta)^2)$ arm pulls.
%		 \end{enumerate}
%	\item For a `batch' of $n/(P+1)$ arms with $a).$ all but two special arms chosen uniformly at random are with mean reward $\frac{1}{2}$; and $b).$ the two special arms are either with reward $\frac{1}{2}$ or with rewards $1/2+\alpha$ and $1/2+\alpha+\beta$, there is
%	\begin{enumerate}
%		\item Conditioning on $1/2+\alpha$ and $1/2+\alpha+\beta$ arms exist, storing any arm with reward $>1/2$ takes $\Omega(n \cdot \eps^2/P^3 (\alpha+\beta)^2)$ arm pulls (if the memory is $o(n/(P+1))$).
%		\item To ``learn'' with advantage $\eps$ from the original distribution, it takes $\Omega(n \cdot \eps^3/P^3 (\alpha+\beta)^2)$ arm pulls.
%	\end{enumerate}
%	\item We want to argue that the algorithm should only pay $\frac{n}{\poly(P)\cdot (\alpha_{p}+\beta)^2}$ in pass $p$ to keep the sample complexity at most $O(H_{2}\cdot \polylog{(n)})$. The reason for this is that $\alpha_{p}>>\beta$ for any $p\leq \frac{\log{n}}{\log\log{n}}$ -- by our choice, we have $\alpha_{1}=n^{1/3} \beta$, $\alpha_{i}=\frac{1}{\polylog(n)}\cdot \alpha_{1}$
%\end{enumerate}
%}



%\chen{Sep/5 What remains to be done:
%\begin{enumerate}
%	\item Write the technical overview of the ub + lb.
%	\item Read each other's technical parts and give comments.
%	\item Items 1 and 2 deadline is Sep/5. 
%	\item \textbf{For Chen:} Write a proof skecth for \Cref{prop:multi-pass-lb} in the appendix -- do this by Sep/7.
%	\item Notation for ``defining a notion'': $:=$ vs. $\triangleq$
%\end{enumerate}
%}
%
%\chen{Some MISC stuff:
%\begin{enumerate}
%	\item Appendix A: standard technical tools. Chernoff bounds should go there. Standard bounds to compare arms should also go there -- do this by Sep/6.
%	\item Add a footnote to explain what is $\tilde{O}$ (in particular, we hide both $\log(n)$ abd $\log(1/\Delta_{[i]})$ terms).
%	\item Our algorithms work with a \emph{lower bound} estimation of $\tilde{\Delta}\leq \Delta_{[2]}$ and we only pay the overhead of $1/\tilde{\Delta}^2$.
%	\item Have a saparate paragraph for the additional notation in the lower bound I section.
%\end{enumerate}
%}


\clearpage

%%
\section{Introduction}


\begin{figure}[t]
\centering
\includegraphics[width=0.6\columnwidth]{figures/evaluation_desiderata_V5.pdf}
\vspace{-0.5cm}
\caption{\systemName is a platform for conducting realistic evaluations of code LLMs, collecting human preferences of coding models with real users, real tasks, and in realistic environments, aimed at addressing the limitations of existing evaluations.
}
\label{fig:motivation}
\end{figure}

\begin{figure*}[t]
\centering
\includegraphics[width=\textwidth]{figures/system_design_v2.png}
\caption{We introduce \systemName, a VSCode extension to collect human preferences of code directly in a developer's IDE. \systemName enables developers to use code completions from various models. The system comprises a) the interface in the user's IDE which presents paired completions to users (left), b) a sampling strategy that picks model pairs to reduce latency (right, top), and c) a prompting scheme that allows diverse LLMs to perform code completions with high fidelity.
Users can select between the top completion (green box) using \texttt{tab} or the bottom completion (blue box) using \texttt{shift+tab}.}
\label{fig:overview}
\end{figure*}

As model capabilities improve, large language models (LLMs) are increasingly integrated into user environments and workflows.
For example, software developers code with AI in integrated developer environments (IDEs)~\citep{peng2023impact}, doctors rely on notes generated through ambient listening~\citep{oberst2024science}, and lawyers consider case evidence identified by electronic discovery systems~\citep{yang2024beyond}.
Increasing deployment of models in productivity tools demands evaluation that more closely reflects real-world circumstances~\citep{hutchinson2022evaluation, saxon2024benchmarks, kapoor2024ai}.
While newer benchmarks and live platforms incorporate human feedback to capture real-world usage, they almost exclusively focus on evaluating LLMs in chat conversations~\citep{zheng2023judging,dubois2023alpacafarm,chiang2024chatbot, kirk2024the}.
Model evaluation must move beyond chat-based interactions and into specialized user environments.



 

In this work, we focus on evaluating LLM-based coding assistants. 
Despite the popularity of these tools---millions of developers use Github Copilot~\citep{Copilot}---existing
evaluations of the coding capabilities of new models exhibit multiple limitations (Figure~\ref{fig:motivation}, bottom).
Traditional ML benchmarks evaluate LLM capabilities by measuring how well a model can complete static, interview-style coding tasks~\citep{chen2021evaluating,austin2021program,jain2024livecodebench, white2024livebench} and lack \emph{real users}. 
User studies recruit real users to evaluate the effectiveness of LLMs as coding assistants, but are often limited to simple programming tasks as opposed to \emph{real tasks}~\citep{vaithilingam2022expectation,ross2023programmer, mozannar2024realhumaneval}.
Recent efforts to collect human feedback such as Chatbot Arena~\citep{chiang2024chatbot} are still removed from a \emph{realistic environment}, resulting in users and data that deviate from typical software development processes.
We introduce \systemName to address these limitations (Figure~\ref{fig:motivation}, top), and we describe our three main contributions below.


\textbf{We deploy \systemName in-the-wild to collect human preferences on code.} 
\systemName is a Visual Studio Code extension, collecting preferences directly in a developer's IDE within their actual workflow (Figure~\ref{fig:overview}).
\systemName provides developers with code completions, akin to the type of support provided by Github Copilot~\citep{Copilot}. 
Over the past 3 months, \systemName has served over~\completions suggestions from 10 state-of-the-art LLMs, 
gathering \sampleCount~votes from \userCount~users.
To collect user preferences,
\systemName presents a novel interface that shows users paired code completions from two different LLMs, which are determined based on a sampling strategy that aims to 
mitigate latency while preserving coverage across model comparisons.
Additionally, we devise a prompting scheme that allows a diverse set of models to perform code completions with high fidelity.
See Section~\ref{sec:system} and Section~\ref{sec:deployment} for details about system design and deployment respectively.



\textbf{We construct a leaderboard of user preferences and find notable differences from existing static benchmarks and human preference leaderboards.}
In general, we observe that smaller models seem to overperform in static benchmarks compared to our leaderboard, while performance among larger models is mixed (Section~\ref{sec:leaderboard_calculation}).
We attribute these differences to the fact that \systemName is exposed to users and tasks that differ drastically from code evaluations in the past. 
Our data spans 103 programming languages and 24 natural languages as well as a variety of real-world applications and code structures, while static benchmarks tend to focus on a specific programming and natural language and task (e.g. coding competition problems).
Additionally, while all of \systemName interactions contain code contexts and the majority involve infilling tasks, a much smaller fraction of Chatbot Arena's coding tasks contain code context, with infilling tasks appearing even more rarely. 
We analyze our data in depth in Section~\ref{subsec:comparison}.



\textbf{We derive new insights into user preferences of code by analyzing \systemName's diverse and distinct data distribution.}
We compare user preferences across different stratifications of input data (e.g., common versus rare languages) and observe which affect observed preferences most (Section~\ref{sec:analysis}).
For example, while user preferences stay relatively consistent across various programming languages, they differ drastically between different task categories (e.g. frontend/backend versus algorithm design).
We also observe variations in user preference due to different features related to code structure 
(e.g., context length and completion patterns).
We open-source \systemName and release a curated subset of code contexts.
Altogether, our results highlight the necessity of model evaluation in realistic and domain-specific settings.






%%
\section{Preliminaries}
\label{sec:prelim}
\label{sec:term}
We define the key terminologies used, primarily focusing on the hidden states (or activations) during the forward pass. 

\paragraph{Components in an attention layer.} We denote $\Res$ as the residual stream. We denote $\Val$ as Value (states), $\Qry$ as Query (states), and $\Key$ as Key (states) in one attention head. The \attlogit~represents the value before the softmax operation and can be understood as the inner product between  $\Qry$  and  $\Key$. We use \Attn~to denote the attention weights of applying the SoftMax function to \attlogit, and ``attention map'' to describe the visualization of the heat map of the attention weights. When referring to the \attlogit~from ``$\tokenB$'' to  ``$\tokenA$'', we indicate the inner product  $\langle\Qry(\tokenB), \Key(\tokenA)\rangle$, specifically the entry in the ``$\tokenB$'' row and ``$\tokenA$'' column of the attention map.

\paragraph{Logit lens.} We use the method of ``Logit Lens'' to interpret the hidden states and value states \citep{belrose2023eliciting}. We use \logit~to denote pre-SoftMax values of the next-token prediction for LLMs. Denote \readout~as the linear operator after the last layer of transformers that maps the hidden states to the \logit. 
The logit lens is defined as applying the readout matrix to residual or value states in middle layers. Through the logit lens, the transformed hidden states can be interpreted as their direct effect on the logits for next-token prediction. 

\paragraph{Terminologies in two-hop reasoning.} We refer to an input like “\Src$\to$\brga, \brgb$\to$\Ed” as a two-hop reasoning chain, or simply a chain. The source entity $\Src$ serves as the starting point or origin of the reasoning. The end entity $\Ed$ represents the endpoint or destination of the reasoning chain. The bridge entity $\Brg$ connects the source and end entities within the reasoning chain. We distinguish between two occurrences of $\Brg$: the bridge in the first premise is called $\brga$, while the bridge in the second premise that connects to $\Ed$ is called $\brgc$. Additionally, for any premise ``$\tokenA \to \tokenB$'', we define $\tokenA$ as the parent node and $\tokenB$ as the child node. Furthermore, if at the end of the sequence, the query token is ``$\tokenA$'', we define the chain ``$\tokenA \to \tokenB$, $\tokenB \to \tokenC$'' as the Target Chain, while all other chains present in the context are referred to as distraction chains. Figure~\ref{fig:data_illustration} provides an illustration of the terminologies.

\paragraph{Input format.}
Motivated by two-hop reasoning in real contexts, we consider input in the format $\bos, \text{context information}, \query, \answer$. A transformer model is trained to predict the correct $\answer$ given the query $\query$ and the context information. The context compromises of $K=5$ disjoint two-hop chains, each appearing once and containing two premises. Within the same chain, the relative order of two premises is fixed so that \Src$\to$\brga~always precedes \brgb$\to$\Ed. The orders of chains are randomly generated, and chains may interleave with each other. The labels for the entities are re-shuffled for every sequence, choosing from a vocabulary size $V=30$. Given the $\bos$ token, $K=5$ two-hop chains, \query, and the \answer~tokens, the total context length is $N=23$. Figure~\ref{fig:data_illustration} also illustrates the data format. 

\paragraph{Model structure and training.} We pre-train a three-layer transformer with a single head per layer. Unless otherwise specified, the model is trained using Adam for $10,000$ steps, achieving near-optimal prediction accuracy. Details are relegated to Appendix~\ref{app:sec_add_training_detail}.


% \RZ{Do we use source entity, target entity, and mediator entity? Or do we use original token, bridge token, end token?}





% \paragraph{Basic notations.} We use ... We use $\ve_i$ to denote one-hot vectors of which only the $i$-th entry equals one, and all other entries are zero. The dimension of $\ve_i$ are usually omitted and can be inferred from contexts. We use $\indicator\{\cdot\}$ to denote the indicator function.

% Let $V > 0$ be a fixed positive integer, and let $\vocab = [V] \defeq \{1, 2, \ldots, V\}$ be the vocabulary. A token $v \in \vocab$ is an integer in $[V]$ and the input studied in this paper is a sequence of tokens $s_{1:T} \defeq (s_1, s_2, \ldots, s_T) \in \vocab^T$ of length $T$. For any set $\mathcal{S}$, we use $\Delta(\mathcal{S})$ to denote the set of distributions over $\mathcal{S}$.

% % to a sequence of vectors $z_1, z_2, \ldots, z_T \in \real^{\dout}$ of dimension $\dout$ and length $T$.

% Let $\mU = [\vu_1, \vu_2, \ldots, \vu_V]^\transpose \in \real^{V\times d}$ denote the token embedding matrix, where the $i$-th row $\vu_i \in \real^d$ represents the $d$-dimensional embedding of token $i \in [V]$. Similarly, let $\mP = [\vp_1, \vp_2, \ldots, \vp_T]^\transpose \in \real^{T\times d}$ denote the positional embedding matrix, where the $i$-th row $\vp_i \in \real^d$ represents the $d$-dimensional embedding of position $i \in [T]$. Both $\mU$ and $\mP$ can be fixed or learnable.

% After receiving an input sequence of tokens $s_{1:T}$, a transformer will first process it using embedding matrices $\mU$ and $\mP$ to obtain a sequence of vectors $\mH = [\vh_1, \vh_2, \ldots, \vh_T] \in \real^{d\times T}$, where 
% \[
% \vh_i = \mU^\transpose\ve_{s_i} + \mP^\transpose\ve_{i} = \vu_{s_i} + \vp_i.
% \]

% We make the following definitions of basic operations in a transformer.

% \begin{definition}[Basic operations in transformers] 
% \label{defn:operators}
% Define the softmax function $\softmax(\cdot): \real^d \to \real^d$ over a vector $\vv \in \real^d$ as
% \[\softmax(\vv)_i = \frac{\exp(\vv_i)}{\sum_{j=1}^d \exp(\vv_j)} \]
% and define the softmax function $\softmax(\cdot): \real^{m\times n} \to \real^{m \times n}$ over a matrix $\mV \in \real^{m\times n}$ as a column-wise softmax operator. For a squared matrix $\mM \in \real^{m\times m}$, the causal mask operator $\mask(\cdot): \real^{m\times m} \to \real^{m\times m}$  is defined as $\mask(\mM)_{ij} = \mM_{ij}$ if $i \leq j$ and  $\mask(\mM)_{ij} = -\infty$ otherwise. For a vector $\vv \in \real^n$ where $n$ is the number of hidden neurons in a layer, we use $\layernorm(\cdot): \real^n \to \real^n$ to denote the layer normalization operator where
% \[
% \layernorm(\vv)_i = \frac{\vv_i-\mu}{\sigma}, \mu = \frac{1}{n}\sum_{j=1}^n \vv_j, \sigma = \sqrt{\frac{1}{n}\sum_{j=1}^n (\vv_j-\mu)^2}
% \]
% and use $\layernorm(\cdot): \real^{n\times m} \to \real^{n\times m}$ to denote the column-wise layer normalization on a matrix.
% We also use $\nonlin(\cdot)$ to denote element-wise nonlinearity such as $\relu(\cdot)$.
% \end{definition}

% The main components of a transformer are causal self-attention heads and MLP layers, which are defined as follows.

% \begin{definition}[Attentions and MLPs]
% \label{defn:attn_mlp} 
% A single-head causal self-attention $\attn(\mH;\mQ,\mK,\mV,\mO)$ parameterized by $\mQ,\mK,\mV \in \real^{{\dqkv\times \din}}$ and $\mO \in \real^{\dout\times\dqkv}$ maps an input matrix $\mH \in \real^{\din\times T}$ to
% \begin{align*}
% &\attn(\mH;\mQ,\mK,\mV,\mO) \\
% =&\mO\mV\layernorm(\mH)\softmax(\mask(\layernorm(\mH)^\transpose\mK^\transpose\mQ\layernorm(\mH))).
% \end{align*}
% Furthermore, a multi-head attention with $M$ heads parameterized by $\{(\mQ_m,\mK_m,\mV_m,\mO_m) \}_{m=1}^M$ is defined as 
% \begin{align*}
%     &\Attn(\mH; \{(\mQ_m,\mK_m,\mV_m,\mO_m) \}_{m\in[M]}) \\ =& \sum_{m=1}^M \attn(\mH;\mQ_m,\mK_m,\mV_m,\mO_m) \in \real^{\dout \times T}.
% \end{align*}
% An MLP layer $\mlp(\mH;\mW_1,\mW_2)$ parameterized by $\mW_1 \in \real^{\dhidden\times \din}$ and $\mW_2 \in \real^{\dout \times \dhidden}$ maps an input matrix $\mH = [\vh_1, \ldots, \vh_T] \in \real^{\din \times T}$ to
% \begin{align*}
%     &\mlp(\mH;\mW_1,\mW_2) = [\vy_1, \ldots, \vy_T], \\ \text{where } &\vy_i = \mW_2\nonlin(\mW_1\layernorm(\vh_i)), \forall i \in [T].
% \end{align*}

% \end{definition}

% In this paper, we assume $\din=\dout=d$ for all attention heads and MLPs to facilitate residual stream unless otherwise specified. Given \Cref{defn:operators,defn:attn_mlp}, we are now able to define a multi-layer transformer.

% \begin{definition}[Multi-layer transformers]
% \label{defn:transformer}
%     An $L$-layer transformer $\transformer(\cdot): \vocab^T \to \Delta(\vocab)$ parameterized by $\mP$, $\mU$, $\{(\mQ_m^{(l)},\mK_m^{(l)},\mV_m^{(l)},\mO_m^{(l)})\}_{m\in[M],l\in[L]}$,  $\{(\mW_1^{(l)},\mW_2^{(l)})\}_{l\in[L]}$ and $\Wreadout \in \real^{V \times d}$ receives a sequence of tokens $s_{1:T}$ as input and predict the next token by outputting a distribution over the vocabulary. The input is first mapped to embeddings $\mH = [\vh_1, \vh_2, \ldots, \vh_T] \in \real^{d\times T}$ by embedding matrices $\mP, \mU$ where 
%     \[
%     \vh_i = \mU^\transpose\ve_{s_i} + \mP^\transpose\ve_{i}, \forall i \in [T].
%     \]
%     For each layer $l \in [L]$, the output of layer $l$, $\mH^{(l)} \in \real^{d\times T}$, is obtained by 
%     \begin{align*}
%         &\mH^{(l)} =  \mH^{(l-1/2)} + \mlp(\mH^{(l-1/2)};\mW_1^{(l)},\mW_2^{(l)}), \\
%         & \mH^{(l-1/2)} = \mH^{(l-1)} + \\ & \quad \Attn(\mH^{(l-1)}; \{(\mQ_m^{(l)},\mK_m^{(l)},\mV_m^{(l)},\mO_m^{(l)}) \}_{m\in[M]}), 
%     \end{align*}
%     where the input $\mH^{(l-1)}$ is the output of the previous layer $l-1$ for $l > 1$ and the input of the first layer $\mH^{(0)} = \mH$. Finally, the output of the transformer is obtained by 
%     \begin{align*}
%         \transformer(s_{1:T}) = \softmax(\Wreadout\vh_T^{(L)})
%     \end{align*}
%     which is a $V$-dimensional vector after softmax representing a distribution over $\vocab$, and $\vh_T^{(L)}$ is the $T$-th column of the output of the last layer, $\mH^{(L)}$.
% \end{definition}



% For each token $v \in \vocab$, there is a corresponding $d_t$-dimensional token embedding vector $\embed(v) \in \mathbb{R}^{d_t}$. Assume the maximum length of the sequence studied in this paper does not exceed $T$. For each position $t \in [T]$, there is a corresponding positional embedding  








%%
% \section{Lower Bound Part I: Generalized Arm-trapping and Distribution Learning Lemmas in MABs}
\section{Technical Lemmas for the Lower Bound}
\label{sec:tech-lemma}
% \nicksays{define all notation like KL-divergence and total variation distance, also we need to introduce \(\smp\)}
% \chen{re-write this paragraph -- you did No only give the baseline hardness problems here, you also showed the batched instances and the technical tools provided by the previous paper.}
In this section, we present several technical lemmas en route to our main lower bound result. In particular, we show the following results in this section.
%that establish the hardness of some ``baseline'' problems, i.e. our main complexity results are obtained by embedding instances of more complicated problems into the baselines.

\begin{enumerate}[label=\alph*).]
	\item A lower bound on the necessary number of arm pulls for an algorithm with sublinear memory to \emph{store} an arm with high mean reward (while possibly without knowing the identity of the arm).
	\item A lower bound on the necessary number of arm pulls for an algorithm to gain ``knowledge'' about the underlying \emph{distribution} of the MABs instance. 
	\item An observation of the framework from \cite{AW23BestArm} as a general sample-memory-pass trade-off for batched instances.
\end{enumerate}

We note that a variant of the first two lower bounds on instances with a single arm with high mean reward is first proved in \cite{AW23BestArm}. However, the subtle difference in the construction (as it will be evident in \Cref{sec:lb-main}) requires lower bounds to work with \emph{two} arms with high mean rewards. This, in particular, requires a careful handling of properties on Double-armed Bandits (DABs), which we prove in \Cref{lem:arm-identify} and \Cref{lem:arm-learn}.

\paragraph{Additional notation.} We introduce several additional notation used in a self-contained manner in the lower bound proof. Unless specified otherwise, we use $\ALG$
to denote a streaming algorithm, and $\smp$ is the random variable for the sample complexity of $\ALG$. As we introduced in \Cref{sec:standatd-tech-tools}, for two random variables $X$ and $Y$, we use $\tvd{X}{Y}$ to denote their total variation distance, $\kl{X}{Y}$ for the KL-divergence, and $\II(X;Y)$ for the mutual information. We also use $X\mid Y=y$ to denote the random variable for $X$ \emph{conditioning} on the realization of $Y=y$. Finally, for the conciseness of notation, we slightly abuse the notation to use $\kl{X|Z}{Y|Z}$ as a short-hand notation for $\Exp_{z \sim Z} \kl{X \mid Z=z}{Y \mid Z=z}$.

\subsection{Lower Bounds on the Sample Complexity of Double-armed Bandits}
\label{subsec:two-arm-lb}

We start with proving the necessary number of arm pulls to distinguish an instance of \emph{two} arms that are $(i).$ either both with reward $1/2$ or $(ii).$ one arm with mean reward $1/2+\alpha$ and the other with $1/2+\alpha+\beta$. The problem is in the same spirit as the single-arm distinguishment problem in \cite{AWneurips22,AW23BestArm}, but we are unaware of any previous result for the exact version we are using. % As such, we include the lemma statements and the proof for completeness. 

Our first lemma shows that if an instance is sampled from a two-arm version of ``good arms'' and ``bad arms'', the algorithm will not be able to distinguish the cases if the number of arm pulls is small.

% \nicksays{introduce KL-deverigence before the use }

\begin{lemma}
	\label{lem:arm-identify}
	% \chen{to-do: Change the notation of conditional KL-divergence in the chain rule.}
	Consider two arms with a Bernoulli reward distribution whose mean is parameterized as follows.
	\begin{itemize}
		\item With probability $\rho$, the \emph{Yes case}, where
		\begin{enumerate}[label=\roman*).]
			\item $\arm_{1}$ is with reward $\frac{1}{2}+\alpha$;
			\item $\arm_{2}$ is with reward $\frac{1}{2}+\alpha+\beta$.
		\end{enumerate}
		\item With probability $1-\rho$, the \emph{No case}, where both $\arm_{1}$ and $\arm_{2}$ are with mean rewards of $\frac{1}{2}$;
	\end{itemize}
	where $\rho\in (0,\frac{1}{2}]$ is the probability for the reward to be more than $\frac{1}{2}$, and $\alpha, \beta >0$ satisfy $\alpha+\beta<\frac{1}{2}$. Any algorithm to determine the reward of the arms with a success probability of at least $(1-\rho+\eps)$ has to use $\frac{1}{4}\cdot \frac{\eps^2}{\rho^2 (\alpha+\beta)^{2}}$ arm pulls.
\end{lemma}
\begin{proof}
	We define $\Xyes=(\Xyes^{1}, \Xyes^{2}, \cdots, \Xyes^{m})$ as the random variable for taking $m$ samples from the Yes case. 
	Similarly, we define $\Xno=(\Xno^{1}, \Xno^{2}, \cdots, \Xno^{m})$ as the random variable for taking $m$ samples from the No case. 
	Furthermore, we also define random variables for ``dummy'' arm pulls: we define $\Xhigh$ as the random variable for taking a sample on an arm with reward $\frac{1}{2}+\alpha+\beta$, and $\Xflat$ as the random variable for taking samples on an arm with reward $\frac{1}{2}$.
	We first observe that for any $i\in [m]$, there is 
	\begin{align*}
		\kl{\Xyes^{i}}{\Xno^{i}}\leq \kl{\Xhigh^{i}}{\Xflat^{i}} =  \kl{\bern{\frac{1}{2}+\alpha+\beta}}{\bern{\frac{1}{2}}}.
	\end{align*}
	To see this, note that the algorithm is allowed to take a sample from either of the arms; however, the case to maximize the KL-divergence is for the algorithm to compare the empirical rewards from a $\bern{\frac{1}{2}+\alpha+\beta}$ arm and a $\bern{\frac{1}{2}}$ arm, which establishes the upper bound.
	
	We can in fact extend the above observation to \emph{conditional} KL-divergence. In particular, we have
	
	\begin{claim}
		\label{clm:cross-trial-ub}
		For any $i\in [m]$, there is
		\begin{align*}
			\kl{\Xyes^{i}\mid (\Xyes^{i+1},\cdots, \Xyes^{m})}{\Xno^{i}\mid (\Xno^{i+1},\cdots, \Xno^{m})} \leq \kl{\Xhigh}{\Xflat}.
		\end{align*}
	\end{claim}
	\begin{proof}
		Intuitively, the dependence between the results of arm pulls is only on the \emph{choice} of arms; once an arm is picked, the results are independent across different arm pulls. Our proof is a formalization of the above intuition. Define $\Xyes^{i, \arm_j}$ and $\Xno^{i, \arm_j}$ as the random variable for the algorithm to pull the $\arm_{j}$ ($j\in \{1,2\}$) on the $i$-th trial under the Yes and No cases, respectively. Furthermore, define $J$ as the random variable for the choice of arm by the algorithm. Note that we have $\Xyes^{i, \arm_j} = \Xyes^{i}\mid J=j$. % \nicksays{add brackets?}. 
		For any $i\in [m]$ and $j\in\{1,2\}$, there is
		\begin{align*}
			& \kl{\Xyes^{i}\mid (\Xyes^{i+1},\cdots, \Xyes^{m})}{\Xno^{i}\mid (\Xno^{i+1},\cdots, \Xno^{m})} \\
			& \leq \kl{\Xyes^{i}\mid (\Xyes^{i+1},\cdots, \Xyes^{m}, J)}{\Xno^{i}\mid (\Xno^{i+1},\cdots, \Xno^{m}, J)}  \tag{extra conditioning can only increase KL-divergence}\\
			& = \kl{\Xyes^{i}\mid J}{\Xno^{i}\mid J} \tag{indenpendence between sampling from bernoulli distributions}\\
			& \leq \kl{\Xhigh}{\Xflat},
		\end{align*}
		as desired. \myqed{\Cref{clm:cross-trial-ub}}
	\end{proof}
	
	%\chen{Write the proof of the above}
	We now use \Cref{clm:cross-trial-ub} to prove \Cref{lem:arm-identify}. By the standard calculation of the KL-divergence of Bernoulli random variables (\Cref{clm:bernoulli-KL}), we accordingly have
	%\begin{align*}
	%\kl{\Xhigh}{\Xflat} & = \left(\frac{1}{2}+\alpha+\beta\right)\cdot \log({1+2\alpha+2\beta})+\left(\frac{1}{2}-\alpha-\beta\right)\cdot \log{(1-2\alpha-2\beta)}\\
	%& = \frac{1}{2}\cdot \log\paren{1-4(\alpha+\beta)^2} + (\alpha+\beta)\cdot \log\left({\frac{1+2\alpha+2\beta}{1-2\alpha-2\beta}}\right)\\
	%& \leq (\alpha+\beta)\cdot \log\left({\frac{1+2\alpha+2\beta}{1-2\alpha-2\beta}}\right) \tag{$\log(1-4(\alpha+\beta)^2)<0$}\\
	%&\leq (\alpha+\beta)\cdot \log (2^{8(\alpha+\beta)}) \tag{$\frac{1+x}{1-x}\leq 2^{4x}$ for any $0<x<1/2$}\\
	%& = 8 \cdot (\alpha+\beta)^2
	%\end{align*}
	\begin{align*}
		\kl{\Xhigh}{\Xflat} \leq 8 \cdot (\alpha+\beta)^2
	\end{align*}
	for any sample index of $i$. As such, we can bound the KL-divergence of the distributions with all samples as follows.
	\begin{align*}
		\kl{\Xyes}{\Xno} &= \sum_{i=1}^{m} \kl{\Xyes^{i}\mid (\Xyes^{i+1},\cdots, \Xyes^{m})}{\Xno^{i}\mid (\Xno^{i+1},\cdots, \Xno^{m})} \tag{by Chain rule}\\
		&\leq \sum_{i=1}^{m} \kl{\Xhigh}{\Xflat} \tag{by \Cref{clm:cross-trial-ub}}\\
		&= 8m\cdot (\alpha+\beta)^2.
	\end{align*}
	
	Therefore, by Pinsker's inequality \Cref{fact:pinsker}, we have % \chen{right-pointer} % \nicksays{don't forget to introduce the total variation distance }, we have
	\begin{align*}
		\tvd{\Xyes}{\Xno} & \leq \sqrt{\frac{1}{2}\cdot \kl{\Xyes}{\Xno}}\\
		& \leq 2 (\alpha+\beta) \cdot \sqrt{m}.
	\end{align*}
	On the other hand, by \Cref{fact:distinguish-tvd}, we know that to distinguish the cases by a sample from the distribution with probability at least $1-\rho-\eps$, there has to be $\tvd{\Xyes}{\Xno}\geq \frac{\eps}{\rho}$. As such, we get a lower bound of
	\begin{align*}
		m \geq \frac{1}{4}\cdot \frac{\eps^2}{\rho^2 (\alpha+\beta)^2},	
	\end{align*}
	as desired.
\end{proof}


We now move to the second result for double-armed bandits, which shows that if the number of arm pulls is small, then the ``knowledge'' % \nicksays{are you sure in this sentence?} 
of the algorithm cannot change the original distribution by too much. More formally, we prove that with a limited number of arm pulls, from the algorithm's perspective, the probability for which case the instance is from remains close to the original distribution.
\begin{lemma}
	\label{lem:arm-learn}
	Let $\alpha, \beta \in (0,\frac16)$, $\beta\leq \alpha$, and $\rho \in (0,\frac12)$. Sample $\Theta$ from $\set{0,1}$ such that $\Theta=1$ with probability $\rho$.
	Consider two arms with Bernoulli reward distributions from the following family:
	\begin{itemize}
		\item If $\Theta=1$, the \emph{Yes} case, where 
		\begin{enumerate}
			\item $\arm_1$ is with mean reward $\frac{1}{2}+\alpha$;
			\item $\arm_2$ is with mean reward $\frac{1}{2}+\alpha+\beta$.
		\end{enumerate}
		\item If $\Theta=0$, the mean rewards of $\arm_1$ and $\arm_2$ are both $\frac{1}{2}$.
	\end{itemize}
	Let $\ALG$ be an algorithm that uses at most $m=\frac{1}{16}\cdot \frac{\eps^3}{\rho \cdot (\alpha+\beta)^{2}}$ arm pulls on an instance $I$ sampled from the family. Let $\pi$ be the transcript of $\ALG$ that records the arm pulls and the results, and let $\Pi$ be the random variable of $\pi$. Then, with probability at least $1-\eps$ over the randomness of transcript $\Pi$, there is
	\begin{align*}
		& \Pr\paren{\Theta=1 \mid \Pi=\pi} \in [\rho -  \eps,  \rho +  \eps]\\
		& \Pr\paren{\Theta=0 \mid \Pi=\pi} \in [1-\rho - \eps,  1- \rho + \eps]
	\end{align*}
\end{lemma}

\begin{proof}
	We prove the lemma by an information-theoretic argument similar to the analysis in \cite{AW23BestArm}, albeit we need to handle the dependence between arm pulls in our case. For an $m$-trial process, let $\Pi=(\Pi^{1}, \Pi^{2}, \cdots, \Pi^{m})$, where $\Pi^{i}$ is the random variable for the transcript of the $i$-th arm pull. Therefore, we can bound the mutual information between $\Theta$ and $\Pi$ as follows. % \chen{wrong way to write the mutual information as KL-divergence $\II(\Theta; \Pi) = \expectR{\theta \in \{0,1\}}{\kl{\Pi\mid \Theta=\theta}{\Pi}}$ not necessarily equalt to $\expectR{\theta \in \{0,1\}}{\kl{\Pi}{\Pi\mid \Theta=\theta}}$}
	\begin{align*}
		\II(\Theta; \Pi) &= \expectR{\theta \in \{0,1\}}{\kl{\Pi\mid \Theta=\theta}{\Pi}} \tag{KL-divergence view of mutual information}\\
		&= \expectR{\theta \in \{0,1\}}{\kl{(\Pi^{1}, \Pi^{2}, \cdots, \Pi^{m})\mid \Theta=\theta}{(\Pi^{1}, \Pi^{2}, \cdots, \Pi^{m})}}\\
		&= \expectR{\theta \in \{0,1\}}{\sum_{i=1}^{m} \kl{\Pi^{i} \mid (\Pi^{i+1}, \cdots,\Pi^{m}, \Theta=\theta)}{\Pi^{i} \mid (\Pi^{i+1}, \cdots,\Pi^{m})}} \tag{by chain rule of KL divergence}.
	\end{align*}
	We now argue that each of the KL-divergence terms in the expectation can be upper-bounded by substituting the transcript with the pull on $\arm_{2}$.
	\begin{claim}
		\label{clm:script-trial-ub}
		Let $\Pi^{i, \arm_2}$ be the random variable for the transcript induced by pulling $\arm_2$ on step $i$. For any $i\in [m]$ and $\theta \in \{0,1\}$, there is
		\begin{align*}
			& \kl{\Pi^{i} \mid \Pi^{i+1}, \cdots,\Pi^{m}, \Theta=\theta}{\Pi^{i} \mid \Pi^{i+1}, \cdots,\Pi^{m}} \\
			& \leq \rho \cdot \kl{\Pi^{i, \arm_1}\mid \Theta = \theta}{\Pi^{i, \arm_1}} + (1-\rho)\cdot \kl{\Pi^{i, \arm_2}\mid \Theta = \theta}{\Pi^{i, \arm_2}}.
		\end{align*}
	\end{claim}
	\begin{proof}
		The proof is similar to the one we showed in \Cref{clm:cross-trial-ub}. Concretely, let $J$ be the random variable for the choice of the arm to be pulled, and observe in the same manner as \Cref{clm:cross-trial-ub} that conditioning on the choice of $J$, the transcript between different $i$ indices are \emph{independent}. As such, For any $i\in [m]$ and $\theta\in\{0,1\}$, there is
		\begin{align*}
			& \kl{\Pi^{i} \mid \Pi^{i+1}, \cdots,\Pi^{m}, \Theta=\theta}{\Pi^{i} \mid \Pi^{i+1}, \cdots,\Pi^{m}} \\
			& \leq \kl{\Pi^{i} \mid \Pi^{i+1}, \cdots,\Pi^{m}, \Theta=\theta, J}{\Pi^{i} \mid \Pi^{i+1}, \cdots,\Pi^{m}, J}  \tag{extra conditioning can only increase KL-divergence}\\
			&= \kl{\Pi^{i} \mid  \Theta=\theta, J}{\Pi^{i} \mid J}. \tag{$\Pi^{i}$ is independent of $\Pi^{\neq i}$ conditioning on the choice of $J$}
		\end{align*}
		For the first random variable, we have
		\begin{align*}
			& \paren{\Pi^{i} \mid \theta=0, J=1} = \bern{1/2} \qquad \paren{\Pi^{i} \mid \theta=0, J=2} = \bern{1/2}; \\
			& \paren{\Pi^{i} \mid \theta=1, J=1} = \bern{1/2+\alpha} \qquad \paren{\Pi^{i} \mid \theta=1, J=2} = \bern{1/2+\alpha+\beta}.
		\end{align*}
		On the other hand, for the second random variable, there is
		\begin{align*}
			\paren{\Pi^{i} \mid J=1} = \bern{\frac{1}{2}+\rho \cdot \alpha}; \qquad \paren{\Pi^{i} \mid J=2} = \bern{\frac{1}{2}+\rho \cdot (\alpha + \beta)}.
		\end{align*}
		By the above calculation, the KL-divergences are maximized with $J=2$ for the $\Theta=0$ case and $J=1$ for $\Theta=1$ case. As such, we have
		\begin{align*}
			& \kl{\Pi^{i} \mid \Pi^{i+1}, \cdots,\Pi^{m}, \Theta=\theta}{(\Pi^{i} \mid \Pi^{i+1}, \cdots,\Pi^{m}} \\
			& \leq \kl{\Pi^{i} \mid J, \Theta=\theta}{\Pi^{i} \mid J}\\
			&= \expectR{j\in \{1,2\}}{\kl{\Pi^{i} \mid \Theta=\theta, J=j}{\Pi^{i} \mid J=j}}\\
			& \leq \rho \cdot \kl{\Pi^{i} \mid J=1}{\Pi^{i} \mid \Theta=\theta, J=1} + (1-\rho) \cdot \kl{\Pi^{i} \mid J=2}{\Pi^{i} \mid \Theta=\theta, J=2} \\
			&= \rho \cdot \kl{\Pi^{i, \arm_1}\mid \Theta = \theta}{\Pi^{i, \arm_1}} + (1-\rho)\cdot \kl{\Pi^{i, \arm_2}\mid \Theta = \theta}{\Pi^{i, \arm_2}},
		\end{align*}
		as desired. \myqed{\Cref{clm:script-trial-ub}}
	\end{proof}
	By \Cref{clm:script-trial-ub}, we can upper bound the mutual information between $\Theta$ and $\Pi$ as % \nicksays{I don't get how we get \(\rho \alpha\) and \(\rho (\alpha + \beta)\) in the proof, can you explain it? }
	\begin{align*}
		\qquad & \II(\Theta; \Pi) \\
		&\leq \expectR{\theta \in \{0,1\}}{\sum_{i=1}^{m} \kl{\Pi^{i, \arm_2}}{\Pi^{i, \arm_2} \mid \Theta = \theta}}\\
		&= \sum_{i=1}^{m} \rho \cdot \kl{\bern{\frac{1}{2}+\alpha}}{\bern{\frac{1}{2}+\rho\cdot \alpha}} + (1-\rho)\cdot \kl{\bern{\frac{1}{2}}}{\bern{\frac{1}{2}+\rho\cdot (\alpha+\beta)}}\\
		&\leq 8m\cdot \paren{\rho\cdot (\rho-1)^2\cdot \alpha^2 + (1-\rho) \cdot \rho^2 \cdot (\alpha+\beta)^2} \tag{by \Cref{clm:bernoulli-KL}}\\
		&\leq 16 m\cdot \rho \cdot (\alpha+\beta)^2 \tag{by $(\rho-1)^2\leq \rho^2$ since $\rho\leq \frac{1}{2}$}.
	\end{align*}
	By plugging in the condition that $m\leq \frac{1}{16} \cdot \frac{\eps^3}{\rho (\alpha+\beta)^2}$, we have $\II(\Theta; \Pi)\leq \eps^3$. We now use another KL-divergence form of the mutual information to get
	\begin{align*}
		\II(\Theta; \Pi) = \expectR{\pi\sim \Pi}{\kl{\Theta}{\Theta\mid \Pi=\pi}} \leq \eps^3.
	\end{align*}
	As such, with probability at least $1-\eps$ over the randomness of $\Pi$, we have
	\begin{align*}
		\kl{\Theta}{\Theta\mid \Pi=\pi}\leq \frac{1}{\eps} \cdot  \expectR{\pi\sim \Pi}{\kl{\Theta}{\Theta\mid \Pi=\pi}} \leq \eps^2.
	\end{align*}
	We condition on the high probability transcripts for the rest of the calculations. Now, we can apply Pinsker's inequality (\Cref{fact:pinsker}) to get % \nicksays{add brackets around \(\Theta \mid \Pi = \pi)\) ? }
	\begin{align*}
		\tvd{\Theta}{\Theta\mid \Pi=\pi} \leq \sqrt{\kl{\Theta}{\Theta\mid \Pi=\pi}} \leq \eps.
	\end{align*}
	By \Cref{fact:distinguish-tvd}, we get the desired upper bound of the ``advantage'', i.e. 
	\begin{align*}
		& \card{\Pr\paren{\Theta=0\mid \Pi=\pi}-\Pr\paren{\Theta=0}}\leq \eps\\
		& \card{\Pr\paren{\Theta=1\mid \Pi=\pi}-\Pr\paren{\Theta=1}}\leq \eps,
	\end{align*} % \nicksays{identical lines}
	which implies the desired lemma statement.
\end{proof}

\subsection{Lower bounds on the Sample Complexity of MABs Trapping and Learning}
\label{subsec:batch-arm-hardness}
We now show how we `amplify' the result for the double-armed bandits to a collection of $k$ arms with two \emph{special} arms. These results are similar both in spirit and in technicality to existing multi-pass lower bounds \cite{AWneurips22,AW23BestArm}, and we include the proofs for completeness.
% \chen{Jul/30 Note: I stopped here -- try to fill up Prop. 7 for the immediate next step.}

\begin{lemma}
	\label{lem:arm-trapping}
	Let $k\geq 3$ be an integer and $\alpha, \beta >0$ such that $\alpha+\beta<\frac{1}{6}$, suppose there is a family of $k$ arms in which
	\begin{itemize}
		\item two indices $\istar, \jstar \in [k]$ chosen uniformly at random (without replacement), and their mean rewards are $\mu_{\istar}=\frac{1}{2}+\alpha$ and $\mu_{\jstar}=\frac{1}{2}+\alpha+\beta$.
		\item for all $i \in [k]\setminus \{\istar, \jstar\}$, their mean rewards are $\mu_{i}=\frac{1}{2}$.
	\end{itemize}
	Then, for any given parameter $\tau\in (0, \frac{1}{2}]$, any algorithm that outputs $\frac{\tau\cdot k}{40}$ arms that contains any arm with reward \emph{strictly more than} $\frac{1}{2}$ with probability at least $\tau$ requires $\frac{1}{600}\cdot \frac{\tau^3}{(\alpha+\beta)^2}\cdot k$ arm pulls.
\end{lemma}
% \chen{Prove for general $\rho$? Maybe not necessary.}
\begin{proof}
	\FloatBarrier
	The proof uses the ``direct-sum'' argument in a similar manner of \cite{AWneurips22} and \cite{AW23BestArm}. Concretely, we provide a reduction from the problem in \Cref{lem:arm-identify}, and show that an algorithm that satisfies the prescribed property in \Cref{lem:arm-trapping} with $s$ samples would imply an algorithm that identifies an arm with $O(s/k)$ samples, which eventually leads to a contradiction with \Cref{lem:arm-identify} for $\rho=1/2$. The formal reduction is as \Cref{red:arm-trapping}.
	\begin{algorithm}[!h]
		\caption{A reduction algorithm to prove \Cref{lem:arm-trapping}}\label{red:arm-trapping} % \nicksays{combine inputs?}
		\KwIn{Two arms $\arm_{1}$ and $\arm_{2}$ from the distribution of \Cref{lem:arm-identify} with $\rho=\frac{1}{2}$.}
		\KwIn{An algorithm $\ALG$ that $a).$ uses at most $\frac{1}{600}\cdot \frac{\tau^3}{(\alpha+\beta)^2}\cdot k$ arm pulls, $b).$ outputs a collection $S$ of $\frac{\tau\cdot k}{20}$, and $c).$ $S$ contains an arm with reward \emph{strictly more than} $\frac{1}{2}$ with probability at least $\tau$.}
		\KwOut{The decision (Yes or No cases) from which $\arm_{1}$ and $\arm_{2}$ are sampled.}
		Sample a coin $\Theta \sim \bern{\frac{1}{2}+\frac{11}{38}\cdot \tau}$\;
		\If{$\Theta=0$}{
			Directly output ``$\arm_1$ is from $\bern{1/2+\alpha}$ and $\arm_2$ is from $\bern{1/2+\alpha+\beta}$''.\;
		}
		\Else{
			Construct an instance $I$: sample two indices $\istar$, $\jstar$ uniformly at random (without replacement), and set the arms with indices $\istar$, $\jstar$ as $\arm_{1}$ and $\arm_{2}$\;
			For all indices $i \in [k]\setminus \{\istar, \jstar\}$, create $k-2$ dummy arms $\bern{1/2}$\;
			Run $\ALG$ on instance $I$, and output with the following rules: \;
			\If{$\arm_1$ or $\arm_2$ uses more than $\frac{1}{30}\cdot \frac{\tau^2}{(\alpha+\beta)^2}$ arm pulls}{
				\label{line:sample-ub-term} Terminate $\ALG$ and output ``$\arm_1$ is from $\bern{1/2+\alpha}$ and $\arm_2$ is from $\bern{1/2+\alpha+\beta}$''. \;
			}
			\ElseIf{$S$ cotains any of $\{\istar, \jstar\}$}{
				\label{line:false-trap} Output ``$\arm_1$ is from $\bern{1/2+\alpha}$ and $\arm_2$ is from $\bern{1/2+\alpha+\beta}$''\;
			}
			\Else{
				Output ``$\arm_1$ and $\arm_2$ are from $\bern{1/2}$''\;
			}
		}
	\end{algorithm}
	
	We first observe that \Cref{red:arm-trapping} never uses more than $\frac{1}{30}\cdot \frac{\tau^2}{(\alpha+\beta)^2}$ arm pulls, as there is a forced termination once this condition happens. We now need to analyze the correctness of the algorithm for \Cref{lem:arm-identify}. Note that we fix $\rho$ in \Cref{lem:arm-identify} to be $\rho=\frac{1}{2}$. We claim that the algorithm correctly identifies the cases with probability at least $\frac12+\frac{\tau}{5}$, and the analysis considers two cases, respectively.
	\begin{enumerate}[label=\roman*).]
		\item $\arm_1$ is $\bern{1/2+\alpha}$ and $\arm_2$ is $\bern{1/2+\alpha+\beta}$. In this case, with probability $\frac{1}{2}-\frac{11}{38}\cdot \tau$, \Cref{red:arm-trapping} directly return the correct answer. On the other hand, if \Cref{red:arm-trapping} runs $\ALG$ on the instance $I$, it will return the correct answer as long as $\ALG$ succeeds, which is with probability at least $\tau$. As such, the correct probability \emph{conditioning} on the ``Yes'' case of \Cref{lem:arm-identify} is at least
		\begin{align*}
			\frac{1}{2}-\frac{11}{38}\cdot \tau + \left(\frac{1}{2}+\frac{11}{38}\cdot \tau\right)\cdot \tau \geq \frac{1}{2} +\left(\frac{1}{2}-\frac{11}{38}\right)\cdot \tau \geq \frac{1}{2}+ \frac{\tau}{5}.
		\end{align*}  
		\item Both $\arm_1$ and $\arm_2$ are $\bern{1/2}$. We show that if \Cref{red:arm-trapping} runs $\ALG$ on the instance $I$, the correct probability is sufficiently high. To this end, we bound the failure probability for \Cref{red:arm-trapping} to not report this case (conditioning on $\Theta=0$). Let $s_{(\ell)}$ be the random variables for the number of arm pulls used by the arm on index $\ell$. Furthermore, let us use $s_1$ and $s_2$ to denote the random variables for the number of samples used by $\arm_1$ and $\arm_2$. Let $\mathcal{E}_{\text{No}}$ be the event that $\arm_1$ and $\arm_2$ are sampled from the ``No'' case of \Cref{lem:arm-identify} (both $\bern{1/2}$). We have
		\begin{align*}
			\expect{s_1\mid \mathcal{E}_{\text{No}}} &= \expect{s_2\mid \mathcal{E}_{\text{No}}} \tag{$s_1$ and $s_2$ are identical random variables}\\
			&= \sum_{\ell=1}^{k} \Pr(i^* = \ell)\cdot \expect{s_{(\ell)}\mid \mathcal{E}_{\text{No}}} \tag{all arms are identical random variables conditioning on $\mathcal{E}_{\text{No}}$}\\
			&= \frac{1}{k}\cdot \expect{\sum_{\ell} s_{(\ell)}\mid \mathcal{E}_{\text{No}}}\\
			&\leq \frac{1}{600}\cdot \frac{\tau^3}{(\alpha+\beta)^2}. \tag{bound on the number of arm pulls}
		\end{align*}
		Therefore, we have $\Pr\paren{s_1\geq \frac{1}{30} \cdot \frac{\tau^2}{(\alpha+\beta)^2}}\leq \frac{\tau}{20}$ by a simple Markov bound. Therefore, the probability for \Cref{line:sample-ub-term} to falsely output the ``Yes'' case is at most $\frac{\tau}{4}$. On the other hand, conditioning on $\mathcal{E}_{\text{No}}$, the arms become identical random variables. More formally, let $X_{\arm_1}$ and $X_{\arm_2}$ be the indicator random variables for $\arm_1$ and $\arm_2$ to be in $S$, and let $X_{(\ell)}$ % \nicksays{We use \(X\) so many times probably we can use another letter.}
		be the indicator random variables for the arm of index $\ell$ to be in $S$, we have
		\begin{align*}
			\Pr(X_{(\ell)}=1) = \Pr(X_{\arm_{1}}=1) = \Pr(X_{\arm_{2}} = 1) = \frac{\card{S}}{k}\leq \frac{\tau}{40}.
		\end{align*} % \nicksays{\(X_{(\ell)}\) ? }
		Therefore, by a union bound, the algorithm to contain \emph{any} of $\arm_{1}$ and $\arm_{2}$ in the ``No'' case is at most $\frac{1}{20}$.
		Now, we apply another union bound, and the failure probability conditioning on $\mathcal{E}_{\text{No}}$ and $\ALG$ is executed on $I$ is at most $\frac{\tau}{20}+\frac{\tau}{20}=\frac{\tau}{10}$. As such, the success probability given $\mathcal{E}_{\text{No}}$ is at least
		\begin{align*}
			(\frac{1}{2}+\frac{11}{38}\cdot \tau)\cdot (1-\frac{\tau}{10}) \geq \frac{1}{2}+\frac{\tau}{5}.
		\end{align*}
	\end{enumerate}
	By plugging in $\eps=\frac{\tau}{5}$ and $\rho=\frac{1}{2}$ to \Cref{lem:arm-identify}, we obtain the number of necessary arm pulls is at least $\frac{1}{25}\cdot \frac{\tau^2}{(\alpha+\beta)^2}$ arm pulls, which forms a contradiction with \Cref{red:arm-trapping}. Therefore, such an $\ALG$ cannot exist.
	
	\FloatBarrier
\end{proof}


\begin{lemma}
	\label{lem:batch-arm-learning}
	Let $k\geq 3$ be an integer, $\alpha, \beta >0$ such that $\alpha+\beta<\frac{1}{6}$, and $\rho\in (0, \frac{1}{2})$, suppose there is a family of $k$ arms in which
	\begin{itemize}
		\item with probability $\rho$, the \emph{Yes} % \nicksays{sometimes we use yse sometimes Yes, we need to unify this} 
		case, where all except \emph{two} arms chosen uniformly at random are with mean rewards $\frac{1}{2}$, and the two special arms are with mean rewards $\frac{1}{2}+\alpha$ and $\frac{1}{2}+\alpha+\beta$.
		\item with probability $1- \rho$, the \emph{No} case, where all the arms are with mean rewards $\frac{1}{2}$.
	\end{itemize}
	Then, for any given parameter $\tau\in (0, \frac{1}{5}]$, let $\ALG$ be any algorithm that given an instance $D$ from the distribution, uses at most $\frac{1}{200}\cdot \frac{\tau^2}{\rho \cdot (\alpha+\beta)^2}\cdot k$ arm pulls, and let $\Pi$ and $\pi$ be the random variable and the realization of the transcripts of $ALG$. With probability at least $1-2 \tau^{1/2}$ over the randomness of the transcript, there is
	\begin{align*}
		& \Pr\paren{\text{$D$ in \emph{Yes} case} \mid \Pi=\pi} \in [\rho-2 \tau^{1/2}, \rho + 2 \tau^{1/2}];\\
		& \Pr\paren{\text{$D$ in \emph{No} case} \mid \Pi=\pi} \in [1-\rho-2 \tau^{1/2}, 1-\rho + 2 \tau^{1/2}],
	\end{align*}
	where the randomness is over the choices of the instances.
\end{lemma}
\begin{proof}
	\FloatBarrier
	Similar to the proof of \Cref{lem:arm-trapping} (and as in \cite{AW23BestArm}), we prove the lemma by applying the ``direct sum'' argument with \Cref{lem:arm-learn}. To this end, we again assume for the purpose of contradiction that an algorithm that breaks the bound on \Cref{lem:batch-arm-learning} exists, and build an algorithm that is ruled out by \Cref{lem:arm-learn}. In the proof, we only focus on the upper bound of $\Pr\paren{\text{$D$ in \emph{Yes} case} \mid \Pi=\pi}$ as the lower bound follows from the same logic.
	
	\begin{algorithm}[!h]
		\caption{A reduction algorithm to prove \Cref{lem:batch-arm-learning}}\label{red:batch-arm-learn}
		\KwIn{Two arms $\arm_{1}$ and $\arm_{2}$ from the distribution of \Cref{lem:arm-learn}.}
		\KwIn{An algorithm $\ALG$ that $a).$ uses at most $\frac{1}{200}\cdot \frac{\tau^2}{\rho\cdot (\alpha+\beta)^2}\cdot k$ arm pulls, $b).$ with probability more than $2\tau^{1/2}$ produce a transcript $\pi$, such that $\Pr\paren{\text{$D$ in \emph{Yes} case} \mid \Pi=\pi}> \rho + 2 \tau^{1/2}$.}
		% \nicksays{join input?}
		\KwOut{A (conditional) probability distribution of instance $D$.}
		Construct an instance $\tilde{D}$: sample two indices $\istar$, $\jstar$ uniformly at random (without replacement), and set the arms with indices $\istar$, $\jstar$ as $\arm_{1}$ and $\arm_{2}$\;
		For all indices $i \in [k]\setminus \{\istar, \jstar\}$, create $k-2$ dummy arms $\bern{1/2}$\;
		Run $\ALG$ on instance $\tilde{D}$, and output with the following rules: \;
		\If{$\arm_1$ or $\arm_2$ uses more than $\frac{1}{5}\cdot \frac{\tau^{3/2}}{\rho \cdot(\alpha+\beta)^2}$ arm pulls}{
			\label{line:sample-ub-knowledge} Terminate $\ALG$ and output ``Yes case''. \;
		}
		\Else{$S$ contains any of $\{\istar, \jstar\}$}{
			\label{line:follow-knowledge} Output the distribution of the ``Yes'' and ``No'' cases of $\tilde{D}$ (which is a distribution generated by $\ALG$) as the distribution of the ``Yes'' and ``No'' cases of the problem in \Cref{lem:arm-learn}\;
		}
	\end{algorithm}
	
	The formal description of the reduction algorithm is as in \Cref{red:batch-arm-learn}. It is straightforward to observe that \Cref{red:batch-arm-learn} uses at most $\frac{1}{5}\cdot \frac{\tau^{3/2}}{\rho \cdot(\alpha+\beta)^2}$ arm pulls, as we terminate and output in \Cref{line:sample-ub-knowledge} otherwise. We now show that \Cref{red:batch-arm-learn} correctly ``learns'' the distribution of the arms with probability at least $2\tau^{1/2}$. To this end, we conduct the following case-based analysis:
	\begin{enumerate}[label=\roman*).]
		\item If the algorithm enters \Cref{line:sample-ub-knowledge}: we show that the algorithm reports ``Yes case'' correctly with probability at least $2\gamma^{1/2}$, which implies $\Pr\paren{\text{$D$ in \emph{Yes} case} \mid \Pi=\pi}=1\geq \rho + 2 \tau^{1/2}$. To see the desired statement, note that in the ``No case'', every arm becomes identical random variables. As such, similar to the proof of \Cref{lem:arm-trapping}, we can define $s_1$ and $s_2$ as the number of arm pulls used on $\arm_1$ and $\arm_2$, and show that
		\begin{align*}
			\expect{s_1 \mid \text{No case}} =\expect{s_2 \mid \text{No case}} \leq \frac{1}{200}\cdot \frac{\tau^2}{\rho\cdot (\alpha+\beta)^2}.
		\end{align*}
		As such, we have
		\begin{align*}
			\Pr\paren{s_1\geq \frac{1}{5}\cdot \frac{\tau^{3/2}}{\rho\cdot (\alpha+\beta)^2}} \leq \frac{\tau^{1/2}}{20}  \qquad \Pr\paren{s_2\geq \frac{1}{5}\cdot \frac{\tau^{3/2}}{\rho\cdot (\alpha+\beta)^2}} \leq \frac{\tau^{1/2}}{20}.
		\end{align*}
		Therefore, the probability for a transcript $\pi$ such that $\Pr\paren{\text{$D$ in \emph{Yes} case} \mid \Pi=\pi}=1$ is at least $1-\frac{\tau^{1/2}}{10}\geq 2\tau^{1/2}$ by the choice of $\tau\leq \frac{1}{5}$. 
		\item If the algorithm enters \Cref{line:follow-knowledge}, then by the guarantee of $\ALG$, we have that
		\begin{align*}
			\Pr_{\Pi}\paren{\Pr\paren{\text{$D$ in \emph{Yes} case} \mid \Pi=\pi}>\rho+2\tau^{1/2}} > 2\tau^{1/2}.
		\end{align*}
	\end{enumerate}
	Note that by using \Cref{lem:arm-learn} with $\eps=2\tau^{1/2}$, for the after mentioned bound to hold, at least $\frac{1}{2}\cdot \frac{\tau^{3/2}}{\rho\cdot (\alpha+\beta)^2}$ arm pulls are necessary. As such, it forms a contradiction with \Cref{red:batch-arm-learn}, which means such $\ALG$ cannot exist.
	
	\FloatBarrier
\end{proof}


\subsection{A Lower Bound Framework on Batched Distributions}

In our lower bound proof, we will crucially use a recent multi-pass lower bound tool developed by \cite{AW23BestArm}. The original lower bound construction of \cite{AW23BestArm} is on \emph{batched} instances distributions. On a high level, these distributions divide the arm into multiple batches with a fixed order. Inside each batch, most of the arms are ``flat'', i.e., with mean reward $\frac{1}{2}$, and one (or a few) \emph{special} arm(s) are planted uniformly at random among the indices. The reward distribution of the special arms is chosen randomly and independently between the reward of $\frac{1}{2}$ and $>\frac{1}{2}$. To make the instance hard, the distributions usually put batches whose special arm \emph{might} possess higher rewards to the late part of the stream. The intuition here is that to make sure the sample complexity upper bound is always followed, the streaming algorithm has to ``eliminate'' batches one by one in the reversed order of the stream.

The original analysis of \cite{AW23BestArm} was presented with only one specific distribution. In this section, we observe that their construction works for general batched distributions as long as they satisfy some properties. To this end, we formally define the \emph{batched} instances distributions.

\begin{definition}[Batched instance distributions]
	\label{def:batch-instance}
	Suppose the following information is given:  
	\begin{enumerate}[label=\roman*).]
		\item Positive integers $B\geq 2$, $C\geq 1$, $S \geq 1$; 
		\item A set of functions $F=\{f_{b}: \mathbb{N}^{+}\rightarrow (0,1)\}_{b=1}^{B+1}$ that computes $f_{b}(B)$ as a probability; 
		\item A set of tuples of positive real numbers $\mathrm{H} = \{(\etaib{1}{b}, \etaib{2}{b}, \cdots, \etaib{S}{b})\}_{b=1}^{B+1}$ from $(0, \frac12)$, and the values are (potentially) functions of $C$.
	\end{enumerate}
	We say instance distribution $\cD(B,C,F, \mathrm{H})$ is a $(B+1)$-\emph{batched instance distribution} if it satisfies the following properties:
	\begin{enumerate}[label=\alph*).]
		\item The arms are divided into $(B+1)$ batches;
		\item Inside each batch $b$, sample $S$ arms uniformly at random, and call them the \emph{special arms};
		\item All arms that are \emph{not} among the special arms follow the reward distribution $\bern{\frac{1}{2}}$.
		\item Sample a coin $\Theta_{b}\in \{0, 1\}$ from the distribution $\bern{f_{b}(B)}$ for each batch $b\in [B+1]$ \emph{independently}:
		\begin{itemize}
			\item If $\Theta_{b}=0$, set the special arms of batch $b$ with distribution $\bern{\frac{1}{2}}$.
			\item If $\Theta_{b}=1$, set the special arms of batch $b$ with distributions $\bern{\frac{1}{2}+\etaib{1}{b}}, \bern{\frac{1}{2}+\etaib{2}{b}}, \cdots, \bern{\frac{1}{2}+\etaib{S}{b}}$ (following an arbitrarily fixed order).
		\end{itemize}
	\end{enumerate}
\end{definition}

An illustration of batched instance distributions can be found in \Cref{fig:batched-instance}. For any $(B+1)$-batched instance distribution, it follows from the analysis of \cite{AW23BestArm} that the following proposition holds.

\begin{proposition}[\cite{AW23BestArm}, rephrased]
	\label{prop:multi-pass-lb}
	For a batched instance distribution $\cD(B,C,F, \mathrm{H})$ under the streaming setting, let the batches be arranged in the \emph{reversed} order of the stream arrival, i.e., $\mathcal{B}_{B+1}$ arrives first, and $\mathcal{B}_{1}$ arrives the last. Let the set of functions in $F$ be satisfying: 
	\begin{align*}
		f_{b}(B)=
		\begin{cases}
			\frac{1}{2B},\quad b\leq B;\\
			1, \quad b=B+1.
		\end{cases}
	\end{align*}
	For every batch $b\in [B+1]$, suppose w.log. that $\etaib{1}{b}\geq \etaib{s}{b}$ for any $s\in [S]$. Additionally, suppose $\cD(B,C,F, \mathrm{H})$ satisfies the following properties:
	\begin{itemize}
		\item \textbf{C1:} For each batch $\mathcal{B}_{b}$, \emph{conditioning on} $\Theta_{b}=1$, if an \emph{offline} algorithm uses at most $\frac{1}{700}\cdot \frac{\tau^3}{(\etaib{1}{b})^2}\cdot \frac{n}{B+1}$ arm pulls and outputs a collection of $\frac{1}{20}\cdot \frac{n}{B+1}\cdot \tau$ arms from $\mathcal{B}_{b}$, the probability for the output to contain any arm with reward \emph{strictly more than} $\frac12$ is at most $\tau$. % \chen{lemma 5}
		\item \textbf{C2:} For each batch $\mathcal{B}_{b}$, let $\nu$ be the probability for $\Theta_{b}=1$ (possibly conditioning on the information the algorithm obtains). Suppose we additionally obtain a new transcript $\pi$ with at most $\frac{1}{200}\cdot \frac{\tau^2}{\rho \cdot \left(\etaib{1}{b}\right)^2}\cdot \frac{n}{B+1}$ arm pulls. Then, with probability at least $1-2\tau^{1/2}$ over the randomness of $\pi$, there is
		\begin{align*}
			& \Pr\paren{\Theta_{b}=1 \mid \Pi=\pi} \in [\nu-2 \tau^{1/2}, \nu + 2 \tau^{1/2}];\\
			& \Pr\paren{\Theta_{b}=0 \mid \Pi=\pi} \in [1-\nu-2 \tau^{1/2}, 1-\nu + 2 \tau^{1/2}],
		\end{align*}
		% \nicksays{identical probabilities} chen: fixed
		% \chen{lemma 6}
		\item \textbf{C3:} For any $b, p$ such that $p>b$, there is $\etaib{1}{p}\leq (\frac{1}{6C B})^{15}\cdot \etaib{1}{b}$.
	\end{itemize}
	Let $\ALG$ be a deterministic streaming algorithm that uses $P\leq B$ passes and a memory of at most $\frac{1}{30000}\cdot \frac{n}{B^3}$ arms. Additionally, suppose $\ALG$ satisfies
	\begin{equation}
		\label{equ:batch-sample-ub}
		\expect{\smp\mid \Theta_{b}=1, \Theta_{<b}=0}\leq C \cdot B^2 \cdot \frac{n}{\left(\etaib{1}{b}\right)^2}.
	\end{equation}
	% \nicksays{what is \(\smp\)} chen: introduced in the additional notation
	Then, the probability for $\ALG$ to return the best arm is strictly less than $\frac{999}{1000}$.
\end{proposition}

\Cref{prop:multi-pass-lb} summarizes all the necessary conditions used by \cite{AW23BestArm}: in the analysis of \cite{AW23BestArm}, conditions \textbf{C1} and \textbf{C2} are used for the analysis the small-size sample case, and condition \textbf{C3} is used in the large-size sample case. Also, in the statement, there are two minor differences between \Cref{prop:multi-pass-lb} and the original theorem statement in \cite{AW23BestArm}:
\begin{enumerate}
	\item In \cite{AW23BestArm}, the result is only stated with $P=B$, i.e., using the number of passes directly as the parameter $B$. Here, we use $P\leq B$ since we work with an upper bound of $P$ that is only dependent on $n$.
	\item In \cite{AW23BestArm}, \Cref{equ:batch-sample-ub} does not have the $B^2$ factor. Nevertheless, it is evident from their proofs that we can add a $B$ factor on the sample bound.
\end{enumerate}
% For more details, we refer the readers to \Cref{sec:multi-pass-lb-tool} for a proof sketch of \Cref{prop:multi-pass-lb}.
% \chen{Reminder: conditioning on the all previous passes are batch- and memory-oblivious, the probability for the new pass to be batch- and memory-oblivious is roughly $(1-1/B)^C$.\\
	% Reminder 2: In our settings, $B=\log{n}/\log\log{n}$}




%% 
\section{Lower Bound: A Sharp Memory-pass Trade-off for Multi-pass Algorithms with Known $\Delta_{[2]}$}
\label{sec:lb-main}

We now introduce the construction and analysis of our main lower bound. 
Our adversarial instance follows the structure of batched instances as in \Cref{def:batch-instance}. On a high level, our instances keep \emph{two} special arms in each batch $b$ with stochastic mean rewards of either $\left(\frac{1}{2}, \frac{1}{2}\right)$ or $\left(\frac{1}{2}+\etaib{1}{b}, \frac{1}{2}+\etaib{2}{b}\right)$. In the latter case, which happens with probability roughly $O(1/B)$, we insist on \emph{invariate} $\etaib{1}{b}-\etaib{2}{b}$, which limits the utility for the knowledge of $\Delta_{[2]}$. 
% \nicksays{what is \(\Delta_{[2]}\), maybe better just say optimality gap in words everywhere or use \(\Delta_{[2]}\)?}. 
Furthermore, we carefully pick the parameters such that the gap between $C\cdot \frac{n}{\left(\etaib{1}{b}\right)^2}$ becomes $\polylog{n}$. Since we only work with a number of passes of $\Theta(\log(n)/\log\log(n))$, the construction allows us to ``reduce'' the $O\left(\sum_{i=2}^{n}\frac{1}{\Delta^2_{[i]}}\right)$ % \nicksays{We didn't define \(O(\sum_{i=2}^{n}\frac{1}{\Delta^2_{[i]}})\)}
sample complexity to the $C\cdot \frac{n}{\left(\etaib{1}{b}\right)^2}$ bound, which in turn allows us to use \Cref{prop:multi-pass-lb} to establish the lower bound.

We now give the formal construction of the instance family.

\begin{tbox}
	$\cP(B, C, \gamma)$: A hard instance distribution for multi-pass MABs algorithms with known $\Delta_{[2]}$. 
	
	\begin{enumerate}
		\item \textbf{Parameters}: Ensure that $\frac{1}{20}\cdot \frac{1}{n^{1/3}}\leq \gamma \leq \frac{1}{10}\cdot \frac{1}{n^{1/3}}$, and let $\chi_{1}=n^{1/3}\cdot \gamma$; furthermore, for any $b\in [B]$, let 
		\[\chi_{b+1} = \paren{\frac{1}{12 C \log(n)}}^{15} \cdot \chi_{b}.\]
		\item \textbf{Division of arms:} Divide the $n$ arms into $(B+1)$ batches of equal sizes, and put them in the \emph{reverse} order of the stream, i.e. $\mathcal{B}_{B+1}$ arrives first, and $\mathcal{B}_{1}$ arrives the last.
		\item \textbf{Sampling special arms: } For each batch $b\in [B+1]$, sample \emph{two} arms uniformly at random (without replacement), and call them \emph{special arms}. Set all the arms \emph{except} the special arms with reward distribution $\bern{1/2}$.
		\item \textbf{Batches $b\in[B]$: } For each $b\in [B]$, sample $\Theta_{b}$ from distribution $\bern{1/2B}$:
		\begin{enumerate}
			\item If $\Theta_{b}=0$, set both special arms with reward distributions $\bern{1/2}$.
			\item Otherwise, if $\Theta_{b}=1$ 
			\begin{itemize}
				\item Set the first special arm with reward distribution $\bern{1/2+\chi_{b}}$.
				\item Set the second special arm with reward distribution $\bern{1/2+\chi_{b}+\gamma}$.
			\end{itemize}
		\end{enumerate}
		\item \textbf{The batch $B+1$: } Always set the reward distributions of the special arms as follows ($\Theta_{B+1}=1$ deterministically) % \nicksays{constant?}) 
		\begin{itemize}
			\item Set the first special arm with reward distribution $\bern{1/2+\chi_{B+1}}$.
			\item Set the second special arm with reward distribution $\bern{1/2+\chi_{B+1}+\gamma}$.
		\end{itemize}
	\end{enumerate}
\end{tbox}


\begin{figure}
	\centering
	\begin{subfigure}{0.95\textwidth}
		\centering
		\includegraphics[scale=0.18]{figs/batched-instance-general.png}
		\caption{General $(B+1)$-Batched Instance Distribution}
		\label{fig:batched-instance}
	\end{subfigure}%
	% leave a blank line to change row 
	
	\begin{subfigure}{0.95\textwidth}
		\centering
		\includegraphics[scale=0.18]{figs/streaming-adversarial-instance.png}
		\caption{$\cP(B, C, \gamma)$ Instance distribution}
		\label{fig:streaming-adv-instance}
	\end{subfigure}
	\caption{An illustration of the general $(B+1)$-batched instance distribution (\Cref{def:batch-instance}) and the $\cP(B, C, \gamma)$ instance distribution. The mean rewards of arms are ranked in the decrement order from left to right for illustration purposes -- their positions inside the batches are uniformly at random.}
	\label{fig:lb-instance-illus}
\end{figure}


An illustration of the distribution $\cP(B, C, \gamma)$ can be shown as \Cref{fig:streaming-adv-instance}. It is straightforward to observe that the $\cP(B, C, \gamma)$ family follows the $(B+1)$-batched instance as in \Cref{def:batch-instance}. More concretely, in $\cP(B, C, \gamma)$, the arms are divided into $(B+1)$ batches, we have $S=2$ and the values of $\etaib{i}{b}$ as functions of $C$, and the probability functions are $f_{b}(B)=\frac{1}{2B}$ for all $b \in [B]$ and $f_{B+1}(B)=1$. Furthermore, we make the crucial observation that $\Delta_{[2]}$ is invariant across different settings.

\FloatBarrier




\begin{observation}
	\label{obs:Delta-invariate}
	For any instance in $\cP(B, C, \gamma)$, the value of $\Delta_{[2]}$ is equal to $\gamma$. In other words, in $\cP(B, C, \gamma)$, for all $b\in[B+1]$, there is 
	\begin{align*}
		\paren{\Delta_{[2]} \mid \Theta_{<b}=0, \Theta_{b}=1} = \gamma.
	\end{align*} % \nicksays{Maybe better to say this in words?}
\end{observation}

% \chen{Factor in the knowledge of $\Delta$, which is not written explicitly in the current version.}

We now use $\cP(B, C, \gamma)$ to state our main multi-pass lower bound. 
\begin{theorem}[Formalization of \Cref{rst:main-lb}]
	\label{thm:lb-main}
	% For any $1\leq P \leq \frac{1}{100}\cdot \frac{\log(n)}{\log\log(n)}$, 
	There exists a family of streaming MABs instances $\cP$, such that any streaming algorithm (deterministic or randomized) that given the quantity of $\Delta_{[2]}$, finds the best arm from an instance sampled from $\cP$ with an \emph{expected} sample complexity of $O\paren{\sum_{i=2}^{n} \frac{1}{\Delta^2_{[i]}} \cdot \log(n)}$, a success probability of at least $1999/2000$, and a memory of $o\paren{n/\log^3 {n}}$ arms has to make $\Omega\paren{\frac{\log(n)}{\log\log(n)}}$ passes over the stream. 
\end{theorem}


To prove \Cref{thm:lb-main}, the rest of this section is dedicated to two parts. We first show that the family of $\cP(B, C, \gamma)$ \emph{with $B=\Theta(\frac{\log n}{\log\log n})$} satisfies the conditions characterized by \Cref{prop:multi-pass-lb}. As such, to ensure the success probability is high, any algorithm must break the sample upper bound of \Cref{equ:batch-sample-ub}. Subsequently, we show that to keep the expected sample complexity of $O(\sum_{i=2}^{n}\frac{1}{\Delta^2_{[i]}})$, % \nicksays{remove this}, 
the sampling upper bound of \Cref{equ:batch-sample-ub} has to be satisfied, which forms a contradiction for the proof of \Cref{thm:lb-main}.

More concretely, the first part of the argument can be summarized as \Cref{lem:hard-B-dist}. (Note that in the lemma, we do not assume the knowledge of $\Delta_{[2]}$, as we will deal with it in the proof of \Cref{thm:lb-main} later.)
\begin{lemma}
	\label{lem:hard-B-dist}
	Let $C\geq 1$ be a fixed integer and $\gamma \leq \frac{1}{10}\cdot \frac{1}{n^{1/3}}$ be a real number. Let $B=\frac{1}{100C}\cdot \frac{\log n}{\log\log(n)}$, and let $\ALG$ be any deterministic $P$-pass streaming algorithm such that $P\leq B$. Suppose that $\ALG$ uses a memory of at most $\frac{1}{30000}\cdot \frac{n}{B^3}$ arms. 
	Additionally, suppose on instances of distribution $\cP(B, C, \gamma)$ and every $b \in [B+1]$, $\ALG$ satisfies:
	\[
	\Exp\bracket{\smp \mid\Theta_{b}=1, \Theta_{<b}=0} \leq C \cdot B^2 \cdot \frac{n}{(\chi_{b}+\gamma)^2}, 
	\]
	where the randomness is taken over the choice of the instance $I \sim \cP(B, C, \gamma) \mid \Theta_{b}=1, \Theta_{<b}=0$. % \nicksays{we need brackets, what is \(\smp\)?} 
	Then, the probability that $\ALG$ can output the best arm for $I \sim \cP(B, C, \gamma)$ is strictly less than $999/1000$.  
\end{lemma}
\begin{proof}
	We prove the lemma by showing that the distribution $\cP(B, C, \gamma)$ satisfied the conditions prescribed by \Cref{prop:multi-pass-lb}, which will allow us to directly use the conclusion therein. To this end, we verify \textbf{C1}, \textbf{C2}, and \textbf{C3}, respectively:
	\begin{itemize}
		\item Condition \textbf{C1}. We use \Cref{lem:arm-trapping} to argue this property. For any batch $b\in [B+1]$, we use $\alpha=\chi_{b}$ and $\beta=\gamma$. Since we set $\gamma\leq \frac{1}{10}\cdot \frac{1}{n^{1/3}}$, there is clearly $\chi_{b}+\gamma<\frac{1}{6}$. Furthermore, we let $k=\frac{n}{B+1}$ as the number of arms in each batch. Suppose for the purpose of contradiction that \textbf{C1} does \emph{not} hold. By our construction, we have $\etaib{1}{b}=\chi_{b}+\gamma$, and the assumption implies an algorithm that
		\begin{enumerate}
			\item uses at most $\frac{1}{700}\cdot \frac{\tau^3}{(\chi_{b}+\gamma)^2}\cdot \frac{n}{B+1}<\frac{1}{600}\cdot\frac{\tau^3}{(\alpha+\beta)^2}\cdot k$ arm pulls;
			\item outputs a collection of $\frac{1}{20}\frac{\tau n}{B+1}=\frac{\tau \cdot k}{20}$ arms, in which contains an arm with reward strictly more than $\frac{1}{2}$ with probability at least $\tau$,
		\end{enumerate}
		which forms a contradiction with \Cref{lem:arm-trapping}. Therefore, the condition \textbf{C1} has to be satisfied.
		
		\item Condition \textbf{C2}. We use \Cref{lem:batch-arm-learning} to verify this property. Again, we use $\alpha=\alpha_{b}$ and $\beta=\gamma$. Furthermore, let $\mathcal{E}$ be any event we want to condition on, and we let $\nu=\Pr\paren{\Theta_{b}=1\mid \mathcal{E}}$ be the probability for the distribution in the \emph{yes} case from the algorithm's internal view. Now, to use \Cref{lem:batch-arm-learning}, we simply set $\rho=\nu$, and if condition \textbf{C2} is not satisfied, the output distribution of the transcripts will violate \Cref{lem:batch-arm-learning}. Thus, condition \textbf{C2} must be followed in $\cP(B, C, \gamma)$.
		
		\item Condition \textbf{C3}. Note that we have $B= \frac{1}{100C}\cdot \frac{\log n}{\log\log n}$. As a result, we also have 
		\begin{equation}
			\label{equ:a-B-lower-bound}
			\begin{aligned}
				\chi_{b}\geq \chi_{B} & = \gamma \cdot n^{1/3} \cdot \paren{\frac{1}{12 C\log n}}^{\frac{10\log n}{100 C\log\log(n)}}\\
				&= \gamma \cdot n^{1/3} \cdot \paren{\frac{1}{12C}}^{\frac{\log n}{10C \log\log(n)}} \cdot \paren{\frac{1}{\log{n}}}^{\frac{\log n}{10C \log\log(n)}}\\
				& \geq \gamma \cdot n^{1/3}\cdot \paren{\frac{1}{2}}^{^{\frac{\log n}{\log\log(n)}}}\cdot \paren{\frac{1}{\log n}}^{\frac{\log n}{10C \log\log(n)}}\\
				& = \gamma \cdot n^{1/3}\cdot \frac{1}{n^{1/10C + o(1)}}\\
				& \geq \gamma\cdot n^{1/5},
			\end{aligned}
		\end{equation}
		where the second inequality if because $(\frac{1}{12C})^{\frac{1}{10C}}\geq \frac{1}{2}$ for any $C\geq 1$.
		Therefore, we have $\gamma\leq n^{-1/5}\chi_{b}$ for any choice of $\gamma$. As such, for any $r>b$, there is 
		\begin{align*}
			\frac{\etaib{1}{r}}{\etaib{1}{b}} &\leq \frac{\chi_{b+1}+\gamma}{\chi_{b}+\gamma}\\
			&\leq \frac{\chi_{b+1}\cdot \log n}{\chi_{b}} \tag{by $\gamma\leq n^{-1/5}\chi_{b}$ for sufficiently large $n$}\\
			&\leq \paren{\frac{1}{12 C \log(n)}}^{15} \\
			&\leq \paren{\frac{1}{6 B C}}^{15}, \tag{by $B\leq \log{n}$}
		\end{align*}
		which verifies the validity of condition \textbf{C3}.
	\end{itemize} 
	Finally, we observe that the memory and sample bound in \Cref{lem:hard-B-dist} matches the bound for deterministic algorithms in \Cref{prop:multi-pass-lb}, concluding the proof. 
\end{proof}

We are now ready to wrap up the proof of our main lower bound of \Cref{thm:lb-main}.

\begin{proof}[Proof of \Cref{thm:lb-main}]
	% \chen{There is a small hidden bug -- fix the issue.}
	We focus on deterministic algorithms in the proof with success probability $\frac{999}{1000}$. The lower bound for randomized algorithms can be obtained by an application of Yao's minimax principle.
	
	We first deal with algorithms that do \emph{not} have the \emph{a priori} knowledge of $\Delta_{[2]}$. Assume the purpose of contradiction that there exists a $P$-pass streaming algorithm that uses
	\begin{enumerate}
		\item A memory of at most $\frac{1}{20000}\cdot \frac{n}{\log^3 n}$ arms;
		\item A success probability of at least $\frac{999}{1000}$;
		\item A sample complexity of $C' \cdot \log(n)\cdot \sum_{i=2}^{n}\frac{1}{\Delta^2_{[i]}}$;
		\item The number of passes $P$ satisfies $P\leq \frac{1}{100C}\cdot \frac{\log(n)}{\log\log(n)}$.
	\end{enumerate}
	Then, we can pick $C=2C'$ to construct the hard family $\mathcal{P}(\frac{1}{100C}\cdot \frac{\log(n)}{\log\log(n)}, C, \gamma)$ (pick any suitable $\gamma$). Observe that following the same calculation of \Cref{equ:a-B-lower-bound} and by the property of the distribution, for any $b\in[B+1]$, there is
	\begin{align*}
		\chi_{b}\geq \chi_{B} & = \gamma \cdot n^{1/3} \cdot \paren{\frac{1}{12 C\log n}}^{\frac{10\log n}{100 C\log\log(n)}}\\
		&= \gamma \cdot n^{1/3} \cdot \paren{\frac{1}{12C}}^{\frac{\log n}{10C \log\log(n)}} \cdot \paren{\frac{1}{\log{n}}}^{\frac{\log n}{10C \log\log(n)}}\\
		& \geq \gamma \cdot n^{1/3}\cdot \paren{\frac{1}{2}}^{^{\frac{\log n}{\log\log(n)}}}\cdot \paren{\frac{1}{\log n}}^{\frac{\log n}{10C \log\log(n)}}\\
		& = \gamma \cdot n^{1/3}\cdot \frac{1}{n^{1/10C + o(1)}}\\
		& \geq \gamma\cdot n^{1/5}.
	\end{align*}
	Furthermore, by the upper and lower bound on $\gamma$, we have that for $b\in[B+1]$, there is
	\begin{align*}
		\frac{1}{\gamma^2} &\leq 400 \cdot n^{2/3} \tag{by $\gamma\geq \frac{1}{20}\cdot \frac{1}{n^{1/3}}$}\\
		&\leq 50 \cdot n \tag{for sufficiently large $n$}\\
		&\leq \frac{n}{(\chi_{1}+\gamma)^2} \tag{$\chi_{1}\leq \frac{1}{10}$ and $\gamma \leq \frac{1}{n^{1/5}}\chi_{1}$}\\
		&\leq \frac{n}{(\chi_{b}+\gamma)^2} = \frac{n}{(\etaib{1}{b})^2}. \tag{$\chi_{b}\leq \chi_{1}$} 
	\end{align*}
	As such, it is straightforward to see that
	\begin{align*}
		\paren{C'\cdot \log(n) \cdot \sum_{i=2}^{n}\frac{1}{\Delta^2_{[i]}} \middle| \Theta_{b}=1, \Theta_{<b}=0} &\leq C'\cdot \log(n) \cdot \left(\frac{1}{\gamma^2} + \frac{n}{(\etaib{1}{b})^2}\right)\\
		&\leq 2 C'\cdot \log(n)\cdot \frac{n}{(\etaib{1}{b})^2}.
	\end{align*}
	Therefore, if a deterministic algorithm $\ALG$ satisfies: 
	\[\Exp\bracket{\smp} \leq C' \cdot \log(n)\cdot \sum_{i=2}^{n}\frac{1}{\Delta^2_{[i]}},\] it implies that the algorithm has sample complexity of 
	\[\Exp\bracket{\smp \mid \Theta_{b}=1, \Theta_{<b}=0}\leq C \cdot \log(n) \cdot \frac{n}{(\etaib{1}{b})^2}.\]
	This leads to a contradiction with \Cref{lem:hard-B-dist}, which implies that such a streaming algorithm cannot exist.
	
	Finally, to complete the proof, we note that on the constructed family $\mathcal{P}(\frac{1}{100C}\cdot \frac{\log(n)}{\log\log(n)}, C, \gamma)$, there is always $\Delta_{[2]}=\gamma$ by \Cref{obs:Delta-invariate}. As such, if we have a streaming algorithm $\widetilde{\ALG}$ that only works with a \emph{known} $\Delta_{[2]}$, we can simulate the algorithm without this prior knowledge by running $\widetilde{\ALG}$ as an inner streaming algorithm with parameter $\Delta_{[2]}=\gamma$. The contradiction still holds, concluding the proof.
\end{proof}


\begin{remark}
	By Remark 5.8 of \cite{AW23BestArm}, the $B^2$ term in \Cref{equ:batch-sample-ub} could be made to $B^C$ for any fixed constant $C$ with larger gaps between $\etaib{1}{b}$ values for different $b\in[B]$. Therefore, we could strenghthen the sample complexity lower bound in \Cref{thm:lb-main} to $O\paren{\sum_{i=2}^{n} \frac{1}{\Delta^2_{[i]}} \cdot \polylog(n)}$.
\end{remark}



%%
%\section{Upper bound}
%\label{sec:ub}
% \subsection{Technical Overview}

\section{Upper Bound: A Multi-pass Pure Exploration Streaming MABs Algorithm with Known $\Delta_{[2]}$}\label{sec:basic}
A natural question to follow from our lower bound in \Cref{sec:lb-main} is whether this bound is tight.
In this section, we show our main upper bound result that nearly matches our lower bound in \Cref{sec:lb-main}. In particular, we prove the following theorem.
\begin{theorem}[Formalization of \Cref{rst:main-ub}]
	\label{thm:basic}
	For any $P\geq 1$, there exists a $(P + 1)$-pass streaming algorithm that given a streaming MABs instance and a known value of $\Delta_{[2]}$, finds the best arm with probability at least $1-\delta$ with a single-arm memory and at most \[O\left(\log \left(\frac{n P}{\delta}\right) \sum_{i = 2}^n \frac{n^{2/P}}{\Delta^2_{[i]}}\right)\]
	arm pulls. 
\end{theorem}

Note that by plugging in $P=\Theta(\log(n))$, \Cref{thm:basic} gives an $O(\log(n))$-pass algorithm with $O(\sum_{i = 2}^n \frac{1}{\Delta^2_{[i]}}\cdot \log{n})$ sample complexity, as we have stated in \Cref{rst:main-ub}.

We describe the general algorithm that could `adjust' the sample complexity bound with the number of passes as a parameter, see \Cref{alg:main}. 
Our approach is similar to the elimination algorithm in~\cite{KarninKS13}, and by leveraging the information about \(\Delta_{[2]}\), we can significantly reduce the number of passes. 
The algorithm proceeds in \(P  +1\) passes over the stream, and maintains a set $I_p$ of `active arms' on pass $p$. In pass \(p\), it samples each arm in the current set \(I_p\) for a carefully chosen \(T_p\) times. After sampling, it computes the estimated mean reward \(\hat{\mu}^{p}_i\) for each arm \(i\) in \(I_p\), and find the maximum estimated mean \(\hat{\mu}^{p}_{\max}\) among the arms in $I_p$. The algorithm then constructs a new set \(I_{p + 1}\) by eliminating arms whose estimated means are more than \(\epsilon_p\) below \(\hat{\mu}^{p}_{\max}\). After \(P + 1\) passes, if the set \(I_{P + 1}\) contains a single arm, the algorithm returns that arm as the best arm. 

% \chen{Based on our model, with the existence of $\pi$, we never need to explicitly maintain the set $I$ -- add a remark about it.}\nicksays{I added this these words in \Cref{lem:main-alg-memory}, but probably it is not the best place} 

\begin{algorithm}
    \caption{The Main Multi-pass Streaming Algorithm}\label{alg:main}
    \KwIn{Stream \(I\), parameter \(P\), gap parameter \(\Delta_{[2]}\), and confidence parameter \(\delta\)}
    \KwOut{Best arm}
    Set \(n \gets \abs{I}\) and \(I_0 \gets \{1, \dotsc, n\}\)\;
    Let \(\epsilon_p = n^{1-p/P}\Delta_{[2]} / 4\) for \(p = 0, \dotsc, P\) \;
    \For{\(p = 0, \dotsc, P\)}{
        \ForEach{\(i \in I\) in the arrival order}{
            \If{\(i \not\in I_p\)}{
                Skip arm\;
            }
            Pull arm \(i\) until the number of pulls reach \(T_p \triangleq \frac{8 \log(2n (P + 1) / \delta)}{\epsilon^2_r \log e}\) times\;
            Compute estimated mean \(\hat{\mu}^p_i\) after \(T_p\) pulls\;
        }
        % Pull each arm from \(I_r\) until the number of pulls reach 
        % Compute estimated mean \(\hat{\mu}^r_i\) after \(T_r\) pulls for each arm from \(I_r\)\;
        Pick \(\hat{\mu}^p_{\max} = \max\limits_{i \in I_r}\{\hat{\mu}^p_i\}\)\;
        Create a new set \(I_{p + 1} \gets \{i \in I_p \mid \hat{\mu}^p_i \ge \hat{\mu}^p_{\max} - {\epsilon_p}\}\)\;
    }
    \If{\(I_{P + 1}\) contains one element}{
        \Return{single index of arm from \(I_{P + 1}\)}\;
    }
\end{algorithm}
% In the rest of the section we prove the following theorem \chen{which theorem?}


It is easy to observe that the streaming algorithm (\Cref{alg:main}) maintains only a single arm memory across the passes. Formally:
\begin{lemma}\label{lem:main-alg-memory}
The memory of \Cref{alg:main} contains at most one arm at any point of the stream.
\end{lemma}
\begin{proof}
The algorithm initializes the set \(I_0\) to contain all the arm indices from the stream \(I\). However, this set doesn't actually store the arms themselves, only their indices. The arms are fetched one by one during the execution. 

For each pass number \(p\), from \(0\) to \(P\), the algorithm iterates through each arm \(i\) in the stream \(I\) and loads it to the memory. During this iteration, the algorithm pulls from it a fixed number of times, updates statistics \(\hat\mu^{p}_i\), and then moves on to the next arm. Therefore, at any given time, the memory contains only one arm, which is the arm currently being proceeded. 
We note that the whole information \(\hat\mu^p_i\) and \(I_p\) can be extracted from the transcript \(\Pi\), and we maintain this information explicitly only for the convenience of the presentation.  
\end{proof}


What remains is to prove the correctness and the sample complexity of \Cref{alg:main}. For simplicity and to remove any issues with dependency, in the analysis, we apply the following standard trick. We assume that we first extract \(T_P\) samples from each arm before the algorithm even starts the work. During the work, \Cref{alg:main} only utilize these samples for the computation. In other words, uses a prefix of length \(T_p\) for computation \(\hat\mu^{p}_i\). % One might wonder about the meaning or purpose of this approach. 
This strategy provides a structured and consistent set of samples and values \(\hat\mu^{p}_i\) for the algorithm's decisions, avoiding any dependency from continuously updated statistics.


%Without loss of generality we can assume that we first make \(T_{R}\) samples from each distribution, and the algorithm only reveals these samples \chen{What does this sentence mean?}. 
We first define a `good event' that captures the high-probability bound for the empirical rewards to deviate from the actual rewards with $T_{p}$ number of arm pulls. More formally, we define event $\cE$ as follows.

\begin{align}\label{eq:event-E}
    \E \triangleq \{\forall{i \in I}, p \in \{0, \dotsc, P\} : \abs{\hat\mu^p_i - \mu_i} \le \epsilon_r / 4\}
\end{align}

We show that $\E$ holds probability at least $(1-\delta)$, which is by an application of the Chernoff-Hoeffding bound (\Cref{lem:chernoff}). The formal lemma and proof are as follows.
\begin{lemma}\label{lem:bound-E}
    Let \(\E\) be the event defined in \Cref{eq:event-E}. Then, \[\Pr\left[~\neg \E~\right] \le \delta \,.\]
\end{lemma}
% \chen{The $\mathcal{E}^{\complement}$ notation is not defined. Also, this is an event, I think this notation is for sets only, no?}
% \chen{suggestion: $\Pr\left[~\neg \E~\right] \le \delta$  }
\begin{proof}
    By the union bound, we have:
    \begin{align}\label{eq:event-sum}
        \Pr\left[\neg {\E}\right] \le \sum_{i \in I} \sum_{p = 0}^{P} \Pr\left[\abs{\hat{\mu}^p_i - \mu_i} > \epsilon_p/4\right]\,.
    \end{align}
    % \chen{And suddenly here the notation changes to $\bar{\E}$! :(}
    
    By the Chernoff-Hoeffding inequality (\Cref{lem:chernoff}), we have:
    \begin{align}\label{eq:event-single}
        \Pr\left[\abs{\hat{\mu}^p_i - \mu_i} > \frac{\epsilon_p}{4} \right] \le 2 \exp\left(-\frac{\epsilon_p^2 T_p}{8}\right) \le 2\exp\left(-\ln\left(\frac{n(P + 1)}{\delta}\right)\right) \le \frac{\delta}{n (P + 1)}\,.
    \end{align}
    
    Combining~\Cref{eq:event-sum} and~\Cref{eq:event-single}, we get:
    \begin{equation*}
        \Pr\left[\neg \E\right] \le n (P + 1) \frac{\delta}{n(P + 1)} \le \delta\, ,
    \end{equation*}
    as desired.
 % \chen{Please make sure the notation is unified in this proof.}
\end{proof}

In the rest of the proof, we condition on the high probability event that $\E$ happens. We now establish the correctness of the algorithm by \Cref{lem:best} and \Cref{lem:eliminate-large-gap}. Our target is to demonstrate that the algorithm correctly identifies the best arm in stream \(I\). We prove that if the event \(\E\) holds, then the algorithm outputs the best arm. Combining this result with the bound from \Cref{lem:bound-E}, we get that the probability of the algorithm making an incorrect identification is at most \(\delta\), thus affirming the algorithm's correctness.

We first prove that the best arm cannot be eliminated during the work of the algorithm. 
\begin{lemma}\label{lem:best}
    Conditioning on the event \(\E\) defined in \Cref{eq:event-E} holds, for any \(p \in \{0, \dotsc, P + 1\}\), we have \(\star \in I_p\). 
\end{lemma}

\begin{proof}
    To prove that the best arm \(\star\) cannot be eliminated in the algorithm, we start by assuming the opposite. Assume that the best arm \(\star\) belongs to \(I_p\), but does not belong to \(I_{p + 1}\) for some \(r\). This implies the existence of an arm \(i\) for which the following inequality holds 
    \begin{equation}\label{eq:upper-bound}
        \hat{\mu}^{p}_\star < \hat{\mu}^{p}_i - \epsilon_p \,.
    \end{equation}

    However, from the property of the event \(\E\), we have \(\hat{\mu}^{p}_\star \ge \mu_\star - \epsilon_p / 4 \) and \(\hat{\mu}^p_i \le \mu_i + \epsilon_p / 4\).
    Consequently, we get: 
    \begin{equation}\label{eq:lower-bound}
        \hat{\mu}^p_\star - \hat{\mu}^p_i \ge \mu_\star - \mu_i -\epsilon_p / 2 \ge -\epsilon_p / 2\,.
    \end{equation}

    The~\Cref{eq:upper-bound} and~\Cref{eq:lower-bound} contradict each other, leading to a contradiction. Therefore, it is not possible for \(\star\) to be in \(I_p\) and eliminate the subsequent round, and as a result, \(\star\) remains in \(I_p\) for any \(p\). Thus, the best arm \(\star\) is always present in the set \(I_p\) for any \(p\) and consequently in set \(I_{P + 1}\). 
\end{proof}
% \chen{I'm not sure \Cref{lem:best} is sufficient to prove the algorithm returns the optimal arm -- we need to say that the size of $I_{R}$ is also $1$, for which we need to combine \Cref{lem:best} and \Cref{lem:eliminate-large-gap}. }


Next, we demonstrate that if event \(\cE\) holds, then an arm with a large gap should be eliminated before a specific iteration in Algorithm~\ref{alg:main}. In other words, if the condition \(\cE\) is satisfied, the algorithm will identify and eliminate arms with significant gaps with a specific number of iterations. 
% In what follows, for simplicity of notation, we use the notation $\Delta_i$ as the gap between the best arm and the arm with index $i$ (\emph{not to be confused with $\Delta_{[i]}$}).

\begin{lemma}\label{lem:eliminate-large-gap}
    Conditioning on the event \(\cE\) defined in \Cref{eq:event-E} holds, and suppose for arm \(i\) and an integer $p\in [P+1]$, there is \(\Delta_i > \frac{3}{2}\epsilon_p\), then \(i \notin I_{p + 1}\).
\end{lemma}

\begin{proof}
    Consider any arm \(i\) and a value \(r\) such that \(\Delta_i > \frac{3}{2} \epsilon_p \). We aim to prove that arm \(i\) will not be included in the set \(I_{p  +1}\). We can first assume that \(i\in I_p\) since otherwise, if \(i\) is not in \(I_p\), arm \(i\) cannot be included in \(I_{p + 1}\) simply by set inclusion.

    Assuming \(i\) is in \(I_p\), by using \Cref{lem:best} and the event \(\cE\), we have the following inequality:
    \begin{equation}\label{eq:a1}
        \hat{\mu}^{p}_{\max} \geq \hat{\mu}^{p}_{\star} \geq \mu_\star - \epsilon_p / 4\,.
    \end{equation}

    This inequality indicates that the maximum estimated mean \(\hat{\mu}^{p}_{\max}\) in \(p\)-th iteration is at least as large as the estimated mean \(\hat{\mu}^p_\star\), of the best arm \(\star\), which in turn, is at least \(\mu_\star - \epsilon_p / 4\) according to the event \(\cE\). 

    Furthermore, according to the event \(\cE\), we have:
    \begin{equation}\label{eq:a2}
        \hat{\mu}^p_i \leq \mu_i + \epsilon_p / 4\,.
    \end{equation}
    
    By combining~\Cref{eq:a1} and~\Cref{eq:a2}, we obtain: 
    \begin{align*}
        \hat{\mu}^{p}_i - \hat{\mu}^{p}_{\max} + \epsilon_p \le \mu_{i} - \mu_{\star} + \epsilon_p / 2 + \epsilon_p = \frac{3}{2} \epsilon_p - \Delta_i < 0.
    \end{align*}
    The above inequality shows that \(\hat{\mu}^p_i < \hat{\mu}^p_{\max} -\epsilon_p\) holds due the large gap \(\Delta_i = \mu_{\star} - \mu_i > \frac{3}{2}\epsilon_{p}\). 
    Consequently, if \(i\) fails to meet the condition for inclusion in \(I_{p + 1}\), \(i\) will not present in \(I_{p + 1}\). This concludes the proof of the lemma. 
\end{proof}

As \(\epsilon_P \le \Delta_{[2]}/4\) and for any \(i \in I\) we have \(\Delta_i > \Delta_{[2]} \ge 4\epsilon_P > \frac{3}{2}\epsilon_P\), it follows that \(i\) satisfies the conditions of~\Cref{lem:eliminate-large-gap}. By combining~\Cref{lem:best} and~\Cref{lem:eliminate-large-gap}, we deduce that if event \(\mathcal{E}\) holds true, then \(I_{P + 1} = \{\star\}\), and \Cref{alg:main} outputs the correct arm.

We next bound the number of samples that \Cref{alg:main} makes.

\begin{lemma}\label{lem:bound-pull}
    Conditioning on the event \(\E\) defined in \Cref{eq:event-E} holds, the sample complexity of \Cref{alg:main} is at most 
    \begin{equation*}
        O\left(\log \left(\frac{nP}{\delta} \right) \sum_{i =2}^n \frac{n^{2/P}}{\Delta^2_{[i]}}\right)\,.
    \end{equation*}
\end{lemma}

\begin{proof}
	% \chen{It seems on Line~4 we do not resample -- does that creat a dependency issue? If we resample that number of times we only pay a constant overhead.}\nicksays{I extended sentence, hope it will work better}
    For any suboptimal arm $\arm_{i}$, define the value \(p(i) \triangleq \min\{p \ge 0 \mid \Delta_i > \frac{3}{2}\epsilon_p\}\). For an optimal arm, we define \(p(\star) \triangleq P\). We note that it is correctly defined because \(\frac{3}{2}\epsilon_P = \frac{3}{2} \frac{\Delta_{[2]}}{4} < \Delta_{[2]} \le \min_i \{\Delta_i\}\), which means \(\forall i : p(i) \le P\). We split the set of arms into two parts. The first set is the set of arms with small gaps
    \begin{align*}
        S \triangleq \{i \in I\mid \Delta_i \le 3 n \Delta / 2\},
    \end{align*}
    and the set of arms with big gaps
    \begin{align*}
        B \triangleq \{i \in I\mid \Delta_i > 3 n \Delta / 2\}\,.
    \end{align*}
    \noindent
    We define \(T_{S}\) and \(T_{B}\) as the number of arm pulls used by the sets \(S\) and \(B\), and \(T\triangleq T_{S}+T_{B}\) is the total number of arm pulls. We bound the two terms in what follows. 
    Furthermore, we first look at arms from \(S\) \emph{except} \(\armstar\). For \(\arm_{i}\) with gap parameter \(\Delta_{i}\), due to the definition of \(\epsilon_p\), we have that 
    \begin{align*}
        \frac{3}{2} n^{1/P} \epsilon_{p(i)}\ge \Delta_i > \frac{3}{2} \epsilon_{p(i)},
    \end{align*}
    and consequently,
    \begin{eqnarray}\label{eq:eps-lower}
        \epsilon_{p(i)} \ge \frac{2\Delta_i}{3 n^{1/P}}\,.
    \end{eqnarray}

    By \Cref{lem:eliminate-large-gap}, we have that the number of pulls for \(\arm_{i}\) is bounded by \(T_{p(i)}\). By the definition of \(T_{p}\) and~\Cref{eq:eps-lower}, we have

    \begin{align}\label{eq:basic-small}
        T_{p(i)} = \frac{8 \log(2 n (P + 1) / \delta)}{\epsilon^2_{p(i)} \log e} \le \frac{72 \ln (2 n (P + 1) / \delta) n^{2/P}}{4 \Delta^2_i \log e}\,.
    \end{align} 
    
    For the optimal arm, by \Cref{lem:best} and \Cref{lem:eliminate-large-gap}, the number of pulls assigned to the optimal arm is equal to
    \begin{align}\label{eq:basic-optimal}
        T_{P} = \frac{8 \ln (2 n (P + 1) / \delta)}{\epsilon^2_{P} \log e} = \frac{128 \ln (2 n (P + 1) / \delta)}{\Delta_{[2]}^2 \log e}.
    \end{align}
    % Thus, the total number of pulls  for arms from \(S\) is bounded by 
    % \begin{align}
        % \sum_{i \neq \star} \frac{72 \ln (16 n (P + 1) / \delta) n^{2/P}}{4 \Delta^2_i} + \frac{128 \ln (16 n (P + 1) / \delta)}{\Delta^2} = O\left(\ln (16 n (P + 1) / \delta) \sum_{i \in I} \frac{1}{\Delta^2_i}\right)
    % \end{align}

    All arms from the set \(B\) are eliminated after the first round, the number of pulls assigned in the first round is bounded by \(T_0 \le \frac{128 \ln (2 n (P + 1) / \delta)}{n^2\Delta_{[2]}^2}\). Thus, the total number of pulls assigned for arms from \(B\) is at most 
    \begin{align}\label{eq:basic-large}
        T_B \leq nT_0 \le \frac{128 \ln (2 n (P + 1) / \delta)}{n\Delta_{[2]}^2}.
    \end{align}

    Therefore, we have 
    \begin{align*}
    	T &= T_{S}+T_{B}\\
    	  &\leq T_{P} + \sum_{i\neq \star} T_{r(i)} + T_{B} \tag{by \Cref{lem:eliminate-large-gap}}\\
    	  &\leq \frac{128 \ln (2 n (P + 1) / \delta)}{\Delta_{[2]}^2 \log e} + \sum_{i\neq \star} \frac{72 \ln (2 n (P + 1) / \delta) n^{2/P}}{4 \Delta^2_i \log e} + T_{B} \tag{by \Cref{eq:basic-small} and \Cref{eq:basic-optimal}}\\
    	  &\leq \frac{128 \ln (2 n (P + 1) / \delta)}{\Delta_{[2]}^2 \log e} + \sum_{i\neq \star} \frac{72 \ln (2 n (P + 1) / \delta) n^{2/P}}{4 \Delta^2_i \log e} + \frac{128 \ln (2 n (P + 1) / \delta)}{n\Delta_{[2]}^2} \tag{by \Cref{eq:basic-large}}\\
    	  & = O\left(\log \left(\frac{nP}{\delta} \right) \sum_{i =2}^n \frac{n^{2/P}}{\Delta^2_{[i]}}\right),
    \end{align*}
    as desired.
%     Combining inequalities \Cref{eq:basic-small}, \Cref{eq:basic-optimal}, and \Cref{eq:basic-optimal} we get the desire result. \chen{Not clear to me!}
\end{proof}

\paragraph{Finalizing the proof of \Cref{thm:basic}.} 
% cite all lemmas as blackboxes and use them to prove  \Cref{thm:basic}.
\Cref{lem:bound-E,lem:best,lem:eliminate-large-gap} guarantee that the \Cref{alg:main} outputs the best arm with probability at least \(1 - \delta\). The combination of \Cref{lem:bound-E} and \Cref{lem:bound-pull} gives the bound on the sample complexity. Finally, \Cref{lem:main-alg-memory} bounds the memory usage. 
\begin{observation}
\label{obs:delta-lower-bound}
We observe that \Cref{alg:main} can be extended to situations where the exact value of the optimality gap \(\Delta_{[2]}\) is unknown. Instead, we only have access to a lower bound \(\gamma\) for \(\Delta_{[2]}\) (we should use \(\epsilon_p = \gamma n^{1-\frac{p}{P}}\)). Under these circumstances, the analysis requires slight modifications. Although the correctness analysis remains unchanged, the bound of sample complexity needs modification. Specifically, when categorizing arms into sets \(B\) and \(S\), we should use the value \(\gamma n\) instead of \(\Delta_{[2]}n\). The contribution of arms from set \(S\) remains unchanged. However, the bound on the contributions of arms from set \(B\) to the total number of pulls is now bounded by \(O\left(\log\left(\frac{nP}{\delta}\right)\frac{1}{n\gamma^2}\right)\). Consequently, if we only have access to the lower bound \(\gamma\), the guarantees of \Cref{thm:basic} remain intact, except for a minor additive overhead of \(O\left(\log\left(\frac{nP}{\delta}\right)\frac{1}{n\gamma^2}\right)\) in the sample complexity.
\end{observation}
% \chen{Add a remark here for the lower bound estimation of $\Delta_{[2]}$}

\begin{remark}
\label{rmk:logn-overhead}
Note that the sample complexity bound for \Cref{alg:main} with $P=O(\log(n))$ does \emph{not} have the dependency on the $\log\log(\frac{1}{\Delta_{[i]}})$ factor as in \cite{KarninKS13,JamiesonMNB14}. Instead, we have an extra $O(\log(n))$ multiplicative factor. We remark that the bound is \emph{not} trivial to obtain. For the worst-case optimal bounds, there exists a simple single-pass streaming algorithm that finds the best arm with $O(n\log(n)/\Delta^2_{[2]})$ sample complexity and a memory of a single arm. The algorithm is simply to keep the best arm on the fly with $O(\log(n)/\Delta^2_{[2]})$ samples on each arm. However, for the instance-sensitive sample complexity, it is unclear how to simulate this simple algorithm with only the knowledge of $\Delta_{[2]}$. Furthermore, since the parameters $\Delta_{[i]}$ can be arbitrarily small for any $i$, our multiplicative factor of $O(\log(n))$ can be better for infinitely many instances.
\end{remark}

\begin{remark}
    We observe that our algorithm is very similar to the elimination algorithms of \cite{JinH0X21,KarninKS13}, albeit our ``search'' of the gap parameter starts from $O(\sqrt{n}\Delta_{[2]})$. We rearmark that if we change the algorithm of \cite{JinH0X21,KarninKS13} by choosing the starting value for the gap as \(O(\sqrt{n}\Delta_{[2]})\), it appears that we can get a similar result with the number of passes \(P=\log(n)\) and sample complexity $\tilde{O}(\sum_{i=2}^{n}\frac{1}{\Delta^2_{[2]}})$. However, if we do not make the extra changes as in our algorithm, it is unclear how to obtain the trade-off between the sample complexity and the number of passes.
\end{remark}

%%
\FloatBarrier
\section{Experiments}
\label{sec:experiments}
The experiments are designed to address two key research questions.
First, \textbf{RQ1} evaluates whether the average $L_2$-norm of the counterfactual perturbation vectors ($\overline{||\perturb||}$) decreases as the model overfits the data, thereby providing further empirical validation for our hypothesis.
Second, \textbf{RQ2} evaluates the ability of the proposed counterfactual regularized loss, as defined in (\ref{eq:regularized_loss2}), to mitigate overfitting when compared to existing regularization techniques.

% The experiments are designed to address three key research questions. First, \textbf{RQ1} investigates whether the mean perturbation vector norm decreases as the model overfits the data, aiming to further validate our intuition. Second, \textbf{RQ2} explores whether the mean perturbation vector norm can be effectively leveraged as a regularization term during training, offering insights into its potential role in mitigating overfitting. Finally, \textbf{RQ3} examines whether our counterfactual regularizer enables the model to achieve superior performance compared to existing regularization methods, thus highlighting its practical advantage.

\subsection{Experimental Setup}
\textbf{\textit{Datasets, Models, and Tasks.}}
The experiments are conducted on three datasets: \textit{Water Potability}~\cite{kadiwal2020waterpotability}, \textit{Phomene}~\cite{phomene}, and \textit{CIFAR-10}~\cite{krizhevsky2009learning}. For \textit{Water Potability} and \textit{Phomene}, we randomly select $80\%$ of the samples for the training set, and the remaining $20\%$ for the test set, \textit{CIFAR-10} comes already split. Furthermore, we consider the following models: Logistic Regression, Multi-Layer Perceptron (MLP) with 100 and 30 neurons on each hidden layer, and PreactResNet-18~\cite{he2016cvecvv} as a Convolutional Neural Network (CNN) architecture.
We focus on binary classification tasks and leave the extension to multiclass scenarios for future work. However, for datasets that are inherently multiclass, we transform the problem into a binary classification task by selecting two classes, aligning with our assumption.

\smallskip
\noindent\textbf{\textit{Evaluation Measures.}} To characterize the degree of overfitting, we use the test loss, as it serves as a reliable indicator of the model's generalization capability to unseen data. Additionally, we evaluate the predictive performance of each model using the test accuracy.

\smallskip
\noindent\textbf{\textit{Baselines.}} We compare CF-Reg with the following regularization techniques: L1 (``Lasso''), L2 (``Ridge''), and Dropout.

\smallskip
\noindent\textbf{\textit{Configurations.}}
For each model, we adopt specific configurations as follows.
\begin{itemize}
\item \textit{Logistic Regression:} To induce overfitting in the model, we artificially increase the dimensionality of the data beyond the number of training samples by applying a polynomial feature expansion. This approach ensures that the model has enough capacity to overfit the training data, allowing us to analyze the impact of our counterfactual regularizer. The degree of the polynomial is chosen as the smallest degree that makes the number of features greater than the number of data.
\item \textit{Neural Networks (MLP and CNN):} To take advantage of the closed-form solution for computing the optimal perturbation vector as defined in (\ref{eq:opt-delta}), we use a local linear approximation of the neural network models. Hence, given an instance $\inst_i$, we consider the (optimal) counterfactual not with respect to $\model$ but with respect to:
\begin{equation}
\label{eq:taylor}
    \model^{lin}(\inst) = \model(\inst_i) + \nabla_{\inst}\model(\inst_i)(\inst - \inst_i),
\end{equation}
where $\model^{lin}$ represents the first-order Taylor approximation of $\model$ at $\inst_i$.
Note that this step is unnecessary for Logistic Regression, as it is inherently a linear model.
\end{itemize}

\smallskip
\noindent \textbf{\textit{Implementation Details.}} We run all experiments on a machine equipped with an AMD Ryzen 9 7900 12-Core Processor and an NVIDIA GeForce RTX 4090 GPU. Our implementation is based on the PyTorch Lightning framework. We use stochastic gradient descent as the optimizer with a learning rate of $\eta = 0.001$ and no weight decay. We use a batch size of $128$. The training and test steps are conducted for $6000$ epochs on the \textit{Water Potability} and \textit{Phoneme} datasets, while for the \textit{CIFAR-10} dataset, they are performed for $200$ epochs.
Finally, the contribution $w_i^{\varepsilon}$ of each training point $\inst_i$ is uniformly set as $w_i^{\varepsilon} = 1~\forall i\in \{1,\ldots,m\}$.

The source code implementation for our experiments is available at the following GitHub repository: \url{https://anonymous.4open.science/r/COCE-80B4/README.md} 

\subsection{RQ1: Counterfactual Perturbation vs. Overfitting}
To address \textbf{RQ1}, we analyze the relationship between the test loss and the average $L_2$-norm of the counterfactual perturbation vectors ($\overline{||\perturb||}$) over training epochs.

In particular, Figure~\ref{fig:delta_loss_epochs} depicts the evolution of $\overline{||\perturb||}$ alongside the test loss for an MLP trained \textit{without} regularization on the \textit{Water Potability} dataset. 
\begin{figure}[ht]
    \centering
    \includegraphics[width=0.85\linewidth]{img/delta_loss_epochs.png}
    \caption{The average counterfactual perturbation vector $\overline{||\perturb||}$ (left $y$-axis) and the cross-entropy test loss (right $y$-axis) over training epochs ($x$-axis) for an MLP trained on the \textit{Water Potability} dataset \textit{without} regularization.}
    \label{fig:delta_loss_epochs}
\end{figure}

The plot shows a clear trend as the model starts to overfit the data (evidenced by an increase in test loss). 
Notably, $\overline{||\perturb||}$ begins to decrease, which aligns with the hypothesis that the average distance to the optimal counterfactual example gets smaller as the model's decision boundary becomes increasingly adherent to the training data.

It is worth noting that this trend is heavily influenced by the choice of the counterfactual generator model. In particular, the relationship between $\overline{||\perturb||}$ and the degree of overfitting may become even more pronounced when leveraging more accurate counterfactual generators. However, these models often come at the cost of higher computational complexity, and their exploration is left to future work.

Nonetheless, we expect that $\overline{||\perturb||}$ will eventually stabilize at a plateau, as the average $L_2$-norm of the optimal counterfactual perturbations cannot vanish to zero.

% Additionally, the choice of employing the score-based counterfactual explanation framework to generate counterfactuals was driven to promote computational efficiency.

% Future enhancements to the framework may involve adopting models capable of generating more precise counterfactuals. While such approaches may yield to performance improvements, they are likely to come at the cost of increased computational complexity.


\subsection{RQ2: Counterfactual Regularization Performance}
To answer \textbf{RQ2}, we evaluate the effectiveness of the proposed counterfactual regularization (CF-Reg) by comparing its performance against existing baselines: unregularized training loss (No-Reg), L1 regularization (L1-Reg), L2 regularization (L2-Reg), and Dropout.
Specifically, for each model and dataset combination, Table~\ref{tab:regularization_comparison} presents the mean value and standard deviation of test accuracy achieved by each method across 5 random initialization. 

The table illustrates that our regularization technique consistently delivers better results than existing methods across all evaluated scenarios, except for one case -- i.e., Logistic Regression on the \textit{Phomene} dataset. 
However, this setting exhibits an unusual pattern, as the highest model accuracy is achieved without any regularization. Even in this case, CF-Reg still surpasses other regularization baselines.

From the results above, we derive the following key insights. First, CF-Reg proves to be effective across various model types, ranging from simple linear models (Logistic Regression) to deep architectures like MLPs and CNNs, and across diverse datasets, including both tabular and image data. 
Second, CF-Reg's strong performance on the \textit{Water} dataset with Logistic Regression suggests that its benefits may be more pronounced when applied to simpler models. However, the unexpected outcome on the \textit{Phoneme} dataset calls for further investigation into this phenomenon.


\begin{table*}[h!]
    \centering
    \caption{Mean value and standard deviation of test accuracy across 5 random initializations for different model, dataset, and regularization method. The best results are highlighted in \textbf{bold}.}
    \label{tab:regularization_comparison}
    \begin{tabular}{|c|c|c|c|c|c|c|}
        \hline
        \textbf{Model} & \textbf{Dataset} & \textbf{No-Reg} & \textbf{L1-Reg} & \textbf{L2-Reg} & \textbf{Dropout} & \textbf{CF-Reg (ours)} \\ \hline
        Logistic Regression   & \textit{Water}   & $0.6595 \pm 0.0038$   & $0.6729 \pm 0.0056$   & $0.6756 \pm 0.0046$  & N/A    & $\mathbf{0.6918 \pm 0.0036}$                     \\ \hline
        MLP   & \textit{Water}   & $0.6756 \pm 0.0042$   & $0.6790 \pm 0.0058$   & $0.6790 \pm 0.0023$  & $0.6750 \pm 0.0036$    & $\mathbf{0.6802 \pm 0.0046}$                    \\ \hline
%        MLP   & \textit{Adult}   & $0.8404 \pm 0.0010$   & $\mathbf{0.8495 \pm 0.0007}$   & $0.8489 \pm 0.0014$  & $\mathbf{0.8495 \pm 0.0016}$     & $0.8449 \pm 0.0019$                    \\ \hline
        Logistic Regression   & \textit{Phomene}   & $\mathbf{0.8148 \pm 0.0020}$   & $0.8041 \pm 0.0028$   & $0.7835 \pm 0.0176$  & N/A    & $0.8098 \pm 0.0055$                     \\ \hline
        MLP   & \textit{Phomene}   & $0.8677 \pm 0.0033$   & $0.8374 \pm 0.0080$   & $0.8673 \pm 0.0045$  & $0.8672 \pm 0.0042$     & $\mathbf{0.8718 \pm 0.0040}$                    \\ \hline
        CNN   & \textit{CIFAR-10} & $0.6670 \pm 0.0233$   & $0.6229 \pm 0.0850$   & $0.7348 \pm 0.0365$   & N/A    & $\mathbf{0.7427 \pm 0.0571}$                     \\ \hline
    \end{tabular}
\end{table*}

\begin{table*}[htb!]
    \centering
    \caption{Hyperparameter configurations utilized for the generation of Table \ref{tab:regularization_comparison}. For our regularization the hyperparameters are reported as $\mathbf{\alpha/\beta}$.}
    \label{tab:performance_parameters}
    \begin{tabular}{|c|c|c|c|c|c|c|}
        \hline
        \textbf{Model} & \textbf{Dataset} & \textbf{No-Reg} & \textbf{L1-Reg} & \textbf{L2-Reg} & \textbf{Dropout} & \textbf{CF-Reg (ours)} \\ \hline
        Logistic Regression   & \textit{Water}   & N/A   & $0.0093$   & $0.6927$  & N/A    & $0.3791/1.0355$                     \\ \hline
        MLP   & \textit{Water}   & N/A   & $0.0007$   & $0.0022$  & $0.0002$    & $0.2567/1.9775$                    \\ \hline
        Logistic Regression   &
        \textit{Phomene}   & N/A   & $0.0097$   & $0.7979$  & N/A    & $0.0571/1.8516$                     \\ \hline
        MLP   & \textit{Phomene}   & N/A   & $0.0007$   & $4.24\cdot10^{-5}$  & $0.0015$    & $0.0516/2.2700$                    \\ \hline
       % MLP   & \textit{Adult}   & N/A   & $0.0018$   & $0.0018$  & $0.0601$     & $0.0764/2.2068$                    \\ \hline
        CNN   & \textit{CIFAR-10} & N/A   & $0.0050$   & $0.0864$ & N/A    & $0.3018/
        2.1502$                     \\ \hline
    \end{tabular}
\end{table*}

\begin{table*}[htb!]
    \centering
    \caption{Mean value and standard deviation of training time across 5 different runs. The reported time (in seconds) corresponds to the generation of each entry in Table \ref{tab:regularization_comparison}. Times are }
    \label{tab:times}
    \begin{tabular}{|c|c|c|c|c|c|c|}
        \hline
        \textbf{Model} & \textbf{Dataset} & \textbf{No-Reg} & \textbf{L1-Reg} & \textbf{L2-Reg} & \textbf{Dropout} & \textbf{CF-Reg (ours)} \\ \hline
        Logistic Regression   & \textit{Water}   & $222.98 \pm 1.07$   & $239.94 \pm 2.59$   & $241.60 \pm 1.88$  & N/A    & $251.50 \pm 1.93$                     \\ \hline
        MLP   & \textit{Water}   & $225.71 \pm 3.85$   & $250.13 \pm 4.44$   & $255.78 \pm 2.38$  & $237.83 \pm 3.45$    & $266.48 \pm 3.46$                    \\ \hline
        Logistic Regression   & \textit{Phomene}   & $266.39 \pm 0.82$ & $367.52 \pm 6.85$   & $361.69 \pm 4.04$  & N/A   & $310.48 \pm 0.76$                    \\ \hline
        MLP   &
        \textit{Phomene} & $335.62 \pm 1.77$   & $390.86 \pm 2.11$   & $393.96 \pm 1.95$ & $363.51 \pm 5.07$    & $403.14 \pm 1.92$                     \\ \hline
       % MLP   & \textit{Adult}   & N/A   & $0.0018$   & $0.0018$  & $0.0601$     & $0.0764/2.2068$                    \\ \hline
        CNN   & \textit{CIFAR-10} & $370.09 \pm 0.18$   & $395.71 \pm 0.55$   & $401.38 \pm 0.16$ & N/A    & $1287.8 \pm 0.26$                     \\ \hline
    \end{tabular}
\end{table*}

\subsection{Feasibility of our Method}
A crucial requirement for any regularization technique is that it should impose minimal impact on the overall training process.
In this respect, CF-Reg introduces an overhead that depends on the time required to find the optimal counterfactual example for each training instance. 
As such, the more sophisticated the counterfactual generator model probed during training the higher would be the time required. However, a more advanced counterfactual generator might provide a more effective regularization. We discuss this trade-off in more details in Section~\ref{sec:discussion}.

Table~\ref{tab:times} presents the average training time ($\pm$ standard deviation) for each model and dataset combination listed in Table~\ref{tab:regularization_comparison}.
We can observe that the higher accuracy achieved by CF-Reg using the score-based counterfactual generator comes with only minimal overhead. However, when applied to deep neural networks with many hidden layers, such as \textit{PreactResNet-18}, the forward derivative computation required for the linearization of the network introduces a more noticeable computational cost, explaining the longer training times in the table.

\subsection{Hyperparameter Sensitivity Analysis}
The proposed counterfactual regularization technique relies on two key hyperparameters: $\alpha$ and $\beta$. The former is intrinsic to the loss formulation defined in (\ref{eq:cf-train}), while the latter is closely tied to the choice of the score-based counterfactual explanation method used.

Figure~\ref{fig:test_alpha_beta} illustrates how the test accuracy of an MLP trained on the \textit{Water Potability} dataset changes for different combinations of $\alpha$ and $\beta$.

\begin{figure}[ht]
    \centering
    \includegraphics[width=0.85\linewidth]{img/test_acc_alpha_beta.png}
    \caption{The test accuracy of an MLP trained on the \textit{Water Potability} dataset, evaluated while varying the weight of our counterfactual regularizer ($\alpha$) for different values of $\beta$.}
    \label{fig:test_alpha_beta}
\end{figure}

We observe that, for a fixed $\beta$, increasing the weight of our counterfactual regularizer ($\alpha$) can slightly improve test accuracy until a sudden drop is noticed for $\alpha > 0.1$.
This behavior was expected, as the impact of our penalty, like any regularization term, can be disruptive if not properly controlled.

Moreover, this finding further demonstrates that our regularization method, CF-Reg, is inherently data-driven. Therefore, it requires specific fine-tuning based on the combination of the model and dataset at hand.
\FloatBarrier

\bibliographystyle{plain}
\bibliography{paper}

\appendix

%%
\section{Standard technical tools}
\label{sec:standatd-tech-tools}

\subsection{Concentration Inequalities}

We use the following variant of the well-known Chernoff-Hoeffding bound.

\begin{lemma}[Chernoff-Hoeffding Inequality]\label{lem:chernoff}
	Let \(X_1, \dotsc, X_n \in [0, 1]\) be independent random variables. Let \(X = \sum_{i = 1}^n X_i\). For any \(t \ge 0\), it holds that 
	\begin{align*}
		\Pr[X \ge \bE[X] + t] \le \exp\left(\frac{-2t^2}{n}\right)
	\end{align*}
	and
	\begin{align*}
		\Pr[X \le \bE[X] - t] \le \exp\left(\frac{-2t^2}{n}\right)\,.
	\end{align*}
	% Suppose \(X_1, X_2, \dotsc, X_n\) are independent sub-Gaussian random variables with mean 0 and sub-Gaussian norm \(\sigma\). Let \(S = X_1 + X_2 + \dotsc + X_n\). Then for any \(t > 0\), we have:
	% \( {\Pr}(S > t) \le \exp\left(-\frac{t^2}{2n\sigma^2}\right) \) and 
	% \( {\Pr}(S < -t) \le \exp\left(-\frac{t^2}{2n\sigma^2}\right) \).
\end{lemma}


% \chen{This version seems to only work for zero-mean random variables...}\nicksays{changed it}

\subsection{Statistical Distances and Properties}
\label{sub-app:stat-dist}
We frequently use the well-known total variation distance (TVD) and Kullback–Leibler divergence (KL divergence) in our proof. In this section, we provide their formal definition and properties.


\paragraph{Total variation distance.} 
We start with the definition of the total variation distance (TVD) between two distributions. 
\begin{definition}
	\label{def:tvd}
	Let $X$ and $Y$ be two random variables supported over the same $\Omega$, and let $\mu_X$ and $\mu_Y$ be their probability measures. The total variation distance (TVD) is between $X$ and $Y$ is defined as
	\begin{align*}
		\tvd{X}{Y} = \sup_{\Omega'\subseteq \Omega}\card{\mu_X(\Omega')-\mu_Y(\Omega')}.
	\end{align*}
	In particular, when the random variables are discrete, we have
	\begin{align*}
		\tvd{X}{Y} = \frac{1}{2}\sum_{\omega\in \Omega}\card{\mu_X(\omega)-\mu_Y(\omega)}.
	\end{align*}
\end{definition}

The total variation distance satisfied the symmetric property, i.e., $\tvd{X}{Y}=\tvd{Y}{X}$, and the triangle inequality, i.e., for three random vairables $X,Y,Z$, there is $\tvd{X}{Y}+\tvd{Y}{Z}\geq \tvd{X}{Z}$.

\paragraph{KL divergence.} We now introduce the Kullback–Leibler divergence (KL divergence) as an alternative statistical distance of TVD.

\begin{definition}[KL divergence]
	\label{def:kl-div}
	Let $X$ and $Y$ be two discrete random variables supported over the same $\Omega$, and let their distributions be $\mu_{X}$ and $\mu_{Y}$.The KL divergence between $X$ and $Y$, denoted as $\kl{X}{Y}$, is defined as 
	\begin{align*}
		\kl{X}{Y} = \sum_{\omega\in \Omega} \mu_{X}(\omega)\log\paren{\frac{\mu_{X}(\omega)}{\mu_{Y}(\omega)}}.
	\end{align*}
\end{definition}


Unlike the TVD, KL divergence is \emph{not} symmetric in general and does \emph{not} follow the triangle inequality.
% KL divergence can also be defined on continuous random variables, but we do not pursue that direction in this paper. Note that the KL divergence does \emph{not} satisfy the triangular inequality. However, the following equality, known as the \emph{chain rule}, is very useful in `factorizing' a joint distribution into marginals.
% \begin{proposition}[Chain rule of KL divergence]
	% \label{prop:chain-rule}
	% Let $X=(X_{1}, X_{2})$ and $Y=(Y_{1},Y_{2})$ be two random variables, there is
	% \begin{align*}
		% \kl{X}{Y} = \kl{X_{1}}{Y_{1}} + \kl{X_{2}\mid X_{1}}{Y_{2}\mid Y_{1}}.
		% \end{align*}
	% \end{proposition}


\paragraph{The Properties of the TVD and KL divergence}
% \label{sub-app:info-theoretic-facts}

We shall use the following standard properties of KL-divergence and TVD defined. For the proof of this results, see the excellent textbook by Cover and Thomas~\cite{CoverT06}. 

We first connect the KL-divergence to TVD with the following celebrated Pinsker's inequality. 

\begin{fact}[Pinsker's inequality]
	\label{fact:pinsker}
	For any random variables $X$ and $Y$ supported over the same $\Omega$, 
	\begin{align*}
		\tvd{X}{Y} \leq \sqrt{\frac{1}{2}\cdot \kl{X}{Y}}.
	\end{align*}
\end{fact} 

We then present the key properties we used in our proof for the KL divergence, which includes the Chain rule and the conditional KL-divergence.
\begin{fact}[Chain rule of KL divergence]
	\label{fact:kl-chain-rule}
	For any random variables $X=(X_{1}, X_{2})$ and $Y=(Y_{1},Y_{2})$ be two random variables, 
	\begin{align*}
		\kl{X}{Y} = \kl{X_{1}}{Y_{1}} + \Exp_{x \sim X_1} \kl{X_{2}\mid X_{1}=x}{Y_{2}\mid Y_{1}=x}.
	\end{align*}
\end{fact}

%\begin{fact}[Convexity KL-divergence]
%	\label{fact:kl-convexity}
%	For any distributions $\mu_1,\mu_2$ and $\nu_1,\nu_2$ and any $\lambda \in (0,1)$, 
%	\begin{align*}
%		\kl{\lambda \cdot \mu_1 + (1-\lambda) \cdot \mu_2}{\lambda \cdot \nu_1 + (1-\lambda) \cdot \nu_2} \leq \lambda \cdot \kl{\mu_1}{\nu_1} + (1-\lambda) \cdot \kl{\mu_2}{\nu_2}. 
%	\end{align*}
%\end{fact}

\begin{fact}[Conditioning cannot decrease KL-divergence]\label{fact:kl-conditioning}
	For any random variables $X,Y,Z$, 
	\[
	\kl{X}{Y} \leq \Exp_{z \sim Z} \kl{X \mid Z=z}{Y \mid Z=z}. 
	\]
\end{fact}

In our proofs, we slighlty abuse the notation to let $\kl{X|Z}{Y|Z}$ denoting $\Exp_{z \sim Z} \kl{X \mid Z=z}{Y \mid Z=z}$.

The following fact characterizes the error of MLE for the source of a sample based on the TVD of the originating distributions. 

\begin{fact}
	\label{fact:distinguish-tvd}
	Suppose $\mu$ and $\nu$ are two distributions over the same support $\Omega$; then, given one sample $s$ from the following distribution
	\begin{itemize}
		\item With probability $\rho$, sample $s$ from $\mu$;
		\item With probability $1-\rho$, sample $s$ from $\nu$;
	\end{itemize}
	The best probability we can decide whether $s$ came from $\mu$ or $\nu$ 
	is 
	\[
	\max(\rho, 1-\rho) + \min(\rho, 1-\rho)\cdot\tvd{\mu}{\nu}.
	\]
\end{fact}


We frequently use the calculation of KL divergence between Bernoulli random variables. The following fact is a standard upper bound for KL divergence on Bernoulli random variables:
\begin{fact}
	\label{fct:bernoulli-KL}
	Let two random variables be distributed with $\bern{p}$ and $\bern{q}$, there is
	\begin{align*}
		\kl{\bern{p}}{\bern{q}}\leq \frac{(p-q)^2}{q\cdot (1-q)}.
	\end{align*}
\end{fact}


\Cref{fct:bernoulli-KL} implies the following the upper bound for KL-divergence, which we frequently use in our proof.

\begin{claim}
\label{clm:bernoulli-KL}
Let two random variables be distributed with $\bern{1/2+\alpha}$ and $\bern{1/2+\beta}$ such that $\max\{\alpha,\beta\}\leq \frac{1}{6}$, there are
\begin{align*}
	& \kl{\bern{1/2+\alpha}}{\bern{1/2+\beta}} \leq 8\cdot (\beta-\alpha)^2\\
	& \kl{\bern{1/2+\beta}}{\bern{1/2+\alpha}} \leq 8\cdot (\beta-\alpha)^2.
\end{align*}
\end{claim}
\begin{proof}
We prove the first inequality, since the second inequality follows the same calculation. The calculation is as follows.
\begin{align*}
	\kl{\bern{1/2+\alpha}}{\bern{1/2+\beta}} & \leq (\beta-\alpha)^2 \cdot \frac{1}{1/4-\beta^2}\\
	& \leq \frac{36}{8}\cdot (\beta-\alpha)^2 \tag{using $\beta\leq \frac{1}{6}$}\\
	&\leq 8\cdot (\beta-\alpha)^2.
\end{align*}
\end{proof}
% \chen{Include the lemma here.}

\subsection{Information Theory Tools}
\label{subsec:info-theory}
% \chen{revise this section to make it not copy-paste and reflect the tools we used in the section.}
We present the definition and basic properties of the information-theoretic tools in our proofs. For a random variable $X$, we let $\HH(X)$ be the \emph{Shannon entropy} of $X$, defined as follows
\begin{definition}[Shannon entropy]
	\label{def:entropy}
	Let $X$ be a discrete random variable with distributions $\mu_{X}$, the \emph{Shannon entropy} of $X$ is defined as
	\begin{align*}
		\HH(X) \triangleq \expect{\log(1/\mu(X))} =\sum_{x \in \text{supp}(X)} \mu(x)\cdot \log(\frac{1}{\mu(x)}),
	\end{align*}
	where $\text{supp}(X)$ is the support of $X$. If $X$ is a Bernoulli random variable, we use $H_2(p)$ to denote its Shannon entropy, where $P$ is the probability for $X=1$.
\end{definition}

We now give the definition of conditional entropy and mutual information.
\begin{definition}
	\label{def:mutual-info}
	Let $X$, $Y$ be two random variables, we define the \emph{conditional entropy} as \[\HH(X|Y)=\mathbb{E}_{y \sim Y}[\HH(X\mid Y=y)].\] 
	With conditional entropy, we can define the \emph{mutual information} between $X$ and $Y$ as \[\II\paren{X;Y}\triangleq \HH(X)-\HH(X\mid Y) = \HH(Y)-\HH(Y|X).\]
\end{definition}



In our proof, we use the following information-theoretic fact (see e.g.~\cite{CoverT06}) that connects mutual information with the KL-divergence.
\begin{fact}
	\label{fct:mutual-info-and-kl}
	Let $X$, $Y$ be  discrete random variables, there is
	\[\II(X;Y) = \Exp_{y\sim Y}\bracket{\kl{X\mid Y=y}{X}}.\]
\end{fact}

%\begin{fact}
%	\label{fct:info-theory-facts}
%	Let $X$, $Y$, $Z$ be three discrete random variables:
%	\begin{itemize}
%		\item KL-divergence view of mutual information: $\II(X;Y) = \Exp_{y\sim Y}\bracket{\kl{X\mid Y=y}{X}}$.
%		\item $0\leq \HH(X)\leq \log(\card{\text{supp}(X)})$. In particular, if $X$ is a Bernoulli random variable, there is $H_2(p)\leq 1$.
%		\item $0\leq \II(X;Y)\leq \min\{\HH(X), \HH(Y)\}$.
%		\item Conditioning on independent random variable: let $X$ be independent of $Z$, then $\II(X;Y)\leq \II(X;Y\mid Z)$.
%		\item Chain rule of mutual information: $\II(X, Y; Z)=\II(X;Z)+\II(Y;Z\mid X)$.
%		\item Sub-additivity of entropy: $\HH(X,Y) \leq \HH(X) + \HH(Y)$, where $\HH(X, Y)$ is the joint entropy of variables $X, Y$.
%		\item Conditional independence of entropy: $\HH(X\mid Y,Z)=\HH(X\mid Y)$ if $X\perp Z\mid Y$, where the $\perp$ notation stands for independence.
%	\end{itemize}
%\end{fact}

%The following statement is known as the \emph{data processing inequality}, which says if $Y$ is obtained as a function of $X$, and $Z$ is obtained as a function of $Y$, then the mutual information between $X$ and $Z$ can only be lower than that between $X$ and $Y$.
%\begin{proposition}
%	\label{prop:DPI}
%	Let $X$, $Y$, and $Z$ be random variables on finite supports, and we slightly abuse the notation to let $X,Y,Z$ to denote the distribution functions as well. Let $f$ be a deterministic function (no internal randomness), and suppose $Z=f(Y)$. Then, we have
%	\begin{align*}
%		\II(X;Z)\leq \II(X;Y).
%	\end{align*}
%\end{proposition}
%
%The following statement characterizes the relationship between the ``zero mutual information'' and the independence of the conditional probability.
%
%\begin{proposition}
%	\label{prop:mi-prob-indep}
%	Let $X$, $Y$, and $Z$ be random variables on finite supports, and suppose $\II(X;Y\mid Z=z)=0$. Then, for any realization $y\in Y$, there is
%	\begin{align*}
%		\Pr(X\mid Z=z, Y=y) = \Pr(X\mid Z=z).
%	\end{align*}
%\end{proposition}





%%
\section{A Streaming Algorithm with Memory Efficiency in Both Arms and Statistics}\label{sec:ub-stat-efficient}
In this section we describe algorithm that achieves memory efficiency for both number of arms and statistics. In particular, we show a $P$-pass algorithm with a single-arm memory and maintaining \(O(P)\) bits of statistics. The sample complexity becomes \(O\left(P \log \left(\frac{nP}{\delta}\right) \cdot \sum_{i = 2}^{n} \frac{n^{2/P}}{\Delta^2_{[i]}}\right)\) in this algorithm, which is still in the range of $\tilde{O}\paren{\sum_{i = 2}^{n} \frac{n^{2/P}}{\Delta^2_{[i]}}}$ by picking $P=O(\log(n))$. The formal statement is as follows.

\begin{theorem}
	\label{thm:improved}
	For any \(P \geq 1\), \Cref{alg:improved} is an algorithm that given a streaming MABs instance and the value of $\Delta_{[2]}$, finds the best arm with probability at least \(1-\delta\) using most
	\[O\left(P \log \left(\frac{nP}{\delta}\right) \cdot \sum_{i = 2}^{n} \frac{n^{2/P}}{\Delta^2_{[i]}}\right)\] 
	arm pulls, a memory of a single arm, and at most $O(P)$ bits of statistics.
\end{theorem}

We refer the readers to \Cref{subsec:model} for the formal definition of the memory complexity on the number of arms and the statistics. Compared to the algorithm in \Cref{sec:basic}, the idea to reduce the memory usage it to not store sets \(I_p\) for each \(p\). Instead, we simply store the maximum estimated mean for each pass, and simulate the results of the previous passes by resampling to get the new empirical mean. As a result, we pay a $P$ factor overhead for the arm pulls on each arm, but do not need to explicitly maintain the indices of the arms or query the transcript $\pi$. We present the description of the algorithm in \Cref{alg:improved}. 


\FloatBarrier
\begin{algorithm}
	\caption{Stream-Elimination-Re}\label{alg:improved}
	\KwIn{Stream \(I\), parameter $P$, gap parameter \(\Delta_{[2]}\), and confidence parameter \(\delta\)}
	\KwOut{Best arm}
	Set \(n \gets \abs{I}\)\; 
	Maintain the index of the returned arm $\itilde \gets \perp$\;
	Let \(\epsilon_p \triangleq n^{1-i/P}\Delta_{[2]} / 4\) for \(p = 0, \dotsc, P\) \;
	Let \(T_p \triangleq \frac{8}{\epsilon^2_p \log e} \log\left(\frac{2 n (P + 1)^2}{\delta}\right) \) for \(p = 0, \dotsc, P\)\; 
	\For{\(p = 0, \dotsc, P\)}{
		Initialize \(\hat\mu^p_{\max} \gets -\infty\)\;
		Initialize the number of eliminated arms \(c \gets 0\)\;
		\ForEach{\(i \in I\) in the arrival order}{
			\For{\(j = 0, \dotsc, p\)}{
				Pull arm \(i\) until the number of pulls reach \(T_j\) times and compute the estimated mean \(\hat\mu^{pj}_i\)\;
				\label{alg:improved-eliminate}\If{\(\hat\mu^{pj}_i < \hat{\mu}^j_{\max} - \epsilon_j\)}{
					Update \(c \gets c + 1\) \;
					Eliminate arm \(i\) and exit loop \; 
				}
				\If{arm \(i\) is not  eliminated}{
					\If{\(\hat\mu^p_{\max} < \hat\mu^{pj}_i\)}{
						Update \(\hat\mu^p_{\max} \gets \hat\mu^{pj}_i\) and \(\itilde \gets i\)\;
					}
				}
			}
		}
		\If{\(c = n - 1\)}{
			\Return{The arm of index \(\itilde\)}\;
		}
	}
\end{algorithm}

\FloatBarrier


% \chen{Add a proof for the number of arms and statistics; also, discuss we always only maintain one record in $\pitilde$.}
We start with observing the memory efficiency for \Cref{alg:improved} on both stored arms and bits of statistics. Formally, we show that
\begin{lemma}
\label{lem:improved-memory}
The maximum size of the memory of \Cref{alg:improved} is at most a single arm, and the maximum size of statistics \Cref{alg:improved} matains is $O(P)$.
\end{lemma}
\begin{proof}
	For the memory complexity of the number of arms, note that we can one-the-fly keep the arm with index $\itilde$, and we never need to store more than the $\arm_{\itilde}$. 
	
	For the size of the statistics, note that we do \emph{not} keep the empirical means for individual arms in the memory, and we only maintain the mean estimation of $\hat\mu^p_{\max}$ and at most $P$ different values of $\hat\mu^{p{j}}$. Therefore, the auxiliary number of bits to maintain is at most $O(P)$. Finally, for the size of $\pitilde$ (defined in \Cref{subsec:model}), we can always update the estimation of mean on-the-fly, and we never query a record of arm pull for more than once. Therefore, the size of $\pitilde$ is at most $1$. The desired memory bound is obtained by summarizing the above cases.
\end{proof}


For the correctness and the sample complexity, the proof strategy is similar to the proof for \Cref{alg:main}. 
We start with the definition of the event when all estimated means are close to the real values of means. Let's define an event \(\F\) as follows:

\begin{align}\label{eq:event-F}
	\F \triangleq \left\{\forall{i \in I}, p \in \{0, \dotsc, P\}, j \in \{0, \dotsc, p\} : \abs{\hat\mu^{pj}_i - \mu_i} \le \epsilon_r / 4\right\}
\end{align}
Where, \(\hat\mu^{pj}_i\) is the empirical mean of the samples drawn from distribution \(i\) in \(p\)-th pass after \(T_j\) pulls. 

We can now state the following lemma: 

\begin{lemma}\label{lem:bound-F}
	Let \(\F\) be the event defined in~\Cref{eq:event-F}. Then, \[\Pr\left[ \neg \F\right] \le \delta\,. \]
\end{lemma}
\begin{proof}
	By the union bound, we have:
	\begin{align}\label{eq:event-F-sum}
		\Pr\left[\neg \F\right] \le \sum_{i \in I} \sum_{p = 0}^{P} \sum_{j = 0}^{p} \Pr\left[\abs{\hat{\mu}^{pj}_i - \mu_i} > \epsilon_r/4\right]\,.
	\end{align}
	By using the Chernoff-Hoeffding inequality (\Cref{lem:chernoff}) we have: 
	\begin{align}\label{eq:event-F-term}
		\Pr\left[\abs{\hat\mu^{pj}_i-\mu_i} > \frac{\epsilon_r}{4}\right] \leq 2\exp\left(\frac{\epsilon^2_r T_r}{8}\right) \leq \frac{\delta}{n(P + 1)^2}\,.
	\end{align}
	
	Combining \Cref{eq:event-F-sum} and \Cref{eq:event-F-term}, we get: 
	\begin{align*}
		\Pr\left[\neg \F\right] \le n {(P + 1)}^2 \frac{\delta}{n(P + 1)^2} = \delta
	\end{align*}
\end{proof}


\begin{lemma}\label{lem:improved-best}
	 Conditioning on the event \(\F\) defined in \Cref{eq:event-F} holds, then for any \(p \in [P]\) we have that the arm \(\star\) is not eliminated in the \(p\)-th pass. 
\end{lemma}

\begin{proof}
	We need to prove that the if condition from \Cref{alg:improved-eliminate} in \Cref{alg:improved} does not hold for \(i = \star\). To prove this, we start by assuming the opposite. Assume that in \(p\)-th iteration for some \(j \in \{0, \dotsc, p\}\) we have 
	\begin{align*}
		\hat\mu^{pj}_{\star} < \hat\mu^j_{\max} - \epsilon_j
	\end{align*}
	and consequently
	\begin{align}\label{eq:improved-upper}
		\hat\mu^{pj}_{\star} < \hat\mu^{jj}_i - \epsilon_j
	\end{align}
	for some \(i \in I\). 
	
	However, from the definition of the event \(\F\) (\Cref{eq:event-F}) we have \(\hat\mu^{pj}_\star \geq \mu_\star - \epsilon_j / 4\) and \(\hat\mu^{jj}_i \leq \mu_i + \epsilon_r / 4\). Consequently, we get: 
	\begin{align}\label{eq:improved-lower}
		\hat\mu^{pj} - \hat\mu^{jj}_i \ge \mu_\star - \mu_i - \epsilon_j / 2 \ge -\epsilon_j / 2 \,.
	\end{align}
	
	\Cref{eq:improved-upper} and \Cref{eq:improved-lower} contradict to each other, leading to a contradiction. Therefore, it is not possible for the arm \(\star\) be eliminated at any moment of work of \Cref{alg:improved}. 
\end{proof}

We next prove that any suboptimal arm with a large gap will be eliminated in each pass after before specific number of pulls. 

\begin{lemma}\label{lem:improved-suboptimal-bound}    
	Conditioning on the event \(\F\) defined in \Cref{eq:event-F} holds, and assume for arm \(i\) and value \(j\), arm \(i\) satisfies \(\Delta_i > \frac{3}{2}\epsilon_j\), then for any pass \(p \in \{j, \dotsc, P\}\), arm \(i\) will be eliminated after at most \(T_j\) pulls.
\end{lemma}
\begin{proof}
	Consider any suboptimal arm \(i\) and a value \(j\) such that \(\Delta_i > \frac{3}{2} \epsilon_j\). We aim to prove that the arm \(i\) should be eliminated after we make at most \(T_j\) pulls for any pass \(p \geq j\). Consider \(p\)-th pass, if the arm \(i\) is eliminated before we make \(T_j\) pulls, then we are done. 
	Consider the opposite case when we make at least \(T_j\) pulls. By using \Cref{lem:improved-best} and the definition of the event \(\F\), we have the following inequality: 
	\begin{align}\label{eq:improved-best-lower}
		\hat\mu^{p}_{\max} \ge \hat\mu^{pj}_{\star} \ge \mu_{\star} - \epsilon_j / 4\,.
	\end{align}
	This inequality indicates that the maximum estimated mean \(\hat\mu^{j}_{\max}\) in \(p\)-th pass is at least as large as the estimated mean \(\hat\mu^{pj}_{\star}\) of the best arm \(\star\), which is at least \(\mu_\star - \epsilon_j / 4\) by the event \(\F\). 
	
	Furthermore, by the event \(\F\), we have:
	\begin{align}\label{eq:improved-sub-upper}
		\hat\mu^{pj}_{i} \le \mu_i + \epsilon_j / 4\,.
	\end{align}
	
	Combining \Cref{eq:improved-best-lower} and \Cref{eq:improved-sub-upper}, we obtain: 
	\begin{align*}
		\hat\mu^{pj}_i - \hat\mu^j_{\max} + \epsilon_j \leq \mu_j - \mu_\star + \epsilon_j / 2 + \epsilon_j = \frac{3}{2}\epsilon_j - \Delta_i < 0\,.
	\end{align*}
	
	The above inequality shows that \(\hat\mu^{pj}_i < \hat\mu^{j}_{\max} - \epsilon_j\) holds due the large value of the gap \(\Delta_i > \frac{3}{2} \epsilon_j\). 
\end{proof}

As \(\epsilon_P \le \frac{\Delta_{[2]}}{4}\) and for any suboptimal arm \(i\) we have \(\Delta_i \ge \Delta_{[2]} \ge 4\epsilon_P > \frac{3}{2}\epsilon_P\), it follows that any suboptimal arm \(i\) will be eliminated in \(P\)-th pass (\Cref{lem:improved-suboptimal-bound}) and optimal arm \(\star\) will not be eliminated by \Cref{lem:improved-best}. Thus, \Cref{alg:improved} outputs the correct arm if the event \(\F\) holds.


\begin{lemma}\label{lem:improved-sample-complexity}
	Conditioning on the event \(\F\) defined in \Cref{eq:event-F} holds, the number of arm pulls used by \Cref{alg:improved} is at most 
	\begin{align*}
		O\left(P \log\left( \frac{1}{\delta}\right) \cdot \sum_{i = 2}^n \frac{n^{2/P}}{\Delta^2_{[i]}}\right).
	\end{align*}
\end{lemma}

\begin{proof}
	% \chen{read this lemma and see if the summation actually converges}
	We split the set of arms \(I\) into two parts, \(B\) be the set arms with big gaps 
	\begin{align*}%\label{eq:improved-big}
		B \triangleq \left\{i \in I \mid \Delta_i > \frac{3n\Delta_{[2]}}{2}\right\}
	\end{align*}
	and small gaps
	\begin{align*}%\label{eq:improved-small}
		S \triangleq \left\{ i \in I \mid \Delta_i \le \frac{3n\Delta_{[2]}}{2}\right\}\,.
	\end{align*}
	
	Similar to the proof of \Cref{lem:bound-pull}, we define $T_{B}$ and $T_{S}$ to be the number of arm pulls used by the arms in $B$ and $S$, and $T\triangleq T_B + T_S$ is the total number of arm pulls. For any sub arm \(i\), we define the value \(p(i) \triangleq \min\{p \ge 0 \mid \Delta_i > \frac{3}{2}\epsilon_p\}\). For the optimal arm \(\star\), we have \(p(\star) = P\) by \Cref{lem:improved-suboptimal-bound}. We note that it is correctly defined because 
	\[\frac{3}{2}\epsilon_P = \frac{3}{2}\frac{\Delta_{[2]}}{4} < \Delta_{[2]}\,,\]
	and so on \(\forall{i \in I} : p(i) \le P\). 
	
	For arms from \(S\) due the definitions of \(\epsilon_r\) and \(p(i)\), we have that 
	\begin{align*}
		\frac{3}{2} n^{1/P} \epsilon_{p(i)} \ge \Delta_i > \frac{3}{2} \epsilon_{p(i)},
	\end{align*}
	and consequently, 
	\begin{align}\label{eq:improved-eps-lower}
		\epsilon_{p(i)} \ge \frac{2\Delta_i}{3 n^{1/P}}\,.
	\end{align}
	
	By \Cref{lem:improved-suboptimal-bound}, we have that the number of pulls for any suboptimal arm \(i \in S\) is bounded by the number of passes \(P + 1\) times \(T_{p(i)}\).  Consequently, by \Cref{eq:improved-eps-lower}, the number of pulls for arm \(i\) from \(S \setminus \{\star\}\)  the number of pulls is bounded
	\begin{align}\label{eq:improved-bound-small-gap}
		P T_{p(i)} \le P \frac{8}{\epsilon^2_{p(i)} \log{e}} \log\left(\frac{2 n (P + 1)^2}{\delta}\right) / \log(e) \le \frac{18 P }{\Delta^2_i \log{e} } \log\left(\frac{2 n (P + 1)^2}{\delta}\right)\,.
	\end{align}
	
	For the optimal arm the number of pulls is trivially bounded by \begin{align}\label{eq:improved-bound-optimal}
		P T_{P} \le P \frac{128}{\Delta_{[2]}^2 \log e} \log\left(\frac{2n{(P + 1)}^2}{\delta}\right)\,.
	\end{align}
	
	For arms from the set \(B\) by \Cref{lem:improved-suboptimal-bound} we have that the number of pulls assigned to an arm \(i \in B\) is bounded by 
	\begin{align*}
		P T_0 \le \frac{128}{n^2\Delta_{[2]}^2 \log e} \log \left(\frac{2n{(P + 1)}^2}{\delta}\right)\,.
	\end{align*}
	We note that the size of \(B\) is bounded by \(n\). Therefore, the total sample complexity for arms from \(B\) is bounded by 
	\begin{align}\label{eq:improved-bound-large-gap}
		n \cdot \frac{128}{n^2\Delta_{[2]}^2 \log e} \log \left(\frac{2n{(P + 1)}^2}{\delta}\right) \le \frac{128}{\Delta_{[2]}^2 \log e} \log \left(\frac{2n{(P + 1)}^2}{\delta}\right)\,.
	\end{align}
	
	As such, combining~\Cref{eq:improved-bound-small-gap}, \Cref{eq:improved-bound-optimal}, and \Cref{eq:improved-bound-large-gap}, we have
	\begin{align*}
		T &= T_{S} + T_{B}\\
		&\leq P \frac{128}{\Delta_{[2]}^2 \log e} \log\left(\frac{2n{(P + 1)}^2}{\delta}\right) + \sum_{i: i\neq \star} P \frac{18}{\Delta_{i}^2 \log e} \log\left(\frac{2n{(P + 1)}^2}{\delta}\right) + T_{B}\tag{by \Cref{eq:improved-bound-small-gap}, \Cref{eq:improved-bound-optimal}}\\
		&\leq O\left(P \log\left( \frac{nP}{\delta}\right) \cdot \sum_{i = 2}^n \frac{n^{2/P}}{\Delta^2_{[i]}}\right) + \frac{128}{\Delta_{[2]}^2 \log e} \log \left(\frac{2n{(P + 1)}^2}{\delta}\right)  \tag{by \Cref{eq:improved-bound-large-gap}}\\
		&= O\left(P \log\left( \frac{nP}{\delta}\right) \cdot \sum_{i = 2}^n \frac{n^{2/P}}{\Delta^2_{[i]}}\right),
	\end{align*} 
	as desired.
\end{proof}

\paragraph{Finalizing the proof of \Cref{thm:improved}.} The algorithm makes $P+1$ passes over the stream by the algorithm design. The memory efficiency is guaranteed by \Cref{lem:improved-memory}, and the correctness and the sample complexity are gueranteed by \Cref{lem:bound-F,lem:improved-best,lem:improved-suboptimal-bound,lem:improved-sample-complexity}. Combining the above completes the proof of \Cref{thm:improved}.

% \chen{Add a remark to discuss that the lower bound still holds for a $\log^2 n$ multiplicative factor.}

\begin{remark}
	\label{rmk:sample-lb-higher-polylog}
	By setting $P=O(\log(n))$, we get a sample complexity bound of $O(\log^2 (n)\cdot \sum_{i=2}^{n}\frac{1}{\Delta^{2}_{[i]}})$. We remark that our lower bound in \Cref{thm:lb-main} can also be extended to the sample complexity of $O(\log^2 (n)\cdot \sum_{i=2}^{n}\frac{1}{\Delta^{2}_{[i]}})$. In fact, by using $B=\Theta(\log(n)/\log\log(n))$, we can already extend the lower bound to $O((\frac{\log(n)}{\log\log(n)})^2\cdot \sum_{i=2}^{n}\frac{1}{\Delta^{2}_{[i]}})$ sample complexity. We can actually strengthen the bound on \Cref{prop:multi-pass-lb} to allow larger exponent on $B$ by using higher gaps between $\chi_{b}$ and $\chi_{b+1}$ (e.g., $\chi_{b+1}= ({1}/{6 C \log(n)})^{100} \cdot \chi_{b}$). To make the lower bound work, we will need to use smaller constant for $P\leq B$, e.g., at most ${1}/{10000}\cdot \log(n)/\log\log(n)$ passes, which is still in the range of $\Omega(\log(n)/\log\log(n))$.
\end{remark}


%%
% \section{Proof Sketch of \Cref{prop:multi-pass-lb}}
% \label{sec:multi-pass-lb-tool}
% We present a high-level proof sketch of \Cref{prop:multi-pass-lb}, and discuss key differences between \cite{AW23BestArm} that allow us to generalize the statement to fit our construction. 

% \paragraph{An overview of the proof in \cite{AW23BestArm}.} We start with an overview of the \emph{proof} of the main statement of \cite{AW23BestArm}. As we have discussed, the construction of the adversarial instances in \cite{AW23BestArm} follows the same family of batched instance that $a).$ divide the arms into $B+1$ batches and $b).$ make the gap between the sample complexity to `trap' or `learn' arms from batches large enough. The exact construction of \cite{AW23BestArm} is as follows.


% \begin{tbox}
% 	$\cP(B, C)$: A hard instance distribution for multi-pass MABs algorithms (without known $\Delta_{[2]}$). 
	
% 	\begin{enumerate}
% 		\item \textbf{Parameters}: Let $\chi_{1}=(\frac{1}{6C\cdot B})^{10}$, and set 
% 		\[\chi_{b+1} = \paren{\frac{1}{6 C \cdot B}}^{10} \cdot \chi_{b}.\]
% 		\item \textbf{Division of arms:} Divide the $n$ arms into $(B+1)$ batches of equal sizes, and put them in the \emph{reverse} order of the stream, i.e. $\mathcal{B}_{B+1}$ arrives first, and $\mathcal{B}_{1}$ arrives the last.
% 		\item \textbf{Sampling special arms: } For each batch $b\in [B+1]$, sample \emph{a single} arm uniformly at random (without replacement), and call it \emph{special arms}. Set all the arms \emph{except} the special arm with reward distribution $\bern{1/2}$.
% 		\item \textbf{Batches $b\in[B]$: } For each $b\in [B]$, sample $\Theta_{b}$ from distribution $\bern{1/2B}$:
% 		\begin{enumerate}
% 			\item If $\Theta_{b}=0$, set the special with reward distributions $\bern{1/2}$.
% 			\item Otherwise, i.e., $\Theta_{b}=1$, set the special arm with reward distribution $\bern{1/2+\chi_{b}}$.
% 		\end{enumerate}
% 		\item \textbf{The batch $B+1$: } Always set the reward distributions of the special arm with $\bern{1/2+\chi_{B+1}}$.
% 	\end{enumerate}
% \end{tbox}

% The construction is similar to our hard distribution in \Cref{sec:lb-main}, albeit their construction uses only one special arm each batch, and they do \emph{not} need to control the relationship between $\chi_{b}$ and $\gamma$ (which is the technical reason for their result to go beyond $O(\log(n))$ passes). 

% The key idea of \cite{AW23BestArm} is to directly work with the \emph{conditional distribution} of the instance family. Crucially, their argument explicitly tracks the `knowledge' the algorithm ever learned, and the information on the batches that they `do not care' is further revealed to the algorithm\footnote{This is a very high-level summary of the technical work in \cite{AW23BestArm} -- keen readers can find a more detailed discussion for the technical idea in their paper.}. To formalize this strategy, they defined the notion of \emph{memory-oblivious} and \emph{batch-oblivious} algorithm as follows.
% \begin{itemize}
% 	\item An algorithm is said to be \emph{memory-oblivious} by the end of pass $p$ if the memory $M$ does not contain any arm with mean reward strictly more than $\frac{1}{2}$.
% 	\item An algorithm is said to be \emph{batch-oblivious} by the end of pass $p$ if for every $b\in(p, B]$, there is
% 	\[\Pr\paren{\Theta_{b}=1\mid \Pi=\pi, \mathsf{M}=M, \Theta_{\leq p}=0} \in \left[\frac{1}{2B}-\frac{p}{4B^2}, \frac{1}{2B}+\frac{p}{4B^2}\right].\]
% \end{itemize}

% As the name suggested, the memory-oblivious property indicates that the algorithm does \emph{not} store any special arm, and the batch oblivious property means the internal distribution of the batches (arriving before $p$) is \emph{not} too far away from the original distribution. Their main technical analysis then assume the algorithm is memory- and batch-oblivious by the end of the $p$-th pass, and proceeds with two different cases based on the expected samples used on batches from $p+2$ to $B+1$, denoted as $\smp_{B+1:p+2}$.
% \begin{enumerate}[label=\alph*).]
% 	\item\label{line:ana-conservative} Small number of samples -- the `conservative case'. In this case, they assume the expected samples used on the batches that arrive before $p+1$ is small, i.e.
% 	\[\expect{\smp_{B+1:p+2}\mid \Pi=\pi, \mathsf{M}=M, \Theta_{\leq p}=0}\leq \frac{1}{5000}\cdot \frac{n}{\chi^2_{p+2}}\cdot \frac{1}{\poly(B)}.\]
% 	For this case, their main goal is to prove that with probability $(1-\frac{1}{2B})^c$ for some constant $c$, the resulting memory $M'$ and transcript $\pi'$ by the end of pass $p+1$ remains memory- and batch-oblivious. Technically, this is where conditions \textbf{C1} and \textbf{C2} in \Cref{prop:multi-pass-lb} are used. Concretely, we use \textbf{C1} to ensure that the streaming algorithm remains memory oblivious by the end of the $(p+1)$-th pass (with probability $(1-1/2B)^{O(1)}$) -- if the number of samples is small, the algorithm cannot trap any special arm. Then, by \textbf{C2} and the condition of memory obliviousness, we can argue that the algorithm cannot `learn' too much for matches $p+2$ and onwards, since the number of samples is again too small. 
% 	\item\label{line:ana-radical} Large number of samples -- the `radical case'. In this case, they assume the expected samples used on the batches that arrive before $p+1$ is large, i.e.
% 	\[\expect{\smp_{B+1:p+2}\mid \Pi=\pi, \mathsf{M}=M, \Theta_{\leq p}=0}> \frac{1}{5000}\cdot \frac{n}{\chi^2_{p+2}}\cdot \frac{1}{\poly(B)}.\]
% 	For this case, their main goal is to prove that the algorithm cannot satisfy the expected sample bound of $C\cdot \frac{n}{\chi^2_{p+1}}$ if $\Theta_{p+1}=1, \Theta_{\leq p}=0$, which would break the sample complexity restriction since $(\Delta_{[2]}\mid\Theta_{p+1}=1, \Theta_{\leq p}=0)=\chi_{p+1}$. To this end, they use \textbf{C3} to argue that $O\left(\frac{n}{\chi^2_{p+2}}\cdot \frac{1}{\poly(B)}\right) \gg C\cdot \frac{n}{\chi^2_{p+1}}$, which holds true by the gap between the $\chi_{b}$ values.
% \end{enumerate}
% Combining \Cref{line:ana-conservative} and \Cref{line:ana-radical} allows \cite{AW23BestArm} to inductively argue that to preserve the desired sample complexity bound:
% \begin{equation}
% \label{equ:equ:batch-sample-ub-simple}
% \expect{\smp\mid \Theta_{p+1}=1, \Theta_{\leq p}=0}\leq C \cdot \frac{n}{\chi_{p+1}^2},
% \end{equation}
% the algorithm has to proceed with the conservative case for every pass. Consequently, by setting $P=B$, the algorithm fails with constant probability if $B+1$ is the only batch that contains an arm with mean reward $>\frac{1}{2}$, which in turn is an event that happens with constant probability. This establishes a lower bound on the failure probability and concludes the proof.


% \paragraph{The additional properties in \Cref{prop:multi-pass-lb}.} We now discuss the additional properties we demonstrated in \Cref{prop:multi-pass-lb}, which are not explicitly stated but are implied in \cite{AW23BestArm}. We first observe that the above proof sketch applies to all batched instance distributions with properties \textbf{C1}, \textbf{C2}, and \textbf{C3}, i.e., the argument is not limited to one specific construction.

% We then discuss two differences in the statement of \Cref{prop:multi-pass-lb} comparing to \cite{AW23BestArm}.
% \begin{enumerate}
% 	\item Allowing $P\leq B$. In the construction of \cite{AW23BestArm} (as shown above), we have $B=P$ for a $P$-pass algorithm. From the proof sketch, it can be observed that the argument still holds if $B\geq P$ \emph{and} the memory and sample complexity bound for conditions \textbf{C1} and \textbf{C2} scale with $B$. The gap between $B$ and $P$ can lead to suboptimal memory and sample complexity lower bounds; however, the proposition holds true.
% 	\item The extra $B^2$ factor on the sample complexity. In our \Cref{prop:multi-pass-lb}, instead of showing the bound of \Cref{equ:equ:batch-sample-ub-simple}, we target the sample complexity upper bound of \Cref{equ:batch-sample-ub}, which contains an extra $B^2$ factor. Note that to make this work, we essentially only need to modify the proof of the radical case \Cref{line:ana-radical} such that
% 	\begin{align*}
% 		O\left(\frac{n}{\chi^2_{p+2}}\cdot \frac{1}{\poly(B)}\right) \gg C\cdot B^2\cdot \frac{n}{\chi^2_{p+1}},
% 	\end{align*}
% 	and replace $\chi_{b}$ with $\etaib{1}{b}$. To this end, we can \emph{augment the multiplicative gap} between $\chi_{b}$ and $\chi_{b+1}$ by a $1/\poly(B)$ factor. This is also the technical reason for us to use $\chi_{b} = \Theta(1/B^{13b})$ as opposed to $\chi_{b} = \Theta(1/B^{10b})$ in \cite{AW23BestArm}. 
% \end{enumerate}

% Combining the above cases give us the desired statement of \Cref{prop:multi-pass-lb}.



\end{document}