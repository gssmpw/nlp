\section{Related Work}
Recent advancements in data-driven weather forecasting have increasingly leveraged deep learning techniques to improve weather elements forecasting and nowcasting \cite{10220887, 9308323, aykas2021multistream, chen2020deep, bilgin2021tent}. Various neural network architectures have been successfully applied in weather forecasting, including Recurrent Neural Networks (RNNs) \cite{singh2019weather}, Long Short-Term Memory (LSTM) networks \cite{zaytar2016sequence}, Convolutional LSTM (ConvLSTM) \cite{shi2015convolutional}, Convolutional Neural Networks (CNNs) \cite{ayzel2020rainnet}, Graph Neural Networks (GNNs) \cite{simeunovic2021spatio}, and encoder-decoder frameworks \cite{larraondo2019data}.

More recently, the UNet architecture \cite{ronneberger2015unetconvolutionalnetworksbiomedical}, originally developed for medical image segmentation, has demonstrated effectiveness in precipitation due to its encoder-decoder structure with skip connections, enabling the preservation of fine-grained spatial details. Variations of UNet have been proposed to enhance performance while maintaining computational efficiency for weather prediction. For example, SmaAt-UNet \cite{trebing2021smaat}, an enhanced version of the UNet model that integrates attention modules and depthwise-separable convolutions, substantially decreases the number of trainable parameters in the traditional UNet while maintaining its performance. SelfAttUNet \cite{nie2021self} integrates a self-attention mechanism into the UNet model, allowing it to highlight key regions in an image for more effective radar-based precipitation nowcasting.  
Several other studies, such as \cite{kaparakis2023wf, reulen2024ga, renault2023sar}, have also explored improvements to deep learning models built on the core UNet architecture, highlighting their effectiveness in weather nowcasting tasks. 

In addition to CNN-based architectures, data fusion techniques have been increasingly explored to integrate multiple meteorological data sources for enhanced precipitation nowcasting. Traditional models primarily rely on radar reflectivity data; however, recent studies have demonstrated that incorporating additional atmospheric variables, such as wind speed, temperature, and humidity, significantly improves short-term forecasting accuracy. By leveraging multiple data sources, fusion models better capture complex weather dynamics, leading to more reliable nowcasting performance. Kaparakis et al. \cite{kaparakis2023wf} proposed the Weather Fusion UNet (WF-UNet), a fusion-based deep learning model for precipitation nowcasting over Western Europe. Their findings underscore the role of fusion models in enhancing short-term precipitation predictions by leveraging complementary data sources. Similarly, Leinonen et al. \cite{leinonen2023thunderstorm} introduced a deep learning-based thunderstorm nowcasting model that integrates diverse meteorological data, including weather radar, lightning detection, satellite imagery, numerical weather prediction outputs, and digital elevation models. Furthermore, the Spatiotemporal Feature Fusion Transformer \cite{xiong2024spatiotemporal} incorporates feature-level fusion mechanisms to enhance precipitation nowcasting performance. Collectively, these works illustrate the growing impact of data fusion techniques in weather nowcasting, highlighting their potential to bridge the gap between numerical weather prediction and deep learning-based methodologies.