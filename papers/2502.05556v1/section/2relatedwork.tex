\section{Related Work}

\subsection{Cognitive Diagnosis}
Cognitive diagnosis, which originated from educational psychology, is a fundamental task in the field of intelligent education. It characterizes students' learning status and knowledge proficiency based on their responses to various questions~\cite{liu2021towards}. Existing cognitive diagnosis methods are mainly divided into two main categories: psychometric theory-based methods~\cite{lord1952theory,de2009dina,reckase200618} and neural network-based methods~\cite{wang2020neural,gao2021rcd,bi2023beta,liu2024inductive,wang2023self}.
Psychometric theory-based methods, such as Item Response Theory (IRT)~\cite{lord1952theory}, Multidimensional IRT (MIRT)~\cite{reckase200618}, and Deterministic Inputs, Noisy And gate model (DINA)~\cite{de2009dina}, are designed to evaluate students' proficiency through latent factors utilizing psychological theories.
Neural network-based methods use deep neural networks to profile students' learning status. NCD~\cite{wang2020neural} first incorporates neural networks into cognitive diagnosis to effectively capture the fine-grained student-exercise relationships. RCD~\cite{gao2021rcd} and RDGT~\cite{yu2024rdgt} employ graph architectures to explore the relationships among exercises, knowledge concepts, and students. Recently, BETA-CD~\cite{bi2023beta} developed a reliable and rapidly adaptable cognitive diagnosis framework for new students through meta-learning. ACD~\cite{wang2024boosting} considered the connection between students' affective states and cognitive states in learning. However, few existing cognitive diagnosis methods take into account prior knowledge, which makes it challenging for them to generate accurate diagnoses. 

\subsection{Large Language Models}
With the rise of Transformer~\cite{vaswani2017attention}, large language models (LLMs) with extensive parameters and vast training data have gradually become mainstream. 
%As the size of the models increases, LLMs have demonstrated powerful reasoning abilities and a wealth of knowledge, which smaller language models do not possess~\cite{wei2022emergent}.
LLMs usually follow a pre-training and fine-tuning approach to accommodate various downstream tasks. They have significantly improved performance in numerous NLP applications, including text summarization~\cite{laskar2022domain,zhang2023summit}, sentiment analysis~\cite{hoang2019aspect,deng2023llms}, translation~\cite{zhang2023prompting,moslem2023adaptive}, and multimodal understanding~\cite{wu2024semantic,huang2024autogeo}. 


The advanced comprehension and reasoning capabilities, along with the extensive knowledge repository of LLMs, naturally lead to potential applications in the realm of education.
LLMs can provide researchers with new perspectives by simulating the roles of teachers or students~\cite{wang2024user,li2023adapting,xu2024eduagent,liu2024personality,lin2024e3}, or generating educational resources~\cite{lin2024non,lin2024action,dai2024mpcoder}.
However, less exploration has been made to utilize LLMs for cognitive diagnosis. 
The demonstrated success of LLMs in text summarization tasks and educational contexts indicates LLMs' capability to undertake cognitive diagnostic tasks.
%Zhang~et.al~\cite{zhuang2023efficiently} attempt to evaluate LLMs cognitive abilities through theories and models of cognitive diagnosis 
%探究如何评测LLM的能力,采用了认知诊断的理论和模型来评测LLM的认知能力
%许多工作在模拟学生状态方面有着

