\section{Dataset}
\label{sec:dataset}

\subsection{Data Collection}

To analyze political discussions on Discord, we followed the methodology in \cite{singh2024Cross-Platform}, collecting messages from politically-oriented public servers in compliance with Discord's platform policies.

Using Discord's Discovery feature, we employed a web scraper to extract server invitation links, names, and descriptions, focusing on public servers accessible without participation. Invitation links were used to access data via the Discord API. To ensure relevance, we filtered servers using keywords related to the 2024 U.S. elections (e.g., Trump, Kamala, MAGA), as outlined in \cite{balasubramanian2024publicdatasettrackingsocial}. This resulted in 302 server links, further narrowed to 81 English-speaking, politics-focused servers based on their names and descriptions.

Public messages were retrieved from these servers using the Discord API, collecting metadata such as \textit{content}, \textit{user ID}, \textit{username}, \textit{timestamp}, \textit{bot flag}, \textit{mentions}, and \textit{interactions}. Through this process, we gathered \textbf{33,373,229 messages} from \textbf{82,109 users} across \textbf{81 servers}, including \textbf{1,912,750 messages} from \textbf{633 bots}. Data collection occurred between November 13th and 15th, covering messages sent from January 1st to November 12th, just after the 2024 U.S. election.

\subsection{Characterizing the Political Spectrum}
\label{sec:timeline}

A key aspect of our research is distinguishing between Republican- and Democratic-aligned Discord servers. To categorize their political alignment, we relied on server names and self-descriptions, which often include rules, community guidelines, and references to key ideologies or figures. Each server's name and description were manually reviewed based on predefined, objective criteria, focusing on explicit political themes or mentions of prominent figures. This process allowed us to classify servers into three categories, ensuring a systematic and unbiased alignment determination.

\begin{itemize}
    \item \textbf{Republican-aligned}: Servers referencing Republican and right-wing and ideologies, movements, or figures (e.g., MAGA, Conservative, Traditional, Trump).  
    \item \textbf{Democratic-aligned}: Servers mentioning Democratic and left-wing ideologies, movements, or figures (e.g., Progressive, Liberal, Socialist, Biden, Kamala).  
    \item \textbf{Unaligned}: Servers with no defined spectrum and ideologies or opened to general political debate from all orientations.
\end{itemize}

To ensure the reliability and consistency of our classification, three independent reviewers assessed the classification following the specified set of criteria. The inter-rater agreement of their classifications was evaluated using Fleiss' Kappa \cite{fleiss1971measuring}, with a resulting Kappa value of \( 0.8191 \), indicating an almost perfect agreement among the reviewers. Disagreements were resolved by adopting the majority classification, as there were no instances where a server received different classifications from all three reviewers. This process guaranteed the consistency and accuracy of the final categorization.

Through this process, we identified \textbf{7 Republican-aligned servers}, \textbf{9 Democratic-aligned servers}, and \textbf{65 unaligned servers}.

Table \ref{tab:statistics} shows the statistics of the collected data. Notably, while Democratic- and Republican-aligned servers had a comparable number of user messages, users in the latter servers were significantly more active, posting more than double the number of messages per user compared to their Democratic counterparts. 
This suggests that, in our sample, Democratic-aligned servers attract more users, but these users were less engaged in text-based discussions. Additionally, around 10\% of the messages across all server categories were posted by bots. 

\subsection{Temporal Data} 

Throughout this paper, we refer to the election candidates using the names adopted by their respective campaigns: \textit{Kamala}, \textit{Biden}, and \textit{Trump}. To examine how the content of text messages evolves based on the political alignment of servers, we divided the 2024 election year into three periods: \textbf{Biden vs Trump} (January 1 to July 21), \textbf{Kamala vs Trump} (July 21 to September 20), and the \textbf{Voting Period} (after September 20). These periods reflect key phases of the election: the early campaign dominated by Biden and Trump, the shift in dynamics with Kamala Harris replacing Joe Biden as the Democratic candidate, and the final voting stage focused on electoral outcomes and their implications. This segmentation enables an analysis of how discourse responds to pivotal electoral moments.

Figure \ref{fig:line-plot} illustrates the distribution of messages over time, highlighting trends in total messages volume and mentions of each candidate. Prior to Biden's withdrawal on July 21, mentions of Biden and Trump were relatively balanced. However, following Kamala's entry into the race, mentions of Trump surged significantly, a trend further amplified by an assassination attempt on him, solidifying his dominance in the discourse. The only instance where Trump’s mentions were exceeded occurred during the first debate, as concerns about Biden’s age and cognitive abilities temporarily shifted the focus. In the final stages of the election, mentions of all three candidates rose, with Trump’s mentions peaking as he emerged as the victor.
\section{Experiments}
\label{sec: exp}

In this section, we conduct experiments to answer the following research questions:
\begin{itemize}
\item \textbf{RQ1}: Can the proposed model effectively improve the performance of the original CDMs?  
\item \textbf{RQ2}: What is the impact of each component within the proposed method? 
\item \textbf{RQ3}: How does the proposed model perform on cold-start scenarios? 
% \item \textbf{RQ4}: What are the differences in diagnostic effectiveness when using different LLMs?
\item \textbf{RQ4}: How effective is the alignment of semantic and behavioral space embeddings during the cognitive level alignment process?
\end{itemize}

\subsection{Experimental Settings}

\subsubsection{Datasets}

\newcommand{\boformat}[1]{$\mathbf{#1}$}
\newcommand{\blformat}[1]{\textcolor{blue}{$\mathbf{#1}$}}

\def\arraystretch{1.1}
\setlength{\tabcolsep}{0.5em} %

\begin{table}[t!] \centering
    \caption{Comparison of TD3 with existing dataset distillation techniques and heuristic sampling based methods.
    }
    \vspace{-8pt}
    \label{tab:baselines}
    \begin{center}
    \resizebox{0.48\textwidth}{!}{
    \begin{tabular}{cc|cc|cc|c}
        \toprule
        \multirow{2.8}{*}{Dataset} & \multirow{2.8}{*}{Metric} & \multicolumn{2}{c|}{Sampling} & \multicolumn{2}{c|}{Distillation} & \multirow{2.8}{*}{Full-Data} \\[3pt]
        
        \cmidrule{3-6}
        & & \multicolumn{1}{c}{Random.} & \multicolumn{1}{c|}{Longest.} & \multicolumn{1}{c}{Farzi} & \multicolumn{1}{c|}{\sampler} \\[2pt]
        
        \midrule
        \multirow{4}{*}{\STAB{Magazine\\\texttt{[30$\times$20]}}} 
        & HR@10 $\uparrow$     & 15.44 \std{$\pm$2.68} & 19.62 \std{$\pm$0.31} &      41.46 \std{$\pm$1.27} & \textbf{\ul{47.87} \std{$\pm$1.54}} & \textit{45.40} \\
        & HR@20 $\uparrow$     & 27.50 \std{$\pm$1.03} & 33.66 \std{$\pm$2.73} & \ul{58.70} \std{$\pm$0.13} & \textbf{\ul{61.17} \std{$\pm$1.83}} & \textit{56.98} \\
        & NDCG@10 $\uparrow$   &  7.75 \std{$\pm$1.16} &  9.58 \std{$\pm$0.58} &      24.27 \std{$\pm$0.20} & \textbf{\ul{27.90} \std{$\pm$0.88}} & \textit{25.63} \\
        & NDCG@20 $\uparrow$   & 10.75 \std{$\pm$0.77} & 13.10 \std{$\pm$0.30} & \ul{28.65} \std{$\pm$0.13} & \textbf{\ul{31.25} \std{$\pm$0.98}} & \textit{28.57} \\
        \midrule
        \multirow{4}{*}{\STAB{Epinions\\\texttt{[15$\times$30]}}} 
        & HR@10 $\uparrow$     & 10.67 \std{$\pm$0.90} & 10.24 \std{$\pm$0.21} &      18.99 \std{$\pm$0.30} & \textbf{\ul{19.86} \std{$\pm$0.10}} & \textit{19.13} \\
        & HR@20 $\uparrow$     & 20.25 \std{$\pm$1.25} & 20.01 \std{$\pm$0.41} &      29.82 \std{$\pm$0.39} & \textbf{\ul{31.06} \std{$\pm$0.19}} & \textit{30.09} \\
        & NDCG@10 $\uparrow$   &  4.93 \std{$\pm$0.36} &  4.79 \std{$\pm$0.06} & \ul{10.38} \std{$\pm$0.17} & \textbf{\ul{10.67} \std{$\pm$0.05}} & \textit{10.25} \\
        & NDCG@20 $\uparrow$   &  7.31 \std{$\pm$0.45} &  7.22 \std{$\pm$0.11} & \ul{13.09} \std{$\pm$0.07} & \textbf{\ul{13.49} \std{$\pm$0.08}} & \textit{13.00} \\
        \midrule
        \multirow{4}{*}{\STAB{ML-100k\\\texttt{[30$\times$50]}}} 
        & HR@10 $\uparrow$     & 10.85 \std{$\pm$2.30} & 13.36 \std{$\pm$0.45} & 62.92 \std{$\pm$1.39} & \textbf{66.14 \std{$\pm$1.16}} & \textit{68.93} \\
        & HR@20 $\uparrow$     & 21.49 \std{$\pm$3.98} & 25.59 \std{$\pm$0.64} & 77.84 \std{$\pm$0.30} & \textbf{81.37 \std{$\pm$0.43}} & \textit{83.78} \\
        & NDCG@10 $\uparrow$   &  4.81 \std{$\pm$1.10} &  5.97 \std{$\pm$0.32} & 34.92 \std{$\pm$0.71} & \textbf{38.56 \std{$\pm$0.86}} & \textit{40.97} \\
        & NDCG@20 $\uparrow$   &  7.48 \std{$\pm$1.28} &  9.02 \std{$\pm$0.36} & 38.72 \std{$\pm$0.40} & \textbf{42.44 \std{$\pm$0.72}} & \textit{44.76} \\
        \midrule
        \multirow{4}{*}{\STAB{ML-1M\\\texttt{[200$\times$50]}}} 
        & HR@10 $\uparrow$     & 15.88 \std{$\pm$0.22} & 16.60 \std{$\pm$0.51} & 38.01 \std{$\pm$0.98} & \textbf{70.52 \std{$\pm$0.36}} & \textit{79.32} \\
        & HR@20 $\uparrow$     & 28.09 \std{$\pm$0.31} & 30.93 \std{$\pm$0.45} & 56.10 \std{$\pm$1.24} & \textbf{82.52 \std{$\pm$0.25}} & \textit{87.60} \\
        & NDCG@10 $\uparrow$   &  7.40 \std{$\pm$0.11} &  7.77 \std{$\pm$0.16} & 19.85 \std{$\pm$0.49} & \textbf{45.93 \std{$\pm$0.37}} & \textit{58.82} \\
        & NDCG@20 $\uparrow$   & 10.47 \std{$\pm$0.10} & 11.36 \std{$\pm$0.14} & 24.41 \std{$\pm$0.55} & \textbf{48.97 \std{$\pm$0.30}} & \textit{60.93} \\
        \bottomrule
    \end{tabular}}
    \end{center}
\end{table}

In our experiments, we utilize four courses, Python Programming (Python), Linux System (Linux), Database Technology and Application (Database), and Literature and History (Literature), from a publicly available dataset PTADisc~\cite{hu2023ptadisc}, which comes from real-world students' responses in the educational website PTA\footnote{\url{https://pintia.cn/}} and contains textual information of exercises and knowledge concepts. 
%Each response log in the dataset contains a student ID, an exercise ID, whether the student correctly answers the question, the content of the exercise, and the knowledge concepts related to the exercise.
The statistics of the datasets are presented in Table~\ref{tab: dataset}.
The datasets are divided into training, validation, and testing sets, with a ratio of 8:1:1.

\subsubsection{Evaluation Metrics}

Following previous works, we evaluate the students' cognitive status by predicting the performance of students on the testing set, as the cognitive status can not be directly observed. We adopt commonly used metrics, namely the Area Under a ROC Curve (AUC), the Prediction Accuracy (ACC), and the Root Mean Square Error (RMSE), to validate the effectiveness of the CDMs.
%In the subsequent tables, \textbf{bold} numbers represent the best performance, while \underline{underlined} numbers represent the second-best performance. 
For all the metrics, $\uparrow$ represents that a greater value is better, while $\downarrow$ represents the opposite.

\subsubsection{Baseline Methods}

To validate the effectiveness of the proposed method, we conduct experiments on several representative CDMs, including IRT~\cite{lord1952theory}, MIRT~\cite{reckase200618}, DINA~\cite{de2009dina}, NCD~\cite{wang2020neural}, RCD~\cite{gao2021rcd}, SCD~\cite{wang2023self} and ACD~\cite{wang2024boosting}.
 

\subsubsection{Implementation Details}

We utilize PyTorch to implement both the baseline methods and our proposed KCD framework. 
For the baseline models, We use the default hyper-parameters as stated in their papers and for KCD, we use the same hyper-parameter settings, such as training epoch, learning rate, and batch size.
We employ ChatGPT to represent LLMs (specifically, gpt-3.5-turbo-16k) and text-embedding-ada002 as the text embedding model. All the experiments are conducted on a GeForce RTX 3090 GPU.
We train the model on train set and at the end of each epoch, we evaluate the model on the validation set.
The hyper-parameter $\alpha$, $\beta$, and $\lambda$ was set to $0.04$, $0.015$, and $0.2$.
Since our dataset does not include affect labels, we utilize the unsupervised contrastive ACD model and employ NCD as the basic cognitive diagnosis module.
The behavioral space alignment approach is denoted as `-Beh' and the semantic space alignment approach is denoted as `-Sem'.
% We investigated the impact of the hyper-parameter $\lambda$, within the range $[0,0.2,\cdots,1]$ with a step size of $0.2$. Our analysis revealed that setting $\lambda$ to $0.1$ resulted in the best performance across all three datasets.

\begin{figure}[t]
  \centering
  
  \includegraphics[width=1.02\linewidth]{figs/experimentx.png}
  \caption{Performance comparison in cold (blue) and warm (red) scenarios on Python dataset.}
  \vspace{-2em}
\label{fig: experiment1}
\end{figure}

\subsection{Performance Comparison (RQ1)}
To demonstrate the effectiveness of our proposed method in improving cognitive diagnosis, we implement the framework on seven cognitive diagnosis models, and the results are shown in Table~\ref{tab:performance}. 
Additionally, we compared the performance of NCD in warm and cold scenarios, with the results illustrated in Figure~\ref{fig: experiment1}. Here we define the cold scenario as less than $3$ interactions in the training set for exercises and define the warm scenario as more than $10$ interactions in the training set for exercises. Following this definition, we divide the testing set into cold and warm subsets.
We have the following observations from the results: 

\begin{itemize}[leftmargin=*]
    \item[1)]  
    Both KCD-Beh and KCD-Sem achieve significant improvements compared to the basic CDMs.
    This indicates that our proposed framework is widely applicable to various CDMs, and both alignment methods can effectively align the behavioral space of CDMs and the semantic space of LLMs.
    In most models, the behavioral space alignment approach performs better, indicating that aligning in the behavioral space of CDMs can better align information from the semantic space of LLMs.
    \item[2)] Compared to basic CDMs, our proposed methods demonstrate improvements in both cold and warm scenarios, especially in cold scenarios. This indicates that our approach of introducing LLMs as knowledge enhancement effectively alleviates the cold-start issue.
\end{itemize}




% \begin{table}[!t]
% \centering
% \scalebox{0.68}{
%     \begin{tabular}{ll cccc}
%       \toprule
%       & \multicolumn{4}{c}{\textbf{Intellipro Dataset}}\\
%       & \multicolumn{2}{c}{Rank Resume} & \multicolumn{2}{c}{Rank Job} \\
%       \cmidrule(lr){2-3} \cmidrule(lr){4-5} 
%       \textbf{Method}
%       &  Recall@100 & nDCG@100 & Recall@10 & nDCG@10 \\
%       \midrule
%       \confitold{}
%       & 71.28 &34.79 &76.50 &52.57 
%       \\
%       \cmidrule{2-5}
%       \confitsimple{}
%     & 82.53 &48.17
%        & 85.58 &64.91
     
%        \\
%        +\RunnerUpMiningShort{}
%     &85.43 &50.99 &91.38 &71.34 
%       \\
%       +\HyReShort
%         &- & -
%        &-&-\\
       
%       \bottomrule

%     \end{tabular}
%   }
% \caption{Ablation studies using Jina-v2-base as the encoder. ``\confitsimple{}'' refers using a simplified encoder architecture. \framework{} trains \confitsimple{} with \RunnerUpMiningShort{} and \HyReShort{}.}
% \label{tbl:ablation}
% \end{table}
\begin{table*}[!t]
\centering
\scalebox{0.75}{
    \begin{tabular}{l cccc cccc}
      \toprule
      & \multicolumn{4}{c}{\textbf{Recruiting Dataset}}
      & \multicolumn{4}{c}{\textbf{AliYun Dataset}}\\
      & \multicolumn{2}{c}{Rank Resume} & \multicolumn{2}{c}{Rank Job} 
      & \multicolumn{2}{c}{Rank Resume} & \multicolumn{2}{c}{Rank Job}\\
      \cmidrule(lr){2-3} \cmidrule(lr){4-5} 
      \cmidrule(lr){6-7} \cmidrule(lr){8-9} 
      \textbf{Method}
      & Recall@100 & nDCG@100 & Recall@10 & nDCG@10
      & Recall@100 & nDCG@100 & Recall@10 & nDCG@10\\
      \midrule
      \confitold{}
      & 71.28 & 34.79 & 76.50 & 52.57 
      & 87.81 & 65.06 & 72.39 & 56.12
      \\
      \cmidrule{2-9}
      \confitsimple{}
      & 82.53 & 48.17 & 85.58 & 64.91
      & 94.90&78.40 & 78.70& 65.45
       \\
      +\HyReShort{}
       &85.28 & 49.50
       &90.25 & 70.22
       & 96.62&81.99 & \textbf{81.16}& 67.63
       \\
      +\RunnerUpMiningShort{}
       % & 85.14& 49.82
       % &90.75&72.51
       & \textbf{86.13}&\textbf{51.90} & \textbf{94.25}&\textbf{73.32}
       & \textbf{97.07}&\textbf{83.11} & 80.49& \textbf{68.02}
       \\
   %     +\RunnerUpMiningShort{}
   %    & 85.43 & 50.99 & 91.38 & 71.34 
   %    & 96.24 & 82.95 & 80.12 & 66.96
   %    \\
   %    +\HyReShort{} old
   %     &85.28 & 49.50
   %     &90.25 & 70.22
   %     & 96.62&81.99 & 81.16& 67.63
   %     \\
   % +\HyReShort{} 
   %     % & 85.14& 49.82
   %     % &90.75&72.51
   %     & 86.83&51.77 &92.00 &72.04
   %     & 97.07&83.11 & 80.49& 68.02
   %     \\
      \bottomrule

    \end{tabular}
  }
\caption{\framework{} ablation studies. ``\confitsimple{}'' refers using a simplified encoder architecture. \framework{} trains \confitsimple{} with \RunnerUpMiningShort{} and \HyReShort{}. We use Jina-v2-base as the encoder due to its better performance.
}
\label{tbl:ablation}
\end{table*}
\subsection{Ablation Study (RQ2)}


To validate the effectiveness of different components of our proposed method, we conduct ablation experiments to verify several components utilized in LLM Diagnosis and Cognitive Level alignment, including the usage of collaborative information (denoted as `Coll. Info'), the local contrast and global contrast (denoted as `Local Con.' and `Global Con.'), and the dynamic masking strategy (denoted as `Dym. Mask').

Table~\ref{tab:ablation} demonstrates the results of the ablation study on Python dataset, comparing the model performance after removing specific components (denoted as `w/o'). `w/o Coll. Info' represents replacing collaborative information in the process of diagnosis generation and `w/o Dym. Mask' represents replacing dynamic masking strategy with a constant mask ratio.
Experimental results show that removing these components individually leads to a decline in the model's performance. This indicates that these components are crucial for the model's performance.


\begin{figure}[t]
  \centering
  
  \includegraphics[width=1\linewidth]{figs/drop.png}
  \caption{Performance on different dropout ratios.}
  
\label{fig: drop}
\end{figure}
\subsection{Performance on Cold-Start Scenarios (RQ3)}

we conduct additional experiments on sub-datasets with varying degrees of sparsity. Specifically, we apply random dropout to the training sets of the Python and Linux datasets at ratios of $10\%$, $20\%$, $30\%$, $40\%$, and $50\%$.

Figure~\ref{fig: drop} shows the results of the experiments on different dropout ratios. It is obvious that as the dropout ratio increases, both AUC and ACC decrease. This is because the training set becomes more sparse, approaching a cold-start scenario. 
Additionally, compared to ACC, AUC experiences a greater decline, which might be due to the different calculation methods of the two metrics. 
% For more sparse datasets, Python, AUC experience a more significant decrease compared to the Linux dataset. From the experimental results, it can be seen that our proposed method is effective across different dropout ratios, leading to significant improvements for CDMs. More specifically, from the different performances of NCD-Beh and NCD-Sem in the Linux and Python datasets, it can be seen that we can choose different alignment methods based on the dataset to achieve better diagnostic results.


\begin{figure}[t]
  \centering
  \vspace{-1em}
  \includegraphics[width=1\linewidth]{figs/experiment2.png}
  \caption{The t-SNE visualization of student embeddings on Literature dataset.}
  \vspace{-2em}
\label{fig: experiment2}
\end{figure}
\subsection{Visualization of Semantic and Behavioral Embeddings (RQ4)}


To validate the effectiveness of the two alignment processes, we utilize t-SNE~\cite{van2008visualizing} to visualize the distribution of features in LLMs semantic space and CDMs behavioral space. We randomly select 200 example students and map their behavioral embeddings and semantic embeddings to 2-dimensional space. NCD (w/o Alignment) represents the original CDMs without alignment.

Figure~\ref{fig: experiment2} demonstrates the integration of semantic and behavioral embeddings of NCD-Beh and NCD-Sem, with their distributions closely merged compared to original CDMs. This proves the effectiveness of the two alignment methods we proposed.

\begin{figure}[t]
  \centering
  
  \includegraphics[width=1\linewidth]{figs/case.png}
  \caption{The case study of a student on multiple knowledge concepts on Linux dataset.}
  \vspace{-2em}
\label{fig: case}
\end{figure}

\subsection{Case Study}


To more intuitively demonstrate the improvements our proposed methods bring to CDMs, we selected a diagnosis for a specific student in the Linux dataset and compared the prediction results of NCD with the diagnosis results of NCD-Beh.
As illustrated in Figure~\ref{fig: case}, we randomly choose a student, and list his mastery of some knowledge concepts predicted by NCD and our proposed NCD-Beh.
This student correctly answered the exercises related to `numerical encoding' and `process communication', showing mastery of these concepts. He answered other exercises incorrectly, indicating a lack of familiarity with the remaining knowledge concepts.
From the LLM's diagnostic results, it can be observed that the LLM captured similar question-answer information from the training set and made corresponding inferences. This played an important role in NCD-Beh's more accurate prediction of the student's mastery level.