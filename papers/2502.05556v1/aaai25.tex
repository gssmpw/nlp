%File: aaai25.tex
%release 2025.0
\pdfoutput=1
\documentclass[letterpaper]{article} % DO NOT CHANGE THIS
\usepackage{aaai25}  % DO NOT CHANGE THIS
\usepackage{times}  % DO NOT CHANGE THIS
\usepackage{helvet}  % DO NOT CHANGE THIS
\usepackage{courier}  % DO NOT CHANGE THIS
\usepackage[hyphens]{url}  % DO NOT CHANGE THIS
\usepackage{graphicx} % DO NOT CHANGE THIS
\urlstyle{rm} % DO NOT CHANGE THIS
\def\UrlFont{\rm}  % DO NOT CHANGE THIS
\usepackage{natbib}  % DO NOT CHANGE THIS AND DO NOT ADD ANY OPTIONS TO IT
\usepackage{caption} % DO NOT CHANGE THIS AND DO NOT ADD ANY OPTIONS TO IT
\frenchspacing  % DO NOT CHANGE THIS
\setlength{\pdfpagewidth}{8.5in}  % DO NOT CHANGE THIS
\setlength{\pdfpageheight}{11in}  % DO NOT CHANGE THIS
%
% These are recommended to typeset algorithms but not required. See the subsubsection on algorithms. Remove them if you don't have algorithms in your paper.
\usepackage{algorithm}
\usepackage{algorithmic}
\usepackage{times}  % DO NOT CHANGE THIS
\usepackage{helvet}  % DO NOT CHANGE THIS
\usepackage{amsmath}
\usepackage{multirow}
\usepackage{graphicx}
\usepackage{subfigure}
\usepackage{bbding}
\usepackage{makecell}
\usepackage{enumitem}
\usepackage{verbatim}
\usepackage{ragged2e}
\usepackage{booktabs} 
\usepackage{amssymb}
\usepackage{courier}  % DO NOT CHANGE THIS
\usepackage[hyphens]{url}  % DO NOT CHANGE THIS
\usepackage{graphicx} % DO NOT CHANGE THIS
\urlstyle{rm} % DO NOT CHANGE THIS
\def\UrlFont{\rm}  % DO NOT CHANGE THIS
\usepackage{natbib}  % DO NOT CHANGE THIS AND DO NOT ADD ANY OPTIONS TO IT
\usepackage{caption} % DO NOT CHANGE THIS AND DO NOT ADD ANY OPTIONS TO IT
\frenchspacing  % DO NOT CHANGE THIS
\setlength{\pdfpagewidth}{8.5in} % DO NOT CHANGE THIS
\setlength{\pdfpageheight}{11in} % DO NOT CHANGE THIS
%
% These are recommended to typeset algorithms but not required. See the subsubsection on algorithms. Remove them if you don't have algorithms in your paper.
\usepackage{algorithm}
\usepackage{algorithmic}
\usepackage{color}

\usepackage{xcolor}
%
% These are are recommended to typeset listings but not required. See the subsubsection on listing. Remove this block if you don't have listings in your paper.
\usepackage{newfloat}
\usepackage{listings}
\DeclareCaptionStyle{ruled}{labelfont=normalfont,labelsep=colon,strut=off} % DO NOT CHANGE THIS
\lstset{%
	basicstyle={\footnotesize\ttfamily},% footnotesize acceptable for monospace
	numbers=left,numberstyle=\footnotesize,xleftmargin=2em,% show line numbers, remove this entire line if you don't want the numbers.
	aboveskip=0pt,belowskip=0pt,%
	showstringspaces=false,tabsize=2,breaklines=true}
\floatstyle{ruled}
\newfloat{listing}{tb}{lst}{}
\floatname{listing}{Listing}
%
% Keep the \pdfinfo as shown here. There's no need
% for you to add the /Title and /Author tags.
\pdfinfo{
/TemplateVersion (2025.1)
}

% DISALLOWED PACKAGES
% \usepackage{authblk} -- This package is specifically forbidden
% \usepackage{balance} -- This package is specifically forbidden
% \usepackage{color (if used in text)
% \usepackage{CJK} -- This package is specifically forbidden
% \usepackage{float} -- This package is specifically forbidden
% \usepackage{flushend} -- This package is specifically forbidden
% \usepackage{fontenc} -- This package is specifically forbidden
% \usepackage{fullpage} -- This package is specifically forbidden
% \usepackage{geometry} -- This package is specifically forbidden
% \usepackage{grffile} -- This package is specifically forbidden
% \usepackage{hyperref} -- This package is specifically forbidden
% \usepackage{navigator} -- This package is specifically forbidden
% (or any other package that embeds links such as navigator or hyperref)
% \indentfirst} -- This package is specifically forbidden
% \layout} -- This package is specifically forbidden
% \multicol} -- This package is specifically forbidden
% \nameref} -- This package is specifically forbidden
% \usepackage{savetrees} -- This package is specifically forbidden
% \usepackage{setspace} -- This package is specifically forbidden
% \usepackage{stfloats} -- This package is specifically forbidden
% \usepackage{tabu} -- This package is specifically forbidden
% \usepackage{titlesec} -- This package is specifically forbidden
% \usepackage{tocbibind} -- This package is specifically forbidden
% \usepackage{ulem} -- This package is specifically forbidden
% \usepackage{wrapfig} -- This package is specifically forbidden
% DISALLOWED COMMANDS
% \nocopyright -- Your paper will not be published if you use this command
% \addtolength -- This command may not be used
% \balance -- This command may not be used
% \baselinestretch -- Your paper will not be published if you use this command
% \clearpage -- No page breaks of any kind may be used for the final version of your paper
% \columnsep -- This command may not be used
% \newpage -- No page breaks of any kind may be used for the final version of your paper
% \pagebreak -- No page breaks of any kind may be used for the final version of your paperr
% \pagestyle -- This command may not be used
% \tiny -- This is not an acceptable font size.
% \vspace{- -- No negative value may be used in proximity of a caption, figure, table, section, subsection, subsubsection, or reference
% \vskip{- -- No negative value may be used to alter spacing above or below a caption, figure, table, section, subsection, subsubsection, or reference

\setcounter{secnumdepth}{0} %May be changed to 1 or 2 if section numbers are desired.

% The file aaai25.sty is the style file for AAAI Press
% proceedings, working notes, and technical reports.
%

% Title

% Your title must be in mixed case, not sentence case.
% That means all verbs (including short verbs like be, is, using,and go),
% nouns, adverbs, adjectives should be capitalized, including both words in hyphenated terms, while
% articles, conjunctions, and prepositions are lower case unless they
% directly follow a colon or long dash
\title{Knowledge is Power: Harnessing Large Language Models\\ for Enhanced Cognitive Diagnosis}
\author{
    %Authors
    % All authors must be in the same font size and format.
    Zhiang Dong,
    Jingyuan Chen\thanks{Corresponding Author.},
    Fei Wu $^{*}$
}
\affiliations{
    %Afiliations
    Zhejiang University\\
    % If you have multiple authors and multiple affiliations
    % use superscripts in text and roman font to identify them.
    % For example,

    % Sunil Issar\textsuperscript{\rm 2}, 
    % J. Scott Penberthy\textsuperscript{\rm 3}, 
    % George Ferguson\textsuperscript{\rm 4},
    % Hans Guesgen\textsuperscript{\rm 5}
    % Note that the comma should be placed after the superscript
    % email address must be in roman text type, not monospace or sans serif
    \{dongza,jingyuanchen,wufei\}@zju.edu.cn
%
% See more examples next
}

%Example, Single Author, ->> remove \iffalse,\fi and place them surrounding AAAI title to use it
\iffalse
\title{My Publication Title --- Single Author}
\author {
    Author Name
}
\affiliations{
    Affiliation\\
    Affiliation Line 2\\
    name@example.com
}
\fi

\iffalse
%Example, Multiple Authors, ->> remove \iffalse,\fi and place them surrounding AAAI title to use it
\title{My Publication Title --- Multiple Authors}
\author {
    % Authors
    First Author Name\textsuperscript{\rm 1,\rm 2},
    Second Author Name\textsuperscript{\rm 2},
    Third Author Name\textsuperscript{\rm 1}
}
\affiliations {
    % Affiliations
    \textsuperscript{\rm 1}Affiliation 1\\
    \textsuperscript{\rm 2}Affiliation 2\\
    firstAuthor@affiliation1.com, secondAuthor@affilation2.com, thirdAuthor@affiliation1.com
}
\fi


% REMOVE THIS: bibentry
% This is only needed to show inline citations in the guidelines document. You should not need it and can safely delete it.
\usepackage{bibentry}
% END REMOVE bibentry

\begin{document}

\maketitle

\begin{abstract}
    Cognitive Diagnosis Models (CDMs) are designed to assess students' cognitive states by analyzing their performance across a series of exercises. However, existing CDMs often struggle with diagnosing infrequent students and exercises due to a lack of rich prior knowledge. 
    With the advancement in large language models (LLMs), which possess extensive domain knowledge, their integration into cognitive diagnosis presents a promising opportunity.
    Despite this potential, integrating LLMs with CDMs poses significant challenges. LLMs are not well-suited for capturing the fine-grained collaborative interactions between students and exercises, and the disparity between the semantic space of LLMs and the behavioral space of CDMs hinders effective integration.
    To address these issues, we propose a novel \textbf{K}nowledge-enhanced \textbf{C}ognitive \textbf{D}iagnosis (KCD) framework, which is a model-agnostic framework utilizing LLMs to enhance CDMs and compatible with various CDM architectures. The KCD framework operates in two stages: LLM Diagnosis and Cognitive Level Alignment. In the LLM Diagnosis stage, both students and exercises are diagnosed to achieve comprehensive and detailed modeling. In the Cognitive Level Alignment stage, we bridge the gap between the CDMs' behavioral space and the LLMs' semantic space using contrastive learning and mask-reconstruction approaches.
    Experiments on several real-world datasets demonstrate the effectiveness of our proposed framework.
\end{abstract}

% Uncomment the following to link to your code, datasets, an extended version or similar.
%
% \begin{links}
%     \link{Code}{https://aaai.org/example/code}
%     \link{Datasets}{https://aaai.org/example/datasets}
%     \link{Extended version}{https://aaai.org/example/extended-version}
% \end{links}


\section{Introduction}

\footnotetext[0]{ Correspondence to: Yaswanth M <yaswanthm03@gmail.com> and
Vaibhav Singh <singhvaibhav@cse.iitb.ac.in>}
 %Availability of high-quality labeled data, often combined with auxiliary sources of weak supervision, are 
Large language models (LLMs) have facilitated the generation of high-quality synthetic data that often supplement available training data \cite{lin-etal-2023-selective} or even surpass crowd-sourced annotations \cite{augmentationPNAS,alizadeh2023open}. However, concerns of limited variance in such exemplars, leading to model collapse \cite{shumailov2023curse} or the failure to capture the tail of the true underlying distribution \cite{ding2024data}, remain. Similarly, forming multiple views of the available data by inducing rules, as a complementary source of supervision has shown to benefit various NLP tasks, including text classification \cite{maheshwari2021semi,dong-etal-2022-syntactic}.  In this work, we propose \our, a bootstrapping approach to iteratively refine synthetically generated exemplars and automatically induced rules, resulting in high quality entries with respect to a given classification task \cite{yarowsky-1995-unsupervised,varma2018snuba}. 

\begin{figure}[h] 
   \centering
   \includegraphics[trim={0 1.1cm 0 1cm},clip,scale=0.75]{rule_induction-Page-1.drawio.pdf}
   \caption{Overview of ARISE (\textbf{A}utomatic \textbf{R}ule \textbf{I}nduction using \textbf{S}yntactic tree g\textbf{E}neralization).} 
   \label{archi}
\end{figure}

Figure \ref{archi} provides an overview of \our. We start by using available training data as our seed. Using LLMs, we leverage in-context learning (ICL), with the seed as input to synthetically generate candidate exemplars \cite{liu-etal-2022-makes}. Similarly, we generate rule candidates, via inductive generalisation using least general generalization (LGG)~\cite{plotkin1971further,Raza_Gulwani_Milic-Frayling_2014} by extracting syntactic n-grams from the seed. Further, the induced rules are then filtered using a submodular graph cut-based function \cite{bajpai-etal-2024-fair, kothawade2021prism}. The exemplars and the rules we generate are task-specific and each exemplar and rule is associated with a label. Newly generated exemplars are filtered using rules that are generated from the already validated seed. These filtered exemplars are then added to the seed for the next iteration. Iteratively, we induce rules from synthetically generated data and use the induced rules for data filtering. 


%we generate rule candidates by extracting syntactic n-grams from sentence-level dependency parses of the input. Rules are induced from the syntactic n-grams via inductive generalization using   Since the synthetic data are generated along with their labels, those sentences are then validated using the rules. Only those labeled data points that match with the predictions of the rules are filtered. 



%Similarly,  are successfully used as sources of weak supervision . However, concerns associated with model collapse \cite{shumailov2023curse} or failing to capture the tail of the underlying true distribution due to limited variance in such data are raised with data augmentation \cite{ding2024data}. Here, we iteratively use rules and data to filter each otehr resulting in a pool of high quality data and rules which we incoroproate in our learning tasks

%In \our, we propose a bootstrapped approach for iterative synthetic data generation and automatic rule induction . Moreover, it enables joint training of the induced rules with pre-trained neural models via data programming \cite{maheshwari2021semi,zhang2022survey}. 
%Few-shot text classification (FSTC) is challenging, especially in tasks with a large, semantically similar and often overlapping label space \cite{zhang-etal-2022-contrastive}. Such tasks often find application in diverse domains including task oriented dialogue (intent classification), e-commerce, social networks, scientific literature etc.   \cite{yehudai2024llms, wrench}.  Moreover, these tasks are expected to have a unique or highly specialized label space, leading to limited availability of annotated data \cite{singhal-etal-2023-intendd,vulic-etal-2022-multi}. Intuitively, FSTC systems should be designed to extract as much information as possible from the limited supervision data available for learning. We propose \our, a framework that combines automatic rule induction  \cite{pryzant-etal-2022-automatic,bajpai-etal-2024-fair}, synthetic data generation, and contrastive representation learning \cite{zhang-etal-2022-contrastive} for FSTC. Moreover, \our~induces rules in the form of syntactic n-grams that complements information captured in prevalent approaches in FSTC. 

%Recently, FewMany benchmark puts together a bunch  of tasks varying from Intent detection, e-commerce products among others. In this we work, we show how such tasks can be modeled with a combination of learning techniques leading to state of the art results in several of these tasks under few-shot settings. 

%FSTC tasks are generally addressed using a diverse set of techniques. These include In-context learning \cite{NEURIPS2020_llmfew, NEURIPS2022_llmzero}, contrastive representation learning \cite{vulic-etal-2021-convfit}, data augmentation and filtering \cite{lin-etal-2023-selective}, transductive learning \cite{singhal-etal-2023-intendd}, weak supervision \cite{pryzant-etal-2022-automatic}, meta-learning \cite{mesgar-etal-2023-devil} among others. Several of these works successfully combine one or more of these techniques for FSTC tasks \cite{singhal-etal-2023-intendd,vulic-etal-2022-multi}. 


% \begin{figure*}[h] 
%    \centering\includegraphics[width=0.9\textwidth,page=4,trim={0 19.5cm 0 2cm}]{files/PresentationAutoRules.pdf}
%    \caption{Three-step workflow for \our, along with various components in it.} 
%    \label{archi}
%    \end{figure*}

%approach, that iteratively filters the generated rules and data . 
%\our's novelty lies in effectively filtering automatically induced rules and synthetically generated data iteratively via bootstrapping  




%The gold standard few shot labeled data forms the seed for the data augmentation step. However, it forms the validation data during rule induction and filtering. 




%Starting with a seed set of a k-shot labeled dataset, we induce rules and perform synthetic data generation using prompt demonstration. While the synthetically generated training data points are label-specific, they further undergo filtering using the rules induced. Similarly, once new data points are filtered they are used for further rule induction and filtering. The rule and synthetic data generation is performed iteratively.  Further, a pre-trained model is then undergoes continual pretraining and fine-tuning using contrastive learning techniques \cite{zhang-etal-2022-contrastive}. Finally, we use data programming \cite{maheshwari2021semi} to learn a generative label aggregation model by using the final set of rules of labeling functions. Finally, we perform joint learning  of the final classifier using the label aggregation model and the contrastively tuned pre-trained model.

%Using the rules, we predict a label for the generated data, and filters only those that agrees on the labelusing the induced rules.  and filter only those augmented datapoints that match
 %Three, the joint learning step, effectively combines contrastive representation learning, \cite{pmlr-v119-chen20jsimclr,NEURIPS2020_supervisedContrastive} supervised fine-tuning, and Data programming \cite{zhang2022survey} using a joint learning framework \cite{maheshwari2021semi}. We perform self-supervised contrastive pretraining \cite{wu2020clear} and supervised contrastive learning \cite{khosla2020supervised, zhang-etal-2022-contrastive} over a standard pre-trained neural classifier. We use the few-shot labeled data, along with the filtered data, for fine-tuning the neural classifier. 
 
 
 %The induced rules are used as a form of weak supervision to learn a generative modellabeling-functions in  to learn a  as a . The neural claUsing the induced rules we learn a generative model The classifier is learned jointly with a The induced rules SPEAR is a framework for Data Programming. .and  Further, we use , a framework that enables to learn both the models jointly.

In \our, we boost supervision signals in two ways. With synthetic data generation we supplement the available training data \cite{DBLP:journals/corr/abs-1711-10160ratner,pryzant-etal-2022-automatic}. First, with rule induction, we obtain complementary signals that need not be explicitly captured from the existing data \cite{maheshwari2021semi, singhal-etal-2023-intendd}. Second, our rules are induced as generalized syntactic n-grams. Here, we aim to potentially capture morpho-syntactic information from the data, a view of data that need not be explicitly captured by state-of-the-art (SotA) systems in use. A classical NLP pipeline typically represents a string at multiple levels of abstraction which includes Part-of-Speech (PoS) tags, syntactic relations, \emph{etc.} \cite{manning-etal-2014-stanford}. \our~uses higher-order dependency structures as features and generalizes over these features using inductive generalization \cite{popplestone1970experiment} to induce the rules as generalized syntactic n-grams. 
 
%Previous works used rules for boosting supervision signals in various ways. Rules are used as an auxiliary source that can add more labeled data, albeit being noisy . Alternatively, rules can be used  Our generated data brings in auxiliary sources of information. The rules we generate are used to improve efficient utilisation of the available data during the learning. 
  %Weak supervision sources such as rules are often used as auxiliary sources of information that brings in additional information not fully captured in the available training data. Similarly, it can be used for maximising utility of the available data and complement the labeledd data. \our~ achieves both 
 

 

We find applicability of both the rules and exemplars from \our, with consistent performance gains in various text classification setups. Specifically, we experiment with ICL and fine-tuning setups. In ICL, we focus on long-context ICL \cite{li2024long,bertsch2024context} and use the generated data as a pool from which exemplars are retrieved. Further, we incorporate our rules as explanations to the input and the exemplars. Similarly, we use the data for fine-tuning models, which include pre-trained LLMs, Qwen \cite{Qwen2,Qwen2.5} and RoBERTa \cite{liu2019roberta}. %The induced rules enable learning a generative label aggregation model as a form of weak supervision using data programming. We jointly learn a classifier with the generative model using SPEAR \cite{maheshwari2021semi}, a data programming framework.

We perform extensive experiments on multiple text classification datasets, which include three full-shot, and eight few-shot datasets from the {\sc FewMany} benchmark \cite{yehudai2024llms}. Further, we perform multilingual experiments on seven languages using the {\sc MASSIVE} \cite{fitzgerald2022massive1mexamplemultilingualnatural} dataset. 
%Our experiments are performed using both 5-shot and 10-shot settings. In all these settings, \our~outperforms strong competitive models, such as IntenDD \cite{singhal-etal-2023-intendd}, \citet{zhang-etal-2022-contrastive}, and  FastFit \cite{yehudai2024llms},  with statistically significant improvements. 

%In section \ref{sec:genSpace}, we elaborate on our rule induction approach for inducing generalized syntactic n-grams. In section \ref{sec:method}, we elaborate \our, a 3-step framework for FSTC. Here, we elaborate our iterative rule and data filtering along with the joint learning setup. 


Our major contributions are as follows:




\begin{itemize}
    \item Use of rules and data from \our~results in statistically significant gains in all the experimental setups, as compared to the corresponding configuration without resources from \our. Specifically, we obtain state of the art (SotA) results in our full-shot and few-shot experiments when using \our.

    \item The rules we generate are shown to be effective, both during ICL and fine-tuning. Further, using the rules as explanations under ICL for CDR dataset results in SotA results. Similarly, fine-tuning Qwen jointly with data and the augmented rules from \our~has shown statistically significant improvements for Qwen and RoBERTa based models.

    \item Use of augmented data for few-shot setups in the {\sc FewMany} benchmark demonstrate the quality of the augmented data we produce. We show that simply using additional data from \our, as low as 20-shot additional data per class, can result in improved performance than incorporating complex approaches such as contrastive representation learning into the training process. 
    
    %Our proposed approach yields statistically significant gains in all the experiments we perform, compared to state-of-the-art systems \cite{yehudai2024llms, singhal-etal-2023-intendd, zhang-etal-2022-contrastive}. Our best performing model reports a 2.04\% increase in 10-shot and 2.52\% increase in 5-shot settings, compared to the next best model, averaged across all the monolingual tasks.

    \item Our extensive experiments show that \our~is generalizable across multiple domains and multiple languages. We report a 7.21\%  increase in performance, compared to the model without any resources from \our, averaged across seven different languages.    

    \item We show that leveraging syntactic information as weak supervision for rule induction, brings a complementary source of supervision, which otherwise need not be captured by using string level data directly (\S \ref{ruleImpact}).
    
    %performance improvements compared to surface-level string n-grams as rules. Further, our bootstrapped approach outperforms competitive approaches for filtering augmented data \cite{lin-etal-2023-selective}. 

    

\end{itemize}


%show that we can complement these  models with relevant information by leveraging a combination of approaches that include contrastive learning, data augmentation, weak supervision, data subset selection and automatic rule induction for learning models in few shot settings. Further, by using automatic rule induction we learn rules that use syntactic information. Moreover, we show that these information can help improve the performance of these pre-trained deep-neural models. %are there is information which can be utilized?




%Our framework employs the following learning approaches for a general few shot text classification approach. Similar to several other few shot text classification approaches popular in intent classification, we perform contrastive learning for better representation learning. Here, we first perform self-supervised pretraining followed by supervised contrastive learning. Two, we propose an automated approach for inducing `rules' in the form of generalised syntactic n-grams. Three, we synthesise new labeled samples via data augmentation. Here, we use the few shot labeled samples as input to an LLM for data augmentation. Three, we filter the augmented sentences using a data subset selection approach. Finally, we use data programming by using the automatically induced rules as labeling functions to learn a generative model. We then learn an intent classifier using our contrastively trained model jointly learning with the generative model built.rules' from a corpus by generalizing over a feature space that consists of .\our~focuses on  over the concept space of these abstractions in the pipelineIt 


%Further, we use least general generalization \cite[LGG][]{plotkin1971further, plotkinnote} to obtain the generalised syntactic n-grams as our rules. T 


%Moreover, these rules can be integrated to pre-trained models during the training via programmatic weak supervision (PWS). Specifically, a feature space over higher-order dependency structures from dependency parses of the inputs, akin to syntactic n-grams, is defined. The features are enumerated and scored using  labeled data. The entire corpus, consisting of both labeled and unlabeled data, is partitioned using these features. For each such partition, we obtain a generalization of the features which form the rule. T. The classifier is trained jointly with a label aggregation model in a semi-supervised setting.   

% Now, given that there are several approaches by which supervision can be incorporated with the limited data. Now, in-context learning or parameter updation approaches, such as fine-tuning, have been two primary categories to address such  tasks. So far, . However, it is still entirely not clear whether these systems capture all the relevant information for the task. A more interesting question would be, is it possible to capture information  that can be complementary to what these systems learn and use it to improve the performance of these systems? \our~ proposes a general-purpose rule induction framework for semi-supervised text classification. 


 %In \our, we use a combination of the entries at various levels of this abstraction as the generalization. 
%For instance, the two phrases `brown foxes' and `brown cats' are different at the string level. Further, their corresponding dependency representations, which we use as features, are also distinct; `brown $\xleftarrow{amod}$ foxes' and `brown $\xleftarrow{amod}$  cats'. However, they both can be `generalized' into a single structure, written as `brown $\xleftarrow{amod}$ {\sc Noun}', where $amod$ is the dependency relation between both the words and {\sc Noun} corresponds to the POS tag for common nouns. The generalized structure forms a rule in \our.


%The structure, `brown $\xleftarrow{amod}$ {\sc Noun}' is not just a generalization of the aforementioned pair of strings, but it is representative of any string containing a common noun with brown as its adjectival modifier. Similarly, we may consider  `{\sc ADJ} $\xleftarrow{amod}$ {\sc Noun}' also as a generalization of the aforementioned pair of strings, which is more general than the former and it covers a wider set of strings, i.e. any pair of common nouns with exactly one adjectival modifier. Here,  we need to identify generalizations that are of utility for a classification task as compared to the possible over-generalizations. Inductive Logic Programming (ILP) has been extensively used in the past in identifying generalizations over such concepts. Specifically, we employ least general generalization \cite[LGG;][]{plotkinnote,plotkin1971further} over the partition of the dataset encoded in the feature space. 




%We observe consistent increase in performance for each of the datasets


%LGG is performed over a partition of the input dataset, and the quality of the rules obtained is highly dependent on the quality of the partition. \our~employs a modularity-based community detection approach, Louvain, to partition the dataset where similar inputs in the 





%However, we need to identify 


%Systems such as GOLEM \cite{muggleton1992efficient}, {\sc Chillin} \cite{zelle1995inducing} are ILP approaches that use variants of least general generalization. 

%It integrates diverse areas in NLP in its framework to enable rule induction. Higher-order dependency features are used for feature space construction, 



%incorporate these rules as weak labelers or rather as labeling functions using PWS enables the integration of various sources of weak supervision, albeit noisy, in the form of programs that enables ameliorating the need for hard data labeling. Specifically, we use the semi-supervised data programming framework by Spear where the loss function integrates predictions from the rules and from the neural classifier. 

%While programmatic weak supervision enables to integrate arbitrary rules to be integrated into a learning system, those typically are expected to be heuristics hand-crafted by subject matter experts or are obtained from existing resources. Hence any arbitrary Python function that returns a label can be used as a labeling function and no further constraints exist for the type of function. However, since we automate the rule generation, the rule generation needs to follow a formalism or rather the rules generated have to stick to a constrained formalism. In our case, we experiment with two different approaches. One generates a restricted form of horn-clauses using path-constrained random walks, and two, uses least general generalization a form of abstracting out a generalized representation in the same scheme as the input. 



%We show that our approach can outperform previous state-of-the-art approaches in similar settings and further, we can integrate our approach with an LLM of 7 billion parameters which can further improve our performance. Here, we use adapters to improve our models which enables parameter-efficient training of these moddels. 



% \noindent {\textbf{Neural Fields.}} 
% 3d shape (\cite{chen2019learning}, \cite{park2019deepsdf}, \cite{mescheder2019occupancy})
% 3D scene reconstruction (NeRF\cite{mildenhall2021nerf}, ; \cite{niemeyer2021giraffe}),

% Neural Fields (NeFs) map coordinates to signals, providing a compact and flexible continuous data representation~\citep{sitzmann2020implicit, tancik2020fourier}. They are widely used for 3D object and scene modeling~\citep{chen2019learning, park2019deepsdf, mescheder2019occupancy, genova2020local, niemeyer2021giraffe}. NeRF~\citep{mildenhall2021nerf} learns neural radiance fields for view synthesis, mapping spatial coordinates to colors and densities via differentiable volumetric rendering. Extensions include Mip-NeRF~\citep{barron2021mip} for multiscale representations, TensoRF~\citep{chen2022tensorf} for low-rank tensor factorization, and NeuRBF~\citep{chen2023neurbf} for radial basis function aggregation. Unlike these methods, which rely on pre-defined structured information, we infer geometric bases to encode spatial structure.



%RBF-based works (\cite{ramasinghe2021learning}, \cite{ramasinghe2022beyond}, NeuRBF~\cite{chen2023neurbf}, 


\noindent {\textbf{Neural Fields (NeFs) and Generalization.}} Neural Fields (NeFs) map coordinates to signals, providing a compact and flexible continuous data representation~\citep{sitzmann2020implicit, tancik2020fourier}. They are widely used for 3D object and scene modeling~\citep{chen2019learning, park2019deepsdf, mescheder2019occupancy, genova2020local, niemeyer2021giraffe}. However, how to generalize to new scenes without retraining remains a problem. 
Many previous methods attempt to use meta-learning to achieve NeF generalization. Specifically, gradient-based meta-learning algorithms such as Model-Agnostic Meta Learning (MAML)~\citep{finn2017model} and Reptile~\citep{nichol2018first} have been used to adapt NeFs to unseen data samples in a few gradient steps~\citep{lee2021meta, sitzmann2020metasdf, tancik2021learned}. Another line of work uses HyperNet~\citep{Ha2016HyperNetworks} to predict modulation vectors for each data instance, scaling and shifting the activations in all layers of the shared MLP~\citep{mehta2021modulated, dupont2022data, dupont2022coin++}. Some methods use HyperNet to predict the weight matrix of NeF functions~\citep{dupont2021generative, zhang20233dshape2vecset}. Transformers~\citep{vaswani2017attention} have also been used as hypernetworks to predict column vectors in the weight matrix of MLP layers~\citep{chen2022transformers, dupont2022coin++}. In addition, \cite{reizenstein2021common,wang2022attention} use transformers specifically for NeRF. Such methods are deterministic and do not consider the uncertainty of a scene when only partially observed. Other approaches model NeRF from a probabilistic perspective~\citep{kosiorek2021nerf, hoffman2023probnerf, dupont2021generative, moreno2023laser,erkocc2023hyperdiffusion}. For instance, NeRF-VAE~\citep{kosiorek2021nerf} learns a distribution over radiance fields using latent scene representations based on VAE~\citep{kingma2013auto} with amortized inference. Normalizing flow~\citep{winkler2019learning} has also been used with variational inference to quantify uncertainty in NeRF representations~\citep{shen2022conditional, wei2023fg}. However, these methods do not consider potential structural information, such as the geometric characteristics of signals, which our approach explicitly models.



\noindent {\textbf{Neural Processes.}} Neural Processes (NPs)~\citep{garnelo2018neural} is a meta-learning framework that characterizes distributions over functions, enabling probabilistic inference, rapid adaptation to novel observations, and the capability to estimate uncertainties. This framework is divided into two classes of research. The first one concentrates on the marginal distribution of latent variables~\citep{garnelo2018neural}, whereas the second targets the conditional distributions of functions given a set of observations~\citep{garnelo2018conditional, gordon2019convolutional}. Typically, MLP is employed in Neural Processes methods. To improve this, Attentive Neural Processes (ANP)~\citep{kim2019attentive} integrate the attention mechanism to improve the representation of individual context points. Similarly, Transformer Neural Processes (TNP)~\citep{nguyen2022transformer} view each context point as a token and utilize transformer architecture to effectively approximate functions.
Additionally, the Versatile Neural Process (VNP)~\citep{guo2023versatile} employs attentive neural processes for neural field generalization but does not consider the information misalignment between the 2D context set and the 3D target points. The hierarchical structure in VNP is more sequential than global-to-local. Conversely, PONP~\citep{gu2023generalizable} is agnostic to neural-field specifics and concentrates on the neural process perspective. In this work, we consider a hierarchical neural process to model the structure information of the scene. 



\section{Method}
\label{sec:method}

\subsection{Weaknesses of Previous Conditioning Methods}

The most popular form of latent image conditioning typically converts conditioning signals to images, before processing them with typical image processing models. While this approach is powerful, it exhibits limitations in handling complex image synthesis tasks, particularly when incorporating heterogeneous or sparse input conditions. Some approaches, such as \textit{LayoutDiffusion} \cite{zheng_layoutdiffusion_2024}, tackle this with custom attention modules that attend to bounding boxes with learned positional embeddings. However, these approaches neglect to include multiple modalities and the relationships between them, which overlooks nuanced interactions between conditioning signals i.e. disambiguating spatial ordering between overlapping boxes. 

% For example, interactions between conditions which may not explicitly exist in the discrete spatial image domain.

% These approaches force diverse modalities, like mixed spatial and categorical information directly into a unified image space, which overlooks nuanced interactions between conditioning signals. For example, interactions between conditions which may not explicitly exist in the discrete spatial image domain.

Previous conditional diffusion research that utilise graph data opt for complex multi-stage training procedures such as masked contrastive pre-training using graph triplets \cite{yang_diffusion-based_2022}. This is not only time-consuming, but also fails to exploit potential benefits of training an end-to-end system that integrates graph data directly into image processing. 
% Furthermore, other work has shown that the repeated conditioning diffusion models (i.e. time or text conditioning) is superior to simply providing   

We tackle these problems by representing images and their conditioning signals as a single graph, which is processed by a bespoke GNN architecture. This allows repeated interactions between conditioning signals and the image throughout the synthesis process, enabling more flexible and dynamic representations that account for both the current image features and interactions between conditioning signals. By maintaining separate pathways for distinct input types, our approach supports heterogeneous and sparse conditioning, leading to better generalisation, finer control, and more precise manipulation of generated images. This simple yet powerful method can be easily integrated into a wide range of existing vision models.

\begin{figure}
    \centering    \includegraphics[width=1\linewidth]{icml2023/hig_fig2.pdf}
\vspace{-20pt}
    \caption{(\textbf{a}) Overview of the proposed architecture. The HIG is encoded into a latent representation through a MP-GNN which is then used as a condition $c_f$ in a ControlNet. (\textbf{b}) Details of the MP-GNN module. Note: HMP is shorthand for heterogenous magnitude preserving operations applied across all nodes.}
    \label{fig:architecture}
\end{figure}

\subsection{Heterogeneous Image Graphs}

To improve on previous approaches we develop a new approach to condition images via the HIG representation. In this manner, we fully exploit variable-length and heterogeneous conditions to aid in image synthesis.

\textbf{Image Graphs.} When faced with the challenge of conditioning images with graphs we first convert images into representations amenable for graph processing. We reshape image features into image nodes pixel-wise in line with other works \cite{liu_cnn-enhanced_2021, han_vision_2022}. In practice, these nodes represent more than a single pixel, for example a latent image patch. This can be due to performing latent image diffusion \cite{rombach_high-resolution_2022, podell_sdxl_2023} where images are first pre-compressed to latent images, or due to prior processing by the image processing model. In contrast to other works \cite{tian_image_nodate, han_vision_2022, tarasiewicz_graph_2021}, we decide to leave image nodes unconnected; this loosely decouples image conditioning from processing. Image nodes are conditioned and later converted back into an image representation, allowing existing architectures to handle processing. Connecting image nodes in a locally dense fashion gains little benefit over highly optimised $3 \times 3$ convolutional operations. Formally, image nodes exist in a discrete space \( f : \mathbb{Z}^2 \to \mathbb{R}^C \). For an image of size \(M \times N\), we define \( f(i, j) \) where \( i, j \in \mathbb{Z} \) and \( 0 \leq i < M \), \( 0 \leq j < N \).

\textbf{Conditioning Graphs}. Conditioning graphs consist of nodes and edges, where each node has features defined as $ g : \mathcal{V} \to \mathbb{R}^F$, where $\mathcal{V}$ represents the set of nodes and $\mathbb{R}^F$ the feature space. Nodes may have spatial ties to the image domain, which we materialise via edges linking image and conditioning nodes. We use conditioning nodes to indicate semantics within the scene, for instance, a node may represent an object (e.g., a \textit{person}). Whereas we utilise different edge types to represent both spatial, abstract relationships and additional semantics. For instance, an edge between two object nodes may encode interactions or attributes (e.g., a person \textit{wearing} a {\textit{yellow}} hat). The graph structure reflects real-world data: often sparse and heterogeneous. We therefore construct graphs on a per task-basis to best leverage the available data and its dependencies.
Formally, each edge \( e \in \mathcal{E} \) connects two nodes \( (v_i, v_j) \in \mathcal{V} \times \mathcal{V} \) and represents a relationship between them. Edges represent any dependency, allowing for abstract relationships to be included.

% To continue the example, if spatial information for both the \textit{person} and the \textit{hat} is available, the graph would contain a node for each object and an edge connecting them, with the edge encoding the relationship \textit{wearing}. 


% \textbf{Conditioning Graphs.} In contrast, conditioning graphs are represented by sets of nodes and edges, with each node having associated features defined by a function $( g : \mathcal{V} \to \mathbb{R}^F$, where $\mathcal{V}$ represents the set of nodes and $\mathbb{R}^F$ the feature space. Although nodes \textit{may} have explicit spatial ties to the discrete image domain, we materialise these through edges between image and conditioning nodes. However, these relationships may be the product of spatial properties of conditioning nodes. As such, subsets of $\mathbb{R^F}$ may represent spatial coordinates \( (x, y) \in \mathbb{R}^2 \) that satisfy \( 0 \leq x < M \) and \( 0 \leq y < N \). Conditioning nodes are not restricted to pixel grid positions, nor the number of spatial dimensions e.g. nodes may represent 3D properties of the real world. Nodes and edges may represent properties independent of spatial dimensions. For example, nodes in the graph can represent concrete objects in the image (e.g., a \textit{person}), while edges between them may represent abstract interactions or attributes (e.g., a person \textit{wearing} a {\textit{yellow}} hat). The graph structure may be sparse, and heterogeneous (multiple types of nodes and edges). Conditioning graphs are constructed on a per-task basis to optimally leverage available data and its dependencies. Formally, each edge \( e \in \mathcal{E} \) connects two nodes \( (v_i, v_j) \in \mathcal{V} \times \mathcal{V} \) and represents a relationship between them. To continue the example, if spatial information for both the \textit{person} and the \textit{hat} is available, the graph would contain a node for each object and an edge connecting them, with the edge encoding the relationship \textit{wearing}. Edges can represent any dependency, allowing for abstract relationships to be included in the graph.

\textbf{Connecting Image and Conditioning Nodes.} With image and conditioning nodes defined, we are close to the complete HIG representation. To enable conditioning between the image and conditioning graphs, we must construct edges between the two. These connections are determined on a per-task basis, depending on the available data, with explicit choices described in Section 4. However, when spatial information is available i.e. segmentation masks or bounding boxes, it enables direct connections between the image graph and the conditioning graph. Specifically, edges are created between image nodes relevant to spatial conditionings (i.e. pixels within the bounding box) and conditioning nodes representing the corresponding semantic class (i.e. class label). This linkage facilitates information flow across the graphs, integrating pixel-level details with higher-level semantic representations. 

% Additionally, the flexibility of heterogeneous GNNs allows for connections from the image back to the graph with different sets of learned weights. This approach enables the image to influence the graph structure while leveraging the rich semantic details present in the image—such as color or object sub-class—throughout much of the diffusion training scheme, while still respecting the different types of information carried by the node types.

\subsection{Model Architecture}

To be compatible with the EDM2 U-Net architecture \footnote{\href{https://github.com/NVlabs/edm2}{https://github.com/NVlabs/edm2}}, we propose the addition of a magnitude-preserving \textit{Heterogenous Image Graph Neural Network} (HIGnn) as the conditioning network to be used in a ControlNet strategy.

\textbf{HIGnn.} The general architecture of the HIG conditioning block requires two primary capabilities: representation switching and HIG processing. To handle switching between image features and image nodes on the HIG we consider the update function $\mathcal{U}_{\text{i}\rightarrow\text{g}}$. This update functions reshapes image features $\mathbf{x_i} \in \mathbb{R}^{N \times C \times H \times W}$ into image nodes pixel wise $\mathbf{x_g} \in \mathbb{R}^{N\cdot H \cdot W \times C}$ and applies an optional projection to ensure correct dimensionality. For the current set of image pixels $\mathbf{x_i}$, we retrieve HIG image nodes $\mathbf{x_g}$ by
\begin{equation}
\mathbf{x_g} = \mathcal{U}_{\text{i}\rightarrow\text{g}}(\mathbf{x_i}) = \hat{W}R(\mathbf{x_i}),  
 \label{eq:HIG_update}
\end{equation}
where $R$ reshapes the image, and $\hat{W}$ is a learned projection with forced magnitude preservation from \cite{karras_analyzing_2024}. Refer to Appendix \ref{appendix:edm2_preliminaries} for greater detail into the mathematical preliminaries of \cite{karras_analyzing_2024}. We consider the reverse operation of converting from graph nodes to an image $\mathcal{U}_{\text{g}\rightarrow\text{i}}$ in a similiar fashion. 

Once we have the HIG updated with current image nodes we can process it with a GNN. We identify several areas where magnitudes can grow and address them each in turn. In practice many varieties of heterogenous message passing GNN could be used, we create our own magnitude preserving graph convolutional operator similiar to Hamilton et al. \cite{hamilton_inductive_2018} for its simplicity and stability. The basic approach propagates information through two branches, a pseudo `skip-connection' applied to the current node, and a learned pooling operation of the local neighbourhood, and we add the ability to include edge information in the neighbourhood pooling. If edge attributes $\mathbf{a}_i$ are present we integrate them via magnitude preserving concatenation to the pooling branch. Formally, the HIGConv operator applied per meta-path to get updated node embeddings $\mathbf{x}_i'$ is defined as:
% \begin{equation}
%     \mathbf{x}_i' = \psi\left(\hat{W}^{\Phi}_1 \mathbf{x}_i +^\text{mp} \hat{W}^{\Phi}_2 \cdot \frac{1}{\sqrt{|\mathcal{N}^{\Phi}|}} \sum_{j \in \mathcal{N}^{\Phi}(i)} [\mathbf{x}_j \|^\text{mp} \mathbf{a}_j] \right),
%     \label{eq:hignn_operator}
% \end{equation}
\begin{equation}
    \mathbf{x}_g' = \psi\left(\hat{W}^{\Phi}_1 \mathbf{x}_g 
    \underset{0 \text{ if } |\mathcal{N}^{\Phi}(i)| = 0}{\underbrace{+^\text{mp} \hat{W}^{\Phi}_2 \cdot \frac{1}{\sqrt{|\mathcal{N}^{\Phi}(i)|}} \sum_{j \in \mathcal{N}^{\Phi}(i)} [\mathbf{x}_j \|^\text{mp} \mathbf{a}_j]}}\right)    \label{eq:hignn_operator}
\end{equation}

% \[
%     \mathbf{x}_i' = \psi\left(\hat{W}^{\Phi}_1 \mathbf{x}_i +^\text{mp} 
%     \underset{+ 0 \text{ if } |\mathcal{N}^{\Phi}(i)| = 0}{\underbrace{\hat{W}^{\Phi}_2 \cdot \frac{1}{\sqrt{|\mathcal{N}^{\Phi}(i)|}} \sum_{j \in \mathcal{N}^{\Phi}(i)} [\mathbf{x}_j \|^\text{mp} \mathbf{a}_j]}}\right).
% \]

where we choose $\psi$ to be magnitude preserving SiLU operator, and $+^\text{mp}$ the magnitude preserving sum (See Appendix \ref{appendix:edm2_preliminaries}), and both meta-path weights $\hat{W}^{\Phi}_1$ and $\hat{W}^{\Phi}_2$ have forced magnitude. $\mathcal{N}$ indicates the local node neighbourhood and is defined by the connectivity of graph. In order to achieve magnitude preservation we first assume all neighbourhood features to be of unit length, we then summate them scale them by the square root of the neighbourhood size ($\sqrt{|\mathcal{N}^{\Phi}|}$), see Appendix \ref{appendix:sum_random} for details. It is important to address unconnected or `zero-degree' nodes, in this case we ignore the right hand side of the equation, and only take the residual path. Note that simply setting the  neighbourhood to zero unintentionally changes the feature magnitudes when mp-sum is applied, since it assumes both vectors to be of unit length. Finally to combine information across meta-paths, we use the same method and sum across paths before normalising by the inverse square root of the number of incoming meta-paths ($|\Phi_i| = |\{\Phi_k \mid x_i \in \Phi_k\}|$)

% To formulate a heterogeneous GNN with learned projections per meta-path ($\mathbf{\Phi} = \{\Phi_1 ... \Phi_n\}$), we must preserve magnitudes when combining meta-paths.

\begin{equation}
\Tilde{\mathbf{x}}_g = \frac{1}{\sqrt{|\Phi_g|}} \sum_{\Phi \in \Phi_g} \mathbf{x}'_g,
\label{eq:meta_path}
\end{equation}

We verify that this approach is guaranteed to maintain magnitudes under certain conditions of the underlying graph data. In particular, for graph-data of sufficient size this approach holds for graphs which do not have identical features attached to the same node since this breaks the independence assumption. 

% An interesting interpretation of this formulation with respect to image synthesis is to observe how different receptive fields change. The typical convolutional operator used in U-Net models define a local image receptive field $\mathcal{R}$, self-attention  defines a global image receptive field $\mathcal{A}$, and the HIGnn defines receptive fields over meta-path relationships $\mathcal{N}^{\Phi}$ for both the image and conditioning variables. We postulate this to an advantage over other conditioning methods as it allows instant communication between different conditioning signals and parts of the image whilst remaining computationally tractable.

\textbf{EDM2 ControlNet Integration.} To integrate conditioning into a generative model, we adopt a strategy similar to ControlNet \cite{zhang_adding_2023}, i.e. a frozen EDM2 pre-trained model, with a trainable copy the encoder integrated with the conditioning HIGnn. Refer to Figure \ref{fig:architecture} for an overview of our proposed architecture, we employ 4 HIG blocks for our base model. The EDM2 checkpoints are only available for class-conditional generation of the 1000 ImageNet classes, yet we find them easy to adapt to our natural image datasets.  To facilitate this we unfreeze the embedding network. To integrate features we adopt $1\times1$ convolutions with a learnable zero-gain in a similar fashion to the original ControlNet, but we note that traditional summation may damage feature magnitudes. We find that naively integrating is harmful to training. Instead, we apply magnitude preserving summation, which, in contrast to the original ControlNet paper, directly alters the primary network features. This yields poor generative quality at step 0, but proves to be quick to train and to be best in practice.

In the trainable encoder we integrate our proposed HIGnn after the initial convolution block. We opt to keep the dimension of the GNN matched to that of the generative model. Finally, to generate samples we opt for the non-stochastic EDM2 sampler, and use the recent advancements in auto-guidance \cite{karras_guiding_2024}, we use our control model as the primary network, and use the unconditional XS ImageNet checkpoint released with EDM2 as the guidance network \cite{karras_analyzing_2024, karras_guiding_2024}. 

% We do not use EMA
\section{Dataset Generation}
\label{sec:dataset}
\revise{
To train the proposed GNN, we constructed a dataset of building structures and a subset of these structures were subjected to fire simulations using FEA. The dataset generation process is illustrated in \figref{fig:dataset_generation_procedure}. Initially, a total of 33,000 building structures with geometrical details, material properties, and gravity loads were created. Due to randomness in generating these structures, a filter is applied to remove unreasonable data after gravity load simulation, which included 15,377 structures. A trade-off between computational feasibility and model performance is made among the remaining 17,623 structures. As further labeling structures with MIDR requires resource-intensive fire simulations via OpenSeesRT, a large proportion of 16,050 structures is selected as unlabeled dataset. On the other hand, each of the other 1,573 structures was further subjected to 30 different fire simulations, forming the labeled dataset containing $1,573\times 30 = 47,190$ fire cases.} This section details the step-by-step process for generating the dataset, including geometry creation, material property assignment, and simulations due to gravity loads and fire scenarios. 
% To train the proposed neural network, we constructed a dataset comprising building structure data and a subset of fire scenario data. The dataset generation process is illustrated in \figref{fig:dataset_generation_procedure}. 
% A total of 33,000 building structures with geometric details, material properties, and gravity loads were initially created. Out of these, 3,000 structures were selected as labeled data, and the remaining 30,000 were designated as unlabeled data. Further, about half of them filtered out due to instability under gravity loads only. 
\begin{figure*}[h!]
    \centering
    \includegraphics[width=0.8\linewidth]{figures/dataset_filter_procedure.pdf}
    \caption{Workflow for dataset generation (geometry, material property, gravity loads, and fire scenarios).}
    \label{fig:dataset_generation_procedure}
\end{figure*}

\subsection{Geometry Generation}
\label{subsec:geometry_generation}
The geometry of the building structures forms the foundation of the dataset. Regular 
\revise{3D structures} resembling multi-story parking structures or shopping malls were generated, with parameters such as building floor dimensions and story heights selected randomly. Each building structure is composed of multiple rooms, which serve as the basic unit in this study. A room herein is a cuboid space defined by specific length, width, and height. Within a structure, rooms of the same dimensions are uniformly arranged along the length, width, and height, corresponding to the $x$-, $y$-, and $z$-axes, respectively. Structures vary in room size and number of rooms along each axis. Specifically, the room length, width, and height are independently sampled from a uniform distribution within the interval $[2, 5]$ meters along the three directions of the structure. Similarly, the room number along each axis is uniformly sampled independently as an integer within the interval $[2, 7]$, i.e., the maximum number of stories of the buildings simulated in this study is 7.

To introduce variability and simulate real-world scenarios, approximately $8\%$ of structural elements (beams or columns) are randomly removed after initial geometry creation. 
\revise{Such removal is not fire-induced damage, but reflects functional diversity often observed in real buildings, such as open spaces designed for activities in shopping malls, e.g., ice skating rinks. Examples of the generated geometries are illustrated in \figref{fig:example_generated_geometry}, showcasing the diversity and realism of the dataset. This element removal does not affect the definition of room's geometry in the structure and nor does it affect the number of considered fire scenarios.} 

\revise{A range of coefficient of variation values ($3.3\%$ to $17.5\%$) was derived from prior studies that investigated the statistics of geometrical and material properties of structural components of buildings (e.g., \cite{mirza1979variations, lee2004probabilistic}). These studies provide empirical data on the natural variability in parameters such as Young's modulus, yield strength, and dimensions of structural elements due to manufacturing tolerances and material inconsistencies. By selecting $8\%$ for the removal of structural elements in our database, we aimed to maintain a level of variability that is representative of real-world uncertainties while ensuring computational feasibility. This choice ensures that the database captures realistic deviations without introducing extreme cases that may not be commonly encountered in practice.}

\begin{figure*}[h!]
    \centering
    \includegraphics[width=\linewidth]{figures/example_generated_geometry.pdf}
    \caption{Examples of generated structural geometry of different sizes (all dimensions in meters).}
    \label{fig:example_generated_geometry} 
\end{figure*}

{\blockRevise

In this study, we opted for a deterministic square, dimension of $0.1$ m, solid cross-sectional steel elements due to their simplicity in modeling and analysis. Square sections exhibit uniform geometrical properties in all directions, simplifying the computation of structural responses and avoiding complications associated with more complex shapes, such as wide-flange sections, facilitating the computational efficiency and scalability to generate a large dataset. This choice also helps to mitigate issues related to stress concentrations and facilitates a more straightforward representation of structural behavior under thermal loads. 

\textit{Remark:} The selected cross-section provides a comparable flexural rigidity to a $W 130 \times 130 \times 28.1$ wide-flange section (metric units), albeit with significantly higher axial rigidity. This cross-section is acceptable for gravity-load-designed frames under service loading conditions where the models assume fully rigid, moment-resisting beam-column connections for the evaluation of the IDR under thermal loading. This assumption is reasonable in this computational study where the primary interest is to understand the global deformation response of frames under fire conditions. The selection of uniform square cross-sections for both beams and columns, rather than adherence to standard capacity design principles, was made here primarily for computational efficiency and to reduce design parameters in the database generation process. This choice allows for simplified and scalable approach to analyze the fire-induced response of generic steel frames without the need for large section variations, where this study mainly focuses on the fire vulnerability assessment using ML-based predictions. However, if additional loading conditions, e.g., seismic or wind loads, were to be considered, larger sections, strong-column/weak-beam principle, and ductile detailing would be required in the generated buildings for realistic structural behavior under combined loading conditions. Future studies may also consider investigating the influence of variable cross-sectional dimensions and semi-rigid connections on the structural performance under fire conditions. 
} % blockRevise

\subsection{Material Properties}
Steel is chosen as the material for the structures. To reflect real-world variations, we randomly assign one of five slightly different steel material types to each structural element. \revise{
The ranges of material properties are provided in \tabref{tab:material_property_ranges} and the properties are sampled from uniform distributions of the corresponding ranges. These variations simulate differences arising from manufacturing batches or regional material properties. That these properties are at ambient temperature and change when the temperature rises due to a fire. The selection of materials with varying properties is aimed at increasing the diversity of the data. Our goal is to represent as wide a range of data as possible with a limited amount of building structure data, thereby enhancing the generalization ability of the GNN. Our assumed material property ranges are expected to be wider than the real-world conditions based on findings in \cite{mirza1979variations, lee2004probabilistic}. Therefore, we are essentially tackling a more challenging and general task. If we can solve this problem, we are confident that our method will perform equally well or even better in real-world scenarios.
}
\begin{table}[h!]
    \centering
    \caption{Material properties ranges for considered steel structures.}
    \begin{tabular}{lc}
        \toprule
        Property & Range \\
        \midrule
        Young's modulus & [168, 252] GPa \\
        Yield strength & [220, 330] MPa \\
        Strain-hardening ratio & [0.8, 1.2] \% \\
        \bottomrule
    \end{tabular}
    \label{tab:material_property_ranges}
\end{table}

\subsection{Gravity Loads}
Gravity loads are applied to columns and beams based on their \revise{influence (tributary) areas as typically conducted in structural analysis. The considered ``service'' load conditions include the column self-weight and the additional loads directly supported on the beams from their self-weight and weights of the reinforced concrete slabs, people as live load, and building content. An edge beam typically carries approximately half the gravity load supported by a parallel interior beam}. The ranges of gravity loads are listed in \tabref{tab:gravity_load_ranges}. \revise{The loads are sampled from uniform distributions of the corresponding ranges.} Structures that failed to meet an MIDR threshold of $1\%$ under gravity loads were deemed unacceptable designs and filtered out, as such configurations of randomly chosen geometry, material, and gravity load combinations were considered unrealistic from a regulatory and practicality points of view.
\begin{table}[h!]
    \centering
    \caption{Gravity load ranges for considered beams and columns.}
    \begin{tabular}{lc}
        \toprule
        Element & Range (kN/m)  \\
        \midrule
        Column & [0.5, 1.0]  \\
        Edge beam & [1.5, 4.5]  \\
        Interior beam & [3.0, 7.5]  \\
        \bottomrule
    \end{tabular}
    \label{tab:gravity_load_ranges}
\end{table} 

\subsection{Rule-based Thermal Load Generation}
\label{subsec:thermal_load_generation}
To evaluate a building's structural response during a fire event, we employed a simplified rule-based approach for thermal load generation. 
% Previous studies \cite{nan_structuralfire_2023} have demonstrated that steel structures rapidly equilibrate with surrounding gases temperatures due to efficient heat exchange. Consequently, gas temperatures can be directly used as inputs for FEA tools, e.g., OpenSees, simplifying the process of modeling thermal loads. 
% Accurately simulating temperature fields in fire scenarios poses significant challenges. Advanced thermodynamic simulations, such as those performed using Fire Dynamics Simulator (FDS) \cite{mcgrattan_fire_2000}, provide precise temperature predictions. However, these methods are hindered by high computational costs, prolonging execution times, and limited scalability, making them impractical for generating large datasets. Additionally, real-world fire loads often display substantial spatial variability across different rooms \cite{dundar_fire_2023}, resulting in scenario-specific temperature fields with limited generalizability. For example, studies on bridge fires \cite{he_study_2024} have demonstrated that environmental factors, such as wind speeds, can significantly influence temperature distributions. Furthermore, even within identical scenarios, variations in fire modeling methodologies can produce distinctly different temperature fields \cite{zhang_temperature_2020, du_new_2012}. These challenges emphasize the need for efficient and adaptable methods to generate fire temperature data.
% To address these issues, we adopted a rule-based approach to model temperature variations. 
According to \cite{spearpoint_fire_2008}, a typical fire development follows a predictable pattern. During the {\em{growth stage}}, the temperature rises slowly and approximately linearly after ignition. This is followed by the {\em{flashover stage}}, where temperatures increase rapidly to peak values. After reaching the peak, the temperature either stabilizes or continues to rise slowly until the {\em{decay stage}} begins. Inspired by this fire development pattern, we describe the temperature evolution in time, $t$, prior to the decay stage in two distinct stages:
\begin{enumerate}
    \item {\bf{Initial linear increase stage}}: For $t \in [0, t_1)$, temperature increases gradually and linearly as the fire spreads through the building. This stage represents the time before the fire directly affects a structural element.  
    \item {\bf{ISO 834 fire curve stage}}: For $t \in [t_1, t_{\thre}]$, temperature rises rapidly following the ISO 834 curve \cite{ISO834}, modeling the direct impact of the fire on the structural element. 
\end{enumerate}
The slope of the linear temperature increase, $c$, and the transition time, $t_1$, are influenced by the spatial relationship between the fire source and the structural element. For the second stage of temperature evolution, we utilize the ISO 834 curve, a widely accepted standard for fire resistance testing. This standardized fire curve describes the temperature rise over time, enabling rapid and consistent thermal fields across various scenarios. The duration of fire simulation in this study is set to $t_{\thre}=60$ minutes. This value represents the upper limit for the temperature evolution of each structural element, providing a consistent basis for analyzing the structural response to fire.

Let $(x, y, z)$ represents the midpoint of a structural element and $(x_{\subfire}, y_{\subfire}, z_{\subfire})$ the fire source point. \revise{Integer parameters $h$ and $h_{\subfire}$ correspond to the respective floor levels of the element and the fire source}. The temperature evolution for each element is expressed as follows:
\begin{enumerate}
    \item Linear increase stage ($0 < t < t_1$):
    \begin{equation}
    T(t) = c \cdot t,
    \end{equation}
    where $c$, the rate of temperature increase ($^\circ\mathrm{C}/\mathrm{min}$), depends on the height difference between the element, $h$, and the fire source, $h_{\subfire}$:
    \begin{equation}
        c = 
        \begin{cases} 
        5\left/\left(h - h_{\subfire} + 1\right)\right., & h \geq h_{\subfire}, \\
        2\left/\left(h_{\subfire} - h\right)\right., & h < h_{\subfire}.
        \end{cases}
    \end{equation}
     \item ISO 834 stage ($t \geq t_1$):
\begin{equation}
    T(t) = c \cdot t_1 + 345 \log_{10} \left(8 \left(t - t_1\right) + 1\right).
\end{equation}
\end{enumerate}

The transition (arrival) time $t_1$, marking the end of the linear stage, depends on the spatial distance between the fire source and the element. We define the following two Euclidean distances $L_p$ in the $xy$ plane and $L_s$ in the $xyz$ space:
\begin{eqnarray}
L_p & \triangleq & \sqrt{(x - x_{\subfire})^2 + (y - y_{\subfire})^2}, \\
\label{eq:Lp}
L_s & \triangleq & \sqrt{(x - x_{\subfire})^2 + (y - y_{\subfire})^2 + (z - z_{\subfire})^2}.
\label{eq:Ls}
\end{eqnarray}
Accordingly, the transition time, $t_1$, is expressed as follows:
\begin{equation}
    t_1 = 
    \begin{cases}
    \beta_{1} \cdot \left(1 - \exp\left\{- L_s\left/\alpha_{1}\right.\right\}\right), & h > h_{\subfire}, \\
    \beta_{2} \cdot \left(1 - \exp\left\{- L_p\left/\alpha_{2}\right.\right\}\right), & h = h_{\subfire}, \\
    \beta_{3} \cdot \left(1 - \exp\left\{- L_s\left/\alpha_{3}\right.\right\}\right), & h < h_{\subfire} .
    \end{cases}
    \label{eq:t1}
\end{equation}
The parameters $\beta_i$ and $\alpha_i$ for determining $t_1$ are summarized in Table~\ref{tab:fire_spread_parameters}. In this study, we take $r_{\mathrm{up}}=0.95$ and $r_{\mathrm{down}}=0.97$.
\begin{table}[ht]
    \centering
    \caption{Fire spread parameters for $t_1$ calculations.}
    \begin{tabular}{lcc}
        \toprule
        Case  & $\beta_i$ & $\alpha_i$  \\
        \midrule
        $i=1$, Upward spread & $16 \left.\left(1-r_{\mathrm{up}}^{\left|h-h_{\subfire}\right|}\right)\right/\left(1-r_{\mathrm{up}}\right)$ & $10$  \\
        $i=2$, Horizontal spread & $18$ & $18$  \\
        $i=3$, Downward spread & $30 \left.\left(1-r_{\mathrm{down}}^{\left|h-h_{\subfire}\right|}\right)\right/\left(1-r_{\mathrm{down}}\right)$ & $5$  \\
        \bottomrule
    \end{tabular}
    \label{tab:fire_spread_parameters}
\end{table}

\figref{fig:t1_curve} illustrates the $t_1$ curves for various fire scenarios: (1) fire originating on the lower floor, $h-h_{\subfire}=1$ with rapid upward spread, (2) fire on the same floor, $h=h_{\subfire}$ with the fastest spread, and (3) fire on the upper floor, $h_{\subfire}-h=1$ with slow downward spread. The exponential decay in $t_1$ reflects the accelerating fire propagation speed as the distance increases. \figref{fig:t1_curve} also indicates that the employed simplified model is consistent with the Markov chain-based dynamic model given by \cite{cheng_dynamic_2011}, where the rooms at the same floor of the fire point start flashover slightly before the corresponding upper floors. Additionally, $\beta_{1}$ and $\beta_{3}$ are the summation of a geometric sequence, where story level $h$ is the index. The common ratios $r_{\mathrm{up}}<1$ in $\beta_{1}$ and $r_{\mathrm{down}}<1$ in $\beta_{3}$ indicate that the fire speeds up to spread through the next story, which is consistent with the real-world fire spread mechanism given in \cite{hokugo_mechanism_2000}. The temperature profile within the range $t \in [0, t_{\thre}]$ is subsequently used as the thermal load in OpenSeesRT simulations to compute displacements at each structural node at time $t_{\thre}$.
\begin{figure}[h!]
    \centering
    \includegraphics[width=0.8\linewidth]{figures/m204_t1_curve.pdf}
    \caption{Three examples for the $t_1$ curve.}
    \label{fig:t1_curve}
\end{figure}

\revise{
\textit{Remark:} The effects of structural elements, such as concrete floor slabs and partitions, are not explicitly modeled in our approach. Instead, their influence is implicitly captured through the careful selection of the parameters $ \alpha, \beta, r_\mathrm{up} $, and $ r_\mathrm{down} $. This parameterization provides a unified framework for generating temperature fields. Indeed, fire propagation is governed by a multitude of factors and remains an open research question. For instance, if the fire resistance of a floor slab is enhanced by fire protective coating, the corresponding model can account for this by decreasing $\alpha_1$ \& $\alpha_3$, increasing $\beta_1$ \& $\beta_3$, and adopting larger values for $r_\mathrm{up}$ \& $r_\mathrm{down}$, which collectively slow down the vertical spread of fire. Conversely, scenarios involving higher amounts of combustible materials would warrant the opposite adjustments. This flexible and integrated approach avoids the need to design separate models for different fire propagation scenarios while still capturing the essential effects.
}

\revise{
In conclusion, our rule-based approach is a computationally efficient method for approximating fire temperature fields, enabling large-scale dataset generation to train predictive models. By combining ISO 834 fire curves with spatial considerations and embedding structural effects through parameter calibration, the method achieves a balanced trade-off between accuracy and scalability, making it a practical solution for thermal load modeling in fire scenarios. After generating the temperature of each beam or column according to the middle point, the temperature is applied as uniform thermal load to the elements of the structure in question using OpenSeesRT. 
}

% In conclusion, this rule-based approach is a computationally efficient method to approximate fire temperature fields, enabling large-scale dataset generation to train predictive models. By combining ISO 834 fire curves with spatial considerations, the method balances accuracy and scalability, making it a practical solution for thermal load modeling in fire scenarios.

% \subsection{Interstory Drift Ratio}
\subsection{OpenSeesRT Simulation}
\label{subsec:opensees_simulation}

The thermal and mechanical responses of 3D frame structures under combined fire and gravity loads are simulated using OpenSeesRT \cite{perez2024openseesrt}. \revise{In the simulation, the IDR of each node at $t_{\thre}$ is computed using the computed nodal displacements. Each structural model features six degrees of freedom per node (3 translational  and 3 rotational), with linear geometrical transformations (\texttt{geomTransf: Linear}) defining how the element local coordinate systems are mapped to the global coordinate system and assuming small displacements and rotations. Although OpenSeesRT allows a variety of options for modeling finite deformations, in the present simulations and mainly for simplicity, we did not consider large deformations. All bottom nodes (nodes on the ground) are fully constrained in all six degrees of freedom, while degrees of freedom os all other nodes are free.} Material behavior is temperature-dependent and modeled with \texttt{Steel01Thermal}, while fiber-based sections (\texttt{FiberThermal}) capture nonlinear interactions between thermal and mechanical responses at the cross-section level. \revise{Structural elements are represented as displacement-based Euler-Bernoulli beam-columns (\texttt{dispBeamColumnThermal}). This element  formulation accounts for thermal strains (temperature gradients) in the section, which is discretized into fibers. Numerical integration is used along the length of each element using three integration (Gauss) points, one at each end and the third in the middle of the element.}

{\revise{Thermal expansion of steel members plays a crucial role in IDR development. In reality, reinforced concrete floor slabs heat at a different rate than steel members due to their higher thermal mass and lower thermal conductivity. This differential heating can lead to restrained thermal expansion, introducing axial compression in beams and affecting the overall structural response. In this study, explicit {\em{composite action}} between steel members and concrete slabs is not modeled. Instead, our approach focuses on isolating the response of the steel structural frame, which is often the critical load-bearing component in fire scenarios. This assumption aligns with prior studies \cite{Possidente_2024} demonstrating that steel structures reach thermal equilibrium with surrounding gases quickly, allowing the use of uniform thermal loading in fire analysis. Future work could enhance this framework by incorporating slab-beam interaction effects, through a refined FEA for an extended dataset where constraints imposed by floor slabs are explicitly considered.}

The analysis begins with the application of gravity loads, followed by incremental thermal loads simulating the fire exposure. A static nonlinear solver using  \texttt{ExpressNewton} algorithm ensures convergence, while the \texttt{NormDispIncr} test maintains accuracy. An incremental \texttt{LoadControl} scheme with small step sizes is employed to guarantee numerical stability, using 10\% for gravity loads and 1\% for thermal loads. 

\revise{
In the thermal load analysis, uniform thermal load is applied to each beam or column, i.e., the temperature of each element is set to be that at the middle point, according to \secref{subsec:thermal_load_generation}. The \texttt{Steel01Thermal} material allows the properties (e.g., Young's modulus and yield strength) to be adjusted at increasing temperatures according to \cite{EN1993} using its Table 3.1: Reduction factors for the stress-strain relationship of carbon steel at elevated temperatures. For example, if the Young’s modulus at ambient temperature is $E_0$, then as the temperature ($T$) increases, the modulus changes as $E(T) = \eta (T) \times E_0$. \cite{EN1993} directly provides the values of $\eta(T) \in \left[0,1\right] $ at every $100 ^\circ\mathrm{C}$ interval and recommends using linear interpolation to obtain $\eta(T)$ for intermediate values of $T$.
} OpenSeesRT documentation \cite{OpenSeesThermalExamples} provides several examples of thermal analyses.

This modeling framework accommodates variations in material properties, cross-sectional geometries, and temperature profiles, providing robust simulations of structural behavior under fire conditions. The primary settings and configurations for the OpenSeesRT simulations are summarized in \tabref{tab:ops_detail}.
\begin{table}[h!]
    \centering
        \caption{Key settings of OpenSeesRT simulations.}
    \begin{tabular}{l|>{\raggedright\arraybackslash}p{0.6\linewidth}} %
    \toprule
    Modeling Aspect     & Details \\
    \midrule
    Geometry            & 3D models; 6 degrees of freedom per node \\
    Transformation      & geomTransf: Linear \\ 
    Material            & Steel01Thermal \\
    Section             & FiberThermal; Cross-section: $0.1$ m $\times$ $0.1$ m \\ 
    Element type        & {dispBeamColumnThermal} \\ 
    Loading             & Gravity loads: {beamUniform}; Thermal loads: {beamThermal} \\
    Integration scheme  & Incremental {LoadControl}; Step size: $10\%$ (gravity analysis), $1\%$ (thermal analysis) \\
    Nonlinear solver    & {ExpressNewton} algorithm; {UmfPack} solver; Convergence test: {NormDispIncr} tolerance: $10^{-8}$; Maximum \# iterations per step: $1000$. \\ 
    \bottomrule
    \end{tabular}
    \label{tab:ops_detail}
\end{table}

For each structure in the labeled dataset, 30 fire points are selected using a dual-granularity approach, \revise{i.e., two-stage sampling strategy,} to ensure they are well-distributed. Specifically, rooms are sequentially selected, with one fire point randomly chosen within each selected room. If a building is large and contains more than 30 rooms, we randomly select 30 rooms without replacement, i.e., ensuring that no more than one fire point is located in the same room. Conversely, if the building is small and has fewer than 30 rooms, all rooms are initially selected, with one fire point randomly assigned to each room. Additionally, rooms are then selected with replacement until a total of 30 fire points are assigned. \revise{The room-level sampling prioritizes selecting distinct rooms to avoid spatial clustering of fire points, while the point-level sampling ensures intra-room variability. This approach aligns with stratified sampling principles commonly used for efficient spatial representation, where multi-stage sampling strategies optimize coverage and variability, e.g., \cite{arunachalam_generalized_2023}, and enables a more comprehensive characterizing of how the structures respond under fire conditions.}
% This selection method prevents fire points from clustering too closely while maintaining an element of randomness. By distributing fire points in this manner, the 30 fire scenarios are effectively utilized, enabling a more comprehensive characterizing of how the structures respond under fire conditions.

\subsection{Summary of the Dataset Generation}
As discussed in this section and related to  \figref{fig:dataset_generation_procedure}, three key steps were considered in the development of the dataset: 
\begin{enumerate}
    \item {\bf{Filtering process}}: Structures with MIDR exceeding $1\%$ under gravity loads were excluded,  resulting in $1,573$ labeled structures retained for fire simulation and $16,050$ unlabeled structures for training the MFSP predictor.
    \item {\bf{Fire simulations}}: For each retained labeled structure, 30 fire scenarios were simulated using OpenSeesRT, yielding $47,190$ fire cases.
    \item {\bf{Data distribution check}}: MIDR distributions for labeled and unlabeled data under gravity loads were highly similar, because both datasets were generated using the same method. Under fire conditions, the MIDR distribution shifted, reflecting significant structural deformation with values reaching a maximum of about 6\%, an average of 1.70\%, and a standard deviation of 1.12\%. This step ensured a diverse and comprehensive dataset for the proposed predictive framework.
\end{enumerate}
The statistical distribution histograms for MIDR (after applying the $1\%$ filtering threshold \revise{for gravity load responses}) under different loading conditions are plotted in \figref{fig:histogram_mdr}. Figures \ref{fig:histogram_mdr}(a) and \ref{fig:histogram_mdr}(b) show the MIDR distributions of the labeled and unlabeled data, respectively, under gravity loads only. \figref{fig:histogram_mdr}(c) shows the MIDR distribution of the labeled data under the combined effects of gravity and fire loads. Fire load causes the structures to significantly deform, leading to a noticeably \revise{right-skewed} MIDR distribution.

\begin{figure*}[h!]
    \centering
    \includegraphics[width=\linewidth]{figures/histogram_mdr.pdf}
    \caption{Histograms of MIDR for labeled and unlabeled structures with gravity loads and fire cases.}
    \label{fig:histogram_mdr}
\end{figure*}

\revise{
This dataset provides the basis for training and testing the performance of the GNN-based framework. Although we employed a simplified rule-based thermal load generation method compared with conventional CFD-based simulations, the temperature field, the changes of the material properties, and the response of the structures, are all still highly nonlinear and complex. Therefore, it is still a challenging task for the NN to predict the MIDRs based on this dataset.
}
\section{Experiments}
\label{sec: exp}

In this section, we conduct experiments to answer the following research questions:
\begin{itemize}
\item \textbf{RQ1}: Can the proposed model effectively improve the performance of the original CDMs?  
\item \textbf{RQ2}: What is the impact of each component within the proposed method? 
\item \textbf{RQ3}: How does the proposed model perform on cold-start scenarios? 
% \item \textbf{RQ4}: What are the differences in diagnostic effectiveness when using different LLMs?
\item \textbf{RQ4}: How effective is the alignment of semantic and behavioral space embeddings during the cognitive level alignment process?
\end{itemize}

\subsection{Experimental Settings}

\subsubsection{Datasets}

\section{Baseline} \label{sec:splitgraph}

The baseline method for batch-$k$DP solves each query using flow-augmenting path-based methods, which rely on the concept of \textit{split-graphs}~\cite{baseline_moreverbose, baseline1step2, baselineOnlySplitP1}. 
% For each query, paths are iteratively found in a split-graph, which is updated after each iteration.
% A split-graph is constructed by two transformations of the original graph:
% (1) reversing result-set paths, simulating flow-augmentation, and 
% (2) splitting vertices within these paths, giving rise to the name ``split-graph."

\textbf{Definition: Split-Graph~\cite{baselineOnlySplitP1}} 
Given a graph \( G = (V, E) \) and a set \( P \) of disjoint paths from \( s \) to \( t \), the split-graph \( \iG_{G,P} = (\iV_{G,P}, \iE_{G,P}) \) is constructed as follows:
(1) Initializing \( \iV_{G,P} = V \) and \( \iE_{G,P} = E \).
(2) For each edge in \( E(P) \), reversing the corresponding edge in \( \iE_{G,P} \).
(3) Splitting vertices \(v \in V(P) \setminus \{s, t\}\) into \(v^{in}\) and \(v^{out}\), and connecting them accordingly.
(4) Replacing edges in \(\iE_{G,P}\) with updated vertex connections, preserving incoming and outgoing edges.

% \textbf{Example}: 
% Fig.~\ref{fig:eg_split} shows the split-graph construction for the graph \( G \) in Fig.~\ref{fig:g} with $P= \{p_1=\{a, e, d, h\}\}$. Changes are shown in red.


% \vspace{-10pt}
\begin{figure}[h!]
\newcommand{\mylinewidth}{\linewidth}
\centering
    \begin{subfigure}[t]{0.35\mylinewidth}
        \centering
        % \resizebox{\mylinewidth}{!}
        {\includegraphics[width=\linewidth]{pic/eg/g}}
        \caption{Disjoint paths for $(a, h)$.}
        \label{fig:g}
    \end{subfigure}
    \begin{subfigure}[t]{0.6\mylinewidth}
        \centering
        % \resizebox{\mylinewidth}{!}
        {\includegraphics[width=\linewidth]{pic/eg/steps_red_new.pdf}}
        \caption{Split-graph with $P= \{p_1=\{$a$, $e$, $d$, $h$\}\}$.}
        \label{fig:eg_split}
    \end{subfigure}
    \caption{Examples of disjoint paths and split-graph.}
    % \label{fig:fg_share_intuition}
\end{figure} 
% \vspace{-5pt}

% 删除 begin
Given a graph \( G \) and vertices \( s \) and \( t \), the algorithm proceeds as follows:
% (1) Initialize \( P = \emptyset \) and \( \iG_{G,P} = G \).
% (2) Find the first path \( p_1 \) using a path-finding algorithm (e.g., BFS) in \( \iG_{G,P} \) and update \( \iG_{G,P} \).
% (3) Find the second path \( p_2 \), update found paths following an approach similar to augmenting paths in the maximum flow problem~\cite{baseline_moreverbose}, then update \( \iG_{G,P} \). More paths are found in a similar manner.
(1) Initialize $P = \emptyset$ and $\iG_{G, P} = G$.
(2) Find the first path $p_1$ in $\iG_{G, P}$ using any path-finding algorithm (e.g., BFS), forming $P_1 = \{p_1\}$, and update $\iG_{G, P}$ to $\iG_{G, P_1}$.
(3) Search for $p_2$ in $\iG_{G, P_1}$, yielding $P_2 = \{p_1, p_2\}$, and adjust $P_2$ following an approach similar to augmenting flows~\cite{baseline_moreverbose}.
Then update $\iG_{G, P_1}$ to $\iG_{G, P_2}$.
(4) Search for $p_3$ in $\iG_{G, P_2}$. More paths are found in a similar manner.
% 删除 end
In our experiments, we utilize four courses, Python Programming (Python), Linux System (Linux), Database Technology and Application (Database), and Literature and History (Literature), from a publicly available dataset PTADisc~\cite{hu2023ptadisc}, which comes from real-world students' responses in the educational website PTA\footnote{\url{https://pintia.cn/}} and contains textual information of exercises and knowledge concepts. 
%Each response log in the dataset contains a student ID, an exercise ID, whether the student correctly answers the question, the content of the exercise, and the knowledge concepts related to the exercise.
The statistics of the datasets are presented in Table~\ref{tab: dataset}.
The datasets are divided into training, validation, and testing sets, with a ratio of 8:1:1.

\subsubsection{Evaluation Metrics}

Following previous works, we evaluate the students' cognitive status by predicting the performance of students on the testing set, as the cognitive status can not be directly observed. We adopt commonly used metrics, namely the Area Under a ROC Curve (AUC), the Prediction Accuracy (ACC), and the Root Mean Square Error (RMSE), to validate the effectiveness of the CDMs.
%In the subsequent tables, \textbf{bold} numbers represent the best performance, while \underline{underlined} numbers represent the second-best performance. 
For all the metrics, $\uparrow$ represents that a greater value is better, while $\downarrow$ represents the opposite.

\subsubsection{Baseline Methods}

To validate the effectiveness of the proposed method, we conduct experiments on several representative CDMs, including IRT~\cite{lord1952theory}, MIRT~\cite{reckase200618}, DINA~\cite{de2009dina}, NCD~\cite{wang2020neural}, RCD~\cite{gao2021rcd}, SCD~\cite{wang2023self} and ACD~\cite{wang2024boosting}.
 

\subsubsection{Implementation Details}

We utilize PyTorch to implement both the baseline methods and our proposed KCD framework. 
For the baseline models, We use the default hyper-parameters as stated in their papers and for KCD, we use the same hyper-parameter settings, such as training epoch, learning rate, and batch size.
We employ ChatGPT to represent LLMs (specifically, gpt-3.5-turbo-16k) and text-embedding-ada002 as the text embedding model. All the experiments are conducted on a GeForce RTX 3090 GPU.
We train the model on train set and at the end of each epoch, we evaluate the model on the validation set.
The hyper-parameter $\alpha$, $\beta$, and $\lambda$ was set to $0.04$, $0.015$, and $0.2$.
Since our dataset does not include affect labels, we utilize the unsupervised contrastive ACD model and employ NCD as the basic cognitive diagnosis module.
The behavioral space alignment approach is denoted as `-Beh' and the semantic space alignment approach is denoted as `-Sem'.
% We investigated the impact of the hyper-parameter $\lambda$, within the range $[0,0.2,\cdots,1]$ with a step size of $0.2$. Our analysis revealed that setting $\lambda$ to $0.1$ resulted in the best performance across all three datasets.

\begin{figure}[t]
  \centering
  
  \includegraphics[width=1.02\linewidth]{figs/experimentx.png}
  \caption{Performance comparison in cold (blue) and warm (red) scenarios on Python dataset.}
  \vspace{-2em}
\label{fig: experiment1}
\end{figure}

\subsection{Performance Comparison (RQ1)}
To demonstrate the effectiveness of our proposed method in improving cognitive diagnosis, we implement the framework on seven cognitive diagnosis models, and the results are shown in Table~\ref{tab:performance}. 
Additionally, we compared the performance of NCD in warm and cold scenarios, with the results illustrated in Figure~\ref{fig: experiment1}. Here we define the cold scenario as less than $3$ interactions in the training set for exercises and define the warm scenario as more than $10$ interactions in the training set for exercises. Following this definition, we divide the testing set into cold and warm subsets.
We have the following observations from the results: 

\begin{itemize}[leftmargin=*]
    \item[1)]  
    Both KCD-Beh and KCD-Sem achieve significant improvements compared to the basic CDMs.
    This indicates that our proposed framework is widely applicable to various CDMs, and both alignment methods can effectively align the behavioral space of CDMs and the semantic space of LLMs.
    In most models, the behavioral space alignment approach performs better, indicating that aligning in the behavioral space of CDMs can better align information from the semantic space of LLMs.
    \item[2)] Compared to basic CDMs, our proposed methods demonstrate improvements in both cold and warm scenarios, especially in cold scenarios. This indicates that our approach of introducing LLMs as knowledge enhancement effectively alleviates the cold-start issue.
\end{itemize}




\begin{table*}
  [t]
  \centering
  \resizebox{\textwidth}{!}{%
  \begin{tabular}{cccccccccccc}
    \toprule \multicolumn{2}{c}{Components}                                                             & \multicolumn{5}{c}{Re-executability Rate (\%)} & \multicolumn{5}{c}{Readability (\#)} \\
    \cmidrule(lr){1-2} \cmidrule(lr){3-7} \cmidrule(lr){8-12}        \hspace{8pt}\labelemoji\hspace{8pt}                                                                & \hspace{8pt}\toolemoji\hspace{8pt}                                      & O0                                 & O1             & O2             & O3             & AVG            & O0             & O1             & O2             & O3             & AVG            \\
    \hline
    \rowcolor[rgb]{0.93,0.93,0.93}\multicolumn{12}{c}{\textbf{Initialize with LLM4Decompile-End-6.7B~\citep{llm4decompile}}}   \\
    \xmark                                                                                              & \xmark                                    & 69.51                              & 46.95          & 50.61          & 46.34          & 53.35          & 3.98 & 3.41 & 3.44 & 3.38 & 3.55 \\
    \cmark                                                                                              & \xmark                                    & 75.61                              & 50.61          & 50.00          & 50.00          & 56.55          & 4.01 & 3.44 & 3.39 & \textbf{3.49} & 3.58 \\
    \xmark                                                                                              & \cmark                                    & 83.54                     & \textbf{56.10}          & 51.22          & 50.61 & 60.37 & 4.05 & 3.51 & 3.51 & 3.42 & 3.62 \\
    \cmark                                                                                              & \cmark                                    & \textbf{85.37}                            & \textbf{56.10}                     & \textbf{51.83} & \textbf{52.43}          & \textbf{61.43} & \textbf{4.13} & \textbf{3.60} & \textbf{3.54} & \textbf{3.49} & \textbf{3.69} \\

    \rowcolor[rgb]{0.93,0.93,0.93}\multicolumn{12}{c}{\textbf{Initialize with Deepseek-Coder-6.7B-base~\citep{deepseekcoder}}} \\
    \xmark                                                                                              & \xmark                                    & 59.15                              & 35.98          & 39.02          & 37.80          & 42.99          & 3.71 & 3.05 & 3.16 & 3.05 & 3.24 \\
    \cmark                                                                                              & \xmark                                    & 66.46                              & 41.46          & 38.41          & 36.59          & 45.73          & 3.76 & 3.17 & \textbf{3.21} & 3.08 & 3.31 \\
    \xmark                                                                                              & \cmark                                    & 70.73                              & 39.63          & 39.02          & 40.24          & 47.41          & 3.90 & 3.17 & 3.08 & 3.11 & 3.31 \\
    \cmark                                                                                              & \cmark                                    & \textbf{79.88}                     & \textbf{45.73} & \textbf{43.90} & \textbf{42.68} & \textbf{53.05} & \textbf{3.96} & \textbf{3.21} & 3.18 & \textbf{3.19} & \textbf{3.38} \\
    \bottomrule
  \end{tabular}%
  }
  \caption{The ablation study of different methods across four optimization levels
  (O0, O1, O2, O3), as well as their average scores (AVG). The results in bold represent the optimal performance. The ~\labelemoji~ and ~\toolemoji~ means Relabedling and Function Call. \textbf{Bold} denotes the best performance.}
  \label{tab:ablation}
\end{table*}
\subsection{Ablation Study (RQ2)}


To validate the effectiveness of different components of our proposed method, we conduct ablation experiments to verify several components utilized in LLM Diagnosis and Cognitive Level alignment, including the usage of collaborative information (denoted as `Coll. Info'), the local contrast and global contrast (denoted as `Local Con.' and `Global Con.'), and the dynamic masking strategy (denoted as `Dym. Mask').

Table~\ref{tab:ablation} demonstrates the results of the ablation study on Python dataset, comparing the model performance after removing specific components (denoted as `w/o'). `w/o Coll. Info' represents replacing collaborative information in the process of diagnosis generation and `w/o Dym. Mask' represents replacing dynamic masking strategy with a constant mask ratio.
Experimental results show that removing these components individually leads to a decline in the model's performance. This indicates that these components are crucial for the model's performance.


\begin{figure}[t]
  \centering
  
  \includegraphics[width=1\linewidth]{figs/drop.png}
  \caption{Performance on different dropout ratios.}
  
\label{fig: drop}
\end{figure}
\subsection{Performance on Cold-Start Scenarios (RQ3)}

we conduct additional experiments on sub-datasets with varying degrees of sparsity. Specifically, we apply random dropout to the training sets of the Python and Linux datasets at ratios of $10\%$, $20\%$, $30\%$, $40\%$, and $50\%$.

Figure~\ref{fig: drop} shows the results of the experiments on different dropout ratios. It is obvious that as the dropout ratio increases, both AUC and ACC decrease. This is because the training set becomes more sparse, approaching a cold-start scenario. 
Additionally, compared to ACC, AUC experiences a greater decline, which might be due to the different calculation methods of the two metrics. 
% For more sparse datasets, Python, AUC experience a more significant decrease compared to the Linux dataset. From the experimental results, it can be seen that our proposed method is effective across different dropout ratios, leading to significant improvements for CDMs. More specifically, from the different performances of NCD-Beh and NCD-Sem in the Linux and Python datasets, it can be seen that we can choose different alignment methods based on the dataset to achieve better diagnostic results.


\begin{figure}[t]
  \centering
  \vspace{-1em}
  \includegraphics[width=1\linewidth]{figs/experiment2.png}
  \caption{The t-SNE visualization of student embeddings on Literature dataset.}
  \vspace{-2em}
\label{fig: experiment2}
\end{figure}
\subsection{Visualization of Semantic and Behavioral Embeddings (RQ4)}


To validate the effectiveness of the two alignment processes, we utilize t-SNE~\cite{van2008visualizing} to visualize the distribution of features in LLMs semantic space and CDMs behavioral space. We randomly select 200 example students and map their behavioral embeddings and semantic embeddings to 2-dimensional space. NCD (w/o Alignment) represents the original CDMs without alignment.

Figure~\ref{fig: experiment2} demonstrates the integration of semantic and behavioral embeddings of NCD-Beh and NCD-Sem, with their distributions closely merged compared to original CDMs. This proves the effectiveness of the two alignment methods we proposed.

\begin{figure}[t]
  \centering
  
  \includegraphics[width=1\linewidth]{figs/case.png}
  \caption{The case study of a student on multiple knowledge concepts on Linux dataset.}
  \vspace{-2em}
\label{fig: case}
\end{figure}

\subsection{Case Study}


To more intuitively demonstrate the improvements our proposed methods bring to CDMs, we selected a diagnosis for a specific student in the Linux dataset and compared the prediction results of NCD with the diagnosis results of NCD-Beh.
As illustrated in Figure~\ref{fig: case}, we randomly choose a student, and list his mastery of some knowledge concepts predicted by NCD and our proposed NCD-Beh.
This student correctly answered the exercises related to `numerical encoding' and `process communication', showing mastery of these concepts. He answered other exercises incorrectly, indicating a lack of familiarity with the remaining knowledge concepts.
From the LLM's diagnostic results, it can be observed that the LLM captured similar question-answer information from the training set and made corresponding inferences. This played an important role in NCD-Beh's more accurate prediction of the student's mastery level.

\section{Conclusion}

In this work, we propose a model-agnostic framework KCD that can efficiently employ LLMs to enhance the knowledge of conventional CDMs.
By utilizing LLM diagnosis and cognitive level alignment, the framework can leverage the rich knowledge of LLMs and align the semantic space of LLMs and the behavioral space of CDMs to achieve optimal diagnostic results.
Several experiments on four real-world datasets for cognitive diagnosis demonstrate the superiority of our proposed framework, surpassing all the baseline CDMs.
%Future work mainly focuses on other effective approaches to integrate LLMs and CDMs to provide accurate, explainable diagnoses which are also easier for downstream applications to utilize. 

\section{Acknowledgments}
This research was partially supported by grants from the National Natural Science
Foundation of China (No.62037001, No.62307032), Shanghai Rising-Star Program
(23QA1409000), the Starry Night Science Fund at Shanghai Institute for Advanced Study (SN-ZJU-SIAS-0010), and the "Pioneer" and "Leading Goose" R\&D Program of Zhejiang under Grant No. 2025C02022.


\bibliography{aaai25}

\end{document}
