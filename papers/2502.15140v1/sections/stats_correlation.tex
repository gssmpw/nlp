\section{Statistical Correlation}
\label{app:stats-corr}
For each multiple-choice question $q$ with answer choices $A_q = \{a_1, a_2, \cdots, a_n \}$, we computed the correlation between student responses distribution $P_s(a_i \mid q)$ and language model likelihood $P_m(a_i \mid q)$ using the following statistical methods.

\subsection{Pearson's Correlation}
We use Pearson product-moment correlation~\cite{freedman2007statistics} coefficient $r_q$ to calculate the linear relationships between student performance distribution $P_s(a_i|q)$ and language model likelihood $P_m(a_i|q)$ . For a sample of $n$ questions, the coefficient is defined as:

$$r_q = \frac{\sum_{i=1}^n (P_s(a_i \mid q) - \mu_s)(P_m(a_i \mid q) - \mu_m)}{\sqrt{\sum_{i=1}^n (P_s(a_i \mid q) - \mu_s)^2}\sqrt{\sum_{i=1}^n (log P_m(a_i \mid q) - \mu_m)^2}}$$

where $\mu_s$ and $\mu_m$ denote the sample means. The coefficient $r_q \in [-1, 1]$ provides a normalized measure of linear dependence, where $|r_q| = 1$ indicates perfect linear correlation and $r_q = 0$ indicates no linear correlation. 

\subsection{Spearman's Rank Correlation}
We apply Spearman's rank correlation~\cite{zar2005spearman} coefficient $\rho_q$ to measure the non-linear monotonic relationships by operating on the ranks instead of raw values. Let $\text{rank}(P_s(a_i \mid q))$ and $\text{rank}(P_m(a_i \mid q))$ denote the rankings of the $i$-th answer choice when ordered by student response distribution and model likelihood respectively. The Spearman's rank coefficient is defined as:

$$\rho_q = 1 - \frac{6\sum_{i=1}^n d_i^2}{n(n^2-1)}$$

where $d_i = \text{rank}(P_s(a_i \mid q)) - \text{rank}(P_m(a_i \mid q))$ represents the rank difference for the $i$-th answer choice. The coefficient $\rho_q \in [-1, 1]$, where -1 indicates perfect negative correlation (inverse rankings), +1 indicates perfect positive correlation (identical rankings), and 0 suggests no correlations between the rankings.


\subsection{Kendall's tau Correlation}
We compute Kendall's tau~\cite{kendall1938new} correlation coefficient $\tau_q$ to measure the concordance between model predictions and student responses through pairwise rankings. For any two answer choices $i$ and $j$, a pair is considered concordant if their relative ordering is consistent across both distributions (i.e., if $P_s(a_i \mid q) > P_s(a_j \mid q)$ when $P_m(a_i \mid q) > P_m(a_j \mid q)$, or $P_s(a_i \mid q) < P_s(a_j \mid q)$ when $P_m(a_i \mid q) < P_m(a_j \mid q)$), and discordant if their ordering differs. Kendall's tau correlation is defined as:

$$\tau_q = \frac{2(n_c - n_d)}{n(n-1)}$$

where $n_c$ is the number of concordant pairs, $n_d$ is the number of discordant pairs. The coefficient $\tau_q \in [-1, 1]$, where values closer to ±1 indicate stronger alignment between model likelihoods and student answer distributions (positive for similar rankings, negative for opposite rankings), while 0 indicates no alignment.


