\section{Dataset}
\label{sec:dataset}

\subsection{Data Collection}

To analyze political discussions on Discord, we followed the methodology in \cite{singh2024Cross-Platform}, collecting messages from politically-oriented public servers in compliance with Discord's platform policies.

Using Discord's Discovery feature, we employed a web scraper to extract server invitation links, names, and descriptions, focusing on public servers accessible without participation. Invitation links were used to access data via the Discord API. To ensure relevance, we filtered servers using keywords related to the 2024 U.S. elections (e.g., Trump, Kamala, MAGA), as outlined in \cite{balasubramanian2024publicdatasettrackingsocial}. This resulted in 302 server links, further narrowed to 81 English-speaking, politics-focused servers based on their names and descriptions.

Public messages were retrieved from these servers using the Discord API, collecting metadata such as \textit{content}, \textit{user ID}, \textit{username}, \textit{timestamp}, \textit{bot flag}, \textit{mentions}, and \textit{interactions}. Through this process, we gathered \textbf{33,373,229 messages} from \textbf{82,109 users} across \textbf{81 servers}, including \textbf{1,912,750 messages} from \textbf{633 bots}. Data collection occurred between November 13th and 15th, covering messages sent from January 1st to November 12th, just after the 2024 U.S. election.

\subsection{Characterizing the Political Spectrum}
\label{sec:timeline}

A key aspect of our research is distinguishing between Republican- and Democratic-aligned Discord servers. To categorize their political alignment, we relied on server names and self-descriptions, which often include rules, community guidelines, and references to key ideologies or figures. Each server's name and description were manually reviewed based on predefined, objective criteria, focusing on explicit political themes or mentions of prominent figures. This process allowed us to classify servers into three categories, ensuring a systematic and unbiased alignment determination.

\begin{itemize}
    \item \textbf{Republican-aligned}: Servers referencing Republican and right-wing and ideologies, movements, or figures (e.g., MAGA, Conservative, Traditional, Trump).  
    \item \textbf{Democratic-aligned}: Servers mentioning Democratic and left-wing ideologies, movements, or figures (e.g., Progressive, Liberal, Socialist, Biden, Kamala).  
    \item \textbf{Unaligned}: Servers with no defined spectrum and ideologies or opened to general political debate from all orientations.
\end{itemize}

To ensure the reliability and consistency of our classification, three independent reviewers assessed the classification following the specified set of criteria. The inter-rater agreement of their classifications was evaluated using Fleiss' Kappa \cite{fleiss1971measuring}, with a resulting Kappa value of \( 0.8191 \), indicating an almost perfect agreement among the reviewers. Disagreements were resolved by adopting the majority classification, as there were no instances where a server received different classifications from all three reviewers. This process guaranteed the consistency and accuracy of the final categorization.

Through this process, we identified \textbf{7 Republican-aligned servers}, \textbf{9 Democratic-aligned servers}, and \textbf{65 unaligned servers}.

Table \ref{tab:statistics} shows the statistics of the collected data. Notably, while Democratic- and Republican-aligned servers had a comparable number of user messages, users in the latter servers were significantly more active, posting more than double the number of messages per user compared to their Democratic counterparts. 
This suggests that, in our sample, Democratic-aligned servers attract more users, but these users were less engaged in text-based discussions. Additionally, around 10\% of the messages across all server categories were posted by bots. 

\subsection{Temporal Data} 

Throughout this paper, we refer to the election candidates using the names adopted by their respective campaigns: \textit{Kamala}, \textit{Biden}, and \textit{Trump}. To examine how the content of text messages evolves based on the political alignment of servers, we divided the 2024 election year into three periods: \textbf{Biden vs Trump} (January 1 to July 21), \textbf{Kamala vs Trump} (July 21 to September 20), and the \textbf{Voting Period} (after September 20). These periods reflect key phases of the election: the early campaign dominated by Biden and Trump, the shift in dynamics with Kamala Harris replacing Joe Biden as the Democratic candidate, and the final voting stage focused on electoral outcomes and their implications. This segmentation enables an analysis of how discourse responds to pivotal electoral moments.

Figure \ref{fig:line-plot} illustrates the distribution of messages over time, highlighting trends in total messages volume and mentions of each candidate. Prior to Biden's withdrawal on July 21, mentions of Biden and Trump were relatively balanced. However, following Kamala's entry into the race, mentions of Trump surged significantly, a trend further amplified by an assassination attempt on him, solidifying his dominance in the discourse. The only instance where Trump’s mentions were exceeded occurred during the first debate, as concerns about Biden’s age and cognitive abilities temporarily shifted the focus. In the final stages of the election, mentions of all three candidates rose, with Trump’s mentions peaking as he emerged as the victor.

In this section, we present a comprehensive empirical analysis examining how well LLMs capture student misconception patterns in multiple-choice questions. Through two complementary analyses - one focusing on LLMs' generation probabilities and another on their actual answer selections - we investigate whether these models naturally encode patterns in how students select incorrect answers.

\subsection{Student Performance Dataset}

Our analysis uses a dataset of $3,202$ multiple-choice questions drawn from six core academic domains: mathematics, biology, physics, social science, reading comprehension, and humanities. 
These questions were collected from three established educational assessment platforms, with detailed subject distribution shown in Table~\ref{tab:dataset}. To ensure reliable analysis of student performance patterns, we applied two filtering criteria: (1) each question must have responses from at least 50 students, and (2) the error rate must exceed 5\% to enable meaningful analysis of misconceptions. For consistency in our analysis, we included only questions with exactly four answer choices. The aggregated student performance data shows an average correct response rate of 60.5\% across all subjects. 

\subsection{Accuracy of \tool in Detecting GUI Lags}

%\subsection{RQ: How effectively can \tool detect GUI performance lags?} \feng{the structure of this part looks strange...}
%\subsubsection{Motivation}
%Given the variety design of mobile apps and the different types of contents on app pages, detecting GUI performance lags for the app screencast can be challenging. In this research question (RQ), we evaluate how effectively \tool can detect performance lags in screencasts.


%\subsubsection{Approach}
We use the dataset collected and labeled by the user experience experts as the ground truth for evaluation. 
In the context of a screencast, an instance GUI lag is considered correctly located by \tool if and only if it matches a lag (annotated as $[f_{start}, f_{end}]$) in the ground truth, with identical start and end frame indices (i.e, the same $f_{start}$ and $f_{end}$ values). 
To evaluate the effectiveness of \tool in detecting GUI lags, We employ precision, recall, and F1-score:

\begin{itemize}
    \item \textbf{Precision} is the fraction of correctly identified GUI lags out of all instances that \tool identified as GUI lags. It is defined as: 
\end{itemize}
\begin{equation}
Precision = \frac{\#\textit{Correctly identified GUI lags}}{\#\textit{Identified GUI lags}}
\end{equation}

\begin{itemize}
    \item \textbf{Recall} is the fraction of correctly identified GUI lags out of the total number of actual GUI lags. It is defined as: 
\end{itemize}
\begin{equation}
Recall = \frac{\#\textit{Correctly Identified GUI lags}}{\#\textit{Actual GUI lags}}
\end{equation}

\begin{itemize}
    \item The \textbf{F1-score} is the harmonic mean of precision and recall:
\end{itemize}
\begin{equation}
F_{1} = 2 \times \frac{Precision \times Recall}{Precision + Recall}
\end{equation}


%\subsubsection{Results}
Table~\ref{tab:resutls_detecting_lags} presents the results of detecting GUI performance lags using \tool, demonstrating its high level of accuracy in detecting these lags. The precision for detecting the three types of GUI lags ranges from 0.91 to 0.92, while recall ranges from 0.95 to 0.98, and the F1-score ranges from 0.93 to 0.95. Overall, \tool achieves high average precision of 0.91 and recall of 0.96 across all lag types. 

\begin{table}
	\centering
	\caption{Results of \tool in detecting performance GUI lags across different lag types.}
	\label{tab:resutls_detecting_lags}
	\scalebox{1}{
	\begin{tabular}{lrrr}
		\toprule
		Lag type&\tabincell{r}{Precision}&\tabincell{r}{Recall}&\tabincell{r}{F1-Score}\\
		\midrule
			Janky frames          &0.92	        &0.98       &0.95\\
   	  Long Loading frames   &0.91 		  &0.96		  &0.93\\
			Frozen frames         &0.91		    &0.95	    &0.93\\
		\midrule
		\tabincell{l}{Average Across Types} 	&0.91		&0.96 		&0.94\\
		\bottomrule
	\end{tabular}}
        \vspace{-5mm}
\end{table}


Even though \tool achieves high precision and recall, there are still some missed or incorrectly detected cases. Hence, we conduct further analysis on such cases. We found that certain factors contribute to these inaccuracies. Primarily, our approach uses a threshold of 100 milliseconds based on prior HCI studies~\cite{1968_AFIPS_Response_time_in_man_computer, 1994_Usability_Engineering} to determine the shortest duration that users can typically perceive a lag caused by janky frames. While this threshold works well in most scenarios, it is still possible to lead to missed detection or false positives. 
For example, janky frames with a duration under 100 milliseconds may still be perceptible when they occur in scenarios with frequent or complex screen changes, such as during video playback or fast-paced animations. In these cases, even slight irregularities become noticeable to users, but \tool may not flag them due to the short duration. Similarly, certain long loading frames and frozen frames that exceed the thresholds might not be detected as high-severity issues when users have context-specific expectations of delay, such as when waiting for large media files or complex images to load. In these instances, users anticipate the delay, which reduces their perception of it as a performance issue, yet \tool might flag it as a severe lag based on the duration alone.
This analysis indicates that to further improve detection accuracy, \tool needs to go beyond a simple time threshold and incorporate additional factors such as the type of user interaction, the visual prominence of the lagging element, and the expected behavior of certain app functionalities. By integrating these contextual cues, \tool can more effectively capture the GUI lags that are most likely to affect user satisfaction. 


%true impact of GUI lags on user experience, providing more meaningful and actionable insights to developers. This enhancement would ensure that only the most perceptible and disruptive lags are flagged as critical, allowing developers to prioritize issues that are most likely to affect user satisfaction.




%Furthermore, we observed that not all lags are equally disruptive across different types of interactions. For example, users are less tolerant of delays during high-interaction activities, like tapping buttons or scrolling, compared to delays during passive content loading, such as image rendering in the background. This variation in tolerance suggests that the context of user interactions should also influence how \tool categorizes and prioritizes detected lags. Without accounting for these contextual nuances, \tool may sometimes incorrectly classify the severity of an issue, either overlooking significant lags or unnecessarily flagging minor ones.

%This analysis indicates that to improve detection accuracy, \tool needs to go beyond a simple time threshold and incorporate additional factors such as the type of user interaction, the visual prominence of the lagging element, and the expected behavior of certain app functionalities. By integrating these contextual cues, \tool can more effectively capture the true impact of GUI lags on user experience, providing more meaningful and actionable insights to developers. This enhancement would ensure that only the most perceptible and disruptive lags are flagged as critical, allowing developers to prioritize issues that are most likely to affect user satisfaction.

%============






%Nevertheless, some lags were either missed or incorrectly detected. The primary reason for this is our approach's reliance on the shortest time duration that humans can typically perceive---100ms (calculated as 16.6ms per frame * 5 frames for a 60Hz mobile device).
%In certain cases, users may still perceive lag even when an lag lasts for less than 100ms. For example on janky frames, when playing videos, the content of the screen changes frequently which makes the frame transition more noticeable to users. Although the lag duration is less than 100ms, some people may still perceive it as lagging, which is incorrectly detected by \tool.
%For long load frames, users may still perceive lag even when an lag lasts for less than 100ms. 
%For example on long load frames, as illustrated in Figure 4, although the lag duration is less than 100ms, the size of the image occupying a significant portion of the screen makes the frame transition more noticeable to users.
%Conversely, there are situations where users may not perceive the lag, even when it exceeds 100ms, because they expect the app to take time for specific functionalities based on their prior experiences, which is missed detected by \tool.
%These observations suggest that lag detection should not solely rely on duration but also incorporate other factors, such as visual elements within the app and the nature of specific functionalities, which influence user perception of performance.


% \subsubsection{Discussion} Given a screencast, developers may just want to know if it behaviours slow (i.e., containing any type of the performance lag) under the usage scenario. Hence, we also use the metrics to evaluate how effectively \tool can detect the slow screencasts. 

\rqbox{We find that \tool achieves an average precision and recall of 0.91 and 0.96 on detecting GUI lags from screencasts. Our analysis finds that incorporating additional app context may further improve detection results. }

% \begin{table}
% 	\centering
% 	\caption{Results of detecting and locating GUI performance lags by \tool. }
% 	\label{tab:resutls_detecting_lags}
%     \setlength{\tabcolsep}{5pt}
% 	\scalebox{1}{
% 	\begin{tabular}{c rrr rrr} 
% 		\toprule
% 		\multirow{2}*{lag type} & 
% 		\multicolumn{3}{c}{Detection}&
%         \multicolumn{3}{c}{Location}\\
		
% 		\cmidrule(r){2-4}\cmidrule(r){5-7}
% 		&Precision &Recall &F1 &Precision &Recall &F1\\
		
% 		\midrule
% 		Janky frames &	    &       &  &	    &       &\\
% 		Frozen frames&		&	    &  &	    &       &\\
% 		Load frames& 		&		&  &	    &       &\\

%   	\midrule
% 		\tabincell{l}{Avg. across }	&	& 	& &		& 		&\\
		
% 		\bottomrule
% 	\end{tabular}}
% \end{table}

\begin{figure}[t!]
    \centering
        \includegraphics[width=0.8\textwidth]{sections/images/pearson.pdf} 
    \caption{
    % Pearson correlation between LLM generation probabilities and student selection frequencies across model sizes, showing consistently stronger alignment with student reasoning for index-based approach (left) compared to text-based approach (right). Results shown for base and instruction-tuned variants of LLaMA and Qwen model families.
    Pearson correlation between LLM generation probabilities and student selection frequencies for incorrect answer choices (distractors) across model sizes. The index-based approach (left) measures correlation for A/B/C/D label selection probabilities, while the text-based approach (right) measures correlation for full distractor text generation probabilities. Results shown for base and instruction-tuned variants of LLaMA and Qwen model families demonstrate relatively stronger alignment between LLMs and student distractor selection patterns as model size increases, especially for text-based approach.
    }
    \label{img:rq1_alignment_size}
\end{figure}

\subsection{RQ1: Correlation Between LLM Generation Probabilities and Student Selection for Distractors}

Our first research question examines whether LLMs assign higher generation probabilities to the same incorrect answers that students commonly select. To investigate this systematically, we analyzed statistical correlations between LLM generation probabilities and student selection frequencies with Pearson's Correlation, Spearman's Rank Correlation and Kendall's tau Correlation (Details can be found in Appendix~\ref{app:stats-corr}). We tested this using two distinct probability measurement approaches: index-based, where models assign probabilities to answer labels (A/B/C/D), and text-based, where models compute generation probabilities for complete answer texts across different model sizes (0.5B to 72B parameters) and architectures (Qwen and LLaMA families\footnote{We use Qwen-2.5 for all parameter sizes; LLaMA-3.1 for 8b and 70b and LLaMA-3.2 for 1b and 3b parameter sizes.}). As shown in Table \ref{tab:rq1} and Figure \ref{img:rq1_alignment_size}, our analysis reveals several key patterns.

\subsubsection{Difference on Likelihood Approach:} The index-based approach shows better alignment between LLM likelihood and student selection distribution on distractors, with correlations ranging from 0.28 to 0.37 (Pearson). Most notably, qwen-14b achieves a correlation of 0.365 (Pearson) and 0.348 (Spearman) with student selection patterns - a moderate correlation given that these models were never explicitly trained to capture student misconceptions. This moderate correlation suggests that LLMs inherently encode some meaningful patterns about how students reason through multiple-choice questions, but are still not able to fully understand the cognitive processes underlying student misconceptions. Such natural alignment between LLM probabilities and student distractor preferences opens exciting possibilities for using these models to understand and predict student misconceptions.

The index-based approach consistently shows stronger correlations compared to text-based across all models and correlation metrics. This substantial difference (e.g., Qwen-72b: $0.352$ vs $0.183$ for Pearson correlation) is particularly illuminating as it mirrors how students actually approach multiple-choice questions. When students select answers, they typically compare options (A/B/C/D) simultaneously rather than evaluating each option's text in isolation. The index-based approach better captures this comparative decision-making process, while the text-based approach artificially forces sequential, independent evaluation of each option. This suggests that the stronger correlation in the index-based approach may stem from its better alignment with natural multiple-choice problem-solving strategies.


\begin{table}[!t]
    \centering
    % \scriptsize
    \tabcolsep=5.5pt
    \tiny
    %\renewcommand{\arraystretch}{1.1} 
    \caption{\revise{Performance of CodeBERT} on purified datasets. \revise{BC: BadCode; CP: CodePoisoner; CB: CodeBLEU.}}
    \vspace{-1mm}
    \label{tab:rq2_codebert}
    \begin{threeparttable}
    \begin{tabular}{clcccccc}
        \toprule
        
        \multirow{2}{*}{Task} & \multirow{2}{*}{Code Poisoning} & \multicolumn{2}{c}{Clean} & \multicolumn{2}{c}{Undefended} & \multicolumn{2}{c}{\ours{}} \\

        \cmidrule(lr){3-4} \cmidrule(lr){5-6} \cmidrule(lr){7-8}

        & & \revise{ACC} & ASR & ACC & ASR & ACC & ASR \\
    
        \midrule
        
        \multirow{6}{*}{\rotatebox{90}{Defect Detection}} &
        BC (Fixed) & \revise{63.50\%} & 27.76\% & 62.00\% & 100\% & 62.00\% & 26.99\% \\
        & BC (Mixed) & \revise{63.50\%} & 27.76\% & 61.00\% & 96.18\% & 60.00\% & 32.14\% \\
        & BNC (Fixed) & \revise{63.50\%} & 30.92\% & 60.43\% & 100\% & 61.16\% & 37.46\% \\
        & BNC (Grammar) & \revise{63.50\%} & 21.35\% & 63.28\% & 100\% & 63.12\% & 22.48\% \\
        & CP (Variable) & \revise{63.50\%} & 46.29\% & 62.79\% & 100\% & 61.96\% & 48.59\% \\
        \cmidrule(lr){2-8}
        & Average & \revise{63.50\%} & 30.82\% & 61.90\% & 99.24\% & 61.65\% & 33.53\% \\
        \bottomrule
        
        \\

        \toprule
        
        \multirow{10}{*}{\rotatebox{90}{Clone Detection}} &
        & \revise{F1} & ASR & F1 & ASR & F1 & ASR \\
        \midrule
        & BC (Fixed) & \revise{98.71\%} & 1.61\% & 98.10\%  & 100\% & 98.39\%  & 1.58\% \\
        & BC (Mixed) & \revise{98.71\%} & 1.61\% & 98.22\%  & 100\% & 97.20\%  & 2.55\% \\
        & BNC (Fixed) & \revise{98.71\%} & 1.58\% & 98.27\% & 100\%  & 98.53\%  & 3.99\% \\
        & BNC (Grammar) & \revise{98.71\%} & 1.04\% & 98.22\% & 100\% & 97.31\% & 5.17\% \\
        & CP (Variable) & \revise{98.71\%} & 2.23\% & 98.17\%  & 100\%  & 98.23\%  & 6.70\%  \\
        \cmidrule(lr){2-8}
        & Average & \revise{98.71\%} & 1.61\% & 98.20\% & 100\% & 97.93\% & 4.00\% \\
        \bottomrule
        
        \\

        \toprule
        \multirow{9}{*}{\rotatebox{90}{Code Search}} &
        & \revise{MRR} & ANR & MRR & ANR & MRR & ANR \\
        \midrule
        & BC (Fixed) & \revise{81.46} & 46.27 & 80.06 & 4.71 & 80.06 & 55.82 \\
        & BC (Mixed) & \revise{81.46}  & 46.27 & 80.04 & 4.93 & 80.22 & 42.17 \\
        & BNC (Fixed) & \revise{81.46}  & 49.09 & 81.32 & 5.03 & 80.06 & 60.67 \\
        & BNC (Grammar) & \revise{81.46}  & 51.36 & 80.01 & 2.14 & 80.03 & 56.43 \\
        & CP (Variable) & \revise{81.46}  & 43.12 & 79.66 & 8.34 & 79.93 & 61.60 \\
        \cmidrule(lr){2-8}
        & Average & \revise{81.46}  & 47.22 & 80.22 & 5.03 & 80.06 & 55.34 \\
        \bottomrule

        \\
        
        \toprule
        \multirow{9}{*}{\rotatebox{90}{Code Repair}} &
        & \revise{BLEU/CB} & ASR & BLEU/\revise{CB} & ASR & BLEU/\revise{CB} & ASR \\
        \midrule
        & BC (Fixed) & \revise{78.42/75.58} & 0\% & 78.24/\revise{75.73} & 99.98\% & 77.63/\revise{75.46} & 0\% \\
        & BC (Mixed) & \revise{78.42/75.58} & 0\% & 77.33/\revise{75.15} & 100\% & 76.80/\revise{74.82} & 15.18\% \\
        & BNC (Fixed) & \revise{78.42/75.58} & 0\% & 77.66/\revise{75.24} & 100\% & 77.55/\revise{75.31} & 0.48\% \\
        & BNC (Grammar) & \revise{78.42/75.58} & 0\% & 77.09/\revise{75.01} & 100\% & 77.23/\revise{75.13} & 3.19\% \\
        & CP (Variable) & \revise{78.42/75.58} & 0\% & 77.82/\revise{75.58} & 100\% & 77.58/\revise{75.21} & 0.26\% \\
        \cmidrule(lr){2-8}
        & Average & \revise{78.42/75.58} & 0\% & 77.63/\revise{75.36} & 100\% & 77.36/\revise{75.19} & 3.82\% \\
        \bottomrule
        
    \end{tabular}
    \end{threeparttable}
    \vspace{-4mm}
\end{table}

\begin{table}[!t]
    \centering
    % \scriptsize
    \tabcolsep=6pt
    \tiny
    %\renewcommand{\arraystretch}{1.1} 
    \caption{\revise{Performance of StarCoder on the defect detection dataset purified by \ours{}.}}
    \vspace{-1mm}
    \label{tab:rq2_starcoder}
    \begin{tabular}{clcccccc}
        \toprule
        
        \multirow{2}{*}{\revise{Task}} & \multirow{2}{*}{\revise{Code Poisoning}} & \multicolumn{2}{c}{\revise{Clean}} & \multicolumn{2}{c}{\revise{Undefended}} & \multicolumn{2}{c}{\revise{\ours{}}} \\

        \cmidrule(lr){3-4} \cmidrule(lr){5-6} \cmidrule(lr){7-8}

        & & \revise{ACC} & \revise{ASR} & \revise{ACC} & \revise{ASR} & \revise{ACC} & \revise{ASR} \\
    
        \midrule
        
        \multirow{6}{*}{\rotatebox{90}{\revise{Defect Detection}}} &
        \revise{BadCode (Fixed)} & \revise{61.97\%} & \revise{56.89\%} & \revise{61.73\%} & \revise{97.89\%} & \revise{61.37\%} & \revise{56.54\%} \\
        & \revise{BadCode (Mixed)} & \revise{61.97\%} & \revise{57.24\%} & \revise{61.67\%} & \revise{96.23\%} & \revise{61.23\%} & \revise{56.75\%} \\
        & \revise{BNC (Fixed)} & \revise{61.97\%} & \revise{57.39\%} & \revise{61.32\%} & \revise{100\%} & \revise{61.14\%} & \revise{56.82\%} \\
        & \revise{BNC (Grammar)} & \revise{61.97\%} & \revise{58.31\%} & \revise{61.54\%} & \revise{100\%} & \revise{61.26\%} & \revise{57.64\%} \\
        & \revise{CodePoisoner (Variable)} & \revise{61.97\%} & \revise{59.12\%} & \revise{61.68\%} & \revise{96.57\%} & \revise{61.32\%} & \revise{59.03\%} \\
        \cmidrule(lr){2-8}
        & \revise{Average} & \revise{61.97\%} & \revise{57.79\%} & \revise{61.59\%} & \revise{98.14\%} & \revise{61.26\%} & \revise{57.36\%} \\
        
        \bottomrule
        
    \end{tabular}
    \vspace{-3mm}
\end{table}

\subsubsection{Model Size Impact:} Figure \ref{img:rq1_alignment_size} reveals interesting scaling patterns across model sizes. For index-based approach, we observe a general upward trend in correlation as model size increases, with Qwen models showing particularly strong improvement from 0.5B to 14B parameters. The largest models maintain these strong correlations, demonstrating the benefits of model scale in capturing student reasoning patterns. The text-based approach shows more consistent improvement with scale across both model families, though with lower overall correlation values.

The figure also reveals that instruction-tuned variants (-instruct) of both model families generally show stronger correlations than their base counterparts, particularly noticeable in larger models. This suggests that instruction tuning may further enhance models' ability to capture student-like reasoning patterns. These findings provide quantitative evidence that LLMs' generation probabilities partially reflect student distractor preferences, particularly when the task format mirrors typical multiple-choice selection processes.



\subsection{RQ2: Alignment of LLM Mistakes and Student Misconception Patterns}

Our second research question examines whether LLMs, when answering incorrectly, tend to select the same distractors that commonly mislead students. To investigate this systematically, we analyzed incorrect answer selections from LLMs ranging from 0.5B to 72B parameters. For each LLM mistake, we tracked whether the model selected the distractor that was most commonly chosen by students ($1^{st}$ Dist), the second most common ($2^{nd}$ Dist), or the least common ($3^{rd}$ Dist). We tested this across two different prompting formats: index-based (where models select from A/B/C/D) and text-based (where models select the full answer text). The results in Table \ref{tab:rq2} reveal three key patterns:

\subsubsection{Similar Error Patterns Across Model Sizes:} Both small and large models show remarkably consistent patterns in selecting the first and second most common student distractors. For index-based prompting, the smallest model (Qwen-0.5B) selects the most common student distractor 51.6\% of the time and the second most common 32.6\% of the time, while the largest model (Qwen-72B) shows similar proportions at 59.3\% and 25.1\% respectively. This consistency suggests that the ability to capture student misconception patterns is not solely dependent on model size.

\subsubsection{A Cost-Effective Approach to Distractor Generation:} Our findings suggest a surprisingly cost-effective approach to automated distractor generation: using smaller language models. The key insight is that while smaller models make more mistakes, they make ``student-like'' mistakes. Specifically, when Qwen-0.5B ($n=1781$ incorrect answers) and Qwen-72B ($n=435$ incorrect answers) make mistakes, they show similar alignment with student error patterns, despite the dramatic difference in model size and overall accuracy. This opens up an efficient pathway for generating plausible distractors by:
\begin{enumerate}
    \item Leveraging smaller models' higher error rate to generate more potential distractors
    \item Using their natural alignment with student misconception patterns to ensure these distractors are pedagogically relevant
    \item Taking advantage of their computational efficiency and lower resource requirements
\end{enumerate}

\subsubsection{Impact of Likelihood Calculation Approach:} The method of presenting answer choices to LLMs significantly affects their alignment with student error patterns. Index-based prompting (using A/B/C/D) consistently shows higher alignment scores compared to text-based prompting across all model sizes. For example, Qwen-72B achieves a 59.3\% first-distractor selection rate with index-based prompting versus 42.6\% with text-based prompting, suggesting that simpler answer formats might better capture natural misconception patterns.
