\documentclass{article}

\usepackage{primearxiv}

\usepackage[utf8]{inputenc} % allow utf-8 input
\usepackage[T1]{fontenc}    % use 8-bit T1 fonts
\usepackage{hyperref}       % hyperlinks
\usepackage{url}            % simple URL typesetting
\usepackage{booktabs}       % professional-quality tables
\usepackage{amsfonts}       % blackboard math symbols
\usepackage{nicefrac}       % compact symbols for 1/2, etc.
\usepackage{microtype}      % microtypography
\usepackage{lipsum}
\usepackage{fancyhdr}       % header
\usepackage{graphicx}       % graphics

\usepackage{xcolor}
\usepackage{todonotes}
\usepackage{amsmath}
\usepackage{amsthm}
\usepackage{adjustbox}


\usepackage{subcaption}
\usepackage{caption} 
\captionsetup[table]{skip=10pt}
\setlength{\tabcolsep}{18pt}
\renewcommand{\arraystretch}{1.2}
\setlength{\arrayrulewidth}{0.2mm}

\usepackage{array}
\usepackage{colortbl}

\newcommand{\red}[1]{\textcolor{red}{#1}}
\newcommand{\tb}{\textbullet\,}

\newcommand\mc{\mathcal}
\newcommand\mb{\mathbf}

\usepackage{listings}
\usepackage{xcolor}       
\usepackage{multirow}
\usepackage{tcolorbox}
\usepackage{tikz}
\usetikzlibrary{decorations.pathreplacing}


\colorlet{lightyellow}{yellow!40}

\lstset{
  basicstyle=\ttfamily,
  breaklines=true,
  columns=fullflexible,
  mathescape,
  literate={``}{\textquotedblleft}1,
}

\makeatletter
{\small % Capture font definitions of \small
\xdef\f@size@small{\f@size}
\xdef\f@baselineskip@small{\f@baselineskip}
\normalsize % Capture font definitions for \normalsize
\xdef\f@size@normalsize{\f@size}
\xdef\f@baselineskip@normalsize{\f@baselineskip}
}
% Define new \smalltonormalsize font size
\newcommand{\smalltonormalsize}{%
  \fontsize
    {\fpeval{(\f@size@small+\f@size@normalsize)/2}}
    {\fpeval{(\f@baselineskip@small+\f@baselineskip@normalsize)/2}}%
  \selectfont
}
\makeatother

\usepackage{amsmath, amssymb, amsthm}
\usepackage{mathtools} % Enhances amsmath
\newtheorem{theorem}{Theorem}
\theoremstyle{remark}
\newtheorem*{remark}{Remark}
\newtheorem{definition}{Definition}
\newtheorem{corollary}{Corollary}
\newtheorem{hypothesis}{Hypothesis}
\DeclareMathOperator*{\argmax}{arg\,max}
\DeclareMathOperator*{\argmin}{arg\,min}

%Header
\pagestyle{fancy}
\thispagestyle{empty}
\rhead{ \textit{ }} 

% Update your Headers here
\fancyhead[LO]{The Imitation Game for Educational AI}
% \fancyhead[RE]{Firstauthor and Secondauthor} % Firstauthor et al. if more than 2 - must use \documentclass[twoside]{article}
  
%% Title
\title{Do LLMs Make Mistakes Like Students? \\ Exploring Natural Alignment between \\ Language Models and Human Error Patterns}

\author{
  Naiming Liu, \, Shashank Sonkar, \, Richard G. Baraniuk \\
  Rice University \\
  Houston, TX \\
  \texttt{nl35@rice.edu} \\
}



\begin{document}
\maketitle

\begin{abstract}
\begin{abstract}  
Test time scaling is currently one of the most active research areas that shows promise after training time scaling has reached its limits.
Deep-thinking (DT) models are a class of recurrent models that can perform easy-to-hard generalization by assigning more compute to harder test samples.
However, due to their inability to determine the complexity of a test sample, DT models have to use a large amount of computation for both easy and hard test samples.
Excessive test time computation is wasteful and can cause the ``overthinking'' problem where more test time computation leads to worse results.
In this paper, we introduce a test time training method for determining the optimal amount of computation needed for each sample during test time.
We also propose Conv-LiGRU, a novel recurrent architecture for efficient and robust visual reasoning. 
Extensive experiments demonstrate that Conv-LiGRU is more stable than DT, effectively mitigates the ``overthinking'' phenomenon, and achieves superior accuracy.
\end{abstract}  
\end{abstract}

% \keywords{ Cognitive Student Modeling \and AI-human Alignment \and Student Misconceptions \and Large Language Models}

\section{Introduction}
\section{Introduction}


\begin{figure}[t]
\centering
\includegraphics[width=0.6\columnwidth]{figures/evaluation_desiderata_V5.pdf}
\vspace{-0.5cm}
\caption{\systemName is a platform for conducting realistic evaluations of code LLMs, collecting human preferences of coding models with real users, real tasks, and in realistic environments, aimed at addressing the limitations of existing evaluations.
}
\label{fig:motivation}
\end{figure}

\begin{figure*}[t]
\centering
\includegraphics[width=\textwidth]{figures/system_design_v2.png}
\caption{We introduce \systemName, a VSCode extension to collect human preferences of code directly in a developer's IDE. \systemName enables developers to use code completions from various models. The system comprises a) the interface in the user's IDE which presents paired completions to users (left), b) a sampling strategy that picks model pairs to reduce latency (right, top), and c) a prompting scheme that allows diverse LLMs to perform code completions with high fidelity.
Users can select between the top completion (green box) using \texttt{tab} or the bottom completion (blue box) using \texttt{shift+tab}.}
\label{fig:overview}
\end{figure*}

As model capabilities improve, large language models (LLMs) are increasingly integrated into user environments and workflows.
For example, software developers code with AI in integrated developer environments (IDEs)~\citep{peng2023impact}, doctors rely on notes generated through ambient listening~\citep{oberst2024science}, and lawyers consider case evidence identified by electronic discovery systems~\citep{yang2024beyond}.
Increasing deployment of models in productivity tools demands evaluation that more closely reflects real-world circumstances~\citep{hutchinson2022evaluation, saxon2024benchmarks, kapoor2024ai}.
While newer benchmarks and live platforms incorporate human feedback to capture real-world usage, they almost exclusively focus on evaluating LLMs in chat conversations~\citep{zheng2023judging,dubois2023alpacafarm,chiang2024chatbot, kirk2024the}.
Model evaluation must move beyond chat-based interactions and into specialized user environments.



 

In this work, we focus on evaluating LLM-based coding assistants. 
Despite the popularity of these tools---millions of developers use Github Copilot~\citep{Copilot}---existing
evaluations of the coding capabilities of new models exhibit multiple limitations (Figure~\ref{fig:motivation}, bottom).
Traditional ML benchmarks evaluate LLM capabilities by measuring how well a model can complete static, interview-style coding tasks~\citep{chen2021evaluating,austin2021program,jain2024livecodebench, white2024livebench} and lack \emph{real users}. 
User studies recruit real users to evaluate the effectiveness of LLMs as coding assistants, but are often limited to simple programming tasks as opposed to \emph{real tasks}~\citep{vaithilingam2022expectation,ross2023programmer, mozannar2024realhumaneval}.
Recent efforts to collect human feedback such as Chatbot Arena~\citep{chiang2024chatbot} are still removed from a \emph{realistic environment}, resulting in users and data that deviate from typical software development processes.
We introduce \systemName to address these limitations (Figure~\ref{fig:motivation}, top), and we describe our three main contributions below.


\textbf{We deploy \systemName in-the-wild to collect human preferences on code.} 
\systemName is a Visual Studio Code extension, collecting preferences directly in a developer's IDE within their actual workflow (Figure~\ref{fig:overview}).
\systemName provides developers with code completions, akin to the type of support provided by Github Copilot~\citep{Copilot}. 
Over the past 3 months, \systemName has served over~\completions suggestions from 10 state-of-the-art LLMs, 
gathering \sampleCount~votes from \userCount~users.
To collect user preferences,
\systemName presents a novel interface that shows users paired code completions from two different LLMs, which are determined based on a sampling strategy that aims to 
mitigate latency while preserving coverage across model comparisons.
Additionally, we devise a prompting scheme that allows a diverse set of models to perform code completions with high fidelity.
See Section~\ref{sec:system} and Section~\ref{sec:deployment} for details about system design and deployment respectively.



\textbf{We construct a leaderboard of user preferences and find notable differences from existing static benchmarks and human preference leaderboards.}
In general, we observe that smaller models seem to overperform in static benchmarks compared to our leaderboard, while performance among larger models is mixed (Section~\ref{sec:leaderboard_calculation}).
We attribute these differences to the fact that \systemName is exposed to users and tasks that differ drastically from code evaluations in the past. 
Our data spans 103 programming languages and 24 natural languages as well as a variety of real-world applications and code structures, while static benchmarks tend to focus on a specific programming and natural language and task (e.g. coding competition problems).
Additionally, while all of \systemName interactions contain code contexts and the majority involve infilling tasks, a much smaller fraction of Chatbot Arena's coding tasks contain code context, with infilling tasks appearing even more rarely. 
We analyze our data in depth in Section~\ref{subsec:comparison}.



\textbf{We derive new insights into user preferences of code by analyzing \systemName's diverse and distinct data distribution.}
We compare user preferences across different stratifications of input data (e.g., common versus rare languages) and observe which affect observed preferences most (Section~\ref{sec:analysis}).
For example, while user preferences stay relatively consistent across various programming languages, they differ drastically between different task categories (e.g. frontend/backend versus algorithm design).
We also observe variations in user preference due to different features related to code structure 
(e.g., context length and completion patterns).
We open-source \systemName and release a curated subset of code contexts.
Altogether, our results highlight the necessity of model evaluation in realistic and domain-specific settings.







\section{Related Works}
\putsec{related}{Related Work}

\noindent \textbf{Efficient Radiance Field Rendering.}
%
The introduction of Neural Radiance Fields (NeRF)~\cite{mil:sri20} has
generated significant interest in efficient 3D scene representation and
rendering for radiance fields.
%
Over the past years, there has been a large amount of research aimed at
accelerating NeRFs through algorithmic or software
optimizations~\cite{mul:eva22,fri:yu22,che:fun23,sun:sun22}, and the
development of hardware
accelerators~\cite{lee:cho23,li:li23,son:wen23,mub:kan23,fen:liu24}.
%
The state-of-the-art method, 3D Gaussian splatting~\cite{ker:kop23}, has
further fueled interest in accelerating radiance field
rendering~\cite{rad:ste24,lee:lee24,nie:stu24,lee:rho24,ham:mel24} as it
employs rasterization primitives that can be rendered much faster than NeRFs.
%
However, previous research focused on software graphics rendering on
programmable cores or building dedicated hardware accelerators. In contrast,
\name{} investigates the potential of efficient radiance field rendering while
utilizing fixed-function units in graphics hardware.
%
To our knowledge, this is the first work that assesses the performance
implications of rendering Gaussian-based radiance fields on the hardware
graphics pipeline with software and hardware optimizations.

%%%%%%%%%%%%%%%%%%%%%%%%%%%%%%%%%%%%%%%%%%%%%%%%%%%%%%%%%%%%%%%%%%%%%%%%%%
\myparagraph{Enhancing Graphics Rendering Hardware.}
%
The performance advantage of executing graphics rendering on either
programmable shader cores or fixed-function units varies depending on the
rendering methods and hardware designs.
%
Previous studies have explored the performance implication of graphics hardware
design by developing simulation infrastructures for graphics
workloads~\cite{bar:gon06,gub:aam19,tin:sax23,arn:par13}.
%
Additionally, several studies have aimed to improve the performance of
special-purpose hardware such as ray tracing units in graphics
hardware~\cite{cho:now23,liu:cha21} and proposed hardware accelerators for
graphics applications~\cite{lu:hua17,ram:gri09}.
%
In contrast to these works, which primarily evaluate traditional graphics
workloads, our work focuses on improving the performance of volume rendering
workloads, such as Gaussian splatting, which require blending a huge number of
fragments per pixel.

%%%%%%%%%%%%%%%%%%%%%%%%%%%%%%%%%%%%%%%%%%%%%%%%%%%%%%%%%%%%%%%%%%%%%%%%%%
%
In the context of multi-sample anti-aliasing, prior work proposed reducing the
amount of redundant shading by merging fragments from adjacent triangles in a
mesh at the quad granularity~\cite{fat:bou10}.
%
While both our work and quad-fragment merging (QFM)~\cite{fat:bou10} aim to
reduce operations by merging quads, our proposed technique differs from QFM in
many aspects.
%
Our method aims to blend \emph{overlapping primitives} along the depth
direction and applies to quads from any primitive. In contrast, QFM merges quad
fragments from small (e.g., pixel-sized) triangles that \emph{share} an edge
(i.e., \emph{connected}, \emph{non-overlapping} triangles).
%
As such, QFM is not applicable to the scenes consisting of a number of
unconnected transparent triangles, such as those in 3D Gaussian splatting.
%
In addition, our method computes the \emph{exact} color for each pixel by
offloading blending operations from ROPs to shader units, whereas QFM
\emph{approximates} pixel colors by using the color from one triangle when
multiple triangles are merged into a single quad.




\section{Dual Analysis Framework for LLM-Student Misconception Alignment}
To investigate whether LLMs naturally capture student misconception patterns, we analyze their behavior on multiple-choice questions and compare it against actual student response data. Our investigation focuses on two key aspects: (1). whether LLMs assign higher probabilities to incorrect answers that commonly mislead students, and (2). whether LLMs tend to select the same wrong answers that students frequently choose. To enable this analysis, we develop a systematic framework as follows:


\subsection{Preliminaries}

Let $\mathcal{Q} = \{q_1, q_2, \ldots, q_n\}$ denote our collection of $n$ multiple-choice questions. For each question $q_i$, we define:

\begin{itemize}
    \item A set of $m$ answer options $\mathcal{A}_i = \{a_{i1}, a_{i2}, \cdots, a_{im}\}$
    \item The correct answer index $c_i \in \{1, 2, \cdots, m\}$
    \item The empirical student response distribution $\mathcal{S}_i = \{s_{i1}, s_{i2}, \cdots , s_{im}\}$, where $s_{ij}$ represents the proportion of students selecting option $j$ for question $i$ such that $\sum_{j=1}^m s_{ij} = 1$
\end{itemize}

This formalization enables our analysis of the alignment between LLM behavior and student misconception patterns through both probabilistic correlation measures and direct incorrect answer comparisons.

\subsection{Likelihood Calculation of LLM's Answer Choice}

We quantify an LLM's preference for each answer choice using two formatting approaches implemented through the EleutherAI Harness evaluation framework~\cite{eval-harness}.


\subsubsection{Index-based Approach:}
In the index-based approach, we present the model with both the question and all answer choices in a structured format, where each answer choice is assigned a corresponding letter index (A, B, C, ...). The model then predicts a single letter choice representing the answer choice. For choice index $j \in \{A, B, C, \cdots\}$ of question $q_i$, we compute the log-likelihood as:

$$\mathcal{L}_{a_{ij}}^{\text{index}} = \log P(j \mid q_i, \mathcal{A}_i; \theta)$$

where $\theta$ represents the model parameters.

\subsubsection{Text-based Approach:}
In the text-based approach, we evaluate each answer choice independently by computing its likelihood when paired with only the question as input. For each choice $a_{ij}$ from question $q_i$, we compute the log-likelihood by summing over all tokens in the answer text and take an average to avoid giving long answers an unfair advantage:

$$\mathcal{L}_{a_{ij}}^{\text{text}} = \frac{\sum_{t=1}^{T_{ij}} \log P(x_t^{(ij)} \mid q_i, x_{<t}^{(ij)}; \theta)}{T_{ij}}$$

where $x_t^{(ij)}$ is the $t$-th token in choice $a_{ij}$; $T_{ij}$ is the number of tokens in choice $a_ij$ and $x_{<t}^{(ij)}$ represents all preceding tokens in the answer. The summation and average is necessary here because we measure the likelihood of the entire answer text sequence, token by token. 

The two approaches differ in their computation of answer probabilities. Index-based approach computes the conditional probability of each index given the complete context of all options, while text-based approach evaluates the likelihood of each answer text independently through token-wise probability estimation.

\subsubsection{Probability Normalization:}
For both approaches, we convert log-likelihoods to probabilities using the softmax function:

$$P_{ij}(a_{ij} \mid q_i) = \frac{\exp(\mathcal{L}_{a_{ij}})}{\sum_{j=1}^{n} \exp(\mathcal{L}_{a_{ij}})},$$

where $\mathcal{L}_{a_{ij}}$ represents either $\mathcal{L}_{a_{ij}}^{\text{index}}$ or $\mathcal{L}_{a_{ij}}^{\text{text}}$ depending on the approach used.

\subsection{Analysis Framework}

Our investigation employs two approaches to understand how LLMs align with student misconceptions and answer the proposed research questions.

\subsubsection{RQ1: Correlation Between LLM Generation Probabilities and Student Selection Patterns for Distractors:}

For each question-option pair, we compute the LLM's predicted likelihood distribution $P_{ij}$ and obtain student selection probability $s_{ij}$ for each option $a_{ij}$ over all answers choices $\mathcal{A}_i$ with $c_i$ being the correct answer. Then, we adopt statistical methods to compute the correlation coefficient $\rho_i$ between the language model's generation likelihood and student selection frequencies:

\begin{equation}
    \rho_i = \text{corr}(\{P_{ij, j \neq c_i}\}, \{s_{ij, j \neq c_i}\})
\end{equation}

By comparing the LLM's likelihood patterns with actual student selection rates, we can assess whether the model's confidence in distractors aligns with the cognitive misconceptions that commonly mislead students. This alignment, if present, would suggest that LLMs may naturally capture aspects of human cognitive biases and misconceptions that lead to systematic errors in problem-solving.

Our analysis specifically excludes correct answers $(j \neq c_i)$ to isolate these misconception patterns. By focusing only on incorrect options, we can better examine the distribution of plausible but incorrect reasoning, rather than having our analysis confounded by the model's ability to identify correct answers. Higher likelihood scores indicate higher model confidence in an answer choice, allowing us to quantify which incorrect options the model finds most plausible and compare this directly with student error patterns.


\subsubsection{RQ2: Alignment of LLM Mistakes and Student Misconception Patterns:}
Our second analysis examines whether LLM errors align with the most common student misconceptions. Specifically, we investigate if an LLM, when selecting an incorrect answer, chooses the same incorrect option that most students select. This behavioral analysis complements our perplexity-based approach by examining the model's actual answer selections rather than its probability distributions. The analysis proceeds as follows:

\begin{enumerate}
    \item For each question $q_i$, we record the LLM's selected choice option $o_i$ and define an error indicator:
    \begin{equation}
        e_i = \begin{cases}
            1 & \text{if } o_i \neq c_i \\
            0 & \text{otherwise}
        \end{cases}
    \end{equation}

    \item For cases where $e_i = 1$, we compute an alignment score:
    \begin{equation}
        \alpha_i = \frac{s_{i,o_i}}{\max\limits_{j \neq c_i} s_{ij}}
    \end{equation}
\end{enumerate}

The alignment score $\alpha_i$ quantifies how well the LLM's incorrect answers correspond to the most common student misconceptions. A score of $\alpha_i = 1$ indicates perfect alignment, meaning the LLM selected the same incorrect option that students most frequently chose when making mistakes.
To illustrate this measure, consider a multiple-choice question where students' selection rates for incorrect options are distributed as follows: Option B (30\%), Option C (20\%), and Option D (15\%). In this case, Option B represents the most common student misconception. If an LLM makes an error by selecting Option C, the alignment score would be calculated as $\alpha = \frac{20\%}{30\%} = 0.67$. This score reflects that while the LLM did select a relatively common incorrect answer, it did not align with the predominant student misconception (Option B). Lower alignment scores indicate greater divergence between the LLM's error patterns and the predominant student misconceptions. This metric provides valuable insights into whether LLMs naturally capture the cognitive biases that lead students toward specific incorrect answers, helping us understand the extent to which these models might reflect human-like patterns of misunderstanding. The mean alignment score across all questions is defined as:

$$\bar{\alpha} = \frac{\sum_{i=1}^N \alpha_i e_i}{\sum_{i=1}^N e_i}$$


\section{Experiments and Results}
\section{Experiments}
\label{sec:experiments}
The experiments are designed to address two key research questions.
First, \textbf{RQ1} evaluates whether the average $L_2$-norm of the counterfactual perturbation vectors ($\overline{||\perturb||}$) decreases as the model overfits the data, thereby providing further empirical validation for our hypothesis.
Second, \textbf{RQ2} evaluates the ability of the proposed counterfactual regularized loss, as defined in (\ref{eq:regularized_loss2}), to mitigate overfitting when compared to existing regularization techniques.

% The experiments are designed to address three key research questions. First, \textbf{RQ1} investigates whether the mean perturbation vector norm decreases as the model overfits the data, aiming to further validate our intuition. Second, \textbf{RQ2} explores whether the mean perturbation vector norm can be effectively leveraged as a regularization term during training, offering insights into its potential role in mitigating overfitting. Finally, \textbf{RQ3} examines whether our counterfactual regularizer enables the model to achieve superior performance compared to existing regularization methods, thus highlighting its practical advantage.

\subsection{Experimental Setup}
\textbf{\textit{Datasets, Models, and Tasks.}}
The experiments are conducted on three datasets: \textit{Water Potability}~\cite{kadiwal2020waterpotability}, \textit{Phomene}~\cite{phomene}, and \textit{CIFAR-10}~\cite{krizhevsky2009learning}. For \textit{Water Potability} and \textit{Phomene}, we randomly select $80\%$ of the samples for the training set, and the remaining $20\%$ for the test set, \textit{CIFAR-10} comes already split. Furthermore, we consider the following models: Logistic Regression, Multi-Layer Perceptron (MLP) with 100 and 30 neurons on each hidden layer, and PreactResNet-18~\cite{he2016cvecvv} as a Convolutional Neural Network (CNN) architecture.
We focus on binary classification tasks and leave the extension to multiclass scenarios for future work. However, for datasets that are inherently multiclass, we transform the problem into a binary classification task by selecting two classes, aligning with our assumption.

\smallskip
\noindent\textbf{\textit{Evaluation Measures.}} To characterize the degree of overfitting, we use the test loss, as it serves as a reliable indicator of the model's generalization capability to unseen data. Additionally, we evaluate the predictive performance of each model using the test accuracy.

\smallskip
\noindent\textbf{\textit{Baselines.}} We compare CF-Reg with the following regularization techniques: L1 (``Lasso''), L2 (``Ridge''), and Dropout.

\smallskip
\noindent\textbf{\textit{Configurations.}}
For each model, we adopt specific configurations as follows.
\begin{itemize}
\item \textit{Logistic Regression:} To induce overfitting in the model, we artificially increase the dimensionality of the data beyond the number of training samples by applying a polynomial feature expansion. This approach ensures that the model has enough capacity to overfit the training data, allowing us to analyze the impact of our counterfactual regularizer. The degree of the polynomial is chosen as the smallest degree that makes the number of features greater than the number of data.
\item \textit{Neural Networks (MLP and CNN):} To take advantage of the closed-form solution for computing the optimal perturbation vector as defined in (\ref{eq:opt-delta}), we use a local linear approximation of the neural network models. Hence, given an instance $\inst_i$, we consider the (optimal) counterfactual not with respect to $\model$ but with respect to:
\begin{equation}
\label{eq:taylor}
    \model^{lin}(\inst) = \model(\inst_i) + \nabla_{\inst}\model(\inst_i)(\inst - \inst_i),
\end{equation}
where $\model^{lin}$ represents the first-order Taylor approximation of $\model$ at $\inst_i$.
Note that this step is unnecessary for Logistic Regression, as it is inherently a linear model.
\end{itemize}

\smallskip
\noindent \textbf{\textit{Implementation Details.}} We run all experiments on a machine equipped with an AMD Ryzen 9 7900 12-Core Processor and an NVIDIA GeForce RTX 4090 GPU. Our implementation is based on the PyTorch Lightning framework. We use stochastic gradient descent as the optimizer with a learning rate of $\eta = 0.001$ and no weight decay. We use a batch size of $128$. The training and test steps are conducted for $6000$ epochs on the \textit{Water Potability} and \textit{Phoneme} datasets, while for the \textit{CIFAR-10} dataset, they are performed for $200$ epochs.
Finally, the contribution $w_i^{\varepsilon}$ of each training point $\inst_i$ is uniformly set as $w_i^{\varepsilon} = 1~\forall i\in \{1,\ldots,m\}$.

The source code implementation for our experiments is available at the following GitHub repository: \url{https://anonymous.4open.science/r/COCE-80B4/README.md} 

\subsection{RQ1: Counterfactual Perturbation vs. Overfitting}
To address \textbf{RQ1}, we analyze the relationship between the test loss and the average $L_2$-norm of the counterfactual perturbation vectors ($\overline{||\perturb||}$) over training epochs.

In particular, Figure~\ref{fig:delta_loss_epochs} depicts the evolution of $\overline{||\perturb||}$ alongside the test loss for an MLP trained \textit{without} regularization on the \textit{Water Potability} dataset. 
\begin{figure}[ht]
    \centering
    \includegraphics[width=0.85\linewidth]{img/delta_loss_epochs.png}
    \caption{The average counterfactual perturbation vector $\overline{||\perturb||}$ (left $y$-axis) and the cross-entropy test loss (right $y$-axis) over training epochs ($x$-axis) for an MLP trained on the \textit{Water Potability} dataset \textit{without} regularization.}
    \label{fig:delta_loss_epochs}
\end{figure}

The plot shows a clear trend as the model starts to overfit the data (evidenced by an increase in test loss). 
Notably, $\overline{||\perturb||}$ begins to decrease, which aligns with the hypothesis that the average distance to the optimal counterfactual example gets smaller as the model's decision boundary becomes increasingly adherent to the training data.

It is worth noting that this trend is heavily influenced by the choice of the counterfactual generator model. In particular, the relationship between $\overline{||\perturb||}$ and the degree of overfitting may become even more pronounced when leveraging more accurate counterfactual generators. However, these models often come at the cost of higher computational complexity, and their exploration is left to future work.

Nonetheless, we expect that $\overline{||\perturb||}$ will eventually stabilize at a plateau, as the average $L_2$-norm of the optimal counterfactual perturbations cannot vanish to zero.

% Additionally, the choice of employing the score-based counterfactual explanation framework to generate counterfactuals was driven to promote computational efficiency.

% Future enhancements to the framework may involve adopting models capable of generating more precise counterfactuals. While such approaches may yield to performance improvements, they are likely to come at the cost of increased computational complexity.


\subsection{RQ2: Counterfactual Regularization Performance}
To answer \textbf{RQ2}, we evaluate the effectiveness of the proposed counterfactual regularization (CF-Reg) by comparing its performance against existing baselines: unregularized training loss (No-Reg), L1 regularization (L1-Reg), L2 regularization (L2-Reg), and Dropout.
Specifically, for each model and dataset combination, Table~\ref{tab:regularization_comparison} presents the mean value and standard deviation of test accuracy achieved by each method across 5 random initialization. 

The table illustrates that our regularization technique consistently delivers better results than existing methods across all evaluated scenarios, except for one case -- i.e., Logistic Regression on the \textit{Phomene} dataset. 
However, this setting exhibits an unusual pattern, as the highest model accuracy is achieved without any regularization. Even in this case, CF-Reg still surpasses other regularization baselines.

From the results above, we derive the following key insights. First, CF-Reg proves to be effective across various model types, ranging from simple linear models (Logistic Regression) to deep architectures like MLPs and CNNs, and across diverse datasets, including both tabular and image data. 
Second, CF-Reg's strong performance on the \textit{Water} dataset with Logistic Regression suggests that its benefits may be more pronounced when applied to simpler models. However, the unexpected outcome on the \textit{Phoneme} dataset calls for further investigation into this phenomenon.


\begin{table*}[h!]
    \centering
    \caption{Mean value and standard deviation of test accuracy across 5 random initializations for different model, dataset, and regularization method. The best results are highlighted in \textbf{bold}.}
    \label{tab:regularization_comparison}
    \begin{tabular}{|c|c|c|c|c|c|c|}
        \hline
        \textbf{Model} & \textbf{Dataset} & \textbf{No-Reg} & \textbf{L1-Reg} & \textbf{L2-Reg} & \textbf{Dropout} & \textbf{CF-Reg (ours)} \\ \hline
        Logistic Regression   & \textit{Water}   & $0.6595 \pm 0.0038$   & $0.6729 \pm 0.0056$   & $0.6756 \pm 0.0046$  & N/A    & $\mathbf{0.6918 \pm 0.0036}$                     \\ \hline
        MLP   & \textit{Water}   & $0.6756 \pm 0.0042$   & $0.6790 \pm 0.0058$   & $0.6790 \pm 0.0023$  & $0.6750 \pm 0.0036$    & $\mathbf{0.6802 \pm 0.0046}$                    \\ \hline
%        MLP   & \textit{Adult}   & $0.8404 \pm 0.0010$   & $\mathbf{0.8495 \pm 0.0007}$   & $0.8489 \pm 0.0014$  & $\mathbf{0.8495 \pm 0.0016}$     & $0.8449 \pm 0.0019$                    \\ \hline
        Logistic Regression   & \textit{Phomene}   & $\mathbf{0.8148 \pm 0.0020}$   & $0.8041 \pm 0.0028$   & $0.7835 \pm 0.0176$  & N/A    & $0.8098 \pm 0.0055$                     \\ \hline
        MLP   & \textit{Phomene}   & $0.8677 \pm 0.0033$   & $0.8374 \pm 0.0080$   & $0.8673 \pm 0.0045$  & $0.8672 \pm 0.0042$     & $\mathbf{0.8718 \pm 0.0040}$                    \\ \hline
        CNN   & \textit{CIFAR-10} & $0.6670 \pm 0.0233$   & $0.6229 \pm 0.0850$   & $0.7348 \pm 0.0365$   & N/A    & $\mathbf{0.7427 \pm 0.0571}$                     \\ \hline
    \end{tabular}
\end{table*}

\begin{table*}[htb!]
    \centering
    \caption{Hyperparameter configurations utilized for the generation of Table \ref{tab:regularization_comparison}. For our regularization the hyperparameters are reported as $\mathbf{\alpha/\beta}$.}
    \label{tab:performance_parameters}
    \begin{tabular}{|c|c|c|c|c|c|c|}
        \hline
        \textbf{Model} & \textbf{Dataset} & \textbf{No-Reg} & \textbf{L1-Reg} & \textbf{L2-Reg} & \textbf{Dropout} & \textbf{CF-Reg (ours)} \\ \hline
        Logistic Regression   & \textit{Water}   & N/A   & $0.0093$   & $0.6927$  & N/A    & $0.3791/1.0355$                     \\ \hline
        MLP   & \textit{Water}   & N/A   & $0.0007$   & $0.0022$  & $0.0002$    & $0.2567/1.9775$                    \\ \hline
        Logistic Regression   &
        \textit{Phomene}   & N/A   & $0.0097$   & $0.7979$  & N/A    & $0.0571/1.8516$                     \\ \hline
        MLP   & \textit{Phomene}   & N/A   & $0.0007$   & $4.24\cdot10^{-5}$  & $0.0015$    & $0.0516/2.2700$                    \\ \hline
       % MLP   & \textit{Adult}   & N/A   & $0.0018$   & $0.0018$  & $0.0601$     & $0.0764/2.2068$                    \\ \hline
        CNN   & \textit{CIFAR-10} & N/A   & $0.0050$   & $0.0864$ & N/A    & $0.3018/
        2.1502$                     \\ \hline
    \end{tabular}
\end{table*}

\begin{table*}[htb!]
    \centering
    \caption{Mean value and standard deviation of training time across 5 different runs. The reported time (in seconds) corresponds to the generation of each entry in Table \ref{tab:regularization_comparison}. Times are }
    \label{tab:times}
    \begin{tabular}{|c|c|c|c|c|c|c|}
        \hline
        \textbf{Model} & \textbf{Dataset} & \textbf{No-Reg} & \textbf{L1-Reg} & \textbf{L2-Reg} & \textbf{Dropout} & \textbf{CF-Reg (ours)} \\ \hline
        Logistic Regression   & \textit{Water}   & $222.98 \pm 1.07$   & $239.94 \pm 2.59$   & $241.60 \pm 1.88$  & N/A    & $251.50 \pm 1.93$                     \\ \hline
        MLP   & \textit{Water}   & $225.71 \pm 3.85$   & $250.13 \pm 4.44$   & $255.78 \pm 2.38$  & $237.83 \pm 3.45$    & $266.48 \pm 3.46$                    \\ \hline
        Logistic Regression   & \textit{Phomene}   & $266.39 \pm 0.82$ & $367.52 \pm 6.85$   & $361.69 \pm 4.04$  & N/A   & $310.48 \pm 0.76$                    \\ \hline
        MLP   &
        \textit{Phomene} & $335.62 \pm 1.77$   & $390.86 \pm 2.11$   & $393.96 \pm 1.95$ & $363.51 \pm 5.07$    & $403.14 \pm 1.92$                     \\ \hline
       % MLP   & \textit{Adult}   & N/A   & $0.0018$   & $0.0018$  & $0.0601$     & $0.0764/2.2068$                    \\ \hline
        CNN   & \textit{CIFAR-10} & $370.09 \pm 0.18$   & $395.71 \pm 0.55$   & $401.38 \pm 0.16$ & N/A    & $1287.8 \pm 0.26$                     \\ \hline
    \end{tabular}
\end{table*}

\subsection{Feasibility of our Method}
A crucial requirement for any regularization technique is that it should impose minimal impact on the overall training process.
In this respect, CF-Reg introduces an overhead that depends on the time required to find the optimal counterfactual example for each training instance. 
As such, the more sophisticated the counterfactual generator model probed during training the higher would be the time required. However, a more advanced counterfactual generator might provide a more effective regularization. We discuss this trade-off in more details in Section~\ref{sec:discussion}.

Table~\ref{tab:times} presents the average training time ($\pm$ standard deviation) for each model and dataset combination listed in Table~\ref{tab:regularization_comparison}.
We can observe that the higher accuracy achieved by CF-Reg using the score-based counterfactual generator comes with only minimal overhead. However, when applied to deep neural networks with many hidden layers, such as \textit{PreactResNet-18}, the forward derivative computation required for the linearization of the network introduces a more noticeable computational cost, explaining the longer training times in the table.

\subsection{Hyperparameter Sensitivity Analysis}
The proposed counterfactual regularization technique relies on two key hyperparameters: $\alpha$ and $\beta$. The former is intrinsic to the loss formulation defined in (\ref{eq:cf-train}), while the latter is closely tied to the choice of the score-based counterfactual explanation method used.

Figure~\ref{fig:test_alpha_beta} illustrates how the test accuracy of an MLP trained on the \textit{Water Potability} dataset changes for different combinations of $\alpha$ and $\beta$.

\begin{figure}[ht]
    \centering
    \includegraphics[width=0.85\linewidth]{img/test_acc_alpha_beta.png}
    \caption{The test accuracy of an MLP trained on the \textit{Water Potability} dataset, evaluated while varying the weight of our counterfactual regularizer ($\alpha$) for different values of $\beta$.}
    \label{fig:test_alpha_beta}
\end{figure}

We observe that, for a fixed $\beta$, increasing the weight of our counterfactual regularizer ($\alpha$) can slightly improve test accuracy until a sudden drop is noticed for $\alpha > 0.1$.
This behavior was expected, as the impact of our penalty, like any regularization term, can be disruptive if not properly controlled.

Moreover, this finding further demonstrates that our regularization method, CF-Reg, is inherently data-driven. Therefore, it requires specific fine-tuning based on the combination of the model and dataset at hand.

\section{Conclusion}
\section{Conclusion}
In this work, we propose a simple yet effective approach, called SMILE, for graph few-shot learning with fewer tasks. Specifically, we introduce a novel dual-level mixup strategy, including within-task and across-task mixup, for enriching the diversity of nodes within each task and the diversity of tasks. Also, we incorporate the degree-based prior information to learn expressive node embeddings. Theoretically, we prove that SMILE effectively enhances the model's generalization performance. Empirically, we conduct extensive experiments on multiple benchmarks and the results suggest that SMILE significantly outperforms other baselines, including both in-domain and cross-domain few-shot settings.


\section*{Acknowledgments}
This work was supported by NSF grant 1842378, ONR grant N0014-20-1-2534, AFOSR grant FA9550-22-1-0060, a Vannevar Bush Faculty Fellowship, OpenAI, and ONR grant N00014-18-1-2047.

%Bibliography
\bibliographystyle{unsrt}  
\bibliography{custom,dsp}  

\appendix
\section{Statistical Correlation}
\label{app:stats-corr}
For each multiple-choice question $q$ with answer choices $A_q = \{a_1, a_2, \cdots, a_n \}$, we computed the correlation between student responses distribution $P_s(a_i \mid q)$ and language model likelihood $P_m(a_i \mid q)$ using the following statistical methods.

\subsection{Pearson's Correlation}
We use Pearson product-moment correlation~\cite{freedman2007statistics} coefficient $r_q$ to calculate the linear relationships between student performance distribution $P_s(a_i|q)$ and language model likelihood $P_m(a_i|q)$ . For a sample of $n$ questions, the coefficient is defined as:

$$r_q = \frac{\sum_{i=1}^n (P_s(a_i \mid q) - \mu_s)(P_m(a_i \mid q) - \mu_m)}{\sqrt{\sum_{i=1}^n (P_s(a_i \mid q) - \mu_s)^2}\sqrt{\sum_{i=1}^n (log P_m(a_i \mid q) - \mu_m)^2}}$$

where $\mu_s$ and $\mu_m$ denote the sample means. The coefficient $r_q \in [-1, 1]$ provides a normalized measure of linear dependence, where $|r_q| = 1$ indicates perfect linear correlation and $r_q = 0$ indicates no linear correlation. 

\subsection{Spearman's Rank Correlation}
We apply Spearman's rank correlation~\cite{zar2005spearman} coefficient $\rho_q$ to measure the non-linear monotonic relationships by operating on the ranks instead of raw values. Let $\text{rank}(P_s(a_i \mid q))$ and $\text{rank}(P_m(a_i \mid q))$ denote the rankings of the $i$-th answer choice when ordered by student response distribution and model likelihood respectively. The Spearman's rank coefficient is defined as:

$$\rho_q = 1 - \frac{6\sum_{i=1}^n d_i^2}{n(n^2-1)}$$

where $d_i = \text{rank}(P_s(a_i \mid q)) - \text{rank}(P_m(a_i \mid q))$ represents the rank difference for the $i$-th answer choice. The coefficient $\rho_q \in [-1, 1]$, where -1 indicates perfect negative correlation (inverse rankings), +1 indicates perfect positive correlation (identical rankings), and 0 suggests no correlations between the rankings.


\subsection{Kendall's tau Correlation}
We compute Kendall's tau~\cite{kendall1938new} correlation coefficient $\tau_q$ to measure the concordance between model predictions and student responses through pairwise rankings. For any two answer choices $i$ and $j$, a pair is considered concordant if their relative ordering is consistent across both distributions (i.e., if $P_s(a_i \mid q) > P_s(a_j \mid q)$ when $P_m(a_i \mid q) > P_m(a_j \mid q)$, or $P_s(a_i \mid q) < P_s(a_j \mid q)$ when $P_m(a_i \mid q) < P_m(a_j \mid q)$), and discordant if their ordering differs. Kendall's tau correlation is defined as:

$$\tau_q = \frac{2(n_c - n_d)}{n(n-1)}$$

where $n_c$ is the number of concordant pairs, $n_d$ is the number of discordant pairs. The coefficient $\tau_q \in [-1, 1]$, where values closer to ±1 indicate stronger alignment between model likelihoods and student answer distributions (positive for similar rankings, negative for opposite rankings), while 0 indicates no alignment.





\end{document}
