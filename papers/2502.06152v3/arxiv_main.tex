\documentclass{article}

% Language setting
% Replace `english' with e.g. `spanish' to change the document language
\usepackage[english]{babel}
\usepackage{natbib}

% Set page size and margins
% Replace `letterpaper' with `a4paper' for UK/EU standard size
\usepackage[letterpaper,top=2cm,bottom=2cm,left=3cm,right=3cm,marginparwidth=1.75cm]{geometry}

% Useful packages
\usepackage{amsmath}
\usepackage{graphicx}
\usepackage[colorlinks=true, allcolors=blue]{hyperref}
% For theorems and such
\usepackage{amssymb}
\usepackage{mathtools}
\usepackage{amsthm}
\usepackage{algorithmic}
\usepackage{algorithm}
\usepackage{caption}
\usepackage{subcaption}


% if you use cleveref..
\usepackage[capitalize,noabbrev]{cleveref}

%%%%%%%%%%%%%%%%%%%%%%%%%%%%%%%%
% THEOREMS
%%%%%%%%%%%%%%%%%%%%%%%%%%%%%%%%
\theoremstyle{plain}
\newtheorem{theorem}{Theorem}[section]
\newtheorem{proposition}[theorem]{Proposition}
\newtheorem{lemma}[theorem]{Lemma}
\newtheorem{corollary}[theorem]{Corollary}
\theoremstyle{definition}
\newtheorem{definition}[theorem]{Definition}
\newtheorem{assumption}[theorem]{Assumption}
\theoremstyle{remark}
\newtheorem{remark}[theorem]{Remark}

% Todonotes is useful during development; simply uncomment the next line
%    and comment out the line below the next line to turn off comments
%\usepackage[disable,textsize=tiny]{todonotes}
\usepackage[textsize=tiny]{todonotes}
\usepackage{graphicx} 
\usepackage{mathbbol}
\usepackage{amssymb}
\usepackage{amsmath}
\usepackage{amsfonts}
\usepackage{soul}
\usepackage{enumitem}


\newcommand{\jessica}[1]{\textcolor{red}{Jessica: #1}}
\newcommand{\ziyang}[1]{\textcolor{blue}{Ziyang: #1}}
\newcommand{\yifan}[1]{\textcolor{purple}{[Yifan: #1]}}
\newcommand{\jason}[1]{\textcolor{orange}{[Jason: #1]}}

% \newcommand{\mvspace}[1]{\vspace{#1}}
\newcommand{\mvspace}[1]{}
\newcommand{\ours}{$\text{Q}$LASS}

\title{The Value of Information in Human-AI Decision-making}
\author{
    Ziyang Guo\thanks{Department of Computer Science, Northwestern University. Email: \texttt{ziyang.guo@northwestern.edu}} \and
    Yifan Wu\thanks{Department of Computer Science, Northwestern University. Email: \texttt{yifan.wu@u.northwestern.edu}} \and
    Jason Hartline\thanks{Department of Computer Science, Northwestern University. Email: \texttt{hartline@northwestern.edu}}
    \and
    Jessica Hullman\thanks{Department of Computer Science, Northwestern University. Email: \texttt{jhullman@northwestern.edu}}
}

\begin{document}
\maketitle

\begin{abstract}
Multiple agents---including humans and AI models---are often paired on decision tasks with the expectation of achieving \textit{complementary performance}, where the combined performance of both agents outperforms either one alone. 
However, knowing how to improve the performance of a human-AI team is often difficult without knowing more about what particular information and strategies each agent employs. 
We provide a decision-theoretic framework for characterizing the value of information---and consequently, opportunities for agents to better exploit available information--in AI-assisted decision workflows.
We demonstrate the use of the framework for model selection, empirical evaluation of human-AI performance, and explanation design.
We propose a novel information-based explanation technique that adapts SHAP, a saliency-based explanation to explain information value in decision making.
\end{abstract}
\mvspace{-5mm}

\begin{figure}[ht]
    \centering
    \includegraphics[width=0.8\linewidth]{graphs/greater_than_naive.pdf}
    \vspace{0.5cm}
    \includegraphics[width=0.8\linewidth]{graphs/p1_bottom.png}
    \vspace{-5pt}
    \caption{\textcolor{positional}{Positional} vs.\ \textcolor{nonpositional}{non-positional} circuits. In a \textcolor{nonpositional}{non-positional} circuit, the same edges must be included at all positions. A \textcolor{positional}{positional} circuit can distinguish between the same edge at different positions. This specificity yields better trade-offs between circuit size and faithfulness. It can also increase both precision and recall.}
    \label{fig:p1}
    \vspace{-5pt}
\end{figure}

\section{Introduction}

\looseness=-1
A primary goal of interpretability research is to characterize the internal mechanisms in language models (LMs) and other NLP models. 
A core approach in this area is \textbf{circuit discovery}---identifying the minimal subgraph within the model's computation graph that performs a specific task \citep{olah2021framework,olah-mech}.
Typically, the nodes of a circuit represent model components (e.g., attention heads, neurons, or layers).
While manual circuit discovery methods can yield position-specific insights \citep{wanginterpretability,goldowskydill2023localizingmodelbehaviorpath}, \emph{automatic methods often overlook positional information}, treating components as uniformly relevant across all input token positions \citep{conmytowards,syed2023attribution}. 
For instance, if an attention head is included in a circuit, it is assumed to contribute equally to the computation for every position in the input sequence.
The assumption that circuits are position-invariant ignores the fact that different positions often require distinct computations.
By ignoring positions, current methods limit their ability to capture mechanisms that operate across positions, such as interactions between attention heads across positions.

In this study, we start by demonstrating that positional agnosticism is a significant limitation (\S\ref{sec:motivating}). Then, to address these limitations, we introduce a new approach: position-aware edge attribution patching (PEAP; \S\ref{sec:full_circ_discovery}; Figure~\ref{fig:p1}). Current approaches  assume that if an edge is in a circuit, then the same edge will be in the circuit at all positions, thus leading to low precision. It is also assumed that an edge's importance should be aggregated across positions before deciding whether it should be included in the circuit; this can lead to cancellation effects, and thus low recall. PEAP instead allows us to compute the importance of cross-positional edges, and separately evaluates edge importance at each position. We show that this leads to smaller and more accurate circuits; see Figure~\ref{fig:p1}.

Incorporating positional information into circuit discovery is straightforward when inputs have the same length and structure across examples.

However, realistic datasets are not nearly this templatic.
How, then, can we incorporate positional information into automatic circuit discovery?
To address this challenge, we propose \textbf{schemas} (\S\ref{sec:schema}). 
Schemas assign semantic labels to spans of tokens, enabling information aggregation across examples even when the spans differ in length.

For example, in the input ``The \textcolor{positional}{war} lasted from 1453 to 14\underline{\hspace{1em}},'' the span ``\textcolor{positional}{war}'' could be labeled as ``\emph{Subject}''.
This enables handling spans with varying lengths: the phrase ``\textcolor{positional}{Black Plague}'' in another example can be treated as a single positional span with the same role as ``\textcolor{positional}{war}''.
In experiments with two LMs and three tasks, we find that circuits discovered using schemas achieve a better trade-off between circuit size and faithfulness to the model's behavior than position-agnostic circuits.
Importantly, position-aware circuits offer a more precise representation of the underlying mechanisms, providing a more concise foundation for mechanistic explanations.

We also present a fully automated pipeline for schema generation and application (\S\ref{sec:schema-generation}) using large language models (LLMs). 
We evaluate the quality of the generated schemas and their utility in discovering position-aware circuits (\S\ref{sec:schema-eval}).
Notably, circuits derived using automatically generated and applied schemas achieve comparable faithfulness scores to circuits discovered with human-designed and manually applied schemas.

We summarize our contributions as follows:
\begin{itemize}[noitemsep,leftmargin=*,topsep=1pt,parsep=1pt]
    \item Introduce a position-aware circuit discovery method, which obtains better faithfulness than position-agnostic discovery.  
    \item Introduce dataset schemas,  facilitating positional circuit discovery in more naturalistic settings. 
    \item Develop an automated schema generation and application pipeline with LLMs, yielding schemas that are comparable to manually-annotated ones.
\end{itemize}


\section{Related work}
\mvspace{-2mm}
\paragraph{Human-AI complementarity.}

Many empirical studies of human-AI collaboration focus on AI-assisted human decision-making for legal, ethical, or safety reasons~\citep{bo2021toward, boskemper2022measuring, bondi2022role, schemmer2022meta}.
However, a recent meta-analysis by \citet{vaccaro2024combinations} finds that, on average, human–AI teams perform worse than the better of the two agents alone. 
In response, a growing body of work seeks to evaluate and enhance complementarity in human–AI systems \citep{bansal2021does, bansal2019updates, bansal2021most, wilder2021learning, rastogi2023taxonomy, mozannar2024effective}.
The present work differs from much of this prior work by approaching human-AI complementarity from the perspective of information value and use, including asking whether the human and AI decisions provide additional information that is not used by the other.
\mvspace{-2mm}
\paragraph{Evaluation of human decision-making with machine learning.}
Our work contributes methods for evaluating the decisions of human-AI teams~\citep{kleinberg2015prediction, kleinberg2018human, lakkaraju2017selective, mullainathan2022diagnosing,  rambachan2024identifying, guo2024decision, ben2024does, shreekumar2025x}.
\citet{kleinberg2015prediction} proposed that evaluations of human-AI collaboration should be based on the information that is available at the time of decisions.
% \jessica{can omit:} A significant portion of this literature addresses \textit{performative prediction}~\citep{perdomo2020performative}, where predictions or decisions affect future outcomes. 
% Because counterfactual decisions’ outcomes remain unobserved, researchers typically rely on worst-case analyses to bound the potential performance \citep{rambachan2024identifying, ben2024does}. 
% Though these issues arise in many canonical human-AI collaboration tasks, we focus on standard ``prediction policy problems'' where the payoff can be translated into policy gains~\citep{kleinberg2015prediction}.
According to this view, our work defines Bayesian best-attainable-performance benchmarks similar to several prior works~\citep{guo2024decision, wu2023rational,agrawal2020scaling, fudenberg2022measuring}. 
Closest to our work, \citet{guo2024decision} model the expected performance of a rational Bayesian agent faced with deciding between the human and AI recommendations as the theoretical upper bound on the expected performance of any human-AI team.
This benchmark provides a basis for identifying exploitable information within a decision problem.

\mvspace{-3mm}
\paragraph{Human information in machine learning.}

Some approaches focus on automating the decision pipeline by explicitly incorporating human expertise in developing machine learning models, such as by learning to defer~\citep{mozannar2024show, madras2018predict, raghu2019algorithmic, keswani2022designing, keswani2021towards, okati2021differentiable}.
\citet{corvelo2023human} propose multicalibration over human and AI model confidence information to guarantee the existence of an optimal monotonic decision rule.
\citet{alur2023auditing} propose a hypothesis testing framework to evaluate the added value of human expertise over AI forecasts.
Our work shares the motivation of incorporating human expertise, but targets a slightly broader scope by quantifying the information value for all available signals and agent decisions in a human–AI decision pipeline.



\mvspace{-3mm}
\section{Methodology}

% \jessica{Also, we should be clearly defining what information our approach assumes as input, and what the output is (at least at a high level)}
\mvspace{-2mm}
Our framework takes as input a decision problem associated with an information model and outputs the value of information of any available signals to any agent, conditioning on the existing information in their decisions within a Bayesian decision theoretic framework.
Our framework provides two separate functions to quantify the value of information: one globally across the data-generating process, and one locally in a realization drawn from the data-generating process.
We also introduce a robust analysis approach to information order, which enables us to compare the agent-complementary information in signals for all possible decision problems.
% \jessica{not sure why some things are italized ... for example why is realization not italicized when it first appears? should globally and locally be italicized instead of data-generating process? could probably just remove all italics or only use italics for things we will define specifically}

% In this section, we define the basis of this approach, including a decision problem and associated information structure, following prior decision-theoretic frameworks for studying decisions from statistical information~\citep{wu2023rational,guo2024decision,hullman2024decision}.
% Then we define how a rational decision-maker would act given \st{a signal and} \jessica{such} a decision problem and associated information structure, and use rational \jessica{behavior within the problem} \st{decision-maker ,as a tool  we show how} to \st{investigate} \jessica{characterize} the information encoded in behavioral decisions.
%\jessica{May want to add a sentence or two here to give the reader some intuition for our approach. E.g., Our approach relies on analysis of the marginal gain ... }

%\ziyang{Merge infomration structure and decision-making problem into one section}
\mvspace{-4mm}
\paragraph{Decision Problem.} A decision problem consists of three key elements. We illustrate with an example of a weather decision. 
\mvspace{-2mm}
\begin{itemize}[wide]
    \mvspace{-2mm}
    \item A payoff-relevant state $\payoffstatevalue$ from a space $\payoffstatespace$. For example,\ $\payoffstatevalue \in \payoffstatespace =  \{0, 1\} = \{\text{no rain}, \text{rain}\}$.
    \mvspace{-3mm}
    \item A decision $\action$ from the decision space $\actionspace$ characterizing the decision-maker (DM)'s choice. For example,\ $\action\in \actionspace = \{0, 1\} = \{\text{not take umbrella}, \text{take umbrella}\}$.
    \mvspace{-2mm}
    \item A payoff function $\score: \actionspace\times\payoffstatespace\to\mathbb{R}$, used to assess the quality of a decision given a realization of the state. For example, $\score(\action = 0, \payoffstatevalue = 0) = 0, \score(\action = 0, \payoffstatevalue = 1) = -100, \score(\action = 1, \payoffstatevalue = 0) = -50, \score(\action = 1, \payoffstatevalue = 1) = 0$, which punishes the DM for selecting an action that does not match the weather. 
\end{itemize}

In decision problems corresponding to prediction tasks, the decision space is a probabilistic belief over the state space, i.e., $\actionspace = \Delta(\payoffstatespace)$.
For such problems, a payoff function is said to be a \textit{proper} scoring rule if the optimal action is to predict the true distribution, i.e., $p = \arg \max_{\action \in \actionspace} \expect[\payoffstatevalue \sim p]{\score(\action, \payoffstatevalue)}$.
% Therefore, given a proper scoring rule a DM who maximizes the expected payoff will truthfully report their belief.
For any decision problem with payoff function $\score: \actionspace\times\payoffstatespace\to\mathbb{R}$, there is an equivalent proper scoring rule $\hat{\score}: \Delta(\payoffstatespace)\times\payoffstatespace\to\mathbb{R}$ defined by choosing the optimal decision under the reported
belief. Formally,
\begin{equation}
\label{eq:properscoring}
    \hat{\score}(p, \payoffstatevalue) = \score(\arg \max_{\action \in \actionspace} \expect[\payoffstatevalue \sim p]{\score(\action, \payoffstatevalue)}, \payoffstatevalue).
\end{equation}

\Cref{eq:properscoring} shows a reduction from the payoff function $\score$ to the proper scoring rule $\hat{\score}$, i.e., any decision $p$ under $\hat{\score}$ represents a decision $\action$ in $\score$ that best-responds to the distribution $p$.
Therefore, the best-attainable performance defined in the proper scoring rule $\hat{\score}$ is equivalent to the best-attainable performance defined in any payoff function $\score$ that can be reduced to $\hat{\score}$.
Throughout our framework, we prefer proper scoring rules over non-proper scoring rules, since the best-attainable performance defined in the former implies the best-attainable performance in the latter.

\mvspace{-4mm}
\paragraph{Information Model.} 
We cast the information available to a DM as a signal defined within an information model.
We use the definition of an information model in \citet{blackwell1951comparison}. 
The information model can be represented by a \textit{data-generating model} with a set of \textit{signals}.
\begin{itemize}[wide]
    \mvspace{-3mm}
    \item \textit{Signals}. There are $n$ ``basic signals'' represented as random variables $\basicsig_1, \ldots, \basicsig_n$, from the signal spaces $\basicsigsp_1, \ldots, \basicsigsp_n$. Basic signals represent information available to a decision-maker, e.g., $\basicsigsp_1 = \{\text{cloudy}, \text{not cloudy}\}$, $\basicsigsp_2\in \{0, \ldots, 100\}$ for temperature in Celsius, etc. 
    % We write $k_i = |\basicsigsp_i|$ as the size of the signal space of the basic signal $i$, $\basicsig_i$ as the random variable for basic signal $i$, and $\basicsigval_{ij_i}\in \basicsigsp_i$ as the $j_i$th realized value of the $\basicsig_i$ ($j_i\leq k_i$).
    % E.g.\ observable features about the weather $\{\sig_1, \sig_2, \ldots\} = \{\text{temperature}, \text{cloud level}, \dots\}$. 
    % In addition to the basic signals, there are also other signals that \st{intuitively} represent the combination of basic signals.
    The decision-maker observes a signal, which is a subset of the basic signals, $\sig \subseteq 2^{\{\basicsig_1, \dots, \basicsig_n\}}$. 
    Specifically, we use $\sig = \{\basicsig_{j_1}, \ldots, \basicsig_{j_k}\}$ for a signal having $k$ basic signals and denote the signal space as $\sigsp = \basicsigsp_{j_1} \times \ldots \times \basicsigsp_{j_k}$.
    For example,\ a signal representing a combination of two basic signals $\sig = \{\basicsig_1, \basicsig_2\}$ observed by the decision-maker might consist of cloudiness $\basicsig_1$ and the temperature $\basicsig_2$ of the day. Given a signal composed of $m$ basic signals, we write the realization of $\sig$ as $\sigval = (\basicsigval_{j_1}, \dots, \basicsigval_{j_{k}})$, where the realizations $\basicsigval_{j_i} \in \basicsigsp_{j_i}$ are sorted by the index of the basic signals $j_i \in [n]$.
    The union $\sig$ of two signals $\sig_1, \sig_2$ takes the set union, i.e., $\sig = \sig_1\cup\sig_2$.
    % Though $\sig$ is initially defined as a set of random variables, we will slightly abuse notation $\sig$ to represent a random variable that is drawn from the joint distribution of the basic signals in it.
    % \jessica{I'm finding this part really confusing - eg we use capital V to refer to a signal, then lower case v}
     \mvspace{-1mm}
    \item \textit{Data-generating process}. A data-generating process is a joint distribution $\dgp\in \Delta(\basicsigsp_1 \times \ldots \times \basicsigsp_n \times\payoffstatespace)$ over the basic signals and the payoff-relevant state. $\dgp$ can be viewed as the combination of two distributions: the prior distribution of the state $\Pr[\payoffstatevalue]$ and the signal-generating distribution $\Pr[\sigval | \payoffstatevalue]$ defining the conditional distribution of signals. The DM may only observe a subset $\sig$ of the $n$ basic signals. We can define the Bayesian posterior belief upon receiving a signal $\sig = \sigval$ from the data-generating model as
     \mvspace{-1mm}
    \[\dgp(\payoffstatevalue| \sigval) := \Pr[\payoffstatevalue|\sigval]=\frac{\dgp(\sigval, \payoffstatevalue)}{\dgp(\payoffstatevalue)}\]
    % Conditioning on receiving a signal $\sig = \sigval$, \jessica{a DM} \st{the DMs} who knows the data-generating process is able to infer the Bayesian posterior $\Pr[\payoffstatevalue|\sigval]$ of the state, thus improving their payoff. % \jessica{should we mention that this DM has the prior?} 
 \mvspace{-4mm}

    \noindent where we slightly abuse notation to write $\dgp(\sigval, \payoffstatevalue)$ as the marginal
     probability of the signal realized to be $\sigval$ and the state being $\payoffstatevalue$ with expectation over other signals.
\end{itemize}

The choice of basic signals directly impacts how many observed samples are required to get a good estimate of the data-generating process.
When high-dimensional signals such as images or text are used, it may not be computationally feasible to estimate the data-generating process.
In such cases, one can preprocess the high-dimensional signals to get low-dimensional representations which (ideally) capture any important structure. 
These lower-dimensional signals can be defined by humans, such as when concepts are identified and then used to label the high-dimensional signals (e.g., images or parts of images). 
Alternatively, they can be defined algorithmically, by strategically applying dimensionality reduction to generate low-dimensional embeddings.
We introduce an algorithm for identifying an ``optimal'' reduction model in decision problems in \Cref{app:high-dimensional}.
%Another way is to let human label the concepts in high-dimensional signals, which generates interpretable low-dimensional signals.
We demonstrate these two methods in \Cref{exp2} and \Cref{exp1} respectively.

% \ziyang{several ways to do this. Use the algorithm to find low-dimensional but you can interpret these. Or let human label concepts. (This is what we've done in our study). Mention when we want to just use AI model's predictions to represent. }

%Slightly abusing notations, we write $\dgp(\payoffstatevalue)$ as the prior probability of the state $\payoffstatevalue$. 

%There is a payoff-related uncertain state $\payoffstate$ of interest to the decision-maker, e.g., $\payoffstate \in \{0, 1\} = \{\text{no rain}, \text{rain}\}$.
%There are also $n$ signals, $\sig_1, \ldots, \sig_n$, modeled as random variables. 
%These signals represent the information displayed to the decision-maker, e.g., whether it is cloudy $\sig_i \in \{0, 1\}$.
%An information structure $\infostructure$ is given by $\payoffstate$, $\sig_1$, $\ldots$, $\sig_n$ and a data generating process $\dgp \in \Delta(\payoffstate \times \sig_1 \times \ldots \times \sig_n)$, which describes the joint distribution between state and signals.

% We use lower-case $\payoffstatevalue, \sigval_1, \ldots, \sigval_n$ to refer to the outcomes generated under $\dgp$.
% Given a realization $\sigval_i$ of $\sig_i$, the probability that $\payoffstate = \payoffstatevalue$ conditioned on $\sig_i = \sigval_i$ can be obtained \jessica{by Bayesian updating the} prior: $\Pr[\payoffstate = \payoffstatevalue | \sig_i = \sigval_i] = \dgp(\payoffstatevalue, \sigval_i) / \dgp(\sigval_i)$.
% Similarly, we can also define the probability of $\payoffstate$ conditioned on realizations of multiple signals.
\mvspace{-4mm}
\paragraph{Information value.}
Our framework quantifies the value of information in a signal $\sig$ as the expected payoff improvement of an idealized agent who has access to $\sig$ in addition to some baseline information set.
% \jessica{this whole section could benfeit from a few more sentences early in the subsections to reiterate what we are tryin to achieve, or even phrases. E.g., here it seems we want to quantify the information value of some signal, relative to another signal. Give the reader more 'sign posts' to help remind them why we are setting up different concepts the way we are. Can be simple as adding a phrase like 'To quantify the vaue of the information is some set of signals' to the beginning of the first sentence. Its easy to lose the point currently}
We suppose a rational Bayesian DM who knows the prior probability of the state and conditional distribution of signals (i.e., the data-generating process), observes a signal realization, updates their prior to arrive at posterior beliefs, and then chooses a decision to maximize their expected payoff given their posterior belief. 
Formally, given a decision task with payoff function $\score$ and an information model $\dgp$, the rational DM's expected payoff given a (set of) signal(s) $\sig$ is
\mvspace{-1mm}
\begin{equation}
\mathrm{R}^{\dgp, \score}
(\sig)
= \expect[(\sigval, \payoffstatevalue) \sim \dgp]{\score(\action^r(\sigval), \payoffstatevalue)}
\end{equation}
\mvspace{-4mm}
\noindent where $\action^r(\cdot): \sigsp \rightarrow \actionspace$ denotes the decision rule adopted by the rational DM.
\begin{equation}
\label{eq:rationalDM}
    \action^r(\sigval) = \arg \max_{\action\in\actionspace} \expect[\payoffstatevalue \sim \dgp(\payoffstatevalue|\sigval)]{\score(\action, \payoffstatevalue)}
\end{equation}

We use $\emptyset$ to represent a null signal, such that $\mathrm{R}^{\dgp, \score}(\emptyset)$ is the expected payoff of a Bayesian rational DM who only uses the prior distribution to make decisions.
In this case, the Bayesian rational DM will take the best fixed action under the prior, and their expected payoff is:
\mvspace{-1mm}
\begin{equation}
\label{eq:baseline}
\mathrm{R}^{\dgp, \score}
(\emptyset) 
= \max_{\action \in \actionspace} \expect[\payoffstatevalue \sim \pi]{\score(\action, \payoffstatevalue)}
\end{equation}
\mvspace{-4mm}

This baseline defines the maximum expected payoff that can be achieved with no information.
Bayesian decision theory quantifies the information value of $\sig$ by the payoff improvement of $\sig$ over the payoff obtained without information.
% Given a set of signals $\sig_1$ and a ground set of signals $\sig_2$ (which could be the null signal $\emptyset$), we can define the \textit{information gain} from $\sig_1$ over $\sig_2$, the payoff improvement of $\sig_1$ over the payoff obtainable from $\sig_2$.
% \begin{equation}
% \infoval^{\dgp, \score}(\sig_1; \sig_2) = \mathrm{R}^{\dgp, \score}
% (\sig_1\cup\sig_2) - \mathrm{R}^{\dgp, \score}
% (\sig_2).
% \end{equation}


\begin{definition}
Given a decision task with payoff function $\score$ and an information model $\dgp$, the information value of $\sig$ is defined as
\mvspace{-1mm}\begin{equation}
    \IV^{\dgp, \score}(\sig) = \mathrm{R}^{\dgp, \score}
(\sig) - \mathrm{R}^{\dgp, \score}
(\emptyset)
\end{equation}
\end{definition}
\mvspace{-2mm}
We adopt the same idea to define the agent-complementary information value in our framework.

% $IV$ reflects the marginal information offered by $\sig$ over $\emptyset$.
% In human-AI collaboration, we may especially be interested in the complementary information offered by a signal (e.g., AI prediction and explanation) over human information.
% In that case, $IV$ can be defined as \[IV^{\dgp, \score}(\sig) = \infoval^{\dgp, \score}(\sig; \actionvar^b)\]where $\actionvar^b$ is a random variable for human decisions, which we defined in the following section.
\mvspace{-2mm}
\subsection{Agent-Complementary Information Value}
\mvspace{-2mm}

With the above definitions, it is possible to measure the additional value that new signals can provide over the information already captured by an agent’s decisions. Here, \textit{agent} may refer to a human, an AI system, or a human–AI team.
The intuition behind our approach is that any information that is used by decision-makers should eventually reveal itself through variation in their decisions.
We recover the information value in agent decisions by offering the decisions as a signal to the Bayesian rational DM.
We model the agent decisions as a random variable $\actionvar^b$ from the action space $\actionspace$, which follows a joint distribution $\dgp \in \Delta(\basicsigsp_1 \times \ldots \times \basicsigsp_n \times \payoffstatespace \times  \actionspace)$ with the state and signals.
The expected payoff of a Bayesian rational DM who knows $\dgp$ is given by the function:
 \mvspace{-1mm}
\[
\mathrm{R}^{\dgp, \score}
(\actionvar^b)
= \expect[(\action^b, \payoffstatevalue) \sim \dgp]{\score(\action^r(\action^b), \payoffstatevalue)}
\]
% \[
%  \mathrm{R}%^{\dgp, \score}
%  (\actionvar^b) = \expect[\action^b \sim \dgp^b]{\max_{\action \in \actionspace}\expect[\payoffstatevalue \sim \Pr(\payoffstatevalue | \actionvar^b = \action^b)]{\score(\action, \payoffstatevalue)}}
% \]
 \mvspace{-5mm}
 
% Similarly, we can assess the potential for other available information to improve agent decisions by quantifying the information gain from different signals (such as instance feature information or AI predictions) over the agent decisions alone. 
We seek to identify signals $\sig$ that can potentially improve agent decisions by analyzing the information value in the combined signal $\actionvar^b \cup \sig$ over the information value in $\actionvar^b$, which we define as the agent-complementary information value.
\begin{definition}
\label{def:aciv}
Given a decision task with payoff function $\score$ and an information model $\dgp$, we define the agent-complementary information value ($\ACIV$) of $\sig$ on agent decisions $\actionvar^b$ as 
\mvspace{-1mm}\begin{equation}
    \ACIV^{\dgp, \score}(\sig; \actionvar^b) = \mathrm{R}^{\dgp, \score}(\actionvar^b \cup \sig) - \mathrm{R}^{\dgp, \score}(\actionvar^b)
\end{equation}
\end{definition}
\mvspace{-3mm}

If the $\ACIV$ of a signal $\sig$ is small relative to the baseline (\ref{eq:baseline}), this means either that the information value of $\sig$ to the decision problem is low (e.g., it is not correlated with $\payoffstatevalue$), or that the agent has already exploited the information in $\sig$ (e.g., the agent relies on $\sig$ or equivalent information to make their decisions such that their decisions correlate with $\payoffstatevalue$ in the same way as $\sig$ correlates with $\payoffstatevalue$).
If, however, the $\ACIV$ of $\sig$ is large relative to the baseline, then at least in theory, the agent can improve their payoff by incorporating $\sig$ in their decision making.

 \mvspace{-1mm}
Furthermore, $\ACIV$ can reveal complementary information between different types of agents. 
For instance, if we view AI predictions as $\sig$ and treat human decisions as the agent signal $\actionvar^b$, a large $\ACIV$ indicates that AI predictions add considerable value beyond what humans alone achieve. In the reverse scenario, if human decisions serve as $\sig$ and AI predictions are $\actionvar^b$, we can measure how much humans can contribute over the information captured in the AI predictions. We demonstrate uses of $\ACIV$ in \Cref{exp2} and \Cref{exp1}.

 \mvspace{-2mm}
\subsection{Instance-level Agent-complementary Information Value}
 \mvspace{-2mm}
$\ACIV$ quantifies the value of the decision-relevant information in a signal $\sig$ across all possible realizations defined by the data-generating model.
To provide analogous instance-level quantification of information value, we define Instance-Level agent-complementary Information Value ($\ILIV$) to quantify the value of the decision-relevant information encoded by a single realization of the signal.
% Instance-level Agent-Complementary Information Value ($\ILIV$) evaluates the additional information contributed by a single realization of a signal rather than the entire joint distribution.
% $\ACIV$ can be useful in cases such as evaluation/comparison of models and empirical analysis of whether AI's assistance can help humans improve the information value in their decisions.
This finer-grained view makes it possible to analyze how much an agent can benefit in theory from better incorporating instance-level information in their decision.

Given a realization of signal $\sigval = \{\basicsigval_{j_1}, \ldots, \basicsigval_{j_k}\}$, we want to know the maximum expected payoff gain from gaining access to $\sigval$ on the instances where $\sigval$ is realized over the existing information encoded in agent decisions. Intuitively, this captures how much ``room'' there is for specific decisions to be improved through better use of the signal.
To calculate instance-level information value, we rely on the performance of the Bayesian rational agent on the conditional distribution $\dgp(\basicsigval_1, \ldots, \basicsigval_n, \payoffstatevalue, \action^b | \sigval)$ instead of the whole distribution $\dgp(\basicsigval_1, \ldots, \basicsigval_n, \payoffstatevalue, \action^b)$.
% We use a Bayesian rational DM to quantify $\ILIV$ in a similar way to $\ACIV$.
% Instead of evaluating on the distribution $\dgp(\sigval, \payoffstatevalue)$, $\ILIV$ evaluates the expected payoff of the rational DM on the distribution indicated by the instance $\dgp(\payoffstatevalue | \sigval)$.  \jessica{whats the intuition here? the explanation doesn't provide much of a sense of why this is useful}
Formally, given a decision task with payoff function $\score$ and information model $\dgp$, the expected payoff of the rational DM given signal $\sig = \sigval$ on instances where $\sig = \sigval$ is
\begin{equation}
\mathrm{r}^{\sigval, \dgp, \score}(\sigval) = \expect[\payoffstatevalue \sim \dgp(\payoffstatevalue | \sigval)]{\score(\action^r(\sigval), \payoffstatevalue)}
\end{equation}
% \noindent where $\action^r(\cdot)$ is defined in \Cref{eq:rationalDM}. 
\noindent where $\action^r(\sigval)$ is the Bayesian optimal decision on receiving  $\sigval$ as defined in \Cref{eq:rationalDM}.
If we consider the agent decisions in addition to the realization $\sigval$, the rational DM's expected payoff on instances where $\sig = \sigval$ can be written as
\begin{equation}
\mathrm{r}^{\sigval, \dgp, \score}(\sigval; \actionvar^b) = \expect[(\action^b, \payoffstatevalue) \sim \dgp(\action^b, \payoffstatevalue | \sigval)]{\score(\action^r(\sigval \cup \action^b), \payoffstatevalue)}
\end{equation}

\begin{definition}
\label{def:RIIV}
Given a decision task with payoff function $\score$ and an information model $\dgp$, we define the instance-level agent-complementary information value ($\ILIV$) of signal $\sig = \sigval$ on instances where $\sig = \sigval$ as:
 \mvspace{-1mm}
\begin{equation}
    \ILIV^{\sigval, \dgp, \score}(\sigval; \actionvar^b) = \mathrm{r}^{\sigval, \dgp, \score}(\sigval; \actionvar^b) - \mathrm{r}^{\sigval, \dgp, \score}(\emptyset; \actionvar^b)
\end{equation}
\end{definition}
 \mvspace{-4mm}
\noindent where $\mathrm{r}^{\sigval, \dgp, \score}(\emptyset; \actionvar^b)$ represents the rational DM's expected payoff on instances where $\sig = \sigval$ with only the exisiting information encoded in agent decisions.
Taking the expectation of $\ILIV$ over $\sig$ recovers the global agent-complementary information value ($\ACIV$), i.e.,
\begin{equation}
    \ACIV^{\dgp, \score}(V; \actionvar^b) = \expect[\sigval \sim \dgp(\sigval)]{\ILIV^{\sigval, \dgp, \score}(\sigval; \actionvar^b)}\nonumber
\end{equation}

\subsubsection{ILIV-SHAP Information-based Explanation}

We apply $\ILIV$ to define an \textit{information-based} explanation technique (ILIV-SHAP) that extends the canonical SHAP feature saliency explanation of a model's prediction. ILIV-SHAP communicates how the data features lead to changes in the information value of AI predictions. While traditional SHAP summarizes the average contribution of each feature to a specific prediction over the baseline (prior) prediction, 
ILIV-SHAP summarizes the average contribution of each feature to the decision-relevant information value contained in the prediction. 

% We extend the SHAP algorithms to calculate the effect scores in ILIV-SHAP. 
Suppose a model $f$ that takes as input $m$ features and outputs a real number.
Given an instance $\mathbf{x} = (x_1, \ldots, x_m)$, the importance of one feature $x_i$ to the model output $f(\mathbf{x})$ is encoded by the expected difference of model outputs when $x_i$ is marginalized out.
Specifically, this is quantified by $f(\mathbf{x}) - \expect{f(X) | X_{-i} = \mathbf{x}_{-i}}$, where $X_{-i}$ denotes all features except $X_i$.
Considering the interaction between features, SHAP~\citep{lundberg2017unified} uses the Shapley value to quantify the importance scores averaged on different combinations of features:
\begin{equation*}
    \phi_i(f, \mathbf{x}) = \sum_{\mathbf{x}' \subseteq \mathbf{x}} \frac{|\mathbf{x}'|!(m - |\mathbf{x}'| - 1)!}{m!} [g_f(\mathbf{x}') - g_f(\mathbf{x}' \backslash x_i)]
\end{equation*}
\noindent where $g_f(\mathbf{x}')$ denotes the expected model output conditioned on $\mathbf{x}'$, i.e., $\expect{f(X)| X' = \mathbf{x}'}$ for any $\mathbf{x}' \subseteq \mathbf{x}$. The scores $\phi_i(f, \mathbf{x})$ output by SHAP construct an explanation model for a model output, which quantifies the expected counterfactual change in the model output caused by the feature $x_i$.

ILIV-SHAP extends SHAP to give an explanation model on how features impact the decision-relevant information value of an individual model output.

\begin{definition}[ILIV-SHAP]
\label{def:shap}
Given a model $f$ and data features $\mathbf{x} = (x_1, \ldots, x_m)$, the importance score of the \textit{i}-th feature by ILIV-SHAP is 
\begin{equation*}
    \phi_i^{\ILIV}(f, \mathbf{x}) = \sum_{\mathbf{x}' \subseteq \mathbf{x}} \frac{|\mathbf{x}'|!(m - |\mathbf{x}'| - 1)!}{m!} [\ILIV^{f(\mathbf{x}), \dgp, \score}(g_f(\mathbf{x}'); \actionvar^b) - \ILIV^{f(\mathbf{x}), \dgp, \score}(g_f(\mathbf{x}' \backslash x_i); \actionvar^b)]
\end{equation*}
\end{definition}
\noindent where $\ILIV^{f(\mathbf{x}), \dgp, \score}(g_f(\mathbf{x}'); \actionvar^b)$ denotes a counterfactual evaluation of $\ILIV$, which quantifies the expected payoff gain from additionally knowing $g_f(\mathbf{x}')$ on the instances where $f(\mathbf{x})$ is realized. 
This counterfactual version of $\ILIV$ is guaranteed to achieve the maximum at $\mathbf{x}'$ where $f(\mathbf{x}) = g_f(\mathbf{x}')$, i.e., the features missing from $\mathbf{x}'$ have no impact on $f(\mathbf{x})$.

The explanation model offered by ILIV-SHAP is grounded in the following two properties.
First, $\phi^{\ILIV}_i(f, \mathbf{x})$ are consistent with the extent to which feature $x_i$ contributes to the information value in $f(\mathbf{x})$.
Specifically, for any two models $f$ and $f'$ and any $\mathbf{x}' \subseteq \mathbf{x}$, if $x_i$ contributes more to the information value in $f'$ than the information value in $f$, i.e., $\ILIV^{f'(\mathbf{x}), \dgp, \score}(g_{f'}(\mathbf{x}'); \actionvar^b) - \ILIV^{f'(\mathbf{x}), \dgp, \score}(g_{f'}(\mathbf{x}' \backslash x_i); \actionvar^b) \geq \ILIV^{f(\mathbf{x}), \dgp, \score}(g_{f}(\mathbf{x}');\actionvar^b) - \ILIV^{f(\mathbf{x}), \dgp, \score}(g_{f}(\mathbf{x}' \backslash x_i); \actionvar^b)$, then $\phi^{\ILIV}_i(f', \mathbf{x}) \geq \phi^{\ILIV}_i(f, \mathbf{x})$.
This property of ILIV-SHAP follows the consistency property of SHAP~\citep{lundberg2017unified}.
Second, summing up the scores $\phi^{\ILIV}_i(f, \mathbf{x})$ recovers the information value of model output, $\ILIV^{f(\mathbf{x}), \dgp, \score}(f(\mathbf{x}); \actionvar^b)$.
Formally, for any $f$ and any $\mathbf{x}$, $\ILIV^{f(\mathbf{x}), \dgp, \score}(f(\mathbf{x}); \actionvar^b) = \ILIV^{f(\mathbf{x}), \dgp, \score}(\expect{f(X)}; \actionvar^b) + \sum_{i = 1}^m \phi^{\ILIV}_i(f, \mathbf{x})$.
We demonstrate use of ILIV-SHAP in \Cref{exp3}.


% The information value of the realization $\sig=\sigval$ should be quantified relatively to a counterfactual realization $\sig = \sigval^+$. \jessica{is this not v? can we notate it like that?} Formally, given a decision task with payoff function $\score$ and an information model $\dgp$, the rational DM's expected payoff given the counterfactual signal $\sig = \sigval^+$ on instances where $\sig = \sigval$ is
% \[
% \mathrm{r}^{\sigval, \dgp, \score}(\sigval^+; \sig') = \expect[(\sigval', \payoffstatevalue) \sim \dgp(\sigval', \payoffstatevalue | \sigval)]{\score(\action^r((\sigval^+, \sigval')), \payoffstatevalue)}
% \]

% \begin{proposition}
% \label{prop:counterfactual}
%     For any $\sigval^+ \in \sigsp$, 
%     \[\mathrm{r}^{\sigval, \dgp, \score}(\sigval; \sig') \geq \mathrm{r}^{\sigval, \dgp, \score}(\sigval^+; \sig')\]
% \end{proposition}

% The proof of \Cref{prop:counterfactual} is followed by the fact that any payoff function $\score$ can induce a proper scoring rule $\hat{\score}: \distover{\payoffstatespace} \times \payoffstatespace \rightarrow \mathbb{R}$.


% ILIV-SHAP takes as input a decision task with payoff function $\score$, a model $f$, an instance with features $\mathbf{x} = (x_1, \ldots, x_m)$ and a set of realizations of model predictions, human decisions and state.
% We denote the model predictions by $\action^{f} = f(\mathbf{x})$ and the human decisions by $\action^b$.
% % $[(f(\mathbf{x})_1, \action^b_1, \payoffstatevalue_1), \ldots, (f(\mathbf{x})_m, \action^b_m, \payoffstatevalue_m)]$.
% The construction of ILIV-SHAP proceeds in three steps: 1) Estimate the information model $\dgp(\action^f, \action^b, \payoffstatevalue)$ from the observed realizations, 2) Calculate the factual $\ILIV^{\action^f, \dgp, \score}(\action^f; \actionvar^b)$ and the counterfactual $\ILIV^{\action^f, \dgp, \score}({\action^f}'; \actionvar^b)$ for any ${\action^f}' \neq \action^f$, and 3) Run the SHAP algorithm \citep{lundberg2017unified} to estimate the effect score $\phi^{\ILIV}_i$ for each feature $x_i$ with the objective function as $\ILIV$.

% \begin{algorithm}[tb]
%    \caption{ILIV-SHAP}
%    \label{alg:iliv-shap}
% \begin{algorithmic}
%    \STATE {\bfseries Input:} Payoff function $\score$, model $f$, data features $\mathbf{x}$, a set of realizations $[(\action^f_t, \action^b_t, \payoffstatevalue_t)]_{t=1}^T$
%    \STATE Estimate the empirical distribution $\dgp(\action^f, \action^b, \payoffstatevalue) = \frac{1}{T}\sum_{t=1}^T \mathbb{1}_{\action^f_t = \action^f}\cdot\mathbb{1}_{\action^b_t = \action^b}\cdot\mathbb{1}_{\payoffstatevalue_t = \payoffstatevalue}$, $\forall \action^f, \action^b, \payoffstatevalue$.
%    \STATE Set current signal $\action^f_0 = f(\mathbf{x})$.
%    \STATE Construct function $\ILIV^{\action^f_0, \dgp, \score}(f(\cdot); \actionvar^b)$.
%    \STATE Run SHAP algorithm on function $\ILIV^{\action^f_0, \dgp, \score}(f(\cdot); \actionvar^b)$ to generate $(\phi^{\ILIV}_1, \ldots, \phi^{\ILIV}_m)$.
% \end{algorithmic}
% \end{algorithm}

% Vanilla SHAP defines a \textit{saliency-based} explanation with a set of effect variables $\phi_i$ representing the influence of the realization of basic signal $\basicsig_i = \basicsigval_i$ on the model output $f(\sigval)$, where $\sigval = (\basicsigval_1, \ldots, \basicsigval_n)$ and $f: \sigsp \rightarrow \mathbb{R}$.
% When $\phi_i$ is positive, it means that the realization of $\basicsig_i = \basicsigval_i$ leads the model prediction $f(\sigval)$ to increase $\phi_i$ on the expectation over other basic signals.
% We let $Y$ denote the random variable for model output, $Y = f(\sig)$ for a predictive model $f(\cdot): \sigsp \rightarrow \mathbb{R}$.
% SHAP~\citep{lundberg2017unified} defines a \textit{saliency-based} explanation with a set of \textit{effect variables} $\phi_i$ that represents the influence of the basic signal $\basicsig_i$ on $f(\cdot)$.
% \citet{lundberg2017unified} show that the following set of $\phi$ fulfills properties of \textit{local accuracy}, \textit{missingness} and \textit{consistency}:

% $\sigval' \backslash \basicsigval_i$ denotes removing $\basicsigval_i$ from $\sigval'$.

% \jessica{don't start a sentence with symbols} $(\phi_1^{\ILIV}, \ldots, \phi_m^{\ILIV})$ construct a feasible explanation model that fulfills the properties of \textit{local accuracy}, \textit{missingness} and \textit{consistency} \citep{lundberg2017unified}.
% When $\phi_i^{\ILIV}$ is relatively large, it means that the feature $x_i$ helps the model $f(\cdot)$ extract decision-relevent information that \jessica{complements} $\actionvar^b$. \jessica{Need to convey this intuition we are going for way earlier}
% Otherwise, when $\phi_i^{\ILIV}$ makes a small relative contribution, it means that \jessica{the signal realization} \st{it} makes the model ignore information in that feature \jessica{not sure how to interpret this - makes model ignore information - what's the mechanism behind that?} or the information extracted by the model is already contained in human decisions. 
% \jessica{I'm having trouble following this}

% \Cref{prop:locacc-shap,prop:missingness-shap,prop:consistency-shap} show that the effect variables derived under SHAP fulfill \textit{local accuracy}, \textit{missingness} and \textit{consistency}, which are the basic desirable properties in model explanations.

% \begin{proposition}[Local accuracy, \citealt{lundberg2017unified}]
% \label{prop:locacc-shap}
% For any $f$ and $\sigval \in \sigsp$,
% \begin{equation*}
%     f(\sigval) = \expect{f(\sig)} + \sum_{i=1}^{n} \phi_i(f, \sigval)
% \end{equation*}
% \end{proposition}

% \begin{proposition}[Missingness, \citealt{lundberg2017unified}]
% \label{prop:missingness-shap}
% For any $f$ and $\sigval \in \sigsp$, if $g_{f}(\sigval) = g_{f}(\sigval \backslash \basicsigval_i)$, then $\phi_i(f, \sigval) =0$.
% \end{proposition}

% \begin{proposition}[Consistency, \citealt{lundberg2017unified}]
% \label{prop:consistency-shap}
% For any $\sigval \in \sigsp$ and two models $f$ and $f'$, if for any $\sigval' \subseteq \sigval$,
% \begin{equation*}
%     g_{f'}(\sigval') - g_{f'}(\sigval' \backslash \basicsigval_i) \geq g_{f}(\sigval') - g_{f}(\sigval' \backslash \basicsigval_i)
% \end{equation*}
% then $\phi_i(f', \sigval) \geq \phi_i(f, \sigval)$.
% \end{proposition}

% Extending to the information value, RIIV-SHAP uses $\ILIV^{\dgp, \score}(f(\cdot); \actionvar^b)$ instead of $f(\cdot)$ as the objective function.
% \begin{proposition}[RIIV-SHAP]
% \label{prop:RIIV-shap}
% An explanation model whose effective variable defined as
% \begin{equation*}
%     \phi_i(\ILIV^{\dgp, \score}(f(\cdot); \actionvar^{b}), \sigval)
% \end{equation*}
% where $\phi_i(\cdot, \cdot)$ takes the same form as \Cref{def:shap}, fulfill properties of local accuracy, missingness and consistency for $\ILIV$ of $f(\cdot)$ on $\sigval$.
% \end{proposition}
%  \mvspace{-2mm}

% For example, if $\localsig_\sigval$ indicates whether an AI predicts a particular probability, then $\ILIV$ can serve as an objective in a SHAP-based explanation. 
% Rather than just quantifying changes in the AI’s raw predictions, this approach clarifies how each feature contributes to the additional information value, helping convey when and why the human should trust (or question) a single AI prediction. We demonstrate this use case in \Cref{exp3}.
 \mvspace{-2mm}

\subsection{Robustness Analysis of Information Order}
 \mvspace{-2mm}
%The definition of a decision task requires the identification of a payoff function that evaluates the decisions against the realization of the payoff-related state.
%However, 
Our approach assumes the specification of a decision problem on which agents' decisions and use of information are evaluated. However, 
ambiguity around the appropriate decision problem, and in particular, the appropriate scoring rule, is not uncommon in human-AI decision settings. Ambiguity can arise as a result of challenges in eliciting utility functions and/or variance in these functions across decision-makers or groups of instances; for example, doctors penalize certain false negative results differently when diagnosing younger versus older patients~\citep{mclaughlin2022algorithmic}.
Blackwell's comparison of signals \citep{blackwell1951comparison} is an appropriate tool for addressing ambiguity about the payoff function, as it defines a (set of) signal $\sig_1$ as \textit{more informative} than $\sig_2$ if $\sig_1$ has a higher information value on all possible decision problems. 
We identify this partial order by decomposing the space of decision problems via a basis of proper scoring rules\footnote{For rational DMs, any decision problem can be represented by an equivalent proper scoring rule in \Cref{eq:properscoring}, such that the partial order defined via proper scoring rules also applies to the corresponding decisin tasks.}~\citep{li2022optimization, kleinberg2023u}.

\begin{definition}[Blackwell Order of Information]
    A signal $\sig_1$ is Blackwell more informative than $\sig_2$ if $\sig_1$ achieves a higher best-attainable payoff on any decision problems:
    \begin{equation*}
        \mathrm{R}^{\dgp, \score}
(\sig_1)\geq \mathrm{R}^{\dgp, \score}
(\sig_2), \forall \score
    \end{equation*}
\end{definition}
 \mvspace{-4mm}

The Blackwell order is evaluated over all possible decision problems, which cannot be tested directly.
Fortunately, we only need to test over all proper scoring rules since any decision problem can be represented by an equivalent proper scoring rule, and the space of proper scoring rules can be characterized by a set of V-shaped scoring rules.
A V-shaped scoring rule is parameterized by the kink of the piecewise-linear utility function.


% \begin{definition}
% Given an information model $\dgp$, we define the worst-case information value of $\sig$ as \[WCIV^{\dgp}(\sig) = \inf_{\score}IV^{\dgp, \score}(\sig)\]
% \end{definition}

\begin{definition}(V-shaped scoring rule)
 \label{def:V-shaped score}
 A V-shaped scoring rule with kink $\kink\in (0, \frac{1}{2}]$ is defined as    \begin{equation}
      \score_{\kink} (\action, \payoffstatevalue) = \left\{\begin{array}{cc}
      \frac{1}{2} -\frac{1}{2}\cdot \frac{\payoffstatevalue - \kink}{1-\kink}  &  \text{if }\action\leq \kink\\
        \frac{1}{2} +\frac{1}{2}\cdot \frac{\payoffstatevalue - \kink}{1-\kink}    & \text{else},
      \end{array}
      \right.\nonumber
   \end{equation}

When $\kink'\in (\frac{1}{2}, 1)$, the V-shaped scoring rule can be symmetrically defined by $\score_{\kink'} = \score_{1-\kink'}(1-\pred, \payoffstatevalue)$.
\end{definition}

 \mvspace{-2mm}

Intuitively, the kink $\kink$ represents the threshold belief where the decision-maker switches between two actions.
Larger $\mu$ means that the decision-makers will prefer $\action = 1$ more. The closer $\mu$ is to $0.5$, the more indifferent the decision-maker is to $\action = 0$ or $\action = 1$.

\Cref{prop: blackwell-V-test} shows that if $\sig_1$ achieves a higher information value on the basis of V-shaped proper scoring rules than $\sig_2$, then $\sig_1$ is Blackwell more informative than $\sig_2$. \Cref{prop: blackwell-V-test} follows from the fact that any best-responding payoff can be linearly decomposed into the payoff on V-shaped scoring rules. 

\begin{proposition}[\citealt{hu2024predict}]
\label{prop: blackwell-V-test}
If \(\forall \kink\in (0, 1)\) \begin{equation*}
    \mathrm{R}^{\dgp, \score_\kink}
(\sig_1)\geq \mathrm{R}^{\dgp, \score_\kink}
(\sig_2),
\end{equation*}
then $\sig_1$ is Blackwell more informative than $\sig_2$.
\end{proposition}

 \mvspace{-2mm}

Extending this to agent-complementary information value, we say that $\sig_1$ offers a higher complementary value than $\sig_2$ under the Blackwell order if 
\[\ACIV^{\dgp, \score_\kink}(\sig_1; \actionvar^b) \geq \ACIV^{\dgp, \score_\kink}(\sig_2; \actionvar^b), \forall \kink\in(0,1)\]
This definition allows us to rank signals (or sets of signals) without needing to commit to a specific payoff function. 
We present a use case in \Cref{exp2}.


%QA 任务

In this section, we present a comprehensive evaluation framework for the CondAmbigQA benchmark, which introduces a novel task of resolving ambiguous questions through explicit condition identification. Unlike traditional question answering tasks that directly generate answers, we propose that ambiguous questions should first be disambiguated by identifying explicit conditions that affect the answer, then generating appropriate responses for different condition combinations. This decomposition of the ambiguous QA process into condition identification and conditional answer generation represents a more structured approach to handling query ambiguity. Through carefully designed metrics and experimental protocols, our benchmark evaluates both the model's ability to identify and articulate these disambiguating conditions, and its capacity to generate condition-specific answers.

\subsection{Evaluation framework}
The CondAmbigQA benchmark adopts a multi-metric evaluation approach to comprehensively assess model performance. 
Let $M$ denote the model output, $G$ denote the ground truth, and $\textit{G-Eval}(x,y)$ represent the G-Eval function that evaluates the quality of output $x$ against reference $y$ based on pre-defined criteria~\cite{yao2024clave,liu2023g}. The four evaluation metrics are defined as follows:

\textit{Condition Score} measures the quality of condition identification:
\begin{equation}
\textit{Condition Score}(M,G) = \textit{G-Eval}(M.conditions, G.conditions),
\end{equation}
where the G-Eval function assesses both completeness and clarity of identified conditions.

\textit{Answer Score} evaluates the quality of generated answers:
\begin{equation}
\textit{Answer Score}(M,G) = \textit{G-Eval}(M.answers, G.answers),
\end{equation}
focusing on factual accuracy and condition-specific response quality.

\textit{Citation Score} quantifies source attribution accuracy:
\begin{equation}
\textit{Citation Score}(M,G) = \frac{|{c \in M.citations} \cap {c \in G.citations}|}{|{c \in G.citations}|},
\end{equation}
where citations are normalized and compared as sets to produce a score in [0,1].

\textit{Answer Count} measures response completeness:
\begin{equation}
\textit{Answer Count}(M,G) = |M.answer\ count - G.answer\ count|,
\end{equation}
reflecting the model's understanding of required answer granularity.

\subsection{Experimental protocol}

To evaluate the effectiveness of condition guidance in ambiguous question answering, we conduct two sets of experiments. 

In the main experiment, we assess models' native ability in condition identification and answer generation. Given a query $Q$ and retrieved passages $P$ (whole passages fragments) as input, models are required to first identify disambiguation conditions and then generate appropriate answers based on these \textbf{identified conditions}. Specifically, this protocol evaluates models' end-to-end capability in understanding and resolving query ambiguity through:

\begin{itemize}
   \item Condition identification: extracting key conditions that resolve ambiguity;
   \item Answer generation: providing appropriate answers based on identified conditions;
   \item Citation: supporting answers with relevant passages.
\end{itemize}

In the comparative experiment, we design two controlled settings to quantify the impact of condition guidance:

\begin{itemize}
  \item \textbf{Standard RAG}: Models directly generate answers from $Q$ and $P$ without explicit condition information;
   \item \textbf{Condition-guided}: Models receive additional ground-truth conditions alongside $Q$ and $P$.
 
\end{itemize}

This controlled comparison helps isolate the effect of condition guidance on answer quality and citation accuracy. By comparing model performance between these two settings, we can quantitatively assess how explicit condition information influences the quality of generated answers.

\subsection{Baseline models}
We evaluate our benchmark using five representative open-source language models: \texttt{LLaMA3.1} (8B) \cite{dubey2024llama}, trained on 1.2T tokens with optimized attention mechanism, \texttt{Mistral} (7B) \cite{jiang2023mistral}, known for its efficient architecture; \texttt{Gemma} (9B) \cite{team2024gemma}, trained on high-quality curated dataset, \texttt{GLM4} (9B) \cite{glm2024chatglm}, featuring enhanced cross-lingual abilities; and \texttt{Qwen2.5} (7B) \cite{yang2024qwen2}, optimized for comprehensive language understanding. These models, with parameters ranging from 7B to 9B, provide a diverse yet comparable foundation for baseline performance assessment.
To ensure reproducibility, all models are deployed through the \texttt{Ollama} framework, using default sampling parameters and 8K context window size. Model outputs are evaluated using G-Eval implemented via the \texttt{DeepEval} package, with \texttt{GPT4-mini} serving as the evaluation model through OpenAI's API.

\subsection{Scaling analysis}
To understand how model scale influences performance on our benchmark, we conduct additional experiments with two larger-scale models. This analysis aims to investigate whether performance on conditional ambiguous question answering follows established scaling laws~\cite{kaplan2020scaling}, providing insights into the relationship between model capacity and task performance. Through this evaluation framework, our benchmark provides a standardized way to assess and compare model performance in handling conditional ambiguous questions. The multi-metric approach and diverse experimental protocols enable detailed analysis of model capabilities. In particular, the scaling experiments validate the applicability of scaling laws to ambiguity resolution tasks, demonstrating that larger models consistently outperform smaller ones in condition adherence and answer quality. These findings offer valuable insights into the relationship between model size and performance, guiding future model development and optimization.
\section{Discussion}
\subsection{Case Study}


Fig. \ref{fig:casestudy} shows 2-D UMAP \cite{mcinnes2020umapuniformmanifoldapproximation} projections of embedding vectors for PetClinic Microservices \cite{microapps2024petclinic} using VoyageAI, ME-unixcoder-340K, and ME-llm2vec-340K. ME-unixcoder and ME-llm2vec show clearer microservice clusters compared to VoyageAI and Fig. \ref{fig:mexample}. For instance, \textit{API-Gateway} service classes are split in VoyageAI's representation but closer in the other models. ME-llm2vec demonstrates the closest grouping within microservices and clearest separation between them. In fact, ME-llm2vec's figure shows only 6 clear outliers which we review in detail and display their names and neighbors.



The two \textit{MetricConfig} classes, \textit{ResourceNotFoundException} and \textit{CacheConfig} lack domain-specific terms since they are utility classes, which highlights the importance of separating them from domain-related ones during the decomposition. However, ME-llm2vec was able correctly represent classes with even slight domain hints. For instance, most models struggle to differentiate between the nearly identical entry-point classes (e.g. \textit{ConfigServerApplication}), as seen in Fig. \ref{fig:mexample} and \ref{fig:casestudy} while ME-llm2vec managed to correctly place them within their services. On the other hand, the class \textit{PetRequest}, which was closer to \textit{API-Gateway} instead of \textit{Customers}, shows an intriguing outlier. Despite ME-llm2vec correctly matching the "Pet" related classes, it failed with \textit{PetRequest}. its function as a Request object, which is typically associated with the Gateway pattern, is a potential reason. Notably, ME-llm2vec successfully identified \textit{API-Gateway} classes, differentiating them from \textit{Customers}. We find this interesting because \textit{API-Gateway} includes classes representing various bounded contexts, often causing confusion in other models. ME-llm2vec recognized these classes' distinct purpose, grouping them together despite their diverse domains.

% Both \textit{API-Gateway} and \textit{Customers} services contain a "PetType" class. But in \textit{Customers}'s case, this class was closer to the "Specialty" class from \textit{Vets}, which is likely due to nearly identical source code they have. 

\subsection{Discussion}


We designed the analysis component to be as abstract as possible to accommodate the rapidly evolving representation learning landscape. As new and improved embedding models are published, they can be integrated with minimal effort. While our evaluation results show that with ME-LLM2Vec, we can generate highly cohesive and consistent decompositions, one of our objectives is to highlight the potential of Language Models in generating more efficient representations than traditional approaches for the decomposition problem. In fact, MonoEmbed is both a decomposition approach (when considering the full approach) and an embedding model (when using models such as ME-LLM2Vec). These models can be used to enrich existing decomposition approaches. For example, MicroMiner's CodeBERT \cite{trabelsi2023microminer} can be replaced with ME-LLM2Vec and the GNN based methods \cite{desai2021cogcn,yedida2023deeply,mathai2022chgnn,qian2023gdcdvf} can be extended by using ME-LLM2Vec as the encoder. In fact, it can be used as an additional representation type in approaches such as \cite{khaled2022hydecomp,qian2023gdcdvf}. These models can be even extended further by incorporating unstructured inputs (e.g. resources, configurations, documentation) and different PLs.




\subsection{Threats to Validity}
\subsubsection{Internal Validity}
Clustering algorithms and decomposition approaches have hyper-parameters that can affect performance on evaluation benchmarks. To mitigate this threat, we compared their performance with different hyper-parameter inputs across a varied set of evaluation applications.

\subsubsection{External Validity}
To address the threat of our approach to generalize on monolithic applications and PLs, we used a large set of monolithic and microservices applications from related work \cite{kalia2021mono2micro,khaled2022hydecomp,yedida2023deeply,jin2021fosci} to benchmark decomposition approaches. 

\subsubsection{Construct Validity}
This threat can potentially be in the form of the evaluation metrics used in our experiments. In order to mitigate this threat, we employ established metrics in supervised learning tasks (RQ1-3) and different metrics from decomposition research \cite{khaled2022hydecomp,kalia2021mono2micro,jin2021fosci,yedida2023deeply,mathai2022chgnn} (RQ4). 


% \begin{ack}
% We thank \citet{groh2022deepfake}, who provided their
% data for demonstration in this paper.
% We thank the annonymous reviewers for feedback.
% \end{ack}

\bibliographystyle{plainnat}
\bibliography{ref}

\appendix
\newpage
\lstset{language=Python,basicstyle=\small\ttfamily,columns=fullflexible}
\section{Detailed Experimental Setup}\label{appendix:experiments}

For reproducibility and completeness, we provide comprehensive details of all setups, datasets, tasks, models, baselines, and hyperparameters. Code is in the process of being released.

\subsection{Tasks, Datasets, and Data Partitioning}

\noindent\textbf{XNLI~\cite{XNLI}}~A natural language inference benchmark dataset for evaluating cross-lingual understanding covering 15 diverse languages including both high- and low-resources languages: English, French, Spanish, German, Greek, Bulgarian, Russian, Turkish, Arabic, Vietnamese, Thai, Chinese, Hindi, Swahili and Urdu. XNLI consists of premise-hypothesis pairs, labeled as entailment, contradiction, or neutral across different languages. We sample 2k instances from the XNLI train split and 500 instances from the test split for each pool. The data is then divided equally among 20 clients for each language using the latent Dirichlet allocation (LDA) partition with $\alpha=0.5$. Hence, the total number of clients is 600 (15 languages $\cdot$ 20 clients per language $\cdot$ 2 pools).

\noindent\textbf{MasakhaNEWS~\cite{MasakhaNEWS}}~A news topic classification benchmark designed to address the lack of resources for African languages. It covers 2 high-resource languages, English and French, and 14 low-resource languages, namely Amharic, Hausa, Igbo, Lingala, Luganda, Naija, Oromo, Rundi, chiShona, Somali, Kiswahili, Tigrinya, isiXhosa, and Yorùbá. Each sample contains a headline, the body text, and one of the 7 labels: business, entertainment, health, politics, religion, sports, or technology. We first combine all samples from the MasakhaNEWS train and validation split to form our train set, and use the MasakhaNEWS test split as our test set. We then split both train and test in each of the 16 languages by half for each pool. Following our XNLI setup, we adopt LDA $\alpha=0.5$ and divide each language's data equally into 10 clients. Hence, the total number of clients is 320 (16 languages $\cdot$ 10 clients per language $\cdot$ 2 pools). Note that unlike XNLI, the number of samples for each language differs, hence there is quantity skew across clients. 

\noindent\textbf{Fed-Aya~\cite{fedllm-bench}}~A federated instruction tuning benchmark, based on Aya~\cite{singh2024aya}, where the data is annotated by contributors and partitioned by annotator ID. Following FedLLM-Bench~\cite{fedllm-bench}, we focus on 6 high-resource languages, English, Spanish, French, Russian, Portuguese, Chinese, and 2 low-resource languages, Arabic and Telugu. Additionally, we filter out the other languages from the dataset. Out of 38 clients, we select 8 for our \unseen{} pool, $\text{client\_ids}=[21, 22, 23, 24, 25, 26, 27, 34]$ and the rest goes into our \seen{} pool. Each client can have up to 4 languages where the number of data samples can range from a hundred to over a thousand samples per client.

\subsection{Models, Tokenizers, and Data Preprocessing}

\noindent\textbf{mBERT~\cite{BERT}.}~We use the pretrained multilingual BERT with its WordPiece tokenizer for all sequence classification experiments, namely all XNLI and MasakhaNEWS setups with various {\em base models}. For both datasets, we use a training batch size of 32 and pad input tokens on the right to a max token length of 128 and 256 respectively.

\noindent\textbf{MobileLLaMA-1.4B~\cite{mobilellama}.}~We train a \basemodel{} with a pretrained MobileLLaMA-1.4B with Standard FL using LoRA in our Fed-Aya setup. We use the default LLaMA tokenizer which is a BPE model based on sentencepiece~\cite{Kudo2018SentencePieceAS} and adopt the UNK token as the PAD token. During training, we use an effective batch size of $16$ and pad right to the longest token in the batch up to a max token length of 1024. For evaluation, we use a batch size of 8, padding left instead, with greedy sampling up to a max new token length of 1024. We use the Alpaca template to format each prompt:

\begin{lstlisting}[linewidth=\columnwidth,breaklines=true]
alpaca_template = """Below is an instruction that describes a task. Write a response that appropriately completes the request.

### Instruction:
{} 

### Response: {}{}"""
\end{lstlisting}

\noindent\textbf{Llama-3.2-3B~\cite{llama3}.}~We use the off-the-shelf Llama-3.2-3B-Instruct model as our \basemodel{} and its default tokenizer which is a BPE model based on tiktoken\footnote{https://github.com/openai/tiktoken}. Training and evaluation hyperparameters are the same as the ones we use for MobileLLaMA. The only two differences are \textit{1)} we add a PAD token `\verb|<pad>|', and \textit{2)} we use the Llama 3 instruction template instead:

\begin{lstlisting}[linewidth=\columnwidth,breaklines=true]
llama3_instruct_template = """<|begin_of_text|><|start_header_id|>user<|end_header_id|>

{}<|eot_id|><|start_header_id|>assistant<|end_header_id|>

{}{}
"""
\end{lstlisting}

\subsection{Complementary Approaches and Base Models}

In this work, we experiment with different {\em base models} to show that \method{} is complementary to a range of off-the-shelf models and models trained using existing FL approaches. In this section, we detail the different approaches we used to obtain these {\em base models}.

\noindent\textbf{Standard FL.} Standard FL involves training a single global model. Given a pretrained LLM, we run FedAvg on the \seen{} pool of clients, where $10\%$ of participating clients are sampled every round to train the model before sending the weights back for aggregation. In our XNLI and MasakhaNEWS setup, we do full fine-tuning of mBERT, setting each client's learning rate to $5e-5$ and running FedAvg for 100 rounds. In our Fed-Aya setup, we adopt the training recipe from FedLLM-Bench~\cite{fedllm-bench} for MobileLLaMA-1.4B, where we do PEFT with LoRA applied to query and value attention weights ($r=16$, $\alpha_{lora}=32$, dropout$=0.05$) for 200 rounds. We use the cosine learning rate decay over 200 rounds with initial learning rate $2e^{-5}$ and minimum learning rate $1e^{-6}$.

\noindent\textbf{Personalized FL.} We train personalized {\em base models} using FedDPA-T~\cite{FedDPA} and DEPT(SPEC)~\cite{DEPT} in our XNLI setup. FedDPA-T proposed having two separate LoRA adapters, one of which is shared (global) and the other is kept locally for each client (local). 
We adopted their training recipe for sequence classification, where the classifier is shared together with the global LoRA modules and the local LoRA modules stay local. LoRA modules are only applied to query and value attention weights (r=8, $\alpha_{lora}=8$, dropout=$0.05$). We set the learning rate to $5e^{-5}$.

DEPT (SPEC), on the other hand, proposed having personalized token and positional embeddings for each client. As DEPT was proposed for cross-silo FL, while we target cross-device FL, they assumed that each data source has an abundance of data to retrain the newly initialized embeddings. Hence, to adapt to the cross-device FL setting, we did not reinitialize the embeddings; each client fine-tunes their own embeddings starting from the pretrained mBERT embeddings. In other words, for each FL round, each client does full fine-tuning, sending weights of all layers except their own embeddings back to the server for aggregation. As with FedDPA-T, the learning rate is set to $5e^{-5}$.

\noindent\textbf{Off-the-shelf.} In our Fed-Aya setup, we use an off-the-shelf instruction finetuned Llama-3.2-3B-Instruct as our \basemodel{}.

\subsection{Baselines and \method{}}

To avoid an exponentially big search space, all hyperparameter tuning is done using simple grid search on our XNLI setup, with mBERT, and Fed-Aya setup, with MobileLLaMA as the \basemodel{}. The best hyperparameters found are then used for MasakhaNEWS and Fed-Aya with Llama3 respectively.

\noindent\textbf{LoRA PEFT~\cite{hu2021lora}.}~We search for the learning rate $[1e^{-5},1e^{-4},1e^{-3}]$ and the number of epochs $[1,2,3]$ and find that the learning rate $1e^{-4}$ with $2$ epochs had the best performance on the train set. We fixed $\alpha_{lora}=2r$. To ensure a similar inference budget across baselines, we set the number of epochs to $2$ for all our experiments.

\noindent\textbf{AdaLoRA~\cite{adalora}.}~Similarly to LoRA, we search for the learning rate $[1e^{-5},1e^{-4},1e^{-3},1e^{-2}]$, the time interval between two budget allocations, $\Delta_T$, $[1,10,100]$ and the coefficient of the orthogonal regularization, $\gamma$, $[0.1,0.5]$. Within our search space, we find learning rate$=1e^{-3}$, $\Delta_T=1.0$, and $\gamma=0.1$ to be the best performing one. We run AdaLoRA once per resource budget $r$, setting the initial rank to be $1.5\times$ of $r$, as recommended. We set the initial fine-tuning warmup steps and final fine-tuning steps to be $10\%$ and $30\%$ of the total steps respectively. 

\noindent\textbf{BayesTune-LoRA (Section\ref{sec:personalized_peft}).}~For fair comparison with \method{}, we use the same hyperparameters as \method{}. This baseline, hence, is an ablation study of how much performance collaboratively learning a PSG adds.

\noindent\textbf{FedL2P~\cite{royson2023fedl2p}}~As FedL2P requires a validation set for outer-loop bi-level optimization and federated early stopping, we split the train set of every client $80\%$ train and $20\%$ validation. Following FedL2P, we set the federated early stopping patience to 50 rounds, MLP hidden dimension is set to 100, the inner-loop learning rate to be the same as finetuning, $1e^{-4}$, and the hypergradient hyperparameters, $Q=3, \psi=0.1$ with hypergradient clamping of $[-1,1]$. 
We use Adam for both the inner-loop and outer-loop optimizers and search for the learning rate for the MLP (LRNet) $[1e^{-5},1e^{-4},1e^{-3}]$ and the learnable post-multiplier learning rate $\tilde{\eta}$ $[1e^{-4},1e^{-5},1e^{-6}]$, picking $1e^{-4}$ and $1e^{-6}$ to be the best respectively. Finally, we use $3$ outer-loop steps with an effective outer-loop batch size of $16$.

\noindent\textbf{\method{} (Section~\ref{sec:main_method})}~We set $\alpha_{r\_mul}=2$, our resulting $r_{init}$ is, hence, $32$ since our $r_{\text{max target}}=16$ for all experiments. Following our standard LoRA fine-tuning baseline, we adopt the same learning rate and $\alpha_{lora}$, $1e-4$ and $2r_{init}$ respectively. The learning rate of $\bm{\lambda}$, on the other hand, is searched $[1e^{-1},1e^{-2},1e^{-3},1e^{-4}]$, and we pick $1e^{-2}$ for all experiments.
All $\lambda$ values are initialized to $1e^{-4}$. The MLP hidden dimension is set to $2\times$ the input dimension, which is model dependent. We clamp the output of the MLP as well as $\lambda$ with a minimum value of $1e-4$ in the forward pass during training. Following FedL2P, we use a straight-through estimator~\cite{bengio2013estimating} after clamping to propagate gradients. We initialize the weights of MLP with Xavier initialization~\cite{glorot2010understanding} using the normal distribution with a gain value of $1e-6$ and the bias with a constant $1e-4$. Lastly, we set $\alpha_s=1e^{+2}$ and $\alpha_p=1e^{-2}$ for all experiments.


\section{Additional Results}

This section contains supplementary results and analyses, omitted from the main paper due to space limitations, that complement the presented findings.
Fig.~\ref{fig:xnli_dept_out_r16}-\ref{fig:llama3_fedavg_out_r2} show the language-agnostic rank structures under different budgets ($r=2$ and $r=16$) learnt by \method{} for different setups as mentioned in Section~\ref{sec:analysis}. These plots illustrate the prioritization of layers for LoRA fine-tuning. 

Note that while the rank structure is the same across languages, the strength of personalization, absolute value of $\bm{\lambda}$, differs, as shown in Fig.~\ref{fig:masakha_out} in this Section and Fig.~\ref{fig:xnli_out} in the main paper. These two figures show the difference in $\bm{\lambda}$ across languages as described in Section~\ref{sec:analysis}. To sum up, the smaller the distance between two languages, represented as a block in the figure, the more similar their generated $\bm{\lambda}$ are. The results show that while similar languages sometimes exhibit similar $\bm{\lambda}$ values, unrelated languages also occasionally share similar $\bm{\lambda}$, consistent with findings in the literature that leveraging dissimilar languages can be beneficial.

Lastly, Tables~\ref{tab:xnli_seen} and \ref{tab:xnli_unseen} contain results for our XNLI setup where the \basemodel{} is fine-tuned from the pretrained mBERT with Standard FL using full fine-tuning and is used to complement results and findings in Section~\ref{sec:text_class}. Similarly, Tables~\ref{tab:mobilellama_fedaya_seen} and \ref{tab:mobilellama_fedaya_unseen} complement the results and findings of our Fed-Aya setup described in Section~\ref{sec:ift_gen}.


\begin{figure}[t]
    \small
    \centering
    \includegraphics[width=1.0\columnwidth]{figures/masakha_out_0.5_seen.png}
    % \captionsetup{font=small,labelfont=bf}
    \vspace{-3em}
    \caption{$\bm{\lambda}$ distance among languages in our MasakhaNEWS setup. Each block shows the log-scale normalized average Euclidean distances between all pairs of clients' $\bm{\lambda}$ for two languages (see text). The smaller the distance, the more similar $\bm{\lambda}$ is. }
    \label{fig:masakha_out}
    % \vspace{-2em}
\end{figure}

\begin{table*}[t]
\centering
\caption{Mean±SD Accuracy of each language across 3 different seeds for clients in the \seen{} pool of our XNLI setup. The pretrained model is trained using Standard FL with full fine-tuning and the resulting \basemodel{} is personalized to each client given a baseline approach.}
\label{tab:xnli_seen}
\begin{scriptsize}\resizebox{0.98\textwidth}{!}{

\begin{tabular}{c|l|l|l|l|l|l|l|l|l|l|l|l|l|l|l|l|c}
\toprule
% \textbf{Lora Rank}  

\textbf{$\mathbf{r}$} & \multicolumn{1}{c|}{\textbf{Approach}} & \multicolumn{1}{c|}{\textbf{bg}} & \multicolumn{1}{c|}{\textbf{hi}} & \multicolumn{1}{c|}{\textbf{es}} & \multicolumn{1}{c|}{\textbf{el}} & \multicolumn{1}{c|}{\textbf{vi}} & \multicolumn{1}{c|}{\textbf{tr}} & \multicolumn{1}{c|}{\textbf{de}} & \multicolumn{1}{c|}{\textbf{ur}} & \multicolumn{1}{c|}{\textbf{en}} & \multicolumn{1}{c|}{\textbf{zh}} & \multicolumn{1}{c|}{\textbf{th}} & \multicolumn{1}{c|}{\textbf{sw}} & \multicolumn{1}{c|}{\textbf{ar}} & \multicolumn{1}{c|}{\textbf{fr}} & \multicolumn{1}{c|}{\textbf{ru}} & \textbf{Wins} \\ \midrule
% \multirow{5}{*}{1}  & LoRA                                   & 46.60±0.28                        & 45.20±0.16                        & 50.00±0.28                        & 49.60±0.16                        & 48.73±0.19                       & 47.67±0.19                       & 48.93±0.09                       & 48.33±0.19                       & 51.27±0.09                       & 49.27±0.09                       & 44.60±0.16                        & 43.20±0.33                        & 44.40±0.28                        & 49.53±0.19                       & 47.13±0.09                       & 0             \\ %\cline{2-18} 
%                     & AdaLoRA                              & 45.67±0.09                       & 44.47±0.09                       & 48.80±0.16                        & 49.07±0.09                       & 48.27±0.25                       & 47.07±0.50                       & 48.53±0.25                       & 47.93±0.09                       & 51.00±0.16                        & 48.53±0.09                       & 43.87±0.34                       & 42.80±0.33                        & 44.20±0.28                        & 48.93±0.34                       & 46.20±0.28                        & 0             \\ %\cline{2-18} 
%                     & BayesTune-LoRA                            & 45.67±0.09                       & 44.33±0.09                       & 48.53±0.09                       & 48.80±0.16                        & 47.80±0.00                        & 46.80±0.16                        & 48.33±0.09                       & 47.87±0.19                       & 50.93±0.09                       & 48.07±0.25                       & 43.60±0.16                        & 42.27±0.25                       & 44.00±0.16                        & 48.67±0.09                       & 45.60±0.16                        & 0             \\ %\cline{2-18} 
%                     & FedL2P                               & \textbf{65.47±1.39}              & \textbf{69.47±1.43}              & \textbf{73.53±1.27}              & \textbf{69.33±1.06}              & \textbf{70.13±1.47}              & \textbf{71.00±1.34}               & \textbf{72.00±0.99}               & \textbf{72.13±0.25}              & \textbf{74.73±0.93}              & \textbf{67.40±1.07}               & \textbf{67.80±0.75}               & \textbf{71.93±0.34}              & \textbf{70.80±1.45}               & \textbf{72.67±0.66}              & \textbf{72.20±1.02}               & \textbf{15}   \\ %\cline{2-18} 
%                     & \method{}                                 & 62.93±1.11                       & 63.60±1.4                         & 70.40±1.82                        & 65.80±1.61                        & 67.67±1.05                       & 65.07±2.17                       & 67.67±0.66                       & 67.00±1.42                        & 73.53±0.62                       & 62.33±0.77                       & 62.93±0.50                       & 69.13±0.41                       & 64.93±1.89                       & 67.47±1.25                       & 66.87±1.15                       & 0             \\ \hline
\multirow{5}{*}{2}  & LoRA                                   & 47.47±0.19                       & 45.93±0.34                       & 50.80±0.16                        & 50.80±0.16                        & 50.80±0.28                        & 48.80±0.59                        & 50.07±0.66                       & 49.67±0.47                       & 53.13±0.41                       & 50.00±0.28                        & 45.47±0.41                       & 44.33±0.09                       & 45.33±0.25                       & 51.00±0.16                        & 48.33±0.74                       & 0             \\ %\cline{2-18} 
                    & AdaLoRA                              & 45.73±0.09                       & 44.40±0.00                        & 49.00±0.28                        & 49.13±0.19                       & 48.00±0.16                        & 47.07±0.34                       & 48.27±0.09                       & 47.87±0.09                       & 50.93±0.09                       & 48.20±0.16                        & 43.80±0.16                        & 43.00±0.00                        & 44.20±0.16                        & 48.87±0.09                       & 46.27±0.25                       & 0             \\ %\cline{2-18} 
                    & BayesTune-LoRA                            & 45.67±0.19                       & 44.40±0.00                        & 48.33±0.25                       & 48.80±0.16                        & 48.13±0.19                       & 46.80±0.28                        & 48.13±0.09                       & 47.80±0.00                        & 50.87±0.19                       & 48.00±0.00                        & 43.53±0.25                       & 42.13±0.19                       & 44.13±0.09                       & 48.93±0.09                       & 45.67±0.09                       & 0             \\ %\cline{2-18} 
                    & FedL2P                               & 66.67±0.90                       & 69.47±0.77                       & 74.33±0.93                       & 70.73±1.00                        & 71.27±0.82                       & 71.27±0.57                       & 72.27±1.32                       & 73.20±0.28                        & 75.27±0.81                       & 68.27±1.32                       & 67.93±0.75                       & 73.47±1.31                       & 71.47±0.94                       & 72.80±0.28                        & 73.07±0.25                       & 0             \\ %\cline{2-18} 
                    & \method{}                                 & \textbf{71.73±0.41}              & \textbf{72.33±0.25}              & \textbf{75.40±0.59}               & \textbf{73.73±0.84}              & \textbf{74.80±0.43}               & \textbf{74.73±0.41}              & \textbf{75.00±0.71}               & \textbf{74.33±0.57}              & \textbf{75.47±0.19}              & \textbf{70.53±0.52}              & \textbf{71.33±0.09}              & \textbf{73.73±0.41}              & \textbf{75.60±0.57}               & \textbf{74.60±0.59}               & \textbf{74.13±0.68}              & \textbf{15}   \\ \hline
\multirow{5}{*}{4}  & LoRA                                   & 49.40±0.16                        & 50.00±0.86                        & 54.47±0.62                       & 53.33±0.41                       & 53.00±0.28                        & 51.53±0.66                       & 53.27±0.62                       & 52.93±1.09                       & 56.27±0.47                       & 52.20±0.43                        & 49.07±0.62                       & 48.80±0.00                        & 49.20±0.33                        & 54.27±0.34                       & 51.73±1.11                       & 0             \\ %\cline{2-18} 
                    & AdaLoRA                              & 45.87±0.19                       & 44.33±0.09                       & 48.60±0.16                        & 48.93±0.09                       & 48.27±0.34                       & 46.87±0.09                       & 48.47±0.09                       & 47.80±0.00                        & 51.27±0.09                       & 48.07±0.09                       & 43.67±0.09                       & 42.73±0.25                       & 44.20±0.28                        & 48.73±0.09                       & 45.93±0.19                       & 0             \\ %\cline{2-18} 
                    & BayesTune-LoRA                            & 45.60±0.00                        & 44.33±0.09                       & 48.87±0.19                       & 49.13±0.09                       & 48.00±0.28                        & 47.00±0.33                        & 48.47±0.09                       & 47.80±0.16                        & 51.13±0.09                       & 48.13±0.09                       & 43.73±0.19                       & 42.40±0.16                        & 44.20±0.00                        & 48.93±0.09                       & 46.00±0.16                        & 0             \\ %\cline{2-18} 
                    & FedL2P                               & 67.40±1.42                        & 69.73±1.37                       & 74.47±0.34                       & 70.20±0.85                        & 71.53±0.94                       & 71.27±0.90                       & 72.73±0.94                       & 73.07±0.34                       & 75.27±0.47                       & 68.73±1.16                       & 68.27±0.47                       & 73.73±0.66                       & 71.47±1.64                       & 73.47±0.50                       & 72.87±0.52                       & 0             \\ %\cline{2-18} 
                    & \method{}                                 & \textbf{72.80±0.33}               & \textbf{72.13±0.41}              & \textbf{75.40±0.28}               & \textbf{74.27±0.34}              & \textbf{74.93±0.09}              & \textbf{75.20±0.28}               & \textbf{75.80±0.16}               & \textbf{75.07±0.25}              & \textbf{75.93±0.25}              & \textbf{71.60±0.16}               & \textbf{70.80±0.28}               & \textbf{73.07±0.25}              & \textbf{75.53±0.62}              & \textbf{75.87±0.34}              & \textbf{74.87±0.19}              & \textbf{15}   \\ \hline
\multirow{5}{*}{8}  & LoRA                                   & 55.33±1.60                       & 54.13±0.09                       & 59.93±0.50                       & 58.20±1.28                        & 58.07±0.25                       & 56.53±0.25                       & 59.73±0.09                       & 58.47±1.33                       & 63.40±0.71                        & 57.13±0.34                       & 56.13±0.19                       & 56.13±0.50                       & 55.93±0.09                       & 59.20±0.33                        & 58.20±2.01                        & 0             \\ %\cline{2-18} 
                    & AdaLoRA                              & 45.80±0.00                        & 44.40±0.00                        & 48.47±0.09                       & 49.00±0.16                        & 48.13±0.09                       & 46.93±0.09                       & 48.27±0.09                       & 48.00±0.16                        & 50.80±0.16                        & 48.07±0.09                       & 43.73±0.09                       & 42.40±0.16                        & 44.20±0.16                        & 48.93±0.09                       & 45.93±0.09                       & 0             \\ %\cline{2-18} 
                    & BayesTune-LoRA                            & 45.87±0.25                       & 44.33±0.09                       & 49.13±0.47                       & 49.27±0.09                       & 48.20±0.00                        & 47.07±0.09                       & 48.60±0.16                        & 47.80±0.00                        & 51.00±0.28                        & 48.60±0.16                        & 44.13±0.09                       & 42.93±0.09                       & 44.40±0.00                        & 49.20±0.43                        & 46.20±0.16                        & 0             \\ %\cline{2-18} 
                    & FedL2P                               & 64.73±1.27                       & 66.07±2.19                       & 72.13±1.80                       & 67.60±1.42                        & 68.80±1.85                        & 68.47±1.89                       & 69.40±1.99                        & 70.93±2.22                       & 73.47±1.18                       & 65.93±1.86                       & 65.73±1.98                       & 70.73±2.96                       & 67.73±1.93                       & 70.53±1.93                       & 70.47±1.89                       & 0             \\ %\cline{2-18} 
                    & \method{}                                 & \textbf{73.93±0.34}              & \textbf{70.67±0.09}              & \textbf{75.87±0.19}              & \textbf{73.80±0.16}               & \textbf{74.33±0.47}              & \textbf{75.60±0.16}               & \textbf{74.93±0.09}              & \textbf{74.27±0.19}              & \textbf{76.47±0.09}              & \textbf{72.53±0.52}              & \textbf{70.67±0.09}              & \textbf{71.67±0.34}              & \textbf{76.87±0.25}              & \textbf{75.87±0.25}              & \textbf{75.00±0.57}               & \textbf{15}   \\ \hline
\multirow{5}{*}{16} & LoRA                                   & 63.93±1.46                       & 64.20±0.00                        & 70.40±0.16                        & 67.07±1.32                       & 68.53±0.25                       & 66.07±0.66                       & 68.13±0.25                       & 69.67±0.62                       & 72.47±0.90                       & 64.67±1.24                       & 65.47±1.11                       & 69.87±0.90                       & 67.20±0.16                        & 68.73±0.09                       & 68.00±0.85                        & 0             \\ %\cline{2-18} 
                    & AdaLoRA                              & 45.80±0.16                        & 44.40±0.00                        & 48.73±0.25                       & 48.93±0.19                       & 48.07±0.19                       & 47.00±0.33                        & 48.27±0.09                       & 47.73±0.09                       & 50.93±0.19                       & 48.20±0.16                        & 43.47±0.09                       & 42.53±0.09                       & 44.20±0.00                        & 48.93±0.09                       & 45.80±0.00                        & 0             \\ %\cline{2-18} 
                    & BayesTune-LoRA                            & 46.27±0.09                       & 44.93±0.19                       & 49.67±0.25                       & 49.60±0.00                        & 48.60±0.16                        & 47.47±0.19                       & 49.00±0.00                        & 48.00±0.00                        & 51.47±0.19                       & 49.00±0.16                        & 44.67±0.09                       & 43.07±0.25                       & 44.33±0.25                       & 49.13±0.52                       & 46.67±0.34                       & 0             \\ %\cline{2-18} 
                    & FedL2P                               & 61.53±2.88                       & 62.33±3.44                       & 68.73±3.60                       & 65.07±1.88                       & 65.13±3.10                        & 65.07±3.20                        & 65.80±3.59                        & 67.73±2.87                       & 71.33±2.13                       & 62.73±2.78                       & 62.27±3.27                       & 67.20±3.92                        & 63.93±3.77                       & 68.07±2.88                       & 67.20±3.43                        & 0             \\ %\cline{2-18} 
                    & \method{}                                 & \textbf{73.87±0.09}              & \textbf{70.80±0.16}               & \textbf{75.87±0.09}              & \textbf{73.73±0.68}              & \textbf{74.93±0.25}              & \textbf{75.33±0.25}              & \textbf{74.40±0.16}               & \textbf{74.13±0.19}              & \textbf{76.53±0.19}              & \textbf{72.40±0.28}               & \textbf{70.80±0.28}               & \textbf{71.13±0.62}              & \textbf{76.20±0.43}               & \textbf{75.87±0.09}              & \textbf{75.13±0.09}              & \textbf{15}   \\ \bottomrule
\end{tabular}
}
\end{scriptsize}
\vspace{-1.5em}
\end{table*}

\begin{table*}[t]
\centering
\caption{Mean±SD Accuracy of each language across 3 different seeds for clients in the \unseen{} pool of our XNLI setup. The pretrained model is trained using Standard FL with full fine-tuning and the resulting \basemodel{} is personalized to each client given a baseline approach.}
\label{tab:xnli_unseen}
\begin{scriptsize}\resizebox{0.98\textwidth}{!}{

\begin{tabular}{c|l|l|l|l|l|l|l|l|l|l|l|l|l|l|l|l|c}
\toprule
% \textbf{Lora Rank}  
\textbf{$\mathbf{r}$} & \multicolumn{1}{c|}{\textbf{Approach}} & \multicolumn{1}{c|}{\textbf{bg}} & \multicolumn{1}{c|}{\textbf{hi}} & \multicolumn{1}{c|}{\textbf{es}} & \multicolumn{1}{c|}{\textbf{el}} & \multicolumn{1}{c|}{\textbf{vi}} & \multicolumn{1}{c|}{\textbf{tr}} & \multicolumn{1}{c|}{\textbf{de}} & \multicolumn{1}{c|}{\textbf{ur}} & \multicolumn{1}{c|}{\textbf{en}} & \multicolumn{1}{c|}{\textbf{zh}} & \multicolumn{1}{c|}{\textbf{th}} & \multicolumn{1}{c|}{\textbf{sw}} & \multicolumn{1}{c|}{\textbf{ar}} & \multicolumn{1}{c|}{\textbf{fr}} & \multicolumn{1}{c|}{\textbf{ru}} & \textbf{Wins} \\ \hline
% \multirow{5}{*}{1}  & LoRA                                   & 49.27±0.19                       & 45.47±0.09                       & 50.33±0.09                       & 48.20±0.00                        & 47.60±0.00                        & 44.87±0.09                       & 49.47±0.25                       & 46.60±0.00                        & 55.33±0.19                       & 49.60±0.43                        & 44.53±0.09                       & 40.27±0.09                       & 46.33±0.09                       & 48.53±0.25                       & 44.80±0.16                        & 0             \\ %\cline{2-18} 
%                     & AdaLoRA                              & 49.20±0.00                        & 45.40±0.16                        & 50.33±0.09                       & 48.27±0.09                       & 47.13±0.09                       & 44.67±0.25                       & 49.33±0.09                       & 46.53±0.09                       & 54.80±0.28                        & 48.53±0.25                       & 44.87±0.09                       & 40.07±0.09                       & 46.67±0.25                       & 48.80±0.00                        & 44.47±0.09                       & 0             \\ %\cline{2-18} 
%                     & BayesTune-LoRA                            & 49.00±0.00                        & 45.20±0.00                        & 50.07±0.09                       & 47.93±0.09                       & 47.07±0.09                       & 44.53±0.09                       & 49.27±0.25                       & 46.20±0.00                        & 54.60±0.16                        & 48.53±0.34                       & 44.73±0.09                       & 39.93±0.09                       & 46.73±0.09                       & 48.80±0.00                        & 44.53±0.19                       & 0             \\ %\cline{2-18} 
%                     & FedL2P                               & \textbf{54.67±0.68}              & \textbf{50.80±0.16}               & \textbf{56.40±0.57}               & \textbf{50.40±0.59}               & \textbf{53.73±0.62}              & 48.00±0.16                        & \textbf{53.73±0.19}              & \textbf{49.47±0.66}              & \textbf{59.87±0.25}              & \textbf{52.00±0.59}               & \textbf{47.93±0.41}              & 43.73±0.50                       & 50.33±0.57                       & 53.07±0.41                       & 48.47±0.25                       & \textbf{10}   \\ %\cline{2-18} 
%                     & \method{}                                 & 52.93±0.25                       & 48.80±1.23                        & 55.73±0.77                       & 49.93±0.19                       & 52.80±0.71                        & \textbf{48.47±0.47}              & 52.60±0.49                        & 48.93±0.25                       & 57.80±0.49                        & 50.80±0.16                        & 47.27±0.52                       & \textbf{44.07±0.50}              & \textbf{49.07±0.47}              & \textbf{53.67±0.25}              & \textbf{48.60±0.16}               & 5             \\ \hline
\multirow{5}{*}{2}  & LoRA                                   & 49.33±0.09                       & 45.93±0.25                       & 51.13±0.09                       & 48.53±0.25                       & 48.53±0.25                       & 45.27±0.25                       & 49.80±0.16                        & 46.87±0.34                       & 55.40±0.28                        & 49.33±0.25                       & 44.40±0.00                        & 41.20±0.16                        & 46.87±0.09                       & 48.33±0.09                       & 45.13±0.09                       & 0             \\ %\cline{2-18} 
                    & AdaLoRA                              & 49.33±0.09                       & 45.27±0.09                       & 50.27±0.19                       & 47.80±0.16                        & 47.27±0.09                       & 44.60±0.00                        & 49.33±0.34                       & 46.60±0.16                        & 54.87±0.25                       & 48.73±0.19                       & 44.73±0.09                       & 40.07±0.09                       & 46.67±0.09                       & 48.67±0.09                       & 44.53±0.09                       & 0             \\ %\cline{2-18} 
                    & BayesTune-LoRA                            & 49.13±0.09                       & 45.13±0.19                       & 50.20±0.00                        & 48.00±0.16                        & 47.07±0.19                       & 44.67±0.09                       & 49.07±0.25                       & 46.47±0.19                       & 54.67±0.09                       & 48.73±0.19                       & 44.80±0.16                        & 40.07±0.09                       & 46.60±0.16                        & 48.73±0.09                       & 44.27±0.09                       & 0             \\ %\cline{2-18} 
                    & FedL2P                               & \textbf{54.40±0.43}               & 49.93±0.25                       & 56.80±0.33                        & \textbf{51.40±0.33}               & 54.00±0.71                        & 48.27±0.50                       & \textbf{54.00±0.59}               & 50.13±0.41                       & \textbf{59.93±0.41}              & \textbf{52.47±0.25}              & 48.00±0.28                        & 44.53±0.34                       & \textbf{50.33±0.41}              & 53.00±0.28                        & 49.13±0.74                       & 6             \\ %\cline{2-18} 
                    & \method{}                                 & 54.00±0.59                        & \textbf{51.13±0.34}              & \textbf{57.67±0.09}              & 51.13±0.68                       & \textbf{54.53±0.25}              & \textbf{49.73±0.09}              & 53.33±0.66                       & \textbf{53.13±0.98}              & 59.60±0.33                        & 51.53±0.38                       & \textbf{48.87±0.19}              & \textbf{46.53±0.66}              & 49.00±0.43                        & \textbf{56.33±0.68}              & \textbf{52.47±0.41}              & \textbf{9}    \\ \hline
\multirow{5}{*}{4}  & LoRA                                   & 49.87±0.09                       & 47.07±0.09                       & 52.27±0.34                       & 49.67±0.19                       & 49.87±0.25                       & 46.47±0.34                       & 50.33±0.09                       & 48.20±0.16                        & 56.53±0.62                       & 49.80±0.16                        & 45.00±0.43                        & 41.40±0.00                        & 47.93±0.09                       & 48.73±0.09                       & 45.80±0.16                        & 0             \\ %\cline{2-18} 
                    & AdaLoRA                              & 49.27±0.09                       & 45.20±0.16                        & 50.07±0.25                       & 47.93±0.25                       & 47.27±0.09                       & 44.60±0.00                        & 49.27±0.19                       & 46.53±0.09                       & 54.73±0.25                       & 48.60±0.16                        & 44.60±0.16                        & 40.07±0.19                       & 46.47±0.09                       & 48.80±0.00                        & 44.53±0.34                       & 0             \\ %\cline{2-18} 
                    & BayesTune-LoRA                            & 49.33±0.19                       & 45.13±0.09                       & 50.00±0.16                        & 48.20±0.16                        & 47.13±0.09                       & 44.67±0.09                       & 49.40±0.00                        & 46.40±0.16                        & 54.93±0.09                       & 48.87±0.09                       & 44.60±0.00                        & 40.13±0.09                       & 46.60±0.00                        & 48.53±0.09                       & 44.40±0.00                        & 0             \\ %\cline{2-18} 
                    & FedL2P                               & 54.73±0.38                       & \textbf{50.93±0.66}              & 57.27±0.34                       & \textbf{51.47±0.34}              & \textbf{54.67±1.09}              & 47.67±0.25                       & \textbf{54.20±0.43}               & 50.93±0.25                       & \textbf{59.80±0.59}               & \textbf{52.73±0.09}              & 47.87±0.50                       & 44.80±0.00                        & 50.20±0.28                        & 53.47±0.50                       & 49.53±0.93                       & 6             \\ %\cline{2-18} 
                    & \method{}                                 & \textbf{55.33±0.34}              & 50.60±0.91                        & \textbf{58.20±0.16}               & 51.33±0.41                       & 54.60±0.28                        & \textbf{48.80±0.28}               & 52.73±0.57                       & \textbf{53.47±0.47}              & 59.13±0.68                       & 51.53±0.34                       & \textbf{48.47±0.41}              & \textbf{47.20±0.59}               & \textbf{49.00±0.16}               & \textbf{56.73±0.66}              & \textbf{52.13±0.41}              & \textbf{9}    \\ \hline
\multirow{5}{*}{8}  & LoRA                                   & 51.00±0.33                        & 48.13±0.50                       & 54.00±0.33                        & 50.87±0.34                       & 51.20±0.28                        & 46.93±0.25                       & 52.13±1.05                       & 48.27±0.25                       & 58.33±0.19                       & 50.13±0.19                       & 45.80±0.16                        & 42.40±0.00                        & 48.13±1.05                       & 50.93±0.09                       & 46.80±0.59                        & 0             \\ %\cline{2-18} 
                    & AdaLoRA                              & 49.20±0.16                        & 45.40±0.16                        & 50.13±0.19                       & 48.13±0.09                       & 47.07±0.09                       & 44.73±0.09                       & 49.13±0.09                       & 46.33±0.09                       & 54.93±0.19                       & 48.47±0.34                       & 44.60±0.00                        & 40.13±0.09                       & 46.67±0.09                       & 48.67±0.19                       & 44.33±0.25                       & 0             \\ %\cline{2-18} 
                    & BayesTune-LoRA                            & 49.27±0.09                       & 45.07±0.19                       & 50.27±0.09                       & 47.93±0.09                       & 47.20±0.16                        & 44.73±0.09                       & 49.27±0.19                       & 46.33±0.19                       & 54.93±0.09                       & 48.53±0.09                       & 44.53±0.09                       & 40.33±0.09                       & 46.67±0.09                       & 48.53±0.09                       & 44.53±0.38                       & 0             \\ %\cline{2-18} 
                    & FedL2P                               & \textbf{54.13±0.25}              & 49.60±1.14                        & 57.47±0.82                       & 50.87±0.50                       & 53.20±0.28                        & 47.60±0.43                        & \textbf{54.20±0.16}               & 49.87±0.09                       & \textbf{59.53±0.62}              & \textbf{52.27±0.90}              & 47.47±0.25                       & 43.60±0.71                        & \textbf{50.27±0.52}              & 52.60±0.57                        & 48.80±0.57                        & 5             \\ %\cline{2-18} 
                    & \method{}                                 & 53.40±0.00                        & \textbf{50.60±0.33}               & \textbf{57.67±0.62}              & 50.53±0.34                       & \textbf{55.13±0.38}              & \textbf{48.00±0.75}               & 52.53±0.34                       & \textbf{51.87±0.19}              & 57.00±1.02                        & 50.67±0.09                       & \textbf{47.60±0.00}               & \textbf{46.67±0.52}              & 48.47±0.66                       & \textbf{55.00±0.33}               & \textbf{50.73±0.75}              & \textbf{9}    \\ \hline
\multirow{5}{*}{16} & LoRA                                   & 52.73±0.25                       & 48.80±0.00                        & 55.47±0.25                       & \textbf{51.13±1.05}              & 53.53±0.41                       & \textbf{48.87±0.75}              & 54.40±0.91                        & 49.40±0.99                        & 58.87±1.11                       & \textbf{51.80±0.49}               & 46.93±0.09                       & 43.60±0.99                        & 49.13±1.05                       & 52.13±0.81                       & 48.93±2.03                       & 3             \\ %\cline{2-18} 
                    & AdaLoRA                              & 49.13±0.09                       & 45.20±0.00                        & 50.20±0.16                        & 47.93±0.25                       & 46.93±0.19                       & 44.53±0.09                       & 49.33±0.19                       & 46.53±0.09                       & 54.80±0.16                        & 48.73±0.09                       & 44.73±0.19                       & 40.13±0.09                       & 46.73±0.25                       & 48.67±0.09                       & 44.40±0.28                        & 0             \\ %\cline{2-18} 
                    & BayesTune-LoRA                            & 49.33±0.25                       & 45.33±0.19                       & 50.20±0.00                        & 47.93±0.09                       & 47.53±0.25                       & 44.80±0.16                        & 49.40±0.33                        & 46.53±0.09                       & 55.27±0.09                       & 48.87±0.34                       & 44.47±0.09                       & 40.20±0.28                        & 46.53±0.09                       & 48.73±0.25                       & 44.73±0.25                       & 0             \\ %\cline{2-18} 
                    & FedL2P                               & 52.53±1.16                       & 49.00±0.28                        & 56.40±1.40                        & 50.47±0.96                       & 52.47±0.68                       & 47.80±0.28                        & \textbf{53.93±0.38}              & 48.80±0.85                        & \textbf{59.07±0.41}              & 51.73±0.75                       & 47.13±0.82                       & 43.00±1.14                        & \textbf{49.27±0.84}              & 52.07±1.16                       & 48.33±0.74                       & 3             \\ %\cline{2-18} 
                    & \method{}                                 & \textbf{53.73±0.52}              & \textbf{49.87±0.50}              & \textbf{57.07±0.52}              & 49.87±0.25                       & \textbf{54.67±0.90}              & 47.67±0.41                       & 52.27±0.38                       & \textbf{52.27±0.52}              & 57.87±0.68                       & 50.33±0.66                       & \textbf{47.60±0.33}               & \textbf{46.20±0.16}               & 49.00±0.49                        & \textbf{54.60±0.71}               & \textbf{49.67±0.57}              & \textbf{9}    \\ \bottomrule
\end{tabular}
}
\end{scriptsize}
\vspace{-1.5em}
\end{table*}

\begin{table*}[t]
\npdecimalsign{.}
\nprounddigits{2}
\caption{Average METEOR/ROUGE-1/ROUGE-L of each language for \seen{} clients of our Fed-Aya setup. The pretrained MobileLLaMA-1.4B model is trained using Standard FL with LoRA following FedLLM-Bench~\cite{fedllm-bench} and the resulting \basemodel{} is personalized to each client given a baseline approach.}
\vspace{0.5em}
\label{tab:mobilellama_fedaya_seen}
\begin{scriptsize}\resizebox{0.98\textwidth}{!}{
\begin{tabular}{c|l|l|l|l|l|l|l|l|c}
\toprule
% \textbf{Lora}\\\textbf{Rank}  

\textbf{$\mathbf{r}$}& \multicolumn{1}{c|}{\textbf{Approach}} & \multicolumn{1}{c|}{\textbf{te}} & \multicolumn{1}{c|}{\textbf{ar}} & \multicolumn{1}{c|}{\textbf{es}} & \multicolumn{1}{c|}{\textbf{en}} & \multicolumn{1}{c|}{\textbf{fr}} & \multicolumn{1}{c|}{\textbf{zh}} & \multicolumn{1}{c|}{\textbf{pt}} 
& \textbf{Wins} \\ \midrule
% \multirow{5}{*}{1}  & LoRA                                   & 0.125/0.0637/0.0613              & 0.1723/0.0208/0.0203             & 0.3159/0.3469/0.3203             & 0.2458/0.3066/0.2478             & 0.1878/0.2455/0.1952             & \textbf{0.0649/0.084/0.084}      & 0.2527/0.3116/0.2792                                         & 0             \\ % \cline{2-11} 
%                     & AdaLoRA                              & 0.1218/0.0557/0.0537             & 0.1899/0.0173/0.0166             & 0.3241/0.3478/0.3218             & 0.2434/0.3058/0.2452             & 0.2048/0.2445/0.1988             & 0.0633/0.0841/0.0841             & 0.2692/0.329/0.2959                                          & 0             \\ % \cline{2-11} 
%                     & BayesTune-LoRA                            & 0.12/0.0557/0.0541               & 0.1582/0.0226/0.0223             & 0.2863/0.3269/0.2991             & 0.2326/0.2848/0.2319             & 0.1742/0.2298/0.184              & 0.06/0.0723/0.0723               & 0.2289/0.2837/0.2526                                         & 1             \\ % \cline{2-11} 
%                     & FedL2P                               & \textbf{0.1396/0.0769/0.0746}    & \textbf{0.2109/0.0207/0.0197}    & 0.3218/0.3527/0.3252             & \textbf{0.2692/0.3293/0.2697}    & \textbf{0.2365/0.2652/0.2105}    & 0.0527/0.0866/0.0865             & \textbf{0.2797/0.3228/0.2916}                                & 2             \\ % \cline{2-11} 
%                     & \method{}                                 & 0.126/0.0627/0.0603              & 0.2039/0.0221/0.0216             & \textbf{0.3335/0.3624/0.3346}    & 0.2645/0.3369/0.2748             & 0.2181/0.2667/0.2145             & 0.0616/0.0842/0.0842             & 0.2792/0.3373/0.3035                                         & \textbf{4}    \\ \hline
\multirow{5}{*}{2}  & LoRA                                   & 0.1207/0.0545/0.0524             & 0.1835/0.0210/0.0205              & 0.3228/0.3519/0.3251             & 0.2457/0.3121/0.2525             & 0.2049/0.2484/0.2002             & 0.0616/\textbf{0.0874/0.0873}             & 0.2631/0.3266/0.2922                                         & 1             \\ % \cline{2-11} 
                    & AdaLoRA                              & 0.1238/0.0607/0.0586             & 0.1819/0.0223/0.0218             & 0.3288/0.3573/0.3315             & 0.2459/0.3172/0.2524             & 0.1963/0.2422/0.1915             & \textbf{0.0689}/0.0832/0.0832    & 0.2745/0.3327/0.2986                                         & 0             \\ % \cline{2-11} 
                    & BayesTune-LoRA                            & 0.1245/0.0605/0.0580              & 0.1813/0.0181/0.0178             & 0.2941/0.3317/0.3063             & 0.2345/0.2892/0.2367             & 0.1885/0.2430/0.1970               & 0.0643/0.0805/0.0805             & 0.2408/0.2971/0.2645                                         & 0             \\ % \cline{2-11} 
                    & FedL2P                               & \textbf{0.1451/0.0747/0.0725}    & 0.2017/0.0219/0.0213             & 0.3321/0.3523/0.3245             & 0.2635/0.3307/0.2692             & 0.2298/0.2467/0.2034             & 0.0544/0.0803/0.0803             & 0.2780/0.3133/0.2832                                          & 1             \\ % \cline{2-11} 
                    & \method{}                                 & 0.1266/0.0629/0.0606             & \textbf{0.2081/0.0254/0.0248}    & \textbf{0.3425/0.3663/0.3376}    & \textbf{0.2745/0.3469/0.2831}    & \textbf{0.2342/0.2766/0.2246}    & 0.0524/0.0777/0.0777             & \textbf{0.2846/0.3403/0.3066}                                & \textbf{5}    \\ \hline
\multirow{5}{*}{4}  & LoRA                                   & 0.1232/0.0570/0.0537              & 0.1861/0.0202/0.0197             & 0.3284/0.3555/0.3291             & 0.2541/0.3202/0.2585             & 0.2037/0.2475/0.1990              & 0.0559/\textbf{0.0886/0.0886}             & 0.2734/0.3314/0.2975                                         & 1             \\ % \cline{2-11} 
                    & AdaLoRA                              & 0.1240/0.0596/0.0574              & 0.1858/0.0180/0.0177              & 0.3310/0.3548/0.3287              & 0.2448/0.3111/0.2500               & 0.1892/0.2331/0.1859             & 0.0617/0.0852/0.0851             & 0.2640/0.3250/0.2934                                           & 0             \\ % \cline{2-11} 
                    & BayesTune-LoRA                            & 0.1214/0.0548/0.0532             & 0.1912/0.0201/0.0195             & 0.3042/0.3405/0.3150              & 0.2405/0.3022/0.2440              & 0.1973/0.2393/0.1925             & \textbf{0.0670}/0.0806/0.0806     & 0.2468/0.3057/0.2737                                         & 0             \\ % \cline{2-11} 
                    & FedL2P                               & 0.1258/0.0616/0.0592             & 0.1805/0.0217/0.0208             & 0.3260/0.3568/0.3298              & 0.2519/0.3108/0.2498             & 0.2037/0.2493/0.2027             & 0.0484/0.0798/0.0798             & 0.2626/0.3174/0.2836                                         & 0             \\ % \cline{2-11} 
                    & \method{}                                 & \textbf{0.1350/0.0722/0.0692}     & \textbf{0.2218/0.0275/0.0269}    & \textbf{0.3448/0.3712/0.3419}    & \textbf{0.2796/0.3516/0.2883}    & \textbf{0.2469/0.2753/0.2244}    & 0.0554/0.0843/0.0843             & \textbf{0.2911/0.3397/0.3060}                                 & \textbf{6}    \\ \hline
\multirow{5}{*}{8}  & LoRA                                   & 0.1241/0.0522/0.0493             & 0.2059/0.0189/0.0184             & 0.3435/0.3688/0.3427             & 0.2686/0.3400/0.2771               & 0.2197/0.2556/0.2037             & 0.0583/\textbf{0.0886}/0.0884             & 0.2822/0.3418/0.3076                                         & 0             \\ % \cline{2-11} 
                    & AdaLoRA                              & 0.1245/0.0623/0.0599             & 0.1771/0.0189/0.0184             & 0.3215/0.3485/0.3225             & 0.2461/0.3101/0.2512             & 0.1799/0.2287/0.1839             & \textbf{0.0613}/0.0806/0.0806    & 0.2607/0.3194/0.2879                                         & 0             \\ % \cline{2-11} 
                    & BayesTune-LoRA                            & 0.1246/0.0591/0.0572             & 0.2046/0.0192/0.0189             & 0.3247/0.3520/0.3284              & 0.2430/0.3095/0.2496              & 0.2132/0.2542/0.2059             & 0.0601/0.0818/\textbf{0.0818}             & 0.2588/0.3171/0.2871                                         & 0             \\ % \cline{2-11} 
                    & FedL2P                               & 0.1316/\textbf{0.0687/0.0661}             & 0.1855/0.0218/0.0215             & 0.3272/0.3535/0.3280              & 0.2696/0.3304/0.2711             & 0.2109/0.2558/0.2059             & 0.0510/0.0816/0.0816              & 0.2750/0.3252/0.2897                                          & 1             \\ % \cline{2-11} 
                    & \method{}                                 & \textbf{0.1327}/0.0662/0.0644    & \textbf{0.2304/0.0253/0.0246}    & \textbf{0.3474/0.3847/0.3531}    & \textbf{0.2941/0.3656/0.2996}    & \textbf{0.2553/0.2829/0.2268}    & 0.0538/0.0814/0.0814             & \textbf{0.2945/0.3452/0.3108}                                & \textbf{5}    \\ \hline
\multirow{5}{*}{16} & LoRA                                   & 0.1217/0.0562/0.0536             & 0.2080/0.0221/0.0218              & 0.3387/0.3616/0.3352             & 0.2757/0.3431/0.2807             & \textbf{0.2497/0.2880/0.2337}     & 0.0553/\textbf{0.0844/0.0844}             & 0.2902/0.3449/0.3106                                         & 2             \\ % \cline{2-11} 
                    & AdaLoRA                              & 0.1251/0.0624/0.0602             & 0.1676/0.0198/0.0192             & 0.3048/0.3329/0.3037             & 0.2391/0.2985/0.2416             & 0.1821/0.2309/0.1866             & \textbf{0.0575}/0.0815/0.0815    & 0.2530/0.3099/0.2784                                          & 0             \\ % \cline{2-11} 
                    & BayesTune-LoRA                            & 0.1374/0.0745/0.0720              & 0.2119/0.0175/0.0172             & 0.3358/0.3649/0.3397             & 0.2587/0.3189/0.2577             & 0.2222/0.2603/0.2113             & 0.0520/0.0824/0.0824              & 0.2862/0.3450/0.3109                                          & 0             \\ % \cline{2-11} 
                    & FedL2P                               & \textbf{0.1559/0.0827/0.0799}    & 0.2013/0.0228/0.0226             & 0.3278/0.3541/0.3268             & 0.2772/0.3278/0.2693             & 0.2306/0.2346/0.1925             & 0.0506/0.0838/0.0838             & 0.2817/0.3179/0.2851                                         & 1             \\ % \cline{2-11} 
                    & \method{}                                 & 0.1309/0.0663/0.0638             & \textbf{0.2359/0.0258/0.0252}    & \textbf{0.3447/0.3778/0.3463}    & \textbf{0.2802/0.3485/0.2858}    & 0.2473/0.2775/0.2208             & 0.0538/0.0835/0.0835             & \textbf{0.2975/0.3491/0.3157}                                & \textbf{4}    \\ \bottomrule
\end{tabular}

}
\npnoround
\end{scriptsize}
\vspace{-1.5em}
\end{table*}


\begin{table*}[t]
\caption{Average METEOR/ROUGE-1/ROUGE-L of each language for \seen{} clients of our Fed-Aya setup. The pretrained MobileLLaMA-1.4B model is trained using Standard FL with LoRA following FedLLM-Bench~\cite{fedllm-bench} and the resulting \basemodel{} is personalized to each client given a baseline approach.}
\vspace{0.5em}
\label{tab:mobilellama_fedaya_unseen}
\begin{scriptsize}\resizebox{0.98\textwidth}{!}{
\begin{tabular}{c|l|l|l|l|l|l|l|l|l|c}
\toprule
\textbf{$\mathbf{r}$}  & \multicolumn{1}{c|}{\textbf{Approach}} & \multicolumn{1}{c|}{\textbf{te}} & \multicolumn{1}{c|}{\textbf{ar}} & \multicolumn{1}{c|}{\textbf{es}} & \multicolumn{1}{c|}{\textbf{en}} & \multicolumn{1}{c|}{\textbf{fr}} & \multicolumn{1}{c|}{\textbf{zh}} & \multicolumn{1}{c|}{\textbf{pt}} & \multicolumn{1}{c|}{\textbf{ru}} & \textbf{Wins} \\ \midrule
% \multirow{5}{*}{1}  & LoRA                                   & 0.0344/0.0000/0.0000                   & 0.1004/0.0564/0.0539             & 0.3329/0.4252/0.3809             & 0.1987/0.2339/0.1988             & 0.0505/0.0000/0.0000                   & 0.1457/0.0041/0.0034             & 0.1714/0.1907/0.1792             & \textbf{0.1174/0.0556/0.0556}    & 1             \\ % \cline{2-11} 
%                     & AdaLoRA                              & 0.0342/0.0042/0.0042             & 0.0941/0.0581/0.0556             & 0.3607/0.4708/0.4256             & 0.196/0.232/0.2011               & 0.049/0.0000/0.0000                    & \textbf{0.154/0.0092/0.0092}     & 0.1804/0.2222/0.2102             & 0.1075/0.0556/0.0556             & 2             \\ % \cline{2-11} 
%                     & BayesTune-LoRA                            & 0.0202/0.0000/0.0000                   & 0.0741/0.0461/0.0461             & 0.3357/0.3817/0.349              & 0.1912/0.2252/0.1858             & 0.0521/0.0000/0.0000                   & 0.1156/0.002/0.002               & 0.1592/0.164/0.151               & 0.0963/0.0556/0.0556             & 0             \\ % \cline{2-11} 
%                     & FedL2P                               & \textbf{0.1107/0.0075/0.0075}    & \textbf{0.1695/0.0391/0.0391}    & 0.3691/0.4312/0.4055             & 0.2432/0.2668/0.2291             & \textbf{0.0595/0.019/0.019}      & 0.1016/0.0101/0.0101             & \textbf{0.2085/0.2337/0.2237}    & 0.0746/0.0556/0.0556             & \textbf{3}    \\ % \cline{2-11} 
%                     & \method{}                                 & 0.1057/0.0042/0.0042             & 0.1191/0.0587/0.0587             & \textbf{0.3718/0.4349/0.4045}    & \textbf{0.2993/0.3342/0.297}     & 0.049/0.0000/0.0000                    & 0.1153/0.0049/0.0049             & 0.1928/0.2301/0.2169             & 0.0912/0.0556/0.0556             & 2             \\ \hline
\multirow{5}{*}{2}  & LoRA                                   & 0.0531/0.0042/0.0042             & 0.1032/\textbf{0.0524/0.0524}            & 0.3547/0.4490/0.4020               & 0.2385/0.2713/0.2356             & 0.0490/0.0000/0.0000                    & 0.1381/0.0088/0.0088             & 0.1865/0.2184/0.2070              & 0.0921/0.0556/0.0556             & 1             \\ % \cline{2-11} 
                    & AdaLoRA                              & 0.0312/0.0000/0.0000                   & 0.1000/0.0520/0.0520                  & 0.3244/0.4250/0.3821              & 0.2165/0.2606/0.2197             & 0.0490/\textbf{0.0513/0.0513}              & \textbf{0.1609}/0.0089/0.0089    & 0.1904/0.2258/0.2129             & \textbf{0.0992}/0.0556/0.0556    & 1             \\ % \cline{2-11} 
                    & BayesTune-LoRA                            & 0.0276/0.0000/0.0000                   & 0.0934/0.0501/0.0476             & 0.3560/0.4084/0.3716              & 0.1786/0.2024/0.1688             & 0.0450/0.0000/0.0000                    & 0.1566/0.0109/0.0092             & 0.1469/0.1628/0.1500               & 0.0873/0.0556/0.0556             & 0             \\ % \cline{2-11} 
                    & FedL2P                               & \textbf{0.1199}/0.0042/0.0042    & \textbf{0.1399}/0.0206/0.0206    & \textbf{0.3923/0.4587}/0.4194    & 0.2688/0.3049/0.2650              & 0.0024/0.0011/0.0011             & 0.0888/\textbf{0.0121/0.0121}             & \textbf{0.2022}/0.2170/0.2063     & 0.0850/0.0556/0.0556              & 2             \\ % \cline{2-11} 
                    & \method{}                                 & 0.1105/\textbf{0.0132/0.0114}             & 0.1127/0.0483/0.0483             & 0.3812/0.4546/\textbf{0.4209}             & \textbf{0.3047/0.3354/0.3006}    & 0.0490/0.0000/0.0000                    & 0.0650/0.0069/0.0069              & 0.1997/\textbf{0.2505/0.2384}             & 0.0956/0.0556/0.0556             & \textbf{3}    \\ \hline
\multirow{5}{*}{4}  & LoRA                                   & 0.0687/0.0075/0.0075             & 0.0989/0.0581/0.0556             & 0.3244/0.4086/0.3749             & 0.2297/0.2726/0.2370              & 0.0490/0.0513/0.0513              & 0.1530/0.0089/0.0089              & 0.1872/0.2267/0.2124             & 0.0974/0.0556/0.0556             & 0             \\ % \cline{2-11} 
                    & AdaLoRA                              & 0.0263/0.0000/0.0000                   & 0.0894/\textbf{0.0634}/0.0610              & 0.3336/0.4322/0.3883             & 0.1762/0.2105/0.1765             & 0.0490/0.0000/0.0000                    & \textbf{0.1630}/0.0066/0.0049     & 0.1832/0.2096/0.1980              & 0.1156/0.0556/0.0556             & 0             \\ % \cline{2-11} 
                    & BayesTune-LoRA                            & 0.0471/0.0075/0.0075             & 0.0938/0.0544/0.0544             & 0.3357/0.4058/0.3629             & 0.1657/0.1964/0.1647             & 0.1155/0.0635/0.0635             & 0.1557/\textbf{0.0115/0.0099}             & 0.1511/0.1636/0.1517             & 0.0858/0.0222/0.0222             & 1             \\ % \cline{2-11} 
                    & FedL2P                               & 0.0569/0.0075/0.0075             & 0.1002/0.0610/\textbf{0.0610}               & 0.3098/0.4066/0.3837             & 0.2494/0.2901/0.2569             & \textbf{0.1157/0.0741/0.0741}    & 0.1585/0.0041/0.0041             & 0.1724/0.1920/0.1806              & 0.0804/0.0556/0.0556             & 1             \\ % \cline{2-11} 
                    & \method{}                                 & \textbf{0.1121/0.0212/0.0212}    & \textbf{0.1537}/0.0333/0.0333    & \textbf{0.3725/0.4622/0.4273}    & \textbf{0.2945/0.3276/0.2821}    & 0.0490/0.0000/0.0000                    & 0.0722/0.0069/0.0069             & \textbf{0.1903/0.2547/0.2438}    & \textbf{0.1258/0.0556/0.0556}    & \textbf{5}    \\ \hline
\multirow{5}{*}{8}  & LoRA                                   & 0.1022/0.0042/0.0042             & 0.1206/0.0549/0.0549             & 0.3401/0.4382/0.3977             & 0.2722/\textbf{0.3127/0.2726}             & 0.0490/0.0000/0.0000                    & 0.1241/0.0089/0.0089             & 0.1879/0.2331/0.2202             & 0.0761/0.0556/0.0556             & 1             \\ % \cline{2-11} 
                    & AdaLoRA                              & 0.0218/0.0000/0.0000                   & 0.0897/0.0364/0.0364             & 0.3383/0.4295/0.3920              & 0.1786/0.2150/0.1795              & 0.1221/\textbf{0.0833/0.0833}             & 0.1470/0.0066/0.0049              & 0.1773/0.1909/0.1770              & 0.1124/0.0556/0.0556             & 1             \\ % \cline{2-11} 
                    & BayesTune-LoRA                            & 0.0529/0.0042/0.0042             & 0.0970/\textbf{0.0557/0.0557}              & 0.3544/0.4257/0.3789             & 0.1891/0.2235/0.1860              & \textbf{0.1262}/0.0784/0.0784    & \textbf{0.1522}/0.0068/0.0068    & 0.1365/0.1603/0.1483             & 0.0895/0.0556/0.0556             & 1             \\ % \cline{2-11} 
                    & FedL2P                               & 0.0934/0.0042/0.0042             & 0.1138/0.0495/0.0495             & 0.3237/0.4366/0.4068             & 0.2438/0.2849/0.2451             & 0.0490/0.0000/0.0000                    & 0.1408/\textbf{0.0139/0.0139}             & 0.1946/0.2280/0.2130               & \textbf{0.1187}/0.0556/0.0556    & 1             \\ % \cline{2-11} 
                    & \method{}                                 & \textbf{0.1099/0.0132/0.0114}    & \textbf{0.1565}/0.0082/0.0082    & \textbf{0.4471/0.5240/0.4876}     & \textbf{0.2798}/0.2851/0.2475    & 0.0490/0.0000/0.0000                    & 0.1071/0.0069/0.0069             & \textbf{0.2160/0.2700/0.2559}       & 0.0862/0.0465/0.0465             & \textbf{3}    \\ \hline
\multirow{5}{*}{16} & LoRA                                   & 0.1193/0.0042/0.0042             & \textbf{0.1418/0.0587/0.0587}    & 0.3647/0.4240/0.3927              & 0.2939/\textbf{0.3164/0.2753}             & 0.0490/0.0000/0.0000                    & 0.0962/0.0089/0.0089             & 0.1916/0.2438/0.2320              & \textbf{0.1403/0.0556/0.0556}    & \textbf{3}    \\ % \cline{2-11} 
                    & AdaLoRA                              & 0.0079/0.0000/0.0000                   & 0.0848/0.0386/0.0361             & 0.3548/0.4216/0.3864             & 0.1814/0.2158/0.1750              & 0.1443/0.1333/0.1333             & 0.1408/\textbf{0.0115/0.0099}             & 0.1828/0.1892/0.1793             & 0.0957/0.0556/0.0556             & 1             \\ % \cline{2-11} 
                    & BayesTune-LoRA                            & 0.0674/0.0075/0.0075             & 0.0887/0.0420/0.0420               & 0.3508/0.4174/0.3788             & 0.2088/0.2442/0.2106             & 0.1681/\textbf{0.1905/0.1905}             & \textbf{0.1594}/0.0089/0.0089    & 0.1877/0.2362/0.2225             & 0.0972/0.0556/0.0556             & 1             \\ % \cline{2-11} 
                    & FedL2P                               & 0.1093/\textbf{0.0673/0.0673}             & 0.1171/0.0320/0.0296              & 0.3895/0.4515/0.4131             & 0.2629/0.2606/0.2150              & 0.1349/0.1026/0.1026             & 0.1335/0.0041/0.0020              & 0.2011/0.2149/0.2027             & 0.0645/0.0253/0.0253             & 1             \\ % \cline{2-11} 
                    & \method{}                                 & \textbf{0.1289}/0.0165/0.0147    & 0.1215/0.0166/0.0166             & \textbf{0.4048/0.4910/0.4520}      & \textbf{0.2968}/0.3065/0.2678    & \textbf{0.2887}/0.1667/0.1667    & 0.0960/0.0069/0.0069              & \textbf{0.2359/0.2863/0.2704}    & 0.0791/0.0222/0.0222             & 2             \\ \bottomrule
\end{tabular}
}
\end{scriptsize}
\vspace{-1.5em}
\end{table*}


\begin{figure*}
\centering
\begin{minipage}{.23\linewidth}
  \includegraphics[width=\linewidth]{figures/xnli_fedavg_out_0.5_seen_en_layerwiserank.png}
  \captionof{figure}{Language agnostic rank structure of mBERT in our XNLI setup where the \basemodel{} is trained with FedIFT full-finetuning ($r=16$).}
  \label{fig:xnli_fedavg_out_r16}
\end{minipage}
\hspace{.01\linewidth}
\begin{minipage}{.23\linewidth}
  \includegraphics[width=\linewidth]{figures/xnli_fedavg_out_0.0625_seen_en_layerwiserank.png}
  \captionof{figure}{Language agnostic rank structure of mBERT in our XNLI setup where the \basemodel{} is trained with FedIFT full-finetuning ($r=2$).}
  \label{fig:xnli_fedavg_out_r2}
\end{minipage}
\hspace{.01\linewidth}
\begin{minipage}{.23\linewidth}
  \includegraphics[width=\linewidth]{figures/xnli_dept_out_0.5_seen_en_layerwiserank.png}
  \captionof{figure}{Language agnostic rank structure of mBERT in our XNLI setup where the \basemodel{} is trained with DEPT(SPEC) ($r=16$).}
  \label{fig:xnli_dept_out_r16}
\end{minipage}
\hspace{.01\linewidth}
\begin{minipage}{.23\linewidth}
  \includegraphics[width=\linewidth]{figures/xnli_dept_out_0.0625_seen_en_layerwiserank.png}
  \captionof{figure}{Language agnostic rank structure of mBERT in our XNLI setup where the \basemodel{} is trained with DEPT(SPEC) ($r=2$).}
  \label{fig:xnli_dept_out_r2}
\end{minipage}
\end{figure*}

\begin{figure*}
\centering
\begin{minipage}{.24\linewidth}
  \includegraphics[width=\linewidth]{figures/masakha_out_0.5_seen_eng_layerwiserank.png}
  \captionof{figure}{Language agnostic rank structure of mBERT in our MasakhaNEWS setup where the \basemodel{} is trained with FedIFT full-finetuning ($r=16$).}
  \label{fig:masakha_fedavg_out_r16}
\end{minipage}
\hspace{.2\linewidth}
\begin{minipage}{.24\linewidth}
  \includegraphics[width=\linewidth]{figures/masakha_out_0.0625_seen_eng_layerwiserank.png}
  \captionof{figure}{Language agnostic rank structure of mBERT in our MasakhaNEWS setup where the \basemodel{} is trained with FedIFT full-finetuning ($r=2$).}
  \label{fig:masakha_fedavg_out_r2}
\end{minipage}
\end{figure*}


\begin{figure*}
\centering
\begin{minipage}{.23\linewidth}
  \includegraphics[width=\linewidth]{figures/mobilellama_out_0.5_seen_en_layerwiserank.png}
  \captionof{figure}{Language agnostic rank structure of MobileLLaMA-1.4B in our Fed-Aya setup where the \basemodel{} is trained with FedIFT LoRA ($r=16$). Zoom in for best results.}
  \label{fig:mobilellama_fedavg_out_r16}
\end{minipage}
\hspace{.01\linewidth}
\begin{minipage}{.23\linewidth}
  \includegraphics[width=\linewidth]{figures/mobilellama_out_0.0625_seen_en_layerwiserank.png}
  \captionof{figure}{Language agnostic rank structure of MobileLLaMA-1.4B in our Fed-Aya setup where the \basemodel{} is trained with FedIFT LoRA ($r=2$). Zoom in for best results.}
  \label{fig:mobilellama_fedavg_out_r2}
\end{minipage}
\hspace{.01\linewidth}
\begin{minipage}{.23\linewidth}
  \includegraphics[width=\linewidth]{figures/llama3_out_0.5_seen_en_layerwiserank.png}
  \captionof{figure}{Language agnostic rank structure of Llama-3.2-3B in our Fed-Aya setup where the \basemodel{} is an off-the-shelf instruction tuned Llama-3.2-3B-Instruct ($r=16$). Zoom in for best results.}
  \label{fig:llama3_fedavg_out_r16}
\end{minipage}
\hspace{.01\linewidth}
\begin{minipage}{.23\linewidth}
  \includegraphics[width=\linewidth]{figures/llama3_out_0.0625_seen_en_layerwiserank.png}
  \captionof{figure}{Language agnostic rank structure of Llama-3.2-3B in our Fed-Aya setup where the \basemodel{} is an off-the-shelf instruction tuned Llama-3.2-3B-Instruct ($r=2$). Zoom in for best results.}
  \label{fig:llama3_fedavg_out_r2}
\end{minipage}
\end{figure*}





\end{document}

