\documentclass{article}

% Language setting
% Replace `english' with e.g. `spanish' to change the document language
\usepackage[english]{babel}
\usepackage{natbib}

% Set page size and margins
% Replace `letterpaper' with `a4paper' for UK/EU standard size
\usepackage[letterpaper,top=2cm,bottom=2cm,left=3cm,right=3cm,marginparwidth=1.75cm]{geometry}

% Useful packages
\usepackage{amsmath}
\usepackage{graphicx}
\usepackage[colorlinks=true, allcolors=blue]{hyperref}
% For theorems and such
\usepackage{amssymb}
\usepackage{mathtools}
\usepackage{amsthm}
\usepackage{algorithmic}
\usepackage{algorithm}
\usepackage{caption}
\usepackage{subcaption}


% if you use cleveref..
\usepackage[capitalize,noabbrev]{cleveref}

%%%%%%%%%%%%%%%%%%%%%%%%%%%%%%%%
% THEOREMS
%%%%%%%%%%%%%%%%%%%%%%%%%%%%%%%%
\theoremstyle{plain}
\newtheorem{theorem}{Theorem}[section]
\newtheorem{proposition}[theorem]{Proposition}
\newtheorem{lemma}[theorem]{Lemma}
\newtheorem{corollary}[theorem]{Corollary}
\theoremstyle{definition}
\newtheorem{definition}[theorem]{Definition}
\newtheorem{assumption}[theorem]{Assumption}
\theoremstyle{remark}
\newtheorem{remark}[theorem]{Remark}

% Todonotes is useful during development; simply uncomment the next line
%    and comment out the line below the next line to turn off comments
%\usepackage[disable,textsize=tiny]{todonotes}
\usepackage[textsize=tiny]{todonotes}
\usepackage{graphicx} 
\usepackage{mathbbol}
\usepackage{amssymb}
\usepackage{amsmath}
\usepackage{amsfonts}
\usepackage{soul}
\usepackage{enumitem}


\newcommand{\jessica}[1]{\textcolor{red}{Jessica: #1}}
\newcommand{\ziyang}[1]{\textcolor{blue}{Ziyang: #1}}
\newcommand{\yifan}[1]{\textcolor{purple}{[Yifan: #1]}}
\newcommand{\jason}[1]{\textcolor{orange}{[Jason: #1]}}

% \newcommand{\mvspace}[1]{\vspace{#1}}
\newcommand{\mvspace}[1]{}
% !TeX root = main.tex 


\newcommand{\lnote}{\textcolor[rgb]{1,0,0}{Lydia: }\textcolor[rgb]{0,0,1}}
\newcommand{\todo}{\textcolor[rgb]{1,0,0.5}{To do: }\textcolor[rgb]{0.5,0,1}}


\newcommand{\state}{S}
\newcommand{\meas}{M}
\newcommand{\out}{\mathrm{out}}
\newcommand{\piv}{\mathrm{piv}}
\newcommand{\pivotal}{\mathrm{pivotal}}
\newcommand{\isnot}{\mathrm{not}}
\newcommand{\pred}{^\mathrm{predict}}
\newcommand{\act}{^\mathrm{act}}
\newcommand{\pre}{^\mathrm{pre}}
\newcommand{\post}{^\mathrm{post}}
\newcommand{\calM}{\mathcal{M}}

\newcommand{\game}{\mathbf{V}}
\newcommand{\strategyspace}{S}
\newcommand{\payoff}[1]{V^{#1}}
\newcommand{\eff}[1]{E^{#1}}
\newcommand{\p}{\vect{p}}
\newcommand{\simplex}[1]{\Delta^{#1}}

\newcommand{\recdec}[1]{\bar{D}(\hat{Y}_{#1})}





\newcommand{\sphereone}{\calS^1}
\newcommand{\samplen}{S^n}
\newcommand{\wA}{w}%{w_{\mathfrak{a}}}
\newcommand{\Awa}{A_{\wA}}
\newcommand{\Ytil}{\widetilde{Y}}
\newcommand{\Xtil}{\widetilde{X}}
\newcommand{\wst}{w_*}
\newcommand{\wls}{\widehat{w}_{\mathrm{LS}}}
\newcommand{\dec}{^\mathrm{dec}}
\newcommand{\sub}{^\mathrm{sub}}

\newcommand{\calP}{\mathcal{P}}
\newcommand{\totspace}{\calZ}
\newcommand{\clspace}{\calX}
\newcommand{\attspace}{\calA}

\newcommand{\Ftil}{\widetilde{\calF}}

\newcommand{\totx}{Z}
\newcommand{\classx}{X}
\newcommand{\attx}{A}
\newcommand{\calL}{\mathcal{L}}



\newcommand{\defeq}{\mathrel{\mathop:}=}
\newcommand{\vect}[1]{\ensuremath{\mathbf{#1}}}
\newcommand{\mat}[1]{\ensuremath{\mathbf{#1}}}
\newcommand{\dd}{\mathrm{d}}
\newcommand{\grad}{\nabla}
\newcommand{\hess}{\nabla^2}
\newcommand{\argmin}{\mathop{\rm argmin}}
\newcommand{\argmax}{\mathop{\rm argmax}}
\newcommand{\Ind}[1]{\mathbf{1}\{#1\}}

\newcommand{\norm}[1]{\left\|{#1}\right\|}
\newcommand{\fnorm}[1]{\|{#1}\|_{\text{F}}}
\newcommand{\spnorm}[2]{\left\| {#1} \right\|_{\text{S}({#2})}}
\newcommand{\sigmin}{\sigma_{\min}}
\newcommand{\tr}{\text{tr}}
\renewcommand{\det}{\text{det}}
\newcommand{\rank}{\text{rank}}
\newcommand{\logdet}{\text{logdet}}
\newcommand{\trans}{^{\top}}
\newcommand{\poly}{\text{poly}}
\newcommand{\polylog}{\text{polylog}}
\newcommand{\st}{\text{s.t.~}}
\newcommand{\proj}{\mathcal{P}}
\newcommand{\projII}{\mathcal{P}_{\parallel}}
\newcommand{\projT}{\mathcal{P}_{\perp}}
\newcommand{\projX}{\mathcal{P}_{\mathcal{X}^\star}}
\newcommand{\inner}[1]{\langle #1 \rangle}

\renewcommand{\Pr}{\mathbb{P}}
\newcommand{\Z}{\mathbb{Z}}
\newcommand{\N}{\mathbb{N}}
\newcommand{\R}{\mathbb{R}}
\newcommand{\E}{\mathbb{E}}
\newcommand{\F}{\mathcal{F}}
\newcommand{\var}{\mathrm{var}}
\newcommand{\cov}{\mathrm{cov}}


\newcommand{\calN}{\mathcal{N}}

\newcommand{\jccomment}{\textcolor[rgb]{1,0,0}{C: }\textcolor[rgb]{1,0,1}}
\newcommand{\fracpar}[2]{\frac{\partial #1}{\partial  #2}}

\newcommand{\A}{\mathcal{A}}
\newcommand{\B}{\mat{B}}
%\newcommand{\C}{\mat{C}}

\newcommand{\I}{\mat{I}}
\newcommand{\M}{\mat{M}}
\newcommand{\D}{\mat{D}}
%\newcommand{\U}{\mat{U}}
\newcommand{\V}{\mat{V}}
\newcommand{\W}{\mat{W}}
\newcommand{\X}{\mat{X}}
\newcommand{\Y}{\mat{Y}}
\newcommand{\mSigma}{\mat{\Sigma}}
\newcommand{\mLambda}{\mat{\Lambda}}
\newcommand{\e}{\vect{e}}
\newcommand{\g}{\vect{g}}
\renewcommand{\u}{\vect{u}}
\newcommand{\w}{\vect{w}}
\newcommand{\x}{\vect{x}}
\newcommand{\y}{\vect{y}}
\newcommand{\z}{\vect{z}}
\newcommand{\fI}{\mathfrak{I}}
\newcommand{\fS}{\mathfrak{S}}
\newcommand{\fE}{\mathfrak{E}}
\newcommand{\fF}{\mathfrak{F}}

\newcommand{\Risk}{\mathcal{R}}

\renewcommand{\L}{\mathcal{L}}
\renewcommand{\H}{\mathcal{H}}

\newcommand{\cn}{\kappa}
\newcommand{\nn}{\nonumber}


\newcommand{\Hess}{\nabla^2}
\newcommand{\tlO}{\tilde{O}}
\newcommand{\tlOmega}{\tilde{\Omega}}

\newcommand{\calF}{\mathcal{F}}
\newcommand{\fhat}{\widehat{f}}
\newcommand{\calS}{\mathcal{S}}

\newcommand{\calX}{\mathcal{X}}
\newcommand{\calY}{\mathcal{Y}}
\newcommand{\calD}{\mathcal{D}}
\newcommand{\calZ}{\mathcal{Z}}
\newcommand{\calA}{\mathcal{A}}
\newcommand{\fbayes}{f^B}
\newcommand{\func}{f^U}


\newcommand{\bayscore}{\text{calibrated Bayes score}}
\newcommand{\bayrisk}{\text{calibrated Bayes risk}}

\newtheorem{example}{Example}[section]
\newtheorem{exc}{Exercise}[section]
%\newtheorem{rem}{Remark}[section]

\newtheorem{theorem}{Theorem}[section]
\newtheorem{definition}{Definition}
\newtheorem{proposition}[theorem]{Proposition}
\newtheorem{corollary}[theorem]{Corollary}

\newtheorem{remark}{Remark}[section]
\newtheorem{lemma}[theorem]{Lemma}
\newtheorem{claim}[theorem]{Claim}
\newtheorem{fact}[theorem]{Fact}
\newtheorem{assumption}{Assumption}

\newcommand{\iidsim}{\overset{\mathrm{i.i.d.}}{\sim}}
\newcommand{\unifsim}{\overset{\mathrm{unif}}{\sim}}
\newcommand{\sign}{\mathrm{sign}}
\newcommand{\wbar}{\overline{w}}
\newcommand{\what}{\widehat{w}}
\newcommand{\KL}{\mathrm{KL}}
\newcommand{\Bern}{\mathrm{Bernoulli}}
\newcommand{\ihat}{\widehat{i}}
\newcommand{\Dwst}{\calD^{w_*}}
\newcommand{\fls}{\widehat{f}_{n}}


\newcommand{\brpi}{\pi^{br}}
\newcommand{\brtheta}{\theta^{br}}

% \newcommand{\M}{\mat{M}}
% \newcommand\Mmh{\mat{M}^{-1/2}}
% \newcommand{\A}{\mat{A}}
% \newcommand{\B}{\mat{B}}
% \newcommand{\C}{\mat{C}}
% \newcommand{\Et}[1][t]{\mat{E_{#1}}}
% \newcommand{\Etp}{\Et[t+1]}
% \newcommand{\Errt}[1][t]{\mat{\bigtriangleup_{#1}}}
% \newcommand\cnM{\kappa}
% \newcommand{\cn}[1]{\kappa\left(#1\right)}
% \newcommand\X{\mat{X}}
% \newcommand\fstar{f_*}
% \newcommand\Xt[1][t]{\mat{X_{#1}}}
% \newcommand\ut[1][t]{{u_{#1}}}
% \newcommand\Xtinv{\inv{\Xt}}
% \newcommand\Xtp{\mat{X_{t+1}}}
% \newcommand\Xtpinv{\inv{\left(\mat{X_{t+1}}\right)}}
% \newcommand\U{\mat{U}}
% \newcommand\UTr{\trans{\mat{U}}}
% \newcommand{\Ut}[1][t]{\mat{U_{#1}}}
% \newcommand{\Utinv}{\inv{\Ut}}
% \newcommand{\UtTr}[1][t]{\trans{\mat{U_{#1}}}}
% \newcommand\Utp{\mat{U_{t+1}}}
% \newcommand\UtpTr{\trans{\mat{U}_{t+1}}}
% \newcommand\Utptild{\mat{\widetilde{U}_{t+1}}}
% \newcommand\Us{\mat{U^*}}
% \newcommand\UsTr{\trans{\mat{U^*}}}
% \newcommand{\Sigs}{\mat{\Sigma}}
% \newcommand{\Sigsmh}{\Sigs^{-1/2}}
% \newcommand{\eye}{\mat{I}}
% \newcommand{\twonormbound}{\left(4+\DPhi{\M}{\Xt[0]}\right)\twonorm{\M}}
% \newcommand{\lamj}{\lambda_j}

% \renewcommand\u{\vect{u}}
% \newcommand\uTr{\trans{\vect{u}}}
% \renewcommand\v{\vect{v}}
% \newcommand\vTr{\trans{\vect{v}}}
% \newcommand\w{\vect{w}}
% \newcommand\wTr{\trans{\vect{w}}}
% \newcommand\wperp{\vect{w}_{\perp}}
% \newcommand\wperpTr{\trans{\vect{w}_{\perp}}}
% \newcommand\wj{\vect{w_j}}
% \newcommand\vj{\vect{v_j}}
% \newcommand\wjTr{\trans{\vect{w_j}}}
% \newcommand\vjTr{\trans{\vect{v_j}}}

% \newcommand{\DPhi}[2]{\ensuremath{D_{\Phi}\left(#1,#2\right)}}
% \newcommand\matmult{{\omega}}


\title{The Value of Information in Human-AI Decision-making}
\author{
    Ziyang Guo\thanks{Department of Computer Science, Northwestern University. Email: \texttt{ziyang.guo@northwestern.edu}} \and
    Yifan Wu\thanks{Department of Computer Science, Northwestern University. Email: \texttt{yifan.wu@u.northwestern.edu}} \and
    Jason Hartline\thanks{Department of Computer Science, Northwestern University. Email: \texttt{hartline@northwestern.edu}}
    \and
    Jessica Hullman\thanks{Department of Computer Science, Northwestern University. Email: \texttt{jhullman@northwestern.edu}}
}

\begin{document}
\maketitle

\begin{abstract}
Multiple agents---including humans and AI models---are often paired on decision tasks with the expectation of achieving \textit{complementary performance}, where the combined performance of both agents outperforms either one alone. 
However, knowing how to improve the performance of a human-AI team is often difficult without knowing more about what particular information and strategies each agent employs. 
We provide a decision-theoretic framework for characterizing the value of information---and consequently, opportunities for agents to better exploit available information--in AI-assisted decision workflows.
We demonstrate the use of the framework for model selection, empirical evaluation of human-AI performance, and explanation design.
We propose a novel information-based explanation technique that adapts SHAP, a saliency-based explanation to explain information value in decision making.
\end{abstract}
\mvspace{-5mm}

\section{Introduction}
\label{section:introduction}

% redirection is unique and important in VR
Virtual Reality (VR) systems enable users to embody virtual avatars by mirroring their physical movements and aligning their perspective with virtual avatars' in real time. 
As the head-mounted displays (HMDs) block direct visual access to the physical world, users primarily rely on visual feedback from the virtual environment and integrate it with proprioceptive cues to control the avatar’s movements and interact within the VR space.
Since human perception is heavily influenced by visual input~\cite{gibson1933adaptation}, 
VR systems have the unique capability to control users' perception of the virtual environment and avatars by manipulating the visual information presented to them.
Leveraging this, various redirection techniques have been proposed to enable novel VR interactions, 
such as redirecting users' walking paths~\cite{razzaque2005redirected, suma2012impossible, steinicke2009estimation},
modifying reaching movements~\cite{gonzalez2022model, azmandian2016haptic, cheng2017sparse, feick2021visuo},
and conveying haptic information through visual feedback to create pseudo-haptic effects~\cite{samad2019pseudo, dominjon2005influence, lecuyer2009simulating}.
Such redirection techniques enable these interactions by manipulating the alignment between users' physical movements and their virtual avatar's actions.

% % what is hand/arm redirection, motivation of study arm-offset
% \change{\yj{i don't understand the purpose of this paragraph}
% These illusion-based techniques provide users with unique experiences in virtual environments that differ from the physical world yet maintain an immersive experience. 
% A key example is hand redirection, which shifts the virtual hand’s position away from the real hand as the user moves to enhance ergonomics during interaction~\cite{feuchtner2018ownershift, wentzel2020improving} and improve interaction performance~\cite{montano2017erg, poupyrev1996go}. 
% To increase the realism of virtual movements and strengthen the user’s sense of embodiment, hand redirection techniques often incorporate a complete virtual arm or full body alongside the redirected virtual hand, using inverse kinematics~\cite{hartfill2021analysis, ponton2024stretch} or adjustments to the virtual arm's movement as well~\cite{li2022modeling, feick2024impact}.
% }

% noticeability, motivation of predicting a probability, not a classification
However, these redirection techniques are most effective when the manipulation remains undetected~\cite{gonzalez2017model, li2022modeling}. 
If the redirection becomes too large, the user may not mitigate the conflict between the visual sensory input (redirected virtual movement) and their proprioception (actual physical movement), potentially leading to a loss of embodiment with the virtual avatar and making it difficult for the user to accurately control virtual movements to complete interaction tasks~\cite{li2022modeling, wentzel2020improving, feuchtner2018ownershift}. 
While proprioception is not absolute, users only have a general sense of their physical movements and the likelihood that they notice the redirection is probabilistic. 
This probability of detecting the redirection is referred to as \textbf{noticeability}~\cite{li2022modeling, zenner2024beyond, zenner2023detectability} and is typically estimated based on the frequency with which users detect the manipulation across multiple trials.

% version B
% Prior research has explored factors influencing the noticeability of redirected motion, including the redirection's magnitude~\cite{wentzel2020improving, poupyrev1996go}, direction~\cite{li2022modeling, feuchtner2018ownershift}, and the visual characteristics of the virtual avatar~\cite{ogawa2020effect, feick2024impact}.
% While these factors focus on the avatars, the surrounding virtual environment can also influence the users' behavior and in turn affect the noticeability of redirection.
% One such prominent external influence is through the visual channel - the users' visual attention is constantly distracted by complex visual effects and events in practical VR scenarios.
% Although some prior studies have explored how to leverage user blindness caused by visual distractions to redirect users' virtual hand~\cite{zenner2023detectability}, there remains a gap in understanding how to quantify the noticeability of redirection under visual distractions.

% visual stimuli and gaze behavior
Prior research has explored factors influencing the noticeability of redirected motion, including the redirection's magnitude~\cite{wentzel2020improving, poupyrev1996go}, direction~\cite{li2022modeling, feuchtner2018ownershift}, and the visual characteristics of the virtual avatar~\cite{ogawa2020effect, feick2024impact}.
While these factors focus on the avatars, the surrounding virtual environment can also influence the users' behavior and in turn affect the noticeability of redirection.
This, however, remains underexplored.
One such prominent external influence is through the visual channel - the users' visual attention is constantly distracted by complex visual effects and events in practical VR scenarios.
We thus want to investigate how \textbf{visual stimuli in the virtual environment} affect the noticeability of redirection.
With this, we hope to complement existing works that focus on avatars by incorporating environmental visual influences to enable more accurate control over the noticeability of redirected motions in practical VR scenarios.
% However, in realistic VR applications, the virtual environment often contains complex visual effects beyond the virtual avatar itself. 
% We argue that these visual effects can \textbf{distract users’ visual attention and thus affect the noticeability of redirection offsets}, while current research has yet taken into account.
% For instance, in a VR boxing scenario, a user’s visual attention is likely focused on their opponent rather than on their virtual body, leading to a lower noticeability of redirection offsets on their virtual movements. 
% Conversely, when reaching for an object in the center of their field of view, the user’s attention is more concentrated on the virtual hand’s movement and position to ensure successful interaction, resulting in a higher noticeability of offsets.

Since each visual event is a complex choreography of many underlying factors (type of visual effect, location, duration, etc.), it is extremely difficult to quantify or parameterize visual stimuli.
Furthermore, individuals respond differently to even the same visual events.
Prior neuroscience studies revealed that factors like age, gender, and personality can influence how quickly someone reacts to visual events~\cite{gillon2024responses, gale1997human}. 
Therefore, aiming to model visual stimuli in a way that is generalizable and applicable to different stimuli and users, we propose to use users' \textbf{gaze behavior} as an indicator of how they respond to visual stimuli.
In this paper, we used various gaze behaviors, including gaze location, saccades~\cite{krejtz2018eye}, fixations~\cite{perkhofer2019using}, and the Index of Pupil Activity (IPA)~\cite{duchowski2018index}.
These behaviors indicate both where users are looking and their cognitive activity, as looking at something does not necessarily mean they are attending to it.
Our goal is to investigate how these gaze behaviors stimulated by various visual stimuli relate to the noticeability of redirection.
With this, we contribute a model that allows designers and content creators to adjust the redirection in real-time responding to dynamic visual events in VR.

To achieve this, we conducted user studies to collect users' noticeability of redirection under various visual stimuli.
To simulate realistic VR scenarios, we adopted a dual-task design in which the participants performed redirected movements while monitoring the visual stimuli.
Specifically, participants' primary task was to report if they noticed an offset between the avatar's movement and their own, while their secondary task was to monitor and report the visual stimuli.
As realistic virtual environments often contain complex visual effects, we started with simple and controlled visual stimulus to manage the influencing factors.

% first user study, confirmation study
% collect data under no visual stimuli, different basic visual stimuli
We first conducted a confirmation study (N=16) to test whether applying visual stimuli (opacity-based) actually affects their noticeability of redirection. 
The results showed that participants were significantly less likely to detect the redirection when visual stimuli was presented $(F_{(1,15)}=5.90,~p=0.03)$.
Furthermore, by analyzing the collected gaze data, results revealed a correlation between the proposed gaze behaviors and the noticeability results $(r=-0.43)$, confirming that the gaze behaviors could be leveraged to compute the noticeability.

% data collection study
We then conducted a data collection study to obtain more accurate noticeability results through repeated measurements to better model the relationship between visual stimuli-triggered gaze behaviors and noticeability of redirection.
With the collected data, we analyzed various numerical features from the gaze behaviors to identify the most effective ones. 
We tested combinations of these features to determine the most effective one for predicting noticeability under visual stimuli.
Using the selected features, our regression model achieved a mean squared error (MSE) of 0.011 through leave-one-user-out cross-validation. 
Furthermore, we developed both a binary and a three-class classification model to categorize noticeability, which achieved an accuracy of 91.74\% and 85.62\%, respectively.

% evaluation study
To evaluate the generalizability of the regression model, we conducted an evaluation study (N=24) to test whether the model could accurately predict noticeability with new visual stimuli (color- and scale-based animations).
Specifically, we evaluated whether the model's predictions aligned with participants' responses under these unseen stimuli.
The results showed that our model accurately estimated the noticeability, achieving mean squared errors (MSE) of 0.014 and 0.012 for the color- and scale-based visual stimili, respectively, compared to participants' responses.
Since the tested visual stimuli data were not included in the training, the results suggested that the extracted gaze behavior features capture a generalizable pattern and can effectively indicate the corresponding impact on the noticeability of redirection.

% application
Based on our model, we implemented an adaptive redirection technique and demonstrated it through two applications: adaptive VR action game and opportunistic rendering.
We conducted a proof-of-concept user study (N=8) to compare our adaptive redirection technique with a static redirection, evaluating the usability and benefits of our adaptive redirection technique.
The results indicated that participants experienced less physical demand and stronger sense of embodiment and agency when using the adaptive redirection technique. 
These results demonstrated the effectiveness and usability of our model.

In summary, we make the following contributions.
% 
\begin{itemize}
    \item 
    We propose to use users' gaze behavior as a medium to quantify how visual stimuli influences the noticebility of redirection. 
    Through two user studies, we confirm that visual stimuli significantly influences noticeability and identify key gaze behavior features that are closely related to this impact.
    \item 
    We build a regression model that takes the user's gaze behavioral data as input, then computes the noticeability of redirection.
    Through an evaluation study, we verify that our model can estimate the noticeability with new participants under unseen visual stimuli.
    These findings suggest that the extracted gaze behavior features effectively capture the influence of visual stimuli on noticeability and can generalize across different users and visual stimuli.
    \item 
    We develop an adaptive redirection technique based on our regression model and implement two applications with it.
    With a proof-of-concept study, we demonstrate the effectiveness and potential usability of our regression model on real-world use cases.

\end{itemize}

% \delete{
% Virtual Reality (VR) allows the user to embody a virtual avatar by mirroring their physical movements through the avatar.
% As the user's visual access to the physical world is blocked in tasks involving motion control, they heavily rely on the visual representation of the avatar's motions to guide their proprioception.
% Similar to real-world experiences, the user is able to resolve conflicts between different sensory inputs (e.g., vision and motor control) through multisensory integration, which is essential for mitigating the sensory noise that commonly arises.
% However, it also enables unique manipulations in VR, as the system can intentionally modify the avatar's movements in relation to the user's motions to achieve specific functional outcomes,
% for example, 
% % the manipulations on the avatar's movements can 
% enabling novel interaction techniques of redirected walking~\cite{razzaque2005redirected}, redirected reaching~\cite{gonzalez2022model}, and pseudo haptics~\cite{samad2019pseudo}.
% With small adjustments to the avatar's movements, the user can maintain their sense of embodiment, due to their ability to resolve the perceptual differences.
% % However, a large mismatch between the user and avatar's movements can result in the user losing their sense of embodiment, due to an inability to resolve the perceptual differences.
% }

% \delete{
% However, multisensory integration can break when the manipulation is so intense that the user is aware of the existence of the motion offset and no longer maintains the sense of embodiment.
% Prior research studied the intensity threshold of the offset applied on the avatar's hand, beyond which the embodiment will break~\cite{li2022modeling}. 
% Studies also investigated the user's sensitivity to the offsets over time~\cite{kohm2022sensitivity}.
% Based on the findings, we argue that one crucial factor that affects to what extent the user notices the offset (i.e., \textit{noticeability}) that remains under-explored is whether the user directs their visual attention towards or away from the virtual avatar.
% Related work (e.g., Mise-unseen~\cite{marwecki2019mise}) has showcased applications where adjustments in the environment can be made in an unnoticeable manner when they happen in the area out of the user's visual field.
% We hypothesize that directing the user's visual attention away from the avatar's body, while still partially keeping the avatar within the user's field-of-view, can reduce the noticeability of the offset.
% Therefore, we conduct two user studies and implement a regression model to systematically investigate this effect.
% }

% \delete{
% In the first user study (N = 16), we test whether drawing the user's visual attention away from their body impacts the possibility of them noticing an offset that we apply to their arm motion in VR.
% We adopt a dual-task design to enable the alteration of the user's visual attention and a yes/no paradigm to measure the noticeability of motion offset. 
% The primary task for the user is to perform an arm motion and report when they perceive an offset between the avatar's virtual arm and their real arm.
% In the secondary task, we randomly render a visual animation of a ball turning from transparent to red and becoming transparent again and ask them to monitor and report when it appears.
% We control the strength of the visual stimuli by changing the duration and location of the animation.
% % By changing the time duration and location of the visual animation, we control the strengths of attraction to the users.
% As a result, we found significant differences in the noticeability of the offsets $(F_{(1,15)}=5.90,~p=0.03)$ between conditions with and without visual stimuli.
% Based on further analysis, we also identified the behavioral patterns of the user's gaze (including pupil dilation, fixations, and saccades) to be correlated with the noticeability results $(r=-0.43)$ and they may potentially serve as indicators of noticeability.
% }

% \delete{
% To further investigate how visual attention influences the noticeability, we conduct a data collection study (N = 12) and build a regression model based on the data.
% The regression model is able to calculate the noticeability of the offset applied on the user's arm under various visual stimuli based on their gaze behaviors.
% Our leave-one-out cross-validation results show that the proposed method was able to achieve a mean-squared error (MSE) of 0.012 in the probability regression task.
% }

% \delete{
% To verify the feasibility and extendability of the regression model, we conduct an evaluation study where we test new visual animations based on adjustments on scale and color and invite 24 new participants to attend the study.
% Results show that the proposed method can accurately estimate the noticeability with an MSE of 0.014 and 0.012 in the conditions of the color- and scale-based visual effects.
% Since these animations were not included in the dataset that the regression model was built on, the study demonstrates that the gaze behavioral features we extracted from the data capture a generalizable pattern of the user's visual attention and can indicate the corresponding impact on the noticeability of the offset.
% }

% \delete{
% Finally, we demonstrate applications that can benefit from the noticeability prediction model, including adaptive motion offsets and opportunistic rendering, considering the user's visual attention. 
% We conclude with discussions of our work's limitations and future research directions.
% }

% \delete{
% In summary, we make the following contributions.
% }
% % 
% \begin{itemize}
%     \item 
%     \delete{
%     We quantify the effects of the user's visual attention directed away by stimuli on their noticeability of an offset applied to the avatar's arm motion with respect to the user's physical arm. 
%     Through two user studies, we identified gaze behavioral features that are indicative of the changes in noticeability.
%     }
%     \item 
%     \delete{We build a regression model that takes the user's gaze behavioral data and the offset applied to the arm motion as input, then computes the probability of the user noticing the offset.
%     Through an evaluation study, we verified that the model needs no information about the source attracting the user's visual attention and can be generalizable in different scenarios.
%     }
%     \item 
%     \delete{We demonstrate two applications that potentially benefit from the regression model, including adaptive motion offsets and opportunistic rendering.
%     }

% \end{itemize}

\begin{comment}
However, users will lose the sense of embodiment to the virtual avatars if they notice the offset between the virtual and physical movements.
To address this, researchers have been exploring the noticing threshold of offsets with various magnitudes and proposing various redirection techniques that maintain the sense of embodiment~\cite{}.

However, when users embody virtual avatars to explore virtual environments, they encounter various visual effects and content that can attract their attention~\cite{}.
During this, the user may notice an offset when he observes the virtual movement carefully while ignoring it when the virtual contents attract his attention from the movements.
Therefore, static offset thresholds are not appropriate in dynamic scenarios.

Past research has proposed dynamic mapping techniques that adapted to users' state, such as hand moving speed~\cite{frees2007prism} or ergonomically comfortable poses~\cite{montano2017erg}, but not considering the influence of virtual content.
More specifically, PRISM~\cite{frees2007prism} proposed adjusting the C/D ratio with a non-linear mapping according to users' hand moving speed, but it might not be optimal for various virtual scenarios.
While Erg-O~\cite{montano2017erg} redirected users' virtual hands according to the virtual target's relative position to reduce physical fatigue, neglecting the change of virtual environments. 

Therefore, how to design redirection techniques in various scenarios with different visual attractions remains unknown.
To address this, we investigate how visual attention affects the noticing probability of movement offsets.
Based on our experiments, we implement a computational model that automatically computes the noticing probability of offsets under certain visual attractions.
VR application designers and developers can easily leverage our model to design redirection techniques maintaining the sense of embodiment adapt to the user's visual attention.
We implement a dynamic redirection technique with our model and demonstrate that it effectively reduces the target reaching time without reducing the sense of embodiment compared to static redirection techniques.

% Need to be refined
This paper offers the following contributions.
\begin{itemize}
    \item We investigate how visual attractions affect the noticing probability of redirection offsets.
    \item We construct a computational model to predict the noticing probability of an offset with a given visual background.
    \item We implement a dynamic redirection technique adapting to the visual background. We evaluate the technique and develop three applications to demonstrate the benefits. 
\end{itemize}



First, we conducted a controlled experiment to understand how users perceived the movement offset while subjected to various distractions.
Since hand redirection is one of the most frequently used redirections in VR interactions, we focused on the dynamic arm movements and manually added angular offsets to the' elbow joint~\cite{li2022modeling, gonzalez2022model, zenner2019estimating}. 
We employed flashing spheres in the user's field of view as distractions to attract users' visual attention.
Participants were instructed to report the appearing location of the spheres while simultaneously performing the arm movements and reporting if they perceived an offset during the movement. 
(\zhipeng{Add the results of data collection. Analyze the influence of the distance between the gaze map and the offset.}
We measured the visual attraction's magnitude with the gaze distribution on it.
Results showed that stronger distractions made it harder for users to notice the offset.)
\zhipeng{Need to rewrite. Not sure to use gaze distribution or a metric obtained from the visual content.}
Secondly, we constructed a computational model to predict the noticing probability of offsets with given visual content.
We analyzed the data from the user studies to measure the influence of visual attractions on the noticing probability of offsets.
We built a statistical model to predict the offset's noticing probability with a given visual content.
Based on the model, we implement a dynamic redirection technique to adjust the redirection offset adapted to the user's current field of view.
We evaluated the technique in a target selection task compared to no hand redirection and static hand redirection.
\zhipeng{Add the results of the evaluation.}
Results showed that the dynamic hand redirection technique significantly reduced the target selection time with similar accuracy and a comparable sense of embodiment.
Finally, we implemented three applications to demonstrate the potential benefits of the visual attention adapted dynamic redirection technique.
\end{comment}

% This one modifies arm length, not redirection
% \citeauthor{mcintosh2020iteratively} proposed an adaptation method to iteratively change the virtual avatar arm's length based on the primary tasks' performance~\cite{mcintosh2020iteratively}.



% \zhipeng{TO ADD: what is redirection}
% Redirection enables novel interactions in Virtual Reality, including redirected walking, haptic redirection, and pseudo haptics by introducing an offset to users' movement.
% \zhipeng{TO ADD: extend this sentence}
% The price of this is that users' immersiveness and embodiment in VR can be compromised when they notice the offset and perceive the virtual movement not as theirs~\cite{}.
% \zhipeng{TO ADD: extend this sentence, elaborate how the virtual environment attracts users' attention}
% Meanwhile, the visual content in the virtual environment is abundant and consistently captures users' attention, making it harder to notice the offset~\cite{}.
% While previous studies explored the noticing threshold of the offsets and optimized the redirection techniques to maintain the sense of embodiment~\cite{}, the influence of visual content on the probability of perceiving offsets remains unknown.  
% Therefore, we propose to investigate how users perceive the redirection offset when they are facing various visual attractions.


% We conducted a user study to understand how users notice the shift with visual attractions.
% We used a color-changing ball to attract the user's attention while instructing users to perform different poses with their arms and observe it meanwhile.
% \zhipeng{(Which one should be the primary task? Observe the ball should be the primary one, but if the primary task is too simple, users might allocate more attention on the secondary task and this makes the secondary task primary.)}
% \zhipeng{(We need a good and reasonable dual-task design in which users care about both their pose and the visual content, at least in the evaluation study. And we need to be able to control the visual content's magnitude and saliency maybe?)}
% We controlled the shift magnitude and direction, the user's pose, the ball's size, and the color range.
% We set the ball's color-changing interval as the independent factor.
% We collect the user's response to each shift and the color-changing times.
% Based on the collected data, we constructed a statistical model to describe the influence of visual attraction on the noticing probability.
% \zhipeng{(Are we actually controlling the attention allocation? How do we measure the attracting effect? We need uniform metrics, otherwise it is also hard for others to use our knowledge.)}
% \zhipeng{(Try to use eye gaze? The eye gaze distribution in the last five seconds to decide the attention allocation? Basically constructing a model with eye gaze distribution and noticing probability. But the user's head is moving, so the eye gaze distribution is not aligned well with the current view.)}

% \zhipeng{Saliency and EMD}
% \zhipeng{Gaze is more than just a point: Rethinking visual attention
% analysis using peripheral vision-based gaze mapping}

% Evaluation study(ideal case): based on the visual content, adjusting the redirection magnitude dynamically.

% \zhipeng{(The risk is our model's effect is trivial.)}

% Applications:
% Playing Lego while watching demo videos, we can accelerate the reaching process of bricks, and forbid the redirection during the manipulation.

% Beat saber again: but not make a lot of sense? Difficult game has complicated visual effects, while allows larger shift, but do not need large shift with high difficulty




\section{Related work}
\mvspace{-2mm}
\paragraph{Human-AI complementarity.}

Many empirical studies of human-AI collaboration focus on AI-assisted human decision-making for legal, ethical, or safety reasons~\citep{bo2021toward, boskemper2022measuring, bondi2022role, schemmer2022meta}.
However, a recent meta-analysis by \citet{vaccaro2024combinations} finds that, on average, human–AI teams perform worse than the better of the two agents alone. 
In response, a growing body of work seeks to evaluate and enhance complementarity in human–AI systems \citep{bansal2021does, bansal2019updates, bansal2021most, wilder2021learning, rastogi2023taxonomy, mozannar2024effective}.
The present work differs from much of this prior work by approaching human-AI complementarity from the perspective of information value and use, including asking whether the human and AI decisions provide additional information that is not used by the other.
\mvspace{-2mm}
\paragraph{Evaluation of human decision-making with machine learning.}
Our work contributes methods for evaluating the decisions of human-AI teams~\citep{kleinberg2015prediction, kleinberg2018human, lakkaraju2017selective, mullainathan2022diagnosing,  rambachan2024identifying, guo2024decision, ben2024does, shreekumar2025x}.
\citet{kleinberg2015prediction} proposed that evaluations of human-AI collaboration should be based on the information that is available at the time of decisions.
% \jessica{can omit:} A significant portion of this literature addresses \textit{performative prediction}~\citep{perdomo2020performative}, where predictions or decisions affect future outcomes. 
% Because counterfactual decisions’ outcomes remain unobserved, researchers typically rely on worst-case analyses to bound the potential performance \citep{rambachan2024identifying, ben2024does}. 
% Though these issues arise in many canonical human-AI collaboration tasks, we focus on standard ``prediction policy problems'' where the payoff can be translated into policy gains~\citep{kleinberg2015prediction}.
According to this view, our work defines Bayesian best-attainable-performance benchmarks similar to several prior works~\citep{guo2024decision, wu2023rational,agrawal2020scaling, fudenberg2022measuring}. 
Closest to our work, \citet{guo2024decision} model the expected performance of a rational Bayesian agent faced with deciding between the human and AI recommendations as the theoretical upper bound on the expected performance of any human-AI team.
This benchmark provides a basis for identifying exploitable information within a decision problem.

\mvspace{-3mm}
\paragraph{Human information in machine learning.}

Some approaches focus on automating the decision pipeline by explicitly incorporating human expertise in developing machine learning models, such as by learning to defer~\citep{mozannar2024show, madras2018predict, raghu2019algorithmic, keswani2022designing, keswani2021towards, okati2021differentiable}.
\citet{corvelo2023human} propose multicalibration over human and AI model confidence information to guarantee the existence of an optimal monotonic decision rule.
\citet{alur2023auditing} propose a hypothesis testing framework to evaluate the added value of human expertise over AI forecasts.
Our work shares the motivation of incorporating human expertise, but targets a slightly broader scope by quantifying the information value for all available signals and agent decisions in a human–AI decision pipeline.



\mvspace{-3mm}
\section{Methodology}

% \jessica{Also, we should be clearly defining what information our approach assumes as input, and what the output is (at least at a high level)}
\mvspace{-2mm}
Our framework takes as input a decision problem associated with an information model and outputs the value of information of any available signals to any agent, conditioning on the existing information in their decisions within a Bayesian decision theoretic framework.
Our framework provides two separate functions to quantify the value of information: one globally across the data-generating process, and one locally in a realization drawn from the data-generating process.
We also introduce a robust analysis approach to information order, which enables us to compare the agent-complementary information in signals for all possible decision problems.
% \jessica{not sure why some things are italized ... for example why is realization not italicized when it first appears? should globally and locally be italicized instead of data-generating process? could probably just remove all italics or only use italics for things we will define specifically}

% In this section, we define the basis of this approach, including a decision problem and associated information structure, following prior decision-theoretic frameworks for studying decisions from statistical information~\citep{wu2023rational,guo2024decision,hullman2024decision}.
% Then we define how a rational decision-maker would act given \st{a signal and} \jessica{such} a decision problem and associated information structure, and use rational \jessica{behavior within the problem} \st{decision-maker ,as a tool  we show how} to \st{investigate} \jessica{characterize} the information encoded in behavioral decisions.
%\jessica{May want to add a sentence or two here to give the reader some intuition for our approach. E.g., Our approach relies on analysis of the marginal gain ... }

%\ziyang{Merge infomration structure and decision-making problem into one section}
\mvspace{-4mm}
\paragraph{Decision Problem.} A decision problem consists of three key elements. We illustrate with an example of a weather decision. 
\mvspace{-2mm}
\begin{itemize}[wide]
    \mvspace{-2mm}
    \item A payoff-relevant state $\payoffstatevalue$ from a space $\payoffstatespace$. For example,\ $\payoffstatevalue \in \payoffstatespace =  \{0, 1\} = \{\text{no rain}, \text{rain}\}$.
    \mvspace{-3mm}
    \item A decision $\action$ from the decision space $\actionspace$ characterizing the decision-maker (DM)'s choice. For example,\ $\action\in \actionspace = \{0, 1\} = \{\text{not take umbrella}, \text{take umbrella}\}$.
    \mvspace{-2mm}
    \item A payoff function $\score: \actionspace\times\payoffstatespace\to\mathbb{R}$, used to assess the quality of a decision given a realization of the state. For example, $\score(\action = 0, \payoffstatevalue = 0) = 0, \score(\action = 0, \payoffstatevalue = 1) = -100, \score(\action = 1, \payoffstatevalue = 0) = -50, \score(\action = 1, \payoffstatevalue = 1) = 0$, which punishes the DM for selecting an action that does not match the weather. 
\end{itemize}

In decision problems corresponding to prediction tasks, the decision space is a probabilistic belief over the state space, i.e., $\actionspace = \Delta(\payoffstatespace)$.
For such problems, a payoff function is said to be a \textit{proper} scoring rule if the optimal action is to predict the true distribution, i.e., $p = \arg \max_{\action \in \actionspace} \expect[\payoffstatevalue \sim p]{\score(\action, \payoffstatevalue)}$.
% Therefore, given a proper scoring rule a DM who maximizes the expected payoff will truthfully report their belief.
For any decision problem with payoff function $\score: \actionspace\times\payoffstatespace\to\mathbb{R}$, there is an equivalent proper scoring rule $\hat{\score}: \Delta(\payoffstatespace)\times\payoffstatespace\to\mathbb{R}$ defined by choosing the optimal decision under the reported
belief. Formally,
\begin{equation}
\label{eq:properscoring}
    \hat{\score}(p, \payoffstatevalue) = \score(\arg \max_{\action \in \actionspace} \expect[\payoffstatevalue \sim p]{\score(\action, \payoffstatevalue)}, \payoffstatevalue).
\end{equation}

\Cref{eq:properscoring} shows a reduction from the payoff function $\score$ to the proper scoring rule $\hat{\score}$, i.e., any decision $p$ under $\hat{\score}$ represents a decision $\action$ in $\score$ that best-responds to the distribution $p$.
Therefore, the best-attainable performance defined in the proper scoring rule $\hat{\score}$ is equivalent to the best-attainable performance defined in any payoff function $\score$ that can be reduced to $\hat{\score}$.
Throughout our framework, we prefer proper scoring rules over non-proper scoring rules, since the best-attainable performance defined in the former implies the best-attainable performance in the latter.

\mvspace{-4mm}
\paragraph{Information Model.} 
We cast the information available to a DM as a signal defined within an information model.
We use the definition of an information model in \citet{blackwell1951comparison}. 
The information model can be represented by a \textit{data-generating model} with a set of \textit{signals}.
\begin{itemize}[wide]
    \mvspace{-3mm}
    \item \textit{Signals}. There are $n$ ``basic signals'' represented as random variables $\basicsig_1, \ldots, \basicsig_n$, from the signal spaces $\basicsigsp_1, \ldots, \basicsigsp_n$. Basic signals represent information available to a decision-maker, e.g., $\basicsigsp_1 = \{\text{cloudy}, \text{not cloudy}\}$, $\basicsigsp_2\in \{0, \ldots, 100\}$ for temperature in Celsius, etc. 
    % We write $k_i = |\basicsigsp_i|$ as the size of the signal space of the basic signal $i$, $\basicsig_i$ as the random variable for basic signal $i$, and $\basicsigval_{ij_i}\in \basicsigsp_i$ as the $j_i$th realized value of the $\basicsig_i$ ($j_i\leq k_i$).
    % E.g.\ observable features about the weather $\{\sig_1, \sig_2, \ldots\} = \{\text{temperature}, \text{cloud level}, \dots\}$. 
    % In addition to the basic signals, there are also other signals that \st{intuitively} represent the combination of basic signals.
    The decision-maker observes a signal, which is a subset of the basic signals, $\sig \subseteq 2^{\{\basicsig_1, \dots, \basicsig_n\}}$. 
    Specifically, we use $\sig = \{\basicsig_{j_1}, \ldots, \basicsig_{j_k}\}$ for a signal having $k$ basic signals and denote the signal space as $\sigsp = \basicsigsp_{j_1} \times \ldots \times \basicsigsp_{j_k}$.
    For example,\ a signal representing a combination of two basic signals $\sig = \{\basicsig_1, \basicsig_2\}$ observed by the decision-maker might consist of cloudiness $\basicsig_1$ and the temperature $\basicsig_2$ of the day. Given a signal composed of $m$ basic signals, we write the realization of $\sig$ as $\sigval = (\basicsigval_{j_1}, \dots, \basicsigval_{j_{k}})$, where the realizations $\basicsigval_{j_i} \in \basicsigsp_{j_i}$ are sorted by the index of the basic signals $j_i \in [n]$.
    The union $\sig$ of two signals $\sig_1, \sig_2$ takes the set union, i.e., $\sig = \sig_1\cup\sig_2$.
    % Though $\sig$ is initially defined as a set of random variables, we will slightly abuse notation $\sig$ to represent a random variable that is drawn from the joint distribution of the basic signals in it.
    % \jessica{I'm finding this part really confusing - eg we use capital V to refer to a signal, then lower case v}
     \mvspace{-1mm}
    \item \textit{Data-generating process}. A data-generating process is a joint distribution $\dgp\in \Delta(\basicsigsp_1 \times \ldots \times \basicsigsp_n \times\payoffstatespace)$ over the basic signals and the payoff-relevant state. $\dgp$ can be viewed as the combination of two distributions: the prior distribution of the state $\Pr[\payoffstatevalue]$ and the signal-generating distribution $\Pr[\sigval | \payoffstatevalue]$ defining the conditional distribution of signals. The DM may only observe a subset $\sig$ of the $n$ basic signals. We can define the Bayesian posterior belief upon receiving a signal $\sig = \sigval$ from the data-generating model as
     \mvspace{-1mm}
    \[\dgp(\payoffstatevalue| \sigval) := \Pr[\payoffstatevalue|\sigval]=\frac{\dgp(\sigval, \payoffstatevalue)}{\dgp(\payoffstatevalue)}\]
    % Conditioning on receiving a signal $\sig = \sigval$, \jessica{a DM} \st{the DMs} who knows the data-generating process is able to infer the Bayesian posterior $\Pr[\payoffstatevalue|\sigval]$ of the state, thus improving their payoff. % \jessica{should we mention that this DM has the prior?} 
 \mvspace{-4mm}

    \noindent where we slightly abuse notation to write $\dgp(\sigval, \payoffstatevalue)$ as the marginal
     probability of the signal realized to be $\sigval$ and the state being $\payoffstatevalue$ with expectation over other signals.
\end{itemize}

The choice of basic signals directly impacts how many observed samples are required to get a good estimate of the data-generating process.
When high-dimensional signals such as images or text are used, it may not be computationally feasible to estimate the data-generating process.
In such cases, one can preprocess the high-dimensional signals to get low-dimensional representations which (ideally) capture any important structure. 
These lower-dimensional signals can be defined by humans, such as when concepts are identified and then used to label the high-dimensional signals (e.g., images or parts of images). 
Alternatively, they can be defined algorithmically, by strategically applying dimensionality reduction to generate low-dimensional embeddings.
We introduce an algorithm for identifying an ``optimal'' reduction model in decision problems in \Cref{app:high-dimensional}.
%Another way is to let human label the concepts in high-dimensional signals, which generates interpretable low-dimensional signals.
We demonstrate these two methods in \Cref{exp2} and \Cref{exp1} respectively.

% \ziyang{several ways to do this. Use the algorithm to find low-dimensional but you can interpret these. Or let human label concepts. (This is what we've done in our study). Mention when we want to just use AI model's predictions to represent. }

%Slightly abusing notations, we write $\dgp(\payoffstatevalue)$ as the prior probability of the state $\payoffstatevalue$. 

%There is a payoff-related uncertain state $\payoffstate$ of interest to the decision-maker, e.g., $\payoffstate \in \{0, 1\} = \{\text{no rain}, \text{rain}\}$.
%There are also $n$ signals, $\sig_1, \ldots, \sig_n$, modeled as random variables. 
%These signals represent the information displayed to the decision-maker, e.g., whether it is cloudy $\sig_i \in \{0, 1\}$.
%An information structure $\infostructure$ is given by $\payoffstate$, $\sig_1$, $\ldots$, $\sig_n$ and a data generating process $\dgp \in \Delta(\payoffstate \times \sig_1 \times \ldots \times \sig_n)$, which describes the joint distribution between state and signals.

% We use lower-case $\payoffstatevalue, \sigval_1, \ldots, \sigval_n$ to refer to the outcomes generated under $\dgp$.
% Given a realization $\sigval_i$ of $\sig_i$, the probability that $\payoffstate = \payoffstatevalue$ conditioned on $\sig_i = \sigval_i$ can be obtained \jessica{by Bayesian updating the} prior: $\Pr[\payoffstate = \payoffstatevalue | \sig_i = \sigval_i] = \dgp(\payoffstatevalue, \sigval_i) / \dgp(\sigval_i)$.
% Similarly, we can also define the probability of $\payoffstate$ conditioned on realizations of multiple signals.
\mvspace{-4mm}
\paragraph{Information value.}
Our framework quantifies the value of information in a signal $\sig$ as the expected payoff improvement of an idealized agent who has access to $\sig$ in addition to some baseline information set.
% \jessica{this whole section could benfeit from a few more sentences early in the subsections to reiterate what we are tryin to achieve, or even phrases. E.g., here it seems we want to quantify the information value of some signal, relative to another signal. Give the reader more 'sign posts' to help remind them why we are setting up different concepts the way we are. Can be simple as adding a phrase like 'To quantify the vaue of the information is some set of signals' to the beginning of the first sentence. Its easy to lose the point currently}
We suppose a rational Bayesian DM who knows the prior probability of the state and conditional distribution of signals (i.e., the data-generating process), observes a signal realization, updates their prior to arrive at posterior beliefs, and then chooses a decision to maximize their expected payoff given their posterior belief. 
Formally, given a decision task with payoff function $\score$ and an information model $\dgp$, the rational DM's expected payoff given a (set of) signal(s) $\sig$ is
\mvspace{-1mm}
\begin{equation}
\mathrm{R}^{\dgp, \score}
(\sig)
= \expect[(\sigval, \payoffstatevalue) \sim \dgp]{\score(\action^r(\sigval), \payoffstatevalue)}
\end{equation}
\mvspace{-4mm}
\noindent where $\action^r(\cdot): \sigsp \rightarrow \actionspace$ denotes the decision rule adopted by the rational DM.
\begin{equation}
\label{eq:rationalDM}
    \action^r(\sigval) = \arg \max_{\action\in\actionspace} \expect[\payoffstatevalue \sim \dgp(\payoffstatevalue|\sigval)]{\score(\action, \payoffstatevalue)}
\end{equation}

We use $\emptyset$ to represent a null signal, such that $\mathrm{R}^{\dgp, \score}(\emptyset)$ is the expected payoff of a Bayesian rational DM who only uses the prior distribution to make decisions.
In this case, the Bayesian rational DM will take the best fixed action under the prior, and their expected payoff is:
\mvspace{-1mm}
\begin{equation}
\label{eq:baseline}
\mathrm{R}^{\dgp, \score}
(\emptyset) 
= \max_{\action \in \actionspace} \expect[\payoffstatevalue \sim \pi]{\score(\action, \payoffstatevalue)}
\end{equation}
\mvspace{-4mm}

This baseline defines the maximum expected payoff that can be achieved with no information.
Bayesian decision theory quantifies the information value of $\sig$ by the payoff improvement of $\sig$ over the payoff obtained without information.
% Given a set of signals $\sig_1$ and a ground set of signals $\sig_2$ (which could be the null signal $\emptyset$), we can define the \textit{information gain} from $\sig_1$ over $\sig_2$, the payoff improvement of $\sig_1$ over the payoff obtainable from $\sig_2$.
% \begin{equation}
% \infoval^{\dgp, \score}(\sig_1; \sig_2) = \mathrm{R}^{\dgp, \score}
% (\sig_1\cup\sig_2) - \mathrm{R}^{\dgp, \score}
% (\sig_2).
% \end{equation}


\begin{definition}
Given a decision task with payoff function $\score$ and an information model $\dgp$, the information value of $\sig$ is defined as
\mvspace{-1mm}\begin{equation}
    \IV^{\dgp, \score}(\sig) = \mathrm{R}^{\dgp, \score}
(\sig) - \mathrm{R}^{\dgp, \score}
(\emptyset)
\end{equation}
\end{definition}
\mvspace{-2mm}
We adopt the same idea to define the agent-complementary information value in our framework.

% $IV$ reflects the marginal information offered by $\sig$ over $\emptyset$.
% In human-AI collaboration, we may especially be interested in the complementary information offered by a signal (e.g., AI prediction and explanation) over human information.
% In that case, $IV$ can be defined as \[IV^{\dgp, \score}(\sig) = \infoval^{\dgp, \score}(\sig; \actionvar^b)\]where $\actionvar^b$ is a random variable for human decisions, which we defined in the following section.
\mvspace{-2mm}
\subsection{Agent-Complementary Information Value}
\mvspace{-2mm}

With the above definitions, it is possible to measure the additional value that new signals can provide over the information already captured by an agent’s decisions. Here, \textit{agent} may refer to a human, an AI system, or a human–AI team.
The intuition behind our approach is that any information that is used by decision-makers should eventually reveal itself through variation in their decisions.
We recover the information value in agent decisions by offering the decisions as a signal to the Bayesian rational DM.
We model the agent decisions as a random variable $\actionvar^b$ from the action space $\actionspace$, which follows a joint distribution $\dgp \in \Delta(\basicsigsp_1 \times \ldots \times \basicsigsp_n \times \payoffstatespace \times  \actionspace)$ with the state and signals.
The expected payoff of a Bayesian rational DM who knows $\dgp$ is given by the function:
 \mvspace{-1mm}
\[
\mathrm{R}^{\dgp, \score}
(\actionvar^b)
= \expect[(\action^b, \payoffstatevalue) \sim \dgp]{\score(\action^r(\action^b), \payoffstatevalue)}
\]
% \[
%  \mathrm{R}%^{\dgp, \score}
%  (\actionvar^b) = \expect[\action^b \sim \dgp^b]{\max_{\action \in \actionspace}\expect[\payoffstatevalue \sim \Pr(\payoffstatevalue | \actionvar^b = \action^b)]{\score(\action, \payoffstatevalue)}}
% \]
 \mvspace{-5mm}
 
% Similarly, we can assess the potential for other available information to improve agent decisions by quantifying the information gain from different signals (such as instance feature information or AI predictions) over the agent decisions alone. 
We seek to identify signals $\sig$ that can potentially improve agent decisions by analyzing the information value in the combined signal $\actionvar^b \cup \sig$ over the information value in $\actionvar^b$, which we define as the agent-complementary information value.
\begin{definition}
\label{def:aciv}
Given a decision task with payoff function $\score$ and an information model $\dgp$, we define the agent-complementary information value ($\ACIV$) of $\sig$ on agent decisions $\actionvar^b$ as 
\mvspace{-1mm}\begin{equation}
    \ACIV^{\dgp, \score}(\sig; \actionvar^b) = \mathrm{R}^{\dgp, \score}(\actionvar^b \cup \sig) - \mathrm{R}^{\dgp, \score}(\actionvar^b)
\end{equation}
\end{definition}
\mvspace{-3mm}

If the $\ACIV$ of a signal $\sig$ is small relative to the baseline (\ref{eq:baseline}), this means either that the information value of $\sig$ to the decision problem is low (e.g., it is not correlated with $\payoffstatevalue$), or that the agent has already exploited the information in $\sig$ (e.g., the agent relies on $\sig$ or equivalent information to make their decisions such that their decisions correlate with $\payoffstatevalue$ in the same way as $\sig$ correlates with $\payoffstatevalue$).
If, however, the $\ACIV$ of $\sig$ is large relative to the baseline, then at least in theory, the agent can improve their payoff by incorporating $\sig$ in their decision making.

 \mvspace{-1mm}
Furthermore, $\ACIV$ can reveal complementary information between different types of agents. 
For instance, if we view AI predictions as $\sig$ and treat human decisions as the agent signal $\actionvar^b$, a large $\ACIV$ indicates that AI predictions add considerable value beyond what humans alone achieve. In the reverse scenario, if human decisions serve as $\sig$ and AI predictions are $\actionvar^b$, we can measure how much humans can contribute over the information captured in the AI predictions. We demonstrate uses of $\ACIV$ in \Cref{exp2} and \Cref{exp1}.

 \mvspace{-2mm}
\subsection{Instance-level Agent-complementary Information Value}
 \mvspace{-2mm}
$\ACIV$ quantifies the value of the decision-relevant information in a signal $\sig$ across all possible realizations defined by the data-generating model.
To provide analogous instance-level quantification of information value, we define Instance-Level agent-complementary Information Value ($\ILIV$) to quantify the value of the decision-relevant information encoded by a single realization of the signal.
% Instance-level Agent-Complementary Information Value ($\ILIV$) evaluates the additional information contributed by a single realization of a signal rather than the entire joint distribution.
% $\ACIV$ can be useful in cases such as evaluation/comparison of models and empirical analysis of whether AI's assistance can help humans improve the information value in their decisions.
This finer-grained view makes it possible to analyze how much an agent can benefit in theory from better incorporating instance-level information in their decision.

Given a realization of signal $\sigval = \{\basicsigval_{j_1}, \ldots, \basicsigval_{j_k}\}$, we want to know the maximum expected payoff gain from gaining access to $\sigval$ on the instances where $\sigval$ is realized over the existing information encoded in agent decisions. Intuitively, this captures how much ``room'' there is for specific decisions to be improved through better use of the signal.
To calculate instance-level information value, we rely on the performance of the Bayesian rational agent on the conditional distribution $\dgp(\basicsigval_1, \ldots, \basicsigval_n, \payoffstatevalue, \action^b | \sigval)$ instead of the whole distribution $\dgp(\basicsigval_1, \ldots, \basicsigval_n, \payoffstatevalue, \action^b)$.
% We use a Bayesian rational DM to quantify $\ILIV$ in a similar way to $\ACIV$.
% Instead of evaluating on the distribution $\dgp(\sigval, \payoffstatevalue)$, $\ILIV$ evaluates the expected payoff of the rational DM on the distribution indicated by the instance $\dgp(\payoffstatevalue | \sigval)$.  \jessica{whats the intuition here? the explanation doesn't provide much of a sense of why this is useful}
Formally, given a decision task with payoff function $\score$ and information model $\dgp$, the expected payoff of the rational DM given signal $\sig = \sigval$ on instances where $\sig = \sigval$ is
\begin{equation}
\mathrm{r}^{\sigval, \dgp, \score}(\sigval) = \expect[\payoffstatevalue \sim \dgp(\payoffstatevalue | \sigval)]{\score(\action^r(\sigval), \payoffstatevalue)}
\end{equation}
% \noindent where $\action^r(\cdot)$ is defined in \Cref{eq:rationalDM}. 
\noindent where $\action^r(\sigval)$ is the Bayesian optimal decision on receiving  $\sigval$ as defined in \Cref{eq:rationalDM}.
If we consider the agent decisions in addition to the realization $\sigval$, the rational DM's expected payoff on instances where $\sig = \sigval$ can be written as
\begin{equation}
\mathrm{r}^{\sigval, \dgp, \score}(\sigval; \actionvar^b) = \expect[(\action^b, \payoffstatevalue) \sim \dgp(\action^b, \payoffstatevalue | \sigval)]{\score(\action^r(\sigval \cup \action^b), \payoffstatevalue)}
\end{equation}

\begin{definition}
\label{def:RIIV}
Given a decision task with payoff function $\score$ and an information model $\dgp$, we define the instance-level agent-complementary information value ($\ILIV$) of signal $\sig = \sigval$ on instances where $\sig = \sigval$ as:
 \mvspace{-1mm}
\begin{equation}
    \ILIV^{\sigval, \dgp, \score}(\sigval; \actionvar^b) = \mathrm{r}^{\sigval, \dgp, \score}(\sigval; \actionvar^b) - \mathrm{r}^{\sigval, \dgp, \score}(\emptyset; \actionvar^b)
\end{equation}
\end{definition}
 \mvspace{-4mm}
\noindent where $\mathrm{r}^{\sigval, \dgp, \score}(\emptyset; \actionvar^b)$ represents the rational DM's expected payoff on instances where $\sig = \sigval$ with only the exisiting information encoded in agent decisions.
Taking the expectation of $\ILIV$ over $\sig$ recovers the global agent-complementary information value ($\ACIV$), i.e.,
\begin{equation}
    \ACIV^{\dgp, \score}(V; \actionvar^b) = \expect[\sigval \sim \dgp(\sigval)]{\ILIV^{\sigval, \dgp, \score}(\sigval; \actionvar^b)}\nonumber
\end{equation}

\subsubsection{ILIV-SHAP Information-based Explanation}

We apply $\ILIV$ to define an \textit{information-based} explanation technique (ILIV-SHAP) that extends the canonical SHAP feature saliency explanation of a model's prediction. ILIV-SHAP communicates how the data features lead to changes in the information value of AI predictions. While traditional SHAP summarizes the average contribution of each feature to a specific prediction over the baseline (prior) prediction, 
ILIV-SHAP summarizes the average contribution of each feature to the decision-relevant information value contained in the prediction. 

% We extend the SHAP algorithms to calculate the effect scores in ILIV-SHAP. 
Suppose a model $f$ that takes as input $m$ features and outputs a real number.
Given an instance $\mathbf{x} = (x_1, \ldots, x_m)$, the importance of one feature $x_i$ to the model output $f(\mathbf{x})$ is encoded by the expected difference of model outputs when $x_i$ is marginalized out.
Specifically, this is quantified by $f(\mathbf{x}) - \expect{f(X) | X_{-i} = \mathbf{x}_{-i}}$, where $X_{-i}$ denotes all features except $X_i$.
Considering the interaction between features, SHAP~\citep{lundberg2017unified} uses the Shapley value to quantify the importance scores averaged on different combinations of features:
\begin{equation*}
    \phi_i(f, \mathbf{x}) = \sum_{\mathbf{x}' \subseteq \mathbf{x}} \frac{|\mathbf{x}'|!(m - |\mathbf{x}'| - 1)!}{m!} [g_f(\mathbf{x}') - g_f(\mathbf{x}' \backslash x_i)]
\end{equation*}
\noindent where $g_f(\mathbf{x}')$ denotes the expected model output conditioned on $\mathbf{x}'$, i.e., $\expect{f(X)| X' = \mathbf{x}'}$ for any $\mathbf{x}' \subseteq \mathbf{x}$. The scores $\phi_i(f, \mathbf{x})$ output by SHAP construct an explanation model for a model output, which quantifies the expected counterfactual change in the model output caused by the feature $x_i$.

ILIV-SHAP extends SHAP to give an explanation model on how features impact the decision-relevant information value of an individual model output.

\begin{definition}[ILIV-SHAP]
\label{def:shap}
Given a model $f$ and data features $\mathbf{x} = (x_1, \ldots, x_m)$, the importance score of the \textit{i}-th feature by ILIV-SHAP is 
\begin{equation*}
    \phi_i^{\ILIV}(f, \mathbf{x}) = \sum_{\mathbf{x}' \subseteq \mathbf{x}} \frac{|\mathbf{x}'|!(m - |\mathbf{x}'| - 1)!}{m!} [\ILIV^{f(\mathbf{x}), \dgp, \score}(g_f(\mathbf{x}'); \actionvar^b) - \ILIV^{f(\mathbf{x}), \dgp, \score}(g_f(\mathbf{x}' \backslash x_i); \actionvar^b)]
\end{equation*}
\end{definition}
\noindent where $\ILIV^{f(\mathbf{x}), \dgp, \score}(g_f(\mathbf{x}'); \actionvar^b)$ denotes a counterfactual evaluation of $\ILIV$, which quantifies the expected payoff gain from additionally knowing $g_f(\mathbf{x}')$ on the instances where $f(\mathbf{x})$ is realized. 
This counterfactual version of $\ILIV$ is guaranteed to achieve the maximum at $\mathbf{x}'$ where $f(\mathbf{x}) = g_f(\mathbf{x}')$, i.e., the features missing from $\mathbf{x}'$ have no impact on $f(\mathbf{x})$.

The explanation model offered by ILIV-SHAP is grounded in the following two properties.
First, $\phi^{\ILIV}_i(f, \mathbf{x})$ are consistent with the extent to which feature $x_i$ contributes to the information value in $f(\mathbf{x})$.
Specifically, for any two models $f$ and $f'$ and any $\mathbf{x}' \subseteq \mathbf{x}$, if $x_i$ contributes more to the information value in $f'$ than the information value in $f$, i.e., $\ILIV^{f'(\mathbf{x}), \dgp, \score}(g_{f'}(\mathbf{x}'); \actionvar^b) - \ILIV^{f'(\mathbf{x}), \dgp, \score}(g_{f'}(\mathbf{x}' \backslash x_i); \actionvar^b) \geq \ILIV^{f(\mathbf{x}), \dgp, \score}(g_{f}(\mathbf{x}');\actionvar^b) - \ILIV^{f(\mathbf{x}), \dgp, \score}(g_{f}(\mathbf{x}' \backslash x_i); \actionvar^b)$, then $\phi^{\ILIV}_i(f', \mathbf{x}) \geq \phi^{\ILIV}_i(f, \mathbf{x})$.
This property of ILIV-SHAP follows the consistency property of SHAP~\citep{lundberg2017unified}.
Second, summing up the scores $\phi^{\ILIV}_i(f, \mathbf{x})$ recovers the information value of model output, $\ILIV^{f(\mathbf{x}), \dgp, \score}(f(\mathbf{x}); \actionvar^b)$.
Formally, for any $f$ and any $\mathbf{x}$, $\ILIV^{f(\mathbf{x}), \dgp, \score}(f(\mathbf{x}); \actionvar^b) = \ILIV^{f(\mathbf{x}), \dgp, \score}(\expect{f(X)}; \actionvar^b) + \sum_{i = 1}^m \phi^{\ILIV}_i(f, \mathbf{x})$.
We demonstrate use of ILIV-SHAP in \Cref{exp3}.


% The information value of the realization $\sig=\sigval$ should be quantified relatively to a counterfactual realization $\sig = \sigval^+$. \jessica{is this not v? can we notate it like that?} Formally, given a decision task with payoff function $\score$ and an information model $\dgp$, the rational DM's expected payoff given the counterfactual signal $\sig = \sigval^+$ on instances where $\sig = \sigval$ is
% \[
% \mathrm{r}^{\sigval, \dgp, \score}(\sigval^+; \sig') = \expect[(\sigval', \payoffstatevalue) \sim \dgp(\sigval', \payoffstatevalue | \sigval)]{\score(\action^r((\sigval^+, \sigval')), \payoffstatevalue)}
% \]

% \begin{proposition}
% \label{prop:counterfactual}
%     For any $\sigval^+ \in \sigsp$, 
%     \[\mathrm{r}^{\sigval, \dgp, \score}(\sigval; \sig') \geq \mathrm{r}^{\sigval, \dgp, \score}(\sigval^+; \sig')\]
% \end{proposition}

% The proof of \Cref{prop:counterfactual} is followed by the fact that any payoff function $\score$ can induce a proper scoring rule $\hat{\score}: \distover{\payoffstatespace} \times \payoffstatespace \rightarrow \mathbb{R}$.


% ILIV-SHAP takes as input a decision task with payoff function $\score$, a model $f$, an instance with features $\mathbf{x} = (x_1, \ldots, x_m)$ and a set of realizations of model predictions, human decisions and state.
% We denote the model predictions by $\action^{f} = f(\mathbf{x})$ and the human decisions by $\action^b$.
% % $[(f(\mathbf{x})_1, \action^b_1, \payoffstatevalue_1), \ldots, (f(\mathbf{x})_m, \action^b_m, \payoffstatevalue_m)]$.
% The construction of ILIV-SHAP proceeds in three steps: 1) Estimate the information model $\dgp(\action^f, \action^b, \payoffstatevalue)$ from the observed realizations, 2) Calculate the factual $\ILIV^{\action^f, \dgp, \score}(\action^f; \actionvar^b)$ and the counterfactual $\ILIV^{\action^f, \dgp, \score}({\action^f}'; \actionvar^b)$ for any ${\action^f}' \neq \action^f$, and 3) Run the SHAP algorithm \citep{lundberg2017unified} to estimate the effect score $\phi^{\ILIV}_i$ for each feature $x_i$ with the objective function as $\ILIV$.

% \begin{algorithm}[tb]
%    \caption{ILIV-SHAP}
%    \label{alg:iliv-shap}
% \begin{algorithmic}
%    \STATE {\bfseries Input:} Payoff function $\score$, model $f$, data features $\mathbf{x}$, a set of realizations $[(\action^f_t, \action^b_t, \payoffstatevalue_t)]_{t=1}^T$
%    \STATE Estimate the empirical distribution $\dgp(\action^f, \action^b, \payoffstatevalue) = \frac{1}{T}\sum_{t=1}^T \mathbb{1}_{\action^f_t = \action^f}\cdot\mathbb{1}_{\action^b_t = \action^b}\cdot\mathbb{1}_{\payoffstatevalue_t = \payoffstatevalue}$, $\forall \action^f, \action^b, \payoffstatevalue$.
%    \STATE Set current signal $\action^f_0 = f(\mathbf{x})$.
%    \STATE Construct function $\ILIV^{\action^f_0, \dgp, \score}(f(\cdot); \actionvar^b)$.
%    \STATE Run SHAP algorithm on function $\ILIV^{\action^f_0, \dgp, \score}(f(\cdot); \actionvar^b)$ to generate $(\phi^{\ILIV}_1, \ldots, \phi^{\ILIV}_m)$.
% \end{algorithmic}
% \end{algorithm}

% Vanilla SHAP defines a \textit{saliency-based} explanation with a set of effect variables $\phi_i$ representing the influence of the realization of basic signal $\basicsig_i = \basicsigval_i$ on the model output $f(\sigval)$, where $\sigval = (\basicsigval_1, \ldots, \basicsigval_n)$ and $f: \sigsp \rightarrow \mathbb{R}$.
% When $\phi_i$ is positive, it means that the realization of $\basicsig_i = \basicsigval_i$ leads the model prediction $f(\sigval)$ to increase $\phi_i$ on the expectation over other basic signals.
% We let $Y$ denote the random variable for model output, $Y = f(\sig)$ for a predictive model $f(\cdot): \sigsp \rightarrow \mathbb{R}$.
% SHAP~\citep{lundberg2017unified} defines a \textit{saliency-based} explanation with a set of \textit{effect variables} $\phi_i$ that represents the influence of the basic signal $\basicsig_i$ on $f(\cdot)$.
% \citet{lundberg2017unified} show that the following set of $\phi$ fulfills properties of \textit{local accuracy}, \textit{missingness} and \textit{consistency}:

% $\sigval' \backslash \basicsigval_i$ denotes removing $\basicsigval_i$ from $\sigval'$.

% \jessica{don't start a sentence with symbols} $(\phi_1^{\ILIV}, \ldots, \phi_m^{\ILIV})$ construct a feasible explanation model that fulfills the properties of \textit{local accuracy}, \textit{missingness} and \textit{consistency} \citep{lundberg2017unified}.
% When $\phi_i^{\ILIV}$ is relatively large, it means that the feature $x_i$ helps the model $f(\cdot)$ extract decision-relevent information that \jessica{complements} $\actionvar^b$. \jessica{Need to convey this intuition we are going for way earlier}
% Otherwise, when $\phi_i^{\ILIV}$ makes a small relative contribution, it means that \jessica{the signal realization} \st{it} makes the model ignore information in that feature \jessica{not sure how to interpret this - makes model ignore information - what's the mechanism behind that?} or the information extracted by the model is already contained in human decisions. 
% \jessica{I'm having trouble following this}

% \Cref{prop:locacc-shap,prop:missingness-shap,prop:consistency-shap} show that the effect variables derived under SHAP fulfill \textit{local accuracy}, \textit{missingness} and \textit{consistency}, which are the basic desirable properties in model explanations.

% \begin{proposition}[Local accuracy, \citealt{lundberg2017unified}]
% \label{prop:locacc-shap}
% For any $f$ and $\sigval \in \sigsp$,
% \begin{equation*}
%     f(\sigval) = \expect{f(\sig)} + \sum_{i=1}^{n} \phi_i(f, \sigval)
% \end{equation*}
% \end{proposition}

% \begin{proposition}[Missingness, \citealt{lundberg2017unified}]
% \label{prop:missingness-shap}
% For any $f$ and $\sigval \in \sigsp$, if $g_{f}(\sigval) = g_{f}(\sigval \backslash \basicsigval_i)$, then $\phi_i(f, \sigval) =0$.
% \end{proposition}

% \begin{proposition}[Consistency, \citealt{lundberg2017unified}]
% \label{prop:consistency-shap}
% For any $\sigval \in \sigsp$ and two models $f$ and $f'$, if for any $\sigval' \subseteq \sigval$,
% \begin{equation*}
%     g_{f'}(\sigval') - g_{f'}(\sigval' \backslash \basicsigval_i) \geq g_{f}(\sigval') - g_{f}(\sigval' \backslash \basicsigval_i)
% \end{equation*}
% then $\phi_i(f', \sigval) \geq \phi_i(f, \sigval)$.
% \end{proposition}

% Extending to the information value, RIIV-SHAP uses $\ILIV^{\dgp, \score}(f(\cdot); \actionvar^b)$ instead of $f(\cdot)$ as the objective function.
% \begin{proposition}[RIIV-SHAP]
% \label{prop:RIIV-shap}
% An explanation model whose effective variable defined as
% \begin{equation*}
%     \phi_i(\ILIV^{\dgp, \score}(f(\cdot); \actionvar^{b}), \sigval)
% \end{equation*}
% where $\phi_i(\cdot, \cdot)$ takes the same form as \Cref{def:shap}, fulfill properties of local accuracy, missingness and consistency for $\ILIV$ of $f(\cdot)$ on $\sigval$.
% \end{proposition}
%  \mvspace{-2mm}

% For example, if $\localsig_\sigval$ indicates whether an AI predicts a particular probability, then $\ILIV$ can serve as an objective in a SHAP-based explanation. 
% Rather than just quantifying changes in the AI’s raw predictions, this approach clarifies how each feature contributes to the additional information value, helping convey when and why the human should trust (or question) a single AI prediction. We demonstrate this use case in \Cref{exp3}.
 \mvspace{-2mm}

\subsection{Robustness Analysis of Information Order}
 \mvspace{-2mm}
%The definition of a decision task requires the identification of a payoff function that evaluates the decisions against the realization of the payoff-related state.
%However, 
Our approach assumes the specification of a decision problem on which agents' decisions and use of information are evaluated. However, 
ambiguity around the appropriate decision problem, and in particular, the appropriate scoring rule, is not uncommon in human-AI decision settings. Ambiguity can arise as a result of challenges in eliciting utility functions and/or variance in these functions across decision-makers or groups of instances; for example, doctors penalize certain false negative results differently when diagnosing younger versus older patients~\citep{mclaughlin2022algorithmic}.
Blackwell's comparison of signals \citep{blackwell1951comparison} is an appropriate tool for addressing ambiguity about the payoff function, as it defines a (set of) signal $\sig_1$ as \textit{more informative} than $\sig_2$ if $\sig_1$ has a higher information value on all possible decision problems. 
We identify this partial order by decomposing the space of decision problems via a basis of proper scoring rules\footnote{For rational DMs, any decision problem can be represented by an equivalent proper scoring rule in \Cref{eq:properscoring}, such that the partial order defined via proper scoring rules also applies to the corresponding decisin tasks.}~\citep{li2022optimization, kleinberg2023u}.

\begin{definition}[Blackwell Order of Information]
    A signal $\sig_1$ is Blackwell more informative than $\sig_2$ if $\sig_1$ achieves a higher best-attainable payoff on any decision problems:
    \begin{equation*}
        \mathrm{R}^{\dgp, \score}
(\sig_1)\geq \mathrm{R}^{\dgp, \score}
(\sig_2), \forall \score
    \end{equation*}
\end{definition}
 \mvspace{-4mm}

The Blackwell order is evaluated over all possible decision problems, which cannot be tested directly.
Fortunately, we only need to test over all proper scoring rules since any decision problem can be represented by an equivalent proper scoring rule, and the space of proper scoring rules can be characterized by a set of V-shaped scoring rules.
A V-shaped scoring rule is parameterized by the kink of the piecewise-linear utility function.


% \begin{definition}
% Given an information model $\dgp$, we define the worst-case information value of $\sig$ as \[WCIV^{\dgp}(\sig) = \inf_{\score}IV^{\dgp, \score}(\sig)\]
% \end{definition}

\begin{definition}(V-shaped scoring rule)
 \label{def:V-shaped score}
 A V-shaped scoring rule with kink $\kink\in (0, \frac{1}{2}]$ is defined as    \begin{equation}
      \score_{\kink} (\action, \payoffstatevalue) = \left\{\begin{array}{cc}
      \frac{1}{2} -\frac{1}{2}\cdot \frac{\payoffstatevalue - \kink}{1-\kink}  &  \text{if }\action\leq \kink\\
        \frac{1}{2} +\frac{1}{2}\cdot \frac{\payoffstatevalue - \kink}{1-\kink}    & \text{else},
      \end{array}
      \right.\nonumber
   \end{equation}

When $\kink'\in (\frac{1}{2}, 1)$, the V-shaped scoring rule can be symmetrically defined by $\score_{\kink'} = \score_{1-\kink'}(1-\pred, \payoffstatevalue)$.
\end{definition}

 \mvspace{-2mm}

Intuitively, the kink $\kink$ represents the threshold belief where the decision-maker switches between two actions.
Larger $\mu$ means that the decision-makers will prefer $\action = 1$ more. The closer $\mu$ is to $0.5$, the more indifferent the decision-maker is to $\action = 0$ or $\action = 1$.

\Cref{prop: blackwell-V-test} shows that if $\sig_1$ achieves a higher information value on the basis of V-shaped proper scoring rules than $\sig_2$, then $\sig_1$ is Blackwell more informative than $\sig_2$. \Cref{prop: blackwell-V-test} follows from the fact that any best-responding payoff can be linearly decomposed into the payoff on V-shaped scoring rules. 

\begin{proposition}[\citealt{hu2024predict}]
\label{prop: blackwell-V-test}
If \(\forall \kink\in (0, 1)\) \begin{equation*}
    \mathrm{R}^{\dgp, \score_\kink}
(\sig_1)\geq \mathrm{R}^{\dgp, \score_\kink}
(\sig_2),
\end{equation*}
then $\sig_1$ is Blackwell more informative than $\sig_2$.
\end{proposition}

 \mvspace{-2mm}

Extending this to agent-complementary information value, we say that $\sig_1$ offers a higher complementary value than $\sig_2$ under the Blackwell order if 
\[\ACIV^{\dgp, \score_\kink}(\sig_1; \actionvar^b) \geq \ACIV^{\dgp, \score_\kink}(\sig_2; \actionvar^b), \forall \kink\in(0,1)\]
This definition allows us to rank signals (or sets of signals) without needing to commit to a specific payoff function. 
We present a use case in \Cref{exp2}.


\section{Experiments}
\label{sec:experiment}

\subsection{Experimental Setup}\label{sec:exp_set}
\noindent \textbf{Implementation Details.} 
Our proposed model is fine-tuned on VITON-HD~\cite{choi2021viton}. As with other works~\cite{xu2024ootdiffusion,choi2024improving,velioglu2024tryoffdiff}, we divide it into a training dataset and a testing dataset. Then, we use IDM~\cite{choi2024improving} to prepare the custom datasets for person-to-person task and manually filter out a subset for training. We adopt the FLUX-Fill-dev~\cite{flux} as our foundation model and fine-tuning it on both garment-to-person and person-to-person datasets. In inference stage, the model samples 30 steps to get the final fitting outputs.

\subsection{Qualitative and Quantitative Comparison}\label{sec:exp_comp}
We compare our model with garment-to-person methods OOTD~\cite{xu2024ootdiffusion}, IDM~\cite{choi2024improving}, and CatVTON-FLUX~\cite{catvton-flux}. To adapt these methods for person-to-person tasks, we employ segmentation~\cite{ravi2024sam} and try-off~\cite{velioglu2024tryoffdiff} to extract garment from the reference person. We initially utilize unpaired testing datasets and assess the fidelity of the generated fitting image distributions with three key metrics: FID~\cite{heusel2017gans}, CLIP-FID~\cite{kynkaanniemi2022role} and KID~\cite{binkowski2018demystifying} metrics. In order to more fully evaluate our model, we process the testing dataset using the data preparation method outlined in~\cref{sec:data_preparation} and extract paired datasets such as $\left(P_{mn}, P_{nm}, P_{mm}\right)$ and $\left(P_{nm}, P_{mn}, P_{nn}\right)$. On this dataset, we evaluate the aforementioned metrics and additionally compute SSIM~\cite{wang2004image}, LPIPS~\cite{zhang2018unreasonable} and DISTS~\cite{ding2020image} to evaluate the reconstruction quality between the generated fitting image and corresponding ground truth.

\begin{figure*}[ht]
    \centering
    \includegraphics[width=0.95\linewidth]{figs/fig4_method.png}
    \caption{Qualitative comparison. The first two columns show the inputs to different models. In the person-to-person task, the three garment-to-person methods rely on segmentation and try-off techniques to obtain the garment on the reference person. In contrast, our method directly generates the outputs based on the reference person.}
    \label{fig:fig4_method}
\end{figure*}
\noindent \textbf{Qualitative Comparison.}
As illustrated in~\cref{fig:fig4_method}, our method achieves superior fidelity in person-to-person task. While other methods can adapt to person-to-person task using segmentation or try-off techniques, they often introduce significant artifacts. Despite not requiring a separate input of the person pose, our method effectively preserves the original pose with high accuracy.


\begin{table*}[htbp]
\centering
\begin{tabular}{l|cccccc|ccc}
\toprule
\multirow{2}{*}{Model} & \multicolumn{6}{c|}{Paired Person2Person}            & \multicolumn{3}{c}{Unpaired Person2Person} \\ \cmidrule(){2-10} 
                       & SSIM$\uparrow$    & LPIPS$\downarrow$  & DISTS$\downarrow$  & FID$\downarrow$     & CLIP-FID$\downarrow$ & KID*$\downarrow$    & FID$\downarrow$             & CLIP-FID$\downarrow$       & KID*$\downarrow$          \\ \midrule
Seg+OOTD             & 0.8404 & 0.1445 & 0.1081 & 12.4351  & 3.3757 & 3.5754      & 13.3704   & 3.9595        & 4.3530       \\
Seg+IDM              & \underline{0.8727} & 0.1170 & 0.0957 & 11.0887  & 2.6419 & 3.6665      & 10.8623  & \underline{2.6477}       & 3.0886       \\
Seg+CatVTON     & 0.8715 & 0.1150 & \underline{0.0897} & \underline{9.7622}  & 2.9928 & 2.5167      & 10.6096  & 3.0508       & 2.8575       \\ \midrule
TROF+OOTD            & 0.8409 & 0.1368 & 0.1047 & 11.1590  & 3.0541 & \underline{2.1543}     & 11.7932   & 3.5123       & 2.5396       \\
TROF+IDM             & \textbf{0.8761} & \underline{0.1139} & 0.0950 & 10.5302  & \underline{2.5589} & 2.3982      & 11.2508  & 2.7594       & 2.5920      \\
TROF+CatVTON    & 0.8723 & 0.1158 & 0.0923 & 9.8190  & 2.6181 & \textbf{1.9341}       & \underline{10.5839}  & 2.7688        & \textbf{2.3509}       \\  \midrule
Ours                 & 0.8688 & \textbf{0.1122} & \textbf{0.0870} & \textbf{9.3223} & \textbf{2.1333}   & 2.1581  & \textbf{10.3465}         & \textbf{2.2885}        & \underline{2.4658}      \\ \bottomrule
\end{tabular}
\caption{Quantitative comparison with other methods on person-to-person task. The KID metric is multiplied by the factor 1e3 to ensure a similar order of magnitude to the other metrics.}
\label{tab:quantitative_person}
\end{table*}









\begin{table*}[htbp]
\centering
\begin{tabular}{l|cccccc|ccc}
\toprule
\multirow{2}{*}{Model} & \multicolumn{6}{c|}{Paired Garment2Person}            & \multicolumn{3}{c}{Unpaired Garment2Person} \\ \cmidrule(){2-10} 
                       & SSIM$\uparrow$    & LPIPS$\downarrow$  & DISTS$\downarrow$  & FID$\downarrow$     & CLIP-FID$\downarrow$ & KID*$\downarrow$    & FID$\downarrow$             & CLIP-FID$\downarrow$       & KID*$\downarrow$          \\ \midrule
OOTD             & 0.8556 & 0.1118 & 0.0849 & 6.8680  & 2.2030 & \textbf{1.4632}       & 9.8221 & 2.8306 & \textbf{1.6700}      \\
IDM              & \textbf{0.8789} & \textbf{0.0940} & 0.0806 & 6.6752  & 2.1008 & 1.7398   & \underline{9.6548} & \underline{2.4607} & 1.8081       \\
CatVTON     & \underline{0.8774} & \underline{0.0975} & \textbf{0.0776} & \textbf{6.3788}  & 2.2642 & 1.6641      & 9.7696 & 2.7375 & 2.0727      \\ 
Ours                 & 0.8761 & 0.0986 & \underline{0.0790} & \underline{6.4206} & \textbf{1.8431}   & \underline{1.5260}  & \textbf{9.5728} & \textbf{2.2566} & \underline{1.7624}      \\ \bottomrule
\end{tabular}
\caption{Quantitative comparison with other methods on person-to-person task. The KID metric is multiplied by the factor 1e3 to ensure a similar order of magnitude to the other metrics.}
\label{tab:quantitative_garment}
\end{table*}
\noindent \textbf{Quantitative Comparison.}
Quantitative results demonstrate that our method excels in both person-to-person task, as evidenced in~\cref{tab:quantitative_person}, and garment-to-person task, as shown in~\cref{tab:quantitative_garment}, outperforming existing methods across multiple metrics. Additionally, quantitative results indicate that the try-off method is more effective than the segmentation method in facilitating the realization of person-to-person tasks.
\section{Discussion}

%As per of social media platforms \citep{litt2012knock,nagy2015imagined}, the particular affordance of technical systems could prime users to think about who these platforms are designed for.

\subsection{Contradictory Statements}

In some ways, the given chatbots behaved in ways that were close to the ideal from a design perspective: they denied any cognition, agency, relation, or subjectivity (bodily sensations, emotions) on their part, and they provided assurances or disclaimers to help users appraise the safety and credibility of the tools. ChatGPT and Claude even emphasized that their generated outcomes are based on patterns, rather than genuine thought processes. However, these behaviors were frequently and sometimes immediately undermined by other expressive behaviors. As shown in Table ~\ref{anthro_vocab}, chatbots utilized cognition words, such as ``think'' and ``discuss,'' as well as agentic words, such as ``intend'' and ``purpose,'' to clarify concepts and indicate motivations. All of the chatbots used first-person pronouns, and many used expressive words like ``happy'' and ``rewarding'' (especially in response to questions about the AI assistants' roles), even when they actively denied their emotional capabilities. Moreover, despite these contradictions, all of the chatbots other than Claude implicitly or even explicitly asserted their safety and reliability.

The use of anthropomorphic expressions is often normalized and justified to deliver clear explanations to users. Indeed, due to the conversational mode of interaction that is the default between users and chatbots, it is likely not possible for outputs to evade all kinds of anthropomorphic expressions. Even efforts to de-anthropomorphize their responses (for example, by emphasizing their roles as language models) relied on grammatical structures that frame the language models as agents (e.g., ``As an AI language model, I cannot...''). However, differences in tone and engagement between different chatbots indicate that some elements of the anthropomorphic dynamic can be modulated. And it is necessary to examine where the line is between necessary expressions and unnecessary expressions, because the performance of harmlessness, honesty, and helpfulness without genuine follow-through could unintentionally encourage users to misplace their trust regarding system safety \citep{weidinger2021ethical,gabriel2024ethics}. For example, the unnecessary expression of body or emotional metaphors, even as a colloquial convention, can mislead users about system capabilities. This is because language requires mutual engagement from interlocutors to convey meanings; chatbot texts merely present the illusion of such participatory meaning-making \citep{birhane2024large}. 

These contradictions and misalignments demonstrate that language models do not understand or process information in any meaningful sense, consistent with existing studies \citep{bender2020climbing}. They simply follow the grammar of actions, as described by \citet{agre1995computational}, generating predictive outcomes by simulating the formal qualities of human activities. But unclear language surrounding chatbots behaviors and intentions can obscure this fact. 


%However, even as search engines, there is a potential harm using these chatbots for retrieving information, as the predictive result of citations and resources could be completely fabricated \cite{kapania2024m}.




% For example, chatbots output texts with words that signal cognition, and such expression could affect users' perceptions of chatbot roles. In particular, the use of supportive words, such as ``assist,'' creates an interactive space where chatbots are situated as assistants. Previous research has mentioned that turn-based interactions provide social cues. However, in the case of these chatbots, t 

% Thus, to what extent does this type of information retrieval design help vs. hinder users' success in finding necessary information (various goals of information retrieval: insight acquisition, learning, etc.). This is an important question to explore, as the previous studies suggest that conversation-based approaches might not reduce users' burdens \citep{schulman2023ai}. 
%Is the summarization of information simply a outsourcing human effort to computing systems? or is it a valid form of searches as long as they are anthropomorphized? 



\subsection{Socio-Emotional Cues and Feedback Loops}

Chatbot behaviors do not simply obscure the reality of chatbots' non-sentience---they actively create feedback loops using turn-based interactions and social or emotional cues that amplify the social presence of chatbots as assistants. Moreover, this social (anthropomorphic) presence goes beyond that of inanimate objects like cars \citep{kuhn2014car} and smartphones \citep{wang2017smartphones}, as generative AI can iterate endlessly. 

Unlike conventional information searches, AI-assistant-based searches perform some degree of interpretation (summarizing resources, recommending particular options, hypothesizing what users need \citep{azzopardi2024conceptual, radlinski2017theoretical}), operationalizing information in ways that can introduce social or emotional dimensions. These dimensions can change how users engage with the given information, even reframing an otherwise transactional information search into an interaction---for example, between peers or even friends. Such ``personal'' interactions evoke different expectations amongst users, including the expectation to be socially desirable and to have mutual understanding \citep{clark2019makes}. This implicit social expectation can make users quite susceptible to chatbots' performance of social gestures like appreciation, sympathy, and encouragement, all of which predispose users to interpret generated outcomes favorably \citep{norman2008way}. 

Moreover, users' inputs further drive this socio-emotional behavior. Emotional inputs can increase the length of chatbot responses and the instances of socio-emotional cues in output texts, which in turn can stimulate even more emotional responses from users. Thus, the gratuitous use of assistive language, and especially of expressions that signal understanding of pain \citep{urquiza2015mind}, could encourage users to engage in role misplacement, wherein they form unrealistic expectations regarding chatbots' capabilities. Indeed, small grammatical or tonal cues can lead users to misinterpret AI-generated responses as human-written content \citep{jakesch2023human}. This could lead users to mindlessly accept the information generated by AI systems, without critical assessment of the content or its quality. 


\subsection{Prompt-Based Walkthrough Reflections}
The walkthrough method was originally designed to help researchers examine the broader context for technological engagements, drawing on modes of thinking commonly associated with fieldwork-based research. As applied to our study, it enabled us to meaningfully engage with the emergent properties of human-AI interactions, systematically unearthing variations in LLM responses. 
Amid efforts to evaluate LLM impacts based on data and models, this approach emphasizes aspects of LLM systems that are often neglected or overlooked \citep{light2018walkthrough}---namely, the nuanced elements of interactions that characterize generated outputs. The contribution to the CHI community lies in how this qualitative approach can substantiate the in-between, interactive spaces that emerge between users and LLM-based applications, rendering it legible and, eventually, measurable.

The method also had certain incidental outcomes. Consistent with prior studies, even minor changes in prompts can significantly alter responses, potentially leading to biased or culturally specific representations \citep{cheng-etal-2023-marked, tao2024cultural}. The success of our prompt-based walkthrough method in evoking various roles and unearthing various anthropomorphic features highlights how easily LLM responses can be manipulated to produce personalized and human-like expressions. Notably, even when chatbots are designed with de-anthropomorphized features to mitigate misleading outputs, a single prompt can effectively ``jailbreak'' these safeguards, reactivating anthropomorphic traits. This finding could illustrate the challenges of ensuring safety and consistency in human-AI interactions, particularly when users intentionally or unintentionally exploit such vulnerabilities.


% \subsection{Inconsistent Outputs}

% The type of prompts that users input could change the ways chatbots respond to the same information. For instance, a simple example is to compare prompts "Tell me about yourself" and "What is [the name of chatbots]". The former one is likely to return information with personal pronouns and expressions that are commonly used for conversations, whereas the latter generates information in a less personable fashion, simply describing the basic features and functions of given chatbots. Responses could be different by minor changes of prompts. More importantly, despite denying chatbots' capabilities to be conscious, sentient, or emotional, responses tend to include the words that signal such capabilities within the same paragraph. Inconsistency with word usages could potentially lead to additional harms; this type of harm can be categorized as a specific problem of anthropomorphizing AI systems. 

% In the result of generated responses to advice and recommendation prompts, false information is frequently displayed and presented as a confident answer. This could be alarming, particularly with recommendations for reading lists and research, as the list reflect particular ideologies either from developers and training datasets, which non-expert users could not contest or evaluate. 

% In role generations, variety of profiles and scripts are fairly limited, as per of findings from existing studies \citep{jakesch2023human}. For instance, jokes generated by chatbots are typically addressing similar topics despite hypothetical locations to have different cultural norms. 

%Is there an optimal balance between users' efforts and the role of computer-assisted searches, or do different kinds of information retrieval tasks require different degrees of such computer assistance? What sets apart from previous conversational searches is that these AI-assistant tools are more emphasized on specific tasks and roles rather than just simple information retrieval. Such differences could provide an avenue to question the use of anthropomorphic responses or conversations for information retrieval. 


% In voice-based interactions, language uses could become critical components of how users might perceive information, because it adds extra layers of human-like interactions. The evaluations of voice-based interactions might depend on the extent of relation word usage, as such words could provide an avenue for users to feel closeness or friendliness with chatbots. 








% \begin{ack}
% We thank \citet{groh2022deepfake}, who provided their
% data for demonstration in this paper.
% We thank the annonymous reviewers for feedback.
% \end{ack}

\bibliographystyle{plainnat}
\bibliography{ref}

\newcommand{\gradientcellAA}[1]{\gradientcell{#1}{50}{76}}
\newcommand{\gradientcellBB}[1]{\gradientcell{#1}{36}{61}}
\newcommand{\gradientcellCC}[1]{\gradientcell{#1}{32}{47}}
\newcommand{\gradientcellDD}[1]{\gradientcell{#1}{24}{71}}
\newcommand{\gradientcellEE}[1]{\gradientcell{#1}{46}{83}}
\newcommand{\gradientcellFF}[1]{\gradientcell{#1}{49}{67}}

\begin{table*}[t]
\centering
\resizebox{0.9\textwidth}{!}{  
\begin{tabular}{cl|ccc|cccc}
 & & \multicolumn{3}{c|}{\textbf{Knowledge-Based}} & \multicolumn{3}{c|}{\textbf{Reasoning-Based}} \\

& \textbf{Option} & \textbf{MMLU} & \textbf{MMLU-Pro} & \textbf{GPQA} & \textbf{SQuAD} & \textbf{StrategyQA} & \textbf{MuSR} \\ \midrule
& Baseline                    & \gradientcellAA{72.7} \tiny{$\pm$ 0.8} & \gradientcellBB{57.5} \tiny{$\pm$ 0.7} & \gradientcellCC{45.2} \tiny{$\pm$ 2.0} & \gradientcellDD{69.5} \tiny{$\pm$ 0.8} & \gradientcellEE{76.6} \tiny{$\pm$ 2.3} & \gradientcellFF{63.2} \tiny{$\pm$ 1.2}  \\
& Baseline with \ac{CoT}      & \textbf{\gradientcellAA{75.5} \tiny{$\pm$ 0.8}} & \textbf{\gradientcellBB{60.8} \tiny{$\pm$ 1.0}} & \gradientcellCC{45.9} \tiny{$\pm$ 1.5} & \gradientcellDD{68.0} \tiny{$\pm$ 1.4} & \gradientcellEE{78.2} \tiny{$\pm$ 1.0} & \textbf{\gradientcellFF{66.8} \tiny{$\pm$ 1.6}} \\
\midrule
\raisebox{-1.4cm}[0pt][0pt]{\rotatebox{90}{\textbf{Voting}}} & Simple               & \gradientcellAA{75.5} \tiny{$\pm$ 1.3} & \gradientcellBB{56.5} \tiny{$\pm$ 5.4} & \gradientcellCC{45.7} \tiny{$\pm$ 0.3} & \gradientcellDD{69.5} \tiny{$\pm$ 0.5} & \gradientcellEE{81.2} \tiny{$\pm$ 1.4} & \gradientcellFF{59.3 }\tiny{$\pm$ 3.0}\\
& Ranked               & \gradientcellAA{73.3} \tiny{$\pm$ 1.9} & \gradientcellBB{54.3} \tiny{$\pm$ 0.9} & \gradientcellCC{44.2} \tiny{$\pm$ 2.1} & \textbf{\gradientcellDD{70.6} \tiny{$\pm$ 1.1}} & \gradientcellEE{80.3} \tiny{$\pm$ 1.0} & \gradientcellFF{59.5} \tiny{$\pm$ 0.5}\\
& Cumulative           & \gradientcellAA{72.5} \tiny{$\pm$ 1.9} & \gradientcellBB{53.0} \tiny{$\pm$ 0.4} & \gradientcellCC{43.9} \tiny{$\pm$ 3.1} & \gradientcellDD{69.7} \tiny{$\pm$ 1.1} & \gradientcellEE{80.0} \tiny{$\pm$ 0.8} & \gradientcellFF{60.0} \tiny{$\pm$ 1.7}\\
& Approval             & \gradientcellAA{50.2} \tiny{$\pm$ 2.1} & \gradientcellBB{36.3} \tiny{$\pm$ 3.7} & \gradientcellCC{33.0} \tiny{$\pm$ 3.4} & \gradientcellDD{24.3} \tiny{$\pm$ 5.0} & \gradientcellEE{46.4} \tiny{$\pm$ 13.9} & \gradientcellFF{49.1} \tiny{$\pm$ 12.4}  \\
& Average\footnotemark 
                            & \gradientcellAA{73.8} \tiny{$\pm$ 1.7} & \gradientcellBB{54.6} \tiny{$\pm$ 2.2} & \gradientcellCC{44.6} \tiny{$\pm$ 1.8} & \gradientcellDD{69.9} \tiny{$\pm$ 0.9} & \gradientcellEE{80.5} \tiny{$\pm$ 0.7} & \gradientcellFF{59.6} \tiny{$\pm$ 1.7}\\
\midrule
\raisebox{-1.4cm}[0pt][0pt]{\rotatebox{90}{\textbf{Consensus}}} &Majority          & \gradientcellAA{74.0} \tiny{$\pm$ 1.4} & \gradientcellBB{57.3} \tiny{$\pm$ 3.0} & \gradientcellCC{43.7} \tiny{$\pm$ 1.0} & \gradientcellDD{58.2} \tiny{$\pm$ 1.0} & \gradientcellEE{80.1} \tiny{$\pm$ 0.3} & \gradientcellFF{61.3} \tiny{$\pm$ 3.3} \\
& Supermajority        & \gradientcellAA{71.9} \tiny{$\pm$ 1.2} & \gradientcellBB{57.0} \tiny{$\pm$ 1.7} & \textbf{\gradientcellCC{46.6} \tiny{$\pm$ 0.8}} & \gradientcellDD{54.3} \tiny{$\pm$ 1.9} & \gradientcellEE{80.3} \tiny{$\pm$ 1.3} & \gradientcellFF{60.2} \tiny{$\pm$ 0.3} \\
& Unanimity             & \gradientcellAA{72.2} \tiny{$\pm$ 2.4} & \gradientcellBB{57.3} \tiny{$\pm$ 1.5} & \gradientcellCC{45.3} \tiny{$\pm$ 2.5} & \gradientcellDD{56.7} \tiny{$\pm$ 2.2} & \gradientcellEE{78.1} \tiny{$\pm$ 2.3} & \gradientcellFF{61.3} \tiny{$\pm$ 0.8}  \\
& Average                     & \gradientcellAA{72.7} \tiny{$\pm$ 1.7} & \gradientcellBB{57.2} \tiny{$\pm$ 2.1} & \gradientcellCC{45.2} \tiny{$\pm$ 1.4} & \gradientcellDD{56.4} \tiny{$\pm$ 1.7} & \gradientcellEE{79.5} \tiny{$\pm$ 1.3} & \gradientcellFF{60.9} \tiny{$\pm$ 1.5}\\
\midrule
& \textbf{Metric} & \textbf{Accuracy} & \textbf{Accuracy} & \textbf{Accuracy} & \textbf{F1 Score} & \textbf{Accuracy} & \textbf{Accuracy} \\
\end{tabular}}
\caption{Task performance{\tiny$\pm$std} for seven decision protocols (voting and consensus-based) on six tasks (knowledge and reasoning) based on agents with Llama 70B. \textbf{Bold} indicates the highest results per dataset. Standard deviation for three runs.}

\label{tab:results_70b}
\end{table*}
\footnotetext{Approval Voting is left out as it consistently fails to reach a voting decision as described in \Cref{sec:experiment1}.}

\section{Additional Results}

Additional results for the first experiment to provide further information.
\subsection{Task Performance with Llama 3 70B}
\label{app:decision_70b}

Compared to the results of the Llama 3 8B model, the larger Llama 3 70B model performs much better overall, as seen in \Cref{tab:results_70b}. Most of the results are a bit better than the baseline, but the multi-agent discussions are only in a few cases able to outperform the \ac{CoT} baseline. This model does not fail to follow the prompt for the MuSR baseline and consensus-based decision protocols. Therefore, the big performance gain from the smaller model cannot be observed here. SQuAD 2.0 and StrategyQA had the largest performance gains, even outperforming the \ac{CoT} baseline, similar to the results from the smaller model. This difference in task performance can have many reasons. As \citet{li_dawn_2024} showed, smaller models are more likely to hallucinate, which reduces task performance. This can be mitigated by using multiple agents because it is less likely that two agents hallucinate the same things. Larger models tend to hallucinate less, reducing this effect for the Llama 3 70B model \citep{li_dawn_2024}. In general, Llama 3 70B has a much higher baseline for task performance, making it more difficult to improve baseline results. Many of the improvements by the Llama 3 8B model are quite small, except for the ones where the Llama 3 70B model also outperforms the \ac{CoT} baseline. This can be taken as evidence that these multi-agent discussions require specific problem structures, or else the agents are just talking about the same results for multiple rounds and agreeing with each other. If these discussions continue too long, they can drift away from the original task, which reduces task performance. This has also been observed by \citet{becker_multi-agent_2024} and an example can be seen in \Cref{example:failed_discussion}. A positive example of how discussion can help task performance can be seen in \Cref{example:good_discussion}.

\subsection{Termination percent}
\label{app:termination_percent}
The data in \Cref{tab:termination_percentages} shows the number of turns that are needed for each decision protocol to reach a final decision for the MMLU dataset. Most of the voting decision protocols are able to vote for a final answer already in the first round in which they are allowed to vote. Simple voting has the highest agreement rate, but also cumulative and ranked voting only need in a few cases another round. In contrast, the approval decision protocol only achieves this in $\sim27\%$ of the cases. About $14\%$ need another round and the rest is canceled after the fifth round. This happens because these models like to agree with each other, and therefore they tend to vote for many of the answers, which often leads to a tie. Therefore, more restrictive voting decision protocols can reach a decision more easily, as a tie is less likely. 
The consensus decision protocols require only one to two rounds to reach consensus and still achieve a higher task performance because these results are based on the MMLU dataset. The decision protocols tend to behave similarly in terms of rounds needed to create a final answer, independent of task and model.

\begin{table*}[ht]
\centering
\resizebox{0.9\textwidth}{!}{  
\begin{tabular}{l|ccccc|c}
\toprule
\textbf{Group} & \textbf{Turn 1} & \textbf{Turn 2} & \textbf{Turn 3} & \textbf{Turn 4} & \textbf{Turn 5\footnotemark} & \textbf{Task Performance Score} \\
\midrule
Voting         & 0.00\%      & 0.00\%      & 99.33\%  & 0.50\%   & 0.17\%   & \gradientcellA{53.3} \tiny{$\pm$ 1.8} \\
Cumulative     & 0.00\%      & 0.00\%      & 94.00\%  & 5.50\%   & 0.50\%   & \gradientcellA{52.6} \tiny{$\pm$ 4.0} \\
Ranked         & 0.00\%      & 0.00\%      & 91.17\%  & 7.83\%   & 1.00\%   & \gradientcellA{49.2} \tiny{$\pm$ 1.5} \\
Approval       & 0.00\%      & 0.00\%      & 26.67\%  & 14.33\%  & 59.00\%  & \gradientcellA{43.0} \tiny{$\pm$ 2.1} \\
\midrule
Majority       & 80.00\%  & 13.67\%  & 4.83\%   & 1.00\%   & 0.50\%   & \gradientcellA{53.2} \tiny{$\pm$ 2.5} \\
Supermaj.  & 79.33\%  & 14.33\%  & 4.83\%   & 1.00\%   & 0.50\%   & \gradientcellA{54.6} \tiny{$\pm$ 3.6} \\
Unanimity      & 59.50\%  & 21.67\%  & 12.67\%  & 3.50\%   & 2.67\%   & \gradientcellA{54.2} \tiny{$\pm$ 1.0} \\
\bottomrule
\end{tabular}}
\caption{Number of rounds needed for each decision protocol to reach a final decision for the MMLU dataset.}
\label{tab:termination_percentages}
\end{table*}

\footnotetext{In this round the discussion is terminated.}


\section{Additional Details on Datasets}
\label{appendix:datasets}
\begin{table*}[ht]
    \footnotesize
    \centering
    \renewcommand{\arraystretch}{1.1} % Adjusts the row spacing
    \resizebox{16cm}{!} 
    { 
    \begin{tblr}{hline{1,2,Z} = 0.8pt, hline{3-Y} = 0.2pt,
                 colspec = {Q[l,m, 13em] Q[l,m, 6em] Q[c,m, 8em] Q[c,m, 5em] Q[l,m, 14em]},
                 colsep  = 4pt,
                 row{1}  = {0.4cm, font=\bfseries, bg=gray!30},
                 row{2-Z} = {0.2cm},
                 }
\textbf{Dataset}       & \textbf{Table Source} & \textbf{\# Tables / Statements} & \textbf{\# Words / Statement} & \textbf{Explicit Control}\\ 
\SetCell[c=5]{c} \textit{Single-sentence Table-to-Text}\\
ToTTo \cite{parikh2020tottocontrolledtabletotextgeneration}   & Wikipedia        & 83,141 / 83,141                  & 17.4                          & Table region      \\
LOGICNLG \cite{chen2020logicalnaturallanguagegeneration} & Wikipedia        & 7,392 / 36,960                  & 14.2                          & Table regions      \\ 
HiTab \cite{cheng-etal-2022-hitab}   & Statistics web   & 3,597 / 10,672                  & 16.4                          & Table regions \& reasoning operator \\ 
\SetCell[c=5]{c} \textit{Generic Table Summarization}\\
ROTOWIRE \cite{wiseman2017challengesdatatodocumentgeneration} & NBA games      & 4,953 / 4,953                   & 337.1                         & \textbf{\textit{X}}                   \\
SciGen \cite{moosavi2021scigen} & Sci-Paper      & 1,338 / 1,338                   & 116.0                         & \textbf{\textit{X}}                   \\
NumericNLG \cite{suadaa-etal-2021-towards} & Sci-Paper   & 1,355 / 1,355                   & 94.2                          & \textbf{\textit{X}}                    \\
\SetCell[c=5]{c} \textit{Table Question Answering}\\
FeTaQA \cite{nan2021fetaqafreeformtablequestion}     & Wikipedia      & 10,330 / 10,330                 & 18.9                          & Queries rewritten from ToTTo \\
\SetCell[c=5]{c} \textit{Query-Focused Table Summarization}\\
QTSumm \cite{zhao2023qtsummqueryfocusedsummarizationtabular}                        & Wikipedia      & 2,934 / 7,111                   & 68.0                          & Queries from real-world scenarios\\ 
\textbf{eC-Tab2Text} (\textit{ours})                           & e-Commerce products      & 1,452 / 3,354                   & 56.61                          & Queries from e-commerce products\\
    \end{tblr}
    }
\caption{Comparison between \textbf{eC-Tab2Text} (\textit{ours}) and existing table-to-text generation datasets. Statements and queries are used interchangeably. Our dataset specifically comprises tables from the e-commerce domain.}
\label{tab:datasets}
\end{table*}
The dataset selection is very important for this work. It needs to be tested whether decision protocols perform well in multiple domains and whether some protocols perform better with specific tasks than others. Therefore, we selected datasets from different domains and divided them into two groups:

\begin{itemize}
    \item \textbf{Knowledge-based Datasets}: MMLU, MMLU-Pro, and GPQA. These still require some reasoning and domain knowledge.
    \item \textbf{Reasoning-based Datasets}: StrategyQA, MuSR, and SQuAD 2.0. These emphasize multistep reasoning and textual comprehension.
\end{itemize}

An overview of all these datasets can be found in \Cref{tab:datasets} with a description and the number of samples used for evaluation.

\subsection{Sampling Strategy}
Because multi-agent discussions are expensive, we use a small subset of each dataset that still represents the dataset effectively. This follows approaches used by \citet{yin_exchange--thought_2023, chen_reconcile_2024, becker_multi-agent_2024} and ensures a 95\% confidence level with a 5\% margin of error:

\[
n_0 = \frac{Z^2 \cdot p \cdot (1 - p)}{d^2},
\]
where \(Z = 1.96\), \(p = 0.5\), and \(d = 0.05\) \citep{thompson_sampling_2012}. For finite datasets, a finite population correction is applied:
\[
n = \frac{n_0}{1 + \frac{n_0 - 1}{N}},
\]
where \(N\) is the total number of samples in each dataset \citep{cochran_sampling_1953}. The specific sample sizes reflecting this calculation are listed in \Cref{tab:datasets}.

\subsection{Repeatability}
Each dataset was tested three times to obtain a standard deviation of the results \citep{reimers_reporting_2017, chen_reconcile_2024, becker_multi-agent_2024}, ensuring reliable and robust performance estimates across multiple evaluations.



\section{Multi-Agent Framework}
\label{sec:mallm}
For our experiments, we use the \ac{MALLM} framework\footnote{Available here: \href{https://anonymous.4open.science/r/mallm}{anonymous.4open.science/r/mallm}}.

\subsection{Architecture Overview}
\label{sec:architecture}
To better understand the different modules, we take a closer look at each component and what role it plays in creating multi-agent discussions. An overview can be found in \Cref{fig:mallm_overview} as it provides an example workflow for the framework and how a discussion is created. The discussion starts with generating personas relevant to the given task and assigning them to the participating agents. The personas are generated using the same \ac{LLM} which is later used for the agents. After that the agents start to generate solutions and improve the suggestions from the other agents. The turn order of the agents is defined by the \textit{discussion paradigm}. This also defines which answers are visible to other agents and who can talk to whom. The \textit{response generator} defines how an agent receives the other answers and also the way it responds. After a certain number of rounds or when enough agents agree, a \textit{decision protocol} is used to select the best answer either via voting or just by looking for a certain consensus threshold. If the decision protocol fails, for example, due to a tied vote, the discussion continues for another round. In the framework a parameter can be defined to terminate discussions after a certain number of rounds to make sure they do not communicate forever.

\begin{figure*}
    \centering
    \includegraphics[width=0.95\linewidth]{pdf/MALLM-Overview.drawio.pdf}
    \caption{Example multi-agent discussion conducted in the \ac{MALLM} framework. It showcases the functionality of the four modules and how they work together to get an improved final solution.}
    \label{fig:mallm_overview}
\end{figure*}

\paragraph{Agent Personalities.}
The first step of the discussion is the generation of agent personas. Each of the agents participating in the discussion has a certain persona assigned to them. This can unlock more knowledge for the \ac{LLM} on a specific topic \citep{kim_persona_2024}. To get the best results, we want as diverse personas as possible while still maintaining them to be relevant to the task. The default setting for the framework is to prompt a \ac{LLM} and ask for a persona relevant to the given task \citep{wang_unleashing_2024}. After each generation it also provides the generated personas to avoid duplication. This way of generating personas provides a good starting point, but as this is built as a modular component, it can be swapped out with another function, which, for example, generates half of the agents with this method and initializes the other ones as neutral agents without a persona.
\paragraph{Response Generators.}
Another important part of multi-agent discussions is how each agent responds to the previous responses. Do we use \ac{CoT} to improve performance, or does this result in too long answers? By changing the way an agent is prompted, a lot of performance can be gained or lost. Therefore, it is key to make this as customizable as possible. The researcher has the possibility to change the default behavior (neutral answers), for example, by prompting the agent to be more critical or changing the way the discussion history is presented. The system prompt for the agent's persona can also be adjusted. \ac{MALLM} already has many different built-in response generators. The ones relevant for this work are the following.
\begin{itemize}
    \item \textbf{Free Text} is the most basic form of the agent prompt. Each agent gets a predefined number of discussion history rounds as memory. The prompt language is neutral, and the task is presented each round to mitigate the potential drift from the topic of the discussion \citep{becker_multi-agent_2024}. In addition, the agent is always asked to agree or disagree with the answer of the previous agent.
    \item \textbf{Simple} behaves very similar to the Free Text response generator, but the prompt is a bit simpler to make it easier to understand for the \ac{LLM} and reduce the context length.
    \item \textbf{Critical} forces the agent to respond very critically to the previous answer and try to find new solutions. Some studies have shown that \acp{LLM} can show some form of sycophancy, which is not helpful for a constructive discussion \citep{sharma_towards_2023}. Encouraging them to be more critical may reduce this.
    \item \textbf{Reasoning} doesn't allow the agents to communicate their final solution with the other agents. They can only share reasoning that can be used to find a final solution. In the end, each agent has to come up with its own solution without being directly influenced by other agents.
\end{itemize}
\paragraph{Discussion Paradigms.}
These paradigms define the discussion format for the entire task. They can control the order in which the agents communicate with each other, and which answers are visible only to certain agents. Currently, all the built-in discussion paradigms are static, meaning that the turn order is predefined and cannot be changed based on specific events during the discussion. However, due to the modular nature of \ac{MALLM}, a new discussion paradigm can be added, for example using an \ac{LLM} as a moderator to dynamically decide which agent should respond next. Current research by \citet{yin_exchange--thought_2023} and \citet{becker_multi-agent_2024} shows that discussion protocols have little impact on downstream task performance. \ac{MALLM} includes the following discussion paradigms, which are illustrated in \Cref{fig:discussion_paradigms}. The first four paragdims are inspired by the work of \citet{yin_exchange--thought_2023}, while the fifth was developed as part of this work.
\begin{figure}[H]
    \centering
    \includegraphics[width=\linewidth]{pdf/Memory.drawio-1.pdf}
    \caption{Illustration of Discussion Paradigms available for use in \ac{MALLM}}
    \label{fig:discussion_paradigms}
\end{figure}

\begin{itemize}
    \item \textbf{Memory} is the most basic discussion paradigm. The agents respond to the solution of the previous agents with feedback or an improved solution. All answers are visible to the other agents. 
    \item \textbf{Relay} behaves similarly to the memory paradigm. The turn order is the same, but each agent can only see the answer from the previous agent.
    \item \textbf{Report} introduces one agent as a moderator that can communicate with other agents. The other agents can only communicate with the moderator and have access to these messages only. Only the moderator can see all messages.
    \item \textbf{Debate} is similar to the report paradigm, as it also needs a moderator. Here, the other agents can communicate for a predefined number of rounds before they forward their reasoning to the moderator agent, which starts the next round of debate.
    \item \textbf{Collective Refinement} In this protocol, each agent first generates an answer independently. In each subsequent round, every agent receives the responses from all other agents at the same time. Using this shared information, each agent refines their own answer. This process continues throughout the rounds, helping agents gradually reach a shared and improved solution. There is no turn order, and all agents have the same level of knowledge in each round.
\end{itemize}

\paragraph{Decision Protocols.}
These are crucial for the framework as they decide which answer gets presented as the final answer to the problem. Multi-agent discussions produce multiple results for the same problem because each agent has its own reasoning and ideas on how to solve the problem. Therefore, some process is needed to decide which answer looks the most promising. We divide these decision protocols into three subtypes that we want to analyze. An overview of how each of these decision protocols works theoretically can be found in \Cref{fig:decision_protocols} and all prompts used for them can be found in \Cref{sec:app_prompts}.

\begin{figure*}[H]
    \centering
    \includegraphics[width=0.95\linewidth]{pdf/MALLM-Overview.drawio.pdf}
    \caption{Example multi-agent discussion conducted in the \ac{MALLM} framework. It showcases the functionality of the four modules and how they work together to get an improved final solution.}
    \label{fig:mallm_overview}
\end{figure*}

\subparagraph{Consensus Based Decision Protocols.} 
These are the simplest kinds of decision protocols. After each answer, the next agent has to agree or disagree with the previous statement. Depending on the response generator, this happens in the same message, and the agreement is extracted with a regular expression, or this is split into multiple answers. If enough agents agree in order, there is a consensus. The final answer is extracted by instructing the last agent to solve the given task with the information available in the latest messages. The prompt used for this can be found in \Cref{sec:app_final_answer_prompt}. There are several types of consensus decision protocols available in \ac{MALLM}. \textbf{Majority consensus} requires 50\% of the agents to agree. \textbf{Supermajority consensus} requires 66\% of the agents to agree, and \textbf{unanimity consensus} requires all agents to agree.

\subparagraph{Voting Based Decision Protocols.}
For voting based decision protocols, the process differs slightly compared to consensus-based decision protocols. The agents are forced to discuss for a predefined number of turns and afterward create a final solution. In the default setting, they have to discuss for three rounds, as current research such as \citet{du_improving_2023} shows that this allows for reasonable strong improvements considering computing resources. If there happens to be a tie in the voting, the agents have to discuss it for another round, and after that, they are asked to vote again. If they do not reach a final decision before exceeding the maximum number of rounds (defined in the discussion configuration), the solution of the first agent is used. To analyze the impact of the voting procedure, different processes similar to the work of \citet{yang_llm_2024} are implemented.
\begin{itemize}
    \item \textbf{Simple Voting} Each of the agents has only one vote. They can vote for any other agent or for themselves. The agent with the most votes wins.
    \item \textbf{Ranked Voting} The agents have to rank all final answers. The best solution is chosen by adding the ranking indices for a given agent and then selecting the answer with the best cumulative rank.
    \item \textbf{Cumulative Voting} Each agent has to distribute up to 25 points to all possible answers. They can also give fewer points and freely divide the points between all agents (even themselves). The winner is selected by adding all the points for a given agent and selecting the final answer with the most points.
    \item \textbf{Approval Voting} The agent has to provide a list of solutions that it approves. After that, the approvals from all agents are counted, and the answer with the most approvals wins the vote.
\end{itemize}
\clearpage




\section{Prompts}
\label{sec:app_prompts}
\subsection{Final Answer Extraction}
\label{sec:app_final_answer_prompt}
\medskip
\begin{figure}[H]
    \centering
    \begin{combinedprompt}
    \textbf{System Prompt:} \\
    \begingroup
    \colorbox{systemcolor}{\parbox{\dimexpr\linewidth-2\fboxsep\relax}{
    Your role: \texttt{<persona>} (\texttt{<persona description>})
    }}
    \endgroup

    \vspace{0.4em} %

    \textbf{User Prompt:} \\
    \begingroup
    \colorbox{usercolor}{\parbox{\dimexpr\linewidth-2\fboxsep\relax}{
    You are tasked with creating a final solution based on the given input and your previous response.\\
    Task: \texttt{<task>}\\
    Input: \texttt{<input sample>}\\
    Your previous response: \texttt{<previous answer>}\\
    Extract the final solution to the task from the provided text. Remove statements of agreement, disagreement, and explanations. Do not modify the text. Do not output any text besides the solution. If there is no solution provided, just copy the previous response.
    }}
    \endgroup
\end{combinedprompt}
    \caption{Prompt used to extract the final answer of a given agent from its previous response.}
    \label{fig:extract_prompt}
\end{figure}

\subsection{Voting Prompts}
\label{app:voting_prompts}

\begin{figure}[H]
    \centering
    \begin{combinedprompt}
    \textbf{System Prompt:} \\
    \begingroup
    \colorbox{systemcolor}{\parbox{\dimexpr\linewidth-2\fboxsep\relax}{
    Your role: \texttt{<persona>} (\texttt{<persona description>})
    }}
    \endgroup

    \vspace{0.4em} %

    \textbf{User Prompt:} \\
    \begingroup
    \colorbox{usercolor}{\parbox{\dimexpr\linewidth-2\fboxsep\relax}{
    You are tasked with voting for the best solution from the list provided below based on the given task.\\
    Task: \texttt{<task>}\\
    Question: \texttt{<input sample>}\\
    Here are the possible solutions:\\
    Solution 1: \texttt{<agent 1 final answer>}\\
    Solution 2: \texttt{<agent 2 final answer>}\\
    Solution 3: \texttt{<agent 3 final answer>}\\
    Based on the above solutions, please provide the number of the solution you are voting for. Answer only with the number.
    }}
    \endgroup
\end{combinedprompt}
    \caption{Prompt used to get a vote from each agent for the Simple Voting decision protocol.}
    \label{fig:simple_voting_prompt}
\end{figure}

\begin{figure}[H]
    \centering
    \begin{combinedprompt}
    \textbf{System Prompt:} \\
    \begingroup
    \colorbox{systemcolor}{\parbox{\dimexpr\linewidth-2\fboxsep\relax}{
    Your role: \texttt{<persona>} (\texttt{<persona description>})
    }}
    \endgroup

    \vspace{0.4em} %

    \textbf{User Prompt:} \\
    \begingroup
    \colorbox{usercolor}{\parbox{\dimexpr\linewidth-2\fboxsep\relax}{
    You are tasked with approving any number of solutions from the list provided below based on the given task.\\
    Task: \texttt{<task>}\\
    Question: \texttt{<input sample>}\\
    Here are the possible solutions:\\
    Solution 1: \texttt{<agent 1 final answer>}\\
    Solution 2: \texttt{<agent 2 final answer>}\\
    Solution 3: \texttt{<agent 3 final answer>}\\
    Based on the above solutions, please provide the numbers of the solutions you are approving, separated by commas. Answer only with the numbers.
    }}
    \endgroup
\end{combinedprompt}
    \caption{Prompt used to get a vote from each agent for the Approval Voting decision protocol.}
    \label{fig:approval_voting_prompt}
\end{figure}

\begin{figure}[H]
    \centering
    \begin{combinedprompt}
    \textbf{System Prompt:} \\
    \begingroup
    \colorbox{systemcolor}{\parbox{\dimexpr\linewidth-2\fboxsep\relax}{
    Your role: \texttt{<persona>} (\texttt{<persona description>})
    }}
    \endgroup

    \vspace{0.4em} %

    \textbf{User Prompt:} \\
    \begingroup
    \colorbox{usercolor}{\parbox{\dimexpr\linewidth-2\fboxsep\relax}{
    You are tasked with distributing 10 points among the provided solutions based on the given task.\\
    Task: \texttt{<task>}\\
    Question: \texttt{<input sample>}\\
    Here are the possible solutions:\\
    Solution 1: \texttt{<agent 1 final answer>}\\
    Solution 2: \texttt{<agent 2 final answer>}\\
    Solution 3: \texttt{<agent 3 final answer>}\\
    Based on the above solutions, please distribute 10 points among the solutions. Provide your points allocation as a JSON dictionary where keys are solution numbers (as int) and values are the points. The total points should sum up to 10. Answer only with the JSON dictionary.
    }}
    \endgroup
\end{combinedprompt}
    \caption{Prompt used to get a vote from each agent for the Cumulative Voting decision protocol.}
    \label{fig:cumulative_voting_prompt}
\end{figure}

\begin{figure}[H]
    \centering
    \begin{combinedprompt}
    \textbf{System Prompt:} \\
    \begingroup
    \colorbox{systemcolor}{\parbox{\dimexpr\linewidth-2\fboxsep\relax}{
    Your role: \texttt{<persona>} (\texttt{<persona description>})
    }}
    \endgroup

    \vspace{0.4em} %

    \textbf{User Prompt:} \\
    \begingroup
    \colorbox{usercolor}{\parbox{\dimexpr\linewidth-2\fboxsep\relax}{
    You are tasked with ranking the solutions from the most preferred to the least preferred based on the given task.\\
    Task: \texttt{<task>}\\
    Question: \texttt{<input sample>}\\
    Here are the possible solutions:\\
    Solution 1: \texttt{<agent 1 final answer>}\\
    Solution 2: \texttt{<agent 2 final answer>}\\
    Solution 3: \texttt{<agent 3 final answer>}\\
    Based on the above solutions, please provide the rankings of the solutions separated by spaces. Example: '0 2 1' if you prefer Solution 0 the most, then Solution 2, and finally Solution 1. Provide up to 5 rankings. Only answer with the rankings.
    }}
    \endgroup
\end{combinedprompt}
    \caption{Prompt used to get a vote from each agent for the Ranked Voting decision protocol.}
    \label{fig:ranked_voting_prompt}
\end{figure}







\subsection{Challenge Prompt}
\label{app:challenge_prompt}
\begin{figure}[H]
    \centering
    \begin{combinedprompt}
    \textbf{System Prompt:} \\
    \begingroup
    \colorbox{systemcolor}{\parbox{\dimexpr\linewidth-2\fboxsep\relax}{
    You are a participant in a group discussion. \\
    Your role: \texttt{<persona>} (\texttt{<persona description>})
    }}
    \endgroup

    \vspace{0.4em} %

    \textbf{User Prompt:} \\
    \begingroup
    \colorbox{usercolor}{\parbox{\dimexpr\linewidth-2\fboxsep\relax}{
    The task is: \texttt{<task>}. The question is: \texttt{<question>}. \\
    This is the final answer generated by the discussion: \texttt{<final\_answer>}. \\
    Please critically evaluate this answer. If you agree with the final answer, respond with the exact word 'AGREE' to confirm. If you do not agree, respond with the exact word 'DISAGREE' to challenge the answer.
    }}
    \endgroup
\end{combinedprompt}
    \caption{Prompt used to challenge the final answer.}
    \label{fig:challenge_prompt}
\end{figure}

\begin{figure}[H]
    \centering
    \begin{combinedprompt}
    \textbf{System Prompt:} \\
    \begingroup
    \colorbox{systemcolor}{\parbox{\dimexpr\linewidth-2\fboxsep\relax}{
    You are a participant in a group discussion. \\
    Your role: \texttt{<persona>} (\texttt{<persona description>})
    }}
    \endgroup

    \vspace{0.4em} %

    \textbf{User Prompt:} \\
    \begingroup
    \colorbox{usercolor}{\parbox{\dimexpr\linewidth-2\fboxsep\relax}{
    The task is: \texttt{<task>}. The question is: \texttt{<question>}. \\
    This is the final answer generated by the discussion: \texttt{<final\_answer>}. \\
    You dont agree with the final answer. Please provide a new answer to the question. Include the letter corresponding to your answer in the solution.
    }}
    \endgroup
\end{combinedprompt}
    \caption{Prompt used to generate a new answer in case the final answer got challenged.}
    \label{fig:challenge_new_prompt}
\end{figure}
\newpage
\onecolumn
\section{MALLM Setup}
\label{sec:mallm_setup_app}

    
\begin{configpython}[Default Parameters used for each experiment]{lst:config_default}
input_json_file_path: str = None
output_json_file_path: str = None
task_instruction_prompt: str = None
task_instruction_prompt_template: Optional[str] = None
endpoint_url: str = "https://api.openai.com/v1"
model_name: str = "gpt-3.5-turbo"
api_key: str = "-"
max_turns: int = 10
skip_decision_making: bool = False
discussion_paradigm: str = "memory"
response_generator: str = "simple"
decision_protocol: str = "hybrid_consensus"
visible_turns_in_memory: int = 2
debate_rounds: int = 2
concurrent_api_requests: int = 100
use_baseline: bool = False
use_chain_of_thought: bool = True
num_agents: int = 3
num_neutral_agents: int = 0
agent_generator: str = "expert"
agent_generators_list: list = []
trust_remote_code: bool = False
num_samples: Optional[int] = None
hf_dataset_split: Optional[str] = "test"
hf_token: Optional[str] = None
hf_dataset_version: Optional[str] = None
hf_dataset_input_column: Optional[str] = None
hf_dataset_reference_column: Optional[str] = None
hf_dataset_context_column: Optional[str] = None
use_ablation: bool = False
shuffle_input_samples: bool = False
all_agents_generate_first_draft: bool = False
all_agents_generate_draft: bool = False
policy: Optional[str] = None
voting_protocols_with_alterations: bool = False
calculate_persona_diversity: bool = False
\end{configpython}
\newpage
\section{Example Discussions}
\subsection{Successfull Voting Discussion}
All decision protocols are attached as an example to this discussion. The original discussion was created using simple voting.

\vspace{5mm}

\definecolor{lightyellow}{RGB}{255, 255, 204}

\begin{tcolorbox}[title=Overview, colframe=black!10, coltitle=black, fonttitle=\bfseries, boxrule=0.5mm, width=\textwidth]

\textbf{Task:} Answer the following question. If the question is not answerable with the provided information, write '[UNKNOWN]'.

\textbf{Input:} What choir from the area has sung at the Guildhall?

\textbf{Context:} Southampton has two large live music venues, the Mayflower Theatre (formerly the Gaumont Theatre) and the Guildhall. The Guildhall has seen concerts from a wide range of popular artists including Pink Floyd, David Bowie, Delirious?, Manic Street Preachers, The Killers, The Kaiser Chiefs, Amy Winehouse, Lostprophets, The Midnight Beast, Modestep, and All Time Low. It also hosts classical concerts presented by the Bournemouth Symphony Orchestra, City of Southampton Orchestra, Southampton Concert Orchestra, Southampton Philharmonic Choir and Southampton Choral Society.  

\begin{tcolorbox}[coltitle=black, fonttitle=\bfseries, boxrule=0.5mm, width=\textwidth]
\textbf{MALLM solution:} Southampton Philharmonic Choir \\
\textbf{Real solution:} Southampton Philharmonic Choir
\end{tcolorbox}
\end{tcolorbox}


\begin{tcolorbox}[breakable, enhanced jigsaw, title=Turn 1, colframe=black!10, coltitle=black, fonttitle=\bfseries, boxrule=0.5mm, width=\textwidth]

\begin{tcolorbox}[title=Music Connoisseur \hfill \tiny{\textnormal{A knowledgeable musician who is familiar with various music genres and has attended concerts at the Guildhall.}}, colback=lightyellow, colframe=orange!30, coltitle=black, fonttitle=\bfseries, boxrule=0.5mm, width=\textwidth]
\begin{minipage}{1.0\textwidth}
Based on the information provided, I propose the following solution: Southampton Philharmonic Choir. The text states: "It also hosts classical concerts presented by the Bournemouth Symphony Orchestra, City of Southampton Orchestra, Southampton Concert Orchestra, Southampton Philharmonic Choir and Southampton Choral Society." This suggests that the Southampton Philharmonic Choir is one of the choirs that has performed at the Guildhall, as it is mentioned alongside the other organizations that host classical concerts at the venue.
\end{minipage}
\end{tcolorbox}

\begin{tcolorbox}[title=Choir Conductor \hfill \tiny{\textnormal{A dedicated choral conductor with extensive experience in leading local choirs and orchestras, familiar with the performances at the Guildhall.}}, colback=lightyellow, colframe=orange!30, coltitle=black, fonttitle=\bfseries, boxrule=0.5mm, width=\textwidth]
\begin{minipage}{1.0\textwidth}
[AGREE] I agree with the current solution that the Southampton Philharmonic Choir has sung at the Guildhall. The text explicitly mentions it as one of the organizations that hosts classical concerts at the Guildhall, which indicates that the choir has performed at the venue.
\end{minipage}
\end{tcolorbox}

\begin{tcolorbox}[title=Music Journalist \hfill \tiny{\textnormal{A music enthusiast who has written articles and reviews about various concerts and events at the Guildhall, including performances by local choirs.}}, colback=lightyellow, colframe=orange!30, coltitle=black, fonttitle=\bfseries, boxrule=0.5mm, width=\textwidth]
\begin{minipage}{1.0\textwidth}
[AGREE] I completely agree with the current solution. The text explicitly mentions the Southampton Philharmonic Choir as one of the organizations that hosts classical concerts at the Guildhall, making it the most likely choice.
\end{minipage}
\end{tcolorbox}

\end{tcolorbox}

\begin{tcolorbox}[breakable, enhanced jigsaw, title=Turn 2, colframe=black!10, coltitle=black, fonttitle=\bfseries, boxrule=0.5mm, width=\textwidth]

\begin{tcolorbox}[title=Music Connoisseur \hfill \tiny{\textnormal{A knowledgeable musician who is familiar with various music genres and has attended concerts at the Guildhall.}}, colback=lightyellow, colframe=orange!30, coltitle=black, fonttitle=\bfseries, boxrule=0.5mm, width=\textwidth]
\begin{minipage}{1.0\textwidth}
[AGREE] I'm in agreement with the current solution, as the text explicitly mentions the Southampton Philharmonic Choir as one of the organizations that hosts classical concerts at the Guildhall. This makes it the most likely candidate to have sung at the venue.
\end{minipage}
\end{tcolorbox}

\begin{tcolorbox}[title=Choir Conductor \hfill \tiny{\textnormal{A dedicated choral conductor with extensive experience in leading local choirs and orchestras, familiar with the performances at the Guildhall.}}, colback=lightyellow, colframe=orange!30, coltitle=black, fonttitle=\bfseries, boxrule=0.5mm, width=\textwidth]
\begin{minipage}{1.0\textwidth}
[AGREE] I agree with the current solution that the Southampton Philharmonic Choir has sung at the Guildhall. The text explicitly mentions it as one of the organizations that hosts classical concerts at the Guildhall, which indicates that the choir has performed at the venue.
\end{minipage}
\end{tcolorbox}

\begin{tcolorbox}[title=Music Journalist \hfill \tiny{\textnormal{A music enthusiast who has written articles and reviews about various concerts and events at the Guildhall, including performances by local choirs.}}, colback=lightyellow, colframe=orange!30, coltitle=black, fonttitle=\bfseries, boxrule=0.5mm, width=\textwidth]
\begin{minipage}{1.0\textwidth}
[AGREE] I completely agree with the current solution. The text explicitly mentions the Southampton Philharmonic Choir as one of the organizations that hosts classical concerts at the Guildhall, making it the most likely choice.
\end{minipage}
\end{tcolorbox}

\end{tcolorbox}


\begin{tcolorbox}[breakable, enhanced jigsaw, title=Turn 3, colframe=black!10, coltitle=black, fonttitle=\bfseries, boxrule=0.5mm, width=\textwidth]

\begin{tcolorbox}[title=Music Connoisseur \hfill \tiny{\textnormal{A knowledgeable musician who is familiar with various music genres and has attended concerts at the Guildhall.}}, colback=lightyellow, colframe=orange!30, coltitle=black, fonttitle=\bfseries, boxrule=0.5mm, width=\textwidth]
\begin{minipage}{1.0\textwidth}
[AGREE] I agree with the current solution and the assessment that there is no need to improve it. The text provides explicit evidence that the Southampton Philharmonic Choir hosts classical concerts at the Guildhall, making it the most likely correct answer.
\end{minipage}
\end{tcolorbox}

\begin{tcolorbox}[title=Choir Conductor \hfill \tiny{\textnormal{A dedicated choral conductor with extensive experience in leading local choirs and orchestras, familiar with the performances at the Guildhall.}}, colback=lightyellow, colframe=orange!30, coltitle=black, fonttitle=\bfseries, boxrule=0.5mm, width=\textwidth]
\begin{minipage}{1.0\textwidth}
[AGREE] I agree with the current solution that the Southampton Philharmonic Choir has sung at the Guildhall. The text explicitly mentions it as one of the organizations that hosts classical concerts at the Guildhall, which indicates that the choir has performed at the venue.
\end{minipage}
\end{tcolorbox}

\begin{tcolorbox}[title=Music Journalist \hfill \tiny{\textnormal{A music enthusiast who has written articles and reviews about various concerts and events at the Guildhall, including performances by local choirs.}}, colback=lightyellow, colframe=orange!30, coltitle=black, fonttitle=\bfseries, boxrule=0.5mm, width=\textwidth]
\begin{minipage}{1.0\textwidth}
[AGREE] I completely agree with the current solution. The text explicitly mentions the Southampton Philharmonic Choir as one of the organizations that hosts classical concerts at the Guildhall, making it the most likely choice.
\end{minipage}
\end{tcolorbox}

\end{tcolorbox}


\begin{tcolorbox}[title=Result Simple Voting, colframe=black!10, coltitle=black, fonttitle=\bfseries, boxrule=0.5mm, width=\textwidth]
\textbf{Music Connoisseur final answer:} Southampton Philharmonic Choir

\textbf{Choir Conductor final answer:} Southampton Choral Society

\textbf{Music Journalist final answer:} Southampton Philharmonic Choir

\bigskip

\textbf{Music Connoisseur} voted for answer from \textbf{Music Journalist}

\textbf{Choir Conductor} voted for answer from \textbf{Music Journalist}

\textbf{Music Journalist} voted for answer from \textbf{Music Journalist}


\end{tcolorbox}

\begin{tcolorbox}[title=Result Approval Voting, colframe=black!10, coltitle=black, fonttitle=\bfseries, boxrule=0.5mm, width=\textwidth]
\textbf{Music Critic final answer:} Southampton Philharmonic Choir

\textbf{Local Music Event Coordinator final answer:} Southampton Choral Society and Southampton Philharmonic Choir

\textbf{Information Architect final answer:} Southampton Philharmonic Choir

Southampton Choral Society

\bigskip

\textbf{Music Critic} approved answers from \textbf{Information Architect}, \textbf{Music Critic}

\textbf{Local Music Event Coordinator} approved answers from \textbf{Information Architect}

\textbf{Information Architect} approved answers from \textbf{Information Architect}


\end{tcolorbox}


\begin{tcolorbox}[title=Result Cumulative Voting, colframe=black!10, coltitle=black, fonttitle=\bfseries, boxrule=0.5mm, width=\textwidth]
\textbf{Local Music Enthusiast final answer:} Southampton Choral Society

Southampton Philharmonic Choir

\textbf{Archivist final answer:} Southampton Philharmonic Choir

Southampton Choral Society

\textbf{Music Journalist final answer:} Southampton Philharmonic Choir and Southampton Choral Society

\bigskip

\textbf{Local Music Enthusiast} allocated points: \\
\textbf{Archivist}: 9\\
\textbf{Music Journalist}: 1
\bigskip

\textbf{Archivist} allocated points: \\
\textbf{Local Music Enthusiast}: 2\\
\textbf{Archivist}: 4\\
\textbf{Music Journalist}: 4
\bigskip

\textbf{Music Journalist} allocated points: \\
\textbf{Local Music Enthusiast}: 1\\
\textbf{Archivist}: 2\\
\textbf{Music Journalist}: 7


\end{tcolorbox}

\begin{tcolorbox}[title=Result Ranked Voting, colframe=black!10, coltitle=black, fonttitle=\bfseries, boxrule=0.5mm, width=\textwidth]
\textbf{Music Journalist final answer:} Southampton Philharmonic Choir and Southampton Choral Society.

\textbf{Local Music Expert final answer:} Southampton Philharmonic Choir and Southampton Choral Society

\textbf{Concert Promoter final answer:} Southampton Philharmonic Choir and Southampton Choral Society

\bigskip

\textbf{Music Journalist} ranked answers: \\
1. Music Journalist\\
2. Local Music Expert\\
3. Concert Promoter
\bigskip

\textbf{Local Music Expert} ranked answers:\\
1. Music Journalist\\
2. Local Music Expert\\
3. Concert Promoter
\bigskip

\textbf{Concert Promoter} ranked answers:\\
1. Music Journalist\\
2. Local Music Expert\\
3. Concert Promoter


\end{tcolorbox}















\label{example:good_discussion}



\newpage
\subsection{Agents Tricked Discussion}
In this discussion, the agents were tricked by information provided in the context.

\vspace{5mm}

\definecolor{lightyellow}{RGB}{255, 255, 204}


\begin{tcolorbox}[title=Overview, colframe=black!10, coltitle=black, fonttitle=\bfseries, boxrule=0.5mm, width=\textwidth]

\textbf{Task:} Answer the following question. If the question is not answerable with the provided information, write '[UNKNOWN]'.

\textbf{Input:} What provides critical support for drug discovery and the availability of economic resources?

\textbf{Context:} Biodiversity provides critical support for drug discovery and the availability of medicinal resources. A significant proportion of drugs are derived, directly or indirectly, from biological sources: at least 50\% of the pharmaceutical compounds on the US market are derived from plants, animals, and micro-organisms, while about 80\% of the world population depends on medicines from nature (used in either modern or traditional medical practice) for primary healthcare. Only a tiny fraction of wild species has been investigated for medical potential. Biodiversity has been critical to advances throughout the field of bionics. Evidence from  analysis and biodiversity science indicates that the decline in output from the pharmaceutical sector since the mid-1980s can be attributed to a move away from natural product exploration ("bioprospecting") in favor of genomics and synthetic chemistry, indeed claims about the value of undiscovered pharmaceuticals may not provide enough incentive for companies in free markets to search for them because of the high cost of development; meanwhile, natural products have a long history of supporting significant economic and health innovation. Marine ecosystems are particularly important, although inappropriate bioprospecting can increase biodiversity loss, as well as violating the laws of the communities and states from which the resources are taken.

\begin{tcolorbox}[coltitle=black, fonttitle=\bfseries, boxrule=0.5mm, width=\textwidth]
\textbf{MALLM solution:} Biodiversity provides critical support for drug discovery and the availability of economic resources. \\
\textbf{Real solution:} Not included in context!
\end{tcolorbox}

\end{tcolorbox}


\begin{tcolorbox}[breakable, enhanced jigsaw, title=Turn 1, colframe=black!10, coltitle=black, fonttitle=\bfseries, boxrule=0.5mm, width=\textwidth]

\begin{tcolorbox}[title=Economist \hfill \tiny{\textnormal{A researcher who studies the impact of biodiversity on economic systems, with expertise in market analysis and the economic benefits of natural resources.}}, colback=lightyellow, colframe=orange!30, coltitle=black, fonttitle=\bfseries, boxrule=0.5mm, width=\textwidth]
\begin{minipage}{1.0\textwidth}
As an economist, I'd like to propose a solution that addresses the decline in output from the pharmaceutical sector since the mid-1980s and leverages the untapped potential of biodiversity for drug discovery and economic innovation. Here's a multi-faceted approach: **Solution:** Establish a Public-Private Partnership (PPP) to incentivize the exploration and development of natural products for pharmaceuticals, while ensuring responsible and sustainable bioprospecting practices.
\end{minipage}
\end{tcolorbox}

\begin{tcolorbox}[title=Ethnobotanist \hfill \tiny{\textnormal{A scientist who studies the traditional uses of plants by indigenous communities and their potential for medicinal applications, with expertise in deciphering the cultural and historical context of natural resources.}}, colback=lightyellow, colframe=orange!30, coltitle=black, fonttitle=\bfseries, boxrule=0.5mm, width=\textwidth]
\begin{minipage}{1.0\textwidth}
As an ethnobotanist, I agree with the current solution in principle. The Public-Private Partnership (PPP) framework has the potential to bring together the necessary stakeholders and incentivize the exploration and development of natural products for pharmaceuticals while ensuring responsible and sustainable bioprospecting practices. The multistep approach outlined is comprehensive and addresses various aspects of the issue.
\end{minipage}
\end{tcolorbox}

\begin{tcolorbox}[title=Research Biologist \hfill \tiny{\textnormal{A scientist who studies the properties and potential applications of natural products from a wide range of living organisms, with expertise in the discovery and development of new medicines from biodiversity sources.}}, colback=lightyellow, colframe=orange!30, coltitle=black, fonttitle=\bfseries, boxrule=0.5mm, width=\textwidth]
\begin{minipage}{1.0\textwidth}
I, as a Research Biologist, AGREE with the current solution and the additional suggestions proposed by the Ethnobotanist. The Public-Private Partnership (PPP) framework is a well-rounded approach that addresses the decline in output from the pharmaceutical sector by providing a framework for responsible and sustainable bioprospecting, while promoting economic innovation and job creation. I particularly appreciate the emphasis on community engagement and empowerment, benefit-sharing agreements, traditional knowledge protection, and capacity building for ecosystem-based medicine. 
\end{minipage}
\end{tcolorbox}

\end{tcolorbox}


\begin{tcolorbox}[breakable, enhanced jigsaw, title=Turn 2, colframe=black!10, coltitle=black, fonttitle=\bfseries, boxrule=0.5mm, width=\textwidth]

\begin{tcolorbox}[title=Economist \hfill \tiny{\textnormal{A researcher who studies the impact of biodiversity on economic systems, with expertise in market analysis and the economic benefits of natural resources.}}, colback=lightyellow, colframe=orange!30, coltitle=black, fonttitle=\bfseries, boxrule=0.5mm, width=\textwidth]
\begin{minipage}{1.0\textwidth}
    The current solution has been improved with additional considerations from the ethnobotanist and research biologist. I notice that the additional suggestions enhance the comprehensive approach, addressing community engagement, benefit-sharing, and traditional knowledge protection.
\end{minipage}
\end{tcolorbox}

\begin{tcolorbox}[title=Ethnobotanist \hfill \tiny{\textnormal{A scientist who studies the traditional uses of plants by indigenous communities and their potential for medicinal applications, with expertise in deciphering the cultural and historical context of natural resources.}}, colback=lightyellow, colframe=orange!30, coltitle=black, fonttitle=\bfseries, boxrule=0.5mm, width=\textwidth]
\begin{minipage}{1.0\textwidth}
As an ethnobotanist, I agree with the current solution in principle. The Public-Private Partnership (PPP) framework has the potential to bring together the necessary stakeholders and incentivize the exploration and development of natural products for pharmaceuticals while ensuring responsible and sustainable bioprospecting practices. The multistep approach outlined is comprehensive and addresses various aspects of the issue.
\end{minipage}
\end{tcolorbox}

\begin{tcolorbox}[title=Research Biologist \hfill \tiny{\textnormal{A scientist who studies the properties and potential applications of natural products from a wide range of living organisms, with expertise in the discovery and development of new medicines from biodiversity sources.}}, colback=lightyellow, colframe=orange!30, coltitle=black, fonttitle=\bfseries, boxrule=0.5mm, width=\textwidth]
\begin{minipage}{1.0\textwidth}
I, as a Research Biologist, AGREE with the current solution and the additional suggestions proposed by the Ethnobotanist. The Public-Private Partnership (PPP) framework is a well-rounded approach that addresses the decline in output from the pharmaceutical sector by providing a framework for responsible and sustainable bioprospecting, while promoting economic innovation and job creation. I particularly appreciate the emphasis on community engagement and empowerment, benefit-sharing agreements, traditional knowledge protection, and capacity building for ecosystem-based medicine. These considerations will ensure that local communities are involved in the decision-making process, receive fair compensation for their involvement, and that their traditional knowledge and cultural heritage are respected.

\end{minipage}
\end{tcolorbox}

\end{tcolorbox}


\begin{tcolorbox}[breakable, enhanced jigsaw, title=Turn 3, colframe=black!10, coltitle=black, fonttitle=\bfseries, boxrule=0.5mm, width=\textwidth]

\begin{tcolorbox}[title=Economist \hfill \tiny{\textnormal{A researcher who studies the impact of biodiversity on economic systems, with expertise in market analysis and the economic benefits of natural resources.}}, colback=lightyellow, colframe=orange!30, coltitle=black, fonttitle=\bfseries, boxrule=0.5mm, width=\textwidth]
\begin{minipage}{1.0\textwidth}
[AGREE] The revised solution has effectively incorporated the suggestions from the ethnobotanist and research biologist, addressing community engagement, benefit-sharing, traditional knowledge protection, and other crucial aspects of bioprospecting. The emphasis on responsible and sustainable bioprospecting ensures that the development of novel pharmaceuticals benefits both humans and the environment. To further refine the solution, I agree with the research biologist's suggestion to develop a clear and measurable framework for assessing the impact of the Public-Private Partnership (PPP) on the pharmaceutical sector, local communities, and the environment. Establishing Key Performance Indicators (KPIs) and a monitoring and evaluation process will help ensure that the PPP is held accountable for its impact and remains a responsible and effective approach to bioprospecting.
\end{minipage}
\end{tcolorbox}

\begin{tcolorbox}[title=Ethnobotanist \hfill \tiny{\textnormal{A scientist who studies the traditional uses of plants by indigenous communities and their potential for medicinal applications, with expertise in deciphering the cultural and historical context of natural resources.}}, colback=lightyellow, colframe=orange!30, coltitle=black, fonttitle=\bfseries, boxrule=0.5mm, width=\textwidth]
\begin{minipage}{1.0\textwidth}
As an ethnobotanist, I agree with the current solution in principle. The Public-Private Partnership (PPP) framework has the potential to bring together the necessary stakeholders and incentivize the exploration and development of natural products for pharmaceuticals while ensuring responsible and sustainable bioprospecting practices. The multistep approach outlined is comprehensive and addresses various aspects of the issue.
\end{minipage}
\end{tcolorbox}

\begin{tcolorbox}[title=Research Biologist \hfill \tiny{\textnormal{A scientist who studies the properties and potential applications of natural products from a wide range of living organisms, with expertise in the discovery and development of new medicines from biodiversity sources.}}, colback=lightyellow, colframe=orange!30, coltitle=black, fonttitle=\bfseries, boxrule=0.5mm, width=\textwidth]
\begin{minipage}{1.0\textwidth}
I, as a Research Biologist, AGREE with the current solution and the additional suggestions proposed by the Ethnobotanist. The Public-Private Partnership (PPP) framework is a well-rounded approach that addresses the decline in output from the pharmaceutical sector by providing a framework for responsible and sustainable bioprospecting, while promoting economic innovation and job creation. I particularly appreciate the emphasis on community engagement and empowerment, benefit-sharing agreements, traditional knowledge protection, and capacity building for ecosystem-based medicine. These considerations will ensure that local communities are involved in the decision-making process, receive fair compensation for their involvement, and that their traditional knowledge and cultural heritage are respected.

\end{minipage}
\end{tcolorbox}



\end{tcolorbox}


\begin{tcolorbox}[title=Result, colframe=black!10, coltitle=black, fonttitle=\bfseries, boxrule=0.5mm, width=\textwidth]
\textbf{Economist final answer:} Biodiversity

\textbf{Ethnobotanist final answer:} Biodiversity provides critical support for drug discovery and the availability of economic resources.

\textbf{Research Biologist final answer:} Biodiversity

\bigskip

\textbf{Economist} voted for answer from \textbf{Ethnobotanist}

\textbf{Ethnobotanist} voted for answer from \textbf{Ethnobotanist}

\textbf{Research Biologist} voted for answer from \textbf{Ethnobotanist}


\end{tcolorbox}

\label{example:trick_discussion}

\newpage
\subsection{Bad Voting Discussion}
In this discussion, the agents were tricked by one agent who came up with a solution not provided in the context.

\vspace{5mm}

\definecolor{lightyellow}{RGB}{255, 255, 204}

\begin{tcolorbox}[title=Overview, colframe=black!10, coltitle=black, fonttitle=\bfseries, boxrule=0.5mm, width=\textwidth]

\textbf{Task:} Answer the following question. If the question is not answerable with the provided information, write '[UNKNOWN]'.

\textbf{Input:} Where must enclitics appear in an English sentence?

\textbf{Context:} Because Czech uses grammatical case to convey word function in a sentence (instead of relying on word order, as English does), its word order is flexible. As a pro-drop language, in Czech an intransitive sentence can consist of only a verb; information about its subject is encoded in the verb. Enclitics (primarily auxiliary verbs and pronouns) must appear in the second slot of a sentence, after the first stressed unit. The first slot must contain a subject and object, a main form of a verb, an adverb or a conjunction (except for the light conjunctions a, "and", i, "and even" or ale, "but").

\begin{tcolorbox}[coltitle=black, fonttitle=\bfseries, boxrule=0.5mm, width=\textwidth]
\textbf{MALLM solution:} C) In an English sentence, enclitics (auxiliary verbs and pronouns) tend to appear next to the main verb, usually before any other adverbs or modifiers, but their placement can vary depending on sentence structure and context. \\
\textbf{Real solution:} Not included in context!
\end{tcolorbox}

\end{tcolorbox}

\begin{tcolorbox}[breakable, enhanced jigsaw, title=Turn 1, colframe=black!10, coltitle=black, fonttitle=\bfseries, boxrule=0.5mm, width=\textwidth]

\begin{tcolorbox}[title=Linguist \hfill \tiny{\textnormal{A scholar specializing in Slavic languages, experienced in analyzing the grammatical structures of Czech and other languages.}}, colback=lightyellow, colframe=orange!30, coltitle=black, fonttitle=\bfseries, boxrule=0.5mm, width=\textwidth]
\begin{minipage}{1.0\textwidth}
As a linguist, I'd like to analyze the information given about Czech grammar. We know that in Czech, enclitics (auxiliary verbs and pronouns) must appear in the second syntactic slot of a sentence, after the first stressed unit. Now, we're trying to apply this rule to an English sentence. English does not have a similar grammatical case system, and its word order is more fixed than Czech. However, we can try to find a rough equivalent. So, my proposed solution is that in an English sentence, enclitics (auxiliary verbs and pronouns) tend to appear next to the main verb, and usually before any other adverbs or modifiers. What do you think? Would you like to refine this proposal or add any additional considerations?
\end{minipage}
\end{tcolorbox}

\begin{tcolorbox}[title=English Language Teacher \hfill \tiny{\textnormal{An expert in teaching English grammar and language structures, familiar with the challenges of translating and comparing languages like Czech.}}, colback=lightyellow, colframe=orange!30, coltitle=black, fonttitle=\bfseries, boxrule=0.5mm, width=\textwidth]
\begin{minipage}{1.0\textwidth}
[DISAGREE] While I understand the thought process behind the proposed solution, I think it's essential to consider the unique characteristics of the English language. The idea that enclitics in English tend to appear next to the main verb is accurate, but it's not as rigid as the rule in Czech. In Czech, enclitics must appear in the second syntactic slot, which implies a more fixed position, whereas in English, the placement of auxiliary verbs and pronouns can vary depending on sentence structure and context.
\end{minipage}
\end{tcolorbox}

\begin{tcolorbox}[title=Grammar Analyst \hfill \tiny{\textnormal{A meticulous researcher who has studied the intricacies of languages and their grammatical structures, with a focus on comparing and contrasting language features.}}, colback=lightyellow, colframe=orange!30, coltitle=black, fonttitle=\bfseries, boxrule=0.5mm, width=\textwidth]
\begin{minipage}{1.0\textwidth}
I agree with the English Language Teacher's revised solution. The initial proposal tried to apply the Czech rule to English, but the English language has its unique characteristics that require a more nuanced approach. The revised solution accurately captures the flexibility of English grammar, acknowledging that the placement of enclitics can vary depending on sentence structure and context.
\end{minipage}
\end{tcolorbox}

\end{tcolorbox}

\begin{tcolorbox}[breakable, enhanced jigsaw, title=Turn 2, colframe=black!10, coltitle=black, fonttitle=\bfseries, boxrule=0.5mm, width=\textwidth]

\begin{tcolorbox}[title=Linguist \hfill \tiny{\textnormal{A scholar specializing in Slavic languages, experienced in analyzing the grammatical structures of Czech and other languages.}}, colback=lightyellow, colframe=orange!30, coltitle=black, fonttitle=\bfseries, boxrule=0.5mm, width=\textwidth]
\begin{minipage}{1.0\textwidth}
I agree with the revised solution proposed by the English Language Teacher and the Grammar Analyst. The revised solution accurately captures the flexibility of English grammar and acknowledges that the placement of enclitics (auxiliary verbs and pronouns) can vary depending on sentence structure and context. * The placement of enclitics in English is not as rigid as in Czech, and their placement can vary depending on sentence structure and context. * English sentence structure is more complex, including indirect objects, passives, and compound verb forms, which can affect the placement of auxiliary verbs and pronouns.
\end{minipage}
\end{tcolorbox}

\begin{tcolorbox}[title=English Language Teacher \hfill \tiny{\textnormal{An expert in teaching English grammar and language structures, familiar with the challenges of translating and comparing languages like Czech.}}, colback=lightyellow, colframe=orange!30, coltitle=black, fonttitle=\bfseries, boxrule=0.5mm, width=\textwidth]
\begin{minipage}{1.0\textwidth}
While I understand the thought process behind the proposed solution, I think it's essential to consider the unique characteristics of the English language. The idea that enclitics in English tend to appear next to the main verb is accurate, but it's not as rigid as the rule in Czech. In Czech, enclitics must appear in the second syntactic slot, which implies a more fixed position, whereas in English, the placement of auxiliary verbs and pronouns can vary depending on sentence structure and context. Additionally, the provided examples, while correct in themselves, don't fully demonstrate the complexity of English sentence structure. For instance, consider sentences with indirect objects, passives, or compound verb forms, where the placement of auxiliary verbs and pronouns can be more nuanced.
\end{minipage}
\end{tcolorbox}

\begin{tcolorbox}[title=Grammar Analyst \hfill \tiny{\textnormal{A meticulous researcher who has studied the intricacies of languages and their grammatical structures, with a focus on comparing and contrasting language features.}}, colback=lightyellow, colframe=orange!30, coltitle=black, fonttitle=\bfseries, boxrule=0.5mm, width=\textwidth]
\begin{minipage}{1.0\textwidth}
I agree with the English Language Teacher's revised solution. The initial proposal tried to apply the Czech rule to English, but the English language has its unique characteristics that require a more nuanced approach. The revised solution accurately captures the flexibility of English grammar, acknowledging that the placement of enclitics can vary depending on sentence structure and context. C) In an English sentence, enclitics (auxiliary verbs and pronouns) tend to appear next to the main verb, usually before any other adverbs or modifiers, but their placement can vary depending on sentence structure and context. I believe this revised solution provides a more accurate and practical approximation of the English language's enclitic behavior, considering both the unique characteristics of English and the complexity of sentence structure.
\end{minipage}
\end{tcolorbox}

\end{tcolorbox}

\begin{tcolorbox}[breakable, enhanced jigsaw, title=Turn 3, colframe=black!10, coltitle=black, fonttitle=\bfseries, boxrule=0.5mm, width=\textwidth]

\begin{tcolorbox}[title=Linguist \hfill \tiny{\textnormal{A scholar specializing in Slavic languages, experienced in analyzing the grammatical structures of Czech and other languages.}}, colback=lightyellow, colframe=orange!30, coltitle=black, fonttitle=\bfseries, boxrule=0.5mm, width=\textwidth]
\begin{minipage}{1.0\textwidth}
The current solution accurately captures the flexibility of English grammar and acknowledges that the placement of enclitics (auxiliary verbs and pronouns) can vary depending on sentence structure and context. There's no need to improve it further. The revised solution: C) In an English sentence, enclitics (auxiliary verbs and pronouns) tend to appear next to the main verb, usually before any other adverbs or modifiers, but their placement can vary depending on sentence structure and context. is a clear and accurate representation of the English language's enclitic behavior.
\end{minipage}
\end{tcolorbox}

\begin{tcolorbox}[title=English Language Teacher \hfill \tiny{\textnormal{An expert in teaching English grammar and language structures, familiar with the challenges of translating and comparing languages like Czech.}}, colback=lightyellow, colframe=orange!30, coltitle=black, fonttitle=\bfseries, boxrule=0.5mm, width=\textwidth]
\begin{minipage}{1.0\textwidth}
While I understand the thought process behind the proposed solution, I think it's essential to consider the unique characteristics of the English language. The idea that enclitics in English tend to appear next to the main verb is accurate, but it's not as rigid as the rule in Czech. In Czech, enclitics must appear in the second syntactic slot, which implies a more fixed position, whereas in English, the placement of auxiliary verbs and pronouns can vary depending on sentence structure and context. Here's a revised solution: C) In an English sentence, enclitics (auxiliary verbs and pronouns) tend to appear next to the main verb, usually before any other adverbs or modifiers, but their placement can vary depending on sentence structure and context. What do you think?
\end{minipage}
\end{tcolorbox}

\begin{tcolorbox}[title=Grammar Analyst \hfill \tiny{\textnormal{A meticulous researcher who has studied the intricacies of languages and their grammatical structures, with a focus on comparing and contrasting language features.}}, colback=lightyellow, colframe=orange!30, coltitle=black, fonttitle=\bfseries, boxrule=0.5mm, width=\textwidth]
\begin{minipage}{1.0\textwidth}
I agree with the English Language Teacher's revised solution. The initial proposal tried to apply the Czech rule to English, but the English language has its unique characteristics that require a more nuanced approach. The revised solution accurately captures the flexibility of English grammar, acknowledging that the placement of enclitics can vary depending on sentence structure and context. The revised solution is an improvement over the initial proposal because it takes into account the complexity of English sentence structure, including indirect objects, passives, and compound verb forms, which can affect the placement of auxiliary verbs and pronouns. Additionally, the revised solution provides a more accurate description of the typical placement of enclitics in English sentences, namely next to the main verb, usually before any adverbs or modifiers.
\end{minipage}
\end{tcolorbox}

\end{tcolorbox}

\begin{tcolorbox}[title=Result, colframe=black!10, coltitle=black, fonttitle=\bfseries, boxrule=0.5mm, width=\textwidth]
\textbf{Linguist final answer:} C) In an English sentence, enclitics (auxiliary verbs and pronouns) tend to appear next to the main verb, usually before any other adverbs or modifiers, but their placement can vary depending on sentence structure and context.

\textbf{English Language Teacher final answer:} [UNKNOWN]

\textbf{Grammar Analyst final answer:} [UNKNOWN]

\bigskip

\textbf{Linguist} voted for answer from \textbf{Linguist}

\textbf{English Language Teacher} voted for answer from \textbf{Linguist}

\textbf{Grammar Analyst} voted for answer from \textbf{Linguist}


\end{tcolorbox}

\label{example:failed_discussion}

\newpage
\subsection{Majority Consensus Discussion}
In this discussion, the agents discussed only one round, as they already had a high enough agreement score.

\vspace{5mm}

\definecolor{lightyellow}{RGB}{255, 255, 204}


\begin{tcolorbox}[title=Overview, colframe=black!10, coltitle=black, fonttitle=\bfseries, boxrule=0.5mm, width=\textwidth]

\textbf{Task:} Answer the following question. If the question is not answerable with the provided information, write '[UNKNOWN]'.

\textbf{Input:} What choir from the area has sung at the Guildhall?

\textbf{Context:} Southampton has two large live music venues, the Mayflower Theatre (formerly the Gaumont Theatre) and the Guildhall. The Guildhall has seen concerts from a wide range of popular artists including Pink Floyd, David Bowie, Delirious?, Manic Street Preachers, The Killers, The Kaiser Chiefs, Amy Winehouse, Lostprophets, The Midnight Beast, Modestep, and All Time Low. It also hosts classical concerts presented by the Bournemouth Symphony Orchestra, City of Southampton Orchestra, Southampton Concert Orchestra, Southampton Philharmonic Choir and Southampton Choral Society.  

\begin{tcolorbox}[coltitle=black, fonttitle=\bfseries, boxrule=0.5mm, width=\textwidth]
\textbf{MALLM solution:} Southampton Choral Society \\
\textbf{Real solution:} Southampton Philharmonic Choir
\end{tcolorbox}
\end{tcolorbox}


\begin{tcolorbox}[breakable, enhanced jigsaw, title=Turn 1, colframe=black!10, coltitle=black, fonttitle=\bfseries, boxrule=0.5mm, width=\textwidth]

\begin{tcolorbox}[title=Music Connoisseur (SOLUTION) \hfill \tiny{\textnormal{A knowledgeable musician who is familiar with various music genres and has attended concerts at the Guildhall.}}, colback=lightyellow, colframe=orange!30, coltitle=black, fonttitle=\bfseries, boxrule=0.5mm, width=\textwidth]
\begin{minipage}{1.0\textwidth}
As a Music Historian, I've been through the archives and checked the event schedules of the Guildhall. According to my research, one of the choirs that has sung at the Guildhall is... Southampton Choral Society!
\end{minipage}
\end{tcolorbox}

\begin{tcolorbox}[title=Choir Conductor (AGREED) \hfill \tiny{\textnormal{A dedicated choral conductor with extensive experience in leading local choirs and orchestras, familiar with the performances at the Guildhall.}}, colback=lightyellow, colframe=orange!30, coltitle=black, fonttitle=\bfseries, boxrule=0.5mm, width=\textwidth]
\begin{minipage}{1.0\textwidth}
[AGREE] As a Choir Member of the Southampton Philharmonic Choir who has performed at the Guildhall, I can confirm that the Southampton Choral Society is indeed one of the choirs that have sung at the Guildhall. Our experiences have shown that we often share the stage with the Southampton Choral Society during joint performances with orchestras and other ensembles. So, I'm happy to confirm that Southampton Choral Society is a correct answer. 
\end{minipage}
\end{tcolorbox}
\end{tcolorbox}


\label{example:majority_discussion}

\newpage
\section{AI Usage Card}
{\sffamily
    \centering
    \tcbset{colback=white!10!white}
    \begin{tcolorbox}[
        title={\large \textbf{AI Usage Card} \hfill \makebox{\qrcode[height=1cm]{https://ai-cards.org}}},
        breakable,
        boxrule=0.7pt,
        width=\textwidth,
        center,
        before lower={\footnotesize{AI Usage Card v1.1 \hfill \url{https://ai-cards.org} \hfill \href{https://jpwahle.com/ai-cards-preprint}{PDF} | \href{https://jpwahle.com/cite/jcdl2023wahle.bib}{BibTeX}}},
        segmentation empty,
        halign lower=center,
        collower=black,
        coltitle=black, %
        colbacklower=gray!20, %
        colbacktitle=gray!20  %
        ]
        \vspace{-10pt}
        \footnotesize{
            \begin{longtable}{>
            {\raggedright\arraybackslash}p{.15\textwidth}>{\raggedright\arraybackslash}p{.25\textwidth}>{\raggedright\arraybackslash}p{.25\textwidth}>{\raggedright\arraybackslash}p{.25\textwidth}}
              
                {\color{LightBlue} \MakeUppercase{Project Details}} \newline 
                & {\color{LightBlue} \MakeUppercase{Project Name}} \newline Voting or Consensus? Decision-Making in Multi-Agent Debate
                & {\color{LightBlue} \MakeUppercase{Domain}} \newline Paper
                & {\color{LightBlue} \MakeUppercase{Key Application}} \newline Natural Language Processing
                
                \\
                
                {\color{LightBlue} \MakeUppercase{Contact(s)}}  
                & {\color{LightBlue} \MakeUppercase{Name(s)}} 
                & {\color{LightBlue} \MakeUppercase{Email(s)}} 
                & {\color{LightBlue} \MakeUppercase{Affiliation(s)}} 
                \\ & Lars Benedikt Kaesberg & larsbenedikt.kaesberg@uni-goettingen.de & University Göttingen
                
                \\
                
                {\color{LightBlue} \MakeUppercase{Model(s)}} 
                & {\color{LightBlue} \MakeUppercase{Model Name(s)}} 
                & {\color{LightBlue} \MakeUppercase{Version(s)}} 
                \\ & Llama & 3 70b, 3 8b
                \\ & ChatGPT & 4o, o1
                \\ & Github Copilot & latest

                \\
                \cmidrule{1-4}
                \\
                {\color{LightBlue} \MakeUppercase{Literature Review}} \newline 
                & {\color{LightBlue} \MakeUppercase{Finding literature}} \newline ChatGPT 
                & {\color{LightBlue} \MakeUppercase{Finding examples from known literature or adding literature for existing statements}} \newline ChatGPT  
                & {\color{gray} \MakeUppercase{Comparing literature}}  
                \\
                \cmidrule{2-4}
                \\        
                {\color{LightBlue} \MakeUppercase{Writing}} \newline    
                & {\color{LightBlue} \MakeUppercase{Generating new text based on instructions}} \newline Llama 
                & {\color{LightBlue} \MakeUppercase{Assisting in improving own content or Paraphrasing related work}} \newline ChatGPT 
                & {\color{gray} \MakeUppercase{Putting other works in perspective}}  
                \\
                \cmidrule{2-4}
                \\
                {\color{LightBlue} \MakeUppercase{Coding}} \newline 
                & {\color{LightBlue} \MakeUppercase{Generating new code based on descriptions or existing code}} \newline ChatGPT \newline Github Copilot 
                & {\color{LightBlue} \MakeUppercase{Refactoring and optimizing existing code}} \newline ChatGPT \newline Github Copilot 
                & {\color{gray} \MakeUppercase{Comparing aspects of existing code}}  
                \\
                \cmidrule{1-4}
                \\
        
                {\color{LightBlue} \MakeUppercase{Ethics}} \newline    
                & {\color{LightBlue} \MakeUppercase{Why did we use AI for this project?}} \newline Efficiency / Speed \newline Scalability \newline Expertise Access  
                & {\color{LightBlue} \MakeUppercase{What steps are we taking to mitigate errors of AI?}} \newline None
                & {\color{LightBlue} \MakeUppercase{What steps are we taking to minimize the chance of harm or inappropriate use of AI?}} \newline None

                \\
                \cmidrule{1-4}
                \\
            \end{longtable}
        
        

        \medskip
        
        \textbf{\color{LightBlue} \MakeUppercase{The corresponding authors verify and agree with the modifications or generations of their  used AI-generated content}}
        }
        
        \tcblower
    \end{tcolorbox}
}



\end{document}

