\documentclass[10pt,twocolumn,letterpaper]{article}

\usepackage[pagenumbers]{iccv}

\usepackage{pgfplots}
\pgfplotsset{compat=1.18}
\usepackage{pifont}
\usepackage{graphicx}
\usepackage{makecell}
\usepackage{amsmath}
\usepackage{amssymb}
\usepackage{booktabs}
\usepackage{multirow}
\usepackage{cuted}
\usepackage[shortcuts]{extdash}
\newcolumntype{a}{>{\columncolor{Gray}}c}
\definecolor{confblue}{rgb}{0.21,0.49,0.74}
\usepackage[pagebackref,breaklinks,colorlinks,allcolors=confblue]{hyperref}

%%%%%%%%% TITLE
\title{GaussianFlowOcc: Sparse and Weakly Supervised Occupancy Estimation using Gaussian Splatting and Temporal Flow}

%%%%%%%%% AUTHORS
\author{Simon Boeder\\
Bosch Research\\
{\tt\small simon.boeder@de.bosch.com}
\and
Fabian  Gigengack\\
Bosch Research\\
{\tt\small fabian.gigengack@de.bosch.com}
\and
Benjamin Risse\\
University of M\"unster\\
{\tt\small b.risse@uni-muenster.de}
}

\begin{document}
\maketitle

\begin{strip}
    \vspace{-11mm}
    \centering
    \includegraphics[page=1, width=\textwidth]{figures/Teaser.pdf}
    \captionof{figure}{
        \textbf{Contributions of GaussianFlowOcc.} 
        We propose an efficient and fast model for estimating dynamic scenes using 3D Gaussian distributions, trained with weak supervision.
        By employing a \emph{Temporal Module}, we efficiently model scene dynamics to account for object motion during training.
        Our approach significantly outperforms previous methods in terms of mIoU and inference speed.
    }
    \label{fig:teaser}
\end{strip}

\begin{abstract}
Occupancy estimation has become a prominent task in 3D computer vision, particularly within the autonomous driving community.
In this paper, we present a novel approach to occupancy estimation, termed \emph{GaussianFlowOcc}, which is inspired by Gaussian Splatting and replaces traditional dense voxel grids with a sparse 3D Gaussian representation.
Our efficient model architecture based on a \emph{Gaussian Transformer} significantly reduces computational and memory requirements by eliminating the need for expensive 3D convolutions used with inefficient voxel-based representations that predominantly represent empty 3D spaces.
GaussianFlowOcc effectively captures scene dynamics by estimating temporal flow for each Gaussian during the overall network training process, offering a straightforward solution to a complex problem that is often neglected by existing methods.
Moreover, GaussianFlowOcc is designed for scalability, as it employs weak supervision and does not require costly dense 3D voxel annotations based on additional data (e.g., LiDAR).
Through extensive experimentation, we demonstrate that GaussianFlowOcc significantly outperforms all previous methods for weakly supervised occupancy estimation on the nuScenes dataset while featuring an inference speed that is 50 times faster than current SOTA.
\end{abstract}

\vspace{-5mm}
\section{Introduction}
Vision-based 3D object detection and occupancy estimation are fundamental challenges in the field of autonomous driving, as an accurate understanding of a vehicle's surroundings is essential for reliable planning and navigation~\cite{xu2025survey, shi2024grid, zhang2024vision}.
Despite recent advancements, existing approaches face several significant limitations.
First, most recent occupancy estimation models rely on fully annotated 3D ground-truth data for training, which is expensive and challenging to obtain at scale.
Second, these models depend on dense 3D voxel-based representations, leading to substantial computational overhead that is often unnecessary, given the natural sparsity of real-world 3D scenes.
Although self-supervised methods have been developed to reduce reliance on 3D labels by leveraging temporal rendering~\cite{huang2023selfocc, zhang2023occnerf}, these approaches fail to account for the inherent dynamics of autonomous driving scenarios. 
This introduces temporal inconsistencies during training, severely limiting their performance.

To address these limitations, we introduce \emph{GaussianFlowOcc}, a novel occupancy estimation method that combines efficiency and expressiveness through several contributions.
At the core of our approach is a sparse representation of the scene that utilizes 3D Gaussian distributions, replacing traditional dense voxel grids.
We propose a \emph{Gaussian Transformer} network, a novel deep learning model designed to directly transform image features into 3D Gaussians.
This shift not only reduces computational overhead, but also provides a more flexible representation for modeling dynamic scenes.
Additionally, it allows us to leverage \emph{Gaussian Splatting}~\cite{kerbl20233d} for efficient training without relying on 3D voxel labels.
The estimated Gaussians are rendered back into the input camera space, where losses can be computed using 2D labels such as pseudo ground truth derived from pretrained models like Grounded-SAM~\cite{ren2024grounded} for semantics and Metric3D~\cite{yin2023metric3d} for depth.
Another key contribution is \emph{Temporal Gaussian Splatting}, where adjacent frames are rendered to enhance the supervisory signal, allowing the model to capture 3D geometry more effectively.
To handle temporal inconsistencies caused by dynamic objects, GaussianFlowOcc extends this framework with a \emph{Temporal Module} that estimates a 3D flow for each Gaussian to each rendered temporal frame.
This additional modeling of scene dynamics mitigates errors caused by object motion during training, strengthening the supervision signal and improving the overall performance.
By combining these techniques, GaussianFlowOcc achieves state-of-the-art performance while maintaining computational efficiency, rendering it highly effective for practical occupancy estimation applications.
The source will be made available at \url{https://github.com/boschresearch/GaussianFlowOcc}.
In summary, our contributions are:
\begin{itemize}
    \item \textbf{Efficient Gaussian Transformer for sparse 3D Gaussian scene representation:} 
    We introduce a novel \emph{Gaussian Transformer} that leverages set-based attention mechanisms to efficiently estimate a sparse set of 3D Gaussian distributions representing the scene from multi-view images.
    It entirely eliminates voxelization and computationally intensive operations, leading to an inference speedup of $50\times$ compared to state-of-the-art.
    \item \textbf{Weakly supervised training with Gaussian Splatting:}
    By leveraging Gaussian Splatting, our method can learn from 2D pseudo labels instead of relying on costly 3D annotations from other sensors like LiDAR.
    Our approach requires only images for training, which is the essential input needed for inference anyway, thus making it more scalable than methods depending on specialized 3D labeled datasets.
    \item \textbf{Modeling of scene dynamics via learned 3D flow:} GaussianFlowOcc integrates a \emph{Temporal Module} that additionally estimates 3D flow for each Gaussian, enabling robust handling of dynamic objects during inference and during \emph{Temporal Gaussian Splatting} at training time.
    GaussianFlowOcc is the first method to incorporate temporal supervision while explicitly modeling scene dynamics, without requiring ground-truth flow annotations.
    \item \textbf{State-of-the-art performance:} On top of being highly efficient at inference, GaussianFlowOcc substantially outperforms previous weakly supervised methods.
\end{itemize}

\section{Related Work}

\subsection{Occupancy Estimation}
The 3D semantic occupancy estimation task has become an important area in the autonomous driving research community in recent years.
Numerous voxel-based occupancy benchmarks have been introduced for datasets such as SemanticKITTI~\cite{behley2019semantickitti} and nuScenes~\cite{tian2023occ3d, wang2023openoccupancy}.
Early works in 3D occupancy estimation utilize established techniques from Birds-Eye-View (BEV) perception and object detection \cite{huang2021bevdet,li2022bevformer} to lift multi-view camera images into a unified 3D voxel grid~\cite{huang2023tri, huang2022bevdet4d, tong2023scene, cao2022monoscene, li2023voxformer}.
Subsequent approaches have improved model efficiency \cite{yu2023flashocc, wang2024opus, lu2023octreeocc, liu2024fully, shi2025occupancy, tang2024sparseocc, huang2024gaussianformer}, optimized the training procedure and label efficiency \cite{pan2023renderocc, boeder2024occflownet, gan2023simple, hayler2024s4c, sun2024gsrender}, and improved occupancy estimation performance through architectural innovations \cite{li2023fb, zhang2023occformer, jiang2023symphonize, tan2024geocc, Zhao_2024_CVPR, ma2024cotr, ma2024cam4docc}.

Despite these advancements, the majority of existing 3D occupancy estimation methods require costly voxel-based 3D ground truth labels for training.
Therefore, a parallel line of research has emerged that focuses on self- and weakly supervised learning for occupancy estimation models, leveraging only 2D labels.
SelfOcc~\cite{huang2023selfocc} and OccNeRF~\cite{zhang2023occnerf} employ volume rendering inspired by NeRF~\cite{mildenhall2021nerf} to render estimated 3D occupancy back to the 2D image space. 
Spatial and temporal photometric consistency losses can be used to train the geometry estimation, while semantic information is incorporated using pretrained vision foundation models like OpenSeeD~\cite{zhang2023simple} or GroundedSAM~\cite{ren2024grounded}.
GaussianOcc~\cite{gan2024gaussianocc} replaces the volume rendering pipeline with 3D Gaussian Splatting to accelerate training, yet it continues to model the scene with dense occupancy grids, thus failing to exploit the benefits of a fully sparse Gaussian representation.
Also, while these methods eliminate the need for 3D labels, they fail to address scene dynamics, a critical aspect when relying on temporal consistency losses.

A growing body of research aims to align 3D occupancy with feature spaces of strong foundation models.
OccFeat~\cite{sirko2024occfeat} distills features of CLIP~\cite{radford2021learning} and DINO~\cite{caron2021emerging, oquab2023dinov2} into an occupancy representation for model pretraining.
POP-3D~\cite{vobecky2024pop}, LOcc~\cite{yu2024language} and OVO~\cite{tan2023ovo} perform open-vocabulary occupancy estimation by aligning voxel-based predictions with vision-language features extracted from pretrained encoders like MaskCLIP~\cite{zhou2022extract}.
VEON~\cite{zheng2025veon} and LangOcc~\cite{boeder2024langocc} follow up on the self-supervised methods and directly use volume rendering of vision-language features to train open-vocabulary models.

\subsection{Differentiable Rendering and 3D Gaussian Splatting}
Differentiable rendering has emerged as a powerful technique for learning 3D scene representations by projecting them into 2D views, followed by an optimization based on photometric or semantic consistency.
Neural Radiance Fields (NeRF)~\cite{mildenhall2021nerf} have been particularly influential, modeling scenes as volumetric representations that encode radiance and density, enabling novel view synthesis through differentiable volume rendering.
Recently, 3D Gaussian Splatting (GS)~\cite{kerbl20233d} has introduced a novel paradigm for 3D scene reconstruction by representing scenes as a collection of 3D Gaussians.
This approach significantly reduces computational overhead while preserving expressive scene modeling, making it highly efficient for tasks requiring dynamic or large-scale reconstructions.
NeRF and GS approaches were originally designed to reconstruct individual scenes for novel-view synthesis, with research focusing on improving efficiency~\cite{muller2022instant, chen2025mvsplat}, rendering quality~\cite{barron2021mip, barron2022mip} or feature enrichment~\cite{qin2024langsplat, kerr2023lerf, ye2023featurenerf, zhou2024feature}.
Several works have further explored the modeling of dynamic scenes for video reconstruction \cite{wu20244d, fridovich2023k, pumarola2021d}.
Another line of work incorporates different priors like depth, stereo-matching, or additional data like LiDAR, to train generalizable reconstruction models~\cite{xu2022point,chang2022rc,chen2021mvsnerf,yu2021pixelnerf, wimbauer2023behind, liu2025mvsgaussian, zheng2024gps}.
As mentioned above, similar ideas have been adapted to train occupancy estimation models.
Methods like OccNeRF~\cite{zhang2023occnerf}, SelfOcc~\cite{huang2023selfocc} and GaussianOcc~\cite{gan2024gaussianocc} employ volume rendering or GS for weakly supervised training.
However, they still rely on voxel grids for scene representation, which limits efficiency and scalability.
GSRender~\cite{sun2024gsrender} additionally introduces a ray compensation method to mitigate duplicate predictions along camera rays, a common issue among rendering-based approaches.
Lastly, the recent approach GaussTR~\cite{jiang2024gausstr} shares similarities with our method by adopting 3D Gaussians as the scene representation.
However, GaussTR uses multiple pretrained feature encoders (e.g., CLIP, Metric3D, SAM) during inference, making the pipeline computationally expensive.
Furthermore, GaussTR employs standard attention layers, which significantly limits the number of Gaussians it can handle. 

\section{Methodology}\label{sec:methodology}
\subsection{Problem Definition and Scene Representation}\label{sec:problem_definition}
The objective of occupancy prediction is to accurately estimate the 3D geometry and semantics surrounding a vehicle based on a set of $L$ multi-view images $I = \{I^1, I^2, ..., I^L\}$ at the current time step.
Previous methods have typically utilized a semantic voxel volume $V=\{c_1, c_2, ... c_{C}\}^{X \times Y \times Z}$ on a predefined grid, with $C$ representing the number of semantic classes, as a structured representation of the scene.
In contrast, our approach involves estimating the 3D scene using 3D Gaussian distributions, drawing inspiration from 3D Gaussian Splatting~\cite{kerbl20233d}.
The scene is defined as a collection of Gaussians $\mathcal{G} = \{G_1, G_2, ..., G_n\}$, each characterized by a set of properties: mean $\mu \in \mathbb{R}^3$, opacity $\sigma \in [0,1]$, scale $s \in \mathbb{R}^{3}$, rotation quaternions $r \in \mathbb{R}^4$, and semantic logits $c \in \mathbb{R}^{C}$:
\begin{align}
G_i = (\mu, \sigma, s, r, c)
\label{eq:scene_definition}
\end{align}
We train a model $\mathbb{M}(I)$ to estimate the 3D Gaussians $\mathcal{G}$.
If desired, the resulting scene representation can be easily converted to a voxel volume as a post-processing step to facilitate comparison with previous methods (see \cref{sec:voxelize}).

\subsection{Model Architecture}\label{sec:model_arch}
The architecture of the proposed model is illustrated in \cref{fig:architecture}.
Initially, image features $\hat{I}$ are extracted from the input images using an image encoder.
The image features, along with the initial positions and features of the Gaussians, are then fed into the proposed \emph{Gaussian Transformer} (\cref{sec:gaussian_transformer}).
The \emph{Gaussian Transformer} iteratively transforms the initial Gaussians to their final positions by employing cross-attention to the image features, self-attention between the Gaussians, and temporal attention to the Gaussians of the previous frame. 
Subsequently, the \emph{Gaussian Heads} (\cref{sec:gaussian_head}) estimate the remaining Gaussian properties and their semantics.
Concurrently, the \emph{Temporal Module} (\cref{sec:gaussian_flow}) computes a 3D flow for each Gaussian to its temporal neighbors.
The final Gaussians and their temporal offsets are then utilized in the Gaussian Splatting pipeline (\cref{sec:gaussian_splatting}) to render depth and semantic maps into the input cameras and a set of temporally adjacent frames.
The model is trained using losses between the rendered and ground truth depth and semantic segmentation maps.

\begin{figure*}
    \centering
    \includegraphics[page=1, trim=0cm 8.86cm 3.68cm 0cm, clip, width=\textwidth]{figures/Figures.pdf}
    \caption{
        \textbf{Overview of GaussianFlowOcc.}
        After encoding a set of input images, the \emph{Gaussian Transformer} iteratively estimates the position and features of 3D Gaussian distributions.
        The \emph{Gaussian Heads} then predict opacity, scale, rotation and semantics of these Gaussians.
        The model is trained using Gaussian Splatting with 2D labels, generated from off-the-shelf models.
        The \emph{Temporal Module} simultaneously estimates 3D temporal offsets for each Gaussian to correct temporal inconsistencies when using \emph{Temporal Gaussian Splatting}.
    }
    \label{fig:architecture}
\end{figure*}

\subsubsection{Initial Gaussians} \label{sec:initial}
We initialize with a set of $N$ Gaussian means $\mathcal{G}^0_\mu \in \mathbb{R}^{N \times 3}$ and latent features $\mathcal{G}^0_f \in \mathbb{R}^{N \times D}$, akin to the queries in previous works \cite{li2022bevformer, liu2022petr,wang2023exploring}.
The initial positions and features of the Gaussians are learnable, enabling the model to incorporate prior knowledge of driving scenes.
Notably, the other Gaussian properties $o, s, r, c$ are left uninitialized; instead, the model learns to infer them from the Gaussian features through the \emph{Gaussian Heads} in the output stage, a strategy that has proven to be more robust for training.

\subsubsection{Image Encoder} \label{sec:image encoder}
Image features $\hat{I}$ are extracted from the input images $I$ using a pretrained backbone architecture like \textit{ResNet50}~\cite{he2016deep}.
These features and the initial Gaussians are then fed into the \emph{Gaussian Transformer}.

\subsubsection{Gaussian Transformer} \label{sec:gaussian_transformer}
The \emph{Gaussian Transformer} iteratively updates the initial Gaussian means and features over $B$ blocks, with each block $b$ comprising five successive modules, elaborated in the subsequent sections.
An overview of a \emph{Gaussian Transformer} block is depicted in \cref{fig:gauss_transformer}. 

\begin{figure}
    \centering
     \resizebox{\linewidth}{!}{\includegraphics[page=2, trim=0cm 4.08cm 6.9cm 0cm, clip]{figures/Figures.pdf}}
    \caption{
        \textbf{Architecture of the proposed \emph{Gaussian Transformer}.}
        In each block, the features and means of the Gaussians are refined using a Positional Encoding, ITA, ISA, GICA and Gaussian Rectification.
    }
    \label{fig:gauss_transformer}
    \vspace{-3mm}
\end{figure}

\paragraph{Positional Encoding}
The Gaussian means of the previous block $\mathcal{G}_\mu^{b-1}$ are encoded into the latent dimension $D$ using a Multi-Layer Perceptron (MLP) and are added onto the previous features $\mathcal{G}_f^{b-1}$.

\paragraph{Gaussian-Image Cross-Attention}
To enable Gaussians to acquire meaningful scene information, we introduce the \emph{Gaussian-Image Cross-Attention} (GICA), which facilitates interactions between Gaussians $\mathcal{G}$ and the image features $\hat{I}$.
For this we leverage deformable cross-attention, originally proposed by \cite{li2022bevformer} and widely adopted in subsequent works.
In this operation, the current positions of the Gaussians $\mathcal{G}_\mu$ are projected onto the image feature maps using the camera parameters.
For each projected point, a set of image features around this point are sampled and used as keys and values for an attention operation.
This allows the Gaussians to extract rich, localized information from the scene in a computationally efficient manner.
For a detailed description of deformable attention, we refer readers to \cite{li2022bevformer}.

\paragraph{Induced Self-Attention}
Computing interactions among Gaussians is a critical aspect of our approach.
However, the standard attention mechanism is characterized by quadratic time and memory complexity $\mathcal{O}(N^2)$.
This imposes severe constraints on the number of Gaussians that can be processed (as demonstrated in \cref{fig:ablation_induced}).
To address this limitation, we draw inspiration from the Set Transformer~\cite{lee2019set} and employ an attention operation called \emph{Induced Self-Attention} (ISA).
ISA reduces the memory complexity to approximately linearly scaling with respect to the number of Gaussians, allowing us to employ a significantly larger number of Gaussians. 
The operation is schematically illustrated in \cref{fig:induced_attn} (A).
In essence, ISA replaces quadratic attention with two attention operations connected via a bottleneck.
Instead of allowing Gaussians to directly attend to one another, we introduce a set of $M$ trainable latent feature vectors, called inducing points $P \in \mathbb{R}^{M \times D}$ (with $M \ll N$).
These inducing points aggregate information from all Gaussians into a bottleneck representation $H$.
Subsequently, all Gaussians interact with the bottleneck features $H$ enabling indirect yet complete Gaussian-to-Gaussian interactions.
This transformation reduces the computational complexity to $\mathcal{O}(MN)$. 
The overall process is formally defined as:
\begin{align}
    &\ \mathrm{ISA}(\mathcal{G}_f) = \mathrm{MHA}(\mathcal{G}_f, H, H), \; \\
    &\ \text{where} \quad H = \mathrm{MHA}(P, \mathcal{G}_f, \mathcal{G}_f),
    \label{eq:induced_self}
\end{align}
where $\mathrm{MHA}(Q, K, V)$ denotes the standard multi-head attention operation, which includes a skip connection and a feed-forward layer.

\paragraph{Induced Temporal Attention}
Building upon the concept of ISA, we introduce \emph{Induced Temporal Attention} (ITA) to enable efficient temporal information propagation across frames, as depicted in \cref{fig:induced_attn} (B).
The ITA operation leverages the structure of induced attention by letting the inducing points first attend to the Gaussian features from the previous frame to compute the bottleneck features $Z$.
Subsequently, Gaussian features from the current frame interact with the bottleneck features, facilitating temporal attention with linear complexity.
Formally, given $\mathcal{G}'_f$ as the final Gaussian features from the previous frame, the ITA operation is expressed as:
\begin{align}
    &\ \mathrm{ITA}(\mathcal{G}_f, \mathcal{G}'_f)= \mathrm{MHA}(\mathcal{G}_f, Z, Z), \; \\
    &\ \text{where} \quad Z = \mathrm{MHA}(P, \mathcal{G}'_f, \mathcal{G}'_f),
    \label{eq:induced_temp}
\end{align}
Note that, as we show in \cref{fig:architecture} and \cref{fig:gauss_transformer}, the previous Gaussian features $\mathcal{G}'_f$ are derived by first correcting the previous final Gaussian means using ego-motion and estimated temporal flow to the current frame, computing the positional encoding and adding them to the previous final features.

\begin{figure}
    \centering
    \resizebox{\linewidth}{!}{\includegraphics[page=3, trim=0cm 13.2cm 15.07cm 0cm, clip]{figures/Figures.pdf}}
    \caption{
        \textbf{Illustration of the Induced Attention Modules.} 
        (A) Induced Self-Attention (ISA) 
        (B) Induced Temporal Attention (ITA)}
    \label{fig:induced_attn}
    \vspace{-4mm}
\end{figure}

\paragraph{Gaussian Rectification}
Thus far, only the Gaussian features have been updated using cross-attention, self-attention and temporal attention.
In the \emph{Gaussian Rectification} step, these features are used to estimate an update to the Gaussian means for the next block. 
Unlike prior approaches such as GaussianFormer~\cite{huang2024gaussianformer} and GaussTR~\cite{jiang2024gausstr}, our method refrains from modifying Gaussian properties other than the means within the transformer itself.
Instead, we focus on gathering features that encapsulate all the necessary information.
The update is performed using an MLP, denoted as $\mathrm{Rect}(\mathcal{G}_f)$, which estimates a residual for each Gaussian and adds it to the current means to refine their position:
\begin{align}
    &\ \mathcal{G}^{b+1}_\mu = \mathcal{G}^{b}_\mu + \Delta \mathcal{G}^{b}_\mu \; \\
    &\ \text{where} \quad \Delta \mathcal{G}^{b}_\mu = \mathrm{Rect}(\mathcal{G}^{b}_f)
\end{align}

\subsubsection{Gaussian Heads} \label{sec:gaussian_head}
After refining the Gaussian positions and features through $B$ blocks of the \emph{Gaussian Transformer}, the \emph{Gaussian Heads} estimate the remaining properties of the Gaussians, completing the scene representation.
This is accomplished using a set of individual MLP heads, each dedicated to predict a specific property from the Gaussian features.
To ensure valid outputs, a sigmoid activation $\sigma(\cdot)$ function is applied to estimate the opacity $o$, while the rotation coefficients $r$ are normalized to unit length.
The outputs are computed as follows:
 \begin{align}
    &\ o = \sigma \left(\mathrm{MLP}_o \left(\mathcal{G}_f \right) \right), \ s = \mathrm{MLP}_s(\mathcal{G}_f), \\ 
    &\ r = \mathrm{norm}(\mathrm{MLP}_r(\mathcal{G}_f)), \ c = \mathrm{MLP}_c(\mathcal{G}_f),
\end{align}
The outputs generated by the \emph{Gaussian Heads} define the final 3D scene representation, which can be used for downstream applications.

\subsection{Gaussian Splatting Supervision} \label{sec:gaussian_splatting}
Representing the scene with 3D Gaussians allows us to leverage the highly efficient Gaussian Splatting (GS)~\cite{kerbl20233d} method during training. 
For a detailed explanation of GS, we refer readers to the original work \cite{kerbl20233d}.
Essentially, the estimated Gaussian properties are used to project the 3D Gaussians back onto the input image views via the corresponding camera parameters.
This enables us to rasterize predicted 2D semantic logits and depth values.
Notably, unlike standard GS, which rasterizes RGB values, we rasterize semantic logits instead.
We generate 2D semantic and depth labels using pretrained models to serve as ground truth. 
The entire model is then trained using 2D rendering losses. 
This enables training in a weakly supervised manner directly in the 2D image space, eliminating the need for 3D labels or additional sensor data, such as LiDAR, entirely.
We optimize the MSE-Loss $\mathcal{L}_{depth}$ between rendered depth $\hat{D}$ and precomputed depth maps $D$ and the binary cross-entropy loss $\mathcal{L}_{seg}$ between rendered logits $\hat{S}$ and precomputed semantic segmentation maps $S$.
 \begin{align}
    \mathcal{L} = \mathcal{L}_{depth}(\hat{D}, D) + \mathcal{L}_{seg}(\hat{S}, S) 
\end{align}


\subsection{Temporal Gaussian Splatting} \label{sec:temp_splatting}
Cameras in autonomous driving scenarios usually share a relatively small frustum overlap, which complicates learning of correct depth and scene geometry.
To address this limitation and increase viewpoint overlap between training cameras, we incorporate \emph{Temporal Gaussian Splatting}, a technique commonly employed in recent works~\cite{pan2023renderocc, boeder2024occflownet, gan2024gaussianocc}.
During training, for a specified temporal horizon $T$, we load the camera parameters and 2D ground truth labels of the $T$ previous and $T$ subsequent frames of the sequence.
We then use GS to rasterize our estimated 3D Gaussians into predicted depth and semantic maps for all temporal cameras and compute the same 2D rendering losses as for the current frame.
This strategy effectively widens the supervisory signal by incorporating data from neighboring temporal views.
While na\"ively applying \emph{Temporal Gaussian Splatting} can already significantly improve prediction performance, it overlooks the fact that many objects in the scene are in motion.
This motion introduces inconsistencies in supervisory signals across frames, as objects predicted in the current frame might have moved in previous/subsequent frames, so the 2D losses give wrong signals sometimes.

\subsection{Temporal Module and Gaussian Flow} \label{sec:gaussian_flow}
To address temporal inconsistencies caused by the motion of dynamic objects during training, we introduce a \emph{Temporal Module} that learns to correct object motion.
This module reduces discrepancies across temporally adjacent frames by estimating 3D offsets for each Gaussian to align them with all frames within a defined temporal horizon.

The \emph{Temporal Module} takes the Gaussian features output by the \emph{Gaussian Transformer} as input and computes a 3D motion offset for each Gaussian relative to each temporal frame in the horizon.
To achieve this, we first define a set of learnable $D$-dimensional \textit{time tokens} $\Psi \in \mathbb{R}^{T \times D}$, one token for each frame in the temporal horizon $T$.
For a given target time step, we create a copy of the Gaussians and append the corresponding time token to their features.
These temporally encoded features are then passed through an MLP to estimate a 3D motion offset vector $\overrightarrow{v} (t) \in \mathbb{R}^3$:
 \begin{align}
    \overrightarrow{v} (t) = \mathrm{MLP}_v  \left(\mathcal{G}_f, \oplus  \Psi(t) \right).
\end{align}
The estimated offsets are added to the copied Gaussian positions, effectively translating the Gaussians to their updated locations in the target time step.
GS is subsequently applied in the usual manner, using the translated Gaussians for the corresponding temporal frame.
The \emph{Temporal Module} is trained jointly with the rest of the model.
Importantly, it does not require additional losses or ground truth motion data.
Dynamic objects are correctly rendered in temporally adjacent frames only when their motion is accurately estimated, allowing the module to learn implicitly through the existing rendering losses.
The benefits of this dynamic object handling are evaluated in \cref{sec:ablation_temp_module}, demonstrating its impact on scene consistency and model performance.

\subsection{Voxelization} \label{sec:voxelize}
The final output of our model consists solely of 3D Gaussian distributions (see \cref{fig:teaser}), with no voxelization or dense operations used during the model’s architecture or training process.
However, these Gaussians can be efficiently voxelized as a post-processing step to facilitate direct comparison with prior work on occupancy estimation benchmarks.
For each voxel center with position $p_i=(x,y,z)$ on a defined voxel grid, the estimated 3D Gaussian distributions are queried and their opacity and semantic logits are accumulated at each point.
Each voxel is assigned a label corresponding to the highest accumulated semantic logits.
Voxels with an accumulated opacity below a defined threshold are classified as free.
Further details about the voxelization are provided in the appendix.

\section{Experiments} \label{sec:experiments}
To evaluate the performance of GaussianFlowOcc, we compare our approach with recent state-of-the-art methods on occupancy estimation.
In addition, an extensive ablation study is performed to study the impact of each module. 

\subsection{Dataset and Metrics}
We conduct experiments on the Occ3D-nuScenes dataset~\cite{tian2023occ3d}, a widely used benchmark for occupancy estimation based on the nuScenes dataset~\cite{caesar2020nuscenes}.
The dataset defines a voxel grid spanning $[-40m, 40m]$ along the X and Y axes and $[-1m, 5.4m]$ along the Z axis, with a voxel size of $0.4m$.
Each voxel is assigned one of $17$ semantic labels from nuScenes Lidarseg~\cite{fong2021panoptic} or classified as free space.
Performance on this benchmark is evaluated using the Intersection over Union (IoU) score for each of the $17$ semantic categories, aggregated to the mean IoU (mIoU).
Additionally, following recent trends in the literature, we also report the RayIoU metric~\cite{liu2024fully}, a ray-based metric focused on rewarding good 3D completion while penalizing thick surface predictions.
Finally, we provide information on inference speed (FPS) on a single A100 GPU.

\subsection{Implementation Details}
We utilize a ResNet50~\cite{he2016deep} backbone for feature extraction with an image resolution of $256\times704$ and set the number of Gaussians to $N=10000$.
The \emph{Gaussian Transformer} consists of three blocks, each with induced attention layers employing $M=500$ inducing points.
The \emph{Gaussian Heads} are implemented as a single linear layer, while all layers in the model share a latent dimension of $256$.
The \emph{Temporal Module} consists of three linear layers and operates with a temporal horizon $T=6$ during training.
The model is trained for $18$ epochs using four A100 GPUs.

\subsection{3D Occupancy Prediction Results}

\begin{table*}[ht]
    \begin{center}
        \caption{
            \textbf{Occupancy estimation performance on the Occ3D-nuScenes validation dataset.}
            The \emph{Training Labels} column specifies the method used for pseudo label generation.
            Performance is reported in terms of IoU (\%).
            The best-performing result in each column is highlighted in \textbf{bold}, while the second-best is shown in \textit{italic}. 
            A dash (-) indicates that no results were provided by the original work.}
        \label{table:main}
        \resizebox{\textwidth}{!}{
            \begin{tabular}{lll|cc|cccc|c}
                \hline
                \noalign{\smallskip}
                 Method & Backbone & Training Labels & mIoU & IoU & RayIoU & RayIoU@1 & RayIoU@2 & RayIoU@4 & FPS\\
                \noalign{\smallskip}
                \hline
                \noalign{\smallskip}
                SelfOcc \cite{huang2023selfocc} & ResNet50 & OpenSeeD & 10.54 & 45.01 & - & - & - & -  & 1.15 \\
                OccNeRF \cite{zhang2023occnerf} & ResNet101 & GroundedSAM & 10.81 & 22.81 & - & - & - & - & 1.27 \\
                GaussianOcc \cite{gan2024gaussianocc} & ResNet101 & GroundedSAM & 11.26 & - & 11.85 & \textit{8.69} & 11.90 & 14.95 & \textit{5.57} \\
                GaussTR \cite{jiang2024gausstr} & $2 \times$ViT & CLIP + Metric3D + GroundedSAM & \textit{13.26} & \textit{45.19} & - & - & - & - & 0.20 \\
                \noalign{\smallskip}
                \hline
                \noalign{\smallskip}
                GaussianFlowOcc (Ours) & ResNet50 & GroundedSAM + Metric3D & \textbf{16.02} & \textbf{46.91} & \textbf{16.47} & \textbf{11.81} & \textbf{16.58} & \textbf{20.98} & \textbf{10.2} \\
                \noalign{\smallskip}
                \hline
            \end{tabular}
            \addtolength{\tabcolsep}{2pt}
        }
    \end{center}
    \vspace{-4mm}
\end{table*}

We evaluate our proposed method against state-of-the-art approaches for weakly supervised 3D occupancy prediction on the Occ3D-nuScenes dataset~\cite{tian2023occ3d}, with the results summarized in \cref{table:main}.
Each of the compared methods relies on a pretrained semantic segmentation model to generate 2D semantic labels for training.
Our approach significantly outperforms all existing models, achieving at least a $42\%$ mIoU improvement over methods that rely on RGB reconstruction and pseudo semantic labels.
Notably, we surpass the recent GaussTR, which also uses Metric3D for depth supervision, by $21\%$ mIoU, improving from $13.26$ to $16.05$.
Furthermore, our model achieves this superior performance without relying on expensive backbones such as CLIP and Metric3D, which GaussTR incorporates at the cost of significantly higher inference time.
A similar trend is observed in the RayIoU metric, where our method improves performance by approximately $39\%$ compared to GaussianOcc.

Beyond accuracy, our model also offers a substantial efficiency advantage.
Due to its lightweight architecture, it achieves inference speeds up to almost $9\times$ faster than previous voxel-based methods.
As mentioned above, especially GaussTR, remains computationally expensive due to its reliance on multiple ViT-L backbones; GaussianFlowOcc is $50 \times$ faster.

We attribute these performance gains primarily to two key innovations: 
Firstly, our \emph{Temporal Module} significantly improves temporal consistency, yielding large performance increases as we show in \cref{sec:ablation_temp_module}. 
Secondly, the \emph{Gaussian Transformer} allows for a much higher number of Gaussians to be processed efficiently, due to approximately linearly scaling induced attention modules.

A detailed per-class performance breakdown is provided in the appendix.
Additionally, qualitative results presented in \cref{fig:teaser} and in the appendix highlight our model’s strong 3D scene completion abilities.
In particular, our approach excels at representing small, thin, and flat objects (e.g., traffic signs, poles, and pedestrians), a major advantage of using 3D Gaussians, which can model arbitrary geometries.
In contrast, voxel-based methods are inherently constrained by a grid resolution, limiting their ability to capture fine details.

\subsection{Ablation Study} \label{sec:exp3}

\begin{table*}[t]
    \begin{minipage}[t]{0.3\textwidth}
        \centering
        \caption{
            \textbf{Ablation on self and temporal attention.}
            Performance changes when omitting self and/or temporal attention within the \emph{Gaussian Transformer}.}
        \label{table:ablation_attention}
        \resizebox{\textwidth}{!}{
        \begin{tabular}{cc|c}
            \hline
            \thead{Induced \\ Self-Attention} & \thead{Induced \\ Temporal Attention} & mIoU \\
            \hline
            & & 13.81 \\
             \checkmark & & 14.60 \\
              & \checkmark & 14.47 \\
            \checkmark & \checkmark &  \textbf{16.02} \\
            \hline
        \end{tabular}
        }
    \end{minipage}
    \hfill
    \begin{minipage}[t]{0.3\textwidth}
        \centering
         \caption{
            \textbf{Ablation study of the proposed \emph{Temporal Module}.}
            Performance comparison when compensating for object motion during training versus ignoring temporal inconsistencies during \emph{Temporal Gaussian Splatting}.
            }
        \label{table:ablation_temp_module}
        \resizebox{\textwidth}{!}{\begin{tabular}{c|cc}
            \hline
            Temporal Module & mIoU & RayIoU \\
            \hline
             & 14.18 & 14.46 \\
            \checkmark & \textbf{16.02} &  \textbf{16.47} \\
            \hline
        \end{tabular}}
    \end{minipage}
    \hfill
    \begin{minipage}[t]{0.3\textwidth}
        \centering
         \caption{
            \textbf{Ablation on the Gaussian Parameters.}
            Performance values when only using and learning a subset of the Gaussian parameters.
            }
        \label{table:ablation_params}
        \resizebox{\textwidth}{!}{\begin{tabular}{ccc|c}
            \hline
             Opacity & Scale & Rotation & mIoU \\
            \hline
             & & & 9.48 \\
             \checkmark & & & 13.12 \\
             \checkmark & \checkmark & & 14.98 \\
             \checkmark & \checkmark & \checkmark & \textbf{16.02}\\
            \hline
        \end{tabular}}
    \end{minipage}
\end{table*}

\subsubsection{Induced Attention vs. Full Attention} \label{sec:ablation_induced}
To verify the effectiveness of the proposed induced attention mechanism, we compare it to a standard full attention mechanism.
The results are shown in \cref{fig:ablation_induced}.
The plot shows both mIoU score and memory consumption for a single sample during training for induced attention (with fixed $M=500$) and standard attention.
As can be seen, with lower numbers of Gaussians, the induced attention and full attention exhibit comparable performance and memory usage, as the quadratic complexity of the full attention is not yet significant.
Increasing the number of Gaussians to $N=5000$ already requires double the amount of memory for full attention, while not providing any significant performance increase.
Using $10000$ Gaussians demands $50\mathrm{GB}$ of GPU memory, forcing us to reduce the batch size to $1$ on our hardware.
Training a model this way did not lead to convergence.
In contrast, using induced attention, the memory requirement scales linearly, ultimately enabling better performance.
The increase in mIoU achieved by the Induced Attention model when increasing the number of Gaussians from $5000$ to $10000$ indicates that a higher number of Gaussians is necessary to represent the complexity of driving scenes.
We find that increasing $N$ beyond $10000$ did not improve performance.

\begin{figure}
    \centering
	\resizebox{\linewidth}{!}{\includegraphics[page=7, trim=0cm 9.89cm 13.23cm 0cm, clip]{figures/Figures.pdf}}
	\caption{
		\textbf{Comparison in mIoU and GPU Memory consumption between Induced and Full Attention.}}
	\label{fig:ablation_induced}
\end{figure}

\subsubsection{Self-Attention and Temporal Attention}
We also show the necessity of the self-attention and temporal attention mechanisms by training a model without each of these modules.
The results are displayed in \cref{table:ablation_attention}, indicating that both modules increase performance significantly.
Note that GICA is always used, as no scene information would be used otherwise.

\subsubsection{Temporal Module} \label{sec:ablation_temp_module}
We demonstrate the effectiveness of the proposed \emph{Temporal Module} by training a model without it while keeping all other components unchanged.
As visible in \cref{table:ablation_temp_module}, using the \emph{Temporal Module} leads to a relative improvement in mIoU of $13\%$ and in RayIoU of $14\%$, which emphasizes the importance of object motion compensation during training.
A demonstration of rasterized views of temporally adjacent frames with and without applied estimated flow is provided in the appendix and in \cref{fig:teaser}.

\subsubsection{Gaussian Parameters}
In contrast to previous works \cite{gan2024gaussianocc, sun2024gsrender}, we find that allowing the model to freely adjust all Gaussian parameters yields the best results. 
To demonstrate this, we conduct an experiment in which we disable the \emph{Gaussian Heads} for certain properties and fix their values for GS, preventing them from being learned.
We fix the opacity with $1$, the scale with $0.3$ and the rotations with unit quaternions.
When only estimating the mean and fixing all other properties, only a performance of $9.48$ mIoU can be achieved, showing the importance of the other properties.
When allowing the model to influence the Gaussians opacity, a large improvement to $13.12$ mIoU is reached, outperforming most previous methods.
Adding the other properties further increases performance, with the maximum score achieved when using all properties.
As mentioned above, the positive effect of using rotation and scale is especially noticeable for thin or flat structures.

\subsubsection{Time Horizon}
We ablate the \emph{Temporal Gaussian Splatting} horizon $T$ in the range of $\{0, 2, ..., 10\}$ to show how temporal rendering leads to better supervision.
With horizons greater than 8, the model diverges, which is likely due to large discrepancy between the time steps, making it increasingly harder to estimate (especially non-linear) motion. 
Additionally, some camera poses are just too distant from the current time step.

\begin{table}
    \begin{center}
        \caption{\textbf{Ablation on different horizons $T$. }
            The best performance is achieved with $T=6$.}
        \label{table:ablation_horizon}
     \resizebox{\columnwidth}{!}{
        \begin{tabular}{c|cccccc}
            \hline
            Time Horizon & 0 & 2 & 4 & \textbf{6} & 8 & 10\\
            \hline
            mIoU & 12.67 & 14.17 & 14.71 & \textbf{16.02} & 14.82 & N/A \\
            \hline
        \end{tabular}
        }
    \end{center}
\end{table}

\subsubsection{Pseudo Depth Labels}
To analyze the impact of pseudo depth labels generated by Metric3D, we train a model using only pseudo semantics.
The results of this experiment are presented in \cref{table:ablation_pseudodepth}.
Even without pseudo depth supervision, our model outperforms all prior works not using depth, demonstrating its strong capability.
Notably, other approaches \cite{zhang2023occnerf, huang2023selfocc} incorporate additional photometric losses to enhance geometric consistency, which we did not find to be beneficial.
Interestingly, GaussTR~\cite{jiang2024gausstr} fails to converge without depth labels, further highlighting the robustness of our approach.
This suggests that \emph{Temporal Gaussian Splatting} alone provides sufficient 3D cues for geometry estimation by leveraging semantic maps from multiple cameras over an extended time horizon.

\begin{table}
    \begin{center}
    \small
        \caption{\textbf{Ablation on pseudo depth labels.}
        Depth labels substantially improve performance.}
        \label{table:ablation_pseudodepth}
        \begin{tabular}{c|cc|c}
            \hline
            Pseudo Depth & mIoU & IoU & RayIoU \\
            \hline
             & 13.20 & 34.12 & 12.75 \\
            \checkmark & \textbf{16.02} & \textbf{46.91} & \textbf{16.47}\\
            \hline
        \end{tabular}
    \end{center}
\end{table}

\section{Conclusion and Future Work} \label{sec:conclusion}
In this work, we introduced a novel, linearly scaling Transformer architecture for efficiently modeling driving scenes using 3D Gaussian distributions, trained with \emph{Temporal Gaussian Splatting} and explicit object motion modeling.
Our approach significantly outperforms previous weakly supervised methods on the Occ3D-nuScenes benchmark while maintaining substantially faster inference speeds.
Despite these advances, there are still areas for improvement.
We found that self-supervised learning (SSL) with RGB rendering did not provide benefits, suggesting that further investigation is needed in this direction.
Moreover, a more sophisticated approach to dynamic object modeling could further improve temporal consistency.
Incorporating physically informed motion priors, leveraging temporal and Gaussian dependencies as well as adapting more properties than just the mean to better represent deformable objects could enhance temporal modeling.
For future work, the sparse query mechanism and efficient attention proposed in this work could be applied to other downstream tasks, such as object detection, lane segmentation, and planning. 
This opens up exciting possibilities for using our model as a generic world representation in autonomous driving and robotics applications.

{
    \small
    \bibliographystyle{ieeenat_fullname}
    \bibliography{main}
}
 
\documentclass[journal,compsoc]{IEEEtran}
\usepackage{epsfig}
\usepackage{graphicx}
\usepackage{amsmath}
\usepackage{amssymb}
\usepackage{algorithm, algorithmic}

\usepackage{diagbox}
\usepackage{float}
\usepackage{afterpage}
\usepackage{bm}
\usepackage{subfig}

%\usepackage{tabu}
\usepackage{multirow}
\usepackage{color}
\usepackage{tablefootnote}
\usepackage{adjustbox}
\usepackage{wrapfig}

\usepackage{hyperref}       % hyperlinks
\usepackage{url}            % simple URL typesetting
\usepackage{booktabs}       % professional-quality tables
\usepackage{amsfonts}       % blackboard math symbols
\usepackage{nicefrac}       % compact symbols for 1/2, etc.
\usepackage{microtype}      % microtypography
\usepackage{times}
\usepackage{epsfig}
%\usepackage{tabu}
%\usepackage{overpic}
\usepackage{bbding}
\usepackage{etoolbox}
\usepackage{paralist}
\usepackage{ulem}
\usepackage{tikz}

\usepackage{makecell}

\usepackage{xcolor,colortbl}

% \usepackage[pagebackref=true,breaklinks=true,colorlinks,bookmarks=false]{hyperref}


\newcolumntype{Y}{p{0.5cm}<{\centering}}
\newcommand{\mc}[2]{\multicolumn{#1}{c}{#2}}
\definecolor{Gray}{gray}{0.5}
\definecolor{LightCyan}{rgb}{0.88,1,1}

\newcolumntype{a}{>{\columncolor{Gray}}c}
\newcolumntype{b}{>{\columncolor{white}}c}



\DeclareMathOperator*{\cat}{Cat}


\def\H{\operatorname{H}}
\def\I{\operatorname{I}}
\def\KL{\operatorname{KL}}


\def\etal{\textit{et al}.}
\def\ie{\textit{i.e.}}
\def\eg{\textit{e.g.}}
\def\etc{\textit{etc}}
\def\wrt{\textit{w.r.t. }}

\def\bz{\textcolor{blue}}
\def\xc{\textcolor{red}}
\newcommand{\tb}[1]{\textbf{#1}}
\newcommand{\bc}[1]{\textcolor[RGB]{192,0,0}{#1}}
\newcommand{\rc}[1]{\textcolor{blue}{#1}}
% \newcommand{\rb}[1]{\textcolor{teal}{#1}}
% \newcommand{\bb}[1]{\textcolor{blue}{#1}}
\newcommand{\bb}[1]{\textcolor[RGB]{192,0,0}{\textbf{#1}}}
\newcommand{\rb}[1]{\textcolor{blue}{\textbf{#1}}}
\newcommand{\todo}[1]{{\color{blue}{[TODO: #1]}}}

% \newcommand{\rev}[1]{\textcolor{red}{#1}}
\newcommand{\rev}[1]{{#1}}
\renewcommand{\thefootnote}{\fnsymbol{footnote}}


\normalem
\begin{document}

\title{Rotation-Adaptive Point Cloud Domain Generalization via Intricate Orientation Learning \\
—— Supplementary Material —— }

\author{{Bangzhen~Liu,~Chenxi~Zheng,~Xuemiao~Xu,~Cheng Xu,~Huaidong~Zhang, \\ and~Shengfeng~He,~\IEEEmembership{Senior Member,~IEEE}}

\thanks{ Bangzhen Liu,~Chenxi~Zheng, and~Xuemiao~Xu are with the School of Computer Science and Engineering, South China University of Technology, Guangzhou, China. E-mail: liubz.scut@gmail.com,~cszcx@mail.scut.edu.cn, and~xuemx@scut.edu.cn.}
\thanks{ Cheng Xu is with the Centre for Smart Health, The Hong Kong Polytechnic University, Hong Kong. E-mail: cschengxu@gmail.com}
\thanks{ Huaidong Zhang is with the School of Future Technology, South China University of Technology, Guangzhou, China. E-mail: huaidongz@scut.edu.cn.}
\thanks{ Shengfeng He is with the School of Computing and Information Systems, Singapore Management University, Singapore. E-mail: shengfenghe@smu.edu.sg.}
}

\markboth{IEEE Transactions on Pattern Analysis and Machine Intelligence}%
{Shell \MakeLowercase{\textit{Liu et al.}}: Rotation-Adaptive Point Cloud Domain Generalization via Intricate Orientation Learning}


\maketitle

\IEEEdisplaynontitleabstractindextext

\IEEEpeerreviewmaketitle


\section{More Experimental Results} \label{sec1}

It is worth noting that the three sub-datasets used in PointDA are all category-wise imbalanced, as shown in Table~\ref{table:dataset}, which indicates that the micro-average precision score (\textit{Acc.}) reported by previous studies is inappropriate to assess the generalizability of cross-domain classification. In the main paper, we instead report the results of PointDA in the form of the macro-average precision score (\textit{Avg.}) for a more convincing evaluation. We also report the extra evaluations in the form of \textit{Acc.} in Table~\ref{tab:pointda10_acc} for reference. Our method still outperforms all the competitors in the average metric over the six cross-domain tasks.

\begin{table}[h]
    \caption{{Number of samples for each category in PointDA~\cite{qin2019pointdan}.}}
    \vspace{-2ex}
    \label{table:dataset}
    \scriptsize
    % \begin{center}
    \setlength{\tabcolsep}{0.05cm}{
      \resizebox{0.48\textwidth}{!}{
        \begin{tabular}{c|c|c|c|c|c|c|c|c|c|c|c}
          \hline  & Tub & Bed & Shelf & Case & Chair & Lamp & Monit. & Plant & Sofa & Table & Total\\
          \hline 
          ModelNet & 106 & 515 & 572 & 200 & 889 & 124 & 465 & 240 & 680 & 392 & 4183 \\
          ShapeNet & 599 & 167 & 310 & 1076 & 4612 & 1620 & 762 & 158 & 2198 & 5876 & 17378 \\
          ScanNet & 98 & 329 & 464 & 650 & 2578 & 161 & 210 & 88 & 495 & 1037 & 6110 \\
          \hline
          \end{tabular}
      }
     }
    % \end{center}
    \vspace{-2mm}
  \end{table}
  

\noindent\textbf{Evaluation on Aligned Dataset.} {We additionally implement our method under the traditional aligned data scenario, where the rotation only happens on the z-axis. In this case, we adapt our intricate orientation mining approach to specifically identify the most intricate orientations along the z-axis. 
The comparisons with state-of-the-art 3DDG methods are shown in Table~\ref{tab:align}, where the results of competitors are directly borrowed from their papers. Our method surpasses the baselines on all six tasks, demonstrating its effectiveness. The proposed orientation-aware contrastive training enables the model to gain a more comprehensive understanding of point clouds from various challenging perspectives, thereby enhancing the generalizability of the learned features. We notice that our method is slightly inferior on M$\to$S* and S$\to$S*. Since the orientational shift is our major concern, we do not have a special design for capturing geometric information under self-occlusions. However, in this case, our method still outperforms the two 3DDG methods on three out of the six tasks, while achieving the best average accuracy. Furthermore, the experimental results also reveal the presence of rotational shifts in the aligned data scenes, demonstrating the potential of our method for solving this problem.}

\begin{table}[h] % table for OSDA setting on Office31
    \caption{Comparison of the \textit{Acc.} ($\%$) under the 3D domain generalization setting. The best records are marked in \textbf{bold}.}
    \label{tab:align} 
    \vspace{-3ex}
    \small
    \begin{center}
    \setlength{\tabcolsep}{0.1cm}{
    \resizebox{0.48\textwidth}{!}{
    \begin{tabular}{c|c|c|c|c|c|c|c}
    \hline

    Methods
    &{M$\to$S} & {M$\to$S*} & {S$\to$M} & {S$\to$S*} & {S*$\to$M} & {S*$\to$S} & {Avg}\\

    \hline
    Supervised                      &{93.9} &{78.4} &{96.2} &{78.4} &{96.2} &{93.9} &{89.5}\\
    w/o Adapt                       &{83.3} &{43.8} &{75.5} &{42.5} &{63.8} &{64.2} &{62.2}\\
    \hline
    {Metasets~\cite{huang2021metasets}} &\tb{86.0} &{52.3} &{67.3} &{42.1} &{69.8} &{69.5} &{64.5}\\
    {PDG~\cite{wei2022learning}}        &{85.6} &\tb{57.9} &{73.1} &{50.0} &{70.3} &{66.3} &{67.2}\\
    % \hline
    {Ours}                              &{83.8} &{46.0} &\tb{83.2} &{45.5} &\tb{76.4} &\tb{70.3} &\tb{67.5}\\
    \hline

    \end{tabular}

    }
    }
    \end{center}
    \vspace{-3mm}
\end{table}


\begin{table*}[h] % table for OSDA setting on Office31
    \caption{Comparison of the micro-average precision score \textit{Acc.}~($\%$) under the orientation-aware 3D domain generalization setting. The top 2 records are marked in \bc{red} and \rc{blue}.}
    \label{tab:pointda10_acc} 
    \vspace{-3ex}
    \small
    \begin{center}
    \setlength{\tabcolsep}{0.35cm}{
    \resizebox{1\textwidth}{!}{
    \begin{tabular}{c|c|c|c|c|c|c|c|c}
    \hline

    {Methods}
    &Type &{M$\to$S} & {M$\to$S*} & {S$\to$M} & {S$\to$S*} & {S*$\to$M} & {S*$\to$S} & {AVG} \\

    
    % \cline{2-19}
    % \cline{14-19}  
    % \cmidrule $\pm$ r{2-13}
    % \cmidrule $\pm$ r{14-19}
    \hline
    Supervised                            &\multirow{2}{*}{-}            &{86.6 $\pm$ 6.3}    &{69.6 $\pm$ 3.2}    &{88.4 $\pm$ 14.7}   &{69.6 $\pm$ 3.2}    &{88.4 $\pm$ 14.7}  &{86.6 $\pm$ 6.3}    &{81.5}\\
    w/o Adapt                             &          &{57.9 $\pm$ 15.1}   &{28.7 $\pm$ 5.1}    &{54.1 $\pm$ 8.6}   &{28.8 $\pm$ 4.7}    &{43.0 $\pm$ 4.8}  &{42.3 $\pm$ 6.0}    &{42.5}\\
    \hline   
    VN~\cite{Deng_2021_ICCV}              &\multirow{2}{*}{RE}             &{70.5 $\pm$ 0.0} &{30.6 $\pm$ 0.0} &{66.7 $\pm$ 0.0} &{32.0 $\pm$ 0.0} &{39.4 $\pm$ 0.0} &{44.8 $\pm$ 0.0}   &{47.3} \\  
    SVN~\cite{su2022svnet}               &            &{66.8 $\pm$ 0.6} &{32.3 $\pm$ 0.4}   &{62.0 $\pm$ 0.5} &{30.0 $\pm$ 0.6}    &{38.0 $\pm$ 0.9} &{42.2 $\pm$ 1.1}   &{45.7} \\ 
    EOMP~\cite{luo2022equivariant}        &              &{61.4 $\pm$ 0.8} &{28.1 $\pm$ 0.3}    &{60.5 $\pm$ 0.6} &{37.0 $\pm$ 0.7}  &{27.9 $\pm$ 0.8} &{37.2 $\pm$ 0.9}    &{42.0} \\
    
    \hline
    SPRIN~\cite{you2021prin}              &\multirow{5}{*}{RI}             &{68.2 $\pm$ 0.4} &{30.1 $\pm$ 0.6}   &{71.8 $\pm$ 0.6} &{30.4 $\pm$ 0.6}    &{46.8 $\pm$ 0.6} &{49.3 $\pm$ 0.5}  &{49.4}\\
    RIPCA~\cite{li2021closer}              &            &{70.3 $\pm$ 1.2} &{33.0 $\pm$ 0.7}   &{70.4 $\pm$ 0.9} &{39.1 $\pm$ 1.3}    &\rc{49.9 $\pm$ 1.6} &{50.6 $\pm$ 2.2}  &\rc{52.2}\\
    RIConv++~\cite{zhang2022riconv}        &              &{28.8 $\pm$ 0.6} &{14.2 $\pm$ 0.5}   &{55.1 $\pm$ 0.7} &{38.9 $\pm$ 0.5}    &{34.8 $\pm$ 0.7} &{47.3 $\pm$ 0.5}  &{36.5}\\
    PaRI~\cite{chen2022devil}              &             &{36.1 $\pm$ 0.0} &{29.3 $\pm$ 0.3}   &{51.8 $\pm$ 0.8} &\bc{44.8 $\pm$ 0.4}    &{43.3 $\pm$ 0.9} &{49.4 $\pm$ 0.1} &{42.5} \\
    LocoTrans~\cite{chen2024local}              &                  &\bc{76.7 $\pm$ 0.0}   &{34.5 $\pm$ 0.3}    &\rc{74.3 $\pm$ 0.4}  &\rc{43.6 $\pm$ 0.2}  &{41.6 $\pm$ 0.6} &{41.5 $\pm$ 0.0}    &{52.0}\\
    \hline
    PointDAN~\cite{qin2019pointdan}       &\multirow{5}{*}{DA}              &{59.8 $\pm$ 15.1}   &{29.5 $\pm$ 4.0}    &{55.2 $\pm$ 6.9}   &{24.0 $\pm$ 4.8}    &{38.0 $\pm$ 4.8}  &{47.4 $\pm$ 6.1}    &{42.3}\\
    DefRec~\cite{achituve2021self}        &              &{57.2 $\pm$ 13.3}   &{33.1 $\pm$ 4.6}    &{54.4 $\pm$ 8.0}    &{33.1 $\pm$ 4.4}   &{38.8 $\pm$ 6.5}  &{48.2 $\pm$ 5.7}    &{44.1}\\
    GAST~\cite{zou2021geometry}           &              &{27.7 $\pm$ 4.2}   &{7.0 $\pm$ 0.6}     &{40.8 $\pm$ 2.6}   &{5.8 $\pm$ 0.8}     &{30.7 $\pm$ 1.5}  &{50.7 $\pm$ 3.6}     &{27.1}\\
    MLSP~\cite{liang2022point}            &              &{66.5 $\pm$ 15.5}   &{32.8 $\pm$ 4.3}    &{59.7 $\pm$ 5.1}   &{30.0 $\pm$ 6.4}    &{46.3 $\pm$ 5.0}  &{52.2 $\pm$ 5.8}     &{47.9}\\
    SDDA~\cite{cardace2023self}           &              &{65.0 $\pm$ 14.5}  &\rc{37.8 $\pm$ 3.4}     &{61.4 $\pm$ 5.4}    &{40.1 $\pm$ 4.1}   &{40.7 $\pm$ 6.3}  &\rc{53.3 $\pm$ 6.4}     &{49.7}\\
    PCFEA~\cite{wang2024progressive}     &                   &{62.0 $\pm$ 13.6}   &{9.3 $\pm$ 0.2}   &{42.7 $\pm$ 8.7}   &{43.1 $\pm$ 4.0}   &{47.1 $\pm$ 4.0}   &\bc{54.0 $\pm$ 4.6}    &{43.0}    \\
    \hline
    {Metasets~\cite{huang2021metasets}}   &\multirow{3}{*}{DG}               &{53.9 $\pm$ 1.4} &\bc{40.3 $\pm$ 0.9}   &{32.2 $\pm$ 12.3}  &{33.5 $\pm$ 1.7} &{24.5 $\pm$ 4.6}    &{39.8 $\pm$ 10.0} &{37.4}\\
    {PDG~\cite{wei2022learning}}          &             &{25.4 $\pm$ 29.5}   &{21.2 $\pm$ 18.0}   &{38.4 $\pm$ 18.5}   &{8.1 $\pm$ 3.2}   &{30.3 $\pm$ 4.8}   &{29.7 $\pm$ 12.0}    &{25.5}\\
    % \hline
    {Ours}                                &              &\rb{70.8 $\pm$ 2.0}   &{37.2 $\pm$ 1.2}    &\bb{80.7 $\pm$ 0.6}   &{34.0 $\pm$ 1.0}     &\bb{50.0 $\pm$ 2.5}   &{47.1 $\pm$ 3.2}     &\bb{53.3}  \\
    \hline

    \end{tabular}

    }
    }
    \end{center}
    \vspace{-3mm}
\end{table*}


\noindent\textbf{Analysis of Hyper-parameter Sensitivity.} 
We evaluate the effects of varying $\lambda_{oc}$ and $\lambda_{ms}$, by changing the value while keeping the other frozen as 0.1. As Fig.~\ref{fig:ablation}(a) and Fig.~\ref{fig:ablation}(b) show, $\lambda_{oc}$ is insensitive across a large range, while larger $\lambda_{ms}$ may slightly decrease the performance of our model. According to the variation of performance curves, we choose $\lambda_{oc}=0.01$ and $\lambda_{ms}=0.01$ as the model setting in our main paper.
\begin{figure}[h]
    % \flushleft
    \centering
    \subfloat[Ablation of $\lambda_{oc}$]{%[b]{0.45\textwidth}
        \label{fig:plot_lambda_oc}
        \includegraphics[width=0.22\textwidth]{./resources/supp/ablation_cons_weights.pdf}
    }
    % \hspace{2mm}
    \subfloat[Ablation of $\lambda_{ms}$]{%[b]{0.45\textwidth}
        \label{fig:plot_lambda_ms}
        \includegraphics[width=0.22\textwidth]{./resources/supp/ablation_reg_weights.pdf}
    }
    % \vspace{-2mm}
    \caption{The curves of performance \wrt varying $\lambda_{oc}$ and $\lambda_{ms}$.}
    \label{fig:ablation}
    % \vspace{-5.5mm}
\end{figure} 


\noindent\textbf{Analysis of Training Stability.} 
We plot the curves of the proposed orientation consistency loss and the marginal separation loss over the training stage to demonstrate the convergence of our intricate orientational learning. As Fig.~\ref{fig:plot}(a) and Fig.~\ref{fig:plot}(b) show, all the losses gradually decrease and converge to a convincing degree. The blue curves are the orientation consistency loss, which periodically bursts every 20 epochs. This is due to the update of the intricate orientation set, which gradually adapts the model to all the intricate orientations. At the end of the training stage, the amplification tends to be stable, indicating the consistency of the object towards various rotations.


\begin{figure}[h]
    % \flushleft
    \centering
    \subfloat[M$\to$S]{%[b]{0.45\textwidth}
        \label{fig:m2s_loss}
        \includegraphics[width=0.23\textwidth]{./resources/supp/m2s_loss.pdf}
    }
    % \hspace{2mm}
    \subfloat[M$\to$S*]{%[b]{0.45\textwidth}
        \label{fig:m2ss_loss}
        \includegraphics[width=0.23\textwidth]{./resources/supp/m2ss_loss.pdf}
    }
    \vspace{-2ex}
    \caption{The training curves (\ie, $L_{cls}$, $L_{oc}$, and $L_{ms}$) on M$\to$S (a) and M$\to$S* (b).}
    \label{fig:plot}
    \vspace{-3mm}
\end{figure} 



\noindent\textbf{Analysis of Time Complexity.} {We report the computational costs of training/testing one batch of data in milliseconds for different compared methods in Table~\ref{tab:complexity}. The results are obtained by accumulating the running times within a single training/testing epoch and calculating the mean value w.r.t. one batch.} Due to the process of diversifying the intricate orientation set, our method introduces extra computational costs in the training phase. Nonetheless, our method yields the best performance among these methods while achieving the second-best inferencing speed, which is more efficient than the other RE and RI methods that require extra time-consuming modules for practical applications. 


\begin{table}[h]
    \caption{Time statistics (ms) of training/testing on one batch of data.}
    \label{tab:complexity}
    \vspace{-3ex}
    \small
    \begin{center}
    \setlength{\tabcolsep}{0.3cm}{
    \resizebox{0.4\textwidth}{!}{
      \begin{tabular}{c|c|c|c|c} 
        \hline 
        Methods & Type & Avg. & $T_{train}$& $T_{test}$ \\
        \hline
        VN~\cite{Deng_2021_ICCV} & RE & 41.3 & 808 & 27.9 \\
        SPRIN~\cite{you2021prin} & RI & 43.9 & 1551 & 370.2 \\
        RIPCA~\cite{li2021closer} & RI & 46.6 & 717 & 23.3 \\
        MLSP~\cite{liang2022point} & DA & 43.2 & 825 & 36.5 \\
        SDDA~\cite{cardace2023self} & DA & 43.1 & \bf{567} & \bf{12.0} \\
        \hline
        Ours & DG & \bf{49.6} & 2114 & 14.5 \\
        \hline
        \end{tabular}
    }
    }
    \end{center}
    \vspace{-3mm}
  \end{table}
  


\section{Extra Visualizations and Analysis} \label{sec2}

\noindent\textbf{The Learned Intricate Augmented Samples.} {In Fig.~\ref{fig:intricat_angle}, we select several point clouds and provide visualizations of how their intricate orientations evolve during training. We trained the model on ModelNet and optimized the intricate set on the testing set every 20 epochs. Each row of the point cloud sequence shows the current pose of the given point cloud augmented by its corresponding intricate orientation at that specific epoch. 
Beneath each sequence, we also visualize the distribution of predicted probabilities and the consistency of prediction over different testing orientations. 
Specifically, for each point cloud, we obtain the predicted probabilities of its 64 testing variants $P = {\{P_a|P_a = \left[p^1_a, ..., p^C_a\right]\}}^A_{a=1}$, where $A=64$ is the number of testing orientation series and $C=10$ is the number of categories. 
The visualized probabilities' distribution $P_m$ is calculated by averaging the predictions over the 64 testing rotation series, such that $P_m = \left[\frac{1}{A}\sum_{j=1}^{A}p^1_j, ..., \frac{1}{A}\sum_{j=1}^{A}p^C_j\right]$. 
To evaluate the predicted consistency, we adopt the entropy as the metric and calculate the consistency $Ent_m$ over the 64 predicted probabilities by 
\begin{equation*}
  Ent_m = \left[\frac{1}{A}\sum_{j=1}^{A}p^1_j log p^1_j, ..., \frac{1}{A}\sum_{j=1}^{A}p^C_j log p^C_j\right].
\end{equation*}
As the number of training epochs increases, both the confidence and output consistency of the model are enhanced. For samples located near the decision boundaries, such as row 6 and row 9, learning with intricate orientation mining could significantly alleviate the ambiguity of learned features, thereby producing a more robust and generalizable classifier for downstream tasks.
}




\begin{figure*}[h]
    % \flushleft
    \centering
    \subfloat[Metasets]{%[b]{0.45\textwidth}
        \label{fig:cm1_m2s}
        % \centering
        \includegraphics[width=0.24\textwidth]{./resources/supp/conf_mat_modelnet2shapenet_Metaset.pdf}
        % \vspace{10mm}
    }
    \subfloat[PDG]{%[b]{0.45\textwidth}
        \label{fig:cm2_m2s}
        % \centering
        \includegraphics[width=0.24\textwidth]{./resources/supp/conf_mat_modelnet2shapenet_PDG.pdf}
        % \vspace{10mm}
    }
    \subfloat[Ours]{%[b]{0.45\textwidth}
        \label{fig:cm3_m2s}
        % \centering
        \includegraphics[width=0.24\textwidth]{./resources/supp/conf_mat_modelnet2shapenet.pdf}
        % \vspace{10mm}
    }
    \vspace{-2ex}
    \caption{The confusion matrices of Metaset, PDG, and our method on M$\to$S. Zoom in for details.}
    \vspace{-3ex}
    \label{fig:visualization_m2s}
\end{figure*} 
% \vspace{-4mm}
\begin{figure*}[h]
    % \flushleft
    \centering
    \subfloat[Metasets]{%[b]{0.45\textwidth}
        \label{fig:cm1_s2m}
        % \centering
        \includegraphics[width=0.24\textwidth]{./resources/supp/conf_mat_shapenet2modelnet_Metaset.pdf}
        % \vspace{10mm}
    }
    \subfloat[PDG]{%[b]{0.45\textwidth}
        \label{fig:cm2_s2m}
        % \centering
        \includegraphics[width=0.24\textwidth]{./resources/supp/conf_mat_shapenet2modelnet_PDG.pdf}
        % \vspace{10mm}
    }
    \subfloat[Ours]{%[b]{0.45\textwidth}
        \label{fig:cm3_s2m}
        % \centering
        \includegraphics[width=0.24\textwidth]{./resources/supp/conf_mat_shapenet2modelnet.pdf}
        % \vspace{10mm}
    }
    \vspace{-2ex}
    \caption{The confusion matrices of Metaset, PDG, and our method on S$\to$M. Zoom in for details.}
    \label{fig:visualization_s2m}
\end{figure*} 


\begin{figure*}[h]
    % \flushleft
    \centering
    \subfloat{\label{fig:0}
        \begin{minipage}[b]{1.0\textwidth}\centering
            \includegraphics[width=0.95\textwidth]{./resources/supp/underline.pdf} 
            \\
            \includegraphics[width=0.9\textwidth]{./resources/supp/all_title.pdf} 
            \\
            \includegraphics[width=0.9\textwidth]{./resources/supp/all_cat_0_cropped.pdf} 
            \\
            \vspace{-3mm}
            \includegraphics[width=1.0\textwidth]{./resources/supp/all_predictions_cat_0_cropped.pdf}
        \end{minipage}
    }\vspace{-3mm}

    \subfloat{\label{fig:1}
        \begin{minipage}[b]{1.0\textwidth}\centering
            \includegraphics[width=0.9\textwidth]{./resources/supp/all_cat_1_cropped.pdf} 
            \\
            \vspace{-3mm}
            \includegraphics[width=1.0\textwidth]{./resources/supp/all_predictions_cat_1_cropped.pdf}
        \end{minipage}
    }\vspace{-3mm}

    \subfloat{\label{fig:2}
        \begin{minipage}[b]{1.0\textwidth}\centering
            \includegraphics[width=0.9\textwidth]{./resources/supp/all_cat_2_cropped.pdf} 
            \\
            \vspace{-3mm}
            \includegraphics[width=1.0\textwidth]{./resources/supp/all_predictions_cat_2_cropped.pdf}
        \end{minipage}
    }\vspace{-3mm}

    \subfloat{\label{fig:4}
        \begin{minipage}[b]{1.0\textwidth}\centering
            \includegraphics[width=0.9\textwidth]{./resources/supp/all_cat_3_cropped.pdf} 
            \\
            \vspace{-3mm}
            \includegraphics[width=1.0\textwidth]{./resources/supp/all_predictions_cat_3_cropped.pdf}
        \end{minipage}
    }\vspace{-3mm}

    \subfloat{\label{fig:6}
        \begin{minipage}[b]{1.0\textwidth}\centering
            \includegraphics[width=0.9\textwidth]{./resources/supp/all_cat_4_cropped.pdf} 
            \\
            \vspace{-3mm}
            \includegraphics[width=1.0\textwidth]{./resources/supp/all_predictions_cat_4_cropped.pdf}
        \end{minipage}
    }\vspace{-3mm}

    \subfloat{\label{fig:7}
        \begin{minipage}[b]{1.0\textwidth}\centering
            \includegraphics[width=0.9\textwidth]{./resources/supp/all_cat_5_cropped.pdf} 
            \\
            \vspace{-3mm}
            \includegraphics[width=1.0\textwidth]{./resources/supp/all_predictions_cat_5_cropped.pdf}
        \end{minipage}
    }\vspace{-3mm}

    \subfloat{\label{fig:8}
        \begin{minipage}[b]{1.0\textwidth}\centering
            \includegraphics[width=0.9\textwidth]{./resources/supp/all_cat_6_cropped.pdf} 
            \\
            \vspace{-3mm}
            \includegraphics[width=1.0\textwidth]{./resources/supp/all_predictions_cat_6_cropped.pdf}
        \end{minipage}
    }

    \caption{Visualization of the learned intricate orientation series on ModelNet (M). Each row of point cloud sequence records the transformation of the point cloud's poses after augmention by its corresponding intricate rotational angle during the training procedure. The interval of recording is 20 epoch. The statistic measurements at every records, including the predicted probability and rotational consistency, are presented beneath each point cloud sequence. The category which the current point cloud belongs to is marked in \bc{red}.}
    \label{fig:intricat_angle}
    % \vspace{-6mm}
\end{figure*}



\noindent\textbf{Confusion Matrices.} {we provide the evaluation results for Metasets~\cite{huang2021metasets}, PDG~\cite{wei2022learning}, and our method in the form of confusion matrix on the target domain. ShapeNet is a dataset whose samples are highly imbalanced across different categories, while ModelNet is much more balanced. The confusion matrices of the three approaches are shown in  Fig.~\ref{fig:visualization_m2s} and Fig.~\ref{fig:visualization_s2m}. Compared with the other two 3D domain generalization methods, our method has more compact confusion matrices under the orientation shift. For M$\to$S, both our method and Metaset are separated relatively well while PDG has much inaccurate classification on class "Plant". The inner reason is that the part-based feature utilized by PDG may encounter confusing local expressions, such as the plane of the table and the bottom of a potted plant. For S$\to$M, our method achieves more balanced and concise results. We observe that the sample of class "monitor" is much easier to misclassify into "bed" due to the similar plane structure of their surface. Similar trends happen for the categories "table" and "cabinet", which have less discriminative features in the view of shape. }

{In summary, the single shape cannot serve as a discriminative representation in some cases. This is the limitation of shape representation under the orientation shift since there are a lot of objects whose shapes are similar but belong to different categories. In this case, extra visual (\eg, texture or color), linguistic information, or spatial cues are important to provide complement representation, which may benefit the problem of cross-domain generalization under orientation shift. We will plan to investigate the function of these features in our future work.  }

\section{Gradient of the rotation parameters} \label{sec3}
In this section, we provide detailed calculations about the optimizable parameters $\Theta$ concerning a given model $F$. Considering the objective of optimizing $\Theta$ within a standard classification task, we have the following objective:
\begin{equation}
  \hat{\Theta} = \mathop{\arg\max_{\Theta}}L(w_{opt}, \hat{P}, y), 
\end{equation}
where $w_{opt}$ is the freeze parameter of $F$, $(\hat{P}, y)$ are the augmented point cloud and label:
\begin{equation}
  \begin{split}
    ~\hat{P}&=f(\hat{\Theta}, P)  \\
    &=R_{\theta_{x}}\cdot R_{\theta_{y}}\cdot R_{\theta_{z}}\cdot P.
  \end{split}
\end{equation}
According to the chain rules, the gradient of $\hat{\Theta}$ is calculated by:
\begin{equation}
  \begin{split}
  \frac{\partial L}{\partial \hat{\Theta}} &= \frac{\partial L}{\partial \hat{P}} \frac{\partial \hat{P}}{\partial \hat{\Theta}} \\
  &=\frac{\partial L}{\partial \hat{P}}
  \left(
  \frac{\partial R_{\theta_{x}}}{\partial \theta_{x}}
  R_{\theta_{y}}
  R_{\theta_{z}} \quad
  R_{\theta_{x}}
  \frac{\partial R_{\theta_{y}}}{\partial \theta_{y}}
  R_{\theta_{z}} \quad
  R_{\theta_{x}}
  R_{\theta_{y}}
  \frac{\partial R_{\theta_{z}}}{\partial \theta_{z}}
  \right)P,
  \end{split}
\end{equation}
where 
\begin{equation}
  \begin{split}
R_{\theta_{x}} = 
\begin{pmatrix}
  1 & 0 & 0 \\
  0 & \cos\theta_{x} & -\sin\theta_{x} \\
  0 & \sin\theta_{x} & \cos\theta_{x} 
\end{pmatrix},
\\
R_{\theta_{y}} = 
\begin{pmatrix}
  \cos\theta_{y} & 0 & \sin\theta_{y} \\
  0 & 1 & 0 \\
  -\sin\theta_{y} & 0 & \cos\theta_{y}
\end{pmatrix},
\\
R_{\theta_{z}} = 
\begin{pmatrix}
  \cos\theta_{z} & -\sin\theta_{z} & 0  \\
  \sin\theta_{z} & \cos\theta_{z} & 0 \\
  0 & 0 & 0
\end{pmatrix},
\end{split}
\end{equation}
and 
\begin{equation}
    \begin{split}
\frac{\partial R_{\theta_{x}}}{\partial \theta_{x}} = 
\begin{pmatrix}
  0 & 0 & 0 \\
  0 & -\sin\theta_{x} & -\cos\theta_{x} \\
  0 & \cos\theta_{x} & -\sin\theta_{x} 
\end{pmatrix}, \\
\frac{\partial R_{\theta_{y}}}{\partial \theta_{y}} = 
\begin{pmatrix}
  -\sin\theta_{y} & 0 & \cos\theta_{y} \\
  0 & 0 & 0 \\
  -\cos\theta_{y} & 0 & -\sin\theta_{y}
\end{pmatrix}, \\
\frac{\partial R_{\theta_{z}}}{\partial \theta_{z}} = 
\begin{pmatrix}
  -\sin\theta_{z} & -\cos\theta_{z} & 0  \\
  \cos\theta_{z} & -\sin\theta_{z} & 0 \\
  0 & 0 & 1
\end{pmatrix}.
\end{split}
\end{equation}



\section{Theoretical Analysis for Rotation-Adaptive Point Cloud Domain Generalization} \label{sec4}

In this section, we provide theoretical proof demonstrating how orientational consistency functions to bridge the domain gap, analyzed from the perspective of mutual information reduction.

Let $X\!=\!(U, V)$ represent a 3D point cloud, where $U$ corresponds to orientation-dependent variables and $V$ to orientation-independent variables. In our case, we assume that the ranges of $U$ and $V$ remain consistent across domains.
For $X_s\!\sim\!p_\mathrm{src}(x)$, where $p_\mathrm{src}(x)$ denotes the source domain data distribution, the marginal distributions \wrt $U_s$ and $V_s$ are expressed by:
\begin{equation}
 p_\mathrm{src}(u)=\int p_\mathrm{src}(x) \mathrm{d}v, \quad p_\mathrm{src}(v)=\int p_\mathrm{src}(x) \mathrm{d}u.
\end{equation}
Considering the data distribution $X_a\!\sim\!p_\mathrm{aug}(x)$ after augmentation, where each sample is assumed to be uniformly sampled \wrt orientations, the marginal distributions \wrt $U_a$ and $V_a$ are given by:
\begin{equation}
 p_\mathrm{aug}(u)=\mathcal{U}(\mathcal{D}_{U_a}), \quad p_\mathrm{aug}(v)=p_\mathrm{src}(v),
\end{equation}
where $\mathcal{U}(\cdot)$ denotes a uniform distribution over the measurable domain $\mathcal{D}_{U_a}$ of ${U_a}$. For simplicity, the subscript of $U_a$ in $\mathcal{D}_{U_a}$ is omitted without causing ambiguity in the subsequent analysis. In this work, we adopt the proposed orientation-aware contrastive learning framework to approximately achieve this, where ${U_a}$ is represented by Euler angles and $\mathcal{D}_{U}:=[-\pi, \pi)^3$. 



Based on the definition of joint entropy, the entropy of $p_\mathrm{src}(x)$ can be expressed in terms of its marginal entropies \wrt $U_s$ and $V_s$, along with an additional term presenting the mutual information between these two components:
\begin{equation}
\begin{aligned}
    \H(X_s) 
 =& \H(U_s)+\H(V_s)-\I(U_s;V_s) \\
 =& \mathbb{E}_{u\sim p_\mathrm{src}(u)}[-\log p_\mathrm{src}(u)] + \mathbb{E}_{v\sim p_\mathrm{src}(v)}[-\log p_\mathrm{src}(v)] \\
    &- \mathbb{E}_{x\sim p_\mathrm{src}(x)}\log \frac{p_\mathrm{src}(x)}{p_\mathrm{src}(u)p_\mathrm{src}(v)},
\end{aligned}
\end{equation}
where $\I(U_s;V_s)$ represents the mutual information between $U_s$ and $V_s$ in $p_\mathrm{src}(x)$. 
Since $p_\mathrm{aug}(u)$ follows a uniform distribution and $U_a$ and $V_a$ of $p_\mathrm{aug}(x)$ are independent under this setting, the entropy of $p_\mathrm{aug}(x)$ is given by $\I_\mathrm{aug}(U_a;V_a)\!=\!0$, and the entropy of $p_\mathrm{aug}(u)$ corresponds to the measure of $\mathcal{D}_{U}$, denoted as $m(\mathcal{D}_{U})$. 
Thus, the entropy of $p_\mathrm{aug}(x)$ can be simplified as follows:
\begin{equation}
\begin{aligned}
    \H(X_a) &= \H(U_a)+\H(V_a) \\
    &= \log m(\mathcal{D}_{U}) + \mathbb{E}_{v\sim p_\mathrm{aug}(v)}[-\log p_\mathrm{aug}(v)],
\end{aligned}
\end{equation}
where $m(\mathcal{D}_{U})\!=\!(2\pi)^3$ in our case.

We use the KL divergence to quantify the distributional shift between the source and the target distribution.
For any $X_t\!\sim\!p_\mathrm{tgt}(x)$, where $p_\mathrm{tgt}(x)$ represents the target domain distribution, the KL divergence between $p_\mathrm{tgt}(x)$ and $p_\mathrm{src}(x)$ (or $p_\mathrm{aug}(x)$) can be computed once the cross-entropy between them is known. 
However, directly calculating the cross-entropy between $p_\mathrm{tgt}(x)$ and $p_\mathrm{src}(x)$ (or $p_\mathrm{aug}(x)$) is intractable, and it is often treated as an optimization objective to minimize. Notably, the cross-entropy between $p_\mathrm{tgt}(x)$ and $p_\mathrm{src}(x)$ (or $p_\mathrm{aug}(x)$) shares the same upper bound, as the samples $X_s$, $X_a$, and $X_t$ all share the same dimensionality:
\begin{equation}
    \sup_{p_\mathrm{src}}{\H(p_\mathrm{tgt}, p_\mathrm{src})} = \sup_{p_\mathrm{aug}}{\H(p_\mathrm{tgt}, p_\mathrm{aug})} = \log ({m(\mathcal{D}_U) \times m(\mathcal{D}_V)}).
\end{equation}
Here, $\mathcal{D}_V$ is the measurable domain of $V_s$, $V_a$, and $V_t$.
It is straightforward to prove that $\H_\mathrm{aug}(X_a) > \H_\mathrm{src}(X_s)$, as the mutual information is non-negative and entropy reaches its upper bound when the distribution is uniform.
Therefore, the relation between the upper bound of the KL divergence from $p_\mathrm{tgt}(x)$ to $p_\mathrm{src}(x)$ and from $p_\mathrm{tgt}(x)$ to $p_\mathrm{aug}(x)$ can be expressed as:
\begin{equation}
\begin{aligned}
    \sup_{p_\mathrm{src}}{\KL(p_\mathrm{tgt}||p_\mathrm{src})} &= \sup_{p_\mathrm{src}}{\H(p_\mathrm{tgt};p_\mathrm{src})} - \sup_{p_\mathrm{src}}{\H(X_\mathrm{s})} \\
    &> \sup_{p_\mathrm{aug}}{\H(p_\mathrm{tgt};p_\mathrm{aug})} - \sup_{p_\mathrm{aug}}{\H(X_\mathrm{a})} \\
    &= \sup_{p_\mathrm{aug}}{\KL(p_\mathrm{tgt}||p_\mathrm{aug})}. \label{eq:ieq}
\end{aligned}
\end{equation}
As revealed in Eq.~\ref{eq:ieq}, the upper bound of $\KL(p_\mathrm{tgt}||p_\mathrm{aug})$ is consistently lower than $\KL(p_\mathrm{tgt}||p_\mathrm{src})$, demonstrating the effectiveness of orientation invariance in reducing the domain shift under the disturbance of varying rotations. Consequently, the final upper bound of $\KL(p_\mathrm{tgt}||p_\mathrm{aug})$ is formally given as follows:
\begin{equation}
    \sup_{p_\mathrm{aug}}{\KL(p_\mathrm{tgt}||p_\mathrm{aug})} = \log m(\mathcal{D}_V) - \mathbb{E}_{v\sim p_\mathrm{aug}(v)}[-\log p(v)].
\end{equation}


\section{Limitation and Future Work} 
Although our method shows commendable advantages in handling cross-domain orientational shifts, it faces challenges with other complex types of domain shifts, such as heavy occlusions. This is because our framework does not offer an explicit design for tackling these domain shifts. Addressing this limitation, possibly through constructing a more powerful and versatile feature space resilient to multiple domain shifts via self-supervised pre-training, is a goal for future work.

{\small
\bibliographystyle{ieee_fullname}
\bibliography{egbib}
}

\end{document} 


\end{document}
