\section{Related Work}
\subsection{Wearable Eating Behavior Monitoring}
	Eating is essential for the health of human beings. The monitoring of eating is thus of great importance to either health management of the general population or eating pathology for patients. Through its long history of thousands of years, the monitoring has been mainly based on self-reporting, which is subjective and with recall bias \cite{basiotis1987number}. This brings forward automatic methods especially those using wearable devices. 
	%
	A wide range of efforts has been made in the last two decades \cite{amft2010wearable,dong2012new,farooq2014novel,farooq2015comparative,liu2012intelligent,passler2012food,sazonov2008non,sun2014ebutton,vu2017wearable}. Various sensors have been applied, such as acoustic sensors using microphones \cite{amft2010wearable}, visual sensors using cameras \cite{sun2014ebutton}, wrist or arm motion sensor using accelerometers \cite{dong2012new}, physiological sensors using EMG \cite{farooq2014novel}, piezoelectric sensors \cite{farooq2015comparative}, and multi-sensor by combining these sensors \cite{liu2012intelligent}.
	
	The choice of sensors is in fact a choice of balance between the comfort and reliability. For example, microphones have been widely used for monitoring by recognizing the sound of chewing and swallowing. The outer-ear \cite{passler2012food} or neck \cite{sazonov2008non} microphones might be comfort to wear, but less reliable due to the disturbance of the environmental noise. The inner-ear microphones are reliable but less user friendly.
	%
	In addition, in most applications (especially when users are patients), more important is that whether the sensors can provide pathological evidence for more in-depth analysis of the users' condition. The EMG or piezoelectric sensors are good in this regard, because they are connected to the inner processes (e.g, muscle movement and tension) of the body. However, these sensors are less acceptable to users as they have to be installed around the neck.
	
	In this paper, we study the possibility of using the 24-hour ECG sensors. As a non-invasive sensor, it is comparatively comfort, and most importantly, it has been widely used in collecting both the
	physiological and psychological evidence \cite{serhani2020ecg}, because it is associated to the heart which is one of the most important organs of our body and determines both the physical and mental health.
	
	
	\subsection{Deep Learning for ECG Analysis}
	In terms of learning methods employed for eating monitoring, although neural network-based models have also been explored by a few researchers \cite{farooq2013comparative}, conventional models, such as HMM \cite{bi2015autodietary}, SVM \cite{lopez2010detection}, random forests \cite{fontana2013estimation}, are still dominating in this area, because it is straightforward to consider the monitoring as a 1D signal processing (detection) problem. Early work in \cite{farooq2013comparative} has used a 3-layer neural network 
	which outperforms SVM, but it remains as a conventional shallow network which has not leveraged the power of deep learning.
	
	Beyond the task of eating behavior monitoring, deep learning has already been adopted in a number of studies for ECG analysis \cite{al2018convolutional,andreotti2017comparing,fan2018multiscaled,hannun2019cardiologist,he2018automatic,naz2021ecg,sarkar2020self,ullah2020classification}. A majority of them are for arrhythmia classification \cite{andreotti2017comparing,hannun2019cardiologist} together with a few for emotion recognition \cite{sarkar2020self} or atrial fibrillation detection \cite{fan2018multiscaled}. Main-stream methods are either to feed the 1D ECG signals to the 1D CNNs (ResNet or VGG) directly \cite{andreotti2017comparing,hannun2019cardiologist,fan2018multiscaled} or to adopt 2D CNNs after transforming the 1D inputs into 2D spectrograms or simply reshaping the signals into 2D \cite{al2018convolutional,he2018automatic,naz2021ecg,ullah2020classification}. 
	%
	None of these methods has modeled the periodic nature of the signal. Furthermore, 
	none of these methods has modeled the high order relation (inter-wave comparison) specifically, and thus is limited to provide explicit evidence for cardiac analysis. 
	%
	In this paper, we will address this problem with collocative learning which is capable of period sensing and evidence backtracking. This has been done through a periodic attention coaching mechanism and a CAM-based decoding process for backtracking and human comprehensible representation generation.
	
	\subsection{Attention and CAM Methods}
	Attention mechanisms have been employed in deep learning for weighting the feature maps so that the highly important regions can stand out from the feed-forward inference \cite{hu2018squeeze,wang2018non,woo2018cbam}. The attention usually appears as a weighting mask which has been learned through back-propagation adaptively, meaning the mask formation is data-driven, network dependent, and without human guidance. The coached attention gates (Section~\ref{sec:cag_intro}) we proposed is different from standard attention mechanisms in a way that we define a parameterized template mask with embedded periodic patterns and allow the back-propagation to learn the parameters only rather than the mask directly. It is a combination of prior knowledge injection and adaptive learning.
	
	CAM-based methods are often used to generate saliency maps which indicate which parts of the inputs have played a more significant role in the inference than others \cite{chattopadhay2018grad,selvaraju2017grad,zhou2016learning}. However, the saliency maps are usually with an arbitrary shape for providing indicative rather than explicit evidence (e.g., exact localization of waves or wave pairs). In this paper, we have built a decoding scheme (see Section~\ref{sec:evidence_backtracking}) on top of the Grad-CAM \cite{selvaraju2017grad}, which is able to convert the indicative evidence into explicit evidence for generating human comprehensive representations.
	
	In terms of collocative learning, this paper is also related to \cite{wei2021deep} in which the prototype framework of collocative learning has been proposed. In this paper, we have followed the thread but have implemented the learning in a different way (see Section~\ref{sec:relation_model}). Most importantly, this framework in this paper has been designed specifically for the ECG-based eating behavior monitoring which is a completely different task from the Immunofixation Electrophoresis analysis in \cite{wei2021deep}.
	
	%of building collocative learning with three steps of relation transformation, attention regulation, and decoding, but the implementation of each step is different.