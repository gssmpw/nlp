\documentclass[conference]{IEEEtran}
\usepackage{times}

% numbers option provides compact numerical references in the text. 
\usepackage{amsmath}
\usepackage{amssymb}
\usepackage[numbers]{natbib}
\usepackage{multicol}
\usepackage[bookmarks=true]{hyperref}
\usepackage{algorithm}
\usepackage{algpseudocode}
\usepackage{graphicx}
\usepackage[utf8]{inputenc}
\usepackage[T1]{fontenc}
\usepackage{booktabs}
\usepackage{caption}
\usepackage[subrefformat=parens]{subcaption}
\usepackage{adjustbox}


\pdfinfo{
   % /Author (Anonymous)
   /Author (Ryo Takizawa, Izumi Karino, Koki Nakagawa, Yoshiyuki Ohmura, Yasuo Kuniyoshi)
   /Title  (Enhancing Reusability of Learned Skills for Robot Manipulation via Gaze and Bottleneck)
   /CreationDate (D:20241227165300)
   /Subject (Imitation Learning for Robot Manipulation)
   /Keywords (Robot Manipulation, Deep Imitation Learning, Generalization, Gaze, Eye-hand Coordination, Autonomous Agent)
}

\begin{document}

% paper title
\title{Enhancing Reusability of Learned Skills for Robot Manipulation via Gaze and Bottleneck}

% % You will get a Paper-ID when submitting a pdf file to the conference system
% \author{Author Names Omitted for Anonymous Review. Paper-ID 72}

\author{
\authorblockN{Ryo Takizawa\authorrefmark{1},
Izumi Karino,
Koki Nakagawa, 
Yoshiyuki Ohmura and
Yasuo Kuniyoshi}
\authorblockA{The University of Tokyo}
\authorblockA{\authorrefmark{1}Indicates Corresponding Author}
}




\maketitle

    


\begin{abstract}
Autonomous agents capable of diverse object manipulations should be able to acquire a wide range of manipulation skills with high reusability. Although advances in deep learning have made it increasingly feasible to replicate the dexterity of human teleoperation in robots, generalizing these acquired skills to previously unseen scenarios remains a significant challenge.
In this study, we propose a novel algorithm, Gaze-based Bottleneck-aware Robot Manipulation (GazeBot), which enables high reusability of the learned motions even when the object positions and end-effector poses differ from those in the provided demonstrations. By leveraging gaze information and motion bottlenecks—both crucial features for object manipulation—GazeBot achieves high generalization performance compared with state-of-the-art imitation learning methods, without sacrificing its dexterity and reactivity. Furthermore, the training process of GazeBot is entirely data-driven once a demonstration dataset with gaze data is provided.
Videos and code are available at \href{https://crumbyrobotics.github.io/gazebot}{https://crumbyrobotics.github.io/gazebot}.
\end{abstract}






\IEEEpeerreviewmaketitle









\section{Introduction} 
Recent advancements utilizing powerful neural networks such as Transformers have made deep imitation learning increasingly capable of reproducing dexterity to a certain extent \cite{Zhao2023, Chi2023, Kim2024}.
However, significant issues persist regarding their generalization capabilities. 
Although generalization in object manipulation occurs at multiple levels, even the most fundamental aspects, such as changes in object position and the end-effector pose, are known to cause drastic reductions in success rates with variations of just a few centimeters \cite{Cheng2024}.
For instance, ACT \cite{Zhao2023}, a model recognized for its strong dexterous capabilities, has only been validated with objects placed on white tape with an accuracy of approximately 5 cm. 
Although ACT demonstrated high success rates under these specific conditions in our experiments, it was unable to reach objects placed in unseen positions (Figure \ref{fig:overview}), highlighting the poor generalization capabilities acquired through this method.

\begin{figure}
    \centering
    \includegraphics[width=0.95\linewidth]{figures/overview2.pdf}
    {\setlength{\belowcaptionskip}{-2pt}
    \caption{\textbf{GazeBot} achieves high reusability of learned skills for unseen object positions and end-effector poses. Demonstrations collected within restricted ranges of object positions and end-effector poses, and then the success rate is evaluated for in-distribution (\textbf{ID}) cases within these ranges and out-of-distribution (\textbf{OOD}) cases outside them.}
    \label{fig:overview}}
\end{figure}

The purpose of this study is to achieve imitation that can accurately perform demonstrated object manipulations under various object positions and initial end-effector poses. 
Owing to the limited generalization capability of conventional imitation learning, exhaustive demonstration collection is currently required to ensure that the robot behaves correctly under various object positions and end-effector poses \cite{Brohan2022, Kim2024b, Moo2024}. 
From a short-term perspective such as automating a certain task, it might be sufficient to collect data until the desired capability is achieved.
However, for enabling robots to autonomously and adaptively expand human-aligned behaviors in unknown environments, it is essential that actions learned from limited demonstrations can be reused in slightly different situations.
To address this need, we propose Gaze-based Bottleneck-aware Robot Manipulation (\textbf{GazeBot}), an object manipulation imitation method that enables the reuse of acquired skills even with object positions and end-effector poses not included in the provided demonstrations.
As illustrated in Figure \ref{fig:overview}, the object positions and initial end-effector poses used in the demonstrations are restricted to a designated region to evaluate the generalization performance.
In this setting, GazeBot can accurately perform the task not only under in-distribution (\textbf{ID}) conditions (i.e., within the designated region) but also under out-of-distribution (\textbf{OOD}) conditions (i.e., outside the designated region).

To reuse skills learned within the ID domain in OOD situations, it is necessary to (1) establish an object representation that is robust to changes in object position and (2) develop an action policy architecture capable of accurate control under previously unseen end-effector poses. 
To achieve this, we draw inspiration from human object manipulation, where \textbf{gaze} on the target object not only provides a visual representation independent of the object's absolute position but also exhibits strong gaze–hand coordination during end-effector movements such as reaching \cite{Johansson2001, Hayhoe2003}. 
Concretely, we first represent the entire field of view using a 3D point cloud and then crop a cubic region around the gaze position---referred to as \textbf{gaze-centered point cloud}---from the entire point cloud.
By using this gaze-centered point cloud as input to the action prediction, we realize an object representation that is robust to variations in object position.
Next, based on the action predictivity in the gaze-centered point cloud, we perform a data-driven action segmentation of the overall manipulation into (1) a reaching motion to the vicinity of the gaze position and (2) a gaze-centered dexterous action. 
We define the temporal boundary between these motions as a \textbf{bottleneck}, and predict the \textbf{bottleneck pose}, the end-effector pose at this bottleneck, from the 3D gaze position and gaze-centered point cloud. 
This approach allows accurate prediction of the bottleneck pose even for unseen object positions, and this prediction is independent of the current end-effector pose.
Consequently, GazeBot can handle various object positions and end-effector poses by first executing a rough reaching motion to the bottleneck pose, then performing a gaze-centered dexterous action using only the gaze-centered point cloud. This approach enables the reuse of the learned motion in unseen conditions. 
Here, reaching the bottleneck is achieved by generating an end-effector trajectory that smoothly connects the current end-effector pose and the bottleneck pose using a first-order Bézier curve, which provides sufficient expressive power for the reaching motion while avoiding unnecessary complexity that could compromise the model's generalization. 
Furthermore, GazeBot updates all actions at every step, and a fully parametric method based on a Transformer is used to directly output subsequent end-effector poses for the gaze-centered dexterous actions. 
This approach ensures that dexterity and reactivity are not sacrificed in the pursuit of generalization.

In summary, our key results and contributions are as follows:
\begin{itemize}
    \item We propose a \textbf{gaze-centered point cloud} as a visual representation that is invariant to the object position.
    \item We introduce a method for data-driven action segmentation based on \textbf{bottleneck} determination that clarifies which actions can be reused.
    \item We implement a reaching method to the bottleneck pose that remains accurate even with unseen object positions and end-effector poses.
    \item We develop a novel imitation learning model, \textbf{GazeBot}, which demonstrates superior generalization performance compared with state-of-the-art models by leveraging gaze and bottlenecks.
\end{itemize}













\section{Related Work}
\textbf{Gaze-based Object Manipulation.} 
Some works have previously proposed imitation learning methods using gaze data collected from a remote human operator during teleoperated demonstrations for action prediction \cite{Kim2020, Kim2021, Kim2024}. These gaze-based methods have previously exhibited advantages such as enhanced robustness by disregarding task-irrelevant objects \cite{Kim2020} and improved dexterity by focusing on task-relevant regions of visual inputs \cite{Kim2021, Kim2024}. 
However, because these methods rely on image cropping for gaze-centered images, they are susceptible to visual variations caused by changes in object position.
In this study, we employ a gaze-centered point cloud that is robust to positional changes, and our method further improves the reusability of acquired skills for unseen object positions and end-effector poses, in addition to the conventional benefits of gaze.

\textbf{Data-driven Action Segmentation.} 
Segmenting actions into reaching motions and dexterous actions has been proposed to improve dexterity \cite{Kim2021}, increase success rates for long-horizon tasks \cite{Kim2024, Belkhale2023, Sundaresan2024}, and enable high generalization capabilities \cite{Johns2021, Sundaresan2024}. 
Kim et al. proposed data-driven segmentation methods based on end-effector velocity \cite{Kim2021} or the visibility of the end-effector within a gaze-centered image \cite{Kim2024}. However, these approaches often fail in tasks where high dexterity is not required or when the end-effector is not visible during dexterous actions, such as manipulating with a long stick.
In contrast, our approach segments motions at bottlenecks, which are determined based on action predictivity in the gaze-centered point cloud. 
This action predictivity-based approach offers a more general and data-driven segmentation scheme compared with these previous methods.

\textbf{Generalization to Out-of-Distribution.} 
Generalization to out-of-distribution scenarios is required across diverse levels and factors \cite{Bharadhwaj2023}, including adaptation to a variety of object poses and unseen objects in the same category \cite{Simeonov2021, Wang2024, Gao2024}, to changes in the environment such as varying backgrounds, camera positions, or distractor objects \cite{Zhu2023, Yu2023, Xie2024}, and to entirely novel objects and tasks \cite{Shen2023, Stone2023}. 
However, to the best of our knowledge, there have been no studies examining enhanced generalization to out-of-distribution object positions or end-effector poses, which is the main focus of this study.
In other words, our proposed GazeBot is the first method in the general deep imitation learning framework to demonstrate the high reusability of imitated skills even under unseen object positions and end-effector poses.
Although several methods exhibit similarities with GazeBot, they have not successfully demonstrated such high reusability, primarily owing to issues in the design of the action policy. 
Hydra \cite{Belkhale2023} and SPHINX \cite{Sundaresan2024}, for instance, segment actions into a reaching phase and a dexterous action phase, using sparse action representations similar to a bottleneck pose for the reaching motion. 
However, in contrast to GazeBot, these sparse representations are directly estimated by a neural network from entire images and the end-effector poses, and dexterous action prediction also relies on absolute information such as the end-effector poses or entire images. As we will see in Section \ref{sec:experiments}, these design changes hinder the accurate extrapolation of the reaching motions and reduce the reusability of the learned dexterous actions when faced with unseen object positions or end-effector poses.













\section{Gaze-based Bottleneck-aware Robot Manipulation}
\label{sec:method}
In this section, we propose a novel policy model named GazeBot that improves the reusability of learned skills while retaining dexterity and reactivity. 
We begin by explaining how gaze and bottlenecks contribute to the enhanced reusability of imitated motions (Section \ref{sec:gaze-bottleneck}). We then detail the data-driven action segmentation based on gaze and bottlenecks for decomposing demonstrations (Section \ref{sec:segmentations}). Finally, we describe the design of the policy model (Section \ref{sec:model}).

\subsection{Gaze and Bottleneck}
\label{sec:gaze-bottleneck}

\subsubsection{Gaze-based Visual Representation}
Unlike a standard camera, human vision does not uniformly perceive the entire visual field, but instead distinguishes between a high-resolution foveal (central) region and a lower-resolution peripheral region \cite{Paillard1996}. This foveal region can be seen as a 3D attention mechanism that selects a specific portion of space for detailed processing. 
Kim et al. previously proposed an imitation learning method that implements a foveal vision system for robot manipulation \cite{Kim2020}. In that approach, gaze data of a remote human operator are measured during teleoperated demonstrations, and a portion of the image input is cropped around the operator’s gaze position to simulate human-like gaze-based vision. During the inference phase, a gaze prediction model trained on the measured gaze data provides the online gaze control.

However, the conventional image-cropping approach lacks 3D-awareness, causing substantial changes in visual representation when the object position is varied (Figure \ref{fig:gaze-vision}).
To address this, we extract a region of the point cloud around the 3D gaze coordinates rather than cropping a 2D image. Here, we use stereo vision to estimate depth and then convert the pixel coordinates of the gaze to 3D coordinates. As shown in Figure \ref{fig:gaze-vision}b, this gaze-centered point cloud reduces the sensitivity to changes in object positions and end-effector poses compared with conventional 2D cropping methods (Figure \ref{fig:gaze-vision}a), thereby enabling neural networks to make predictions independent of object position and end-effector pose.

\begin{figure}
    \centering
    \includegraphics[width=0.85\linewidth]{figures/gaze-based-vision.pdf}
    \caption{Although both the left and right scenes represent a similar state, their positions on the table differ. In conventional gaze-centered image (a), which lacks 3D-awareness, the scenes appear substantially different, whereas in our proposed \textbf{gaze-centered point cloud} (b), their underlying three-dimensional structure is captured as similar.}
    \label{fig:gaze-vision}
\end{figure}

\subsubsection{Bottleneck-aware Action Segmentation}
\label{sec:bottleneck}
% Conventional methods for robot object manipulation often suffer from limited generalization. One contributing factor is the lack of inductive bias in the architectures used: most of today’s widely adopted policy models are designed to handle diverse tasks ranging from object manipulation to locomotion and navigation \cite{Muhammad2022, Doshi2024}. Yet, just as biological species evolve specialized characteristics, it is crucial to incorporate suitable biases into the policy model so that learning for object manipulation can be done more efficiently—without sacrificing generality.

In human object manipulation, gaze position and hand movement are strongly coupled both temporally and spatially \cite{Johansson2001, Paillard1996}.
As illustrated in Figure \ref{fig:bottleneck}, a gaze-centered point cloud makes it possible to segment the movement into two distinct phases: (1) a reaching motion toward the vicinity of the gaze, and (2) a gaze-centered dexterous action. This segmentation is possible because gaze-centered vision captures only the object during the reaching phase, whereas it captures both the object and end-effector during the dexterous action phase.
Here, we define the temporal boundary between these two actions as a bottleneck.

\begin{figure}
    \centering
    \includegraphics[width=0.9\linewidth]{figures/bottleneck.pdf}
    \caption{By observing object manipulation in gaze-centered point cloud, an action can be segmented into two phases at the \textbf{bottleneck}: (1) the reaching motion and (2) the gaze-centered dexterous action. The motion after the bottleneck is reusable irrespective of the object position and the initial end-effector pose.}
    \label{fig:bottleneck}
\end{figure}

While fully parametric methods, in which neural networks directly predict action trajectories, can achieve high dexterity and reactivity, it has been challenging to extrapolate learned behaviors to unseen object positions and end-effector poses.
To address this, we restrict the use of fully parametric action prediction to the gaze-centered dexterous actions derived from the aforementioned bottleneck-aware action segmentation. In this context, the robot first reaches the bottleneck pose in a goal-fixed manner using Bézier curve approximation, and then switches to the fully parametric approach for the gaze-centered dexterous action.
Here, if (A) the bottleneck estimation is independent of both object position and end-effector pose, and (B) the fully parametric gaze-centered dexterous action prediction relies solely on the relative spatial relationships between the object and end-effector (or the grasped object), it becomes possible to accurately perform the learned motions even under previously unseen object positions and initial end-effector poses without sacrificing dexterity or reactivity.

As the bottleneck pose is defined in a gaze-centered manner, it can be described as an offset from the 3D gaze position:
\begin{align}
    p_b &= p_{\text{gaze}} + p^{\text{relative}}_b \label{eq:bottleneck1}, \\ 
        &= p_{\text{gaze}} + f(g) \label{eq:bottleneck2}, 
\end{align}
where $p_b, p_{\text{gaze}},$ and $p^{\text{relative}}_b$ denote the bottleneck pose, the 3D gaze position in the end-effector coordinate frame, and the offset of the bottleneck pose from the gaze position, respectively, and $f(\cdot)$ represents a neural network that takes the gaze-centered point cloud $g$ as input.
Here, the bottleneck pose is estimated as an offset from the gaze position rather than being directly predicted based on the gaze-centered point cloud (Eq. \ref{eq:bottleneck2}).
Notably, this offset is independent of the object position. 
As a result, the bottleneck pose can be accurately estimated even when the object position is unseen. Moreover, as Eq. \ref{eq:bottleneck1} does not involve the current end-effector pose, the accuracy of bottleneck estimation is not affected even with unseen end-effector poses.

\subsection{Data-driven Demonstration Segmentation}
\label{sec:segmentations}

We first temporally segment the provided demonstrations in a data-driven manner based on gaze and bottlenecks before training an action policy.
Specifically, we decompose the demonstrations into multiple sub-tasks (Section \ref{sec:sub-task}), and then further segment each sub-task into reaching motions and gaze-centered dexterous actions according to bottlenecks (Section \ref{sec:bottleneck-determination}). 
The corresponding action policy modules are trained using the sets of reaching motions and gaze-centered dexterous actions obtained through these processes.
The decomposition into sub-tasks is performed so that each sub-task includes exactly one bottleneck by using the task decomposition method based on gaze transitions \cite{Takizawa2024}. 
The proposed method for demonstration segmentation is summarized in Algorithm \ref{alg:segmentations}.

\subsubsection{Gaze-Based Sub-task Segmentation}
\label{sec:sub-task}
A typical object manipulation task comprises multiple bottlenecks. For example, in a standard pick-and-place task, there is usually one bottleneck for the “pick” phase and another for the “place” phase. In such cases, before segmenting the motions using bottlenecks, it is necessary to decompose a sequence of object manipulation behaviors into multiple smaller sub-tasks (e.g., pick/place in this example).

As a robust and simple way to achieve such segmentation in a data-driven manner, previous work proposed a gaze-based task decomposition method for object manipulation \cite{Takizawa2024}. This approach exploits the pattern of gaze during object manipulation, where the human teleoperator fixates on specific task-relevant gaze landmarks. By simply detecting transitions in this gaze behavior, the object manipulation task can be segmented so that each fixation period corresponds to a distinct sub-task. In this work, we first apply this method to decompose expert demonstrations into multiple sub-tasks.

\subsubsection{Bottleneck Determination}
\label{sec:bottleneck-determination}
Once the demonstrations are segmented into sub-tasks, we need to determine the time step at which each sub-task reaches its bottleneck pose. 
To do so, we propose a bottleneck determination approach based on action predictivity from a gaze-centered point cloud.
In each sub-task, predicting actions solely from a gaze-centered point cloud exhibits low accuracy when the end-effector or grasped object is not visible within the gaze-centered point cloud.
Therefore, the transition from this low-predictivity phase to a high-predictivity phase can serve as the segmentation point, which we define as the bottleneck (Figure \ref{fig:bottleneck-determination}).

In this approach, we use a policy model $a_t = h(g_t)$ that predicts the action solely from the gaze-centered point cloud $g_t$. This policy model is trained via behavior cloning \cite{Zhang2018}, that is, by minimizing $\|h(g_t) - a_t^*\|$ to replicate the actions in the demonstration, where $a_t^*$ denotes the expert action recorded in the demonstration. 
Once this policy $h$ has been trained, we compute the action-prediction loss $\|h(g_t) - a_t^*\|$ at every time step of each demonstration. 
As illustrated in Figure \ref{fig:bottleneck-determination}, we split each sub-task based on the median of this action prediction loss, enabling entirely data-driven segmentation for every sub-task.

\begin{figure}
    \centering
    \includegraphics[width=\linewidth]{figures/bottleneck-determination.pdf}
    \caption{We can determine the bottleneck based on the action predictivity by training a model to predict actions solely from the gaze-centered point cloud.}
    \label{fig:bottleneck-determination}
\end{figure}

\begin{algorithm}[t]
\caption{Data-driven Demonstration Segmentation}
\label{alg:segmentations}
\begin{algorithmic}[1]
  \Statex
  \textbf{Given:} Demonstrations 
  \[
  D = \{(o_t^{(i)}, p_t^{(i)}, g_t^{(i)})\}_{i=1}^{N},\quad t\in[0,T^{(i)}],
  \]
  where $o_t$, $p_t$, $g_t$ denote observation (point cloud), end-effector pose, and gaze position.
  \Statex
  \textbf{Notation:} For the $k$-th sub-task of the $i$-th demonstration: 
  \[
  s^{(i)}_k \text{ (start)},\quad e^{(i)}_k \text{ (end)},\quad b^{(i)}_k \text{ (bottleneck)}
  \]
  
  \State \textbf{(1) Gaze-based Action Segmentation:}
  \For{each demonstration \(i\)}
    \State Partition \([0, T^{(i)}]\) into \([s^{(i)}_0,e^{(i)}_0],\dots,[s^{(i)}_K,e^{(i)}_K]\)
  \EndFor
  
  \State \textbf{(2) Train Action Prediction Model $h_\theta$:}
  \[
  \theta^*=\arg\min_\theta \sum_{(o_t,a_t,g_t)\in D}\|a_t-h_\theta(\text{crop}(o_t,g_t))\|^2
  \]
  
  \State \textbf{(3) Compute Action Predictivity:}
  \For{each demonstration \(i\) and sub-task \(k\)}
    \State \(\text{scores}_k=\{\|a_t-h_{\theta^*}(\text{crop}(o_t,g_t))\|^2\}_{t=s^{(i)}_k}^{e^{(i)}_k}\)
  \EndFor
  
  \State \textbf{(4) Detect Bottleneck:}
  \For{each sub-task \([s^{(i)}_k,e^{(i)}_k]\)}
    \State Partition \([s^{(i)}_k,e^{(i)}_k]\) into \([s^{(i)}_k, b^{(i)}_k-1]\) and \([b^{(i)}_k, e^{(i)}_k]\)
  \EndFor

  \Statex 
  \textbf{Return:} $D$ and $\{s^{(i)}_k, e^{(i)}_k, b^{(i)}_k\}_{i,k}$
\end{algorithmic}
\end{algorithm}

\subsection{Policy Design}
\label{sec:model}

\begin{figure*}[t]
    \begin{center}
        \includegraphics[width=\linewidth]{figures/model-architecture.pdf}
        \caption{\textbf{GazeBot} architecture. (left) The gaze prediction model is trained to estimate the gaze position across the entire image as a classification problem. (right) The action policy model achieves robust reaching motion by estimating the bottleneck pose and the shape of the trajectory up to the bottleneck, and uses a Transformer to predict the gaze-centered dexterous action in a full-parametric manner. Both actions and gaze transitions are predicted by the gaze-centered point cloud to improve the reusability.}
        \label{fig:model-architecture}
    \end{center}
\end{figure*}

At every time step, GazeBot first predicts the gaze position from the entire image using the gaze prediction model (Section \ref{sec:gaze-model}). It then uses the gaze-centered point cloud at the predicted gaze position, along with the current end-effector pose, as inputs to the action policy model (Section \ref{sec:policy-model1} and \ref{sec:policy-model2}), which reactively predicts the subsequent end-effector poses.
The action policy model consists of two main modules: (i) a module that predicts gaze-centered dexterous actions solely from gaze-centered point clouds, and (ii) a module that predicts bottleneck poses based on gaze-centered point clouds and 3D gaze positions, as well as \textit{control points} of first-order Bézier curves to connect the current end-effector and the predicted bottleneck pose. 
These modules are trained using, respectively, (i) post-bottleneck motions and (ii) reaching trajectories leading to the bottleneck pose, both obtained from demonstrations as described in the previous section \ref{sec:bottleneck-determination}. 

\subsubsection{Gaze Prediction Model}  
\label{sec:gaze-model}
For gaze-based object manipulation, it is necessary for the robot to acquire gaze information during both training and inference. 
In response, Kim et al. proposed a method that takes advantage of the common practice of collecting real-robot demonstrations via human teleoperation \cite{Kim2020}. 
During demonstration collection, the teleoperator’s gaze is measured simultaneously, and this gaze data is then used as supervision for a gaze prediction model, which predicts gaze positions across the entire image. The gaze positions estimated by the gaze prediction model are used during both training and inference.

We adopt a similar approach in this study. In particular, we employ an architecture similar to the one used in DAA \cite{Kim2024b}. As shown in the left side of Figure \ref{fig:model-architecture}, we replace the image token extraction via EfficientNetV2 with DINOv2 \cite{Oquab2023}, a newer and more powerful feature extraction model, and replace the cross-attention layers (originally intended for language inputs) with a set of four-layer multilayer perceptrons (MLPs) that map each of the 768-dimensional tokens to a single probability value.
To intentionally transition the gaze to the next gaze landmark upon completing each sub-task, we prepare one such MLP for each sub-task and switch these MLPs based on the sub-task index $i_{\mathrm{seg}}$.  
At inference time, the sub-task index $i_{\mathrm{seg}}$ is initialized to zero at the beginning and incremented based on the progress $c_t$ output by the action policy model, indicating the completion of each sub-task. 
By managing gaze transitions using $c_t$ estimated from the gaze-centered point cloud, which solely captures the relative spatial relationships between the object and end-effector (or grasped object), the robot can perform gaze transitions at the correct timing even under unseen conditions.

\subsubsection{Action Policy Model (i)}  
\label{sec:policy-model1}
The pixel-space gaze position predicted by the gaze prediction model is transformed into 3D coordinates using stereo-based depth estimation. 
From the stereo-derived point cloud, we then crop a cubic region (20 cm on each side in this study) centered on the 3D gaze coordinates, which is referred to as the gaze-centered point cloud. 
This gaze-centered point cloud is embedded as a sequence of tokens using a PointTransformer \cite{Zhao2020} and then passed to a Transformer encoder (Figure \ref{fig:model-architecture} right). 
The Transformer encoder includes a CLS token (\textit{classification token}, similar to ViT \cite{Dosovitskiy2020}) that aggregates information from all point-cloud tokens into a feature vector $f_{\mathrm{pcd}} \in \mathbb{R}^{512}$. 
All output tokens, including the CLS token, are then fed into a Transformer decoder, which generates both the action sequence $a_{t:t+H} \in \mathbb{R}^{H\times14}$ for the left and right end-effector over $H$ future time steps and the progress $c_t \in \mathbb{R}^{N_{seg}}$ for controlling gaze transitions in each sub-task. Here, each end-effector has 7 degrees of freedom including gripper angle, and $N_{seg}$ denotes the number of sub-tasks. 
As the input vision is gaze-centered, the output actions are represented as relative end-effector poses, with the current end-effector pose serving as the origin of the coordinate frame. 
By computing these actions in relative coordinates, the learned dexterous actions remain reusable even when the absolute position of the object changes.

\subsubsection{Action Policy Model (ii)}  
\label{sec:policy-model2}
The feature vector $f_{\mathrm{pcd}}$ is used to predict the bottleneck poses for each end-effector. 
As described in Section \ref{sec:bottleneck}, the bottleneck pose is estimated by first predicting its offset relative to the 3D gaze position, then adding this offset to the 3D coordinates of the gaze. The module depicted in the top-right of Figure \ref{fig:model-architecture} implements this procedure and enables significantly more accurate extrapolation of the bottleneck pose under unseen object positions.

After predicting the bottleneck pose, the corresponding reaching trajectory is generated. 
In this context, a simple straight-line movement sometimes suffices for the reaching motion, because the movement from the current end-effector pose to the bottleneck pose is non-contact and relatively coarse.
However, for some tasks the reaching trajectory consistently takes a particular shape. For example, when placing an object into a deep container, the reaching motion toward the container consistently forms an upwardly convex curve.
To capture such a characteristic shape without introducing unnecessary complexity that could compromise the model’s generalization, we generate the reaching trajectory from the current end-effector pose to the bottleneck pose by using a first-order Bézier curve. 
Furthermore, because the endpoint of the Bézier curve is fixed at the bottleneck pose, the robot can accurately reach the bottleneck without incurring a compounding error \cite{Zhao2023} that accumulates over time.

As illustrated in the lower-right portion of Figure \ref{fig:model-architecture}, we predict a 7-dimensional \textit{bezier vector} that determines the shape of the first-order Bézier curve for each end-effector.
A first-order Bézier curve is defined by its two endpoints and a single \textit{control point}. 
Because the curve lies in a 7-dimensional end-effector pose space, the combined control points for both arms are in 14 dimensions.
Rather than predicting the control point directly, we first estimate the displacement---bezier vector---from the mean pose of the start (current end-effector pose) and end (bottleneck pose) poses, and then add this bezier vector to the mean pose to obtain the control point (Figure \ref{fig:model-architecture} bottom-right).
For the training, we approximate each reaching motion in the demonstrations by fitting them with first-order Bézier curves, and then use the resulting bezier vectors from these curves for supervision.













\section{Experiments}
\label{sec:experiments}
\subsection{Robot System}
We used a dual-arm robot system designed for imitation learning via human teleoperation. %\cite{Kim2020}. 
In this system, a human operator remotely controls the robot while observing its surrounding environment through a head-mounted display (HMD). During this operation, the operator’s gaze data were recorded in sync with the video feed displayed on the HMD. The system is compatible with both a physical dual-arm robot—consisting of two UR5 (Universal Robots Inc.) arms—and its simulated counterpart, allowing the collection of teleoperated demonstration data in both real and virtual environments. % \cite{Hamano2022}. 
The recorded gaze data were output as pixel coordinates corresponding to the images displayed on the HMD. The same single stereo camera (ZED Mini, Stereolabs Inc.) was used for both the teleoperation and inference phases, and we directly utilized the depth images generated by the deep learning–based depth estimation algorithm provided by ZED SDK (Stereolabs Inc.). The time-series demonstration data used for training were recorded at 10 Hz.

\subsection{Task Setup}
\label{sec:data}
In this study, we conducted imitation learning experiments on three tasks (two real and one simulated) to evaluate the reusability of skills acquired by GazeBot in comparison with conventional models.
To quantitatively assess the generalization performance, we collected demonstrations under controlled conditions by restricting the object positions and initial end-effector poses within predefined regions, thus clearly defining ID and OOD situations:
\begin{itemize}
    \item \textbf{PenInCup} (real, 109 demos): The robot, operating on a bare tabletop (no tablecloth), picks up a blue marker pen with its left arm and places it into a red cup. Using a gaze-based action segmentation method, this series of actions is divided into two sub-tasks. The pen is always placed such that the part without the cap remains within a 15 cm square area on the left side of the table, and the cup is placed so that part of its base overlaps a 5 cm area on the right side of the table. The robot’s initial pose starts with its left and right end-effectors positioned above the left and right edges of the table, respectively, oriented inward. The object positions and orientations, as well as the robot’s initial pose, are manually randomized within their respective areas. All other factors (table location, background, etc.) are maintained as consistent as possible during the demonstration collection. To avoid introducing bias in the training images, we marked the boundaries of the object placement areas in a way that would be minimally visible.
    \item \textbf{OpenCap} (real, 110 demos): The robot, operating on a tabletop covered with a green tablecloth, holds an upright empty plastic bottle in place with its right arm and removes the cap with its left arm. This series of actions is divided into two sub-tasks. The bottle is always placed within the right side of the table. All other conditions are identical to those described in the PenInCup task.
    \item \textbf{PileBox} (sim, 100 demos): The robot, operating on a bare tabletop, picks up a red box with its left arm and stacks it on top of a green box. This series of actions is divided into two sub-tasks. The red box is always placed such that its center remains within a 10 cm × 20 cm rectangular region on the left side of the table, and the green box is placed such that its center remains within a 10 cm square region on the right side of the table. The robot’s initial pose starts with its left and right end-effectors positioned above the left and right edges of the table, respectively, oriented inward. Because this task is performed in a simulation environment, the object positions and orientations are fully randomized according to an uniform distribution, whereas the initial robot pose, table position, and background are all fixed.
\end{itemize}

\subsection{Evaluation of Generalization and Ablation}
To verify the improved reusability of learned skills provided by GazeBot, we first trained GazeBot and its ablation models, including conventional models (ACT \cite{Zhao2023}, DAA \cite{Kim2024b}), using the demonstrations collected as previously described in Section \ref{sec:data}.
We then measured their success rates for scenarios within the training distribution (ID) and those outside it (OOD), as shown in Figure \ref{fig:eval-generalization}. 
In the experiment, we standardized the initial conditions for all ID and OOD trials to enable an accurate comparison of each model’s generalization performance. 
Consequently, both GazeBot and the seven ablation models performed the task under conditions as similar as possible. 
In the OOD trials, we first evaluated the cases where each object was individually placed in unseen positions, and then examined the cases where all objects were simultaneously placed in unseen positions (Figure \ref{fig:eval-generalization}). Finally, we conducted trials involving unseen initial end-effector poses. All initial conditions used in the ID and OOD trials are provided in Appendix \ref{appendix-experiment}.

The overall success rates in ID and OOD trials for the PenInCup, OpenCap, and PileBox tasks are presented in Table \ref{tab:eval-generalization}.
Here, the reusability of acquired skills is evaluated by how well the model achieves a success rate as high as possible on OOD while maintaining a high success rate on ID.
GazeBot incorporates several design choices and components, whose contributions can be assessed by comparing its performance with that of the ablation models.

\begin{figure*}
    \begin{center}
        \includegraphics[width=\linewidth]{figures/eval-generalization.pdf}
        \caption{Examples of \textbf{ID} and \textbf{OOD} trials in the PenInCup, OpenCap, and PileBox tasks, where object positions and the initial end-effector poses are controlled. The images show the initial states of each trial. The checkboxes correspond to the method order in Table \ref{tab:eval-generalization} and indicate whether each method succeeded in the task from that initial state.}
        \label{fig:eval-generalization}
    \end{center}
\end{figure*}

\begin{table*}
    \centering
    \resizebox{\textwidth}{!}{
    \begin{tabular}{lcccccccccccc}
    \toprule
        & \multicolumn{4}{c}{\textbf{PenInCup} (real)} 
        & \multicolumn{4}{c}{\textbf{OpenCap} (real)} 
        & \multicolumn{4}{c}{\textbf{PileBox} (sim)} \\
    \cmidrule(lr){2-5}\cmidrule(lr){6-9}\cmidrule(lr){10-13}
        & \multicolumn{2}{c}{ID} & \multicolumn{2}{c}{OOD} 
        & \multicolumn{2}{c}{ID} & \multicolumn{2}{c}{OOD} 
        & \multicolumn{2}{c}{ID} & \multicolumn{2}{c}{OOD} \\
    \cmidrule(lr){2-3}\cmidrule(lr){4-5}
    \cmidrule(lr){6-7}\cmidrule(lr){8-9}
    \cmidrule(lr){10-11}\cmidrule(lr){12-13}
        \textbf{Method} 
        & Pick & Put & Pick & Put 
        & Hold & Open 
        & Hold & Open 
        & Lifted & Pile 
        & Lifted & Pile \\
    \midrule
        GazeBot (Ours) 
            & \textbf{92} & \textbf{75} & \textbf{83} & \textbf{83} 
            & \textbf{100} & \textbf{83} 
            & \textbf{100} & \textbf{75} 
            & \textbf{100} & \textbf{100} 
            & 80 & \textbf{75} \\
        - w/o point cloud (Ablation1) 
            & 75 & \textbf{75} & 42 & 42 
            & \textbf{100} & 67 
            & 58 & 42 
            & 95 & 65 
            & 70 & 40 \\
        - w/ state input partially (Ablation2)
            & 50 & 33 & 25 & 8 
            & \textbf{100} & 50
            & 58 & 42 
            & \textbf{100} & 90 
            & \textbf{85} & 50 \\
        - w/ state input (Ablation3)
            & 75 & 67 & 17 & 0 
            & \textbf{100} & 50 
            & 50 & 33 
            & \textbf{100} & 80 
            & 65 & 10 \\
        - w/o relative bottleneck (Ablation4)
            & 58 & 50 & 42 & 17 
            & \textbf{100} & 67 
            & 67 & 42 
            & \textbf{100} & 95 
            & 65 & 35 \\
        DAA 
            & 67 & 50 & 17 & 0 
            & 75 & 67 
            & 8 & 8 
            & \textbf{100} & 60 
            & 15 & 0 \\
        - w/o local action 
            & 67 & 50 & 17 & 0 
            & 67 & 50 
            & 8 & 8 
            & \textbf{100} & 40 
            & 20 & 0 \\
        ACT 
            & 75 & 67 & 8 & 0 
            & \textbf{100} & 75 
            & 25 & 25 
            & \textbf{100} & 80 
            & 40 & 15 \\
    \bottomrule
    \end{tabular}
    }
    \caption{Comparison of success rates (\%) for ID and OOD trials in PenInCup, OpenCap and PileBox tasks. Our method is compared with seven ablation models, two of which are conventional baselines. We conducted 12 trials for PenInCup and OpenCap, and 20 trials for PileBox. Notably, only the proposed method maintains a high success rate in OOD situations for both tasks.}
    \label{tab:eval-generalization}
\end{table*}


\subsubsection{3D Awareness of Gaze-based Vision (Ablation1)}
One core feature of GazeBot is the ability to produce visual representations robust to changes in object location, achieved by the gaze-centered point cloud. In Ablation1, we replaced the gaze-centered point cloud with the conventional image-cropping approach. The cropped left and right images were tokenized via ResNet18 \cite{He2015}, following ACT, and then fed into the Transformer encoder. 
Although Ablation1 maintained a relatively high success rate for ID trials, as presented in Table \ref{tab:eval-generalization}, it suffered from lower accuracy in OOD owing to the absence of 3D awareness, which led to degraded performance when the objects appeared differently under unseen object positions.

\subsubsection{Inputs to the Policy (Ablation2, Ablation3)}
In GazeBot, the action policy model is designed so that the current end-effector pose is used as input only during the estimation of the bezier vector. 
In other words, bottleneck estimation relies solely on the 3D gaze position and the gaze-centered point cloud, and the prediction of the gaze-centered dexterous actions after the bottlenecks also relies only on the gaze-centered point cloud.

In Ablation3, we added tokens representing the current left and right end-effector poses and the 3D gaze position (3 × 512-dimensions) to the sequence of 3D point cloud tokens fed to the Transformer encoder. This addition prevented the model from accurately estimating the bottleneck in unseen initial poses. Even when the robot successfully reached the correct bottleneck for unseen object positions, the subsequent gaze-centered dexterous actions degraded because the bottleneck pose itself is an untrained input for the policy network (Table \ref{tab:eval-generalization}).

In Ablation2, we similarly added tokens but then applied an attention mask so that the CLS token $f_{pcd}$ in the Transformer encoder did not attend to these additional tokens. As a result, $f_{pcd}$ aggregated only information from the 3D point cloud, enabling accurate bottleneck reaching for unseen end-effector poses. Nonetheless, as mentioned in Ablation3, the accuracy of the gaze-centered dexterous actions still decreased even after the bottleneck was reached correctly (Table \ref{tab:eval-generalization}), often causing the end-effector to be “pulled” back toward the ID region.

\subsubsection{Bottleneck Estimation Method (Ablation4)}
GazeBot estimates the bottleneck pose by first predicting an offset from the 3D gaze position based on the gaze-centered point cloud and then adding that offset to the 3D gaze position. Another implementation could be considered, where these inputs are fed into a neural network that directly outputs the bottleneck pose. However, under unseen object positions, Ablation4 struggled to extrapolate the bottleneck and even failed to reach the object correctly. By contrast, our method successfully extrapolated the bottleneck pose in almost all OOD cases (Table \ref{tab:eval-generalization}).

\subsubsection{Design of the Bottleneck Reaching Module (DAA)}
In GazeBot, we achieve bottleneck reaching by first predicting a bottleneck pose and then generating a first-order Bézier curve as a trajectory leading to the bottleneck.
One could adopt a fully parametric approach that directly outputs an action sequence leading to the bottleneck pose.
DAA, for instance, proposes switching between global and local actions—similar to our segmentation by bottlenecks—and employs a fully parametric approach for the global actions by directly predicting the time series of end-effector poses.
In our tasks, however, DAA exhibited a significant drop in success rates from ID to OOD (Table \ref{tab:eval-generalization}). In most OOD trials, the model even failed to perform the reaching motion itself, reflecting the issues seen in Ablation4.

\subsubsection{Gaze-Based Vision (ACT, DAA w/o local action)}
Lastly, we compared two baselines: (1) ACT, which is effectively equivalent to removing all proposed modules from GazeBot, and (2) DAA w/o local action, which can be interpreted as applying a gaze-based image-cropping strategy to ACT.
As can be observed from Table \ref{tab:eval-generalization}, both approaches exhibited a sharp decline in success rates when moving from ID to OOD.
In OOD trials, whether local action was present or not had little impact on the success rate of DAA because it already failed during the global (reaching) phase.
Moreover, ACT slightly outperformed DAA overall. 
In DAA, because only a small, gaze-centered region is input to the model and 3D awareness is nearly absent, the input varies significantly with the object position, undermining data efficiency. Consequently, with approximately 100 demonstrations—as in this experiment—it was not possible to achieve the improvements in dexterity previously reported \cite{Kim2021}.
Meanwhile, GazeBot, which also adopted gaze-based vision, outperformed ACT (Table \ref{tab:eval-generalization}). This demonstrates that our approach additionally contributed to improved data efficiency of gaze-based models.











\section{Limitations and Future Directions}
In this study, we proposed GazeBot, an imitation learning method for robotic object manipulation that significantly improves the reusability of learned skills without sacrificing dexterity and reactivity. 
GazeBot demonstrates a significantly high generalization capability, particularly for unseen object positions and unseen end-effector poses, compared to state-of-the-art models. However, several challenges remain.

\textbf{Precise and Flexible Gaze Control.} While GazeBot demonstrated the importance of three-dimensional gaze control in object manipulation, robotic gaze prediction is still in its infancy. 
Robots not only require task-specific gaze data from human teleoperators for supervision but also lack the flexibility to adjust their gaze positions when objects are partially or fully obscured. Moreover, they struggle to align their gaze \textit{exactly} on tiny targets, such as a needle in midair.

\textbf{More Flexible and Adaptive Segmentations.} In this study, we achieved highly reusable imitation by employing gaze and bottlenecks to spatially and temporally segment vision and action. However, our current approach uses rigid segmentation along explicit boundaries. One limitation concerns the gaze-centered point cloud. In our method, gaze-based vision only captures information corresponding to foveal vision, whereas human vision simultaneously benefits from peripheral vision, which can roughly capture a wider range of information. Another limitation involves bottleneck determination. The proposed approach assumes that each sub-task has a single bottleneck, yet some tasks involve sub-tasks with multiple bottlenecks. Future work could explore more flexible segmentation methods to enable more adaptive learning.

\textbf{Integrating Advanced Trajectory Planning.} For simplicity, the reaching trajectories in this work are generated by connecting the current end-effector pose and the bottleneck pose with a first-order Bézier curve. To further enhance generalization, we could replace these simple connections with trajectories provided by advanced planning methods that support collision avoidance and bimanual coordination. Such improvements would enable more robust reusability of learned motions.






% \section*{Acknowledgments}

%% Use plainnat to work nicely with natbib. 

\documentclass{MITstyle}

%\usepackage[table]{xcolor}
\usepackage{chngcntr}
\usepackage{hyperref}
\usepackage{microtype}

\title{A Lightweight and Extensible Cell Segmentation and Classification Model for Whole Slide Images}

\author{Nikita Shvetsov~$^{1, }$\footnote{Correspondence e-mail: nikita.shvetsov@uit.no}, Thomas K. Kilvaer~$^{2, 3}$, Masoud Tafavvoghi~$^{4}$, Anders Sildnes~$^{1}$, \\ Kajsa Møllersen~$^{4}$, Lill-Tove Rasmussen Busund~$^{5, 6}$, Lars Ailo Bongo~$^{1}$ \\
%
\vspace{1em} % Space between authors and afilliations
%
\normalfont{\small $^{1}$Department of Computer Science, UiT The Arctic University of Norway}\\
\normalfont{\small $^{2}$Department of Oncology, University Hospital of North Norway}\\
\normalfont{\small $^{3}$Department of Clinical Medicine, UiT The Arctic University of Norway}\\
\normalfont{\small $^{4}$Department of Community Medicine, UiT The Arctic University of Norway}\\
\normalfont{\small $^{5}$Department of Medical Biology, UiT The Arctic University of Norway} \\
\normalfont{\small $^{6}$Department of Clinical Pathology, University Hospital of North Norway} %\vspace{2em}
}

\begin{document}
\maketitle

\section*{Abstract}

% \begin{abstract}
% Developing clinically useful cell-level analysis tools in digital pathology remains challenging due to limitations in dataset granularity, inconsistent annotations, computational demands of advanced models, and difficulties in integrating new technologies into clinical workflows. To address these challenges, we propose a multi-faceted solution that enhances data quality, model performance, and usability to create a lightweight and extensible cell segmentation and classification model.

% First, we update data labels by employing a cross-relabeling process that refines the labels of two existing datasets, PanNuke and MoNuSAC, to create a new unified dataset with enhanced granularity, encompassing seven distinct cell types. Second, we leverage the H-Optimus foundation model as a fixed encoder to improve feature representation for simultaneous cell segmentation and classification tasks. Third, to address the computational demands of foundation models, we employ knowledge distillation to reduce model size and complexity while maintaining comparable performance. Finally, to facilitate integration into clinical workflows, we integrate the distilled model into the QuPath software, a widely used open-source platform in digital pathology.

% Our results demonstrate improvements in cell segmentation and classification performance using the H‑Optimus-based model compared to a CNN-based model. Specifically, the average $R^2$ improved from 0.575 to 0.871, and the average $PQ$ score improved from 0.450 to 0.492, indicating better alignment with actual cell counts and enhanced segmentation and classification quality. Furthermore, the distilled student model maintains performance comparable to the larger foundation model while reducing the parameter count by a factor of 48.
% Overall, by reducing computational complexity and integrating it into existing workflows, the proposed approach may significantly impact diagnostic processes, reduce the workload of pathologists, and contribute to improved patient outcomes. Though our approach shows potential enhancements in efficiency and usability of cell segmentation and classification models in digital pathology, extensive validation is needed to deploy these models in clinical practice.
% \end{abstract}

%%% shortened abstract
\begin{abstract}
Developing clinically useful cell-level analysis tools in digital pathology remains challenging due to limitations in dataset granularity, inconsistent annotations, high computational demands, and difficulties integrating new technologies into workflows. To address these issues, we propose a solution that enhances data quality, model performance, and usability by creating a lightweight, extensible cell segmentation and classification model. 

First, we update data labels through cross-relabeling to refine annotations of PanNuke and MoNuSAC, producing a unified dataset with seven distinct cell types. Second, we leverage the H-Optimus foundation model as a fixed encoder to improve feature representation for simultaneous segmentation and classification tasks. Third, to address foundation models' computational demands, we distill knowledge to reduce model size and complexity while maintaining comparable performance. Finally, we integrate the distilled model into QuPath, a widely used open-source digital pathology platform. 

Results demonstrate improved segmentation and classification performance using the H-Optimus-based model compared to a CNN-based model. Specifically, average $R^2$ improved from 0.575 to 0.871, and average $PQ$ score improved from 0.450 to 0.492, indicating better alignment with actual cell counts and enhanced segmentation quality. The distilled model maintains comparable performance while reducing parameter count by a factor of 48. By reducing computational complexity and integrating into workflows, this approach may significantly impact diagnostics, reduce pathologist workload, and improve outcomes. Although the method shows promise, extensive validation is necessary prior to clinical deployment.
\end{abstract}
\clearpage

\section{Introduction}
In digital pathology, accurate segmentation and classification of cells are crucial for many diagnostic, prognostic, and predictive analyses \cite{Jaber_Beziaeva_etal._2019,Lin_Pan_etal._2022,Park_Ock_etal._2022,Shen_Choi_etal._2024}. Nowadays, developments in computational pathology offer multiple solutions \cite{H._Qu_P._Wu_etal._2020,Javed_Mahmood_etal._2020} to utilize cell-level datasets to train machine learning models that solve these problems. The quality and specificity of training datasets are critical for robust and accurate models. Adhering to the principle of "garbage in, garbage out", it is essential to ensure that these datasets are extensively and accurately labeled with distinct classes that reflect the diverse biological characteristics of different cell types. Unfortunately, the number of open-source datasets comprising such high-quality annotations is limited. Existing cell segmentation datasets \cite{Gamper_Koohbanani_etal._2019,Graham_Vu_etal._2019,Verma_Kumar_etal._2021} may offer extensive annotations for certain cell types while providing more general labels for others. For example, in PanNuke, which is one of the largest open-source datasets comprising labeled cells, various types of morphologically and functionally different inflammatory cells like macrophages and lymphocytes are clustered in a broad "inflammatory" class. Consequently, these classes are frequently omitted from analyses or aggregated into broader meta-classes \cite{Gamper_Koohbanani_etal._2020} and likely interfere with other cell classes included in the dataset. This and similar inconsistencies in annotation granularity limit the ability of machine learning models to learn the comprehensive and nuanced features necessary for accurate cell segmentation and classification. To address these challenges, methods for refining and standardizing dataset annotations are essential to enhance the quality of training data.

A complementary approach to mitigate the absence of high-quality training data is the use of foundation models. Foundation models as encoders are defined as large-scale, versatile networks pre-trained on vast, diverse datasets using self-supervised learning, contrasting with convolutional neural network (CNN) pre-trained encoders that rely on supervised learning with labeled data. In practice, foundation models leverage enormous amounts of weakly or unlabeled data from millions of whole slide images (WSIs) and employ self-attention mechanisms to capture long-range dependencies and global context \cite{Chen_Ding_etal._2024,Saillard_Jenatton_etal._2024,Vorontsov_Bozkurt_etal._2024,Xu_Usuyama_etal._2024}. As a consequence, foundation models are able to produce transferable feature representations across different cell types and tissue environments. The feature representations can be leveraged by decoder networks to produce segmentation masks and pixel-level classifications. Because foundation models have comprehensive feature representations, they can be effectively fine-tuned using much smaller amounts of cell-level data compared to the large datasets needed to train models from scratch. Furthermore, foundation models incorporate adversarial training elements or contrastive learning \cite{Chen_Ding_etal._2024,Xu_Usuyama_etal._2024}, enhancing their resilience and adaptability by exposing them to challenging and varied scenarios during training. This may result in more generalizable models, often making them well-suited for diverse and complex tasks in digital pathology.

Despite the inherent advantages of foundation models, their deployment for practical use faces its own obstacles. In particular, they require substantial computational power, financial investments and rigorous testing to ensure reliability and efficacy for a given task \cite{Akkus_Dangott_etal._2022,Dragomir_Cocuz_etal._2022,Go_2022,Jafri_Farooqui_etal._2024}. Moreover, while foundation models enhance feature representation and performance, they depend on the quality of available annotations for decoder fine-tuning and, like any other model, cannot resolve existing inconsistencies or ambiguities in data labels. Therefore, there remains a critical need for solutions that address both data quality and practical deployment considerations.
Further, integrating new technologies into existing clinical workflows often encounters resistance, as it necessitates adjustments to established diagnostic processes. So, there is a need to develop solutions that could be integrated into current practices, minimizing the burden on medical professionals to adopt new tools \cite{King_Williams_etal._2023}.

Existing solutions \cite{Goldsborough_Philps_etal._2024,Hörst_Rempe_etal._2024}, while addressing some aspects of these challenges, fall short in providing a comprehensive approach. To address the data quality and clinical deployment issues, we propose a multi-faceted solution that encompasses data refinement, model optimization, and integration with existing pathology tools (\hyperref[fig:fig1]{Figure 1}). The outcome is a lightweight cell segmentation and classification model that can be integrated into digital pathology workflows for practical clinical use.

\begin{figure}[h!]
    \centering
    \includegraphics[width=\textwidth, height=0.82\textheight, keepaspectratio]{images/Figure_1.pdf}
    \caption{Overview of the proposed solution, including 1) Data refinement using cross-relabeling, 2) Teacher model development and fine tuning, 3) Student model optimization with knowledge distillation and 4) Student model and QuPath integration}
    \label{fig:fig1}
\end{figure}
\clearpage

Our approach begins with preparing the data for the fine-tuning and training of the machine learning models. We create a refined dataset, acquired via cross-relabeling two cell-level datasets, enhancing annotation specificity and consistency of the labeled data. Subsequently, we create a cell segmentation and classification model based on the foundation model. We leverage the foundation model as a fixed encoder and fine-tune a decoder using the refined dataset to improve generalization across diverse tissue- and cell types.
To ensure that the model remains lightweight and deployable in a possibly resource-constrained environment, we employ knowledge distillation to approximate the functionality of the foundation model. Finally, to facilitate the practical application of our model in digital pathology workflows, we integrate it with the QuPath \cite{Bankhead_Loughrey_etal._2017} application. Each methodological component contributes to the overarching goal of enhancing model performance, generalizability, and usability in clinical settings.

The primary contributions of this paper are:
\begin{enumerate}
    \item \textit{Data labels refinement through cross-relabeling:}
    
    We propose a new method for refining labels of cell-level datasets through cross-relabeling. This method employs classification models to re-label broad and ambiguous instances, resulting in a more diverse dataset. Our evaluation demonstrates that these classification models achieve high accuracy on test subsets, indicating the reliability of the method for label refinement.

    \item \textit{Enhanced model performance via foundation models:}
    
    We employ a foundation model as a feature extractor for the cell segmentation and classification task. In comparison with training a CNN model from scratch, the foundation model backbone only needs fine-tuning, which significantly reduces training time, computational resources and data requirements. We show that using a foundation model encoder leads to better performance in cell segmentation and classification networks than using a CNN-based encoder. This improvement may enable the model to generalize more effectively across various tissue types and imaging methods.
    
    \item \textit{Model optimization through knowledge distillation:}
    
    We show that a smaller student model trained using knowledge distillation on the refined dataset obtained via our cross-relabeling approach from a foundation model achieves comparable performance in cell segmentation and quantification tasks. As a result, this model is more suitable for deployment in environments without high-performance computing resources.
    
    \item \textit{Integration with QuPath:}
    
    We integrate the distilled cell segmentation and classification model into QuPath, a widely used open-source digital pathology platform, to accelerate clinical adaptation by enabling pathologists to more easily incorporate advanced computational tools into their existing workflows.
\end{enumerate}

Through these methodological steps, we aim to bridge the gap between advanced machine learning techniques and practical clinical applications, making accurate and efficient digital pathology accessible in a broader range of healthcare settings.

\section{Refining Existing Datasets Using Cross-Relabeling}
To address the limitations of sparse and ambiguous labeling of cell-level datasets, we propose a generalizable cross-relabeling strategy that can be applied to any dataset containing broadly categorized or imprecisely labeled cell types. This approach involves training and subsequently leveraging classification models to refine broad categories into more specific or biologically relevant classes.
When applied to cell-level data, the methodology includes extracting individual cell images from the dataset patches, preprocessing these images to standardize the size and accommodate partial cells, and then training deep learning classifiers capable of distinguishing between the finer cell subtypes within the coarser categories. 
To illustrate our approach, we focus on the PanNuke \cite{Gamper_Koohbanani_etal._2020, Gamper_Koohbanani_etal._2019} and MoNuSAC \cite{Verma_Kumar_etal._2021} datasets that we have used to train models for cell quantification in our previous works \cite{Shvetsov_Grønnesby_etal._2022,Shvetsov_Sildnes_etal._2024}. We find that for better cell differentiation we have to introduce more granular labels. PanNuke includes a broad classification of "inflammatory" cells, encompassing lymphocytes, macrophages, and neutrophils. Each cell type differs significantly in structure, function, and clinical relevance. Conversely, MoNuSAC uses the label "epithelial" for a class that comprises both benign epithelial cells and malignant neoplastic cells. This practice makes it challenging to differentiate between benign and malignant epithelial cells in the dataset, which is a critical distinction when identifying tumor areas within tissue samples. To address these issues, we implement a cross-relabeling strategy as shown in \hyperref[fig:fig2]{Figure 2}. The key components are two classification models: one is trained on singular cell images from PanNuke data to classify the epithelial meta-class into epithelial and neoplastic classes. The other is trained on MoNuSAC to refine the inflammatory class into lymphocytes, neutrophils, and macrophages.

\begin{figure}[h!]
    \centering
    \includegraphics[width=\textwidth]{images/Figure_2.pdf}
    \caption{Refined dataset generation via cross relabeling}
    \label{fig:fig2}
\end{figure}

The refining approach consists of three consecutive steps. The first is the preprocessing step, in which we extract individual cells from both datasets (\hyperref[fig:fig3]{Figure 3}). The specifics of PanNuke and MoNuSAC patch preparation before cell preprocessing are provided in \hyperref[chap:S1]{Appendix S1}.

\begin{figure}[h!]
    \centering
    \includegraphics[width=\textwidth]{images/Figure_3.pdf}
    \caption{Cell instances preprocessing including (1) cell map extraction, (2) bounding box delineation, (3) adjusting cell boxes and (4) cropping and resizing of cell images}
    \label{fig:fig3}
\end{figure}

During preprocessing, we extract cell type maps from the ground truth label mask and calculate bounding boxes around each cell instance. To accommodate partial cells at patch borders, a common issue in cropped patch images, we employ mirror padding and extend the field of view of the cell label by 15 pixels to capture adjacent cells. We then crop and resize the identified regions to $64 \times 64$ pixels using bicubic interpolation.

The preprocessed PanNuke dataset comprises 68,031 neoplastic and 23,207 epithelial cell images, while MoNuSAC comprises  33,104 lymphocytes, 1,252 neutrophils, and 1,695 macrophages, which we subsequently use in training cell classification models and classifying the cell image data \hyperref[fig:S2]{Appendix Figure S2 (1)}. 

The next step is to train two distinct ResNet50-based classifiers tailored to address the specific labeling challenges inherent in each dataset. We use ResNet50 for classification models due to its proven effectiveness for image classification tasks in histopathology \cite{pan2022reviewmachinelearningapproaches}, and its compatibility with small images. For the PanNuke dataset, we design the classifier, trained on MoNuSAC data, to disaggregate the heterogeneous "inflammatory" cell category into distinct subtypes: lymphocytes, macrophages, and neutrophils. Similarly, for the MoNuSAC dataset, the classifier is trained on PanNuke data and distinguishes between benign and malignant epithelial cells within the overarching "epithelial" label. By applying these targeted classifiers to their respective datasets, we assign more specific labels to individual cell instances, thus enabling us to create a unified dataset.
To ensure a balanced representation of classes, we train both models on datasets that had been equalized to match the size of the least represented class. Thus, we obtain datasets comprising 23,207 samples per class for PanNuke and 1,252 samples per class for MoNuSAC data. Next, we partition both of them into training (70\%), validation (20\%), and testing (10\%) subsets. To mitigate the risk of overfitting, we use a single dropout layer with a rate of p=0.5 in both models and data augmentation using randomized color perturbations, rotation, and horizontal and vertical flipping. We employ AdamW optimizer and the cross-entropy loss function for the training criterion.

To evaluate the two trained models, we measure the classification accuracy on the respective test subsets. The accuracies on the test subset for both classifiers are presented in \hyperref[tab:1]{Table 1}. The PanNuke model achieves an average accuracy of 93.57\%, with higher accuracy for neoplastic cells (96.06\%) compared to epithelial cells (86.26\%). The confusion matrix in Figure A3.1 shows that the model predominantly distinguishes accurately between epithelial and neoplastic tissues, with a substantial number of correct classifications and relatively few misclassifications. The MoNuSAC model demonstrates an average accuracy of 98.92\%, excelling in classifying lymphocytes (99.67\%) and macrophages (94.12\%), with lower performance for neutrophils (85.71\%). The confusion matrix in Figure A3.2 shows that the model identifies lymphocytes and performs reasonably well with macrophages and neutrophils.

\begin{table}[h!]
\renewcommand{\arraystretch}{1.5}
  \centering
  \caption{Cell classification results for PanNuke and MoNuSAC trained models (CI 95\%).}
  \label{tab:1}
  \begin{tabular}{|l|c|c|}
   \hline
   %\rowcolor{gray!30}
    Accuracy               & PanNuke model              & MoNuSAC model              \\
    \hline
    Average      & 0.936 (0.931--0.941)         & 0.989 (0.986--0.993)        \\
    \hline
    Neoplastic   & 0.961 (0.956--0.965)         & -                          \\
    \hline
    Epithelial   & 0.863 (0.849--0.877)         & -                          \\
    \hline
    Lymphocytes  & -                          & 0.997 (0.995--0.999)        \\
    \hline
    Neutrophils  & -                          & 0.857 (0.796--0.918)        \\
    \hline
    Macrophages  & -                          & 0.941 (0.906--0.976)        \\
    \hline
  \end{tabular}
\end{table}

Finally, during the last step, we use the model trained on PanNuke data for epithelial cells in MoNuSAC and the model trained on MoNuSAC for the inflammatory cells class in PanNuke. Specifically, we use classifier models to relabel epithelial cells in MoNuSAC and inflammatory cells in PanNuke data. Then we combine cells with refined labels and the rest of the cells in both datasets to create a refined dataset (\hyperref[fig:S2]{Appendix Figure S2 (2)}). The process of relabeling cells and visualizing them on a patch is shown in \hyperref[fig:fig4]{Figure 4}. The cell counts in the refined dataset are provided in \hyperref[tab:S4]{Appendix Table S4}.

\begin{figure}[h!]
    \centering
    \includegraphics[width=\textwidth, height=0.42\textheight, keepaspectratio]{images/Figure_4.pdf}
    \caption{Cell relabeling procedure for epithelial and inflammatory cell classes}
    \label{fig:fig4}
\end{figure}

%\hfill

Relabeling and combining datasets have been explored in a prior study \cite{Parulekar_Kanwat_etal._2023}, where consecutive fine-tuning on multiple datasets was employed to account for hierarchical class label structures. While the method presented in \cite{Parulekar_Kanwat_etal._2023} is intuitive, it often lacks consistency and requires multiple fine-tuning runs, which can be cumbersome and time-consuming. 
In contrast, cross-relabeling simplifies this process by using specialized classification models tailored to each dataset's specific labeling challenges. This approach provides better transparency and produces a unified dataset encompassing seven distinct cell types across multiple tissue samples, enhancing data diversity for further model training or fine-tuning.

Despite these improvements, cross-relabeling does not entirely resolve issues related to poor labeling quality or the amount of labeled data. Specifically, our results show lower accuracies persist for underrepresented classes, such as macrophages, which may stem from a limited sample availability and intrinsic challenges in distinguishing these cells based solely on H\&E staining. Furthermore, while our method enhances label specificity, it relies on the initial quality of the broad labels; thus, any fundamental inaccuracies in the original annotations can propagate through the relabeling process. Addressing the overall problem of limited data labels may require integrating additional data sources or utilizing complementary immunohistochemical staining methods.
Although the reported performance metrics are obtained from evaluations on the native test sets of each dataset, it is important to note that the primary application of these classifiers is to perform cross-relabeling, where a model trained on one dataset (e.g., PanNuke) is applied to another (e.g., MoNuSAC) and vice versa. We acknowledge that a more systematic evaluation of cross-dataset generalization is needed and could be performed in future work.

Overall, the refined dataset produced by our approach can enhance the supervised training or fine-tuning of cell segmentation and classification models, especially those that utilize pre-trained foundation models to improve feature extraction robustness. In addition, these models can detect nuanced classes that enable researchers to conduct more detailed analyses of biological processes in computational pathology.

\section{Foundation models for robust cell segmentation and classification}

Accurate cell segmentation and classification in digital pathology are hindered by limited labeled data and the fact that conventional CNNs are unable to capture global contextual information due to their local receptive field constraints \cite{Gheflati_Rivaz_2022,Yang_Marcus_etal.}. Traditional approaches in cell quantification have predominantly relied on CNN encoders, such as ResNet50, given their proven effectiveness in semantic segmentation tasks \cite{Deshmane_2023,Graham_Vu_etal._2019,Mukasheva_Koishiyeva_etal._2024,Stringer_Wang_etal._2021}. However, approaches that include fine-tuning of pretrained CNNs, data augmentation, and stain normalization to partially increase data variability and address staining differences often fail to achieve the necessary generalization and robustness across diverse tissue types and staining conditions \cite{G._Wang_W._Li_etal._2018,Gao_Bagci_etal._2018,Karim_El_Khoury_Martin_Fockedey_etal._2021}.

To overcome these challenges, we leverage an encoder-decoder network that uses a foundation model as the encoder and a CNN upsampling decoder (\hyperref[fig:fig5]{Figure 5}) for simultaneous cell segmentation and classification in 2D patches extracted from WSIs. Foundation models with transformer-based architectures are viable alternatives to CNN-based encoders \cite{Shamshad_Khan_etal._2023,Sourget_2023}. They enable the creation of more advanced architectures that can decode or transform learned features more effectively \cite{Chen_Duan_etal._2023,Cheng_Misra_etal._2022,Xie_Wang_etal._2021}.

\begin{figure}[h!]
    \centering
    \includegraphics[width=\textwidth]{images/Figure_5.pdf}
    \caption{UNETR-like model with foundational model as backbone}
    \label{fig:fig5}
\end{figure}

By utilizing a transformer-based encoder, we incorporate global contextual information into the feature extraction process, which is a key advantage of such architectures \cite{Chen_Lu_etal._2021}. This foundation model integration facilitates accurate pixel-wise segmentation and classification without the need for extensive encoder training, thereby potentially improving generalization across varied cellular structures and tissue types.
In our implementation, we employ a modified UNETR \cite{Hatamizadeh_Tang_etal._2021} architecture that combines a vision transformer (ViT) \cite{Dosovitskiy_Beyer_etal._2021} encoder with a CNN-based decoder. The encoder utilizes the pretrained H-Optimus foundation model, which contains 1.1 billion parameters and is trained on over 500,000 H\&E stained WSIs \cite{Saillard_Jenatton_etal._2024}. We extract outputs from four evenly spaced transformer blocks $Z_i$, where $i \in [1, 14, 26, 38]$, to serve as residual connections for the CNN decoder. We select these blocks based on our observation that features from non-adjacent levels of the encoder lead to better overall performance on the test subset.

The CNN decoder upsamples the feature representations, acquired from the transformer blocks, to generate an intermediate vector that is handled by two task-specific layers that generate cell segmentation and classification masks. The first task-specific layer is the ‘Cellpose head’,  which is used to delineate cell instances. The layer generates horizontal and vertical gradient maps to form vector fields that are refined through gradient tracking in a post-processing step using the Cellpose algorithm \cite{Stringer_Wang_etal._2021}, known for its efficacy in cell segmentation tasks and generalizability across multiple domains \cite{Pachitariu_Stringer_2022,Stringer_Pachitariu_2024}. The second task-specific layer is the "Cell type head", which assigns labels to individual pixels. In the post-processing step, we determine the output classification label of each segmented cell instance by majority voting over the labeled pixels that comprise the cell in the segmentation map.

To evaluate model performance and measure the impact of adding a foundation model as backbone, we compare it to a ResNet50-based model. ResNet50 is a widely used solution for encoders in segmentation architectures in the medical domain \cite{Deshmane_2023,Graham_Vu_etal._2019,Mukasheva_Koishiyeva_etal._2024,Stringer_Wang_etal._2021}. For the H-Optimus-based model, we utilize frozen weights for the encoder and only fine-tune the decoder to take advantage of the extensive pre-training of the foundation model. For the ResNet50-based model we start with ImageNet \cite{Deng_Dong_etal.} weights and train both encoder and decoder parts. Hyperparameters for the training step are set to be identical, where possible, for comparable evaluation. 
For this evaluation, we deliberately use the PanNuke dataset to provide a standardized and controlled comparison between the H‑Optimus and ResNet50-based models (\hyperref[fig:S2]{Appendix Figure S2 (3)}). Specifically, we use two of the default PanNuke dataset splits (66\%) for training and validation, and reserve the third split (33\%) for testing.

To address the challenge of cell class imbalance in the PanNuke dataset, which is a common characteristic in most cell-level H\&E patch datasets, both models’ training processes employ a weighted loss function comprising cross-entropy and focal loss \cite{Lin_Goyal_etal._2018}. The focal loss component is adjusted with coefficients derived from each cell class' instance frequency, emphasizing learning from underrepresented classes and enhancing the model's sensitivity to rare but significant cellular patterns. The cross-entropy loss is augmented with spectral decoupling regularization \cite{Pezeshki_Kaba_etal._2021,Pohjonen_Stürenberg_etal._2022} and spatially varying label smoothing \cite{Islam_Glocker_2021}, which potentially stabilizes training and improves generalization in case of complex tissue morphologies. For optimization, we employ the \textit{AdamW} \cite{Loshchilov_Hutter_2019} to counter unbalanced class scenarios, with cosine annealing learning rate scheduler.

We utilize the scikit-learn library \cite{Van_der_Walt_Schönberger_etal._2014} and HoVer-Net \cite{Graham_Vu_etal._2019} implementations of $R^2$ (the coefficient of determination) and $PQ$ (panoptic quality) to evaluate our experiments. Complete mathematical formulations and detailed explanations of these metrics are provided in \hyperref[chap:S5]{Appendix S5}. To compute confidence intervals, we use nonparametric bootstrapping, where after calculating the metric on the full sample, we generated 1000 bootstrap replicates by resampling with replacement and then determined the 95\% confidence intervals as the 2.5th and 97.5th percentiles of the resulting empirical distribution.

%\hfill

The model comparisons are summarized in \hyperref[tab:2]{Table 2}. The H‑Optimus-based model achieves higher $R^2$ across all cell classes compared to the ResNet50-based model, which means that its predictions are more closely aligned with the PanNuke cell counts, indicating a stronger correlation with the observed data. Notably, the improvement of $R^2_{dead}$ may be an indicator of better global contextual representations provided by the foundation model backbone. In terms of segmentation and classification quality combined, measured by the PQ score, the H‑Optimus-based model demonstrates notable improvements across most cell classes. Overall, the average $R^2$ improved from 0.575 to 0.871, while the average $PQ$ score improved from 0.450 to 0.492, demonstrating better performance of the H-Optimus-based model.

\begin{table}[h!]
\renewcommand{\arraystretch}{1.5}
  \centering
  \caption{Cell quantification metrics for baseline and proposed models (CI 95\%).}
  \label{tab:2}
  \begin{tabular}{|l|c|c|}
    \hline
    %\rowcolor{gray!30}
    Metric             & Resnet50-based            & H-optimus-based              \\
    \hline
    $R^2_{neoplastic}$    & 0.681 (0.576--0.769)       & \textbf{0.941 (0.917--0.960)} \\
    \hline
    $R^2_{inflammatory}$  & 0.863 (0.778--0.903)       & \textbf{0.949 (0.918--0.966)} \\
    \hline
    $R^2_{connective}$    & 0.600 (0.488--0.698)       & 0.609 (0.436--0.772)          \\
    \hline
    $R^2_{dead}$          & 0.097 (-11.389--0.669)     & 0.925 (0.404--0.982)          \\
    \hline
    $R^2_{epithelial}$    & 0.635 (0.490--0.747)       & \textbf{0.930 (0.886--0.964)} \\
    \hline
    $PQ_{neoplastic}$       & 0.517 (0.499--0.535)       & \textbf{0.589 (0.575--0.604)} \\
    \hline
    $PQ_{inflammatory}$     & 0.455 (0.429--0.482)       & \textbf{0.528 (0.507--0.549)} \\
    \hline
    $PQ_{connective}$       & 0.416 (0.400--0.431)       & \textbf{0.451 (0.436--0.465)} \\
    \hline
    $PQ_{dead}$             & 0.374 (0.342--0.408)       & 0.292 (0.209--0.365)          \\
    \hline
    $PQ_{epithelial}$       & 0.488 (0.460--0.519)       & \textbf{0.599 (0.579--0.618)} \\
    \hline
  \end{tabular}
\end{table}

Our results  show that integrating the H‑Optimus foundation model within the UNETR architecture enhances the model's ability to segment and classify cells across diverse tissues from PanNuke data. The pretrained transformer encoder provides robust feature representations, resulting in higher average $R^2$ and $PQ$ scores compared to the CNN-based model. This leads to more reliable cell quantification and more accurate downstream analysis. Additionally, the streamlined fine-tuning process reduces computational overhead and training time, making the model more adaptable for new data.

Despite these advancements, the foundation model-based approach does not fully resolve all challenges related to cell segmentation and classification. We observe lower metric scores for underrepresented classes in the training data. Furthermore, foundation models typically encompass billions of parameters, resulting in substantial computational and memory requirements. It therefore poses challenges for deployment in resource-constrained environments, limiting their practical applicability in certain clinical settings.

\section{Model optimization via Knowledge Distillation}

To address the limitations posed by the extensive size of foundation models, we implement knowledge distillation — a model compression technique that leverages the teacher-student paradigm \cite{Hinton_Vinyals_etal._2015}. By training a smaller, more efficient student model to replicate the output of a larger, pre-trained teacher model, we retain performance while significantly reducing the model's complexity and resource requirements (\hyperref[fig:fig6]{Figure 6}).

\begin{figure}[h!]
    \centering
    \includegraphics[width=\textwidth, height=0.45\textheight, keepaspectratio]{images/Figure_6.pdf}
    \caption{Knowledge distillation framework for training a student model using a pre-trained teacher}
    \label{fig:fig6}
\end{figure}

We employ knowledge distillation to compress the H‑Optimus-based teacher model into a more efficient student model. The teacher model is the modified UNETR architecture with the H‑Optimus foundation model described in the previous chapter. The student model is based on a UNet architecture augmented with residual connections and incorporates a smaller ViT encoder with 9 million parameters \cite{Steiner_Kolesnikov_etal._2022,Wightman_2019}. 

First, we fine-tune the teacher model using the refined dataset from the cross-relabeling procedure (Section 2). Initially we train the decoder of the teacher model while keeping the encoder weights frozen. We split the refined dataset into train (70\%), validation (20\%) and test (10\%) subsets (\hyperref[fig:S2]{Appendix Figure S2 (4)}). During fine-tuning, we use the train and validation subsets, while leaving the test subset for model evaluation. We set the training procedure and model hyperparameters to be identical to those that were used to demonstrate the utility of foundation models for the simultaneous cell segmentation and classification task.

Next, we perform knowledge distillation from teacher to student using the refined dataset used to fine-tune the teacher model. The student model is trained to replicate the teacher model's outputs. We utilize a specialized loss function that aligns the student's predicted probability distribution with the teacher's, incorporating the teacher's class probability distribution derived from the output. Following the methodology of Hinton et al. \cite{Hinton_Vinyals_etal._2015}, we experiment with various hyperparameter settings for the temperature ($T$) and the balancing coefficients ($\alpha$ and $\beta$) in the loss function. We vary $T$ from 1 to 20 and adjust $\alpha$ and $\beta$ to balance the distillation and student losses. Through iterative tuning and evaluation, we identify that setting $T=14$, $\alpha=0.3$, and $\beta=0.7$ yields a configuration that converges and closely approximates the teacher model's performance during training.

Finally, we assess the performance of both models using the $R^2$ and $PQ$ (defined in \hyperref[chap:S5]{Appendix S5}) on the test set of the refined dataset (\hyperref[tab:3]{Table 3}). We observe that the 95\% confidence intervals overlap for most cell types, so we cannot claim statistically significant performance differences between the teacher and student models. One exception appears in the neoplastic class. The teacher model produces an $R^2$ of 0.919, while the student model shows an $R^2$ of 0.852. In addition, the student model achieves higher $PQ$ values for the neoplastic and connective classes, though the confidence intervals show overlap.

\begin{table}[h!]
\renewcommand{\arraystretch}{1.5}
  \centering
  \caption{Cell quantification metrics for teacher and distilled student models (CI 95\%).}
  \label{tab:3}
  \begin{tabular}{|l|c|c|}
    \hline
    %\rowcolor{gray!30}
    Metric & Teacher & Student \\
    \hline
    $R^2_{neoplastic}$    & \textbf{0.919} (0.898--0.939) & 0.852 (0.800--0.891) \\
    \hline
    $R^2_{lymphocyte}$    & 0.969 (0.956--0.977)         & 0.969 (0.956--0.978) \\
    \hline
    $R^2_{connective}$    & 0.694 (0.548--0.809)         & 0.618 (0.469--0.741) \\
    \hline
    $R^2_{dead}$          & 0.755 (0.400--0.908)         & 0.424 (0.100--0.731) \\
    \hline
    $R^2_{epithelial}$    & 0.922 (0.870--0.958)         & 0.843 (0.738--0.917) \\
    \hline
    $R^2_{macrophage}$    & 0.384 (-0.369--0.724)        & 0.704 (0.352--0.859) \\
    \hline
    $R^2_{neutrofil}$     & 0.854 (0.578--0.929)         & 0.833 (0.502--0.925) \\
    \hline
    $PQ_{neoplastic}$       & 0.581 (0.569--0.593)         & 0.601 (0.588--0.613) \\
    \hline
    $PQ_{lymphocyte}$       & 0.536 (0.520--0.553)         & 0.563 (0.544--0.579) \\
    \hline
    $PQ_{connective}$       & 0.436 (0.421--0.451)         & 0.457 (0.441--0.474) \\
    \hline
    $PQ_{dead}$             & 0.272 (0.235--0.315)         & 0.279 (0.201--0.369) \\
    \hline
    $PQ_{epithelial}$       & 0.522 (0.500--0.545)         & 0.530 (0.506--0.555) \\
    \hline
    $PQ_{macrophage}$       & 0.524 (0.459--0.588)         & 0.474 (0.405--0.543) \\
    \hline
    $PQ_{neutrofil}$        & 0.541 (0.490--0.592)         & 0.565 (0.522--0.607) \\
    \hline
  \end{tabular}
\end{table}


We further decompose the $PQ$ metric into its $SQ$ and $DQ$ components (\hyperref[tab:S6]{Appendix Table S6}). Both models produce nearly identical $SQ$ values, which indicates that they predict instance boundaries with similar precision. Although the student model shows some improvement in $DQ$ scores for certain classes, the confidence intervals overlap and do not confirm a statistically significant difference.

We observe that the student and teacher models yield comparable detection performance despite the student model using a much smaller and simpler architecture. A model with fewer parameters reduces the risk of overfitting when training data are scarce relative to the model’s complexity \cite{Farias_Ludermir_etal._2022}. The knowledge distillation process also encourages the student model to focus on the most generalizable detection features learned from the teacher. These factors enable the student model to achieve similar detection performance across different cell types.

Additionally, considering the model sizes reported in \hyperref[tab:4]{Table 4}, the distilled model achieves a significant reduction compared to the teacher model, with a 48-fold decrease in parameter count and a 5.5-fold reduction in on-disk size. In inference mode, the teacher model requires 16 GB of VRAM for a batch size of 32, while the distilled model only needs 3 GB of VRAM for the same batch size. These reductions make the distilled model significantly more practical for fine-tuning and deployment in resource-constrained environments.

\begin{table}[h!]
\renewcommand{\arraystretch}{1.5}
  \centering
  \caption{Parameter counts and size of teacher and distilled model}
  \label{tab:4}
  \adjustbox{max width=\textwidth}{%
  \begin{tabular}{|l|c|c|c|}
    \hline
    %\rowcolor{gray!30}
    Metric & H-optimus-based (Teacher) & mobileViT-based (Student) & Magnitude of difference \\
    \hline
    Parameters count       & 1,158,917,906   & \textbf{24,093,393}   & \textbf{48x}  \\
    \hline
    Estimated Total Size (MB) & 87,912       & \textbf{15,935}    & \textbf{5.5x} \\
    \hline
  \end{tabular}%
}
\end{table}

%\hfill

With recent advancements in complex network architectures and the use of pretrained encoders to achieve state-of-the-art performance \cite{Baumann_Dislich_etal._2024,Hörst_Rempe_etal._2024} in cell segmentation and classification tasks, model size, computational complexity, and processing times have increased. This limits the scalability and accessibility of these models. As we demonstrate, this may be mitigated using knowledge distillation. Studies in the field of natural language processing have demonstrated the efficacy of knowledge distillation in retaining the capabilities of the teacher model while achieving significant reductions in size and complexity \cite{Huangpu_Gao_2024,Sun_Yu_etal.}. 

We demonstrate the feasibility of knowledge distillation in digital pathology, specifically for cell segmentation and classification tasks. Moreover, we achieve this performance while also significantly reducing the parameter count. In addressing the challenge of knowledge transfer, we found that distillation from a transformer-based model to a smaller transformer is more straightforward than attempting to map transformer features to CNN blocks. In our experiments, using a CNN-based network as a student results in worse cell quantification performance due to the structural constraints of CNN feature space dimensions. 

Although our primary approach relies on a transformer-based student model that performs well, it can be further optimized to incorporate advantages from CNN architectures. For example, employing alternative techniques such as using ViT adapters \cite{Chen_Duan_etal._2023} or $1 \times 1$ convolutions to adjust feature map sizes may be beneficial for harnessing CNN advantages like enhanced local feature extraction. Moreover, if additional performance improvements are desired, the process can be further enhanced by applying supplementary knowledge distillation techniques, such as self-distillation \cite{Zhang_Song_etal._2019} or online distillation \cite{Houyon_Cioppa_etal._2023}.

Despite these promising results, further validation on independent datasets is necessary to fully understand the model's limitations. Underrepresented classes may pose challenges when addressing complex cases. Pathologists need to validate these models to adopt them in clinical settings. While the distilled models are smaller and more deployable, a technological gap persists because pathologists traditionally rely on established methods for inspecting WSIs and diagnosing diseases. Addressing the complexities involved in deploying models for inference and supporting pathologists in adopting new tools is essential for integrating these models into clinical workflows.

\section{Model integration with QuPath}
Digital pathology tools with graphical user interfaces are essential for visualizing and analyzing WSIs. To make our student model useful in clinical pathology workflows, it needs to be integrated into a tool that enables inspecting regions, creating annotations, and providing quantitative analyses of biomarkers. Therefore, we integrate the trained student model from the previous chapter into the QuPath open‑source platform \cite{Bankhead_Loughrey_etal._2017}. QuPath provides the required annotation, visualization, and analysis tools to interpret complex histological data, including workflows for cell segmentation, classification, and quantification (\hyperref[fig:fig7]{Figure 7}). 

\begin{figure}[h!]
    \centering
    \includegraphics[width=\textwidth]{images/Figure_7.pdf}
    \caption{Visualization of model-generated cell quantification annotations (left) and the corresponding unannotated slide (right) in QuPath}
    \label{fig:fig7}
\end{figure}

To identify the regions in a WSI critical for prognosticating tumor development, such as specific tumor areas or border regions without overlapping healthy tissue, the pathologist uses QuPath to outline these regions. Then, the pathologist initiates a cell segmentation and classification script through the QuPath interface for the selected regions. The resulting annotations and quantified cell information are then directly overlaid onto the WSI in the QuPath interface. Additional design and implementation details are in \hyperref[chap:S7]{Appendix S7}. 

Two common approaches for integrating deep learning models into QuPath are Java‑based native QuPath extensions \cite{Goldsborough_Philps_etal._2024} and the execution of RESTful API requests to a model server coupled with handling the response via an extension, as demonstrated in the application of cell segmentation models applied to immunofluorescence images \cite{Sugawara_2023}. While the community is actively working on these integration strategies, there is currently no universal solution that fully addresses all integration and performance requirements.

Extensions may offer better integration with QuPath, allowing slightly improved performance and more widespread usage of the built-in QuPath models, but they lack the flexibility to customize models and modify their behavior. For example, the newest version of QuPath includes models such as StarDist \cite{Weigert_Schmidt} and InstanSeg \cite{Goldsborough_Philps_etal._2024} that can perform cell segmentation. Both models pose limitations when applied to simultaneous cell segmentation and classification. StarDist performs well only on convex, round shapes by design, whereas some neoplastic, inflammatory, and connective cells exhibit complex and non-convex shapes. InstanSeg provides only semantic segmentation without assigning classes to the segmented cells.

%\hfill

In contrast, our approach offers an alternative integration strategy. It utilizes the paquo library to directly interact with QuPath’s internal application programming interface from within Python. This enables data exchange and processing without the need for intermediate conversion steps and provides greater control over model customization, retraining, and the incorporation of custom processing steps.

The integration of our custom model with QuPath underscores its potential to significantly enhance the diagnostic process by reducing the time burden on pathologists and enabling them to focus on more complex interpretative tasks using familiar software. Leveraging a tool that is already well-established among pathologists increases the likelihood of its adoption into daily clinical workflows. The quantitative data generated through the automated workflow is critical for both clinical decision-making and research, facilitating more accurate biomarker analysis, enabling robust statistical evaluations, and supporting hypothesis generation and testing. Additionally, by streamlining cell segmentation and classification, the tool enhances the scalability and reproducibility of pathological assessments, ultimately contributing to improved diagnostic accuracy and patient outcomes.

\section{Conclusion and future work}

In this study, we address critical challenges in digital pathology and tackle the usability and deployment issues of the developed models in standard computing environments without the need for high-performance computing systems. Our multi-faceted approach encompasses data refinement through cross-relabeling, leveraging foundation models for robust cell segmentation and classification, optimizing model performance via knowledge distillation, and integrating the optimized model into the QuPath software for practical application. This approach is used to construct a capable, versatile, and adjustable model for cell segmentation and classification, with enhanced performance and usability.

\begin{sloppypar}
While our approach shows potential in the field of computational pathology, certain limitations persist. 
For example, our implementation currently exhibits lower performance in detecting macrophages. 
This serves as an instance of the broader challenge of accurately identifying complex cell types. In order to address this issue, extending our approach to incorporate additional data sources, exploring alternative modeling approaches, and integrating other imaging modalities such as immunohistochemical staining may help improve detection accuracy. Moreover, although the distilled model reduces computational demands, integrating advanced deep learning models into clinical practice requires addressing technological gaps and potential resistance to adopting new tools within established diagnostic processes.
\end{sloppypar}

Future work could focus on several key areas to refine the proposed approach and facilitate its adoption in clinical environments. Enhancing the cell-relabeling process with additional datasets \cite{Graham_Jahanifar_etal._2021} could improve the representation of underrepresented cell types and enhance overall model performance. Also, incorporating additional data sources, such as multi-modal imaging or complementary staining methods, may address limitations related to cell type differentiation and class imbalance. Exploring other foundation models \cite{Vorontsov_Bozkurt_etal._2024,Zimmermann_Vorontsov_etal._2024} or introducing additional modalities \cite{Ding_Wagner_etal._2024,Vaidya_Zhang_etal._2025} may provide alternative architectures better suited to specific tasks or offer improved efficiency. Implementing more complex knowledge distillation techniques \cite{Houyon_Cioppa_etal._2023,Zhang_Song_etal._2019} could further optimize the model's performance and adaptability. Additionally, deeper integration with QuPath or other digital pathology software could provide pathologists more control over cell quantification analysis directly within the QuPath interface, thereby increasing accessibility and usability. Such enhancements would not only refine model performance but also ensure greater adaptability and scalability within various clinical environments. Finally, extensive validation of the model by pathologists and benchmarking against independent datasets are essential steps toward establishing the model's reliability and fostering confidence in its clinical utility.

\section*{Acknowledgments} 
This work was funded in part by the Research Council of Norway grant no. 309439 SFI Visual Intelligence, and the North Norwegian Health Authority grant no. HNF1521-20.

\bibliographystyle{IEEEtran}
\begin{sloppypar}
\begin{thebibliography}{99}

\bibitem{chaplot2020neural} Chaplot, Devendra Singh, et al. "Neural topological slam for visual navigation." Proceedings of the IEEE/CVF conference on computer vision and pattern recognition. 2020.

\bibitem{maksymets2021thda} Maksymets, Oleksandr, et al. "Thda: Treasure hunt data augmentation for semantic navigation." Proceedings of the IEEE/CVF International Conference on Computer Vision. 2021.

\bibitem{mezghan2022memory} Mezghan, Lina, et al. "Memory-augmented reinforcement learning for image-goal navigation." 2022 IEEE/RSJ International Conference on Intelligent Robots and Systems (IROS). IEEE, 2022.

\bibitem{al2022zero} Al-Halah, Ziad, Santhosh Kumar Ramakrishnan, and Kristen Grauman. "Zero experience required: Plug \& play modular transfer learning for semantic visual navigation." Proceedings of the IEEE/CVF Conference on Computer Vision and Pattern Recognition. 2022.

\bibitem{ye2021auxiliary} Ye, Joel, et al. "Auxiliary tasks and exploration enable objectgoal navigation." Proceedings of the IEEE/CVF international conference on computer vision. 2021.

\bibitem{chaplot2020object} Chaplot, Devendra Singh, et al. "Object goal navigation using goal-oriented semantic exploration." Advances in Neural Information Processing Systems 33 (2020)

\bibitem{ramakrishnan2022poni} Ramakrishnan, Santhosh Kumar, et al. "Poni: Potential functions for objectgoal navigation with interaction-free learning." Proceedings of the IEEE/CVF Conference on Computer Vision and Pattern Recognition. 2022.

\bibitem{ramrakhya2022habitat} Ramrakhya, Ram, et al. "Habitat-web: Learning embodied object-search strategies from human demonstrations at scale." Proceedings of the IEEE/CVF Conference on Computer Vision and Pattern Recognition. 2022.

\bibitem{mousavian2019visual} Mousavian, Arsalan, et al. "Visual representations for semantic target driven navigation." 2019 International Conference on Robotics and Automation (ICRA). IEEE, 2019.

\bibitem{dhariwal2021diffusion} Dhariwal, Prafulla, and Alexander Nichol. "Diffusion models beat gans on image synthesis." Advances in neural information processing systems 34 (2021)

\bibitem{ho2022classifier} Ho, Jonathan, and Tim Salimans. "Classifier-free diffusion guidance." arXiv preprint arXiv:2207.12598 (2022).

\bibitem{nichol2021glide} Nichol, Alex, et al. "Glide: Towards photorealistic image generation and editing with text-guided diffusion models." arXiv preprint arXiv:2112.10741 (2021)

\bibitem{brooks2023instructpix2pix} Brooks, Tim, Aleksander Holynski, and Alexei A. Efros. "Instructpix2pix: Learning to follow image editing instructions." Proceedings of the IEEE/CVF Conference on Computer Vision and Pattern Recognition. 2023.

\bibitem{fu2023guiding} Fu, Tsu-Jui, et al. "Guiding instruction-based image editing via multimodal large language models." arXiv preprint arXiv:2309.17102 (2023).

\bibitem{geng2024instructdiffusion} Geng, Zigang, et al. "Instructdiffusion: A generalist modeling interface for vision tasks." Proceedings of the IEEE/CVF Conference on Computer Vision and Pattern Recognition. 2024.

\bibitem{zhou2024minedreamer} Zhou, Enshen, et al. "Minedreamer: Learning to follow instructions via chain-of-imagination for simulated-world control." arXiv preprint arXiv:2403.12037 (2024).

\bibitem{zhou2023esc} Zhou, Kaiwen, et al. "Esc: Exploration with soft commonsense constraints for zero-shot object navigation." International Conference on Machine Learning. PMLR, 2023.

\bibitem{yu2023l3mvn} Yu, Bangguo, Hamidreza Kasaei, and Ming Cao. "L3mvn: Leveraging large language models for visual target navigation." 2023 IEEE/RSJ International Conference on Intelligent Robots and Systems (IROS). IEEE, 2023.

\bibitem{gadre2023cows} Gadre, Samir Yitzhak, et al. "Cows on pasture: Baselines and benchmarks for language-driven zero-shot object navigation." Proceedings of the IEEE/CVF Conference on Computer Vision and Pattern Recognition. 2023.

\bibitem{shah2023navigation} Shah, Dhruv, et al. "Navigation with large language models: Semantic guesswork as a heuristic for planning." Conference on Robot Learning. PMLR, 2023.

\bibitem{cai2024bridging} Cai, Wenzhe, et al. "Bridging zero-shot object navigation and foundation models through pixel-guided navigation skill." 2024 IEEE International Conference on Robotics and Automation (ICRA). IEEE, 2024.

\bibitem{yu2023co} Yu, Bangguo, Hamidreza Kasaei, and Ming Cao. "Co-NavGPT: Multi-robot cooperative visual semantic navigation using large language models." arXiv preprint arXiv:2310.07937 (2023).

\bibitem{wu2024voronav} Wu, Pengying, et al. "Voronav: Voronoi-based zero-shot object navigation with large language model." arXiv preprint arXiv:2401.02695 (2024).

\bibitem{qin2023mp5} Qin, Yiran, et al. "Mp5: A multi-modal open-ended embodied system in minecraft via active perception." arXiv preprint arXiv:2312.07472 (2023).

\bibitem{du2024learning} Du, Yilun, et al. "Learning universal policies via text-guided video generation." Advances in Neural Information Processing Systems 36 (2024).

\bibitem{ajay2024compositional} Ajay, Anurag, et al. "Compositional foundation models for hierarchical planning." Advances in Neural Information Processing Systems 36 (2024).

\bibitem{liang2024skilldiffuser} Liang, Zhixuan, et al. "Skilldiffuser: Interpretable hierarchical planning via skill abstractions in diffusion-based task execution." Proceedings of the IEEE/CVF Conference on Computer Vision and Pattern Recognition. 2024.

\bibitem{heusel2017gans} Heusel, Martin, et al. "Gans trained by a two time-scale update rule converge to a local nash equilibrium." Advances in neural information processing systems 30 (2017).

\bibitem{zhang2018unreasonable} Zhang, Richard, et al. "The unreasonable effectiveness of deep features as a perceptual metric." Proceedings of the IEEE conference on computer vision and pattern recognition. 2018.

\bibitem{brown2020language} Brown, Tom B. "Language models are few-shot learners." arXiv preprint arXiv:2005.14165 (2020).

\bibitem{podell2023sdxl} Podell, Dustin, et al. "Sdxl: Improving latent diffusion models for high-resolution image synthesis." arXiv preprint arXiv:2307.01952 (2023).

\bibitem{brohan2022rt} Brohan, Anthony, et al. "Rt-1: Robotics transformer for real-world control at scale." arXiv preprint arXiv:2212.06817 (2022).

\bibitem{brohan2023rt} Brohan, Anthony, et al. "Rt-2: Vision-language-action models transfer web knowledge to robotic control." arXiv preprint arXiv:2307.15818 (2023).

\bibitem{li2024manipllm} Li, Xiaoqi, et al. "Manipllm: Embodied multimodal large language model for object-centric robotic manipulation." Proceedings of the IEEE/CVF Conference on Computer Vision and Pattern Recognition. 2024.

\bibitem{shah2023vint} Shah, Dhruv, et al. "ViNT: A foundation model for visual navigation." arXiv preprint arXiv:2306.14846 (2023).

\bibitem{liu2024visual} Liu, Haotian, et al. "Visual instruction tuning." Advances in neural information processing systems 36 (2024).

\bibitem{hu2021lora} Hu, Edward J., et al. "Lora: Low-rank adaptation of large language models." arXiv preprint arXiv:2106.09685 (2021).

\bibitem{qin2023supfusion} Qin, Yiran, et al. "SupFusion: Supervised LiDAR-camera fusion for 3D object detection." Proceedings of the IEEE/CVF International Conference on Computer Vision. 2023.

\bibitem{qin2024worldsimbench} Qin, Yiran, et al. "Worldsimbench: Towards video generation models as world simulators." arXiv preprint arXiv:2410.18072 (2024).

\bibitem{yu2025gamefactory} Yu, Jiwen, et al. "GameFactory: Creating New Games with Generative Interactive Videos." arXiv preprint arXiv:2501.08325 (2025).

\bibitem{zhou2024code} Zhou, Enshen, et al. "Code-as-Monitor: Constraint-aware Visual Programming for Reactive and Proactive Robotic Failure Detection." arXiv preprint arXiv:2412.04455 (2024).

\bibitem{zhang2024ad} Zhang, Zaibin, et al. "AD-H: Autonomous Driving with Hierarchical Agents." arXiv preprint arXiv:2406.03474 (2024).

\bibitem{wang2024toward} Wang, Chaoqun, et al. "Toward Accurate Camera-based 3D Object Detection via Cascade Depth Estimation and Calibration." arXiv preprint arXiv:2402.04883 (2024).

\bibitem{huang2024story3d} Huang, Yuzhou, et al. "Story3d-agent: Exploring 3d storytelling visualization with large language models." arXiv preprint arXiv:2408.11801 (2024).

\bibitem{savinov2018semi} Savinov, Nikolay, Alexey Dosovitskiy, and Vladlen Koltun. "Semi-parametric topological memory for navigation." arXiv preprint arXiv:1803.00653 (2018).

\bibitem{majumdar2022zson} Majumdar, Arjun, et al. "Zson: Zero-shot object-goal navigation using multimodal goal embeddings." Advances in Neural Information Processing Systems 35 (2022): 32340-32352.

\bibitem{yadav2023offline} Yadav, Karmesh, et al. "Offline visual representation learning for embodied navigation." Workshop on Reincarnating Reinforcement Learning at ICLR 2023. 2023.

\bibitem{yadav2023ovrl} Yadav, Karmesh, et al. "Ovrl-v2: A simple state-of-art baseline for imagenav and objectnav." arXiv preprint arXiv:2303.07798 (2023).

\bibitem{sun2024fgprompt} Sun, Xinyu, et al. "FGPrompt: fine-grained goal prompting for image-goal navigation." Advances in Neural Information Processing Systems 36 (2024).

\bibitem{zhu2017target} Zhu, Yuke, et al. "Target-driven visual navigation in indoor scenes using deep reinforcement learning." 2017 IEEE international conference on robotics and automation (ICRA). IEEE, 2017.

\bibitem{koh2024generating} Koh, Jing Yu, Daniel Fried, and Russ R. Salakhutdinov. "Generating images with multimodal language models." Advances in Neural Information Processing Systems 36 (2024).

\bibitem{krantz2022instance} Krantz, Jacob, et al. "Instance-specific image goal navigation: Training embodied agents to find object instances." arXiv preprint arXiv:2211.15876 (2022).

\bibitem{schulman2017proximal} Schulman, John, et al. "Proximal policy optimization algorithms." arXiv preprint arXiv:1707.06347 (2017).

\bibitem{anderson2018evaluation} Anderson, Peter, et al. "On evaluation of embodied navigation agents." arXiv preprint arXiv:1807.06757 (2018).

\bibitem{lin2024navcot} Lin, Bingqian, et al. "NavCoT: Boosting LLM-Based Vision-and-Language Navigation via Learning Disentangled Reasoning." arXiv preprint arXiv:2403.07376 (2024).

\bibitem{NavGPT} Zhou, Gengze, Yicong Hong, and Qi Wu. "Navgpt: Explicit reasoning in vision-and-language navigation with large language models." Proceedings of the AAAI Conference on Artificial Intelligence.

\bibitem{hahn2021no} Hahn, Meera, et al. "No rl, no simulation: Learning to navigate without navigating." Advances in Neural Information Processing Systems 34 (2021): 26661-26673.

\bibitem{li2025t2isafety} Li, Lijun, et al. "T2ISafety: Benchmark for Assessing Fairness, Toxicity, and Privacy in Image Generation." arXiv preprint arXiv:2501.12612 (2025).

\bibitem{an2024agfsync} An, Jingkun, et al. "AGFSync: Leveraging AI-Generated Feedback for Preference Optimization in Text-to-Image Generation." arXiv preprint arXiv:2403.13352 (2024).


\end{thebibliography}
\end{sloppypar}

\clearpage
\beginsupplement
\section*{Appendix}
\renewcommand{\thesubsection}{S\arabic{subsection}}

\subsection{\label{chap:S1}PanNuke and MoNuSAC preprocessing}
The PanNuke dataset comprises a set of 7,901 RGB patches, each with dimensions of $256 \times 256$ pixels, which we set as the standard patch size for our analysis. In contrast, the MoNuSAC dataset encompasses 294 images of heterogeneous dimensions. To standardize the MoNuSAC images with our experiments, we implement a standardization protocol. Specifically, for images exceeding the dimensions of $256 \times 256$ pixels, we segment them into equal-sized patches and apply mirror padding to the remaining portions to avoid information loss at the peripherals. Patches with dimensions less than $128 \times 128$ pixels are excluded from the dataset due to the insufficient resolution to capture relevant cellular details. For patches where either dimension falls between 128 and 256 pixels, we employ upsampling to achieve the standard patch size. As a result, we obtain a total of 2,823 RGB patches derived from the MoNuSAC dataset for subsequent analysis. For additional details on the MoNuSAC data preparation process, refer to the source code \cite{Shvetsov_2025a}.
\clearpage

\subsection{\label{chap:S2}Data usage for the methodology}

\counterwithin{figure}{subsection}
\renewcommand{\thefigure}{S\arabic{subsection}}

\begin{figure}[h!]
    \centering
    \includegraphics[width=\textwidth, height=0.85\textheight, keepaspectratio]{images/A2.pdf}
    \caption{Overview of the methodology for cross-labeling, dataset refinement, and model comparison. (1) Cross-relabeling - training and testing cell classification models, (2) Cross-relabeling - using cell classification models to create refined dataset, (3) Fine-tuning and training models for comparison, (4) Student knowledge distillation with refined dataset}
    \label{fig:S2}
\end{figure}
\clearpage

\subsection{\label{chap:S3}Confusion matrices for classification models}
\counterwithin{figure}{subsection}
\renewcommand{\thefigure}{S\arabic{subsection}.\arabic{figure}}

\begin{figure}[h!]
    \centering
    \includegraphics[width=\textwidth, height=0.4\textheight, keepaspectratio]{images/A3_1.pdf}
    \caption{Confusion matrix for PanNuke trained model}
    \label{fig:S3.1}
\end{figure}

\begin{figure}[h!]
    \centering
    \includegraphics[width=\textwidth, height=0.4\textheight, keepaspectratio]{images/A3_2.pdf}
    \caption{Confusion matrix for MoNuSAC trained model}
    \label{fig:S3.2}
\end{figure}

\clearpage

\subsection{\label{chap:S4}Datasets cell counts}

\counterwithin{table}{subsection}
\renewcommand{\thetable}{S\arabic{subsection}}

\begin{table}[h!]
\renewcommand{\arraystretch}{2.0}
\centering
\caption{\label{tab:S4}Cell counts for PanNuke, MoNuSAC and refined datasets. Numbers in parentheses indicate preprocessed cell counts for cell classifier models training and testing.}
%\adjustbox{max width=\textwidth}{%
\begin{tabular}{|l|c|c|c|}
\hline
%\rowcolor{gray!30}
Cell type & PanNuke & MoNuSAC & Refined \\
\hline
Neoplastic & 77,403 (68,031) & - & 105,451 \\
\hline
Epithelial & 26,572 (23,207) & - & 29,926 \\
\hline
Epithelial (benign and malignant) & - & 31,402 & - \\
\hline
Inflammatory & 32,276 & - & - \\
\hline
Lymphocytes & - & 37,045 (33,104) & 65,275 \\
\hline
Neutrophils & - & 1,355 (1,252) & 3,833 \\
\hline
Macrophage & - & 1,842 (1,695) & 3,410 \\
\hline
Dead & 2,908 & - & 2,908 \\
\hline
Connective & 50,585 & - & 50,585 \\
\hline
\end{tabular}
%
%}
\end{table}



\clearpage

\subsection{\label{chap:S5}Definition of validation metrics}
\counterwithin{equation}{subsection}
\renewcommand{\theequation}{\arabic{equation}}

\subsubsection{\label{chap:S5.1}R\textsuperscript{2}}
The coefficient of determination, denoted as $R^2$, is a statistical measure that represents the proportion of variance in the dependent variable that is predictable from the independent variables. In the context of cell quantification in pathology, $R^2$ is used to assess how well the predicted quantities of different cell types in a patch align with the actual quantities observed in the ground truth data, with higher values representing more accurate quantification. $R^2$ is defined as
\begin{equation*}
R^2 = 1 - \frac{\sum_{i=1}^n (y_i - \hat{y}_i)^2}{\sum_{i=1}^n (y_i - \bar{y})^2},
\end{equation*}
where $y_i$ represents the actual number of cells of a specific type in the $i$-th image, $\hat{y}_i$ represents the predicted number of cells of that type in the $i$-th image, $\bar{y}$ is the mean of the actual numbers across all images, and $n$ is the total number of images in the dataset.

The $R^2$ metric has a range of $(-\infty, 1]$. An $R^2$ of 1 indicates perfect prediction, where all predicted values exactly match the actual values. An $R^2$ of 0 suggests that the model explains none of the variability of the response data around its mean. If $R^2$ is negative, it indicates that the model performs worse than a model that simply predicts the mean of the actual values for all observations.

\subsubsection{\label{chap:S5.2}PQ}
Panoptic Quality ($PQ$) is a comprehensive metric used to evaluate the performance of segmentation models in tasks that require both instance segmentation and classification. $PQ$ provides a single score that encapsulates both the detection accuracy (i.e., how many objects were correctly identified) and the segmentation quality (i.e., how accurately the objects' boundaries were delineated). This metric is particularly useful in multiclass scenarios where each pixel is classified into distinct categories, such as different cell types in pathology images.

$PQ$ is calculated as the product of two terms: Detection Quality ($DQ$) and Segmentation Quality ($SQ$). It can be expressed as
\begin{equation*}
PQ = DQ \cdot SQ,
\end{equation*}
where
\begin{equation*}
DQ = \frac{TP}{TP + 0.5\, FP + 0.5\, FN},
\end{equation*}
\begin{equation*}
SQ = \frac{\sum_{(p, g) \in \mathcal{M}} IoU(p, g)}{TP}.
\end{equation*}
In these formulas, $TP$ denotes the number of correctly matched instances between ground truth and prediction, $FP$ denotes the predicted instances that have no corresponding ground truth, $FN$ denotes the ground truth instances that were not detected, $IoU(p, g)$ is the Intersection over Union for a pair of matched instances $p$ (prediction) and $g$ (ground truth), and $\mathcal{M}$ is the set of matched pairs.

The $PQ$ metric is calculated for each class and is averaged across classes to provide a global performance measure.

The $PQ$ score has a range of $[0, 1.0]$, where a higher score indicates better performance in both detecting and segmenting the instances correctly. A $PQ$ of 1 signifies perfect identification and segmentation of all instances, whereas a $PQ$ of 0 indicates that no instances were correctly identified and segmented.

\clearpage

\subsection{\label{chap:S6}Segmentation and Detection quality metrics for teacher and student models}

\begin{table}[h!]
\renewcommand{\arraystretch}{2.0}
\centering
\caption{Segmentation and detection quality for student and teacher models (CI 95\%)}
\label{tab:S6}
%\adjustbox{max width=\textwidth}{%
\begin{tabular}{|l|c|c|}
\hline
%\rowcolor{gray!30}
Metric & Teacher & Student \\
\hline
$SQ_{neoplastic}$ & 0.819 (0.815--0.823) & 0.824 (0.819--0.828) \\
\hline
$SQ_{lymphocyte}$ & 0.795 (0.788--0.802) & 0.790 (0.783--0.796) \\
\hline
$SQ_{connective}$ & 0.770 (0.762--0.776) & 0.780 (0.772--0.786) \\
\hline
$SQ_{dead}$ & 0.659 (0.623--0.688) & 0.657 (0.624--0.695) \\
\hline
$SQ_{epithelial}$ & 0.780 (0.770--0.790) & 0.788 (0.779--0.797) \\
\hline
$SQ_{macrophage}$ & 0.788 (0.760--0.810) & 0.757 (0.730--0.783) \\
\hline
$SQ_{neutrofil}$ & 0.782 (0.761--0.801) & 0.775 (0.759--0.792) \\
\hline
$DQ_{neoplastic}$ & 0.706 (0.692--0.719) & 0.727 (0.712--0.741) \\
\hline
$DQ_{lymphocyte}$ & 0.675 (0.656--0.698) & 0.713 (0.691--0.734) \\
\hline
$DQ_{connective}$ & 0.566 (0.546--0.584) & 0.583 (0.565--0.602) \\
\hline
$DQ_{dead}$ & 0.410 (0.361--0.465) & 0.435 (0.306--0.561) \\
\hline
$DQ_{epithelial}$ & 0.668 (0.639--0.694) & 0.673 (0.644--0.702) \\
\hline
$DQ_{macrophage}$ & 0.657 (0.583--0.727) & 0.615 (0.531--0.703) \\
\hline
$DQ_{neutrofil}$ & 0.691 (0.625--0.753) & 0.729 (0.679--0.778) \\
\hline
\end{tabular}
%
%}
\end{table}

\clearpage

\subsection{\label{chap:S7}QuPath integration method}
We adopt an integration strategy leveraging the paquo \cite{Bayer_AG} library, a Python package that enables direct interaction with QuPath’s internal API, thereby facilitating seamless data exchange without intermediate conversion steps. The data processing pipeline (\hyperref[fig:S7]{Appendix Figure S7}) begins with the acquisition of WSIs and their associated annotations from QuPath, which are represented as Shapely \cite{Gillies_Wel_etal._2024} polygons. Utilizing paquo, we directly read, create, and modify these annotations and detections within a QuPath project in the Python environment. Images are then cropped using these polygons and processed by cell segmentation and classification models employing standard vision processing toolkits such as OpenCV, pyvips, and PyTorch. Additionally, QuPath employs Groovy scripts to initiate a Python process that starts the entire pipeline from QuPath graphical interface: fetching polygons, extracting images from them, and running deep learning model inference on the cropped images. 
The results are returned to QuPath, leveraging paquo's Python bindings to manipulate QuPath data while minimizing the computational overhead typically associated with cross-environment communication.

\counterwithin{figure}{subsection}
\renewcommand{\thefigure}{S\arabic{subsection}}

\begin{figure}[h!]
    \centering
    \includegraphics[width=\textwidth]{images/A7.pdf}
    \caption{QuPath integration workflow using Python environment}
    \label{fig:S7}
\end{figure}

Compared to traditional workflows that involve exporting annotations as GeoJSON, classifying them in Python, and reimporting them into QuPath, our approach offers several advantages. We eliminate the need to switch between programming languages, providing a cohesive and streamlined development process entirely within QuPath software and removing the necessity to use other tools. Meanwhile, we avoid storing annotations as intermediate JSON files unless required for external use or archiving. By conducting the entire inference and post-processing workflow within the Python environment, we leverage the power and flexibility of Python libraries for image processing and machine learning. This approach also enables adjustments to any set of labels and models, thereby improving its applicability.

%\hfill

The distilled model and QuPath integration code are packaged into a Docker container, enabling streamlined execution with the Docker engine. Detailed integration code and deployment instructions can be found in the GitHub repository \cite{Shvetsov_2025b}.

Despite these benefits, we acknowledge that the paquo library is a proof‑of‑concept project in its early development stage and has not been tested across all versions of QuPath.

\clearpage

\subsection{\label{chap:S8}Data and code availability statement}
All datasets, models, and code used in this study are publicly available and can be obtained from the repositories listed below. 
The PanNuke \cite{Gamper_Koohbanani_etal._2019} and MoNuSAC \cite{Verma_Kumar_etal._2021} datasets are publicly accessible, and download information along with detailed descriptions can be found in their respective articles. Preprocessing scripts for PanNuke and MoNuSAC data, as well as individual cell extraction scripts, are available on GitHub \cite{Shvetsov_2025a}. The H-Optimus foundation model used in our experiments can be downloaded from the HuggingFace repository \cite{hoptimus2024}, and model information is available on GitHub \cite{Saillard_Jenatton_etal._2024}. In addition, the integration code for QuPath and the distilled model packaged in a Docker container are provided in the repository \cite{Shvetsov_2025b}, and paquo Python library is available from the authors GitHub repository \cite{Bayer_AG}.
\clearpage

\end{document}

% \bibliographystyle{plainnat}
% \bibliography{references}

\appendix

\subsection{Implementation Details}
\label{appendix-implement}

% We summarize training and inference of GazeBot in Algorithms \ref{alg:model-training} and \ref{alg:model-inference}.

% \begin{algorithm}
% \caption{Training Algorithm}
% \label{alg:model-training}
% \begin{algorithmic}
% \Require $n \geq 0$
% \Ensure $y = x^n$
% \State $y \gets 1$
% \State $X \gets x$
% \State $N \gets n$
% \While{$N \neq 0$}
% \If{$N$ is even}
%     \State $X \gets X \times X$
%     \State $N \gets \frac{N}{2}$  \Comment{This is a comment}
% \ElsIf{$N$ is odd}
%     \State $y \gets y \times X$
%     \State $N \gets N - 1$
% \EndIf
% \EndWhile
% \end{algorithmic}
% \end{algorithm}

% \begin{algorithm}
% \caption{Inference Algorithm}
% \label{alg:model-inference}
% \begin{algorithmic}
% \Require $n \geq 0$
% \Ensure $y = x^n$
% \State $y \gets 1$
% \State $X \gets x$
% \State $N \gets n$
% \While{$N \neq 0$}
% \If{$N$ is even}
%     \State $X \gets X \times X$
%     \State $N \gets \frac{N}{2}$  \Comment{This is a comment}
% \ElsIf{$N$ is odd}
%     \State $y \gets y \times X$
%     \State $N \gets N - 1$
% \EndIf
% \EndWhile
% \end{algorithmic}
% \end{algorithm}

% デモの例を見せる
We collected demonstrations through human teleoperation, and the teleoperation system is summarized in Figure \ref{fig:teleop-system}. 

Examples of the collected demonstrations for the PenInCup, OpenCap, and PileBox tasks are shown in Figure \ref{fig:demos}.
The images were captured using a ZED Mini camera mounted on the robot. These images correspond to both the perspective observed by the remote operator through the HMD and the viewpoint used by the robot during inference.
The white circles in the images indicate the gaze positions of the remote operator measured at the respective moments.

% Hypara一覧表を加える
The hyperparameters used for GazeBot and the seven ablation models are summarized in Table \ref{tab:hyperparameters-models}.
We found that the same hyperparameters yielded good results across all models.
We also present the hyperparameters used for task decomposition based on gaze transitions \cite{Takizawa2024} in Table \ref{tab:hyperparameters-seg}. This method successfully segmented all demonstrations into sub-tasks consistently and accurately.

\subsection{Experiment Details}
\label{appendix-experiment}
% すべてのtrial(ID・OOD)の初期状態一覧
In the experiment, we standardized the initial conditions for all ID and OOD trials to enable an accurate comparison of each model’s generalization performance. Both GazeBot and the seven ablation models performed the task under conditions as similar as possible. Figure \ref{fig:init-states} illustrates all the initial conditions used in the ID and OOD trials. 
In the OOD trials, we first evaluated the cases where each object was individually placed in unseen positions. We then examined the cases where all objects were simultaneously placed in unseen positions. Finally, we conducted trials involving unseen initial end-effector poses.



\newpage



\begin{figure*}[b]
    \centering
    \includegraphics[width=0.6\linewidth]{figures/teleop.pdf}
    \caption{The robot system used in this work: a dual-arm robot (right side of the image) and a teleoperation system (left side of the image).}
    \label{fig:teleop-system}
\end{figure*}

\begin{table*}[b]
\centering
\begin{tabular}{lll}
\toprule
\textbf{Parameter Name} & \textbf{Value} & \textbf{Description} \\ 
\midrule
image size & 1280 x 720 & Size of the entire image before converting it to point cloud. \\ 
num points (real) & 10000 & Number of points included in the gaze-centered point cloud in real tasks. \\ 
num points (sim) & 4000 & Number of points included in the gaze-centered point cloud in simulator tasks. \\
crop size (point cloud) & 0.2 & Size of the cubic region in meters cropped by the entire point cloud . \\ 
crop size (image) & 250 & Size of the square region in meters cropped by the entire image (used in Ablation1 and DAA). \\ 
progress thresh & 0.9 / 5 & The sub-task index is incremented when the progress advances by 0.9 continuously over five time steps. \\
learning rate & 1e-5 & Learning rate of training (the same for all models). \\ 
weight decay & 1e-4 & Weight decay of training (the same for all models). \\ 
batch size & 8 & Batch size of training (the same for all policy models). \\ 
hidden dimension & 512 & Hidden dimensions used for the Transformer. \\ 
num heads & 8 & Number of heads used for transformer. \\ 
feedforward dimension & 3200 & Feedforward dimensions used for the Transformer. \\ 
num enc layers & 4 & Number of encoder layers of the Transformer. \\ 
num dec layers & 4 & Number of decoder layers of the Transformer. \\ 
hidden dimension & 512 & Hidden dimensions used for the Transformer. \\ 
dropout & 0.1 & Dropout used for the Transformer. \\
batch size (gaze) & 16 & Batch size of training in the gaze model. \\
upsample & 4 & Upsampling of the number of tokens in the gaze model. \\
\bottomrule
\end{tabular}
\caption{Hyperparameters.}
\label{tab:hyperparameters-models}
\end{table*}
% gaze分節化のやつも

\begin{table*}[b]
\centering
\begin{tabular}{lll}
\toprule
\textbf{Parameter Name} & \textbf{Value} & \textbf{Description}\\ 
\midrule
init thresh (gaze position) & 50 & Initial threshold of gaze position for change detection. \\ 
init thresh (gaze feature) & 0.05 & Initial threshold of gaze feature for change detection. \\ 
crop size (in 1280x720) & 256 & Size of the cropped image around the gaze position to extract gaze features.  \\
window size & 20 & Window size for median filtering. \\ 
\bottomrule
\end{tabular}
\caption{Hyperparameters for gaze-based task decomposition \cite{Takizawa2024}.}
\label{tab:hyperparameters-seg}
\end{table*}

\begin{figure*}[b]
    \begin{tabular}{c}
        \vspace{0.3cm}
        \begin{minipage}[c]{0.95\linewidth}
            \centering
            \includegraphics[width=\linewidth]{figures/demo-a.png}
            \subcaption{PenInCup}
        \end{minipage}
        \\ \vspace{0.3cm}
        \begin{minipage}[c]{0.95\linewidth}
            \centering
            \includegraphics[width=\linewidth]{figures/demo-b.png}
            \subcaption{OpenCap}
        \end{minipage}
        \\ 
        \begin{minipage}[c]{0.95\linewidth}
            \centering
            \includegraphics[width=\linewidth]{figures/demo-c.png}
            \subcaption{PileBox}
        \end{minipage}
    \end{tabular}
    \caption{Examples of the demonstrations collected by a human teleoperator. White circles indicate the remote operator's gaze position.}
    \label{fig:demos}
\end{figure*}



\begin{figure*}[b]
    \begin{tabular}{c}
        \vspace{0.2cm}
        \begin{minipage}[c]{0.95\linewidth}
            \centering
            \includegraphics[width=0.7\linewidth]{figures/init-states-a.pdf}
            \subcaption{ID trials (PenInCup)}
        \end{minipage}
        \\ \vspace{0.2cm}
        \begin{minipage}[c]{0.95\linewidth}
            \centering
            \includegraphics[width=0.7\linewidth]{figures/init-states-b.pdf}
            \subcaption{OOD trials (PenInCup)}
        \end{minipage}
        \\ \vspace{0.2cm}
        \begin{minipage}[c]{0.95\linewidth}
            \centering
            \includegraphics[width=0.7\linewidth]{figures/init-states-c.pdf}
            \subcaption{ID trials (OpenCap)}
        \end{minipage}
        \\ \vspace{0.2cm}
        \begin{minipage}[c]{0.95\linewidth}
            \centering
            \includegraphics[width=0.7\linewidth]{figures/init-states-d.pdf}
            \subcaption{OOD trials (OpenCap)}
        \end{minipage}
        \\  \vspace{0.2cm}
        \begin{minipage}[c]{0.95\linewidth}
            \centering
            \includegraphics[width=0.7\linewidth]{figures/init-states-e.pdf}
            \subcaption{ID trials (PileBox)}
        \end{minipage}
        \\ \vspace{0.2cm}
        \begin{minipage}[c]{0.95\linewidth}
            \centering
            \includegraphics[width=0.7\linewidth]{figures/init-states-f.pdf}
            \subcaption{OOD trials (PileBox)}
        \end{minipage}
    \end{tabular}
    \caption{Initial states of all trials in ID and OOD.}
    \label{fig:init-states}
\end{figure*}


\end{document}