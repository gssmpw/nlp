%%%%%%%%%%%%%%%%%%%%%%%%%%%%%%%%%%%%%%%%%%%%%%%%%%%%%%%%%%%%%%%%%%%%%%%%

%%% LaTeX Template for AAMAS-2025 (based on sample-sigconf.tex)
%%% Prepared by the AAMAS-2025 Program Chairs based on the version from AAMAS-2025. 

%%%%%%%%%%%%%%%%%%%%%%%%%%%%%%%%%%%%%%%%%%%%%%%%%%%%%%%%%%%%%%%%%%%%%%%%

%%% Start your document with the \documentclass command.


%%% == IMPORTANT ==
%%% Use the first variant below for the final paper (including auithor information).
%%% Use the second variant below to anonymize your submission (no authoir information shown).
%%% For further information on anonymity and double-blind reviewing, 
%%% please consult the call for paper information
%%% https://aamas2025.org/index.php/conference/calls/submission-instructions-main-technical-track/

%%%% For anonymized submission, use this
%\documentclass[sigconf,anonymous]{aamas} 

%%%% For camera-ready, use this
% \documentclass[sigconf]{aamas}  % for camera-ready
\documentclass[sigconf,nonacm]{aamas} % for preprint

%%% Load required packages here (note that many are included already).

\usepackage{balance} % for balancing columns on the final page
% NGUYEN PHAM ADDING
% \usepackage{scalefnt}

% \usepackage{natbib}
\usepackage{multirow}
\usepackage{graphicx}

\usepackage{amsmath}
\usepackage{amsthm}
\theoremstyle{plain}
% \theoremstyle{definition}

\newtheorem{lemma}{Lemma}
% \newtheorem{remark}{Remark}
\usepackage{subcaption}
\usepackage{algorithm}
\usepackage{algorithmic}
\usepackage{threeparttable}
\newcommand{\setlabel}[1]{\edef\@currentlabel{#1}\label}

\theoremstyle{acmplain}
\newtheorem{theo}{Theorem}
\newtheorem{re}{Theorem}
\newtheorem{pos}{Theorem}
\newtheorem{defi}{Theorem}
\newtheorem{remark}[re]{Remark}
\newtheorem{proposition}[pos]{Proposition}
\newtheorem{definition}[defi]{Definition}

% \newcounter{definition}
% \newcounter{proposition}
% \newcounter{remark}

% % Define theorem environments with specific counters
% \newtheorem{definition}[definition]{Definition}
% \newtheorem{proposition}[proposition]{Proposition}
% \newtheorem{remark}[remark]{Remark}


% %%%%%%%%%%%%%%%%%%%%%%%%%%%%%%%%
% % THEOREMS
% %%%%%%%%%%%%%%%%%%%%%%%%%%%%%%%%
% \setlength{\marginparwidth}{2cm}
% \usepackage[textsize=tiny]{todonotes}
\usepackage{import}
% \usepackage{amsfonts}
% \usepackage{mathrsfs}
% \usepackage{bm}
\mathchardef\mhyphen="2D
\def\OT{\textup{OT}}
\def\cN{\mathcal{N}}
\def\cA{\mathcal{A}}
\def\cB{\mathcal{B}}
\newcommand{\E}{\mathbb{E}}
\newcommand{\Var}{\mathbb{V}}
\newcommand{\Cov}{\mathbb{C}}
\DeclareMathOperator{\SW}{\texttt{SW}}
\DeclareMathOperator{\SSW}{\texttt{SSW}}
% NGUYEN PHAM ADDING
\usepackage{pifont}
\newcommand{\cmark}{\ding{51}} % Checkmark
\newcommand{\xmark}{\ding{55}} % Crossmark

\newif\ifshowappendix
\showappendixtrue   % show appendix
% \showappendixfalse  % not show appendix
\ifshowappendix
  % Code that runs if showappendix is TRUE
\else
    \usepackage{xr-hyper}
    \usepackage{hyperref} 
    \externaldocument{exfile} 
\fi

%%% AAMAS-2025 copyright block (do not change!)

\makeatletter
\gdef\@copyrightpermission{
  \begin{minipage}{0.2\columnwidth}
   \href{https://creativecommons.org/licenses/by/4.0/}{\includegraphics[width=0.90\textwidth]{by}}
  \end{minipage}\hfill
  \begin{minipage}{0.8\columnwidth}
   \href{https://creativecommons.org/licenses/by/4.0/}{This work is licensed under a Creative Commons Attribution International 4.0 License.}
  \end{minipage}
  \vspace{5pt}
}
\makeatother

\setcopyright{ifaamas}
\acmConference[AAMAS '25]{Proc.\@ of the 24th International Conference
on Autonomous Agents and Multiagent Systems (AAMAS 2025)}{May 19 -- 23, 2025}
{Detroit, Michigan, USA}{Y.~Vorobeychik, S.~Das, A.~Nowé  (eds.)}
\copyrightyear{2025}
\acmYear{2025}
\acmDOI{}
\acmPrice{}
\acmISBN{}


%%%%%%%%%%%%%%%%%%%%%%%%%%%%%%%%%%%%%%%%%%%%%%%%%%%%%%%%%%%%%%%%%%%%%%%%

%%% == IMPORTANT ==
%%% Use this command to specify your submission number.
%%% In anonymous mode, it will be printed on the first page.

\acmSubmissionID{<<1303>>}

%%% Use this command to specify the title of your paper.

\title{\texttt{DUPRE}: Data Utility Prediction for Efficient Data Valuation}

%%% Provide names, affiliations, and email addresses for all authors.

\author{Kieu Thao Nguyen Pham}
\authornote{Equal contribution.}
\affiliation{
  \institution{National University of Singapore}
  \country{Singapore}}
\email{nguyen.pkt@u.nus.edu}

\author{Rachael Hwee Ling Sim}
\authornotemark[1]
\affiliation{
  \institution{National University of Singapore}
  \country{Singapore}}
\email{rachael.sim@u.nus.edu}


\author{Quoc Phong Nguyen}
\affiliation{
  \institution{A2I2, Deakin University}
  \country{Australia}}
\email{ qphongmp@gmail.com}

\author{See Kiong Ng}
\affiliation{
  \institution{National University of Singapore}
  \country{Singapore}}
\email{seekiong@nus.edu.sg}

\author{Bryan Kian Hsiang Low}
\affiliation{
  \institution{National University of Singapore}
  \country{Singapore}}
\email{lowkh@comp.nus.edu.sg}

%%% Use this environment to specify a short abstract for your paper.


\begin{abstract}
Data valuation is increasingly used in machine learning (ML) to decide the fair compensation for data owners and identify valuable or harmful data for improving ML models. Cooperative game theory-based data valuation, such as Data Shapley, requires evaluating the data utility (e.g., validation accuracy) and retraining the ML model for multiple data subsets. While most existing works on efficient estimation of the Shapley values have focused on reducing the number of subsets to evaluate, our framework, \texttt{DUPRE}, takes an alternative yet complementary approach that reduces the cost per subset evaluation by predicting data utilities instead of evaluating them by model retraining. Specifically, given the evaluated data utilities of some data subsets, \texttt{DUPRE} fits a \emph{Gaussian process} (GP) regression model to predict the utility of every other data subset. Our key contribution lies in the design of our GP kernel based on the sliced Wasserstein distance between empirical data distributions. In particular, we show that the kernel is valid and positive semi-definite, encodes prior knowledge of similarities between different data subsets, and can be efficiently computed. We empirically verify that \texttt{DUPRE} introduces low prediction error and speeds up data valuation for various ML models, datasets, and utility functions.

  
  % \emph{Data valuation} is increasingly used in machine learning (ML) to decide the fair compensation for data owners and identify valuable or harmful data for improving ML models. \emph{Data Shapley} is a commonly used data valuation method based on cooperative game theory that requires evaluating the data utility (e.g., validation accuracy) obtained through ML model retraining on multiple subsets of data owners. Previous works have proposed efficient approximations of Data Shapley and other semivalues by reducing the number of subsets to evaluate from exponential to polynomial scale in the number of data owners. This paper presents an framework called \texttt{DUPRE} that can be used in conjunction with any existing sampling-based approximation method to further boost the efficiency of performing data valuation. To achieve this, we reduce the cost \emph{per} evaluation by predicting data utilities instead of evaluating them through model retraining. Specifically, given the evaluated data utilities of some data subsets, our method, \texttt{DUPRE}, trains a \emph{Gaussian process} (GP) regression model to predict the utility of additional subsets by hypothesizing two similar data subsets have similar data utilities. 
  % However, a key challenge lies in how to define a valid kernel or the positive semi-definite kernel (PSD) that determines the similarities between datasets and offer computational savings compared to directly evaluating the utility from model retraining.  We address this by proposing a kernel, \texttt{SSW}, based on the sliced Wasserstein distance (SW), which has been proven to be PSD and computationally efficient. We also propose an active selection algorithm to select evaluated subsets for training GP that minimize uncertainty in the \emph{Data Shapley} estimation. We empirically verify that \texttt{DUPRE} introduces low prediction error and significantly speeds up data valuation for various ML models, datasets, and utility functions.

  % % We select the GP kernel to encode prior knowledge of similarities between different data subsets. 



\end{abstract}
\keywords{Data Valuation; Gaussian Process Regression; Shapley Value; Semivalue; Data Utility Prediction; Kernel; Collaborative Machine Learning}

%%%%%%%%%%%%%%%%%%%%%%%%%%%%%%%%%%%%%%%%%%%%%%%%%%%%%%%%%%%%%%%%%%%%%%%%

%%% Include any author-defined commands here.
         
\newcommand{\BibTeX}{\rm B\kern-.05em{\sc i\kern-.025em b}\kern-.08em\TeX}

%%%%%%%%%%%%%%%%%%%%%%%%%%%%%%%%%%%%%%%%%%%%%%%%%%%%%%%%%%%%%%%%%%%%%%%%

\begin{document}

%%% The following commands remove the headers in your paper. For final 
%%% papers, these will be inserted during the pagination process.

\pagestyle{fancy}
\fancyhead{}

%%% The next command prints the information defined in the preamble.

\maketitle 

\section{Introduction} \label{sec:intro}
In recent years, there has been growing interest in \emph{data valuation} and understanding how much data is worth in \emph{machine learning}~(ML). Data valuation can be used to determine the fair compensation that data owners deserve for sharing their data \cite{jia2019towards,sim2020cml} and to identify valuable datasets to explain and improve the performance of their models~\cite{ghorbani2019data}.
A common category of data valuation methods that values a data point/set (relative to the data contributed by others) is \emph{cooperative game theory} (CGT) based valuation \cite{sim2022data}. 
Suppose the ML model is trained on data from a set $N$ of $n$ data owners (owners).
Data Shapley \cite{ghorbani2019data}, a popular CGT-based valuation technique, values an owner $i$ by its Shapley value
\begin{equation}\label{eq:shapley}
    % \textstyle
    \phi_u(i) \triangleq \sum_{C \subseteq N \setminus i} \frac{1}{n} {\binom{n-1}{|C|}}^{-1} \left[ u(C \cup \{i\}) - u(C) \right].
\end{equation}

The data utility function $u$ maps any coalition (i.e., set) $C \subseteq N$ of owners to the ML performance achievable by their data. A concrete example of $u$ is the validation accuracy on a trained neural network.
Other CGT-based valuations include the least-core solution \cite{yan2021core} and semivalues \cite{Kwon2021betashapley,wang2023data}, such as the Banzhaf value, which is similar to the Shapley value but uses a different set of weights $(\omega_c)_{c=0}^{n-1}$ such that each $\omega_c \geq 0$ and $\sum_{c=0}^{n-1} \omega_{c}{\binom{n-1}{c}}=1$ in the following expression
\begin{equation}\label{eq:semi}
\varphi_u(i) \triangleq \sum_{C \subseteq N \setminus i}\omega_{|C|} \left[ u(C \cup \{i\}) - u(C) \right].
\end{equation}

While CGT-based valuations satisfy desirable axioms, they all require an exponential number of evaluations of~$u$.
This high complexity motivates existing works to study more efficient methods for estimating the Shapley value and other semivalues by~\textbf{(i)} reducing the number of evaluations or~\textbf{(ii)} reducing the cost per evaluation. Each data utility evaluation may involve expensive model retraining from scratch to evaluate its predictive performance. Our work addresses~\textbf{(ii)} and is complementary to methods that address~\textbf{(i)}.

Most existing works focus on \textbf{(i)} and propose Monte Carlo methods, such as permutation sampling, group testing \cite{jia2019towards}, and reusing samples (data utility evaluations) to compute the semivalues for multiple owners efficiently \cite{LiYu24,kolpaczki2024without}. 
To address \textbf{(ii)}, some works propose heuristics to avoid retraining the ML model from scratch. 
For example, TMC-Shapley~\cite{ghorbani2019data} simply approximates $\Delta_u(i|C) \triangleq u(C \cup i) - u(C)$ with $0$ when $u(C)$ is sufficiently close to $u(N)$.
Gradient Shapley~\cite{ghorbani2019data} considers training the ML model (e.g., deep neural network) over a single training epoch so multiple utilities (e.g., $u(C)$, $u(C \cup i)$ then $u(C \cup \{i,j\})$) can be incrementally computed. ~\citet{jia2019towards} suggest using the influence function heuristic to approximate $\Delta_u(i|C)$ when owner $i$ owns a single data point. It also describes how to compute the Shapley value of all points exactly in log-linear time for their $k$-nearest neighbor utility function.
However, these methods cannot be applied for \emph{all} models (e.g., neural networks trained over \emph{multiple} epochs), utility functions (e.g., when $u(C) \ll u(N)$) and datasets (e.g., the influence approximation may be inaccurate when each owner owns a larger dataset instead \cite{koh2019accuracy}).
These limitations raise an important question: For faster data valuation, is there a \emph{general} method to \emph{reduce the cost per data utility evaluations} that works for \emph{all} models, utility functions, and datasets?

\citet{wang2021predict} have proposed predicting data utility for any input dataset and using a hybrid of actual utility evaluations and predicted evaluations during data valuation. They briefly analyzed how this hybrid slightly worsens the CGT-based valuation approximation error. However, \citet{wang2021predict} did not optimize the predictor and simply trained a neural network that takes in an indicator vector for each coalition $C$ (i.e., an $n$-dimensional vector where each entry is $1$ if the corresponding owner is present in $C$ and $0$ if absent).

Our work seeks to optimize the predictor and addresses the following questions: (1) Can the predictor further exploit the similarity between data of different owners? For example, if owners~$i$ and~$j$ have highly similar data, can the predictor leverage the prior knowledge that $u(C \cup i) \approx u(C \cup j)$ (instead of learning the relationship from more data utility evaluations)? (2) Additionally, can the predictor quantify the uncertainty in its prediction?

\emph{Gaussian process regression} (GPR) seems well-suited to this task. We can (1) incorporate prior knowledge by specifying an appropriate kernel over datasets and (2) quantify the additional uncertainty in the estimated semivalue (due to the predictor) using the GPR posterior covariance.
However, adapting GPR presents challenges: The kernel must be \emph{positive semi-definite} (PSD), to ensure a nondegenerate GP posterior and offer computational savings as compared to directly evaluating the utility from model retraining.
We propose measuring the similarity between empirical data distributions with a \emph{sliced Wasserstein distance} ($\SW$) kernel, as \citet{meunier2022slicedw} have proven that the kernel is PSD. 
Additionally, our method is computationally efficient: after precomputing the sorted projections once, the $\SW$ distance used in each kernel entry can be computed in linear time w.r.t.~the total dataset size. 
Our method, \texttt{DUPRE}, fits a GP model based on some actual data utility evaluations. Subsequently, \texttt{DUPRE} predicts the utility of unevaluated coalitions and can be used to identify those with high uncertainty for further actual evaluations. \texttt{DUPRE} complements any exact CGT-based valuation and approximation techniques proposed to address \textbf{(i)}.

In summary, we make the following key contributions.
In Section~\ref{sec:pf}, we formulate the problem of predicting the utility of coalitions using GPR and propose a suitable valid kernel to measure the similarity between datasets for regression and classification problems. In Section~\ref{sec:semi}, we describe how to estimate the semivalues based on the GP model.\footnote{The supplementary materials can be found at \url{https://kakaeriol.github.io/dupre/}.} In Section~\ref{sec:exp}, we empirically verify that \texttt{DUPRE} introduces a low prediction error and speeds up data valuation for various few models, datasets, and utility functions. 

\section{Problem formulation} 
\label{sec:pf}
We consider $n$ data owners $N = \{1, \dots, n\}$. Data owner $i$ has a dataset $D_i \triangleq (X_i, y_i)$ where $X_i$ is the input matrix and $y_i$ is the target outputs. Each owner $i$ shares their dataset $D_i$ with the mediator, who train ML models on data from multiple owners to assign a fair data value $\phi_i$ to each data owner $i$.
Let $C \subseteq N $ denote a \emph{coalition} of data owners with the aggregated dataset~$D_C \triangleq (X_C, y_C)$, where~$X_C \triangleq \cup_{i \in C} X_i$ and~$y_C \triangleq \cup_{i \in C} y_i$.\footnote{For notation convenience, we say $(x,y) \in D_C$ if there exists an index $j$ such that the $j$ element of $X_C$ and $y_C$ are $x$ and $y$, respectively.}

\begin{figure}[!ht]
    \centering
    \includegraphics[width=\columnwidth]{figures/clear_DUPRE.pdf}
    \caption{We partition all (sampled) coalitions into two families of sets. The utilities of coalitions in  $\cA$ are actually evaluated by training the ML model and measuring the utility, e.g., validation accuracy. In contrast, the utilities of coalitions in $\cB$ are predicted by the Gaussian Process (GP) model.
    }
    \label{fig:framework}
    \Description{DUPRE framework}
\end{figure}
The data utility function $u$ maps any coalition $C$ to the performance (e.g., negated mean squared error) of the ML model trained on their data, $u(C)$. The function $u$ may be expensive to evaluate for complex models such as deep neural networks.

For any $j$, let coalitions $A_j$ and $B_j$ be subsets of $N$.
Given $a$ actual utility evaluations of coalitions in $\cA \triangleq (A_j)_{j=1}^a$ (i.e., $(u(A))_{A \in \cA}$), our goal is to learn a predictor $\hat{u}$ that predicts $b$ data utilities of coalitions in $\cB \triangleq (B_j)_{j=1}^b$.
Subsequently, we use both the actual utility evaluations at $\cA$ and predicted utility evaluations at $\cB$ to exactly compute or approximate the CGT-based data valuation, as shown in Figure~\ref{fig:framework}.


For each coalition $C$, we model the data utility $u(C) = \upsilon(C) + \epsilon_C$ where $\epsilon_C$ is sampled from a Gaussian distribution with zero mean and variance $\sigma^2$. We specify the underlying generating function $\upsilon$ as a \emph{Gaussian process} \cite{williams_gaussian_1996} with a covariance kernel $k$ (see Appendix~\ref{appendix:gp_background}). Given the observed utilities $\mathbf{u}_\cA$ for coalitions $\cA$, the posterior belief of the utilities $\hat{\mathbf{u}}_{\cB|\cA}$ for coalitions $\cB$ follows a Gaussian distribution
\begin{align*}
  \mathbf{\hat{u}}_{\cB|\cA} & \sim \mathcal{N}
  \left(\mathbb{E}[\mathbf{\hat{u}}_{\cB|\cA}], \mathbb{V}[\mathbf{\hat{u}}_{\cB|\cA}]   \right) \\
  \mathbb{E}[\mathbf{\hat{u}}_{\cB|\cA}] &=
  \mathbf{K}_{\cB,\cA} [\mathbf{K}_{\cA,\cA} + \sigma^2 \mathbf{I}]^{-1} \mathbf{u}_\cA \\ 
  \mathbb{V}[\mathbf{\hat{u}}_{\cB|\cA}] &= 
  \mathbf{K}_{\cB,\cB} - \mathbf{K}_{\cB,\cA}  [\mathbf{K}_{\cA,\cA} + \sigma^2 \mathbf{I}]^{-1} \mathbf{K}_{\cA, \cB}\ .
\end{align*}

Here, $\mathbf{K}_{\cB,\cA}$ is a matrix whose $j,i$ entry is $k(B_j, A_i)$, the similarity between the aggregated datasets of $D_{B_j}$ and that of $D_{A_i}$. Notice that there are $O(a^2 + ab + b^2)$ unique kernel entries in  $\mathbf{K}_{\cB,\cA}$ and  $\mathbf{K}_{\cA,\cA}$.
The inverse of the kernel matrix only needs to be computed once in $O(a^3)$ time. Subsequently, each utility prediction only involves matrix multiplication.

Our next challenge is to decide the kernel function $k$ over datasets or data distributions such that the kernel is \textbf{(I)} valid (see Appendix~\ref{appendix:pos} for properties a valid kernel must satisfy such as positive semi-definite, PSD) and \textbf{(II)} computationally efficient. The former results in a valid GP while the latter ensures that the method is useful in practice. 

\subsection{Choice of Kernel}
For simplicity, we start by ignoring the target outputs and considering only the input matrix, i.e., $D_i = X_i$ for every $i$.
How do we measure the similarity between the aggregated dataset $D_A$ (from owners in $A$) and $D_B$? Equivalently, let $\delta$ be the Dirac delta distribution, how do we measure the distance between the empirical data distributions $\mathbb{P}(D_A) =\frac{1}{|D_A|}\sum_{x \in D_A}{\delta (x)}$ and $\mathbb{P}(D_B)$?
We quantify the distance between the empirical data distributions using optimal transport distances (see Appendix~\ref{appendix:otdd_background}) rather than f-divergences. Optimal transport distances, such as the Wasserstein distance, exhibit desirable mathematical properties including symmetry and the triangle inequality, which are essential for defining valid kernels and comparing distributions even when their supports are disjoint. Optimal transport distances measure the minimal total cost required to transform one distribution into another. In particular, the Wasserstein distance captures the intrinsic geometry of the space of distributions \cite{meunier2022slicedw} and admits an intuitive interpretation: it is the minimum total cost of transporting mass from the distribution $\mathbb{P}(D_A)$ to $\mathbb{P}(D_B)$.

However, the kernel based on the squared Wasserstein distance, i.e., $k(A, B) \propto e^{-W_2^2(\mathbb{P}(D_A), \mathbb{P}(D_B))}$ may not be PSD when the data dimension exceeds $1$ \cite{Ginsbourger2018GaussianPW,meunier2022slicedw}, thus violating \textbf{(I)}.
Moreover, as computing the Wasserstein distance involves an optimization problem, the most efficient method \cite{dvurechensky2018compot} still takes $\Tilde{O}(\max(|D_A|,|D_B|)^2)$ time.\footnote{$\Tilde{O}$ hides polylogarithmic factors.} This computation burden becomes expensive when repeated $O(a^2 + ab + b^2)$ times for each kernel entry, violating \textbf{(II)}.
Thus, we must use alternatives to the Wasserstein distance, such as the \emph{sliced Wasserstein distance} (SW) \cite{meunier2022slicedw} that provably satisfies \textbf{(I)}.

\begin{proposition}[\cite{meunier2022slicedw}]\label{prop:sw}
    The exponential kernels $k(A, B)$ based on the $\SW$ distance, including 
    $\exp\left(-\gamma \cdot \SW_2^{2\rho}\left(\mathbb{P}(D_A), \mathbb{P}(D_B)\right)\right)$ and \linebreak
    $\exp\left(-\gamma \cdot \SW_1^{\rho}\left(\mathbb{P}(D_A), \mathbb{P}(D_B)\right)\right)$ 
    are positive semi-definite (PSD) and valid for $\gamma > 0$ and $\rho \in [0,1]$.
\end{proposition}

To address \textbf{(II)}, the SW distance can be efficiently approximated using Monte Carlo sampling with $L$ projections. After each projection, the Wasserstein distance between one-dimensional distributions can be computed analytically.
We additionally observe that only line~10 in Algorithm~\ref{alg:SW}, which merges two sorted projection lists, is unique to the coalition pair $(A, B)$. Thus, only this step, which takes $O(L \cdot (|D_A|+|D_B|))$ time, is repeated for the  $O(a^2 + ab + b^2)$ unique kernel entries. The $L$ factor can be further reduced by parallelizing the computation for multiple projected directions. 
Given $D_A$ has dimension $m$, steps~1-9 in Algorithm~\ref{alg:SW} only need to be precomputed once in $O(L|D_A| m  + L|D_A| \log |D_A|)$ time.

\begin{algorithm}
\caption{Sliced Wasserstein Distance Computation}\label{alg:SW}
\begin{flushleft}    
\textbf{Input}: Two dataset matrices $X_A, X_B$ with $m$ features/columns\\
\textbf{Parameter}: Number of projected directions $L$\\
\textbf{Output}: Sliced Wasserstein Distance $\SW(X_A$, $X_B)$ between $X_A$ and $X_B$, 
\end{flushleft}

\begin{algorithmic}[1]
\STATE $s \gets 0$
\FOR{$l = 1$ {\bfseries to} $L$}
\STATE Uniformly sample $\theta^{(l)}$ distributed on unit sphere 
    \STATE $\Pi_{\theta^{(l)}}(X_A) \gets X_A \theta^{(l)}$
    \STATE $\Pi_{\theta^{(l)}}(X_A) \gets \texttt{sort}(\Pi_{\theta^{(l)}}(X_A))$
    \STATE $\Pi_{\theta^{(l)}}(X_B) \gets X_B \theta^{(l)}$
    \STATE $\Pi_{\theta^{(l)}}(X_B) \gets \texttt{sort}(\Pi_{\theta^{(l)}}(X_B))$
    \STATE  $F_{X_A} \gets$ the empirical c.d.f.  with ${\Pi}_{\theta^{(l)}}(X_A)$
    \STATE  $F_{X_B} \gets$ the empirical c.d.f.  with ${\Pi}_{\theta^{(l)}}(X_B)$ 
    \STATE $s \gets s + \frac{1}{L}\left( \int_0^ 1 | F_{X_A}^{-1}(z)- F_{X_B}^{-1}(z)|^p dz \right) ^{1/p} $ \label{step:ot}
    
\ENDFOR
\STATE  \textbf{return} $s$
\end{algorithmic}
\end{algorithm}

\subsection{Supervised Learning}
In this section, we will additionally consider the target outputs $y_i$ for each owner $i$. Specifically, we will define a transformation $\mathcal{G}_{\eta}$ that will map the dataset (consists of both the input matrix and target outputs) to a common feature space. 

\begin{definition}
    Given the transformation $\mathcal{G}_{\eta}$, the \emph{supervised sliced Wasserstein} (SSW) distance between the empirical data distribution $D_A$ and $D_B$ is 
    \[
    \SSW_2^{2 \rho}(\mathbb{P}(D_A), \mathbb{P}(D_B); \mathcal{G}_{\eta}) = {\SW}_2^{2\rho}(\mathbb{P}(\mathcal{G}_{\eta}(D_A)), \mathbb{P}(\mathcal{G}_{\eta}(D_B)))\ .
    \]
\end{definition}

\textbf{Supervised regression.} Data valuation can be applied on regression problems, where the output $y_i$ is a vector of real values. We concatenate $y_i$ with $X_i$ and vary the weight on $y_i$ using a parameter $\eta$. Formally, let $\mathcal{G}_{\eta}(D_C) = \eta X_C \oplus (1-\eta) y_C$, where $\eta \in (0,1]$ is the scaling weight for the feature space.

\textbf{Supervised classification.} 
Data valuation is also often applied in supervised classification problems, where the output $y_i$ is a vector of discrete class labels.
At first glance, we can concatenate $y_i$ with $X_i$, however, how do we measure the distance between different labels such as `airplane', `bird', and `truck' in CIFAR-10? Inspired by \citet{alvarez2020otdd}, we quantify the distance between labels $\mathtt{y}^k$ and $\mathtt{y}^j$ as the sliced Wasserstein distance between the aggregated datasets with corresponding labels $\mathtt{y}^k$ and $\mathtt{y}^j$, i.e., $\SW(\mathbb{P}({x_i \mid (x_i, \mathtt{y}^k) \in D_N}), \mathbb{P}({x_i \mid (x_i, \mathtt{y}^j) \in D_N}))$. Thus, the 'bird' label is closer to `airplane' than `truck'.
We then use multi-dimensional scaling (MDS) \cite{pmlr-v139-demaine21a} to embed each class $\mathtt{y}^j$ as a vector $e(\mathtt{y}^j)$, preserving the distances between class labels. The function $e$ is also applied element-wise to embed the target outputs $y_C$. Formally, let $\mathcal{G}_{\eta}(D_C) = \eta X_C \oplus (1-\eta) e(y_C)$, where $\eta \in (0,1]$ are scaling weights for the feature and label spaces, respectively.

\begin{proposition}\label{prop:ssw}
    The kernel $k(A,B) = e^{-\gamma \SSW^{2\rho}_2(\mathbb{P}(D_A), \mathbb{P}(D_B); \mathcal{G}_{\eta})}$ is a valid kernel when $\gamma > 0$ and $\rho \in [0, 1]$ and $\eta > 0$.
\end{proposition}
The proof based on  Proposition~\ref{prop:sw} can be found in Appendix~\ref{app:proof_ssw}. The following toy example demonstrates why our chosen kernel makes intuitive sense. In Figure~\ref{fig:intuition}, as owners $l$ and $l'$ own similar data, the similarity $k(C \cup l, C \cup l')$ is large for any coalition $C$. In contrast, as owners $l$ and $j$ have very different data, the similarity $k(C \cup l, C \cup j)$ is always small. Thus, in a GP model, an observation of the utility of the coalition with $l$ would greatly reduce the uncertainty of that with $l'$ but not $j$.

\begin{figure}[ht]
    \centering
    \begin{tabular}{@{}c@{\hspace{3mm}}c@{}}
        \includegraphics[width=0.4\columnwidth]{figures/circle.png} & \includegraphics[width=0.6\columnwidth]{figures/MDS_2_heatmap.png} \\ 
        (a) Synthetic Dataset & (b) Kernel Value
    \end{tabular}
    \caption{Datasets \textcolor{blue}{$l$} and \textcolor{orange}{$l'$} are similar while datasets \textcolor{blue}{$l$} and \textcolor{green}{$j$} are very different. The $\SSW$ kernel similarity $k(C \cup l, C \cup l') = .99$ is large as compared to the similarity $k(C \cup l, C \cup j) = 0.19$. 
    }
    \label{fig:intuition}
    \Description{Intuition of our Kernel in Gaussian Process}
\end{figure}

\section{Semivalue Estimation}
\label{sec:semi}
We have defined the GP model and kernel to predict utilities for coalitions in $\cB$. Given the $a$ actual utility evaluations of $\cA$ (i.e., $\mathbf{u}_\cA$), we first learn the kernel hyperparameters (e.g., $\gamma$) and choose the norm of the $\SW$ distance and scaling weights $\eta$ for the feature and output spaces to maximize the log-likelihood. Then, we predict the posterior belief of the utilities $\hat{\mathbf{u}}_{\cB|\cA}$ of coalitions in $\cB$. How do we compute the semivalue, such as the Shapley value, based on the observed utilities $\mathbf{u}_\cA$ of coalitions in $\cA$ and the posterior belief $\hat{\mathbf{u}}_{\cB|\cA}$ of coalitions in $\cB$?

\begin{proposition}[Semivalue Prediction]\label{pos:shapley-dist} 
Let $\mathbf{w}^i_\cA$ and $\mathbf{w}^i_{\cB|\cA}$ denote the vectors containing the weights of all coalitions in $\cA$ and $\cB$ respectively.
The $j$-th entry of $\mathbf{w}^i_\cA$ corresponds to the weight of coalition $A_j$ in the computation of (i) the Shapley value $\phi_u(i)$ and (ii) the semivalue $\varphi_u(i)$, as defined in Equations~\ref{eq:shapley} and \ref{eq:semi}). 
When $i \notin A_j$, the $j$-th entry is given by (i) $-1/(n {\binom{n-1}{|A_j|}})$ or (ii) $-\omega_{|A_j|}$. When $i \in A_j$, the $j$-th entry is (i) $1/(n {\binom{n-1}{|A_j|-1}})$ or (ii) $\omega_{|A_j|-1}$.

The estimated Shapley value and semivalue for owner $i$, denoted as $\hat{\phi}_i$ and $\hat{\varphi}_i$, are given by the weighted sum $\mathbf{w}_{\cA}^i{}^\top \mathbf{u}_{\cA} + \mathbf{w}_{\cB}^i{}^\top \hat{\mathbf{u}}_{\cB|\cA}$ which follows the distribution
\[
\mathcal{N}( \mathbf{w}_{\cA}^i{}^\top \mathbf{u}_{\cA} + \mathbf{w}_{\cB}^i{}^\top \mathbb{E}[\hat{\mathbf{u}}_{\cB|\cA}], \mathbf{w}_{\cB}^i{}^\top \mathbb{V}[\hat{\mathbf{u}}_{\cB|\cA}] \mathbf{w}_{\cB}^i ).
\]
\end{proposition}

\begin{remark}[Monte Carlo Approximation]\label{rem:estimate}
    The weights $\mathbf{w}^i_\cA$ and $\mathbf{w}^i_{\cB|\cA}$ used to estimate $\hat{\phi}_i'$ can also correspond to the weights used in Monte Carlo estimates of Shapley value\cite{jia2019towards, kolpaczki2024without}, $\phi_i'$.
    In practice, when using Monte Carlo estimates, the sampled coalitions are collected and subsequently partitioned for actual evaluation and prediction, as illustrated in Figure~\ref{fig:framework}. 
\end{remark}

Proposition~\ref{pos:shapley-dist} considers only the uncertainty arising from the use of GP model predictions instead of actual utility evaluations (refer to \textbf{(ii)} in Section~\ref{sec:intro}). 
When Monte Carlo approximation is used, the total uncertainty in the estimate $\hat{\phi}_i'$ should also consider the uncertainty introduced by Monte Carlo sampling.
Formally, let $\sigma_{MC}$ denote the standard deviation introduced by Monte Carlo methods, such as those described by \citet{kolpaczki2024without}, which evaluates and utilizes only a subset of all coalitions (see \textbf{(i)} in Section~\ref{sec:intro}). Let $\sigma_{GP}$ represent the standard deviation associated with our GP model, defined as the square root of the variance in Proposition~\ref{pos:shapley-dist}. The total uncertainty $\mathbb{V}[\hat{\phi}_i']$ is upper bounded by $\sigma_{GP}^2 + \sigma_{MC}^2 + 2 \sigma_{GP} \sigma_{MC}$. 

\subsection{Active Querying to Accelerate Uncertainty Reduction}
We can reduce the variance $\sigma^2_{GP}$ associated with our GP model further by additionally evaluating $\bar{b}$ coalitions $\widebar{\cB} \subset \cB$. In particular, for each $B \in \widebar{\cB}$, the mediator trains a model on the aggregated data $D_B$ to evaluate $u(B)$ and only predicts the utility of the remaining coalitions $\cB \setminus \widebar{\cB}$.
Instead of randomly selecting  $\bar{b}$ coalitions, the mediator can actively select the $\bar{b}$ coalitions that lead to the largest reduction in the semivalue variance:
\begin{align}\label{eq:active}
    \widebar{\cB} &= \mathrm{argmax}_{\mathcal{C} \subseteq \cB, |\mathcal{C}| = \bar{b}}\ 
    \mathbf{w}^i_{\cB}{}^\top \cdot (\mathbb{V}[\mathbf{\hat{u}}_{\cB|\cA}] - \mathbb{V}[\mathbf{\hat{u}}_{\cB|\cA \oplus \mathcal{C}}]) \cdot \mathbf{w}^i_{\cB}. 
\end{align}
Here, $\oplus$ denote the concatenation operator and $\mathbb{V}[\mathbf{\hat{u}}_{\cB|\cA}]$ is the predictive variance of the utilities in $\cB$ given observed utilities from $\cA$.
As the weighted variance reduction function is often monotone submodular \cite{das2008algorithms}, Equation~\eqref{eq:active} can be maximized by the greedy algorithm in Algorithm~\ref{alg:active}. 
% The method to incrementally update $\mathbf{K}^{-1}$ in line 6 of Algorithm~\ref{alg:active} is described in Appendix~\ref{appdix:algo2}.

\begin{algorithm}
\caption{Greedy Active Selection with Efficient Inverse Update}\label{alg:active}
\begin{flushleft}
\textbf{Input}: Evaluated coalitions and utilities, $(\mathcal{A}, \mathbf{u}_{\mathcal{A}})$; unevaluated coalitions, $\mathcal{B}$; kernel function $k$; semivalue weight vector $\mathbf{w}_{\mathcal{B}}^i$ \\
\textbf{Parameter}: Number of additional evaluations $\bar{b}$ \\
\textbf{Output}: Selected coalitions $\bar{\mathcal{B}}$ for additional evaluations
\end{flushleft}
\begin{algorithmic}[1]
    \STATE Initialize: $\widebar{\cB} \gets ()$, $\mathbf{K}^{-1} \gets (K_{\mathcal{A}, \mathcal{A}} + \sigma^2 \mathbf{I})^{-1}$
    \FOR{$j = 1$ {\bfseries to} $\bar{b}$}
        \STATE Set $\mathtt{max\_VR} \gets 0$, $G^* \gets \textit{null}$
        \FOR{$G \in \mathcal{B} \setminus \widebar{\cB}$}
            \STATE Let $\mathcal{C} \gets \widebar{\cB} \oplus (G,)$
            \STATE Incrementally update $\mathbf{K}^{-1}$ using the previous inverse and $k(G, \mathcal{A} \oplus \widebar{\cB})$ (as described in Appendix~\ref{appdix:algo2})
            \STATE Compute the variance reduction:
            \[
            VR \gets (\mathbf{w}_{\mathcal{B}}^i)^\top \cdot
            \mathbf{K}_{\mathcal{B}, \mathcal{A} \oplus \mathcal{C}}
            \mathbf{K}^{-1}
            \mathbf{K}_{\mathcal{A} \oplus \mathcal{C}, \mathcal{B}}
            \cdot
            \mathbf{w}_{\mathcal{B}}^i
            \]
            \IF{$VR \geq \mathtt{max\_VR}$}
                \STATE Update $\mathtt{max\_VR} \gets VR$, $G^* \gets G$
            \ENDIF
        \ENDFOR
        \STATE Add the selected coalition: $\widebar{\cB} \gets \widebar{\cB} \oplus (G^*,)$
    \ENDFOR
    \STATE \textbf{return} $\widebar{\cB}$
\end{algorithmic}
\end{algorithm}

\section{Experiments}\label{sec:exp}
We conduct experiments across several datasets to evaluate the effectiveness of our methods. For classification tasks, we utilize (a) the Moon dataset \cite{scikit_learn}, (b) the MNIST dataset \cite{deng2012mnist}, (c) the CIFAR-10 dataset \cite{Krizhevsky09}, and (d) the IMDb dataset \cite{maas2011learning}. For regression tasks, we employ (e) the California Housing dataset (CaliH) \cite{SPL97_pace1997sparse}, which provides real-world housing data. We use accuracy as the utility function for classification tasks and the $R^2$ score for regression tasks, defined as $R^2 = 1 - \frac{SS_{\text{res}}}{SS_{\text{tot}}} = 1 - \frac{\sum_{i=1}^{n} (y_i - \hat{y}_i)^2}{\sum_{i=1}^{n} (y_i - \overline{y})^2}$. Finally, in our experiments, we employ the Shapley value, the most widely used semivalue.

Our experiments aim to achieve three primary objectives: (i) to investigate how factors such as the kernel and the number of randomly and actively selected coalitions affect the performance and computation time of our framework in computing Shapley values; (ii) to demonstrate the application of our framework in data valuation, particularly for classification tasks; and (iii) to highlight the advantages of uncertainty quantification.

\textbf{Baseline:} We evaluate three baselines to compare with our proposed approach. The first baseline, denoted as \texttt{OTDD}, is based on the label-feature distance concept from \citet{alvarez2020otdd}. Although this baseline employs the optimal transport distance between feature-label pairs, it is important to note that the exponential kernel derived from \texttt{OTDD} is not valid, unlike the kernel we propose. We summarize the computational complexity of each distance metric in Appendix \ref{sec:A:compare}, Table~\ref{tab:dis_compare}. For this baseline, we implement the Sliced Wasserstein distance as the label distance for efficiency.\footnote{\citet{alvarez2020otdd} originally defined the distance as $\left(\|x - x'\|^p + W_p^p\right)^{1/p}$, where $W_p^p$ is the $p$-Wasserstein distance between label distributions. In our implementation, we use the Sliced Wasserstein (SW) distance for improved efficiency.} The second and third baselines, inspired by \citet{wang2021predict}, represent each dataset $D_i$ with a binary indicator vector (01 encoding) $b_i \in \{0,1\}^n$, indicating the indices present for each data owner $i$. These methods are referred to as \texttt{GP-binary} and \texttt{NN-binary}, employing Gaussian Process Regression and Neural Networks, respectively. Finally, for our SSW kernel, we set $\eta=0.5$.

\subsection{Coalition Utility Prediction}\label{sec:cup}
This section empirically evaluates the utility prediction performance of our proposed kernel $\SSW$ and the baseline methods. We compare the computation time between our $\texttt{DUPRE}$ framework and an exhaustive evaluation of all possible coalitions. Our experiments include two settings: one where $\alpha$ randomly selected coalitions' utilities are evaluated by model training and another (with -a suffix) where half of the coalitions are randomly selected and the other $\alpha/2$ coalitions are actively selected as described in Algorithm~\ref{alg:active} and Figure~\ref{fig:framework}. After evaluating these $\alpha$ coalitions, we train our $\texttt{DUPRE}$ framework on their utilities and then estimate the utilities of all remaining coalitions. 

To assess the effectiveness of our methods, we compute the mean and standard deviation of the mean squared error (MSE) between our predictions and the actual utilities over ten runs, each using a different set of randomly evaluated coalitions (with random seeds from 0 to 9). We also calculate the Pearson correlation coefficient \cite{kirchPearson2008} to evaluate the correlation between predicted and actual utilities and use Kendall's tau metric \cite{kendall1938new} to assess the ranking order of predicted and actual Shapley values, which is critical for understanding a data owner’s contribution.

We first consider predicting the validation accuracy on the MNIST dataset, using a neural network as the ML model for evaluation. Each of the $6$ data owners holds a distinct subset of digit labels: $\{1, 4, 5, 7, 8\}, \{2\}, \{9\}, \{6\}, \{0\}, \{3\}$. As shown in Figure~\ref{fig:up_mnist_6}, our kernel ($\SSW$) outperforms the other baselines, evidenced by a lower mean squared error (MSE) in (a) and higher Pearson correlation coefficients in (c). While the active selection process improves performance, it also increases computation time. Nonetheless, our \texttt{DUPRE} framework remains faster than an exhaustive actual evaluation of all possible coalitions. Moreover, our kernel provides a superior ranking of Shapley values than other kernels, as indicated by the higher Kendall's tau coefficients in (d).
\begin{figure}[ht]
\includegraphics[width=1\columnwidth]{figures/mnist_non_iid_6_all.png} 
    \caption{A comparison of the quality of the utility predictions and time taken for various methods on the MNIST dataset with $6$ owners.
    The \texttt{-a} suffix indicates that we have actively selected 50\% of the coalitions to accelerate uncertainty reduction.
    }
    \label{fig:up_mnist_6}
    \Description{Comparison of methods on MNIST with six data owners, showing MSE, correlation, and computation time for coalition utility prediction.}
\end{figure}

\begin{figure}[ht]
\includegraphics[width=1\columnwidth]{figures/calihouse_iid_6_all.png}
    \caption{
    A comparison of the quality of the utility predictions and time taken for various methods on the CaliH dataset with $6$ owners.
    The \texttt{-a} suffix indicates that we have actively selected 50\% of the coalitions to accelerate uncertainty reduction.}
    \label{fig:up_calih_6}
    \Description{Comparison of methods on California Housing with six data owners, showing MSE, correlation, and computation time for coalition utility prediction.}
\end{figure}
We repeat this experiment on the regression dataset, CaliH, as illustrated in Figure~\ref{fig:up_calih_6}, and use Multi-Layer Perceptron (MLP) as the ML model. The kernel based on \texttt{OTDD} is not applicable here, as it requires classification labels. Once again, our proposed kernel results in lower MSE and higher correlations than the baselines. 

\subsection{Evaluating the Quality of Shapley Value Predictions}
\label{sec:quality_shapley}
In this section, we explore two approaches for computing the Shapley value: predicting the utilities of all coalitions to estimate exact Shapley values and predicting only the utilities of a subset to estimate approximate Shapley values.
\subsubsection{Exact Shapley Value Estimation}
\label{sec:exp_exact_shap}
We evaluated our framework on the CIFAR-10 and CaliH datasets with $8$ data owners. For CIFAR-10, the utility function is the accuracy of the trained ResNet model. In Table~\ref{tab:compare_corr}, the first segment computes the Shapley value based on the predicted utilities from a GP model that is trained on the utility of $100$ actually evaluated coalitions.
In contrast, the GP model in the second segment is additionally trained on the utility of $10$ more coalitions, randomly selected or selected by Algorithm~\ref{alg:active}.


\begin{table*}[ht]
\caption{A comparison of the quality of exact Shapley value approximation and the time taken on the CIFAR-10 and CaliH datasets with $8$ data owners. A higher correlation is preferred. }
    \centering
    \label{tab:compare_corr}
    \begin{tabular}{l|ccc|ccc}
        \toprule
        \multirow{2}{*}{\textbf{Method}} & \multicolumn{3}{c}{\textbf{CIFAR-10}} & \multicolumn{3}{c}{\textbf{CaliH}} \\ 
 %\cline{2-7}
        & \textbf{Pearson} & \textbf{Kendall Tau} & \textbf{Time (s)}  & \textbf{Pearson} & \textbf{Kendall Tau} & \textbf{Time (s)} \\ 
        \midrule
        $\texttt{SSW}$ & \textbf{0.901} $\pm$ 0.07 & \textbf{0.664} $\pm$ 0.152 & 5893 $\pm$ 1375 &  \textbf{0.775} $\pm$ 0.181 & \textbf{0.6714} $\pm$ 0.18 & 543 $\pm$ 10 \\ 
        $\texttt{OTDD}$ & 0.640 $\pm$ 0.09 & 0.523 $\pm$ 0.21 & 8056 $\pm$ 2141 & - & - & - \\ 
        $\texttt{GP-binary}$ & 0.785 $\pm$ 0.006 & 0.565 $\pm$ 0.1 & 4120 $\pm$ 593 & 0.528 $\pm$ 0.134 & 0.593 $\pm$ 0.129 & 264 $\pm$ 35 \\ 
        $\texttt{NN-binary}$ & 0.612 $\pm$ 0.01 & 0.544 $\pm$ 0.12 & 2541 $\pm$ 256 & 0.579 $\pm$ 0.154 & 0.602 $\pm$ 0.163 & 351 $\pm$ 45\\ 
        $\texttt{LAVA}$ & -0.0785 $\pm$ 0.0405 & -0.3045 $\pm$ 0.03 & 5580 $\pm$ 394 & 0.1644 $\pm$ 0.226 & 0.107 $\pm$ 0.15 & 50 $\pm$ 10 \\ \hline
        % Additional active learning entries can be added here
        \textbf{Evaluate $10$ additional coalitions} \\
         $\texttt{SSW}$ - random & 0.911 $\pm$ 0.04 & 0.674  $\pm$ 0.132  & 6137 $\pm$ 1098  &    0.805 $\pm$ 0.155 & 0.7124 $\pm$ 0.21 & 585 $\pm$ 20\\  
	$\texttt{SSW}$ - active & \textbf{0.934} $\pm$ 0.07 & \textbf{0.677}  $\pm$ 0.126  & 6317 $\pm$ 1567  &    \textbf{0.831} $\pm$ 0.165 & \textbf{0.7624} $\pm$ 0.24 &     627 $\pm$ 23\\ 
        $\texttt{GP-binary}$ - active &   0.855 $\pm$ 0.006 & 0.615 $\pm$ 0.15 & 4320 $\pm$ 635 & 0.655 $\pm$ 0.1 & 0.653 $\pm$ 0.120 & 388 $\pm$ 40 \\\hline 
        \textbf{Evaluate all coalitions} &   - & - & 12458   & - & -  & 950   \\
        \bottomrule

    \end{tabular}
\end{table*}

We use Pearson and Kendall's tau correlation coefficients as evaluation metrics to assess the agreement between our predicted and the exact Shapley values, i.e., $(\hat{\phi}_i)_{i \in N}$ and $(\phi_i)_{i \in N}$. We also compare against \texttt{LAVA} \cite{just2023lava}, a model-agnostic method that estimates Shapley values based on the distance between datasets and the task dataset, without requiring ML model training. Our method, \texttt{SSW}, produces estimates that are more correlated with the exact Shapley values, as evidenced by higher Pearson and Kendall's tau coefficients as compared to other kernels. In contrast, the \texttt{LAVA} method results in the lowest correlations. 
Additionally, we observe that active selection, which incurs a slightly higher computation cost, improves the correlation more than random selection. 

\subsubsection{Approximate Shapley Value Estimation}\label{sec:est_sampling_shapley}
Next, we evaluate our approach using the MNIST dataset distributed among $10$ data owners. We compute the approximate Shapley values using permutation sampling, as outlined by \citeauthor{castro2009polynomial} \cite{castro2009polynomial}. We limit the number of permutation samples considered by the total number of evaluated and predicted coalitions. Refer to Appendix~\ref{appdix:samp_algo} for more details.

In our experiments, we actually evaluate the utilities of $512$ coalitions by ML model training and use these utilities to train our GP model. Then, we either consider evaluating the utilities of another additional $250$ coalitions (purple line) or predicting the utilities of $\beta$ additional coalitions ($\beta \in [0, 250]$) using the GP model with different kernels. The key objective of the experiment is to determine if these predicted utilities can effectively substitute for actual utility evaluations.

\begin{figure}[ht]
\includegraphics[width=1.\columnwidth]{figures/permutation_predicted.png} 
\caption{
A comparison of the quality of the approximate Shapley value predictions for various methods on the MNIST dataset. The approximate Shapley value is computed using $512$ actual utility evaluations and an additional $\beta$ actual/predicted coalition utilities. 
}
    \label{fig:per}
    \Description{Comparison of approximate Shapley value predictions for various methods on MNIST, computed using 512 actual utility evaluations plus additional actual/predicted coalition utilities.}
\end{figure}

In Figure~\ref{fig:per}, the x-axis represents the number of additional coalitions used to compute the approximate Shapley value $\phi_i'$. We measure the mean squared error (MSE) between the estimated approximate Shapley value $\hat{\phi}_i'$ and the exact Shapley value $\phi_i$ computed using Equation~\ref{eq:shapley} as well as the Pearson correlation between the estimated approximate Shapley values and exact Shapley values across data owners. The purple lines exhibit a strictly decreasing MSE and a strictly increasing Pearson correlation as more coalitions are evaluated. Similarly, for our SSW kernel, both the MSE and Pearson correlation improve as more coalitions are predicted. Given that the performance of the SSW kernel mirrors that of the actual evaluations, it is a suitable substitute. In contrast, the other baselines do not exhibit the same trend; for instance, with NN-binary, increasing the number of predicted utilities can worsen both the MSE and Pearson correlation.

\subsection{Benefits of Uncertainty Quantification}
\label{sec:benfun}
In this experiment, we examine the uncertainty of the predicted Shapley values of the MNIST classification task where the dataset is split among 5 data owners. We specifically focus on the data owner with (a) the highest contribution and (b) the lowest contribution. Since our kernel $\SSW$ outperforms \texttt{OTDD}, we consider only \texttt{GP-binary} as the baseline. We compute the Shapley values and their variances using the formula in Section~\ref{sec:pf}. The result is illustrated in Figure~\ref{fig:bef}.

\begin{figure}[ht]
\includegraphics[width=1.\columnwidth]{figures/mnist_non_iid_5_uncertainty.png} 
\caption{Plot of the GP model's predictive mean and standard deviation (shaded region around the line) of Shapley values for five data owners.
}
    \label{fig:bef}
    \Description{Uncertainty quantification among five MNIST data owners. 
    }
\end{figure}

As the number of evaluated coalitions increases, the predicted Shapley value gets closer to the actual Shapley value and the variance decreases. For our $\SSW$ kernel, the actual Shapley value always lie within the shaded region, suggesting our uncertainty quantification is well-calibrated.

\subsection{Further Analysis} 
\label{sec:further-analysis}
In this section, we present additional experiments to analyze and stress-test our framework. 
First, we perform an ablation study on the parameter $\eta$ (Section~\ref{subsec:ablation}), 
which controls the relative importance of label information in our kernel. Next, we demonstrate how to handle more complex datasets like IMDb (Section~\ref{subsec:nlp-datasets}). Finally, we show the robustness of our method in a heavily imbalanced and heterogeneous setting using the IMDb dataset (Section~\ref{subsec:imdb-robustness}).

\subsubsection{Effect of the label-weight parameter $\eta$}
\label{subsec:ablation}
We perform an ablation study to understand the influence of different $\eta$ (i.e., different weight of the output label) affects the mean squared error (MSE) of the utility and Shapley value predictions. This experiment uses a classification task on the synthetic Moon dataset with $6$ data owners.

\begin{figure}[ht]
\includegraphics[width=1.\columnwidth]{figures/synthetic_6_aba.png} 
\caption{
    A comparison of the quality of the utility and Shapley value predictions for various $\eta$ values on the Moon dataset with $6$ data-owners. }
    \label{fig:abamoon}
    \Description{Different label-weight parameter on the Moon dataset}
\end{figure}

A smaller $\eta$ value means that the label information has a greater influence on the dataset distance, allowing the GP model to better capture label-dependent patterns. As illustrated in Figure~\ref{fig:abamoon}, $\eta = 0.3$ results in the lowest MSE and highest Pearson correlation, indicating that assigning more weight to label information leads to improved predictive performance.

\subsubsection{Evaluation on an unstructured dataset with 10 data owners}
\label{subsec:nlp-datasets}
\begin{table*}[ht!]
\centering
\caption{
\textbf{A comparison of the quality of utility predictions and Shapley value predictions for various methods on the IMDb dataset with $10$ owners.}
\textbf{Setup~1 (similar to Section \ref{sec:cup})} compares the actual utility $u(C)$ with GP predicted expected utility $\hat{u}(C)$ of various coalition $C$.
\textbf{Setup~2 (similar to Section \ref{sec:est_sampling_shapley})} compares the actual Shapley value $\phi_i$ and the predicted Shapley value $(\hat{\phi}_i)$ for various owner $i$. 
The results are the mean $\pm$ std.\ over 5 runs.
Lower MSE and higher correlation are preferred.
}
\label{tab:imdb-combined-main}
\begin{tabular}{l|ccc|cc}
\toprule
& \multicolumn{3}{c|}{\textbf{Setup 1 (Section \ref{sec:cup})}} 
& \multicolumn{2}{c}{\textbf{Setup 2 (Section \ref{sec:est_sampling_shapley}) }} \\
\textbf{Method} 
& $\text{MSE}(u(C), \hat{u}(C))$ 
& $\text{Pearson}\bigl(u(C), \hat{u}(C) \bigr)$
& $\text{Shapley Corr.}$
& $\text{MSE}((\phi_i)_{i \in N}, (\hat{\phi}_i)_{i \in N})$
& $\text{Pearson}((\phi_i)_{i \in N}, (\hat{\phi}_i)_{i \in N})$ \\
\midrule
\texttt{SSW (Ours)} 
& $\mathbf{8.6\times10^{-6}\pm0.008}$ 
& $\mathbf{0.60\pm0.13}$ 
& $\mathbf{0.75\pm0.21}$
& $\mathbf{0.00022\pm0.00016}$ 
& $\mathbf{0.59\pm0.22}$\\
\texttt{GP-binary}  
& $1.8\times10^{-5}\pm0.0042$ 
& ${0.466\pm0.17}$ 
& $0.66\pm0.26$
& $0.00029\pm0.00014$
& $0.50\pm0.24$ \\
\texttt{NN-binary}  
& $0.006\pm0.008$ 
& $-0.199\pm0.07$ 
& $-0.137\pm0.35$
& $0.0001\pm0.00015$
& $0.52\pm0.025$ \\
\bottomrule
\end{tabular}
\end{table*}
We now evaluate our framework on the IMDb dataset, which comprises $50000$ movie reviews labeled as either positive or negative. Following the OpenDataVal benchmark \cite{opendataval2023}, we use DistilBERT \cite{sanh2019distilbert} embeddings for each review. We split the dataset among 10 data owners and consider two experimental settings as in Sections~\ref{sec:cup} and \ref{sec:est_sampling_shapley}). For the former, we use 256 evaluated coalitions. For the latter, we use 512 evaluated coalitions and 100 predicted coalitions.

Table~\ref{tab:imdb-combined-main} shows that our method $\SSW$ always achieves lower MSE and higher correlation than 
\texttt{GP-binary} and \texttt{NN-binary}, across five runs where the evaluated and predicted coalitions are randomly varied. In Appendix~\ref{appendix:SST-2}, we further validate our framework on the \emph{Stanford Sentiment Treebank (SST-2)} \cite{socher-etal-2013-recursive} dataset. We also observe that $\SSW$ consistently outperforms the baselines across different text-based tasks and pre-trained embeddings.

\subsubsection{Robustness to Heterogeneous Data Size and Distribution}
\balance
While the earlier experiments already account for varying dataset sizes and distributions (see Appendix~\ref{appendix:exp}, Table~\ref{tab:datasize_per_owner}), we further validate our framework under more extreme heterogeneity using the IMDb dataset. Specifically, we consider $20$ data owners, where the $j$-th data owner has $100j$ data points. Additionally, owners $j=1...5$ hold only data of the positive class, while owners $j=6...10$
hold only data of the negative class. The remaining $29000$ data points are used as the validation set. 
\label{subsec:imdb-robustness}
\begin{table}[ht!]
    \centering
    \caption{
    A comparison of the quality of the Shapley value predictions for various methods on the IMDb dataset with $20$ heterogeneous data owners. The Shapley value are computed based on $5000$ actual utility evaluations and $20000$ predicted utility evaluations.
    }
    \label{tab:imdb}
    \begin{tabular}{lcc}
    \toprule
    \textbf{Method} & $\text{MSE}$ & $\text{Pearson}$\\
    \midrule
    \texttt{SSW (Ours)}        &  ${9.60\times10^{-5}}$ &  $\mathbf{0.736}$\\
    \texttt{GP-binary }                 & $\mathbf{9.48\times10^{-5}}$ & $0.710$\\
    \texttt{NN-binary}                  & $1.00\times10^{-4}$ & $0.652$ \\
    \bottomrule
    \end{tabular}
\end{table}

We consider estimating the Shapley value with $25000$ coalitions ($\sim1800$ permutations, see Appendix~\ref{appdix:samp_algo}). Out of these $25000$ coalitions, we actually evaluate the utilities of $5000$ random coalitions and predict the utilities of the remaining coalitions. 
In Table~\ref{tab:imdb}, we observe that despite the large variation in dataset sizes and distributions, $\SSW$ achieves the strongest correlation (0.736) and low MSE ($9.60\times10^{-5}$). 
Additionally, we observe that the MSE does not increase when the owners have more data points. The Pearson correlation between dataset size and MSE for $\SSW$ is low and only \textbf{0.241}. 

\section{Related Work}
Our work is complementary to related works on \emph{data valuation} that propose new data utility functions (such as data volume \cite{NEURIPS2021_59a3adea} and information gain \cite{sim2020cml}) and strategies (such as the Shapley value \cite{ghorbani2019data, Kwon2021betashapley, yan2021core}, the Banzhaf value \cite{wang2023data} and Least Core \cite{yan2021core}). \texttt{DUPRE} can be used to efficiently predict the utilities of any data utility function needed in the data valuation strategies.

Our work is also complementary to \emph{semivalue approximation techniques} that reduce the number of coalitions to evaluate such as permutation sampling \cite{castro2009polynomial}, stratified sampling \cite{Maleki.2013}, structured sampling \citep{vanCampen.2018}, or approximating Shapley without marginal contribution \cite{kolpaczki2024without}. Instead of evaluating all the sampled coalitions' utilities by training a model, we propose evaluating a subset and predicting the remaining utilities using a GP model. Our work offers an alternative to methods that reduce the cost per evaluation, such as TMC-Shapley, Gradient Shapley \cite{ghorbani2019data} and the influence function heuristic used by \citet{jia2019towards}.

Our work can be extended to make use of other \emph{dataset distances}, such as optimal transport dataset distance (OTDD) \cite{alvarez2020otdd}, if they satisfy the valid properties of a kernel. There are also other data valuation works that have used the SW distance or aim to reduce the cost of each utility function but they differ in their application. In data valuation works, \citet{just2023lava,kessler2024sava} have also defined their data utility function based on the optimal transport distance between training data subsets and validation data. However, our purpose of considering dataset distances is different.

\section{Conclusion}
In this paper, we introduce \texttt{DUPRE}, a novel framework that complements existing sampling-based approximation methods to further boost the efficiency of computing CGT-based data valuation. We design a valid kernel based on the sliced Wasserstein distance and adapt the distance to consider the target outputs in supervised learning. As our kernel can encode prior knowledge of similarities between different data subsets, our GP model outperforms other approaches in our experiments.

While \texttt{DUPRE} demonstrates strong empirical performance, its predictions are not guaranteed to be accurate for every dataset or utility function. We recommend using a validation set of coalitions and utilities to continually assess and improve its predictions. 
Future work can consider other applications of data utility prediction, such as in online data valuation scenarios where new data owners frequently join or leave the collaboration.

\begin{acks}
This research is supported by the National Research Foundation Singapore and DSO National Laboratories under the AI Singapore Program (AISG Award No: AISG2-RP-2020-018).
\end{acks}
\bibliographystyle{ACM-Reference-Format}
% \bibliographystyle{unsrtnat}
\bibliography{refs}

%%%%%%%%%%%%%%%%%%%%%%%%%%%%%%%%%%%%%%%%%%%%%%%%%%%%%%%%%%%%
\ifshowappendix
\subsection{Lloyd-Max Algorithm}
\label{subsec:Lloyd-Max}
For a given quantization bitwidth $B$ and an operand $\bm{X}$, the Lloyd-Max algorithm finds $2^B$ quantization levels $\{\hat{x}_i\}_{i=1}^{2^B}$ such that quantizing $\bm{X}$ by rounding each scalar in $\bm{X}$ to the nearest quantization level minimizes the quantization MSE. 

The algorithm starts with an initial guess of quantization levels and then iteratively computes quantization thresholds $\{\tau_i\}_{i=1}^{2^B-1}$ and updates quantization levels $\{\hat{x}_i\}_{i=1}^{2^B}$. Specifically, at iteration $n$, thresholds are set to the midpoints of the previous iteration's levels:
\begin{align*}
    \tau_i^{(n)}=\frac{\hat{x}_i^{(n-1)}+\hat{x}_{i+1}^{(n-1)}}2 \text{ for } i=1\ldots 2^B-1
\end{align*}
Subsequently, the quantization levels are re-computed as conditional means of the data regions defined by the new thresholds:
\begin{align*}
    \hat{x}_i^{(n)}=\mathbb{E}\left[ \bm{X} \big| \bm{X}\in [\tau_{i-1}^{(n)},\tau_i^{(n)}] \right] \text{ for } i=1\ldots 2^B
\end{align*}
where to satisfy boundary conditions we have $\tau_0=-\infty$ and $\tau_{2^B}=\infty$. The algorithm iterates the above steps until convergence.

Figure \ref{fig:lm_quant} compares the quantization levels of a $7$-bit floating point (E3M3) quantizer (left) to a $7$-bit Lloyd-Max quantizer (right) when quantizing a layer of weights from the GPT3-126M model at a per-tensor granularity. As shown, the Lloyd-Max quantizer achieves substantially lower quantization MSE. Further, Table \ref{tab:FP7_vs_LM7} shows the superior perplexity achieved by Lloyd-Max quantizers for bitwidths of $7$, $6$ and $5$. The difference between the quantizers is clear at 5 bits, where per-tensor FP quantization incurs a drastic and unacceptable increase in perplexity, while Lloyd-Max quantization incurs a much smaller increase. Nevertheless, we note that even the optimal Lloyd-Max quantizer incurs a notable ($\sim 1.5$) increase in perplexity due to the coarse granularity of quantization. 

\begin{figure}[h]
  \centering
  \includegraphics[width=0.7\linewidth]{sections/figures/LM7_FP7.pdf}
  \caption{\small Quantization levels and the corresponding quantization MSE of Floating Point (left) vs Lloyd-Max (right) Quantizers for a layer of weights in the GPT3-126M model.}
  \label{fig:lm_quant}
\end{figure}

\begin{table}[h]\scriptsize
\begin{center}
\caption{\label{tab:FP7_vs_LM7} \small Comparing perplexity (lower is better) achieved by floating point quantizers and Lloyd-Max quantizers on a GPT3-126M model for the Wikitext-103 dataset.}
\begin{tabular}{c|cc|c}
\hline
 \multirow{2}{*}{\textbf{Bitwidth}} & \multicolumn{2}{|c|}{\textbf{Floating-Point Quantizer}} & \textbf{Lloyd-Max Quantizer} \\
 & Best Format & Wikitext-103 Perplexity & Wikitext-103 Perplexity \\
\hline
7 & E3M3 & 18.32 & 18.27 \\
6 & E3M2 & 19.07 & 18.51 \\
5 & E4M0 & 43.89 & 19.71 \\
\hline
\end{tabular}
\end{center}
\end{table}

\subsection{Proof of Local Optimality of LO-BCQ}
\label{subsec:lobcq_opt_proof}
For a given block $\bm{b}_j$, the quantization MSE during LO-BCQ can be empirically evaluated as $\frac{1}{L_b}\lVert \bm{b}_j- \bm{\hat{b}}_j\rVert^2_2$ where $\bm{\hat{b}}_j$ is computed from equation (\ref{eq:clustered_quantization_definition}) as $C_{f(\bm{b}_j)}(\bm{b}_j)$. Further, for a given block cluster $\mathcal{B}_i$, we compute the quantization MSE as $\frac{1}{|\mathcal{B}_{i}|}\sum_{\bm{b} \in \mathcal{B}_{i}} \frac{1}{L_b}\lVert \bm{b}- C_i^{(n)}(\bm{b})\rVert^2_2$. Therefore, at the end of iteration $n$, we evaluate the overall quantization MSE $J^{(n)}$ for a given operand $\bm{X}$ composed of $N_c$ block clusters as:
\begin{align*}
    \label{eq:mse_iter_n}
    J^{(n)} = \frac{1}{N_c} \sum_{i=1}^{N_c} \frac{1}{|\mathcal{B}_{i}^{(n)}|}\sum_{\bm{v} \in \mathcal{B}_{i}^{(n)}} \frac{1}{L_b}\lVert \bm{b}- B_i^{(n)}(\bm{b})\rVert^2_2
\end{align*}

At the end of iteration $n$, the codebooks are updated from $\mathcal{C}^{(n-1)}$ to $\mathcal{C}^{(n)}$. However, the mapping of a given vector $\bm{b}_j$ to quantizers $\mathcal{C}^{(n)}$ remains as  $f^{(n)}(\bm{b}_j)$. At the next iteration, during the vector clustering step, $f^{(n+1)}(\bm{b}_j)$ finds new mapping of $\bm{b}_j$ to updated codebooks $\mathcal{C}^{(n)}$ such that the quantization MSE over the candidate codebooks is minimized. Therefore, we obtain the following result for $\bm{b}_j$:
\begin{align*}
\frac{1}{L_b}\lVert \bm{b}_j - C_{f^{(n+1)}(\bm{b}_j)}^{(n)}(\bm{b}_j)\rVert^2_2 \le \frac{1}{L_b}\lVert \bm{b}_j - C_{f^{(n)}(\bm{b}_j)}^{(n)}(\bm{b}_j)\rVert^2_2
\end{align*}

That is, quantizing $\bm{b}_j$ at the end of the block clustering step of iteration $n+1$ results in lower quantization MSE compared to quantizing at the end of iteration $n$. Since this is true for all $\bm{b} \in \bm{X}$, we assert the following:
\begin{equation}
\begin{split}
\label{eq:mse_ineq_1}
    \tilde{J}^{(n+1)} &= \frac{1}{N_c} \sum_{i=1}^{N_c} \frac{1}{|\mathcal{B}_{i}^{(n+1)}|}\sum_{\bm{b} \in \mathcal{B}_{i}^{(n+1)}} \frac{1}{L_b}\lVert \bm{b} - C_i^{(n)}(b)\rVert^2_2 \le J^{(n)}
\end{split}
\end{equation}
where $\tilde{J}^{(n+1)}$ is the the quantization MSE after the vector clustering step at iteration $n+1$.

Next, during the codebook update step (\ref{eq:quantizers_update}) at iteration $n+1$, the per-cluster codebooks $\mathcal{C}^{(n)}$ are updated to $\mathcal{C}^{(n+1)}$ by invoking the Lloyd-Max algorithm \citep{Lloyd}. We know that for any given value distribution, the Lloyd-Max algorithm minimizes the quantization MSE. Therefore, for a given vector cluster $\mathcal{B}_i$ we obtain the following result:

\begin{equation}
    \frac{1}{|\mathcal{B}_{i}^{(n+1)}|}\sum_{\bm{b} \in \mathcal{B}_{i}^{(n+1)}} \frac{1}{L_b}\lVert \bm{b}- C_i^{(n+1)}(\bm{b})\rVert^2_2 \le \frac{1}{|\mathcal{B}_{i}^{(n+1)}|}\sum_{\bm{b} \in \mathcal{B}_{i}^{(n+1)}} \frac{1}{L_b}\lVert \bm{b}- C_i^{(n)}(\bm{b})\rVert^2_2
\end{equation}

The above equation states that quantizing the given block cluster $\mathcal{B}_i$ after updating the associated codebook from $C_i^{(n)}$ to $C_i^{(n+1)}$ results in lower quantization MSE. Since this is true for all the block clusters, we derive the following result: 
\begin{equation}
\begin{split}
\label{eq:mse_ineq_2}
     J^{(n+1)} &= \frac{1}{N_c} \sum_{i=1}^{N_c} \frac{1}{|\mathcal{B}_{i}^{(n+1)}|}\sum_{\bm{b} \in \mathcal{B}_{i}^{(n+1)}} \frac{1}{L_b}\lVert \bm{b}- C_i^{(n+1)}(\bm{b})\rVert^2_2  \le \tilde{J}^{(n+1)}   
\end{split}
\end{equation}

Following (\ref{eq:mse_ineq_1}) and (\ref{eq:mse_ineq_2}), we find that the quantization MSE is non-increasing for each iteration, that is, $J^{(1)} \ge J^{(2)} \ge J^{(3)} \ge \ldots \ge J^{(M)}$ where $M$ is the maximum number of iterations. 
%Therefore, we can say that if the algorithm converges, then it must be that it has converged to a local minimum. 
\hfill $\blacksquare$


\begin{figure}
    \begin{center}
    \includegraphics[width=0.5\textwidth]{sections//figures/mse_vs_iter.pdf}
    \end{center}
    \caption{\small NMSE vs iterations during LO-BCQ compared to other block quantization proposals}
    \label{fig:nmse_vs_iter}
\end{figure}

Figure \ref{fig:nmse_vs_iter} shows the empirical convergence of LO-BCQ across several block lengths and number of codebooks. Also, the MSE achieved by LO-BCQ is compared to baselines such as MXFP and VSQ. As shown, LO-BCQ converges to a lower MSE than the baselines. Further, we achieve better convergence for larger number of codebooks ($N_c$) and for a smaller block length ($L_b$), both of which increase the bitwidth of BCQ (see Eq \ref{eq:bitwidth_bcq}).


\subsection{Additional Accuracy Results}
%Table \ref{tab:lobcq_config} lists the various LOBCQ configurations and their corresponding bitwidths.
\begin{table}
\setlength{\tabcolsep}{4.75pt}
\begin{center}
\caption{\label{tab:lobcq_config} Various LO-BCQ configurations and their bitwidths.}
\begin{tabular}{|c||c|c|c|c||c|c||c|} 
\hline
 & \multicolumn{4}{|c||}{$L_b=8$} & \multicolumn{2}{|c||}{$L_b=4$} & $L_b=2$ \\
 \hline
 \backslashbox{$L_A$\kern-1em}{\kern-1em$N_c$} & 2 & 4 & 8 & 16 & 2 & 4 & 2 \\
 \hline
 64 & 4.25 & 4.375 & 4.5 & 4.625 & 4.375 & 4.625 & 4.625\\
 \hline
 32 & 4.375 & 4.5 & 4.625& 4.75 & 4.5 & 4.75 & 4.75 \\
 \hline
 16 & 4.625 & 4.75& 4.875 & 5 & 4.75 & 5 & 5 \\
 \hline
\end{tabular}
\end{center}
\end{table}

%\subsection{Perplexity achieved by various LO-BCQ configurations on Wikitext-103 dataset}

\begin{table} \centering
\begin{tabular}{|c||c|c|c|c||c|c||c|} 
\hline
 $L_b \rightarrow$& \multicolumn{4}{c||}{8} & \multicolumn{2}{c||}{4} & 2\\
 \hline
 \backslashbox{$L_A$\kern-1em}{\kern-1em$N_c$} & 2 & 4 & 8 & 16 & 2 & 4 & 2  \\
 %$N_c \rightarrow$ & 2 & 4 & 8 & 16 & 2 & 4 & 2 \\
 \hline
 \hline
 \multicolumn{8}{c}{GPT3-1.3B (FP32 PPL = 9.98)} \\ 
 \hline
 \hline
 64 & 10.40 & 10.23 & 10.17 & 10.15 &  10.28 & 10.18 & 10.19 \\
 \hline
 32 & 10.25 & 10.20 & 10.15 & 10.12 &  10.23 & 10.17 & 10.17 \\
 \hline
 16 & 10.22 & 10.16 & 10.10 & 10.09 &  10.21 & 10.14 & 10.16 \\
 \hline
  \hline
 \multicolumn{8}{c}{GPT3-8B (FP32 PPL = 7.38)} \\ 
 \hline
 \hline
 64 & 7.61 & 7.52 & 7.48 &  7.47 &  7.55 &  7.49 & 7.50 \\
 \hline
 32 & 7.52 & 7.50 & 7.46 &  7.45 &  7.52 &  7.48 & 7.48  \\
 \hline
 16 & 7.51 & 7.48 & 7.44 &  7.44 &  7.51 &  7.49 & 7.47  \\
 \hline
\end{tabular}
\caption{\label{tab:ppl_gpt3_abalation} Wikitext-103 perplexity across GPT3-1.3B and 8B models.}
\end{table}

\begin{table} \centering
\begin{tabular}{|c||c|c|c|c||} 
\hline
 $L_b \rightarrow$& \multicolumn{4}{c||}{8}\\
 \hline
 \backslashbox{$L_A$\kern-1em}{\kern-1em$N_c$} & 2 & 4 & 8 & 16 \\
 %$N_c \rightarrow$ & 2 & 4 & 8 & 16 & 2 & 4 & 2 \\
 \hline
 \hline
 \multicolumn{5}{|c|}{Llama2-7B (FP32 PPL = 5.06)} \\ 
 \hline
 \hline
 64 & 5.31 & 5.26 & 5.19 & 5.18  \\
 \hline
 32 & 5.23 & 5.25 & 5.18 & 5.15  \\
 \hline
 16 & 5.23 & 5.19 & 5.16 & 5.14  \\
 \hline
 \multicolumn{5}{|c|}{Nemotron4-15B (FP32 PPL = 5.87)} \\ 
 \hline
 \hline
 64  & 6.3 & 6.20 & 6.13 & 6.08  \\
 \hline
 32  & 6.24 & 6.12 & 6.07 & 6.03  \\
 \hline
 16  & 6.12 & 6.14 & 6.04 & 6.02  \\
 \hline
 \multicolumn{5}{|c|}{Nemotron4-340B (FP32 PPL = 3.48)} \\ 
 \hline
 \hline
 64 & 3.67 & 3.62 & 3.60 & 3.59 \\
 \hline
 32 & 3.63 & 3.61 & 3.59 & 3.56 \\
 \hline
 16 & 3.61 & 3.58 & 3.57 & 3.55 \\
 \hline
\end{tabular}
\caption{\label{tab:ppl_llama7B_nemo15B} Wikitext-103 perplexity compared to FP32 baseline in Llama2-7B and Nemotron4-15B, 340B models}
\end{table}

%\subsection{Perplexity achieved by various LO-BCQ configurations on MMLU dataset}


\begin{table} \centering
\begin{tabular}{|c||c|c|c|c||c|c|c|c|} 
\hline
 $L_b \rightarrow$& \multicolumn{4}{c||}{8} & \multicolumn{4}{c||}{8}\\
 \hline
 \backslashbox{$L_A$\kern-1em}{\kern-1em$N_c$} & 2 & 4 & 8 & 16 & 2 & 4 & 8 & 16  \\
 %$N_c \rightarrow$ & 2 & 4 & 8 & 16 & 2 & 4 & 2 \\
 \hline
 \hline
 \multicolumn{5}{|c|}{Llama2-7B (FP32 Accuracy = 45.8\%)} & \multicolumn{4}{|c|}{Llama2-70B (FP32 Accuracy = 69.12\%)} \\ 
 \hline
 \hline
 64 & 43.9 & 43.4 & 43.9 & 44.9 & 68.07 & 68.27 & 68.17 & 68.75 \\
 \hline
 32 & 44.5 & 43.8 & 44.9 & 44.5 & 68.37 & 68.51 & 68.35 & 68.27  \\
 \hline
 16 & 43.9 & 42.7 & 44.9 & 45 & 68.12 & 68.77 & 68.31 & 68.59  \\
 \hline
 \hline
 \multicolumn{5}{|c|}{GPT3-22B (FP32 Accuracy = 38.75\%)} & \multicolumn{4}{|c|}{Nemotron4-15B (FP32 Accuracy = 64.3\%)} \\ 
 \hline
 \hline
 64 & 36.71 & 38.85 & 38.13 & 38.92 & 63.17 & 62.36 & 63.72 & 64.09 \\
 \hline
 32 & 37.95 & 38.69 & 39.45 & 38.34 & 64.05 & 62.30 & 63.8 & 64.33  \\
 \hline
 16 & 38.88 & 38.80 & 38.31 & 38.92 & 63.22 & 63.51 & 63.93 & 64.43  \\
 \hline
\end{tabular}
\caption{\label{tab:mmlu_abalation} Accuracy on MMLU dataset across GPT3-22B, Llama2-7B, 70B and Nemotron4-15B models.}
\end{table}


%\subsection{Perplexity achieved by various LO-BCQ configurations on LM evaluation harness}

\begin{table} \centering
\begin{tabular}{|c||c|c|c|c||c|c|c|c|} 
\hline
 $L_b \rightarrow$& \multicolumn{4}{c||}{8} & \multicolumn{4}{c||}{8}\\
 \hline
 \backslashbox{$L_A$\kern-1em}{\kern-1em$N_c$} & 2 & 4 & 8 & 16 & 2 & 4 & 8 & 16  \\
 %$N_c \rightarrow$ & 2 & 4 & 8 & 16 & 2 & 4 & 2 \\
 \hline
 \hline
 \multicolumn{5}{|c|}{Race (FP32 Accuracy = 37.51\%)} & \multicolumn{4}{|c|}{Boolq (FP32 Accuracy = 64.62\%)} \\ 
 \hline
 \hline
 64 & 36.94 & 37.13 & 36.27 & 37.13 & 63.73 & 62.26 & 63.49 & 63.36 \\
 \hline
 32 & 37.03 & 36.36 & 36.08 & 37.03 & 62.54 & 63.51 & 63.49 & 63.55  \\
 \hline
 16 & 37.03 & 37.03 & 36.46 & 37.03 & 61.1 & 63.79 & 63.58 & 63.33  \\
 \hline
 \hline
 \multicolumn{5}{|c|}{Winogrande (FP32 Accuracy = 58.01\%)} & \multicolumn{4}{|c|}{Piqa (FP32 Accuracy = 74.21\%)} \\ 
 \hline
 \hline
 64 & 58.17 & 57.22 & 57.85 & 58.33 & 73.01 & 73.07 & 73.07 & 72.80 \\
 \hline
 32 & 59.12 & 58.09 & 57.85 & 58.41 & 73.01 & 73.94 & 72.74 & 73.18  \\
 \hline
 16 & 57.93 & 58.88 & 57.93 & 58.56 & 73.94 & 72.80 & 73.01 & 73.94  \\
 \hline
\end{tabular}
\caption{\label{tab:mmlu_abalation} Accuracy on LM evaluation harness tasks on GPT3-1.3B model.}
\end{table}

\begin{table} \centering
\begin{tabular}{|c||c|c|c|c||c|c|c|c|} 
\hline
 $L_b \rightarrow$& \multicolumn{4}{c||}{8} & \multicolumn{4}{c||}{8}\\
 \hline
 \backslashbox{$L_A$\kern-1em}{\kern-1em$N_c$} & 2 & 4 & 8 & 16 & 2 & 4 & 8 & 16  \\
 %$N_c \rightarrow$ & 2 & 4 & 8 & 16 & 2 & 4 & 2 \\
 \hline
 \hline
 \multicolumn{5}{|c|}{Race (FP32 Accuracy = 41.34\%)} & \multicolumn{4}{|c|}{Boolq (FP32 Accuracy = 68.32\%)} \\ 
 \hline
 \hline
 64 & 40.48 & 40.10 & 39.43 & 39.90 & 69.20 & 68.41 & 69.45 & 68.56 \\
 \hline
 32 & 39.52 & 39.52 & 40.77 & 39.62 & 68.32 & 67.43 & 68.17 & 69.30  \\
 \hline
 16 & 39.81 & 39.71 & 39.90 & 40.38 & 68.10 & 66.33 & 69.51 & 69.42  \\
 \hline
 \hline
 \multicolumn{5}{|c|}{Winogrande (FP32 Accuracy = 67.88\%)} & \multicolumn{4}{|c|}{Piqa (FP32 Accuracy = 78.78\%)} \\ 
 \hline
 \hline
 64 & 66.85 & 66.61 & 67.72 & 67.88 & 77.31 & 77.42 & 77.75 & 77.64 \\
 \hline
 32 & 67.25 & 67.72 & 67.72 & 67.00 & 77.31 & 77.04 & 77.80 & 77.37  \\
 \hline
 16 & 68.11 & 68.90 & 67.88 & 67.48 & 77.37 & 78.13 & 78.13 & 77.69  \\
 \hline
\end{tabular}
\caption{\label{tab:mmlu_abalation} Accuracy on LM evaluation harness tasks on GPT3-8B model.}
\end{table}

\begin{table} \centering
\begin{tabular}{|c||c|c|c|c||c|c|c|c|} 
\hline
 $L_b \rightarrow$& \multicolumn{4}{c||}{8} & \multicolumn{4}{c||}{8}\\
 \hline
 \backslashbox{$L_A$\kern-1em}{\kern-1em$N_c$} & 2 & 4 & 8 & 16 & 2 & 4 & 8 & 16  \\
 %$N_c \rightarrow$ & 2 & 4 & 8 & 16 & 2 & 4 & 2 \\
 \hline
 \hline
 \multicolumn{5}{|c|}{Race (FP32 Accuracy = 40.67\%)} & \multicolumn{4}{|c|}{Boolq (FP32 Accuracy = 76.54\%)} \\ 
 \hline
 \hline
 64 & 40.48 & 40.10 & 39.43 & 39.90 & 75.41 & 75.11 & 77.09 & 75.66 \\
 \hline
 32 & 39.52 & 39.52 & 40.77 & 39.62 & 76.02 & 76.02 & 75.96 & 75.35  \\
 \hline
 16 & 39.81 & 39.71 & 39.90 & 40.38 & 75.05 & 73.82 & 75.72 & 76.09  \\
 \hline
 \hline
 \multicolumn{5}{|c|}{Winogrande (FP32 Accuracy = 70.64\%)} & \multicolumn{4}{|c|}{Piqa (FP32 Accuracy = 79.16\%)} \\ 
 \hline
 \hline
 64 & 69.14 & 70.17 & 70.17 & 70.56 & 78.24 & 79.00 & 78.62 & 78.73 \\
 \hline
 32 & 70.96 & 69.69 & 71.27 & 69.30 & 78.56 & 79.49 & 79.16 & 78.89  \\
 \hline
 16 & 71.03 & 69.53 & 69.69 & 70.40 & 78.13 & 79.16 & 79.00 & 79.00  \\
 \hline
\end{tabular}
\caption{\label{tab:mmlu_abalation} Accuracy on LM evaluation harness tasks on GPT3-22B model.}
\end{table}

\begin{table} \centering
\begin{tabular}{|c||c|c|c|c||c|c|c|c|} 
\hline
 $L_b \rightarrow$& \multicolumn{4}{c||}{8} & \multicolumn{4}{c||}{8}\\
 \hline
 \backslashbox{$L_A$\kern-1em}{\kern-1em$N_c$} & 2 & 4 & 8 & 16 & 2 & 4 & 8 & 16  \\
 %$N_c \rightarrow$ & 2 & 4 & 8 & 16 & 2 & 4 & 2 \\
 \hline
 \hline
 \multicolumn{5}{|c|}{Race (FP32 Accuracy = 44.4\%)} & \multicolumn{4}{|c|}{Boolq (FP32 Accuracy = 79.29\%)} \\ 
 \hline
 \hline
 64 & 42.49 & 42.51 & 42.58 & 43.45 & 77.58 & 77.37 & 77.43 & 78.1 \\
 \hline
 32 & 43.35 & 42.49 & 43.64 & 43.73 & 77.86 & 75.32 & 77.28 & 77.86  \\
 \hline
 16 & 44.21 & 44.21 & 43.64 & 42.97 & 78.65 & 77 & 76.94 & 77.98  \\
 \hline
 \hline
 \multicolumn{5}{|c|}{Winogrande (FP32 Accuracy = 69.38\%)} & \multicolumn{4}{|c|}{Piqa (FP32 Accuracy = 78.07\%)} \\ 
 \hline
 \hline
 64 & 68.9 & 68.43 & 69.77 & 68.19 & 77.09 & 76.82 & 77.09 & 77.86 \\
 \hline
 32 & 69.38 & 68.51 & 68.82 & 68.90 & 78.07 & 76.71 & 78.07 & 77.86  \\
 \hline
 16 & 69.53 & 67.09 & 69.38 & 68.90 & 77.37 & 77.8 & 77.91 & 77.69  \\
 \hline
\end{tabular}
\caption{\label{tab:mmlu_abalation} Accuracy on LM evaluation harness tasks on Llama2-7B model.}
\end{table}

\begin{table} \centering
\begin{tabular}{|c||c|c|c|c||c|c|c|c|} 
\hline
 $L_b \rightarrow$& \multicolumn{4}{c||}{8} & \multicolumn{4}{c||}{8}\\
 \hline
 \backslashbox{$L_A$\kern-1em}{\kern-1em$N_c$} & 2 & 4 & 8 & 16 & 2 & 4 & 8 & 16  \\
 %$N_c \rightarrow$ & 2 & 4 & 8 & 16 & 2 & 4 & 2 \\
 \hline
 \hline
 \multicolumn{5}{|c|}{Race (FP32 Accuracy = 48.8\%)} & \multicolumn{4}{|c|}{Boolq (FP32 Accuracy = 85.23\%)} \\ 
 \hline
 \hline
 64 & 49.00 & 49.00 & 49.28 & 48.71 & 82.82 & 84.28 & 84.03 & 84.25 \\
 \hline
 32 & 49.57 & 48.52 & 48.33 & 49.28 & 83.85 & 84.46 & 84.31 & 84.93  \\
 \hline
 16 & 49.85 & 49.09 & 49.28 & 48.99 & 85.11 & 84.46 & 84.61 & 83.94  \\
 \hline
 \hline
 \multicolumn{5}{|c|}{Winogrande (FP32 Accuracy = 79.95\%)} & \multicolumn{4}{|c|}{Piqa (FP32 Accuracy = 81.56\%)} \\ 
 \hline
 \hline
 64 & 78.77 & 78.45 & 78.37 & 79.16 & 81.45 & 80.69 & 81.45 & 81.5 \\
 \hline
 32 & 78.45 & 79.01 & 78.69 & 80.66 & 81.56 & 80.58 & 81.18 & 81.34  \\
 \hline
 16 & 79.95 & 79.56 & 79.79 & 79.72 & 81.28 & 81.66 & 81.28 & 80.96  \\
 \hline
\end{tabular}
\caption{\label{tab:mmlu_abalation} Accuracy on LM evaluation harness tasks on Llama2-70B model.}
\end{table}

%\section{MSE Studies}
%\textcolor{red}{TODO}


\subsection{Number Formats and Quantization Method}
\label{subsec:numFormats_quantMethod}
\subsubsection{Integer Format}
An $n$-bit signed integer (INT) is typically represented with a 2s-complement format \citep{yao2022zeroquant,xiao2023smoothquant,dai2021vsq}, where the most significant bit denotes the sign.

\subsubsection{Floating Point Format}
An $n$-bit signed floating point (FP) number $x$ comprises of a 1-bit sign ($x_{\mathrm{sign}}$), $B_m$-bit mantissa ($x_{\mathrm{mant}}$) and $B_e$-bit exponent ($x_{\mathrm{exp}}$) such that $B_m+B_e=n-1$. The associated constant exponent bias ($E_{\mathrm{bias}}$) is computed as $(2^{{B_e}-1}-1)$. We denote this format as $E_{B_e}M_{B_m}$.  

\subsubsection{Quantization Scheme}
\label{subsec:quant_method}
A quantization scheme dictates how a given unquantized tensor is converted to its quantized representation. We consider FP formats for the purpose of illustration. Given an unquantized tensor $\bm{X}$ and an FP format $E_{B_e}M_{B_m}$, we first, we compute the quantization scale factor $s_X$ that maps the maximum absolute value of $\bm{X}$ to the maximum quantization level of the $E_{B_e}M_{B_m}$ format as follows:
\begin{align}
\label{eq:sf}
    s_X = \frac{\mathrm{max}(|\bm{X}|)}{\mathrm{max}(E_{B_e}M_{B_m})}
\end{align}
In the above equation, $|\cdot|$ denotes the absolute value function.

Next, we scale $\bm{X}$ by $s_X$ and quantize it to $\hat{\bm{X}}$ by rounding it to the nearest quantization level of $E_{B_e}M_{B_m}$ as:

\begin{align}
\label{eq:tensor_quant}
    \hat{\bm{X}} = \text{round-to-nearest}\left(\frac{\bm{X}}{s_X}, E_{B_e}M_{B_m}\right)
\end{align}

We perform dynamic max-scaled quantization \citep{wu2020integer}, where the scale factor $s$ for activations is dynamically computed during runtime.

\subsection{Vector Scaled Quantization}
\begin{wrapfigure}{r}{0.35\linewidth}
  \centering
  \includegraphics[width=\linewidth]{sections/figures/vsquant.jpg}
  \caption{\small Vectorwise decomposition for per-vector scaled quantization (VSQ \citep{dai2021vsq}).}
  \label{fig:vsquant}
\end{wrapfigure}
During VSQ \citep{dai2021vsq}, the operand tensors are decomposed into 1D vectors in a hardware friendly manner as shown in Figure \ref{fig:vsquant}. Since the decomposed tensors are used as operands in matrix multiplications during inference, it is beneficial to perform this decomposition along the reduction dimension of the multiplication. The vectorwise quantization is performed similar to tensorwise quantization described in Equations \ref{eq:sf} and \ref{eq:tensor_quant}, where a scale factor $s_v$ is required for each vector $\bm{v}$ that maps the maximum absolute value of that vector to the maximum quantization level. While smaller vector lengths can lead to larger accuracy gains, the associated memory and computational overheads due to the per-vector scale factors increases. To alleviate these overheads, VSQ \citep{dai2021vsq} proposed a second level quantization of the per-vector scale factors to unsigned integers, while MX \citep{rouhani2023shared} quantizes them to integer powers of 2 (denoted as $2^{INT}$).

\subsubsection{MX Format}
The MX format proposed in \citep{rouhani2023microscaling} introduces the concept of sub-block shifting. For every two scalar elements of $b$-bits each, there is a shared exponent bit. The value of this exponent bit is determined through an empirical analysis that targets minimizing quantization MSE. We note that the FP format $E_{1}M_{b}$ is strictly better than MX from an accuracy perspective since it allocates a dedicated exponent bit to each scalar as opposed to sharing it across two scalars. Therefore, we conservatively bound the accuracy of a $b+2$-bit signed MX format with that of a $E_{1}M_{b}$ format in our comparisons. For instance, we use E1M2 format as a proxy for MX4.

\begin{figure}
    \centering
    \includegraphics[width=1\linewidth]{sections//figures/BlockFormats.pdf}
    \caption{\small Comparing LO-BCQ to MX format.}
    \label{fig:block_formats}
\end{figure}

Figure \ref{fig:block_formats} compares our $4$-bit LO-BCQ block format to MX \citep{rouhani2023microscaling}. As shown, both LO-BCQ and MX decompose a given operand tensor into block arrays and each block array into blocks. Similar to MX, we find that per-block quantization ($L_b < L_A$) leads to better accuracy due to increased flexibility. While MX achieves this through per-block $1$-bit micro-scales, we associate a dedicated codebook to each block through a per-block codebook selector. Further, MX quantizes the per-block array scale-factor to E8M0 format without per-tensor scaling. In contrast during LO-BCQ, we find that per-tensor scaling combined with quantization of per-block array scale-factor to E4M3 format results in superior inference accuracy across models. 

\fi
\end{document}


