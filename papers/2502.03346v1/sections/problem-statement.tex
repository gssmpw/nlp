\section{Problem Statement}\label{sec:problem-statement}

We consider a human $H$ and a robot $R$ collaboratively transporting an object. The robot and the human grasp the object at a fixed height; this allows us to instantiate the problem on a planar workspace $\mathcal{W}\subseteq SE(2)$.
Assuming a quasistatic setting, the object's state $p \in \mathcal{W}$ evolves according to $p_{k+1} = f(p_k, a_k, u_k)$, where $a\in\mathcal{A}$, $u\in\mathcal{U}$ represent human and robot velocities, respectively, and $k$ is a time index.
The workspace includes a set of obstacle-occupied regions $\mathcal{O}\subset\mathcal{W}$. The goal of the human-robot team is to transport the object from an initial pose $p_0$ to a desired pose $g$ in $\mathcal{W}$ (see~\figref{fig:setup}) while avoiding collisions with $\mathcal{O}$. We assume that the two agents do not communicate explicitly (e.g., via language), but they observe the actions of one another. Our goal is to design a control policy to enable the robot to efficiently and fluently collaborate with its human partner.


\begin{figure}[t]
  \includegraphics[width=\columnwidth]{figures/problem-statement.png}
  \caption{A human (H) and a robot (R) collaboratively move an object from an initial pose $p_0$ to a final pose $g$ in a workspace $\mathcal{W}$. An obstacle $\mathcal{O}$ stands in their way. To avoid collisions with $\mathcal{O}$ and reach $g$, they have to coordinate on a strategy of workspace traversal. In this work, we engineer implicit coordination through the velocities $a$ and $u$ that the human and the robot exert on the object.
  }
  \label{fig:setup}
\end{figure}
