\section{Limitations}\label{sec:limitations}

While our decision-making framework is agnostic to the types of actions that agents are executing, in this work we considered a quasistatic setting integrating only velocities that the agents transmit to the transported object. Future work will integrate actions stemming from additional modalities, such as force, language, body posture, eye gaze, and gestures. Additionally, while our framework is not prescriptive on the space of joint strategies $\Psi$, this work emphasizes the topological relationships between the human-robot team and the obstacles around it. Future work will integrate additional strategy attributes, such as leadership~\citep{mortl2012role}, and \emph{timing} of critical maneuvers. Our control implementation was based on a flexible and practical MPC framework, but alternative approaches could be explored, such as POMDPs~\citep{nikolaidis2017human}. Our controller demonstrated practical performance assuming a simple model of human motion prediction (constant velocity); future work will integrate more fine-grained prediction models~\citep{salzmann2023hst,yasar2024posetron}. The relatively low max base speed (0.3 m/s) and payload (2 kg) of our robot might have influenced the types of interactions with users. We aspire to deploy our framework on a robot with higher payload and maximum speed to expand to tasks involving heavier objects and allow for movement across all 6 object DoFs. We also plan on expanding to more complex environments with more obstacles and dynamic agents.
