
\subsection{Experiments on the fanout\label{ssec:exp2}}



\definecolor{fanout1}{RGB}{165,42,42}
\definecolor{fanoutfp1}{RGB}{255,165,0}
\definecolor{fanout2fp1}{RGB}{160,32,240}
\definecolor{fanout3fp1}{RGB}{255,192,203}

\begin{figure}[h]
    \centering

\setlength\tabcolsep{1.5pt}
\begin{tabular}{|c|c|c|}
\hline
{\scriptsize score}
&
{\scriptsize$OF_{S_{ND}}^{F_{IN}}$ violations}
&
{\scriptsize Legend}
\\
\hline 
\includegraphics[scale=.25]{plots/exp2/exp2_score_plot_no_leg.png}
&
\includegraphics[scale=.25]{plots/exp2/exp2_finord_wrt_send_plot.png}
&
\scalebox{.7}{
\begin{tikzpicture}
\node[align=center] (leg1) at (0,0) {\textcolor{fanout1}{$\blacksquare$} {\footnotesize fanout$=1$}}; 
%
\node[below=.25cm of leg1.south west,anchor=west,align=center] (leg2) {\textcolor{fanoutfp1}{$\blacksquare$} {\footnotesize fanout$=f+1$}};
%
\node[below=.25cm of leg2.south west,anchor=west,align=center] (leg3) {\textcolor{fanout2fp1}{$\blacksquare$} {\footnotesize fanout$=2*f+1$}};
%
\node[below=.25cm of leg3.south west,anchor=west,align=center] (leg4) {\textcolor{fanout3fp1}{$\blacksquare$} {\footnotesize fanout$=3*f+1$}};
%
%
\node[below right=.3cm and -2cm of leg4] (leg5) {\footnotesize FullShuffle};
\node[draw=black,fill=black,left=.2cm of leg5,circle,inner sep=2pt] (leg5s) {};
\draw[thick] ($(leg5s) + (-.3,0) $) -- ($ (leg5s) + (.3,0) $);
%
\node[below=.25cm of leg5.south west,anchor=west,align=center] (leg6) {\footnotesize PerColumnShuffle};
\node[left=.2cm of leg6,inner sep=0pt] (leg6s) {$\blacktriangle$};
\draw[thick,dotted] ($(leg6s) + (-.3,0) $) -- ($ (leg6s) + (.3,0) $);
%
\node[below=.25cm of leg6.south west,anchor=west,align=center] (leg7) {\footnotesize VoteCount};
\node[left=.2cm of leg7,inner sep=0pt] (leg7s) {$\blacksquare$};
\draw[thick,dashed] ($(leg7s) + (-.3,0) $) -- ($ (leg7s) + (.3,0) $);
\end{tikzpicture}
}
\\
\hline 
\hline
{\scriptsize duplications in DAG}
&
{\scriptsize$OF_{S_{ND}}^{W_{AV}}$ violations}
&
{\scriptsize$OF_{D_{LV}}^{W_{AV}}$ violations}
\\
\hline 
\includegraphics[scale=.25]{plots/exp2/exp2_duplicated_plot.png}
&
\includegraphics[scale=.25]{plots/exp2/exp2_wavord_wrt_send_plot.png}
&
\includegraphics[scale=.25]{plots/exp2/exp2_wavord_wrt_bdlv_plot.png}
\\
\hline 
\end{tabular}
\setlength\tabcolsep{6pt}
    
    \caption{Experiments w.r.t.~the fanout}
    \label{fig:exp2}
\end{figure}



The second experiments investigate the effect of parameter \textbf{(3)} i.e., the fanout from clients to nodes.
Parameter \textbf{(2)} is fixed as the slow network.
We consider $4$ possible values for parameter \textbf{(3)}: $1$, $f+1$, $2*f+1$ or $3*f+1$, the latter coinciding with the first set of experiments (clients broadcasting their transactions to all the nodes).
Fig.\ref{fig:exp2} summarizes experimental results.

The simulations highlight a tradeoff between robustness to transaction reordering and a higher rate of duplications in the DAG (i.e., the same transaction is found in multiple vertices).
Indeed, on the bottom left diagram, we can see, that the number of duplications increases with the fanout.
At higher fanouts, there is a slight decrease along the $x$ axis because Byzantine nodes do not include certain transactions.
On the top left diagram, we can see that, the lower the fanout is, the more vulnerable the system is (i.e., the decreases in the score becomes higher as the adversarial power increases).
For $OF_{S_{ND}}^{F_{IN}}$, we can see that when $b=0$, the lower the fanout, the higher the number of violations is.
However, while for fanout higher than $f$, this number increases with $b$, when it is $1$, it decreases. 
This is explained by the fact that in the former cases, all transactions are guaranteed to be included in the DAG (any transaction being send to at least one honest node) while it is not the case for the latter.
The transactions that disappear in the case of fanout $1$ are not taken into account when counting the pairs of transactions which violate $OF_{S_{ND}}^{F_{IN}}$, which explains the decrease.




