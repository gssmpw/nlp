

\section{Introduction}


A Distributed Ledger (DL) is a set of replicated state machines. 
In cryptocurrencies and Web3, individual machines are called nodes and transitions in the state machine correspond to the finalization
of a transaction.
Clients may modify the ledger state by sending transactions to nodes.
Nodes maintain consensus on the current state by agreeing on the initial state and on the order with which transactions are finalized.


Distributed computing literature recognizes two main classes of properties of DL which are ``\textit{consistency}'' (i.e., consensus agreement and validity which are safety properties \cite{recognizing_safety_and_liveness}) and ``\textit{liveness}'' (an umbrella term that can refer to consensus wait-freedom, starvation-freedom or local progress) \cite{safety_liveness_exclusion_in_distributed_computing}.
However, as detailed in \cite{order_fairness_for_byzantine_consensus}, these properties have no say on the actual order of transactions that is agreed upon.
It is then possible, for a malicious adversary, to manipulate the order of transactions if given the opportunity.
In decentralized finance \cite{sok_preventing_transaction_reordering_manipulations_in_decentralized_finance}, front-running attacks \cite{flash_boys_frontrunning_in_decentralized_exchanges_miner_extractable_value_and_consensus_instability} consist in, having knowledge of a pending transaction $x$, placing a new transaction $x'$ in front of it (i.e., so that $x'$ is finalized before $x$).
Combining such attacks can be used to extract profits via manipulating the value of financial assets.
In practice, this is done by automated Maximum Extractable Value (MEV) bots \cite{flash_boys_frontrunning_in_decentralized_exchanges_miner_extractable_value_and_consensus_instability,sok_preventing_transaction_reordering_manipulations_in_decentralized_finance}. 
In Ethereum PoS \cite{exploiting_ethereum_after_the_merge_the_interplay_between_pos_and_mev_strategies}, front running can be performed by providing a higher gas reward for $x'$, thus providing an incentive to place $x'$ before $x$.
However, even without relying on reward mechanisms (gas), control over nodes still allows front running \cite{adversary_augmented_simulation_to_evaluate_client_fairness_on_hyperledger_fabric}.
According to Forbes, between January 2020 and September 2022, MEV bots have extracted $\sim675$ million \$ profits on the Ethereum blockchain alone.





Blockchains constitute a means to implement a DL.
In a blockchain, transactions are batched into successive blocks and their finalization is ordered according to the order in which they appear on and within blocks.
Each block is proposed by a unique node called a leader who gathers new transactions in that block.
In contrast to Blockchains, DAG-based DL \cite{sok_dag_based_blockchain_systems} such as \cite{all_you_need_is_dag,narwhal_and_tusk,bullshark,reducing_latency_of_dag_based_consensus_in_the_asynchronous_setting_via_the_utxo_model} rely on a Directed Acyclic Graph which vertices contain batches of transactions.
Each node can broadcast new vertices that are then added to the DAG. The definition of edges from more recent vertices to older vertices ensures consistency as it enables the nodes to agree on a shared view of their local copies of the DAG.
Similarly to Blockchains, transactions are then finalized in batches except that these batches rather correspond to sets of vertices, called ``waves'' in \cite{all_you_need_is_dag,bullshark,narwhal_and_tusk}.
With most current Blockchain algorithms, manipulations of transactions order may trivially occur if, e.g., the current leader has an incentive to do so.
\cite{exploiting_ethereum_after_the_merge_the_interplay_between_pos_and_mev_strategies} discusses it for the Ethereum PoS Blockchain and \cite{adversary_augmented_simulation_to_evaluate_client_fairness_on_hyperledger_fabric} experiments on such manipulations in HyperLedger Fabric.
The absence of block-proposing leaders in DAG-based DLT should, theoretically, make them less vulnerable to such attacks.
Yet, in the literature, such evaluations are lacking.



Transaction reordering attacks in DL may be mitigated using cryptography (commit \& reveal \cite{sok_preventing_transaction_reordering_manipulations_in_decentralized_finance,maximal_extractable_value_protection_on_a_DAG,fairness_notions_in_dag_based_dlts}). 
However, this solution comes with a high overhead.
As such, to alleviate these risks, ``\textit{order fairness}'' (OF) properties and Blockchain protocols that uphold them (called Algorithmic Committee Ordering in \cite{sok_preventing_transaction_reordering_manipulations_in_decentralized_finance}) have been designed \cite{order_fairness_for_byzantine_consensus,quick_order_fairness}.
OF properties relate various partial orders of events to one another.
These include the orders with which transactions are submitted (by clients), received (by nodes) and finalized.
These orders are not necessarily observable and their precise definitions may vary depending on the algorithms that are considered.
OF is a relatively new subject of study in traditional Blockchains and this is all the more true for DAG-based ledgers \cite{fairness_notions_in_dag_based_dlts}.


In this paper, we define several novel OF metrics that are particularly adapted to DAG algorithms such as DagRider \cite{all_you_need_is_dag}, Bullshark \cite{bullshark} and Tusk \cite{narwhal_and_tusk}.
Then, in the spirit of \cite{adversary_augmented_simulation_to_evaluate_client_fairness_on_hyperledger_fabric}, we devise DAG-specific attack scenarios aiming at manipulating transactions ordering and observe their effect on OF.
In particular, we consider the DagRider protocol \cite{all_you_need_is_dag}, complemented with mechanisms inspired by \cite{reducing_latency_of_dag_based_consensus_in_the_asynchronous_setting_via_the_utxo_model} and \cite{order_fairness_for_byzantine_consensus,themis_fast_strong_order_fairness_in_byzantine_consensus}.
We study the impact of attacks orchestrated by an adversary under various hypotheses on the network and algorithm parameterization.
The adversary may infect a number of nodes (so as to modify their behavior) bounded by the fault tolerance thresholds of the involved algorithms.
Our study suggests that, even though they are more robust than their Blockchain counterparts (e.g., HyperLedger Fabric in \cite{adversary_augmented_simulation_to_evaluate_client_fairness_on_hyperledger_fabric}), permissioned DAG-based ledgers such as DagRider \cite{all_you_need_is_dag} are still vulnerable to transaction reordering attacks.



This paper is organized as follows:
After preliminaries and related works in Sec.\ref{sec:related}, Sec.\ref{sec:prel} introduces our system model and application use case.
We then present DagRider and OF properties adapted to its specificities in Sec.\ref{sec:dag}.
Sec.\ref{sec:attacks} introduces attack scenarios before Sec.\ref{sec:experiments} details our experiments.
We conclude in Sec.\ref{sec:conc}.











