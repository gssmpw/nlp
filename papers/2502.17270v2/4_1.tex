


The goal of the adversary is to prevent a specific target client $\chi$ to win puzzle games.
To do so, puzzle solutions send by $\chi$ must be ordered after solutions of the same puzzle send by other clients.
We call such attacks transaction reordering attacks \cite{sok_preventing_transaction_reordering_manipulations_in_decentralized_finance} as the adversary changes the natural order of transactions which would otherwise have occurred.


\begin{wrapfigure}{r}{.19\textwidth}
    \centering
    \scalebox{.75}{\begin{tikzpicture}
\node[fill=violet,fill opacity=.25,draw=black,text opacity=1,minimum height=1.75cm, minimum width=.8cm] (b0) at (0,0) {
$
\begin{array}{l}
~
\\
~
\\
~
\\
\end{array}
$
};
%
\node[fill=cyan,fill opacity=.25,draw=black,text opacity=1,minimum height=1.75cm, minimum width=.8cm,right=.5cm of b0] (b1) {
$
\begin{array}{l}
\hline 
x_1
\\
x_2
\\
x_3
\\
\end{array}
$
};
\draw[->] (b1.130) -- (b0.east);
%
\node[fill=red,fill opacity=.25,draw=black,text opacity=1,minimum height=1.75cm, minimum width=.8cm,right=1cm of b1] (b2) {
$
\begin{array}{l}
\hline 
x_4
\\
x_5
\\
~
\\
\end{array}
$
};
\draw[->] (b2.130) -- (b1.east);
%
\node[red] (anchor) at (1.37,.42) {\Huge$\times$};
\draw (anchor.center) edge[bend left=35,->,red,dashed] ($(anchor) + (.25,-1)$);
\draw (anchor.center) edge[bend left=25,->,red,dashed] ($(anchor) + (1.7,-.1)$);
\node[draw=red,fill=white] (mal) at ($(anchor) + (.7,-.3)$) {\shortColRed{or}};
%
\node[draw=red,fill=white,circle,inner sep=.05cm] (var1) at ($(anchor) + (.5,-1.1)$) {\small\shortColRed{1}};
\node[draw=red,fill=white,circle,inner sep=.05cm] (var1) at ($(anchor) + (1.45,-.25)$) {\small\shortColRed{2}};
\end{tikzpicture}}
    \caption{Desired effect on a Blockchain}
    \label{fig:blockchain_attacked}
\vspace*{-.5cm}
\end{wrapfigure}



In some Blockchains such attacks may succeed if the leader of the next block is infected by the adversary.
Fig.\ref{fig:blockchain_attacked} illustrates this, supposing that the adversary wishes to delay the finalization of $x_1$ and has infected the leader node for the cyan block.
In this configuration, the adversary may (as represented by the red dashed arrows) either \shortColRed{\textcircled{1}} put $x_1$ at the end of the block or \shortColRed{\textcircled{2}} ignore $x_1$, which may then be added in the next block by the leader for the next height.
In doing so, the adversary accomplishes its goal without violating neither consistency nor liveness.
In both cases, it is likely (provided the honest leader would otherwise have respected it) that receive-order fairness is, as a result, violated.


In DAG-based DL, we have waves instead of blocks.
The equivalent of variant \shortColRed{\textcircled{1}} may amount to influencing the outcome of the deterministic order used to order vertices of a wave.
Variant \shortColRed{\textcircled{2}} in contrast, may amount to changing which vertices are included in the wave.




To achieve its goal, the adversary may infect a limited number of nodes and modify their behavior (on either or both the DagRider and Reliable Broadcast layers). This number is bound by the fault tolerance thresholds of the involved algorithms, here the same $f = (n-1)/3$ for Bracha and DagRider.
Honest nodes faithfully participate in both Bracha and DagRider.
This signifies that \textbf{(1)} they emit ECHO and READY messages as soon as possible, \textbf{(2)} they do not exclude received transactions from their vertex proposals, \textbf{(3)} they propose non-empty vertices as soon as possible and \textbf{(4)} upon proposing a vertex, they include all possible strong (between $2*f+1$ and $n$) and weak edges (between $0$ and $f$).
In the following, we present how infected nodes can be coordinated to achieve effect analogous to those in Fig.\ref{fig:blockchain_attacked}.




