


\subsection{Byzantine behavior on the DagRider layer\label{ssec:byz_dagrider_layer}}

Infected nodes may propose vertices differently.
In addition to not including transactions from the target client $\chi$, infected nodes will also avoid including edges that point towards other vertices that contain transactions from $\chi$.


\begin{wrapfigure}{l}{.25\textwidth}
\vspace*{-.3cm}
    \centering
    \scalebox{.75}{\begin{tikzpicture}[semifill/.style n args={2}{path picture={
 \fill[#1] (path picture bounding box.north east) rectangle
 (path picture bounding box.south);
  \fill[#2] (path picture bounding box.south west) rectangle
 (path picture bounding box.north);
 }}]
\node[fill=violet,fill opacity=.25,draw=black,text opacity=1,minimum height=1.75cm, minimum width=.8cm,inner sep=0cm] (v_0_A) at (0,0) {
$
\begin{array}{l}
v_0^A
\\
~
\\
~
\\
\end{array}
$
};
%
%
\node[semifill={red}{cyan!25},fill opacity=.25,draw=black,text opacity=1,minimum height=1.75cm, minimum width=.8cm, right=.5cm of v_0_A,inner sep=0cm] (v_1_A) {
$
\begin{array}{l}
v_1^A
\\
\hline 
x_1
\\
~
\\
\end{array}
$
};
%
\node[fill=red,fill opacity=.25,draw=black,text opacity=1,minimum height=1.75cm, minimum width=.8cm, right=.5cm of v_1_A,inner sep=0cm] (v_2_A) {
$
\begin{array}{l}
v_2^A
\\
\hline 
x_3
\\
~
\\
\end{array}
$
};
%
\node[fill=red,fill opacity=.25,draw=black,text opacity=1,minimum height=1.75cm, minimum width=.8cm, right=.5cm of v_2_A,inner sep=0cm] (v_3_A) {
$
\begin{array}{l}
v_3^A
\\
\hline 
x_2
\\
x_4
\\
\end{array}
$
};
%
\node[fill=red,fill opacity=.25,draw=black,text opacity=1,minimum height=1.75cm, minimum width=.8cm, right=.5cm of v_3_A,inner sep=0cm] (v_4_A) {
$
\begin{array}{l}
v_4^A
\\
\hline 
x_5
\\
~
\\
\end{array}
$
};
%
% ========================================================
%
\node[fill=violet,fill opacity=.25,draw=black,text opacity=1,minimum height=1.75cm, minimum width=.8cm,inner sep=0cm] (v_0_B) at (0,-2.25) {
$
\begin{array}{l}
v_0^B
\\
~
\\
~
\\
\end{array}
$
};
%
\node[below left=-.75cm and -.75cm of v_0_B] {\includegraphics[width=.6cm]{images/thief.png}};
%
\node[fill=cyan,fill opacity=.25,draw=black,text opacity=1,minimum height=1.75cm, minimum width=.8cm, right=.5cm of v_0_B,inner sep=0cm] (v_1_B) {
$
\begin{array}{l}
v_1^B
\\
\hline 
\text{\st{$x_1$}}
\\
x_2
\\
\end{array}
$
};
%
\node[fill=cyan,fill opacity=.25,draw=black,text opacity=1,minimum height=1.75cm, minimum width=.8cm, right=.5cm of v_1_B,inner sep=0cm] (v_2_B) {
$
\begin{array}{l}
v_2^B
\\
\hline 
x_3
\\
~
\\
\end{array}
$
};
%
\node[fill=red,fill opacity=.25,draw=black,text opacity=1,minimum height=1.75cm, minimum width=.8cm, right=.5cm of v_2_B,inner sep=0cm] (v_3_B) {
$
\begin{array}{l}
v_3^B
\\
\hline 
x_4
\\
~
\\
\end{array}
$
};
%
% ========================================================
%
\node[fill=violet,fill opacity=.25,draw=black,text opacity=1,minimum height=1.75cm, minimum width=.8cm,inner sep=0cm] (v_0_C) at (0,-4.5) {
$
\begin{array}{l}
v_0^C
\\
~
\\
~
\\
\end{array}
$
};
%
\node[semifill={cyan}{red!25},fill opacity=.25,draw=black,text opacity=1,minimum height=1.75cm, minimum width=.8cm, right=.5cm of v_0_C,inner sep=0cm] (v_1_C) {
$
\begin{array}{l}
v_1^C
\\
\hline 
x_2
\\
~
\\
\end{array}
$
};
%
% ========================================================
%
\draw[->] (v_1_A.150) -- (v_0_A.east);
\draw[->] (v_1_A.150) -- (v_0_B.east);
\draw[->] (v_2_A.150) -- (v_1_A.east);
\draw[->] (v_2_A.150) -- (v_1_B.east);
\draw[->] (v_3_A.150) -- (v_2_A.east);
\draw[->] (v_3_A.150) -- (v_2_B.east);
\draw[->] (v_4_A.150) -- (v_3_A.east);
\draw[->] (v_4_A.150) -- (v_3_B.east);
%
\draw[->] (v_1_B.150) -- (v_0_B.east);
\draw[->] (v_1_B.150) -- (v_0_C.east);
\draw[->] (v_2_B.150) -- (v_1_B.east);
\draw[->,red] (v_2_B.150) -- (v_1_C.east);
\draw[->] (v_3_B.150) -- (v_2_A.east);
\draw[->] (v_3_B.150) -- (v_2_B.east);
%
\draw[->] (v_1_C.150) -- (v_0_C.east);
\draw[->] (v_1_C.150) -- (v_0_B.east);
%
\node[align=left,below=1cm of v_3_B,draw] (leg) {excludes $x_1$ from $v_1^B$\\makes $v_2^B$ target $v_1^C$};
\end{tikzpicture}


}
    \caption{DagRider layer attack}
    \label{fig:dag_attacked_dagrider_layer}
\vspace*{-.4cm}
\end{wrapfigure}

Fig.\ref{fig:dag_attacked_dagrider_layer} illustrates this.
Here node B is controlled by the adversary.
If B were to behave honestly, the resulting DAG would be the one from Fig.\ref{fig:distributed_ledger_dag_and_blockchains}.
By contrast, Fig.\ref{fig:dag_attacked_dagrider_layer} results from B not including $x_1$ in $v_1^B$ and favoring a strong edge to $v_1^C$ rather than to $v_1^A$ in its $v_2^B$ vertex proposal.
Overall, this has the effect of excluding transaction $x_1$ from the first wave (in cyan), delaying its delivery to the second wave (in red). As a result, this has the same effect as variant \shortColRed{\textcircled{2}} of Fig.\ref{fig:blockchain_attacked}.
See Appendix \ref{anx:detail_dag_rider_layer_attack} for details.



