


\subsection{Byzantine behavior on the DagRider layer\label{ssec:byz_dagrider_layer}}

Infected nodes may propose vertices differently.
In addition to not including transactions from the target client $\chi$, infected nodes will also avoid including edges that point towards other vertices that contain transactions from $\chi$.


\begin{wrapfigure}{l}{.25\textwidth}
\vspace*{-.3cm}
    \centering
    \scalebox{.75}{\input{figures/attack_goal/dag_attacked_dagrider_layer}}
    \caption{DagRider layer attack}
    \label{fig:dag_attacked_dagrider_layer}
\vspace*{-.4cm}
\end{wrapfigure}

Fig.\ref{fig:dag_attacked_dagrider_layer} illustrates this.
Here node B is controlled by the adversary.
If B were to behave honestly, the resulting DAG would be the one from Fig.\ref{fig:distributed_ledger_dag_and_blockchains}.
By contrast, Fig.\ref{fig:dag_attacked_dagrider_layer} results from B not including $x_1$ in $v_1^B$ and favoring a strong edge to $v_1^C$ rather than to $v_1^A$ in its $v_2^B$ vertex proposal.
Overall, this has the effect of excluding transaction $x_1$ from the first wave (in cyan), delaying its delivery to the second wave (in red). As a result, this has the same effect as variant \shortColRed{\textcircled{2}} of Fig.\ref{fig:blockchain_attacked}.
See Appendix \ref{anx:detail_dag_rider_layer_attack} for details.



