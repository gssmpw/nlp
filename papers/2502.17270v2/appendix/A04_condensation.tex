

\section{Our solution for the ``VoteCount'' mechanism\label{anx:vote_count_mechanism}}


\begin{figure}[h]
    \centering
    \includegraphics[scale=.275]{figures/deterministic/ord_sccs.png}
    \caption{$\leq_{\text{\faEnvelopeO}}$ while highlighting SCCs}
    \label{fig:ord_graph_condensation}
\end{figure}


\begin{figure}[h]
    \centering
    \includegraphics[scale=.275]{figures/deterministic/topo_ranks.png}
    \caption{Topological ranks}
    \label{fig:ord_topological_ranks}
\end{figure}

In this appendix,
we detail the ``VoteCount'' mechanism used to order vertices in a wave of the DAG.
It is inspired by 
\textbf{(1)}
the initial idea from \cite{reducing_latency_of_dag_based_consensus_in_the_asynchronous_setting_via_the_utxo_model}
of dressing a table of votes from the structure of the DAG wave and use it to define a precedence relation $\leq_{\text{\faEnvelopeO}}$,
\textbf{(2)} the idea from \cite{order_fairness_for_byzantine_consensus,themis_fast_strong_order_fairness_in_byzantine_consensus} to solve the issue caused by Condorcet cycles that could exist in the precedence relation 
via identifying them as Strongly Connected Components of the corresponding graph representation 
and \textbf{(3)} the topological ranking notion from
\cite{diversified_top_k_graph_pattern_matching} as an alternative to the Hamiltonian path method used in \cite{themis_fast_strong_order_fairness_in_byzantine_consensus}.



We illustrate this mechanism's use on Fig.\ref{fig:ord_graph_condensation} and Fig.\ref{fig:ord_topological_ranks}, applying it to the example from Fig.\ref{fig:do_example_1}.


Representing $\geq_{\text{\faEnvelopeO}}$ as a directed graph, the first step is to compute its Strongly Connected Components (SCCs) which we highlight in blue on Fig.\ref{fig:ord_graph_condensation}.
This allows identifying Condorcet cycles (on Fig.\ref{fig:ord_graph_condensation} there is a cycle with $4$ vertices).
The condensation graph \cite{order_fairness_for_byzantine_consensus,themis_fast_strong_order_fairness_in_byzantine_consensus}, obtained via merging vertices in the same SCC, is guaranteed to be acyclic.
This allows to use the topological rank notion from \cite{diversified_top_k_graph_pattern_matching} which is defined as follows:
leaves of the condensation graph have rank 0, while the rank of the other vertices correspond to their shortest distance to any of the leaves.
On Fig.\ref{fig:ord_topological_ranks} we give the topological rank of each vertex (the rank is written on the left, from 0 at the top to 4 at the bottom). 
This allows sorting the vertices in a manner that is consistent with $\geq_{\text{\faEnvelopeO}}$.
The order between vertices with the same topological rank can be resolved in any deterministic manner (e.g., lexicographic order on their hash value).