
\clearpage 

\section{Additional metrics on the first experiments\label{anx:exp1}}




\begin{figure}[h]
    \centering

\setlength\tabcolsep{1.5pt}
\begin{tabular}{|c|c|}
\hline
{\scriptsize $\#$ of solved puzzles}
&
{\scriptsize $\#$ of waves}
\\
\hline 
\includegraphics[scale=.3]{plots/exp1/exp1_num_solved_puzzles_plot.png}
&
\includegraphics[scale=.3]{plots/exp1/exp1_num_waves_plot.png}
\\
\hline 
\hline
{\scriptsize $\#$ of transactions per wave}
&
{\scriptsize$\#$ of $OF_{I_{NI}}^{W_{AV}}$ violations}
\\
\hline 
\includegraphics[scale=.3]{plots/exp1/exp1_block_size_plot.png}
&
\includegraphics[scale=.3]{plots/exp1/exp1_wavord_wrt_binit_plot.png}
\\
\hline 
\hline
{\scriptsize$\#$ of $OF_{R_{EC}}^{F_{IN}}$ violations}
&
{\scriptsize$\#$ of $OF_{I_{NI}}^{F_{IN}}$ violations}
\\
\hline 
\includegraphics[scale=.3]{plots/exp1/exp1_finord_wrt_rec_plot.png}
&
\includegraphics[scale=.3]{plots/exp1/exp1_finord_wrt_binit_plot.png}
\\
\hline 
\hline 
{\scriptsize$\#$ of $OF_{D_{LV}}^{F_{IN}}$ violations}
&
\\
\hline 
\includegraphics[scale=.3]{plots/exp1/exp1_finord_wrt_bdlv_plot.png}
&
\\
\hline 
\end{tabular}
\setlength\tabcolsep{6pt}
    
    \caption{Other metrics for the first experiment}
    \label{fig:exp1_other}
\end{figure}


Fig.\ref{fig:exp1_other} provide additional metrics (in addition of those of Fig.\ref{fig:exp1}) that characterize the simulations performed in the first set of experiments.

The total number of solved puzzles (top left of Fig.\ref{fig:exp1_other}) indicates that in all simulations, all metrics are measured after having solved at least 3010 puzzles.
Thus, we have statistically significant results in regards to fairness.



Let us then consider two next diagrams. 
The top right diagram gives the total number of waves at the end of the simulation.
The left diagram on the the second row gives, in the form of ribbon plots, the 1st quartile, median value and 3rd quartile of the distribution of the number of transactions per wave (across the 300 to 800 waves in the simulation).
Let us also recall that the duration of the simulations are configured so that at least 3000 puzzles are solved and the frequency at which new puzzles are revealed does not change.
In the slower network with higher delays, the number of waves taken to solve these 3000 puzzles is smaller (dotted lines with triangles on the top right diagram), and each such wave contains more transactions (light green, orange and cyan ribbons on the left diagram of the second row).

The high values on the number of transactions per wave (left diagram of the second row) are explained by the fact that certain leaders may be skipped over (see discussion in Appendix \ref{anx:max_sim}). Whenever a leader of wave $w$ is skipped, the content of wave $w$ is eventually included in that of wave $w+1$, leading to larger waves that contain more transactions.
Because there are less waves in the simulations on the slower network (around 350 compared to the 700 waves in the quicker network) it is also more likely that the 3rd quartile corresponds to such enlarged waves.
The fact that there are larger delays (light green, orange and cyan ribbons) may also increase the risk that certain leaders are skipped over (because of the unreliable delivery, the strong path condition is less likely to be met).

These diagrams also highlight a side effect of the impact of the adversary (the $x$-axis corresponding to the power of the adversary).
Indeed, as the adversary causes the delivery of certain vertices to be delayed, it slows down the overall throughput (indeed, in addition of these vertices themselves being delayed, as nodes require at least $2*f+1$ strong edges to vertices at column $c$ to propose a new vertex at column $c+1$, the production of new vertices may also be slowed down).
In turn, this causes delayed vertices (and thus also waves) to contain more transactions (increase in the number of transactions per wave) and the total number of waves to decrease.

The number of violations of $OF_{I_{NI}}^{W_{AV}}$ and $OF_{D_{LV}}^{F_{IN}}$ follow the same trends as discussed in Sec.\ref{ssec:exp1}.
As for $OF_{R_{EC}}^{F_{IN}}$ and $OF_{I_{NI}}^{F_{IN}}$, we observe that the statistical noise is greater than for $OF_{S_{ND}}^{F_{IN}}$ and prevents reaching conclusions.

