

\section{Bracha's Byzantine Reliable Broadcast\label{anx:bracha}}

Bracha's Byzantine Reliable Broadcast algorithm \cite{asynchronous_byzantine_agreement_protocols} relies on two ``echoing'' phases to ascertain that all nodes agree on the same message that is broadcast.
A node that wants to initiate the reliable broadcast start with a simple broadcast of an ``INIT'' message carrying the message it wants to reliably broadcast (this corresponds to the call of $\mathtt{rbcast}$ on Fig.\ref{fig:layers_dagrider}).
The first echoing phase then consists in nodes emitting and collecting ``ECHO'' messages that answer the first ``INIT'' message.
The second echoing phase likewise consists in collecting ``READY'' messages.

Once enough ``READY'' messages have been received locally by a given node, that node can deliver the reliably broadcast message (this corresponds to the call of $\mathtt{rdlver}$ on Fig.\ref{fig:layers_dagrider}) because it is then certain that all the other nodes will also eventually deliver that same message.

The thresholds for the collection of ``ECHO'' and ``READY'' messages depend on the maximum number of byzantine nodes $f$ as follows: $\lfloor \frac{n+f}{2} \rfloor + 1$ ``ECHO'' or $f+1$ ``READY'' messages from distinct nodes are required to broadcast a new ``READY'' message and $2*f + 1$ ``READY'' messages are required for delivery ($\mathtt{rdlver}$).


