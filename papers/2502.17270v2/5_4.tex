
\subsection{Varying the number of nodes\label{ssec:exp3}}



\definecolor{nodesnum10}{RGB}{248,118,109}
\definecolor{nodesnum13}{RGB}{124,174,0}
\definecolor{nodesnum16}{RGB}{0,191,196}
\definecolor{nodesnum19}{RGB}{199,124,255}

\begin{figure}[h]
    \centering

\setlength\tabcolsep{1.5pt}
\begin{tabular}{|c|c|c|}
\hline
{\scriptsize score}
&
{\scriptsize$OF_{S_{ND}}^{F_{IN}}$ violations}
&
{\scriptsize Legend}
\\
\hline 
\includegraphics[scale=.25]{plots/exp3/exp3_score_plot_no_leg.png}
&
\includegraphics[scale=.25]{plots/exp3/exp3_finord_wrt_send_plot.png}
&
\scalebox{.7}{
\begin{tikzpicture}
\node[align=center] (leg1) at (0,0) {\textcolor{nodesnum10}{$\blacksquare$} {\footnotesize $f=3$, $n=10$}};
%
\node[below=.175cm of leg1.south west,anchor=west,align=center] (leg2) {\textcolor{nodesnum13}{$\blacksquare$} {\footnotesize $f=4$, $n=13$}};
%
\node[below=.175cm of leg2.south west,anchor=west,align=center] (leg3) {\textcolor{nodesnum16}{$\blacksquare$} {\footnotesize $f=5$, $n=16$}};
%
\node[below=.175cm of leg3.south west,anchor=west,align=center] (leg4) {\textcolor{nodesnum19}{$\blacksquare$} {\footnotesize $f=6$, $n=19$}};
%
%
\node[below right=.1cm and -2cm of leg4] (leg5) {\footnotesize fanout$=3*f+1$};
\node[draw=black,fill=black,left=.2cm of leg5,circle,inner sep=2pt] (leg5s) {};
%
\node[below=.2cm of leg5.south west,anchor=west,align=center] (leg6) {\footnotesize fanout$=1$};
\node[left=.2cm of leg6,inner sep=0pt] (leg6s) {$\blacktriangle$};
%
\node[below=.25cm of leg6.south west,anchor=west,align=center] (leg7) {\footnotesize Smaller Delays};
\node[left=.2cm of leg7,inner sep=0pt] (leg7s) {$~$};
\draw[thick] ($(leg7s) + (-.3,0) $) -- ($ (leg7s) + (.3,0) $);
%
\node[below=.25cm of leg7.south west,anchor=west,align=center] (leg8) {\footnotesize Larger Delays};
\node[left=.2cm of leg8,inner sep=0pt] (leg8s) {$~$};
\draw[thick,dotted] ($(leg8s) + (-.3,0) $) -- ($ (leg8s) + (.3,0) $);
\end{tikzpicture}
}
\\
\hline 
\hline
{\scriptsize solved puzzles}
&
{\scriptsize $OF_{S_{ND}}^{W_{AV}}$ violations}
&
{\scriptsize waves}
\\
\hline 
\includegraphics[scale=.25]{plots/exp3/exp3_num_solved_puzzles_plot.png}
&
\includegraphics[scale=.25]{plots/exp3/exp3_wavord_wrt_send_plot.png}
&
\includegraphics[scale=.25]{plots/exp3/exp3_num_waves_plot.png}
\\
\hline 
\end{tabular}
\setlength\tabcolsep{6pt}
    
    \caption{Varying the number of nodes.}
    \label{fig:exp3}
\end{figure}

In Sec.\ref{ssec:exp1} and Sec.\ref{ssec:exp2}, we had fixed the number of nodes $n$ to $13 = 1 + 3*4$ and varied the number of Byzantine nodes $b$ between $0$ and $f=4$ (the proportion of Byzantine nodes thus varying from $0$ to $33\%$).
This arbitrary $n=13$ is a tradeoff w.r.t.~the computational costs of performing the simulations.
Indeed, in order to have statistically significant results, we need long simulations (in each, the system solves at least 3000 puzzles so that the score values converge, as per the law of large numbers). 
With $n=13$, at 2 puzzles per wave, we must simulate around $7$ million message exchanges on the Bracha layer.

In this third set of experiments, we show that our results for $n=13$ can be generalized to any value of $n$. To do so, we show that varying $n$ do not influence the impact of the other parameters on the robustness of the system w.r.t.~the proportion of Byzantine nodes $b/n$.

Fig.\ref{fig:exp3} summarizes our results.
Here, we consider the ``PerColumnShuffle'' deterministic order.
On the legend, in the top right, the colors correspond to different values of $n$ that we consider (10, 13, 16 and 19).
While the shape of the points (circles or triangles) correspond to the fanout value parameter,
the style of the lines (continuous or dotted) correspond to the \faClockO~delay distribution.
On the top of Fig.\ref{fig:exp3}, we can see that varying $n$ does not impact the effect of increasing the proportion $b/n$ of Byzantine nodes (on the $x$ axis) on the number of $OF_{S_{ND}}^{F_{IN}}$ violations (on the $y$ axis).
Indeed, for all values of $n$, at a fixed value of fanout, the number of violations is lower in the case of smaller delays and higher in the case of larger delays.
Reciprocally, at a fixed value of the delay distribution, the number of violations is lower at fanout $3*f+1$ and increases with $b/n$ while it is higher at fanout $1$ and decreases with $b/n$.
Similar observations can be made on all the other metrics.
















