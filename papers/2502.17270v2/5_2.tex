

\subsection{Experiments on the delay distribution\label{ssec:exp1}}








\begin{figure}[h]
    \centering

\setlength\tabcolsep{1.5pt}
\begin{tabular}{|c|c|c|}
\hline
{\scriptsize score}
&
{\scriptsize$OF_{S_{ND}}^{F_{IN}}$ violations}
&
{\scriptsize Legend}
\\
\hline 
\includegraphics[scale=.25]{plots/exp1/exp1_score_plot_no_leg.png}
&
\includegraphics[scale=.25]{plots/exp1/exp1_finord_wrt_send_plot.png}
&
\scalebox{.7}{
\begin{tikzpicture}
\node[align=center] (leg1) at (0,0) {\textcolor{darkspringgreen}{$\blacksquare$} {\footnotesize FullShuffle}}; 
%
\node[below=.25cm of leg1.south west,anchor=west,align=center] (leg2) {\textcolor{blue}{$\blacksquare$} {\footnotesize PerColumnShuffle}};
%
\node[below=.25cm of leg2.south west,anchor=west,align=center] (leg3) {\textcolor{red}{$\blacksquare$} {\footnotesize VoteCount}};
%
\node[below right=.3cm and -.9cm of leg3] (leg4) {\footnotesize Smaller Delays};
\node[draw=black,fill=black,left=.2cm of leg4,circle,inner sep=2pt] (leg4s) {};
\draw[thick] ($(leg4s) + (-.3,0) $) -- ($ (leg4s) + (.3,0) $);
\node[below=.1cm of leg4.south east,anchor=east,align=right] (leg4b) {\scriptsize (quick network)};
%
\node[below=.75cm of leg4.south west,anchor=west,align=center] (leg5) {\footnotesize Larger Delays};
\node[left=.2cm of leg5,inner sep=0pt] (leg5s) {$\blacktriangle$};
\draw[dotted,thick] ($(leg5s) + (-.3,0) $) -- ($ (leg5s) + (.3,0) $);
\node[below=.1cm of leg5.south east,anchor=east,align=right] (leg5b) {\scriptsize (slow network)};
\end{tikzpicture}
}
\\
\hline 
\hline
{\scriptsize$OF_{S_{ND}}^{W_{AV}}$ violations}
&
{\scriptsize$OF_{R_{EC}}^{W_{AV}}$ violations}
&
{\scriptsize$OF_{D_{LV}}^{W_{AV}}$ violations}
\\
\hline 
\includegraphics[scale=.25]{plots/exp1/exp1_wavord_wrt_send_plot.png}
&
\includegraphics[scale=.25]{plots/exp1/exp1_wavord_wrt_rec_plot.png}
&
\includegraphics[scale=.25]{plots/exp1/exp1_wavord_wrt_bdlv_plot.png}
\\
\hline 
\end{tabular}
\setlength\tabcolsep{6pt}
    
    \caption{Experiments w.r.t.~the deterministic order and \faClockO}
    \label{fig:exp1}
\end{figure}


This first set of experiments investigates the effect of parameters \textbf{(1)} (the deterministic order) and \textbf{(2)} (the \faClockO~delay distribution).
Parameter \textbf{(3)} is fixed : clients send their transactions to all the nodes.
Fig.\ref{fig:exp1} summarizes experimental results.
On each diagram, the proportion $b/n$ of Byzantine nodes (with $b \in [0,f]$) corresponds to the $x$-axis (it represents the power of the adversary). 
The colors and styles of the different curves resp.~correspond parameters \textbf{(1)} and \textbf{(2)}.
The first diagram, on the top left of Fig.\ref{fig:exp1}, gives the $\mathtt{score}$ of the client targeted by the attack on the $y$ axis.
As for the other four diagrams, they plot the number of pairs of transactions for which the corresponding property is violated.

In all cases, the adversary succeeds in reducing $\mathtt{score}(\chi)$.
Still, we remark that VoteCount has a protective effect.
However, and this is especially true for VoteCount, higher delays in the network may cause a higher vulnerability of the system.
Indeed, this gives the adversary more leeway to manipulate the order of delivery of the vertices, which has a stronger effect on VoteCount because it determines the in-wave order based on the columns at which nodes include delivered vertices in their latest vertices' causal sub-graph.

The decrease in the score is correlated to an increase in the number of $OF_{S_{ND}}^{F_{IN}}$ violations.
As expected, the FullShuffle deterministic order is more susceptible to violate $OF_{S_{ND}}^{F_{IN}}$ because the in-wave order is entirely random.
However, we observe that VoteCount yields more violations than PerColumnShuffle.
This can be explained by the fact that VoteCount determines the order based on the order of delivery of the vertices and not on the order of reception of individual transactions.
On the other hand, PerColumnShuffle always order a vertex at column $c+1$ after a vertex at column $c$.
Then, because honest nodes always include transactions as early as possible, the number of violations only depends on the client to node delay distribution and the randomness introduced by PerColumnShuffle when shuffling vertices at the same column.
On the other hand, because VoteCount takes into account the delivery order, the node to node \faClockO delay distribution also introduces noise, which explains the higher number of violations.

When considering the four diagrams on the left, what we observe is the combined effect of the DagRider layer and Bracha layer attacks (Sec.\ref{ssec:byz_dagrider_layer} and Sec.\ref{ssec:attack_bracha}).
Overall, we observe that the increase in wave-order-fairness violations ($OF_{S_{ND}}^{W_{AV}}$ and $OF_{R_{EC}}^{W_{AV}}$) is sharper than that for finalization-order-fairness ($OF_{S_{ND}}^{F_{IN}}$).
Indeed, the attack may exclude vertices from certain waves (especially whenever the wave leader is Byzantine).

Because the Bracha-layer attack's effect is to manipulate the order of delivery, and because $OF_{D_{LV}}^{W_{AV}}$ only relates this vertex delivery order to the finalization order, this means that the initial increase in $OF_{D_{LV}}^{W_{AV}}$ violations on the bottom right diagram is only due to the DagRider layer attack. Then, the decrease at higher adversarial power can be explained by the fact that the Bracha layer attack already successfully reorders the relevant pairs of transactions (and thus, new violations are less likely to be added at the DagRider layer).

See Appendix \ref{anx:exp1} for further details.





