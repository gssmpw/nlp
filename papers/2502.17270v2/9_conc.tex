

\section{Conclusion\label{sec:conc}}


Our contribution concerns a theoretical and empirical security evaluation of DAG-based ledgers' robustness to reordering attacks (i.e., in which the adversary causes transactions to be delivered in a specific order) and therefore of order fairness.
We formalize novel order fairness properties that specifically describe DAG-based ledgers.
We demonstrate that, via targeting both the DAG construction logic and the underlying Byzantine Reliable Broadcast protocol, even a weak adversary can have a significant impact on the order with which transactions are delivered
(the adversary being weak as it does not control the network but only coordinates a minority of Byzantine nodes, below the fault tolerance thresholds of the involved algorithms).
Moreover, we highlight the tradeoff between transaction duplication and the robustness to such attacks.

Our study underlines the importance of considering order fairness as properties of interest when defining algorithms that implement distributed ledgers. 
It especially suggests that DAG-based ledgers are also vulnerable (in the same way as classical Blockchains) to transaction reordering attacks.




