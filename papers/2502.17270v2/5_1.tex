




\subsection{System parameterization\label{ssec:network_param}}


We consider $m=3$ clients and we simulate Peer To Peer communications with no loss.
Delays for the transmission of individual messages are sampled from hypoexponential probability distributions (as recommended in \cite{models_of_network_delay}), which we bound with an upper value $\Delta$.
This reflects the non-determinism in network communications while staying within the conservative hypothesis of a partially synchronous communication model \cite{consensus_in_the_presence_of_partial_synchrony}.
New puzzles are revealed at regular intervals.
The time required for a client to solve a puzzle is sampled from a Poisson distribution.
In addition of the transactions send by the clients, which contain puzzle solutions, the nodes also regularly receive third party transactions that are not part of the puzzle game (to simulate the DL being used concurrently by other applications at the application layer).


Our aim is to evaluate the robustness of the system w.r.t.~the proportion of nodes that are infected by the adversary (i.e., the Byzantine nodes)
We also vary three additional parameters:
\textbf{(1)} the deterministic order,
%, which is either one of the three options from Sec.\ref{ssec:deterministic_order}, 
\textbf{(2)} the \textcolor{black}{\faClockO} distribution of delays between nodes (in this way, we vary the network ratio \cite{themis_fast_strong_order_fairness_in_byzantine_consensus} i.e., how the propagation time of transactions relates to the rate at which they are created),
%, corresponding to two types of networks : quick and slow  
and \textbf{(3)} the fanout from clients to nodes.
%i.e., we consider three cases : the clients systematically broadcast their transactions to all the nodes or to randomly selected subsets of $1$, $f+1$ or $2*f+1$ nodes.
Details in Appendix \ref{anx:max_sim}.

In the following, we present three sets of experiments.
For the first two, we fix the value of $n$ to $13$, i.e., we have up to $f=4$ byzantine nodes.
Then, in the third set of experiments, we vary the value of $n$ to show that the results of the first two experiments do not depend on $n$.
All the materials required to reproduce them are available at \cite{max_dagrider_order_fairness_exp}.



