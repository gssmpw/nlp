

\subsection{Evaluation metrics\label{ssec:metrics}}


To asses the vulnerability of DAG-based DL to transaction reordering attacks, beyond reasoning on examples such as Fig.\ref{fig:dag_attacked_dagrider_layer} and Fig.\ref{fig:dag_attacked_bracha_layer}, we need concrete metrics.
As explained in Sec.\ref{sec:prel}, we can monitor the number of pairs of transactions $(x,x')$ for which OF properties are violated. We now introduce 8 DAG-specific OF properties (details in Appendix \ref{anx:enumeration_order_fairness_props}).

In Sec.\ref{ssec:dagrider} we characterized the lifecycle of a transaction $x$ on a specific DagRider node by four instants:
\textbf{(1)} reception of $x$ from a client,
\textbf{(2)} triggering the broadcast of the first vertex that contains $x$,
\textbf{(3)} delivery of the first vertex that contains $x$ 
and
\textbf{(4)} finalization of $x$ (the first occurrence of $x$, given that transactions may be duplicated in the DAG).
The definition of OF properties depend on these instants as OF relates the order of transaction reception on each individual node (``reception'', which can be understood as instant \textbf{(1)}, \textbf{(2)} or \textbf{(3)}) and the order of their eventual finalization (instant \textbf{(4)}).
Instants \textbf{(2)} and \textbf{(3)} are guaranteed to exist in all cases.
However, for instant \textbf{(1)}, it is true iff clients broadcast their transactions to all the nodes.


We consider properties $OF^\beta_\alpha$ of the form: for any transactions $x$ and $x'$, \textbf{if} $\alpha(x,x')$ \textbf{then} $\beta(x,x')$.

The $\beta$ predicate concerns transaction finalization, with two cases: $F_{IN}$ (for ``finalization'') and $W_{AV}$ (for ``wave''), so that $\beta \in \{F_{IN},~W_{AV}\}$.
For any two transactions $x$ and $x'$, $F_{IN}(x,x')$ signifies that all honest nodes must finalize $x$ before $x'$ while $W_{AV}(x,x')$ implies that no honest node can finalize $x$ in a wave after that in which $x'$ is finalized.
While $F_{IN}$ corresponds to {\em receive-order fairness}, $W_{AV}$ coincides with adapting a certain interpretation of the {\em block-order fairness} from \cite{order_fairness_for_byzantine_consensus} to DAGs.
$W_{AV}$ also provides an interesting metric. Indeed, violations of $OF^{W_{AV}}_*$ properties can be linked to variant variant \shortColRed{\textcircled{2}} of Fig.\ref{fig:blockchain_attacked} while those of $OF^{F_{IN}}_*$ properties may correspond to either \shortColRed{\textcircled{1}} or \shortColRed{\textcircled{2}}.


The $\alpha$ predicate involves relations between communication events. 
We consider four cases with $\alpha \in \{S_{ND},R_{EC},I_{NI},D_{LV}\}$.
$S_{ND}(x,x')$ signifies that the initial emission of $x$ by a certain client precedes that of $x'$.
$R_{EC}(x,x')$ (and resp.~$I_{NI}$ and $D_{LV}$) signifies that a majority of nodes receive $x$ from a client (and resp.~ begin their participation in the reliable broadcast of a vertex that contains $x$ and deliver a vertex containing $x$) before they do so for $x'$ i.e., instant \textbf{(1)} (and resp.~\textbf{(2)} and \textbf{(3)}).

%Let us remark that many variants of these properties may be defined. For instance, we could consider differential (as in the \textit{differential-order fairness} of \cite{quick_order_fairness}) variants  of $R_{EC}$, $I_{NI}$ and $D_{LV}$, which, instead of being of the form ``a majority of nodes do $X(x,x')$'' are of the form ``the number of nodes that do $X(x,x')$ is greater than the number of nodes that do $X(x',x)$ plus $2*f$''.




