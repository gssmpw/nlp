\section{Shortest Path versus Effective Resistance Queries}
\label{sec:sp-versus-er}



In this section, we motivate the study of effective resistance queries by showing that they are incomparable in power to shortest path queries. Specifically, we show that sometimes it is (provably) possible to use fewer effective resistance queries to solve a given task, and sometimes it is possible to use fewer shortest path queries.

Our first result asserts that it is more efficient to check whether a graph is a clique using effective resistance queries than shortest path queries.
\begin{theorem} \label{thm:er-sp-clique}
There is an algorithm for testing whether a graph $G$ with $n$ vertices is a clique (i.e., whether $G = K_n$) using $O(n)$ effective resistance queries, but any such algorithm requires $\Omega(n^2)$ shortest path queries. 
\end{theorem}

\begin{proof}
As shown in \cref{thm:proper-subgraph-verification}, it is possible to check using $O(n)$ effective resistance queries whether two graphs $G$ and $H$ are equal, assuming that $G$ is a subgraph of $H$. The algorithm for checking that $G$ is a clique follows from the special case of this result when $H = K_n$, as every graph is a subgraph of $K_n$.

The lower bound for shortest path queries works as follows. Suppose that $G$ is either equal to $K_n$ or equal to $K_n \setminus \{e\}$ for a single unknown edge $e$. Any shortest path query on $u, v$ will return the same value (i.e., $1$) in either of these two cases, except when $u$ and $v$ are the endpoints of $e$. It follows that any algorithm requires $\binom{n}{2}$ shortest path queries.
\end{proof}

Our second result asserts that it is more efficient to check whether two given vertices in $G$ are adjacent using shortest path queries than effective resistance queries.
\full{
\begin{figure}
    \begin{center}
    \setlength\tabcolsep{2cm}.
    \begin{tabular}{cc}
    \includegraphics[scale=0.55]{figs/spath-eres-1} &
    \includegraphics[scale=0.55]{figs/spath-eres-2}
    \end{tabular}
    \end{center}
    \caption{\small A pair of unweighted graphs $G$ and $H_{i,j}$ on $n$ vertices that require $\Omega(n)$ effective resistance queries to distinguish; see \cref{thm:er-sp-adjacency}.  The effective resistance between $v_1, v_2$ is the same in both graphs (i.e., $R_G(v_1, v_2) = R_H(v_1, v_2) = 1$) despite $v_1, v_2$ being adjacent in $G$ but not in $H_{i,j}$.}
    \label{fig:effect-res-inefficient}
\end{figure}
}
\begin{theorem} \label{thm:er-sp-adjacency}
There is an algorithm for testing whether two given vertices $u, v \in V$ in a simple graph are adjacent using $1$ shortest path query, but any algorithm for this problem requires $\Omega(n)$ effective resistance queries.
\end{theorem}
\full{
\begin{proof}
It is possible to check whether $u, v$ are adjacent using a single shortest path query since $u$ and $v$ are adjacent if and only if the length of the shortest path between them is $1$.

On the other hand, consider the problem of deciding whether the underlying graph is (1) $G$ or (2) of the form $H = H_{i,j}$ for some $i, j$ using effective resistance queries, where these graphs are as shown in \cref{fig:effect-res-inefficient}.
I.e., $G$ consists of two star graphs on $n/2$ vertices whose centers $v_1, v_2$ are connected by an edge, and $H = H_{i,j}$ consists of two star graphs on $n/2 - 1$ vertices whose centers $v_1, v_2$ form a square with the vertices $v_i, v_j$, where $i, j \in \set{3, \ldots, n}$, $i \neq j$ are unknown.
In particular, the graphs $G$ and $H$ are on the same set of vertices $V = \set{v_1, \ldots, v_n}$, but in $G$, $v_1$ and $v_2$ are adjacent, and in $H$, $v_1$ and $v_2$ are not adjacent.
So, the problem of deciding whether the underlying graph is $G$ or $H$ reduces to checking whether $v_1$ and $v_2$ are adjacent.
To show that checking adjacency requires $\Omega(n)$ effective resistance queries, it therefore suffices to show that determining whether the underlying graph is $G$ or $H$ requires $\Omega(n)$ such queries.



One can check that $\res_G(x, y) = \res_H(x, y)$ for all $x, y \in V \setminus \set{v_i, v_j}$, and in particular that $\res_G(v_1, v_2) = \res_H(v_1, v_2) = 1$. 
So, to decide whether the underlying graph is $G$ or $H$, it is necessary to perform a query involving either $v_i$ or $v_j$. However, $i, j$ are unknown, and so any (possibly adaptive) sequence of queries $\res(x, y)$ necessarily including $v_i$ and $v_j$ must be of length at least $(n - 2)/2 = \Omega(n)$.
\end{proof}
}

Put together, \cref{thm:er-sp-clique,thm:er-sp-adjacency} show that shortest path queries and effective resistance queries can each be asymptotically more efficient than the other for some verification tasks, and so neither type of query subsumes the other---they are incomparable.
Understanding their relative power is therefore quite interesting.
We also note that \cref{thm:er-sp-adjacency} shows that testing whether a given pair of vertices is adjacent takes $\Omega(n)$ effective resistance queries, but we do not know if this is tight. In fact, we do not have a better upper bound than the trivial $\binom{n}{2}$ queries needed to recover the whole graph. Finding tight bounds for this problem is therefore an interesting question.
