\section{Property testing}
Next, we present our algorithms for testing properties of a hidden graph using ER queries. A \emph{graph property} is a set of graphs that is closed under isomorphism, such as connectivity, acyclicity or planarity. The algorithms in this section are approximate in the sense that they guarantee correctness only if the input graph either possesses the desired property or is far from having it. This approximate testing model has been extensively studied within the property testing framework~\cite{books/cu/Goldreich17}. Our results are most closely related to the \emph{bounded-degree model}.

In this model the input graph is assumed to have maximum degree $d$, which is bounded by a constant.  The algorithm can query the neighbors of a vertex.  A graph is $\eps$-far from having a property if more than $\eps d n$ edge modifications are required to make it satisfy that property.  The query complexity in this models is often described as a function of $n$ and $\eps$. See \citet[Section 9]{books/cu/Goldreich17} for more on this model.

\fullornot{
    In this section, we provide property testing algorithms for vertex biconnectivity and edge biconnectivity. Additionally, we present a theorem that allows us to adapt existing property testing algorithms to our query model. The number of queries required depends on the \emph{ER density}, a graph parameter that we introduce and define. We demonstrate that bounded-degree, bounded-treewidth graphs have small ER density, making them well-suited for efficient property testing algorithms under our framework.
}{
    In the full version of the paper, we provide property testing algorithms for vertex biconnectivity and edge biconnectivity. Additionally, we present a  meta theorem that allows for the adaptation of existing property testing algorithms to our query model for bounded treewidth graphs. In the short version, we only present this final result.
}

\full{
\subsection{Testing vertex biconnectivity}
First, we use \cref{thm:cut-vertex-verification} to show that the vertex biconnectivity of a graph can be tested with a linear number of queries. 

\begin{theorem}
\label{thm:vertex_bi_conn_test}
Let
$\eps > 0$ and let $G=(V,E)$ be an undirected graph with $n$ vertices.
There exists an algorithm that makes $O(n/\eps)$ effective resistance queries that accepts with probability $1$ if $G$ is vertex biconnected, and rejects with probability at least $ 2/3$ if $G$ is $\eps$-far from being vertex biconnected, 
i.e., one needs to add %
at least $\eps m$ edges to make $G$ vertex biconnected.
\end{theorem} 
\begin{proof}
First, we check that $G$ is connected by making $n-1$ queries. Specifically, we check that an arbitrary vertex $v$ is at finite distance from all other $n - 1$ vertices.  If $G$ is not connected we reject. So, we assume $G$ is connected in the rest of the proof, which in particular implies that $m \geq n-1$.

Our algorithm takes an arbitrary vertex $v\in V$, and samples $s = 4/\eps$ other vertices $u_1, \ldots, u_s$ uniformly at random. 
It rejects if any of the following conditions holds: (1) $v$ is a cut vertex, 
or (2) there exists $1\leq i\leq s$ such that $v$ and $u_i$ belong to different biconnected components.  If none of these two conditions hold, our algorithm accepts.  Both of these conditions can be checked with $O(s\cdot n) = O(n/\eps)$ ER queries using the algorithm of~\Cref{thm:cut-vertex-verification}.

Clearly, if $G$ is biconnected none of the two conditions holds, thus our algorithm accepts. 

Now, suppose $G$ is $\eps$-far from being biconnected.  If $v$ is a cut vertex our algorithm rejects, so suppose $v$ is not a cut vertex, and it blongs to a biconnected component $B$.  

As $G$ is $\eps$-far from being biconnected, it contains at least 
\(
\eps m + 1 \geq \eps(n-1) + 1
\) biconnected components.
Therefore, its largest biconnected component, and in particular $B$, has at most 
\(
n - \eps(n-1) = n - \eps n + \eps \leq n - \eps n + \eps(n/2) = (1-\eps/2)n
\)
vertices. Therefore, 
with probability at least
\(
1- (1 - \eps/2)^{s} \geq 1 - e^{-(\eps/2)\cdot s} \geq 1 - e^{-2} \geq 2/3,
\)
there exists a $1\leq i\leq s$ such that $u_i\notin B$.  Since, $v$ is not a cut vertex, in this case, $v$ and $u_i$ belong to two different biconnected components, hence our algorithm rejects.
\end{proof}
}

\subsection{Local reconstruction via Schur complement}
\label{subsec:local_reconst_schur_comp}
Our result for adapting known property testing algorithms relies on an algorithm that identifies the neighbors of a given vertex with ER queries. 
\full{
    This algorithm exploits properties of the Schur complement stated in \cref{lem:schur_comp_basics}, as well as the algorithm of \cref{lem:full_reconstruciton}
    for full reconstruction of a graph from its pairwise ER distances.
    \begin{corollary}
    \label{lem:schur_comp_full_sonstruct}
    Let $G = (V, E)$ be a weighted graph, and let $U\subseteq V$.  The Schur complement of $G$ on $U$, $G_U$, can be computed with $\binom{|U|}{2}$ effective resistance queries.
    \end{corollary}
}

We denote the \emph{unit ball} around a vertex $v \in V$ with respect to the effective resistance metric by $B_{\res}(v) := \{u\in V \stbar \res(u,v)\leq 1\}$.
\fullornot{
    It follows from \cref{lem:schur_comp_full_sonstruct} that we can find neighbors of a given vertex, provided that there are not too many vertices in its unit ball in the ER metric of the graph.
}{
    One can reconstruct the Schur complement of $G$ on $B_{\res}(v)$ with $O(|B_{\res}(v)|^2)$ queries.  Since the neighbors of $v$ are all in $B_{\res}(v)$, the set of neighbors of $v$ in $G$ and the set of neighbors of $v$ in this Schur complement are identical.  Hence, we have the following corollary.
}
\begin{corollary}
\label{cor:neighborhood_construct}
Let $G=(V,E)$ be an unweighted graph and $u\in V$.
Then, one can find all neighbors of $u$ with $O(n + |B_{\res}(u)|^2)$ ER queries.
\end{corollary}
\full{
\begin{proof}
We start by computing $B_{\res}(u)$, which we can do by querying for the ER distance between $u$ and each of the other $n - 1$ vertices. 
All neighbors of $u$ in $G$ are at ER distance at most one by \cref{lem:er_and_edge_cuts}, and hence contained in $B_{\res}(u)$.  Therefore, by \cref{lem:schur_comp_basics}, $u$'s neighborhood is preserved in the Schur complement $G_{B_{\res}(u)}$.
So, we can find the neighbors of $u$ by constructing the Schur complement, $G_{B_{\res}(u)}$, which can be done with $O(|B_{\res}(u)|^2)$ ER queries by \cref{lem:schur_comp_full_sonstruct}.
\end{proof}

We will use \cref{cor:neighborhood_construct} in two different ways: to test edge biconnectivity for a general graph, and to adapt property testing algorithms for graphs whose unit balls contain a bounded number of vertices. 
}
\full{
\subsection{Testing edge biconnectivity}
Next, we show an algorithm to test edge biconnectivity of a graph. Our algorithm relies on \cref{cor:neighborhood_construct} to find neighbors of vertices, which facilitates breadth-first search %
in a hidden graph.

\begin{theorem}
\label{thm:edge_bi_conn_test}
Let $G=(V,E)$ be an undirected graph with $n$ vertices and let $0<\eps\leq 1$.
There exists an algorithm that makes $O(n/\varepsilon^2+1/\eps^4)$ ER queries and behaves as follows. It accepts with probability $1$ if $G$ is edge biconnected, and rejects with probability $\geq 2/3$ if $G$ is $\eps$-far from being edge biconnected, i.e., if one needs to add more than $\eps\cdot m$ edges to make $G$ edge biconnected.
\end{theorem}
\begin{proof}
As in the proof of \cref{thm:vertex_bi_conn_test}, we first check that $G$ is connected using $n - 1$ ER queries, and if it is not we reject. 
So, we assume $G$ is connected in the rest of the proof, which in particular implies that $m \geq n-1$.

Our algorithm samples a set $S\subseteq V$ of $16/\eps$ vertices uniformly at random.
We call a biconnected component \emph{small} if its size is at most $4/\eps$.
We call a biconnected component \emph{low-degree} if it is incident to at most two cut edges.  A tree can be obtained by contracting all edge-biconnected components of the graph. Since the average degree of any tree is less than two, it follows that at least half of the biconnected components must have a low degree.

In high level, our algorithm checks if any of the sampled vertices in $S$ belong to a biconnected component that is both small and low-degree with a BFS-like algorithm.  It rejects if it finds such a biconnected component and accepts otherwise.
For each $s\in S$, our algorithm simulates a BFS starting at $s$, that can end in three different ways: (1) the BFS finds a cut edge, hence it decides that $G$ is not edge biconnected, (2) the BFS finds more than $4/\eps + 2$ vertices that are either in the same biconnected component of $s$, or are incident to cut edges of this biconnected component, hence it decides that the biconnected component of $s$ is not small or not low-degree.  

To simulate BFS, upon taking a vertex $u$ from the queue, we query all effective resistances from $u$, and compute the unit ER ball centered at $u$, $B_{\res}(u)$. 
Note, every vertex that is strictly inside this ball is in the same edge biconnected component of $u$.  Also, every vertex on the surface of this ball, is either in the same edge biconnected component, or it is adjacent to $u$ via a cut edge (\cref{lem:er_and_edge_cuts}).  Therefore, if $|B_{\res}(u)|>4/\eps + 2$, the edge biconnected component of $u$ (and also the sampled point that is the root of this BFS call) cannot be both small and low-degree.  In this case, we continue to the next sample point. 
Otherwise, if $|B_{\res}(u)|\leq 4/\eps + 2$, we compute the neighbors of $u$ with another $|B_{\res}(u)|^2 = O(1/\eps^2)$ queries, using the algorithm of \cref{cor:neighborhood_construct}, and check if any of the incident edges of $u$ is a cut edge, by checking if its effective resistance is exactly one, using \cref{lem:er_and_edge_cuts}. 
If we find a cut edge then we reject. Otherwise, we continue the BFS until we visit at least $4/\eps$ vertices. In that case, we continue to the next sampled point, as the biconnected component of the current sampled point is larger than $4/\eps$, and hence not small.

For each vertex $v$ that the BFS visits, it spends $O(n)$ queries to compute $B_{\res}(v)$, and $O(1/\eps^2)$ queries to compute the neighbors of $v$.  For each sampled vertex in $S$, BFS visits $O(1/\eps)$ vertices, the size limit for low-degree small biconnected components.  Therefore, the total number of queries is
\(
|S|\cdot O(1/\varepsilon) \cdot O(n+1/\eps^2) = O(n/\varepsilon^2+1/\eps^4).
\)

If $G$ is edge biconnected, our algorithm accepts with probability one, as it will not find a cut edge.

If $G$ is $\eps$-far from edge biconnectivity, then it contains at least 
\(
\eps m + 1 \geq \eps(n-1)
\) edge biconnected components.  At least half of them, i.e.~$\eps(n-1)/2$ of them, are low-degree.
On the other hand, $G$ (which has $n$ vertices) can contain at most $\eps n/4$ edge biconnected components of size at least $4/\eps$. Thus, assuming $\eps \leq 1$ and $n\geq 8$, there are at least %
\(
\eps(n-1)/2 - \eps n/4 \geq \eps n/8
\)
small edge biconnected components, i.e., of size at most $4/\eps$, that are incident to at most two cut edges.
In particular, these low-degree small biconnected components have at least $\eps n/8$ vertices in total.  Hence, our sample $S$ (of size $16/\eps$) contains at least one vertex from a low-degree small biconnected component with probability at least 
\(
1-(1-\eps/8)^{-16/\eps} \geq 1-e^{-2}\geq 2/3.
\)
Therefore, our algorithm rejects with at least this probability.
\end{proof}
}
\subsection{The ER density and property testing with ER queries} \cref{cor:neighborhood_construct} implies the following meta theorem, which allows us to adapt any property testing algorithm within the
bounded-degree model to a property testing algorithm with ER queries.  
The query complexity of our algorithm depends on the ER density $\rho_{\res}(G)$ of the graph $G$, which we define as the maximum size of any ER unit ball, i.e., $\rho_{\res}(G) := \max_{v\in V}{|B_{\res}(v)|}$.

\begin{theorem}
\label{thm:adj_list_pt_to_er_pt}
    Let $G=(V,E)$ be a graph with ER density $\rho = \rho_{\res}(G)$.  Let $A$ be any property testing algorithm for problem $P$ in the bounded-degree model with $f(\varepsilon, n)$ running time.  Then, there is a property testing algorithm for problem $P$ that uses $O(f(\varepsilon, n)\cdot (n+\rho^2))$ effective resistance queries.
\end{theorem}

\begin{remark}
We remark that the linear dependency on $n$ in 
\full{\cref{thm:edge_bi_conn_test} and} \cref{thm:adj_list_pt_to_er_pt} can be eliminated if we have an oracle that returns the vertices in the order of their distance to a vertex $v$, allowing us to stop querying at any point.   Then, to discover the vertices in a unit ball around $v$, $B_{\res}(v)$, we only need to spend $|B_{\res}(v)|$ queries as opposed to $n-1$.
\end{remark}

Property testing in the bounded-degree model is a well-studied subject with many fundamental results.  To name a few, there are $f(1/\eps)$ time testers to decide if a graph (1) contains a fixed subgraph, (2) is regular, (3) is $k$-connected, or (4) belongs to a minor closed family (e.g.,~planar graphs). (See~\citet{books/cu/Goldreich17} for a more extensive list of results.)

\Cref{thm:adj_list_pt_to_er_pt} can be used to adapt all these results to algorithms with ER queries. However, this comes with an additional cost of a factor of $(n + \rho^2)$. There exist graphs for which $\rho^2$ is large, making the theorem less useful—for example, in the case of expander graphs.

However, we show that $\rho$ is bounded for bounded-degree bounded treewidth graphs. As a result, all the aforementioned results from the bounded-degree model can be adapted to bounded treewidth graphs with an additional factor of $n$ in the number of queries. In most cases, this leads to a linear number of ER queries, as opposed to the quadratic number required when reconstructing the entire graph.
\paragraph{ER density of bounded treewidth graphs.}
We use the following two fundamental results of \citet{journals/dmtcs/Bodlaender99} on domino tree decompostion and \citet{journals/mpcps/Nash-Williams59} on a lower bound for ER distance of two vertices towards showing that bounded-degree bounded treewidth graphs have boudned ER density.
\begin{lemma}[{\citet[Theorem 3.1]{journals/dmtcs/Bodlaender99}}]
\label{lem:good_tree_decomp}
    Let $G=(V,E)$ be a graph with maximim degree $d_G$ and treewidth $w_G$.  Then, $G$ has a tree decomposition of width $O(w_G d_G^2)$, in which each vertex is in at most two bags.%
    \footnote{Such a decomposition is called a \emph{domino decomposition}, and the smallest width of a domino decomposition is called \emph{domino treewidth}.} Moreover, the degree of each bag in the tree decomposition is $O(w_G d_G^2)$.
\end{lemma}
    
The last part of the statement is not explicit in \citet{journals/dmtcs/Bodlaender99}.  However, it is easy to conclude by observing that each neighbor of a bag $B$ has at least one vertex in common with $B$, assuming the underlying graph is connected.  Therefore, since each vertex appears in at most two bags, the number of neighbors of a bag cannot be larger than its size.
\begin{lemma}[\citet{journals/mpcps/Nash-Williams59}]
\label{lem:nash_williams}
    Let $G=(V, E)$ be a graph and let $s, t\in V$.
    Let $S_1, S_2, \ldots, S_k$ be such that $s\in S_i$ and $t\not\in S_i$ for all $i\in[k]$, and such that the edge sets $E(S_i, V\backslash S_i)$ are pairwise disjoint. Then,\full{
    \[
    \res(s,t) \geq \sum_{i=1}^k{\frac{1}{|E(S_i, V\backslash S_i)|}} \ \text{.}
    \]}{ \(
    \res(s,t) \geq \sum_{i=1}^k|E(S_i, V\backslash S_i)|^{-1}\).}
\end{lemma}

\begin{lemma}
\label{lem:er_vs_tree_decomp_distance}
    Let $G=(V,E)$ be a graph with maximum degree $d_G$, and let $({\cal B}, T)$ be a tree decomposition of $G$ with width $w_T$ and maximum degree $d_T$.
    Also, suppose that each vertex of $G$ appears in at most $b_T$ bags of $T$.
    Let $s,t\in V$, let $B_s, B_t\in {\cal B}$ be any two bags that contain $s$ and $t$, respectively, and let $r_T(s,t)$ be the distance of these bags in $T$. 
    \[
    r_T(s,t) \leq {4b_T\cdot d_G\cdot w_T}\cdot \res(s,t)  \ \text{.}
    \]
\end{lemma}
\begin{proof}
    If $r_T(s,t) < 4b_T$, the statement of the lemma holds. Thus, assume $r_T(s,t) \geq 4b_T$.
    
    Let $p = r_T(s,t)$, and let $B_s = B_1, \ldots, B_p = B_t$ be the unique $B_s, B_t$ path in $T$.
     We know that each vertex of $G$ is in at most $b_T$ bags.  Moreover, we know that these bags form a connected induced tree in the tree decomposition.
     Therefore, for each 
     $i\in [p-b_T]$, $B_i$ and $B_{i+b_T+1}$ are disjoint.  
     We conclude that there is a subsequence $B'_1,\ldots, B'_q$ of $B_1, \ldots, B_p$ such that (1) $q \geq \lfloor p/b_T\rfloor  - 2$, (2) $s,t\not\in B'_i$ for $i\in [q]$, and (3) for each $1\leq i< j\leq q$, $B'_i$ and $B'_j$ are disjoint (we need $-2$ in condition (1) because of condition (2)).  Note, $q\geq 1$ because $p\geq 4b_T$. 
     
     Note, for each $i\in [q]$, $B'_i$ is a vertex cut that separates $s$ and $t$ in $G$. Let $S^i_1, \ldots, S^i_{r_i}$ be the connected components of %
     $V \setminus B_i'$ with $s\in S^i_1$ and $t\in S^i_{r_i}$.  Then, let
     $S'_i := S^i_1\cup \cdots \cup S^i_{r_i -1}\cup B'_i$, and observe that $s\in S'_i$ and $t\not\in S'_i$.

     Moreover, for each $i\in [q]$, each edge in %
     $E(S'_i, V\backslash S'_i)$%
     has one endpoint in $B'_i$ (note, in particular, $V \setminus S_i' = S_{r_i}^i$).  Since the bags $B'_i$ are disjoint, we conclude that the edge sets $E(S'_i, V\backslash S'_i)$ for $i\in [q]$ are disjoint.  We also conclude that for every $i \in [q]$,
     \[
        |E(S'_i, V\backslash S'_i)| \leq d_G\cdot |B'_i| \leq d_G\cdot w_T.
    \]
     It follows by \cref{lem:nash_williams} that %
     \begin{align*}
     \res(s,t) \geq& \frac{q}{d_G\cdot w_T} \geq \frac{\lfloor p/b_T\rfloor  - 2}{d_G\cdot w_T}
     \geq \frac{p/b_T  - 3}{d_G\cdot w_T} \\
     \geq& \frac{p/b_T  - (3/4)\cdot (p/b_T)}{d_G\cdot w_T} 
      = \frac{(1/4)\cdot p}{b_T\cdot d_G\cdot w_T} =  \frac{(1/4)\cdot r_T(s, t)}{b_T\cdot d_G\cdot w_T},
     \end{align*}
     where the last inequality holds as $p\geq 4b_T \Rightarrow 3 \leq (3/4)\cdot(p/b_T)$.
\end{proof}
\begin{theorem}
\label{thm:er_density_tw}
    Let $G=(V,E)$ be a graph with maximum degree $d_G$ and treewidth $w_G$.
    Then the ER density of $G$ is $(w_Gd_G)^{O(w_G d_G^3)}$.  In particular, a graph with constant maximum degree and constant treewidth has constant ER density.
\end{theorem}
\begin{proof}
    By \cref{lem:good_tree_decomp} there exists a tree decomposition $T$ of $G$ with maximum degree $d_T=O(w_Gd_G^2)$ and width $w_T=O(w_Gd_G^2)$ in which each vertex appears in at most $b_T = 2$ bags.

    Let $s\in V$, and let $B_s$ b ethe bag in $T$ that contains $s$.  By \cref{lem:er_vs_tree_decomp_distance}, vertices with ER distance at most one from $s$ (i.e., vertices in $B_{\res}(s)$), are in bags of distance at most $\alpha := \max(4b_T\cdot d_G\cdot w_T, 4 b_T) = 4b_T\cdot d_G\cdot w_T$ from $B_s$ in $T$.

    There are at most %
    $
    \sum_{i=0}^{\alpha} d_T^i = {(d_T^{\alpha+1}-1)}/{(d_T - 1)} \leq d_T^{\alpha+1}
    $ 
    such bags, assuming $d_T \geq 2$, that collectively contain at most
    \[
    w_T\cdot d_T^\alpha = 
    w_T\cdot d_T^{4b_T\cdot d_G\cdot w_T + 1} = 
    w_Gd_G^2\cdot (w_Gd_G^2)^{O(d^3_G\cdot w_G)} = 
    (w_Gd_G)^{O(d^3_G\cdot w_G)}
    \]
    vertices.
\end{proof}
Therefore, the ER density of a graph $G$ is a constant if it has constant maximum degree and constant treewidth.  In that case, \cref{thm:adj_list_pt_to_er_pt} implies a property testing algorithm with a linear number of ER queries, if a constant-time property testing algorithm is known in the bounded-degree model.

