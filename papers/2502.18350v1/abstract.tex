    

\begin{abstract}
The goal of \emph{graph inference} is to design algorithms for learning properties of a hidden graph using queries to an oracle that returns information about the graph. Graph reconstruction, verification, and property testing are all special cases of graph inference.

In this work, we study graph inference using an oracle that returns the \emph{effective resistance} (ER) distance between a given pair of vertices. 
Effective resistance is a natural notion of distance that arises from viewing graphs as electrical circuits, and  has many applications. 
However, it has received little attention from a graph inference perspective. Indeed, although it is known that an $n$-vertex graph can be uniquely reconstructed by making all $\binom{n}{2} = \Theta(n^2)$ possible ER queries, very little else is known.
We address this and show a number of fundamental results in this model, including:
\begin{enumerate}
\item $O(n)$-query algorithms for testing whether a graph is a tree; deciding whether two graphs are equal assuming one is a subgraph of the other; and testing whether a given vertex (or edge) is a cut vertex (or cut edge).
\item Property testing algorithms, including for testing whether a graph is vertex-biconnected and whether it is edge-biconnected. We also give a reduction that shows how to adapt property testing results from the well-studied bounded-degree model to our model with ER queries. This yields ER-query-based algorithms for testing $k$-connectivity, bipartiteness, planarity, and containment of a fixed subgraph.
\item Graph reconstruction algorithms, including an algorithm for reconstructing a graph from a low-width tree decomposition; a $\Theta(k^2)$-query, polynomial-time algorithm for recovering the entire adjacency matrix $A$ of the hidden graph, given $A$ with $k$ of its entries deleted; and a $k$-query, exponential-time algorithm for the same task.
\end{enumerate}
We additionally compare the relative power of ER queries and shortest path queries, which are closely related and better studied. Interestingly, we show that the two query models are incomparable in power.
\end{abstract}
