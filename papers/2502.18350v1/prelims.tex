\section{Preliminaries}
\label{sec:prelims}

\subsection{Graph Laplacian and Effective Resistance}
Let $G=(V, E, w)$ be an undirected 
graph with edge weights
$w:E\rightarrow \R^{\geq 0}$. We denote the number of vertices as $n:=|V|$ and the number of edges as $m:=|E|$. The \emph{graph Laplacian} is the $n \times n$ matrix defined $L := D - A$, where the \emph{weighted degree matrix} is the diagonal matrix with entries $D_{u,u} = \sum_{(u,v)\in E}{w(u,v)}$ and the \emph{weighted adjacency matrix} is the matrix with entries $A_{u,v} = w(u,v)$ if $(u,v)\in E$ and $A_{u,v}=0$ otherwise.
Alternatively, the graph Laplacian is $L = \partial W\partial^{T}$, where $W$ is the $m \times m$ diagonal weight matrix with entries $W_{e,e} = w(e)$ and $\partial$ is the $n \times m$ \emph{signed incidence matrix} (corresponding to the \emph{boundary operator}), which has  entries\footnote{The choice of which of the first two cases is equal to $1$ and which equal to $-1$ is arbitrary. The graph Laplacian will be the same for either choice, and other quantities involving the signed incidence matrix will only differ by a sign flip.}
\[
\partial_{v, e} := \begin{cases}
1 & \text{if $e = (v, w)$ for some $w$ ,} \\
-1 & \text{if $e = (u, v)$ for some $u$ ,} \\
0 & \text{otherwise .} \\
\end{cases}
\]
Finally, a third way of writing the graph Laplacian is as the sum $L = \sum_{(u,v)\in E} w(u,v)L_{uv}$, where $L_{uv} = (\vec{1}_{u} - \vec{1}_{v})(\vec{1}_{u} - \vec{1}_{v})^{T}$ is the \emph{edge Laplacian}.

    Let $M \in \mathbb{R}^{m \times n}$ be a matrix.
    The \emph{psuedoinverse} of $M$, denoted $M^+$, is any matrix such that the following conditions hold:
    (1) $AA^+A=A$,
    (2) $A^+AA^+=A^+$,
    (3) $(AA^+)^T=AA^+$, and 
    (4) $(A^+A)^T=A^+A$.
    

The \emph{effective resistance} between a pair of nodes $u$ and $v$ is defined as
\begin{equation} \label{eq:eff_res_def}
\res(u,v) := (\vec{1}_u - \vec{1}_v)^{T} L^{+} (\vec{1}_u-\vec{1}_v) \ \text{,}
\end{equation}
where $L^{+}$ is the pseudoinverse of $L$ and $\vec{1}_u,\vec{1}_v \in \R^n$ are the indicator vectors of $u$ and $v$. One can show that $\res$ is a metric on the vertices in the graph.
Interpreting graphs as electrical circuits, the effective resistance measures the resistance between $u$ and $v$ in the electrical network where each edge $e$ has conductance equal to its weight, $w(e)$, giving each edge a resistance of $1/w(e)$. %
The \emph{$uv$-potentials} are $\Vec{p}_{u,v} := L^{+} (\vec{1}_u-\vec{1}_v)$, as these are the voltage potentials on the vertices resulting in a unit flow of current from $u$ to $v$ in the graph.

Notably, the kernel of any graph Laplacian $L$ contains the span of the all-ones vector (i.e., $\lspan(\vec{1}) \subseteq \ker(L)$), since the weighted degree of a vertex is equal to the sum of the weights of its incident edges. Moreover, if the graph is connected then $\ker(L) = \lspan(\vec{1})$.

To see the matrix form of the relationship between $L^+$ and effective resistance, let $J$ be the all-ones matrix. The matrix $I - \frac{1}{n}J$ projects any vector onto the orthogonal complement of $\lspan(\vec{1})$.
Its conjugation with $R$ be the matrix of all effective resistances with $R_{i,j}=\res(i,j)$, gives $L^+$ up to a constant factor:
\begin{equation}
    \label{eq:eff-res-matrix-form}
     -\frac{1}{2}\left(I-\frac{1}{n}J\right)R \left(I-\frac{1}{n}J\right)=L^+
\end{equation}

This formula implies the following lemma~\citep{journals/tcs/WittmannSBT09, Spielman2012TreesRecNotes, Hoskins2018Inferring}.
\begin{lemma}
\label{lem:full_reconstruciton}
    Any weighted graph with $n$ vertices can be reconstructed with $\binom{n}{2}$ effective resistance queries.
\end{lemma}
\full{
Due to the $O(n^\omega)$ time algorithm for obtaining a matrix's rank-normal form given by~\citep{KellerGehrig1985}, the psuedoinverse of $L$, and hence the matrix of all effective resistances, can be obtained in $O(n^\omega)$ runtime where $\omega<2.371339$~\citep{ADWXX2024} is the matrix multiplication exponent.
}


\paragraph{The regularized graph Laplacian}
As mentioned above, the graph Laplacian is never full-rank regardless of the graph; the all-ones vector $\vec{1}$ is always in the kernel. The \emph{regularized graph Laplacian} is a slightly modified version of the graph Laplacian defined $\widetilde{L} = L +  \frac{1}{n} \cdot J$. For a connected graph, $\widetilde{L}$ is full-rank. 

\begin{lemma}[{\citet[Equation 9]{ghosh2008minimizing}}]
    Let $G = (V, E)$ be a connected graph with Laplacian $L$. The following two facts are true of the regularized graph Laplacian: (1) $\tilde{L}^{-1} = L^{+} + \frac{1}{n} J$,
        and (2), for all $u, v \in V$, $\res(u,v) = (\vec{1}_u - \vec{1}_v)^{T} \tilde{L}^{-1} (\vec{1}_u-\vec{1}_v)$.
\end{lemma}

Alternatively, we can characterize the effective resistance in terms of the amount of the amount of current on the edges. A \emph{unit uv-flow} is
a flow $f:E\to\R$ such that $\partial f = \vec{1}_u - \vec{1}_v$. The  \emph{energy} of a flow $f:E\rightarrow \R$ is defined to be %
$\mathcal{E}(f):=\sum_{e\in E}{f(e)^2/w(e)}$. 

\begin{lemma}[Thompson's Principle~\citep{thomson1867treatise}]
\label{lem:thompson_dirichlet}
Let $\mathcal{F}_{uv}$ be the set of unit $uv$-flows. The effective resistance between two vertices $u,v\in V$ is equal to the minimum energy of any unit $uv$-flow, $\res(u,v)= \min_{f \in \mathcal{F}_{uv}} \mathcal{E}(f)$.
\end{lemma}
The minimum energy $uv$-flow $f_{u,v}:= \argmin_{f \in \mathcal{F}_{uv}} \mathcal{E}(f)$ is called the \emph{electrical $uv$-flow}.
Thompson's principle immediately implies \emph{Rayleigh's monotonicity law}: if the resistance values increase the effective resistance between any pair of vertices can only increase. 

\begin{lemma}[Rayleigh's Monotonicity Law~\citep{rayleigh1877theory}]
    Let $G=(V, E_G,w_G)$ and $H=(V, E_H, w_H)$ be weighted graphs on the same set of vertices $V$. If $E_G\subseteq E_H$ and $w_G(e) \leq w_H(e)$ for all $e\in E_G$, then $\res_G(u,v) \geq \res_H(u,v)$ for all $u,v\in V$. 
\end{lemma}
We find the following basic properties of effective resistance useful in this paper.
\begin{lemma}
\label{lem:er_and_edge_cuts}
Let $G=(V,E)$. Then for all $u, v \in V$,
\begin{enumerate} [(1)]
    \item \label{lem:er_of_an_edge}
    If $(u,v)\in E$, then $\res(u,v) \leq 1$.  Moreover, $\res(u,v) = 1$ if and only if $(u,v)$ is a cut edge.
    \item \label{lem:er_of_path}
    If there is exactly one path $\gamma$ between two vertices $u,v\in V$, then the $\res(u,v)$ is the length of $\gamma$.
    \item The ER distance between any pair of vertices in two different edge biconnected components is at at least one.
\end{enumerate}
\end{lemma}



\paragraph{Schur complement.} The \emph{Schur complement} of the Laplacian $L$ of a matrix on a vertex subset $U$, denoted $L_U$, is defined as follows.\footnote{The Schur complement is usually defined as $L_U := L(U, U) - L(U,\overline{U})L(\overline{U}, \overline{U})^{-1}L(\overline{U}, U)$ (e.g., in~\citep[Lemma 11.5.3]{spielman2019sagt}), using the inverse $L(\overline{U}, \overline{U})^{-1}$ instead of the pseudoinverse. We use the pseudoinverse as it is more general while preserving the relevant properties of the Schur complement. To see why it preserves these properties, note that the submatrix $L(\overline{U}, \overline{U})$ is invertible iff there is an edge between $U$ and $\overline{U}$. However, if there is no edge between $U$ and $\overline{U}$, then $L(U,\overline{U})=0$ and $L_U = L(U,U)$. Likewise, if there are no edge between $U$ and $\overline{U}$, then $L_U$ is a graph Laplacian of subgraph of $G$ induced by $U$, so all the properties of the Schur complement mentioned in this section hold.}

\begin{equation}
\label{eqn:schur_comp}
L_U := L(U, U) - L(U,\overline{U})L(\overline{U}, \overline{U})^+L(\overline{U}, U) \ \text{,}
\end{equation}
where $L(X,Y)$ is the submatrix of $L$ indexed by rows in $X$ and columns in $Y$ for $X, Y \subseteq V$. 
Note that for a connected graph, $L(\overline{U},\overline{U})$ is always full rank.
It is known that the Schur complement of a Laplacian of a graph $G$ is the Laplacian of a (possibly weighted) graph with vertex set $U$, which we refer to as $G_U$.  The following lemma summarizes some properties of the Schur complement that we find helpful in this paper~\citep{spielman2019sagt}.
\begin{lemma}
\label{lem:schur_comp_basics}
Let $G=(V,E)$ be a weighted graph, let $U\subseteq V$, and let $G_U$ be the Schur complement of $G$ on $U$.  Then the following properties hold.
\begin{enumerate}[(1)]
    \item For any $u,v\in U$, their effective resistance in $G$ is equal to their effective resistance in $G_U$.

    \item Let $u\in U$ be such that $w_G(u,v) = 0$ for all $v \in \overline{U}$. Then, for any $u'\in U$, $w_G(u,u') = w_{G_U}(u,u')$, i.e., the neighborhood of $u$ is identical in $G$ and $G_U$.
\end{enumerate}
\end{lemma}

The Schur complement matrix can be obtained by using Gaussian elimination on the graph Laplacian. Likewise, for each step of Gaussian elimination, there is a corresponding operation on the graph, with the operations corresponding to entire process of Gaussian elimination resulting in the graph $G_U$. For a vertex $v\in\overline{U}$, performing Gaussian elimination on the $v$th row and column corresponds to removing the vertex $v$ and, for each pair of neighbors $(a,b) \in N(v) \times N(v)$, replacing its weight with $w(a, b)' := \frac{w(a,b)}{d(v)}$, the ratio between the edge weight and the degree of $v$.

 
\paragraph{Tree decomposition.} Let $G=(V,E)$ be a graph. A \emph{tree decomposition} is a pair $({\cal B}, T)$, where ${\cal B}$ is a set of subsets of $V$ called \emph{bags} and $T$ is a tree with vertex set ${\cal B}$ such that
\begin{enumerate}[(1)]
    \item $\bigcup_{B\in{\cal B}} B = V$.
    \item For every $\{u,v\}\in E$, there exists $B\in{\cal B}$ such that $u,v\in B$.
    \item Let $B, B''\in {\cal B}$, and suppose $B'\in{\cal B}$ is a bag on the unique $B$ to $B''$ path in $T$.  Then, $B\cap B''\subseteq B'$.
\end{enumerate}

We show that given a tree decomposition of a graph with bounded treewidth, it is possible to fully recover the graph using $o(n^2)$ effective resistance queries.
\paragraph{Graph cuts and connectedness.}
Let $G=(V,E)$ be a graph.
A vertex $v \in V$ is a \emph{cut vertex} if the subgraph induced by $V \setminus \set{v}$ has more connected components than $G$.
An edge $(u,v) \in E$ is a \emph{cut edge} if the subgraph given by $(V, E \setminus \{(u,v)\})$ has more connected components than $G$.
A graph is \emph{vertex biconnected} if it contains no cut vertices.
A graph is \emph{edge biconnected} if it contains no cut edges.

    
