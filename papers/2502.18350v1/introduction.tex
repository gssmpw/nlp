\section{Introduction}
\label{sec:intro}

Let $G = (V,E)$ be a hidden graph with $n$ vertices, and let $d: V \times V \rightarrow \R^{\geq 0}$ be a metric on $G$.\fullornot{}{\footnote{Due to space constraints, this version of the paper omits a number of results and proofs. We strongly encourage the reader to view the full version of our paper, which we have submitted as a supplementary file. The full version contains all discussion, results, and proofs.}}
We study \emph{graph inference} problems, where the goal is to reconstruct $G$ or deduce properties of $G$ using a limited number of queries to $d$. In this work, we focus on the case where $d$ is the \emph{effective resistance} (ER) metric on $G$.

Graph inference using the ER metric and its close cousin, the \emph{shortest path} (SP) metric, arises naturally in many applications. Inferring a graph from SP queries has been widely studied in the context of network discovery~\citep{journals/jsac/BeerlixovaEEHHMR06}, where it is used to map unknown regions of the internet. It is also applied in the reconstruction of evolutionary trees~\citep{hein1989optimal,conf/soda/KingZZ03,journals/ipl/ReyzinS07}, where understanding evolutionary relationships based on pairwise distances is crucial. On the other hand, ER-based inference is particularly relevant in the study of social network privacy~\citep{book/affarwal2014social_network_data,Hoskins2018Inferring}, where the goal is to assess the potential of revealing hidden network features through random-walk-based queries.

Graph inference via SP queries has also been explored extensively from a theoretical perspective, leading to notable algorithmic and lower bound results. For instance, bounded-degree graphs can be reconstructed using $\widetilde{O}(n^{3/2})$ queries~\citep{journals/talg/KannanMZ18}, and bounded-degree almost-chordal graphs can be reconstructed using $O(n \log n)$ queries~\citep{journals/arxiv/bestide2023reconstnolongcycle}. On the other hand, $\Omega(n^2)$ queries are needed to even reconstruct trees if they are not bounded degree~\citep{alt/ReyzinS07}. However, the study of graph inference via ER queries is still in its early stages. Beyond the $\Theta(n^2)$-query algorithm that guarantees full reconstruction given all pairwise ER distances (see~\citep{journals/tcs/WittmannSBT09,Spielman2012TreesRecNotes,Hoskins2018Inferring}), little is known that comes with theoretical guarantees. Moreover, the quadratic query complexity needed to reconstruct a whole graph is impractical for large graphs and when the ability to make ER measurements is limited. The focus of this work is to address the corresponding natural question:

\begin{quote}
What properties of a graph can be inferred from a subquadratic number of ER queries?
\end{quote}
We interpret this question quite generally, and study graph reconstruction, property testing, and more using ER queries. 

A closely related question asks what the comparative power of SP and ER queries is in the context of machine learning on graphs. In the context of graph neural networks and graph transformers, distance measures such as SP and ER are often used as inputs to the network in the form of positional encodings~\citep{zhang2023rethinking,velingker2024affinity}. A primary challenge in graph neural networks is to identify which positional encodings offer the highest expressive power, enabling the model to detect the most properties of a graph. For example, graph transformers using ER as a positional encoding can determine which vertices are cut vertices, while graph transfomers using SP cannot~\citep{zhang2023rethinking}. While there have been several works aiming to understand and compare the relative expressive power of SP and ER encodings~\citep{zhang2023rethinking, black2024comparing, velingker2024affinity}, the capabilities of SP or ER remain only partially understood. A deeper understanding of their potential could lead to better feature selection strategies and improved learning efficiency.


\subsection{Our Contribution}
In this paper, we study a number of fundamental graph inference problems using ER queries.
To begin with, we demonstrate that with $O(n)$ ER queries, the following tasks can be performed:

\begin{enumerate}[(1)]
    \item Determining whether a graph is a tree \fullornot{(\Cref{thm:tree-verification})}{(see full version)}.
    \item Determining whether two graphs are identical, assuming one is a subgraph of the other. In other words, we can detect if some edges have been added to the graph or if some edges have been deleted, but not both~(\Cref{thm:proper-subgraph-verification}).
    \item Deciding whether a given vertex is a cut vertex~(\Cref{thm:cut-vertex-verification}).
    \item Deciding whether a pair of vertices are adjacent with a cut edge\fullornot{~(\Cref{thm:cut-edge-verification})}{~(\cref{lem:cutedge-iff-res-is-plusorminus-one})}.
\end{enumerate}

Furthermore, we explore (approximate) \emph{property testing} of graphs, where our algorithms can distinguish between the case where the graph satisfies a given property and the case where it is $\eps$-far from satisfying the property, meaning that at least $\eps m$ edge modifications are required to satisfy the property. In this setting, we present the following results:

\begin{enumerate}[(1)]
    \item Determining whether a graph is vertex-biconnected or $\eps$-far from being vertex-biconnected with $O(n/\eps)$ ER queries\fullornot{~(\Cref{thm:vertex_bi_conn_test})}{ (see full version)}.
    \item Determining whether a graph is edge-biconnected or $\eps$-far from being edge-biconnected with $O(n/\eps^2 + 1/\eps^4)$ ER queries\fullornot{~(\Cref{thm:edge_bi_conn_test})}{ (see section 3.2)}.
    \item Showing that for any property that can be tested in $f(\eps, n)$ time in the well-studied bounded-degree model, there is an algorithm to test it for bounded-treewidth graph with $f(\eps, n) \cdot n$ ER queries (\Cref{thm:adj_list_pt_to_er_pt}). 
    As a result, we obtain property testers that require a subquadratic number of ER queries for bounded-treewidth graphs, addressing a range of problems such as the inclusion of a fixed subgraph, $k$-connectivity, bipartiteness, the presence of long cycles, and any minor-closed property (e.g., planarity)~\cite{books/cu/Goldreich17}.

\end{enumerate}

We remark that the running time of our property testing algorithms for the latter two results can be reduced by a factor of $n$, making them dependent only on $\eps$, given a stronger ER query model. In this model, queries return vertices in the order of their distance from a vertex $v$, allowing the algorithm to request only the closest $k$ vertices to $v$. This stronger query model is natural, as in most applications, closer vertices are often accessible with less effort, for example if close vertices are sampled by a random walk.
\\ \\
In addition to our property inference algorithms, we analyze the relative power of ER and SP query models. Interestingly, we show that the two query models are incomparable, i.e., neither model is strictly stronger than the other for all tasks. Specifically, we demonstrate the following results:

\begin{enumerate}[(1)]
    \item Determining whether a graph is a clique can be achieved with $O(n)$ ER queries, whereas it requires $\Omega(n^2)$ SP queries (\Cref{thm:er-sp-clique}).
    \item Checking whether two given vertices are adjacent can be achieved with a single SP query, whereas it requires $\Omega(n)$ ER queries (\Cref{thm:er-sp-adjacency}).
\end{enumerate}

Finally, we study the problem of graph reconstruction using ER queries. Specifically, we give an algorithm for each of the following tasks:
\begin{enumerate}[(1)]
\item Given a width-$k$ tree decomposition of a graph $G$, reconstructs $G$ using $O(k^2 n)$ ER queries \fullornot{(\cref{thm:reconstruction-from-tree-decomp})}{ (see full version)}.
\item Given an adjacency matrix $A$ of a graph $G$ with $k$ missing entries, recovers $A$ using $O(k^2)$ queries and runs polynomial time \fullornot{(\cref{thm:k-unknowns-quadratic_queries})}{ (see full version)}.
\item Given an adjacency matrix $A$ of a graph $G$ with $k$ missing entries, recovers $A$ using $k$ queries and runs in exponential time (\cref{thm:k-unknowns-exponential-completion}).
\end{enumerate}

Our reconstruction algorithm in \cref{thm:k-unknowns-exponential-completion} uses techniques from convex analysis and relies on a highly nontrivial structural property of ER distances, which allows us to reduce the number of queries. It remains an open question whether a polynomial-time algorithm can be achieved with a subquadratic number of queries, \fullornot{i.e., whether it is possible to combine the strengths of \cref{thm:k-unknowns-quadratic_queries,thm:k-unknowns-exponential-completion}}{that is, whether quadratic Schur complement reconstruction and \cref{thm:k-unknowns-exponential-completion} can be combined to give a low query complexity and runtime complexity}.


\subsection{Related work}
\paragraph{Graph inference with different query models.}
Graph inference has been studied under various models, with {edge detection} and {edge counting} being two prominent approaches motivated by applications in biology. In these models, queries allow one to check whether an induced subgraph contains any edges or to determine the number of edges in the subgraph~\citep{sicomp/AlonBKRS04, journals/jcss/AngluinC08, journals/siamdm/AlonA05, conf/wg/BouvelGK05}.~\citet{alt/ReyzinS07} provides an extensive survey of results within these models, as well as the shortest path (SP) query model. Notably, they show that reconstructing a hidden tree requires at least $\Omega(n^2)$ SP queries, establishing that the bounded-degree assumption is necessary for obtaining nontrivial results using SP queries.

\citet{journals/talg/KannanMZ18}
demonstrated that bounded-degree graphs can be fully reconstructed using $\widetilde{O}(n^{3/2})$ SP queries. They further showed that bounded-treewidth chordal graphs and outerplanar graphs can be reconstructed with $\widetilde{O}(n\log n)$ SP queries. This result has been recently generalized to bounded-treewidth graphs without long cycles by \citet{conf/iwpec/Bestide24}, with additional related work by \citet*{journals/tcs/RongLYW21}.

In contrast to the SP model, significantly fewer theoretical results are known for the {effective resistance (ER) model}. It has been established that a hidden graph can be fully reconstructed if the ER distances between all pairs of its vertices are known~\citep{journals/tcs/WittmannSBT09,Spielman2012TreesRecNotes,Hoskins2018Inferring}. On the other hand, since ER and SP distances are equivalent for trees, it follows that reconstructing a general tree requires $\Omega(n^2)$ ER queries, and hence the known $O(n^2)$ query algorithm is tight for general graphs, as is the case for the SP model. However, in contrast to the SP model, subquadratic algorithms are not known for any family of graphs beyond bounded-degree trees.

A continuous variant of the graph reconstruction problem, known as \emph{Calder\'{o}n's inverse problem}, is to recover the conductivity of an object from measurements of current and potential on its surface.  Calder\'{o}n's inverse problem has been studied extensively by mathematicians~\citep{Uhlmann2012} and found important applications in Electrical Impedance Tomography (EIT) in medical imaging~\citep{uhlmann2009electrical} and Electrical Resistivity Tomography (ERT) in geophysics~\citep{wikipediaERT}.

\paragraph{Property testing.}
We study the gap version of checking properties of graphs using ER queries. In this setting, the goal is to distinguish between graphs that possess a certain property and those that are ``far'' from having that property, using only a few ER queries. This part of our work is related to property testing of graphs in the bounded-degree model. Many properties have been studied in this model, including $k$-connectivity, bipartiteness, subgraph exclusion, minor exclusion, and planarity. We refer the reader to~\citet{books/cu/Goldreich17} for an extensive exposition of known results in this area. 
For contrast, we study property testing algorithms that rely only on resistance distance queries rather than adjacency queries. 

More broadly, our work is also related to sublinear-time algorithms for finite metric spaces. Such algorithms become particularly relevant when one is interested in estimating parameters such as the average ER distance or identifying the central point in the ER metric space. Notably, these algorithms operate without reconstructing the graph or explicitly learning its topology. There is a substantial body of work on testing properties of metric spaces; see, for example,~\cite{conf/stoc/Indyk99,journals/iandc/ParnasR03,conf/icalp/Onak08}.

\paragraph{Graph machine learning.}

One important (albeit perhaps less direct) motivation for our work is graph machine learning. Currently, there are no polynomial-time machine learning algorithms (e.g.,~graph neural networks) that can completely capture the structure of a graph; that is to say, all existing methods will give the same output on some set of non-isomorphic graphs (see~\citep{xu2018how} for an example). This means that no graph learning algorithm captures all properties of a graph.
\par 
As an imperfect solution, one approach is compute some quantities associated with the graph and use these as input to the machine learning algorithm. For example, one could compute the effective resistance between all pairs of vertices (as in~\citep{zhang2023rethinking,velingker2024affinity}) and use this as an input to a neural network. These graph quantities are sometimes called \textit{positional encodings}. Since the use of positional encodings is known not to completely capture the topology of a graph, researchers instead ask which properties of a graph these encodings do capture. For example, effective resistance increases the expressive power of message-passing neural networks~\citep{velingker2024affinity}, transformers~\citep{vaswani2017attention} using effective resistance as a positional encoding can determine which vertices are cut vertices (a property neither message-passing neural networks nor shortest-path distance can detect)~\citep{zhang2023rethinking}, and transformers using effective resistance can determine which edges are cut edges~\citep{black2024comparing}. At a high level, the goal is to understand the power of effective resistance in deducing graph properties when used as a positional encoding in a graph neural network. This question becomes particularly relevant to our work when it focuses on identifying a small subset of effective resistances that are sufficiently informative for the task at hand, thereby reducing computational costs.


