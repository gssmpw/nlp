\section{Stochastic arrivals: lower bound}
\label{sec: stochasting_setting}


In this section, we prove \Cref{theorem: stochastic CR LB} via a derivation of the stochastic matching ($\mathrm{SM}$) algorithm proposed by \citet{brubach2016new}. In their approach, after solving the LP benchmark and obtaining the optimal solution $\xbf$, they match an arriving demand node $v$ to a node $u \in N(v)$ with probability~$\frac{x_{u,v}}{T p_v}$. We extend their algorithm and proof to accommodate imbalanced instances.
\BlankLine

\RestyleAlgo{ruled}
\begin{algorithm}[H]
\label{alg: sm}
\SetKwInOut{Input}{input}\SetKwInOut{Output}{output}
\SetKwData{Left}{left}\SetKwData{This}{this}\SetKwData{Up}{up}
\SetKwFunction{Union}{Union}\SetKwFunction{FindCompress}{FindCompress}
\SetKwInOut{Input}{Input}\SetKwInOut{Output}{Output}
\Input{An instance $\inst$ and $\market > 0$.}
\Output{A set of offline decisions for the delayed problem.}
\BlankLine

\caption{$\mathrm{SM}$}

  Compute an optimal solution $\xbf[\market]$ to $\offIkappa$.

  When a demand of type $v$ arrives, assign $v$ to any supply node $u \in N(v)$ with probability $\frac{x_{u,v}[\market]}{T p_v}$.

 %\Return 
\end{algorithm}
\BlankLine


\begin{proposition}
\label{prop: fst_part_stochastic}
Given an instance $\inst$ and $\market>0$, denote by $\mathrm{SM}(\inst, \market)$ the reward of Algorithm~\ref{alg: sm} in a single execution. Then, we have that
    \begin{equation*}
        \frac{\mathbb E[\mathrm{SM}(\inst, \market)]}{\offI(\inst)} ~ \geq ~ \begin{cases}
            \frac{1-e^{-\market}}{\market}& \text{ if $\inst$ is $\kappa$-oversupplied,}\\
            1-e^{-\market} & \text{ if $\inst$ is $\kappa$-undersupplied.}
        \end{cases}
    \end{equation*}
\end{proposition}

\noindent{\textbf{Proof of \Cref{prop: fst_part_stochastic}.}
    Consider any instance $\inst$ and $\market > 0$ as inputs to $\mathrm{SM}(\inst, \market)$. We define the following:
    \begin{itemize}
        \item[-] $A(u,t)$ is the event that demand arrival $t$ consumes supply node $u$, i.e., that $\mathrm{SM}(\inst, \market)$ matches arrival~$t$ to $u$, that $u$ has not been consumed in periods $1,\ldots,t-1$, and that $t$ chooses to consume $u$;
        \item[-] $B(u,t)$ is the event that $u$ has not been consumed before the $t$-th arrival, i.e., that arrivals $1,\ldots,t-1$ do not consume node $u$;
        \item[-] $C(u,t)$ is the event that  $\mathrm{SM}(\inst, \market)$ matches arrival $t$ to $u$ and $t$ chooses to consume $u$ (regardless of whether that is possible given that $u$ may have been consumed previously).
    \end{itemize}
    With these events defined, we observe that $A(u,t) \subseteq C(u,t)$. Then, from the feasibility of $\xbf^*[\market]$, we derive 
    \begin{equation*}
        \mathbb P(A(u,t)) ~\leq ~ \mathbb P(C(u,t)) ~ = ~ \sum_{v \in V} p_v \cdot \frac{x^*_{u,v}[\kappa]}{T p_v} \cdot \prob_{u,v}  ~\leq ~ \frac{\kappa}{T}.
    \end{equation*}
    Denoting for any event $X$ the complement of $X$ as $\overline{X}$, we obtain that $\mathbb P( \overline{A(u,t)}) \geq \mathbb P( \overline{C(u,t)}) \geq (1-\frac{\kappa}{T})$. 
    \begin{align*}
    \text{Hence,}\quad
        \mathbb P(B(u,t)) ~ = ~\mathbb P(\cap_{i=1}^{t-1}  \overline{A(u,i)}) ~ \geq ~ \mathbb P(\cap_{i=1}^{t-1}  \overline{C(u,i)}) ~ = ~ \prod_{i=1}^{t-1} \mathbb P(\overline{C(u,i)}) ~ \geq ~ \left(1- \frac{\kappa}{T} \right)^{t-1}.
    \end{align*}
    Lastly, for any $u\in \supply$ and $v\in V$, consider the event $D_{uv}$ that a node of type $v$ consumes the node $u$ in any period. We can bound the probability of $D_{uv}$ as
    \begin{align*}
        \mathbb P[D_{uv}] &= \sum_{t=1}^T \mathbb P[\text{arrival $t$ is of type $v$}, B(u,t), \text{$\mathrm{SM}(\inst, \market)$ matches $t$ to $u$, $t$ chooses to consume $u$}]\\
        &\geq \sum_{t=1}^T p_v \cdot \left(1- \frac{\kappa}{T} \right)^{t-1} \cdot \frac{x^*_{u,v}[\kappa]}{T p_v} \cdot \prob_{u,v}
         = \frac{1}{T} \cdot \frac{1 - (1-\kappa/T)^T}{\kappa/T} \cdot x^*_{u,v}[\kappa] \prob_{u,v}\\
        & \geq (1-e^{-\kappa}) \cdot \frac{x^*_{u,v}[\kappa ] \prob_{u,v}}{\kappa}.
    \end{align*}
    Summing over all $D_{uv}$ we conclude
    \begin{align*}
        \mathbb E[\mathrm{SM}(\inst, \market)]=\sum_{u\in \supply, v\in V} \mathbb P[D_{uv}] &\geq \frac{1-e^{-\market}}{\market} \cdot \offIkappa\\
        &= \begin{cases}
            (1-e^{-\market}) \cdot \offI(\inst) & \text{if $\inst$ is $\kappa$-undersupplied,}\\
            \frac{1-e^{-\market}}{\market} \cdot \offI(\inst) & \text{if $\inst$ is $\kappa$-oversupplied.}\quad\Halmos
        \end{cases}
    \end{align*}

\section{Numerical experiments}
\label{sec: numerical_exp}
In this part of our analysis, we investigate the imbalance of real-world matching markets and how it may affect the CR of $\greedy$. We rely on data from a volunteer platform that matches individual providers with individual demand requests (to ensure anonymity, we omit a description of the nature of the requests); the nature of the platform is such that (i) the platform centrally matches supply and demand, (ii) upon arrival of a node t, the edges between t and the supply nodes present are known to the platform (demand comes with a latitude/longitude and each supply node has a declared radius within which they are willing to fulfill demand; moreover, each demand/supply may have particular restrictions which are known to the platform), and (iii) when a demand node is matched to a supply node, the platform does not immediately learn whether or not the supplier actually fulfills the desired request (we model this through the delayed realization). 

In practice, supply renews on our platform with heterogeneous periodicities as some suppliers volunteer on a monthly, a bi-weekly, a weekly or a one-off basis. For the sake of our analysis here, we ignore this replenishment and instead focus on a 2-week horizon in 2023, wherein we consider a static set of supply nodes that we assume are each available to fulfill at most one demand request. We obtain the set of supply nodes by including each supplier that was available to fulfill at least one demand request during the time interval of interest. For simplicity, we model each supplier to have a radius of 5mi and ignore the additional constraints that may arise. 

With respect to demand, we include the requests that came onto the platform during the 2-week horizon; thus, our analysis does not include unfulfilled demand requests that may have remained on the platform from previous weeks. For the arrival order of the demand requests, we rely on the same ordering in which requests occurred historically. % We vary the value of 



In total, we use our data to create 69 graphs (many of which are not connected), on which we then compute 6,900 distinct values of $\offI(\inst)$, $\greedy(\inst)$, and $\market$ by varying $\mu$ to take on 100 different values from 1/1000 to 1, i.e., $\mu\in\{ 10^{-3} + i \cdot \frac{1-10^{-3}}{100} \, : \, i \in \{0, \ldots, 100\}\}$. $\mu$. We discuss in \Cref{ssec: computation_kappa_instance} how we compute $\market$ for a given instance (based on \Cref{def: new_over_undersupplied}). In \Cref{fig:side_by_side_scatter} we show two scatter plots that display the relationship between the empirically observed  ratio $\greedy(\inst)/\offI(\inst)$ and the imbalance as well as the analytically proven bound on the CR as a function of the imbalance; one scatter plot shows all 6,900 values of $\offI(\inst)$, $\greedy(\inst)$ whereas the other only shows those for one particular graph (\Cref{fig: map_graph_instance}). Our analysis gives rise to the following observations.

\begin{figure}[h!]
    \centering
    % Left scatter plot
    \begin{minipage}{0.48\textwidth}
        \centering
        \includegraphics[width=\linewidth]{images/CR_GREEDY_OFF_ALL_2.pdf}
    \end{minipage}
    \hfill
    % Right scatter plot
    \begin{minipage}{0.48\textwidth}
        \centering
        \includegraphics[width=\linewidth]{images/CR_GREEDY_OFF_STATE_2.pdf}

    \end{minipage}
    \caption{\small{Scatter plots for the CR of $\greedy$: aggregated instances (left) and a particular instance (right).}}
    \label{fig:side_by_side_scatter}
\end{figure}

\noindent\textbf{Greater imbalance yields better CRs.} In line with our theoretical results, we find that the ratio $\greedy(\inst)/\offI(\inst)$ generally improves when instances are more imbalanced. For example, for balanced instances $(\market=1$), we see the ratio go down to $.7$, but for $\market\approx10$, the lowest ratios lie above $.95$. Especially in the range of $\market\leq10$, we see that the improvement in the ``worst instances'' seems to track the trajectory of the  worst-case lower bound fairly well. Of course, as all of these instances are based on real-world data, and not on adversarial upper-triangular instances, we would not expect the lower bound to ever be tight.

\noindent\textbf{Preponderance of balanced instances when} $\prob = 1$. Though our analysis shows a wide range of values of $\market$ that are achieved, we find that most (though not all) instances are at $\market=1$ when~$\prob=1$. This occurs because, as we alluded to before, many of our instances are disconnected and contain sparse subgraphs with very few nodes. For example, the graph in \Cref{fig: map_graph_instance} contains a connected component in which a demand and a supply node have an edge between each other and no other edges. These two nodes by themselves render the instance balanced according to our definition. Of course, in reality, the CR would be higher than $\frac{1}{\market+1}=1/2$: the CR across different connected components is a convex combination of the CR on each individual component with the weight dictated by the value of $\offI(\inst)$ restricted to the component. Thus the pair would be discounted in a way that is not captured in our imbalance definitions (though, in the specific example of a connected component consisting of just one demand node and one supply node, the greedy algorithm makes the optimal decision anyway).


\noindent\textbf{Oscillating values of} $\market$ \textbf{for increasing $\prob$}.  Focusing on a particular choice among our 69 graphs (see \Cref{fig: map_graph_instance}), we find the trajectory of $\market$ to display an oscillating behavior  as $\prob$ increases (see right panel of \Cref{fig:side_by_side_scatter}). This holds due to the following dynamics. For simplicity denote by $(\undersup_\prob, \oversup_\prob)$ the imbalanced pair for a given $\prob$. When $\prob$ is close to $0$, all supply nodes are in $\oversup_\prob$, and increasing $\prob$ yields a decrease in imbalance. As $\prob$ grows larger, the imbalance~$\market$ decreases and hits~1 exactly when the first supply node becomes part of $\undersup_\prob$. Consider the nodes that are now in $\undersup_\prob$: as we increase $\prob$, these nodes become more undersupplied and consequently $\market$ increases again. This continues until some node in $\oversup_\prob$ becomes less oversupplied than nodes in $\undersup_\prob$ are undersupplied; at that point,  increasing $\prob$ lowers $\market$ once again, and this process repeats itself up to the point where $\prob=1$. % This is due to transitions between imbalance pairs $(\undersup, \oversup)$ in the graph.  each time a node switches from $\undersup$ to $\oversup$, the  To explain this, we further analyze the instance's connected components for varying $\prob$ in .




\begin{figure}[h]
    \centering
    % Left column: Map/Graph of the instance
    \begin{minipage}[b]{0.48\textwidth}
        \centering
        \includegraphics[width=\linewidth]{images/plot_map_2.pdf}
    \end{minipage}
    \hfill
    % Right column: Bipartite instances
    \begin{minipage}[b]{0.48\textwidth}
        \centering
        \includegraphics[width=\linewidth]{images/bipartite_1_2.pdf}
        \includegraphics[width=\linewidth]{images/bipartite_2_2.pdf}
        \includegraphics[width=\linewidth]{images/bipartite_3_2.pdf}
    \end{minipage}

    \caption{\small{Map/graph of the particular instance (left) and the instance's connected components for different values of $\mu$ (right). 
{Coordinates were randomly perturbed to preserve the privacy of the platform users.}} }
    \label{fig: map_graph_instance}
\end{figure}



\section{Concluding Remarks}
\label{section: conclusion}

Our paper explores the impact of supply and demand imbalances on online bipartite matching with stochastic rewards. We introduce a novel parametrization to measure imbalance and prove that better CRs attainable on imbalanced instances. We further reveal that when contemplating decisions related to the expansion of a platform's demand or supply, the dependence of the CR on the resulting imbalance ought to be taken into account. To the best of our knowledge, ours is the first study to (i) derive CRs that are parameterized by the supply-demand imbalance, (ii) illustrate how these ought to affect an upstream supply planning question.