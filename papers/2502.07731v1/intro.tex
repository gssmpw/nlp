\section{Introduction}
\label{sec: introduction}

Over the past decades, the emergence of online platforms has brought about a new economy that relies on algorithmic matching procedures for its core operations. These platforms cover various industries such as labor markets, accommodation, online advertising, and transportation, and have caused a revolution in the way we travel, work, and consume goods and services. Platforms like Uber, Airbnb, and Google AdWords are designed to optimize their operations by matching incoming demand with available supply. 
The success of these platforms relies on (i) effectively executing the online matching that connects incoming demand to available supply and (ii) ensuring an appropriate level of supply to handle the demand. 

The complex question of how to match incoming demand with available supply has spurred ample research on online matching algorithms that dates back to the seminal work of \citet{karp1990optimal}. 
In the original problem, a decision-maker is given a fixed set of supply nodes offline and observes the arrival of one new demand node per period over a known time horizon. 
Each time a demand node arrives, the decision-maker needs to irrevocably match the incoming demand after observing which supply nodes the demand node is adjacent to. The decision-maker aims to design an online matching algorithm that maximizes the number of matches made. Since this foundational work, the field has evolved to explore a diverse set of extensions, including the vertex-weighted variant \citep{aggarwal2011online,jaillet2014online}, the edge-weighted variant \citep{feldman2009free,haeupler2011online} and the use of \textit{stochastic rewards} on the edges \citep{mehta2012online,mehta2014online}, among others. In all of these variants, algorithmic performance is measured through the competitive ratio (CR), i.e., the worst-case gap between an algorithm's achieved matching size and that of an offline oracle. Upper bounds on the CR are usually proven through worst-case instances, on which no online algorithm can outperform a given bound; these worst-case instances are usually balanced in the sense that the expected demand precisely matches the available supply.


Real-world instances, however, usually do not  exhibit the kind of balance that is inherent in worst-case instances (see our analysis in \Cref{sec: numerical_exp}). For instance, the demand for Uber rides may vastly outstrip the supply during rush hour whereas Uber drivers may experience a low utilization in-between peak demand periods. 
As a result,  real-world instances are often imbalanced in one direction or the other. Indeed, the market imbalance between supply and demand may even result from the platform's offline planning: firms like Uber and Airbnb leverage their forecasting abilities to guide promotional tactics on both market sides. This process strives to establish an equilibrium where supply is neither excessive nor lacking. The exact tradeoff between the two is captured by the platform's overage and underage costs, i.e., the cost of underutilized resources and that of missed matching opportunities, respectively. Unless the overage and underage cost are approximately equal, it is likely in the firms interest to strive for a somewhat imbalanced marketplace.\footnote{Of course, the planning process of a matching platform is more complicated than a simple newsvendor calculation, but directionally one may still expect overage and underage to guide the desired imbalance of demand and supply.}

In order to identify the appropriate (im)balance of supply and demand, a platform needs to know how its online matching algorithms translate the available supply and demand into a resulting matching size. Specifically, as we describe below, algorithms may obtain significantly higher CRs in imbalanced markets. As a result, we find that a matching platform may find itself in a situation where its (e.g., monthly expected) demand is twice as large as its supply; a natural expectation would then be that by doubling its supply, the platform would be able to double the number of matches it makes between supply and demand. After all, it faces sufficient demand for all of the additional supply to be consumed. However, this reasoning ignores that the matching algorithm's performance deteriorates in a more balanced market. Indeed, we find that, depending on the market dynamics, a doubling in supply may grow the matching size by just 50\%. This gap leads us to answer the following questions in our work:

\begin{enumerate}
    \item \emph{How does the market imbalance of supply and demand affect the performance of matching algorithms? Are higher CRs achievable in imbalanced settings?}
    \item \emph{How does the effect of imbalance on a firm's matching ability affect its underage/overage tradeoff?}
\end{enumerate}

\subsubsection*{Outline of our contributions.}
\label{ssec: outline}
Section \ref{sec: model} sets up our model as an online bipartite matching problem spanning $T$ steps, wherein the supply nodes are known to the platform. Conversely, the arriving demand nodes remain unknown to the platform, and at each timestep, a node arrives, compelling the platform to make an irrevocable decision regarding the match for the incoming demand.
Each specific match carries a success probability that determines whether the demand consumes the matched resource. This is often referred to as the stochastic reward setting. Our study focuses on a class of \textit{delayed algorithms}, which address the complexities of having delayed realizations, e.g., when a customer may not immediately purchase an item but instead delays its purchase to a later point --- this class of algorithms, with the benchmark of our choice (see below) is particularly amenable to analyzing the effect of imbalance. At time $t$, a delayed algorithm has access to information about previous matches up until time $t$ but lacks information about their stochastic realizations. 
Below, we begin by describing how our paper models market imbalance. We then address our first research question by summarizing the impact that  imbalance has on the performance of matching algorithms. Thereafter, we use these results to derive insights on the trade-offs between underage and overage of a matching platform.

\emph{\textbf{Modeling imbalance.}} Our model captures the imbalance of a given instance through the effect of loosening/tightening a particular constraint in a deterministic linear program (DLP) that upper bounds the performance of any algorithm (see  \Cref{sec: model}). 
Specifically, we parameterize instances by a parameter $\market$, where $\market\leq 1$ indicates that matching supply nodes $\market$ times in expectation is sufficient for the DLP to achieve its objective. We thus label these instances as \textit{$\market$-oversupplied}, as the DLP only requires a portion of the supply to effectively meet the same demand. Similarly, we call an instance \textit{$\market$-undersupplied}  when allowing supply nodes to be matched $\market\geq 1$ times increases the objective of the DLP upper bound by a factor of $\market$. We also extend our definition to mixed instances that have both an undersupplied and an oversupplied component to capture the fact that instances may be locally over- and locally undersupplied. Moreover, we use the data of an online matching platform (see Section \ref{sec: numerical_exp}) to evaluate the level of imbalance in real-world matching instances.



\emph{\textbf{The effect of imbalance on the performance of matching algorithms.}} Building on our definition of imbalance, we formally present our findings in \Cref{sec: overview_results}.  
We analyze two different arrival regimes, adversarial and stochastic. In the adversarial arrival setting (Section \ref{sec3}) we show that for any instance characterized by the imbalance parameter~$\market$ (which we refer to as $\market$-imbalanced) there exists a delayed algorithm that achieves a CR of $\max \left\{ \frac{1}{\market+1}, \frac{\market}{\market+1}\right\}$ when compared to the DLP. This result, with a slightly different information structure, also holds for the generalized imbalanced definition that encapsulates graphs that are locally over- and locally undersupplied (\Cref{sec: alternative_def}).  

In the stochastic setting (Section \ref{sec: stochasting_setting}), we show that for any instance characterized by the imbalance parameter~$\market$ there exists a delayed algorithm with a CR of $\max \left\{\frac{1-e^{-\kappa}}{\kappa}, 1-e^{-\kappa} \right\}$.\footnote{We do not extend this result to our generalized imbalance definition because the best-possible CRs in the stochastic setting lack the symmetry in $\market$ that the adversarial setting provides (see \Cref{sec: overview_results}).}
We prove that these bounds are optimal by constructing a sequence of instances in which the CR of the optimal delayed algorithm converges to the specified values (\Cref{section: impossibility}). Hence, balanced instances, characterized by $\market = 1$, give rise to the worst-case CRs whereas imbalanced instances give rise to significantly higher ones (e.g., as $\market\to\infty$, they converge to~$1$). 

\emph{\textbf{Imbalance and underage/overage considerations.}} The previous results may affect a platform's demand and supply planning. To illustrate, consider a platform $A$ that encounters adversarial arrivals and delayed stochastic realizations of matches ({detailed in \Cref{sec: example_introduction}}). Further, assume that our presented lower bounds hold tightly and denote the value of the DLP by \texttt{OPT}. Now, suppose platform $A$ is $2$-undersupplied, i.e., the platform faces enough demand to consume twice as much supply (in expectation) as it has at its disposal. With the assumptions stated above, this implies that platform $A$ matches demand in such a way that $\texttt{OPT} \times \frac{2}{1+2}= \frac{2}{3} \cdot \texttt{OPT}$ supply nodes get consumed (in expectation). Given that platform $A$ faces enough demand to consume twice as much supply (in expectation) as it has at its disposal, it is natural to assume that by doubling its available supply it would be able to double the demand it is able to serve. Indeed, with twice as much supply the objective of the DLP doubles to $2 \cdot \texttt{OPT}$. However, in doing so the platform creates a balanced market ($\market=1$), in which the CR of its matching algorithm is just $1/2$. Therefore, it may occur that the added supply only translates into $2\cdot \texttt{OPT} \times\frac{1}{1+1}=\texttt{OPT}$ supply nodes being consumed, i.e., rather than doubling the number of successful matches, it only increases the latter by 50\%. Of course, these considerations rely on the arrivals being adversarial. A $2$-undersupplied platform $B$ that faces stochastic arrivals would be able to create $(1-e^{-2}) \cdot \texttt{OPT} \approx 0.86 \cdot  \texttt{OPT}$ successful matches in expectation, yet by doubling its supply that number would only increase by about $47\%$ to $2(1-1/e) \cdot \texttt{OPT}\approx 1.26 \cdot \texttt{OPT}$. Thus, in both arrival settings, disappointment might ensue when a doubling of supply does not yield twice as many successful matches. In Section~\ref{sec: overview_results}, we adapt these insights to a setting where decision-makers face a linear cost for the supply on the platform. 


\subsubsection*{Related Literature.}
\label{ssec: related_literature}


{
We survey three separate areas: online matching with stochastic rewards under (i) adversarial and (ii) stochastic arrivals, and (iii) the effect of imbalance in other matching problems.

\emph{\textbf{Online bipartite matching with stochastic rewards (stochastic arrivals)}.}  From an algorithmic point of view, our work falls into the literature on online bipartite matching with stochastic rewards. Introduced by \citet{mehta2012online}, papers in this stream generalize the online bipartite matching model \citep{karp1990optimal} by allowing matches to succeed or fail with a known consumption probability. \citet{mehta2012online} demonstrated that the famous RANKING algorithm of \citet{karp1990optimal} achieves a CR of 0.534 when all consumption probabilities are identical and introduced another algorithm with a CR of 0.567 for identical and vanishingly small probabilities; \citet{mehta2014online} show that 0.534 is also achievable for unequal but vanishingly small probabilities. Recently, \citet{huang2020online} designed an algorithm -- based on the randomized online primal dual framework of \citet{devanur2013randomized} -- with a CR of~$0.576$ for the case of identical and vanishingly small probabilities. The CR for the more general setting of unequal but vanishingly small probabilities was improved by \citet{goyal2023online} to $0.596$ and then by \citet{huang2023online} to  $0.611$ (the latter two works compare to a different benchmark).  For delayed algorithms (which they call non-adaptive), \citet{mehta2012online} showed that no algorithm can obtain a CR better than $1/2$, and \citet{mehta2014online} devised an algorithm that matches that CR under general probabilities. {For the more general problem of Online Submodular Welfare Maximization, \citet{udwani2024online} shows that adaptivity to stochastic rewards does not offer improved CRs.} 

Our results for the adversarial setting (Theorem \ref{theorem: adversarial CR LB}, Proposition \ref{prop: adversarial CR UB}) provide new CRs and corresponding upper bounds that are parameterized by $\market$. In particular, we provide a delayed algorithm that achieves a CR of $\max\{1/(\market+1),\market/(\market+1)\}$ on $\market$-imbalanced instances and show that no delayed algorithm can do better.
}

{
\emph{\textbf{Online bipartite matching with stochastic rewards (stochastic arrivals)}.} 
Our stochastic arrival setting aligns with the \textit{known I.I.D.} framework introduced by \citet{feldman2009online} for online bipartite matching. In contrast to the adversarial setting, at each time step, the arriving demand node is drawn independently from previous arrivals from a given known distribution of demand types. 
\citet{feldman2009online}, \citet{manshadi2012online}, and  \citet{jaillet2014online} respectively developed algorithms with CRs of 0.67, 0.702, and 0.706. However, these algorithms do not consider stochastic rewards and, moreover, they are based on comparisons to more intricate benchmarks. 
\citet{haeupler2011online} developed an algorithm for an edge-weighted setting with stochastic arrivals an deterministic consumption; their algorithm achieves a CR of $.667$. Despite not modeling stochastic rewards, their algorithm is non-adaptive in a way that facilitates extending it to a delayed algorithm in the setting with stochastic rewards. Indeed, \citet{brubach2020online} build on \citet{haeupler2011online} to derive a delayed algorithm with CR $(1-1/e)$ for the stochastic rewards setting. Our corresponding bound (\Cref{theorem: stochastic CR LB}) is  $\max \left\{\frac{1-e^{-\kappa}}{\kappa}, 1-e^{-\kappa} \right\}$, which coincides with theirs when $\market = 1$. 
We remark that \citet{brubach2020online} also provide a CR of 0.702 for the stochastic reward setting, but this does not hold for a delayed algorithm and requires additional assumptions.
}

{
\textbf{\emph{Imbalance in other matching problems.}} We know of few papers that analyze the effect of imbalance on matching markets. The celebrated result of \citet{ashlagi2017unbalanced} focused on (a tiny) imbalance in stable matching, identifying that it yields a significant advantage for the short side; related settings have since been studied in the literature \citep{kanoria2021matching,cai2022short}. %should be citet or citep here?
However, the stable matching setting differs significantly from online bipartite matching, both with respect to objective and constraints. More closely related to our work, \citet{ma2020group}  consider a steady-state fairness objective and prove that online matching algorithms may reach asymptotic optimality as the imbalance between supply or demand increases. Our results also characterize how algorithms benefit from imbalance, but we focus on a slightly different setting (standard online stochastic matching problem) and (ii) rather than providing only an asymptotic result, we characterize  explicitly the CR as a function of the level of imbalance. In that regard, our work is closer to \citet{abolhassani2022online} who also provide a parameterized guarantee, though only for over-, not undersupplied, instances and only when the imbalance is an integer-multiple.
