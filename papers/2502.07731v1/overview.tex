\section{Overview of Main Results}
\label{sec: overview_results}

In this section, we provide an overview of how demand-supply imbalances yield improved CRs for matching algorithms and how these improvements in turn affect a platform's optimal demand-supply imbalance. 

We begin with an algorithmically trivial example to illustrate the impact of imbalance:  an instance $\inst$ that is \textit{complete bipartite} with $L$ supply nodes. For some fixed positive integer $n$ we have a time horizon of~$T = n\cdot L$ arrivals with consumption probabilities equalling $\prob_{u,t} = 1/n$ for each arrival $t$ and each supply node~$u$. Note that $\offI(\inst)$ is able to match all the demand and gets a value of $L$. In contrast, consider the delayed algorithm $\alg_D$ that assigns each arrival to the supply node with the fewest demand assigned so far; this ensures in this instance that over the course of the horizon every supply node gets matched to exactly $n$ demand nodes. The expected reward of this algorithm in this particular instance is given by
\begin{equation}
\label{eq: alg_D in overview_results}
\mathbb E[\alg_D(\inst)] = \sum_{u \in \supply} \left( 1 - \left(1-\frac{1}{n}\right)^n \right)= L \left(  1 - \left(1-\frac{1}{n}\right)^n \right).
\end{equation}
As $n \rightarrow \infty$ the CR of this algorithm converges to $(1-1/e)$ in this instance, from above. Furthermore, no delayed algorithm can do better in this instance (see  \Cref{section: impossibility}). 

However, this upper bound on the CR occurs in a balanced instance, as is common in the online matching literature. When instances are imbalanced, the CR of the algorithms may improve. To illustrate, we can generalize the above instance to have $\market \cdot nL$ arrivals\footnote{By our assumption on $\market$, we can express $\market = p/q$ with $p,q \in \mathbb Z_{>0}$. When $n$ is a multiple of $q$, then $\market \cdot nL$ is an integer and the instance is well-defined for all values of $L$. In particular, we take $\market$ as small as $1/n$.} and compare the performance of $\alg_D$ against that of our offline benchmark. %, as illustrated in \Cref{figure: delayed_instantaneous_heatmap}. 
Intuitively, as the market imbalance ratio~$\market$ increases beyond~$1$, the benchmark remains constant at~$L$. However, the performance of $\alg_D$ improves due to a higher number of arrivals compared to the original instance, which leads to more supply nodes being consumed in expectation. Conversely, when $\market$ is less than~$1$, both the algorithm's and the benchmark's value decrease, but the latter decreases faster which allows the  expected reward of the algorithm to approach that of the {benchmark} (as $\market\to0$). This occurs because each supply node is matched with fewer demand nodes, which leads to the number of supply nodes consumed being approximately equal to $(\kappa nL)\cdot \mu$.\footnote{When $\market < 1$, the probability that a node will be consumed in the imbalanced instance is given by $ 1 - \left(1-\frac{1}{n}\right)^{\market n}$. For $\market = 1/n$, this simplifies to $1/n$, matching the value for a single match in the offline benchmark.} Though we focus on delayed algorithms, similar improvements occur for algorithms that adapt to consumption realizations.\footnote{{As these algorithms have more information, our lower bounds on what is achievable by delayed algorithms naturally extend. At the end of this section, we discuss the regimes in which imbalance may be so large that these lower bounds become higher than the best-known lower bounds for existing adaptive algorithms.}}

Motivated by this special case, we investigate the achievable CRs for delayed algorithms on imbalanced instances. Our main guarantee for the setting with adversarial arrivals is as follows:
\begin{theorem}
    \label{theorem: adversarial CR LB}
    Under the adversarial arrival model, there exists a delayed algorithm with a CR of at least $\max \left \{ \frac{1}{1+\market}, \frac{\market}{1+\market} \right\}$ for the sets of $\market$-undersupplied and $\market$-oversupplied instances.
\end{theorem}


We establish \Cref{theorem: adversarial CR LB} by lower bounding, for any $\market$-oversupplied or $\market$-undersupplied instance $\inst$ and arrival sequence $\sigmabf$, the CR of a greedy algorithm ($\greedy$) that assigns each arriving demand node to an adjacent supply node that has been assigned the fewest demand nodes so far. We lower-bound the ratio $\mathbb E[\greedy(\inst, \sigmabf)]/\offIkappa$ using a factor-revealing linear program; the formulation of this linear program is obtained by leveraging the techniques developed in the analysis of the BALANCE algorithm \cite{mehta2007adwords}  to derive constraints that involve the operational dynamics of both $\greedy$ and $\offI$ on a given instance and arrival sequence. We then characterize a feasible solution to the dual of the factor-revealing linear program, which establishes a lower bound on the CR. Further details on these techniques, as well as the full proof of  the result can be found in Section~\ref{sec3} and its associated appendix. 

For our generalized imbalance definition (\Cref{def: new_over_undersupplied}), we can provide an analogous guarantee that requires the algorithm to have knowledge of the sets $(\undersup, \oversup)$ that form a $\market$-imbalanced pair.\footnote{We omit an analogous guarantee for the stochastic arrivals since the lower bound in \Cref{theorem: stochastic CR LB} is not symmetric in~$\market$. Hence, a similar guarantee would depend on the relative size of the over- and undersupplied components of the instance.} %an equivalent guarantee to that in \Cref{theorem: adversarial CR LB} can be obtained. 
This result applies to a much wider class of instances than \Cref{theorem: adversarial CR LB} given that it allows for the sets~$\undersup$ and $\mathcal{O}$ to both be non-empty. In fact, if one of these sets is empty (i.e., $\undersup = \supply$ or $\mathcal{O} = \supply$), the instance is either $\market$-undersupplied or $1/\market$-oversupplied and we recover the guarantee from \Cref{theorem: adversarial CR LB}. 
The proof of this result is deferred to \Cref{sec: alternative_def}.

\begin{theorem}
    \label{theo: CRguarantee_extended_def}
    Let $\inst$ be an instance that admits a $\market$-imbalanced pair $(\undersup, \oversup)$. Then, there exists a delayed algorithm with $(\undersup, \oversup)$ as an input and a CR of at least $\frac{\market}{1+\market}$ in the adversarial arrival model.
\end{theorem}
Our algorithm leverages knowledge of $(\undersup, \oversup)$ to always prioritize matching demand to oversupplied rather than undersupplied nodes. The main technical ingredients of \Cref{sec: alternative_def} show that the DLP solution also does not ever assign a demand node, that is adjacent to an oversupplied node, to an undersupplied node. With that, the CR is a quick corollary of \Cref{theorem: adversarial CR LB}.

    
Under stochastic arrivals we use the parametrization in \Cref{def: under/oversupplied} to obtain the following: %  guarantee:
\begin{theorem}
\label{theorem: stochastic CR LB}
Under the stochastic arrival model, there exists a delayed algorithm that is at least $\max \left\{\frac{1-e^{-\market}}{\market}, 1-e^{-\market} \right\}$-competitive for the sets of $\market$-undersupplied and $\market$-oversupplied instances.
\end{theorem} 


To establish \Cref{theorem: stochastic CR LB}, we adapt the algorithm described by \citet{brubach2020online} to account for the imbalance $\market$. Our algorithm computes an optimal solution to $\offIkappa$ for a given instance; then, when a demand node arrives, it is matched to an adjacent supply node with a probability proportional to this solution. Using the techniques in \citet{brubach2020online} and \Cref{def: under/oversupplied}, we establish a lower bound on the probability that each individual edge results in a successful match, which  allows us to  demonstrate the desired bound on the CR. Further details and the full proof of the result are in Section~\ref{sec: stochasting_setting}.


{We demonstrate the tightness of our CR bounds by presenting impossibility results that establish upper bounds on the performance of any delayed algorithm within the given arrival model.}

\begin{proposition}
    \label{prop: adversarial CR UB}
    In the adversarial arrival model, no delayed algorithm can achieve a competitive ratio strictly greater than $\max \left \{ \frac{1}{1+\market}, \frac{\market}{1+\market} \right\}$  for the sets of $\market$-undersupplied and $\market$-oversupplied instances.
\end{proposition}

We prove \Cref{prop: adversarial CR UB} in \Cref{section: impossibility} via an upper-triangular instance \citep{karp1990optimal}, on which no delayed  algorithm can exceed our bound. Extending upper-triangular instances to our setting requires us to carefully calibrate various parameters to ensure that the instance is well-defined and of the right imbalance. We highlight that similar instances were previously considered by \citet{mehta2012online} to show that no delayed-feedback algorithm can achieve a CR greater than $1/2$, a result that we generalize to imbalanced instances. Under stochastic arrivals we similarly obtain the following upper bound (proof in \Cref{section: impossibility}): 

\begin{proposition}
    \label{prop: stochastic CR UB}
    In stochastic settings, no delayed algorithm can achieve a competitive ratio strictly greater than $\max \left\{\frac{1-e^{-\market}}{\market}, 1-e^{-\market} \right\}$  for the sets of $\market$-undersupplied and $\market$-oversupplied instances.
\end{proposition}

Our proof builds on the instance from the beginning of this section (with $L=1$).


\subsubsection*{Interpretation of results.} In practice, platforms can influence the level of imbalance under which their matching algorithms operate through various strategic levers. Increasing supply typically incurs additional costs but, at the same time, the ability to capture more demand can significantly boost revenue.
% by maximizing the utilization of available supply. 
Consequently, platforms must carefully balance the potential costs of expanding supply against the revenue benefits derived from serving a larger customer base. As the resulting market imbalance affects the downstream performance of the matching algorithm, it is in the platform's interest to incorporate the expected downstream performance into its upstream inventory level decision.  In the remainder of this section we explore the interplay between the upstream decision and the downstream performance and compare our bounds to existing CRs.

\emph{Overage and underage costs consideration.}
We begin by combining our CR bounds on the performance of a downstream matching algorithm with an inventory decision made upstream. To do so, we assume that two platforms $A$ and $B$ are in a setting with the following features:
\begin{itemize}
    \item In $\offI(\inst)$, supply and demand are perfectly matched, i.e., there exists an optimal solution $\mathbf{x}$ to $\offI(\inst)$ in which  $\forall u\in \supply:\; \sum_{t:(u,t)\in E}\prob_{u,t} x_{u,t}=1$, and $\forall t\in[T]:\;\sum_{u:(u,t)\in E}x_{u,t}=1$;
    \item Denoting by $d=\sum_{t:(u,t)\in E}\prob_{u,t}x_{u,t}$, the expected (realizing) demand, we assume that the inventory decision of each platform has a proportional effect on the imbalance of $\offIkappa$ downstream, i.e., purchasing $d/\market$ units of inventory leads to a $\market$-imbalanced instance. E.g., buying $2d$ units of inventory, which is twice as much as  $\offI(\inst)$ has at its disposal, yields a $1/2$-oversupplied instance; 
    \item The platforms face demand under which the CRs of their matching algorithms hold tightly, i.e., in an $\market$ over- or under-supplied instance, the number of matches that platforms A and B respectively expect is
    % \vspace{-.1in}
    \begin{align*}
    & \max \left \{ \frac{1}{1+\market}, \frac{\market}{1+\market} \right\} \cdot \left(d \cdot \min \left\{1, \frac{1}{\market}\right\} \right)=\frac{d}{1+\kappa}\\
    \text{and}\quad &\max \left\{\frac{1-e^{-\market}}{\market}, 1-e^{-\market} \right\} \cdot \left(d \cdot \min \left\{1, \frac{1}{\market}\right\} \right) = \frac{(1-e^{-\market}) d }{\market}.
    \end{align*} 
\item Each platform generates  $r$ revenue for each successful match and incurs a cost of $c<r$ for each unit of inventory.
\end{itemize}

The first two structural assumptions hold true in the family of  worst-case instances (see \Cref{section: impossibility}) we construct and the last holds true asymptotically when taking the limit in that family. For the revenue and cost, we adopt standard assumptions from inventory management. In practice, these assumptions are unlikely to jointly hold exactly true, but nonetheless they capture an abstraction that allows us to cleanly connect our CR results downstream to the upstream decision. 

The platforms need to decide how much inventory to order, relative to the demand.\footnote{We implicitly assume here that the platform only has one global supply lever as opposed to having the ability to add supply in a targeted manner. For volunteer or ridehailing platforms this is a reasonable assumption as platforms need to target potential new volunteers/drivers without access to detailed personal information. }
We let $d\inv$ denote the inventory order; based on the above, $\inv>1$ corresponds to creating a $1/\inv$-oversupplied instance and $\inv<1$ corresponds to creating a $1/\inv$-undersupplied instance. The choice of $\inv$ is based on (i) the cost of inventory $d\inv\cdot c$ and (ii) the revenue that the inventory translates to (via the matching algorithm's performance), i.e., $\frac{dr}{1+1/\inv}=dr\frac{\inv}{\inv+1}$ for platform A and $(1-e^{-1/\inv})\inv \cdot dr$ for platform B. Combining the cost and the revenue, we obtain the two platform's profit functions:
\begin{equation*}
\text{Profit}_{A}(\inv) = d\left(r \cdot \frac{\inv}{1+\inv} - \inv c\right),\quad
    \text{Profit}_{B}(\inv) = d\left((1-e^{-1/\inv})\inv \cdot r - \inv c\right).
\end{equation*}
As both profit functions are concave, we can identify the optimal $\inv$ for each platform by taking the derivative and setting it to 0 in each case. This yields for $A$ and $B$ respectively (see Appendix \ref{appendix: underage-costs}):
\begin{equation*}
        \inv_A^* = \frac{\frac{r-c}{r} - 1 + \sqrt{1 - \frac{r-c}{r}}}{1 - \frac{r-c}{r}} = \sqrt{\left( 1 - \frac{r-c}{r}\right)^{-1}} - 1 \quad \text{and}\quad \inv_B^* =  - \left(W_{-1}\left(-\frac{1}{e} \cdot \frac{r-c}{r}\right)+1 \right)^{-1},
    \end{equation*}
    where $W_{-1
    }$ stands for the negative branch of the Lambert $W$ function.\footnote{$W$ is the multivalued function that satisfies $x = W(x) \exp(W(x))$ for any $x$. For $x\in [-\frac{1}{e},0)$, there are two possible real values for $W(x)$; the branch with $W(x) \leq -1$ is referred to as the negative branch $W_{-1}(x)$.} The quantities reveal that, similar to the newsvendor setting, the optimal stocking decision, relative to the demand $d$, depends only on the ratio $(r-c)/r$.  To explore the behavior of the optimal stocking levels, we plot these levels (Figure \ref{fig:optimal_stocking_levels}) for both adversarial and stochastic settings. 
We first find a pronounced initial peak in stocking levels for stochastic arrivals (platform B); this is due to the exponential term in the profit expression and contrasts with a more gradual increase for adversarial arrivals. Second, as $(r-c)/r$ increases, the optimal stocking levels for platform B transitions to a linear behavior; in particular, the optimal stocking levels for both platforms intersect around a ratio $(r-c)/r$ of around 0.7839. {Lastly, it is worth comparing these decisions to naive upstream supply decisions that do not take into account the effect that imbalance has on the CR: assuming an $\alpha$-approximation, the profit-maximizing quantity would be either $1$ or $0$ with the threshold of $(r-c)/r$ occurring at $1-\alpha$, e.g., with $\alpha=1/2$ (the worst-case upper bound on non-adaptive algorithms in our setting), we would have $\eta_\alpha^*=0$ for $(r-c)/r<1/2$ and $\eta_\alpha^*=1$ for $(r-c)/r\geq 1/2$. In other words, the obtained stocking level could be 0 for $(r-c)/r<1/2$ though it would be profitable to hold inventory; it could be too large for $(r-c)/r\in(.5, .76)$ when $\eta_\alpha^*=1$ though a lower inventory would suffice (given the benefits of imbalance); and it could be too small for $(r-c)/r>.76$ when $\eta_A^*$ and $\eta_B^*$ are both greater 1 because the low cost of supply makes it worthwhile to create greater imbalance, but the non-parameterized CR does not adapt to that regime.
}

\emph{Comparison to adaptive algorithms.} 
    For $(r-c)/r \leq 0.606$ and $(r-c)/r \geq 0.831$, the resulting market imbalance values $\market(\inv)$  fall below 0.736 and exceed 1.358, respectively. In these regimes, our CRs are higher than 0.702 for stochastic instances and 0.576 for adversarial instances (see Figure \ref{fig:adversarial_stochastic_comparison}), which reflects the best-known CRs for (adaptive) algorithms (which require additional assumptions to achieve these CRs). Of course, these algorithms are likely to significantly outperform their CRs on instances that exhibit imbalance; as such, the takeaway of our analysis should not be we provide improved CRs relative to the literature, but rather that imbalance offers itself as a natural parametrization within which stronger CRs are obtainable.
% Third, 

\begin{figure}[h]
    \centering
    \begin{minipage}{0.48\textwidth}
        \centering
        \includegraphics[width=\textwidth]{images/optimal_stocking_levels_2.pdf}
        \caption{\small{Optimal stocking levels $\inv_A^*$ and $\inv_B^*$ as functions of $(r-c)/r$. The horizontal dotted lines correspond to $\inv\in\{0.736,1.358\}$. The vertical dotted lines correspond to their intersection with the optimal stocking level.}}
        \label{fig:optimal_stocking_levels}
    \end{minipage}\hfill
    \begin{minipage}{0.48\textwidth}
        \centering
        \includegraphics[width=\textwidth]{images/lower_bounds_2_3.pdf}
        \caption{\small{Comparison of our tight CRs for adversarial and stochastic settings across $\market$. The dotted lines intersect with the $x$-axis at $0.736$, $0.754$, $1.210$, and $1.358$ respectively.}}
        \label{fig:adversarial_stochastic_comparison}
        \vfill \hfill
    \end{minipage}
    \vspace{-.15in}
\end{figure}
