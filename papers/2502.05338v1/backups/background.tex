%\dimitra{Integrate that into the perforamnce part!Design Requirements for a Trusted NIC} \label{sec:background}

\if 0
\myparagraph{Requirements for security} In our \projecttitle{}, we materialize the theoretical principles of Clement et al.~\cite{clement2012} to design replication protocols for the modern cloud that meet the security and performance requirements of hosted applications without increasing the replication factor. This theoretical work sets the lower bound for transforming a CFT protocol into one that is Byzantine tolerant requiring the same replication degree as the CFT counterpart. Specifically, they have shown that such a translation {\em exists} if the following two properties are guaranteed:
\begin{itemize}
    \item {\bf{Transferable authentication.}} Potentially malicious nodes cannot impersonate other (honest) nodes. Essentially, any node can verify that a message is signed by the correct sender, even for forwarded messages.
    \item {\bf{Non-equivocation.}} A sender cannot send different messages to different nodes in the same round while it is supposed to send the same message according to the protocol.
\end{itemize}

In \projecttitle{} we guarantee non-equivocation by assigning a monotonically increased trusted hardware counter (or sequencer) to each message before its transmission~\cite{clement2012, levin2009trinc, minBFT}. This technique forces a total order on network messages of sender allowing other (honest) nodes to detect missing (past) messages. For the transferable authentication property, the theoretical translation leverages digital signatures~\cite{10.1007/978-3-540-87779-0_2}. In our \projecttitle{} we decided against them due to their high computational latency. Instead, as in~\cite{levin2009trinc} we used cryptographic MACs along with a specific id that is uniquely bound to each device. We explain the \projecttitle{} mechanisms in $\S$~\ref{subsec:tfpga}.

%According to Clement et al~\cite{clement2012} the translated protocol needs to be adapted to ensure consistency and correctness, i.e., that the message that is received is legitimate. We have shown fapplications 
\fi 



\subsection{Summary: Research Question}
We advocate that modern distributed systems must be fast and trustworthy, operating correctly under the BFT model that characterizes the modern cloud infrastructure. We further showed that while TEEs could help in this direction, they cannot meet the requirements of such systems in terms of variations in performance. At the same time, they also have overheads in programmability and their security analysis.

To this end, we pose the following research question:
{\em How to design a network abstraction for high-perfomant trustworthy distributed systems under BFT that meets the {\bf performance} requirements of modern systems while it formally guarantees their {\bf secure and correct execution in the heterogeneous (Byzantine) cloud infrastructure}?}
%\pramod{I think this section should end with a new subsection or overall question on high-level design requirements for building high-performance trustworthy distributed systems.}

%In our work, we extend the scope of the FPGA-based SmartNICS~\cite{u280_smartnics, alveo_sn1000} to extend the security semantics of the network operations they provide . We implement a minimal TCB on the NIC level to materialize the aforementioned security properties (i.e., hardware-assisted authenticated equivocation-free messages) on the network data path eliminating the device communication costs. Our design choice of leveraging SmartNICs  is not hypothetical; SmartNIC devices have already been launched by major cloud providers~\cite{alibaba_smartnics, nitro_smartnics, msr_smartnics}.

%because their programmable FPGA allowing us to design and implement on hardware the security properties we need while all network traffic is on the communication path optimizing for latency~\cite{fpga_netstack}. %As such, FPGA-based SmartNICs are the best choice for our 

%mateterializing the security properties at the NIC level trusted hardware would allow us to generate and verify hardware-assisted authenticated equivocation-free messages on the network data path eliminating the device communication costs.

%To achieve that our work leverages the 
%SmartNIC devices~\cite{liquidIO_smartnics, u280_smartnics, bluefield_smartnics, broadcom_smartnics, netronome_smartnics, alibaba_smartnics, nitro_smartnics, msr_smartnics} that have already showed their performance benefits in networked systems~\cite{211249, 10.1145/3341302.3342079} and have already been launched by major cloud providers~\cite{alibaba_smartnics, nitro_smartnics, msr_smartnics}. % have launched SmartNICs in their datacenters to offload tasks from the host CPU onto dedicated hardware. 

%These devices present great opportunities for high-performant networking because they offer a fully programmable hardware ``on-path'' with the NIC device, e.g., (multicore) (ARM) cores~\cite{bluefield_smartnics, alibaba_smartnics, broadcom_smartnics, liquidIO_smartnics} and FPGAs~\cite{u280_smartnics, alveo_sn1000, msr_smartnics}. As such, we rely on FPGA-based SmartNICs because their on-path programmable FPGA allowing us to design and implement on hardware the security properties we need while all network traffic is on the communication path optimizing for latency~\cite{fpga_netstack}. %As such, FPGA-based SmartNICs are the best choice for our 

%These SmartNICs can be categorized into three different hardware designs: {\em (1)} ASIC-based NICs~\cite{netronome_smartnics} that have dedicated hardware for common network processing functions (TCP checksum and segmentation, RSS, etc.), {\em (2)} (multicore) SoC-based NICs~\cite{bluefield_smartnics, alibaba_smartnics, broadcom_smartnics, liquidIO_smartnics} that have an on-chip set of embedded (ARM) cores, and {\em (3)} FPGAs~\cite{u280_smartnics, alveo_sn1000, msr_smartnics} that introduce a fully programmable hardware ``on-path'' with the NIC device. 

%All these hardware designs for SmartNICs present different characteristics in performance and programmability. The ASIC NICs provide the highest performance potential but they suffer from a lack of programmability and adaptability over time~\cite{211249}. Current ASIC NIC designs cannot guarantee the properties of non-equivocation and transferable authentication and as such they are not an option for \projecttitle{}. On the other hand, the SoC-based SmartNICs are able to support our \projecttitle{} security requirements because their on-chip cores have full OS support and are fully programmable. However, we only had access to Bluefield 2~\cite{bluefield_smartnics} cards where the embedded cores are ``off-path'' with no support for DMA transfers between the on-chip and host memory. As such, a network operation would involve an extra (intra-host) operation for forwarding the request from the application to the device cores increasing the operation's latency.

%\projecttitle{} is built on top of the FPGA-based SmartNICs because they are combining the best of both worlds. First, the FPGAs are fully programmable allowing us to design and implement on hardware the exact security properties the \projecttitle{} requires. Secondly, the FPGA resides on-path with the NIC allowing \projecttitle{} to process all incoming/outgoing network traffic on the communication path optimizing for latency~\cite{fpga_netstack}. %As such, FPGA-based SmartNICs are the best choice for our \projecttitle{} design.

%These cloud solutions build primarily on top of programmable FPGA-based SmartNICs for acceleration or SoC-based SmartNICs that integrate off-path arrays of wimpy ARM cores (up to 16) to improve programmability (at the cost of performance).

%Our \trustednic{} has been built on top of FPGA-based SmartNICs to optimize for performance. \trustednic{} is an on-path SmartNIC that processes all incoming/outgoing messages directly on the communication path, ensuring that their security properties are met. In contrast, off-path SoC-based SmartNICs, which do not offer any security guarantees out of the box, do not optimize for performance. For example, the host needs to send the outgoing messages to the SoC cores for further processing. As DMA transfers are not supported between host and SoC cores, the re-routing is resolved at the NIC level, incurring extra overheads.

%Overall, \trustednic{} is a simpler SmartNIC because it just involves an FPGA on the communication path as in~\cite{msr_smartnics}. %Commercial SmartNICs are complete Systems-on-Chip, involving far more hardware (not
%only arrays of cores but also dedicated ASICs for network
%processing, packet switching, memory controllers, potentially
%storage controllers, etc.). As such, they can do more, but they
%also cost more, require more energy, and are more complex
%to deploy, program, and use.
