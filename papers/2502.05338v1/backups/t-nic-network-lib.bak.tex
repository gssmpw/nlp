\section{Trusted Network Library}
\label{sec:t-nic-software}
The section provides an overview of the \projecttitle{} programming APIs ($\S$~\ref{sec:net-lib}) as well as an example of a generic protocol translation ($\S$~\ref{subsec:transformation}).

\subsection{Programming APIs}
\label{sec:net-lib}
\projecttitle{} implements a programming API that  resembles the traditional programming frameworks on top of the classical RDMA API that improve adoption~\cite{erpc}. Our API, shown in Table~\ref{tab:apis}, allows system designers to leverage modern RDMA-enabled networks as they have been proven to be the most efficient design in modern distributed systems\cite{f04eb9b864204bab958e72055062748c, farm, hermes, rdma-design, rdma-scale, octopus, Mitchell2013}. In addition, our \projecttitle{} API extends the security semantics of the traditional RDMA operations materializing the minimum necessary security properties required for robust replication, i.e.,  transferable authentication and non-equivocation (discussed in $\S$~\ref{sec:background}). 

%We specifically provide a unified abstraction where we designed and implemented peer-to-peer and group communication primitives that follow the standardized classical and widely-adopted~\cite{f04eb9b864204bab958e72055062748c, farm, hermes, rdma-design, rdma-scale, octopus, Mitchell2013} RDMA API.





%The local virtual address and the are also passed by the user.


\if 0
\begin{center}
\begin{table*}[ht]
\centering
\begin{tabular}{ |m{3.4cm}||m{12cm}|}
 \hline
 \multicolumn{2}{|c|}{{\bf Initialization API}} \\
 \hline
 \texttt{init\_local\_queue()} & \\
 \texttt{init\_connection()} & \dimitra{write me} \\
 \hline
 \texttt{send\_pk()} &  \\
 \hline
 \multicolumn{2}{|c|}{{\bf Network API}} \\
 \hline
 \texttt{remote\_write()} &  \\
 \texttt{remote\_read()} &  \\
 \texttt{send()} &  \\
 \hline
 \texttt{poll()}&  \\
  \hline
  \multicolumn{2}{|c|}{{\bf Security API}} \\
 \hline
 \texttt{check\_attestation()} &  \\
 \hline
 \end{tabular}
\caption{\projectlibrary{} API.}
\end{table*}\label{table:api}
\end{center}

Table~\ref{table:api} shows the core API implemented by our \projectlibrary{}.
\fi

\myparagraph{Initialization API}  Prior to exchanging network messages, the \projecttitle{} system needs to be configured by the application to establish RDMA connections. Specifically, the application creates one ibv struct for each peer-to-peer connection {\tt ibv\_qp\_conn()} which sets up and stores the queue pair, the local and remote IP addresses, and device UDP ports and others. In addition, the application needs to allocate the ibv memory that will be registered to the \projecttitle{} hardware. The code calls into the {\tt init\_lqueue()} that initializes the local queue pairs and  registers the ibv memory to the \projecttitle{}. Once the application configures its \projecttitle{} appropriately, it synchronizes itself with the remote side using the {\tt ibv\_sync()}. This function utilizes a common NIC to communicate with the remote machine and exchange important information such as its registered memory base address, queue pair numbers, etc. After all these steps are completed the application can use the \projecttitle{} network API to exchange data in a secure manner. %In addition, the application needs to exchange its registered memory base address ({\tt exchange\_qp()}) with a remote machine. 

\begin{table}
\begin{center}
\begin{tabular}{ |c|c| } 
 \hline
 \multicolumn{2}{|c|}{{\bf Initialization API}} \\ [0.5ex] \hline 
 {\tt ibv\_qp\_conn()} & Creates an ibv struct \\
 {\tt alloc\_mem()}    & Allocates host ibv memory \\
 {\tt init\_lqueue()}    & Registers local memory to \projecttitle{} \\
% {\tt exchange\_qp()} & Connects with remote IPs \\
 {\tt ibv\_sync()} & Exchanges ibv memory addresses \\
 %{\tt add\_qp()} &  $2f+1$ \\
 \hline
 \hline
 \multicolumn{2}{|c|}{{\bf Network API}} \\ [0.5ex] \hline 
 {\tt local\_send()}    & Generates an attested message \\
 {\tt local\_verify()}    & Verifies a message \\
 {\tt auth\_send()} &  Transmits an attested message \\
 {\tt poll()} &  Polls for incoming messages \\
 {\tt rem\_read/write()} &  Fetches/writes remote memory \\
 \hline
  %\hline
 %\multicolumn{2}{|c|}{{\bf Configuration API}} \\ [0.5ex] %\hline 
 %{\tt secrets()}    & (shared) keys' exchange \\
 %\hline
\end{tabular}
\end{center}
%\vspace{-10pt}
\caption{\projecttitle{} programming API.}
\label{tab:apis}
%\vspace{-18pt}
\end{table}




\myparagraph{Network API} Our network API adopts the programming paradigm of high-performant general libraries on top of RDMA~\cite{erpc} on top of the one-sided RDMA-verbs~\cite{rdma}. Specifically, \projecttitle{} executes shielded one-sided RDMA reliable operations that offer the same reliability guarantees as the classical one-sided RDMA over Reliable Connection (RC). The networking stack imposes a FIFO ordering (per connection), similar to TCP/IP networking.

We expose to the user the {\tt auth\_send()} operation which sends an attested message by implementing an RDMA reliable write of the message concatenated with its attestation. We also support the classical RDMA operations for reads and writes. The remote side can {\tt poll()} to fetch the number of the completed operations. In contrast to classical RDMA we have not used background threads for polling. The function returns the number of the completed operations. The poll function is only updated when the verification at the \projecttitle{} hardware has been successful. 

In addition, we implemented local operations where we generate or verify an attested message without transmitting in the network. The local attestations and verification's pass through the \projecttitle{} hardware and are copied back to host local memory.

In \projecttitle{}, we do not implement a hardware-assisted multicast. Even in the standard RDMA libraries, hardware-based multicast is barely used in practice~\cite{f04eb9b864204bab958e72055062748c} and is only supported only over unreliable connections. In \projecttitle{}, we can support equivocation-free multicast through uni-casts of the same attested message generated by {\tt local\_send()} as in~\cite{levin2009trinc}.

\if 0
Compared to classical, little-adopted, RDMA that only supports h/w multicast for unreliable connections~\cite{f04eb9b864204bab958e72055062748c}, \projecttitle{} allows for multicast reliable operations that prevent equivocation. However, we have not implemented multicast on hardware. Instead, multicasts in \projecttitle{} can be implemented through unicasts of the same attested message generated by {\tt local\_send()}.
\fi

%\myparagraph{Authentication API} All \trustedfpga{}s in the membership have exchanged keys as part of their initialization. Applications can request the public keys of their local and/or remote \trustedfpga{}s to verify the attestation of a received message (if it is required by the protocol).





\subsection{A Generic Transformation Recipe}
\label{subsec:transformation}
We now show how to use \projecttitle{} programming API to construct a generic transformation of a distributed system operating under the CFT model to a system that targets Byzantine settings.  Our transformation is defined as a set of wrapper functions on top of the send and receive operations~\cite{clement2012}. For safety, our transformation further needs to meet the following properties in order to provide the same guarantees that are expected by the original CFT system~\cite{clement2012, making_distributed_app_rob, 268272}.

\begin{itemize}
    \item \myparagraph{Safety} If the sender $S$ is correct and a correct receiver receives a message $m$ from $S$, then the sender $S$ must have executed send with $m$.
    \item \myparagraph{Integrity} If a correct receiver receives and delivers a message $m$, then $m$ is a valid (state machine) message.
\end{itemize}

Listing~\ref{lst:transformation} shows our proposed generic send and receive operations that we applied in a scenario between two nodes where the first node (sender) sends messages to the second node (receiver) that follows the sender. For each message the sender (executes the {\tt send}) needs to send the (client) msg, its current state (e.g., sender's output as a response to the current client msg) and  the follower's previous state known to the sender. The last piece of data is optional and can be used to ensure that both the sender and the receiver have the same view of the system (Listing~\ref{lst:transformation}:L4-6). The receiver, on the other hand, needs to {\em simulate} the leader's state and upon a valid message reception (Listing~\ref{lst:transformation}:L12-13), it needs to verify that the sender's output/state is the expected (Listing~\ref{lst:transformation}:L14). To ensure that the receiver is not lagging behind, it might also want to ensure that it is in the same view with the leader. To achieve so it verifies that its latest sent message has been received by the leader (Listing~\ref{lst:transformation}:L15-16).

\myparagraph{Correctness arguments for the transformation} Next we explain how our generic transformation satisfies the listed above requirements. First, the safety requirement is directly ensured by \projecttitle{} transferable authentication property. Since a faulty node cannot impersonate a correct node, if \projecttitle{} validates and delivers a message m from a sender, then the sender must have send m. Thanks to our  reliable network operations where the \projecttitle{} hardware will retry the send attempts (transparently to the application), liveness between correct nodes is ensured. 

%by attaching the corresponding  authentication tokens of pi’s inputs that led to sending the output message.
The integrity property requires the follower to obtain a guarantee that a message from the leader has been derived from a valid protocol execution scenario. In our transformation, the follower simulates the leader's state and jointly with the transferable authentication mechanism, we can always prove the validity of leader's output and state. Following this, the integrity requirement is met in the follower by replaying the client msg and verifying that the sender's output is legitimate.

For the case of the replication protocols we further require the following property.
\begin{itemize}
\item \myparagraph{Consistency (for replication protocols)} If correct replicas $R_1$ and $R_2$ receive valid messages $m_i$ and $m_j$  respectively from sender $S$, then either (a) $Bpg_{,i}$ is a prefix of $Bpg_{,j}$ , (b) $Bpg_{,j}$ is a prefix of $Bpg_{,i}$, or (c) $Bpg_{,i}$ = $Bpg_{,j}$ (where $Bpg_{,x}$ is the process behavior that supports the validity of message $m_x$).
\end{itemize}

The consistency requirement is seamlessly enforced  through the \projecttitle{} non-equivocation primitive. Our approach to prevent equivocation requires \projecttitle{} to  register a monotonic sequence number to each outgoing message. This enforces a total order on the leader's messages and along with the integrity requirement this is sufficient to consistently. Importantly, we ensure that correct followers/receivers will never miss any past messages. As such, when two followers receive from the same sender (using the equivocation-free multicast we discuss in $\S$~\ref{subsec:net_lib}) they will follow the exact same transition path. %By forcing the sender to also transmit all its state and previous known message from the receiver, the receiver can validate detect missing messages and inconsistencies in each communication step.


%Resultantly, when two receivers p1 and p2 hear
%from the same sender pg their behaviors corresponding to the
%sender pg will follow the exact same state transition path.
%However, as pg may not send all messages, a behavior of
%process (say) p1 can form a prefix of a behavior by process
%p2, and vice versa.

{\small
\begin{lstlisting}[frame=h,style=customc,
                    label={lst:transformation},
                    caption= Generic application send and recv wrapper functions for transformation using \projecttitle{}: blue sections (native) and orange sections (\projecttitle additions).]
// generic send wrapper function
void send(P_id, char[] msg) {
    // current state
@\Hilight@    state = hash(my_state);
@\Hilight@    to_be_sent = msg || state || prev_receiver_msg;
@\HilightYlinewidth@    auth_send(follower, to_be_send);
}

// upon receiveing a send(follower, msg) message
void recv(msg) {
@\HilightYlinewidth@ // message authenticity and integrity checked in TNIC hw
auto [attestation, msg || state || prev_receiver_msg] = deliver();
@\HilightYlinewidth@    [msg, cnt] = verify_msg(recv_msg);
@\Hilight@    verify_sender_state(state);
@\HilightYlinewidth@    local_verify(prev_receiver_msg);
@\Hilight@    verify_system_view(prev_receiver_msg);
@\Hilight@    apply(msg);
}
\end{lstlisting}
}