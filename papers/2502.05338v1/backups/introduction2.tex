\section{Introduction}
%as they present the {\em illusion} of a single  giant machine 
%\antonis{"giant machine" seems informal/general, maybe "large computer"?}
%with unlimited resources that {\em transparently} meet the applications' performance and resource requirements
Distributed systems are an integral part of the third-party cloud infrastructure~\cite{amazon_ec2, microsoft_azure, rackspace, google_engine}. While these systems in the cloud manifest in diverse forms, ranging from storage systems~\cite{dynamo, azure_storage, tao, spanner, 51, zippy, AmazonS3} and management services~\cite{Hunt:2010, Burns2016} to computing frameworks~\cite{aws_lambda, azure_functions, google_cloud_functions}, they all must be fast and remain correct when failures occur. %all built on top of networking libraries for communication.

%must be fast and trustworthy
%Undoubtedly, distributed systems . 
Unfortunately, the widespread adoption of the cloud has drastically increased the surface area of faults and attacks~\cite{Gunawi_bugs-in-the-cloud, Shinde2016, high_resolution_side_channels} that go beyond the traditional crash-stop fault model\cite{delporte}. The modern (untrusted) third-party cloud infrastructure severely suffers from arbitrary ({\em Byzantine}) faults~\cite{Lamport:1982} that can range from malicious (network) attacks to configuration errors and bugs that are capable of irreversibly disrupting the correct execution of the system~\cite{Gunawi_bugs-in-the-cloud, Shinde2016, high_resolution_side_channels, Castro:2002}.  %A such, distributed systems in the cloud need now to remain correct even under the presence of arbitrary ({\em Byzantine}) failures~\cite{ford2010availability, Mazieres2002b, Garay2000}.

%prior research has repeatedly showed that TEEs adoption can be complicated and challenging especially in the context of networked  applications ~\cite{avocado, treaty, minBFT,10.1145/3492321.3519568}. 

%Specifically,

A promising solution to build trustworthy distributed systems that remain correct even under the presence of\linebreak Byzantine failures is the {\em silicon root of trust}---specifically, the Trusted Execution Environments (TEEs)~\cite{cryptoeprint:2016:086, arm-realm, amd-sev, riscv-multizone, intelTDX}. While the TEEs offer a (single-node) isolated Trusted Computed Base (TCB),  we have identified three core challenges that render their adoption impractical for building trustworthy distributed systems.
%\dimitra{MAKE the challenges IN a more DIRECT, ABSTRACT WAY}
%% Programmability + heterogeneity -> integration

%\pramod{Could we highlight the keywords in these three design challenges?}

{\bf \em First, TEEs in the cloud are heterogeneous, introducing programmability and security challenges}. TEEs are CPU-dependant environments that come with variations in security~\cite{10.1145/3600160.3600169, 7807249, 10.1007/978-3-031-16092-9_7} and programmability properties~\cite{Baumann2014, scone, 10.1145/3079856.3080208, 10.1145/3460120.3485341, tsai2017graphene, Rkt-io}. Integrating just a few heterogeneous TEEs to build a trustworthy distributed system is a challenging engineering task~\cite{10.1145/3460120.3485341} and complicates the security and correctness analysis of the derived system. We overcome this challenge by designing a {\em host-agnostic} and {\em unified trusted} silicon root of trust at the NIC level, which exposes a trusted network library to resolve the programmability burden.

%Particularly, we desingoffer a {\em trusted  unified} \dimitra{How can we have host agnostic silicon root of trust? Solution we provide a network level abstraction unified, etc. because all machines go through Network... that is attached to the hardware network level} programming interface for the heterogeneous cloud infrastructure without relying on CPU-sided TEEs.

%\dimitra{minimalism + verification}
%% Security properties/vulnerabilities + security analysis -> minimalism
{\bf \em Secondly, TEEs come with a large TCB and an increased attack surface}. Prior research shows a correlation between TCB size and potential vulnerabilities~\cite{10.1145/3379469, 10.5555/1756748.1756832} whereas hundreds of security bugs have already been uncovered in state-of-the-art TEEs~\cite{10.1145/3456631}.  As such, modern TEEs are hard to be formally verified. Our work addresses the challenges by building a {\em minimalistic verifiable TCB}. Our TCB, residing at the NIC level hardware, is equipped with {\em the lower bound of security primitives} (e.g., the non-equivocation and transferable authentication) for building trustworthy distributed systems. We also verify that the offered security properties offer a verifiable silicon root of trust; in fact, we perform an end-to-end formal verification of our offered security and safety properties. 
%For example, we calculated that Intel TDX~\cite{intelTDX}, that ports within the trusted hardware the entire Linux kernel (v5.19~\cite{linuxlifecircle}), ``hardens'' at least 2K lines of (used) code, leading to a TCB of 19MB.


%\antonis{Last sentence here says how we address this problem (which is nice), but previous paragraph does not.}

%temincreases the surface of vulnerabilities and drastically complicates the security analysis of the system. Indeed, prior research has reported hundreds of SGX security flaws that CPU vendors continuously try to patch them by releasing revisions or even introducing new TEEs' architectures. 

%{\bf First}, {HETEROGENEOUS+PROMABABILITY CHALLENGES} TEEs cannot {\em naturally} guarantee correctness properties for the designed system across the heterogeneous machines in the cloud~\cite{}. TEEs are CPU-dependant instructions with varying security properties, threat model and vulnerabilities~\cite{}. As an example, Intel SGX~\cite{} offers different security properties from ARM TrustZone~\cite{}. Stitching all these heterogeneous TEEs together can be as challenging as error prone and violating system safety. 

%{\bf Secondly}, Hundrends of SGX security flaws and many revisions (AMD-sev-snp) and keep patching these things \dimitra{huge TCB->vulnernabliies -> security analysis} TEEs have a high TCB with different programming interfaces that not only does it deteriorate programmability but it complicates the security analysis of the derived distributed system. In this paper, we advocate that minimalism is the key to building secure and trustworthy systems. As such, we target to find a minimal abstraction for building trusted systems.

%\dimitra{two insignts -> (1) CFT-to-BFT and (2) network offloading}
{\bf \em Thirdly, TEEs report significant performance bottlenecks.} The overheads are especially exacerbated in the context of networked systems~\cite{avocado, treaty, minBFT,10.1145/3492321.3519568}--- the foundational building block in the core of any distributed system---due to their slow syscalls execution~\cite{hotcalls}. In our work we offer both security and performance based on two insights: first we materialize a transformation from CFT distributed systems to BFT distributed systems with the minimum required security properties of non-equivocation and transferable authentication. Secondly we offload the security computation at the NIC-level hardware, specifically we extend the scope of the state-of-the-art SmartNICs.


%Recent efforts in academia~\cite{10.1145/3341302.3342079, 10.1145/3476886.3477505, 9040000, 9912712} and industry~\cite{211249, alibaba_smartnics, nitro_smartnics, msr_smartnics} have established that offloading network processing to specialized hardware, i.e., SmartNICs~\cite{liquidIO_smartnics, u280_smartnics, bluefield_smartnics, broadcom_smartnics, netronome_smartnics}, is a promising design to boost performance. Inspired by this trend, we explore the SmartNICs' potential to further extend their scope in building fast and trustworthy distributed systems by offloading security-related processing at the hardware network level. 


%{\bf \em Lastly, TEEs are hard to verify}. As discussed the TEEs integrate a huge TCB that varies between different TEEs complicating its security analysis. In addition to that, while TEEs offer (remote) attestation to   verify the integrity of the TEE and its executing code, the attestation in TEEs (when offered) requires the executing code to be exposed to the TEEs' hardware provider (e.g., Intel Attestation Service~\cite{ias}) which might not be compliant  with any imposed privacy policies. In \projecttitle{} we overcome these limitations. While its offered security properties are decoupled from the application's code, we further formally verify the safety and security guarantees of our system.

%However, there are real-world application scenarios (e.g., proprietary algorithm~\cite{}) where the program itself need to comply with strict privacy policies and cannot be {\em openly} exposed to multiple parties. 

Overall, we seek to answer the following question: {\em What is the minimal host-agnostic high-performant silicon-root-of-trust for building trusted distributed systems in the untrusted heterogeneous cloud?}


Our contribution is the \projecttitle{}, a trusted NIC architecture for distributed systems. Our \projecttitle{} overcomes all the above challenges, offering a {\em unified minimalistic networking abstraction for highly-performant trusted distributed systems} \antonis{"heterogeneity" is not mentioned here.}. 
We design our \projecttitle{} on top of state-of-the-art SmartNICs, removing the dependencies on the CPU-based TEEs for trust. In addition, \projecttitle{} materializes a minimalistic TCB as the key component to design trusted distributed systems in the Byzantine cloud infrastructure. In fact, we opt for the opportunity to prove our argument and formally verify the safety and security guarantees of our \projecttitle{} architecture. %For performance, we offload \projecttitle{}'s TCB  along with the network-level protocol implementation in the specialized SmartNICs hardware. 



Our \projecttitle{} follows a three-layer design: {\em (a)} the \projecttitle{} hardware architecture that implements a trusted RDMA network stack for trusted network operations ($\S$~\ref{sec:t-nic-hardware}), {\em (b)} a trusted (software) network stack that operates as a middle layer between the hardware and the user application bypassing the kernel for performance ($\S$~\ref{sec:t-nic-network}) and {\em (c)} a trusted networking library that exposes programming APIs to system designers ($\S$~\ref{sec:t-nic-software}). 

We have implemented the \projecttitle{}'s hardware architecture on top of Alveo U280 FPGA-based SmartNICs~\cite{u280_smartnics}. Our core component is the \projecttitle{}'s {\em attestation kernel}, the minimal required hardware-assisted TCB that guarantees the necessary security properties across the network ($\S$~\ref{sec:requirements-ds}). In addition to \projecttitle{}'s attestation kernel, we offload to the FPGA an end-to-end implementation of the reliable one-sided RDMA protocol~\cite{rdma_specification} to boost performance. Leveraging \projecttitle{} trusted network library, we show a generic {\em recipe} to transform a distributed system operating under the fail-stop model for Byzantine settings ($\S$~\ref{subsec:transformation}). Lastly, we apply our \projecttitle{}-assisted transformation into four use-cases ($\S$~\ref{sec:use_cases})---specifically, a single-node trusted log system, a state-machine-replication (SMR) protocol, a chain replication system and an accountability protocol---showing the generality of our approach. 


We evaluate \projecttitle{} with a competitive state-of-the-art software-based network stack, eRPC~\cite{erpc}, on top of RDMA~\cite{rdma} and DPDK~\cite{dpdk} that utilizes TEEs to shield the network operations. Our evaluation shows that our \projecttitle{} offers 3$\times$---5$\times$ lower latency than the software-based network stack that relies on CPU-based TEEs for trust. In addition, we evaluate the \projecttitle{}-based versions of the four implemented systems against their (conventional) TEE-based versions. To show the generality of our approach, we implemented the TEE-based versions of the systems using two competitive state-of-the-art TEEs, the Intel SGX~\cite{intel-sgx} and AMD-sev~\cite{amd-sev}. Our evaluation shows that \projecttitle{} can improve the system's throughput by up to $6\times$ compared to their TEE-based implementations.

\pramod{Try to unify the contributions into three? and also use bold keywords/terms to highlight each contribution. -- Maybe we could map them to design contributions?}
In summary, we make the following contributions:
\begin{itemize}
    \item We designed and built \projecttitle{}, a trusted NIC architecture leveraging the state-of-the-art FPGA-based SmartNICs to boost performance while offering security for distributed systems in the untrusted cloud ($\S$~\ref{sec:t-nic-hardware}). 
    \dimitra{Hardware level implementation + minimalism taht allows verifiable + NIC level implementation tfor high performance}
    \item We offer a unified minimalistic high-performant trusted library to system designers ($\S$~\ref{sec:t-nic-software}) on top of a user-space network system stack ($\S$~\ref{sec:t-nic-network}) and we show how to use \projecttitle{} to build a generic transformation of distributed systems for Byzantine settings ($\S$~\ref{subsec:transformation}).
    \item We further demonstrate the generality of our approach in practice by implementing four different secure systems using \projecttitle{} ($\S$~\ref{sec:use_cases}). Specifically, we design and build a single-node trusted log system, an state-machine-replication (SMR) protocol, a chain replication system, and an accountability protocol that are fundamental building blocks in the cloud.
    \item We formally verify the safety and security guarantees of our \projecttitle{} architecture from the bootstrapping process to the normal-operation network operations ($\S$~\ref{subsec::formal_verification_remote_attestation}).
    
    %\item We designed and built \projecttitle{}, a trusted NIC architecture leveraging the state-of-the-art FPGA-based SmartNICs to boost performance while offering security for distributed systems in the untrusted cloud ($\S$~\ref{sec:t-nic-hardware}). 
    %\item We offer a unified minimalistic high-performant trusted library to system designers ($\S$~\ref{sec:t-nic-software}) on top of a user-space network system stack ($\S$~\ref{sec:t-nic-network}) and we show how to use \projecttitle{} to build a generic transformation of distributed systems for Byzantine settings ($\S$~\ref{subsec:transformation}).
    %\item We further demonstrate the generality of our approach in practice by implementing four different secure systems using \projecttitle{} ($\S$~\ref{sec:use_cases}). Specifically, we design and build a single-node trusted log system, an state-machine-replication (SMR) protocol, a chain replication system, and an accountability protocol that are fundamental building blocks in the cloud.
    %\item We formally verify the safety and security guarantees of our \projecttitle{} architecture from the bootstrapping process to the normal-operation network operations ($\S$~\ref{subsec::formal_verification_remote_attestation}).
%   
%    \item \pramod{this could be the concluding para of the intro before summarizing the contributions?} Lastly, we conduct an extensive evaluation of \projecttitle{} on real hardware ($\S$~\ref{sec:eval}). We evaluate \projecttitle{} against a competitive software-based network stack ($\S$~\ref{subsec:net_lib}). We further evaluate \projecttitle{} for practical systems: we compare our \projecttitle{}-based versions of our four implemented systems with their TEE-based implementations. To generalize our findings, we make use of two state-of-the-art TEEs, Intel SGX, and AMD-sev, showing that \projecttitle{} can improve the end-application performance up to $6\times$ w.r.t. when using TEEs.
    
\end{itemize}

