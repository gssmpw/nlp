% LaTeX template for Artifact Evaluation V20201122
%
% Prepared by 
% * Grigori Fursin (cTuning foundation, France) 2014-2020
% * Bruce Childers (University of Pittsburgh, USA) 2014
%
% See examples of this Artifact Appendix in
%  * SC'17 paper: https://dl.acm.org/citation.cfm?id=3126948
%  * CGO'17 paper: https://www.cl.cam.ac.uk/~sa614/papers/Software-Prefetching-CGO2017.pdf
%  * ACM ReQuEST-ASPLOS'18 paper: https://dl.acm.org/citation.cfm?doid=3229762.3229763
%
% (C)opyright 2014-2020
%
% CC BY 4.0 license
%

%\documentclass[sigplan,twocolumn]{acmart}

%\usepackage{hyperref}

%\newcommand{\projecttitle}{\textsc{tnic}\xspace}
%\newcommand{\scone}{\textsc{scone}\xspace}

%\begin{document}


%\special{papersize=8.5in,11in}

%%%%%%%%%%%%%%%%%%%%%%%%%%%%%%%%%%%%%%%%%%%%%%%%%%%%
% When adding this appendix to your paper, 
% please remove above part
%%%%%%%%%%%%%%%%%%%%%%%%%%%%%%%%%%%%%%%%%%%%%%%%%%%%

\appendix
\section{Artifact Appendix}

%%%%%%%%%%%%%%%%%%%%%%%%%%%%%%%%%%%%%%%%%%%%%%%%%%%%%%%%%%%%%%%%%%%%%
\subsection{Abstract}
Our artifacts include the \projecttitle{} codebase as well as the software artifact with the four \projecttitle{} applications, i.e., A2M, BFT, CR, and PeerReview. In addition, we provide the codebases of all the microbenchmarks we discuss in the paper including those of the TEE-based systems. Lastly, we attach the security proofs of \projecttitle{} system operations and attestation protocol based on Tamarin~\cite{tamarin-prover}. This appendix provides the necessary information to set up, build, and run the experiments we present in the paper.



\subsection{Artifact check-list (meta-information)}

%{\em Obligatory. Use just a few informal keywords in all fields %applicable to your artifacts
%and remove the rest. This information is needed to find appropriate reviewers and gradually 
%unify artifact meta information in Digital Libraries.}

{\small
\begin{itemize}
 % \item {\bf Algorithm: }
  \item {\bf Program:} \projecttitle{} hardware implementation codebase. \projecttitle{} software codebases that include the systems where \projecttitle{} has been applied (run in emulated hardware) and microbenchmarks (e.g., network benchmark). \projecttitle{}'s security proofs based on Tamarin~\cite{tamarin-prover}.
  \item {\bf Compilation:} Requires Vitis HLS~\cite{vitis-hls}, Vivado~\cite{vivado}, CMake, C++, Boost, eRPC~\cite{erpc}, DPDK~\cite{dpdk}, Tamarin~\cite{tamarin-prover}.
  %\item {\bf Transformations: }
  %\item {\bf Binary: }
  %\item {\bf Model: }
  %\item {\bf Data set: }
  \item {\bf Run-time environment:} Requires NixOS, 5.15.4, {\sc scone}~\cite{scone} (for SGX-based experiments).
  \item {\bf Hardware:} Requires Alveo U280 cards~\cite{u280_smartnics}, Intel(R) Core(TM) i9-9900K with Intel Corporation Ethernet Controllers (XL710) (or any other DPDK compatible NIC) and AMD EPYC 7413. 
  % \item {\bf Run-time state: }
  \item {\bf Execution:} The time of the experiments are configurable. Each of our experiments did not take more than 10 minutes. However, the compilation and synthesis phases of the \projecttitle{} hardware implementation might take up to 4 hours. 
  \item {\bf Metrics:} Throughput and latency
  %\item {\bf Output: }
  % \item {\bf Experiments:} 
  %\item {\bf How much disk space required (approximately)?: }
  %\item {\bf How much time is needed to prepare workflow (approximately)?: }
  %\item {\bf How much time is needed to complete experiments (approximately)?: }
  \item {\bf Publicly available:} Yes.
  \item {\bf Code licenses:} MIT License. \projecttitle{} doesn't use any external license.
  %\item {\bf Data licenses (if publicly available)?: }
  %\item {\bf Workflow framework used?: }
  \item {\bf Archived (DOI):} \url{10.5281/zenodo.14775354}
\end{itemize}
}

%%%%%%%%%%%%%%%%%%%%%%%%%%%%%%%%%%%%%%%%%%%%%%%%%%%%%%%%%%%%%%%%%%%%%
\subsection{Description}

\subsubsection{How to access}

The open-source version of the \projecttitle{} codebase can be found on GitHub at the following address:

\url{https://github.com/TUM-DSE/TNIC-main.git}

%{\em Obligatory}

\subsubsection{Hardware dependencies}
For AMD-SEV and \projecttitle{}-hardware setups, you need three machines with AMD EPYC 7413 CPU. Each machine is equipped with an Alveo U280 card~\cite{u280_smartnics} and one of every U280's QSFP28 ports connects to the 100Gbps network. For Intel SGX setups, you need machines with Intel(R) Core(TM) i9-9900K with Intel Corporation Ethernet Controllers (XL710) (or any other DPDK compatible NIC) for network connection. 

\subsubsection{Software dependencies}
The software build process involves building the low-level 
Linux kernel driver and the high-level user application layers.
All codebases run on top of NixOS, 5.15.4. We provide the appropriate \texttt{.nix} files to set up a \texttt{nix-shell} environment with all necessary system dependencies. 

The code has been built with \texttt{Makefile} and \texttt{cmake}. The applications, as well as the TEE-based code and application layer, are  written in C++17. We depend on Boost libraries and gflgas for
the parsing of the command line arguments. We rely on several other dependencies, which we explain in our README files, including; {\sc scone}~\cite{scone} for SGX-based experiments, Vivado~\cite{vivado} and Vitis HLS~\cite{vitis-hls} for building \projecttitle{} hardware, eRPC~\cite{erpc}, DPDK~\cite{dpdk}, and Tamarin~\cite{tamarin-prover}.

%\subsubsection{Data sets}

%\subsubsection{Models}

%%%%%%%%%%%%%%%%%%%%%%%%%%%%%%%%%%%%%%%%%%%%%%%%%%%%%%%%%%%%%%%%%%%%%
\subsection{Installation}
The artifact is linked to the repository as submodules. Each repository provides analytical instructions in their \texttt{README.md} files of how to build and run the binaries.

To build the \projecttitle{}'s hardware implementation, please follow the instructions provided in ~\cite{build-hardware-for-fpga}.

To build the software including the driver and the benchmarks, please follow the instructions in~\cite{build-software}.

To run the experiments for the \projecttitle{} hardware implementation, you need to first load the \projecttitle{}'s kernel module and then run the compiled binary. Detailed instructions are available in ~\cite{tnic-run}.  

Similar instructions have been documented for the applications~\cite{tnic-apps} and the security proofs~\cite{tnic-proofs}.

%{\em Obligatory}

%%%%%%%%%%%%%%%%%%%%%%%%%%%%%%%%%%%%%%%%%%%%%%%%%%%%%%%%%%%%%%%%%%%%%
%\subsection{Experiment workflow}

%%%%%%%%%%%%%%%%%%%%%%%%%%%%%%%%%%%%%%%%%%%%%%%%%%%%%%%%%%%%%%%%%%%%%
\subsection{Evaluation and expected results}
Each of the experiments will output information about its progress; this is a hint that the script is still
running and hasn’t halted. The output of the experiment reports important measurements about the execution. The results are expected not to vary significantly (less than 5$\%$) when compared to the results presented in the paper. However, as discussed, we observed quite a significant variance in some TEE-based systems (Intel SGX and AMD-SEV).
%{\em Obligatory}

%%%%%%%%%%%%%%%%%%%%%%%%%%%%%%%%%%%%%%%%%%%%%%%%%%%%%%%%%%%%%%%%%%%%%
%\subsection{Experiment customization}
%A wide variety of experiemnt customization can be available
%through different execution parameters. The users can cre-
%ate different versions of the system through combinations of
%vFPGAs, network and memory stacks.
%%%%%%%%%%%%%%%%%%%%%%%%%%%%%%%%%%%%%%%%%%%%%%%%%%%%%%%%%%%%%%%%%%%%%
%\subsection{Notes}

%%%%%%%%%%%%%%%%%%%%%%%%%%%%%%%%%%%%%%%%%%%%%%%%%%%%%%%%%%%%%%%%%%%%%
\subsection{Methodology}

Submission, reviewing, and badging methodology:

\begin{itemize}
  \item \url{https://www.acm.org/publications/policies/artifact-review-badging}
  \item \url{http://cTuning.org/ae/submission-20201122.html}
  \item \url{http://cTuning.org/ae/reviewing-20201122.html}
\end{itemize}

%%%%%%%%%%%%%%%%%%%%%%%%%%%%%%%%%%%%%%%%%%%%%%%%%%%%
% When adding this appendix to your paper, 
% please remove below part
%%%%%%%%%%%%%%%%%%%%%%%%%%%%%%%%%%%%%%%%%%%%%%%%%%%%

%\bibliographystyle{ACM-Reference-Format}
%\bibliography{sample}
%\end{document}
