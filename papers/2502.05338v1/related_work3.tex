\subsection{\revcont{Discussion}}
\myparagraph{\rev{B2}{\projecttitle{}'s applicability}}
\revcont{As FPGA-based SmartNICs are widely adopted by major cloud providers for hardware acceleration~\cite{211249}, we believe that \projecttitle{} has the potential for broader industry application. In addition,  ASIC-based NICs can also provide the same functionalities by integrating \projecttitle{}'s hardware modules into an optimized System-on-Chip (SoC).}
%\revcont{As FPGA-based SmartNICs are widely adopted by major cloud providers for hardware acceleration~\cite{211249}, we believe that \projecttitle{} has the potential for broader industry application. In addition, \projecttitle{}’s APIs are generic and CPU-agnostic, making them more suitable for cloud environments than programming heterogeneous TEEs.}

%\revcont{\projecttitle{} demonstrates its minimalistic root of trust, which suffices for building BFT distributed systems. 
%ASIC-based NICs can also provide the same functionalities by integrating \projecttitle{}'s hardware modules into an optimized System-on-Chip (SoC). }
%which would be more optimized than the FPGA-based approach

\myparagraph{\rev{C1}{Use cases}}
\revcont{The paper deliberately focuses on distributed cloud applications as \projecttitle{}'s primary use cases. Trust in shared third-party clouds is a more critical concern than in other environments, posing unique challenges in trust, performance, and scalability. While the current scope is specific, the underlying principles could extend to other use cases, such as HPC or on-premise computing.}
%However,  \projecttitle{} addresses these using SmartNICs, which are already widely deployed in modern clouds~\cite{211249}. 


\myparagraph{\rev{D3}{Message drops}}
\revcont{
\projecttitle{} guarantees packet retransmission between two correct nodes until their successful reception  extending a RoCE implementation that supports reliable operations. The application need not re-send the message as it receives a different sequence number, which is not accepted (or verified) by the remote \projecttitle{} until all preceding messages have been received. %Instead, \projecttitle{} extends a RoCE implementation that supports reliable operations where the remote side acknowledges the receipt of a message using specific ACK messages. 
% \projecttitle{} guarantees packet retransmission between two correct nodes until their successful reception. The application need not re-send the message as it would receive a different sequence number, which would not be accepted (or verified) by the remote \projecttitle{} until all preceding messages have been received. Instead, \projecttitle{} extends a RoCE implementation that supports reliable operations where the receipt of a message is acknowledged by the remote side using specific ACK messages.
}

\myparagraph{\rev{D4, E3}{View-change and recovery}}
\revcont{Detailing view-change and recovery in \projecttitle{} protocols are beyond the scope of our work. \projecttitle{} could adopt similar techniques as in TrInc~\cite{trinc} without disrupting these operations. In a new leader's election, replicas can establish new connections with new identifiers. As such, previous connections will not block execution, and state transfers, e.g., view-change, can be performed effectively. } 

%\revcont{Detailing view-change and recovery in \projecttitle{} protocols are beyond the scope of this paper, while \projecttitle{} could adopt techniques such as the ones used by TrInc~\cite{trinc}. Importantly, \projecttitle{} does not interfere with these operations. During a new leader's election, replicas can establish new connections with new connection identifiers. As a result, previous connections will not block execution, and state transfers (e.g., view-change) can be carried out effectively. } 
% \rev{D4, E3}{We do not implement view-change or recovery in \projecttitle{} protocols, while \projecttitle{} does not interfere with these operations. During a new leader's election, replicas can establish new connections with new connection identifiers. As a result, previous connections will not block execution, and state transfers (e.g., view-change) can be carried out effectively. For recovery, \projecttitle{} could adopt techniques such as the ones used by TrInc~\cite{trinc}. } 

% \myparagraph{\rev{E1}{Impact of RDMA on security}}
\myparagraph{\rev{E1}{PCIe transaction encryption}}
\revcont{\projecttitle{} encrypts PCIe transactions for CPU-to-device communication, allowing attackers to modify the PCIe transactions. This vulnerability is not unique to \projecttitle{}; it applies to any network stack, including the OS-based ones, since the \emph{untrusted} OS drives PCIe transactions. }

\section{Related work}
%\pramod{the related work bucket should not start with individual papers. It should first summarize the field/area/bucket area, and then give example systems, and lastly, it should end with an overall comparison of the the research field with our solution. In this way, you avoid direct comparison with individual work. Can you please re-write it?}

\myparagraph{Trustworthy distributed systems} Classical BFT systems~\cite{Castro:2002, Suri_Payer_2021, DBLP:journals/corr/abs-1803-05069, Chan2018PaLaAS, DBLP:journals/corr/abs-1807-04938, Chan2018PiLiAE, bft-smart, 6681599} provide BFT guarantees at the cost of high complexity, performance, and scalability overheads. \projecttitle{} bridges the gap between BFT and prior limitations, designing a {\em silicon root-of-trust} with generic trusted networking abstractions that materialize the BFT security properties.

\myparagraph{Trusted hardware for distributed systems} Trustworthy systems~\cite{10.1145/3492321.3519568, minBFT, 10.1145/3552326.3587455, 10.1145/3492321.3519568, treaty, avocado, ccf} leverage trusted hardware to optimize the performance of classical BFT at the cost of generalization and easy adoption. The systems suffer from high latencies (50us---105ms)~\cite{levin2009trinc, 10.1145/2168836.2168866}, build large TCBs~\cite{treaty, avocado}, and rely on specific TEEs~\cite{minBFT, hybster}. In contrast, our \projecttitle{} aims to offer performance and generality, while our minimalistic TCB is verifiable and unified in the heterogeneous cloud. 

\begin{figure}
    \centering
    \includegraphics[width=0.9\linewidth]{eval-plots/plots/hw_eval_attest_util.pdf}
    \vspace{-10pt}
  \caption{\rev{(d)}{The scalability analysis of \projecttitle{} hardware. The resource usage is normalized to the U280 FPGA capacity.} }
  % \caption{Latency evaluation of send operations for various packet sizes across five competitive network stacks with various security properties.}
    % \felix{Figure: BRAM = RAMB36 (including RAMB18) + URAM. "CARRY8" is the 4th most used resource of the attestation kernel and is used for "Others" as an upper bound for all other resources.}
    % More resource utilization details: https://github.com/dgiantsidi/replication-protos/blob/main/plots/asplos_submission/attestation_kernel_util.md
    \label{fig:scalability}
   %\vspace{-2pt}
\end{figure}

\myparagraph{SmartNIC-assisted systems} Networked systems offer fast network operations with emerging (programmable) SmartNIC devices~\cite{liquidIO_smartnics,u280_smartnics,bluefield_smartnics,broadcom_smartnics,netronome_smartnics,alibaba_smartnics,nitro_smartnics,msr_smartnics}. Some of them offload the network functions to the hardware and reduce the host processing and energy overheads~\cite{246498,211249,10.1145/3387514.3405895,10.1145/3365609.3365851,10.1145/3127479.3132252,258971,246486,179716,227655,10.1145/3286062.3286068,shan2022supernic,10.1145/3390251.3390257} or re-design generic networking protocols, from RDMA/RoCE to TCP/IP network stacks, on top of FPGA-based SmartNICs for performance~\cite{coyote,corundum,storm,8891991,280712,9114811,opennic_project}. Others~\cite{10.1145/3341302.3342079,10.1145/2872362.2872367,234944,9220629,6853195,10292786,10.1145/3477132.3483555,280678,10.1145/3132747.3132756,honeycomb,288659,10329593} build generic execution frameworks to optimize various distributed systems. Our \projecttitle{} follows a similar approach by building a high-performant unified network stack with SmartNICs and extending its security semantics with the properties of non-equivocation and transferable authentication.% and offloading security to the NIC hardware.


%\myparagraph{SmartNIC-assisted network stacks} SmartNICs effectively provide high-performance network stacks. Another line of research~\cite{coyote, corundum, storm, 8891991, 280712, 9114811, opennic_project} re-designs generic networking protocols, from RDMA/RoCE to TCP/IP network stacks, on top of FPGA-based SmartNICs for performance. Our \projecttitle{} further extends its security semantics with the properties of non-equivocation and transferable authentication. 

\myparagraph{Programmable HW for network security} Programmable hardware, SmartNICs, and switches are used to shield networking. Recent systems~\cite{10.1145/3603269.3604874, 10.1145/3620678.3624786, 10.1145/3563647.3563654, 10.1145/3321408.3323087, 278292} leverage programmable switches and FPGAs to offload security processing and boost performance in the context of blockchain systems~\cite{10.1145/3603269.3604874} or security functions (e.g., access control, DNS traffic inspection)~\cite{10.1145/3620678.3624786, 10.1145/3563647.3563654, 10.1145/3321408.3323087}. Our \projecttitle{} similarly offloads security into the hardware, but it carefully uses SmartNICs to overcome the processing bottlenecks of the switches. 





