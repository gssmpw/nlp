% \section{Supplementary Material}
% \label{sec:appendix}
% As part of the appendix, we present (1) formally verified proofs for the \projecttitle security protocols and  (2) the implementation details of four distributed
% systems using \projecttitle.

\subsection*{Acknowledgement}
This work was supported by an ERC Starting Grant (ID: 101077577), the Intel Trustworthy Data Center of the Future (TDCoF), and the Chips Joint Undertaking (JU), European Union (EU) HORIZON-JU-IA, under grant agreement No. 101140087 (SMARTY).
% This project has also received funding from the CHIPS Joint Undertaken as part of the European Union's Horizon Europe research and innovation programme, SMARTY Project, grant agreement No. 101140087. 
The authors acknowledge the financial support by the Federal Ministry of Education and Research of Germany in the programme of "Souverän. Digital. Vernetzt.". Joint project 6G-life, project identification number: 16KISK002. 
%This work was also supported by the Intel Trustworthy Data Center of the Future (TDCoF) project. 

\if 0

\section{Discussion}
%\pramod{maybe instead of general conclusion - we can write this section as two paragraphs: 1) Discussion on generality (see the last sentence), and 2) limitations.}
% \pramod{Please add citations and check it.}

\underline{\bf Firstly}, we implemented \projecttitle{} with FPGA-based SmartNICs, i.e., Alveo U280~\cite{alveo_smartnics}; we believe other commercial SmartNIC vendors (e.g., Mellanox/Intel) can implement our minimalistic and generic NIC-level interface. \underline{\bf Secondly}, the emergence of SPDM/TDISP protocols~\cite{spdm,tdisp} to secure I/O devices in conjunction with confidential virtual machines (CVMs) as offered by a new generation of CPUs, e.g., AMD-SEV-SNP~\cite{amd-sev}, Intel TDX~\cite{intelTDX2}, and Arm CCA~\cite{arm-cca}, could adapt our work by augmenting \projecttitle{} with an "SPDM-broker" to build an E2E secure design since \projecttitle{} already supports remote attestation and can also support an encrypted channel between a CVM and the target NIC. \underline{\bf Lastly}, the PCIe-enabled CXL standard~\cite{cxl} is pitched to include secure extensions in the PCIe 6.0 specs~\cite{pcie6.0}. We believe \projecttitle{}'s architecture can significantly influence the design of future ``secure CXL bridges" because our minimalistic interface can also be realized on a CXL-expander card.

\fi

%We present \projecttitle{}, a trusted NIC architecture that offers a host CPU-agnostic unified security architecture based on a minimal and formally verified silicon root-of-trust with low TCB. Our \projecttitle{} exposes just the two fundamental properties of transferable authentication and non-equivocation that suffice to efficiently transform distributed systems under the fail-stop operation model for untrusted (Byzantine) cloud environments. N



%In this paper, we present the design and implementation of \projecttitle{}, a trusted NIC architecture. \projecttitle{} proposes a host CPU-agnostic unified security architecture based on trustworthy network-level isolation. We materialize the \projecttitle{} architecture using SmartNICs by exposing a minimal and formally verified silicon root-of-trust with low TCB, relying on just two fundamental properties of transferable authentication and non-equivocation. Using \projecttitle{}, we implement a hardware-accelerated trustworthy network stack that efficiently transforms a range of distributed systems under the fail-stop operation model for untrusted (Byzantine) cloud environments. Notably, we realize the \projecttitle{} architecture-based FPGA-based SmartNICs on Alveo U280 cards; we believe other commercial SmartNIC vendors can implement our NIC-level interface.


%\projecttitle{} outperforms the state-of-the-art CPU-based trusted hardware, TEEs, while offering a unified, general, programmable, formally verified, and minimalist interface to build trustworthy distributed systems. 
%\myparagraph{Ethical considerations and artifact availability} This work raises no ethical issues and will be open-sourced.% The \projecttitle{} (HW and SW) artifact and formal security proofs will be publicly available.

% We present \projecttitle{}, a trusted NIC architecture for trustworthy distributed systems. \projecttitle{} leverages the state-of-the-art network hardware, i.e., SmartNICs and extends their security properties, implementing a minimalistic TCB, the attestation kernel that materializes the lower bound of properties---the non-equivocation and transferable authentication properties---to transform distributed systems for Byzantine settings. We formally verified the security and safety properties of \projecttitle{} while we also applied it to transform four systems: a single-node trusted log system, a state-machine-replication (SMR) protocol, a chain replication system and an accountability protocol. Our evaluation shows that \projecttitle{} outperforms the state-of-the-art CPU-based trusted hardware, TEEs while offering unification, generality and minimalism.



% \section{Conclusion}
% In this paper, we present the design and implementation of \projecttitle{}, a trusted NIC architecture. Our, we make the following contributions:
% \begin{itemize}[leftmargin=*]
%     \item We designed and built \projecttitle{}, a {\em host-agnostic unified trusted high-performant} NIC architecture for building fast and trustworthy systems in the heterogeneous cloud infrastructure. We expose a trusted network programming library that offers the powerful properties of non-equivocation and transferable authentication to system designers while we show {\em how} to strengthen existent systems with \projecttitle{}.
%     %\dimitra{Hardware level implementation + minimalism taht allows verifiable + NIC level implementation tfor high performance}
%     \item We offer a {\em unified minimalistic verifiable silicon root of trust (TCB)}  that realizes an efficient transformation between existing system under the fail-stop operation model for the Byzantine settings in the cloud. We further conduct an end-to-end formal verification of our \projecttitle{} operations from bootstrapping and remote attestation to normal operation.
%     \item We built \projecttitle{} extending the scope of the state-of-the art SmartNIC devices to boost performance. We built our minimalistic TCB, \projecttitle{} attestation kernel, in NIC-level hardware by extending and shielding an FPGA-based RDMA network stack.
    
% \end{itemize}

%  We formally verified the security and safety properties of \projecttitle{} while we also applied it to transform four distributed systems: Attested Append-Only Memory (A2M), Byzantine Fault Tolerance (BFT), Chain Replication, and Accountable Distributed Systems using PeerReview. Our evaluation shows that \projecttitle{} outperforms the state-of-the-art CPU-based trusted hardware, TEEs while offering unification, generality and minimalism.