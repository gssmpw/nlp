\section{Motivation and Background}
\label{sec:requirements-ds}

We first examine the design requirements for high-performance, trustworthy distributed systems for cloud environments. % hosted in the untrusted heterogeneous cloud infrastructure.

\subsection{Trustworthy Distributed Systems}\label{subsec:trustworthy_ds}
\myparagraph{Byzantine fault model} 
In the untrusted cloud infrastructure, arbitrary (Byzantine) faults are a frequent occurrence in the wild~\cite{Gunawi_bugs-in-the-cloud, Shinde2016, 10.1145/1189256.1189259, 10.5555/1267308.1267318}. To this end, system designers introduced Byzantine Fault Tolerant (BFT) systems that remain correct even under the presence of (a bounded number of) Byzantine failures~\cite{Lamport:1982}. \rev{(b)}{Traditional BFT protocols need \emph{at least} $3f+1$ nodes in order to provide consistent replication while tolerating up to $f$ Byzantine failures.} While BFT accurately captures the realistic security needs in the cloud~\cite{bft_made_practical}, it is rarely adopted in practice~\cite{bftForSkeptics} due to its complexity and limited performance~\cite{268273, 10.1145/2658994}. 

\myparagraph{Crash fault model} 
The vast majority of cloud applications operate under the fail-stop (crash fault) model~\cite{spanner, 27897, cockroachdb_raft, zippydb, foundationdb}, optimistically {\em assuming} that the entire cloud infrastructure is trusted and only fails by crashing~\cite{delporte}. \rev{(b)}{Compared to BFT replication, Crash Fault Tolerant (CFT) protocols~\cite{10.1145/279227.279229, raft, primary-backup, Hunt:2010}, require $2f+1$ replicas to tolerate $f$ (yet non-Byzantine) failures.} While CFT systems can offer performance and scalability~\cite{f04eb9b864204bab958e72055062748c}, they are fundamentally incapable of ensuring safety in the presence of non-benign faults, hence, are ill-suited for the modern cloud. 

% In the untrusted cloud infrastructure, arbitrary (Byzantine) faults are a frequent occurrence in the wild~\cite{Gunawi_bugs-in-the-cloud, Shinde2016, 10.1145/1189256.1189259, 10.5555/1267308.1267318}. To this end, system designers introduced Byzantine Fault Tolerant (BFT) systems that remain correct even under the presence of (a bounded number of) Byzantine failures~\cite{Lamport:1982}. While BFT accurately captures the realistic security needs in the cloud~\cite{bft_made_practical}, it is rarely adopted in practice~\cite{bftForSkeptics} due to its complexity and limited performance~\cite{268273, 10.1145/2658994}. In contrast, the vast majority of cloud applications operate under the fail-stop (crash fault) model, optimistically {\em assuming} that the entire cloud infrastructure is trusted and only fails by crashing~\cite{delporte}. While Crash Fault Tolerant (CFT) systems usually offer performance and scalability~\cite{f04eb9b864204bab958e72055062748c}, they are ill-suited for the modern cloud as they are fundamentally incapable of ensuring safety in the presence of non-benign faults. 
 
% \noindent{\bf{Security properties for BFT.}} 

\myparagraph{Security properties for BFT} 
\rev{(b), A2, A4}{
We seek to build BFT systems while reducing their programmability and performance overheads. Our approach, inspired by the theoretical findings of Clement et al.~\cite{clement2012}, {\em transforms} CFT systems into BFT systems by providing the {\em lower bound} of security properties, i.e., {\em transferable authentication} and {\em non-equivocation}.
}

% \rev{A2}{We next explain the properties:}
% , which are minimal security properties required to build trustworthy systems under the BFT model. 

% \myparagraph{Transferable authentication}
\revcont{
We next explain the two security properties. First, {\em transferable authentication} allows a node to verify the original sender of a received message, even if it is forwarded by other than the original sender. Assuming that the sender $p_i$ sends an authenticated message $m$ to a recipient $p_j$, the authenticated message $m$ is accompanied by an authentication token $\sigma (p_i)$ that allows  $p_j$ to verify that $p_i$ generated the message, e.g., {verify($m, \sigma (p_i)$)}. Authentication tokens are unforgeable:
\begin{itemize}[leftmargin=*]
  \item if $p_i$ is correct, then {verify($m, \sigma (p_i)$)} is true if and only if $p_i$ generated $m$.
  \item if $p_i$ is faulty, {verify($m, \sigma (p_i)$)} $\wedge$ {verify($m', \sigma (p_i)$)} $\Rightarrow$ $m = m'$. As such, a compromised $p_i$ cannot produce two valid different messages that can be verified with the same token $\sigma (p_i)$.
\end{itemize}
As an authentication token is transferable, it allows another recipient $p_k$ to evaluate {verify($m, \sigma (p_i)$)} in the same way even when $m$ and $\sigma (p_i)$ are forwarded from $p_j$.
}

\revcont{
Second, {\em non-equivocation} guarantees that a node cannot make conflicting statements to different nodes. Equivocation also manifests as network adversaries or replay attacks that send invalid messages or re-send valid but stale messages.
}

\revcont{The seminal paper~\cite{clement2012} proves that, given these two properties, a transformation from any CFT protocol to a BFT protocol is {\emph {always}} possible without increasing the number of participating nodes; e.g., a reliable broadcast can be implemented to tolerate up to $f$ Byzantine failures in an asynchronous system with $2f+1$ replicas, rather than the conventional $3f+1$.}
% An authentication token provides transferable authentication if the correct processes $p_j$ and $p_k$ always evaluate \texttt{verify($m, \sigma (p_i)$)} in the same way even when $p_k$ receives message $m$ and authentication token $\sigma (p_i)$ from $p_j$.

% To sum up, providing these two properties at the network level, we can {\em always} and {\em correctly transform} (any) CFT distributed system to operate in the BFT model~\cite{clement2012, byzantine-pratical}. 

\if 0
\noindent{\bf{Security properties for BFT.}} We seek to offer BFT while reducing its programmability and performance overheads. As such, we materialize the {\em minimum} security properties required to build trustworthy systems under the BFT model~\cite{clement2012}: 
\begin{itemize}[leftmargin=*]
    \item {\bf Transferable authentication} refers to a machine's capability to verify the original sender of a received message, even if it is forwarded by other than the original sender. %Authentication is transferable if the original sender can also be verified for forwarded messages. 
    \item {\bf Non-equivocation} guarantees that a node cannot make conflicting statements to different nodes. Equivocation also manifests as network adversaries or replay attacks that send invalid messages or re-send valid but stale messages.
\end{itemize}
\fi

\subsection{High-Performance Distributed Systems} \label{subsec::tees}
%The security properties discussed above suffice for building distributed systems that operate {\em correctly} under the BFT model. 
The aforementioned two security properties are sufficient to {\em correctly transform} (any) CFT distributed system to operate in the BFT model~\cite{clement2012, byzantine-pratical}. 
However, a fundamental design trade-off exists between efficiency and robustness for practical deployments in the cloud. Our work aims to resolve this tension.

\myparagraph{Trusted hardware for BFT} System designers established trusted hardware, TEEs, as the most effective way to eliminate a system's Byzantine counterparts~\cite{avocado, minBFT, hybster, 10.1145/3492321.3519568}. While TEEs can be used to offer BFT, prior research illustrated significant performance and architectural limitations in the context of networked systems~\cite{avocado, 10.1145/3492321.3519568, hybster, minBFT}. Based on performance and security studies~\cite{9460547, 9935045}, TEEs' overheads in the heterogeneous cloud, in addition to their heterogeneity in programmability and security guarantees, are incapable of offering high-performant trusted networking under the BFT model. 


\myparagraph{SmartNICs for high-performance and BFT} We leverage the state-of-the-art hardware-level networking accelerators, i.e., SmartNICs~\cite{liquidIO_smartnics, u280_smartnics, bluefield_smartnics, broadcom_smartnics, netronome_smartnics, alibaba_smartnics, nitro_smartnics, msr_smartnics}, to address the trade-off between performance and security, overcoming the limitations of TEEs. Our design choice of leveraging SmartNICs is not hypothetical; SmartNIC devices have already been launched by major cloud providers~\cite{alibaba_smartnics, nitro_smartnics, msr_smartnics}, presenting great opportunities for performance thanks to their integrated fully programmable hardware (e.g., ARM cores~\cite{bluefield_smartnics, alibaba_smartnics, broadcom_smartnics, liquidIO_smartnics}, FPGAs~\cite{u280_smartnics, alveo_sn1000, msr_smartnics}). Precisely, we rely on two promising directions: {\em(1)} security and network processing offloading at the NIC-level hardware and {\em(2)} an efficient transformation for BFT. 

\if 0
We extend the scope of FPGA-based SmartNICs~\cite{u280_smartnics, alveo_sn1000} by offloading an RDMA protocol implementation to the FPGA and extending its security properties, offering non-equivocation and transferable authentication. 
% Our system not only leverages hardware acceleration for fast, trusted networking, 
Our system not only leverages hardware acceleration for performance, but {\em seamlessly} offers the foundations of a scalable transformation of distributed systems for BFT. These properties also guarantee that a CFT-to-BFT transformation for State-Machine-Replication (SMR) {\em always exists} with the same replication factor of the original CFT system~\cite{clement2012, byzantine-pratical} (2$f$+1), offering better scalability and less message complexity than the traditional BFT (3$f$+1). %In simple words, one can have BFT guarantees ensuring safety for up to $f$ Byzantine faults with $2f+1$ nodes as in the original CFT protocol; $f$ fewer nodes compared to the traditional (non-transformed) BFT protocols ($3f+1$ nodes).
\fi 

%In fact, the properties of the non-equivocation and transferable authentication suffice to transform even for BFT state machine replication (SMR) systems. They guarantee that a CFT-to-BFT transformation for SMR {\em exists} with the same replication factor of the original CFT system~\cite{clement2012, byzantine-pratical}, offering better scalability than the traditional. In simple words, one can have BFT guarantees ensuring safety for up to $f$ Byzantine faults with $2f+1$ nodes as in the original CFT protocol; $f$ fewer nodes compared to the traditional (non-transformed) BFT protocols ($3f+1$ nodes).
