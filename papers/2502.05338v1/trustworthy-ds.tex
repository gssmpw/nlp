\section{Trusted Distributed Systems}
\label{sec:use_cases}

%\pramod{@Dimitra - please make a carefull pass over the text, please cite all related seminal papers, and also explain how TNIC helps in improving these systems explicitly. }

Using \projecttitle{}, we transform the following four distributed systems into BFT systems (see Appendix $\S$~\ref{sec:use_cases-appendix} for details).  
% in Appendix~\ref{sec:use_cases-appendix}.

% \antonis{Below we use \projecttitle{}-A2M, but further down, we do not! (i.e., no \projecttitle{}-BFT, \projecttitle{}-CR..)}

\myparagraph{Attested Append-Only Memory (A2M)} We design an Attested Append-Only Memory (A2M)~\cite{A2M} leveraging \projecttitle{}, which can be used to shield and optimize various systems~\cite{AbdElMalek2005FaultscalableBF, Castro:2002, Li2004, 10.5555/1298455.1298473}. The original A2M, and hence our implementation over \projecttitle{}, builds append-only (trusted) logs, associating each entry with a monotonically increasing sequence number to combat equivocation. While A2M has a large TCB and ports the log within the TEE, our implementation has only a minimal TCB in hardware and it can robustly store the log in the untrusted host memory, improving memory efficiency~\cite{levin2009trinc}.

As in the original A2M, we build the \texttt{append} and \texttt{lookup} operations. The \texttt{append} calls into  \projecttitle{} to generate an attestation for the log entry while associating it with a monotonically increased sequence number ({\tt sent\_cnt}). The sequence number denotes the entry's position in the log. The \texttt{lookup} operation retrieves entries locally without verification.
%, and the \texttt{truncate} operation deletes entries based on a provided sequence number. The append operation involves adding entries with a sequence number, context, and an authenticator field. %The lookup operation retrieves log entries without verification, while the truncate operation removes entries based on sequence numbers and utilizes a \textsc{manifest} log to keep track of state changes.


\myparagraph{Byzantine Fault Tolerance (BFT)} 
We design a Byzantine Fault-Tolerant protocol (BFT) using \projecttitle{}. The protocol builds a replicated counter as a foundational service for various systems~\cite{rafthyperledger, Kafka, boki, 10.1145/3286685.3286686, scalog}. Our system model considers a network of replicas with at most $f$ Byzantine replicas out of $N=2f+1$ total replicas. One replica serves as the leader, and the others act as followers. 
The system prevents equivocation through \projecttitle{}, which enforces and validates the ordering of messages. This reduces the number of replicas required and the message complexity compared to the classical BFT ($3f+1$).
% , thus cutting deployment costs and message complexity.

%The protocol operates in a partial synchrony model for liveness and assumes deterministic protocol specifications.

Clients send increment counter requests to the leader, who performs the requests and broadcasts the change along with a {\em proof of execution} (PoE) message to followers. The proof of execution is a \projecttitle{}-attested message with the original client's request, the leader's counter value, and metadata. The followers leverage their local state machine to detect a faulty leader (or follower)~\cite{268272}. Subsequently,
if and only if a follower has not applied the message before, it applies the incremented counter value to its state machine before forwarding its own PoE message to all other replicas and replying to the client.
% , who will also validate the followers' outputs. 
A quorum of at least $f+1$ identical messages from different replicas guarantees a correctly committed result for the client. %Overall, \projecttitle{} optimizes the replication factor and message complexity of BFT.

\myparagraph{Chain Replication (CR)} 
We design a Byzantine CR system~\cite{10.1007/978-3-642-35476-2_24} using \projecttitle{} as the replication layer of a Key-Value store. As in the CFT version of CR, the replicas, e.g., head, middle, and tail, are connected in a chained fashion. 
%We assume a centralized configuration service for error detection and reconfiguration, which always provides clients with a correct configuration.
%The system model assumes Byzantine fault tolerance with a centralized configuration service for error detection and reconfiguration. The head triggers reconfiguration if it intentionally fails to forward messages.

Clients execute requests by forwarding them to the head. The head orders and executes the request, creating his own {\em proof of execution message} (PoE), which is sent along the chain. The PoE consists of the original request and the head's output that \projecttitle{} attests. Each node in the chain verifies the previous node's PoE, executes the request, and creates its own PoE, which consists of the last PoE and the node's output. 

%Unlike the CFT CR assuming that cryptographic operations on the CPU are not compromised, local operations in the tail cannot be trusted in the BFT model.
\rev{D5}{Unlike the CFT CR, local operations in the tail (e.g., reads) are untrusted in the BFT model. Therefore, all operations must traverse the entire chain. Replicas reply to clients with their output after forwarding their PoE message, and clients wait for identical replies from all chained nodes. We discuss the performance-security trade-offs of an alternative TEE-based design of porting the entire CR protocol into the TEE (that would allow clients to read only from the tail) in $\S$~\ref{subsec:use_cases_eval}.} %While such a system , it targets a weaker threat model compared to \projecttitle{}.}
%Such a system would adhere to the protocol specification, with clients only needing to communicate with the tail. However, this design would target a weaker threat model compared to \projecttitle{}.
% Unlike the CFT CR, all operations must traverse the chain as local operations in the tail cannot be trusted. Replicas reply to clients with their output after forwarding their PoE message. Clients wait for identical replies from all chained nodes.
% % For \texttt{get} requests, clients traverse the chain or consult the majority, broadcasting the request to $f+1$ replicas, including the tail.

% \rev{D5}{We base our protocol implementation on~\cite{10.1007/978-3-642-35476-2_24}, where operations must traverse the entire chain, similar to Chain Replication for the Crash Fault Model. While \cite{10.1007/978-3-642-35476-2_24} assumes that cryptographic operations on the CPU are not compromised, we have implemented the system practically using \projecttitle{}. A hypothetical TEE-based design would involve porting the entire Chain Replication protocol into the TEE. Such a system would adhere to the protocol specification, with clients only needing to communicate with the tail, as in CFT Chain Replication. However, this hypothetical design would target a weaker threat model compared to \projecttitle{}.}

\myparagraph{Accountability (PeerReview)}
Lastly, we design an accountability system with \projecttitle{} based on the PeerReview system~\cite{peer-review} to {\em detect} malicious actions in large deployments~\cite{nfs, 10.1145/1218063.1217950}.  We detect faults impacting the system's network messages logged into the participant's tamper-evident log. We frame the protocol within an overlay multicast protocol for streaming systems where the nodes are organized in a tree topology. Witnesses assigned to each node audit the node's log to detect faults or non-responsive nodes. The witnesses replay the log entries, comparing them with a reference deterministic implementation to identify inconsistencies. 
Our \projecttitle{} prevents equivocation at NIC hardware efficiently, which eliminates the expensive all-to-all communication of the original PeerReview that does not use trusted hardware~\cite{levin2009trinc}.
% Moreover, \projecttitle{} optimizes accountability by efficiently handling equivocation.~\cite{trinc}



\begin{figure*}[t!]
%\begin{center}
\minipage{0.33\textwidth}
%\begin{figure}
    \centering
     \vspace{-3mm}
    \includegraphics[width=\linewidth]{atc-submission-plots/hw_eval_attest_latency.pdf} 
  \caption{{\tt Attest} function latency.}
  \label{fig:attest_kernel}
%\end{figure}
\endminipage
\minipage{0.33\textwidth}
%\begin{figure}
    \centering
     \vspace{-3mm}
  \includegraphics[width=\linewidth]{atc-submission-plots/latency_breakdown.pdf}
  \caption{{\tt Attest} latency breakdown.}\label{fig:latency_breakdown}
%\end{figure}
\endminipage
\minipage{0.33\textwidth}
%\begin{figure}
    \centering
  \includegraphics[width=\linewidth]{atc-submission-plots/foo100.pdf}
    \vspace{-7mm}
  \caption{Latency over time (SGX).}\label{fig:latency_distribution}
%\end{figure}
\endminipage
\vspace{-3mm}
%\end{center}
\end{figure*}
