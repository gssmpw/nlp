\section{\rev{(a)}{\projecttitle{} Network Library}}
% \section{Trusted Network Library}
\label{sec:t-nic-software}
We present \projecttitle{}'s programming APIs ($\S$~\ref{sec:net-lib}), and a generic recipe to transform existing distributed systems ($\S$~\ref{subsec:transformation}).

\begin{table}
\begin{center}
\footnotesize
\begin{tabular}{ |c|c| } 
 \hline
 \multicolumn{2}{|c|}{{\bf Initialization APIs}} \\ [0.5ex] \hline 
 {\tt ibv\_qp\_conn()} & Creates an ibv struct \\
 {\tt alloc\_mem()}    & Allocates host ibv memory \\
 {\tt init\_lqueue()}    & Registers local memory to \projecttitle{} \\
 {\tt ibv\_sync()} & Exchanges ibv memory addresses \\
 \hline
 \hline
 \multicolumn{2}{|c|}{{\bf Network APIs}} \\ [0.5ex] \hline 
 {\tt local\_send/verify()}    & Generates/verifies attested messages \\
 %{\tt local\_verify()}    & Verifies a message \\
 {\tt auth\_send()} &  Transmits an attested message \\
 {\tt poll()} &  Polls for incoming messages \\
 {\tt rem\_read/write()} &  Fetches/writes remote memory \\
 \hline
\end{tabular}
\end{center}
%\vspace{-10pt}
\caption{\projecttitle{} programming APIs.}
\label{tab:apis}
\vspace{-10pt}
\end{table}


\subsection{Programming APIs}
\label{sec:net-lib}
\projecttitle{} implements a programming API (Table~\ref{tab:apis}) that resembles the traditional RDMA programming API~\cite{erpc} used in modern distributed systems\cite{f04eb9b864204bab958e72055062748c, farm, hermes, rdma-design, rdma-scale, octopus, Mitchell2013}. We further extend the security semantics, offering the properties of non-equivocation and transferable authentication  ($\S$~\ref{subsec:trustworthy_ds}). %Our API is shown in Table~\ref{tab:apis}.
%as the lower bound to build trustworthy distributed systems
% Our API, shown in Table~\ref{tab:apis}, allows system designers to leverage modern RDMA-enabled networks as they have been proven to be the most efficient design in modern distributed systems\cite{f04eb9b864204bab958e72055062748c, farm, hermes, rdma-design, rdma-scale, octopus, Mitchell2013}.  
%In contrast to traditional RDMA, our \projecttitle{} extends the security semantics, offering the properties of non-equivocation and transferable authentication as the lower bound to build trustworthy distributed systems ($\S$~\ref{subsec:trustworthy_ds}).

\myparagraph{Initialization APIs} 
The \projecttitle{} application first needs to configure the \projecttitle{} system to establish peer-to-peer RDMA connections. % using the initialization APIs. 
% Prior to exchanging network messages, the \projecttitle{} system needs to be configured by the application to establish RDMA connections. 
The application creates one ibv struct for each connection with {\tt ibv\_qp\_conn()}, which sets up and stores the queue pair, the local and remote IP addresses, device UDP ports, and others. 
%The application also invokes {\tt alloc\_mem()} to allocate the ibv memory and then calls into {\tt init\_lqueue()} that initializes the local queue pairs and registers the ibv memory to the \projecttitle{} hardware. 
The application also invokes {\tt alloc\_mem()} to allocate the ibv memory and then register the ibv memory to the \projecttitle{} hardware. 
Lastly, the application synchronizes with the remote machine using {\tt ibv\_sync()} to exchange necessary data (e.g., ibv memory address, queue pair numbers).
% Lastly, the application synchronizes itself with the remote machine using {\tt ibv\_sync()} to exchange necessary data (ibv memory address, queue pair numbers, etc.). 
%which exchanges important information such as the ibv memory address, queue pair numbers, etc. 
% In addition, the application needs to allocate the ibv memory that will be registered to the \projecttitle{} hardware. The code calls into the {\tt init\_lqueue()} that initializes the local queue pairs and registers the ibv memory to the \projecttitle{}. 

% Once the application configures its \projecttitle{} appropriately, it synchronizes itself with the remote side using {\tt ibv\_sync()}. 
% This function communicates with the remote machine and exchanges important information such as its registered memory base address, queue pair numbers, etc. 


%After all these steps are complete, the application can use the \projecttitle{} network API to exchange data in a secure manner. 





\myparagraph{Network APIs}
\projecttitle{} executes trusted one-sided, reliable RDMA with the same reliability guarantees as the classical one-sided RDMA over Reliable Connection (RC), i.e., a FIFO ordering (per connection), similar to TCP/IP networking. % The networking stack imposes 

% We expose to the user the {\tt auth\_send()} operation which sends an attested message by implementing an RDMA reliable write of the message concatenated with its attestation. 
\projecttitle{} offers {\tt auth\_send()} to send an attested message with RDMA reliable writes.
% that concatenates the message with its attestation. 
We support classical RDMA operations for reads and writes: {\tt rem\_read()} and {\tt rem\_write()}. The remote side runs {\tt poll()} to fetch the number of completed operations; {\tt poll()} is updated only when the message verification succeeds at the \projecttitle{} hardware. We offer local operations for generating and verifying attested messages within a single-node setup: {\tt local\_send()} and {\tt local\_verify()}.
% In contrast to classical RDMA, we have not used background threads for polling. The function returns the number of the completed operations. 
% The poll function is only updated when the verification at the \projecttitle{} hardware has been successful. 


\projecttitle{} does not support a hardware-assisted multicast, but it can offer equivocation-free multicast uni-casting the same attested message generated by {\tt local\_send()} as in~\cite{levin2009trinc}.
 % Even in the standard RDMA libraries, hardware-based multicast is barely used in practice~\cite{f04eb9b864204bab958e72055062748c}. %, whereas it is only supported over unreliable connections.

\subsection{A Generic Transformation Recipe}
\label{subsec:transformation}
\myparagraph{Transformation properties} 
We show how to use \projecttitle{} APIs to transform an existing (CFT) distributed system into one that targets Byzantine settings. Our transformation is defined as wrapper functions on top of the send and receive operations~\cite{clement2012}. For safety, our transformation needs to meet the following properties to provide the same guarantees expected by the original CFT system~\cite{clement2012, making_distributed_app_rob, 268272}:
% We demonstrate how to use \projecttitle{} programming API to construct a generic transformation of a distributed system operating under the CFT model to a system that targets Byzantine settings.  Our transformation is defined as a set of wrapper functions on top of the send and receive operations~\cite{clement2012}. For safety, our transformation further needs to meet the following properties in order to provide the same guarantees that are expected by the original CFT system~\cite{clement2012, making_distributed_app_rob, 268272}.

\noindent\underline{{\bf{Safety.}}} If a correct receiver receives a message $m$ from a correct sender $S$, then $S$ must have sent with $m$.

\noindent\underline{{\bf{Integrity.}}} If a correct receiver receives and delivers a message $m$, then $m$ is a {\em valid} message.
%\begin{itemize}[leftmargin=*]
%    \item {\bf{Safety.}} If a correct receiver receives a message $m$ from a correct sender $S$, then $S$ must have sent with $m$.
%    \item {\bf{Integrity.}} If a correct receiver receives and delivers a message $m$, then $m$ is a {\em valid} message.
%\end{itemize}

\if 0
{\small
\begin{lstlisting}[frame=h,style=customc,
                    label={lst:transformation},
                    caption= Generic application send and recv wrapper functions for transformation using \projecttitle{}: blue sections (native) and orange sections (\projecttitle additions).]
// generic send wrapper function
void send(P_id, char[] msg) {
    // current state
@\Hilight@    state = hash(my_state);
@\Hilight@    tx_msg = msg || state || prev_receiver_msg;
@\HilightYlinewidth@    auth_send(follower, tx_msg);
}
// upon receiving a send(follower, msg) message
void recv(recv_msg) {
@\HilightYlinewidth@ // message authenticity and integrity checked in TNIC hw
auto [attestation, msg || state || prev_receiver_msg] = deliver();
@\HilightYlinewidth@    [msg, cnt] = verify_msg(msg);
@\Hilight@    verify_sender_state(state);
@\HilightYlinewidth@    local_verify(prev_receiver_msg);
@\Hilight@    verify_system_view(prev_receiver_msg);
@\Hilight@    apply(msg);
}
\end{lstlisting}
}
\pramod{should we it to the appendix and summarize with a forward pointer?}
\fi

{\small
\begin{lstlisting}[frame=h,style=customc,
                    label={lst:transformation},
                    caption= Generic send and recv wrapper functions using \projecttitle{}. \projecttitle additions are highlighted in orange.]
void send(P_id, char[] msg) {
    state = hash(my_state);  
    tx_msg = msg || state || receiver_state;
@\HilightYlinewidth@    auth_send(follower, tx_msg);
}
void recv(recv_msg) {
    auto [att, msg || state || receiver_state] = deliver();
@\HilightYlinewidth@    [msg, cnt] = verify_msg(msg);
    verify_sender_state(state);
@\HilightYlinewidth@    local_verify(receiver_state);
    verify_system_view(receiver_state); apply(msg);
}
\end{lstlisting}
}
\vspace{-1mm}

Listing~\ref{lst:transformation} shows our proposed {\tt send} (L1-5) and {\tt recv} (L7-13) operations, providing a general method for transforming a CFT system into a BFT system. 
We assume a two-node scenario where the first node (sender) receives client requests and forwards them to the second node (receiver). For transmission, the sender sends the client message {\tt msg}, its current state (e.g., the sender's action to the \texttt{msg}), and the receiver's previous state (known to the sender). 
% To avoid replaying the entire message history in order to reconstruct the system's state (as done in \cite{clement2012}), we require that nodes simulate the sender's state to verify that the sender’s response to the request is as expected
The receiver's state is optional depending on the consistency guarantees of the derived system and can be used to ensure that both nodes have the same system view. 

% Listing~\ref{lst:transformation} shows our proposed generic {\tt send} (L2-7) and {\tt recv} (L10-18) operations that we applied in a scenario between two nodes where the first node (sender) receives client requests and forwards the requests to the second node (receiver) who is following the sender. For each transmitted message ({\tt tx\_msg}), the sender sends the (client) msg, its current state (e.g., sender's output as a response to the current client msg), and the follower's previous state known to the sender. The last piece of data is optional depending on the consistency guarantees of the derived system and can be used to ensure that both the sender and the receiver have the same view of the system (L4-6). 

Upon receiving a valid message (L8-9), the receiver {\em simulates} the sender's state to verify that the sender's action to the request is as expected (L10). \rev{A5, (b)}{The way to simulate the states depends on the applications, e.g., in our BFT protocol implementation ($\S$~\ref{sec:use_cases}), each replica maintains copies of counters that represent the expected counter values for all other participating nodes. The simulation allows nodes to avoid replaying the entire message history in order to reconstruct the system's state, as done in \cite{clement2012}.} The receiver also ensures that it does not lag, and both nodes have the same ``view'' of the system inputs by verifying that the sender has {\em seen} the receiver's latest state (e.g., message) (L11-12).
% \rev{A5}{The state of the sender/receiver depends on the applications, e.g., in our BFT protocol implementation ($\S$~\ref{sec:use_cases}), each replica maintains copies of counters that represent the expected counter values for all other participating nodes. }
% The receiver, on the other hand, {\em simulates} the sender's state to verify its actions. Importantly, upon a valid message reception (L12-13), the receiver verifies that the sender's action as a response to the client's request is the expected (L14). In addition, the receiver ensures that it is not lagging behind and it has the same ``view'' of the system inputs as the sender by verifying that the sender has {\em seen} the receiver's latest sent message (L15-16).

% \myparagraph{Correctness arguments for the transformation} 
Our generic transformation satisfies the requirements listed above. First, \projecttitle{}'s transferable authentication property directly satisfies the safety requirement. A faulty node cannot impersonate correct nodes; if \projecttitle{} validates a message $m$ from a sender, the sender must have sent $m$. \projecttitle{}'s reliable network operations ensure liveness properties between correct nodes. % are , where the \projecttitle{} hardware transparently retries failed sends,
% Next we explain how our generic transformation recipe satisfies the listed above requirements. First, the safety requirement is directly satisfied by \projecttitle{}'s transferable authentication property. A faulty node cannot impersonate a correct node; if \projecttitle{} validates and delivers a message $m$ from a sender, then the sender must have sent $m$. \projecttitle{} reliable network operations, where the \projecttitle{} hardware transparently retries failed sends ensures liveness properties between correct nodes. 
% 
Second, our transformation satisfies the integrity property. \rev{A7, (b)}{The integrity property is ensured by validating that the sender’s response to the client’s request follows the protocol specification. The transferable authentication mechanism allows correct receivers to prove the integrity flow by simulating the sender's output and state, e.g., by maintaining a copy of the sender’s state. }
% \rev{A7}{The integrity property is ensured by validating that the sender’s response to the client’s request follows the protocol specification. Note that correct replicas can verify the validity and integrity of the request by simulating the sender’s state, e.g., by maintaining a copy of the sender’s state.}

% Second, our transformation satisfies the integrity property. The property requires the receiver to ensure that a sender's message derives from a valid execution scenario of the business logic. Our transformation with the transferable authentication mechanism allows the receiver to simulate the sender's output and state and prove their integrity and validity. 

% Second, the integrity property requires the receiver to obtain a guarantee that a message from the sender has been derived from a valid execution scenario of the business logic. In our transformation, the follower simulates the leader's state and jointly with the transferable authentication mechanism, we can always prove the integrity and the validity of the sender's output and state. 
% Second, the integrity property requires the receiver to obtain a guarantee that a message from the sender has been derived from a valid execution scenario of the business logic. In our transformation, the follower simulates the leader's state and jointly with the transferable authentication mechanism, we can always prove the integrity and the validity of the sender's output and state. 

% Since \projecttitle{} aims to provide a general approach to transform distributed systems for Byzantine settings, it also needs to ensure the consistency property as an additional property to be met for the specific case of the CFT replication protocols~\cite{clement2012}.
\myparagraph{Consistency property for replication} 
Our transformation further needs to meet the consistency property~\cite{clement2012}. If correct receivers $R_1$ and $R_2$ receive valid messages $m_i$ and $m_j$  respectively from sender $S$, then either (a) $Bpg_{i}$ is a prefix of $Bpg_{j}$ , (b) $Bpg_{j}$ is a prefix of $Bpg_{i}$, or (c) $Bpg_{i}$ = $Bpg_{j}$ (where $Bpg_{x}$ is the process behavior that supports the validity of message $m_x$).

%for the specific case of the CFT replication protocols
% to transform distributed systems for Byzantine settings:
% \begin{itemize}[leftmargin=*]
% \item \myparagraph{Consistency} If correct receivers $R_1$ and $R_2$ receive valid messages $m_i$ and $m_j$  respectively from sender $S$, then either (a) $Bpg_{,i}$ is a prefix of $Bpg_{,j}$ , (b) $Bpg_{,j}$ is a prefix of $Bpg_{,i}$, or (c) $Bpg_{,i}$ = $Bpg_{,j}$ (where $Bpg_{,x}$ is the process behavior that supports the validity of message $m_x$).
% \end{itemize}

The consistency requirement is enforced through the \projecttitle{}'s non-equivocation primitive that assigns a (unique) monotonic sequence number to each outgoing message, enforcing a total order on the sender's outgoing messages. 
\rev{D2, (b)}{Along with the integrity requirement, the total order can prevent equivocation and suffice for consistency. } 
% \atsushi{@Dimitra: can we detail more why the total order can suffice for consistency? }
Importantly, \projecttitle{} ensures that correct receivers cannot miss any past messages. Following this, two followers that receive from the same sender (using the equivocation-free multicast discussed in $\S$~\ref{subsec:net_lib}) follow the same transition (execution) path. 
\rev{D2, (b)}{\projecttitle{} cannot transform systems with non-deterministic specifications.}
%\rev{D2, (b)}{\projecttitle{} cannot transform systems that might have non-deterministic outputs for a given input.} %\atsushi{@Dimitra: D2: Can you check the text? I think they should be more detailed}
% Our suggested transformation based on the sender's state simulation (leader) can be effectively applied to state machine replication.

% The consistency requirement is enforced through the \projecttitle{}'s non-equivocation primitive that assigns a (unique) monotonic sequence number to each outgoing message, enforcing a total order on the sender's outgoing messages. Along with the integrity requirement, the total order suffices for consistency. 
% Importantly, \projecttitle{} ensures that correct receivers cannot miss any past messages. Following this, two followers that receive from the same sender (using the equivocation-free multicast discussed in $\S$~\ref{subsec:net_lib}) follow the same transition (execution) path. 


