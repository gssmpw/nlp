
\section*{Appendix}
\section{Formal verification Proofs}
\label{sec:formal-verification-details}
In this appendix, we present the detailed security proofs for \projecttitle{} security protocols using
the Tamarin Prover~\cite{tamarin-prover}. 



\myparagraph{Proof artifacts} The complete proofs, including the detailed formal models used to generate them, can be found under the following link:\\{\color{blue}\url{https://anonymous.4open.science/r/tnic-protocols}}


\myparagraph{Symbolic model} We prove the security properties of \projecttitle{}  in a symbolic model because Tamarin analyzes protocols in symbolic models and can prove properties by verifying user-defined lemmas. We leverage Tamarin's built-in primitives and automated and interactive analysis to verify the security protocols.

We impose a set of assumptions on our proofs motivated by Tamarin's symbolic model: \emph{(i)}~The symbolic model does not reason about bitstrings directly. Instead, it assumes a set of atomic terms and functions that operate on these terms. All messages that are part of the model are composed of such atomic terms and functions applied on these terms \emph{(ii)}~These cryptographic functions are assumed to be perfect with no side-effects, e.g., hash functions are irreversible, and hash collisions are impossible. This allows for proving lemmas without considering the probabilities of violating specific properties and thus significantly reducing the complexity. The computational model is an alternative to the symbolic model that considers such probabilities. \emph{(iii)}~Attackers can read and delete all messages that are sent on the network and modify them in accordance with the set of defined functions.

Tamarin works on symbolic models specified using multiset rewriting rules that operate on the system's state. Different states of the system are expressed as a set of facts with rules capturing the available transitions from one system state to another. Rules are used to model the actions of agents running the protocol and the adversary’s capabilities. In addition to the rules, Tamarin also makes use of restrictions. Restrictions further refine the sources of facts in the protocol to improve the efficiency of the proof generation.

 Our verification work relies on properties of the already analyzed TLS handshake~\cite{tamarin_tls_proof}. It provides a model and lemmas for the security properties of the protocols presented in this paper.

To prove the correctness of our lemmas, Tamarin computes possible executions for each rule. Tamarin employs constraint solving to refine its knowledge about the sequence of protocol transitions. To check the correctness of the protocol model we also employ sanity lemmas which ensure that there exists a sequence of transitions to reach a predefined valid state. These lemmas ensure that the protocol can be executed as intended. 

In the following paragraphs, we give an overview of the rules and lemmas used to model the \projecttitle{} protocols.

\myparagraph{Rules}
The bootstrapping rules, in accordance with the bootstrapping steps in Section~\ref{subsec:nic_controller}:
\begin{itemize}

\item \emph{bootstrapping\_1}: Models step (1), the generation and burning of the hardware key by the \projecttitle{} manufacturer.
\item \emph{bootstrapping\_2}: Models step (2), the loading and verification of firmware from the insecure storage medium.
\item \emph{bootstrapping\_3\_4\_5}: Models step (3-5), the loading, key and certificate generation of the controller.
\item \emph{publish\_firmware}: Models the hardware manufacturer publishing a new firmware version.
\item \emph{get\_tnic\_public\_key}: Models the retrieval of the public \projecttitle{} key for verification.
\item \emph{compromise\_tnic\_private\_key}: Models an attacker compromising a specific \projecttitle{} device and retrieving the private key.

\end{itemize}
The attestation rules, in accordance with the attestation steps in Section~\ref{subsec:nic_controller}:
\begin{itemize}

\item \emph{attestation\_6\_7}: Models step (6-7), the receiving of the configuration data from the protocol designer, and the start of the secure channel establishment with the \projecttitle{} device.
\item \emph{attestation\_8a}: Models step (8) on the \projecttitle{} side.
\item \emph{attestation\_8b\_9}: Models step (8) on the IP Vendor side and step (9), the sending of the encrypted configuration bitstream.
\item \emph{attestation\_10\_11\_12}: Models step (10-12), the report generation of the \projecttitle{} device. 
\item \emph{attestation\_13\_14\_15\_16}: Models step (13-16), the report retrieval and verification, as well as the sending of the bitstream encryption key.
\item \emph{attestation\_17}: Models step (17), the decryption and configuration of the bitstream, as well as the acknowledgment.
\item \emph{attestation\_18}: Models the final step of the IP Vendor after which the attestation protocol is completed.
\item \emph{add\_bitstream}: Models the addition of a new bitstream to the IP Vendor, which potentially contains sensitive information.

\end{itemize}
The communication rules, in accordance with the functions provided in Algorithm~\ref{algo:primitives}:
\begin{itemize}

\item \emph{init\_ctrs}: Models the initialization of the send and receive counters for each session. Is restricted to guarantee the uniqueness of the session counters.
\item \emph{send\_msg}: Models send an arbitrary message by attesting it before sending it over the secure channel. Is restricted to guarantee the session counters are increased.
\item \emph{recv\_msg}: Models receive an arbitrary message by only accepting it after a successful verification.

\end{itemize}

\myparagraph{Lemmas}
The sanity lemmas, which ensure the protocol can be executed as intended:
\begin{itemize}

\item \emph{sanity} (verified in 26 steps): Ensures that the protocol allows for successfully completing the bootstrapping \& attestation phase, such that the IP Vendor and uncompromised \projecttitle{} device are in an expected state.
\item \emph{send\_sanity} (verified in 23 steps):  Ensures that the protocol allows for successfully verifying a message sent during the communication phase after two \projecttitle{} devices are successfully initialized. 

\end{itemize}
The attestation lemmas, which ensure the bootstrapping \& attestation phase behaves as expected:
\begin{itemize}

\item \emph{HW\_key\_priv\_secret} (verified in 3 steps): Ensures that the private key of the \projecttitle{} device is not obtainable from any messages sent as part of the \projecttitle{} protocols of the model. 
\item \emph{S\_key\_secret} (verified in 97 steps): Ensures all symmetric keys established during initialization phases are secret. It also ensures that past symmetric keys stay secret even if the hardware key is compromised in the future after the session is completed.
\item \emph{bitstream\_secret} (verified in 83 steps): Ensures all bitstreams shared during initialization phases are secret. It also ensures that past shared bitstreams stay secret even if the hardware key is compromised in the future after the session is completed.
\item \emph{initialization\_attested} (verified in 5540 steps): Ensures that after the IP Vendor finished the attestation during the initialization phase, the \projecttitle{} device is in an expected state and loaded the correct configuration.

\end{itemize}
The transferable authentication lemma:
\begin{itemize}

\item \emph{verified\_msg\_is\_auth} (verified in 31795 steps): Ensuring that each message that is successfully accepted by a \projecttitle{} device is sent by a genuine \projecttitle{} device, assuming
the hardware of the \projecttitle{} devices was not compromised.

\end{itemize}
The non-equivocation lemmas:
\begin{itemize}

\item \emph{no\_lost\_messages} (verified in four steps): Ensures that for all messages that are successfully accepted by a genuine \projecttitle{} device, there are no messages that were sent before but not accepted by the same \projecttitle{} device.
\item \emph{no\_message\_reordering} (verified in 5447 steps): Ensures that for all messages that are successfully accepted by a genuine \projecttitle{} device, there are no messages that were sent after that message but accepted before.
\item \emph{no\_double\_messages} (verified in 10850 steps): Ensures that a genuine \projecttitle{} device does not accept the same message multiple times.

\end{itemize}











