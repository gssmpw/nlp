% TEMPLATE for Usenix papers, specifically to meet requirements of
%  USENIX '05
% originally a template for producing IEEE-format articles using LaTeX.
%   written by Matthew Ward, CS Department, Worcester Polytechnic Institute.
% adapted by David Beazley for his excellent SWIG paper in Proceedings,
%   Tcl 96
% turned into a smartass generic template by De Clarke, with thanks to
%   both the above pioneers
% use at your own risk.  Complaints to /dev/null.
% make it two column with no page numbering, default is 10 point

% Munged by Fred Douglis <douglis@research.att.com> 10/97 to separate
% the .sty file from the LaTeX source template, so that people can
% more easily include the .sty file into an existing document.  Also
% changed to more closely follow the style guidelines as represented
% by the Word sample file. 

% Note that since 2010, USENIX does not require endnotes. If you want
% foot of page notes, don't include the endnotes package in the 
% usepackage command, below.

%\documentclass[letterpaper,twocolumn,10pt]{article}
%\usepackage{usenix,epsfig,endnotes}

%\documentclass[sigconf, screen]{acmart}

%\usepackage[english]{babel}
%\usepackage{blindtext}
    % use the base acmart.cls
    % use the sigplan proceeding template with the default 10 pt fonts
    % nonacm option removes ACM related text in the submission. 
\documentclass[sigplan,screen]{acmart}

% Prevent the compiler from complaining about missing countries in affiliation
\makeatletter
\def\@ACM@checkaffil{% Only warnings
    \if@ACM@instpresent\else
    \ClassWarningNoLine{\@classname}{No institution present for an affiliation}%
    \fi
    \if@ACM@citypresent\else
    \ClassWarningNoLine{\@classname}{No city present for an affiliation}%
    \fi
    \if@ACM@countrypresent\else
        \ClassWarningNoLine{\@classname}{No country present for an affiliation}%
    \fi
}
\makeatother

\makeatletter
\def\@ACM@copyright@check@cc{}
\makeatother

\copyrightyear{2025}
\acmYear{2025}
\setcopyright{cc}
\setcctype{by}
\acmConference[ASPLOS '25]{Proceedings of the 30th ACM International Conference on Architectural Support for Programming Languages and Operating Systems, Volume 2}{March 30-April 3, 2025}{Rotterdam, Netherlands}
\acmBooktitle{Proceedings of the 30th ACM International Conference on Architectural Support for Programming Languages and Operating Systems, Volume 2 (ASPLOS '25), March 30-April 3, 2025, Rotterdam, Netherlands}\acmDOI{10.1145/3676641.3716277}
\acmISBN{979-8-4007-1079-7/2025/03}

%\copyrightyear{2025}
%\acmYear{2025}
%\setcopyright{acmlicensed}
%\acmConference[ASPLOS '25] {Proceedings of the 30th ACM International Conference on Architectural %Support for Programming Languages and Operating Systems, Volume 2}{March 30--April 3, 2025}%{Rotterdam, Netherlands.}
%\acmBooktitle{Proceedings of the 30th ACM International Conference on Architectural Support for %Programming Languages and Operating Systems, Volume 2 (ASPLOS '25), March 30--April 3, 2025, %Rotterdam, Netherlands}
%\acmISBN{979-8-4007-1079-7/25/03}
%\acmDOI{10.1145/XXXXXX.XXXXXX}
% 1 Authors, replace the red X's with your assigned DOI string during the rightsreview eform process.
% 2 Your DOI link will become active when the proceedings appears in the DL.
% 3 Retain the DOI string between the curly braces for uploading your presentation video.

\settopmatter{printacmref=true}
% \settopmatter{printfolios=true}

%don't want date printed
\date{}

\usepackage{bbm}
\usepackage{graphicx}
\usepackage{amsmath,amssymb,amsthm,amsfonts}

\usepackage{paralist}
\usepackage{bm}
\usepackage{xspace}
\usepackage{url}
\usepackage{prettyref}
\usepackage{boxedminipage}
\usepackage{wrapfig}
\usepackage{ifthen}
\usepackage{color}
\usepackage{xspace}

\newcommand{\ii}{{\sc Indicator-Instance}\xspace}
\newcommand{\midd}{{\sf mid}}


\usepackage{amsmath,amsthm,amsfonts,amssymb}
\usepackage{mathtools}
\usepackage{graphicx}


% \usepackage{fullpage}

\usepackage{nicefrac}

\newtheorem{inftheorem}{Informal Theorem}
\newtheorem{claim}{Claim}
\newtheorem*{definition*}{Definition}
\newtheorem{example}{Example}

\DeclareMathOperator*{\argmax}{arg\,max}
\DeclareMathOperator*{\argmin}{arg\,min}
\usepackage{subcaption}

\newtheorem{problem}{Problem}
\usepackage[utf8]{inputenc}
\newcommand{\rank}{\mathsf{rank}}
\newcommand{\tr}{\mathsf{Tr}}
\newcommand{\tv}{\mathsf{TV}}
\newcommand{\opt}{\mathsf{OPT}}
\newcommand{\rr}{\textsc{R}\space}
\newcommand{\alg}{\textsf{Alg}\space}
\newcommand{\sd}{\textsf{sd}_\lambda}
\newcommand{\lblq}{\mathfrak{lq} (X_1)}
\newcommand{\diag}{\textsf{diag}}
\newcommand{\sign}{\textsf{sgn}}
\newcommand{\BC}{\texttt{BC} }
\newcommand{\MM}{\texttt{MM} }
\newcommand{\Nexp}{N_{\mathrm{exp}}}
\newcommand{\Nrep}{N_{\mathrm{replay}}}
\newcommand{\Drep}{D_{\mathrm{replay}}}
\newcommand{\Nsim}{N_{\mathrm{sim}}}
\newcommand{\piBC}{\pi^{\texttt{BC}}}
\newcommand{\piRE}{\pi^{\texttt{RE}}}
\newcommand{\piEMM}{\pi^{\texttt{MM}}}
\newcommand{\mmd}{\texttt{Mimic-MD} }
\newcommand{\RE}{\texttt{RE} }
\newcommand{\dem}{\pi^E}
\newcommand{\Rlint}{\mathcal{R}_{\mathrm{lin,t}}}
\newcommand{\Rlipt}{\mathcal{R}_{\mathrm{lip,t}}}
\newcommand{\Rlin}{\mathcal{R}_{\mathrm{lin}}}
\newcommand{\Rlip}{\mathcal{R}_{\mathrm{lip}}}
\newcommand{\Rmax}{R_{\mathrm{max}}}
\newcommand{\Rall}{\mathcal{R}_{\mathrm{all}}}
\newcommand{\Rdet}{\mathcal{R}_{\mathrm{det}}}
\newcommand{\Fmax}{F_{\mathrm{max}}}
\newcommand{\Nmax}{\mathcal{N}_{\mathrm{max}}}
\newcommand{\piref}{\pi^{\mathrm{ref}}}
\newcommand{\green}{\text{\color{green!75!black} green}\;}
\newcommand{\thetaBC}{\widehat{\theta}^{\textsf{BC}}}
\newcommand{\ent}{\mathcal{E}_{\Theta,n,\delta}}
\newcommand{\eNt}{\mathcal{E}_{\Theta_t,\Nexp,\delta}}
\newcommand{\eNtH}{\mathcal{E}_{\Theta_t,\Nexp,\delta/H}}

\newcommand{\eref}[1]{(\ref{#1})}
\newcommand{\sref}[1]{Sec. \ref{#1}}
\newcommand{\dr}{\widehat{d}_{\mathrm{replay}}}
\newcommand{\figref}[1]{Fig. \ref{#1}}

\usepackage{xcolor}
\definecolor{expert}{HTML}{008000}
\definecolor{error}{HTML}{f96565}
\newcommand{\GKS}[1]{{\textcolor{violet}{\textbf{GKS: #1}}}}
\newcommand{\Q}[1]{{\textcolor{red}{\textbf{Question #1}}}}
\newcommand{\ZSW}[1]{{\textcolor{orange}{\textbf{ZSW: #1}}}}
\newcommand{\JAB}[1]{{\textcolor{teal}{\textbf{JAB: #1}}}}
\newcommand{\jab}[1]{{\textcolor{teal}{\textbf{JAB: #1}}}}
\newcommand{\SAN}[1]{{\textcolor{blue}{\textbf{SC: #1}}}}
\newcommand{\scnote}[1]{\SAN{#1}}
\newcommand{\norm}[1]{\left\lVert #1 \right\rVert}

\usepackage{color-edits}
\addauthor{sw}{blue}

\usepackage{thmtools}
\usepackage{thm-restate}

\usepackage{tikz}
\usetikzlibrary{arrows,calc} 
\newcommand{\tikzAngleOfLine}{\tikz@AngleOfLine}
\def\tikz@AngleOfLine(#1)(#2)#3{%
\pgfmathanglebetweenpoints{%
\pgfpointanchor{#1}{center}}{%
\pgfpointanchor{#2}{center}}
\pgfmathsetmacro{#3}{\pgfmathresult}%
}

\declaretheoremstyle[
    headfont=\normalfont\bfseries, 
    bodyfont = \normalfont\itshape]{mystyle} 
\declaretheorem[name=Theorem,style=mystyle,numberwithin=section]{thm}

% \usepackage{algorithm}
% \usepackage{algorithmic}
\usepackage[linesnumbered,algoruled,boxed,lined,noend]{algorithm2e}

\usepackage{listings}
\usepackage{amsmath}
\usepackage{amsthm}
\usepackage{tikz}
\usepackage{caption}
\usepackage{mdwmath}
\usepackage{multirow}
\usepackage{mdwtab}
\usepackage{eqparbox}
\usepackage{multicol}
\usepackage{amsfonts}
\usepackage{tikz}
\usepackage{multirow,bigstrut,threeparttable}
\usepackage{amsthm}
\usepackage{bbm}
\usepackage{epstopdf}
\usepackage{mdwmath}
\usepackage{mdwtab}
\usepackage{eqparbox}
\usetikzlibrary{topaths,calc}
\usepackage{latexsym}
\usepackage{cite}
\usepackage{amssymb}
\usepackage{bm}
\usepackage{amssymb}
\usepackage{graphicx}
\usepackage{mathrsfs}
\usepackage{epsfig}
\usepackage{psfrag}
\usepackage{setspace}
\usepackage[%dvips,
            CJKbookmarks=true,
            bookmarksnumbered=true,
            bookmarksopen=true,
%						bookmarks=false,
            colorlinks=true,
            citecolor=red,
            linkcolor=blue,
            anchorcolor=red,
            urlcolor=blue
            ]{hyperref}
%\usepackage{algorithm}
\usepackage[linesnumbered,algoruled,boxed,lined]{algorithm2e}
\usepackage{algpseudocode}
\usepackage{stfloats}
\RequirePackage[numbers]{natbib}

\usepackage{comment}
\usepackage{mathtools}
\usepackage{blkarray}
\usepackage{multirow,bigdelim,dcolumn,booktabs}

\usepackage{xparse}
\usepackage{tikz}
\usetikzlibrary{calc}
\usetikzlibrary{decorations.pathreplacing,matrix,positioning}

\usepackage[T1]{fontenc}
\usepackage[utf8]{inputenc}
\usepackage{mathtools}
\usepackage{blkarray, bigstrut}
\usepackage{gauss}

\newenvironment{mygmatrix}{\def\mathstrut{\vphantom{\big(}}\gmatrix}{\endgmatrix}

\newcommand{\tikzmark}[1]{\tikz[overlay,remember picture] \node (#1) {};}

%% Adapted form https://tex.stackexchange.com/questions/206898/braces-for-cases-in-tabular-environment/207704#207704
\newcommand*{\BraceAmplitude}{0.4em}%
\newcommand*{\VerticalOffset}{0.5ex}%  
\newcommand*{\HorizontalOffset}{0.0em}% 
\newcommand*{\blocktextwid}{3.0cm}%
\NewDocumentCommand{\InsertLeftBrace}{%
	O{} % #1 = draw options
	O{\HorizontalOffset,\VerticalOffset} % #2 = optional brace shift options
	O{\blocktextwid} % #3 = optional text width
	m   % #4 = top tikzmark
	m   % #5 = bottom tikzmark
	m   % #6 = node text
}{%
	\begin{tikzpicture}[overlay,remember picture]
	\coordinate (Brace Top)    at ($(#4.north) + (#2)$);
	\coordinate (Brace Bottom) at ($(#5.south) + (#2)$);
	\draw [decoration={brace, amplitude=\BraceAmplitude}, decorate, thick, draw=black, #1]
	(Brace Bottom) -- (Brace Top) 
	node [pos=0.5, anchor=east, align=left, text width=#3, color=black, xshift=\BraceAmplitude] {#6};
	\end{tikzpicture}%
}%
\NewDocumentCommand{\InsertRightBrace}{%
	O{} % #1 = draw options
	O{\HorizontalOffset,\VerticalOffset} % #2 = optional brace shift options
	O{\blocktextwid} % #3 = optional text width
	m   % #4 = top tikzmark
	m   % #5 = bottom tikzmark
	m   % #6 = node text
}{%
	\begin{tikzpicture}[overlay,remember picture]
	\coordinate (Brace Top)    at ($(#4.north) + (#2)$);
	\coordinate (Brace Bottom) at ($(#5.south) + (#2)$);
	\draw [decoration={brace, amplitude=\BraceAmplitude}, decorate, thick, draw=black, #1]
	(Brace Top) -- (Brace Bottom) 
	node [pos=0.5, anchor=west, align=left, text width=#3, color=black, xshift=\BraceAmplitude] {#6};
	\end{tikzpicture}%
}%
\NewDocumentCommand{\InsertTopBrace}{%
	O{} % #1 = draw options
	O{\HorizontalOffset,\VerticalOffset} % #2 = optional brace shift options
	O{\blocktextwid} % #3 = optional text width
	m   % #4 = top tikzmark
	m   % #5 = bottom tikzmark
	m   % #6 = node text
}{%
	\begin{tikzpicture}[overlay,remember picture]
	\coordinate (Brace Top)    at ($(#4.west) + (#2)$);
	\coordinate (Brace Bottom) at ($(#5.east) + (#2)$);
	\draw [decoration={brace, amplitude=\BraceAmplitude}, decorate, thick, draw=black, #1]
	(Brace Top) -- (Brace Bottom) 
	node [pos=0.5, anchor=south, align=left, text width=#3, color=black, xshift=\BraceAmplitude] {#6};
	\end{tikzpicture}%
}%

\usetikzlibrary{patterns}

\definecolor{cof}{RGB}{219,144,71}
\definecolor{pur}{RGB}{186,146,162}
\definecolor{greeo}{RGB}{91,173,69}
\definecolor{greet}{RGB}{52,111,72}

% provide arXiv number if available:
% \arxiv{cs.IT/1502.00326}

% put your definitions there:

%\newtheorem{remark}{Remark} \def\remref#1{Remark~\ref{#1}}
%\newtheorem{conjecture}{Conjecture} \def\remref#1{Remark~\ref{#1}}
%\newtheorem{example}{Example}

%\theorembodyfont{\itshape}
%\newtheorem{theorem}{Theorem}
%\newtheorem{proposition}{Proposition}
%\newtheorem{lemma}{Lemma} \def\lemref#1{Lemma~\ref{#1}}
%\newtheorem{corollary}{Corollary}


%\theorembodyfont{\rmfamily}
%\newtheorem{definition}{Definition}
%\numberwithin{equation}{section}
% \theoremstyle{plain}
% \newtheorem{theorem}{Theorem}
% \newtheorem{Example}{Example}
% \newtheorem{lemma}{Lemma}
% \newtheorem{remark}{Remark}
% \newtheorem{corollary}{Corollary}
% \newtheorem{definition}{Definition}
% \newtheorem{conjecture}{Conjecture}
% \newtheorem{question}{Question}
% \newtheorem*{induction}{Induction Hypothesis}
% \newtheorem*{folklore}{Folklore}
% \newtheorem{assumption}{Assumption}

\def \by {\bar{y}}
\def \bx {\bar{x}}
\def \bh {\bar{h}}
\def \bz {\bar{z}}
\def \cF {\mathcal{F}}
\def \bP {\mathbb{P}}
\def \bE {\mathbb{E}}
\def \bR {\mathbb{R}}
\def \bF {\mathbb{F}}
\def \cG {\mathcal{G}}
\def \cM {\mathcal{M}}
\def \cB {\mathcal{B}}
\def \cN {\mathcal{N}}
\def \var {\mathsf{Var}}
\def\1{\mathbbm{1}}
\def \FF {\mathbb{F}}


\newenvironment{keywords}
{\bgroup\leftskip 20pt\rightskip 20pt \small\noindent{\bfseries
Keywords:} \ignorespaces}%
{\par\egroup\vskip 0.25ex}
\newlength\aftertitskip     \newlength\beforetitskip
\newlength\interauthorskip  \newlength\aftermaketitskip















%%%%%%%%%%%%%%%%%%%%%%%%%%%% by Wu %%%%%%%%%%%%%%%%%%%%%%%%%%%%
\usepackage{xspace}

\newcommand{\Lip}{\mathrm{Lip}}
\newcommand{\stepa}[1]{\overset{\rm (a)}{#1}}
\newcommand{\stepb}[1]{\overset{\rm (b)}{#1}}
\newcommand{\stepc}[1]{\overset{\rm (c)}{#1}}
\newcommand{\stepd}[1]{\overset{\rm (d)}{#1}}
\newcommand{\stepe}[1]{\overset{\rm (e)}{#1}}
\newcommand{\stepf}[1]{\overset{\rm (f)}{#1}}


\newcommand{\floor}[1]{{\left\lfloor {#1} \right \rfloor}}
\newcommand{\ceil}[1]{{\left\lceil {#1} \right \rceil}}

\newcommand{\blambda}{\bar{\lambda}}
\newcommand{\reals}{\mathbb{R}}
\newcommand{\naturals}{\mathbb{N}}
\newcommand{\integers}{\mathbb{Z}}
\newcommand{\Expect}{\mathbb{E}}
\newcommand{\expect}[1]{\mathbb{E}\left[#1\right]}
\newcommand{\Prob}{\mathbb{P}}
\newcommand{\prob}[1]{\mathbb{P}\left[#1\right]}
\newcommand{\pprob}[1]{\mathbb{P}[#1]}
\newcommand{\intd}{{\rm d}}
\newcommand{\TV}{{\sf TV}}
\newcommand{\LC}{{\sf LC}}
\newcommand{\PW}{{\sf PW}}
\newcommand{\htheta}{\hat{\theta}}
\newcommand{\eexp}{{\rm e}}
\newcommand{\expects}[2]{\mathbb{E}_{#2}\left[ #1 \right]}
\newcommand{\diff}{{\rm d}}
\newcommand{\eg}{e.g.\xspace}
\newcommand{\ie}{i.e.\xspace}
\newcommand{\iid}{i.i.d.\xspace}
\newcommand{\fracp}[2]{\frac{\partial #1}{\partial #2}}
\newcommand{\fracpk}[3]{\frac{\partial^{#3} #1}{\partial #2^{#3}}}
\newcommand{\fracd}[2]{\frac{\diff #1}{\diff #2}}
\newcommand{\fracdk}[3]{\frac{\diff^{#3} #1}{\diff #2^{#3}}}
\newcommand{\renyi}{R\'enyi\xspace}
\newcommand{\lpnorm}[1]{\left\|{#1} \right\|_{p}}
\newcommand{\linf}[1]{\left\|{#1} \right\|_{\infty}}
\newcommand{\lnorm}[2]{\left\|{#1} \right\|_{{#2}}}
\newcommand{\Lploc}[1]{L^{#1}_{\rm loc}}
\newcommand{\hellinger}{d_{\rm H}}
\newcommand{\Fnorm}[1]{\lnorm{#1}{\rm F}}
%% parenthesis
\newcommand{\pth}[1]{\left( #1 \right)}
\newcommand{\qth}[1]{\left[ #1 \right]}
\newcommand{\sth}[1]{\left\{ #1 \right\}}
\newcommand{\bpth}[1]{\Bigg( #1 \Bigg)}
\newcommand{\bqth}[1]{\Bigg[ #1 \Bigg]}
\newcommand{\bsth}[1]{\Bigg\{ #1 \Bigg\}}
\newcommand{\xxx}{\textbf{xxx}\xspace}
\newcommand{\toprob}{{\xrightarrow{\Prob}}}
\newcommand{\tolp}[1]{{\xrightarrow{L^{#1}}}}
\newcommand{\toas}{{\xrightarrow{{\rm a.s.}}}}
\newcommand{\toae}{{\xrightarrow{{\rm a.e.}}}}
\newcommand{\todistr}{{\xrightarrow{{\rm D}}}}
\newcommand{\eqdistr}{{\stackrel{\rm D}{=}}}
\newcommand{\iiddistr}{{\stackrel{\text{\iid}}{\sim}}}
%\newcommand{\var}{\mathsf{var}}
\newcommand\indep{\protect\mathpalette{\protect\independenT}{\perp}}
\def\independenT#1#2{\mathrel{\rlap{$#1#2$}\mkern2mu{#1#2}}}
\newcommand{\Bern}{\text{Bern}}
\newcommand{\Poi}{\mathsf{Poi}}
\newcommand{\iprod}[2]{\left \langle #1, #2 \right\rangle}
\newcommand{\Iprod}[2]{\langle #1, #2 \rangle}
\newcommand{\indc}[1]{{\mathbf{1}_{\left\{{#1}\right\}}}}
\newcommand{\Indc}{\mathbf{1}}
\newcommand{\regoff}[1]{\textsf{Reg}_{\mathcal{F}}^{\text{off}} (#1)}
\newcommand{\regon}[1]{\textsf{Reg}_{\mathcal{F}}^{\text{on}} (#1)}

\definecolor{myblue}{rgb}{.8, .8, 1}
\definecolor{mathblue}{rgb}{0.2472, 0.24, 0.6} % mathematica's Color[1, 1--3]
\definecolor{mathred}{rgb}{0.6, 0.24, 0.442893}
\definecolor{mathyellow}{rgb}{0.6, 0.547014, 0.24}


\newcommand{\red}{\color{red}}
\newcommand{\blue}{\color{blue}}
\newcommand{\nb}[1]{{\sf\blue[#1]}}
\newcommand{\nbr}[1]{{\sf\red[#1]}}

\newcommand{\tmu}{{\tilde{\mu}}}
\newcommand{\tf}{{\tilde{f}}}
\newcommand{\tp}{\tilde{p}}
\newcommand{\tilh}{{\tilde{h}}}
\newcommand{\tu}{{\tilde{u}}}
\newcommand{\tx}{{\tilde{x}}}
\newcommand{\ty}{{\tilde{y}}}
\newcommand{\tz}{{\tilde{z}}}
\newcommand{\tA}{{\tilde{A}}}
\newcommand{\tB}{{\tilde{B}}}
\newcommand{\tC}{{\tilde{C}}}
\newcommand{\tD}{{\tilde{D}}}
\newcommand{\tE}{{\tilde{E}}}
\newcommand{\tF}{{\tilde{F}}}
\newcommand{\tG}{{\tilde{G}}}
\newcommand{\tH}{{\tilde{H}}}
\newcommand{\tI}{{\tilde{I}}}
\newcommand{\tJ}{{\tilde{J}}}
\newcommand{\tK}{{\tilde{K}}}
\newcommand{\tL}{{\tilde{L}}}
\newcommand{\tM}{{\tilde{M}}}
\newcommand{\tN}{{\tilde{N}}}
\newcommand{\tO}{{\tilde{O}}}
\newcommand{\tP}{{\tilde{P}}}
\newcommand{\tQ}{{\tilde{Q}}}
\newcommand{\tR}{{\tilde{R}}}
\newcommand{\tS}{{\tilde{S}}}
\newcommand{\tT}{{\tilde{T}}}
\newcommand{\tU}{{\tilde{U}}}
\newcommand{\tV}{{\tilde{V}}}
\newcommand{\tW}{{\tilde{W}}}
\newcommand{\tX}{{\tilde{X}}}
\newcommand{\tY}{{\tilde{Y}}}
\newcommand{\tZ}{{\tilde{Z}}}

\newcommand{\sfa}{{\mathsf{a}}}
\newcommand{\sfb}{{\mathsf{b}}}
\newcommand{\sfc}{{\mathsf{c}}}
\newcommand{\sfd}{{\mathsf{d}}}
\newcommand{\sfe}{{\mathsf{e}}}
\newcommand{\sff}{{\mathsf{f}}}
\newcommand{\sfg}{{\mathsf{g}}}
\newcommand{\sfh}{{\mathsf{h}}}
\newcommand{\sfi}{{\mathsf{i}}}
\newcommand{\sfj}{{\mathsf{j}}}
\newcommand{\sfk}{{\mathsf{k}}}
\newcommand{\sfl}{{\mathsf{l}}}
\newcommand{\sfm}{{\mathsf{m}}}
\newcommand{\sfn}{{\mathsf{n}}}
\newcommand{\sfo}{{\mathsf{o}}}
\newcommand{\sfp}{{\mathsf{p}}}
\newcommand{\sfq}{{\mathsf{q}}}
\newcommand{\sfr}{{\mathsf{r}}}
\newcommand{\sfs}{{\mathsf{s}}}
\newcommand{\sft}{{\mathsf{t}}}
\newcommand{\sfu}{{\mathsf{u}}}
\newcommand{\sfv}{{\mathsf{v}}}
\newcommand{\sfw}{{\mathsf{w}}}
\newcommand{\sfx}{{\mathsf{x}}}
\newcommand{\sfy}{{\mathsf{y}}}
\newcommand{\sfz}{{\mathsf{z}}}
\newcommand{\sfA}{{\mathsf{A}}}
\newcommand{\sfB}{{\mathsf{B}}}
\newcommand{\sfC}{{\mathsf{C}}}
\newcommand{\sfD}{{\mathsf{D}}}
\newcommand{\sfE}{{\mathsf{E}}}
\newcommand{\sfF}{{\mathsf{F}}}
\newcommand{\sfG}{{\mathsf{G}}}
\newcommand{\sfH}{{\mathsf{H}}}
\newcommand{\sfI}{{\mathsf{I}}}
\newcommand{\sfJ}{{\mathsf{J}}}
\newcommand{\sfK}{{\mathsf{K}}}
\newcommand{\sfL}{{\mathsf{L}}}
\newcommand{\sfM}{{\mathsf{M}}}
\newcommand{\sfN}{{\mathsf{N}}}
\newcommand{\sfO}{{\mathsf{O}}}
\newcommand{\sfP}{{\mathsf{P}}}
\newcommand{\sfQ}{{\mathsf{Q}}}
\newcommand{\sfR}{{\mathsf{R}}}
\newcommand{\sfS}{{\mathsf{S}}}
\newcommand{\sfT}{{\mathsf{T}}}
\newcommand{\sfU}{{\mathsf{U}}}
\newcommand{\sfV}{{\mathsf{V}}}
\newcommand{\sfW}{{\mathsf{W}}}
\newcommand{\sfX}{{\mathsf{X}}}
\newcommand{\sfY}{{\mathsf{Y}}}
\newcommand{\sfZ}{{\mathsf{Z}}}


\newcommand{\calA}{{\mathcal{A}}}
\newcommand{\calB}{{\mathcal{B}}}
\newcommand{\calC}{{\mathcal{C}}}
\newcommand{\calD}{{\mathcal{D}}}
\newcommand{\calE}{{\mathcal{E}}}
\newcommand{\calF}{{\mathcal{F}}}
\newcommand{\calG}{{\mathcal{G}}}
\newcommand{\calH}{{\mathcal{H}}}
\newcommand{\calI}{{\mathcal{I}}}
\newcommand{\calJ}{{\mathcal{J}}}
\newcommand{\calK}{{\mathcal{K}}}
\newcommand{\calL}{{\mathcal{L}}}
\newcommand{\calM}{{\mathcal{M}}}
\newcommand{\calN}{{\mathcal{N}}}
\newcommand{\calO}{{\mathcal{O}}}
\newcommand{\calP}{{\mathcal{P}}}
\newcommand{\calQ}{{\mathcal{Q}}}
\newcommand{\calR}{{\mathcal{R}}}
\newcommand{\calS}{{\mathcal{S}}}
\newcommand{\calT}{{\mathcal{T}}}
\newcommand{\calU}{{\mathcal{U}}}
\newcommand{\calV}{{\mathcal{V}}}
\newcommand{\calW}{{\mathcal{W}}}
\newcommand{\calX}{{\mathcal{X}}}
\newcommand{\calY}{{\mathcal{Y}}}
\newcommand{\calZ}{{\mathcal{Z}}}

\newcommand{\bara}{{\bar{a}}}
\newcommand{\barb}{{\bar{b}}}
\newcommand{\barc}{{\bar{c}}}
\newcommand{\bard}{{\bar{d}}}
\newcommand{\bare}{{\bar{e}}}
\newcommand{\barf}{{\bar{f}}}
\newcommand{\barg}{{\bar{g}}}
\newcommand{\barh}{{\bar{h}}}
\newcommand{\bari}{{\bar{i}}}
\newcommand{\barj}{{\bar{j}}}
\newcommand{\bark}{{\bar{k}}}
\newcommand{\barl}{{\bar{l}}}
\newcommand{\barm}{{\bar{m}}}
\newcommand{\barn}{{\bar{n}}}
\newcommand{\baro}{{\bar{o}}}
\newcommand{\barp}{{\bar{p}}}
\newcommand{\barq}{{\bar{q}}}
\newcommand{\barr}{{\bar{r}}}
\newcommand{\bars}{{\bar{s}}}
\newcommand{\bart}{{\bar{t}}}
\newcommand{\baru}{{\bar{u}}}
\newcommand{\barv}{{\bar{v}}}
\newcommand{\barw}{{\bar{w}}}
\newcommand{\barx}{{\bar{x}}}
\newcommand{\bary}{{\bar{y}}}
\newcommand{\barz}{{\bar{z}}}
\newcommand{\barA}{{\bar{A}}}
\newcommand{\barB}{{\bar{B}}}
\newcommand{\barC}{{\bar{C}}}
\newcommand{\barD}{{\bar{D}}}
\newcommand{\barE}{{\bar{E}}}
\newcommand{\barF}{{\bar{F}}}
\newcommand{\barG}{{\bar{G}}}
\newcommand{\barH}{{\bar{H}}}
\newcommand{\barI}{{\bar{I}}}
\newcommand{\barJ}{{\bar{J}}}
\newcommand{\barK}{{\bar{K}}}
\newcommand{\barL}{{\bar{L}}}
\newcommand{\barM}{{\bar{M}}}
\newcommand{\barN}{{\bar{N}}}
\newcommand{\barO}{{\bar{O}}}
\newcommand{\barP}{{\bar{P}}}
\newcommand{\barQ}{{\bar{Q}}}
\newcommand{\barR}{{\bar{R}}}
\newcommand{\barS}{{\bar{S}}}
\newcommand{\barT}{{\bar{T}}}
\newcommand{\barU}{{\bar{U}}}
\newcommand{\barV}{{\bar{V}}}
\newcommand{\barW}{{\bar{W}}}
\newcommand{\barX}{{\bar{X}}}
\newcommand{\barY}{{\bar{Y}}}
\newcommand{\barZ}{{\bar{Z}}}

\newcommand{\hX}{\hat{X}}
\newcommand{\Ent}{\mathsf{Ent}}
\newcommand{\awarm}{{A_{\text{warm}}}}
\newcommand{\thetaLS}{{\widehat{\theta}^{\text{\rm LS}}}}

\newcommand{\jiao}[1]{\langle{#1}\rangle}
\newcommand{\gaht}{\textsc{GoodActionHypTest}\;}
\newcommand{\iaht}{\textsc{InitialActionHypTest}\;}
\newcommand{\true}{\textsf{True}\;}
\newcommand{\false}{\textsf{False}\;}

% \usepackage[capitalize,noabbrev]{cleveref}
% \crefname{lemma}{Lemma}{Lemmas}
% \Crefname{lemma}{Lemma}{Lemmas}
% \crefname{thm}{Theorem}{Theorems}
% \Crefname{thm}{Theorem}{Theorems}
% \Crefname{assumption}{Assumption}{Assumptions}
% \Crefname{inftheorem}{Informal Theorem}{Informal Theorems}
% \crefformat{equation}{(#2#1#3)}

% % if you use cleveref..
% \usepackage[capitalize,noabbrev]{cleveref}
% \crefname{lemma}{Lemma}{Lemmas}
% \crefname{proposition}{Proposition}{Propositions}
% \crefname{remark}{Remark}{Remarks}
% \crefname{corollary}{Corollary}{Corollaries}
% \crefname{definition}{Definition}{Definitions}
% \crefname{conjecture}{Conjecture}{Conjectures}
% \crefname{figure}{Fig.}{Figures}

%%%%%%%%%%%%%%%%%%%%%%%%%%%%%%%%%%%%%%%%%%%%%%%%%%%%%%%%%%%%%%%%%%%%%%
%% mac.tex
%%
%% Umut A. Acar
%% Macros for adaptive computation paper.
%%%%%%%%%%%%%%%%%%%%%%%%%%%%%%%%%%%%%%%%%%%%%%%%%%%%%%%%%%%%%%%%%%%%%%

\newcommand{\readarrow}{\ensuremath{\Longrightarrow}}
\newcommand{\currenttime}{\ttt{currentTime}\xspace}
\newcommand{\tags}[1]{\ensuremath{\mathsf{tags}(#1)}}
\newcommand{\weight}[1]{\ensuremath{\mathsf{weight}(#1)}}
\newcommand{\dist}[2]{\ensuremath{\delta(#1,#2)}}


\newcommand{\cutspace}{\vspace{-4mm}}

\newcommand{\bomb}[1]{\fbox{\mbox{\emph{\bf {#1}}}}}

\newcommand{\myparagraph}[1]{\smallskip \noindent{\bf {#1}.}}
% formatting stuff
\newcommand{\codecolsep}{1ex}



\newcommand{\rlabel}[1]{\hspace*{-1mm}\mbox{\small{\bf ({#1})}}}

\newcommand{\tablerow}{\\[5ex]}
\newcommand{\tableroww}{\\[7ex]}
\newcommand{\tableline}{
\vspace*{2ex}\\
\hline\\ 
\vspace*{2ex}}

% Don't care
\newcommand{\dontcare}{\_
}

%% filter and quicksort stuff
\newcommand{\ncf}[2]{C^{fil}_{\ensuremath{{#1},{#2}}}}
\newcommand{\ncq}[1]{C^{qsort}_{\ensuremath{#1}}}
\newcommand{\nuf}[3]{P^{fil}_{#1,(\ensuremath{{#2},{#3}})}}
\newcommand{\nuq}[2]{P^{qsort}_{(\ensuremath{{#1},{#2}})}}

%% shorthands
\newcommand{\ddg}{{\sc ddg}}
\newcommand{\ncpa}{change-propagation algorithm}
\newcommand{\adg}{{\sc adg}}
\newcommand{\nwrite}{\texttt{write}}
\newcommand{\nread}{\texttt{read}}
\newcommand{\nmodr}{\texttt{mod}}
\newcommand{\ttt}[1]{\texttt{#1}}
\newcommand{\nmodl}{\texttt{modl}}
\newcommand{\nnil}{\ttt{NIL}}
\newcommand{\ncons}[2]{\ttt{CONS({\ensuremath{#1},\ensuremath{#2}})}}
\newcommand{\nfilter}{\texttt{filter}}
\newcommand{\nfilterp}{\texttt{filter'}}
\newcommand{\naqsort}{\texttt{qsort'}}
\newcommand{\nqsort}{\texttt{qsort}}
\newcommand{\nqsortp}{\texttt{qsort'}}
\newcommand{\nnewMod}{\texttt{newMod}}
\newcommand{\nchange}{\texttt{change}}
\newcommand{\npropagate}{\texttt{propagate}}
\newcommand{\ndest}{\texttt{d}}
\newcommand{\ninit}{\texttt{init}}



%% Comment sth out. 
\newcommand{\out}[1] {}
\newcommand{\sthat}{\ensuremath{~|~}}

%% definitions
\newcommand{\defi}[1]{{\bfseries\itshape #1}}


% Code listings.
\newcounter{codeLineCntr}
\newcommand{\codeLine}
 {\refstepcounter{codeLineCntr}{\thecodeLineCntr}}
\newcommand{\codeLineL}[1]
 {\refstepcounter{codeLineCntr}\label{#1}{\thecodeLineCntr}}
\newcommand{\codeLineNN}{} %% NN = No-Number (and no change to counter)

\newenvironment{codeListing}
 {\setcounter{codeLineCntr}{0}
%  \fontsize{10}{12}
 % the first one is width the second is height
 \fontsize{9}{11}
  \fontsize{8}{8}
  \vspace{-.1in}
  \ttfamily\begin{tabbing}}
  {\end{tabbing}
   \vspace{-.1in}}

\newenvironment{codeListing8}
 {\setcounter{codeLineCntr}{0}
%  \fontsize{8}{10}
  \fontsize{8}{8}
  \vspace{-.1in}
  \ttfamily
  \begin{tabbing}}
 {\end{tabbing}
 \vspace{-.1in}
}

\newenvironment{codeListing8h}
 {\setcounter{codeLineCntr}{0}
  \fontsize{8.5}{10.5}
  \vspace{-.1in}
  \ttfamily
  \begin{tabbing}}
 {\end{tabbing}
 \vspace{-.1in}
}


\newenvironment{codeListing9}
 {\setcounter{codeLineCntr}{0}
  \fontsize{9}{11}
  \vspace{-.1in}
  \ttfamily
  \begin{tabbing}}
 {\end{tabbing}
 \vspace{-.1in}
}

\newenvironment{codeListing10}
 {\setcounter{codeLineCntr}{0}
  \fontsize{10}{12}
  \vspace{-.1in}
  \ttfamily
  \begin{tabbing}}
 {\end{tabbing}
 \vspace{-.1in}
}


\newenvironment{codeListingNormal}
 {\setcounter{codeLineCntr}{0}
  \vspace{-.1in}
  \ttfamily
  \begin{tabbing}}
 {\end{tabbing}
 \vspace{-.1in}
}

\newcommand{\codeFrame}[1]
{\begin{center}\fbox{\parbox[t]{\columnwidth}{#1}}\end{center}
% \vspace*{-.15in}
}

\newcommand{\halfBox}[1]
{\begin{center}\fbox{\parbox[t]{\columnwidth}{#1}}\end{center}
% \vspace*{-.15in}
}

\newcommand{\fullBox}[1]
{\begin{center}\fbox{\parbox[t]{\textwidth}{#1}}\end{center}
% \vspace*{-.15in}
}

%%Note this is redefined in local-mac.tex for each paper.
\newcommand{\fixedCodeFrame}[1]
{
\begin{center}
\fbox{
\parbox[t]{0.9\columnwidth}{
#1
}
}\end{center}
}

% Footnote commands.
\newcommand{\footnotenonumber}[1]{{\def\thempfn{}\footnotetext{#1}}}

% Margin notes - use \notesfalse to turn off notes.
\setlength{\marginparwidth}{0.6in}
\reversemarginpar
\newif\ifnotes
\notestrue
\newcommand{\longnote}[1]{
  \ifnotes
    {\medskip\noindent Note:\marginpar[\hfill$\Longrightarrow$]
      {$\Longleftarrow$}{#1}\medskip}
  \fi}
\newcommand{\note}[1]{
  \ifnotes
    {\marginpar{\raggedright{\tiny #1}}}
  \fi}
\newcommand{\notered}[1]{\ifnotes
    {\marginpar{\raggedright{\tiny
          {\sf\color{red} #1}}}}
    \fi}


% Stuff not wanted.
\newcommand{\punt}[1]{}

% Sectioning commands.
\newcommand{\subsec}[1]{\subsection{\boldmath #1 \unboldmath}}
\newcommand{\subheading}[1]{\subsubsection*{#1}}
\newcommand{\subsubheading}[1]{\paragraph*{#1}}

% Reference shorthands.
\newcommand{\spref}[1]{Modified-Store Property~\ref{sp:#1}}
\newcommand{\prefs}[2]{Properties~\ref{p:#1} and~\ref{p:#2}}
\newcommand{\pref}[1]{Property~\ref{p:#1}}


\newcommand{\partref}[1]{Part~\ref{part:#1}}
\newcommand{\chref}[1]{Chapter~\ref{ch:#1}}
\newcommand{\chreftwo}[2]{Chapters \ref{ch:#1} and~\ref{ch:#2}}
\newcommand{\chrefthree}[3]{Chapters \ref{ch:#1}, and~\ref{ch:#2}, and~\ref{ch:#3}}
\newcommand{\secref}[1]{Section~\ref{sec:#1}}
\newcommand{\subsecref}[1]{Subsection~\ref{subsec:#1}}
\newcommand{\secreftwo}[2]{Sections \ref{sec:#1} and~\ref{sec:#2}}
\newcommand{\secrefthree}[3]{Sections \ref{sec:#1},~\ref{sec:#2},~and~\ref{sec:#3}}
\newcommand{\appref}[1]{Appendix~\ref{app:#1}}
\newcommand{\figref}[1]{Figure~\ref{fig:#1}}
\newcommand{\figreftwo}[2]{Figures \ref{fig:#1} and~\ref{fig:#2}}
\newcommand{\figrefthree}[3]{Figures \ref{fig:#1}, \ref{fig:#2} and~\ref{fig:#3}}
\newcommand{\figreffour}[4]{Figures \ref{fig:#1},~\ref{fig:#2},~\ref{fig:#3}~and~\ref{fig:#4}}
\newcommand{\figpageref}[1]{page~\pageref{fig:#1}}
\newcommand{\tabref}[1]{Table~\ref{tab:#1}}
\newcommand{\tabreftwo}[2]{Tables~\ref{tab:#1} and~\ref{tab:#1}}

\newcommand{\stref}[1]{step~\ref{step:#1}}
\newcommand{\caseref}[1]{case~\ref{case:#1}}
\newcommand{\lineref}[1]{line~\ref{line:#1}}
\newcommand{\linereftwo}[2]{lines \ref{line:#1} and~\ref{line:#2}}
\newcommand{\linerefthree}[3]{lines \ref{line:#1},~\ref{line:#2},~and~\ref{line:#3}}
\newcommand{\linerefrange}[2]{lines \ref{line:#1} through~\ref{line:#2}}
\newcommand{\thmref}[1]{Theorem~\ref{thm:#1}}
\newcommand{\thmreftwo}[2]{Theorems \ref{thm:#1} and~\ref{thm:#2}}
\newcommand{\thmrefthree}[3]{Theorems \ref{thm:#1}, \ref{thm:#2} and~\ref{thm:#3}}
\newcommand{\thmpageref}[1]{page~\pageref{thm:#1}}
\newcommand{\lemref}[1]{Lemma~\ref{lem:#1}}
\newcommand{\lemreftwo}[2]{Lemmas \ref{lem:#1} and~\ref{lem:#2}}
\newcommand{\lemrefthree}[3]{Lemmas \ref{lem:#1},~\ref{lem:#2},~and~\ref{lem:#3}}
\newcommand{\lempageref}[1]{page~\pageref{lem:#1}}
\newcommand{\corref}[1]{Corollary~\ref{cor:#1}}
\newcommand{\defref}[1]{Definition~\ref{def:#1}}
\newcommand{\defreftwo}[2]{Definitions \ref{def:#1} and~\ref{def:#2}}
\newcommand{\defpageref}[1]{page~\pageref{def:#1}}
\renewcommand{\eqref}[1]{Equation~(\ref{eq:#1})}
\newcommand{\eqreftwo}[2]{Equations (\ref{eq:#1}) and~(\ref{eq:#2})}
\newcommand{\eqpageref}[1]{page~\pageref{eq:#1}}
\newcommand{\ineqref}[1]{Inequality~(\ref{ineq:#1})}
\newcommand{\ineqreftwo}[2]{Inequalities (\ref{ineq:#1}) and~(\ref{ineq:#2})}
\newcommand{\ineqpageref}[1]{page~\pageref{ineq:#1}}
\newcommand{\itemref}[1]{Item~\ref{item:#1}}
\newcommand{\itemreftwo}[2]{Item~\ref{item:#1} and~\ref{item:#2}}

% Useful shorthands.
\newcommand{\abs}[1]{\left| #1\right|}
\newcommand{\card}[1]{\left| #1\right|}
\newcommand{\norm}[1]{\left\| #1\right\|}
\newcommand{\floor}[1]{\left\lfloor #1 \right\rfloor}
\newcommand{\ceil}[1]{\left\lceil #1 \right\rceil}
  \renewcommand{\choose}[2]{{{#1}\atopwithdelims(){#2}}}
%\newcommand{\ang}[1]{\langle#1\rangle}
\newcommand{\paren}[1]{\left(#1\right)}
\newcommand{\prob}[1]{\Pr\left\{ #1 \right\}}
\newcommand{\expect}[1]{\mathrm{E}\left[ #1 \right]}
\newcommand{\expectsq}[1]{\mathrm{E}^2\left[ #1 \right]}
\newcommand{\variance}[1]{\mathrm{Var}\left[ #1 \right]}
\newcommand{\twodots}{\mathinner{\ldotp\ldotp}}

% Standard number sets.
\newcommand{\reals}{{\mathrm{I}\!\mathrm{R}}}
\newcommand{\integers}{\mathbf{Z}}
\newcommand{\naturals}{{\mathrm{I}\!\mathrm{N}}}
\newcommand{\rationals}{\mathbf{Q}}
\newcommand{\complex}{\mathbf{C}}

% Special styles.
\newcommand{\proc}[1]{\ifmmode\mbox{\textsc{#1}}\else\textsc{#1}\fi}
\newcommand{\procdecl}[1]{
  \proc{#1}\vrule width0pt height0pt depth 7pt \relax}
  \newcommand{\func}[1]{\ifmmode\mathrm{#1}\else\textrm{#1}fi} %
%  Multiple cases.  
\renewcommand{\cases}[1]{\left\{
  \begin{array}{ll}#1\end{array}\right.}
  \newcommand{\cif}[1]{\mbox{if $#1$}} 

%% spacing hacks
\newcommand{\longpage}{\enlargethispage{\baselineskip}}
\newcommand{\shortpage}{\enlargethispage{-\baselineskip}}



%% Notes, todos, and remarks
\newcounter{remark}[section]

\newcommand{\myremark}[3]{
\refstepcounter{remark}
\[
\left\{
\sf 
\parbox{\columnwidth}{
{\bf {#1}'s remark~\theremark:} 
{#3}
}
\right\}
\]
%\marginpar{\bf {#2}.~\theremark}
}


% - - - - - - - - - - - - - - - - - - - - - - - - - - - - - - - - - - - - - - - - - - - - 
% For amsthm package:

%\theoremstyle{plain}
%\newtheorem{thm}{Theorem}[section]
%% \newtheorem{lem}[thm]{Lemma}
%% \newtheorem{prop}[thm]{Proposition}
%% \newtheorem*{cor}{Corollary}

%% \theoremstyle{definition}
%% \newtheorem{defn}{Definition}[section]
%% \newtheorem{conj}{Conjecture}[section]
%% \newtheorem{falseconj}{False~Conjecture}[section]
%% \newtheorem{exmp}{Example}[section]

%% \theoremstyle{remark}
%% \newtheorem*{rem}{Remark}
%% %\newtheorem*{note}{Note}
%% \newtheorem{case}{Case}


\newcommand{\uremark}[1]{\myremark{Umut}{U}{#1}}
\newcommand{\ur}[1]{\uremark{#1}}
\newcommand{\rremark}[1]{\myremark{Ruy}{R}{#1}}
\newcommand{\mremark}[1]{\myremark{Matthew}{M}{#1}}
\newcommand{\todoremark}[1]{\myremark{TODO}{TODO}{#1}}
%\newcommand{\todo}[1]{\myremark{TODO}{TODO}{#1}}
\newcommand{\todo}[1]{{\bf{[TODO:{#1}]}}}

%%

%%%%%%%%%%%%%%%%%%%%%%%%%%%%%%%%%%%%%%%%%%%%
%% For Submissions, acmart template
%%%%%%%%%%%%%%%%%%%%%%%%%%%%%%%%%%%%%%%%%%%%
\setcopyright{none}
\renewcommand\footnotetextcopyrightpermission[1]{} % This line removes the footnote about the conference and year.
\def\@titlefont{\huge\sffamily\bfseries} % THIS LINE CHANGES THE FONT OF THE TITLE

%%%%%%%%%%%%%%%%%%%%%%%%%%%%%%%%%%%%%%%%%%%%
%% PENALTY
%%%%%%%%%%%%%%%%%%%%%%%%%%%%%%%%%%%%%%%%%%%%

% See their definitions and default values in: https://en.wikibooks.org/wiki/TeX/penalty
\binoppenalty=700
\brokenpenalty=0 %100
\clubpenalty=150   %150
\displaywidowpenalty=50   %50
\exhyphenpenalty=0 %50
\floatingpenalty=0 %20000
\hyphenpenalty=0 %50
\interlinepenalty=0
\linepenalty=10
\postdisplaypenalty=0
\predisplaypenalty=0 %10000
\relpenalty=0 %500
\widowpenalty=0  %150

%%%%%%%%%%%%%%%%%%%%%%%%%%%%%%%%%%%%%%%%%%%%
%% Floating: Figure / Table / Algorithm
%%%%%%%%%%%%%%%%%%%%%%%%%%%%%%%%%%%%%%%%%%%%
\usepackage{float}
\usepackage[labelfont=bf,font={small},aboveskip=0em, belowskip=0em]{caption}

%%%%%%%%%% Floating Spacing %%%%%%%%%%%%%
% Can also set space around the caption separately
%\setlength\abovecaptionskip{0em}
%\setlength\belowcaptionskip{0em}
% Space between multiple floatings
\setlength{\floatsep}{0em}
% space below floating (distance to the rest of text)
\setlength{\textfloatsep}{0.5em}
% space above tables/figures (distance from the text above)
% for tables in the middle of the page (i.e., not top or bottom), this number is both the top spacing and bottom spacing
\setlength{\intextsep}{0.5em}
%%%%% These two are for double-column floatings, e.g., figure* and table*
\setlength{\dbltextfloatsep}{1em} % floating to text
\setlength{\dblfloatsep}{0.5em} % between floatings


%%%%%%%%%%%%%%%%%%%%%%%%%%%%%%%%%%%%%%%%%%%%
%% Section titles
%%%%%%%%%%%%%%%%%%%%%%%%%%%%%%%%%%%%%%%%%%%%
\usepackage{titlesec}
%% Change section and subsection title to normal font size
%\titleformat{\section}{\normalfont\large\bfseries}{\thesection}{1em}{}
%\titleformat{\subsection}{\normalfont\large\bfseries}{\thesection}{1em}{}

% Change title spacing
\titlespacing{\section}{0pt}{0.3em}{0.2em} % left margin, space before, space after
\titlespacing{\subsection}{0pt}{0.3em}{0.15em} % left margin, space before, space after
\titlespacing{\subsubsection}{0pt}{0.3em}{0.5em} % left margin, space before, space after (horizontal)
%\newcommand{\mysubsubsection}[1]{\underline{#1}.}
%\titleformat{\subsubsection}[runin]
%{\normalfont\normalsize\bfseries}{\thesubsubsection}{1em}{\mysubsubsection}

%\newcommand{\para}[1]{{\bf \emph{#1}}\,}
%\newcommand{\myparagraph}[1]{\noindent\emp{#1} \quad}


\usepackage{grumble}
\usepackage{subfigure}
\usepackage{caption}
\usepackage{booktabs}
\usepackage{multirow}
\usepackage{listings}
\usepackage{xcolor}
\usepackage{float}
% \usepackage{subfigure}
% \usepackage{subcaption}
\usepackage{xspace}
\usepackage[binary-units=true]{siunitx}
% linux libertine for normal text
%\usepackage{libertine}
%\usepackage{libertinust1math}
% inconsolate as teletype font
% \usepackage{inconsolata}
\usepackage{tikz}

% \usepackage{subcaption}
\usepackage{xcolor,colortbl}
\usepackage{algorithmic}
\usepackage[ruled,vlined,linesnumbered]{algorithm2e}
%\newtheorem{theorem}{Theorem}
%\newtheorem{corollary}{Corollary}
%\newtheorem{lemma}{Lemma}
\usepackage{grumble}

% footnote w/o a marker
\newcommand\blfootnote[1]{%
  \begingroup
  \renewcommand\thefootnote{}\footnote{#1}%
  \addtocounter{footnote}{-1}%
  \endgroup
}

\newcommand*\circled[1]{\tikz[baseline=(char.base)]{
            \node[shape=circle,draw,inner sep=0.2pt] (char) {#1};}}

\newcommand{\projecttitle}{\textsc{tnic}\xspace}
\newcommand{\projectlibrary}{\textsc{net-lib}\xspace}
\newcommand{\trustedfpga}{\textsc{t-fpga}\xspace}
\newcommand{\trustednic}{\textsc{trusted-nic}\xspace}

\definecolor{codegreen}{rgb}{0,0.6,0}
\definecolor{codegray}{rgb}{0.5,0.5,0.5}
\definecolor{codepurple}{rgb}{0.58,0,0.82}
\definecolor{backcolour}{rgb}{0.95,0.95,0.92}

%\renewcommand{\ttdefault}{cmtt} % Courier clon
\definecolor{lightGrey}{rgb}{0.9, 0.9, 0.9}
\definecolor{beaublue}{rgb}{0.74, 0.83, 0.9}
\definecolor{lightred}{RGB}{229, 220, 220}
\newcommand{\Hilight}{\makebox[0pt][l]{\color{beaublue}\rule[-0.35em]{\linewidth}{1.2em}}}
\newcommand{\Hilightpart}{\makebox[0pt][l]{\color{beaublue}\rule[-0.35em]{\linewidth}{1.2em}}}
\newcommand{\Hilightt}{\makebox[0pt][l]{\color{lightred}\rule[-0.35em]{\linewidth}{1.2em}}}
\definecolor{burlywood}{rgb}{0.87, 0.72, 0.53}
\newcommand{\HilightY}{\makebox[0pt][l]{\color{burlywood}\rule[-0.35em]{0.3\linewidth}{1.2em}}}
\newcommand{\Hilightsmall}{\makebox[0pt][l]{\color{burlywood}\rule[-0.35em]{0.1\linewidth}{1.2em}}}
\newcommand{\Hilightmedium}{\makebox[0pt][l]{\color{burlywood}\rule[-0.35em]{0.25\linewidth}{1.2em}}}
\newcommand{\HilightYlinewidth}{\makebox[0pt][l]{\color{burlywood}\rule[-0.35em]{\linewidth}{1.2em}}}

\lstdefinestyle{customc}{
    backgroundcolor=\color{backcolour},   
    commentstyle=\color{codegreen},
    keywordstyle=\color{magenta},
    numberstyle=\tiny\color{codegray},
    stringstyle=\color{codepurple},
    basicstyle=\ttfamily\footnotesize,
    breaklines,
    tabsize=2,
    numbers=left,
    columns=fullflexible,
    keepspaces=true,
    frame=lines,
    numbersep=4pt,
    escapechar=@,
    mathescape=true,
    captionpos=b,
    language=c++,
    keywords = {auto, new, void, Raft_ctx, Msg, for}
}

%%%%%%%%%%%%%%%%%%%%%%%%%%%%%%%%%%%%%%%%%%%%%
%% For Submissions, acmart template
%%%%%%%%%%%%%%%%%%%%%%%%%%%%%%%%%%%%%%%%%%%%
\setcopyright{none}
\renewcommand\footnotetextcopyrightpermission[1]{} % This line removes the footnote about the conference and year.
\def\@titlefont{\huge\sffamily\bfseries} % THIS LINE CHANGES THE FONT OF THE TITLE

%%%%%%%%%%%%%%%%%%%%%%%%%%%%%%%%%%%%%%%%%%%%
%% PENALTY
%%%%%%%%%%%%%%%%%%%%%%%%%%%%%%%%%%%%%%%%%%%%

% See their definitions and default values in: https://en.wikibooks.org/wiki/TeX/penalty
\binoppenalty=700
\brokenpenalty=0 %100
\clubpenalty=150   %150
\displaywidowpenalty=50   %50
\exhyphenpenalty=0 %50
\floatingpenalty=0 %20000
\hyphenpenalty=0 %50
\interlinepenalty=0
\linepenalty=10
\postdisplaypenalty=0
\predisplaypenalty=0 %10000
\relpenalty=0 %500
\widowpenalty=0  %150

%%%%%%%%%%%%%%%%%%%%%%%%%%%%%%%%%%%%%%%%%%%%
%% Floating: Figure / Table / Algorithm
%%%%%%%%%%%%%%%%%%%%%%%%%%%%%%%%%%%%%%%%%%%%
\usepackage{float}
\usepackage[labelfont=bf,font={small},aboveskip=0em, belowskip=0em]{caption}

%%%%%%%%%% Floating Spacing %%%%%%%%%%%%%
% Can also set space around the caption separately
%\setlength\abovecaptionskip{0em}
%\setlength\belowcaptionskip{0em}
% Space between multiple floatings
\setlength{\floatsep}{0em}
% space below floating (distance to the rest of text)
\setlength{\textfloatsep}{0.5em}
% space above tables/figures (distance from the text above)
% for tables in the middle of the page (i.e., not top or bottom), this number is both the top spacing and bottom spacing
\setlength{\intextsep}{0.5em}
%%%%% These two are for double-column floatings, e.g., figure* and table*
\setlength{\dbltextfloatsep}{1em} % floating to text
\setlength{\dblfloatsep}{0.5em} % between floatings


%%%%%%%%%%%%%%%%%%%%%%%%%%%%%%%%%%%%%%%%%%%%
%% Section titles
%%%%%%%%%%%%%%%%%%%%%%%%%%%%%%%%%%%%%%%%%%%%
\usepackage{titlesec}
%% Change section and subsection title to normal font size
%\titleformat{\section}{\normalfont\large\bfseries}{\thesection}{1em}{}
%\titleformat{\subsection}{\normalfont\large\bfseries}{\thesection}{1em}{}

% Change title spacing
\titlespacing{\section}{0pt}{0.3em}{0.2em} % left margin, space before, space after
\titlespacing{\subsection}{0pt}{0.3em}{0.15em} % left margin, space before, space after
\titlespacing{\subsubsection}{0pt}{0.3em}{0.5em} % left margin, space before, space after (horizontal)
%\newcommand{\mysubsubsection}[1]{\underline{#1}.}
%\titleformat{\subsubsection}[runin]
%{\normalfont\normalsize\bfseries}{\thesubsubsection}{1em}{\mysubsubsection}

%\newcommand{\para}[1]{{\bf \emph{#1}}\,}
%\newcommand{\myparagraph}[1]{\noindent\emp{#1} \quad}


\begin{document}
\author{Dimitra Giantsidi}
\affiliation{%
  \institution{The University of Edinburgh}
  % \country{Germany}
}
% \email{d.giantsidi@sms.ed.ac.uk}
\author{Julian Pritzi}
\affiliation{%
  \institution{Technical University of Munich}
  % \country{Germany}
}
% \email{julian.pritzi@tum.de}
\author{Felix Gust}
\affiliation{%
  \institution{Technical University of Munich}
  % \country{Germany}
}
% \email{gustf@in.tum.de}
\author{Antonios Katsarakis$^*$}
\affiliation{%
  \institution{Huawei Research}
  % \country{Germany}
}
% \email{antonios.katsarakis@huawei.com}
\author{Atsushi Koshiba}
\affiliation{%
  \institution{Technical University of Munich}
  % \country{Germany}
}
% \email{atsushi.koshiba@tum.de}
\author{Pramod Bhatotia}
\affiliation{%
  \institution{Technical University of Munich}
  % \country{Germany}
}
% \email{pramod.bhatotia@tum.de}
\renewcommand{\shortauthors}{Giantsidi, et al.}

%make title bold and 14 pt font (Latex default is non-bold, 16 pt)
% \title{{TNIC: A Trusted NIC Architecture}\\ \vspace{-2mm}
% {\large A hardware-network substrate for building high-performance trustworthy distributed systems}}
\title{{TNIC: A Trusted NIC Architecture}}
\subtitle{{\large \bf A hardware-network substrate for building high-performance trustworthy distributed systems}}
\renewcommand{\shorttitle}{TNIC: A Trusted NIC Architecture}

\begin{abstract}  
Test time scaling is currently one of the most active research areas that shows promise after training time scaling has reached its limits.
Deep-thinking (DT) models are a class of recurrent models that can perform easy-to-hard generalization by assigning more compute to harder test samples.
However, due to their inability to determine the complexity of a test sample, DT models have to use a large amount of computation for both easy and hard test samples.
Excessive test time computation is wasteful and can cause the ``overthinking'' problem where more test time computation leads to worse results.
In this paper, we introduce a test time training method for determining the optimal amount of computation needed for each sample during test time.
We also propose Conv-LiGRU, a novel recurrent architecture for efficient and robust visual reasoning. 
Extensive experiments demonstrate that Conv-LiGRU is more stable than DT, effectively mitigates the ``overthinking'' phenomenon, and achieves superior accuracy.
\end{abstract}  

%%
%% The code below is generated by the tool at http://dl.acm.org/ccs.cfm.
%% Please copy and paste the code instead of the example below.
%%
\begin{CCSXML}
<ccs2012>
   <concept>
       <concept_id>10002978.10003006.10003007.10003009</concept_id>
       <concept_desc>Security and privacy~Trusted computing</concept_desc>
       <concept_significance>500</concept_significance>
       </concept>
%    <concept>
%        <concept_id>10010520.10010521.10010537.10003100</concept_id>
%        <concept_desc>Computer systems organization~Cloud computing</concept_desc>
%        <concept_significance>500</concept_significance>
%        </concept>
%  </ccs2012>
\end{CCSXML}

\ccsdesc[500]{Security and privacy~Trusted computing}
% \ccsdesc[500]{Computer systems organization~Cloud computing}

%%
%% Keywords. The author(s) should pick words that accurately describe
%% the work being presented. Separate the keywords with commas.
\keywords{trusted computing, hardware-software co-design}

\maketitle

% Use the following at camera-ready time to suppress page numbers.
% Comment it out when you first submit the paper for review.
%\thispagestyle{empty}

\section{Introduction}
\blfootnote{$^*$This work started when the author was at the University of Edinburgh.}
Distributed systems are integral to the third-party cloud infrastructure~\cite{amazon_ec2, microsoft_azure, rackspace, google_engine}. While these systems manifest in diverse forms (e.g., storage systems~\cite{dynamo, azure_storage, tao, spanner, 51, zippy, AmazonS3}, management services~\cite{Hunt:2010, Burns2016}, computing frameworks~\cite{aws_lambda, azure_functions, google_cloud_functions}) they all must be fast and remain correct upon failures. %when failures occur. 

Unfortunately, the widespread adoption of the cloud has drastically increased the surface area of attacks and faults~\cite{Gunawi_bugs-in-the-cloud, Shinde2016, high_resolution_side_channels} that are beyond the traditional fail-stop (or crash fault) model~\cite{delporte}. The modern (untrusted) third-party cloud infrastructure severely suffers from arbitrary  ({\em Byzantine}) \linebreak faults~\cite{Lamport:1982} that can range from malicious (network) attacks to configuration errors and bugs and are capable of irreversibly disrupting the correct execution of the system~\cite{Gunawi_bugs-in-the-cloud, Shinde2016, high_resolution_side_channels, Castro:2002}.
% ford2010availability, Mazieres2002b, Garay2000}.

A promising solution to build trustworthy distributed systems that can sustain Byzantine failures is based on the {\em silicon root of trust}---specifically, the Trusted Execution Environments (TEEs)~\cite{cryptoeprint:2016:086, arm-realm, amd-sev, riscv-multizone, intelTDX}. While the TEEs offer a (single-node) isolated Trusted Computing Base (TCB),  we have identified three core challenges ($\S$~\ref{subsec:challenges}) that complicate their adoption for building trustworthy distributed systems spanning multiple nodes in Byzantine cloud environments.

{\bf \em First, TEEs in heterogeneous cloud environments introduce programmability and security challenges}. A cloud environment offers diverse heterogeneous host-side CPUs with different TEEs (e.g., Intel SGX/TDX, AMD SEV-SNP, AWS Nitro Enclaves, Arm TrustZone/CCA, RISC-V Keystone). These heterogeneous host-side TEEs require different programming models and offer varying security properties. Therefore, they cannot (easily) provide a generic substrate for building trustworthy distributed systems. Our work overcomes this challenge by designing a {\em host CPU-agnostic} {\em silicon root of trust} at the network interface (NIC) level ($\S$~\ref{sec:t-nic-hardware}). We provide a generic programming API ($\S$~\ref{sec:t-nic-software}) and a {\em recipe} ($\S$~\ref{subsec:transformation}) for building high-performance, trustworthy distributed systems ($\S$~\ref{sec:use_cases}).
%, exposing a {\em unified trusted} network-level isolation 

{\bf \em Secondly, TEEs with a large TCB are plagued with security vulnerabilities, rendering them non-verifiable}. With hundreds of security bugs already uncovered~\cite{10.1145/3456631}, TEEs' large TCBs further increase their security vulnerabilities~\cite{10.1145/3379469, 10.5555/1756748.1756832}, impeding a formal verification of their security. We overcome this with a {\em minimalistic verifiable TCB} ($\S$~\ref{subsec:nic_attest_kernel}). Our TCB resides at the NIC hardware and is equipped with {\em the lower bound of security primitives};  we provide only two key security properties of non-equivocation and transferable authentication for building trustworthy distributed systems ($\S$~\ref{subsec:trustworthy_ds}). Since we strive for a minimal trusted interface, we can (and we did) formally verify the security properties of our TCB ($\S$~\ref{subsec::formal_verification_remote_attestation}). 

{\bf \em Thirdly, TEEs report significant performance bottlenecks.} TEEs syscalls execution for (network) I/O is extremely costly~\cite{hotcalls}, whereas even state-of-the-art network stacks showed a lower bound of 4$\times$ slowdown~\cite{avocado}. We attack this challenge based on two aspects. First, we build a scalable transformation with our minimal TCB's security properties ($\S$~\ref{subsec:transformation}) to transform Byzantine faults (3$f$+1) to much cheaper crash faults (2$f$+1) for tolerating $f$  (distributed) Byzantine nodes.  Secondly, we design hardware-accelerated offload of the security computation at the NIC level by extending the scope of SmartNICs with {\em the lower bound of security primitives} ($\S$~\ref{sec:t-nic-hardware}) while offering kernel-bypass networking ($\S$~\ref{sec:t-nic-network}).


To overcome these challenges, we present  \projecttitle{}, a trusted NIC architecture for building trustworthy distributed systems deployed in Byzantine cloud environments. \projecttitle{} realizes an abstraction of trustworthy network-level isolation by building a hardware-accelerated silicon root of trust at the NIC level. Overall, \projecttitle{} follows a layered design:
\begin{itemize}[leftmargin=*]
    \item {\bf Trusted NIC hardware architecture ($\S$~\ref{sec:t-nic-hardware}):}  We materialize a \underline{minimalistic}, \underline{verifiable}, and \underline{host-CPU-agnostic} TCB at the network interface level as the key component to design trusted distributed systems for Byzantine settings. Our TCB guarantees the security properties of non-equivocation and transferable authentication that suffice to implement an efficient transformation of systems for Byzantine settings. We build \projecttitle{} on top of FPGA-based SmartNICs~\cite{u280_smartnics}. We formally verify the safety and security guarantees of \projecttitle{} protocols using Tamarin Prover~\cite{tamarin-prover}. 

    \item \rev{(a)}{{\bf Network stack ($\S$~\ref{sec:t-nic-network}) and library ($\S$~\ref{sec:t-nic-software}):} Based on the \projecttitle{} architecture, we design a \underline{HW-accelerated} network stack to access the hardware bypassing kernel for performance. On top of \projecttitle{}'s network stack, we present a networking library that exposes a \underline{simplified} programming model. We show {\em how to use} \projecttitle{} APIs to construct a \underline{generic transformation} of a distributed system operating under the CFT model to target Byzantine settings.}
    % \item {\bf Trusted network stack ($\S$~\ref{sec:t-nic-network}) and library ($\S$~\ref{sec:t-nic-software}):} Based on the \projecttitle{} architecture, we design a trusted \underline{HW-accelerated} network stack to access the hardware bypassing kernel for performance. On top of \projecttitle{}'s network stack, we present a trusted networking library that exposes a \underline{simplified} programming model. We show {\em how to use} \projecttitle{} APIs to construct a \underline{generic transformation} of a distributed system operating under the CFT model to target Byzantine settings.
    
    \item {\bf Trusted distributed systems using \projecttitle{} ($\S$~\ref{sec:use_cases}):} We build with \projecttitle{} the following (distributed) systems for Byzantine environments: Attested Append-only Memory (A2M)~\cite{A2M}, Byzantine Fault Tolerance (BFT)~\cite{pbft}, Chain Replication~\cite{chain-replication}, and Accountability with PeerReview~\cite{peer-review}---showing the \underline{generality of our approach}.
\end{itemize}



% We build \projecttitle{} on top of Alveo U280 FPGA-based SmartNICs~\cite{u280_smartnics}.  extending the Coyote system~\cite{coyote}.
% Our core component is the \projecttitle{}'s {\em attestation kernel} ($\S$~\ref{subsec:nic_attest_kernel}), the minimal required hardware-assisted TCB that guarantees the lower bound of security properties for BFT across the network ($\S$~\ref{sec:requirements-ds}). 
% We formally verify the safety and correctness properties of all \projecttitle{}'s operations (i.e., from remote attestation to networking) using Tamarin prover~\cite{tamarin-prover} ($\S$~\ref{subsec::formal_verification_remote_attestation}). Our 
% \projecttitle{}'s attestation kernel resides on the network data path to optimize for latency while we further design a unified trusted network stack to implement user-space networking following the RDMA programming paradigm ($\S$~\ref{subsec:roce_protocol_kernel} and $\S$~\ref{sec:t-nic-network}). Lastly, we leverage our \projecttitle{} trusted network library($\S$~\ref{sec:t-nic-software}) to show a generic {\em recipe} ($\S$~\ref{subsec:transformation})to transform a distributed system operating under the fail-stop model for Byzantine settings ($\S$~\ref{subsec:transformation}). 



We evaluate \projecttitle{} with a  state-of-the-art software-based network stack, eRPC~\cite{erpc}, on top of RDMA~\cite{rdma}/DPDK~\cite{dpdk} with two different TEEs (Intel SGX~\cite{intel-sgx} and AMD-sev~\cite{amd-sev}). Our evaluation shows that \projecttitle{} offers 3$\times$---5$\times$ lower latency than the software-based approach with the CPU-based TEEs. For trusted distributed systems, \projecttitle{} improves throughput by up to $6\times$ compared to their TEE-based implementations.
 % We evaluate the \projecttitle{}-based versions of the four implemented systems against their TEEs-based versions. 


\section{Motivation and Background}
\label{sec:requirements-ds}

We first examine the design requirements for high-performance, trustworthy distributed systems for cloud environments. % hosted in the untrusted heterogeneous cloud infrastructure.

\subsection{Trustworthy Distributed Systems}\label{subsec:trustworthy_ds}
\myparagraph{Byzantine fault model} 
In the untrusted cloud infrastructure, arbitrary (Byzantine) faults are a frequent occurrence in the wild~\cite{Gunawi_bugs-in-the-cloud, Shinde2016, 10.1145/1189256.1189259, 10.5555/1267308.1267318}. To this end, system designers introduced Byzantine Fault Tolerant (BFT) systems that remain correct even under the presence of (a bounded number of) Byzantine failures~\cite{Lamport:1982}. \rev{(b)}{Traditional BFT protocols need \emph{at least} $3f+1$ nodes in order to provide consistent replication while tolerating up to $f$ Byzantine failures.} While BFT accurately captures the realistic security needs in the cloud~\cite{bft_made_practical}, it is rarely adopted in practice~\cite{bftForSkeptics} due to its complexity and limited performance~\cite{268273, 10.1145/2658994}. 

\myparagraph{Crash fault model} 
The vast majority of cloud applications operate under the fail-stop (crash fault) model~\cite{spanner, 27897, cockroachdb_raft, zippydb, foundationdb}, optimistically {\em assuming} that the entire cloud infrastructure is trusted and only fails by crashing~\cite{delporte}. \rev{(b)}{Compared to BFT replication, Crash Fault Tolerant (CFT) protocols~\cite{10.1145/279227.279229, raft, primary-backup, Hunt:2010}, require $2f+1$ replicas to tolerate $f$ (yet non-Byzantine) failures.} While CFT systems can offer performance and scalability~\cite{f04eb9b864204bab958e72055062748c}, they are fundamentally incapable of ensuring safety in the presence of non-benign faults, hence, are ill-suited for the modern cloud. 

% In the untrusted cloud infrastructure, arbitrary (Byzantine) faults are a frequent occurrence in the wild~\cite{Gunawi_bugs-in-the-cloud, Shinde2016, 10.1145/1189256.1189259, 10.5555/1267308.1267318}. To this end, system designers introduced Byzantine Fault Tolerant (BFT) systems that remain correct even under the presence of (a bounded number of) Byzantine failures~\cite{Lamport:1982}. While BFT accurately captures the realistic security needs in the cloud~\cite{bft_made_practical}, it is rarely adopted in practice~\cite{bftForSkeptics} due to its complexity and limited performance~\cite{268273, 10.1145/2658994}. In contrast, the vast majority of cloud applications operate under the fail-stop (crash fault) model, optimistically {\em assuming} that the entire cloud infrastructure is trusted and only fails by crashing~\cite{delporte}. While Crash Fault Tolerant (CFT) systems usually offer performance and scalability~\cite{f04eb9b864204bab958e72055062748c}, they are ill-suited for the modern cloud as they are fundamentally incapable of ensuring safety in the presence of non-benign faults. 
 
% \noindent{\bf{Security properties for BFT.}} 

\myparagraph{Security properties for BFT} 
\rev{(b), A2, A4}{
We seek to build BFT systems while reducing their programmability and performance overheads. Our approach, inspired by the theoretical findings of Clement et al.~\cite{clement2012}, {\em transforms} CFT systems into BFT systems by providing the {\em lower bound} of security properties, i.e., {\em transferable authentication} and {\em non-equivocation}.
}

% \rev{A2}{We next explain the properties:}
% , which are minimal security properties required to build trustworthy systems under the BFT model. 

% \myparagraph{Transferable authentication}
\revcont{
We next explain the two security properties. First, {\em transferable authentication} allows a node to verify the original sender of a received message, even if it is forwarded by other than the original sender. Assuming that the sender $p_i$ sends an authenticated message $m$ to a recipient $p_j$, the authenticated message $m$ is accompanied by an authentication token $\sigma (p_i)$ that allows  $p_j$ to verify that $p_i$ generated the message, e.g., {verify($m, \sigma (p_i)$)}. Authentication tokens are unforgeable:
\begin{itemize}[leftmargin=*]
  \item if $p_i$ is correct, then {verify($m, \sigma (p_i)$)} is true if and only if $p_i$ generated $m$.
  \item if $p_i$ is faulty, {verify($m, \sigma (p_i)$)} $\wedge$ {verify($m', \sigma (p_i)$)} $\Rightarrow$ $m = m'$. As such, a compromised $p_i$ cannot produce two valid different messages that can be verified with the same token $\sigma (p_i)$.
\end{itemize}
As an authentication token is transferable, it allows another recipient $p_k$ to evaluate {verify($m, \sigma (p_i)$)} in the same way even when $m$ and $\sigma (p_i)$ are forwarded from $p_j$.
}

\revcont{
Second, {\em non-equivocation} guarantees that a node cannot make conflicting statements to different nodes. Equivocation also manifests as network adversaries or replay attacks that send invalid messages or re-send valid but stale messages.
}

\revcont{The seminal paper~\cite{clement2012} proves that, given these two properties, a transformation from any CFT protocol to a BFT protocol is {\emph {always}} possible without increasing the number of participating nodes; e.g., a reliable broadcast can be implemented to tolerate up to $f$ Byzantine failures in an asynchronous system with $2f+1$ replicas, rather than the conventional $3f+1$.}
% An authentication token provides transferable authentication if the correct processes $p_j$ and $p_k$ always evaluate \texttt{verify($m, \sigma (p_i)$)} in the same way even when $p_k$ receives message $m$ and authentication token $\sigma (p_i)$ from $p_j$.

% To sum up, providing these two properties at the network level, we can {\em always} and {\em correctly transform} (any) CFT distributed system to operate in the BFT model~\cite{clement2012, byzantine-pratical}. 

\if 0
\noindent{\bf{Security properties for BFT.}} We seek to offer BFT while reducing its programmability and performance overheads. As such, we materialize the {\em minimum} security properties required to build trustworthy systems under the BFT model~\cite{clement2012}: 
\begin{itemize}[leftmargin=*]
    \item {\bf Transferable authentication} refers to a machine's capability to verify the original sender of a received message, even if it is forwarded by other than the original sender. %Authentication is transferable if the original sender can also be verified for forwarded messages. 
    \item {\bf Non-equivocation} guarantees that a node cannot make conflicting statements to different nodes. Equivocation also manifests as network adversaries or replay attacks that send invalid messages or re-send valid but stale messages.
\end{itemize}
\fi

\subsection{High-Performance Distributed Systems} \label{subsec::tees}
%The security properties discussed above suffice for building distributed systems that operate {\em correctly} under the BFT model. 
The aforementioned two security properties are sufficient to {\em correctly transform} (any) CFT distributed system to operate in the BFT model~\cite{clement2012, byzantine-pratical}. 
However, a fundamental design trade-off exists between efficiency and robustness for practical deployments in the cloud. Our work aims to resolve this tension.

\myparagraph{Trusted hardware for BFT} System designers established trusted hardware, TEEs, as the most effective way to eliminate a system's Byzantine counterparts~\cite{avocado, minBFT, hybster, 10.1145/3492321.3519568}. While TEEs can be used to offer BFT, prior research illustrated significant performance and architectural limitations in the context of networked systems~\cite{avocado, 10.1145/3492321.3519568, hybster, minBFT}. Based on performance and security studies~\cite{9460547, 9935045}, TEEs' overheads in the heterogeneous cloud, in addition to their heterogeneity in programmability and security guarantees, are incapable of offering high-performant trusted networking under the BFT model. 


\myparagraph{SmartNICs for high-performance and BFT} We leverage the state-of-the-art hardware-level networking accelerators, i.e., SmartNICs~\cite{liquidIO_smartnics, u280_smartnics, bluefield_smartnics, broadcom_smartnics, netronome_smartnics, alibaba_smartnics, nitro_smartnics, msr_smartnics}, to address the trade-off between performance and security, overcoming the limitations of TEEs. Our design choice of leveraging SmartNICs is not hypothetical; SmartNIC devices have already been launched by major cloud providers~\cite{alibaba_smartnics, nitro_smartnics, msr_smartnics}, presenting great opportunities for performance thanks to their integrated fully programmable hardware (e.g., ARM cores~\cite{bluefield_smartnics, alibaba_smartnics, broadcom_smartnics, liquidIO_smartnics}, FPGAs~\cite{u280_smartnics, alveo_sn1000, msr_smartnics}). Precisely, we rely on two promising directions: {\em(1)} security and network processing offloading at the NIC-level hardware and {\em(2)} an efficient transformation for BFT. 

\if 0
We extend the scope of FPGA-based SmartNICs~\cite{u280_smartnics, alveo_sn1000} by offloading an RDMA protocol implementation to the FPGA and extending its security properties, offering non-equivocation and transferable authentication. 
% Our system not only leverages hardware acceleration for fast, trusted networking, 
Our system not only leverages hardware acceleration for performance, but {\em seamlessly} offers the foundations of a scalable transformation of distributed systems for BFT. These properties also guarantee that a CFT-to-BFT transformation for State-Machine-Replication (SMR) {\em always exists} with the same replication factor of the original CFT system~\cite{clement2012, byzantine-pratical} (2$f$+1), offering better scalability and less message complexity than the traditional BFT (3$f$+1). %In simple words, one can have BFT guarantees ensuring safety for up to $f$ Byzantine faults with $2f+1$ nodes as in the original CFT protocol; $f$ fewer nodes compared to the traditional (non-transformed) BFT protocols ($3f+1$ nodes).
\fi 

%In fact, the properties of the non-equivocation and transferable authentication suffice to transform even for BFT state machine replication (SMR) systems. They guarantee that a CFT-to-BFT transformation for SMR {\em exists} with the same replication factor of the original CFT system~\cite{clement2012, byzantine-pratical}, offering better scalability than the traditional. In simple words, one can have BFT guarantees ensuring safety for up to $f$ Byzantine faults with $2f+1$ nodes as in the original CFT protocol; $f$ fewer nodes compared to the traditional (non-transformed) BFT protocols ($3f+1$ nodes).

%\section{Introduction}
\label{sec:introduction}
The business processes of organizations are experiencing ever-increasing complexity due to the large amount of data, high number of users, and high-tech devices involved \cite{martin2021pmopportunitieschallenges, beerepoot2023biggestbpmproblems}. This complexity may cause business processes to deviate from normal control flow due to unforeseen and disruptive anomalies \cite{adams2023proceddsriftdetection}. These control-flow anomalies manifest as unknown, skipped, and wrongly-ordered activities in the traces of event logs monitored from the execution of business processes \cite{ko2023adsystematicreview}. For the sake of clarity, let us consider an illustrative example of such anomalies. Figure \ref{FP_ANOMALIES} shows a so-called event log footprint, which captures the control flow relations of four activities of a hypothetical event log. In particular, this footprint captures the control-flow relations between activities \texttt{a}, \texttt{b}, \texttt{c} and \texttt{d}. These are the causal ($\rightarrow$) relation, concurrent ($\parallel$) relation, and other ($\#$) relations such as exclusivity or non-local dependency \cite{aalst2022pmhandbook}. In addition, on the right are six traces, of which five exhibit skipped, wrongly-ordered and unknown control-flow anomalies. For example, $\langle$\texttt{a b d}$\rangle$ has a skipped activity, which is \texttt{c}. Because of this skipped activity, the control-flow relation \texttt{b}$\,\#\,$\texttt{d} is violated, since \texttt{d} directly follows \texttt{b} in the anomalous trace.
\begin{figure}[!t]
\centering
\includegraphics[width=0.9\columnwidth]{images/FP_ANOMALIES.png}
\caption{An example event log footprint with six traces, of which five exhibit control-flow anomalies.}
\label{FP_ANOMALIES}
\end{figure}

\subsection{Control-flow anomaly detection}
Control-flow anomaly detection techniques aim to characterize the normal control flow from event logs and verify whether these deviations occur in new event logs \cite{ko2023adsystematicreview}. To develop control-flow anomaly detection techniques, \revision{process mining} has seen widespread adoption owing to process discovery and \revision{conformance checking}. On the one hand, process discovery is a set of algorithms that encode control-flow relations as a set of model elements and constraints according to a given modeling formalism \cite{aalst2022pmhandbook}; hereafter, we refer to the Petri net, a widespread modeling formalism. On the other hand, \revision{conformance checking} is an explainable set of algorithms that allows linking any deviations with the reference Petri net and providing the fitness measure, namely a measure of how much the Petri net fits the new event log \cite{aalst2022pmhandbook}. Many control-flow anomaly detection techniques based on \revision{conformance checking} (hereafter, \revision{conformance checking}-based techniques) use the fitness measure to determine whether an event log is anomalous \cite{bezerra2009pmad, bezerra2013adlogspais, myers2018icsadpm, pecchia2020applicationfailuresanalysispm}. 

The scientific literature also includes many \revision{conformance checking}-independent techniques for control-flow anomaly detection that combine specific types of trace encodings with machine/deep learning \cite{ko2023adsystematicreview, tavares2023pmtraceencoding}. Whereas these techniques are very effective, their explainability is challenging due to both the type of trace encoding employed and the machine/deep learning model used \cite{rawal2022trustworthyaiadvances,li2023explainablead}. Hence, in the following, we focus on the shortcomings of \revision{conformance checking}-based techniques to investigate whether it is possible to support the development of competitive control-flow anomaly detection techniques while maintaining the explainable nature of \revision{conformance checking}.
\begin{figure}[!t]
\centering
\includegraphics[width=\columnwidth]{images/HIGH_LEVEL_VIEW.png}
\caption{A high-level view of the proposed framework for combining \revision{process mining}-based feature extraction with dimensionality reduction for control-flow anomaly detection.}
\label{HIGH_LEVEL_VIEW}
\end{figure}

\subsection{Shortcomings of \revision{conformance checking}-based techniques}
Unfortunately, the detection effectiveness of \revision{conformance checking}-based techniques is affected by noisy data and low-quality Petri nets, which may be due to human errors in the modeling process or representational bias of process discovery algorithms \cite{bezerra2013adlogspais, pecchia2020applicationfailuresanalysispm, aalst2016pm}. Specifically, on the one hand, noisy data may introduce infrequent and deceptive control-flow relations that may result in inconsistent fitness measures, whereas, on the other hand, checking event logs against a low-quality Petri net could lead to an unreliable distribution of fitness measures. Nonetheless, such Petri nets can still be used as references to obtain insightful information for \revision{process mining}-based feature extraction, supporting the development of competitive and explainable \revision{conformance checking}-based techniques for control-flow anomaly detection despite the problems above. For example, a few works outline that token-based \revision{conformance checking} can be used for \revision{process mining}-based feature extraction to build tabular data and develop effective \revision{conformance checking}-based techniques for control-flow anomaly detection \cite{singh2022lapmsh, debenedictis2023dtadiiot}. However, to the best of our knowledge, the scientific literature lacks a structured proposal for \revision{process mining}-based feature extraction using the state-of-the-art \revision{conformance checking} variant, namely alignment-based \revision{conformance checking}.

\subsection{Contributions}
We propose a novel \revision{process mining}-based feature extraction approach with alignment-based \revision{conformance checking}. This variant aligns the deviating control flow with a reference Petri net; the resulting alignment can be inspected to extract additional statistics such as the number of times a given activity caused mismatches \cite{aalst2022pmhandbook}. We integrate this approach into a flexible and explainable framework for developing techniques for control-flow anomaly detection. The framework combines \revision{process mining}-based feature extraction and dimensionality reduction to handle high-dimensional feature sets, achieve detection effectiveness, and support explainability. Notably, in addition to our proposed \revision{process mining}-based feature extraction approach, the framework allows employing other approaches, enabling a fair comparison of multiple \revision{conformance checking}-based and \revision{conformance checking}-independent techniques for control-flow anomaly detection. Figure \ref{HIGH_LEVEL_VIEW} shows a high-level view of the framework. Business processes are monitored, and event logs obtained from the database of information systems. Subsequently, \revision{process mining}-based feature extraction is applied to these event logs and tabular data input to dimensionality reduction to identify control-flow anomalies. We apply several \revision{conformance checking}-based and \revision{conformance checking}-independent framework techniques to publicly available datasets, simulated data of a case study from railways, and real-world data of a case study from healthcare. We show that the framework techniques implementing our approach outperform the baseline \revision{conformance checking}-based techniques while maintaining the explainable nature of \revision{conformance checking}.

In summary, the contributions of this paper are as follows.
\begin{itemize}
    \item{
        A novel \revision{process mining}-based feature extraction approach to support the development of competitive and explainable \revision{conformance checking}-based techniques for control-flow anomaly detection.
    }
    \item{
        A flexible and explainable framework for developing techniques for control-flow anomaly detection using \revision{process mining}-based feature extraction and dimensionality reduction.
    }
    \item{
        Application to synthetic and real-world datasets of several \revision{conformance checking}-based and \revision{conformance checking}-independent framework techniques, evaluating their detection effectiveness and explainability.
    }
\end{itemize}

The rest of the paper is organized as follows.
\begin{itemize}
    \item Section \ref{sec:related_work} reviews the existing techniques for control-flow anomaly detection, categorizing them into \revision{conformance checking}-based and \revision{conformance checking}-independent techniques.
    \item Section \ref{sec:abccfe} provides the preliminaries of \revision{process mining} to establish the notation used throughout the paper, and delves into the details of the proposed \revision{process mining}-based feature extraction approach with alignment-based \revision{conformance checking}.
    \item Section \ref{sec:framework} describes the framework for developing \revision{conformance checking}-based and \revision{conformance checking}-independent techniques for control-flow anomaly detection that combine \revision{process mining}-based feature extraction and dimensionality reduction.
    \item Section \ref{sec:evaluation} presents the experiments conducted with multiple framework and baseline techniques using data from publicly available datasets and case studies.
    \item Section \ref{sec:conclusions} draws the conclusions and presents future work.
\end{itemize}
\section{Overview}

\subsection{System Overview}
%We advocate that distributed systems must be fast and trustworthy  in the Byzantine heterogeneous cloud infrastructure. %We further show that while TEEs could help in this direction, they cannot meet the requirements of such systems in terms of variations in performance, programmability overheads and complicated security analysis.

We propose \projecttitle{}, a trusted NIC architecture for high-performance, trustworthy distributed systems, formally guaranteeing their secure and correct execution in the heterogeneous Byzantine cloud infrastructure. 
% To this end, we propose \projecttitle{}, a trusted NIC architecture that offers a network abstraction for high-performance, trustworthy distributed systems under BFT that meets the performance requirements of modern systems while it formally guarantees their secure and correct execution in the heterogeneous (Byzantine) cloud infrastructure. 
\projecttitle{} is comprised of three layers (shown in Figure~\ref{fig:overview}): {\bf (1)  the \projecttitle{} hardware architecture} (green box) that implements trusted network operations on top of SmartNIC devices ($\S$~\ref{sec:t-nic-hardware}), \rev{(a)}{{\bf (2) the \projecttitle{} network stack}} (yellow box) that intermediates between the application layer and the \projecttitle{} hardware ($\S$~\ref{sec:t-nic-network}), and \rev{(a)}{{\bf (3)  the \projecttitle{} network library}} (blue box) that exposes \projecttitle{}'s programming APIs ($\S$~\ref{sec:t-nic-software}). 

Our \projecttitle{} hardware architecture implements the networking IB/RDMA protocol~\cite{rdma_specification} on FPGA-based SmartNICs~\cite{u280_smartnics}. 
It extends the conventional protocol implementation with a minimal hardware module, the attestation kernel, that materializes the security properties of the non-equivocation and transferable authentication. The \projecttitle{} network stack configures the \projecttitle{} device on the control path while it offers the data path as kernel-bypass device access for low-latency operations. Lastly, the  \projecttitle{} network library exposes \rev{(a)}{programming APIs} built on top of (reliable) one-sided RDMA primitives. 
% Our \projecttitle{} hardware architecture implements and extends the networking IB/RDMA protocol~\cite{rdma_specification} on top of the FPGA-based SmartNICs, Alveo U280~\cite{u280_smartnics}. Critically, it extends the conventional protocol implementation with a minimal hardware security module, the attestation kernel, that materializes the security properties of the non-equivocation and the transferable authentication. The \projecttitle{} trusted network stack runs in user space. It configures the \projecttitle{} device (MAC address, IP, etc.) on the control path, while the data path is offered as kernel-bypass device access for low-latency operations. Lastly, The  \projecttitle{} trusted network library exposes a {\em trusted} API that is built on top of one-sided RDMA (reliable) operations. 


%\section{Design}
%\dimitra{
%\begin{itemize}
%    \item Intro
%    \item Background + Motivation
%    \item Overview (including sys/data model etc.)
%    \item Design + Implementation (TNIC: hw architecture, TNIC libraries: sw abstraction, Applications: use-cases)
%\end{itemize}
%}

%\subsection{System Model}

%\myparagraph{Model sketch}
%We model the distributed system as a set of {\tt N} nodes each of which is attached to a single \projecttitle{} instance that is loaded into an FPGA-based SmartNIC as Alveo U280~\cite{u280_smartnics}. The nodes communicate by exchanging messages through bi-directional network links that connect their FPGAs. The system is managed and owned by the third-party cloud infrastructure which is untrusted.


\subsection{Threat Model} 

%\antonis{It is a bit weird we have Fault model in section 2 and Thread model in section 3}

We inherit the fault and threat model from the classical BFT~\cite{Castro:2002} and trusted computing domains~\cite{intel-sgx}. The cloud infrastructure (machines, network, etc.) can exhibit Byzantine behavior and also being subject to attackers that can control over the host CPU (e.g., the OS, VMM, etc.) and the SmartNICs (post-manufacturing). The adversary can attempt to re-program the SmartNIC, but they cannot compromise the cryptographic primitives~\cite{levin2009trinc, minBFT, Castro:2002}. The physical package, supply chain, and manufacturer of the SmartNICs are trusted~\cite{10.1145/3503222.3507733, 10.1145/2168836.2168866}. The \projecttitle{} implementation (bitstream) is synthesized by a trusted IP vendor with a trusted tool flow for covert channels resilience. %(in a trusted environment) The system designers source the \projecttitle{} from trusted IP vendors.

\rev{(a), B3, B4, C2}{
Since \projecttitle{} does not rely on CPU-based TEEs and its network stack and library run on the unprotected CPU, both software can be compromised by a potentially Byzantine actor on the machine. As such, \projecttitle{} does not distinguish between different types of untrusted software components. Whether the network library, the network stack, or the application code is compromised, the node is considered faulty (Byzantine) and must conform to the BFT application system model, which should specify its tolerance to Byzantine failures.
}

%We do not consider denial-of-service (DoS) attacks; the cloud provider has physical control of the hardware and can simply unpower it. Nevertheless such attacks affect availability and not the correctness.

\if 0
\begin{figure}[t!]
    \centering
    \includegraphics[width=.45\textwidth]{figures/trusted-nic-attestation_kernel.drawio.pdf}
    \caption{\trustedfpga{} attestation kernel overview (transmission path).}
    \label{fig:attestation_kernel}
\end{figure}    
\fi

\begin{figure}[t!]
    \centering
    %\includegraphics[width=0.5\textwidth]{figures/trusted-nic-overview.drawio.pdf}
    % \includegraphics[width=0.45\textwidth]{figures/trusted-nic-system_overview.drawio.pdf}
    \includegraphics[width=0.7\linewidth]{figures/tnic_system_overview_atsushi.pdf}
    \caption{\projecttitle{} system overview.}
    \label{fig:overview}
\end{figure}

\subsection{Design Challenges and Key Ideas} \label{subsec:challenges} While designing \projecttitle{}, we overcome the following challenges:

\myparagraph{\#1: Heterogeneous hardware} 
CPU-based TEEs in the cloud infrastructure are heterogeneous with different programmability~\cite{Baumann2014, scone, 10.1145/3079856.3080208, 10.1145/3460120.3485341, tsai2017graphene, Rkt-io} and security properties~\cite{10.1145/3600160.3600169, 7807249, 10.1007/978-3-031-16092-9_7} that complicate their adoption and the system's correctness~\cite{10.1145/3460120.3485341}. 
% For example, Intel SGX~\cite{cryptoeprint:2016:086} offers code confidentiality and attestation while ARM TrustZone~\cite{arm-realm} does not. Likewise, AMD-sev~\cite{amd-sev} and Intel TDX~\cite{intelTDX} offer OS-based programming interfaces, whereas Keystone~\cite{riscv-multizone} and SGX require specific SDKs~\cite{KeystoneSDK}.
Prior systems~\cite{hybster, 10.1145/3492321.3519568, minBFT, DBLP:journals/corr/LiuLKA16a} {\em could} not address this heterogeneity challenge as they require {\em homogeneous} {\tt x86} machines with SGX extensions of a specific version. This is rather unrealistic in modern heterogeneous distributed systems where system designers are compelled to {\em stitch heterogeneous TEEs together}. TEE's heterogeneity in programmability and security semantics hampers their adoption and adds complexity to ensuring the system's overall correctness. 
% An alternative approach is to {\em stitch heterogeneous TEEs together}, however, it imposes two significant challenges. First, TEEs'  {programmability differences limit} their adoption as a widely accepted general approach for trustworthy systems. Second, TEE's security semantics heterogeneity complicates the correctness of the derived systems; designers must carefully consider their security guarantees.
% , placing a heavy burden on system designers.

% The host CPU-side TEEs in the cloud infrastructure are heterogeneous with different programmability~\cite{Baumann2014, scone, 10.1145/3079856.3080208, 10.1145/3460120.3485341, tsai2017graphene, Rkt-io} and security properties~\cite{10.1145/3600160.3600169, 7807249, 10.1007/978-3-031-16092-9_7} that affect their adoption while also raising concerns about the system's correctness~\cite{10.1145/3460120.3485341}. As an example of this heterogeneity, Intel SGX~\cite{cryptoeprint:2016:086} offers code confidentiality and attestation, but ARM TrustZone~\cite{arm-realm} does not. Likewise, AMD-sev~\cite{amd-sev} and Intel TDX~\cite{intelTDX} offer OS-based programming interfaces, whereas Keystone~\cite{riscv-multizone} and SGX require specific SDKs~\cite{KeystoneSDK}. Prior work {\em could} not address this heterogeneity challenge; in fact, we found that almost all prior systems' implementations~\cite{hybster, 10.1145/3492321.3519568, minBFT, DBLP:journals/corr/LiuLKA16a} require {\em homogeneous} {\tt x86} machines with SGX extensions of a specific version.

% The requirement of homogeneous TEEs is rather unrealistic in modern cloud environments, resulting in system designers being compelled to {\em stitch heterogeneous TEEs together} to build trustworthy distributed systems. Unfortunately, this tactic introduces significant overheads in two directions. First, due to significant engineering efforts, TEEs'  {programmability differences limit} their adoption as a widely accepted general approach for trustworthy systems. Secondly, TEE's fundamental heterogeneity in {security semantics} complicates the correctness of the derived system; system designers must carefully consider the derived system's security guarantees.



\myparagraph{Key idea: A host CPU-agnostic unified security architecture based on trustworthy network-level isolation} 
Our \projecttitle{} offers a unified and host-agnostic network-interface level isolation that guarantees the specific yet well-defined security properties of the non-equivocation and transferable authentication. 
\rev{B1}{\projecttitle{} shifts the security properties from CPU-hosted TEEs to NIC hardware, thereby addressing the heterogeneity and programmability issues associated with CPU-based TEEs.}
% \projecttitle{} is built upon SmartNIC hardware, which is highly favorable in the heterogeneous Byzantine cloud infrastructure. 
%\projecttitle{} is built upon SmartNIC hardware, which is highly favorable in the heterogeneous Byzantine cloud infrastructure. 
\projecttitle{} also offers generic programming APIs ($\S$~\ref{sec:net-lib}) that are used to {\em correctly} transform a wide variety of distributed systems for Byzantine settings. 
We demonstrate the power of \projecttitle{} with a generic transformation {\em recipe} ($\S$~\ref{subsec:transformation}) and its usage to transform prominent distributed systems ($\S$~\ref{sec:use_cases}).
% We build our \projecttitle{} system on SmartNIC hardware to achieve those goals, offering network-level isolation for our offered security properties. 
% Importantly, our host-agnostic \projecttitle{} network interface is highly favorable in the heterogeneous Byzantine cloud infrastructure. In contrast, the \projecttitle{}'s security properties have been proven to be sufficient for {\em correctly} transforming a wide variety of distributed systems for Byzantine settings. 

% Our \projecttitle{} offers a unified and host-agnostic network-interface level isolation that guarantees the specific and well-defined security properties of the non-equivocation and the transferable authentication. At the same time, we resolve the programmability burden through generic programming APIs. We build our \projecttitle{} system on SmartNIC hardware to achieve those goals, offering network-level isolation for our offered security properties. Importantly, our host-agnostic \projecttitle{} network interface is highly favorable in the heterogeneous Byzantine cloud infrastructure. In contrast, \projecttitle{}'s security properties of the non-equivocation and the transferable authentication have been proven to be sufficient for {\em correctly} transforming a wide variety of distributed systems for Byzantine settings. In fact, we show the power of \projecttitle{} with a generic transformation {\em recipe} ($\S$~\ref{subsec:transformation}) as well as its application to transform four widely adopted distributed systems ($\S$~\ref{sec:use_cases}).


\myparagraph{\#2: Large TCB in the TEE-based silicon root-of-trust} 
TEEs based on a {\em silicon root of trust} are promising for building trustworthy systems~\cite{avocado, minBFT, hybster, 10.1145/3492321.3519568}. Unfortunately, the state-of-the-art TEEs integrate a {\em large} TCB; for example, we calculate the TCB size of the state-of-the-art Intel TDX~\cite{intelTDX}. The TEE ports within the trusted hardware the entire Linux kernel (specifically, v5.19~\cite{linuxlifecircle}) and ``hardens'' at least 2000K lines of usable code, leading to a final TCB of 19MB. Such large TCBs have been accused of increasing the area of faults and attacks~\cite{10.1145/3379469, 10.5555/1756748.1756832} of commercial TEEs that are already under fire for their {security vulnerabilities}~\cite{intel_sgx_vulnerabilities1, intel_sgx_vulnerabilities2, intel_sgx_vulnerabilities3, intel_sgx_vulnerabilities4, intel_sgx_vulnerabilities5}. Importantly, TEE's large TCBs complicate their {security analysis and verification}, rendering their security properties {\em incoherent}. 
% Prior works~\cite{avocado, minBFT, hybster, 10.1145/3492321.3519568} have established the {\em silicon root of trust}, e.g., TEEs, to be a promising direction to build trustworthy systems. Unfortunately, even the state-of-the-art TEEs integrate a {\em large} TCB. For example, we calculated the TCB size of the state-of-the-art Intel TDX~\cite{intelTDX}. The TEE ports within the trusted hardware the entire Linux kernel (specifically, v5.19~\cite{linuxlifecircle}) and ``hardens'' at least 2K lines of usable code, leading to a final TCB of 19MB. Such large TCBs have been accused of increasing the area of faults and attacks~\cite{10.1145/3379469, 10.5555/1756748.1756832} of commercial TEEs that are already under fire for their {major security vulnerabilities}~\cite{intel_sgx_vulnerabilities1, intel_sgx_vulnerabilities2, intel_sgx_vulnerabilities3, intel_sgx_vulnerabilities4, intel_sgx_vulnerabilities5}. As such, modern TEEs suffer from security flaws, whereas their large TCBs complicate their {security analysis and formal verification}, rendering the derived ``trustworthy'' distributed system {\em incoherent}. 


\myparagraph{Key idea: A minimal and formally verifiable silicon root-of-trust with low TCB} 
In our work, we advocate that a {\em minimalistic silicon root of trust} (TCB) at the NIC level hardware is the foundation for building verifiable, trustworthy distributed systems. 
In fact, \projecttitle{} builds a minimalistic and verifiable attestation kernel ($\S$~\ref{subsec:nic_attest_kernel}) that guarantees the \projecttitle{} security properties at the SmartNIC hardware. 
% that guarantees the non-equivocation and transferable authentication properties for network messages. 
% In fact, \projecttitle{} builds a minimalistic and verifiable attestation kernel at the SmartNIC hardware that guarantees the non-equivocation and transferable authentication properties for network messages. 
Moreover, we have formally verified the \projecttitle{} secure hardware protocols ($\S$~\ref{subsec::formal_verification_remote_attestation}).
% since we rely on a minimalistic interface for trusted computing, 


% In our work, we advocate that a {\em minimalistic silicon root of trust} (TCB) at the NIC level hardware is the foundation for building verifiable, trustworthy distributed systems. In fact, \projecttitle{} builds a minimalistic and verifiable attestation kernel (TCB) at the SmartNIC hardware that guarantees the non-equivocation and the transferable authentication properties for the network messages. Moreover, since we rely on a minimalistic interface for trusted computing, we have formally verified the \projecttitle{} secure hardware protocols in the Tamarin theorem prover ($\S$~\ref{subsec::formal_verification_remote_attestation}).


\myparagraph{\#3: Performance} 
TEE's overheads are significant in the context of networked systems~\cite{avocado, treaty, minBFT,10.1145/3492321.3519568}. Prior research~\cite{avocado} reported 4$\times$---8$\times$ performance degradation with even a sophisticated network stack. Others~\cite{10.1145/3492321.3519568, hybster, minBFT} limit performance due to the communication costs between their untrusted and TEE-based counterparts~\cite{10.1145/2168836.2168866}. The actual performance overheads in heterogeneous distributed systems are expected to be more exacerbated~\cite{9460547, 9935045}. As such, TEEs cannot {\em naturally} offer high-performant, trusted networking. 
% Distributed systems in the third-party cloud infrastructure must be fast and trustworthy. Their overheads are significantly exacerbated in the context of networked systems~\cite{avocado, treaty, minBFT,10.1145/3492321.3519568}--- the foundational building block in the core of any distributed system. Prior research~\cite{avocado} have reported an average of 4$\times$---8$\times$ performance degradation for networking between two host-sided TEEs instances (Intel SGX) even with using a sophisticated network stack implementation that had carefully been optimized for this specific TEE version. Other systems on top of TEEs~\cite{10.1145/3492321.3519568, hybster, minBFT} also can limit performance due to the communication costs between the untrusted and TEE-based counterparts of the system~\cite{10.1145/2168836.2168866}. Based on performance analysis of heterogeneous TEEs~\cite{9460547, 9935045}, the actual performance overheads of a distributed system in the heterogeneous cloud can be even more exacerbated. As such host-sided trusted hardware cannot {\em naturally} offer high-performant trusted networking. 

\myparagraph{Key idea: Hardware-accelerated trustworthy network stack} 
Our \projecttitle{} bridges the gap between performance and security with two design insights. 
First, \projecttitle{} attestation kernel offers the foundations to transform CFT distributed systems to BFT systems without affecting the number of participating nodes, significantly improving scalability. 
% which significantly improves scalability. 
% First, \projecttitle{} minimalistic TCB, the attestation kernel, offers the foundations to transform CFT distributed systems to BFT systems without affecting the number of participating nodes, which significantly improves scalability. 
Second, \projecttitle{} user-space network stack ($\S$~\ref{sec:t-nic-network}) bypasses the OS and offloads security and network processing to the NIC-level hardware. 
% This is especially important in the context of replication protocol as the transformation significantly improves scalability; we can have BFT with the same number of participating nodes as in the CFT system. 
% Second, we offer a user-space network stack that bypasses the OS while we build our attestation kernel at the NIC-level hardware---specifically, on SmartNICs---offloading at the hardware the security processing of the non-equivocation and transferable authentication. 
% Our \projecttitle{} bridges the gap between performance and security with two design insights. First \projecttitle{} minimalistic TCB, the attestation kernel, offers the foundations to transform CFT distributed systems to BFT distributed systems without affecting the number of participating machines. This is especially important in the context of replication protocol as the transformation significantly improves scalability; we can have BFT with the same number of participating nodes as in the CFT system. Secondly, we offer a user-space network stack that bypasses the OS while we build our attestation kernel at the NIC-level hardware---specifically, on SmartNICs---offloading at the hardware the security processing of the non-equivocation and transferable authentication. 







\if 0
\myparagraph{Challenge \#1: Heterogeneity and Programmability} Prior trustworthy systems were built on top of CPU-specific TEEs. Importantly, we found that almost all of the open-source systems' implementations~\cite{hybster, 10.1145/3492321.3519568, minBFT, DBLP:journals/corr/LiuLKA16a} require {\em homogeneous} {\tt x86} machines with SGX extensions of a specific version.

The requirement for homogeneous TEEs is not realistic in modern cloud environments as the cloud data centers are comprised by heterogeneous machines and continuously update their hardware infrastructure~\cite{}. Following this, system developers are compelled to {\em stitch together} heterogeneous TEEs to build trustworthy systems. Unfortunately, this task is challenging in two aspects. First, it introduces a significant programmability overhead that limits TEEs' adoption as a general approach for trustworthy systems. Secondly, it raises concerns about the safety and the correctness of the designed system; heterogeneous TEEs come with different security properties which complicate the design and safety analysis of the derived system.


\myparagraph{Solution} To attack this challenge, \projecttitle{}'s key idea is unification. We offer a trusted unified programming interface that exposes the generic, yet powerful, security properties of non-equivocation and transferable authentication. The properties, implemented at NIC-hardware level within the \projecttitle{}'s attestation kernel, suffice to transform distributed systems for Byzantine settings ($\S$~\ref{sec:background}). Our \projecttitle{}' approach is highly favorable in the heterogeneous Byzantine cloud infrastructure; we offer BFT guarantees without relying on the host-sided TEEs. In fact, we show the power of \projecttitle{} with a generic transformation {\em recipe} ($\S$~\ref{subsec:transformation}) as well as its application to transform four widely-adopted systems ($\S$~\ref{sec:use_cases}).
%onsequently, . In addition, our minimalistic hardware-assisted TCB allows us to fully verify the safety and security properties of our \projecttitle{} from its initialization and remote attestation process to its normal operation ($\S$~\ref{subsec::formal_verification_remote_attestation}).

\myparagraph{Challenge \#2: Minimalistic TCB} The state-of-the-art TEEs~\cite{amd-sev, intelTDX} integrate a huge TCB that vastly increases the area of faults and attacks. In fact, state-of-the-art TEEs are still under fire for major security vulnerabilities~\cite{intel_sgx_vulnerabilities1, intel_sgx_vulnerabilities2, intel_sgx_vulnerabilities3, intel_sgx_vulnerabilities4, intel_sgx_vulnerabilities5}. As an example of this, we calculated the TCB size of the state-of-the-art Intel TDX~\cite{intelTDX}. The TEE ports within the trusted hardware the entire Linux kernel (specifically, v5.19~\cite{linuxlifecircle}) and ``hardens'' at least 2K lines of usable code, leading to a final TCB of 19MB. Safety in BFT systems has always been a challenge---e.g., Zyzzyva~\cite{unsafe_Zyzzyva} has proven to be unsafe almost ten years after its original publication---let alone when the {\em supposed} BFT system relies on ``unstable'' TEEs. 

\myparagraph{Solution} In our work, we advocate that a {\em minimal TCB}, that materializes the {\em foundational security primitives}, is the key to building verifiable trustworthy distributed systems. In fact, our minimalistic attestation kernel implements the properties of non-equivocation and transferable authentication that have already been proven to be the lower bound for transforming distributed systems to BFT ones~\cite{clement2012}. As such, our \projecttitle{} offers strong foundations to system designers without relying on the TEEs' extended TCB's with lots of unnecessary features.

%To attack this challenge, \projecttitle{}'s key idea is minimalism. We design a minimalistic fully verifiable (TCB) attestation kernel at NIC-hardware level that exposes a unified API to system designers. As such, we offer BFT while removing the distributed system's dependencies on the host CPU, rendering our approach highly favorable in the heterogeneous cloud infrastructure. In addition, our minimalistic hardware-assisted TCB allows us to fully verify the safety and security properties of our \projecttitle{} from its initialization and remote attestation process to its normal operation ($\S$~\ref{subsec::formal_verification_remote_attestation}).

\myparagraph{Challenge \#3: Performance and security} Distributed systems in the third-party cloud infrastructure need to be fast and trustworthy. With the performance being critical in all of them, offering security with TEEs is a poor design choice due to their performance limitations. In addition to the examples in $\S$~\ref{subsec::tees}, we also acknowledge from experience the difficulty to optimise TEE-based programs even with using the state-of-the-art programming frameworks~\cite{scone}. For example, while porting a simple distributed client/server application into \textsc{scone}---an optimized framework that shields (unmodified) applications with Intel SGX---we found that increasing the (protected) swap memory (paging area), even unused, led to a performance degradation by a factor of two for the exact same unmodified application. Consequently, TEEs cannot {\em intuitively} offer performance and security.

%, the problem is even more exacerbated in fault-tolerant systems. Building replication protocols for distributed systems under the Byzantine fault model has always been a complex endeavor with performance, scalability, and engineering challenges~\cite{bftForSkeptics}. Even well-studied protocols, e.g., PBFT~\cite{Castro:2002}, are limited in scalability (it requires at least $f$ nodes w.r.t to its CFT counterpart), incur high latency (it runs three all-to-all broadcasting phases with $O(n^2)$ message complexity) and are hard to verify and optimize~\cite{10.1145/2658994}.

%\myparagraph{Challenge \#1: Performance vs. security} Distributed systems in the third-party cloud infrastructure need to be fast, scalable, and trustworthy. While performance is critical in all of them, with TEEs themselves introducing significant performance limitations, the problem is even more exacerbated in fault-tolerant systems. Building replication protocols for distributed systems under the Byzantine fault model has always been a complex endeavor with performance, scalability, and engineering challenges~\cite{bftForSkeptics}. Even well-studied protocols, e.g., PBFT~\cite{Castro:2002}, are limited in scalability (it requires at least $f$ nodes w.r.t to its CFT counterpart), incur high latency (it runs three all-to-all broadcasting phases with $O(n^2)$ message complexity) and are hard to verify and optimize~\cite{10.1145/2658994}.



\myparagraph{Solution} Our \projecttitle{} bridges this gap between performance and security. We implement a minimal TCB, the attestation kernel ($\S$~\ref{subsec:nic_attest_kernel}), that materializes all the necessary properties for transforming systems for Byzantine settings. We build the attestation kernel as part of the state-of-the-art NIC hardware for high performance ($\S$~\ref{subsec:roce_protocol_kernel}). Our evaluation shows that \projecttitle{}, applied in four widely-adopted systems, outperforms the TEE-based versions of those systems ($\S$~\ref{sec:eval}).

%protocol, to help system designers building more robust protocols. To achieve this, we build an extended implementation of the classical, widely-adopted, RDMA network stack~\cite{rdma} on programmable hardware, i.e., FPGA-based SmartNICs~\cite{u280_smartnics}, offloading on this hardware the necessary required security processing and mechanisms. 

%\myparagraph{Challenge \#1: Security vs high-performance} The distributed applications hosted in the third-party cloud infrastructure need to be highly available. As such, cloud services and applications build on top of replication protocols that offer fault tolerance and can remain available when failures occur~\cite{Jimenez-Peris2001}. The vast majority of such deployed protocols in the cloud operate under the Crash Fault Tolerant model (CFT) where the machines can {\em only} fail by crashing or omitting some steps. However, the CFT model is inadequate in modern cloud infrastructure as it has been observed that the machines or other parts of the infrastructure can fail arbitrarily (i.e., exhibit Byzantine behavior~\cite{Lamport:1982}) due to malicious adversaries, compromised OS/hypervisor in machines, faulty network links and compromised memory and storage medium~\cite{Gunawi_bugs-in-the-cloud, ciad, fast-08-corruption, security-one-sided-communication, accountable-cloud}. Consequently, the current replication protocols target a quite limited fault model which does not match modern's applications security needs that are hosted in the (untrusted) cloud.

%\myparagraph{Solution} Our \projecttitle{} overcomes this limitation by offering  trusted and easily adoptable network operations, which are at the core of any distributed protocol, to help system designers building more robust protocols. To achieve this, we build an extended implementation of the classical, widely-adopted, RDMA network stack~\cite{rdma} on programmable hardware, i.e., FPGA-based SmartNICs~\cite{u280_smartnics}, offloading on this hardware the necessary required security processing and mechanisms. 

%More importantly, conventional BFT protocols present the following characteristics. First, they require an extra set of $f$ participant machines to tolerate up to $f$ failures. Compared to CFT protocols that operate with $2f+1$ participants, classical BFT protocols present limited scalability as they require at least $3f+1$ participants~\cite{BFT_THEORY}. In addition to this, BFT protocols can be slow as they usually run at least three phases of broadcasts~\cite{Castro:2002, DBLP:journals/corr/abs-1803-05069} and incur high message complexity (e.g., $O(n^2)$). Lastly, BFT protocols are complex: they are hard to understand, let alone be optimised~\cite{10.1145/2658994}. Even intuitive algorithmic improvements to optimize for the common case or recovery can significantly affect other parts of the protocol (e.g., view-change in~\cite{10.1145/1658357.1658358}, normal case adds 2 extra phases in~\cite{DBLP:journals/corr/abs-1803-05069}) .% Consequently, they have seen little adoption in commercial cloud applications due to their limited scalability and performance.




  


\myparagraph{Challenge \#4: Hardware verifiability} Some (but not all) TEEs offer (remote) attestation so that the application owner can verify the integrity of the TEE and its executing code. To attest the system, the application code needs to be known to the TEEs' hardware provider (e.g., Intel Attestation Service~\cite{ias}) so that its functionality can be evaluated, {\em measured} before the trust is established. However, there are real-world application scenarios (e.g., proprietary algorithm~\cite{}) where the program itself need to comply with strict privacy policies and cannot be {\em openly} exposed to multiple parties. 

\myparagraph{Solution} Our \projecttitle{} overcomes the limitation. While its offered security properties are decoupled from the CPU running code, we further formally verify the safety and security guarantees of our system.

%\myparagraph{Challenge \#2: Performance and scalability} Researchers~\cite{Castro:2002, DBLP:journals/corr/abs-1803-05069, 10.1145/1658357.1658358} presented a range of robust replication protocols that remain correct when arbitrary failures occur targeting the BFT model~\cite{Lamport:1982}. Unfortunately, these BFT protocols have recognised little adoption because they cannot meet the performance requirements of deployed applications~\cite{bft-time-is-now}. In addition, the vast majority~\cite{Castro:2002, DBLP:journals/corr/abs-1803-05069} introduces resources overheads and limits scalability because it requires at least $3f+1$ machines to tolerate up to $f$ faults. That is, at least more $f$ machines compared to currently deployed CFT protocols. Consequently, BFT protocols are not well suited for performance in modern high-end distributed systems~\cite{bftForSkeptics}.

%\myparagraph{Solution} We bridge the gap between performance, scalability and robustness. Our \projecttitle{} offers robustness by materialising the necessary foundations for building BFT protocols~\cite{clement2012} in programmable, yet fast, hardware, while it also improves performance and scalability by limiting the number of required participant machines to the minimum, i.e., $2f+1$. More specifically, \projecttitle{} implements the theoretical foundations of Clement et. al~\cite{clement2012} to translate a CFT protocol to a BFT protocol without having to increase the CFT protocol's replication degree. We explain this mechanism in $\S$~\ref{sec:background}.  %Their work has shown that a translation between any CFT protocol to a BFT protocol {\em always} exists if the security properties of the transferable authentication and the non-equivocation are guaranteed. We discuss the properties and the translation mechanism in$\S$~\ref{}. 

%\begin{itemize}
 %   \item {\bf{Transferable authentication.}} Potentially malicious nodes cannot impersonate other (honest) nodes. Essentially, any node can verify that a message is signed by the correct sender, even for forwarded messages.
 %   \item {\bf{Non-equivocation.}} A sender cannot send different messages to different nodes in the same round while it is supposed to send the same message according to the protocol.
%\end{itemize}

%we designed and implemented \projecttitle{} to offer the two properties of non-equivocation and transferable authentication that allow us to design and build BFT protocols with the minimum possible participant nodes ($2f+1$), resolving the trade-off of scalability, performance, and BFT at once. 

%More precisely, our design relies on the theoretical findings 

%Our \projecttitle{} materializes these properties on the NIC-level by implementing and integrating an attestation kernel for generating message authentication certificates or attestations and verifying those when messages are received ($\S$~\ref{subsec:tfpga}). That way \projecttitle{} builds and exposes the minimal abstraction required for implementing robust protocols under the Byzantine Fault model with $2f+1$ participant nodes.


%\myparagraph{Challenge \#3: Adaptability} Due to the traditional BFT protocols limitations, a new line of research has attempted to optimize them~\cite{10.1145/3492321.3519568, minBFT, hybster, 10.1145/2168836.2168866, DBLP:journals/corr/LiuLKA16a, trinc} making use of trusted hardware, precisely, Trusted Execution Environments (TEEs)~\cite{cryptoeprint:2016:086, arm-realm, amd-sev, riscv-multizone, intelTDX}. Unfortunately, the safety requirements of these optimized protocols highly depend on very specific and CPU-dependant TEEs. Consequently, in addition to their limit adaptability and generality, these protocols' correct implementation and deployment requires that there will {\em always} be available the required number of machines equipped with specific CPU generation and TEE hardware versions. In any other case, system designers are compelled to be able to quickly learn and program any another available TEE. This complicates the widespread adoption of such protocols because the task of programming heterogeneous TEEs as rather challenging~\cite{10.1145/3460120.3485341} as error prone; various TEEs present different programming models and security properties~\cite{10.1007/978-3-031-16092-9_7}. %To sum up, this heterogeneity complicates the widespread adoption of such protocols.
%it raises questions regarding performance and correctness. 

%---in fact, we found that all such open-sourced protocols~\cite{hybster, 10.1145/3492321.3519568, minBFT, DBLP:journals/corr/LiuLKA16a} are built on top Intel SGX~\cite{intel-sgx}.

%We argue that the protocols' reliance on such specific TEEs limits generality and adoption in modern heterogeneous cloud infrastructure. For example, to deploy these protocols, a cloud provider must guarantee that there would always be available a required number of machines equipped with specific CPU generation and TEE hardware versions. In cases where there are no machines available, or the TEE has been discontinued (e.g., this is the case with Intel SGX~\cite{sgx_deprecated}), protocol designers are compelled to be able to quickly learn and program another available TEE. However, programming heterogeneous TEEs is a task rather challenging~\cite{10.1145/3460120.3485341}, and it raises questions regarding performance and correctness. Heterogeneous TEEs not only have different programming and performance characteristics but the security properties they offer can greatly vary too~\cite{10.1007/978-3-031-16092-9_7}. As an example of this, Keystone (RISC-V)~\cite{keystone_eurosys} and Intel SGX (x86)~\cite{intel-sgx} have quite different programming APIs. In addition, Intel SGX can only support a very limited Trusted Computing Base (TCB), only up to 256MB, compared to Intel TDX~\cite{intelTDX} and AMD-SEV~\cite{amd-sev}. 
%\antonis{may mention arm as well?}
%Importantly, commercially available TEEs~\cite{arm-trustzone, intel-sgx, amd-sev, keystone_eurosys, 197162, timber} offer different levels of security (i.e., integrity, freshness, and confidentiality), whereas not all of them come with built-in support for secure bootstrapping and remote attestation~\cite{10.1007/978-3-031-16092-9_7, 7807249, secTEE}. All these characteristics make the widespread adoption of existing BFT protocols impractical.


%\myparagraph{Solution} We attack this challenge by removing any dependencies on CPU-based TEEs and unshackling the designers from having to continuously learn and program various TEEs. Our \projecttitle{} makes use of programmable hardware, i.e., FPGAs, to implement a trusted network stack offloading any security-related processing in the NIC hardware ($\S$~\ref{subsec:tfpga}) and to offer a unified abstraction (network library) to the system designers ($\S$~\ref{sec:net-lib}). While our \projecttitle{} shifts the homogeneity from the CPU layer to the FPGA-based NIC layer, our architectural design is not hypothetical; \projecttitle{} fits well in recent deployments in commercial clouds, e.g. Microsoft's Catapult design~\cite{msr_smartnics, 211249}. 

%machines. To achieve this, we implement our entire network stack, including the trusted subsystem for non-equivocation and authentication, on the NIC's hardware-level, leveraging the SmartNIC technology~\cite{}. Essentially, our \projecttitle{} shifts the homogeneity from the CPU layer to the NIC layer:  while the host CPUs participating in a protocol can be heterogeneous, our \projecttitle{} is built on top of homogeneous SmartNICs \antonis{seems stricter than what we assume? maybe FPGA-enabled networking?}, exposing a unified abstraction for exchanging and verifying network messages.

%We implemented our network stack on top of FPGA-based SmartNICs, specifically Alveo U280~\cite{alveo_smartnics}. While our design could be adapted to be applicable to  SoC-based SmartNICs (e.g., Mellanox BlueField~\cite{bluefield_smartnics}, Alveo SN1000~\cite{alveo_sn1000}), we decided against it due to their performance limitations in 100G era~\cite{211249}.\antonis{weak argument, Bluefield 3 is up to 400Gbits. Let's say: "Our design is applicable to ... and our evaluation shows that hardware implementation is more beneficial.} Instead, given the increasingly wide deployment of such specialized hardware in DCs as an efficient way to offload network processing our architectural design is not hypothetical. \projecttitle{} fits well in recent deployments in commercial clouds. An example of this is Microsoft’s Catapult, where the FPGA, which sits on the data path in front of the network card and applies ``smart'' processing, could be extended (as in $\S$~\ref{subsec:tfpga}) to improve security.

%Our system achieves these goals by implementing a minimal hardware-based authentication subsystem, the \emph{attestation kernel}, that guarantees the necessary security properties required for BFT. The attestation kernel generates and verifies authentication certificates for the network messages to ensure two core security properties: the non-equivocation and the (transferable) authentication properties. These two properties have been proven to be necessary and sufficient for decreasing the replication factor for BFT protocols~\cite{clement2012}. Further, we carefully implemented \projecttitle{} to optimize for performance and scalability by integrating the attestation kernel ``on path'' at the NIC-level. That way, our system offers security without introducing unnecessary overheads both in performance and adoption: data is processed \emph{on their way} to the network whereas the protocols do not have to rely on a specific TEE in the host CPU layer. 
\fi 

\if 0

\subsection{Design Challenges}
\dimitra{here}
\myparagraph{Challenge \#1: Security} The distributed applications hosted in the third-party cloud infrastructure need to be highly available. As such, cloud services and applications build on top of replication protocols that offer fault tolerance and, importantly, can remain available when failures occur~\cite{Jimenez-Peris2001}. The vast majority of such deployed protocols in the cloud operate under the Crash Fault Tolerant model (CFT) where the machines can {\em only} fail by crashing or omitting some steps. However, the CFT model is inadequate in modern cloud infrastructure as it has been observed that the machines or other parts of the infrastructure can fail arbitrarily (i.e., exhibit Byzantine behavior~\cite{Lamport:1982}) due to malicious adversaries, compromised OS/hypervisor in machines, faulty network links and compromised memory and storage medium~\cite{Gunawi_bugs-in-the-cloud, ciad, fast-08-corruption, security-one-sided-communication, accountable-cloud}. Consequently, the current replication protocols target a quite limited fault model which does not match modern's applications security needs that are hosted in the (untrusted) cloud.

\myparagraph{Solution} Our \projecttitle{} overcomes this limitation by offering  trusted and easily adoptable network operations, which are at the core of any distributed protocol, to help system designers building more robust protocols. To achieve this, we build an extended implementation of the classical, widely-adopted, RDMA network stack~\cite{rdma} on programmable hardware, i.e., FPGA-based SmartNICs~\cite{u280_smartnics}, offloading on this hardware the necessary required security processing and mechanisms. 

%More importantly, conventional BFT protocols present the following characteristics. First, they require an extra set of $f$ participant machines to tolerate up to $f$ failures. Compared to CFT protocols that operate with $2f+1$ participants, classical BFT protocols present limited scalability as they require at least $3f+1$ participants~\cite{BFT_THEORY}. In addition to this, BFT protocols can be slow as they usually run at least three phases of broadcasts~\cite{Castro:2002, DBLP:journals/corr/abs-1803-05069} and incur high message complexity (e.g., $O(n^2)$). Lastly, BFT protocols are complex: they are hard to understand, let alone be optimised~\cite{10.1145/2658994}. Even intuitive algorithmic improvements to optimize for the common case or recovery can significantly affect other parts of the protocol (e.g., view-change in~\cite{10.1145/1658357.1658358}, normal case adds 2 extra phases in~\cite{DBLP:journals/corr/abs-1803-05069}) .% Consequently, they have seen little adoption in commercial cloud applications due to their limited scalability and performance.


\myparagraph{Challenge \#2: Performance and scalability} Researchers~\cite{Castro:2002, DBLP:journals/corr/abs-1803-05069, 10.1145/1658357.1658358} presented a range of robust replication protocols that remain correct when arbitrary failures occur targeting the BFT model~\cite{Lamport:1982}. Unfortunately, these BFT protocols have recognised little adoption because they cannot meet the performance requirements of deployed applications~\cite{bft-time-is-now}. In addition, the vast majority~\cite{Castro:2002, DBLP:journals/corr/abs-1803-05069} introduces resources overheads and limits scalability because it requires at least $3f+1$ machines to tolerate up to $f$ faults. That is, at least more $f$ machines compared to currently deployed CFT protocols. Consequently, BFT protocols are not well suited for performance in modern high-end distributed systems~\cite{bftForSkeptics}.

\myparagraph{Solution} We bridge the gap between performance, scalability and robustness. Our \projecttitle{} offers robustness by materialising the necessary foundations for building BFT protocols~\cite{clement2012} in programmable, yet fast, hardware, while it also improves performance and scalability by limiting the number of required participant machines to the minimum, i.e., $2f+1$. More specifically, \projecttitle{} implements the theoretical foundations of Clement et. al~\cite{clement2012} to translate a CFT protocol to a BFT protocol without having to increase the CFT protocol's replication degree. We explain this mechanism in $\S$~\ref{sec:background}.  %Their work has shown that a translation between any CFT protocol to a BFT protocol {\em always} exists if the security properties of the transferable authentication and the non-equivocation are guaranteed. We discuss the properties and the translation mechanism in$\S$~\ref{}. 

%\begin{itemize}
 %   \item {\bf{Transferable authentication.}} Potentially malicious nodes cannot impersonate other (honest) nodes. Essentially, any node can verify that a message is signed by the correct sender, even for forwarded messages.
 %   \item {\bf{Non-equivocation.}} A sender cannot send different messages to different nodes in the same round while it is supposed to send the same message according to the protocol.
%\end{itemize}

%we designed and implemented \projecttitle{} to offer the two properties of non-equivocation and transferable authentication that allow us to design and build BFT protocols with the minimum possible participant nodes ($2f+1$), resolving the trade-off of scalability, performance, and BFT at once. 

%More precisely, our design relies on the theoretical findings 

%Our \projecttitle{} materializes these properties on the NIC-level by implementing and integrating an attestation kernel for generating message authentication certificates or attestations and verifying those when messages are received ($\S$~\ref{subsec:tfpga}). That way \projecttitle{} builds and exposes the minimal abstraction required for implementing robust protocols under the Byzantine Fault model with $2f+1$ participant nodes.


\myparagraph{Challenge \#3: Adaptability} Due to the traditional BFT protocols limitations, a new line of research has attempted to optimize them~\cite{10.1145/3492321.3519568, minBFT, hybster, 10.1145/2168836.2168866, DBLP:journals/corr/LiuLKA16a, trinc} making use of trusted hardware, precisely, Trusted Execution Environments (TEEs)~\cite{cryptoeprint:2016:086, arm-realm, amd-sev, riscv-multizone, intelTDX}. Unfortunately, the safety requirements of these optimized protocols highly depend on very specific and CPU-dependant TEEs. Consequently, in addition to their limit adaptability and generality, these protocols' correct implementation and deployment requires that there will {\em always} be available the required number of machines equipped with specific CPU generation and TEE hardware versions. In any other case, system designers are compelled to be able to quickly learn and program any another available TEE. This complicates the widespread adoption of such protocols because the task of programming heterogeneous TEEs as rather challenging~\cite{10.1145/3460120.3485341} as error prone; various TEEs present different programming models and security properties~\cite{10.1007/978-3-031-16092-9_7}. %To sum up, this heterogeneity complicates the widespread adoption of such protocols.
%it raises questions regarding performance and correctness. 

%---in fact, we found that all such open-sourced protocols~\cite{hybster, 10.1145/3492321.3519568, minBFT, DBLP:journals/corr/LiuLKA16a} are built on top Intel SGX~\cite{intel-sgx}.

%We argue that the protocols' reliance on such specific TEEs limits generality and adoption in modern heterogeneous cloud infrastructure. For example, to deploy these protocols, a cloud provider must guarantee that there would always be available a required number of machines equipped with specific CPU generation and TEE hardware versions. In cases where there are no machines available, or the TEE has been discontinued (e.g., this is the case with Intel SGX~\cite{sgx_deprecated}), protocol designers are compelled to be able to quickly learn and program another available TEE. However, programming heterogeneous TEEs is a task rather challenging~\cite{10.1145/3460120.3485341}, and it raises questions regarding performance and correctness. Heterogeneous TEEs not only have different programming and performance characteristics but the security properties they offer can greatly vary too~\cite{10.1007/978-3-031-16092-9_7}. As an example of this, Keystone (RISC-V)~\cite{keystone_eurosys} and Intel SGX (x86)~\cite{intel-sgx} have quite different programming APIs. In addition, Intel SGX can only support a very limited Trusted Computing Base (TCB), only up to 256MB, compared to Intel TDX~\cite{intelTDX} and AMD-SEV~\cite{amd-sev}. 
%\antonis{may mention arm as well?}
%Importantly, commercially available TEEs~\cite{arm-trustzone, intel-sgx, amd-sev, keystone_eurosys, 197162, timber} offer different levels of security (i.e., integrity, freshness, and confidentiality), whereas not all of them come with built-in support for secure bootstrapping and remote attestation~\cite{10.1007/978-3-031-16092-9_7, 7807249, secTEE}. All these characteristics make the widespread adoption of existing BFT protocols impractical.


\myparagraph{Solution} We attack this challenge by removing any dependencies on CPU-based TEEs and unshackling the designers from having to continuously learn and program various TEEs. Our \projecttitle{} makes use of programmable hardware, i.e., FPGAs, to implement a trusted network stack offloading any security-related processing in the NIC hardware ($\S$~\ref{subsec:tfpga}) and to offer a unified abstraction (network library) to the system designers ($\S$~\ref{sec:net-lib}). While our \projecttitle{} shifts the homogeneity from the CPU layer to the FPGA-based NIC layer, our architectural design is not hypothetical; \projecttitle{} fits well in recent deployments in commercial clouds, e.g. Microsoft's Catapult design~\cite{msr_smartnics, 211249}. 

%machines. To achieve this, we implement our entire network stack, including the trusted subsystem for non-equivocation and authentication, on the NIC's hardware-level, leveraging the SmartNIC technology~\cite{}. Essentially, our \projecttitle{} shifts the homogeneity from the CPU layer to the NIC layer:  while the host CPUs participating in a protocol can be heterogeneous, our \projecttitle{} is built on top of homogeneous SmartNICs \antonis{seems stricter than what we assume? maybe FPGA-enabled networking?}, exposing a unified abstraction for exchanging and verifying network messages.

%We implemented our network stack on top of FPGA-based SmartNICs, specifically Alveo U280~\cite{alveo_smartnics}. While our design could be adapted to be applicable to  SoC-based SmartNICs (e.g., Mellanox BlueField~\cite{bluefield_smartnics}, Alveo SN1000~\cite{alveo_sn1000}), we decided against it due to their performance limitations in 100G era~\cite{211249}.\antonis{weak argument, Bluefield 3 is up to 400Gbits. Let's say: "Our design is applicable to ... and our evaluation shows that hardware implementation is more beneficial.} Instead, given the increasingly wide deployment of such specialized hardware in DCs as an efficient way to offload network processing our architectural design is not hypothetical. \projecttitle{} fits well in recent deployments in commercial clouds. An example of this is Microsoft’s Catapult, where the FPGA, which sits on the data path in front of the network card and applies ``smart'' processing, could be extended (as in $\S$~\ref{subsec:tfpga}) to improve security.

%Our system achieves these goals by implementing a minimal hardware-based authentication subsystem, the \emph{attestation kernel}, that guarantees the necessary security properties required for BFT. The attestation kernel generates and verifies authentication certificates for the network messages to ensure two core security properties: the non-equivocation and the (transferable) authentication properties. These two properties have been proven to be necessary and sufficient for decreasing the replication factor for BFT protocols~\cite{clement2012}. Further, we carefully implemented \projecttitle{} to optimize for performance and scalability by integrating the attestation kernel ``on path'' at the NIC-level. That way, our system offers security without introducing unnecessary overheads both in performance and adoption: data is processed \emph{on their way} to the network whereas the protocols do not have to rely on a specific TEE in the host CPU layer. 
\fi
%\section{Overview}
\dimitra{here}
\subsection{Design challenges for a trusted NIC}
\label{sec:requirements}
Our work is motivated by 
Our proposal for \projecttitle{} seeks to resolve the tension between security and performance for (distributed) applications in the modern cloud infrastructure, specifically targeting replication protocols\pramod{focus on building trusted distributed systems -- PLEASE DONT MAKE THIS PAPER a REPLICATION protocol paper.}. To this end, our goal is to help system designers to build efficient, fault-tolerant, and robust systems that remain correct even when the untrusted cloud infrastructure fails arbitrarily. 

To sum up, we target the following design properties: 
\begin{itemize}
    \item {\bf{Robustness}} for building and deploying fault-tolerant distributed protocols and applications that remain correct and available in the untrusted cloud infrastructure.
    \item {\bf{Performance and scalability}} for efficiently exchanging and authenticating network messages. Particularly, our \projecttitle{} allows for BFT with the minimum possible number of involved servers ($2f+1$), mitigating clients' costs and resource usage.
    \item {\bf{Adaptability}} for allowing protocol designers to build robust protocols that can seamlessly run in modern heterogeneous DCs without them having to be BFT experts. 
\end{itemize}


\myparagraph{Challenge \#1: Robustness} The distributed applications hosted in the third-party cloud infrastructure need to be highly available. As such, cloud services and applications build on top of replication protocols that offer fault tolerance and, importantly, can remain available when failures occur~\cite{Jimenez-Peris2001}. The vast majority of such deployed protocols in the cloud operate under the Crash Fault Tolerant model (CFT) where the machines can {\em only} fail by crashing or omitting some steps. However, the CFT model is inadequate in modern cloud infrastructure as it has been observed that the machines or other parts of the infrastructure can fail arbitrarily (i.e., exhibit Byzantine behavior~\cite{Lamport:1982}) due to malicious adversaries, compromised OS/hypervisor in machines, faulty network links and compromised memory and storage medium~\cite{Gunawi_bugs-in-the-cloud, ciad, fast-08-corruption, security-one-sided-communication, accountable-cloud}. Consequently, the current replication protocols target a quite limited fault model which does not match modern's applications security needs that are hosted in the (untrusted) cloud.

\myparagraph{Solution} Our \projecttitle{} overcomes this limitation by offering  trusted and easily adoptable network operations, which are at the core of any distributed protocol, to help system designers building more robust protocols. To achieve this, we build an extended implementation of the classical, widely-adopted, RDMA network stack~\cite{rdma} on programmable hardware, i.e., FPGA-based SmartNICs~\cite{u280_smartnics}, offloading on this hardware the necessary required security processing and mechanisms. 

%More importantly, conventional BFT protocols present the following characteristics. First, they require an extra set of $f$ participant machines to tolerate up to $f$ failures. Compared to CFT protocols that operate with $2f+1$ participants, classical BFT protocols present limited scalability as they require at least $3f+1$ participants~\cite{BFT_THEORY}. In addition to this, BFT protocols can be slow as they usually run at least three phases of broadcasts~\cite{Castro:2002, DBLP:journals/corr/abs-1803-05069} and incur high message complexity (e.g., $O(n^2)$). Lastly, BFT protocols are complex: they are hard to understand, let alone be optimised~\cite{10.1145/2658994}. Even intuitive algorithmic improvements to optimize for the common case or recovery can significantly affect other parts of the protocol (e.g., view-change in~\cite{10.1145/1658357.1658358}, normal case adds 2 extra phases in~\cite{DBLP:journals/corr/abs-1803-05069}) .% Consequently, they have seen little adoption in commercial cloud applications due to their limited scalability and performance.


\myparagraph{Challenge \#2: Performance and scalability} Researchers~\cite{Castro:2002, DBLP:journals/corr/abs-1803-05069, 10.1145/1658357.1658358} presented a range of robust replication protocols that remain correct when arbitrary failures occur targeting the BFT model~\cite{Lamport:1982}. Unfortunately, these BFT protocols have recognised little adoption because they cannot meet the performance requirements of deployed applications~\cite{bft-time-is-now}. In addition, the vast majority~\cite{Castro:2002, DBLP:journals/corr/abs-1803-05069} introduces resources overheads and limits scalability because it requires at least $3f+1$ machines to tolerate up to $f$ faults. That is, at least more $f$ machines compared to currently deployed CFT protocols. Consequently, BFT protocols are not well suited for performance in modern high-end distributed systems~\cite{bftForSkeptics}.

\myparagraph{Solution} We bridge the gap between performance, scalability and robustness. Our \projecttitle{} offers robustness by materialising the necessary foundations for building BFT protocols~\cite{clement2012} in programmable, yet fast, hardware, while it also improves performance and scalability by limiting the number of required participant machines to the minimum, i.e., $2f+1$. More specifically, \projecttitle{} implements the theoretical foundations of Clement et. al~\cite{clement2012} to translate a CFT protocol to a BFT protocol without having to increase the CFT protocol's replication degree. We explain this mechanism in $\S$~\ref{sec:background}.  %Their work has shown that a translation between any CFT protocol to a BFT protocol {\em always} exists if the security properties of the transferable authentication and the non-equivocation are guaranteed. We discuss the properties and the translation mechanism in$\S$~\ref{}. 

%\begin{itemize}
 %   \item {\bf{Transferable authentication.}} Potentially malicious nodes cannot impersonate other (honest) nodes. Essentially, any node can verify that a message is signed by the correct sender, even for forwarded messages.
 %   \item {\bf{Non-equivocation.}} A sender cannot send different messages to different nodes in the same round while it is supposed to send the same message according to the protocol.
%\end{itemize}

%we designed and implemented \projecttitle{} to offer the two properties of non-equivocation and transferable authentication that allow us to design and build BFT protocols with the minimum possible participant nodes ($2f+1$), resolving the trade-off of scalability, performance, and BFT at once. 

%More precisely, our design relies on the theoretical findings 

%Our \projecttitle{} materializes these properties on the NIC-level by implementing and integrating an attestation kernel for generating message authentication certificates or attestations and verifying those when messages are received ($\S$~\ref{subsec:tfpga}). That way \projecttitle{} builds and exposes the minimal abstraction required for implementing robust protocols under the Byzantine Fault model with $2f+1$ participant nodes.


\myparagraph{Challenge \#3: Adaptability} Due to the traditional BFT protocols limitations, a new line of research has attempted to optimize them~\cite{10.1145/3492321.3519568, minBFT, hybster, 10.1145/2168836.2168866, DBLP:journals/corr/LiuLKA16a, trinc} making use of trusted hardware, precisely, Trusted Execution Environments (TEEs)~\cite{cryptoeprint:2016:086, arm-realm, amd-sev, riscv-multizone, intelTDX}.  Unfortunately, the safety requirements of these optimized protocols highly depend on very specific and CPU-dependant TEEs. Consequently, in addition to their limit adaptability and generality, these protocols' correct implementation and deployment requires that there will {\em always} be available the required number of machines equipped with specific CPU generation and TEE hardware versions. In any other case, system designers are compelled to be able to quickly learn and program any another available TEE. This complicates the widespread adoption of such protocols because the task of programming heterogeneous TEEs as rather challenging~\cite{10.1145/3460120.3485341} as error prone; various TEEs present different programming models and security properties~\cite{10.1007/978-3-031-16092-9_7}. %To sum up, this heterogeneity complicates the widespread adoption of such protocols.
%it raises questions regarding performance and correctness. 

%---in fact, we found that all such open-sourced protocols~\cite{hybster, 10.1145/3492321.3519568, minBFT, DBLP:journals/corr/LiuLKA16a} are built on top Intel SGX~\cite{intel-sgx}.

%We argue that the protocols' reliance on such specific TEEs limits generality and adoption in modern heterogeneous cloud infrastructure. For example, to deploy these protocols, a cloud provider must guarantee that there would always be available a required number of machines equipped with specific CPU generation and TEE hardware versions. In cases where there are no machines available, or the TEE has been discontinued (e.g., this is the case with Intel SGX~\cite{sgx_deprecated}), protocol designers are compelled to be able to quickly learn and program another available TEE. However, programming heterogeneous TEEs is a task rather challenging~\cite{10.1145/3460120.3485341}, and it raises questions regarding performance and correctness. Heterogeneous TEEs not only have different programming and performance characteristics but the security properties they offer can greatly vary too~\cite{10.1007/978-3-031-16092-9_7}. As an example of this, Keystone (RISC-V)~\cite{keystone_eurosys} and Intel SGX (x86)~\cite{intel-sgx} have quite different programming APIs. In addition, Intel SGX can only support a very limited Trusted Computing Base (TCB), only up to 256MB, compared to Intel TDX~\cite{intelTDX} and AMD-SEV~\cite{amd-sev}. 
%\antonis{may mention arm as well?}
%Importantly, commercially available TEEs~\cite{arm-trustzone, intel-sgx, amd-sev, keystone_eurosys, 197162, timber} offer different levels of security (i.e., integrity, freshness, and confidentiality), whereas not all of them come with built-in support for secure bootstrapping and remote attestation~\cite{10.1007/978-3-031-16092-9_7, 7807249, secTEE}. All these characteristics make the widespread adoption of existing BFT protocols impractical.


\myparagraph{Solution} We attack this challenge by removing any dependencies on CPU-based TEEs and unshackling the designers from having to continuously learn and program various TEEs. Our \projecttitle{} makes use of programmable hardware, i.e., FPGAs, to implement a trusted network stack offloading any security-related processing in the NIC hardware ($\S$~\ref{subsec:tfpga}) and to offer a unified abstraction (network library) to the system designers ($\S$~\ref{sec:net-lib}). While our \projecttitle{} shifts the homogeneity from the CPU layer to the FPGA-based NIC layer, our architectural design is not hypothetical; \projecttitle{} fits well in recent deployments in commercial clouds, e.g. Microsoft's Catapult design~\cite{msr_smartnics, 211249}. 

%machines. To achieve this, we implement our entire network stack, including the trusted subsystem for non-equivocation and authentication, on the NIC's hardware-level, leveraging the SmartNIC technology~\cite{}. Essentially, our \projecttitle{} shifts the homogeneity from the CPU layer to the NIC layer:  while the host CPUs participating in a protocol can be heterogeneous, our \projecttitle{} is built on top of homogeneous SmartNICs \antonis{seems stricter than what we assume? maybe FPGA-enabled networking?}, exposing a unified abstraction for exchanging and verifying network messages.

%We implemented our network stack on top of FPGA-based SmartNICs, specifically Alveo U280~\cite{alveo_smartnics}. While our design could be adapted to be applicable to  SoC-based SmartNICs (e.g., Mellanox BlueField~\cite{bluefield_smartnics}, Alveo SN1000~\cite{alveo_sn1000}), we decided against it due to their performance limitations in 100G era~\cite{211249}.\antonis{weak argument, Bluefield 3 is up to 400Gbits. Let's say: "Our design is applicable to ... and our evaluation shows that hardware implementation is more beneficial.} Instead, given the increasingly wide deployment of such specialized hardware in DCs as an efficient way to offload network processing our architectural design is not hypothetical. \projecttitle{} fits well in recent deployments in commercial clouds. An example of this is Microsoft’s Catapult, where the FPGA, which sits on the data path in front of the network card and applies ``smart'' processing, could be extended (as in $\S$~\ref{subsec:tfpga}) to improve security.

%Our system achieves these goals by implementing a minimal hardware-based authentication subsystem, the \emph{attestation kernel}, that guarantees the necessary security properties required for BFT. The attestation kernel generates and verifies authentication certificates for the network messages to ensure two core security properties: the non-equivocation and the (transferable) authentication properties. These two properties have been proven to be necessary and sufficient for decreasing the replication factor for BFT protocols~\cite{clement2012}. Further, we carefully implemented \projecttitle{} to optimize for performance and scalability by integrating the attestation kernel ``on path'' at the NIC-level. That way, our system offers security without introducing unnecessary overheads both in performance and adoption: data is processed \emph{on their way} to the network whereas the protocols do not have to rely on a specific TEE in the host CPU layer. 

\section{Overview}
To this end we propose \projecttitle{}, a trusted NIC architecture that fullfils the afforementioned design goals. Figure~\ref{fig:overview} shows our \projecttitle{} system design. The system is comprised of three layers: (1) the application layer that runs the (distributed) protocol code (We show four uses cases of \projecttitle{} application in $\S$~\ref{sec:use_cases}), (2) the \projecttitle{} software library which exposes the \projecttitle{} API ($\S$~\ref{sec:net-lib}) and (3) the \projecttitle{} hardware implementation on top of FPGA devices ($\S$~\ref{sec:t-nic-hardware}). Each machine comprises a (host) CPU to which it is attached a \projecttitle{} through the PCIe~\cite{pcie}. The \projecttitle{} devices are connected through direct (Ethernet) links or a switch. %a $X$Gbps Ethernet switch through its QSFP port 0.

The  \projecttitle{} software library implements the programming framework and the \projecttitle{} device access interface. The programming framework exposes a user-space network library for trusted one-sided RDMA (reliable) operations. In addition, we expose a memory management API for correctly initializing and registering the registered-to-the-NIC memory. The software part of our \projecttitle{} also includes a driver that inserts a kernel module to set up the device with the static configuration (MAC address, IP, etc.) and further allocates and maps a specific (host) memory area to the device. While the driver runs only once at the initialization, the mapped memory is used to pass down to the device requests and any related parameters. 


The hardware implementation of \projecttitle{} runs the RDMA protocol on top of FPGA-based SmartNIC hardware. We further extend the network protocol kernel with a minimal hardware security module, the attestation kernel which shields and verifies the network traffic across the participant nodes. Specifically, the security module  constructs a secure message format that is verifiable and the network stack module implements the RDMA network protocol. In contrast to the control and configuration paths shown in Figure~\ref{fig:overview}, the message to be sent, is fetched from the application's memory directly to the device through DMA transfers. The message, after being shielded, is then forwarded to the network stack module for transmission.  % to preserve the authentication and non-equivocation properties across all network operations.
%On top of our trusted network stack, we implement a networking library, \projectlibrary{}, that exposes the developers to the classical RDMA API enhanced with two security properties, i.e., the transferable authentication and the non-equivocation.


\begin{figure}[t!]
    \centering
    %\includegraphics[width=0.5\textwidth]{figures/trusted-nic-overview.drawio.pdf}
    \includegraphics[width=0.5\textwidth]{figures/system_overview-2.pdf}
    \caption{\projecttitle{} system overview.}
    \label{fig:overview}
\end{figure}
%\section{Background}\label{sec:backgrnd}

\subsection{Cold Start Latency and Mitigation Techniques}

Traditional FaaS platforms mitigate cold starts through snapshotting, lightweight virtualization, and warm-state management. Snapshot-based methods like \textbf{REAP} and \textbf{Catalyzer} reduce initialization time by preloading or restoring container states but require significant memory and I/O resources, limiting scalability~\cite{dong_catalyzer_2020, ustiugov_benchmarking_2021}. Lightweight virtualization solutions, such as \textbf{Firecracker} microVMs, achieve fast startup times with strong isolation but depend on robust infrastructure, making them less adaptable to fluctuating workloads~\cite{agache_firecracker_2020}. Warm-state management techniques like \textbf{Faa\$T}~\cite{romero_faa_2021} and \textbf{Kraken}~\cite{vivek_kraken_2021} keep frequently invoked containers ready, balancing readiness and cost efficiency under predictable workloads but incurring overhead when demand is erratic~\cite{romero_faa_2021, vivek_kraken_2021}. While these methods perform well in resource-rich cloud environments, their resource intensity challenges applicability in edge settings.

\subsubsection{Edge FaaS Perspective}

In edge environments, cold start mitigation emphasizes lightweight designs, resource sharing, and hybrid task distribution. Lightweight execution environments like unikernels~\cite{edward_sock_2018} and \textbf{Firecracker}~\cite{agache_firecracker_2020}, as used by \textbf{TinyFaaS}~\cite{pfandzelter_tinyfaas_2020}, minimize resource usage and initialization delays but require careful orchestration to avoid resource contention. Function co-location, demonstrated by \textbf{Photons}~\cite{v_dukic_photons_2020}, reduces redundant initializations by sharing runtime resources among related functions, though this complicates isolation in multi-tenant setups~\cite{v_dukic_photons_2020}. Hybrid offloading frameworks like \textbf{GeoFaaS}~\cite{malekabbasi_geofaas_2024} balance edge-cloud workloads by offloading latency-tolerant tasks to the cloud and reserving edge resources for real-time operations, requiring reliable connectivity and efficient task management. These edge-specific strategies address cold starts effectively but introduce challenges in scalability and orchestration.

\subsection{Predictive Scaling and Caching Techniques}

Efficient resource allocation is vital for maintaining low latency and high availability in serverless platforms. Predictive scaling and caching techniques dynamically provision resources and reduce cold start latency by leveraging workload prediction and state retention.
Traditional FaaS platforms use predictive scaling and caching to optimize resources, employing techniques (OFC, FaasCache) to reduce cold starts. However, these methods rely on centralized orchestration and workload predictability, limiting their effectiveness in dynamic, resource-constrained edge environments.



\subsubsection{Edge FaaS Perspective}

Edge FaaS platforms adapt predictive scaling and caching techniques to constrain resources and heterogeneous environments. \textbf{EDGE-Cache}~\cite{kim_delay-aware_2022} uses traffic profiling to selectively retain high-priority functions, reducing memory overhead while maintaining readiness for frequent requests. Hybrid frameworks like \textbf{GeoFaaS}~\cite{malekabbasi_geofaas_2024} implement distributed caching to balance resources between edge and cloud nodes, enabling low-latency processing for critical tasks while offloading less critical workloads. Machine learning methods, such as clustering-based workload predictors~\cite{gao_machine_2020} and GRU-based models~\cite{guo_applying_2018}, enhance resource provisioning in edge systems by efficiently forecasting workload spikes. These innovations effectively address cold start challenges in edge environments, though their dependency on accurate predictions and robust orchestration poses scalability challenges.

\subsection{Decentralized Orchestration, Function Placement, and Scheduling}

Efficient orchestration in serverless platforms involves workload distribution, resource optimization, and performance assurance. While traditional FaaS platforms rely on centralized control, edge environments require decentralized and adaptive strategies to address unique challenges such as resource constraints and heterogeneous hardware.



\subsubsection{Edge FaaS Perspective}

Edge FaaS platforms adopt decentralized and adaptive orchestration frameworks to meet the demands of resource-constrained environments. Systems like \textbf{Wukong} distribute scheduling across edge nodes, enhancing data locality and scalability while reducing network latency. Lightweight frameworks such as \textbf{OpenWhisk Lite}~\cite{kravchenko_kpavelopenwhisk-light_2024} optimize resource allocation by decentralizing scheduling policies, minimizing cold starts and latency in edge setups~\cite{benjamin_wukong_2020}. Hybrid solutions like \textbf{OpenFaaS}~\cite{noauthor_openfaasfaas_2024} and \textbf{EdgeMatrix}~\cite{shen_edgematrix_2023} combine edge-cloud orchestration to balance resource utilization, retaining latency-sensitive functions at the edge while offloading non-critical workloads to the cloud. While these approaches improve flexibility, they face challenges in maintaining coordination and ensuring consistent performance across distributed nodes.


%\section{Design goals for the \projecttitle{} architecture}
\label{sec:requirements}
Our proposal for the design of a \projecttitle{} architecture targets to resolve the tension between security and performance when it comes to the design and implementation of replication protocols that are widely deployed in modern heterogeneous datacenters. To this end, our \projecttitle{} is specifically designed to help system designers build efficient, fault-tolerant, and robust systems under the presence of Byzantine actors in the untrusted cloud infrastructure. Specifically, our system targets the following design properties: 
\begin{itemize}
    \item {\bf{Robustness}} for building and deploying fault-tolerant distributed protocols and applications that remain correct and available in the untrusted cloud infrastructure.
    \item {\bf{Performance and scalability}} for efficiently exchanging and authenticating network messages. Particularly, our \projecttitle{} allows for BFT with the minimum possible number of involved servers ($2f+1$), mitigating clients' costs and resource usage.
    \item {\bf{Adaptability}} for allowing protocol designers to build robust protocols that can seamlessly run in modern heterogeneous DCs without them having to be BFT experts. 
\end{itemize}

\vspace{1pt}
Our system achieves these goals by implementing a minimal hardware-based authentication subsystem, the \emph{attestation kernel}, that guarantees the necessary security properties required for BFT. The attestation kernel generates and verifies authentication certificates for the network messages to ensure two core security properties: the non-equivocation and the (transferable) authentication properties. These two properties have been proven to be necessary and sufficient for decreasing the replication factor for BFT protocols~\cite{clement2012}. Further, we carefully implemented \projecttitle{} to optimize for performance and scalability by integrating the attestation kernel ``on path'' at the NIC-level. That way, our system offers security without introducing unnecessary overheads both in performance and adoption: data is processed \emph{on their way} to the network whereas the protocols do not have to rely on a specific TEE in the host CPU layer. 

\antonis{Consider another subsection here.}
Next, we explain the two core problems aroused from prior research efforts on BFT protocols and how our \projecttitle{} overcomes those.


%\dimitra{
%\begin{enumerate}
%    \item Problem
%    \item Prior work 
%    \item Our solution
%\end{enumerate}
%}

\myparagraph{Problem 1: Security and performance} Prior research~\cite{Castro:2002, DBLP:journals/corr/abs-1803-05069, 10.1145/1658357.1658358} introduced replication protocols that are executed correctly in the presence of Byzantine actors (e.g., malicious adversaries, compromised participant nodes, faulty network links and memory). Unfortunately, while these protocols offer BFT, they give up on performance and scalability rendering them impractical for modern high-end distributed systems~\cite{bftForSkeptics}. 

More importantly, conventional BFT protocols present the following characteristics. First, they require an extra set of $f$ participant machines to tolerate up to $f$ failures. Compared to CFT protocols that operate with $2f+1$ participants, classical BFT protocols present limited scalability as they require at least $3f+1$ participants~\cite{BFT_THEORY}. In addition to this, BFT protocols can be slow as they usually run at least three phases of broadcasts~\cite{Castro:2002, DBLP:journals/corr/abs-1803-05069} and incur high message complexity (e.g., $O(n^2)$). Lastly, BFT protocols are complex: they are hard to understand, let alone be optimised~\cite{10.1145/2658994}. Even intuitive algorithmic improvements to optimize for the common case or recovery can significantly affect other parts of the protocol (e.g., view-change in~\cite{10.1145/1658357.1658358}, normal case adds 2 extra phases in~\cite{DBLP:journals/corr/abs-1803-05069}) .% Consequently, they have seen little adoption in commercial cloud applications due to their limited scalability and performance.

\myparagraph{Proposal} In our work, we target to bridge this gap between robustness (BFT) and performance. Our \projecttitle{} offers the system designers the necessary foundations for building BFT protocols while it optimizes the performance and scalability of the derived protocols. Specifically, we designed and implemented \projecttitle{} to offer the two properties of non-equivocation and transferable authentication that allow us to design and build BFT protocols with the minimum possible participant nodes ($2f+1$), resolving the trade-off of scalability, performance, and BFT at once. 

More precisely, our design relies on the theoretical findings of Clement et al.~\cite{clement2012}. Their work has shown that a translation between any CFT protocol to a BFT protocol always exists iff the following properties are guaranteed:
\begin{itemize}
    \item {\bf{Transferable authentication.}} Potentially malicious nodes cannot impersonate other (honest) nodes. Essentially, any node can verify that a message is signed by the correct sender, even for forwarded messages.
    \item {\bf{Non-equivocation.}} A sender cannot send different messages to different nodes in the same round while it is supposed to send the same message according to the protocol.
\end{itemize}

Our \projecttitle{} materializes these properties on the NIC-level by implementing and integrating an attestation kernel for generating message authentication certificates or attestations and verifying those when messages are received ($\S$~\ref{subsec:tfpga}). That way \projecttitle{} builds and exposes the minimal abstraction required for implementing robust protocols under the Byzantine Fault model with $2f+1$ participant nodes.

%In contrast, our trusted NIC allows heterogeneous nodes to communicate with each other in a secure manner over the untrusted cloud infrastructure. 
\myparagraph{Problem 2: Seamless adoption in heterogeneous DCs} While the classical BFT protocols significantly limit performance, a new line of research has made an attempt to optimize such protocols introducing a new family of hybrid BFT protocols that operate with fewer replicas. Hybrid BFT protocols~\cite{10.1145/3492321.3519568, minBFT, hybster, 10.1145/2168836.2168866, DBLP:journals/corr/LiuLKA16a, trinc} rely on Trusted Execution Environments (TEEs)~\cite{cryptoeprint:2016:086, arm-realm, amd-sev, riscv-multizone, intelTDX}  to prevent equivocation.  Unfortunately, when it comes to their implementation, hybrid protocols have been built on top of specific and CPU-dependant TEEs---in fact, we found that all open-sourced hybrid protocols~\cite{hybster, 10.1145/3492321.3519568, minBFT, DBLP:journals/corr/LiuLKA16a} are specifically built on top Intel SGX~\cite{intel-sgx}.

We argue that the protocols' reliance on such specific TEEs limits generality and adoption in modern heterogeneous cloud infrastructure. For example, to deploy these protocols, a cloud provider must guarantee that there would always be available a required number of machines equipped with specific CPU generation and TEE hardware versions. In cases where there are no machines available, or the TEE has been discontinued (e.g., this is the case with Intel SGX~\cite{sgx_deprecated}), protocol designers are compelled to be able to quickly learn and program another available TEE. However, programming heterogeneous TEEs is a task rather challenging~\cite{10.1145/3460120.3485341}, and it raises questions regarding performance and correctness. Heterogeneous TEEs not only have different programming and performance characteristics but the security properties they offer can greatly vary too~\cite{10.1007/978-3-031-16092-9_7}. As an example of this, Keystone (RISC-V)~\cite{keystone_eurosys} and Intel SGX (x86)~\cite{intel-sgx} have quite different programming APIs. In addition, Intel SGX can only support a very limited Trusted Computing Base (TCB), only up to 256MB, compared to Intel TDX~\cite{intelTDX} and AMD-SEV~\cite{amd-sev}. 
\antonis{may mention arm as well?}
Importantly, commercially available TEEs~\cite{arm-trustzone, intel-sgx, amd-sev, keystone_eurosys, 197162, timber} offer different levels of security (i.e., integrity, freshness, and confidentiality), whereas not all of them come with built-in support for secure bootstrapping and remote attestation~\cite{10.1007/978-3-031-16092-9_7, 7807249, secTEE}. All these characteristics make the widespread adoption of existing BFT protocols impractical.

\myparagraph{Proposal} Our \projecttitle{} attacks this challenge and removes the protocols' dependencies on homogeneous CPUs and TEEs. We offer a unified abstraction, a network library ( ..), that allows designers to implement BFT protocols on various CPU machines. To achieve this, we implement our entire network stack, including the trusted subsystem for non-equivocation and authentication, on the NIC's hardware-level, leveraging the SmartNIC technology~\cite{}. Essentially, our \projecttitle{} shifts the homogeneity from the CPU layer to the NIC layer:  while the host CPUs participating in a protocol can be heterogeneous, our \projecttitle{} is built on top of homogeneous SmartNICs \antonis{seems stricter than what we assume? maybe FPGA-enabled networking?}, exposing a unified abstraction for exchanging and verifying network messages.

We implemented our network stack on top of FPGA-based SmartNICs, specifically Alveo U280~\cite{alveo_smartnics}. While our design could be adapted to be applicable to  SoC-based SmartNICs (e.g., Mellanox BlueField~\cite{bluefield_smartnics}, Alveo SN1000~\cite{alveo_sn1000}), we decided against it due to their performance limitations in 100G era~\cite{211249}.\antonis{weak argument, Bluefield 3 is up to 400Gbits. Let's say: "Our design is applicable to ... and our evaluation shows that hardware implementation is more beneficial.} Instead, given the increasingly wide deployment of such specialized hardware in DCs as an efficient way to offload network processing our architectural design is not hypothetical. \projecttitle{} fits well in recent deployments in commercial clouds. An example of this is Microsoft’s Catapult, where the FPGA, which sits on the data path in front of the network card and applies ``smart'' processing, could be extended (as in $\S$~\ref{subsec:tfpga}) to improve security.

%\dimitra{
%\begin{enumerate}
%    \item Problem
%    \item Prior work 
%    \item Our solution
%\end{enumerate}
%}

%Our proposal for the Trusted NIC architecture design needs to consider three design properties: security, performance and adaptability to heterogeneous host CPUs. Our system is based on a hardware-based trusted subsystem that guarantees the minimum properties that are required to be exposed to system and protocols designers to implement robust distributed protocols~\cite{} while it is designed for heterogeneous (untrusted) cloud infrastructures. 


%Prior research has introduced distributed (replication) protocols~\cite{} that are executed correctly even when some participants or parts of the infrastructure (e.g., network links, memory) are malicious, e.g., \emph{Byzantine}~\cite{}. Unfortunately, these protocols usually present a trade-off between performance and heterogenity. To operate under the Byzantine Fault Model~\cite{},  they need to increase their replication factor~\cite{}. Compared to the deployed protocols that only target crashed, but honest nodes (Crash Fault Model~\cite{}), traditional BFT protocols will add an extra $f$ set of replicas, in total $3f+1$ replicas to operate when up to $f$ nodes behave arbitrarily. As such, while necessary in the modern untrusted cloud infrastructure~\cite{}, these protocols have seen little recognition and adoption due to their performance and scalability limitations~\cite{}. On the other hand, state-of-the-art BFT protocols rely on trusted CPU-dependant hardware components (i.e., Trusted Execution Environments~\cite{}) to cut down on the replication factor using only $2f+1$ nodes. Unfortunately, their reliance on TEEs requires that all participant machines support such hardware which comes with different security guarantees and programmability. As such, these protocols give up on their adaptability to modern heterogeneous data centers. 


\dimitra{
\begin{itemize}
    \item section 2: requirements for robust distributed systems (problem/solution briefly)
    \item section 3: background on the technologies: smartNICs -- translation properties 
    \item section 4: overview --> Discuss the system/fault/threat model here too!
\end{itemize}
}






\section{Background} 

\myparagraph{Translation requirements}


\myparagraph{\trustednic{} compared to other SmartNICs}
SmartNIC devices~\cite{liquidIO_smartnics, u280_smartnics, bluefield_smartnics, broadcom_smartnics, netronome_smartnics, alibaba_smartnics, nitro_smartnics, msr_smartnics} have started to appear in several applications showing their benefits to both applications and network acceleration~\cite{211249, 10.1145/3341302.3342079}. Indeed, major cloud providers~\cite{alibaba_smartnics, nitro_smartnics, msr_smartnics} have launched SmartNICs in their datacenters to offload tasks from the host CPU cores onto dedicated hardware.
These cloud solutions build primarily on top of programmable FPGA-based SmartNICs for accelaration~\cite{msr_smartnics} or SoC-based SmartNICs~\cite{bluefield_smartnics} that integrate off-path arrays of wimpy ARM cores (up to 16) to improve programmability (at the cost of performance).

Our \trustednic{} has been built on top of FPGA-based SmartNICs to optimize for performance. \trustednic{} is an on-path SmartNIC that processes all incoming/outgoing messages directly on the communication path, ensuring that their security properties are met. In contrast, off-path SoC-based SmartNICs, which do not offer any security guarantees out of the box, do not optimize for performance. For example, the host needs to send the outgoing messages to the SoC cores for further processing. As DMA transfers are not supported between host and SoC cores, the re-routing is resolved at the NIC level, incurring extra overheads.

Overall, \trustednic{} is a simpler SmartNIC because it just involves an FPGA on the communication path as in~\cite{msr_smartnics}. %Commercial SmartNICs are complete Systems-on-Chip, involving far more hardware (not
%only arrays of cores but also dedicated ASICs for network
%processing, packet switching, memory controllers, potentially
%storage controllers, etc.). As such, they can do more, but they
%also cost more, require more energy, and are more complex
%to deploy, program, and use.

%

\subsection{System Model}
%\manos{Rename to model and assumptions/definitions}
\label{sec:system_model}
\myparagraph{Replication protocol}
A replication protocol ensures that a set of $N$  replicas $(R_i,...,R_n)$ maintains a consistent and available state $S$ despite failures or concurrency. A replication protocol operates over a distributed state machine, where replicas execute a deterministic sequence of operations from the set of operations $O$, s.t. $S \times O \rightarrow S$, and apply them in a consistent order.
Since not all replication protocols solve consensus and thus are not state-machine-replication (SMR) protocols, we do not assume the typical consensus properties (termination, agreement, and validity). Instead, we assume that the replication protocols should guarantee the following properties:
%Thus, general CFT protocols solve the consensus problem, where starting with an initial input value, all nodes have to agree on a single output value, and by extension, the CFT protocol has to satisfy the typical consensus properties (termination, agreement, and validity). CFT protocols should additionally guarantee the following properties:

\begin{itemize}
    \item \textbf{Consistency:} All correct replicas agree on the same sequence of operations.
   \item \textbf{Availability:} Replicas process requests as long as $<f$ replicas fail.
    \item \textbf{Fault tolerance:} The system operates correctly with up to $f$ failures (crash or Byzantine).    
\end{itemize}



\myparagraph{Model sketch}
We model the distributed system as a set of $N$ TEEs in $N$ nodes (or replicas), each hosting either a \emph{follower} or a \textit{coordinator} process $P_i$ which executes a CFT protocol. The system is modeled as a state machine, where each replica $R_i$ maintains a local state $S_i$ and transitions between states based on received messages and protocol execution rules.
%and communicates by exchanging messages. %The terms \textit{replica} and \textit{node} 
%and \emph{process}
%can be used interchangeably. 
We assume that \projecttitle{}'s nodes run in a third-party untrusted cloud infrastructure. %We do not make any assumptions about the client processes.% (From now on simply clients.)---however, it makes sense that they live on separate nodes and forward their requests to the coordinator.
A coordinator serves client requests by initiating the implemented CFT replication protocol. Upon completion, it replies back to clients. In leaderless protocols, coordinators are selected randomly (any node can act as a follower and/or a coordinator). In leader-based protocols, only the active leader can act as a coordinator, the rest of the nodes are followers. %A coordinator is always a replica, but not vice versa.

\myparagraph{Communication model} 
Nodes communicate via a fully-con\-nected, bidirectional, point-to-point and unreliable message-passing network, where messages can be arbitrarily delayed, re-ordered or dropped. In line with previous BFT protocols, we adopt the partial synchrony model~\cite{10.1145/42282.42283}, where there is a known bound $\Delta$ and an unknown Global Stabilization Time (GST), such that after GST, all communications arrive within time $\Delta$.

%We consider an asynchronous network where there are unbounded but finite message transmission delays (messages are delivered \emph{eventually}). 

\myparagraph{Fault and threat model}
%We adopt two failure models: crash-failures ~\cite{crash-consistency} and %Byzantine behavior ~\cite{Lamport:1982}.
We say that a node $N_i$ is \textit{faulty} if it exhibits Byzantine behavior~\cite{Lamport:1982}. 
The unprotected (\emph{out-of-the-TEE}) infrastructure (e.g., host memory, OS, NIC, network infrastructure/adversaries) can exhibit arbitrary Byzantine behavior while we assume that the TEEs can only crash-fail. We say that a node is faulty if one of the following holds true:
\textit{(i)} the TEE fails by crashing or \textit{(ii)} the unprotected infrastructure is Byzantine.
%For safety (and liveness), we assume that for $N \ge (2f+1)$ nodes up to $f$ can be \emph{faulty}.%, while the rest are \textit{correct}.
Safety is defined as follows:  If a correct replica $R_j$ delivers a message $m$  from $R_i$, then $R_i$  must have previously sent $m$, and $m$ is consistent with the protocol state. Liveness is defined as follows: If a client submits a request $r$, and a majority of replicas are correct, then $r$ is eventually committed. Last, for safety and liveness, we assume that for $N \ge (2f+1)$ nodes up to $f$ can be \emph{faulty}.

\myparagraph{Cryptographic model} 
We assume collision resistance for the hash functions; no computationally bounded adversary can find two distinct inputs $m \neq m'$ such that $hash(m)=hash(m')$, except with negligible probability. We also make the conventional assumption that signatures and keys are unforgeable, the initial keys are generated securely, and the private keys are stored securely in the TEE. 

\begin{comment}
\addition{\footnote{\addition{This assumption implies that the executing code within the trusted TEE cannot deviate from the protocol specification and thus a process cannot generate and send conflicting statements for the same protocol round to different replicas. Other than that the unprotected area can equivocate in any possible way, i.e., by replaying, re-ordering and compromising in any way the network traffic and the unprotected memory.}}}
\end{comment}


\if 0
\pramod{can be omitted for space issue!}

\myparagraph{Limitations} \projecttitle{} runs the protocol code in a distributed setting of TEEs which increases the trusted computing base (TCB) size. We assume that the TEEs' hardware and the protocols' implementation are correct; the state transitions are in compliance with the protocol specifications while Byzantine nodes/attackers cannot compromise the properties of TEEs.  We do not protect against Denial-of-Service (DoS) attacks. Nevertheless, DoS attacks are indistinguishable from a network partition and can only affect availability but not safety. \projecttitle{} protocols remain live as long as up to f are subset to DoS attacks. Lastly, our work targets RPC-based CFT strongly-consistent protocols as defined in $\S$~\ref{sec:background:CFT} and we do not explore one-sided RDMA-based protocols~\cite{10.1145/3545008.3545041, 10.1145/3127479.3128609, 10.1145/2749246.2749267}. Lastly, since RDMA is not supported in geo-distributed settings, our \projecttitle{} targets single datacenter deployments.

\fi
%In contrast to classical BFT, we do not protect against implementation errors.

%In this work, we show that TEEs can simplify the faults a replica can experience. Our insight is that TEEs can help us eliminate the Byzantine behavior by shielding the executing protocol inside the trusted area. As such, the (trusted) code cannot deviate from the protocol, the state transitions are in compliance with the protocol specifications.  Other than that, a node, i.e., all the infrastructure outside the TEE, can still exhibit Byzantine behavior. 

%\dimitra{here not sure}This means that a node may appear to not send messages to some or all other nodes during a given operation. This is defined as weak and strong non-equivocation, respectively~\cite{madsen2020}. \projecttitle{} is responsible for neither and relies on the underlying CFT protocol to handle them. This is because weak non-equivocation is indistinguishable from network partitioning, while strong is indistinguishable from a crashed node.


%This is in contrast to prior work~\cite{minBFT, hybster, DBLP:journals/corr/LiuLKA16a, A2M, 10.1145/2168836.2168866} on hybrid BFT/TEE protocols which only port a small trusted module (traditionally a trusted counter for non-equivocation) inside the TEE.  Further, \projecttitle{} assumes that the software implementation of the underlying CFT protocol is correct and, in contrast to classical BFT, does not protect against implementation errors.

\begin{comment}
\section{Practical Security Challenges}\label{sec:security_challenges}

\emph{Can we provide a general recipe for hardening the security properties of non-BFT protocols offering both strong security (integrity, confidentiality and freshness) and performance?}

First we present the practical challenges that derive from our thread model.

\subsection{Untrusted network infrastructure}
\myparagraph{Problem} Under our extreme threat model, an adversary can tamper with the network traffic by re-ordering, delaying, re-sending and compromising the messages.

\myparagraph{Solution} We designed and implemented a secure network stack that leverages the advances of networking technologies. Our library extends the trust to the network infrastructure. Precisely, we ensure the integrity and confidentiality of a message and we guarantee that an operation will be executed at-most-once. The latter protects the system from adversaries that re-send old messages and, thus, force the system to re-execute stale operations.

\subsection{Secure bootstrapping and configuration}
\myparagraph{Problem} Nodes come and go. We need to establish a secure configuration and attestation process to offer fast attestation of incoming nodes.

\myparagraph{Solution} ....

\subsection{Run-time security; correct protocols' state transition}
\myparagraph{Problem} Non-BFT protocols assume that all replicas are honest. However, under our threat model this is far from truth. An adversary is able to ``change'' the execution path of the running code -- for example,  by manipulating the registers and the memory. As a result, even a correctly implemented protocol can reply incorrectly to both participant nodes and clients.

\myparagraph{Solution} With the \underline{assumption} that both the TEE and protocol implementations are correct, we can guarantee that the nodes running the protocol inside the TEE either will be honest or will not reply at all. In any case, the execution of the protocol inside the TEE guarantees that no replica will respond intentionally dishonestly. This is our key insight: non-BFT protocols running inside TEEs can tolerate Byzantine settings while guaranteeing strict security.


\subsection{Protocol correctness; equivocation and split-brain problem} 
\myparagraph{Problem} Such a privileged attacker can \emph{easily} compromise the correctness and consistency properties of the protocols since our proposed network stack cannot protect us from forking attacks. Attackers might cause a network partition and further convince the system that both partitions are active. Note that decentralized protocols, e.g. like ABD, which rely on quorum, tolerate forking attacks: at most one partition (w/ majority) would be able to execute both read and write requests. However, all well-known leader-based protocols, such as Raft, CR, CRAQ, etc. cannot directly handle forking attacks.

\underline{A motivating example is the case of Raft}. Let's assume that after a network partition, the old leader with the minority is active in one partition, while a new leader is elected after some time in the partition with the majority. In a trusted system, the old leader is able to serve read requests but not write requests (since writes would require acknowledgments by the majority). The new leader is only allowed to execute writes after the old leader becomes inactive. That is typically achieved through the use of leases -- the elected leader is always aware of the expiration time of the previous leader's lease and delays writes until it is expired. 

Unfortunately, in our untrusted world, we do not have a trusted time source. Attackers might compromise the timers' correctness in many ways (\dimitra{T-Lease}). As a result, attackers could be able to ``keep'' two leaders active in two different network partitions. In that case, both leaders are able to respond to read requests and, additionally, the leader in the partition with the majority can also execute writes. Similar scenarios are also possible to happen in other leader-based protocols like Chain Replication where the attacker can ``fork'' two tail nodes in two different network partitions.


\myparagraph{Solution} To address this major challenge, our recipe should ensure the following: \emph{(i)} nodes that are not part of the configuration/majority should deterministically become inactive and \emph{(ii)} clients cannot get two different responses for the same request. The first property \emph{(i)} safeguards the protocol from forking attacks that violate consistency. The protocol developers can adopt T-Leases that offer with a trusted lease mechanism, exclusively designed to ensure correctness in untrusted systems with our threat model. More precisely, a T-Lease ensures that even with an untrusted timer source the duration of a lease from the granters point of view is always a superset of the duration of this lease from the holder's point of view. Further, T-Lease mechanism protects against TOCTOU attacks. T-Leases have minimal overheads compared to native leases (5\%).

The second property \emph{(ii)} ensures that the attacker cannot tamper with client replies by sending to them an older, yet correct, reply to that request, e.g. an old read-request for the same key. our leader-based protocols should force the client to communicate with the request's leader/coordinator. In particular, a client sends the request to the node that believes or knew that it should be responsible for replying to their request. If this node is the leader, then the client will get a reply. Otherwise the node will send an error message to the client indicating which node is the leader (if they know). In these cases, the client should re-execute the operation.


\myparagraph{Solution} To address this major challenge, our recipe should ensure that in this untrusted environment any nodes that are not part of the configuration/majority should deterministically become inactive. This requirement safeguards the protocol from forking attacks that might violate consistency. To achieve this, protocol developers should adopt T-Leases that offer with a trusted lease mechanism, exclusively designed to ensure correctness in untrusted systems with our threat model. More precisely, a T-Lease ensures that even with an untrusted timer source the duration of a lease from the granters point of view is always a superset of the duration of this lease from the holder's point of view. Further, T-Lease mechanism protects against TOCTOU attacks. T-Leases have minimal overheads compared to native leases (5\%).


\subsection{Performance}
\myparagraph{Problem} TEEs do not directly communicate with the OS since the latter cannot be not trusted. As a result, the application threads should exit the enclave to execute a system call and these enclave transitions are costly.

\myparagraph{Solution} We overcome this limitation by adopting the recent advances in network technologies such as RDMA and DPDK. Both technologies offer kernel bypass decoupling the networking operations from the kernel.
\fi
\end{comment}
\section{The \search\ Search Algorithm}
\label{sec:search}

%In traditional ML, structure changes and step (operator) changes are performed before model training, \ie, fixed to the training process, and weights are updated with SGD, because weights are continous, differentiable values, and there are significantly more weights than structure and operator changes. In workflow autotuning, all three types of cogs can be chosen with a unified search-based approach, because all of them are non-differentiable configurations and the number of cogs in different types are all small.
%Thus, \sysname\ only needs to navigate the search space of combination of cogs as the search space to produce its workflow optimization results.

%We propose, \textit{\textbf{\search}}, an adaptive hierarchical search algorithm that autotunes gen-AI workflows based on observed end-to-end workflow results. In each search iteration, \search\ selects a combination of cogs to apply to the workflow and executes the resulting workflow with user-provided training inputs. \search\ evaluates the final generation quality using the user-specified evaluator and measures the execution time and cost for each training input. These results are aggregated and serve as BO observations and pruning criteria.
%the optimizer can condition on and propose better configurations in later trials. The optimizer will also be informed about the violation of any user-specified metric thresholds. More details of this mechanism can be found in Appendix ~\ref{appdx:TPE}.

With our insights in Section~\ref{sec:theory}, we believe that search methods based on Bayesian Optimizer (BO) can work for all types of cogs in gen-AI workflow autotuning because of BO's efficiency in searching discrete search space.
A key challenge in designing a BO-based search is the limited search budgets that need to be used to search a high-dimensional cog space. 
For example, for 4 cogs each with 4 options and a workflow of 3 LLM steps, the search space is $4^{12}$. Suppose each search uses GPT-4o and has 1000 output tokens, the entire space needs around \$168K to go through. A user search budget of \$100 can cover only 0.06\% of the search space. A traditional BO approach cannot find good results with such small budgets.
%The entire search space grows exponentially with the number of cogs and the number of steps in a workflow. Moreover, different cogs and different combinations of cogs can have varying impacts on different workflows. 
%Without prior knowledge, it is difficult to determine the amount of budget to give to each cog.

To confront this challenge, we propose \textit{\textbf{\search}}, an adaptive hierarchical search algorithm that efficiently assigns search budget across cogs based on budget size and observed workflow evaluation results, as defined in Algorithms~\ref{alg:main} and \ref{alg:outer} and described below.
%autotunes gen-AI workflows based on observed end-to-end workflow results.
%\search\ includes a search layer partitioning method, a search budget initial assignment method, an evaluation-guided budget re-allocation mechanism, and a convergence-based early-exiting strategy. We discuss them in details below.

%\zijian{\search\ allows users to specify the optimization budget allowed in terms of the maximum number of search iterations. Based on the relationship between the complexity of the search space and the available budget, we will separate all tunable parameters into different layers each optimized by independent Bayesian optimization routines. Then we will decide the maximum budget each layer can get with a bottom-up partition strategy. Besides search space and resource partition, we also employ a novel allocation algorithm that integrates successive halving~\cite{successivehalving} and a convergence-based early exiting strategy to facilitate efficient usage of assigned budget.}


% The outermost layer searches and selects structures for a workflow; the middle layer searches and selects step options under the workflow structure selected in the outermost layer; the innermost layer searches and selects weights with the given workflow structure and steps. 

\begin{algorithm}[h]
    \caption{\search\ Algorithm}
    \label{alg:main}
      \small
\begin{algorithmic}[1]
\STATE \textbf{Global Value:} $R = \emptyset$ \COMMENT{Global result set}
%\STATE \textbf{Global Value:} $F = \emptyset$ \COMMENT{Global observation set}

%Reduct factor $\eta > 1$, explore width $W$
\STATE \textbf{Input:} User-specified Total Budget $TB$
\STATE \textbf{Input:} Cog set $C = \{c_{11},c_{12},...\}, \{c_{21},c_{22},...\}, \{c_{31},c_{32},...\}$

    \STATE
%\FOR{$i = 1,2,3$}
    %\COMMENT{$\alpha$ is a configurable value default to 1.1}
%\ENDFOR
%\STATE
%    \STATE \{$B_1,B_2,B_3$\} = LayerPartition($C$) \COMMENT{Calculate ideal layer budget}
    %\STATE \textbf{Glob}.budgets = budgets
%    \STATE opt\_layers = init\_opt\_routines() \COMMENT{A list of optimize routine each layer will use for search}
%\STATE
%    \FOR{$i \in L, \dots, 1$}
%        \IF{$i == L$}
 %           \STATE opt\_layers[L] = InnerLayerOpt
  %      \ELSE
   %         \STATE opt\_layers[i] = OuterLayerOpt
            %\STATE opt\_layers[i].next\_layer\_budgets = B[i+1]
            %\STATE opt\_layers[i].next\_layer\_routine = opt\_layers[i+1]
    %    \ENDIF
    %\ENDFOR
%\STATE opt\_layers[1].invoke($\emptyset$, B[1])
\STATE $U = 0$ \COMMENT{Used budget so far, initialize to 0}

\STATE \COMMENT{Perform search with 1 to 3 layers until budget runs out}
\FOR{$L = 1,2,3$} 
        \IF{$L=1$}
            \STATE $C_1 = C_1 \cup C_2 \cup C_3$ \COMMENT{Merge all cogs into a single layer}
        \ENDIF
        \IF{$L==2$}
            \STATE $C_1 = C_1 \cup C_2$ \COMMENT{Merge step and weight cogs}
            \STATE $C_2 = C_3$ \COMMENT{Architecture cog becomes the second layer}
        \ENDIF
        \STATE
    \FOR{$i = 1,..,L$}
    \STATE $NC_i = |C_i|$ \COMMENT{Total number of cogs in layer $L$} 
%    NO_i &= \sum_{L} \{\text{number of possible options in cog } c_{ij}\} \\
    \STATE $S_i = NC_i^\alpha$ \COMMENT{Estimated expected search size in layer $i$}
    \ENDFOR
    \STATE $E_L = \prod\limits_{i=1}^{L}S_i$ \COMMENT{Expected total search size in the current round}
    \STATE $E = TB - U > E_L$ ? $E_L$ : $(TB - U)$ \COMMENT{Consider insufficient budget} 
    \IF{$L==3$ and $(TB - U)$ > $E_L$}
         \STATE $E = TB - U$ \COMMENT{Spend all remaining budget if at 3 layer}
    \ENDIF
    %\STATE$TL = |N|$ \COMMENT{number of layers}
    \FOR{$i = 1,..,L$}
        \STATE $B_i =  \lfloor S_i \times \sqrt[L]{\frac{E}{E_L}}\rfloor$
        %$B$ = BudgetAssign($N$, $TL$, $TB$)
        \COMMENT{Assign budget proportionally to $S_i$}
    \ENDFOR
    \STATE
\STATE \texttt{LayerSearch} ($\emptyset$, $B$, $L$, $B_L$) \COMMENT{Hierarchical search from layer $L$}
\STATE
\STATE $U = U + E$
\IF{$U \geq TB$}
\STATE break \COMMENT{Stop search when using up all user budget}
\ENDIF
\ENDFOR
%\STATE
%\STATE $O$ = \texttt{SelectBestConfigs} ($R$)
%\IF{$L == 1$}
%    \STATE InnerLayerOpt($\emptyset$, B[1])
%\ELSE
%    \STATE OuterLayerOpt($\emptyset$, B[1], 1)
%\ENDIF
\STATE
\STATE \textbf{Output:} $O$ = \texttt{SelectBestConfigs} ($R$) \COMMENT{Return best optimizations}
\end{algorithmic}
\end{algorithm}

\subsection{Hierarchical Layer and Budget Partition}
\label{sec:ssp}

%We motivate \search's adaptive hierarchical search 
A non-hierarchical search has all cog options in a single-layer search space for an optimizer like BO to search, an approach taken by prior workflow optimizers~\cite{dspy-2-2024,gptswarm}.
With small budgets, a single-layer hierarchy allows BO-like search to spend the budget on dimensions that could potentially generate some improvements.
%While given enough budget, the single-layer space can be extensively searched to find global optimal, with little budget, 
However, a major issue with a single-layer search space is that a search algorithm like BO can be stuck at a local optimum even when budgets increase.
% (unless the budget is close to covering a very large space across dimensions).
To mitigate this issue, our idea is to perform a hierarchical search that works by choosing configurations in the outermost layer first, then under each chosen configuration, choosing the next layer's configurations until the innermost layer. 
With such a hierarchy, a search algorithm could force each layer to sample some values. Given enough budget, each dimension will receive some sampling points, allowing better coverage in the entire search space. However, with high dimensionality (\ie, many types of cogs) and insufficient budget, a hierarchical search may not be able to perform enough local search to find any good optimizations.

To support different user-specified budgets and to get the best of both approaches, we propose an adaptive hierarchical search approach, as shown in Algorithm~\ref{alg:main}.
\search\ starts the search by combining all cogs into one layer ($L=1$, line 9 in Algorithm~\ref{alg:main}) and estimating the expected search budget of this single layer to be the total number of cogs to the power of $\alpha$ (lines 16-19, by default $\alpha = 1.1$). This budget is then passed to the \texttt{LayerSearch} function (Algorithm~\ref{alg:outer}) to perform the actual cog search. When the user-defined budget is no larger than this estimated budget, we expect the single-layer, non-hierarchical search to work better than hierarchical search.
%as the budget for this single layer.

If the user-defined budget is larger, \search\ continues the search with two layers ($L=2$), combining step and weight cogs into the inner layer and architecture cogs as the outer layer (lines 11-14).
\search\ estimates the total search budget for this round as the product of the number of cogs in each of the two layers to the power of $\alpha$ (lines 16-20). It then distributes the estimated search budget between the two layers proportionally to each layer's complexity (lines 22-24) and calls the upper layer's \texttt{LayerSearch} function. Afterward, if there is still budget left, \search\ performs a last round of search using three layers and the remaining budget in a similar way as described above but with three separate layers (architecture as the outermost, step as the middle, and weight cogs as the innermost layer). Two or three layers work better for larger user-defined budgets, as they allow for a larger coverage of the high-dimensional search space.

Finally, \search\ combines all the search results to select the best configurations based on user-defined metrics (line 34).

%\search\ organizes cogs by having architecture cogs in the outer-most search layer, step cogs in the middle layer, and weight cogs in the inner-most layer (line 4 in Algorithm~\ref{alg:main}).
%This is because step cogs' input and output format are dependent on the workflow structure, and the effectiveness of weights (\eg, prompting) are dependent on steps (\eg, LLM model). 

% increases the number of layers until hitting the user-specified total search budget, $TB$

%Thus, the first step of \search\ is to determine the number of layers in its hierarchy and what cogs to include in a layer.
%Intuitively, structure cogs should be placed in the outer-most search layer to be determined first before exploring other cogs. This is because other cogs change node and edge values, and it is easier for 
%However, instead of a fixed number of layers in the hierarchy, we adapt the cog layering according to user-specified total search budgets, $TB$, and the complexity of each layer, using Algorithm~\ref{alg:main}.

% the following \texttt{LayerPartition} method.
%We begin by modeling the relationship between the expected number of evaluations and the number of cogs as well as the number of options in each layer:

%We first consider the identity of each cog in the search space. All structure-cogs will be placed in the outer-most search layer exclusively, which is similar to non-differentiable NAS in traditional ML training. This layer will fix the workflow graph and pass it to the following layer, allowing a stabilized search space for faster convergence.

%Since step-cogs will not create a changing search space, the partition of step-cogs and weight-cogs is conditioned on the search space complexity and the given total budget. Separating step-cogs out can benefit from a more flexible budget allocation strategy and broader exploration for local search at weight-cogs but performs poorly when the given budget is more constrained, in that case, we will optimize them jointly in the same layer.


%\small
%\begin{align*}
%    C &= \{c_{11},c_{12},...\}, \{c_{21},c_{22},...\}, \{c_{31},c_{32},...\} \\
%    NC_i &= \text{total number of cogs in layer i} \\
%    NO_i &= \sum_{j} \{\text{number of possible options in cog } c_{ij}\} \\
%    N_i &= max(NC_i^\alpha,NO_i) \\
%    N_i &= \sum_{j} \{\text{number of possible options in } C_{ij}\} \\
%    N_i &= max(|C_i|^\alpha, N_i) \\
%    B_j &= \prod\limits_{i=1}^{j}N_i, j \in \{1,2,3\}
%\end{align*}

%\normalsize
%where $L$ represents the total number of layers and can be 1, 2, or 3. 
%$C$ represents the entire cog search space, with each row $c_{i*}$ being one of the three types of cogs and lower layers having lower-numbered rows (\eg, $c_{1*}$ being weight cogs). $NC_i$ is the number of cogs in layer $i$, and $NO_i$ is the total number of options across all cogs in layer $i$. $N_i$ is our estimation of the complexity of layer $i$ based on $NC_i$ and $NO_i$ ($\alpha$ is a configurable weight to control the importance between $NC_i$ and $NO_i$; by default $\alpha = 1.1$). 
%$\alpha$ stands for a control parameter, setting the intensity of this scaling behavior w.r.t the number of cogs, we found that $\alpha = 1.2$ is empirically sufficient and efficient for optimizing real workloads. 
%$B_j$ is the expected total number of workflow evaluations for all the lower $j$ layers.
%After calculating $B_1$, $B_2$, and $B_3$, we compare the total budget $TB$ with them.
%When $TB \geq B_3$, we set the total number of layers, $TL$, to 3. When $B_2 \leq TB < B_3$, we set the total number of layers to 2 and merge the step and weight cogs into one layer. When $TB < B_1$, we put all cogs in one layer.
%We only create a separate layer for step-cogs when the given budget $TB$ is greater or equal to the total expected budget for three layers.

%\subsection{Seach Budget Partition}
%\label{sec:sbp}
%After determining cog layers, we distribute the total budget, $TB$, across the layers proportionally to each layer's expected budget $N_i$: , which is the \texttt{BudgetAssign} function.
%We follow a bottom-up partition strategy, where lower layers will try to greedily take the expected budget. This stems from two simple heuristics: (1) feedback to the upper layer is more accurate when the succeeding layer is trained with enough iterations, and (2) the effectiveness of a structure change depends on the setting of individual steps in the workflow (\eg, majority voting is more powerful when each LLM-agent is embedded with diverse few-shot examples or reasoning styles). In cases where the given resource exceeds the total expected budget, 
%We assign $TB$ across layers proportionally to their expected budget $N_i$. 
%The budget assigned at each layer $B_i$ given the total available number of evaluations $TB$ is obtained as follows:

%\small
%\begin{align}
%B_i &=  \lfloor N_i \times \sqrt[L]{\frac{TB}{B^*}}\rfloor
%    B_L &= \begin{cases}
%        min(N_L, TB) & TB < B^* \\
%        \lfloor N_L \times \sqrt[L]{\frac{TB}{B^*}}\rfloor & TB \geq B^*
%    \end{cases}
%    \\
%    B_i &= \begin{cases}
%        min(N_i, \lfloor\frac{TB}{\prod_{j=i+1}^L B_j}\rfloor) & TB < B^* \\
%        \lfloor N_i \times \sqrt[L]{\frac{TB}{B^*}}\rfloor & TB \geq B^*
%    \end{cases}
%\end{align}

%\normalsize


\subsection{Recursive Layer-Wise Search Algorithm}
%The calculation above pre-assigns cogs to layers and search budgets to each layer. 
We now introduce how \search\ performs the actual search in a recursive manner until the inner-most layer is searched, as presented in Algorithm~\ref{alg:outer} \texttt{LayerSearch}. 
Our overall goal is to ensure strong cog option coverage within each layer while quickly directing budgets to more promising cog options based on evaluation results.
%So far, we have determined the optimization layer structure and the maximum allowed search iteration each layer will get. Next, we introduce how the budget is consumed in each layer. The inner-most layer, where weight-cogs, and potentially step-cogs, reside, follows the conventional Bayesian optimization process, exhausting all budgets unless an early stop signal is sent. This signal will be triggered when the current optimizer witnesses $p$ consecutive iterations without any improvements above the threshold. All optimization layers use early stopping to avoid budget waste.
%Algorithm~\ref{alg:inner} describes the search happening at the inner-most (bottom) layer, and 
Specifically, every layer's search is under a chosen set of cog configurations from its upper layers ($C_{chosen}$) and is given a budget $b$. 
In the inner-most layer (lines 7-20), \search\ samples $b$ configurations and evaluates the workflow for each of them together with the configurations from all upper layers ($C_{chosen}$). The evaluation results are added to the feedback set $F$ as the return of this layer.

\begin{algorithm}[h]
  %\algsetup{linenosize=\tiny}
  \small
    \caption{\texttt{LayerSearch} Function}
    \label{alg:outer}
\begin{algorithmic}[1]
%\STATE \textbf{Global Config:} Reduct factor $\eta > 1$, explore width $W$
\STATE \textbf{Global Value:} $R$ \COMMENT{Global result set}
%\STATE \textbf{Global Value:} $F$ \COMMENT{Global observation set}
\STATE \textbf{Input:} $C_{chosen}$: configs chosen in upper layers
\STATE \textbf{Input:} $B$: Array storing assigned budgets to different layers
\STATE \textbf{Input:} $curr\_layer$: this layer's level
\STATE \textbf{Input:} $curr\_b$: this layer's assigned budget
%\STATE
%\FUNCTION{LayerSearch\hspace{0.4em}($C_{chosen}$, $B$, $curr\_layer$, $curr\_b$)}

    \STATE
    \STATE \COMMENT{Search for inner-most layer}
    \IF{curr\_layer == 1}
        \STATE $F = \emptyset$ \COMMENT{Init this layer's feedback set to empty}
        %\STATE $F^{\prime} = match(C_{chosen}, F)$ \COMMENT{Local feedback set}
        \FOR{$k = 0, \dots, curr\_b$}
            \STATE $\lambda$ = \texttt{TPESample} (1) \COMMENT{Sample one configuration using TPE}
            \STATE $f = $ \texttt{EvaluateWorkflow} ($C_{chosen} \cup \lambda$)
            \STATE $R = R \cup \{C_{chosen} \cup \lambda\}$ \COMMENT{Add configuration to global $R$}
            \IF{\texttt{EarlyStop} (f)}
            \STATE break
            \ENDIF
            \STATE $F = F \cup \{f\}$ \COMMENT{Add evaluate result to feedback $F$}
        \ENDFOR
        %\STATE $F = F \cup F^{\prime}$
        \STATE \textbf{Return} $F$
    \ENDIF
    \STATE
    \STATE \COMMENT{Search for non-inner-most layer}
    %\STATE $K = \lfloor \frac{b}{W} \rfloor$, 
    \STATE $b\_used = 0$, $TF = \emptyset$ \COMMENT{Init this layer's used budget and feedback set}
    \STATE $R = \lceil\frac{curr\_b}{\eta}\rceil$, $S = \lfloor\frac{curr\_b}{R}\rfloor$ \COMMENT{Set $R$ and $S$ based on $curr\_b$}
    \STATE
    \WHILE{$b\_{used}$ $\leq$ $curr\_b$}
        \STATE \COMMENT{Sample $W$ configs at a time until running out of $curr\_b$}
        \STATE $n = (curr\_b - b_{used})$ > $W$ ? $W$ : $(curr\_b - b_{used})$
        %\IF{$b - b_{used} < W$}
        %    \STATE $n = b_l - b_{used}$
        %\ELSE
         %   \STATE $n=W$
        %\ENDIF
        \STATE $b\_used$ += $n$
        %\STATE $n = \text{min}(W,\ b_l - kW)$ \COMMENT{Propose $W$ configs and meet $b_l$ constraint}
        \STATE $\Theta = $ \texttt{TPESample} ($n$) \COMMENT{Sample a chunk of $n$ configs in the layer} 
        %\STATE $F^{\prime} = match(C_{chosen}, F)$ \COMMENT{Per-chunk feedback set}
        \STATE $F = \emptyset$ \COMMENT{Init this layer's feedback set to empty}
        \STATE
        \FOR{$s = 0, 1, \dots, S$}
            \STATE $r_s = R\cdot \eta^s$
            \FOR{$\theta \in \Theta$}
                %\IF{$curr\_layer < max\_layer$}
                    \STATE $f =$ \texttt{LayerSearch} ($C_{chosen} \cup \{\theta\}$, $B$, curr\_layer$-1$, $r_s$)
                %\ELSE
                %    \STATE $f =$ InnerOpt($\gamma \cup \{\theta\}$, $r_s$)
                %\STATE $f$ = $opt\_layers[current\_layer+1](\gamma \cup \{\theta\}, r_s)$ \{Optimize the current config at the next layer with $r_s$ budget \}
                %\ENDIF
                \STATE $F = F \cup f$ \COMMENT{Add evaluate result to feedback}
                \IF{\texttt{EarlyStop} ($f$)}
                    \STATE $\Theta = \Theta - \{\theta\}$ \COMMENT{Skip converged configs}
                \ENDIF
            \ENDFOR
            \STATE $\Theta$ = Select top $\lfloor \frac{|\Theta|}{\eta}\rfloor$ configs from $F$ based on user-specified metrics
        \ENDFOR
        \STATE
        \IF{\texttt{EarlyStop} ($F$)}
            \STATE break \COMMENT{Skip remaining search if results converged}
        \ENDIF
        \STATE $TF = TF \cup F$
    \ENDWHILE
    %\STATE $F = F \cup TF$
        \STATE \textbf{Return} $TF$

%\ENDFUNCTION

%\STATE \textbf{Output:} Best metrics in all trials
\end{algorithmic}
\end{algorithm}

% consumption\_nextlayer\_bucket = WSR

% for s in 0, 1,...S do
%     w = W*\eta^{s}
%     r = R*\eta^{-s}

% total budget at next layer = b_l / W * WSR = b_l * SR

% b_l * SR <= b_l * B_l+1

% S = B_{l+1} / R



For a non-inner-most layer, \search\ samples a chunk ($W$) of points at a time using the TPE BO algorithm~\cite{bergstra2011tpe} until all this layer's pre-assigned budget is exhausted (lines 27-30). Within a chunk, \search\ uses a successive-halving-like approach to iteratively direct the search budget to more promising configurations within the chunk (the dynamically changing set, $\Theta$). In each iteration, \search\ calls the next-level search function for each sampled configuration in $\Theta$ with a budget of $r_s$ and adds the evaluation observations from lower layers to the feedback set $F$ for later TPE sampling to use (lines 35-37).
In the first iteration ($s=0$), $r_s$ is set to $R\cdot \eta^0=R$ (line 34). After the inner layers use this budget to search, \search\ filters out configurations with lower performance and only keeps the top $\lfloor \frac{|\Theta|}{\eta}\rfloor$ configurations as the new $\Theta$ to explore in the next iteration (line 42). In each next iteration, \search\ increases $r_s$ by $\eta$ times (line 34), essentially giving more search budget to the better configurations from the previous iteration.

The successive halving method effectively distributes the search budget to more promising configurations, while the chunk-based sampling approach allows for evaluation feedback to accumulate quickly so that later rounds of TPE can get more feedback (compared to no chunking and sampling all $b$ configurations at the same time). To further improve the search efficiency, we adopt an {\em early stop} approach where we stop a chunk or a layer's search when we find its latest few searches do not improve workflow results by more than a threshold, indicating convergence (lines 14,38,45).

%algorithm takes as input other cog settings from previous layers and the assigned budget at the current layer. It tiles the search loop into fixed-size blocks (line 4), each runs the SuccessiveHalving (SH) subroutine in the inner loop (line 7-15). In each SH iteration, only top-$1/\eta$-quantile configurations in $\Theta$ will continue in the next round with $\eta$ times larger budget consumption. As a result, exponentially more trials will be performed by more promising configurations. 

%On average, \textit{Outer-layer search} will create $K$ brackets, each granting approximately $WRS$ budget to the next layer. $R$ represents the smallest amount of resource allocated to any configurations in $\Theta$. 

% layer - 1: budget = 4
% K * W <= b\_current layer
% layer -1: itear 0: propose 2

%     SH:
%     2 config -> R
%     1 config -> 2R

%     iteration 1: propose 2 = W
%     SH:
%     2 config -> R
%     1 config -> 2R

% W configs; each has R resource

% W / eta configs; each has R * eta resource

% R -> least resource one config can get = B2 - smth
% R + R*eta + ... + R*eta\^s -> most promising = B2 + smth


% $L2$ is the middle layer where structure-cogs and step-cogs may be placed exclusively. We employ hyperband for its robustness in exploration and exploitation trace-off. If this layer exists, it will instruct $L1$ the number of search iterations to run in each invocation. Specifically, in each iteration at line 4, \sysname will propose $n$ configurations and run SuccessiveHalving (SH) subroutine (line 8-15). SH will optimize each proposal and use the search results from $L1$ to rank their performance. Each time only the top-performing $n \cdot \eta^{-i}$ can continue in the next round with a larger budget. With this strategy, exponentially more search budgets are allocated to more promising configs at $L2$.

% \input{algo-l2-search}

% $L3$ is the outer-most layer for structure-cogs only when $L2$ is created. For this layer, we employ plain SH without hyperband because of its predictable convergence behavior. This is mainly due to two factors: (1) structure change to the workflow is more significant thus different configurations are more likely to deviate after training with the following layers. (2) with the search space partition strategy in Sec ~\ref{sec:ssp}, we can assume the available budget at each layer is substantial when $L3$ exists. Given this prior knowledge, we can avoid grid searching control parameter $n$ as in the hyperband but adopt a more aggressive allocation scheme to bias towards better proposals and moderate search wastes.



%\subsubsection{Runtime Budget Adaptation}
%Using static estimation of the expected budget for each layer is not enough, we also adjust the assignment during the optimization based on real convergence behavior. Specifically, for layer $i$, we record the number of configurations evaluated in each optimize routine. We set the convergence indicator $C_{ij}$ of $j^{th}$ routine with this number if the search early exits before reaching the budget limit, otherwise 2\x of its assigned resource. Then we update $E_i$ with $\frac{\sum_{j}^M C_{ij}}{M}$. \sysname\ will update the budget partition according to Sec~\ref{sec:sbp} for any newly spawned optimizer routines. Besides controlling the proportion of budgets across layers, a smaller/larger $B_{l+1}$ will also guide the SH in Alg~\ref{alg:outer} to shrink/extend the budget $R$ for differentiating config performance.


\section{\sysname\ Design}
\label{sec:cognify}

We build \sysname, an extensible gen-AI workflow autotuning platform based on the \search\ algorithm. The input to \sysname\ is the user-written gen-AI workflow (we currently support LangChain \cite{langchain-repo}, DSPy \cite{khattab2024dspy}, and our own programming model), a user-provided workflow training set, a user-chosen evaluator, and a user-specified total search budget. \sysname\ currently supports three autotuning objectives: generation quality (defined by the user evaluator), total workflow execution cost, and total workflow execution latency. Users can choose one or more of these objectives and set thresholds for them or the remaining metrics (\eg, optimize cost and latency while ensuring quality to be at least 5\% better than the original workflow). 
\sysname\ uses the \search\ algorithm to search through the cog space.
When given multiple optimization objectives, \sysname\ maintains a sorted optimization queue for each objective and performs its pruning and final result selection from all the sorted queues (possibly with different weighted numbers).
To speed up the search process, we employ parallel execution, where a user-configurable number of optimizers, each taking a chunk of search load, work together in parallel. %Below, we introduce each type of cogs in more details.
\sysname\ returns multiple autotuned workflow versions based on user-specified objectives.
\sysname\ also allows users to continue the auto-tuning from a previous optimization result with more budgets so that users can gradually increase their search budget without prior knowledge of what budget is sufficient.
Appendix~\ref{sec:apdx-example} shows an example of \sysname-tuned workflow outputs. 
\sysname\ currently supports six cogs in three categories, as discussed below. 

%In \sysname, we call every workflow optimization technique a {\em cog}, including structure-changing cogs like task decomposition, step-changing cogs like model selection, and weight-changing cogs like adding few-shot examples to prompts. 
%\sysname\ places structure-changing cogs in the outermost layer, step cogs in the middle layer, and weight cogs in the innermost layer, because \fixme{TODO}.


\subsection{Architecture Cogs}
\label{sec:structure-cog}
%Changing the structure of a workflow can potentially improve its generation quality (\eg, by using multiple steps to attempt at a task in parallel or in chain) or reduce its execution cost and latency (\eg, by merging or removing steps).
\sysname\ currently supports two architecture cogs: task decomposition and task ensemble.
Task decomposition~\cite{khot2023decomposed} breaks a workflow step into multiple sub-steps and can potentially improve generation quality and lower execution costs, as decomposed tasks are easier to solve even with a small (cheaper) model.
There are numerous ways to perform task decomposition in a workflow. 
%, as all LM steps can potentially be decomposed and into different numbers of sub-steps in different ways. Throwing all options to the Bayesian Optimizer would drastically increase the search space for \sysname. 
To reduce the search space, we propose several ways to narrow down task decomposition options. Even though we present these techniques in the setting of task decomposition, they generalize to many other structure-changing tuning techniques.

%First, we narrow down a selected set of steps in a workflow to decompose. 
Intuitively, complex tasks are the hardest to solve and worth decomposition the most. We use a combination of LLM-as-a-judge \cite{vicuna_share_gpt} and static graph (program) analysis to identify complex steps. We instruct an LLM to give a rating of the complexity of each step in a workflow. We then analyze the relationship between steps in a workflow and find the number of out-edges of each step (\ie, the number of subsequent steps getting this step's output). More out-edges imply that a step is likely performing more tasks at the same time and is thus more complex. We multiply the LLM-produced rating and the number of out-edges for each step and pick the modules with scores above a learnable threshold as the target for task decomposition. We then instruct an LLM to propose a decomposition (\ie, generate the submodules and their prompts) for each candidate step. %We provide the LLM with few-shot examples for what proposed modules for a separate task could look like. We also add a refinement step that validates whether the proposition decomposition maintains the semantics of the original trajectory. Once candidate decompositions are generated, those are used for the entirety of the optimization.

{
\begin{figure*}[t!]
\begin{center}
\centerline{\includegraphics[width=0.95\textwidth]{Figures/big_grid.pdf}}
\vspace{-0.1in}
\mycap{Generation Quality vs Cost/Latency.}{Dashed lines show the Pareto frontier (upper left is better). Cost shown as model API dollar cost for every 1000 requests. Cognify selects models
from GPT-4o-mini and Llama-8B. DSPy and Trace do not support model selection and are given GPT-4o-mini for all steps. Trace results for Text-2-SQL and FinRobot have 0 quality and are not included.} 
\Description{Eight graphs with different shapes representing baselines compared to points on a Pareto frontier.}
\label{fig-biggrid}
\end{center}
\end{figure*}
}


The second structure-changing cog that \sysname\ supports is task ensembling. This cog spawns multiple parallel steps (or samplers) for a single step in the original workflow, as well as an aggregator step that returns the best output (or combination of outputs). By introducing parallel steps, \sysname\ can optimize these independently with step and weight cogs. This provides the aggregator with a diverse set of outputs to choose from. 
%The aggregator is prompted with the role of the samplers, as well as the inputs to each. It also receives a criteria by which it should make a decision. We choose to prompt it with a qualitative description of how it should resolve discrepancies between outputs. 


\subsection{Step Cogs}
We currently support two step-changing cogs: model selection for language-model (LM) steps and code rewriting for code steps.

For model selection, to reduce its search space, we identify ``important'' LM steps---steps that most critically impact the final workflow output to reduce the set \search\ performs TPE sampling on. Our approach is to test each step in isolation by freezing other steps with the cheapest model and trying different models on the step under testing. 
We then calculate the difference between the model yielding the best and worst workflow results as the importance of the step under testing. %For each model, we get the workflow output quality score using sampled user-supplied inputs and user-specific evaluator. We then calculate the difference between the highest and lowest scores as this module's importance. 
After testing all the steps, we choose the steps with the highest K\% importance as the ones for TPE to sample from.
%, where K is determined based on user-chosen stop criteria. We then initialize the Bayesian optimization to start with the state where important modules use the largest model and all other modules use the cheapest model. We set the TPE optimization bandwidth of each module to be the inverse of importance, \ie, the more important a module is the more TPE spends on optimizing.

The second step cog \sysname\ supports is code rewriting, where it automatically changes code steps to use better implementation. To rewrite a code step, \sysname\ finds the $k$ worst- and best-performing training data points and feeds their corresponding input and output pairs of this code step to an LLM. We let the LLM propose $n$ new candidate code pieces for the step at a time.
%in parallel to generate a set of $n$ candidates.
In subsequent trials, the optimizer dynamically updates the candidate set using feedback from the evaluator.


\subsection{Weight Cogs}
\sysname\ currently supports two weight-changing cogs: reasoning and few-shot examples.
First, \sysname\ supports adding reasoning capability to the user's original prompt, with two options: zero-shot Chain-of-Thought \cite{wei2022chain} (\ie, ``think step-by-step...'') and dynamic planning \cite{huang2022language} (\ie, ``break down the task into simpler sub-tasks...''). These prompts are appended to the user's prompt. In the case where the original module relies on structured output, we support a reason-then-format option that injects reasoning text into the prompt while maintaining the original output schema.

Second, \sysname\ supports dynamically adding few-shot examples to a prompt. At the end of each iteration, we choose the top-$k$-performing examples for an LM step in the training data and use their corresponding input-output pairs of the LM step as the few-shot examples to be appended to the original prompt to the LM step for later iterations' TPE sampling. As such, the set of few-shot examples is constantly evolving during the optimization process based on the workflow's evaluation results. 
%Few-shot examples are available to all modules, even intermediate steps in the workflow. We use the full trajectory of each request to generate examples for the intermediate steps. Furthermore, we automatically filter out examples that do not meet a user-specified threshold. 



\section{Trusted Distributed Systems}
\label{sec:use_cases}

%\pramod{@Dimitra - please make a carefull pass over the text, please cite all related seminal papers, and also explain how TNIC helps in improving these systems explicitly. }

Using \projecttitle{}, we transform the following four distributed systems into BFT systems (see Appendix $\S$~\ref{sec:use_cases-appendix} for details).  
% in Appendix~\ref{sec:use_cases-appendix}.

% \antonis{Below we use \projecttitle{}-A2M, but further down, we do not! (i.e., no \projecttitle{}-BFT, \projecttitle{}-CR..)}

\myparagraph{Attested Append-Only Memory (A2M)} We design an Attested Append-Only Memory (A2M)~\cite{A2M} leveraging \projecttitle{}, which can be used to shield and optimize various systems~\cite{AbdElMalek2005FaultscalableBF, Castro:2002, Li2004, 10.5555/1298455.1298473}. The original A2M, and hence our implementation over \projecttitle{}, builds append-only (trusted) logs, associating each entry with a monotonically increasing sequence number to combat equivocation. While A2M has a large TCB and ports the log within the TEE, our implementation has only a minimal TCB in hardware and it can robustly store the log in the untrusted host memory, improving memory efficiency~\cite{levin2009trinc}.

As in the original A2M, we build the \texttt{append} and \texttt{lookup} operations. The \texttt{append} calls into  \projecttitle{} to generate an attestation for the log entry while associating it with a monotonically increased sequence number ({\tt sent\_cnt}). The sequence number denotes the entry's position in the log. The \texttt{lookup} operation retrieves entries locally without verification.
%, and the \texttt{truncate} operation deletes entries based on a provided sequence number. The append operation involves adding entries with a sequence number, context, and an authenticator field. %The lookup operation retrieves log entries without verification, while the truncate operation removes entries based on sequence numbers and utilizes a \textsc{manifest} log to keep track of state changes.


\myparagraph{Byzantine Fault Tolerance (BFT)} 
We design a Byzantine Fault-Tolerant protocol (BFT) using \projecttitle{}. The protocol builds a replicated counter as a foundational service for various systems~\cite{rafthyperledger, Kafka, boki, 10.1145/3286685.3286686, scalog}. Our system model considers a network of replicas with at most $f$ Byzantine replicas out of $N=2f+1$ total replicas. One replica serves as the leader, and the others act as followers. 
The system prevents equivocation through \projecttitle{}, which enforces and validates the ordering of messages. This reduces the number of replicas required and the message complexity compared to the classical BFT ($3f+1$).
% , thus cutting deployment costs and message complexity.

%The protocol operates in a partial synchrony model for liveness and assumes deterministic protocol specifications.

Clients send increment counter requests to the leader, who performs the requests and broadcasts the change along with a {\em proof of execution} (PoE) message to followers. The proof of execution is a \projecttitle{}-attested message with the original client's request, the leader's counter value, and metadata. The followers leverage their local state machine to detect a faulty leader (or follower)~\cite{268272}. Subsequently,
if and only if a follower has not applied the message before, it applies the incremented counter value to its state machine before forwarding its own PoE message to all other replicas and replying to the client.
% , who will also validate the followers' outputs. 
A quorum of at least $f+1$ identical messages from different replicas guarantees a correctly committed result for the client. %Overall, \projecttitle{} optimizes the replication factor and message complexity of BFT.

\myparagraph{Chain Replication (CR)} 
We design a Byzantine CR system~\cite{10.1007/978-3-642-35476-2_24} using \projecttitle{} as the replication layer of a Key-Value store. As in the CFT version of CR, the replicas, e.g., head, middle, and tail, are connected in a chained fashion. 
%We assume a centralized configuration service for error detection and reconfiguration, which always provides clients with a correct configuration.
%The system model assumes Byzantine fault tolerance with a centralized configuration service for error detection and reconfiguration. The head triggers reconfiguration if it intentionally fails to forward messages.

Clients execute requests by forwarding them to the head. The head orders and executes the request, creating his own {\em proof of execution message} (PoE), which is sent along the chain. The PoE consists of the original request and the head's output that \projecttitle{} attests. Each node in the chain verifies the previous node's PoE, executes the request, and creates its own PoE, which consists of the last PoE and the node's output. 

%Unlike the CFT CR assuming that cryptographic operations on the CPU are not compromised, local operations in the tail cannot be trusted in the BFT model.
\rev{D5}{Unlike the CFT CR, local operations in the tail (e.g., reads) are untrusted in the BFT model. Therefore, all operations must traverse the entire chain. Replicas reply to clients with their output after forwarding their PoE message, and clients wait for identical replies from all chained nodes. We discuss the performance-security trade-offs of an alternative TEE-based design of porting the entire CR protocol into the TEE (that would allow clients to read only from the tail) in $\S$~\ref{subsec:use_cases_eval}.} %While such a system , it targets a weaker threat model compared to \projecttitle{}.}
%Such a system would adhere to the protocol specification, with clients only needing to communicate with the tail. However, this design would target a weaker threat model compared to \projecttitle{}.
% Unlike the CFT CR, all operations must traverse the chain as local operations in the tail cannot be trusted. Replicas reply to clients with their output after forwarding their PoE message. Clients wait for identical replies from all chained nodes.
% % For \texttt{get} requests, clients traverse the chain or consult the majority, broadcasting the request to $f+1$ replicas, including the tail.

% \rev{D5}{We base our protocol implementation on~\cite{10.1007/978-3-642-35476-2_24}, where operations must traverse the entire chain, similar to Chain Replication for the Crash Fault Model. While \cite{10.1007/978-3-642-35476-2_24} assumes that cryptographic operations on the CPU are not compromised, we have implemented the system practically using \projecttitle{}. A hypothetical TEE-based design would involve porting the entire Chain Replication protocol into the TEE. Such a system would adhere to the protocol specification, with clients only needing to communicate with the tail, as in CFT Chain Replication. However, this hypothetical design would target a weaker threat model compared to \projecttitle{}.}

\myparagraph{Accountability (PeerReview)}
Lastly, we design an accountability system with \projecttitle{} based on the PeerReview system~\cite{peer-review} to {\em detect} malicious actions in large deployments~\cite{nfs, 10.1145/1218063.1217950}.  We detect faults impacting the system's network messages logged into the participant's tamper-evident log. We frame the protocol within an overlay multicast protocol for streaming systems where the nodes are organized in a tree topology. Witnesses assigned to each node audit the node's log to detect faults or non-responsive nodes. The witnesses replay the log entries, comparing them with a reference deterministic implementation to identify inconsistencies. 
Our \projecttitle{} prevents equivocation at NIC hardware efficiently, which eliminates the expensive all-to-all communication of the original PeerReview that does not use trusted hardware~\cite{levin2009trinc}.
% Moreover, \projecttitle{} optimizes accountability by efficiently handling equivocation.~\cite{trinc}



\begin{figure*}[t!]
%\begin{center}
\minipage{0.33\textwidth}
%\begin{figure}
    \centering
     \vspace{-3mm}
    \includegraphics[width=\linewidth]{atc-submission-plots/hw_eval_attest_latency.pdf} 
  \caption{{\tt Attest} function latency.}
  \label{fig:attest_kernel}
%\end{figure}
\endminipage
\minipage{0.33\textwidth}
%\begin{figure}
    \centering
     \vspace{-3mm}
  \includegraphics[width=\linewidth]{atc-submission-plots/latency_breakdown.pdf}
  \caption{{\tt Attest} latency breakdown.}\label{fig:latency_breakdown}
%\end{figure}
\endminipage
\minipage{0.33\textwidth}
%\begin{figure}
    \centering
  \includegraphics[width=\linewidth]{atc-submission-plots/foo100.pdf}
    \vspace{-7mm}
  \caption{Latency over time (SGX).}\label{fig:latency_distribution}
%\end{figure}
\endminipage
\vspace{-3mm}
%\end{center}
\end{figure*}

%\chapter{Implementation}{\label{ch:implementation}}
In this chapter, we present the implementation of the final product. We start by discussing how the four steps introduced in \hyperref[ch:high_level_approach]{chapter \ref*{ch:high_level_approach}} are integrated. We then outline the main system components of our score follower, presenting each as an independent, self-contained module. We then combine this into an overall system architecture and finally introduce the open-source score renderer used to display the score and evaluate the score follower.       

% \section{Aims and Requirements}
% The overall aim of the score follower was to 


\section{Score Follower Framework Details}
Our score follower conforms to the high-level framework presented in \hyperref[section:score_follower_framework]{section \ref*{section:score_follower_framework}}. In step 1, two score features are extracted from a MIDI file (see \hyperref[subsection:midi]{subsection \ref*{subsection:midi}}), namely MIDI note numbers\footnote{\href{https://inspiredacoustics.com/en/MIDI_note_numbers_and_center_frequencies}{https://inspiredacoustics.com/en/MIDI\_note\_numbers\_and\_center\_frequencies}} (corresponding to pitch) and note onsets (corresponding to duration). In step 2, the audio is streamed (whether from a file or into a microphone) and audioframes that exceed some predefined energy threshold are extracted. Here, audioframes are groups of contiguous audio samples, whose length can be specified by the argument \verb|frame_length|, usually between 800 and 2000 samples. The period between consecutive audioframes can also be defined by the argument \verb|hop_length|, typically between 2000 and 5000 audio samples. In step 3, score following is performed via a `Windowed' Viterbi algorithm (see  \hyperref[subsection:adjusting_viterbi]{subsection \ref*{subsection:adjusting_viterbi}}) which uses the Gaussian Process (GP) log marginal likelihoods (LMLs) for emission probabilities (see \hyperref[section:state_duration_model]{section \ref*{section:state_duration_model}}) and a state duration model for transition probabilities (see \hyperref[section:state_duration_model]{section \ref*{section:state_duration_model}}). Finally, in step 4 we render our results using an adapted version of the open source user interface, \textit{Flippy Qualitative Testbench}.

\section{Following Modes}
Two modes are available to the user: Pre-recorded Mode and Live Mode. The former requires a pre-recorded $\verb|.wav|$ file, whereas the latter takes an input stream of audio via the device's microphone. Note that both modes are still forms of score following, as opposed to score alignment, since in each mode we receive audioframes at the sampling rate, not all at once.\\

Live Mode offers a practical example of a score follower, displaying a score and position marker which a musician can read off while playing. However, this mode is not suitable for evaluation because the input and results cannot be easily replicated. Even ignoring repeatability, Live Mode is not suitable for one-off testing since a musician using this application may be influenced by the movement of the marker. For instance, the performer may speed up if the score follower `gets ahead' or slow down if the position marker lags or `gets lost'. To avoid this, we use Pre-recorded Mode when evaluating the performance of our score follower. Furthermore, Pre-recorded Mode offers the advantage of testing away from the music room, providing the opportunity to evaluate a variety of recordings available online. 

\section{System Architecture}
Our guiding principle for development was to build modular code in order to create a streamlined system where each component performs a specific task independently. This structure facilitates easy testing and debugging. \hyperref[fig:black_box]{Figure \ref*{fig:black_box}} presents a high-level architecture diagram, where each black box abstracts a key component of the score follower. When operating in Pre-recorded Mode, there is the option to stream the recording during run-time, which outputs to the device's speakers (as indicated by the dashed lines).

\begin{figure}[H]
    \centering
    \includegraphics[width=1\textwidth]{figs/Part_4_Implementation_And_Results/black_box.png}
    \caption{Abstracted system architecture diagram displaying inputs in grey, the 4 main components of the score follower in black and the outputs in green.}
    \label{fig:black_box}
\end{figure}

\subsection{Score Preprocessor}
The architecture for the Score Preprocesor is given in \hyperref[fig:score_preprocessor]{Figure \ref*{fig:score_preprocessor}}. First, MIDI note number and note onset times are extracted from each MIDI event. Simultaneous notes can be gathered into states and returned as a time-sorted list of lists called \verb|score|, where each element of the outer list is a list of simultaneous note onsets. Similarly, a list of note durations calculated as the time difference between consecutive states is returned as \verb|times_to_next|. Finally, all covariance matrices are precalculated and stored in a dictionary, where the key of the dictionary is determined by the notes present. This is because the distribution of notes and chords in a score is not random: notes tend to belong to a home \gls{key} and melodies tend to be repeated or related (similar to subject fields in speech processing). Therefore, states tend to be reused often, allowing us to achieve amortised time and space savings (by avoiding repeated calculation of the same covariance matrices). 

\begin{figure}[H]
    \centering
    \includegraphics[width=1\textwidth]{figs/Part_3_Implementation/Stage_2_Alignment/score_preprocessor.png}
    \caption{System architecture diagram representing the Score Preprocessor with inputs in grey, processes in blue and objects in yellow.}
    \label{fig:score_preprocessor}
\end{figure}


\subsection{Audio Preprocessor}
The architecture for the Audio Preprocessor is illustrated in \hyperref[fig:audio_preprocessor]{Figure \ref*{fig:audio_preprocessor}}. In Pre-recorded Mode, the Slicer receives a $\verb|.wav|$ file and returns audioframes separated by the \verb|hop_length|. These audioframes are periodically added to a multiprocessing queue, \verb|AudioFramesQueue|, to simulate real-time score following. In Live Mode, we use the python module \verb|sounddevice| to receive a stream of audio, using a periodic callback function to place audioframes on \verb|AudioFramesQueue|. 

\begin{figure}[H]
    \centering
    \includegraphics[width=1\textwidth]{figs/Part_4_Implementation_And_Results/audio_preprocessor.png}
    \caption{System architecture diagram representing the Audio Preprocessor with inputs in grey, processes in blue and objects in yellow.}
    \label{fig:audio_preprocessor}
\end{figure}

\subsection{Follower and Backend}
The joint Follower and Backend architecture diagram is shown in \hyperref[fig:follwer_and_backend]{Figure \ref*{fig:follwer_and_backend}}. The Viterbi Follower (detailed in \hyperref[subsection:adjusting_viterbi]{section \ref*{subsection:adjusting_viterbi}}) calculates the most probable state in the score, given audioframes continually taken from \verb|AudioFramesQueue|. These states are placed on another multiprocessing queue, the \verb|FollowerOutputQueue|, for the Backend to process and send. This prevents any bottle-necking occurring at the Follower stage. The Backend first sets up a UDP connection and then reads off values from \verb|FollowerOutputQueue|, sending them via UDP packets to the score renderer.

\begin{figure}[H]
    \centering
    \includegraphics[width=1\textwidth]{figs/Part_4_Implementation_And_Results/follower_and_backend.png}
    \caption{System architecture diagram representing the Follower and Backend processes with processes in blue, objects in yellow and outputs in green.}
    \label{fig:follwer_and_backend}
\end{figure}

\subsection{Player}
In Pre-recorded Mode, the Player sets up a new process and begins streaming the recording once the Follower process begins. This provides a baseline for testing purposes, as a trained musician can observe the score position marker and judge how well it matches the music. 

\subsection{Overall System Architecture}
The overall system architecture is presented in \hyperref[fig:overall_system_architecture]{Figure \ref*{fig:overall_system_architecture}}. Since the Follower runs a real-time, time sensitive process, parallelism is employed to reduce the total system latency. We use two \verb|multiprocessing| queues\footnote{\href{https://docs.python.org/3/library/multiprocessing.html}{https://docs.python.org/3/library/multiprocessing.html}} to avoid bottle-necking, which allows us to run 4 concurrent processes (Audio Preprocessor, Follower, Backend, and Audio Player). Hence, this architecture allows the components to run independently of one another to avoid blocking. Furthermore, this allows the system to take advantage of the multiple cores and high computational power offered by most modern machines.  

\begin{figure}[H]
    \centering
    \includegraphics[width=1\textwidth]{figs/Part_4_Implementation_And_Results/overall_score_follower_2.png}
    \caption{System architecture diagram representing the overall score follower running in Pre-recorded mode, with inputs in grey, processes in blue, objects in yellow and outputs in green.}
    \label{fig:overall_system_architecture}
\end{figure}


\section{Rendering Results}{\label{section:renderer}}
To visualise the results of our score follower, we adapted an open source tool for testing different score followers.\footnote{\href{https://github.com/flippy-fyp/flippy-qualitative-testbench/blob/main/README.md}{https://github.com/flippy-fyp/flippy-qualitative-testbench/blob/main/README.md}} \hyperref[fig:flippy_example]{Figure \ref*{fig:flippy_example}} shows the user interface of the score position renderer, where the green bar indicates score position. 

\begin{figure}[H]
    \centering
    \includegraphics{figs/Part_4_Implementation_And_Results/example_renderer.png}
    \caption{Screenshot of the score renderer user interface which displays a score (here we show a keyboard arrangement of \textit{O Haupt voll Blut und Wunden} by Bach). The green marker represents the score follower position.}
    \label{fig:flippy_example}
\end{figure}



%\section{Conclusion}
In this work, we propose a simple yet effective approach, called SMILE, for graph few-shot learning with fewer tasks. Specifically, we introduce a novel dual-level mixup strategy, including within-task and across-task mixup, for enriching the diversity of nodes within each task and the diversity of tasks. Also, we incorporate the degree-based prior information to learn expressive node embeddings. Theoretically, we prove that SMILE effectively enhances the model's generalization performance. Empirically, we conduct extensive experiments on multiple benchmarks and the results suggest that SMILE significantly outperforms other baselines, including both in-domain and cross-domain few-shot settings.



\begin{table}
\begin{center}
\footnotesize
\begin{tabular}{ |c|c|c| } 
 \hline
 System &  (host) TEE-free & Tamper-proof \\ [0.5ex] \hline \hline
 SSL-lib & Yes & No\\
 SSL-server/Intel-x86*/AMD  &  Yes & No\\
 SGX/AMD-sev &  No & Yes\\
% TEE-P [Hybster, Damysus, Trinc] & Yes & Yes & Yes & Assumes trusted local persistent state\\
 %TEE-DS & Yes & Yes & Yes & Builds a DS of $f+2u+1$ TEEs\footnote{$f$ is the compromised TEEs and $u$ is the number of unresponsive TEEs}\\
 \projecttitle{} & Yes & Yes\\
 \hline
\end{tabular}
\end{center}
%\vspace{-10pt}
\caption{Host-sided baselines and \projecttitle{}. (*) We use the term SSL-server for this run unless stated otherwise.}
% \caption{(Trusted) Host-sided hardware baselines and \projecttitle{}. (*) We also use the term SSL-server for this run unless stated otherwise.}
\label{tab:hw_options}
\vspace{-8pt}
\end{table}

\section{Evaluation}
\label{sec:eval}

We evaluate \projecttitle{} across three dimensions: {\em (i)} 
hardware (\S~\ref{subsec:hw_eval}), {\em (ii)} network stack (\S~\ref{subsec:net_lib}) and {\em (iii)} distributed systems (\S~\ref{subsec:use_cases_eval}).

% \if 0
% \myparagraph{Implementation}
% We implement our prototype of \projecttitle{} extending the Coyote~\cite{coyote} codebase on Alveo U280~\cite{u280_smartnics}. We build the attestation kernel based on the HMAC module from the Vitis Security Library~\cite{vitis-security-lib} with SHA-384 as the hashing algorithm. For the data transfers, \projecttitle{} builds on top of an XDMA IP~\cite{xdma, fpga_dma} that enables DMA over PCIe. The 100Gb MAC is implemented with a CMAC IP~\cite{license} that exposes two 512-bit AXI4-Stream interfaces to the RoCE protocol kernel for the transmitting (Tx) and receiving (Rx) network paths.


% Our implementation is based on~\cite{coyote}, a fork of which has been used in prior works~\cite{strom} showing that 500 queue pairs (QPs) occupy 9\% of the on-chip memory, while the logic resource usage remains below 1\% when scaling from 500 to 16,000 QPs. Our evaluation and modern deployments use more powerful FPGAs, suggesting that even a larger number of connections could be supported compared to the work in~\cite{strom}. 
% Nevertheless, our \projecttitle{} does not assume a specific FPGA board. Therefore, the findings from previous works on other boards are still relevant.

% \fi 
%Precisely, we extend the data path of Coyote, adopting the abstraction of {\em virtual} FPGA areas to plug in our attestation functionality. \projecttitle{} leverages the Coyote paradigm and wires the control domain and the data domain into different paths.





%The data flow from the host memory to the device memory using the Xilinx streaming protocol (i.e., \texttt{hls\_stream<>}) that divides data into 512-bit packets. 
%At the transmission path, we implemented and plugged in two FIFO queues that redirect the packets from the host memory to the device memory -- specifically into the HMAC module and the RoCE IP. The role of the first queue is to buffer data and calculate the HMAC of all packets of the transmission.
%The module is based on the HMAC module in the Xilinx Vitis Security Library using SHA-384 as the hash function~\cite{vitis-security-lib}.
%The data of the second queue is passed through to the RoCE IP until the last packet of the transmission is reached.
%When the HMAC has been calculated, the last packet of the transmission is replaced by the 384-bit HMAC padded to 512 bits.

%At the reception path, we also augment the stack with two queues.
%The first queue sends the data to the verification module that will calculate the {\em expected} HMAC for verification. If the verification process succeeds we reconstruct the received message and deliver it to the application layer (host memory). Otherwise, all corresponding packets are dropped which naturally affects liveness.

\if 0
\myparagraph{Implementation}
We implement \projecttitle{} extending Coyote~\cite{coyote} codebase. Precisely, we extend the data path of Coyote, adopting the abstraction of {\em virtual} FPGA areas to plug in our attestation functionality. \projecttitle{} leverages the Coyote paradigm and wires the control domain and the data domain into different paths.

The RoCE kernel input and output data cables, e.g., \texttt{axis\_rdma\_sink} and \texttt{axis\_host\_sink}, are connected through the Attestation kernel through 64B data paths (512-bit AXI4-Stream interfaces). Then, the attestation kernel and the host memory communication are achieved through an AXI4 memory-mapped Master interface using the Xilinx streaming protocol. It is the responsibility of the host code to allocate and initialize host memory. Further, the kernel is connected to a 32B bus to receive the parameters, a 20 B bus to issue RDMA write operations, and a 12 B bus to issue local DMA commands.

%The data flow from the host memory to the device memory using the Xilinx streaming protocol (i.e., \texttt{hls\_stream<>}) that divides data into 512-bit packets. 
At the transmission path, we implemented and plugged in two FIFO queues that redirect the packets from the host memory to the device memory -- specifically into the HMAC module and the RoCE IP. The role of the first queue is to buffer data and calculate the HMAC of all packets of the transmission.
The module is based on the HMAC module in the Xilinx Vitis Security Library using SHA-384 as the hash function~\cite{vitis-security-lib}.
The data of the second queue is passed through to the RoCE IP until the last packet of the transmission is reached.
When the HMAC has been calculated, the last packet of the transmission is replaced by the 384-bit HMAC padded to 512 bits.

At the reception path, we also augment the stack with two queues.
The first queue sends the data to the verification module that will calculate the {\em expected} HMAC for verification. If the verification process succeeds we reconstruct the received message and deliver it to the application layer (host memory). Otherwise, all corresponding packets are dropped which naturally affects liveness.
\fi 

\myparagraph{Evaluation setup}
We perform our experiments on a real hardware testbed using two clusters: AMD-FPGA Cluster and Intel Cluster. AMD-FPGA Cluster consists of two machines powered by AMD EPYC 7413 (24 cores, 1.5 GHz) and 251.74 GiB memory. Each machine also has two Alveo U280 cards~\cite{u280_smartnics} that are connected through 100 Gbps QSFP28 ports. Intel Cluster consists of three machines powered by Intel(R) Core(TM) i9-9900K (8 cores, 3.2 GHz) with 64 GiB memory and three Intel Corporation Ethernet Controllers (XL710).





%, caches: 32 KiB (L1 data and code), 256 KiB (L2) and 16 MiB (L3). Cluster 2 nodes are connected over a 40GbE QSFP+ network switch.


%, caches: 32 KiB (L1 data and code), 512 KiB (L2) and 32 MiB (L3). Each machine also has two U280 Alveo cards~\cite{u280_smartnics} that are connected over a 100 Gbps cable.
%\atsushi{Can we simplify the above like this: Cluster 1 is equipped with two machines powered by AMD EPYC 7413 (24 cores, 1.5 GHz) and 251.74 GiB memory.}

%Cluster 2 is equipped with three machines with CPU (marked as CPU-2 in the respective plots): Intel(R) Core(TM) i9-9900K (8 cores, 3.2 GHz), memory: 64 GiB, caches: 32 KiB (L1 data and code), 256 KiB (L2) and 16 MiB (L3). Cluster 2 nodes are connected over a 40GbE QSFP+ network switch.
%\antonis{CPU-1 and CPU-2 are not memorable names, consider AMD, x86?} %\atsushi{I agree with Antonis. Also, can we rename Cluster 1 and 2 like this: AMD-FPGA Cluster and Intel Cluster?} \atsushi{Cluster 2 is equipped with three machines powered by Intel(R) Core(TM) i9-9900K (8 cores, 3.2 GHz) with 64GiB memory.}

%We perform our experiments on a real hardware testbed using two clusters: Cluster 1 and Cluster 2. Cluster 1 is equipped with two machines with CPU (marked as CPU-1 in the respective plots): AMD EPYC 7413 (24 cores, 1.5 GHz), memory: 251.74 GiB, caches: 32 KiB (L1 data and code), 512 KiB (L2) and 32 MiB (L3). Each machine also has two U280 Alveo cards~\cite{u280_smartnics} that are connected over a 100 Gbps cable.
%\atsushi{Can we simplify the above like this: Cluster 1 is equipped with two machines powered by AMD EPYC 7413 (24 cores, 1.5 GHz) and 251.74 GiB memory.}

%Cluster 2 is equipped with three machines with CPU (marked as CPU-2 in the respective plots): Intel(R) Core(TM) i9-9900K (8 cores, 3.2 GHz), memory: 64 GiB, caches: 32 KiB (L1 data and code), 256 KiB (L2) and 16 MiB (L3). Cluster 2 nodes are connected over a 40GbE QSFP+ network switch.
%\antonis{CPU-1 and CPU-2 are not memorable names, consider AMD, x86?} %\atsushi{I agree with Antonis. Also, can we rename Cluster 1 and 2 like this: AMD-FPGA Cluster and Intel Cluster?} \atsushi{Cluster 2 is equipped with three machines powered by Intel(R) Core(TM) i9-9900K (8 cores, 3.2 GHz) with 64GiB memory.}



%We evaluate the performance of \projecttitle{} across three dimensions: {\em (i)} the hardware evaluation for the Attestation kernel performance (\S~\ref{subsec:hw_eval}), {\em (ii)} the network library evaluation which assesses and compares the \projecttitle{} with competitive network stack baselines (\S~\ref{subsec:net_lib}) and {\em (iii)} our four secure primitive-examples constructed on top of \projecttitle{} (\S~\ref{subsec:use_cases_eval}). 
% We evaluate the performance of \projecttitle{} across three dimensions: {\em (i)} the hardware evaluation which focuses on the Attestation kernel evaluation (\S~\ref{subsec:hw_eval}), {\em (ii)} the network library evaluation which assesses and compares the \projecttitle{} with competitive network stack baselines (\S~\ref{subsec:net_lib}) and {\em (iii)} different use-cases we built to evaluate our four secure primitive-examples we constructed on top of \projecttitle{} (\S~\ref{subsec:use_cases_eval}). 


\subsection{Hardware Evaluation: \trustedfpga{}}
\label{subsec:hw_eval}




\myparagraph{Baselines} 
We evaluate the performance of {\tt Attest()} of the \projecttitle{}'s attestation kernel  ($\S$~\ref{subsec:nic_attest_kernel}) compared with four host-sided systems shown in Table~\ref{tab:hw_options}. For these host-sided versions, we build OpenSSL \rev{C3}{v3.1} servers that run natively or within a TEE \rev{C3}{with the same BIOS configuration (AES-NI enabled)}. The servers attest and forward network messages to the host application. We use the terms Intel-x86 and AMD for a native run of the server process (SSL-server) and SGX and AMD-sev for their tamper-proof versions within a TEE. 
\rev{B5}{The TEE baselines follow the same system model as in state-of-the-art hybrid systems~\cite{10.1145/3492321.3519568, minBFT, 10.1145/2168836.2168866, levin2009trinc}, where the host BFT application runs on the untrusted CPU and communicates with a separate TEE-based process to generate and verify message attestations.}
\rev{D6}{\projecttitle{} implements similar abstractions for counter and message attestation. Thus, \projecttitle{} does not introduce additional protocol alterations compared to them.}%as those used in the hybrid systems.
% (SGX, AMD-sev)
The server and host process run in the same machine to eliminate network latency and communicate through TCP sockets. We implement SGX using the {\sc scone} framework~\cite{scone} while AMD-sev runs in a QEMU VM using the official VM image~\cite{AMDSEV}. In addition, we build (non-temper-proof) SSL-lib, which integrates the {\tt Attest} function as a library. 
% : two untrusted native systems (Intel-x86 and AMD) and two TEE-based systems (SGX and AMD-sev). 
% Recall that both SSL-lib and SSL-server are not tamper-proof. 
% Table~\ref{tab:hw_options} summarizes our baselines with \projecttitle{}. 
%To avoid distortion, our plots do not include Trinc~\cite{levin2009trinc} because its reported latencies are an order of magnitude slower (85-105 ms) than all other evaluated solutions.

% \rev{B5}{We use the same system model as in state-of-the-art hybrid systems~\cite{10.1145/3492321.3519568,minBFT}, where the BFT application code runs on the (untrusted) CPU and communicates with a separate TEE-based process to generate and verify message attestations. Avocado~\cite{avocado}, which constructs a trusted computing base (TCB) encompassing both the protocol code and the attestation/verification layer, targets a weaker threat model than \projecttitle{}. Importantly, Avocado assumes that all participating TEEs adhere to the protocol as specified, whereas \projecttitle{} can tolerate up to $f$ nodes that deviate from the protocol.}

% We evaluate the performance of {\tt Attest()} of the \projecttitle{}'s attestation kernel  ($\S$~\ref{subsec:nic_attest_kernel}) compared with four host-sided systems shown in Table~\ref{tab:hw_options}. For these host-sided versions, we build OpenSSL servers that run natively or within a TEE. The servers attest and forward network messages to the host application. We use the terms Intel-x86 and AMD for a native run of the server process (SSL-server) and SGX and AMD-sev for their tamper-proof versions within a TEE. The server and host process run in the same machine to eliminate network latency and communicate through TCP sockets. We implement SGX using the {\sc scone} framework~\cite{scone} while AMD-sev runs in a QEMU VM using the official VM image~\cite{AMDSEV}. In addition, we build (non-temper-proof) SSL-lib, which integrates the {\tt Attest} function as a library. 

%(Table~\ref{tab:hw_options})
%Nat-lib is an OpenSSL-based library integrated into the code logic that generates and verifies messages. Nat is an OpenSSL-based server that communicates through kernel TCP sockets with the interested process to attest and verify its messages. \atsushi{can we shorten the description of Nat and Nat-lib?} Similarly, SGX and AMD-sev are OpenSSL-based servers that run within a tamper-proof TEE---specifically, Intel SGXv1~\cite{cryptoeprint:2016:086} and AMD-sev~\cite{amd-sev} (configured without offering confidentiality), respectively.  We implement the SGX-based server using the {\tt SCONE} framework~\cite{scone} to execute (exit-less) syscalls in a performant fashion. We run AMD-sev as a QEMU VM using the official supported kernel image~\cite{AMDSEV}, which efficiently runs un-modified Linux applications. Experiments with CPU-1, CPU-2, and SGX ran on Cluster 2, whereas we ran AMD-sev experiments on Cluster 1 \atsushi{CPU-1 should also be on Cluster 1, right?}. 
%On purpose, we exclude  Trinc~\cite{levin2009trinc} because it reported latencies that are an order of magnitude slower (85-105 ms) than \projecttitle{}. 
%To avoid distortion, our plots do not include 
%Trinc~\cite{levin2009trinc} because its reported latencies are an order of magnitude slower (85-105 ms) than all other evaluated solutions.
% \projecttitle{}

%\antonis{Differences between Nat and Nat-lib are not clear. Also, names could be more descriptive/memorable (Nat-lib is also very similar to NET-LIB).}

%\atsushi{can we remove this paragraph? (due to the space)} In short, we seek to answer the following research questions:
%\begin{itemize}
%\item {\bf{RQ1.}} What is the performance for generating and verifying attested messages in \projecttitle{}?
%\item {\bf{RQ2.}} What is the performance breakdown and latencies?
%    \item[\bf{RQ3}]  How much resources \projecttitle{} uses for the attestation kernel (Table~\ref{table:resources_usage})?
 %   \item[\bf{RQ4}] What is the hardware network latency and throughput (Fig~\ref{fig:hw_lat_breakdown})?
%\end{itemize}

\myparagraph{Methodology and experiments}
We use Vitis XRT v2022.2 and the shell \texttt{xilinx\_u280\_gen3x16\_xdma\_base\_1} for the stand-alone evaluation of the \projecttitle{} attestation kernel: synchronous data transfers between the host and device (U280). We measure and report the average latency and communication costs by executing an empty function body of \texttt{Attest()}.
%To isolate the latencies between data transfers and computation we further execute the same experiment without computing the HMAC (empty kernel). 

% We use Vitis XRT v2022.2 for the stand-alone evaluation of \projecttitle{} attestation kernel. We load the shell \texttt{xilinx\_u280\_gen3x16\_xdma\_base\_1} and use Vitis XRT synchronous data transfers from host to device and vice versa. We measure and report the {\tt Attest} function average latency as well as the communication costs executing an empty funcation body.

%We use Vitis XRT v2022.2 for the stand-alone evaluation of \projecttitle{} attestation kernel on top of which we build the host and the \trustedfpga{} processes (\texttt{xilinx\_u280\_gen3x16\_xdma\_base\_1}). We use Vitis XRT synchronous data transfers from host to device and vice versa. We measure and report the HMAC average latency on these competitive systems. To isolate the latencies between data transfers and computation we further execute the same experiment without computing the HMAC (empty kernel). \atsushi{I wanna simplify the texts here... work on it later}


\myparagraph{Results}
Figure~\ref{fig:attest_kernel} shows the average latency of {\tt Attest()} based on the HMAC algorithm for 64B and 128B data inputs. The latency of {\tt Verify()} is similar, and as such, it is omitted. Our \projecttitle{} achieves performance in the microseconds range (23 us) and outperforms its equivalent TEE-based competitors at least by a factor of 2. Importantly, \projecttitle{} is approximately 1.2$\times$ faster than AMD, which is not tamper-proof. 
% although it is approximately 2$\times$ slower than Intel-x86. Recall that neither AMD nor Intel-x86 are tamper-proof.

Figure~\ref{fig:latency_breakdown} shows the latency breakdown of {\tt Attest()}. Accessing the \projecttitle{} device and TEEs can be expensive, ranging from 30\% to 90\% of the total operation latency among the systems. 
Regarding \projecttitle{}, the transfer time (16us) accounts for 70\% of the execution time. We expect that \projecttitle{} effectively eliminates this cost by enabling asynchronous (user-space) DMA data transfers. 
% Specifically, for \projecttitle{}, the transfer time takes about 16 us, which accounts for 70\% of the execution time. This is not a concern in \projecttitle{} as it effectively eliminates this cost by enabling asynchronous (user-space) DMA data transfers. 
% Regarding the native runs, i.e., Intel-x86 and AMD, the communication costs account for $\sim$90\% of the latency.  
% Figure~\ref{fig:latency_breakdown} shows the latency breakdown of {\tt Attest()}. Accessing the \projecttitle{} device and the TEEs can be expensive ranging from 30\% to 90\% of the total operation latency among the systems. Specifically, for \projecttitle{}, the transfer time takes about 16 us, which accounts for 70\% of the total execution time. This is not a concern in \projecttitle{} as it effectively eliminates this cost by enabling asynchronous (user-space) DMA data transfers. Regarding the native runs, i.e., Intel-x86 and AMD, the communication costs including the syscalls execution and data transfers between the two processes account for the $~$90\% of the latency.  
% 
Regarding the TEE-based systems (SGX, AMD-sev), the communication and system call execution costs account for up to 40\% of the total execution. To our surprise, this implies that the HMAC computation within any of the two TEEs experiences more than 30$\times$ overheads compared to its native run. To analyze TEEs' behavior, we instrument the client's code to measure the operations' individual latency at various periods of time during the experiment accurately. 
% Our evaluation shows that in the TEE-based systems, e.g., SGX and AMD-sev, the communication and system call execution costs account for up to 40\% of the total execution (on average). To our surprise, this implies that the HMAC computation itself within any of the two TEEs experiences more than 30$\times$ overheads compared to its native run. We further analyzed TEEs' behavior instrumenting the client's code to accurately measure the operations' individual latency at various periods of time during the experiment. 

Figure~\ref{fig:latency_distribution} illustrates the individual operation latency, where SGX-empty indicates SGX without HMAC computation. 
% for three systems: SGX, SGX-empty (SGX without HMAC computation), and Intel-x86. 
% SGX (the SSL-server runs in the SGX enclave), SGX-empty (the SGX SSL-server without HMAC computation) and Intel-x86 (the SSL-server runs natively). 
As shown in Figure~\ref{fig:latency_distribution}, the HMAC execution within the TEE often experiences huge latency spikes. 
% quite often experiences huge spikes in latency. 
% While these spikes are very frequent, they are not guaranteed. 
% We calculate an average mean of 45us and a geometric mean of 30 us. 
\rev{A6}{We attribute this behavior to the scheduling effects and asynchronous exitless system calls inherent in our SGX framework, {\sc scone}~\cite{scone}. We observe similar latency variations during executions on AMD systems, spiking up to 200--500 us. }
% \atsushi{@Dimitra: The reviewer says, "transitions to in and out of SGX system typically involve TLB flushes but what does an asynchronous exitless system call in SCONE correspond to?" Do we have an answer?}
% We observe similar variations for AMD with latencies spiking up to 200--500 us.

% We attribute this behavior to the scheduling effects and asynchronous exitless system calls inherent in our SGX framework, Scone~\cite{scone}. Similar latency variations were also observed during executions on AMD systems


% Figure~\ref{fig:latency_distribution} illustrates the individual operations latency for three systems: SGX (which runs the HMAC within an SGX SSL-server), SGX-empty (the SGX SSL-server copies and returns the input data without HMAC computation) and Intel-x86 (the SSL-server runs natively).  As shown in Figure~\ref{fig:latency_distribution}, the execution of HMAC within the SGX quite often experiences huge spikes in latency. While these spikes are very frequent, they are not guaranteed. We calculate an average mean of 45us and a geometric mean of 30 us. We attribute this behavior to scheduling effects and the asynchronous exitless system call API which is involved~\cite{scone}. Similar variations were observed for the AMD runs with latencies spiking up to 200--500 us.

\if 0
\begin{figure*}[t!]
\begin{center}
\minipage{0.33\textwidth}
  \centering
  \includegraphics[width=\linewidth]{atc-submission-plots/hw_eval_attest_latency.pdf} 
  \caption{HMAC (Attest) latency}
  \label{fig:attest_kernel}
\endminipage
\minipage{0.33\textwidth}
  \centering
  \includegraphics[width=\linewidth]{atc-submission-plots/latency_breakdown.pdf}
  \caption{Latency breakdown}\label{fig:latency_breakdown}
\endminipage

\minipage{0.50\textwidth}%
  \centering
  %\includegraphics[width=\linewidth]{atc-submission-plots/lat_distribution_sgx.pdf}
  \includegraphics[width=\linewidth]{atc-submission-plots/foo50.pdf}
  \caption{Latency distribution over time\dimitra{increase font size}}\label{fig:latency_distribution}
\endminipage
\end{center}
%\caption{Performance evaluation of the trusted component in different hardware setups.} \label{fig:hw_eval}
\end{figure*}
\fi

\if 0
\begin{figure}[t!]
    \centering
    \includegraphics[width=.5\textwidth]{eval-plots/plots/hw_net_lat_throughput.pdf}
    \caption{Throughput latency plots for network operation.}
    \label{fig:hw_lat_breakdown}
\end{figure}
\fi

\if 0
\begin{center}
\begin{table}[ht]
\centering
\begin{tabular}{ |m{1.5cm}||m{1cm}| m{1cm}| m{1cm}|}
 \hline
  & LUTs &  BRAM & Regs  \\
 \texttt{Attest()} &  &  &\\
 \texttt{Verify()} &  &  &\\
 \hline
 \end{tabular}
\caption{Resources usage.}
\end{table}\label{table:resources_usage}
\end{center}
\fi




\begin{figure}
    \centering
   \includegraphics[width=0.75\linewidth]{atc-submission-plots/rpc_thr.pdf}
   \vspace{-10pt}
    \caption{Throughput of send operations across the three selected network stacks.}
    % \caption{Throughput evaluation of send operations for various packet sizes across five competitive network stacks with various security properties.}
    \vspace{-4pt}
  \label{fig:net_throughput}
\end{figure}

\begin{figure*}
    \centering
   \includegraphics[width=0.80\linewidth]{atc-submission-plots/rpc_lat.pdf} 
   \vspace{-12pt}
  \caption{Latency of send operations across five competitive network stacks with various security properties.}
  % \caption{Latency evaluation of send operations for various packet sizes across five competitive network stacks with various security properties.}
  \label{fig:net_latencies}
   \vspace{-10pt}
\end{figure*}



%%%%%%%%



\vspace{-4pt}
\subsection{Software Evaluation: \projecttitle{} Network Stack}\label{subsec:net_lib}
\vspace{-2pt}



\myparagraph{Baselines} 
To evaluate the \projecttitle{} performance, we discuss (1) the effectiveness of offloading the network stack to the \projecttitle{} hardware and (2) the overheads incurred by the CFT systems transformation for the BFT model. We compare \projecttitle{} across four other software/hardware network stacks with different security properties as follows: 
% We evaluate the \projecttitle{} performance to show {\em{(i)}} the effectiveness of offloading the network stack from host CPUs to accelerators and {\em{(ii)}} the overheads that our system incurs due to materializing the requirements for CFT systems transformation (discussed in $\S~\ref{sec:requirements-ds}$). As such, we evaluate and compare \projecttitle{} across four other network stacks implemented on software or hardware with different security properties. 
%Specifically, we use the acronym D-I/O to refer to a Direct I/O network stack that bypasses the kernel stack (for performance). The acronyms D-I/O w/ A. and \projecttitle{} w/ A. means that the network stack generates and sends attested messages without verifying them at the receiving side. 
%\atsushi{What do you think new labels like this: RDMA-hw, DRCT-IO, DRCT-IO-acc, TNIC, TNIC-acc}
% 
% Specifically, we evaluate five different network stacks
{\em (i)} RDMA-hw, an untrusted RoCE protocol on FPGAs, {\em (ii)} DRCT-IO (direct I/O), untrusted software-based kernel-bypass stack, {\em (iii)} DRCT-IO-att, altered DRCT-IO that offers trust by sending attested messages but does not verify them, and {\em (iv)} \projecttitle{}-att, altered \projecttitle{} that similarly sends attested messages without verification. We build {\em (i)} RDMA-hw on top of Coyote~\cite{coyote} network stack similarly to \projecttitle{}. For {\em (ii) (iii)} DRCT-IOs, we base our design on eRPC~\cite{erpc} with DPDK~\cite{dpdk} that offers similar reliability guarantees with RDMA-hw. The hardware network stacks run on AMD-FPGA Cluster, whereas the rest run on Intel Cluster.
% The benchmarks with hardware implementation run on AMD-FPGA Cluster, whereas the rest run on Intel Cluster.
% Specifically, we evaluate five different network stacks: {\em (i)} RDMA-hw which implements a reliable, untrusted RoCE protocol on FPGAs, {\em (ii)} our \projecttitle{}, {\em (iii)} DRCT-IO, a direct I/O, untrusted, software-based network stack that bypasses the kernel stack, {\em (iv)} DRCT-IO-att, the previous stack that offers trust by sending attested messages (using an SGX-based SSL-server) without verifying them at the receiving side and {\em (v)} \projecttitle{}-att that similarly sends attested messages but does not verify them. We build the RDMA-hw experiment on top of Coyote~\cite{coyote} network stacks imilarly to \projecttitle{}, . For the DRCT-IO versions, we base our design on eRPC~\cite{erpc} with DPDK~\cite{dpdk} that offers similar reliability guarantees with RDMA-hw. The benchmarks with hardware implementation run on AMD-FPGA Cluster, whereas the rest run on Intel Cluster.
%\antonis{again names are not very memorable. Something more %descriptive/relevant to the paper than D-I/O, etc.?}
%\atsushi{Which term we should use for our proposal: \projecttitle{} or \projectlibrary{}?}

%Specifically, our evaluation answers the following questions:
%\begin{itemize}
%    \item {\bf RQ1.} How much does \projecttitle{}'s offered security cost?
%    \item {\bf RQ2.} How does \projectlibrary{} performs compared to competitive network %stacks?
%\end{itemize}



\if 0
\begin{figure}[t!]
\begin{center}
\minipage{0.7\linewidth}
  \centering
  \includegraphics[width=\linewidth]{atc-submission-plots/tnic_overheads.pdf} 
\endminipage
\minipage{0.3\linewidth}
  \centering
  \includegraphics[width=\linewidth]{atc-submission-plots/tnic_speedup.pdf} 
  %\label{fig:net_speedup}
\endminipage
\end{center}
\caption{Left figure shows the slowdown of \projecttitle{} w.r.t. to RDMA-hw. Right figure shows the speedup of \projecttitle{} w.r.t. DRCT-IO-att.}\label{fig:net_slowdown}
\end{figure}
\fi




\myparagraph{Methodology and experiments} 
Our experiments measure the latency and throughput for respective network stacks, which run through a single-threaded client-server implementation.
For the latency measurement, the client sends one operation at a time, whereas for the throughput measurement, one client can have multiple outstanding operations.




%issue multiple operations as parallelism in (totally-ordered) consensus protocols (e.g., ~\cite{10.1145/3190508.3190538}) has little performance improvements~\cite{f04eb9b864204bab958e72055062748c, 10.1145/3299869.3319893}. 

%we reliably measure the \projecttitle{} latencies in this way as the ordering operations dominate the overall latency of BFT systems~\cite{10.1145/3190508.3190538}.
%Whereas, 

% , while for the throughput, the client issues multiple operations. These settings are due to two reasons. First, commercial BFT-supported blockchain systems~\cite{10.1145/3190508.3190538} report that ordering operations dominate the overall latency, so we reliably measure the \projecttitle{} latencies in such a way. 
% As such, we want to reliably illustrate \projecttitle{} latencies. 
% First, commercial blockchain systems whose design relies on BFT systems to implement ordering services~\cite{10.1145/3190508.3190538} for the ledger have shown that ordering operations dominates the overall latency. As such, we want to reliably illustrate \projecttitle{} latencies. 
% Second, parallelism in (totally-ordered) consensus protocols (e.g., ~\cite{10.1145/3190508.3190538}) have been evaluated to have little or minimal performance improvements~\cite{f04eb9b864204bab958e72055062748c, 10.1145/3299869.3319893}.

% All our runs in the section have been conducted through a single-threaded client-server implementation that uses the respective network stack. In the throughput experiments the client issues more than one on-going operation whereas to optimize for the latency (and accurately measure it) the client process sends one operation at a time in the respective experiments.  We decided on these settings for two reasons. First, commercial blockchain systems whose design relies on BFT systems to implement ordering services~\cite{10.1145/3190508.3190538} for the ledger have shown that ordering operations dominates the overall latency. As such we want to reliably illustrate \projecttitle{} latencies. Secondly, parallelism in (totally-ordered) consensus protocols (e.g., ~\cite{10.1145/3190508.3190538}) have been evaluated to have little or minimal performance improvements~\cite{f04eb9b864204bab958e72055062748c, 10.1145/3299869.3319893}.

%have already shown that parallelism in (totally-ordered) consensus protocols (as in~\cite{10.1145/3190508.3190538})  offer minimal to no performance improvements.
%\atsushi{Can we simplify the paragraph like this: we run experiments in such settings... for two reasons. First, commercial blockchain systems... Second, parallelism in consensus... has already been evaluated. }

\myparagraph{Results} 
Figure~\ref{fig:net_latencies} and~\ref{fig:net_throughput} show the latency and throughput of the send operation with various packet sizes. First, regarding (1) the effectiveness of network stack offloading, RDMA-hw is 3$\times$---5$\times$ faster than DRCT-IO, which indicates that the network offloading boosts performance. Although DRCT-IO offers minimal latency (16-16.6us) for small packet sizes up to 1~KiB due to its zero-copy transmission/reception optimizations~\cite{erpc}, they are only effective for up to 1460B (MTU is 1500B, but 40B are reserved for metadata), and RDMA-hw still achieves 3$\times$ lower latency (5-5.5us). For bigger data transfers, the RDMA-hw latency increases steadily up to 19~us, whereas DRCT-IO does not scale well with latencies up to 100us. 


Second, regarding (2) the \projecttitle{} performance overhead, \projecttitle{} offers trusted networking with 3$\times$---20$\times$ higher latencies than the untrusted RDMA-hw. 
% The latencies of both \projecttitle{} and \projecttitle{}-att increase linearly with the packet size. Whereas, the (untrusted) RDMA-hw latencies remain stable (5---7us) for packet sizes up to 4KiB and are tripled thereafter (18---20us). 
The latency increase stems from the HMAC calculation of the \projecttitle{} hardware. As this algorithm fundamentally cannot be parallelized, the higher the message size, the higher the latency our \projecttitle{} incurs. More specifically, for packet sizes less than 1~KiB, doubling the packet size in \projecttitle{} results in a 13\%---20\% increment in the overall latency. For packet sizes bigger or equal to 1~KiB, doubling the packet size increases the latency by 30\%---40\%. 
Compared to DRCT-IO-att (82us), \projecttitle{} is up to 5.6$\times$ faster. Importantly, DRCT-IO-att reports extreme latencies (2000us or more) for packet sizes larger than 521B, which are omitted to avoid plot distortion. We attribute these latencies to the scheduling effects of {\sc scone}~\cite{scone}. 


% Figure~\ref{fig:net_latencies} and~\ref{fig:net_throughput} show the latency and throughput of the send operation with various packet sizes. As shown in Figure~\ref{fig:net_latencies}, \projecttitle{} offers trusted networking with 3$\times$---20$\times$ higher latencies than the untrusted RDMA-hw. 
% % The latencies of both \projecttitle{} and \projecttitle{}-att increase linearly with the packet size. Whereas, the (untrusted) RDMA-hw latencies remain stable (5---7us) for packet sizes up to 4KiB and are tripled thereafter (18---20us). 
% The latency increase stems from the HMAC calculation of the \projecttitle{} hardware. As this algorithm fundamentally cannot be parallelized, the higher the message size, the higher the latency our \projecttitle{} incurs. More specifically, for packet sizes less than 1~KiB, doubling the packet size in \projecttitle{} results in a 13\%---20\% increment in the overall latency. For packet sizes bigger or equal to 1~KiB, doubling the packet size increases the latency by 30\%---40\%. 
% 
% Compared to DRCT-IO-att (82us), \projecttitle{} is up to 5.6$\times$ faster. Importantly, DRCT-IO-att reports extreme latencies (2000us or more) for packet sizes larger than 521B, which are omitted to avoid plot distortion. We attribute these latencies to the scheduling effects of {\sc scone}~\cite{scone}. 
% 
% For packet sizes up to 1~KiB, DRCT-IO offers minimal latency (16-16.6us) due to zero-copy transmission/reception optimizations~\cite{erpc} which are only effective for up to 1460B (MTU is 1500B but 40B are reserved for metadata). RDMA-hw achieves 3$\times$ lower latency (5-5.5us) for packet sizes up to 2~KiB. For bigger data transfers, the RDMA-hw latency increases steadily up to 19~us, whereas DRCT-IO does not scale well with latencies up to 100us. Overall, RDMA-hw is 3$\times$---5$\times$ faster than DRCT-IO showing that network offloading to the hardware boosts performance.


%We now compare the untrusted software and hardware implementations of the network stacks. The RDMA-hw is 3$\times$---5$\times$ faster than the DRCT-IO. For packet sizes up to 1~KiB the DRCT-IO offers minimal latency (16-16.6us) due to zero-copy transmission/reception optimizations~\cite{erpc} that are only effective for up to 1460B (MTU is 1500B but 40B are reserved for metadata). Recall that even with such optimizations .  For bigger data transfers, the RDMA-hw latency increases steadily up to 19~us for packet size to be equal to 32KiB. DRCT-IO, on the other hand, does not scale well for bigger packet sizes, reporting latency up to 100us, mainly because network stack operations (e.g., packet fragmentation, packet construction, etc.) are running in software in contrast to RDMA-hw. 


%Lastly, D-I/O-att. shows stable latency (82us) for packet sizes up to 512. For larger packet sizes, we measured extreme latencies (2000us or more), which we attribute to the scheduling effects of the framework to access the TEE~\cite{scone}. We omit these numbers to avoid plot distortion.


%Latencies of both \projecttitle{} and \projecttitle{} w/ A. increase linearly with the packet size. This is primarily due to HMAC calculation, which fundamentally is not improved by parallelism. As such, the higher the message size, the higher the latency our \projecttitle{} incurs. Specifically, in \projecttitle{}, for packet sizes that are less than 1~KiB, doubling the packet size results in 13\%---20\% increment in the overall latency. For packet sizes bigger or equal to 1024, doubling the packet size increases the latency by 30\%---40\%. \projecttitle{} w/ A. shows similar behavior. Lastly, D-I/O w/ A. shows stable latency (82us) for packet sizes up to 512. For larger packet sizes, we measured extreme latencies (2000us or more), which we attribute to the scheduling effects of the framework to access the TEE~\cite{scone}. We omit these numbers to avoid plot distortion.


%Lastly, we calculated the slowdown of \projecttitle{} compared to the RDMA-hw (Figure~\ref{fig:net_slowdown} (left)) and the speedup of \projecttitle{} compared to DRCT-IO-att  (Figure~\ref{fig:net_slowdown} (right)). As expected, the bigger the packet size, the higher the respective overhead; this is due to HMAC calculations, as explained.  \projecttitle{} has 3---20$\times$ overhead compared to RDMA-hw and 3$\times$---5$\times$ speedup compared to its equivalent software based implementation that uses SGX as network packets authenticator. The overheads are the result of the HMAC calculation whereas the speedup is the result of both the network stack offloading in hardware and the on-data-path security processing thanks to \projecttitle{} attestation kernel.
\subsection{Distributed Systems Evaluation}
\label{subsec:use_cases_eval}

\if 0

\fi
%We implement the four systems of $\S~\ref{sec:use_cases}$ with \projecttitle{} in a three-node setup ($N=3$) except for the single-node A2M system.
% the systems' properties where $N$ refers to the number of machines used for the protocol, and $f$ is the number of failures the system can tolerate.

We next evaluate four distributed systems described in $\S~\ref{sec:use_cases}$.% based on \projecttitle{}.


\myparagraph{Methodology and experiments} 
We execute all four of our codebases on Intel Cluster in three servers (as the minimum required setup capable of withstanding a single failure, $N=2f+1$, where $f=1$). \rev{E4}{We only use a single port of the U280 for network communication because of a limitation introduced in our system by the Coyote codebase~\cite{coyote}, on top of which we base \projecttitle{} implementation.} Due to our limited resources, we cannot install Alveo U280 cards on all these servers.  Instead, we build our codebase using the DRCT-IO stack (detailed in $\S$~\ref{subsec:net_lib}) and inject busy waits to emulate the delays incurred by \projecttitle{} for generating and verifying attested messages.
% in the \projecttitle{} system. %Our code uses busy waiting to accurately emulate latency rather than sleep functions. 
% We execute all four of our codebases on Intel Cluster, utilizing all its three servers (as the minimum required setup capable of withstanding a single failure, $N=2f+1$, where $f=1$). Furthermore, due to our limited resources, with only two U280 cards available and the physical separation of Intel Cluster from AMD-FPGA Cluster, we cannot access U280 cards for these specific experiments. Instead, we compiled our code using the DRCT-IO network stack as previously detailed in $\S$~\ref{subsec:net_lib} and do busy waiting to accurately replicate the \projecttitle{} delays within the CPU for generating and verifying attested messages.% in the \projecttitle{} system. %Our code uses busy waiting to accurately emulate latency rather than sleep functions. 

We evaluate each codebase using five systems that generate and verify the attestations: {\em (i)} SSL-lib (no tamper-proof), {\em (ii)} SSL-server (no tamper-proof), {\em (iii)} SGX, {\em (iv)} AMD-sev, and {\em (v)} \projecttitle{}. To perform a fair comparison, we integrate into our codebases a library that accurately emulates all latencies (measured in $\S$~\ref{subsec:hw_eval}) within the CPU. For the AMD latency, we use 30us, representing the lower bound of the latencies measured in $\S$~\ref{subsec:hw_eval}. We do not emulate the SSL-lib latency. 
% We evaluate each protocol using five systems that generate and verify the attestations: {\em (i)} SSL-lib, an SSL library that is integrated into the codebase (no tamper-proof), {\em (ii)} SSL-server (no tamper-proof), {\em (iii)} SGX, {\em (iv)} AMD-sev, and {\em (v)} \projecttitle{}. To perform a fair comparison, we integrate into our protocols' codebases an library that accurately emulate all latencies (measured in $\S$~\ref{subsec:hw_eval}) within the CPU. We do not emulate the SSL-lib latency and for the AMD latency we use 30us which represents the lower bound of the latencies measured in $\S$~\ref{subsec:hw_eval}.

 %which are all an integrated library to the codebase and further adds a configurable delay to represent the operation's latency in each system. The added delay for each system is the respective latency we measured in

Given that DRCT-IO, which is used for the emulation, is at least 3$\times$ slower than the hardware RDMA network stack (RDMA-hw), the latencies outlined in this section are anticipated to reflect the upper limit for all four systems with \projecttitle{}.

\rev{(c)}{We additionally evaluate two TEEs-hosted CFT replication protocols (TEEs-Raft and TEEs-CR) where the entire protocol codebase (Raft~\cite{raft} and Chain replication~\cite{chain-replication} respectively) resides within the TEE. We compare the TEEs-hosted systems with \projecttitle{} and discuss the trade-offs between their threat model, TCB, and performance.}
% Given that DRCT-IO for small messages operates roughly three times the speed of the hardware-based RDMA network stack implementation, the latencies outlined in this section are anticipated to reflect the upper limit for all four systems with \projecttitle{}.


\begin{table}[t!]
\begin{center}
\small
\minipage{0.22\textwidth}
  \centering
\begin{tabular}{lrr}
\hline
& \multicolumn{2}{c}{Throughput (Op/s)} \\
System          & append    & lookup  \\
\hline
SSL-lib         & 790K      & 256M      \\
SGX-lib             & 380K      & 3.8M       \\
AMD-sev         & 30K       & 263M      \\
\projecttitle{} & 158K      & 257M      \\
\hline
\end{tabular}
\endminipage
\hfill
\minipage{0.18\textwidth}
\centering
\begin{tabular}{rrr}
\hline
\multicolumn{2}{c}{Latency (us)} \\
 append     & lookup  \\
\hline
 1.26       & 0.0039      \\
 2.6        & 0.26       \\
 32.37      & 0.0038      \\
 6.34       & 0.0039      \\
 \hline
\end{tabular}
\endminipage
\end{center}
\caption{Throughput and latency of A2M.}%\dimitra{@experiments: maybe you try run A2M into an SGX server to also show the communication costs}}
\label{fig:a2m_eval}
\vspace{-4pt}
\end{table}

\myparagraph{A2M} We first evaluate our \projecttitle{}-A2M system. 
% We evaluate our prototype of the A2M system with \projecttitle{}. 
% For SGX, we port the entire log within the TEE, labeled as SGX-lib. All other versions place the attested log in the untrusted host memory, using the trusted systems to generate attestations as in~\cite{levin2009trinc}.
\rev{B6}{
We evaluate two TEE baselines: SGX-lib, which places the entire log within the TEE, and AMD-sev, which places the attested log outside the TEE as in the implementation of TrInc~\cite{levin2009trinc} and has been shown to be effective. 
% adapting the implementation of TrInc~\cite{levin2009trinc}
% We evaluate another TEE baseline using SGX, labeled SGX-lib, which places the entire log within the TEE. We also evaluate AMD-sev adapting the implementation of TrInc~\cite{levin2009trinc}, which places the attested log outside the TEE and has been shown to be effective. 
}
%, using the trusted systems to generate attestations as in~\cite{levin2009trinc}. 
% In this experiment, we first construct a log of size 9GB with 100M entries and then sequentially we lookup for them individually. Each log entry is comprised of 64B of appended data (context) and an extra 36B for the metadata.
In this experiment, we construct a 9.3GiB log with 100 million entries and then lookup them sequentially/individually. %Each log entry is 100B, which contains 64B appended context and 36B metadata.
% \rev{B6}{We evaluated A2M (which ports the entire log within the SGX) with the implementation of TrInc~\cite{levin2009trinc} (which places the log outside the SGX and has been shown to be effective) with \projecttitle{} (AMD-sev).}
% \atsushi{@Dimitra: let me make sure, does AMD-sev represent the implementation of TrInc (the log is outside the enclave)?}
% SSL-lib (no tamper-proof), SGX-lib (the entire log in the enclave)


%\myparagraph{A2M} We evaluate our prototype of the A2M system with \projecttitle{}. 
% an effective building block for improving the scalability and performance in various systems~\cite{A2M, sundr, Castro:2002, AbdElMalek2005FaultscalableBF}. 
%For the SGX, similarly to the prior work~\cite{A2M, sundr, Castro:2002, AbdElMalek2005FaultscalableBF}, we port the entire log within the TEE, building a large TCB. We refer to this system as SGX-lib. All other versions place the attested log in the untrusted host memory, using the trusted systems to generate attestations as in~\cite{levin2009trinc}. %\atsushi{Does this paragraph repeat explanations in s7.1? If so, like the following applications, can we simply say "We evaluate the A2M system (TNIC-log described in $\S$~\ref{sec:use_cases::a2m})..."?}
% In this experiment, we first construct a log of size 9GB with 100M entries and then sequentially we lookup for them individually. Each log entry is comprised of 64B of appended data (context) and an extra 36B for the metadata.
%In this experiment, we construct a 9.3GiB log with 100 million entries, and then lookup them sequentially/individually. Each log entry is 100B, which contains 64B appended context and 36B metadata.

%specifically the computed authenticate (32B) and the sequence number of the entry (4B). In short, we append 100M of 100B entries, leading to a log size of approximately 9GB. \atsushi{Can we somehow shorten the description?}


\noindent{\underline{Results.}} 
Table~\ref{fig:a2m_eval} shows the throughput and mean latency of the append/lookup operations. The native execution (SSL-lib) achieves the highest throughput as it incurs no communication costs. 
% Specifically, its latency is 1.26us, which is dominated by the HMAC computation.
Compared to SSL-lib, SGX-lib experiences only a 2$\times$ slowdown because we avoid the costly communication w.r.t. an SGX-based server implementation. On the other hand, AMD-sev, which runs the SSL server, incurs a 15$\times$ slowdown. Lastly, \projecttitle{} incurs 5$\times$ and 2.4$\times$ slowdown compared to SSL-lib and SGX-lib, respectively, due to the HMAC calculation.

% Table~\ref{fig:a2m_eval} shows the throughput (operations/s) and the mean latency of our A2M system using various systems. The append operation throughput in the SSL-lib case illustrates the throughput upper bound as it incurs no communication costs. Specifically, the A2M with SSL-lib running natively in the CPU reports a latency of 1.26us, which is dominated by the HMAC computation latency.
% Placing the log within the SGX (SSL-lib), we avoid the costly communication w.r.t. to an SGX-based server implementation, and as such, the system only experiences a 2$\times$ slowdown compared to the native case. The communications costs are reflected in the AMD-sev case; that runs the SSL-server. Compared to when porting to SGX, AMD-sev incurs 15$\times$. Lastly for \projecttitle{}, we observe approximately 5$\times$ and 2.4$\times$ slowdown compared to the SSL-lib and SGX-lib execution, respectively which is due to the HMAC calculation.

Regarding the lookup operation, SSL-lib, AMD-sev, and \projecttitle{} report similar throughput and latency because they lookup untrusted host memory for the requested entry. However, SGX-lib reports a 66$\times$ slowdown due to its trusted memory size constraints and expensive paging mechanism~\cite{treaty} \rev{C3}{because we have to support a log of 9GB within the SGX enclave that only provides 94MB of memory. In contrast, AMD-sev is faster as it only accesses the untrusted host memory. Similar findings have also been demonstrated in~\cite{levin2009trinc}}. As a result, while \projecttitle{} increases append latencies, it greatly optimizes lookup latencies due to its minimal TCB.
% As a result, while \projecttitle{} offers slower append operations than porting the entire log into the TEE, it greatly optimizes lookup latencies due to its minimal TCB.


\begin{figure*}
\centering
\minipage{0.33\textwidth}
\centering
    %\includegraphics[width=\linewidth]{atc-submission-plots/bft_pb.pdf} 
    \includegraphics[width=\linewidth]{eval-plots/cf-protos-in-amd-eval/bft_pb_exte.pdf} 
    \vspace{-4mm}
    \caption{Throughput (and latency numbers) of BFT.} \label{fig:byz_smr_throuthput}
\endminipage%
\minipage{0.33\textwidth}
  \centering
  %\includegraphics[width=\linewidth]{atc-submission-plots/bftcr_lat_throughput.pdf} 
  \includegraphics[width=\linewidth]{eval-plots/cf-protos-in-amd-eval/bftcr_lat_throughput_exte.pdf} 
    \vspace{-6mm}
    \caption{Throughput (and latency numbers) of Chain Replication.} \label{fig:byz_chain_replication}
\endminipage %
\minipage{0.33\textwidth}
     \includegraphics[width=\linewidth]{atc-submission-plots/bftpr_lat_throughput.pdf} 
    \vspace{-6mm}
    \caption{Throughput (and latency numbers) of PeerReview.} \label{fig:accountability_protocol}
\endminipage
\vspace{-2mm}
\end{figure*}


\myparagraph{BFT} We evaluate the performance of our BFT protocol with various network batching factors. We implement network batching as part of the application's message format. % In this experiment, we allocate one {\em message} structure for each client's request, which contains the initial command, the results of the command's execution in a node, the incremented counter values, etc. % and the signed hashes of the replicas' states known to each node.

% In this experiment, we allocate one {\em message} structure for each client's request. The {\em message} contains command and output fields (16B each), which store the initial command and results of the command's execution in a node, respectively. In addition, the message contains metadata, which includes a sequencer and signed hashes of the replicas' states known to each node. The sequencer is used for serialization, and it is equivalent to the counter value assigned to each message from \projecttitle{}. The signed hashes representing the replicas' states are obtained from the last messages received from each replica.

\noindent{\underline{Results.}} Figure~\ref{fig:byz_smr_throuthput} shows the throughput and latency of the protocol, which highlights that \projecttitle{} significantly outperforms TEE-based versions (SGX, AMD-sev), improving the throughput and latency 4---6$\times$. On the other hand, \projecttitle{} incurs 2.4$\times$ throughput overhead and up to 7$\times$ higher latency compared to SSL-lib. 
We recall that SSL-lib is not tamper-proof (Table~\ref{tab:hw_options}) and eliminates the communication overheads incurred by other tamper-proof solutions (SGX, AMD-sev).
% as the security-related processing runs in the untrusted host CPU. % and can be accessed and compromised by malicious adversaries.
% Recall that SSL-lib is the upper bound in performance because it calculates the attestations natively and eliminates the communication overheads incurred by a realistic solution (SGX, AMD-sev). 

% Figure~\ref{fig:byz_smr_throuthput} show the throughput and the latency of the protocol. Our \projecttitle{} incurs about 2.4$\times$ throughput overhead and up to 7$\times$ higher latency compared to the SSL-lib version. Similarly, SGX and AMD-sev versions incur 13.5$\times$ and 9.6$\times$ throughput overheads compared to the SSL-lib. As such, our \projecttitle{} improves throughput and latency 4---6$\times$ compared to TEE-based versions. Recall that the SSL-lib version is the upper bound in performance we can possibly achieve as it calculates the attestations natively and eliminates the communication overheads that a realistic solution (SGX, AMD-sev) occurs. However, in contrast to our \projecttitle{}, SSL-lib is not tamper-proof (Table~\ref{tab:hw_options}) as the security-related processing runs in the untrusted host CPU and can be accessed and compromised by malicious adversaries.

% However, the \projecttitle{} throughput is better than SGX and AMD-sev, which incur 13.5$\times$ and 9.6$\times$ overheads compared to SSL-lib, respectively. 
% As such, our \projecttitle{} improves throughput and latency 4---6$\times$ compared to TEE-based versions. 

%\atsushi{Can we highlight that TNIC is temper-proof but the Nat-lib (best perf) is not?}.
%\antonis{Are the differences/benefits of TEE vs. TFPGA tamper-proofness discussed earlier than the evaluation? Seems its something important.}


We also observe that batching improves the throughput and latency proportionally to the number of batched messages. For all except SSL-lib, the batching factors equal to 8 and 16 achieve 7$\times$ and 15$\times$ higher throughput than without batching, respectively. For SSL-lib, they are moderately effective: approximately 4---6$\times$ faster. It is primarily because the native execution of the attestation function is fast enough to saturate the network bandwidth. 
As such, conventional techniques can drastically eliminate the overheads for BFT and improve \projecttitle{}'s adoption into practical systems.
% In addition, we observe that batching increases throughput (and decreases latency) proportionally to the number of batched messages. For all but the SSL-lib version, we report a 7$\times$ and 15$\times$ throughput boost for batching factor to be equal to 8 and 16 respectively compared with the experiment run with batching factor equal to 1. This is because our batching technique improves the network utilization and reduces the overall attestation calculations, e.g., one attestation per 8 and 16 messages. The technique is moderately effective when using the SSL-lib, approximately 4---6$\times$ faster than without batching. This is primarily because, in that case, the latency is dominated by the network stack latency as the attestation function running natively generates attestations fast. As such, we show that conventional techniques can drastically eliminate the overheads for BFT and improve \projecttitle{}'s adoption into practical systems.




%\begin{figure}[t!]
%    \centering
%    \includegraphics[width=\linewidth]{atc-submission-plots/bft_pb.pdf} 
%    \caption{Throughput (and latency numbers) of BFT using various trusted components.} \label{fig:byz_smr_throuthput}
%\end{figure}

%\begin{figure}[t!]
%    \centering
%    \includegraphics[width=0.7\linewidth]{atc-submission-plots/bft_pb_lat.pdf} 
%    \caption{Average latency of the BFT SMR using various trusted components.} \label{fig:byz_smr_lat}
%\end{figure}


\myparagraph{CR} 
In this experiment, we evaluate the performance of our CR. 
% We evaluate the performance of our Chain Replication. In this experiment, 
We allocate one message structure per client request comprising 60B context, 4B operation type, and a 32B signature.

% (that includes metadata, e.g., source/destination nodes, message ID), 
%We implemented the replication protocol without any underlying Key-Value store data structure.
%of the message allocates 60B comprised of an 8B key and a 32B value as well as 16B for metadata (e.g., source and destination nodes, message idx, etc.) and
\noindent{\underline{Results.}} Figure~\ref{fig:byz_chain_replication} shows the throughput and latency of our Chain Replication. We highlight that our \projecttitle{} is 5$\times$ and 3.4$\times$ faster than SGX and AMD-sev, respectively. While \projecttitle{} incurs 4.6$\times$ overheads compared to SSL-lib, it is 30\% faster than SSL-server, which is not tamper-proof. The performance benefit stems primarily from hardware acceleration by the \projecttitle{}'s attestation kernel on the transmission/reception data path.

%\begin{figure}
%    \centering
%  \includegraphics[width=\linewidth]{atc-submission-plots/bftcr_lat_throughput.pdf} 
%    \caption{Throughput (and latency numbers) of Chain Replication using various trusted components.} \label{fig:byz_chain_replication}
%\end{figure}

%\begin{figure}
%  \includegraphics[width=\linewidth]{atc-submission-plots/bftpr_lat_throughput.pdf} 
%    \caption{Throughput (and latency numbers) of PeerReview using various trusted components.} \label{fig:accountability_protocol}
%\end{figure}

\if 0
\begin{figure*}[t!]
\begin{center}
\minipage{0.5\textwidth}
  \centering
  \includegraphics[width=0.8\linewidth]{atc-submission-plots/bftcr_lat_throughput.pdf} 
    \caption{Throughput-latency evaluation of a Byzantine Chain Replication using various trusted components.} \label{fig:byz_chain_replication}
\endminipage
\minipage{0.5\textwidth}
  \centering
  \includegraphics[width=0.8\linewidth]{atc-submission-plots/bftpr_lat_throughput.pdf} 
    \caption{Throughput-latency evaluation of the accountability protocol using various trusted components.} \label{fig:accountability_protocol}
\endminipage
\end{center}
\end{figure*}
\fi


\if 0
\begin{figure}[t!]
    \centering
    \includegraphics[width=\linewidth]{atc-submission-plots/bftcr_lat_throughput.pdf} 
    \caption{Throughput-latency evaluation of a BFT version of CR using various trusted components.} \label{fig:lat_throughput_kernel}
\end{figure}
\fi






\myparagraph{PeerReview} 
We evaluate our PeerReview system's performance by both activating and deactivating the audit protocol. The system uses one witness for the source node that {\em periodically} audits its log. 
% The witness process is co-located in the same node, and as such, reading the log implies reading a shared memory. We decide in favor of that implementation to carefully isolate the overheads for inspection and replay of the log. 
In our experiments, the witness audits the log after every send operation in the source node until both clients acknowledge the receipt of all source messages. % Each message allocates about 200B, comprising a context, its hash, the cumulative digest of the hashes, etc. 
% We evaluate the performance of our PeerReview system. % using various trusted components. 
% We evaluate the protocol by both activating and deactivating the audit protocol. In our implementation, we use one witness for the source node that {\em periodically} audits its log. The witness process is co-located in the same node, and as such, reading the log implies reading a shared memory. We decide in favor of that implementation to carefully isolate the overheads for inspection and replay of the log. In our experiments, the witness is configured to audit the log after every send operation in the source node until both clients acknowledge the receipt of all source messages.

% The log entries consist of the context, a sequencer, a signed hash of the context (authenticator), and the cumulative digest, which is the signed hash of the authenticator and the previous entry's digest in the log. In total, the messages allocate about 200B.
% The log entries consist of the context, a sequencer, a signed hash of the context (authenticator), and the cumulative digest, which is the signed hash of the authenticator and the previous entry's digest in the log. In total, the messages allocate about 200B.
% and we do not use batching. 
%\atsushi{I feel that the description of these data formats (as well as the other applications) is too detailed, so it'll be fine to omit for cutting the text.}

%The log entries consist of the context, a sequencer, a signed hash of the context, and the sequencer (authenticator) \antonis{is this different than the aforementioned "sequencer"?} as well as the cumulative digest, which is the signed hash of the authenticator and the previous entry's digest in the log. The context is comprised of the data (16B), the output of the protocol specification for that input data (16B), some metadata, and the response sent from the clients that have a similar format. In total, the messages allocate about 200B, and we do not use batching. \atsushi{I feel that the description of these data formats (as well as the other applications) is too detailed, so it'll be fine to omit for cutting the text.}

\noindent{\underline{Results.}} Figure~\ref{fig:accountability_protocol} shows the throughput and latency of our PeerReview system with and without enabling the audit protocol.
Without the audit protocol, the TEE-based systems (SGX, AMD-sev) result in up to 30$\times$ slower throughput than SSL-lib, whereas our \projecttitle{} mitigates the overheads: 3---5$\times$ better throughput compared to AMD-sev and SGX.


Similarly, \projecttitle{} outperforms AMD-sev and SGX by 3.7---5$\times$ with the audit protocol. Importantly, when using \projecttitle{}, the audit protocol itself consumes about 25\% (17us) of the overall latency, leading to 1.33$\times$ performance slowdown. % compared to when being disabled. 
% The audit protocol itself consumes approximately 25\% (17us) of the overall execution latency. 
However, even with the audit protocol, \projecttitle{} offers 3.7---5.42$\times$ lower latency compared to its TEE-based competitors.
% With the audit protocol, we observe similar system behavior; \projecttitle{} outperforms both AMD-sev and SGX by 3.7---5$\times$. Importantly, when using \projecttitle{}, the audit protocol leads to a performance slowdown of 1.33$\times$ compared to when being disabled. The audit protocol itself consumes approximately 25\% (17us) of the overall execution latency. Even with the audit protocol, \projecttitle{} offers 3.7---5.42$\times$ lower latency compared to its TEEs-based competitors.

%\atsushi{a bit difficult to read because 'log' is used as both noun and verb. Can we somehow use another word?} When the auditing protocol is de-activated, the source node sends the (streaming) data to the client nodes. The clients execute and append the message to their log, and then reply to the source. The source appends its own message to the log along with the client's reply. This is required to make sure that any participant cannot lie to a witness about the data it sends to the clients. \atsushi{The sentences until here explain the application workflow, but we should show the results first and then start the discussion with an extra explanation.} We see that using a TEE to generate attested messages increases the throughput slowdown up to 30$\times$ compared to the native embedded library (Nat-lib). Using our \projecttitle{} we improve the overheads---specifically, we incur 5.8$\times$ slowdown, that is 3---5$\times$ better throughput compared to when using AMD-sev and SGX as trusted messages' singers.

\begin{table}[t]
    \centering
    \small
    \setlength\tabcolsep{4.2pt}
    \begin{tabular}{ccrrrrr}
        \hline
        & &\multicolumn{4}{c}{TCB size (LoC)} \\ \cmidrule{3-6}
        System          &Threat model &OS & Att. kernel & App & Total \\ \midrule
        TEEs-Raft       &CFT &2,307K & 1,268 & 856 & 2,309K \\
        TEEs-CR         &CFT &2,307K & 1,268 & 992 & 2,309K \\ 
        \projecttitle{} &BFT & -     & 2,114 & -   & 2,114  \\ \hline
    \end{tabular}
    % \caption{\rev{(c)}{\projecttitle{} compared with TEE-hosted applications.}}
    \caption{\rev{(c)}{\projecttitle{} compared with TEE-hosted applications.}}
    \label{table:qualitative_comparison}
    \vspace{-2mm}
\end{table}

%~ configured with virtio with batch factor equal to one
\myparagraph{\rev{(c)}{TEEs-hosted baselines}} 
\revcont{We compare \projecttitle{} with TEEs-hosted systems implementing two prototypes based on the failure-free execution of Raft (TEEs-Raft) and CR (TEEs-CR). The code runs within three AMD-sev machines. Prior works~\cite{avocado, nimble} suggested this setup for performance---however, at the cost of (1) significantly increased TCB size and (2) a weaker threat model from the application perspective. 
Table~\ref{table:qualitative_comparison} summarizes the security costs. Regarding (1), the TCB of TEEs-hosted systems includes the entire OS~\cite{gramine_tdx}, OpenSSL libraries for messages authentication~\cite{openssl_hmac} (labeled as Att. kernel), and the application codebase, which is over $2$M LoCs in total. In contrast, \projecttitle{}'s TCB only includes our hardware attestation kernel, which is 2,114 LoC of HLS/HDL code. It is only 0.09\% of TEE-hosted systems. 
% In contrast, our \projecttitle{}'s attestation kernel is minimal; only 2,114 LoC of HLS/HDL code (0.09\% of TEE-hosted systems). 
Regarding (2), the TEE-hosted application can only fail by crashing; it can be thought to remain protected from a potentially Byzantine cloud environment, whereas \projecttitle{} targets BFT settings, handling up to $f$ arbitrary failures. 
}

\revcont{We compare TEE-Raft with our \projecttitle{}-based BFT (Figure~\ref{fig:byz_smr_throuthput}) as both are broadcast-based protocols, and TEEs-CR with our \projecttitle{}-based CR (Figure~\ref{fig:byz_chain_replication}) as both require all messages to traverse the entire chain of nodes. TEE-Raft achieves approximately $2.5\times$ higher throughput than \projecttitle{}-based BFT. The performance difference is primarily due to Raft's one-phase commitment compared to our \projecttitle{}-based BFT. Similarly, TEE-CR achieves $2\times$ higher throughput than the \projecttitle{}-based CR. While both versions of CR involve the same number of network Round-Trip Times (RTTs), 
\projecttitle{} involves a higher number of the attestation kernel invocations to verify all the chained messages in the PoE.
}

% \begin{table}[t]
% \centering
% \small
% \begin{tabular}{ c c  c  }
% \hline
%  System & Threat model &  TCB size \\ 
%  \hline
%  \projecttitle{} & BFT  & $<2.5$K LoC  \\  
%  TEEs-app & CFT  & OS  $>2000$K LoC\\
%  \hline
% \end{tabular}
% \caption{\projecttitle{} compared with TEE-hosted applications.}\label{table:qualitative_comparison}
% \vspace{-3mm}
% \end{table}

\subsection{\revcont{FPGA Resource Usage}}
\rev{(d), E2}{Lastly, we perform a resource utilization analysis to show \projecttitle{}'s scalability capabilities. We measure the resource consumption of \projecttitle{}'s primary hardware components~\cite{easynet} and estimate maximum connections on the latest FPGA. }
% , e.g., how many instances of the attestation kernel as well as the RoCE can fit on the same FPGA.

\revcont{
Table~\ref{tab:resource-utilization} shows the resource consumption details. The overall \projecttitle{} design consumes 16.6\% of LUTs, 16.3\% of Flip-Flops (FF), and 16.6\% of RAMB36 (3.46~\% of the entire on-chip memory) on the U280 FPGA. Note that \projecttitle{} only requires commodity FPGA NIC designs to add the attestation kernel, whose utilization is comparable with the other modules, XDMA and RoCE. 
% , which occupies 15.7\% of LUTs, 13.4\% of FFs, and 24.1\% of RAMB36 compared to the entire design.
}

\revcont{
Figure~\ref{fig:scalability} shows the scaling capabilities of \projecttitle{} hardware. 
% The XDMA module, CMAC module, and RoCE kernel are single for each in the design because the XDMA and CMAC modules are independent of the number of connections and the RoCE kernel is configured to hold 500 queue pairs to establish the same number of network connections~\cite{storm}. 
As the number of network connections increases, we only need to replicate the attestation kernel because the XDMA and CMAC modules are independent of the number of connections, and the RoCE kernel is configured to hold up to 500 connections~\cite{storm}. The result demonstrates that \projecttitle{} can support up to 32 concurrent connections on a single U280 FPGA. 
%  to sustain the throughput per connection
% 500 queue pairs to establish the same number of
}
% \rev{E2}{Our implementation is based on~\cite{coyote}, a fork of which has been used in prior works~\cite{storm} showing that 500 queue pairs (QPs) occupy 9\% of the on-chip memory, while the logic resource usage remains below 1\% when scaling from 500 to 16,000 QPs. Our evaluation and modern deployments use more powerful FPGAs, suggesting that even a larger number of connections could be supported compared to the work in~\cite{storm}. }

\newcolumntype{C}[1]{>{\centering\arraybackslash}p{#1}}
\newcolumntype{L}[1]{>{\raggedright\arraybackslash}p{#1}}
\newcolumntype{R}[1]{>{\raggedleft\arraybackslash}p{#1}}

\begin{table}[t]
    \small
    \centering
    \setlength\tabcolsep{4.2pt}
    \begin{tabular}{C{14mm}R{10mm}R{7mm}R{10mm}R{7mm}R{6mm}R{7mm}}
    \hline
        Name & \multicolumn{2}{c}{LUT (\%)} & \multicolumn{2}{c}{FF (\%)} & \multicolumn{2}{c}{RAMB36 (\%)} \\ \cmidrule(lr){1-1} \cmidrule(lr){2-3} \cmidrule(lr){4-5} \cmidrule(lr){6-7}
        U280               & 1303680 &  (100) & 2607360 &  (100) & 2016 &  (100) \\ \cmidrule(lr){1-1} \cmidrule(lr){2-3} \cmidrule(lr){4-5} \cmidrule(lr){6-7}
        TNIC               &  216905 & (16.6) &  423891 & (16.3) &  335 & (16.6) \\               
        XDMA               &   48258 &  (3.7) &   50701 &  (1.9) &   64 &  (3.1) \\ 
        Att. kernel        &   34138 &  (2.6) &   56914 &  (2.2) &   81 &  (4.0) \\ 
        RoCE               &   30379 &  (2.3) &   75804 &  (2.9) &   46 &  (2.3) \\ 
        CMAC               &    1484 &  (0.1) &    3433 &  (0.1) &    0 &  (0.0) \\ \hline
    \end{tabular}
    \caption{\rev{(d)}{\projecttitle{}'s resource usage. The relative (\%) compares with the U280 FPGA capacity. \projecttitle{} means the entire design.}} 
    % available resources on the 
    \label{tab:resource-utilization}
    \vspace{-2mm}
\end{table}


% Nevertheless, our \projecttitle{} does not assume a specific FPGA board. Therefore, the findings from previous works on other boards are still relevant.


\if 0
\begin{figure}[t!]
    \centering
    \includegraphics[width=\linewidth]{atc-submission-plots/bftpr_lat_throughput.pdf} 
    \caption{Throughput-latency evaluation of a BFT version of PR using various trusted components.} \label{fig:lat_throughput_kernel}
\end{figure}
\fi
\subsection{\revcont{Discussion}}
\myparagraph{\rev{B2}{\projecttitle{}'s applicability}}
\revcont{As FPGA-based SmartNICs are widely adopted by major cloud providers for hardware acceleration~\cite{211249}, we believe that \projecttitle{} has the potential for broader industry application. In addition,  ASIC-based NICs can also provide the same functionalities by integrating \projecttitle{}'s hardware modules into an optimized System-on-Chip (SoC).}
%\revcont{As FPGA-based SmartNICs are widely adopted by major cloud providers for hardware acceleration~\cite{211249}, we believe that \projecttitle{} has the potential for broader industry application. In addition, \projecttitle{}’s APIs are generic and CPU-agnostic, making them more suitable for cloud environments than programming heterogeneous TEEs.}

%\revcont{\projecttitle{} demonstrates its minimalistic root of trust, which suffices for building BFT distributed systems. 
%ASIC-based NICs can also provide the same functionalities by integrating \projecttitle{}'s hardware modules into an optimized System-on-Chip (SoC). }
%which would be more optimized than the FPGA-based approach

\myparagraph{\rev{C1}{Use cases}}
\revcont{The paper deliberately focuses on distributed cloud applications as \projecttitle{}'s primary use cases. Trust in shared third-party clouds is a more critical concern than in other environments, posing unique challenges in trust, performance, and scalability. While the current scope is specific, the underlying principles could extend to other use cases, such as HPC or on-premise computing.}
%However,  \projecttitle{} addresses these using SmartNICs, which are already widely deployed in modern clouds~\cite{211249}. 


\myparagraph{\rev{D3}{Message drops}}
\revcont{
\projecttitle{} guarantees packet retransmission between two correct nodes until their successful reception  extending a RoCE implementation that supports reliable operations. The application need not re-send the message as it receives a different sequence number, which is not accepted (or verified) by the remote \projecttitle{} until all preceding messages have been received. %Instead, \projecttitle{} extends a RoCE implementation that supports reliable operations where the remote side acknowledges the receipt of a message using specific ACK messages. 
% \projecttitle{} guarantees packet retransmission between two correct nodes until their successful reception. The application need not re-send the message as it would receive a different sequence number, which would not be accepted (or verified) by the remote \projecttitle{} until all preceding messages have been received. Instead, \projecttitle{} extends a RoCE implementation that supports reliable operations where the receipt of a message is acknowledged by the remote side using specific ACK messages.
}

\myparagraph{\rev{D4, E3}{View-change and recovery}}
\revcont{Detailing view-change and recovery in \projecttitle{} protocols are beyond the scope of our work. \projecttitle{} could adopt similar techniques as in TrInc~\cite{trinc} without disrupting these operations. In a new leader's election, replicas can establish new connections with new identifiers. As such, previous connections will not block execution, and state transfers, e.g., view-change, can be performed effectively. } 

%\revcont{Detailing view-change and recovery in \projecttitle{} protocols are beyond the scope of this paper, while \projecttitle{} could adopt techniques such as the ones used by TrInc~\cite{trinc}. Importantly, \projecttitle{} does not interfere with these operations. During a new leader's election, replicas can establish new connections with new connection identifiers. As a result, previous connections will not block execution, and state transfers (e.g., view-change) can be carried out effectively. } 
% \rev{D4, E3}{We do not implement view-change or recovery in \projecttitle{} protocols, while \projecttitle{} does not interfere with these operations. During a new leader's election, replicas can establish new connections with new connection identifiers. As a result, previous connections will not block execution, and state transfers (e.g., view-change) can be carried out effectively. For recovery, \projecttitle{} could adopt techniques such as the ones used by TrInc~\cite{trinc}. } 

% \myparagraph{\rev{E1}{Impact of RDMA on security}}
\myparagraph{\rev{E1}{PCIe transaction encryption}}
\revcont{\projecttitle{} encrypts PCIe transactions for CPU-to-device communication, allowing attackers to modify the PCIe transactions. This vulnerability is not unique to \projecttitle{}; it applies to any network stack, including the OS-based ones, since the \emph{untrusted} OS drives PCIe transactions. }

\section{Related work}
%\pramod{the related work bucket should not start with individual papers. It should first summarize the field/area/bucket area, and then give example systems, and lastly, it should end with an overall comparison of the the research field with our solution. In this way, you avoid direct comparison with individual work. Can you please re-write it?}

\myparagraph{Trustworthy distributed systems} Classical BFT systems~\cite{Castro:2002, Suri_Payer_2021, DBLP:journals/corr/abs-1803-05069, Chan2018PaLaAS, DBLP:journals/corr/abs-1807-04938, Chan2018PiLiAE, bft-smart, 6681599} provide BFT guarantees at the cost of high complexity, performance, and scalability overheads. \projecttitle{} bridges the gap between BFT and prior limitations, designing a {\em silicon root-of-trust} with generic trusted networking abstractions that materialize the BFT security properties.

\myparagraph{Trusted hardware for distributed systems} Trustworthy systems~\cite{10.1145/3492321.3519568, minBFT, 10.1145/3552326.3587455, 10.1145/3492321.3519568, treaty, avocado, ccf} leverage trusted hardware to optimize the performance of classical BFT at the cost of generalization and easy adoption. The systems suffer from high latencies (50us---105ms)~\cite{levin2009trinc, 10.1145/2168836.2168866}, build large TCBs~\cite{treaty, avocado}, and rely on specific TEEs~\cite{minBFT, hybster}. In contrast, our \projecttitle{} aims to offer performance and generality, while our minimalistic TCB is verifiable and unified in the heterogeneous cloud. 

\begin{figure}
    \centering
    \includegraphics[width=0.9\linewidth]{eval-plots/plots/hw_eval_attest_util.pdf}
    \vspace{-10pt}
  \caption{\rev{(d)}{The scalability analysis of \projecttitle{} hardware. The resource usage is normalized to the U280 FPGA capacity.} }
  % \caption{Latency evaluation of send operations for various packet sizes across five competitive network stacks with various security properties.}
    % \felix{Figure: BRAM = RAMB36 (including RAMB18) + URAM. "CARRY8" is the 4th most used resource of the attestation kernel and is used for "Others" as an upper bound for all other resources.}
    % More resource utilization details: https://github.com/dgiantsidi/replication-protos/blob/main/plots/asplos_submission/attestation_kernel_util.md
    \label{fig:scalability}
   %\vspace{-2pt}
\end{figure}

\myparagraph{SmartNIC-assisted systems} Networked systems offer fast network operations with emerging (programmable) SmartNIC devices~\cite{liquidIO_smartnics,u280_smartnics,bluefield_smartnics,broadcom_smartnics,netronome_smartnics,alibaba_smartnics,nitro_smartnics,msr_smartnics}. Some of them offload the network functions to the hardware and reduce the host processing and energy overheads~\cite{246498,211249,10.1145/3387514.3405895,10.1145/3365609.3365851,10.1145/3127479.3132252,258971,246486,179716,227655,10.1145/3286062.3286068,shan2022supernic,10.1145/3390251.3390257} or re-design generic networking protocols, from RDMA/RoCE to TCP/IP network stacks, on top of FPGA-based SmartNICs for performance~\cite{coyote,corundum,storm,8891991,280712,9114811,opennic_project}. Others~\cite{10.1145/3341302.3342079,10.1145/2872362.2872367,234944,9220629,6853195,10292786,10.1145/3477132.3483555,280678,10.1145/3132747.3132756,honeycomb,288659,10329593} build generic execution frameworks to optimize various distributed systems. Our \projecttitle{} follows a similar approach by building a high-performant unified network stack with SmartNICs and extending its security semantics with the properties of non-equivocation and transferable authentication.% and offloading security to the NIC hardware.


%\myparagraph{SmartNIC-assisted network stacks} SmartNICs effectively provide high-performance network stacks. Another line of research~\cite{coyote, corundum, storm, 8891991, 280712, 9114811, opennic_project} re-designs generic networking protocols, from RDMA/RoCE to TCP/IP network stacks, on top of FPGA-based SmartNICs for performance. Our \projecttitle{} further extends its security semantics with the properties of non-equivocation and transferable authentication. 

\myparagraph{Programmable HW for network security} Programmable hardware, SmartNICs, and switches are used to shield networking. Recent systems~\cite{10.1145/3603269.3604874, 10.1145/3620678.3624786, 10.1145/3563647.3563654, 10.1145/3321408.3323087, 278292} leverage programmable switches and FPGAs to offload security processing and boost performance in the context of blockchain systems~\cite{10.1145/3603269.3604874} or security functions (e.g., access control, DNS traffic inspection)~\cite{10.1145/3620678.3624786, 10.1145/3563647.3563654, 10.1145/3321408.3323087}. Our \projecttitle{} similarly offloads security into the hardware, but it carefully uses SmartNICs to overcome the processing bottlenecks of the switches. 






%\section{Related Work}
\label{sec:formatting}
XAI methods for computer vision can be categorized by two axes: local (heatmap-based) vs. global (concept-based) approaches, and \textit{post-hoc} vs. \textit{ante-hoc} explanations.
In this section, we overview existing methods along these lines.
\\
\noindent \textbf{Heatmap-based explainability.}
This refers to a family of \textit{post-hoc} explainability techniques, often called attribution methods, that visualize the parts of the input image that contribute most to the model's output.
\textit{Gradient-based methods} generate explainable heatmaps by backpropagating gradients with respect to the input of each layer.
Some of these methods, such as FullGrad~\cite{srinivas2019full}, are class-agnostic as they produce roughly identical results regardless of the output class \cite{sundararajan2017axiomatic, smilkov2017smoothgrad}, while others, such as GradCAM~\cite{selvaraju2017grad}, generate class-dependent heatmaps~\cite{simonyan2013deep, chattopadhay2018grad}. This property is essential when the true class is ambiguous.
While widely used, their main drawback is high sensitivity to gradient noise, which may render their outcomes impractical~\cite{adebayo2018sanity}. To address this issue, some Class Activation Maps (CAM) methods~\cite{zhou2016learning}, such as ScoreCAM~\cite{wang2020score}, produce gradient-free explanation maps.

\textit{Attribution propagation methods} decompose the output of a model into the contributions of its layers by propagating “relevance” in a recursive manner, without exclusively relying on gradients. Common attribution propagation methods, such as Layer-wise Relevance Propagation (LRP)~\cite{binder2016layer}, are primarily applicable to Convolutional Neural Networks (CNNs)~\cite{montavon2017explaining, shrikumar2017learning, zhang2018top}. Later approaches have been adapted to accommodate vision transformers (ViTs)~\cite{dosovitskiy2020image}, exploiting their built-in self-attention mechanism~\cite{voita2019analyzing, abnar2020quantifying, chefer2021transformer, radford2021CLIP}.
% We note that, unlike our SALF-CBM, both gradient-based and attribution propagation methods do not provide concept-based explanations. Additionally, since these are \textit{post-hoc} techniques, they do not enable test-time user intervention.
We note that, unlike our SALF-CBM, both gradient-based and attribution propagation methods do not provide concept-based explanations. Additionally, since these are \textit{post-hoc} techniques, they do not enable test-time user intervention.

%-------------------------------------------------------------------------
\noindent \textbf{Concept-based explainability.}
An alternative way of explaining vision models is by using human-interpretable concepts. Various methods provide such explanations in a \textit{post-hoc} manner.
For example, Testing Concept Activation Vectors (TCAV)~\cite{kim2018interpretability} measures the importance of user-defined concepts to the model's prediction by training a linear classifier to distinguish between concepts in their activation space. However, this requires labeling images with their corresponding concepts in advance.
ACE~\cite{ghorbani2019towards} extends this idea by applying multi-resolution segmentation to images from the same class, followed by clustering similar segments into concepts to compute their TCAV scores. Similarly, Invertible Concept Embeddings (ICE)~\cite{zhang2021invertible} and Concept Recursive Activation Factorization (CRAFT)~\cite{fel2023craft} provide concept-based explanations using matrix factorization of feature maps. CRAFT also generates attribution maps that localize concepts in the input image.
These methods, however, are mostly applicable to CNN architectures~\cite{ghorbani2019towards}, which use non-negative activations~\cite{zhang2021invertible, fel2023craft}, and therefore cannot be directly applied to other types of models. Additionally, since they provide \textit{post-hoc} explanations, they do not enable test-time user intervention.

In contrast, \textit{Concept-Bottleneck Models (CBMs)} is a family of \textit{ante-hoc} interpretable models whose explainability mechanism is an integral part of the model itself. CBMs operate by introducing a concept-bottleneck layer into pre-trained models, before the final prediction layer. The goal of this bottleneck is to project features into an interpretable concept space, where each neuron corresponds to a single concept. 
Unlike \textit{post-hoc} methods, the output of CBMs is directly based on interpretable concepts, making them easily explainable and allowing user intervention by modifying concept neurons activations. 
In the original CBM work~\cite{ koh2020concept}, the concept bottleneck layer was trained using manual concept annotations, limiting its ability to scale to large datasets. 
Recently, Post-Hoc CBM (P-CBM)~\cite{yuksekgonul2022post} and Label-Free CBM (LF-CBM)~\cite{oikarinen2023label} addressed this issue by leveraging CLIP to assign concept scores for training images, thus not requiring concept annotations. LF-CBM also presented an automatic process for creating a list of task-relevant concepts using GPT-3. While showing good interpretability results, both P-CBM and LF-CBM present a performance drop on the final classification task compared to the original (non-CBM) model. Additionally, unlike our SALF-CBM, these methods are limited to global concept explanations, and are unable to localize these concepts within the image.
\section{Conclusion}
In this work, we propose a simple yet effective approach, called SMILE, for graph few-shot learning with fewer tasks. Specifically, we introduce a novel dual-level mixup strategy, including within-task and across-task mixup, for enriching the diversity of nodes within each task and the diversity of tasks. Also, we incorporate the degree-based prior information to learn expressive node embeddings. Theoretically, we prove that SMILE effectively enhances the model's generalization performance. Empirically, we conduct extensive experiments on multiple benchmarks and the results suggest that SMILE significantly outperforms other baselines, including both in-domain and cross-domain few-shot settings.
% LaTeX template for Artifact Evaluation V20201122
%
% Prepared by 
% * Grigori Fursin (cTuning foundation, France) 2014-2020
% * Bruce Childers (University of Pittsburgh, USA) 2014
%
% See examples of this Artifact Appendix in
%  * SC'17 paper: https://dl.acm.org/citation.cfm?id=3126948
%  * CGO'17 paper: https://www.cl.cam.ac.uk/~sa614/papers/Software-Prefetching-CGO2017.pdf
%  * ACM ReQuEST-ASPLOS'18 paper: https://dl.acm.org/citation.cfm?doid=3229762.3229763
%
% (C)opyright 2014-2020
%
% CC BY 4.0 license
%

%\documentclass[sigplan,twocolumn]{acmart}

%\usepackage{hyperref}

%\newcommand{\projecttitle}{\textsc{tnic}\xspace}
%\newcommand{\scone}{\textsc{scone}\xspace}

%\begin{document}


%\special{papersize=8.5in,11in}

%%%%%%%%%%%%%%%%%%%%%%%%%%%%%%%%%%%%%%%%%%%%%%%%%%%%
% When adding this appendix to your paper, 
% please remove above part
%%%%%%%%%%%%%%%%%%%%%%%%%%%%%%%%%%%%%%%%%%%%%%%%%%%%

\appendix
\section{Artifact Appendix}

%%%%%%%%%%%%%%%%%%%%%%%%%%%%%%%%%%%%%%%%%%%%%%%%%%%%%%%%%%%%%%%%%%%%%
\subsection{Abstract}
Our artifacts include the \projecttitle{} codebase as well as the software artifact with the four \projecttitle{} applications, i.e., A2M, BFT, CR, and PeerReview. In addition, we provide the codebases of all the microbenchmarks we discuss in the paper including those of the TEE-based systems. Lastly, we attach the security proofs of \projecttitle{} system operations and attestation protocol based on Tamarin~\cite{tamarin-prover}. This appendix provides the necessary information to set up, build, and run the experiments we present in the paper.



\subsection{Artifact check-list (meta-information)}

%{\em Obligatory. Use just a few informal keywords in all fields %applicable to your artifacts
%and remove the rest. This information is needed to find appropriate reviewers and gradually 
%unify artifact meta information in Digital Libraries.}

{\small
\begin{itemize}
 % \item {\bf Algorithm: }
  \item {\bf Program:} \projecttitle{} hardware implementation codebase. \projecttitle{} software codebases that include the systems where \projecttitle{} has been applied (run in emulated hardware) and microbenchmarks (e.g., network benchmark). \projecttitle{}'s security proofs based on Tamarin~\cite{tamarin-prover}.
  \item {\bf Compilation:} Requires Vitis HLS~\cite{vitis-hls}, Vivado~\cite{vivado}, CMake, C++, Boost, eRPC~\cite{erpc}, DPDK~\cite{dpdk}, Tamarin~\cite{tamarin-prover}.
  %\item {\bf Transformations: }
  %\item {\bf Binary: }
  %\item {\bf Model: }
  %\item {\bf Data set: }
  \item {\bf Run-time environment:} Requires NixOS, 5.15.4, {\sc scone}~\cite{scone} (for SGX-based experiments).
  \item {\bf Hardware:} Requires Alveo U280 cards~\cite{u280_smartnics}, Intel(R) Core(TM) i9-9900K with Intel Corporation Ethernet Controllers (XL710) (or any other DPDK compatible NIC) and AMD EPYC 7413. 
  % \item {\bf Run-time state: }
  \item {\bf Execution:} The time of the experiments are configurable. Each of our experiments did not take more than 10 minutes. However, the compilation and synthesis phases of the \projecttitle{} hardware implementation might take up to 4 hours. 
  \item {\bf Metrics:} Throughput and latency
  %\item {\bf Output: }
  % \item {\bf Experiments:} 
  %\item {\bf How much disk space required (approximately)?: }
  %\item {\bf How much time is needed to prepare workflow (approximately)?: }
  %\item {\bf How much time is needed to complete experiments (approximately)?: }
  \item {\bf Publicly available:} Yes.
  \item {\bf Code licenses:} MIT License. \projecttitle{} doesn't use any external license.
  %\item {\bf Data licenses (if publicly available)?: }
  %\item {\bf Workflow framework used?: }
  \item {\bf Archived (DOI):} \url{10.5281/zenodo.14775354}
\end{itemize}
}

%%%%%%%%%%%%%%%%%%%%%%%%%%%%%%%%%%%%%%%%%%%%%%%%%%%%%%%%%%%%%%%%%%%%%
\subsection{Description}

\subsubsection{How to access}

The open-source version of the \projecttitle{} codebase can be found on GitHub at the following address:

\url{https://github.com/TUM-DSE/TNIC-main.git}

%{\em Obligatory}

\subsubsection{Hardware dependencies}
For AMD-SEV and \projecttitle{}-hardware setups, you need three machines with AMD EPYC 7413 CPU. Each machine is equipped with an Alveo U280 card~\cite{u280_smartnics} and one of every U280's QSFP28 ports connects to the 100Gbps network. For Intel SGX setups, you need machines with Intel(R) Core(TM) i9-9900K with Intel Corporation Ethernet Controllers (XL710) (or any other DPDK compatible NIC) for network connection. 

\subsubsection{Software dependencies}
The software build process involves building the low-level 
Linux kernel driver and the high-level user application layers.
All codebases run on top of NixOS, 5.15.4. We provide the appropriate \texttt{.nix} files to set up a \texttt{nix-shell} environment with all necessary system dependencies. 

The code has been built with \texttt{Makefile} and \texttt{cmake}. The applications, as well as the TEE-based code and application layer, are  written in C++17. We depend on Boost libraries and gflgas for
the parsing of the command line arguments. We rely on several other dependencies, which we explain in our README files, including; {\sc scone}~\cite{scone} for SGX-based experiments, Vivado~\cite{vivado} and Vitis HLS~\cite{vitis-hls} for building \projecttitle{} hardware, eRPC~\cite{erpc}, DPDK~\cite{dpdk}, and Tamarin~\cite{tamarin-prover}.

%\subsubsection{Data sets}

%\subsubsection{Models}

%%%%%%%%%%%%%%%%%%%%%%%%%%%%%%%%%%%%%%%%%%%%%%%%%%%%%%%%%%%%%%%%%%%%%
\subsection{Installation}
The artifact is linked to the repository as submodules. Each repository provides analytical instructions in their \texttt{README.md} files of how to build and run the binaries.

To build the \projecttitle{}'s hardware implementation, please follow the instructions provided in ~\cite{build-hardware-for-fpga}.

To build the software including the driver and the benchmarks, please follow the instructions in~\cite{build-software}.

To run the experiments for the \projecttitle{} hardware implementation, you need to first load the \projecttitle{}'s kernel module and then run the compiled binary. Detailed instructions are available in ~\cite{tnic-run}.  

Similar instructions have been documented for the applications~\cite{tnic-apps} and the security proofs~\cite{tnic-proofs}.

%{\em Obligatory}

%%%%%%%%%%%%%%%%%%%%%%%%%%%%%%%%%%%%%%%%%%%%%%%%%%%%%%%%%%%%%%%%%%%%%
%\subsection{Experiment workflow}

%%%%%%%%%%%%%%%%%%%%%%%%%%%%%%%%%%%%%%%%%%%%%%%%%%%%%%%%%%%%%%%%%%%%%
\subsection{Evaluation and expected results}
Each of the experiments will output information about its progress; this is a hint that the script is still
running and hasn’t halted. The output of the experiment reports important measurements about the execution. The results are expected not to vary significantly (less than 5$\%$) when compared to the results presented in the paper. However, as discussed, we observed quite a significant variance in some TEE-based systems (Intel SGX and AMD-SEV).
%{\em Obligatory}

%%%%%%%%%%%%%%%%%%%%%%%%%%%%%%%%%%%%%%%%%%%%%%%%%%%%%%%%%%%%%%%%%%%%%
%\subsection{Experiment customization}
%A wide variety of experiemnt customization can be available
%through different execution parameters. The users can cre-
%ate different versions of the system through combinations of
%vFPGAs, network and memory stacks.
%%%%%%%%%%%%%%%%%%%%%%%%%%%%%%%%%%%%%%%%%%%%%%%%%%%%%%%%%%%%%%%%%%%%%
%\subsection{Notes}

%%%%%%%%%%%%%%%%%%%%%%%%%%%%%%%%%%%%%%%%%%%%%%%%%%%%%%%%%%%%%%%%%%%%%
\subsection{Methodology}

Submission, reviewing, and badging methodology:

\begin{itemize}
  \item \url{https://www.acm.org/publications/policies/artifact-review-badging}
  \item \url{http://cTuning.org/ae/submission-20201122.html}
  \item \url{http://cTuning.org/ae/reviewing-20201122.html}
\end{itemize}

%%%%%%%%%%%%%%%%%%%%%%%%%%%%%%%%%%%%%%%%%%%%%%%%%%%%
% When adding this appendix to your paper, 
% please remove below part
%%%%%%%%%%%%%%%%%%%%%%%%%%%%%%%%%%%%%%%%%%%%%%%%%%%%

%\bibliographystyle{ACM-Reference-Format}
%\bibliography{sample}
%\end{document}

% \if 0
% \newpage

% \appendix
\subsection{Lloyd-Max Algorithm}
\label{subsec:Lloyd-Max}
For a given quantization bitwidth $B$ and an operand $\bm{X}$, the Lloyd-Max algorithm finds $2^B$ quantization levels $\{\hat{x}_i\}_{i=1}^{2^B}$ such that quantizing $\bm{X}$ by rounding each scalar in $\bm{X}$ to the nearest quantization level minimizes the quantization MSE. 

The algorithm starts with an initial guess of quantization levels and then iteratively computes quantization thresholds $\{\tau_i\}_{i=1}^{2^B-1}$ and updates quantization levels $\{\hat{x}_i\}_{i=1}^{2^B}$. Specifically, at iteration $n$, thresholds are set to the midpoints of the previous iteration's levels:
\begin{align*}
    \tau_i^{(n)}=\frac{\hat{x}_i^{(n-1)}+\hat{x}_{i+1}^{(n-1)}}2 \text{ for } i=1\ldots 2^B-1
\end{align*}
Subsequently, the quantization levels are re-computed as conditional means of the data regions defined by the new thresholds:
\begin{align*}
    \hat{x}_i^{(n)}=\mathbb{E}\left[ \bm{X} \big| \bm{X}\in [\tau_{i-1}^{(n)},\tau_i^{(n)}] \right] \text{ for } i=1\ldots 2^B
\end{align*}
where to satisfy boundary conditions we have $\tau_0=-\infty$ and $\tau_{2^B}=\infty$. The algorithm iterates the above steps until convergence.

Figure \ref{fig:lm_quant} compares the quantization levels of a $7$-bit floating point (E3M3) quantizer (left) to a $7$-bit Lloyd-Max quantizer (right) when quantizing a layer of weights from the GPT3-126M model at a per-tensor granularity. As shown, the Lloyd-Max quantizer achieves substantially lower quantization MSE. Further, Table \ref{tab:FP7_vs_LM7} shows the superior perplexity achieved by Lloyd-Max quantizers for bitwidths of $7$, $6$ and $5$. The difference between the quantizers is clear at 5 bits, where per-tensor FP quantization incurs a drastic and unacceptable increase in perplexity, while Lloyd-Max quantization incurs a much smaller increase. Nevertheless, we note that even the optimal Lloyd-Max quantizer incurs a notable ($\sim 1.5$) increase in perplexity due to the coarse granularity of quantization. 

\begin{figure}[h]
  \centering
  \includegraphics[width=0.7\linewidth]{sections/figures/LM7_FP7.pdf}
  \caption{\small Quantization levels and the corresponding quantization MSE of Floating Point (left) vs Lloyd-Max (right) Quantizers for a layer of weights in the GPT3-126M model.}
  \label{fig:lm_quant}
\end{figure}

\begin{table}[h]\scriptsize
\begin{center}
\caption{\label{tab:FP7_vs_LM7} \small Comparing perplexity (lower is better) achieved by floating point quantizers and Lloyd-Max quantizers on a GPT3-126M model for the Wikitext-103 dataset.}
\begin{tabular}{c|cc|c}
\hline
 \multirow{2}{*}{\textbf{Bitwidth}} & \multicolumn{2}{|c|}{\textbf{Floating-Point Quantizer}} & \textbf{Lloyd-Max Quantizer} \\
 & Best Format & Wikitext-103 Perplexity & Wikitext-103 Perplexity \\
\hline
7 & E3M3 & 18.32 & 18.27 \\
6 & E3M2 & 19.07 & 18.51 \\
5 & E4M0 & 43.89 & 19.71 \\
\hline
\end{tabular}
\end{center}
\end{table}

\subsection{Proof of Local Optimality of LO-BCQ}
\label{subsec:lobcq_opt_proof}
For a given block $\bm{b}_j$, the quantization MSE during LO-BCQ can be empirically evaluated as $\frac{1}{L_b}\lVert \bm{b}_j- \bm{\hat{b}}_j\rVert^2_2$ where $\bm{\hat{b}}_j$ is computed from equation (\ref{eq:clustered_quantization_definition}) as $C_{f(\bm{b}_j)}(\bm{b}_j)$. Further, for a given block cluster $\mathcal{B}_i$, we compute the quantization MSE as $\frac{1}{|\mathcal{B}_{i}|}\sum_{\bm{b} \in \mathcal{B}_{i}} \frac{1}{L_b}\lVert \bm{b}- C_i^{(n)}(\bm{b})\rVert^2_2$. Therefore, at the end of iteration $n$, we evaluate the overall quantization MSE $J^{(n)}$ for a given operand $\bm{X}$ composed of $N_c$ block clusters as:
\begin{align*}
    \label{eq:mse_iter_n}
    J^{(n)} = \frac{1}{N_c} \sum_{i=1}^{N_c} \frac{1}{|\mathcal{B}_{i}^{(n)}|}\sum_{\bm{v} \in \mathcal{B}_{i}^{(n)}} \frac{1}{L_b}\lVert \bm{b}- B_i^{(n)}(\bm{b})\rVert^2_2
\end{align*}

At the end of iteration $n$, the codebooks are updated from $\mathcal{C}^{(n-1)}$ to $\mathcal{C}^{(n)}$. However, the mapping of a given vector $\bm{b}_j$ to quantizers $\mathcal{C}^{(n)}$ remains as  $f^{(n)}(\bm{b}_j)$. At the next iteration, during the vector clustering step, $f^{(n+1)}(\bm{b}_j)$ finds new mapping of $\bm{b}_j$ to updated codebooks $\mathcal{C}^{(n)}$ such that the quantization MSE over the candidate codebooks is minimized. Therefore, we obtain the following result for $\bm{b}_j$:
\begin{align*}
\frac{1}{L_b}\lVert \bm{b}_j - C_{f^{(n+1)}(\bm{b}_j)}^{(n)}(\bm{b}_j)\rVert^2_2 \le \frac{1}{L_b}\lVert \bm{b}_j - C_{f^{(n)}(\bm{b}_j)}^{(n)}(\bm{b}_j)\rVert^2_2
\end{align*}

That is, quantizing $\bm{b}_j$ at the end of the block clustering step of iteration $n+1$ results in lower quantization MSE compared to quantizing at the end of iteration $n$. Since this is true for all $\bm{b} \in \bm{X}$, we assert the following:
\begin{equation}
\begin{split}
\label{eq:mse_ineq_1}
    \tilde{J}^{(n+1)} &= \frac{1}{N_c} \sum_{i=1}^{N_c} \frac{1}{|\mathcal{B}_{i}^{(n+1)}|}\sum_{\bm{b} \in \mathcal{B}_{i}^{(n+1)}} \frac{1}{L_b}\lVert \bm{b} - C_i^{(n)}(b)\rVert^2_2 \le J^{(n)}
\end{split}
\end{equation}
where $\tilde{J}^{(n+1)}$ is the the quantization MSE after the vector clustering step at iteration $n+1$.

Next, during the codebook update step (\ref{eq:quantizers_update}) at iteration $n+1$, the per-cluster codebooks $\mathcal{C}^{(n)}$ are updated to $\mathcal{C}^{(n+1)}$ by invoking the Lloyd-Max algorithm \citep{Lloyd}. We know that for any given value distribution, the Lloyd-Max algorithm minimizes the quantization MSE. Therefore, for a given vector cluster $\mathcal{B}_i$ we obtain the following result:

\begin{equation}
    \frac{1}{|\mathcal{B}_{i}^{(n+1)}|}\sum_{\bm{b} \in \mathcal{B}_{i}^{(n+1)}} \frac{1}{L_b}\lVert \bm{b}- C_i^{(n+1)}(\bm{b})\rVert^2_2 \le \frac{1}{|\mathcal{B}_{i}^{(n+1)}|}\sum_{\bm{b} \in \mathcal{B}_{i}^{(n+1)}} \frac{1}{L_b}\lVert \bm{b}- C_i^{(n)}(\bm{b})\rVert^2_2
\end{equation}

The above equation states that quantizing the given block cluster $\mathcal{B}_i$ after updating the associated codebook from $C_i^{(n)}$ to $C_i^{(n+1)}$ results in lower quantization MSE. Since this is true for all the block clusters, we derive the following result: 
\begin{equation}
\begin{split}
\label{eq:mse_ineq_2}
     J^{(n+1)} &= \frac{1}{N_c} \sum_{i=1}^{N_c} \frac{1}{|\mathcal{B}_{i}^{(n+1)}|}\sum_{\bm{b} \in \mathcal{B}_{i}^{(n+1)}} \frac{1}{L_b}\lVert \bm{b}- C_i^{(n+1)}(\bm{b})\rVert^2_2  \le \tilde{J}^{(n+1)}   
\end{split}
\end{equation}

Following (\ref{eq:mse_ineq_1}) and (\ref{eq:mse_ineq_2}), we find that the quantization MSE is non-increasing for each iteration, that is, $J^{(1)} \ge J^{(2)} \ge J^{(3)} \ge \ldots \ge J^{(M)}$ where $M$ is the maximum number of iterations. 
%Therefore, we can say that if the algorithm converges, then it must be that it has converged to a local minimum. 
\hfill $\blacksquare$


\begin{figure}
    \begin{center}
    \includegraphics[width=0.5\textwidth]{sections//figures/mse_vs_iter.pdf}
    \end{center}
    \caption{\small NMSE vs iterations during LO-BCQ compared to other block quantization proposals}
    \label{fig:nmse_vs_iter}
\end{figure}

Figure \ref{fig:nmse_vs_iter} shows the empirical convergence of LO-BCQ across several block lengths and number of codebooks. Also, the MSE achieved by LO-BCQ is compared to baselines such as MXFP and VSQ. As shown, LO-BCQ converges to a lower MSE than the baselines. Further, we achieve better convergence for larger number of codebooks ($N_c$) and for a smaller block length ($L_b$), both of which increase the bitwidth of BCQ (see Eq \ref{eq:bitwidth_bcq}).


\subsection{Additional Accuracy Results}
%Table \ref{tab:lobcq_config} lists the various LOBCQ configurations and their corresponding bitwidths.
\begin{table}
\setlength{\tabcolsep}{4.75pt}
\begin{center}
\caption{\label{tab:lobcq_config} Various LO-BCQ configurations and their bitwidths.}
\begin{tabular}{|c||c|c|c|c||c|c||c|} 
\hline
 & \multicolumn{4}{|c||}{$L_b=8$} & \multicolumn{2}{|c||}{$L_b=4$} & $L_b=2$ \\
 \hline
 \backslashbox{$L_A$\kern-1em}{\kern-1em$N_c$} & 2 & 4 & 8 & 16 & 2 & 4 & 2 \\
 \hline
 64 & 4.25 & 4.375 & 4.5 & 4.625 & 4.375 & 4.625 & 4.625\\
 \hline
 32 & 4.375 & 4.5 & 4.625& 4.75 & 4.5 & 4.75 & 4.75 \\
 \hline
 16 & 4.625 & 4.75& 4.875 & 5 & 4.75 & 5 & 5 \\
 \hline
\end{tabular}
\end{center}
\end{table}

%\subsection{Perplexity achieved by various LO-BCQ configurations on Wikitext-103 dataset}

\begin{table} \centering
\begin{tabular}{|c||c|c|c|c||c|c||c|} 
\hline
 $L_b \rightarrow$& \multicolumn{4}{c||}{8} & \multicolumn{2}{c||}{4} & 2\\
 \hline
 \backslashbox{$L_A$\kern-1em}{\kern-1em$N_c$} & 2 & 4 & 8 & 16 & 2 & 4 & 2  \\
 %$N_c \rightarrow$ & 2 & 4 & 8 & 16 & 2 & 4 & 2 \\
 \hline
 \hline
 \multicolumn{8}{c}{GPT3-1.3B (FP32 PPL = 9.98)} \\ 
 \hline
 \hline
 64 & 10.40 & 10.23 & 10.17 & 10.15 &  10.28 & 10.18 & 10.19 \\
 \hline
 32 & 10.25 & 10.20 & 10.15 & 10.12 &  10.23 & 10.17 & 10.17 \\
 \hline
 16 & 10.22 & 10.16 & 10.10 & 10.09 &  10.21 & 10.14 & 10.16 \\
 \hline
  \hline
 \multicolumn{8}{c}{GPT3-8B (FP32 PPL = 7.38)} \\ 
 \hline
 \hline
 64 & 7.61 & 7.52 & 7.48 &  7.47 &  7.55 &  7.49 & 7.50 \\
 \hline
 32 & 7.52 & 7.50 & 7.46 &  7.45 &  7.52 &  7.48 & 7.48  \\
 \hline
 16 & 7.51 & 7.48 & 7.44 &  7.44 &  7.51 &  7.49 & 7.47  \\
 \hline
\end{tabular}
\caption{\label{tab:ppl_gpt3_abalation} Wikitext-103 perplexity across GPT3-1.3B and 8B models.}
\end{table}

\begin{table} \centering
\begin{tabular}{|c||c|c|c|c||} 
\hline
 $L_b \rightarrow$& \multicolumn{4}{c||}{8}\\
 \hline
 \backslashbox{$L_A$\kern-1em}{\kern-1em$N_c$} & 2 & 4 & 8 & 16 \\
 %$N_c \rightarrow$ & 2 & 4 & 8 & 16 & 2 & 4 & 2 \\
 \hline
 \hline
 \multicolumn{5}{|c|}{Llama2-7B (FP32 PPL = 5.06)} \\ 
 \hline
 \hline
 64 & 5.31 & 5.26 & 5.19 & 5.18  \\
 \hline
 32 & 5.23 & 5.25 & 5.18 & 5.15  \\
 \hline
 16 & 5.23 & 5.19 & 5.16 & 5.14  \\
 \hline
 \multicolumn{5}{|c|}{Nemotron4-15B (FP32 PPL = 5.87)} \\ 
 \hline
 \hline
 64  & 6.3 & 6.20 & 6.13 & 6.08  \\
 \hline
 32  & 6.24 & 6.12 & 6.07 & 6.03  \\
 \hline
 16  & 6.12 & 6.14 & 6.04 & 6.02  \\
 \hline
 \multicolumn{5}{|c|}{Nemotron4-340B (FP32 PPL = 3.48)} \\ 
 \hline
 \hline
 64 & 3.67 & 3.62 & 3.60 & 3.59 \\
 \hline
 32 & 3.63 & 3.61 & 3.59 & 3.56 \\
 \hline
 16 & 3.61 & 3.58 & 3.57 & 3.55 \\
 \hline
\end{tabular}
\caption{\label{tab:ppl_llama7B_nemo15B} Wikitext-103 perplexity compared to FP32 baseline in Llama2-7B and Nemotron4-15B, 340B models}
\end{table}

%\subsection{Perplexity achieved by various LO-BCQ configurations on MMLU dataset}


\begin{table} \centering
\begin{tabular}{|c||c|c|c|c||c|c|c|c|} 
\hline
 $L_b \rightarrow$& \multicolumn{4}{c||}{8} & \multicolumn{4}{c||}{8}\\
 \hline
 \backslashbox{$L_A$\kern-1em}{\kern-1em$N_c$} & 2 & 4 & 8 & 16 & 2 & 4 & 8 & 16  \\
 %$N_c \rightarrow$ & 2 & 4 & 8 & 16 & 2 & 4 & 2 \\
 \hline
 \hline
 \multicolumn{5}{|c|}{Llama2-7B (FP32 Accuracy = 45.8\%)} & \multicolumn{4}{|c|}{Llama2-70B (FP32 Accuracy = 69.12\%)} \\ 
 \hline
 \hline
 64 & 43.9 & 43.4 & 43.9 & 44.9 & 68.07 & 68.27 & 68.17 & 68.75 \\
 \hline
 32 & 44.5 & 43.8 & 44.9 & 44.5 & 68.37 & 68.51 & 68.35 & 68.27  \\
 \hline
 16 & 43.9 & 42.7 & 44.9 & 45 & 68.12 & 68.77 & 68.31 & 68.59  \\
 \hline
 \hline
 \multicolumn{5}{|c|}{GPT3-22B (FP32 Accuracy = 38.75\%)} & \multicolumn{4}{|c|}{Nemotron4-15B (FP32 Accuracy = 64.3\%)} \\ 
 \hline
 \hline
 64 & 36.71 & 38.85 & 38.13 & 38.92 & 63.17 & 62.36 & 63.72 & 64.09 \\
 \hline
 32 & 37.95 & 38.69 & 39.45 & 38.34 & 64.05 & 62.30 & 63.8 & 64.33  \\
 \hline
 16 & 38.88 & 38.80 & 38.31 & 38.92 & 63.22 & 63.51 & 63.93 & 64.43  \\
 \hline
\end{tabular}
\caption{\label{tab:mmlu_abalation} Accuracy on MMLU dataset across GPT3-22B, Llama2-7B, 70B and Nemotron4-15B models.}
\end{table}


%\subsection{Perplexity achieved by various LO-BCQ configurations on LM evaluation harness}

\begin{table} \centering
\begin{tabular}{|c||c|c|c|c||c|c|c|c|} 
\hline
 $L_b \rightarrow$& \multicolumn{4}{c||}{8} & \multicolumn{4}{c||}{8}\\
 \hline
 \backslashbox{$L_A$\kern-1em}{\kern-1em$N_c$} & 2 & 4 & 8 & 16 & 2 & 4 & 8 & 16  \\
 %$N_c \rightarrow$ & 2 & 4 & 8 & 16 & 2 & 4 & 2 \\
 \hline
 \hline
 \multicolumn{5}{|c|}{Race (FP32 Accuracy = 37.51\%)} & \multicolumn{4}{|c|}{Boolq (FP32 Accuracy = 64.62\%)} \\ 
 \hline
 \hline
 64 & 36.94 & 37.13 & 36.27 & 37.13 & 63.73 & 62.26 & 63.49 & 63.36 \\
 \hline
 32 & 37.03 & 36.36 & 36.08 & 37.03 & 62.54 & 63.51 & 63.49 & 63.55  \\
 \hline
 16 & 37.03 & 37.03 & 36.46 & 37.03 & 61.1 & 63.79 & 63.58 & 63.33  \\
 \hline
 \hline
 \multicolumn{5}{|c|}{Winogrande (FP32 Accuracy = 58.01\%)} & \multicolumn{4}{|c|}{Piqa (FP32 Accuracy = 74.21\%)} \\ 
 \hline
 \hline
 64 & 58.17 & 57.22 & 57.85 & 58.33 & 73.01 & 73.07 & 73.07 & 72.80 \\
 \hline
 32 & 59.12 & 58.09 & 57.85 & 58.41 & 73.01 & 73.94 & 72.74 & 73.18  \\
 \hline
 16 & 57.93 & 58.88 & 57.93 & 58.56 & 73.94 & 72.80 & 73.01 & 73.94  \\
 \hline
\end{tabular}
\caption{\label{tab:mmlu_abalation} Accuracy on LM evaluation harness tasks on GPT3-1.3B model.}
\end{table}

\begin{table} \centering
\begin{tabular}{|c||c|c|c|c||c|c|c|c|} 
\hline
 $L_b \rightarrow$& \multicolumn{4}{c||}{8} & \multicolumn{4}{c||}{8}\\
 \hline
 \backslashbox{$L_A$\kern-1em}{\kern-1em$N_c$} & 2 & 4 & 8 & 16 & 2 & 4 & 8 & 16  \\
 %$N_c \rightarrow$ & 2 & 4 & 8 & 16 & 2 & 4 & 2 \\
 \hline
 \hline
 \multicolumn{5}{|c|}{Race (FP32 Accuracy = 41.34\%)} & \multicolumn{4}{|c|}{Boolq (FP32 Accuracy = 68.32\%)} \\ 
 \hline
 \hline
 64 & 40.48 & 40.10 & 39.43 & 39.90 & 69.20 & 68.41 & 69.45 & 68.56 \\
 \hline
 32 & 39.52 & 39.52 & 40.77 & 39.62 & 68.32 & 67.43 & 68.17 & 69.30  \\
 \hline
 16 & 39.81 & 39.71 & 39.90 & 40.38 & 68.10 & 66.33 & 69.51 & 69.42  \\
 \hline
 \hline
 \multicolumn{5}{|c|}{Winogrande (FP32 Accuracy = 67.88\%)} & \multicolumn{4}{|c|}{Piqa (FP32 Accuracy = 78.78\%)} \\ 
 \hline
 \hline
 64 & 66.85 & 66.61 & 67.72 & 67.88 & 77.31 & 77.42 & 77.75 & 77.64 \\
 \hline
 32 & 67.25 & 67.72 & 67.72 & 67.00 & 77.31 & 77.04 & 77.80 & 77.37  \\
 \hline
 16 & 68.11 & 68.90 & 67.88 & 67.48 & 77.37 & 78.13 & 78.13 & 77.69  \\
 \hline
\end{tabular}
\caption{\label{tab:mmlu_abalation} Accuracy on LM evaluation harness tasks on GPT3-8B model.}
\end{table}

\begin{table} \centering
\begin{tabular}{|c||c|c|c|c||c|c|c|c|} 
\hline
 $L_b \rightarrow$& \multicolumn{4}{c||}{8} & \multicolumn{4}{c||}{8}\\
 \hline
 \backslashbox{$L_A$\kern-1em}{\kern-1em$N_c$} & 2 & 4 & 8 & 16 & 2 & 4 & 8 & 16  \\
 %$N_c \rightarrow$ & 2 & 4 & 8 & 16 & 2 & 4 & 2 \\
 \hline
 \hline
 \multicolumn{5}{|c|}{Race (FP32 Accuracy = 40.67\%)} & \multicolumn{4}{|c|}{Boolq (FP32 Accuracy = 76.54\%)} \\ 
 \hline
 \hline
 64 & 40.48 & 40.10 & 39.43 & 39.90 & 75.41 & 75.11 & 77.09 & 75.66 \\
 \hline
 32 & 39.52 & 39.52 & 40.77 & 39.62 & 76.02 & 76.02 & 75.96 & 75.35  \\
 \hline
 16 & 39.81 & 39.71 & 39.90 & 40.38 & 75.05 & 73.82 & 75.72 & 76.09  \\
 \hline
 \hline
 \multicolumn{5}{|c|}{Winogrande (FP32 Accuracy = 70.64\%)} & \multicolumn{4}{|c|}{Piqa (FP32 Accuracy = 79.16\%)} \\ 
 \hline
 \hline
 64 & 69.14 & 70.17 & 70.17 & 70.56 & 78.24 & 79.00 & 78.62 & 78.73 \\
 \hline
 32 & 70.96 & 69.69 & 71.27 & 69.30 & 78.56 & 79.49 & 79.16 & 78.89  \\
 \hline
 16 & 71.03 & 69.53 & 69.69 & 70.40 & 78.13 & 79.16 & 79.00 & 79.00  \\
 \hline
\end{tabular}
\caption{\label{tab:mmlu_abalation} Accuracy on LM evaluation harness tasks on GPT3-22B model.}
\end{table}

\begin{table} \centering
\begin{tabular}{|c||c|c|c|c||c|c|c|c|} 
\hline
 $L_b \rightarrow$& \multicolumn{4}{c||}{8} & \multicolumn{4}{c||}{8}\\
 \hline
 \backslashbox{$L_A$\kern-1em}{\kern-1em$N_c$} & 2 & 4 & 8 & 16 & 2 & 4 & 8 & 16  \\
 %$N_c \rightarrow$ & 2 & 4 & 8 & 16 & 2 & 4 & 2 \\
 \hline
 \hline
 \multicolumn{5}{|c|}{Race (FP32 Accuracy = 44.4\%)} & \multicolumn{4}{|c|}{Boolq (FP32 Accuracy = 79.29\%)} \\ 
 \hline
 \hline
 64 & 42.49 & 42.51 & 42.58 & 43.45 & 77.58 & 77.37 & 77.43 & 78.1 \\
 \hline
 32 & 43.35 & 42.49 & 43.64 & 43.73 & 77.86 & 75.32 & 77.28 & 77.86  \\
 \hline
 16 & 44.21 & 44.21 & 43.64 & 42.97 & 78.65 & 77 & 76.94 & 77.98  \\
 \hline
 \hline
 \multicolumn{5}{|c|}{Winogrande (FP32 Accuracy = 69.38\%)} & \multicolumn{4}{|c|}{Piqa (FP32 Accuracy = 78.07\%)} \\ 
 \hline
 \hline
 64 & 68.9 & 68.43 & 69.77 & 68.19 & 77.09 & 76.82 & 77.09 & 77.86 \\
 \hline
 32 & 69.38 & 68.51 & 68.82 & 68.90 & 78.07 & 76.71 & 78.07 & 77.86  \\
 \hline
 16 & 69.53 & 67.09 & 69.38 & 68.90 & 77.37 & 77.8 & 77.91 & 77.69  \\
 \hline
\end{tabular}
\caption{\label{tab:mmlu_abalation} Accuracy on LM evaluation harness tasks on Llama2-7B model.}
\end{table}

\begin{table} \centering
\begin{tabular}{|c||c|c|c|c||c|c|c|c|} 
\hline
 $L_b \rightarrow$& \multicolumn{4}{c||}{8} & \multicolumn{4}{c||}{8}\\
 \hline
 \backslashbox{$L_A$\kern-1em}{\kern-1em$N_c$} & 2 & 4 & 8 & 16 & 2 & 4 & 8 & 16  \\
 %$N_c \rightarrow$ & 2 & 4 & 8 & 16 & 2 & 4 & 2 \\
 \hline
 \hline
 \multicolumn{5}{|c|}{Race (FP32 Accuracy = 48.8\%)} & \multicolumn{4}{|c|}{Boolq (FP32 Accuracy = 85.23\%)} \\ 
 \hline
 \hline
 64 & 49.00 & 49.00 & 49.28 & 48.71 & 82.82 & 84.28 & 84.03 & 84.25 \\
 \hline
 32 & 49.57 & 48.52 & 48.33 & 49.28 & 83.85 & 84.46 & 84.31 & 84.93  \\
 \hline
 16 & 49.85 & 49.09 & 49.28 & 48.99 & 85.11 & 84.46 & 84.61 & 83.94  \\
 \hline
 \hline
 \multicolumn{5}{|c|}{Winogrande (FP32 Accuracy = 79.95\%)} & \multicolumn{4}{|c|}{Piqa (FP32 Accuracy = 81.56\%)} \\ 
 \hline
 \hline
 64 & 78.77 & 78.45 & 78.37 & 79.16 & 81.45 & 80.69 & 81.45 & 81.5 \\
 \hline
 32 & 78.45 & 79.01 & 78.69 & 80.66 & 81.56 & 80.58 & 81.18 & 81.34  \\
 \hline
 16 & 79.95 & 79.56 & 79.79 & 79.72 & 81.28 & 81.66 & 81.28 & 80.96  \\
 \hline
\end{tabular}
\caption{\label{tab:mmlu_abalation} Accuracy on LM evaluation harness tasks on Llama2-70B model.}
\end{table}

%\section{MSE Studies}
%\textcolor{red}{TODO}


\subsection{Number Formats and Quantization Method}
\label{subsec:numFormats_quantMethod}
\subsubsection{Integer Format}
An $n$-bit signed integer (INT) is typically represented with a 2s-complement format \citep{yao2022zeroquant,xiao2023smoothquant,dai2021vsq}, where the most significant bit denotes the sign.

\subsubsection{Floating Point Format}
An $n$-bit signed floating point (FP) number $x$ comprises of a 1-bit sign ($x_{\mathrm{sign}}$), $B_m$-bit mantissa ($x_{\mathrm{mant}}$) and $B_e$-bit exponent ($x_{\mathrm{exp}}$) such that $B_m+B_e=n-1$. The associated constant exponent bias ($E_{\mathrm{bias}}$) is computed as $(2^{{B_e}-1}-1)$. We denote this format as $E_{B_e}M_{B_m}$.  

\subsubsection{Quantization Scheme}
\label{subsec:quant_method}
A quantization scheme dictates how a given unquantized tensor is converted to its quantized representation. We consider FP formats for the purpose of illustration. Given an unquantized tensor $\bm{X}$ and an FP format $E_{B_e}M_{B_m}$, we first, we compute the quantization scale factor $s_X$ that maps the maximum absolute value of $\bm{X}$ to the maximum quantization level of the $E_{B_e}M_{B_m}$ format as follows:
\begin{align}
\label{eq:sf}
    s_X = \frac{\mathrm{max}(|\bm{X}|)}{\mathrm{max}(E_{B_e}M_{B_m})}
\end{align}
In the above equation, $|\cdot|$ denotes the absolute value function.

Next, we scale $\bm{X}$ by $s_X$ and quantize it to $\hat{\bm{X}}$ by rounding it to the nearest quantization level of $E_{B_e}M_{B_m}$ as:

\begin{align}
\label{eq:tensor_quant}
    \hat{\bm{X}} = \text{round-to-nearest}\left(\frac{\bm{X}}{s_X}, E_{B_e}M_{B_m}\right)
\end{align}

We perform dynamic max-scaled quantization \citep{wu2020integer}, where the scale factor $s$ for activations is dynamically computed during runtime.

\subsection{Vector Scaled Quantization}
\begin{wrapfigure}{r}{0.35\linewidth}
  \centering
  \includegraphics[width=\linewidth]{sections/figures/vsquant.jpg}
  \caption{\small Vectorwise decomposition for per-vector scaled quantization (VSQ \citep{dai2021vsq}).}
  \label{fig:vsquant}
\end{wrapfigure}
During VSQ \citep{dai2021vsq}, the operand tensors are decomposed into 1D vectors in a hardware friendly manner as shown in Figure \ref{fig:vsquant}. Since the decomposed tensors are used as operands in matrix multiplications during inference, it is beneficial to perform this decomposition along the reduction dimension of the multiplication. The vectorwise quantization is performed similar to tensorwise quantization described in Equations \ref{eq:sf} and \ref{eq:tensor_quant}, where a scale factor $s_v$ is required for each vector $\bm{v}$ that maps the maximum absolute value of that vector to the maximum quantization level. While smaller vector lengths can lead to larger accuracy gains, the associated memory and computational overheads due to the per-vector scale factors increases. To alleviate these overheads, VSQ \citep{dai2021vsq} proposed a second level quantization of the per-vector scale factors to unsigned integers, while MX \citep{rouhani2023shared} quantizes them to integer powers of 2 (denoted as $2^{INT}$).

\subsubsection{MX Format}
The MX format proposed in \citep{rouhani2023microscaling} introduces the concept of sub-block shifting. For every two scalar elements of $b$-bits each, there is a shared exponent bit. The value of this exponent bit is determined through an empirical analysis that targets minimizing quantization MSE. We note that the FP format $E_{1}M_{b}$ is strictly better than MX from an accuracy perspective since it allocates a dedicated exponent bit to each scalar as opposed to sharing it across two scalars. Therefore, we conservatively bound the accuracy of a $b+2$-bit signed MX format with that of a $E_{1}M_{b}$ format in our comparisons. For instance, we use E1M2 format as a proxy for MX4.

\begin{figure}
    \centering
    \includegraphics[width=1\linewidth]{sections//figures/BlockFormats.pdf}
    \caption{\small Comparing LO-BCQ to MX format.}
    \label{fig:block_formats}
\end{figure}

Figure \ref{fig:block_formats} compares our $4$-bit LO-BCQ block format to MX \citep{rouhani2023microscaling}. As shown, both LO-BCQ and MX decompose a given operand tensor into block arrays and each block array into blocks. Similar to MX, we find that per-block quantization ($L_b < L_A$) leads to better accuracy due to increased flexibility. While MX achieves this through per-block $1$-bit micro-scales, we associate a dedicated codebook to each block through a per-block codebook selector. Further, MX quantizes the per-block array scale-factor to E8M0 format without per-tensor scaling. In contrast during LO-BCQ, we find that per-tensor scaling combined with quantization of per-block array scale-factor to E4M3 format results in superior inference accuracy across models. 

\section{Protocols Implementation}\label{sec:use_cases-appendix}
We next present the implementation details of four distributed systems shown in Table~\ref{tab:use_cases_options} using \projecttitle{}, presented in Section~\ref{sec:use_cases}.

\begin{table}
\begin{center}
\small
\begin{tabular}{ |c|c|c|c| } 
 \hline
 System & $N$ & $f$ ($N=3$) & Byzantine faults \\ [0.5ex] \hline \hline
 A2M    & 1 & 0 & Prevention\\
 BFT &  $2f+1$ & $f=1$ & Prevention\\
 Chain Replication &  $f+1$ & $f=2$ & Prevention\\
 PeerReview & $f+1$ & $f=2$ & Detection\\
 \hline
\end{tabular}
\end{center}
\caption{Properties of the four trustworthy distributed systems implemented with \projecttitle{}.}
\label{tab:use_cases_options}
\end{table}

\subsection{Clients} Clients in a \projecttitle{} distributed system execute requests by sending singed request messages to \projecttitle{} nodes through the network. \projecttitle{} assumes Byzantine (untrusted) clients; as such, its installed shared keys cannot be outsourced. We assume that at the initialization, the System Designer also loads to \projecttitle{} devices a (per-device) key pair $C_{pub, priv}$ where the $C_{pub}$ is distributed to clients. \projecttitle{} then replies to a client by verifying the (under transmission) attested message and signing it with $C_{priv}$. As such, \projecttitle{} is restricted to only sending valid attested messages to clients where clients can prove the transferable authentication and validity of the message. The only attack vector open to a Byzantine machine is to try to equivocate by sending a stale, valid, attested message that does not reflect the current execution round. However, clients can detect this by verifying that the original request is theirs.    % the such replies to clients are signed with a private key within the \projecttitle{} Furthermore, clients do not need to have access to a \projecttitle{}

%Similarly to classical BFT systems, \projecttitle{} clients require a {\em quorum certificate}, a set of identical messages collected from different participants~\cite{10.1145/800215.806583} to consider their request as committed. In contrast to the traditional BFT, where any $f$ out of the total 3$f$+1 nodes could equivocate, our strategy to prevent equivocation improves message complexity, allowing clients to wait for (at least) $f+1$ identical replies to consider their request committed.


\subsection{Attested Append-Only Memory (A2M)}\label{sec:use_cases::a2m}
We designed a single-node trusted log system based on the A2M system (Attested Append-Only Memory)~\cite{A2M} using \projecttitle{}. A2M has been proven to be an effective building block in improving the scalability and performance of various classical BFT systems~\cite{sundr, Castro:2002, AbdElMalek2005FaultscalableBF}. We show the {\em how} to use \projecttitle{} to build this foundational system while we also show that \projecttitle{} minimizes the system's TCB jointly with the performance improvements demonstrated in $\S$~\ref{sec:eval}.

\myparagraph{System model} Our \projecttitle{} version and the original A2M systems are single-node systems that target a similar goal; they both build a trusted append-only log as an effective mechanism to combat equivocation. The clients can only append entries to a log; each log entry is associated with a monotonically increasing sequence number. Each data item, e.g., a network message, is bound to a unique sequence number, a well-known approach for equivocation-free operations~\cite{clement2012, hybster}. 

A2M was originally built using CPU-side TEEs---specifically, Intel SGX--- whereas we build its \projecttitle{} derivative. While the original A2M system keeps its entire state and the log within the TEE, we use \projecttitle{} to keep the (trusted) log in the untrusted memory. As such, in contrast to the original A2M, \projecttitle{} effectively reduces the overall system's TCB. Our evaluation showed that naively porting the application within the TEE has adverse performance implications in lookup operations.


%the trusted component \atsushi{Here explains that TNIC-log brings better memory efficiency than A2M, which could also be written in the first paragraph to highlight the advantage of TNIC}.

\myparagraph{Execution} Similarly to A2M, we expose three core operations: the \texttt{append}, \texttt{lookup}, and \texttt{truncate} operations to add, retrieve, and delete items of the log, respectively. A2M stores the lowest and highest sequence numbers for each log. Upon appending an entry, A2M increases the highest sequence number and associates it with the newly appended entry. When truncating the log, the system advances the lowest sequence number accordingly. We next discuss how we designed the operations using \projecttitle{} APIs.




\begin{algorithm}
\SetAlgoLined
\small
%\vspace{0.02cm}
\textbf{function} \texttt{append(id, ctx)} \{ \\
\Indp
 [$\alpha$,\texttt{i},\texttt{ctx}] $\leftarrow$ \texttt{local\_send(id,ctx)};\\
 \texttt{log[id].append(log\_entry($\alpha$,\texttt{i},\texttt{ctx}))};\\
 {\bf return} \texttt{[$\alpha$,\texttt{i},\texttt{ctx}]};\\
\Indm
\} \\

\vspace{0.15cm}

\textbf{function} \texttt{lookup(id, i)} \{ \\
\Indp
    {\bf return} \texttt{log[id].get(i)};\\
\Indm
\} \\
\vspace{0.15cm}
\textbf{function} \texttt{truncate(id, head, z)} \{ \\
\Indp
    [$\alpha$,\texttt{tail},\texttt{ctx}] $\leftarrow$ \texttt{append(id,} \textsc{trnc}\texttt{||id||z||head)};\\
        
    %[$\alpha_{2}$,\texttt{idx},\texttt{ctx}$_{2}$]
    \texttt{e} $\leftarrow$ \texttt{append(}\textsc{manifest}\texttt{,[$\alpha$,\texttt{tail},\texttt{ctx}])};\\
    {\bf return} \texttt{e};\\
\Indm
\} \\

\vspace{0.15cm}
\textbf{function} \texttt{verify\_lookup(id, e, head, tail)} \{ \\
\Indp
    \textbf{assert}(\texttt{e.i}$>=$\texttt{tail)};\\
    \texttt{local\_verify(id, e)};\\
\Indm
\} \\
\vspace{-1pt}
\caption{Attested Append-Only Memory (A2M) using \projecttitle{}.}
\label{algo:tnic_log}
\end{algorithm}

\myparagraph{Append operation} The \texttt{append(id,ctx)} operation takes a data item, \texttt{ctx}, and appends it to the log with identifier \texttt{id}. A log entry at index \texttt{i} is comprised of three items: the sequence number of that entry (\texttt{i}), the context of the entry (\texttt{ctx}), and the {\em authenticator} field, namely the digest of the \texttt{ctx||i} as in~\cite{levin2009trinc}. In our implementation, we additionally support the original A2M {\em authenticator} format calculated as the cumulative digest \texttt{c\_digest[i]} for that entry which is calculated as \texttt{c\_digest[i]=hash(ctx||sq||c\_digest[i-1])} where \texttt{c\_digest[0]=0}. The sequence number \texttt{i} is then increased to distinguish any entry that will be appended in the future. %With the cumulative digest, we create a set of chains, and as such, our method does not cause any values to be forgotten.





\myparagraph{Lookup operation} The \texttt{lookup(id, i)} retrieves the log entry at index \texttt{i} of log with identifier \texttt{id}. Compared to A2M, where lookups are compelled to access the trusted hardware, \projecttitle{}-log only performs a local memory access. 
The function does not verify whether the entry is legitimate. Developers need to implement the \texttt{verify\_lookup(id, entry, head, tail)} to verify the attestation. The boundaries of the log (i.e., \texttt{head} and \texttt{tail}) can constantly be retrieved by replaying a specific log, which keeps the state changes, the \textsc{manifest}. We explain how \textsc{manifest} works in the next paragraph.

\myparagraph{Truncate operation} The \texttt{truncate(id, head, z)}, where \texttt{z} is a nonce provided by the client for freshness, ``forgets'' all log entries with sequence numbers lower than \texttt{head}. A non-Byzantine client can never successfully verify a forgotten log entry. To do that, \projecttitle{}-log uses an additional log \textsc{manifest}, which keeps the logs' state changes. First, the operation attests to the {\em tail} of the log by appending a specific entry, which includes the nonce for a correct client to be later able to verify the operation. Then, the algorithm will append the last attested message of the log to the \textsc{manifest} log and return the attested message for the second append. To retrieve the boundaries of a log, clients can always attest to the tail of the \textsc{manifest} and read backward until they find a \textsc{trnc} entry.


\noindent\fbox{\begin{minipage}{\columnwidth}
\myparagraph{System design takeaway} \projecttitle{} minimizes the required TCB in the A2M system while offering faster lookup operations than its original version.
\end{minipage}}
%\atsushi{Can we explain more which point is improved thanks to TNIC?}

%Since the logs reside in the untrusted host memory, their integrity can be compromised by malicious adversaries. However, these adversaries cannot impersonate \projecttitle{} and generate verifiable attestations in any way. As such we do not worry about entries that are not written yet (these will never be verifiable). 
%If an adversary compromises the \textsc{MANIFEST}, \projecttitle{}-log might become responsive. However, this affects availability but not safary and it is beyond the scope of this work similar to other systems [A2M, Trinc, Damysus]. 


\subsection{Byzantine Fault Tolerance (BFT)}\label{sec:use_cases::byz_smr}
%\atsushi{unclear to me what is the advantage of using TNIC to implement the Byzantine SMR. Can we somehow highlight this point?}
%\dimitra{@Atnoni: Can we afford a network partition? 2f+1 w/ f=2, Assume Byz. leader and Byz. follower that drive execution with one correct replica---the others are on purpose exluded by the faulty leader. The client will have a correct reply always because it will wait for f+1 (=3) identical replies. Although if the most up to date correct replica afterwards is partitioned out, then we just block; the remaining correct replicas will have lost one message and block until they get it ... }
%\pramod{ToDo: Pesudocode.}

As a second example of \projecttitle{} applications, we build a Byzantine Fault-Tolerant protocol (BFT) that implements a robust counter based on {\em state machine replication} (SMR). Clients send increment counter requests to the SMR and receive the updated value of the counter. Despite its simplicity, this particular system can represent an ordering service, which is a fundamental building block of various distributed applications ranging from event logging and database systems to serverless and blockchain~\cite{rafthyperledger, Kafka, boki, 10.1145/3286685.3286686, scalog}. Our BFT combats equivocation by leveraging the attestation kernel of \projecttitle{}. As such, via \projecttitle{}, it reduces \textit{(i)} the number of replicas and \textit{(ii)} the message complexity (and latency) required by classical BFT.

\myparagraph{System model} We consider a system of $N=2f+1$ replicas (or {\em nodes}) that communicate with each other over unreliable point-to-point network links. At most $f$ of these replicas can be Byzantine (aka {\em faulty}), i.e., can behave arbitrarily. The rest of the replicas are {\em correct}. Recall that classical BFT protocols require an extra set of $f$ replicas, in total $3f+1$, to handle $f$ Byzantine failures.  One of the replicas is the {\em leader} that drives the protocol, whereas the remaining replicas are (passive) followers. There is only a single active leader at a time.

For liveness, we assume a partial synchrony model~\cite{FLP, 10.1145/226643.226647}. We have only explored deterministic protocol specifications; the correct replicas begin in the same state, and receiving the same inputs in the same order will arrive at the same state, generating the same outputs. Lastly, as in classical BFT protocols, we cannot prevent Byzantine clients who otherwise follow the protocol from overwriting correct clients' data.


\myparagraph{Execution} We implement BFT with \projecttitle{} as a leader-based SMR protocol for a Byzantine model that stores and increases the counter's value. The leader receives clients' requests to increment the counter. The leader, in turn, executes the protocol and applies the changes to its state machine---in our case, the leader computes and stores the next available counter value. Subsequently, the leader broadcasts the request along with some metadata to the passive followers. The metadata includes, among others, the leader's calculated output in response to the client's command, namely, the increased counter value the leader has calculated.
% and the {\em state} of the followers known to the leader.
% In our implementation, the {\em state} is represented as the signed hash of the counter value of each follower. %(known to the leader).

The followers, in turn, execute and apply the incremented counter value to their state machines. However, they first attest to the leader's (and other followers') actions to detect misbehavior. Importantly, followers validate if the state (counter) of the replicas (including the leader and all other replicas) match the expected value.

%The followers, in turn, execute and apply the incremented counter value to their state machines. owever, they first attest to the leader's actions to detect misbehavior. o do so, they audit and validate its sent output through re-execution. recisely, the followers except for their state machine, {\em simulate} the leader's state machine. ach follower replica must add an extra counter representing the state the value counter is expected to have, leading to a $2\times$ extra space complexity.   follower will update the leader's value based on the commands received and then compare its calculated leader's value with the received one. n addition, the followers will validate the state of the replicas (including the leader and all other replicas). hey only have to check if their previous state equals the other replicas' state. 


After a follower applies the increments to its local counter, it replies to the client.
% and the leader with the result of the operation. 
In addition, it forwards the leader's request to every other replica to ensure that all correct replicas will eventually receive and apply the same command. Replicas that have already applied the request ignore it; otherwise, they validate it and apply it. The leader, upon successful validation, will also reply to the client. The client can trust the result if they receive identical replies from a majority quorum, i.e., at least $f+1$ identical messages from different replicas (including the leader). This guarantees that at least one correct replica has responded with the correct result.


\myparagraph{Failure handling} Our strategy to verify the replica's execution jointly with the primitives of non-equivocation and transferable authentication offered by \projecttitle{} shields the protocol against Byzantine behavior. The leader cannot equivocate; even if it attempts to send different requests for the same round to different followers, executing the {\tt local\_send()} will assign different counter values, which healthy followers will detect. As such, a leader in that case will be exposed. 

Likewise, the equivocation mechanism allows correct followers to discard stale message requests sent through replay attacks on the network. If a follower is Byzantine, a healthy leader or replica can detect it. For $f\geq2$, it is impossible for a faulty leader and, at most, $f-1$ remaining Byzantine followers to compromise the protocol. Either these faults will be detected by a healthy replica during the validation phase, or the protocol will be unavailable, i.e., if the leader in purpose only communicates with the Byzantine followers. This directly affects BFT correctness requirements; a client will never get at least $f+1$ matching replies. Even in the extreme case of a network partition or a faulty leader that purposely excludes some healthy replicas from its multicast group, when the network is restored, these replicas will not accept any future messages unless they receive all missed ones. Suppose the leader fails in the middle of the broadcast. In that case, the last step in the follower's protocol ensures that if a correct replica accepts the requests, all correct replicas will eventually apply the same request. Since the reliability aspect and FIFO ordering are implemented in hardware, healthy replicas will ultimately receive all past messages in the proper order. For protocols to progress in the case of a faulty leader, they must pass through a recovery protocol or view-change protocols similar to those described in previous works~\cite{minBFT, Castro:2002}. Recovering is beyond the scope of this work, and as such, we did not implement it.


\noindent\fbox{\begin{minipage}{\columnwidth}
\myparagraph{System design takeaway} \projecttitle{} optimizes the replication factor and the message rounds compared to classical BFT.
\end{minipage}}


\begin{algorithm}
\SetAlgoLined
\small
%\vspace{0.02cm}
\textbf{function} \texttt{leader(req)} \{ \\
\Indp
 {\tt output} $\leftarrow$ \texttt{execute(req)};\\
 {\tt msg} $\leftarrow$ \texttt{req||output};\\
 {\tt attested\_msg} $\leftarrow$ \texttt{local\_send(msg)};\\
 \texttt{rem\_write(}\textsc{followers[:]}{\tt, attested\_msg)};\\

{\bf upon reception of {\tt ack} from \textsc{followers}:}\\
    \Indp
        {\tt [{$\alpha$ || f\_attested\_msg || f\_output || f\_id}]} \\\hspace{22pt} $\leftarrow$ \texttt{upon\_delivery(ack)};\\
        {\bf assert(}\texttt{validate\_follower(f\_attested\_msg,\\\hspace{22pt} f\_output)}{\bf)};\\
        \texttt{incr\_req\_acks\_if\_not\_incr\_before(f\_id)};\\
    \Indm

 \texttt{auth\_send(}\textsc{client}{\tt,msg)};\\
\Indm
\} \\

\vspace{0.15cm}

% \textbf{function} \texttt{follower(}\textsc{sender}{\tt, attested\_msg)} \{ \\
\textbf{function} \texttt{follower()} \{ \\
\Indp
{\bf upon reception of {\tt attested\_msg}:}\\
    \Indp
        {\tt [{$\alpha$ || req || output}]} $\leftarrow$ \\\hspace{22pt}\texttt{upon\_delivery(attested\_msg)};\\

    {\bf assert(}\texttt{validate\_sender(req, output)}{\bf)};\\
    {\bf if }{\tt (in\_order\_not\_applied(req))}\\
    \Indp
    {\tt current\_output} $\leftarrow$ \texttt{execute(req)};\\
    {\tt f\_attested\_msg} $\leftarrow$ \texttt{local\_send(req||current\_output)};\\
    ack $\leftarrow$ {\tt f\_attested\_msg}\\
    % \texttt{auth\_send(}\textsc{client} $\cup$ \textsc{leader} $\cup$ \textsc{followers[:]},\\{\tt\hspace{40pt} attested\_msg)};\\    
    \texttt{auth\_send(}\textsc{leader}, {\tt ack)};\\    
    % \texttt{auth\_send(}\textsc{client} $\cup$ \textsc{followers[:]},\\{\tt\hspace{40pt} attested\_msg)};\\    
    \texttt{auth\_send(}\textsc{client} $\cup$ \textsc{followers[:]},\\{\tt\hspace{40pt} f\_attested\_msg)};\\    
    % \texttt{auth\_send(}\textsc{leader}{\tt, req||current\_output)};\\    
    % \texttt{auth\_send(}\textsc{client}{\tt, req||current\_output)};\\
    % {\bf if {\tt not\_seen(req)} {\sc and  sender = leader}:}\\
    % \Indp
        % {\bf for} (\textsc{followers[:]}) 
            % \texttt{auth\_send(}\textsc{followers[:]}{\tt, msg)};\\
            % \texttt{rem\_write(}\textsc{LEADERfollowers[:]}{\tt, attested\_msg)};\\
    \Indm
    \Indm
\Indm
\} \\
\vspace{-1pt}
\caption{BFT using \projecttitle{}.}
\label{algo:tnic_bft}
\end{algorithm}

%
%\dimitra{>Github code issues:
%\begin{itemize}
%    \item Line 239: has a logical bug regarding the message batching
%    \item Lines 205--210: unnecessary hash re-calcucations--- might improve performance if fixed
%    \item Continuation function needs improvement for correctness/completeness (leader should store the output for each on-going command).
%    \item For correctness, leader should have only one outstanding operation at a time.
%\end{itemize}}



%\lstinputlisting[language=C++]{codelets/pb.cc}




\subsection{Chain Replication (CR)}\label{sec:use_cases::byz_chain_rep}

\begin{algorithm}
\SetAlgoLined
\small
%\vspace{0.02cm}
\textbf{function} \texttt{head\_operation(req)} \{ \\
\Indp
 {\tt output} $\leftarrow$ \texttt{execute(req)};\\
 {\tt msg} $\leftarrow$ \texttt{req||output};\\
 \texttt{auth\_send(}\textsc{middle}{\tt,msg)};\\ \texttt{auth\_send(}\textsc{client}{\tt,msg)};\\
\Indm
\} \\

\vspace{0.15cm}

\textbf{function} \texttt{middle\_tail\_operation(msg)} \{ \\
\Indp
    {\bf assert(}\texttt{validate\_chain(msg)}{\bf)};\\
    {\tt output} $\leftarrow$ \texttt{execute(req)};\\
    {\tt chained\_msg} $\leftarrow$ \texttt{msg||output};\\
    {\bf if} (!\textsc{tail})\\
    \Indp
    \texttt{auth\_send(}\textsc{middle}{\tt,chained\_msg)};\\
    \Indm
    \texttt{auth\_send(}\textsc{client}{\tt,req||output)};\\
\Indm
\} \\
\vspace{0.15cm}
\textbf{function} \texttt{validate(msg)} \{ \\
\Indp
    \texttt{len} $\leftarrow$ \texttt{sz};\\
    \texttt{[req, out, cmt]} $\leftarrow$ \texttt{unmarshall(msg[0:len])};\\
    {\bf assert(}\texttt{memcmp(req, out)}{\bf)};\\
    {\bf assert(}\texttt{(cmt == expected\_cmt)}{\bf)};\\
    {\bf for} {\tt(i = 1; i < }{\sc node\_id}; {\tt i++)} \{\\
    \Indp
    \texttt{[out, cmt]} $\leftarrow$ \texttt{unmarshall(msg[len:len+\textsc{sz}])};\\
    {\bf assert(}\texttt{memcmp(req, out)}{\bf)};\\
    {\bf assert(}\texttt{(cmt == expected\_cmt)}{\bf)};\\
    \texttt{len} $\leftarrow$ \texttt{len} + \texttt{sz};\\
    \Indm
    {\bf return} {\tt True};\\
\Indm
\} \\
\vspace{-1pt}
\caption{Chain Replication using \projecttitle{}.}
\label{algo:tnic_chain_replication}
\end{algorithm}



We implement a Byzantine Chain Replication using \projecttitle{} that represents the replication layer of a Key-Value store. Chain Replication is a foundational protocol for building state machine replication and initially operates under the CFT model using $f+1$ nodes to tolerate up to $f$ failures. We show {\em how} to use \projecttitle{} to shield the protocol without changes to the core of the algorithm (states, rounds, etc.) while keeping the same replication factor.




\myparagraph{System model} We make the same assumptions for the system as in the previous BFT system. For error detection and reconfiguration, we assume a centralized (trusted) configuration service as in~\cite{10.1007/978-3-642-35476-2_24} that generates new configurations upon receiving reconfiguration requests from replicas. Recall that the classical Chain Replication under the CFT model relies on reliable failure detectors~\cite{chain-replication}. For liveness, we also assume that the configuration service will eventually create a configuration of correct replicas that do not intentionally issue reconfiguration requests to perform Denial-of-Service attacks. 

Clients send requests to {\tt put} or {\tt get} a value and receive the result. The replicas (e.g., head, middle, and {\em tail} nodes) are chained, and the requests flow from the head node to the tail through the intermediate middle replicas. 

Malicious primaries, i.e., the head that does not forward the message intentionally, are detected on the client's side and trigger reconfiguration~\cite{Castro:2002, minBFT}.


\myparagraph{Execution} To execute a request \texttt{req}, e.g., {\tt put}/{\tt get}, a client first obtains the current configuration from the configuration service and sends the {\tt req} to the head of the chain. The head orders and executes the request, and then it creates a {\em proof of execution message}, which is sent along the chain. The proof of execution includes the {\tt req} and the leader's action ({\tt out}) in response to that request. In our case, the leader sends the {\tt req} along with the assigned commit index. The message is then sent (signed) to the middle node that follows in the chain.

The middle node checks the message's validity by verifying that the head's output is correct, executes the {\tt req}, and forwards the request to the following replica. Similarly, every other node executes the original request, verifies the output of all previous nodes, and sends the original request and a vector of all previous outputs. A replica must construct a {\em proof of execution message} that achieves one goal. t allows the following replicas in the chain to verify all previous replicas. s such the messages is of the form < < <{\tt req}, {\tt out$_{leader}$}>$_{\sigma_0}$, {\tt out$_{middle1}$}>$_{\sigma_1}$, .., {\tt out$_{tail}$}>$_{\sigma_N}$. The tail is the last node in the chain that will execute and verify the execution of the request. 


In contrast to the CFT version of the Chain Replication protocol, local operations in the tail, {\tt get} or {\tt ack} in a {\tt put} request cannot be trusted. As such, the replicas in the chain need to reply to the clients with their output after they have forwarded their proof of execution message. Clients can wait for at least $f$ replicas replies and the tail reply to collide. Clients can execute the {\tt get} requests similarly to {\tt write} requests, traversing the entire chain, or clients can consult the majority and broadcast the request to $f+1$ replicas, including the tail. 


\myparagraph{Failure handling} By the protocol definition, all nodes will see and execute all messages in the same order imposed by the head node. As such, all correct replicas will always be in the same state. In addition, network partitions that may split the chain into two (or more) individual chains that operate independently cannot affect safety: the clients must verify at least $f+1$ identical replies. Suppose a correct replica or a client detects a violation (by examining the proof of execution message or having to hear for too long from a node). In that case, they can expose the faulty node and request a reconfiguration.

\noindent\fbox{\begin{minipage}{\columnwidth}
\myparagraph{System design takeaway} \projecttitle{} {\em seamlessly} shields the Chain Replication system for Byzantine settings with the same replication factor as the original CFT system.
\end{minipage}}

%\dimitra{>Github code issues:
%\begin{itemize}
%    \item check\_outputs function (L:67): Did you miss a validation step regarding transferable authentication?
%\end{itemize}}



\subsection{Accountability (PeerReview)}
\label{sec:use_cases::accountability}





We implement an accountability protocol based on the PeerReview system~\cite{bftdetection, peer-review}. Compared to the previous three BFT systems that prohibit an improper action from taking effect, accountability protocols~\cite{268272, bftdetection, peer-review} slightly weaken the system (fault) model in favor of performance and scalability. Specifically, our protocol {\em allows} Byzantine faults to happen (e.g., correct nodes might be convinced by a malicious replica to permanently delete data). Still, it guarantees that malicious actions can always be detected. Accountability protocols can be applied to different systems as generic guards that trade security for performance~\cite{peer-review}, e.g., NFS, BitTorrent, etc. 

The original version of the system did not use trusted components. t incurs a high message complexity, i.e., {\em all-to-all} communication to combat equivocation. We use \projecttitle{} to improve that message complexity.

\myparagraph{System model} We only detect faults that directly or indirectly affect a message, implying that {\em (i)} correct nodes 
can observe all messages sent and received by that node and {\em (ii)}  Byzantine faults that are not observable through the network cannot be detected. For example, a faulty storage node might report that it is out of disk space, which cannot be verified without knowing the actual state of its disks.

We further assume that each protocol participant acts according to a deterministic specification protocol. As such, detection can be accomplished even with a single correct machine, requiring only $f+1$ machines.  This does not contradict the impossibility results for agreement~\cite{FLP} because detection systems do not guarantee safety.


\myparagraph{Execution} The participants communicate through network messages generated by \projecttitle{}.  In addition, each participant maintains a {\em tamper-evident} log that stores all messages sent and received by that node as a chain. A log entry is associated with an entry index, the entry data, and an authenticator, calculated as the signed hash of the tail of the log and the current entry data. 

We frame our protocol in the context of an overlay multicast protocol~\cite{10.1145/945445.945474} widely used in streaming systems. The nodes are organized as a tree where the streaming content (e.g., audio, video) flows from a source, i.e., {\em root} node, to clients ({\em children} nodes). To support many clients, each can be a source to other clients, which will be connected as children nodes. 

In our implementation, we consider nodes in a tree topology. The tree's height equals one, comprising one source node and two client (children) nodes connected to the source. Algorithm~\ref{algo:tnic_accountability_protocol} ($\S$~\ref{sec:use_cases-appendix}) shows the operations of our implemented accountability protocol.  hen the source sends a context (executes the \texttt{root()} function), it implicitly includes a signed statement that this message has a particular sequence number (generated by \projecttitle{}). he clients execute the {\tt child()} function that validates the received message, logs the received message, executes the result, and responds to the source. 


\begin{algorithm}[t]
\SetAlgoLined
\small
%\vspace{0.02cm}
\textbf{function} \texttt{root(ctx)} \{ \\
\Indp
 \texttt{auth\_send(}\textsc{child}{\tt,ctx)};\\
 {\bf upon reception of \texttt{response}:};\\
 \Indp
    {\bf assert(}\texttt{validate\_reception(response)}{\bf)};\\
    \texttt{log(response)};\\
\Indm
\Indm
\} \\

\vspace{0.15cm}

\textbf{function} \texttt{child($\alpha$||cmd||seq)} \{ \\
\Indp
    {\bf assert(}\texttt{validate\_reception($\alpha$||cmd||seq)}{\bf)};\\
    \texttt{log($\alpha$||cmd||seq)};\\
    {\tt result} $\leftarrow$ \texttt{execute(cmt)};\\
    {\tt response} $\leftarrow$ \texttt{log(result||cmd)};\\
    \texttt{auth\_send(}\textsc{root}{\tt, response)};\\
\Indm
\} \\
\vspace{0.15cm}
\textbf{function} \texttt{log\_audit()} \{ \\
\Indp
    {\bf{while}} \texttt{last\_id < log\_tail} \{\\
    \Indp

        \texttt{entry} $\leftarrow$ \texttt{validate\_log\_entry\_at(last\_id)};\\
        \texttt{last\_id++};\\
        {\bf assert(}\texttt{replay(entry)}{\bf{)}};\\
    \Indm
    \}\\
\Indm
\} \\
%\vspace{-1pt}
\caption{PeerReview using \projecttitle{}.}
\label{algo:tnic_accountability_protocol}
\end{algorithm}


Each node is assigned to a set of {\em witness} processes to detect faults. Similarly to the original system, we assume that the set of nodes and its witnesses set {\em always} contain a correct process. The witnesses audit and monitor the node's log. To detect destructive behaviors (or expose non-responsive nodes), the witnesses read the node's log and replay it to run the participant's state machine. As such, they ensure the participant's state is consistent with proper operation. 

Specifically, each witness for a participant node keeps track of n, a log sequence number, and s, the state that the participant should have been in after sending or receiving the message in log entry n. t initializes n to 0 and s to the initial state of the participant.

Whenever a witness wants to audit a node, it sends its n and a nonce (for freshness).
The participant returns an attestation of all entries between n and its current log entry using the nonce. The witness then runs the reference implementation, starting at state $s$, and progressing through all the log entries. f the reference implementation sends the same messages in the log, then the witness updates n, %\antonis{Is this the point of truncating the log?}
which is the state of the reference implementation at that point. If not, then the witness has proof it can present of the participant's failure to act correctly.




The original PeerReview system requires a receiver node to forward messages to the original sender's witnesses so they can ensure this message is {\em legitimate}, i.e., it appears in the sender's log. No other conflicting message is sent to another peer (equivocation). As such, a peer must communicate (in every round) with the witness set of any other peer, leading to a quadratic message complexity. \projecttitle{} eliminates the overhead; a participant that sends or receives a message needs to attest and append the message and its attestation in each log. A participant can process received messages only if they are accompanied by attestations generated by the sender's \projecttitle{} hardware. 



\noindent\fbox{\begin{minipage}{\columnwidth}
\myparagraph{System design takeaway} \projecttitle{} can be used to optimize the message complexity in accountable systems.
\end{minipage}}







%\clearpage
% \newpage
% \fi
\bibliographystyle{ACM-Reference-Format}
\balance
\bibliography{sample}


%\theendnotes

\end{document}