

\section{Trusted NIC Hardware}
\label{sec:t-nic-hardware}
Figure~\ref{fig:hardware-design} shows our \projecttitle{} hardware architecture that implements trusted network operations on a SmartNIC device. \projecttitle{} introduces two key components: \emph{(i)} the attestation kernel that guarantees the non-equivocation and transferable authentication properties over the untrusted network ($\S$~\ref{subsec:nic_attest_kernel}) and \emph{(ii)} the RoCE protocol kernel that implements the RDMA protocol including transport and network layers ($\S$~\ref{subsec:roce_protocol_kernel}). We also introduce a bootstrapping and a remote attestation protocol for \projecttitle{} ($\S$~\ref{subsec:nic_controller}) and formally verify them ($\S$~\ref{subsec::formal_verification_remote_attestation}).
% Lastly, we design a remote attestation protocol for our \projecttitle{} based on the \projecttitle{} Controller module that we discuss $\S$~\ref{subsec:nic_controller}. Importantly, we formally verify \projecttitle{}'s remote attestation and initialization protocol ($\S$~\ref{subsec::formal_verification_remote_attestation}).

\subsection{NIC Attestation Kernel}
\label{subsec:nic_attest_kernel}
%\myparagraph{Overview} 
The attestation kernel {\em shields} network messages and materializes the properties of non-equivocation and transferable authentication by generating {\em attestations} for transmitted messages. As shown in Figure~\ref{fig:hardware-design}, the attestation kernel resides in the data pipeline between the RoCE protocol kernel that transmits/receives network messages and the PCIe DMA that transfers data from/to the host memory.
The kernel processes the messages as they {\em flow} from the memory to the network and vice versa to optimize throughput. 
% The module resides in the data pipeline between the RoCE protocol kernel that transmits/receives network messages and the PCIe DMA which fetches and pushes data {\em asynchronously} from the host memory to the FPGA memory.

\myparagraph{Hardware design} The attestation kernel is comprised of three components that represent its state and functionality: the HMAC component that generates the message authentication codes (MAC), the Keystore that stores the keys used by the HMAC module, and the Counters store that keeps the message's latest sent and received timestamp. 

The system designer initializes each \projecttitle{} device during bootstrapping with a unique identifier (ID) and a shared secret key---ideally, one shared key for each session---stored in static memory (Keystore). The keys are shared and, hence, unknown to the untrusted parties. 
% Each \projecttitle{} device is initialized by the system designer during bootstrapping ($\S$~\ref{subsec:nic_controller}) with a unique identifier (ID) and a shared secret key---ideally, one shared key for each session---which is stored in static memory (Keystore). The keys are shared and, hence, unknown to the untrusted parties. 


\projecttitle{} holds two counters per session in the Counters store: \texttt{send\_cnts}, which holds sending messages, and \texttt{recv\_cnts}, which holds the latest seen counter value for each session. The counters represent the messages' timestamp and are increased monotonically and deterministically after every send and receive operation to ensure that unique messages are assigned to unique counters for non-equivocation. Consequently, no messages can be lost, re-ordered, or doubly executed.
% \projecttitle{} holds two counters per session in the Counters store: {\tt send\_cnts} to be used for sending messages, and \texttt{recv\_cnts} that holds the latest seen counter value for each session. The counters represent the messages' timestamp and are increased monotonically and deterministically after every send and receive operation to ensure that unique messages are assigned to unique counters for non-equivocation. Consequently, no messages can be lost, re-ordered, or doubly executed.


\begin{figure}[t!]
    \centering
    %\includegraphics[width=.27\textwidth]{figures/trusted-nic-single-node-overview.drawio-1.pdf}
    %\includegraphics[width=0.45\textwidth]{figures/hw-implementation.drawio.pdf}
    %\caption{\projecttitle{} hardware architecture.}
    % \includegraphics[width=0.95\linewidth]{figures/trusted-nic-hardware-impl-improved.drawio.pdf}
     \includegraphics[width=0.75\linewidth]{figures/trusted-nic-hardware-impl-atsushi.pdf}
    \caption{\projecttitle{} hardware architecture.}
     \label{fig:hardware-design}
\end{figure}





\myparagraph{Algorithm} 
Algorithm~\ref{algo:primitives} shows the functionality of the attestation kernel. The module implements two core functions: {\tt Attest()}, which generates a unique and verifiable attestation for a message, and {\tt Verify()}, which verifies the attestation of a received message. The message transmission invokes {\tt Attest()}, and likewise, the reception invokes {\tt Verify()}. 
% When an application transmits a message, the message is attested by {\tt Attest()}. Likewise, upon a message reception, the message is verified with {\tt Verify()} before it is copied to the application's (host) memory. 
% Algorithm~\ref{algo:primitives} shows the functionality of the attestation kernel. The module implements two core functions: the {\tt Attest} function which generates a unique and verifiable attestation for a message and the {\tt Verify} function which verifies the attestation of a received message. When the application transmits a message, the data flow passes through the {\tt Attest} function, and upon a message reception, the message is verified (with {\tt Verify} function) before it is copied to the application's (host) memory. 

Upon transmission, as shown in Figure~\ref{fig:hardware-design}, the message is first forwarded to the attestation kernel. 
% , and then, the output is forwarded to the RoCE protocol kernel. 
The attestation kernel executes \texttt{Attest()} and generates an {\em attested} message comprised of the message data and its attestation certificate~$\alpha$. The function calculates $\alpha$ as the HMAC of the message concatenated with the counter {\tt send\_cnt} and the device ID (for transferable authentication) with the {\tt key} for that connection (Algo~\ref{algo:primitives}:~L4). It also increases the next available counter for subsequent future messages (Algo~\ref{algo:primitives}:~L2). The function forwards the message with its $\alpha$ to the RoCE protocol kernel (Algo~\ref{algo:primitives}:~L4).
% As shown in Figure~\ref{fig:hardware-design}, a request passes from the application layer to the \projecttitle{} through two separate paths, i.e., the data and control paths. The metadata and the request opcode pass through the control path from the host to the device and vice versa. On the other hand, the message data is first forwarded to the attestation kernel, and then, the output is forwarded to the RoCE protocol kernel. The attestation kernel upon transmission executes the \texttt{Attest} function and generates an {\em attested} message comprised of the message data and its attestation certificate, or simply attestation $\alpha$. The function calculates $\alpha$ as the HMAC of the message concatenated with the counter {\tt send\_cnt} using the {\tt key} for that connection (Algorithm~\ref{algo:primitives}:~L3). It also increases the next available counter for subsequent future messages (Algorithm~\ref{algo:primitives}:~L2). The function forwards to the RoCE protocol the original message along with $\alpha$ for transmission (Algorithm~\ref{algo:primitives}:~L4).

Upon reception, the received message passes through the attestation kernel for verification before it is delivered to the application. Specifically, \texttt{Verify()} checks the authenticity and the integrity of the received message by re-calculating the {\em expected} attestation $\alpha$' based on the message payload and comparing it with the received attestation $\alpha$ of the message (Algo~\ref{algo:primitives}:~L7---8). The verification also ensures that the received counter matches the expected counter for that connection to ensure \emph{continuity} (Algo~\ref{algo:primitives}:~L8). 





\subsection{RoCE Protocol Kernel}
\label{subsec:roce_protocol_kernel}

%\myparagraph{Overview} 
The RoCE protocol kernel implements a reliable transport service on top of the IB Transport Protocol with UDP/IPv4 layers (RoCE v2)~\cite{infiniband} (transport and network layers). As shown in Figure~\ref{fig:hardware-design}, to transmit data, the {\tt Req handler} module in the RoCE kernel receives the request opcode ({\tt metadata}) issued by the host. The message is fetched through the PCIe DMA engine and passes through the attestation kernel. The request opcode and the attested message are forwarded to the {\tt Request generation} module that constructs a network packet. 
% The RoCE protocol kernel implements a reliable transport service on top of the IB Transport Protocol with UDP/IPv4 layers (RoCE v2)~\cite{infiniband} (transport and network layers), and it is connected to the attestation kernel as discussed in $\S$~\ref{subsec:nic_attest_kernel}. As shown in Figure~\ref{fig:hardware-design}, to send data at the transmission path, the {\tt Req handler} module in the RoCE kernel receives the request opcode (metadata) issued by the host. The message to be sent is then fetched through the PCie DMA engine and passes through the attestation kernel for processing. The request opcode and the attested message are forwarded to the {\tt Request generation} module that constructs the network packet. 

Upon receiving a message from the network, the RoCE kernel parses the packet header and updates the protocol state (stored in the State tables). The attested message is then forwarded to the attestation kernel. The message is delivered to the application’s (host) memory upon successful verification.
% Upon successful verification, the message is delivered to the application's (host) memory. 

\myparagraph{Hardware design} The RoCE protocol kernel is also connected to a 100Gb MAC IP and an ARP server IP. 

\noindent{\underline{100Gb MAC.}} The 100Gb MAC kernel implements the link layer connecting \projecttitle{} to the network fabric over a 100G Ethernet Subsystem~\cite{license}. The kernel also exposes two interfaces for transmitting (Tx) and receiving (Rx) network packets. 

\noindent{\underline{ARP server.}} 
The ARP server has a lookup table containing MAC and IP address correspondences. Right before the transmission, the RDMA packets at the {\tt Request generation} first pass through a MAC and IP encoding phase, where the {\tt Request generation} module extracts the remote MAC address from the lookup table in the ARP server.
% The ARP server contains a lookup table with the correspondences between MAC and IP addresses. Right before the transmission, the RDMA packets at the {\tt Request generation} first pass through a MAC and IP encoding phase. At this encoding phase, the {\tt Request generation} module extracts from the lookup (ARP table) in the ARP server the remote MAC address.



\begin{algorithm}[t]
\SetAlgoLined
\footnotesize
\textbf{function} \texttt{Attest(c\_id, msg)} \{ \\
\Indp
{\tt cnt} $\leftarrow$ {\tt send\_cnts[c\_id]++};\\
$\alpha$ $\leftarrow$ {\tt hmac(keys[c\_id], msg||ID||cnt)}; \\
\textbf{return} $\alpha${\tt ||msg||ID||cnt};\\
\Indm
\} \\

%\vspace{0.3cm}

\textbf{function} \texttt{Verify(c\_id, $\alpha$||msg||ID||cnt)} \{ \\
\Indp
    $\alpha$' $\leftarrow$ {\tt hmac(keys[c\_id], msg||ID||cnt)};\\
    \textbf{if} {\tt (}$\alpha$' $=$ $\alpha$ {\tt \&\&} {\tt cnt} $=$ {\tt recv\_cnts[c\_id]++} {\tt )} \\
    \Indp
        \textbf{return} ($\alpha${\tt ||msg||cnt)}; \\
    \Indm
    \textbf{assert(False)}; \\
\Indm
\} \\
%\vspace{0.1cm}
\vspace{-1pt}
\caption{\texttt{Attest()} and \texttt{Verify()} functions.}
\label{algo:primitives}
\end{algorithm}

\noindent{\underline{IB protocol.}} 
The RoCE protocol kernel implements the reliable version of the IB protocol. Similarly to its original specification~\cite{rdma_specification}, the kernel implements State tables to store protocol queues (e.g., receive/send/completion queues) as well as important metadata, i.e.,  packet sequence numbers (PSNs), message sequence numbers (MSNs), and a Retransmission Timer. % to detect packet losses.
% The RoCE protocol kernel implements the reliable version of the IB protocol. Similarly to its original specification~\cite{rdma_specification}, the kernel implements the State tables to store the protocol queues QPs (i.e., receive queue, send queue, completion queue, etc.). The State tables also store important metadata such as (1) the packet sequence numbers (PSNs) to distinguish valid, invalid, and duplicate PSN regions, (2) the message sequence numbers (MSNs) to reconstruct fragmented network messages, and (3) a Retransmission Timer to detect packet losses.

\myparagraph{Dataflow}
The transmission path is shown with the blue-colored axes in Figure~\ref{fig:hardware-design}. The {\tt Req handler} receives requests issued by the host and initializes a DMA command to fetch the payload data from the host memory to the attestation kernel. Once the attestation kernel forwards the attested message to the {\tt Req handler}, the module passes the message from several states to append the appropriate headers {\tt IB hdr} along with metadata (e.g., RDMA op-code, PSN, QP number). The last part of the processing generates and appends UDP/IP headers (e.g., IP address, UDP port, and packet length). The message is then forwarded to the 100Gb MAC module. 
 
In the reception path (red-colored axes in Figure~\ref{fig:hardware-design}), the {\tt Request decoder} extracts the headers, metadata, and attested message. The message is forwarded to the attestation kernel for verification and finally copied to the host memory.

% \myparagraph{{Message format:}} The IB protocol in \projecttitle{} processes the following headers: IP, UDP, BTH (Base Transport Header), RETH (RDMA Extended Transport Header), and AETH (ACK Extended Transport Header) as defined in the RDMA Protocol Specification~\cite{rdma_specification, storm}.  The RoCE kernel adds the Ethernet header (L2 header), the IPv4/UDP headers (L3 headers), and the IB headers (L4 headers). It also adds a 32b end-to-end CRC (ICRC) that covers all invariant fields of the packet and offers protection beyond the coverage of the Ethernet Frame Checksum (FCS). 
The message format in \projecttitle{} follows the original RDMA specification~\cite{rdma_specification}; only the difference between our \projecttitle{} and the original RDMA messages is that the attestation kernel {\em extends} the payload by appending a 64B attestation $\alpha$ and the metadata. The metadata includes a 4B id for the session id of the sender, a 4B ID for the device id (unique per device), and the appropriate {\tt send\_cnt}. This payload extension is negligible and does not harm the network throughput.  
% \atsushi{Can we say something positive: e.g., the RoCE protocol kernel does not need to be aware of this difference, or its extension doesn't badly affect the performance, etc.}\dimitra{Both are true, feel free to write in-place}



\subsection{\projecttitle{} Attestation Protocol} 
\label{subsec:nic_controller}
We design a remote attestation protocol to ensure that the \projecttitle{} device is genuine and the \projecttitle{} bitstream and secrets are flashed securely in the device. % along with the designer's configuration data (IPs, secrets, etc.).

\if 0
\myparagraph{Design assumptions} We base our design on a NIC Controller hardware component that drives the device initialization with no access to confidential information as in~\cite{10.1145/3503222.3507733}. 
The Controller can be implemented as a soft CPU~\cite{microblaze, nios, 10.1145/3503222.3507733} while after \projecttitle{}'s initialization, it monitors JTAG/ICAP interfaces to prevent physical attacks~\cite{secMon}. 
% The Controller can be implemented as a soft CPU~\cite{microblaze, nios, 10.1145/3503222.3507733} while after \projecttitle{}'s initialization, it monitors JTAG/ICAP interfaces to prevent physical attacks as commercial IPs~\cite{secMon} \atsushi{what do you mean 'as commercial IPs'?}. %ensure that the bitstream is not modified before use and 

\fi

\myparagraph{Boostrapping} 
% \rev{D1}{We assume that the FPGA cards have access to an AES key and the hash of a public encrypted key (ECDSA for Intel, RSA for Xilinx), which are embedded in secure, on-chip, non-volatile storage. The manufacturer, system designer, and IP vendor are trusted entities among each other.}
\rev{D1}{The \projecttitle{} hardware is securely bootstrapped in an untrusted third-party cloud by the Manufacturer, System designer, and IP vendor, who trust each other. At the device construction, the Manufacturer burns \texttt{HW$_{key}$}, a secret key unique to the device. It is possible with commercial FPGA cards that have access to an AES key and the hash of a public encrypted key embedded in secure, on-chip, non-volatile storage (Intel~\cite{intel-secure-dev}, AMD~\cite{amd-bootgen}).} The System designer shares the configuration with the IP vendor and instructs the cloud provider to load the (encrypted) FPGA firmware which is then decrypted with the \texttt{HW$_{key}$}. 
The firmware loads the controller binary \texttt{Ctrl$_{bin}$}, generates a key pair \texttt{Ctrl$_{pub, priv}$} for the specific device and binary, and signs the measurement of the \texttt{Ctrl$_{bin}$} and the \texttt{Ctrl$_{pub}$} with the  \texttt{HW$_{key}$} (\texttt{Ctrl$_{bin}$cert}). %Ctrl_bin_cert = sign{Ctrl_bin, pk(Ctrl_priv)}HW_key_priv
%The cert should include the freshly generated public key of the ctrl keypair %The Protocol designer shares the configurations with the IP vendor over a TLS/SSL connection. %participating machines and any configuration data. 

%\myparagraph{Boostrapping} At the manufacturing construction, the Manufacturer burns a secret, unique to the device~\cite{secure_FPGAs}, key \texttt{HW$_{key}$}. The System designer instructs the cloud provider to load the (encrypted) FPGA firmware from the storage medium using the embedded code in the BootROM device. The firmware is then decrypted using the \texttt{HW$_{key}$}. The firmware loads the controller binary \texttt{Ctrl$_{bin}$}, generates a key pair \texttt{Ctrl$_{pub, priv}$} that is bound to the specific device and binary, and lastly, generates and signs the measurement of the \texttt{Ctrl$_{bin}$} producing \texttt{Ctrl$_{bin}$cert}. The Protocol designer then establishes a TLS/SSL connection with the trusted IP vendor and sends the list of the participating machines and any configuration data. 


\begin{figure}[t!]
    \centering
    % \includegraphics[width=.4\textwidth]{figures/remote_attestation.drawio.pdf}
    \includegraphics[width=0.75\linewidth]{figures/trusted-nic-remote_attestation_fixed-cropped.pdf}
    \caption{\projecttitle{} remote attestation protocol.}
    \label{fig:remote_attestation}
    % \vspace{-3mm}
\end{figure}

\myparagraph{Remote attestation} 
Figure~\ref{fig:remote_attestation} shows \projecttitle{} remote attestation. The IP vendor sends a random nonce \texttt{n} for freshness to the Controller. The IP vendors public key \texttt{IPVendor$_{pub}$} is embedded into the \texttt{Ctrl$_{bin}$}. The Controller generates a certificate \texttt{cert} over the \texttt{Ctrl$_{bin}$cert} and \texttt{n} (2) which signs with \texttt{Ctrl$_{pub}$} and sends it to the IP vendor (3).


The IP vendor verifies the authenticity of the \texttt{cert} (4)---(5) and establishes a TLS connection with the Controller. First, the vendor verifies the authenticity of \texttt{m}  with the \texttt{HW$_{key}$}, ensuring that a genuine \texttt{Ctrl$_{bin}$} and a genuine device has signed \texttt{m} (4). As such, the vendor ensures that the \texttt{Ctrl$_{bin}$}  runs in a genuine \projecttitle{} device by verifying that the measurement of the \texttt{Ctrl$_{bin}$} has been signed with an appropriate device key installed by the Manufacturer. Lastly, the vendor verifies the nonce \texttt{n} and \texttt{cert} to ensure freshness (with the \texttt{Ctrl$_{pub}$} included in \texttt{m}). 

Now, a mutually authenticated TLS connection is established; the IP vendor verifies authenticity by checking for the desired \texttt{Ctrl$_{pub}$} and the Controller checks for it's embedded \texttt{IPVendor$_{pub}$} (6.1)---(6.3).
Once the TLS connection is established the IP Vendor sends the Controller the secrets and the \projecttitle{} bitstream, \texttt{TNIC$_{bit}$}.


\if 0
\myparagraph{Remote attestation} 
Figure~\ref{fig:remote_attestation} shows \projecttitle{} remote attestation. The IP vendor sends a random nonce \texttt{n} for freshness and his public key \texttt{V}$_{pub}$ to the Controller. The Controller performs Diffie–Hellman key exchange and establishes a shared secret communication key \texttt{S$_{key}$} (1).
%The Controller uses the \texttt{V}$_{pub}$ and his own \texttt{Ctrl}$_{priv}$ to perform Diffie–Hellman key exchange with the vendor and establish a shared secret key \texttt{S$_{key}$} (1).
% which allows them from now on to securely communicate over insecure channels (8).
% The IP vendor and the Controller also establish a secure communication channel. The IP vendor generates a random nonce \texttt{n} for freshness and an asymmetric key pair, \texttt{V$_{pub, priv}$} for secure communication with the Controller, and sends the \texttt{n} and the public key \texttt{V}$_{pub}$ to the Controller (7). The Controller uses the \texttt{Vendor}$_{pub}$ and his own \texttt{Controller}$_{priv}$ to perform Diffie–Hellman key exchange with the vendor and establish a shared secret key \texttt{S$_{key}$} which allows them from now on to securely communicate over insecure channels (8).
It then receives the encrypted bitstream \projecttitle{}$_{enc}$ from the vendor (2),  computes its measurement \texttt{h} (3), and generates an attestation report \texttt{m} that contains the following data: the nonce \texttt{n}, the \texttt{h}, the \texttt{Ctrl$_{pub}$}, the measurement of the \texttt{Ctrl$_{bin}$} \texttt{H(Ctrl$_{bin}$)}, and \texttt{Ctrl$_{bin}$cert} (4). The Controller signs \texttt{m} with \texttt{Ctrl$_{priv}$} and sends it to the vendor (5).

The IP vendor verifies the authenticity of the attestation report, completing the remote attestation process (6)---(8). First, the vendor verifies the authenticity of the attestation report \texttt{m} with the \texttt{Ctrl$_{pub}$}, ensuring that a genuine \texttt{Ctrl$_{bin}$} has signed \texttt{m} (6). Afterward, the vendor ensures that the \texttt{Ctrl$_{bin}$} runs in a genuine \projecttitle{} device by verifying that the measurement of the \texttt{Ctrl$_{bin}$} has been signed with an appropriate device key installed by the Manufacturer (7). Lastly, the vendor verifies the nonce \texttt{n} and \texttt{h}, the measurement of \projecttitle{}$_{enc}$, are as expected (8). 
% Upon a successful attestation, it shares the vendor encryption key \texttt{VE$_{key}$} with the Controller (16). 

The remote attestation is now completed. The IP vendor securely shares the vendor encryption key \texttt{VE$_{key}$} with the Controller by encrypting it with the previously established shared secret key \texttt{S$_{key}$} (9). The Controller decrypts the \projecttitle{} bitstream and loads it onto the FPGA (10). 
% Similarly, the vendor shares any configuration data. 
Finally, the IP vendor notifies the System designer about the result of the remote attestation.
\fi


\subsection{Formal Verification of \projecttitle{} Protocols}
\label{subsec::formal_verification_remote_attestation}

\begingroup
\setlength{\abovedisplayskip}{0.2em}
\setlength{\belowdisplayskip}{0.1em}
%\atsushi{@Julian: can you update the section to address A1, A3? We plan to add a summary of our appendix (Tamarin's verification proofs) here to make the paper self-sustaining. }

% \pramod{Please tie to the high-level properties (tranferrable authentication and non-equivocation) of TNIC and then map it down to low-level TNIC protocols.}

\rev{A1, A3}{
We formally verify the safety and security properties of \projecttitle{} hardware using Tamarin~\cite{tamarin-prover}.
The proof details are covered in Appendix \S~\ref{sec:formal-verification-details}.
% , a security protocol verification tool that analyzes symbolic models of protocols specified as multiset rewriting systems. 
% We verify \projecttitle{}'s bootstrapping and the remote attestation protocol that provides a formal model to argue about non-equivocation and transferable authentication. 
% 
Our verification consists of a model for bootstrapping, remote attestation, message transmission, and reception, according to Figure~\ref{fig:remote_attestation}.
% Our verification approach consists of a model for bootstrapping, remote attestation, message transmission, and reception, according to Figure~\ref{fig:remote_attestation}.
This model is augmented with custom \textit{action facts}, which mark the occurrence of defined events in the execution trace. These include:
    \begin{enumerate}[leftmargin=*]
        %\item \(\text{D}_e(x)\), which marks the end of the attestation phase for device \(e\), i.e., a \projecttitle{} device or the IP vendor, with associated connection information \(x\). 
        \item \(\text{D}_e(x)\), which marks the end of the attestation phase for endpoint \(e\), with associated connection information \(x\). 
        \item \(\text{S}_e(\textit{m})\) and \(\text{A}_e(\textit{m})\) marking the sending and accepting of a message \(\textit{m}\), following Algorithm~\ref{algo:primitives}, respectively.
    \end{enumerate}
    %The intended temporal relationship of these action facts is expressed using lemmas. Tamarin is then used to apply automated deduction and equational reasoning to determine whether these lemmas can be falsified for our model. The relation \(\textit{a} ~@~ t_i\) expresses that action fact \(\textit{a}\) occurred at time \(t_i\) in the trace. % We use this relation to express our desired security properties. 
    The intended temporal relationship of these action facts is expressed using lemmas, which Tamarin then proves using automated deduction and equational reasoning. The relation \(\textit{a} ~@~ t_i\) expresses that action fact \(\textit{a}\) occurred at time \(t_i\).
    %in the trace. 
    Using this relation, we can express our desired security properties as follows:
}
    

\if 0
We next present the verified security properties of \projecttitle{}.
\fi
% ; proofs are available in Appendix~\ref{sec:formal-verification-details}. 

% This model imposes the following assumptions: \emph{(i)}~all messages are composed of atomic terms using predefined functions \dimitra{What does that mean :) ?}\julian{I will update this but to quickly answer: tamarin does not consider individual bits, but instead works with more abstract terms like 'keys', 'hashes' and treats them as atomic, i.e. it is impossible to decompose them into individual parts/bits. Hence there is also only a set of defined functions \& equations to manipulate those terms, e.g. dec(enc(m)) = m, but functions that e.g. reverse the byte order of the bits or something similar, are not considered when evaluating what an attacker can do. This is usually fine because they do not change security on the high level we have here, but it would e.g. be less suited to analyzed low level cryptographic operations. In such cases you might want to also analyze the crypto functions you are using etc.} \emph{(ii)}~these cryptographic functions are perfect with no side-effects \dimitra{are not compromised?} \julian{That is one implication, another one is that e.g. hash functions do not produce hash collisions or that you can't extract key information from side channels etc. It's just that in reality any cryptographic primitive can be broken with some probability (usually very small) and tamarin assumes it to be impossible. This is so it tamarin is able to generate true/false statements about security properties. The alternative would be considering all the probabilities of breaking idividual crypto functions and composing them to obtain a overall probability of breaking a  security property. This is possible using a cryptographic model (e.g. using CryptoVerif) but more complex and generally more restrictive in what you can show.} \emph{(iii)}~attackers can read and delete all messages that are sent on the network and modify them in accordance with the set of available functions.



% Our verification work relies on properties of the already analyzed TLS handshake~\cite{tamarin_tls_proof} and proofs the desired security properties of the \projecttitle{} bootstrapping and attestation protocol.  The desired properties are modeled as lemmas that define valid and invalid system states. Tamarin then verifies the specified lemmas, by \emph{(i)} finding at least one trace  (series of state transitions) for the valid control states and \emph{(ii)} showing that there exists no trace leading to invalid states.

\begingroup
% NOTE: temporary fix for the spacing, may need to be adjusted in case it breaks
\setlength{\belowdisplayskip}{-1.2em} 

\myparagraph{Remote attestation}  
\if 1
We model the bootstrapping and remote attestation protocols based on Figure~\ref{fig:remote_attestation}. We define lemmas to ensure the secrecy of the private information involved in the protocol and the main attestation lemma, which holds if and only if: once the last step of the remote attestation protocol is completed, the \projecttitle{} device is in a valid, expected state. 
\fi
\revcont{
We define the main attestation lemma for any \projecttitle{} device \(\textit{tnic}\) and associated IP Vendor \(\textit{ipv}\). The lemma holds if, after the last step of the remote attestation protocol, the \projecttitle{} device is in a valid, expected state:
\begin{equation}
    \begin{split}
        \forall~\textit{ipv}, \textit{tnic}, \textit{c}, t_i.~\text{D}_\textit{ipv}(\textit{c}) ~@~ t_i
        \implies \exists~t_j.~ t_j < t_i \land \text{D}_\textit{tnic}(\textit{c}) ~@~ t_j
    \end{split}
\end{equation}
}

% $
%     \forall~\textit{ipv}, \textit{tnic}, \textit{c}, t_i.~\text{D}_\textit{ipv}(\textit{c}) ~@~ t_i \implies 
%     \exists~t_j.~ t_j < t_i \land \text{D}_\textit{tnic}(\textit{c}) ~@~ t_j
% $.
% }

\myparagraph{Transferable authentication} 
% We extend the analysis of the bootstrapping and attestation protocol in Tamarin with models for the network operations that upon the message transmission and reception execute the {\tt Attest} and {\tt Verify} functions (Algorithm~\ref{algo:primitives}), respectively. In simple words we prove that  \projecttitle{} attestation kernel prevents equivocation by ensuring that two attestations of different messages will never be assigned identical message sequencers ({\tt send\_cnts}) and secondly, the transferable authentication is met because each generated attestation is uniquely verified by the secret keys jointly with the unique sender's \projecttitle{} device identifier.
\if 0
We extend the model with rules for the network operations that execute \texttt{Attest()} and \texttt{Verify()} (Algorithm~\ref{algo:primitives}) upon the message transmission and reception, respectively. The model extension allows us to reason about transferable authentication by defining an additional lemma: any accepted message was sent by an authentic \projecttitle{} device in a valid configuration.
\fi
\revcont{
%The \(\text{S}_e(\textit{m})\) action fact only marks send operations of authentic \projecttitle{} devices in a valid configuration because it is only used after attestation. Therefore, we define the lemma, which states that any accepted message was sent by an authentic \projecttitle{} device in a valid configuration:
We define the lemma, which states that any accepted message was sent by an authentic \projecttitle{} device in a valid configuration:
% $
%     \forall~e1, \textit{m}, t_i.~\text{A}_{e1}(\textit{m}) ~@~ t_i \implies 
%     \exists~e2, t_j.~ t_j < t_i \land \text{S}_{e2}(\textit{m}) ~@~ t_j
% $.
\begin{equation}
    \begin{split}
        \forall~e1, \textit{m}, t_i.~\text{A}_{e1}(\textit{m}) ~@~ t_i
        \implies \exists~e2, t_j.~ t_j < t_i \land \text{S}_{e2}(\textit{m}) ~@~ t_j
    \end{split}
\end{equation}
}

\endgroup

% We extend the model with rules for the network operations that upon the message transmission and reception execute the {\tt Attest} and {\tt Verify} functions (Algorithm~\ref{algo:primitives}), respectively. This allows us to reason about transferable authentication by defining the following additional lemma: Any message that is accepted was sent by an authentic \projecttitle{} device in a valid configuration.


% In simple words we prove that  \projecttitle{} attestation kernel prevents equivocation by ensuring that two attestations of different messages will never be assigned identical message sequencers ({\tt send\_cnts}) and secondly, the transferable authentication is met because each generated attestation is uniquely verified by the secret keys jointly with the unique sender's \projecttitle{} device identifier.

% Formally, we define one additional lemma to help reason about the transferable authentication: \emph{(i)} this message was sent by an authentic \projecttitle{} device in a valid configuration \emph{(ii)} there is no message that was sent before, but not accepted \emph{(iii)} there is no message that was sent after, but accepted before this one \emph{(iii)} this message has not been accepted before.


\myparagraph{Non-equivocation} 
\if 0
We further extend the model by three lemmas that help to reason about non-equivocation as follows: for any message that is accepted, it holds that \emph{(i)} there is no message that was sent before but not accepted, \emph{(ii)} there is no message that was sent after, but accepted before this one, \emph{(iii)} this message has not been accepted before. 
% The model is further extended by three lemmas that help to reason about non-equivocation: for any message that is accepted it holds that \emph{(i)} there is no message that was sent before, but not accepted \emph{(ii)} there is no message that was sent after, but accepted before this one \emph{(iii)} this message has not been accepted before. 
\fi
\revcont{
    We further extend the model by three lemmas that help to reason about non-equivocation. For any message that is accepted, it holds that \emph{(i)} there is no message that was sent before but not accepted:
    % \\ $
    %     \forall~e1, e2, \textit{m}_j, t_i, t_j.~\text{A}_{e1}(\textit{m}_j) ~@~ t_i \land \text{S}_{e2}(\textit{m}_j) ~@~ t_j \implies 
    %     (\forall \textit{m}_k, t_k.~ t_k < t_j \land \text{S}_{e2}(\textit{m}_k) ~@~ t_k \implies \exists~t_l.~ t_l < t_i \land \text{A}_{e1}(\textit{m}_k) ~@~ t_l)
    % $ \\
    \begin{equation}
        \begin{split}
            \forall~e1, e2, \textit{m}_j, t_i, t_j.&~\text{A}_{e1}(\textit{m}_j) ~@~ t_i \land \text{S}_{e2}(\textit{m}_j) ~@~ t_j \\\implies (&\forall \textit{m}_k, t_k.~ t_k < t_j \land \text{S}_{e2}(\textit{m}_k) ~@~ t_k \\
                      &\implies \exists~t_l.~ t_l < t_i \land \text{A}_{e1}(\textit{m}_k) ~@~ t_l)
        \end{split}
    \end{equation}
    \emph{(ii)} there is no message that was sent after, but accepted before:
    % \\ $
    %     \forall~e1, \textit{m}_i, \textit{m}_j, t_i, t_j.~t_i < t_j \land \text{A}_{e1}(\textit{m}_i) ~@~ t_i \land \text{A}_{e1}(\textit{m}_j) ~@~ t_j \implies 
    %     \exists~e2, t_k, t_l.~ t_k < t_l \land \text{S}_{e2}(\textit{m}_k) ~@~ t_k \land \text{S}_{e2}(\textit{m}_l) ~@~ t_l
    % $ \\
    \begin{equation}
        \begin{split}     
        \forall~e1, \textit{m}_i, \textit{m}_j, t_i, t_j.~t_i < t_j \land \text{A}_{e1}(\textit{m}_i) ~@~ t_i \land \text{A}_{e1}(\textit{m}_j) ~@~ t_j \\
        \implies \exists~e2, t_k, t_l.~ t_k < t_l \land \text{S}_{e2}(\textit{m}_k) ~@~ t_k \land \text{S}_{e2}(\textit{m}_l) ~@~ t_l
        \end{split}
    \end{equation}
    \emph{(iii)} this message has not been accepted before:
    % \\ $
    %     \forall~e1, \textit{m}, t_i, t_j.~\text{A}_{e1}(\textit{m}) ~@~ t_i \land \text{A}_{e1}(\textit{m}) ~@~ t_j \implies t_i = t_j
    % $ \\
    \begin{equation}
        \begin{split}
               \forall~e1, \textit{m}, t_i, t_j.~\text{A}_{e1}(\textit{m}) ~@~ t_i \land \text{A}_{e1}(\textit{m}) ~@~ t_j \implies t_i = t_j
        \end{split}
    \end{equation}
    Our complete verification includes additional action facts and lemmas to verify properties like the secrecy of private information and the implications of out-of-band key compromises.
}

To sum up, Tamarin successfully shows that there is no sequence of transitions that leads to any state where our lemmas are violated. Thus, the attestation and transferable authentication lemmas hold for our model, and the counters behave as expected for non-equivocation. 

\endgroup