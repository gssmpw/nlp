\section{Related work}
%\pramod{the related work bucket should not start with individual papers. It should first summarize the field/area/bucket area, and then give example systems, and lastly, it should end with an overall comparison of the the research field with our solution. In this way, you avoid direct comparison with individual work. Can you please re-write it?}

\myparagraph{Trustworthy distributed systems} Classical BFT systems____ provide BFT guarantees at the cost of high complexity, performance, and scalability overheads. \projecttitle{} bridges the gap between BFT and prior limitations, designing a {\em silicon root-of-trust} with generic trusted networking abstractions that materialize the BFT security properties.

\myparagraph{Trusted hardware for distributed systems} Trustworthy systems____ leverage trusted hardware to optimize the performance of classical BFT at the cost of generalization and easy adoption. The systems suffer from high latencies (50us---105ms)____, build large TCBs____, and rely on specific TEEs____. In contrast, our \projecttitle{} aims to offer performance and generality, while our minimalistic TCB is verifiable and unified in the heterogeneous cloud. 

\begin{figure}
    \centering
    \includegraphics[width=0.9\linewidth]{eval-plots/plots/hw_eval_attest_util.pdf}
    \vspace{-10pt}
  \caption{\rev{(d)}{The scalability analysis of \projecttitle{} hardware. The resource usage is normalized to the U280 FPGA capacity.} }
  % \caption{Latency evaluation of send operations for various packet sizes across five competitive network stacks with various security properties.}
    % \felix{Figure: BRAM = RAMB36 (including RAMB18) + URAM. "CARRY8" is the 4th most used resource of the attestation kernel and is used for "Others" as an upper bound for all other resources.}
    % More resource utilization details: https://github.com/dgiantsidi/replication-protos/blob/main/plots/asplos_submission/attestation_kernel_util.md
    \label{fig:scalability}
   %\vspace{-2pt}
\end{figure}

\myparagraph{SmartNIC-assisted systems} Networked systems offer fast network operations with emerging (programmable) SmartNIC devices____. Some of them offload the network functions to the hardware and reduce the host processing and energy overheads____ or re-design generic networking protocols, from RDMA/RoCE to TCP/IP network stacks, on top of FPGA-based SmartNICs for performance____. Others____ build generic execution frameworks to optimize various distributed systems. Our \projecttitle{} follows a similar approach by building a high-performant unified network stack with SmartNICs and extending its security semantics with the properties of non-equivocation and transferable authentication.% and offloading security to the NIC hardware.


%\myparagraph{SmartNIC-assisted network stacks} SmartNICs effectively provide high-performance network stacks. Another line of research____ re-designs generic networking protocols, from RDMA/RoCE to TCP/IP network stacks, on top of FPGA-based SmartNICs for performance. Our \projecttitle{} further extends its security semantics with the properties of non-equivocation and transferable authentication. 

\myparagraph{Programmable HW for network security} Programmable hardware, SmartNICs, and switches are used to shield networking. Recent systems____ leverage programmable switches and FPGAs to offload security processing and boost performance in the context of blockchain systems____ or security functions (e.g., access control, DNS traffic inspection)____. Our \projecttitle{} similarly offloads security into the hardware, but it carefully uses SmartNICs to overcome the processing bottlenecks of the switches.