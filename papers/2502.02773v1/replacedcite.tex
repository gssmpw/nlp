\section{Related Work}
\label{related_work}
\subsection{High Definition (HD) Maps Generation}
HD map generation has been extensively studied in the context of autonomous
driving, with research broadly categorized into online and offline methods. Both
approaches aim to create detailed, accurate maps but differ significantly in
their operational processes and use cases. Online HD map
generations____ estimate locations of map elements based on sensor inputs to
avoid continuous map maintenance. Offline HD map generation methods allow the
aggregation of more information. For example, Zhou et
al.____ automate the HD Map building pipeline with instance
segmentation, mapping, and particle filter-based lane aggregation.

\subsection{LLM for Autonomous Driving}
Recently, the autonomous driving domain saw many diverse applications of LLMs.
DriveGPT4____ and LMDrive____ attempt to approach
autonomous driving in an end-to-end fashion. Elhafsi et al.
____ detect visual anomalies using LLMs to prevent certain
failure modes in autonomous driving. Furthermore, we see numerous works focusing
on either of perception____, prediction____,
or planning and control____ aspects of
autonomous driving. However, to the best of our knowledge, we have not found any
work that leverages LLMs for mapping. The closest work we see are around map
annotations for more awareness of the surroundings. Talk2BEV____
annotates an instantaneous Bird's Eye View (BEV) map with natural language
descriptions of identified map elements (like cars, bikes, etc) using LLMs.
However, their language-enhanced map is frame-dependent, which implies its
single usage. \our\ on the other hand, is a static map and can be reused for
downstream tasks. Moreover, our method focuses on enhancing existing SD maps
without using any sensors.

% \red{hitvarth}
% \begin{itemize}
%     \item end to end AV
%     \item anomaly avoidance
%     \item not much work on mapping
%     \item map annotations: Talk2BEV
%     \item how our work is different
% \end{itemize}
% \begin{figure}
%     \centering
%     \includegraphics[width=0.9\linewidth]{figs/osm.png}
%     \caption{An extract from an example OSM}
%     \label{fig:osm}
% \end{figure}