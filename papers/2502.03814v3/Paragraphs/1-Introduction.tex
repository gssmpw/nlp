\section{Introduction}\label{sec:1-intro}
The rapid advancement of Large Language Models (LLMs) has significantly impacted various fields, including natural language processing and robotics. Initially designed for text generation and completion tasks, LLMs have evolved to demonstrate problem-understanding and problem-solving capabilities~\cite{zhao2023survey, wei2022emergent}. This evolution is particularly vital for enhancing robot intelligence by enabling robots to process information and make decisions on coordination and action accordingly~\cite{kim_survey_2024, jeong2024survey}. With these capabilities, robots can more effectively interpret complex instructions, interact with humans, collaborate with robotic teammates, and adapt to dynamic environments~\cite{wang2024large}. As robotic systems evolve toward more sophisticated applications, integrating LLMs has become a transformative step, bridging the gap between high-level reasoning and real-world robotic tasks.

On the other hand, Multi-Robot Systems (MRS), which consist of multiple autonomous robots working collaboratively \cite{queralta2020collaborative, baxter2007multi}, have shown great potential in applications such as environmental monitoring~\cite{ma2018multi, espina2011multi, tiwari2019multi}, warehouse automation~\cite{li2020mechanism, tsang2018novel, rosenfeld2016human}, and large-scale exploration~\cite{burgard2005coordinated, gao2022meeting}. Unlike single-robot systems, MRS leverages collective intelligence to achieve high scalability, resilience, and efficiency~\cite{queralta2020collaborative}. The distributed nature of tasks across multiple robots allows these systems to be cost-effective by relying on simpler, specialized robots instead of a single highly versatile one. Moreover, MRS provides increased robustness, as the redundancy and adaptability of the collective can often mitigate the failures of individual robots \cite{zhou2018resilient, liu2021distributed}. These features make MRS indispensable in scenarios where the scale, complexity, or risk is beyond the capabilities of a single robot.

Despite their importance, MRS introduces unique challenges, such as ensuring robot communication, maintaining coordination in dynamic and uncertain environments, and making collective decisions that adapt to real-time conditions~\cite{gielis2022critical, an2023multi}. Researchers are working to integrate LLMs into MRS to address the unique challenges associated with deploying and coordinating MRS~\cite{chen_scalable_2024, mandi_roco_2024}. For example, effective communication is essential for the MRS to share knowledge, coordinate tasks, and maintain cohesion in the dynamic environment among individual robots~\cite{gielis2022critical}. LLMs can provide a natural language interface for inter-robot communication, allowing robots to exchange high-level information more intuitively and efficiently instead of predefined communication structures and protocols~\cite{mandi_roco_2024}. Furthermore, the problem-understanding and problem-solving abilities of LLM can enhance the adaptability of MRS when given a particular goal without specific instructions. The LLMs can understand the mission, divide it into sub-tasks, and assign them to individual robots within the team based on their capabilities~\cite{liu_coherent_2024, chen_emos_2024}. The generalization ability across different contexts of LLMs can also allow MRS to adapt to new scenarios without extensive reprogramming, making them highly flexible during the deployment~\cite{wang_dart-llm_2024, yu_co-navgpt_2023}.

\begin{figure} \label{fig:sec-1-llm-mrs}
    \centering
    \includegraphics[width=1\linewidth]{figures/LLM-MRS.png}
    \caption{Overview of the applications of LLMs in MRS as introduced in Sec.~\ref{sec:4-LLM-MRS}.}
\end{figure}

The application of LLMs in MRS also aligns with the growing need for human-robot collaboration~\cite{hunt_survey_2024}. As the operators often do not have expertise in robot systems, using LLMs as a shared interface can enable operators using natural languages to communicate and command the robots to make decisions and complete complex real-world missions~\cite{ahn_vader_2024}. These capabilities enhance the efficiency of MRS and broaden their applicability to domains requiring close human-robot collaboration.


Our paper is inspired by the survey~\cite{guo_large_2024} that comprehensively reviewed LLMs for multi-agent systems where abstract agents primarily serve virtual roles. Multi-agent systems differ from MRS in that the former emphasizes the roles of the agents, while the latter focuses on the interactions between the robots and the physical world. The limited coverage we find regarding MRS in their work pertains to LLM-embodied agents, but it still skims over related work and lacks detailed summaries. Hence, we recognize the necessity of summarizing recent works on using LLMs in MRS for decision-making, task planning, human-robot collaboration, and task execution. Fig.~\ref{fig:sec-1-llm-mrs} illustrates the four categories outlined in this survey paper. We hope this survey can assist researchers in understanding the current progress of using LLMs in MRS, the challenges we face, and the potential opportunities to enhance multi-robot collective intelligence.

We structured our survey paper as follows to better provide a comprehensive introduction to researchers interested in applying LLMs to MRS. Sec.~\ref{sec:2-back} lays the background for the MRS and LLMs for individuals to understand the topics better. Also, we summarize and compare several other existing survey papers about applying LLMs in robotics systems and multi-agent systems in general and explain the necessity of our work on MRS. Then, Sec.~\ref{sec:3-comm} reviews the communication structure among the LLMs in the MRS. After that, we review the usage of LLMs in three levels: (1) high-level task allocation and planning, (2) mid-level motion planning, and (3) low-level action execution in Sec.~\ref{sec:4-LLM-MRS}. Following reviewing the usage of LLMs, we review based on the applications of the MRS embodied by the LLMs in the real world in Sec.~\ref{sec:5-application}. In Sec.~\ref{sec:6-benchmark}, we summarize the existing benchmark standards for evaluating the performance of LLMs in MRS and the existing simulation environments. In Sec.~\ref{sec:7-discussion}, we identify the challenges and limitations we face and the opportunities and future directions to enhance the LLMs' ability to handle MRS coordination and decision-making. Finally, we conclude our paper in Sec.~\ref{sec:8-conclusion}.
