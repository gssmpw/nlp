\section{Communication Types for LLMs in Multi-robot Systems}\label{sec:3-comm}


LLMs demonstrate remarkable abilities in understanding and reasoning over complex information. However, their performance can significantly vary depending on the communication architecture employed~\cite{chen_scalable_2024, liu_bolaa_2023}. This variability becomes particularly pronounced in scenarios involving embodied agents, where each agent operates with its own LLM for autonomous decision-making. The independence of these LLMs introduces unique challenges in maintaining consistency, coordination, and efficiency across the MRS. Understanding these dynamics is critical to optimizing LLM-based communication and decision-making frameworks in MRS.

Liu \etal~\cite{liu_bolaa_2023} provided a comprehensive comparison of LLM-augmented Autonomous Agents (LAAs), analyzing the architectures employed to integrate LLMs into agents. While their work primarily focuses on multi-agent systems rather than exclusively MRS, their insights into LLM architectures and agent orchestration offer valuable inspiration for multi-robot applications. Their study begins with a basic structure where LLMs perform zero-shot inference based solely on task instructions and observations. This architecture is then enhanced with a self-thinking loop, incorporating previous actions and observations into subsequent decision-making rounds to improve contextual consistency. They extended the architecture by incorporating few-shot prompts, including example actions to enhance the LLMs' ability to generate effective decisions.
Regarding multi-agent orchestration, Liu \etal ~proposed a centralized architecture featuring a message distributor, which relays information to individual agents equipped with their own LLMs. These agents independently process the distributed messages to generate actions, as illustrated in Fig.\ref{fig:bolaa_sec3}. As discussed in Sec.\ref{sec:4-LLM-MRS}, several studies have adopted similar self-thinking strategies to improve the consistency and reliability of decisions made by LLMs, demonstrating the utility of this approach in collaborative systems.
\begin{figure}[htbp]
    \centering
    \includegraphics[width=1\linewidth]{figures/liu_bolaa_2023_Orchestrator.pdf}
    \caption{The BOLAA architecture, which employs a controller to orchestrate multiple LAAs~\cite{liu_bolaa_2023}.}
    \label{fig:bolaa_sec3}
\end{figure}

%%%%%%%%%%%%% insert figure from BOLAA %%%%%%%%%%%%%%%%%%%%

Additionally, Chen \etal~\cite{chen_scalable_2024} proposed four communication architectures: a fully decentralized framework (DMAS), a fully centralized framework (CMAS), and two hybrid frameworks that combine the decentralized and centralized frameworks (HMAS-1 and HMAS-2). These frameworks are visually represented in Fig.~\ref{fig:scalable_sec3}. Their study evaluated the performance of these structures in warehouse-related tasks, revealing notable distinctions among them. For scenarios involving six or fewer agents, both CMAS and HMAS-2 demonstrated comparable performance, although CMAS required more steps to complete tasks. In contrast, the performance of DMAS and HMAS-1 was notably inferior. Furthermore, their experiments indicated that HMAS-2 outperformed CMAS in handling more complex tasks, suggesting that hybrid frameworks with optimized structures offer greater scalability and adaptability for intricate multi-robot operations.

\begin{figure}[h]
    \centering
    \includegraphics[width=0.5\linewidth]{figures/chen_scalable_2024_Systems.png}
    \caption{Four LLM-based multi-agent communication architectures introduced in Chen \etal~\cite{chen_scalable_2024}. The circles represent robots that may have actions in the current step and the `LLM' text represents each LLM agent. The overlap between one circle and one `LLM' text means that the robot is delegated with one LLM agent to express its special opinions to other agents. The `LLM' text without the overlapped circle represents a central planning agent.}
    \label{fig:scalable_sec3}
\end{figure}

