
\vspace{-0.1pt}
\section{Background}
\vspace{-0.1pt}

In this section, we provide a high-level overview on mmWave imaging and its properties. 


\subsection{mmWave Imaging} \label{sec:bg_imaging}
\vspace{-0.1pt}

\begin{figure}
\begin{minipage}[t]{0.23\textwidth}
\centering
    \includegraphics[width=\textwidth]{new_figures/aoa_new.png}
    \vspace{-0.23in}
    \caption{\footnotesize{\textbf{mmWave Imaging}} \textnormal{a) mmWave radars  estimate range and angle-of-arrival to produce b) reflection maps.}} 
    \label{fig:aoa}
    \vspace{-0.225in}
\end{minipage}
\hfill
\begin{minipage}[t]{0.23\textwidth}
\centering
        \includegraphics[width=\textwidth]{figures/sos3.jpeg}
    \vspace{-0.23in}
    \caption{\footnotesize{\textbf{Diffuse and Specular Reflections.}} \textnormal{The frequency of a signal changes the type of reflections it experiences for the same surface.}} 
    \label{fig:diffuse}
    \vspace{-0.225in}
\end{minipage}
\end{figure}

A mmWave radar transmits wireless signals with millimeter-wave wavelengths, captures their reflections, and uses them to generate 3D images, as shown in Fig.~\ref{fig:aoa}a. It uses two core signal processing techniques. First, it estimates the distance of each reflection by comparing the frequency between the transmitted and received signals. Second, it measures the angle of reflection (in both azimuth and elevation) by comparing the received signals between different antennas. 3D images can then be performed by calculating the reflection power emanating from each voxel in 3D space. 

The captured image can be represented as a heatmap as shown in Fig.~\ref{fig:aoa}b, where red represents high intensity reflection and blue represents low intensity. It is important to note that this visualization is a 2D projection of the 3D image. Mathematically, since every sample from the radar is a complex number with a phase and an amplitude, we can combine them with the following equation: 


\vspace{-0.1in}
{\eqsize
\begin{equation}
    I(x,y,z) = \sum_{k=1}^{K} \sum_{j=1}^{N} S_{j,k} e^{\frac{j 2 \pi d(x,y,z,k)}{\lambda_j}}
\label{eq:sar}
\end{equation}
}

\noindent where $I(x,y,z)$ is the complex mmWave image at location $(x,y,z)$, $K$ is the number of antennas, $N$ is the number of samples in each signal, $S_{j,k}$ is the j\textsuperscript{th} sample of the received signal from the k\textsuperscript{th} antenna, and  $\lambda_j$ is the wavelength of the signal at sample j, which is proportional to the speed of light c and the frequency f as $\lambda = \frac{c}{f}$. $d(x,y,z,k)=2||(x,y,z)-p_k||$ is the round-trip distance from the k\textsuperscript{th} antenna (at location $p_k$) to the point $(x,y,z)$.

\vspace{-0.1pt}

\subsubsection{Resolution} 
\vspace{-0.1pt}
One important characteristic of a mmWave image is its resolution. The resolution in range (depth) is determined by the \textit{bandwidth}, or the range of frequencies that the signal covers. We compute the depth resolution $\delta_z$ as:

\vspace{-0.1pt}
{\eqsize
\begin{equation}
    \delta_z = \frac{c}{2B}
\end{equation}}

\noindent where $c$ is the speed of light, and $B$ is the bandwidth of the signal. The resolution in the horizontal and vertical dimensions is dependent on the \textit{aperture}, or the distance between the first and last antenna in that dimension.

\vspace{-0.1pt}
{ \eqsize
\begin{equation}
    \delta_x = \frac{\lambda z_0}{2 D_x} \ \ \ \  \ \ \   \delta_y = \frac{\lambda z_0}{2 D_y}
    \label{eq:res}
\end{equation}}

\noindent where $\delta_x$ and $\delta_y$ are the resolution in the x and y dimensions, respectively. $z_0$ is the range to the target, and $D_x$ and $D_y$ are the apertures in x and y dimensions, respectively.

\vspace{-0.1pt}
\subsubsection{Synthetic Aperture Radar (SAR) Imaging}
\vspace{-0.1pt}
As described above, it is desirable to increase the aperture to increase the image resolution (Eq.~\ref{eq:res}). Instead of creating a large aperture through a large number of physical antennas on the radar, it is possible to perform SAR imaging. In this method, one set of transmitters/receivers are moved through the environment, taking measurements from different locations, and constructing a ``synthetic aperture". These measurements can be combined through Eq.~\ref{eq:sar} to provide the same resolution as a physical aperture. 


\vspace{-0.08pt}
\subsection{Impact of Frequency} \label{sec:bg_freq} 
\vspace{-0.04pt}
One important question is what frequency to use for imaging. The frequency introduces an important tradeoff. First, higher frequencies result in higher resolution (as seen in Eq.~\ref{eq:res}). On the other hand, higher frequency signals experience more attenuation, or power loss, as they travel through materials~\cite{textbook_attenuation}. 
This means that high frequency signals 
may have higher noise through thick occlusions than
lower frequencies. This is why our dataset and simulator include multi-spectral (multi-frequency) mmWave images.

When selecting frequencies, we chose mmWave signals because lower frequencies
(e.g., WiFi) have poor resolution~\cite{adib20143d}, while higher ones (e.g., Terahertz) struggle with occlusions~\cite{thz_penetration}. In mmWave spectrum, we selected 24~GHz and 77~GHz since these are the bands approved by the FCC for industrial and commercial use~\cite{fcc_reg} and most commercial mmWave radars operate within them. 

\vspace{-0.01in}
\subsection{Types of Reflections} \label{sec:bg_refl}
\vspace{-0.03in}


mmWave signals primarily exhibit specular (mirror-like) reflections~\cite{mmwave_specular}, as illustrated in Fig.~\ref{fig:diffuse}. Due to the significantly longer wavelengths of mmWave signals compared to visible light (millimeters vs nanometers), surfaces that appear diffuse (i.e., omnidirectional) for visible light (blue) often appear specular for mmWave signals (red).

In addition to specular reflections, mmWaves also exhibit strong reflections when a wave strikes a sharp edge, scattering it in multiple directions.
These reflections, known as edge diffraction, can impact signal propagation and perception differently from diffuse or specular reflections.

These reflection properties are one fundamental difference between mmWave signals and visible light that necessitates algorithms and models specifically designed for mmWave images.
