
\vspace{-0.05in}
\section{Conclusion}
\vspace{-0.05in}


We present \name, the first millimeter-wave (mmWave) dataset of diverse, everyday objects in both line-of-sight and fully-occluded settings. We collect over 24 million frames of mmWave images, and used them to synthesize 550 high-resolution real world 3D (SAR) mmWave images for 76 objects. We also develop an open-source simulator that can be used to generate synthetic images for any 3D mesh. We demonstrate the utility of our contributions in two applications -- NLOS object segmentation and NLOS shape classification -- establishing baselines for future research and paving the way for important applications in robotic perception, logistics, scene understanding.

As the research evolves, it would be interesting to expand the dataset -- incorporating additional items, surfaces, occlusion types, etc. -- which would continue improving the accuracy of the demonstrated NLOS tasks and unlock additional ones. More generally, we hope this work serves as a foundation and motivation for future research in mmWave-based NLOS perception, similar to how early RGB datasets accelerated the research in optical-based Computer Vision, and continued driving the field forward as they expanded.
% } \cusuh{looks good to me}

%\cut{As we expand \name\ with real and synthetic images, we hope that it enables the broader community to create the same advancements in mmWave non-line-of-sight perception that other datasets enabled for RGB perception.}

%\ld{We believe that \name\ has the potential to create a new bridge between the wireless sensing and computer vision communities, opening the opportunity to develop geometric and model-driven advancements in non-line-of-sight perception. As we expand \name\ with real and synthetic images, we hope that it enables the broader community to create the same advancements in mmWave non-line-of-sight perception that other datasets enabled for RGB perception.}

%\fa{The first sentence sound like a copy-paste from the intro :)}