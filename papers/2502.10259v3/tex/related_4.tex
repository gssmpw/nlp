
\vspace{-0.1pt}
\section{Related Work}
\vspace{-0.1pt}


\noindent \textbf{mmWave Datasets.}
Recognizing the importance of millimeter-wave perception, past research has made initial steps towards millimeter-wave datasets. However, the vast majority of them are focused either on imaging humans~\cite{human_pose, milipoint, humans_and_cars, milipoint,rahman2024mmvr, wu2024mmhpe} for action recognition, outdoor imaging for autonomous driving~\cite{multimodal_av,av1,hawkeye_haitham,coloradar,kradar,radset,view-of-delft,yang2024autonomous, wang2024vision,guan2024talk2radar}, or imaging of building interiors (walls, hallways)~\cite{brescia2024millinoise,prabhakara2023radarhd}. In the context of higher resolution imaging of objects, the only dataset that we are aware of is focused on guns and knives for TSA and detecting concealed weapons through body scanners for airport security applications~\cite{human_concealed, concealed_class_1,SquiggleMilli}. Our work, however, is focused on generating a dataset of a larger, diverse set of everyday objects for application such as robotic manipulation and perception. 


\vspace{0.03in}
\noindent \textbf{mmWave Image Simulators.}
Given the growth in millimeter wave devices for networking and sensing, the vast majority of existing millimeter wave simulations have also focused on understanding channel characteristics (e.g., for 5G networks)~\cite{ansys,ChanSim2,ChanSim1} or for imaging humans for action recognition and HCI applications~\cite{hsim_1,hsim_2,hsim_3}. In principle, these simulators could be used also for generating mmWave images of everyday objects; however, existing simulators are extremely expensive, costing tens of thousands of dollars~\cite{ansys} and/or are limited to simulating metallic items and cannot simulate everyday objects with different properties~\cite{SquiggleMilli}. In contrast, we developed an open-source simulator and demonstrated that it generates synthetic data that matches real-world collected data. We believe that this tool combined with advances in vision algorithms can lead to very high accuracy in non-line-of-sight scenarios.


\vspace{0.03in}
\noindent \textbf{mmWave Perception.}
Prior work has demonstrated the use of millimeter waves for various perception tasks, including detection and classification. However, these approaches are limited to specific domains, such as vehicles~\cite{human_detect_av, hawkeye_haitham, car_completion}, pedestrians~\cite{ped_classification}, weapons~\cite{concealed_class_1, concealed_class_2}, or entirely metallic objects~\cite{obj_in_box}. Through a new dataset and simulator, our work aims to expand the application of mmWave perception to a more open domain consisting of diverse categories.