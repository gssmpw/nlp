Diffusion models have emerged as powerful tools for solving inverse problems, yet prior work has primarily focused on observations with Gaussian measurement noise, restricting their use in real-world scenarios.
This limitation persists due to the intractability of the likelihood score, which until now has only been approximated in the simpler case of Gaussian likelihoods.
In this work, we extend diffusion models to handle inverse problems where the observations follow a distribution from the exponential family, such as a Poisson or a Binomial distribution. 
By leveraging the conjugacy properties of exponential family distributions, we introduce the \textit{evidence trick}, a method that provides a tractable approximation to the likelihood score.
In our experiments, we demonstrate that our methodology effectively performs Bayesian inference on spatially inhomogeneous Poisson processes with intensities as intricate as ImageNet images. Furthermore, we demonstrate the real-world impact of our methodology by showing that it performs competitively with the current state-of-the-art in predicting malaria prevalence estimates in Sub-Saharan Africa.


%Diffusion models have recently emerged as powerful generative tools for solving inverse problems, renowned for their high-quality reconstructions and seamless integration with existing iterative solvers. However, most studies primarily address simple linear inverse problems under Gaussian noise assumptions, which fail to capture the complexity of real-world scientific challenges.
% In this work, we extend diffusion solvers to efficiently address general noisy, non-linear inverse problems by approximating posterior sampling. Leveraging conjugate prior distributions, our approach accommodates data with likelihoods from the exponential family, including normal, Poisson, gamma, exponential, binomial, and negative binomial distributions. This innovation significantly broadens the applicability of diffusion models to inverse problems appearing a wider range of scientific fields.




