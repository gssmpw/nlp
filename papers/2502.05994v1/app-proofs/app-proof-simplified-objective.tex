\subsection{Proof of Theorem~\ref{prop-new-objective}} \label{proof-prop-new-objective}
Throughout this section, we denote by $p_{\boldsymbol{\theta}}(\boldsymbol{\theta})$ the marginal distribution of $\boldsymbol{\theta}$. Before proving Theorem~\ref{prop-new-objective} we need to show the following result.
\begin{lemma}
\label{lemma-bound-fubini}
Let $\boldsymbol{\zeta}(\mathbf{x}_t)$ be a Lipschitz continuous function. Suppose the following conditions hold:
\begin{equation}
\label{eq-lemma-finite-conditions}
\begin{aligned}
\mathbb{E}_{\boldsymbol{\theta}\sim  p_{\boldsymbol{\theta}}}\left[\norm{\mathbf{T}_{\boldsymbol{\theta}}(\boldsymbol{\theta})}\right] &< \infty, \\
\mathbb{E}_{\boldsymbol{\theta}\sim  p_{\boldsymbol{\theta}}}\left[\norm{g(\boldsymbol{\theta})}\norm{\mathbf{T}_{\boldsymbol{\theta}}(\boldsymbol{\theta})}\right] &< \infty.
\end{aligned}
\end{equation}
Then, the following integral is finite for all $ t \in [\epsilon, 1] $:
\begin{equation}
\label{eq-finite-integral-zeta-T}
\int p_{\mathbf{x}_0}(\mathbf{x}_0) 
\int p_{\mathbf{x}_t|\mathbf{x}_0}(\mathbf{x}_t)
\vert \boldsymbol{\zeta}(\mathbf{x}_t)^{\top} 
\mathbf{T}_{\boldsymbol{\theta}}(g^{-1}({\mathbf{x}}_{0}))\vert 
\mathrm{d}\mathbf{x}_t \mathrm{d}\mathbf{x}_0 < \infty.
\end{equation}
\end{lemma}


% \begin{lemma}
% \label{lemma-bound-fubini}
% Let $\boldsymbol{\zeta}(\mathbf{x}_t)$ be a Lipschitz continuous function. Assume the following conditions hold:
% \begin{equation}
% \label{eq-lemma-finite-conditions}
% \begin{aligned}
% \mathbb{E}_{\boldsymbol{\theta}\sim  p_{\boldsymbol{\theta}}}\left[\norm{\mathbf{T}_{\boldsymbol{\theta}}(\boldsymbol{\theta})}\right] &< \infty \\
% \mathbb{E}_{\boldsymbol{\theta}\sim  p_{\boldsymbol{\theta}}}\left[\norm{g(\boldsymbol{\theta})}\norm{\mathbf{T}_{\boldsymbol{\theta}}(\boldsymbol{\theta})}\right] &< \infty 
% \end{aligned}
% \end{equation}
% Then, 
% \begin{equation}
% \label{eq-finite-integral-zeta-T}
% \int p_{\mathbf{x}_0}(\mathbf{x}_0)\int  p_{\mathbf{x}_t|\mathbf{x}_0}(\mathbf{x}_t)\vert \boldsymbol{\zeta}(\mathbf{x}_t)^{\top} \mathbf{T}_{\boldsymbol{\theta}}(g^{-1}({\mathbf{x}}_{0}))\vert \mathrm{d}\mathbf{x}_t \mathrm{d}\mathbf{x}_0< \infty
% \end{equation}
% for all $t\in[\epsilon,1]$.
% \end{lemma}

\begin{proof}
We express the conditions in~\eqref{eq-lemma-finite-conditions} in terms of $\mathbf{x}_{0}$. Recall that $\mathbf{x}_0 = g(\boldsymbol{\theta})$, therefore:
\begin{equation}
\label{eq-lemma-finite-conditions-x0}
\begin{aligned}
\mathbb{E}_{\mathbf{x}_{0}\sim  p_{\mathbf{x}_{0}}}\left[\norm{\mathbf{T}_{\boldsymbol{\theta}}(g^{-1}(\mathbf{x}_{0}))}\right] &< \infty \\
\mathbb{E}_{\mathbf{x}_{0}\sim  p_{\mathbf{x}_{0}}}\left[\norm{\mathbf{x}_{0}}\norm{\mathbf{T}_{\boldsymbol{\theta}}(g^{-1}(\mathbf{x}_{0}))}\right] &< \infty.
\end{aligned}
\end{equation}
Let $\boldsymbol{\zeta}(\mathbf{x}_t)$ be a $K$-Lipschitz continuous function with respect to the Euclidean norm $\norm{\cdot}$. Recall from~\eqref{eq-forward-transition-kernel} that $\mathbf{x}_t\vert\mathbf{x}_0$ follows a multivariate normal distribution\footnote{Note that in~\eqref{eq-forward-transition-kernel}, $\mathbf{x}_t$ is defined with dimension $d_x$. However, throughout Section~\ref{sec:method}, we assume $\mathbf{x}_t$ has the same dimension $d$ as $\mathbf{y}$ for consistency with the problem setup.} with mean $\sqrt{\alpha_t}\mathbf{x}_0$ and covariance matrix $v_{t}\mathbf{I}_{d}$. We use the triangle inequality and then the Cauchy-Schwarz inequality to obtain the following bound 
\begin{equation}
\label{eq-triangle-cauchy-bound}
\begin{aligned}
\vert\boldsymbol{\zeta}(\mathbf{x}_t)^{\top} \mathbf{T}_{\boldsymbol{\theta}}(g^{-1}(\mathbf{x}_0)) \vert &\leq  \vert(\boldsymbol{\zeta}(\mathbf{x}_t)- \boldsymbol{\zeta}(\sqrt{\alpha_t}\mathbf{x}_0))^{\top} \mathbf{T}_{\boldsymbol{\theta}}(g^{-1}(\mathbf{x}_0)) \vert + \vert\boldsymbol{\zeta}(\sqrt{\alpha_t}\mathbf{x}_0)^{\top} \mathbf{T}_{\boldsymbol{\theta}}(g^{-1}(\mathbf{x}_0))\vert \\
&\leq\norm{\boldsymbol{\zeta}(\mathbf{x}_t)- \boldsymbol{\zeta}(\sqrt{\alpha_t}\mathbf{x}_0)}\norm{\mathbf{T}_{\boldsymbol{\theta}}(g^{-1}(\mathbf{x}_0))} + \norm{\boldsymbol{\zeta}(\sqrt{\alpha_t}\mathbf{x}_0)}\norm{\mathbf{T}_{\boldsymbol{\theta}}(g^{-1}(\mathbf{x}_0))}
\end{aligned}
\end{equation}
for all $t\in[\epsilon,1]$. Additionally, the Lipschitz continuity of $\boldsymbol{\zeta}$ can be applied to the bound in~\eqref{eq-triangle-cauchy-bound}, resulting in the following bound
\begin{equation}
\label{eq-lipschitz-triangle-bound}
\begin{aligned}
\vert\boldsymbol{\zeta}(\mathbf{x}_t)^{\top} \mathbf{T}_{\boldsymbol{\theta}}(g^{-1}(\mathbf{x}_0)) \vert &\leq K \norm{\mathbf{x}_t-\sqrt{\alpha_t}\mathbf{x}_0}\norm{\mathbf{T}_{\boldsymbol{\theta}}(g^{-1}(\mathbf{x}_0))} + \norm{\boldsymbol{\zeta}(\sqrt{\alpha_t}\mathbf{x}_0)}\norm{\mathbf{T}_{\boldsymbol{\theta}}(g^{-1}(\mathbf{x}_0))} \\ 
&\leq K \norm{\mathbf{x}_t}\norm{\mathbf{T}_{\boldsymbol{\theta}}(g^{-1}(\mathbf{x}_0))} +  K\sqrt{\alpha_t}\norm{\mathbf{x}_0}\norm{\mathbf{T}_{\boldsymbol{\theta}}(g^{-1}(\mathbf{x}_0))} \\ &+\norm{\boldsymbol{\zeta}(\sqrt{\alpha_t}\mathbf{x}_0)}\norm{\mathbf{T}_{\boldsymbol{\theta}}(g^{-1}(\mathbf{x}_0))}
\end{aligned}
\end{equation}
where the final inequality follows from the triangle inequality. The inequality in~\eqref{eq-lipschitz-triangle-bound} represents a bound for the integrand in~\eqref{eq-finite-integral-zeta-T}.
The comparison test for Lebesgue integrability (see~\citep[Proposition 4.16]{Royden_2010}) states that, to show the integral in~\eqref{eq-finite-integral-zeta-T} is finite given the bound in~\eqref{eq-lipschitz-triangle-bound}, it is sufficient to verify that the following integrals are finite:
\begin{subequations}
\begin{align}
\mathcal{I}_{1} &:=  K\sqrt{\alpha_t}\int p_{\mathbf{x}_0}(\mathbf{x}_0) \norm{\mathbf{x}_0}\norm{\mathbf{T}_{\boldsymbol{\theta}}(g^{-1}(\mathbf{x}_0))} \mathrm{d}\mathbf{x}_0 \label{eq-I1}\\
\mathcal{I}_{2} &:= \int p_{\mathbf{x}_0}(\mathbf{x}_0)\norm{\boldsymbol{\zeta}(\sqrt{\alpha_t}\mathbf{x}_0)}\norm{\mathbf{T}_{\boldsymbol{\theta}}(g^{-1}(\mathbf{x}_0))} \mathrm{d}\mathbf{x}_0 \label{eq-I2}\\
\mathcal{I}_{3} &:= K  \int p_{\mathbf{x}_0}(\mathbf{x}_0)\int  p_{\mathbf{x}_t|\mathbf{x}_0}(\mathbf{x}_t) \norm{\mathbf{x}_t}\norm{\mathbf{T}_{\boldsymbol{\theta}}(g^{-1}(\mathbf{x}_0))}  \mathrm{d}\mathbf{x}_t \mathrm{d}\mathbf{x}_0 \label{eq-I3} 
\end{align}
\end{subequations}
We will now show that each $\mathcal{I}_{1}$, $\mathcal{I}_{2}$ and $\mathcal{I}_{3}$ is finite for any $t\in[\epsilon,1]$.
\paragraph{$\mathcal{I}_{1}$ is finite.} This follows immediately from~\eqref{eq-I1} and \eqref{eq-lemma-finite-conditions-x0}.
\paragraph{$\mathcal{I}_{2}$ is finite.}
Let $\mathbf{c}$ be an arbitrary point in the domain of $\boldsymbol{\zeta}$. By using the triangle inequality and the Lipschitz continuity of $\boldsymbol{\zeta}$, the integrand of~\eqref{eq-I2} can be bounded as follows
\begin{equation}
\label{eq-bound-integrand-I2}
\begin{aligned}
\norm{\boldsymbol{\zeta}(\sqrt{\alpha_t}\mathbf{x}_0)}\norm{\mathbf{T}_{\boldsymbol{\theta}}(g^{-1}(\mathbf{x}_0))} 
&\leq \norm{\boldsymbol{\zeta}(\sqrt{\alpha_t}\mathbf{x}_0)-\boldsymbol{\zeta}(\mathbf{c})}\norm{\mathbf{T}_{\boldsymbol{\theta}}(g^{-1}(\mathbf{x}_0))}  + \norm{\boldsymbol{\zeta}(\mathbf{c})}\norm{\mathbf{T}_{\boldsymbol{\theta}}(g^{-1}(\mathbf{x}_0))} \\ 
&\leq   K\norm{\sqrt{\alpha_t}\mathbf{x}_0-\mathbf{c}}\norm{\mathbf{T}_{\boldsymbol{\theta}}(g^{-1}(\mathbf{x}_0))}  +  \norm{\boldsymbol{\zeta}(\mathbf{c})}\norm{\mathbf{T}_{\boldsymbol{\theta}}(g^{-1}(\mathbf{x}_0))} \\ 
&\leq (K\sqrt{\alpha_t} +K\norm{\mathbf{c}} +\norm{\boldsymbol{\zeta}(\mathbf{c})})\norm{\mathbf{x}_0}\norm{\mathbf{T}_{\boldsymbol{\theta}}(g^{-1}(\mathbf{x}_0))}  
\end{aligned}
\end{equation}
where the last inequality follows again by the triangle inequality. Notice that it follows immediately from~\eqref{eq-bound-integrand-I2} and~\eqref{eq-lemma-finite-conditions-x0} that
\begin{equation}
\label{eq-bound-integral-I2}
(K\sqrt{\alpha_t} +K\norm{\mathbf{c}} +\norm{\boldsymbol{\zeta}(\mathbf{c})})\int p_{\mathbf{x}_{0}}(\mathbf{x}_{0})\norm{\mathbf{x}_0}\norm{\mathbf{T}_{\boldsymbol{\theta}}(g^{-1}(\mathbf{x}_0))} d\mathbf{x}_{0} < \infty
\end{equation}
for all $t\in[\epsilon,1]$. Hence, it follows immediately from~\eqref{eq-bound-integral-I2} and the comparison test for Lebesgue integrability that $\mathcal{I}_{2}$ is finite for all $t\in[\epsilon,1]$.
\paragraph{$\mathcal{I}_{3}$ is finite.} Notice that $\mathbf{x}_{t}\overset{d}{=}\sqrt{\alpha_t}\mathbf{x}_{0} + \sqrt{v_t} \mathbf{z}$ with $\mathbf{z}\sim\mathcal{N}_{d}(0,\mathbf{I}_{d})$. Therefore, it follows from~\eqref{eq-I3} that
\begin{equation}
\label{eq-expression-I3-normal}
\mathcal{I}_{3} = K  \int p_{\mathbf{x}_0}(\mathbf{x}_0)\int  p_{\mathbf{z}}(\mathbf{z}) \norm{\sqrt{\alpha_t}\mathbf{x}_{0} + \sqrt{v_t} \mathbf{z}}\norm{\mathbf{T}_{\boldsymbol{\theta}}(g^{-1}(\mathbf{x}_0))}  \mathrm{d}\mathbf{z} \mathrm{d}\mathbf{x}_0 
\end{equation}
where $p_{\mathbf{z}}(\mathbf{z})$ is the density of the standard multivariate normal random variable $\mathbf{z}$. We wish to show that $\mathcal{I}_{3}$ is finite by using the expression in \eqref{eq-expression-I3-normal}. It follows from the triangle inequality that the the integrand of \eqref{eq-expression-I3-normal} satisfies the following inequality
\begin{equation}
\label{bound-integrand-I3}
\norm{\sqrt{\alpha_t}\mathbf{x}_{0} + \sqrt{v_t} \mathbf{z}}\norm{\mathbf{T}_{\boldsymbol{\theta}}(g^{-1}(\mathbf{x}_0))}  \leq \sqrt{\alpha_t}\norm{\mathbf{x}_{0}}\norm{\mathbf{T}_{\boldsymbol{\theta}}(g^{-1}(\mathbf{x}_0))}  + \sqrt{v_t} \norm{\mathbf{z}}\norm{\mathbf{T}_{\boldsymbol{\theta}}(g^{-1}(\mathbf{x}_0))} 
\end{equation}
Given \eqref{bound-integrand-I3}, it is sufficient to show that
\begin{equation}
\label{eq-finite-integral-bound-I3}
\sqrt{\alpha_t}\int p_{\mathbf{x}_{0}}(\mathbf{x}_{0})\norm{\mathbf{x}_{0}}\norm{\mathbf{T}_{\boldsymbol{\theta}}(g^{-1}(\mathbf{x}_0))} \mathrm{d}\mathbf{x}_{0}  + \sqrt{v_t} \int p_{\mathbf{x}_{0}}(\mathbf{x}_{0})\int p_{\mathbf{z}}(\mathbf{z})\norm{\mathbf{z}}\norm{\mathbf{T}_{\boldsymbol{\theta}}(g^{-1}(\mathbf{x}_0))} \mathrm{d}\mathbf{z}\mathrm{d}\mathbf{x}_{0} <\infty 
\end{equation}
for all $t\in[\epsilon,1]$ to conclude, by the comparison test for Lebesgue integrability, that $\mathcal{I}_{3}$ is finite for all $t\in[\epsilon,1]$.

Notice that it follows from  \eqref{eq-lemma-finite-conditions-x0} that
\begin{equation}
\label{eq-first-bound-I3}
\sqrt{\alpha_t}\int p_{\mathbf{x}_{0}}(\mathbf{x}_{0})\norm{\mathbf{x}_{0}}\norm{\mathbf{T}_{\boldsymbol{\theta}}(g^{-1}(\mathbf{x}_0))} \mathrm{d}\mathbf{x}_{0} <\infty
\end{equation}
for all $t\in[\epsilon,1]$. Furthermore, recall that for a standard multivariate normal random variable $\mathbf{z}\sim\mathcal{N}_{d}(0,\mathbf{I}_{d})$ with $\mathbf{z} = (z_{1},\ldots,z_{d})$, the expectation of its $\ell^{2}$-norm satisfies the following bound:
\begin{equation}
\label{eq-bounded-norm-gaussian}
\mathbb{E}_{p_{\mathbf{z}}}[\norm{\mathbf{z}}] \leq \sqrt{\sum_{i=1}^{d}\text{Var}(z_{i}^{2})} =\sqrt{d}.
\end{equation}
where the first inequality follows from Jensen's inequality.
It follows from \eqref{eq-bounded-norm-gaussian}  that
\begin{equation}
\label{eq-bound-z-T-integral}
\begin{aligned}
\int p_{\mathbf{x}_0}(\mathbf{x}_0)\int  p_{\mathbf{z}}(\mathbf{z}) \norm{\sqrt{v_t} \mathbf{z}}\norm{\mathbf{T}_{\boldsymbol{\theta}}(g^{-1}(\mathbf{x}_0))}  \mathrm{d}\mathbf{z} \mathrm{d}\mathbf{x}_0  &= \sqrt{v_{t}}\left(\mathbb{E}_{p_{\mathbf{z}}}[\norm{\mathbf{z}}] \right) \left(\int p_{\mathbf{x}_0}(\mathbf{x}_0)\norm{\mathbf{T}_{\boldsymbol{\theta}}(g^{-1}(\mathbf{x}_0))}\mathrm{d}\mathbf{x}_0 \right) 
\\ &\leq \sqrt{v_{t}d}  \left(\int p_{\mathbf{x}_0}(\mathbf{x}_0)\norm{\mathbf{T}_{\boldsymbol{\theta}}(g^{-1}(\mathbf{x}_0))}\mathrm{d}\mathbf{x}_0 \right) 
\\ &<\infty
\end{aligned}
\end{equation}
for all $t\in[\epsilon,1]$, where the last inequality follows from \eqref{eq-lemma-finite-conditions-x0}. Then, it follows from \eqref{eq-first-bound-I3} and \eqref{eq-bound-z-T-integral} that the inequality in \eqref{eq-finite-integral-bound-I3} is satisfied for all $t\in[\epsilon,1]$. This shows that $\mathcal{I}_{3}$ is finite for all $t\in[\epsilon,1]$.



We have verified that $\mathcal{I}_{1}$, $\mathcal{I}_{2}$ and $\mathcal{I}_{3}$ are finite for any $t\in[\epsilon,1]$. This concludes the proof.
\end{proof}

\begin{proof}[Proof of Theorem~\ref{prop-new-objective}]
In order to show the desired result, it is sufficient to show that 
\begin{equation}
\label{eq-sufficient-expectation}
\mathbb{E}_{t\sim U(\epsilon, 1), \mathbf{x}_t \sim p_{\mathbf{x}_t}}\left[\boldsymbol{\zeta}(\mathbf{x}_t)^{\top}\mathbb{E}_{p_{\tilde{\mathbf{x}}_{0}\vert\mathbf{x}_t}}[\mathbf{T}_{\boldsymbol{\theta}}(g^{-1}(\tilde{\mathbf{x}}_{0}))]\right]  = \mathbb{E}_{t\sim U(\epsilon, 1),\mathbf{x}_0\sim  p_{\mathbf{x}_0}, \mathbf{x}_t \sim p_{\mathbf{x}_t|\mathbf{x}_0}, }\left[\boldsymbol{\zeta}(\mathbf{x}_t)^{\top}\mathbf{T}_{\boldsymbol{\theta}}(g^{-1}({\mathbf{x}}_{0}))\right].
\end{equation}
To begin, we express the LHS of~\eqref{eq-sufficient-expectation} explicitly as an integral:
\begin{equation}
\label{eq-explicit-integral-proof}
\begin{aligned}
&\mathbb{E}_{t\sim U(\epsilon, 1), \mathbf{x}_t \sim p_{\mathbf{x}_t}}\left[\boldsymbol{\zeta}(\mathbf{x}_t)^{\top}\mathbb{E}_{p_{\tilde{\mathbf{x}}_{0}\vert\mathbf{x}_t}}[\mathbf{T}_{\boldsymbol{\theta}}(g^{-1}(\tilde{\mathbf{x}}_{0}))]\right] \\
&=\int_{t}    p_{ U(\epsilon, 1)}(t)\int_{\mathbf{x}_t}  p_{\mathbf{x}_t}(\mathbf{x}_{t})  \int_{\tilde{\mathbf{x}}_0} p_{\tilde{\mathbf{x}}_0\vert \mathbf{x}_t}(\tilde{\mathbf{x}}_{0})\boldsymbol{\zeta}(\mathbf{x}_t)^{\top} \mathbf{T}_{\boldsymbol{\theta}}(g^{-1}(\tilde{\mathbf{x}}_{0})) \mathrm{d}\tilde{\mathbf{x}}_{0}\mathrm{d}\mathbf{x}_t\mathrm{d}t\\
&=\int_{t} p_{ U(\epsilon, 1)}(t)\int_{\mathbf{x}_t} \int_{\tilde{\mathbf{x}}_0} p_{\mathbf{x}_t}(\mathbf{x}_{t})       p_{\tilde{\mathbf{x}}_0\vert \mathbf{x}_t}(\tilde{\mathbf{x}}_{0}) \boldsymbol{\zeta}(\mathbf{x}_t)^{\top}\mathbf{T}_{\boldsymbol{\theta}}(g^{-1}(\tilde{\mathbf{x}}_{0})) \mathrm{d}\tilde{\mathbf{x}}_{0}\mathrm{d}\mathbf{x}_t\mathrm{d}t \\
&=\int_{t} p_{ U(\epsilon, 1)}(t)\int_{\mathbf{x}_t} \int_{\tilde{\mathbf{x}}_0} p_{\mathbf{x}_t|\tilde{\mathbf{x}}_{0}}(\mathbf{x}_{t}) p_{\tilde{\mathbf{x}}_{0}}(\tilde{\mathbf{x}}_{0}) \boldsymbol{\zeta}(\mathbf{x}_t)^{\top}\mathbf{T}_{\boldsymbol{\theta}}(g^{-1}(\tilde{\mathbf{x}}_{0})) \mathrm{d}\tilde{\mathbf{x}}_{0}\mathrm{d}\mathbf{x}_t\mathrm{d}t 
\end{aligned}
\end{equation}
where in the last equality we have used Bayes' theorem as follows
\begin{equation*}
 p_{\mathbf{x}_t}(\mathbf{x}_{t})  p_{\tilde{\mathbf{x}}_0\vert \mathbf{x}_t}(\tilde{\mathbf{x}}_{0})  = p_{\mathbf{x}_t|\tilde{\mathbf{x}}_{0}}(\mathbf{x}_{t}) p_{\tilde{\mathbf{x}}_{0}}(\tilde{\mathbf{x}}_{0}).
\end{equation*}
Lemma~\ref{lemma-bound-fubini} verifies, under the assumptions in the statement of the theorem, a sufficient condition to apply Fubini's theorem as follows
\begin{multline}
\label{eq-fubini-swap-integral}
\int_{\mathbf{x}_t} \int_{\tilde{\mathbf{x}}_0} p_{\mathbf{x}_t|\tilde{\mathbf{x}}_{0}}(\mathbf{x}_{t}) p_{\tilde{\mathbf{x}}_{0}}(\tilde{\mathbf{x}}_{0}) \boldsymbol{\zeta}(\mathbf{x}_t)^{\top}\mathbf{T}_{\boldsymbol{\theta}}(g^{-1}(\tilde{\mathbf{x}}_{0})) \mathrm{d}\tilde{\mathbf{x}}_{0}\mathrm{d}\mathbf{x}_t = \\ \int_{\tilde{\mathbf{x}}_0}   \int_{\mathbf{x}_t}  p_{\tilde{\mathbf{x}}_{0}}(\tilde{\mathbf{x}}_{0}) p_{\mathbf{x}_t|\tilde{\mathbf{x}}_{0}}(\mathbf{x}_{t}) \boldsymbol{\zeta}(\mathbf{x}_t)^{\top}\mathbf{T}_{\boldsymbol{\theta}}(g^{-1}(\tilde{\mathbf{x}}_{0}))\mathrm{d}\mathbf{x}_t\mathrm{d}\tilde{\mathbf{x}}_{0}
\end{multline}
for all $t\in[\epsilon,1]$.
We now plug~\eqref{eq-fubini-swap-integral} into~\eqref{eq-explicit-integral-proof} to obtain
\begin{equation*}
\begin{aligned}
&\mathbb{E}_{t\sim U(\epsilon, 1), \mathbf{x}_t \sim p_{\mathbf{x}_t}}\left[\boldsymbol{\zeta}(\mathbf{x}_t)^{\top}\mathbb{E}_{p_{\tilde{\mathbf{x}}_{0}\vert\mathbf{x}_t}}[\mathbf{T}_{\boldsymbol{\theta}}(g^{-1}(\tilde{\mathbf{x}}_{0}))]\right] \\ 
 &= \int_{t} p_{ U(\epsilon, 1)}(t)\int_{\tilde{\mathbf{x}}_0}  p_{\tilde{\mathbf{x}}_{0}}(\tilde{\mathbf{x}}_{0})  \int_{\mathbf{x}_t}   p_{\mathbf{x}_t|\tilde{\mathbf{x}}_{0}}(\mathbf{x}_{t}) \boldsymbol{\zeta}(\mathbf{x}_t)^{\top}\mathbf{T}_{\boldsymbol{\theta}}(g^{-1}(\tilde{\mathbf{x}}_{0}))\mathrm{d}\mathbf{x}_t\mathrm{d}\tilde{\mathbf{x}}_{0}\mathrm{d}t \\ 
&=\mathbb{E}_{t\sim U(\epsilon, 1),\mathbf{x}_0\sim  p_{\mathbf{x}_0}, \mathbf{x}_t \sim p_{\mathbf{x}_t|\mathbf{x}_0} }\left[\boldsymbol{\zeta}(\mathbf{x}_t)^{\top}\mathbf{T}_{\boldsymbol{\theta}}(g^{-1}({\mathbf{x}}_{0}))\right]
\end{aligned}
\end{equation*}
which shows the desired result.
\end{proof}