The technical details of each experiment discussed in this section can be found in Appendix~\ref{app-experiment-set-up}. The first experiment demonstrates the effectiveness of our method in approximating a posterior distribution that closely aligns with the ground truth obtained via MCMC, using a hierarchical model where the prior can be evaluated. The final two experiments address scenarios where an empirical prior is used, making MCMC infeasible.

\begin{figure*}[t!]
\centering
\includegraphics[width=\textwidth]{plots/main_poisson_images.pdf}
\caption{\textbf{Score-Based Cox Process Results.} \textbf{(a)} (Left) True Cox Process intensity from the ImageNet validation set, transformed using an exponential link function. (Right) Median of the estimated Cox Process intensity posterior distribution using the Score-Based Cox Process method. \textbf{(b)} (Left) True Cox Process Intensity from Sentinel-2 Satellite Imagery of Manhattan, New York City (Right) Median of the estimated Cox Process intensity posterior distribution using the Score-Based Cox Process method.}
\label{fig:imagenet-cox-process}
\end{figure*} 

\subsection{One-dimensional Benchmark Analysis}
\label{sec-1d-synthetic-data}
We developed a simple one-dimensional experiment to illustrate the effectiveness of our method in approximating a posterior distribution that closely aligns with the ground truth obtained through MCMC.
This experiment also serves as a basis for comparing our posterior approximation with that generated by the DPS method. 
We consider the following hierarchical generative model:
\begin{equation*}
y_{i,j} \sim \text{Poisson}(\theta_j), \quad \boldsymbol{\theta} = \exp(\mathbf{x}_0), \quad \mathbf{x}_0 \sim \mathcal{GP}(0, \mathbf{K})
\end{equation*}
for $i = 1, \ldots, N$ and $j = 1, \dots d$, where $d = 30$ and $\mathbf{K}$ is the Gaussian Process (GP) covariance matrix defined by a radial basis function (RBF) kernel with variance 1 and length-scale 0.1. 
Using this model, we generated synthetic observations. We aim to solve the inverse problem of recovering the unknown Poisson intensity $\boldsymbol{\theta}$ from the generated synthetic observations. We used as a prior the true latent variable Gaussian Process distribution. 

We compared the posterior distribution of $\boldsymbol{\theta}$ estimated by our method against the ground-truth MCMC posterior as well as the DPS posterior approximation. 
The results of this comparison are provided in Appendix~\ref{app-results-synthetic-1d}.
Our approach demonstrates significantly better alignment with the ground-truth MCMC posterior. In contrast, DPS fails to accurately capture both the credible intervals and the point estimates.
To further assess robustness, we repeated this experiment using other distributions within the exponential family, for which DPS could not be used. Our method consistently aligned with the ground-truth MCMC posterior distribution. 
% The right panel contrasts the predictive distributions: our method better recovers the data-generating process, whereas DPS shows systematic bias in regions of low observation density.
% To assess robustness, we repeated the experiment with Pareto and Exponential likelihoods (Appendix~\ref{app-results-synthetic-1d}). Our method maintains consistent performance across all likelihood families.


\subsection{Score-Based Cox Process}
\label{sec-experiment-cox-process}
A Cox process, also called a doubly stochastic Poisson process, is a point process that generalizes the Poisson process by allowing its intensity function to be governed by a stochastic process, varying across the underlying mathematical space. The space over which the intensity function is defined is discretized to be a $256 \times 256$ grid. Each grid cell's observation is a Poisson random variable, parameterized by the corresponding intensity value.

To generate synthetic Cox Process observations, we explored multiple intensities including samples from the ImageNet validation dataset, a satellite image, and a map of buildings' heights in London. For each choice of intensity, we drew $N = 50$ event samples according to a Cox Process and allocated 80\% of the grid cells to the training set and the remaining 20\% to the test set.

To address the inverse problem, we employed the ImageNet prior. This prior assumes that $\mathbf{x}_0$ are samples from the ImageNet train dataset. We use the exponential inverse link function. The hierarchical generative model was:
\begin{equation*}
y_{i,j} \sim \text{Poisson}(\theta_j), \quad
\boldsymbol{\theta} = \exp(\mathbf{x}_0), \quad
\mathbf{x}_0 \sim \text{ImageNet}
\end{equation*}
for $i = 1, \ldots, N$, $j = 1, \ldots d$, and where $d = 256\times 256$. We refer to this method as the \textit{``Score-Based Cox Process"}. It should be noted that MCMC inference cannot be used due to the intractability of the prior density.
Figure~\ref{fig:imagenet-cox-process} shows the results of the \textit{``Score-Based Cox Process"} on recovering the true intensity surface. Further experimental results given different values of $N$ and different intensities are provided in Appendix~\ref{app-further-experiment-cox-process}. 



\subsection{Prevalence of Malaria Prevalence in Sub-Saharan Africa}
\label{sec-experiment-malaria}
The \emph{Plasmodium falciparum} parasite rate (PfPR) quantifies the proportion of individuals who have the malaria parasite. The data used to estimate the PfPR consist of the number of positive cases in location $j$, denoted as $y_j$ (detected using rapid diagnostic tests or PCR), out of the total number of individuals examined in the same location, $n_j$. Spatio-temporal mapping of PfPR is typically conducted using GPs~\cite{Bhatt2015-uk}. However, the growing volume of data has rendered full-rank Bayesian inference with GPs computationally impractical. Furthermore, the simple covariance functions commonly used in GPs may be inadequate, necessitating increasingly complex models to accurately predict PfPR across spatial and temporal dimensions \cite{Bhatt2017-tk}. 
Here, we reanalyzed a real-world dataset on PfPR from the Malaria Atlas Project, previously used to monitor malaria trends in Sub-Saharan Africa~\citep{Bhatt2015-uk,Pfeffer2018-cm, Weiss2019-au} --- the continent bearing the highest burden of the disease. 
We ignored temporal aspects, and only aimed to interpolate spatial data across all of Sub-Saharan Africa. We used a grid resolution of $256 \times 256$, equivalent to a $\sim 111 \text{ km}^2$ resolution, and aggregated positive cases and individuals examined to this resolution. Out of the grid, $7,048$ ($10.75$\%) entries had non-missing observations, which were then split into training and test sets in an 80/20 ratio. The hierarchical generative model was:
\begin{equation*}
y_{j} \sim \text{Binomial}(n_j,\theta_j), \; \boldsymbol{\theta} = \sigma(s\,\mathbf{x}_0), \;
\mathbf{x}_0 \sim \text{ImageNet}
\end{equation*}
for $j = 1, \ldots d$, $s = 5$ and where $d = 256\times 256$, and where $\sigma(\cdot)$ is the sigmoid (inverse logit) function.
Figure~\ref{fig:malaria-results} presents the PfPR posterior median and credible interval estimated using our approach. A benchmark analysis comparing our method to the Gaussian Markov Random Field (GMRF) --- considered the state-of-the-art for disease mapping~\citep{Rue2009-ty, Lindgren2011-fv, Heaton2017-vl} --- is provided in Appendix~\ref{app-further-experiment-malaria}. Our results show that our approach performs competitively with the GMRF model.

\begin{figure*}[ht!]
\centering
\includegraphics[width=\textwidth]{plots/malaria_plot_main.pdf}
\caption{\textbf{Prevalence of Malaria in Sub-Saharan Africa Results.} \textbf{(a)} Empirical PfPR. \textbf{(b)} Median of the estimated PfPR posterior distribution. \textbf{(c)} $25$\% quantile of the estimated PfPR posterior distribution. \textbf{(d)} $75$\% quantile of the estimated PfPR posterior distribution. 
The inset plots highlight Nigeria, one of the countries with the highest malaria burden worldwide.
The empty entries either correspond to locations outside Sub-Saharan Africa or the stable spatial limits of \emph{P. falciparum} transmission~\cite{Bhatt2015-uk} }
\label{fig:malaria-results}
\end{figure*}


