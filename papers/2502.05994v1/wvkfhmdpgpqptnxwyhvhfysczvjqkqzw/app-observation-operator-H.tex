Our methodology can be extended to accommodate the presence of an observation matrix $ \mathbf{H} \in \mathbb{R}^{d_y \times d_x} $ with $ d_y \leq d_x $. When $ \mathbf{H} $ is not full-rank, a standard inverse does not exist. To address this, we employ the Moore-Penrose pseudoinverse, defined as:
\begin{equation*}
\mathbf{H}^\dagger := (\mathbf{H}^\top \mathbf{H})^{-1} \mathbf{H}^\top,
\end{equation*}
which is well-defined for any $ \mathbf{H} $ with $ d_y \leq d_x $.  Given the original link function $ g(\cdot) $, as defined in Section~\ref{sec-sampling-link-function}, we introduce a modified link function $ g_{\mathbf{H}}: \mathbb{R}^{d_y} \to \mathbb{R}^{d_x} $, defined as:
\begin{equation*}
g_{\mathbf{H}}(\boldsymbol{\theta}) := \mathbf{H}^{\dagger} g(\boldsymbol{\theta}),
\end{equation*}
where $ \mathbf{H}^\dagger $ ensures a consistent mapping even when $ \mathbf{H} $ is not full-rank. In this framework, the inverse of $ g_{\mathbf{H}} $ is naturally given by:
\begin{equation*}
g_{\mathbf{H}}^{-1}(\mathbf{x}) = g^{-1}(\mathbf{H} \mathbf{x}),
\end{equation*}
where $ g_{\mathbf{H}}^{-1} $ maps $ \mathbf{x}$ into the observation space defined by $ \mathbf{H}$.
All results presented in Section~\ref{sec:method} remain valid under this modification, provided the link function $ g $ is replaced with the modified link function $ g_{\mathbf{H}} $.

Finally, we note that the use of the Moore-Penrose pseudoinverse is analogous to the standard approach employed in many overdetermined statistical problems (see, for example, \citet[Section 3.1.1]{bishop_pattern_recognition}).