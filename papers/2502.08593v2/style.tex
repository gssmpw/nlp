%% Eye-candy
%\usepackage[T1]{fontenc}
\usepackage{lmodern}  % Latin Modern fonts
\usepackage{textcomp} % Additional text symbols
%\usepackage{lmodern}
\usepackage{xspace}                                     % Smart spacing with \xspace
%\usepackage[protrusion=true,expansion=true]{microtype}  % Improve font rendering


% Striking out text
\usepackage[normalem]{ulem}

%% Math
\usepackage{amsmath}
\usepackage{enumitem}
%\usepackage{enumerate}
\usepackage{amsfonts, mathtools,amssymb, amsthm}
\usepackage{thm-restate}
\usepackage{dsfont} % For the indicator symbol
\usepackage{mleftright}

%% Theorems
% \theoremstyle{plain}
\newtheorem{theorem}{Theorem}[section]
\newtheorem{lemma}[theorem]{Lemma}
\newtheorem{proposition}[theorem]{Proposition}
% \newtheorem{model}[theorem]{Model}
% \newtheorem{fact}[theorem]{Fact}
\newtheorem{definition}[theorem]{Definition}
\newtheorem{corollary}[theorem]{Corollary}
\newtheorem{assumption}[theorem]{Assumption}
% \newtheorem{claim}[theorem]{Claim}
% \newtheorem{question}{Question}
%\theoremstyle{remark}
\newtheorem{remark}{Remark}
\newtheorem{problem}[theorem]{Problem}
\newtheorem{example}{Example}

\newtheorem{postulate}[theorem]{Postulate}

\newtheorem{principle}[theorem]{Principle}


% Algorithm environment
\usepackage{algorithm}
\usepackage[noend]{algpseudocode}

% Colors (with names)
\usepackage[dvipsnames,table]{xcolor}
% Bibliography
% \usepackage[numbers]{natbib}
% \usepackage[backref=true,backend=biber,style=alphabetic,maxbibnames=99,minbibnames=10,maxalphanames=100]{biblatex}

% Required for the table of results
\usepackage{multirow}
\usepackage{multicol}
\usepackage{chngpage} % allows for temporary adjustment of side margins

% For the commands such as \capitalisewords
\usepackage{mfirstuc}

% Graphics
\usepackage{tikz}
\usetikzlibrary{arrows}
\usetikzlibrary{calc,decorations.pathmorphing,patterns}

% References and links
%\usepackage[colorlinks,citecolor=blue,bookmarks=true,linktocpage]{hyperref}
\usepackage{hyperref}
\hypersetup{
    colorlinks=true,
    citecolor=ForestGreen,
    filecolor=black,
    linkcolor=VioletRed,
    urlcolor=blue
}

\urlstyle{same}
\usepackage{aliascnt}
\usepackage[numbered]{bookmark}
%\usepackage[capitalise]{cleveref}
\usepackage[capitalise,nameinlink]{cleveref}

\Crefname{assumption}{Assumption}{Assumptions}
\Crefname{postulate}{Postulate}{Postulates}
\crefname{figure}{Figure}{Figures}
\crefname{principle}{Principle}{Principles}

% Shortcuts
\newcommand{\indep}{\perp\mkern-9.5mu\perp}
\newcommand{\nindep}{\centernot{\indep}}
\newcommand{\ceq}{\stackrel{+}{=}}
\newcommand{\cle}{\stackrel{+}{\leq}}
\newcommand{\cge}{\stackrel{+}{\geq}}

\newcommand{\eps}{\ensuremath{\varepsilon}\xspace}
\newcommand{\Algo}{\ensuremath{\mathcal{A}}\xspace} % Algorithm A
\newcommand{\Tester}{\ensuremath{\mathcal{T}}\xspace} % Testing algorithm T
\newcommand{\Learner}{\ensuremath{\mathcal{L}}\xspace} % Learning algorithm L
\newcommand{\property}{\ensuremath{\mathcal{P}}\xspace} % Property P
\newcommand{\task}{\ensuremath{\mathcal{P}}\xspace} % Task P
\newcommand{\class}{\ensuremath{\mathcal{C}}\xspace} % Class C
%\newcommand{\eqdef}{\stackrel{\rm def}{=}}
\newcommand{\eqdef}{:=}
\newcommand{\defeq}{\vcentcolon=}
\newcommand{\eqlaw}{\stackrel{\mathcal{L}}{=}}
\newcommand{\accept}{\textsf{accept}\xspace}
\newcommand{\fail}{\textsf{fail}\xspace}
\newcommand{\reject}{\textsf{reject}\xspace}
\newcommand{\opt}{{\textsc{opt}}\xspace}
\newcommand{\half}{\frac{1}{2}}
\newcommand{\domain}{\ensuremath{\mathcal{X}}\xspace} % Domain of a distribution (default notation)
\newcommand{\distribs}[1]{\Delta\mleft(#1\mright)} % Domain of a distribution (default notation)
\newcommand{\yes}{{\sf{}yes}\xspace}
\newcommand{\no}{{\sf{}no}\xspace}
\newcommand{\dyes}{{\cal Y}}
\newcommand{\dno}{{\cal N}}

\newcommand{\Testing}{\ensuremath{\mathcal{T}}\xspace} % Testing problem
\newcommand{\UTesting}{\ensuremath{\mathcal{T}^{\rm u}}\xspace} % Uniformity testing problem
\newcommand{\Learning}{\ensuremath{\mathcal{L}}\xspace} % Learning problem

% Complexity
\newcommand{\littleO}[1]{{o\mleft( #1 \mright)}}
\newcommand{\bigO}[1]{{O\mleft( #1 \mright)}}
\newcommand{\bigOSmall}[1]{{O\big( #1 \big)}}
\newcommand{\bigTheta}[1]{{\Theta\mleft( #1 \mright)}}
\newcommand{\bigOmega}[1]{{\Omega\mleft( #1 \mright)}}
\newcommand{\bigOmegaSmall}[1]{{\Omega\big( #1 \big)}}
\newcommand{\tildeO}[1]{\tilde{O}\mleft( #1 \mright)}
\newcommand{\tildeTheta}[1]{\operatorname{\tilde{\Theta}}\mleft( #1 \mright)}
\newcommand{\tildeOmega}[1]{\operatorname{\tilde{\Omega}}\mleft( #1 \mright)}
\providecommand{\poly}{\operatorname*{poly}}

% Influence
\newcommand{\totinf}[1][f]{{\mathbf{Inf}[#1]}}
\newcommand{\infl}[2][f]{{\mathbf{Inf}_{#1}(#2)}}
\newcommand{\infldeg}[3][f]{{\mathbf{Inf}_{#1}^{#2}(#3)}}

% Sets and indicators
\newcommand{\setOfSuchThat}[2]{ \left\{\; #1 \;\colon\; #2\; \right\} } 			% sets such as "{ elems | condition }"
\newcommand{\indicSet}[1]{\mathds{1}_{#1}}                                              % indicator function
\newcommand{\indic}[1]{\indicSet{\left\{#1\right\}}}                                             % indicator function
\newcommand{\disjunion}{\amalg}%\coprod, \dotcup...

% Distance
\newcommand{\dtv}{\operatorname{d}_{\rm TV}}
\newcommand{\hellinger}[2]{{\operatorname{d_{\rm{}H}}\mleft({#1, #2}\mright)}}
\newcommand{\kolmogorov}[2]{{\operatorname{d_{\rm{}K}}\mleft({#1, #2}\mright)}}
\newcommand{\totalvardistrestr}[3][]{{\dtv^{#1}\mleft({#2, #3}\mright)}}
\newcommand{\totalvardist}[2]{\totalvardistrestr[]{#1}{#2}}
\newcommand{\kldiv}[2]{{\operatorname{KL}\mleft({#1 \mid\mid #2}\mright)}}
\newcommand{\kldivsym}[2]{{\operatorname{KL}^{\rm sym}\mleft({#1 \mid\mid #2}\mright)}}
%\newcommand{\chisquarerestr}[3][]{{\operatorname{d}^{#1}_{\chi^2}\mleft({#2 \mid\mid #3}\mright)}}
\newcommand{\chisquare}[2]{\chi^2\mleft({#1 \mid\mid #2}\mright)}
%\newcommand{\chisquaresym}[2]{\chisquarerestr[\rm sym]{#1}{#2}}
\newcommand{\dist}[2]{\operatorname{dist}\mleft({#1, #2}\mright)}

% Restriction (functions, sequences, etc.)
\newcommand\restr[2]{{% we make the whole thing an ordinary symbol
  \left.\kern-\nulldelimiterspace % automatically resize the bar with \right
  #1 % the function
  \vphantom{\big|} % pretend it's a little taller at normal size
  \right|_{#2} % this is the delimiter
  }}

% Probability
\newcommand{\proba}{\Pr}
\newcommand{\probaOf}[1]{\proba\mleft[\, #1\, \mright]}
\newcommand{\probaCond}[2]{\proba\mleft[\, #1 \;\middle\vert\; #2\, \mright]}
\newcommand{\probaDistrOf}[2]{\proba_{#1}\mleft[\, #2\, \mright]}
\newcommand{\probaDistrOfCond}[3]{\proba_{#1}\mleft[\, #2 \;\middle\vert\; #3\, \mright]}
\newcommand{\bPr}[1]{\Pr\mleft[#1\mright]}
\newcommand{\bP}[2]{\Pr_{#1}\mleft[#2\mright]}
\newcommand{\bPP}[1]{P_{#1}}
\newcommand{\bQQ}[1]{Q_{#1}}

% Support of a distribution/function
\newcommand{\supp}[1]{\operatorname{supp}\mleft(#1\mright)}

% Expectation & variance
\newcommand{\expect}[1]{\mathbb{E}\mleft[#1\mright]}
\newcommand{\expectCond}[2]{\mathbb{E}\mleft[\, #1 \;\middle\vert\; #2\, \mright]}
\newcommand{\shortexpect}{\mathbb{E}}
\newcommand{\var}{\operatorname{Var}}
\newcommand{\cov}{\operatorname{Cov}}
\newcommand{\bE}[2]{\shortexpect_{#1}\mleft[#2\mright]}
\newcommand{\bEE}[1]{\bE{}{#1}}
\newcommand{\bEEC}[2]{\bE{}{ #1 \;\middle\vert\; #2 }}

% Distributions
\newcommand{\uniform}{\ensuremath{\mathbf{u}}}
\newcommand{\uniformOn}[1]{\ensuremath{\uniform_{ #1 } }}
\newcommand{\geom}[1]{\ensuremath{\operatorname{Geom}\mleft( #1 \mright)}}
\newcommand{\bernoulli}[1]{\ensuremath{\operatorname{Bern}( #1 ) }}
\newcommand{\binomial}[2]{\ensuremath{\operatorname{Bin}\mleft( #1, #2 \mright)}}
\newcommand{\poisson}[1]{\ensuremath{\operatorname{Poisson}\mleft( #1 \mright) }}
\newcommand{\gaussian}[2]{\ensuremath{ \mathcal{N}\mleft(#1,#2\mright) }}
\newcommand{\gaussianpdf}[2]{\ensuremath{ g_{#1,#2}}}
\newcommand{\betadistr}[2]{\ensuremath{ \operatorname{Beta}\mleft( #1, #2 \mright) }}

% Information theory
\newcommand{\condmutualinfo}[3]{ I\mleft(#1 \land #2\mid #3\mright) }
\newcommand{\mutualinfo}[2]{ I\mleft(#1\land #2\mright) }
\newcommand{\entropy}[1]{ H\mleft(#1\mright) }
\newcommand{\condentropy}[2]{ H\mleft(#1 \mid #2 \mright) }
% Norms
\newcommand{\norm}[1]{\lVert#1{\rVert}}
\newcommand{\normone}[1]{{\norm{#1}}_1}
\newcommand{\normtwo}[1]{{\norm{#1}}_2}
\newcommand{\norminf}[1]{{\norm{#1}}_\infty}
\newcommand{\abs}[1]{\left\lvert #1 \right\rvert}
\newcommand{\dabs}[1]{\lvert #1 \rvert}
\newcommand{\dotprod}[2]{ \left\langle #1,\xspace #2 \right\rangle } 			% <a,b>
\newcommand{\ip}[2]{\dotprod{#1}{#2}} 			% shortcut

\newcommand{\vect}[1]{\mathbf{#1}} 			% shortcut

% Ceiling and floor
\newcommand{\clg}[1]{\left\lceil #1 \right\rceil}
\newcommand{\flr}[1]{\left\lfloor #1 \right\rfloor}

% Common sets
\newcommand{\R}{\ensuremath{\mathbb{R}}\xspace}
\newcommand{\C}{\ensuremath{\mathbb{C}}\xspace}
\newcommand{\Q}{\ensuremath{\mathbb{Q}}\xspace}
\newcommand{\Z}{\ensuremath{\mathbb{Z}}\xspace}
\newcommand{\N}{\ensuremath{\mathbb{N}}\xspace}
\newcommand{\cont}[1]{\ensuremath{\mathcal{C}^{#1}}}

% Oracles and variants
\newcommand{\ICOND}{{\sf INTCOND}\xspace}
\newcommand{\EVAL}{{\sf EVAL}\xspace}
\newcommand{\CDFEVAL}{{\sf CEVAL}\xspace}
\newcommand{\STAT}{{\sf STAT}\xspace}
\newcommand{\SAMP}{{\sf SAMP}\xspace}
\newcommand{\COND}{{\sf COND}\xspace}
\newcommand{\PCOND}{{\sf PAIRCOND}\xspace}
\newcommand{\ORACLE}{{\sf ORACLE}\xspace}

%% Terminology
\newcommand{\pdfsamp}{dual\xspace}
\newcommand{\cdfsamp}{cumulative dual\xspace}
\newcommand{\Pdfsamp}{\expandafter\capitalisewords\expandafter{\pdfsamp}}
\newcommand{\Cdfsamp}{\expandafter\capitalisewords\expandafter{\cdfsamp}}

% L_p norms
\newcommand{\lp}[1][1]{\ell_{#1}}

% Convolution
\DeclareMathOperator{\convolution}{\ast}

%% Terminology
\newcommand{\D}{\ensuremath{D}}
\newcommand{\distrD}{\ensuremath{\mathcal{D}}}
\newcommand{\birge}[2][\D]{\Phi_{#2}(#1)}
\newcommand{\iid}{i.i.d.\xspace}

\newcommand{\bool}{\{\pm 1\}}

% Sign
\DeclareMathOperator{\sign}{sign}

%% Roman numerals
\makeatletter
\newcommand{\rom}[1]{\romannumeral #1}
\newcommand{\Rom}[1]{\expandafter\@slowromancap\romannumeral #1@}
\newcommand{\century}[2][th]{\Rom{#2}\textsuperscript{#1}}
\makeatother

% Hyperref and \autoref{} -- names
\renewcommand{\sectionautorefname}{Section} % To have "Section 5" instead of "section 5" with \autoref{}
\renewcommand{\chapterautorefname}{Chapter} % To have "Chapter 5" instead of "chapter 5" with \autoref{}
\renewcommand{\subsectionautorefname}{Section} % To have "Section 5" instead of "subsection 5" with \autoref{}
\renewcommand{\subsubsectionautorefname}{Section} % To have "Section 5" instead of "subsubsection 5" with \autoref{}
\def\algorithmautorefname{Algorithm}

% Line comments
\algnewcommand{\LineComment}[1]{\Statex \(\triangleright\) #1}


\newcommand{\p}{\mathbf{p}}
\newcommand{\q}{\mathbf{q}}
\newcommand{\h}{\mathbf{h}}
\newcommand{\one}{\mathbf{1}}

% Channel/mechanism
\newcommand{\mech}{W}

\newcommand{\cU}{\mathcal{U}}
\newcommand{\cX}{\mathcal{X}}
\newcommand{\cI}{\mathcal{I}}
\newcommand{\cY}{\mathcal{Y}}
\newcommand{\hY}{{\hat{Y}}}
\newcommand{\cJ}{\mathcal{J}}
\newcommand{\cO}{\mathcal{O}}
\newcommand{\cP}{\mathcal{P}}
\newcommand{\cZ}{\mathcal{Z}}
\newcommand{\cW}{\mathcal{W}}
\newcommand{\cE}{\mathcal{E}}
\newcommand{\cN}{\mathcal{N}}
\newcommand{\cM}{\mathcal{M}}

\newcommand{\Exp}{\mathbf{E}}  

\newcommand{\cA}{{\cal A}}
\newcommand{\cB}{{\cal B}}

% \newcommand{\R}{{\mathbb R}}
% \newcommand{\N}{{\mathbb N}}
% \newcommand{\C}{{\mathbb C}}
% \newcommand{\Z}{{\mathbb Z}}
% \newcommand{\Q}{{\mathbb Q}}
% \newcommand{\cH}{{\cal H}}
% \newcommand{\cM}{{\cal M}}
% \newcommand{\cX}{{\cal X}}
% \newcommand{\cY}{{\cal Y}}

% Command added by Dominik from maim.tex
\newcommand{\bx}{\textbf{x}}
\newcommand{\by}{\textbf{y}}
\newcommand{\bX}{\textbf{X}}
\newcommand{\bY}{\textbf{Y}}
\newcommand{\bZ}{\textbf{Z}}
\newcommand{\bC}{\textbf{C}}
\newcommand{\bJ}{\textbf{J}}
\newcommand{\bN}{\textbf{N}}

\newcommand{\cF}{{\cal F}}
%\newcommand{\cN}{{\cal N}}


\newcommand{\be}{\textbf{e}}
\newcommand{\bof}{\textbf{f}}
\newcommand{\bF}{\textbf{F}}

\newcommand{\bb}{\textbf{b}}
\newcommand{\ba}{\textbf{a}}
\newcommand{\bc}{\textbf{c}}
\newcommand{\bg}{\textbf{g}}
\newcommand{\bz}{\textbf{z}}

\newcommand{\sx}{\Sigma_{\bX\bX}}
\newcommand{\se}{\Sigma_{\bE\bE}}
\newcommand{\sz}{\Sigma_{\bZ\bZ}}
\newcommand{\su}{\Sigma_{\bU\bU}}
\newcommand{\sn}{\Sigma_{\bN\bN}}
\newcommand{\sxy}{\Sigma_{\bX Y}}
\newcommand{\szx}{\Sigma_{Z \bX}}

\newcommand{\transp}{\intercal}
%\DeclarePairedDelimiter{\ceil}{\lceil}{\rceil}
\DeclareMathOperator{\Tr}{Tr}
\newcommand{\chisquaredec}{\chi^{(2)}}
\newcommand{\ochisquaredec}{\overline{\chi}^{(2)}}
\newcommand{\uchisquaredec}{\underline{\chi}^{(2)}}
\newcommand{\ouchisquaredec}{\underline{\overline{\chi}}^{(2)}}
\newcommand{\suchisquaredec}{\underline{\breve{\chi}}^{(2)}}

\newcommand{\eg}{\textit{e.g.},\xspace}
\newcommand{\ie}{\textit{i.e.},\xspace}


\newcommand{\Paren}[1]{\mleft(#1\mright)}
%\newcommand{\Brace}[1]{\left[#1\right]}

\newcommand{\ns}{{\textcolor{red}{n}}} % Number of samples
\newcommand{\ab}{\textcolor{cyan!80!black}{k}} % alphabet (domain) size
\newcommand{\numbits}{{\textcolor{orange}{\ell}}}
\newcommand{\dims}{{\textcolor{cyan!70!blue!100}{d}}}
\newcommand{\pbits}{{\textcolor{pink}{s}}}
\newcommand{\priv}{{\textcolor{orange}{\varrho}}}
\newcommand{\dst}{{\textcolor{ForestGreen!80!black}{\eps}}}
\newcommand{\errprob}{{\textcolor{brown}{\delta}}}

\newcommand{\mb}[1]{\ensuremath{\mathbf{#1}}\xspace}
\DeclareMathSymbol{\ast}{\mathbin}{symbols}{"03}
\newcommand{\obs}{\mathbb{S}}

\def\P{\mathbb P}
\newcommand{\GMT}{Gaussian-Mean-Testing}
\newcommand{\TGMT}{Truncated-Gaussian-Mean-Testing}
\DeclareMathOperator*{\E}{\mathbf{E}}
\DeclareMathOperator\erf{erf}
\DeclareMathOperator\erfc{erfc}

% Command related to causal graph
\DeclareMathOperator{\pa}{{pa}}      % for parents
\DeclareMathOperator{\PA}{{PA}}      % for parent sets
\DeclareMathOperator{\ch}{{ch}}      % for parents
\DeclareMathOperator{\sib}{sib}	   % for siblings 
\DeclareMathOperator{\an}{{an}}   % for ancestors
\DeclareMathOperator{\An}{{An}}   % for ancestors
\DeclareMathOperator{\de}{{de}}   % for desendents 
\DeclareMathOperator{\nd}{{nd}}   % not reachable from i 
\makeatletter
\newcommand*\bigcdot{\mathpalette\bigcdot@{.5}}
\newcommand*\bigcdot@[2]{\mathbin{\vcenter{\hbox{\scalebox{#2}{$\m@th#1\bullet$}}}}}
\makeatother

% Tikz graph
\usetikzlibrary{arrows,shapes,plotmarks,positioning}
\usetikzlibrary{decorations.markings}

\tikzset{>=stealth'} 
\tikzstyle{graphnode} = 
   [circle,draw=black,minimum size=22pt,text centered,text
     width=22pt,inner sep=0pt] 
\tikzstyle{var}   =[graphnode,fill=white]
\tikzstyle{vardashed}   =[graphnode,draw=gray,fill=white]
\tikzstyle{obs}   =[graphnode,fill=black,text=white]
\tikzstyle{obsgrey}   =[graphnode,draw=white,fill=lightgray,text=black]
\tikzstyle{par}    =[graphnode,draw=white,fill=red,text=black] 
 \tikzstyle{crucial} =[graphnode,draw=white,fill=yellow,text=black] 
\tikzstyle{fac}   =[rectangle,draw=black,fill=black!25,minimum size=5pt]
\tikzstyle{facprior} =[rectangle,draw=black,fill=black,text=white,minimum size=5pt]
\tikzstyle{edge}  =[draw=white,double=black,very thick,-]
\tikzstyle{blueedge}  =[draw=white,double=blue,very thick,-]
\tikzstyle{rededge}  =[draw=white,double=red,very thick,-]
\tikzstyle{prior} =[rectangle, draw=black, fill=black, minimum size=
5pt, inner sep=0pt]
\tikzstyle{dirprior} = [circle, draw=black, fill=black, minimum
size=5pt, inner sep=0pt]


\tikzstyle{dot_node}=[draw=black,fill=black,shape=circle]

\usepackage[textwidth=1.9cm]{todonotes}
\newcommand{\dominik}[1]{\textcolor{orange}{\textbf{Dominik: #1}}}
\newcommand{\aram}[1]{\textcolor{cyan}{\textbf{Aram: #1}}}
\newcommand{\yuhao}[1]{\textcolor{brown}{\textbf{Yuhao: } #1}}

% \newcommand{\dominik}[1]{}
%  \newcommand{\aram}[1]{}
%  \newcommand{\yuhao}[1]{}

\newcommand{\nceq}{\stackrel{+}{\not=}}

\newcommand\independent{\protect\mathpalette{\protect\independenT}{\perp}}
\def\independenT#1#2{\mathrel{\rlap{$#1#2$}\mkern2mu{#1#2}}}
\newcommand{\ind}{\independent}