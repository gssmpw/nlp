%\documentclass{uai2025} % for initial submission
\documentclass[accepted]{uai2025} % after acceptance, for a revised version; 
\usepackage[utf8]{inputenc}
\usepackage{palatino}
\usepackage{fullpage}
\usepackage[backref=page]{hyperref}
\hypersetup{
    unicode=false,          % non-Latin characters in Acrobat’s bookmarks
    colorlinks=true,        % false: boxed links; true: colored links
    linkcolor=blue,         % color of internal links (change box color with linkbordercolor)
    citecolor=purple,       % color of links to bibliography
    filecolor=magenta,      % color of file links
    urlcolor=cyan           % color of external links
}
\renewcommand\backrefxxx[3]{%
	\hyperlink{page.#1}{$\uparrow$#1}%
}
\usepackage{algorithm}
\usepackage[noend]{algorithmic}
\usepackage{booktabs}       % professional-quality tables
\usepackage{nicefrac}       % compact symbols for 1/2, etc.
\usepackage{microtype}      % microtypography
\usepackage{xcolor}         % colors
\usepackage{color}
\usepackage{mathtools}
\usepackage{amsfonts}
\usepackage{amsthm}
\usepackage{amssymb}
\usepackage{url}
\usepackage[capitalize,noabbrev,nameinlink]{cleveref}
\usepackage{multicol, multirow}
\usepackage{xfrac}
\renewcommand{\algorithmiccomment}[1]{\bgroup\hfill$\rhd$~\footnotesize{\textcolor{violet}{#1}}\egroup}
\usepackage[group-separator={,}, group-minimum-digits=4]{siunitx}
\usepackage[font=small]{caption}
\usepackage{enumitem}

\newcommand{\RETURN}{\STATE \textbf{return} }

%%%%%%%%%%%%%%%%%%%%%%%%%%%%%%%%
\theoremstyle{plain}
\newtheorem{theorem}{Theorem}[section]
\newtheorem{proposition}[theorem]{Proposition}
\newtheorem{lemma}[theorem]{Lemma}
\newtheorem{corollary}[theorem]{Corollary}
\theoremstyle{definition}
\newtheorem{definition}[theorem]{Definition}
\newtheorem{assumption}[theorem]{Assumption}
\theoremstyle{remark}
\newtheorem{remark}[theorem]{Remark}

\newcommand{\eps}{\varepsilon}
\newcommand{\ouralgo}{\texttt{MAD}}
\newcommand{\ouralgolong}{\texttt{MaxAdaptiveDegree}}
\newcommand{\ouralgotworounds}{\texttt{MAD2R}}
\newcommand{\ouralgotworoundslong}{\texttt{MaxAdaptiveDegreeTwoRounds}}
\newcommand{\userweights}{\texttt{UserWeights}}
%\newcommand{\userweights}{\texttt{GetUserWeights}}
\newcommand{\selectalgo}{\texttt{WeightAndThreshold}}
% Since we use \weightalgo to represent the generic algorithm. I suggest calling it ALG. 
%\newcommand{\weightalgo}{\texttt{Weight}}
\newcommand{\weightalgo}{\texttt{ALG}}
\newcommand{\basicalgo}{\texttt{Basic}}
\newcommand{\sets}{\mathcal{S}}
\newcommand{\dmin}{d_{min}}
\newcommand{\dmax}{d_{max}}
\newcommand{\bmin}{b_{min}}
\newcommand{\bmax}{b_{max}}
\newcommand{\dadapt}{d_{adapt}}
\newcommand{\U}{\mathcal{U}}
\newcommand{\pluseq}{\mathrel{+}=}
\newcommand\err[1]{{\scriptstyle (\pm #1)}}

\newcommand{\DeclareAutoPairedDelimiter}[3]{%
  \expandafter\DeclarePairedDelimiter\csname Auto\string#1\endcsname{#2}{#3}%
  \begingroup\edef\x{\endgroup
    \noexpand\DeclareRobustCommand{\noexpand#1}{%
      \expandafter\noexpand\csname Auto\string#1\endcsname*}}%
  \x}
\DeclareAutoPairedDelimiter\p{\lparen}{\rparen}
\DeclareAutoPairedDelimiter\br{\lbrack}{\rbrack}
\DeclareAutoPairedDelimiter\bc{\lbrace}{\rbrace}
\DeclareAutoPairedDelimiter\angle{\langle}{\rangle}
\DeclareAutoPairedDelimiter\ceil{\lceil}{\rceil}
\DeclareAutoPairedDelimiter\floor{\lfloor}{\rfloor}
\DeclareAutoPairedDelimiter\abs{\lvert}{\rvert}   % For absolute value
\DeclareAutoPairedDelimiter\norm{\lVert}{\rVert}  % For norm (double bars)

\newif\iftodos
\todosfalse


\iftodos
\newcommand\todo[1]{\textcolor{red}{TODO: #1}}
\newcommand\morteza[1]{\textcolor{olive}{Morteza: #1}}
\newcommand\mortezaEdit[1]{\textcolor{olive}{#1}}
\newcommand\justin[1]{\textcolor{blue}{Justin: #1}}
\newcommand\ale[1]{\textcolor{cyan}{Ale: #1}}
\newcommand\vincent[1]{\textcolor{pink}{V: #1}}
\else 
\newcommand\todo[1]{}
\newcommand\morteza[1]{}
\newcommand\mortezaEdit[1]{}
\newcommand\justin[1]{}
\newcommand\ale[1]{}
\newcommand\vincent[1]{}
\fi 


% also before submission to see how the non-anonymous paper would look like 
                        
%% There is a class option to choose the math font
% \documentclass[mathfont=ptmx]{uai2025} % ptmx math instead of Computer
                                         % Modern (has noticeable issues)
% \documentclass[mathfont=newtx]{uai2025} % newtx fonts (improves upon
                                          % ptmx; less tested, no support)
% NOTE: Only keep *one* line above as appropriate, as it will be replaced
%       automatically for papers to be published. Do not make any other
%       change above this note for an accepted version.

%% Choose your variant of English; be consistent
\usepackage[american]{babel}
% \usepackage[british]{babel}

%% Some suggested packages, as needed:
\usepackage{natbib} % has a nice set of citation styles and commands
    \bibliographystyle{plainnat}
    \renewcommand{\bibsection}{\subsubsection*{References}}
\usepackage{mathtools} % amsmath with fixes and additions
% \usepackage{siunitx} % for proper typesetting of numbers and units
\usepackage{booktabs} % commands to create good-looking tables
\usepackage{tikz} % nice language for creating drawings and diagrams

%% Self-defined macros
\newcommand{\swap}[3][-]{#3#1#2} % just an example


\title{Toward Universal Laws of Outlier Propagation}

\author[1,*]{Aram~Ebtekar}
\author[2,3,*]{Yuhao~Wang}
\author[3]{Dominik~Janzing}
\affil[ ]{%
    \textsuperscript{1}Independent Researcher\qquad 
    \textsuperscript{2}National University of Singapore\qquad
    \textsuperscript{3}Amazon\qquad\qquad\qquad\qquad
}

\begin{document}
\maketitle

{
    \def\thefootnote{*}\footnotetext{These authors contributed equally to this work.}
}

\begin{abstract}
We argue that Algorithmic Information Theory (AIT)
admits a principled way to quantify outliers 
in terms of so-called \textit{randomness deficiency}. 
For the probability distribution generated by a causal Bayesian network, we show that
the randomness deficiency of the joint state decomposes 
into randomness deficiencies of each causal mechanism, subject to the Independence of Mechanisms Principle.
Accordingly, anomalous 
joint observations can be quantitatively attributed to their root causes, i.e., the mechanisms that behaved anomalously. 
As an extension of
Levin's law of randomness conservation, we show that weak outliers cannot cause strong ones when 
Independence of Mechanisms holds. We show how these information theoretic laws provide a better understanding of the behaviour of outliers defined with respect to existing scores. 
\end{abstract}

\section{Introduction}

Anomaly detection plays a crucial role in business, technology, and medicine. Typical
use cases range from fraud detection in finance and online trading \citep{donoho2004early}, performance drops in manufacturing lines \citep{susto2017anomaly} or cloud computing applications \citep{Gan2021, ma2020automap, hardt2023petshop}, health monitoring in intensive care units \citep{maslove2016errors}, 
%\dominik{ref missing}, 
to explaining extreme weather and climate events \citep{Zscheischler2021}.
It has motivated a vast effort towards developing methodologies relevant to outlier analysis. As example, we refer the reader to some of the early works in statistics and computer science \citep{freeman1995outliers, rocke1996identification, rousseeuw2003robust, aggarwal2017introduction}.
In complex systems, an anomaly will typically cause a large cascades of related anomalies \citep{panjei2022survey}. 
In order to mitigate them, it is not sufficient to merely 
\textit{detect} the anomalies; we must also identify which of the anomalies was the root cause \citep{root_cause_analysis, ikram2022root, li2022causal, hardt2023petshop, wang2023interdependent, wang2023incremental}. 
Thus, we implicitly face the \textit{counterfactual} question of what conditions could have been different to prevent the (usually undesired) anomalous event.

To render a complex system accessible to human understanding, we begin with a causal model of its relevant mechanisms, specifying not only their default behavior, but also their behavior under modifications called \emph{interventions}. Such a model should be modular in two respects. First, we may want to understand the \textit{causal pathway}, along which a perturbation of any part of the system propagates through its components until it generates the event. Second, we want to 
``blame'' some component(s) of the system, while acknowledging that others worked as expected. 

Causal Bayesian networks offer a framework that supports both kinds of modular description, specifying causal relations via a directed acyclic graph (DAG) $G$ with random variables $X_1,\dots,X_n$ as nodes \citep{pearl2009causality, Spirtes1993}. Due to the causal Markov condition \citep{pearl2009causality}, the joint distribution factorizes according to 
\vspace{-0.3cm}
{\small
\begin{equation}\label{eq:markov}
P(X_1,\dots,X_n) = \prod_{i=1}^n P(X_j\mid \PA_j), 
\end{equation}
}
where $\PA_j$ denotes the parents of $X_j$ in $G$, i.e., its direct causes. 
We will think of each conditional distribution $P(X_j\mid \PA_j)$ as an \textit{independent mechanism} of the system, which 
can in principle be changed or replaced without changing the others (see 2.1 and 2.2 in \cite{peters2017elements} for a historic overview). Following this modular view, our root cause analysis will assume that in the case of an anomalous observation, most of the mechanisms worked as expected; thus, the anomaly can be blamed on a small number of mechanisms that act as root causes, in alignment with \cite{Schoelkopf2021}'s 
``sparse mechanism shift hypothesis''.

To our knowledge, \cite{root_cause_analysis} provide the most elaborate formalization of the idea of {\it attributing anomalies to mechanisms}. We develop our concepts starting from this baseline. 

\subsection{Outlier scores from p-values}
To quantitatively attribute an anomalous event to upstream nodes, \cite{root_cause_analysis} first introduce what they call an Information Theoretic (IT) outlier score via
\vspace{-0.2cm}
\begin{equation}
\label{eq:itscore}
\lambda_\tau (x) := -\log P(\tau(X)\geq \tau(x)),
\end{equation}
where $x$ denotes an observation of the random variable $X$, and $\tau: {\cal X} \to \R$ is an appropriate feature statistic, whose choice will be discussed later\footnote{While \cite{root_cause_analysis} use the natural base $e$, we use base $2$ logarithms to align with binary program lengths in algorithmic information theory. In effect, we express the outlier score in units of bits \citep{frank2005indefinite}.}.

$\lambda_\tau$ can be viewed as a statistical test of the null hypothesis that $x$ was sampled from $P$: setting the base of the logarithm to $2$ yields the p-value $2^{-\lambda_\tau(x)}$. A small p-value (or large $\lambda_\tau$) corresponds to an unusual sample under $P$, which can be labeled an outlier. Since $x$ is a single observation, anomaly scoring thus reduces to classical hypothesis testing on a sample size of $1$ \citep{shen2020randomness,vovk2020non}\footnote{Note that \citet{tartakovsky2012efficient} discuss anomalies as change points over multiple observations. However, we focus on anomalies confined to an individual observation.}. 

\subsection{Quantitative root cause analysis in causal Bayesian networks}

To quantitatively attribute an anomalous observation $(x_1,\dots,x_n)$ to different mechanisms,
where each $x_j$ represents a value of the corresponding variable $X_j$, the arXiv version of \cite{root_cause_analysis} define \textit{conditional outlier scores}:
{\small
\[
\lambda_\tau (x_j\mid\pa_j) := 
-\log P(\tau(X_j)\ge \tau(x_j) \mid 
\PA_j=\pa_j).
\]}

They demonstrate that this score can be equivalently interpreted as measuring the outlierness of the noise term\footnote{This aligns in spirit with \cite{backtracking}, where events are also backtracked by attributing them to the noise variables.}.
The feature functions $\tau_j$ can be node-specific, which is essential when the variables $X_j$ have different characteristics (e.g., different dimensionality or data types). 
When the specific choice of $\tau$ is not crucial for the discussion, we will simplify notation by dropping the subscript.
The arXiv version of \cite{root_cause_analysis} extends this framework by introducing a joint outlier score through ``convolution'':
\begin{eqnarray}\label{eq:conv}
\lambda(x_1,\dots,x_n)
&:=& \sum_{j=1}^n \lambda(x_j\mid\pa_j) \\
&-&\log \sum_{i=1}^{n-1} \frac{(\sum_{j=1}
^n \lambda(x_j\mid\pa_j))^i}{i!}.\nonumber
\vspace{-0.2cm}
\end{eqnarray}
The second term serves as a correction factor that ensures the score maintains the properties of an IT score, provided that 
the conditional distributions have densities with respect to the Lebesgue measure. 
\subsection{Monotonicity of scores }

\cite{Okati2024} formalize an intuitive principle about anomaly propagation: unless the connecting mechanisms themselves are also anomalous, a moderate outlier in a cause (measured by $\lambda(x_1)$) should not produce an extreme outlier in its effect (measured by $\lambda(x_2)$). This can be demonstrated for the bivariate causal relation $X_1\to X_2$ as follows:
\begin{lemma}[Weak IT outliers rarely cause strong ones]
\label{lm:con}
If $\lambda$ are IT scores and $X_1$ is a continuous variable, 
the following inequality holds:
\vspace{-0.2cm}
\[
P( \lambda(X_2)\geq \lambda(x_2)\mid \lambda(X_1)\geq c ) \leq 2^{c-\lambda(x_2)}.
\]
\end{lemma}

In other words, whenever we generate an anomaly $x_1$ randomly from $P(X_1\mid \lambda(X_1)\geq c)$
and then generate $X_2$ via its mechanism $P(X_2\mid X_1)$, the resulting outlier score is unlikely to be much larger than $c$.
The intuitive reason is that the probability occupied by events with high scores is much smaller than that of a given
event having a lower score.

Note that  there is no statement for any single $x_1$ -- some of them may generate outliers that are higher with large probability -- but ``most'' of them do not. 
The reason that we cannot make any statement about particular values is crucial for the present paper, because it shows how independence of mechanisms comes into play: whenever one chooses, on purpose, an $x_1$ such that it results in a large outlier downstream, there is no reason that this particular $x_1$ needs to have high outlier score. One can easily think, for instance, of the deterministic relation $X_2=f(X_1)$, where $f$ is a highly non-linear function that 
results in rather anomalous values $x_2$ for some tiny region of values $x_1$ which are not anomalous. 


\subsection{Limitations of current approach}

\paragraph{Rigid definition of outliers}
$\lambda_\tau$ defines outliers in terms of a feature statistic $\tau$ that is chosen in advance. Sometimes, we only know what makes a sample abnormal after seeing it, so we would like to be able to choose $\tau$ with hindsight. For example, suppose $P$ is univariate Gaussian. It seems natural to define $\tau(x) := |x-\mu|$, so that the anomaly detector flags extremely high or low values. However, we might also like to flag observations such as $x=0$ or $x=\mu$: while such $x$ may lie in a region of high density, the probability of obtaining such a specific value is zero under $P$.
Although any other value has also probability zero, one may agree that observing $x=0$ and $x=\mu$ would be surprising because these values are {\it special} (in a sense specified later). 

As another example, suppose each component of a multivariate observation $x$ is the reading of a different sensor. The signal transmission from all $d$ sensors may be broken in such a way that all components equal to the same constant ``idle'' state $c$. Such coincidences also indicate an unusual event.

\cite{Aggarwal2016}
describe a broad variety of different outliers beyond the above toy examples: These 
can be unusual frequencies of words in text documents \citep{mohotti2020efficient}, unexpected patterns in images \citep{yakovlev2021abstraction},
unusually large cliques
in graphs \citep{hooi2016fraudar}, or points lying 
in low density regions \citep{breunig2000lof}. 

\vspace{-0.3cm}
\paragraph{No general decomposition rule}

While \eqref{eq:conv} nicely decomposes the joint outlier score into mechanism-specific conditional scores, 
a key limitation is that this decomposition relies on {\it defining} the joint score based on a sum of conditional scores.  
This does not imply that any reasonable outlier score (e.g., using a generic feature function $\tau$) for the joint observation can be decomposed in this manner.

\subsection{Our contributions}


The contribution of this paper is purely conceptual. We do not propose 
another outlier detection or root cause analysis method, but instead provide a theoretical framework for calibrating and interpreting outlier scores. The framework ensures that outlier scores meet three critical criteria:
(i) Comparability across diverse probability spaces and data modalities. (ii) Non-increasingness along causal chains of downstream effects, regardless of variable modalities. (iii) Well-defined attribution of joint system outlier scores to anomalies of mechanisms.


Our ideas were guided by the following general working hypothesis, which we believe applies far beyond the subject of this paper: 
\begin{principle}[Information Theory as a Guide]\label{pr:it}
Good information theoretic concepts enable many nice theorems, but they are often hard to work with in practice. 
However, together with distributional assumptions 
(e.g. Gaussianity) they can boil down to simple concepts (e.g. linear algebraic expressions). The resulting formulae may be valid and useful beyond the distributional assumptions (e.g. by virtue of linear algebra).  
\end{principle}
While Shannon information is sometimes hard to
estimate from small sample sizes, {\it algorithmic} information is even worse: Kolmogorov complexity is not even computable. Furthermore, its identities hold only up to machine-dependent additive constants. Therefore, we owe it to the reader to show that our algorithmic information theoretic concepts trigger insights that can be applied in practice, as we will try in \cref{sec:ex}. 

The paper is structured as follows. \cref{sec:key}
describes concepts from statistics, information theory, and causality that we build upon. \cref{sec:dec}
derives the decomposition, and \cref{sec:mon} shows that the joint score is non-increasing under marginalization. \cref{sec:ex} discusses simple examples and \cref{sec:toy} a toy experiment. For a cleaner exposition, we defer some formal proofs and definitions to the supplementary material.


\section{Key ingredients}
\label{sec:key}

\subsection{Statistical testing with e-values instead of p-values}

While p-values are the most famous measure of evidence in statistical testing, e-values are recently gaining popularity for their superior ability to aggregate evidence across multiple tests \citep{ramdas2024hypothesis}. These values are inconsistently scaled in the literature, with e-values being comparable to reciprocals of p-values and exponentials of algorithmic randomness scores. To remove any obfuscation coming from scaling conventions, we introduce the following definitions.

\begin{definition}
\label{def:tests}
A probability-bounded test (\textbf{p-test}) in ratio form is a statistic $\Lambda:\domain\rightarrow[0,\infty]$  satisfying $\forall \eps>0,
P\left( \Lambda(X)
\ge 1/\eps \right)
\le \eps.$
An expectation-bounded test (\textbf{e-test}) in ratio form is a statistic $\Lambda:\domain\rightarrow[0,\infty]$ satisfying $\E_{X\sim P}\left( \Lambda(X) \right)
\le 1.$
\end{definition}

We say a statistic $\Lambda:\domain\rightarrow[0,\infty]$ is a p-test (or e-test) in \emph{probability form}, if $1/\Lambda$ is a p-test (or e-test) in ratio form. Note that what is commonly called a ``p-value'' is a p-test in probability form, satisfying $\Lambda(X) \leq \epsilon$ with probability at most $\epsilon$. In contrast, what the literature calls an ``e-value'' is an e-test in \emph{ratio} form.

Similarly, a statistic $\lambda:\domain\rightarrow[-\infty,\infty]$ is a p-test (or e-test) in \emph{log form}, if $2^\lambda$ is a p-test (or e-test) in ratio form. Our convention is to use lowercase symbols like $\lambda$ to indicate log form.


By Markov's inequality, every e-test is also a p-test. Conversely, \citet{vovk2021values} describe a number of ways to \emph{calibrate} any given p-test into an e-test. For more details on tests, see \cref{sec:etestsandptests} in the supplementary material.


p-tests provide a straightforward way to control the Type I error rate, i.e., false positives. To ensure that samples from $P$ are labeled as anomalies at a rate no larger than a desired threshold $\epsilon$, we flag only those samples whose p-test scores are above $1/\epsilon$ (i.e., below $\epsilon$ when expressed in probability form).

Since every e-test is also a p-test, e-tests also achieve this false positive rate; however, they are more conservative. In return, e-tests offer many advantages related to composability and optional stopping \citep{grunwald2020safe,ramdas2023game,ramdas2024hypothesis}, and we will find them more convenient for decomposing anomaly scores by mechanism. 

\subsection{Basic notions from algorithmic information theory} 

The intuition behind universal tests is that an observation is anomalous precisely when it is more compressible than usual for the underlying random process. 
To formalize description lengths, we fix a universal prefix-free Turing machine. The conditional Kolmogorov complexity $K(x\mid y)$ is the bit length of the shortest program \( p \) that outputs \( x \) when given access to another string $y$  \citep{Vitanyi19}. It satisfies the Kraft inequality \( \sum_x 2^{-K(x \mid y)} < 1 \). When $y$ is an empty string, we write $K(x)$. Just as Shannon's entropy measures information content for a probability distribution, $K(x)$ measures it for an individual sample $x$.

Using a standard prefix-free encoding of \( n \)-tuples, \citet{Vitanyi19} also define the joint Kolmogorov complexity \( K(x_1, \dots, x_n) \). By analogy with Shannon's mutual information, they define the algorithmic mutual information between $x$ and $y$, conditional on $z^*$, by
\vspace{-0.2cm}
\begin{align*}
I(x:y\mid z^*) 
&:= K(x\mid z^*) + K(y\mid z^*) - K(x,y\mid z^*) \\
&\ceq K(x\mid z^*) - K(x\mid (y,z)^*).
\end{align*}
Here, $z^*$ denote a shortest program that outputs $z$, and $\ceq$ denotes equality up to a constant dependent on the universal machine, but not on $x$ or $y$\footnote{Note that $x^*$ contains more information than $x$, because $x$ is easily generated from $x^*$ but not vice versa. Conditioning on $x$ instead of $x^*$ would result in error terms that are often constant, and at worst logarithmic in the length of $x$ \citep{gacs2021lecture}.}. We say $x$ and $y$ are conditionally independent, given $z$, if $I(x:y\mid z^*)\ceq 0$.


\subsection{Universal tests}

A Martin-Löf or Levin test (i.e., semicomputable p-test or e-test, respectively) can be thought of as combining all computable anomaly scoring algorithms, in order to detect the broadest possible variety of anomalies. \cref{sec:constructuniversal} in the supplementary material contains full details; here we give a brief overview.

\begin{definition}[Domination property]
For two statistical tests $\lambda_1$ and $\lambda_2$ expressed in log form, we say $\lambda_1$ \emph{dominates} $\lambda_2$ if there exists a constant $c \in \mathbb{R}$ such that for all observations $x$ in the sample space,
\begin{align*}
    \lambda_1(x) \geq \lambda_2(x) - c.
\end{align*}
Intuitively, this means $\lambda_1$ can detect at least all the anomalies that $\lambda_2$ can detect, up to a constant term.
\end{definition}

The domination property provides a natural way to compare the power of different statistical tests. The class of semicomputable tests contains a \emph{universal} test that dominates all the others:

\begin{theorem}[Universality of randomness deficiency]
Let $\cX$ be a discrete space that can be interpreted as a subset of $\{0,1\}^*$ in a canonical way, and $P$
be a computable probability distribution on $\cX$ (i.e., with finite description length). 
Then, the randomness deficiency of $x\in\cX$, defined by
\begin{equation}
\label{eq:universalfirst}
\delta (x) := -\log P(x) - K(x\mid P^*),
\end{equation}
is a universal e-test, dominating all other semicomputable e-tests.
\end{theorem}

The intuition behind \cref{eq:universalfirst} is that typical samples from a distribution $P$ are optimally compressed by encodings of length $-\log P(x)$. When a sample $x$ can be compressed beyond this theoretical limit, we consider it ``anomalous''.
This includes the case where we observe $x=0$ from a discretized centered Gaussian distribution. Despite being the mode of the distribution, $0$ is considered anomalous because its negative log-likelihood $-\log P(x)$ is substantially larger than its Kolmogorov complexity $K(0)\ceq 0$.

Note that when a distributional estimate $\hat{P}$ is inferred from data, the ``reconstruction loss'' $-\log \hat{P}(x)$ can be used as an outlier score \citep{Bishop1993}. Without further considerations, this score is not calibrated because there may be a huge region with low density, and thus all low density points together may still be likely. However, if the term $K(x\mid P^*)$ is also small, we know that $x$ is special within this huge set.\footnote{Note that conformal p-values \citep{Bates2021} do not require any parametric assumptions or density estimation, since their coverage guarantees rely on exchangeability alone. However, statements on {\it conditional} coverage \citep{Barber2019} are too weak for our purpose of calibrating {\it conditional} outlier scores.} 

\begin{example}[Uniform distribution]
When $P$ is uniform over all $2^d$ binary strings of length $d$, \labelcref{eq:universalfirst} becomes $\delta(x) = d - K(x\mid P^*)\ceq d - K(x\mid d)$, measuring the degree to which $x$ is compressible.
\end{example}

\subsection{Algorithmic Markov condition and independence of mechanisms} 

Our goal is to demonstrate how the randomness deficiency decomposes across causal mechanisms in a Bayesian network, subject to postulates that connect algorithmic information theory to causality. To achieve this, we establish a theoretical framework based on fundamental postulates.

\cite{Algorithmic} propose an adaptation of \eqref{eq:markov} that characterizes algorithmic dependencies between {\it individual objects}, rather than statistical dependencies between {\it random variables}:
\begin{postulate}[Algorithmic Markov condition]
\label{algmarkov}
Let $x_1,\dots,x_n$ be binary words describing objects whose causal relations are given by the DAG $G$. Then the joint complexity of $(x_1,\dots,x_n)$ decomposes as
\[
K(x_1,\dots,x_n) \ceq \sum_{j=1}^n K(x_j\mid\pa^*_j).
\]
Furthermore, for any three sets $R,S,T$ of nodes, we have
\[
I(R : S \mid T^*)\ceq 0,
\]
whenever $R$ and $S$ are d-separated by $T$.
\end{postulate}
\cite{Algorithmic} argue that
within a causal Bayesian network, where nodes represent random variables, 
the mechanisms $P_{X_j|\PA_j}$ 
constitute independent information-bearing objects.
Given $m$ observations of $n$-tuples $(x^1_1,\dots,x_n^1), \dots, (x^m_1,\dots,x_n^m)$,
they 
define a DAG $G^m$
that connects
the $n$ mechanisms $P_{X_i|\PA_i}$ with 
the $n\times m$ observations in the following way: Each $x_i^l$ has
the observations $\pa^l_i$ and the \textit{node} $P_{X_i|\PA_i}$ as parents, formalizing the idea that the mechanisms
$P_{X_i|\PA_i}$ also determine how the parents in $G$
influence the respective observation. 
\cref{fig:x--y} shows $G^m$ for the DAG $X\to Y$. 
For our purpose, it is sufficient to consider $G^1$, where we have only one observation $x_j$ from each node $X_j$ in $G$.
Accordingly, we construct \( G^1 \) from \( G \) as follows: replace each variable \( X_j \) with its observation \( x_j \), and make \( P_{X_j | \PA_j} \) a parent of \( X_j \). Each \( P_{X_j | \PA_j} \) becomes a root node, and these are the only root nodes in \( G^1 \). This placement of \( P_{X_j | \PA_j} \) as root nodes (and thus algorithmically independent) formalizes the Independence of Mechanisms\footnote{\cite{Algorithmic} call the algorithmic independence of different $P_{X_j\mid \PA_j}$ \textit{Independence of Mechanisms}. Here we prefer using this term for the underlying idea of representing them as root nodes.}. Applying the algorithmic Markov condition to $G^1$ 
yields:
\begin{lemma}[Conditional irrelevance of other mechanisms and predecessors when parents are given]
Let $X_1,\dots,X_n$ be causally ordered. 
Given its parents $\pa_j$ and the mechanism
$P_{X_j\mid\PA_j}$, none of the 
other mechanisms $(P_{X_i\mid\PA_i})_{i\neq j}$ 
and none of the causal 
predecessors $(x_i)_{i<j}$ enable further compression of $x_j$. That is,
\label{lm:indepinputs}
\begin{align*}
x_j \independent %(x_i)_{x_i\notin \pa_j},
(x_i)_{i<j}, (P_{X_i\mid\PA_i})_{i\ne j}\mid(\pa_j,
P_{X_j\mid\PA_j})^*.
\end{align*}
\end{lemma}


\section{Decomposition of Randomness Deficiency}
\label{sec:dec}

We start by presentation the decomposition of randomness deficiency in the bivariate case. 
Specifically, consider the DAG $X \rightarrow Y$ between a cause $X$ and its effect $Y$. Extending \cref{eq:universalfirst}
with each distribution's shortest representation, define the joint randomness deficiency of outcomes $(x,y)$ with respect to $P_{X,Y}$ by
\begin{align}
\label{eq:deltaxy}
    \delta(x, y)
    \defeq
    - \log P(x, y) - K(x, y \mid (P_{X, Y})^*),
\end{align}
and the conditional randomness deficiency by
\begin{align}
    \label{eq:deltaymidx}
    \delta (y \mid x)
    \defeq 
    -\log P (y \mid x) - K(y \mid (x, P_{Y \mid X})^*).
\end{align}
\begin{restatable}{lemma}{decompositionpair}
\label{lm:deltaxy}(Decomposition of randomness deficiency for a cause-effect pair)
For any two random variables $X \rightarrow Y$ (i.e., $X$ being the cause of $Y$), and for specific observations $x$ and $y$,  
the following equality holds under \cref{algmarkov} for the DAG in \cref{fig:x--y}:
\begin{align*}
    \delta (x, y) \ceq \delta (x) + \delta (y \mid x) 
\end{align*}
\end{restatable}

\noindent The proof relies on \cref{lm:indepinputs}, and is provided in \cref{sec:decomp} of the supplementary.

\begin{figure}
    \centering
\begin{tikzpicture}[
    node distance=0.4cm and 0.5cm,
    every node/.style={draw, circle, minimum size=0.5cm},
    every edge/.style={->, thick}
]
% Nodes
\node (x2) [inner sep=4.5pt] {$x_2$};
\node (S) [left=of x2] {$P_X$};
\node (x1) [above=of x2, inner sep=4.5pt] {$x_1$};
\node (xm) [below=of x2] {$x_m$};

\node (y1) [right=of x1, inner sep=4.5pt] {$y_1$};
\node (y2) [right=of x2, inner sep=4.5pt] {$y_2$};
\node (ym) [right=of xm] {$y_m$};

\node (M) [right=of y2, minimum size=0.4cm, inner sep=1pt] {\small $P_{Y\mid X}$};

\draw[->] (S) -- (x1);
\draw[->] (S) -- (x2);
\draw[->] (S) -- (xm);
\draw[draw=none] (x2) --node[midway, draw=none, fill=none] {$\dots$} (xm);

\draw[->] (x1) -- (y1);
\draw[->] (x2) -- (y2);
\draw[->] (xm) -- (ym);

\draw[->] (M) -- (y1);
\draw[->] (M) -- (y2);
\draw[draw=none] (y2) -- node[midway, draw=none, fill=none] {$\dots$} (ym);
\draw[->] (M) -- (ym);
\end{tikzpicture}
\caption{$x_1, \dots, x_m$ sampled from $P_X$, $y_1, \dots, y_m$ sampled from  $P_{Y \mid X}$.}
\label{fig:x--y}
\end{figure}
\noindent The following example shows why causal assumptions are needed for randomness deficiency to be additive.
\begin{example}[No additivity without Independence of Mechanisms]\label{ex:icm}
Let $\cX=\cY=\{0,1\}^d$
and $P_X$ be the uniform distribution. Furthermore, let $P_{Y|X}$ be the deterministic mechanism
\[
P(y\mid x) = 1 \quad \hbox{ iff } \quad 
y= x\oplus x_0,
\]
where $\oplus$ denotes bitwise XOR and $x_0$
is an algorithmically random string with $K(x_0) = d$. The mechanism $P_{Y\mid X}$ being deterministic implies that we always have $\delta(y\mid x)\ceq 0$.

 
Consider an input $x=x_0$ that violates the conditional independence relation $x \independent P_{Y|X} \mid P_X$. 
Then, $\delta(x)\ceq 0$ because $x$ is random with respect to $P_X$, but $\delta(x,y)\ceq d$ because $(x,y)=(x_0,0^d)$ is a simple function of $x_0$, which is easily deciphered from $P_{X,Y}$. Since $x$ is non-generic relative to $P_{Y|X}$, we find that the randomness deficiency of $(x,y)$ cannot be attributed to $x$, nor to the mechanism generating $y$ from $x$. 

\end{example}

We now generalize \cref{lm:deltaxy} to the multivariate case:

\begin{restatable}[Decomposition of multivariate joint randomness deficiency]{theorem}{decomposition}
\label{thm:dec}
Let the set of strings $x_1, x_2, \dots, x_n$ be causally connected by a directed acyclic graph $G$, so that the causal Markov condition holds for $G^m$. Then the joint randomness deficiency of all strings $x_1, x_2, \ldots, x_n$ decomposes into the conditional randomness deficiencies of the mechanisms:
\[
    \delta(x_1,\dots,x_n) \ceq \sum_{j=1}^n \delta (x_j\mid\pa_j),
\]
where $\delta (x_j\mid\pa_j)$ denotes the randomness deficiency of $x_j$ given its parents.
\end{restatable}
The proof is a simple induction over the number $n$ of nodes according to the causal ordering. The induction step from $n-1$ to $n$ uses \cref{lm:deltaxy} where $y:=x_n$ and $x$ is the single multivariate parent $x:=(x_1,\dots,x_{n-1})$; see \cref{sec:decomp}
in the supplementary for details. 

\section{Monotonicity of Randomness Deficiency}
\label{sec:mon}

In particular, \cref{thm:dec} allows us to draw the following straightforward conclusion:

\begin{theorem}[Weak anomalies do not cause stronger ones]\label{thm:mon}
If there is a unique root cause
$j \in \{1,\dots,n\}$,
in the sense that 
\[
\delta(x_i\mid \pa_i) \ceq 0 \quad \hbox{ for } i\neq j,
\]
and the conditions of \cref{thm:dec} are met, 
\begin{equation}\label{eq:mon}
\text{then, }\qquad 
\delta(x_i)\cle  \delta (x_j\mid \pa_j) \quad \forall i\in \{1,\dots,n\}.
\end{equation}
\end{theorem}
The proof follows easily from \cref{thm:dec} together with Corollary 4.1.11 from \cite{gacs2021lecture} which states non-increasingness of $\delta$ under marginalization and thus 
\[
\delta (x_i) \cle \delta(x_1,\dots,x_n).
\]
In essence, \eqref{eq:mon} states that none of the nodes displays a randomness deficiency that exceeds the conditional randomness deficiency of the root cause. 
In a nutshell, ``weak outliers cannot cause strong ones''. --
We have thus found an AIT version of \cref{lm:con}.\footnote{Hence, in the language of modern resource theories in physics \citep{Coecke2019}, Definition 5.1., an anomaly is a ``resource'' and $\delta$ a ``monotone'' measuring its value.} 
It is worth noting that \citet{levin1984randomness} (see also \citet{gacs2021lecture}) demonstrated a similar principle, calling it \emph{randomness conservation}. It says that the output of a mechanism cannot exhibit a substantially larger randomness deficiency than the input - under the condition that the mechanism itself has a constant description length. In physics, randomness deficiency corresponds to a lack of entropy \citep{zurek1989algorithmic,gacs1994boltzmann}. Therefore, the second law of thermodynamics amounts to an instance of randomness conservation \citep{ebtekar2025foundations}.

In our context, {\it simplicity} of mechanisms is generalized by the {\it independence} of mechanisms. \cref{ex:icm} illustrates why the independence of mechanisms principle is essential. It describes a scenario where a value $y$ with randomness deficiency $d$ emerges, yet this deficiency cannot be attributed to either the input $x$ or the mechanism generating $y$ from $x$.

\section{Relation to computable anomaly scores}
\label{sec:ex}
This section describes simple scenarios in which randomness deficiency boils down to simple and well-known scores. 
One may ask why one should start with an uncomputable concept in the first place to  end up with something simple anyway. The answer is 
that the properties we have shown for randomness deficiency guide us in calibrating simple scores, with desirable properties such as {\it decomposition of joint scores into scores of the mechanisms} and {\it monotonicity of scores}.
\begin{example}[z-score]
For a Gaussian variable $X\sim {\mathcal N}(\mu,\sigma)$, the squared z-score reads
$
z^2(x) := (x-\mu)^2/\sigma^2.
$
At a fixed level of precision, $K(x\mid P^*)\cle 
2\log |x-\mu|$. 
Substituting the log likelihood
\begin{align*}
-\log P(x) &= \log\sqrt{2\pi}\sigma + \frac{\log e}{2} z^2(x),
\end{align*}
and treating $\sigma$ as a constant, yields
\begin{align*}
\delta (x) &\cge 
\frac{\log e}{2} z^2(x)  -
2 \log |x-\mu|.
\end{align*}
\end{example}

Note that while the well-known identity $\Exp[z^2]=1$ tells us that $z^2$ is an e-value, the universal e-value $2^{\delta}\approx e^{z^2/2}$ provides far stronger evidence at the tails.
Now we generalize to $n > 1$ dimensions:
\begin{example}[Decomposing squared Mahalanobis distance]\label{ex:maha}
The randomness deficiency of a random vector $\bx \in \R^n$, drawn from a centered Gaussian, satisfies
\[
\delta(\bx)
\cge \frac{\log e}{2} \bx^T \Sigma^{-1}_\bX \bx
- O(\log \|\bx\|_\infty),
\]
where $\Sigma_\bX$ denotes the covariance matrix
and $n$ is treated as a constant. 
The leading term is $(\log e) / 2$ times the squared Mahalanobis distance (M-distance, for short).

Let $\bX$ be generated by  
a causal Bayesian network with
linear structural equations  
$\bX = A \bX + {\bf N}$, where $A$ is strictly lower triangular and 
 $N_i$ are independent noise variables with variance $\sigma_i^2$. 
Using transformation 
${\bf N} = (I-A) \bX$, we obtain a diagonal form
\[
\bx^T \Sigma^{-1}_\bX \bx = \sum_{i=1}^n \frac{n_i^2}{\sigma_i^2}.
\]
For large $n_i$, the expression $n_i^2/\sigma_2$
is roughly proportional to the randomness deficiency of the mechanism $P(X_i\mid PA_i)$. Hence, 
decomposition of randomness deficiencies translates asymptotically into decompositions of the squared M-distance (as used for multivariate anomaly detection \cite{Aggarwal2016}) into $z^2$-scores. 
Further, M-distance 
is non-increasing with respect to 
marginalization to a subset of variables (see \cref{sec:mahalanobis} in the supplementary), resembling the monotonicity of 
randomness deficiency. 
\end{example}
\cref{ex:maha} supports \cref{pr:it}: a conservative bound on the randomness deficiency provides us with a computable e-value, as a function of the $z^2$-scores. Moreover, the conclusion that the $z^2$-score of any variable cannot exceed the sum of all these ``noise scores'' still holds, because our reasoning above is entirely based on linear algebra. 

The following example shows
that root causes may have 
smaller {\it marginal} randomness deficiency than their downstream effect:
\begin{example}[Root cause with small marginal score]
Let $X_2 = 2 X_1 + E_2$
with
$X_1, E_2 \sim {\mathcal N}(0,1)$. 
Then $X_2\sim {\mathcal N}(0,\sqrt{3})$.
Increasing $x_2$ by manual perturbation induces a marginal randomness deficiency that increases with
$x_2^2/3$ for $x_2\to \infty$, while the conditional deficiency increases even like $x^2_2/1$.  Thus, we have 
$\delta(x_2\mid x_1) > \delta(x_2)$ 
and $\delta(x_1,x_2) > \delta(x_2)$ for large perturbations.
If $X_1$ and $X_2$ together influence a third variable
$X_3= X_1 - X_2 + E_3$ 
with $E_3\sim \mathcal{N}(0,1)$, we have $X_3 \sim {\mathcal N}(0,\sqrt{2})$, thus 
$\delta(x_3)$ increases 
with
$x^2_2/2$. Hence, $\delta(x_3)$
increases stronger than 
$\delta(x_2)$, but weaker than  $\delta(x_2\mid x_1)$
and $\delta(x_1,x_2)$.
\end{example}

This paradox resembles Example 2.1 in \cite{Li2024}, where it is phrased in terms of $z^2$-scores. 
Following \cref{pr:it}, we can describe it entirely in terms of M-distances (which are just $z^2$-scores in the one-dimensional case):
the M-distance  of $x_3$ 
can be larger than the M-distance of its root cause $x_2$, but not larger than the M-distance of the pair $(x_1,x_2)$, that is, the full cause of $x_3$. In other words, to fully quantify  
the anomaly introduced by perturbing $x_2$ we need 
to also account for the destroyed coupling between $x_1$ and $x_2$, not only for the size of the value $x_2$ itself.

The merit of algorithmic information theory is the insight that anomaly scores of the vector $(x_1,x_2)$ and
the scalar $x_3$ are indeed comparable, and also comparable to the conditional score of $x_2$ given $x_1$, because their calibration was guided by the randomness deficiency. 

In the following example, which could be easily generalized to words over an arbitrary alphabet, 
randomness deficiency essentially boils down to relative entropy:
\begin{example}[$m$-bit binary word]
\label{ex:bin}
Let us first consider an $m$-bit word whose bits are set to $1$ independently with probability $p$.
Counting the number of words $w$
with fixed Hamming weight $\ell$ yields the lower bound
\[
\delta(w) \cge - \ell \log p -
(m-\ell) \log (1-p) 
- \log {m \choose \ell}
- O(\log m).
\]
Using Stirling's approximation $\log m! = m \log m - m \log e + O(\log m)$,
one can show that
\begin{equation}\label{eq:kl}
\delta(w)
\cge m\cdot \kldiv{\frac \ell m}{p} - O(\log m), 
\end{equation} 
with the binary Kullback-Leibler distance
\[
\kldiv{q}{p} :=  
q\log\frac qp + (1-q) \log \frac{1-q}{1-p}.
\]
Here we can see that  words with
unexpectedly {\it low or high} Hamming weight, both are assigned high outlier scores, enabling us to compare both types of outliers. This works despite the probability mass function being monotonic in the Hamming weight $\ell$.
\end{example}
Multiple works \citep{Aggarwal2016,Akoglu2012,Noble2003} propose detecting anomalies by compression, but consider anomalies that have {\it higher} compression length than usual, although randomness deficiency seems to flag only anomalies  that have unexpectedly {\it low} compression length. \cref{ex:bin}
also resolves this paradox:
For $p< 1/2$, any string with $K(w) > m \cdot H_2(p) $
must have Hamming weight larger than $m p$. Accordingly, its randomness deficiency is bounded from below by \eqref{eq:kl}. Hence, paradoxically, an unusually {\it high} compression length
also implies non-zero randomness deficiency, because it entails 
an increase of the log likelihood term that outweighs 
the observed increase in compression length.

More sophisticated notions of anomalies can be obtained, for instance, when $X$ is graph-valued and an anomaly is given by large cliques \citep{Aggarwal2016} -- e.g. when fraudsters work together frequently on illegal activities, their communication
 is densely connected. 
To estimate the
random deficiency of a graph $G$ with a clique of size $k$, one only needs to
define a probability distribution on the set of graphs with $m$ nodes, and bound the description length of 
a graph by counting the number of graphs with cliques of size $k$.


\section{Experiment with Lempel-Ziv Compression}
\label{sec:toy}

So far we have accounted for the term $K(x|P^*)$ only by very rough upper bounds  rather than trying to approximate it.
Here
we describe a toy scenario in which we can detect anomalies using the Lempel-Ziv compression algorithm; more difficult scenarios may demand more powerful compression algorithms that are closer to general AI.

Consider the causal DAG:
$
X_1 \to X_2 \to \cdots \to X_n,
$ 
with the structural equations
\begin{equation}
\label{eq:sem}
X_1 = N_1; \quad X_j = X_{j-1} + N_j \quad \text{for } j = 2,\dots,n.
\end{equation}
Moreover, suppose that every $N_j$ for $j=1,\dots,n$
is drawn from the uniform distribution on the set of numbers in $[0,1]$ discretized to $d\gg 1$ digits of precision. 
By choosing the uniform distribution, the terms $-\log P(x_j\mid x_{j-1})$ become constant; moreover, we have $K(x_j\mid (x_{j-1}, P)^*)\ceq K(x_j\mid x_{j-1})$.
Since the conditional distribution of each $X_j$, given its parent, is uniform over the (discretized) interval $[x_{j-1},\, x_{j-1}+1]$, the conditional randomness deficiency of 
$x_j$ reads
\[
\delta(x_j\mid x_{j-1}) \ceq d\cdot \log 10 - K(x_j\mid x_{j-1}).
\]

Suppose now we inject an anomaly at some node $j$ by
setting $n_j$ to some numbers in $\{0,\dots,0.9\}$. 
As a result, $x_j$ and $x_{j-1}$ now coincide in at most $2$ digits,
so that $K(x_j\mid x_{j-1}) \ceq 0$ and $\delta(x_j\mid x_{j-1}) \ceq d\cdot \log 10$. 

Following \cite{submodular}, we approximate 
$K(x_j\mid x_{j-1}^*) \ceq K(x_j, x_{j-1}) - K(x_{j-1})$ by 
$
R(x_j,x_{j-1}) - R(x_{j-1}),
$
where $R$ denotes the length of a compressed encoding using the Lempel-Ziv algorithm.
Whenever $n_j$ corrupts to a $1$-digit number, 
Lempel-Ziv recognizes that $x_j$ and $x_{j-1}$
coincide with respect to all but $1$ or $2$ digits. Thus,
$R(x_j,x_{j-1})\approx R(x_{j-1})$, which lets us infer the randomness deficiency of almost $d\log 10$. 

We conducted experiments to verify our findings for
$n=4$, with the noise uniformly drawn from
numbers in $[0,1]$ with $d=10$ digits.  
We randomly chose one of the $4$ nodes as ``root cause'' and set all but one digit of its noise variables to zero. To detect the root cause, we selected the label $j$ that minimized Lempel-Ziv compression length
and found the right one in $100$ out of $100$ runs. 

From a theoretical point of view, the example shows that the joint observation can be anomalous (here in the sense of showing two variables whose digits coincide largely), although none of the variables 
show unexpected behaviour with respect to their \textit{ marginal} distribution. 
Hence, the root cause can only be found by inspecting which \textit{ mechanism} behaves unexpectedly. 


\section{Conclusion and Future Work}
Algorithmic randomness deficiency offers a principled and flexible definition of outliers, without prior specification of the feature that exposes the anomaly. 
On a causal Bayesian network, we saw that the randomness deficiency of a joint observation decomposes along individual causal mechanisms, subject to the Independent Mechanisms Principle. This allows us to trace anomalous observations back to their root causes, identifying specific mechanisms that exhibit atypical behavior. Furthermore, we showed that 
weak outliers cannot be responsible for producing strong outliers,
thus extending Levin’s law of randomness conservation.
This foundational insight can help
calibrating anomaly scores in a way that supports root cause analysis in complex systems. 

{\bf Acknowledgments:} DJ and YW would like to thank the authors of \cite{Li2024} for an interesting discussion about their Example 2.1.


%\bibliography{uai2025}
\begin{thebibliography}{61}
\providecommand{\natexlab}[1]{#1}
\providecommand{\url}[1]{\texttt{#1}}
\expandafter\ifx\csname urlstyle\endcsname\relax
  \providecommand{\doi}[1]{doi: #1}\else
  \providecommand{\doi}{doi: \begingroup \urlstyle{rm}\Url}\fi

\bibitem[Adam et~al.(2019)Adam, Alexandropoulos, Pardalos, and Vrahatis]{adam2019no}
Stavros~P Adam, Stamatios-Aggelos~N Alexandropoulos, Panos~M Pardalos, and Michael~N Vrahatis.
\newblock No free lunch theorem: A review.
\newblock \emph{Approximation and optimization: Algorithms, complexity and applications}, pages 57--82, 2019.

\bibitem[Aggarwal(2016)]{Aggarwal2016}
Charu~C. Aggarwal.
\newblock \emph{Outlier Analysis}.
\newblock Springer Publishing Company, Incorporated, 2nd edition, 2016.
\newblock ISBN 3319475770.

\bibitem[Aggarwal(2017)]{aggarwal2017introduction}
Charu~C Aggarwal.
\newblock \emph{An introduction to outlier analysis}.
\newblock Springer, 2017.

\bibitem[Akoglu et~al.(2012)Akoglu, Tong, Vreeken, and Faloutsos]{Akoglu2012}
Leman Akoglu, Hanghang Tong, Jilles Vreeken, and Christos Faloutsos.
\newblock Fast and reliable anomaly detection in categorical data.
\newblock In \emph{Proceedings of the 21st ACM International Conference on Information and Knowledge Management}, CIKM '12, page 415–424, New York, NY, USA, 2012. Association for Computing Machinery.
\newblock ISBN 9781450311564.

\bibitem[Barber et~al.(2019)Barber, Cand{\`e}s, Ramdas, and Tibshirani]{Barber2019}
Rina~Foygel Barber, Emmanuel~J. Cand{\`e}s, Aaditya Ramdas, and Ryan~J. Tibshirani.
\newblock The limits of distribution-free conditional predictive inference.
\newblock \emph{Information and Inference: A Journal of the IMA}, 2019.

\bibitem[Bates et~al.(2021)Bates, Cand{\`e}s, Lei, Romano, and Sesia]{Bates2021}
Stephen Bates, Emmanuel~J. Cand{\`e}s, Lihua Lei, Yaniv Romano, and Matteo Sesia.
\newblock Testing for outliers with conformal p-values.
\newblock \emph{The Annals of Statistics}, 2021.

\bibitem[Bernstein(2009)]{bernstein2009}
D.S. Bernstein.
\newblock \emph{Matrix Mathematics: Theory, Facts, and Formulas - Second Edition}.
\newblock Princeton University Press, 2009.

\bibitem[Bishop(1993)]{Bishop1993}
C.~M. Bishop.
\newblock Novelty detection and neural network validation.
\newblock In Stan Gielen and Bert Kappen, editors, \emph{ICANN '93}, pages 789--794, London, 1993. Springer London.

\bibitem[Breunig et~al.(2000)Breunig, Kriegel, Ng, and Sander]{breunig2000lof}
Markus~M Breunig, Hans-Peter Kriegel, Raymond~T Ng, and J{\"o}rg Sander.
\newblock Lof: identifying density-based local outliers.
\newblock In \emph{Proceedings of the 2000 ACM SIGMOD international conference on Management of data}, pages 93--104, 2000.

\bibitem[Budhathoki et~al.(2022)Budhathoki, Minorics, Bl{\"o}baum, and Janzing]{root_cause_analysis}
Kailash Budhathoki, Lenon Minorics, Patrick Bl{\"o}baum, and Dominik Janzing.
\newblock Causal structure-based root cause analysis of outliers.
\newblock In \emph{International Conference on Machine Learning}, pages 2357--2369. PMLR, 2022.

\bibitem[Coecke et~al.(2016)Coecke, Fritz, and Spekkens]{Coecke2019}
Bob Coecke, Tobias Fritz, and Robert~W. Spekkens.
\newblock A mathematical theory of resources.
\newblock \emph{Information and Computation}, 250:\penalty0 59--86, 2016.
\newblock ISSN 0890-5401.
\newblock Quantum Physics and Logic.

\bibitem[Cover and Thomas(2006)]{thomas2006elements}
Thomas~M Cover and Joy~A Thomas.
\newblock \emph{Elements of information theory}.
\newblock Wiley-Interscience, 2 edition, 2006.

\bibitem[Donoho(2004)]{donoho2004early}
Steve Donoho.
\newblock Early detection of insider trading in option markets.
\newblock In \emph{Proceedings of the tenth ACM SIGKDD international conference on Knowledge discovery and data mining}, pages 420--429, 2004.

\bibitem[Ebtekar and Hutter(2025)]{ebtekar2025foundations}
Aram Ebtekar and Marcus Hutter.
\newblock Foundations of algorithmic thermodynamics.
\newblock \emph{Physical Review E}, 111:\penalty0 014118, 2025.

\bibitem[Frank(2005)]{frank2005indefinite}
Michael~P Frank.
\newblock The indefinite logarithm, logarithmic units, and the nature of entropy.
\newblock 2005.

\bibitem[Freeman(1995)]{freeman1995outliers}
Jim Freeman.
\newblock Outliers in statistical data.
\newblock \emph{Journal of the Operational Research Society}, 46\penalty0 (8):\penalty0 1034--1035, 1995.

\bibitem[G{\'a}cs(1994)]{gacs1994boltzmann}
P{\'e}ter G{\'a}cs.
\newblock The {B}oltzmann entropy and randomness tests.
\newblock In \emph{Proceedings Workshop on Physics and Computation. PhysComp'94}, pages 209--216. IEEE, November 1994.

\bibitem[G{\'a}cs(2021)]{gacs2021lecture}
P{\'e}ter G{\'a}cs.
\newblock Lecture notes on descriptional complexity and randomness.
\newblock 2021.

\bibitem[Gan et~al.(2021)Gan, Liang, Dev, Lo, and Delimitrou]{Gan2021}
Yu~Gan, Mingyu Liang, Sundar Dev, David Lo, and Christina Delimitrou.
\newblock Sage: practical and scalable ml-driven performance debugging in microservices.
\newblock In \emph{Proceedings of the 26th ACM International Conference on Architectural Support for Programming Languages and Operating Systems}, ASPLOS '21, page 135–151, New York, NY, USA, 2021. Association for Computing Machinery.
\newblock ISBN 9781450383172.

\bibitem[Genovese et~al.(2006)Genovese, Roeder, and Wasserman]{genovese2006false}
Christopher~R Genovese, Kathryn Roeder, and Larry Wasserman.
\newblock False discovery control with p-value weighting.
\newblock \emph{Biometrika}, 93\penalty0 (3):\penalty0 509--524, 2006.

\bibitem[Gr{\"u}nwald et~al.(2020)Gr{\"u}nwald, de~Heide, and Koolen]{grunwald2020safe}
Peter Gr{\"u}nwald, Rianne de~Heide, and Wouter~M Koolen.
\newblock Safe testing.
\newblock In \emph{2020 Information Theory and Applications Workshop (ITA)}, pages 1--54. IEEE, 2020.

\bibitem[Hardt et~al.(2023)Hardt, Orchard, Blöbaum, Kasiviswanathan, and Kirschbaum]{hardt2023petshop}
Michaela Hardt, William Orchard, Patrick Blöbaum, Shiva Kasiviswanathan, and Elke Kirschbaum.
\newblock The petshop dataset -- finding causes of performance issues across microservices, 2023.

\bibitem[Hooi et~al.(2016)Hooi, Song, Beutel, Shah, Shin, and Faloutsos]{hooi2016fraudar}
Bryan Hooi, Hyun~Ah Song, Alex Beutel, Neil Shah, Kijung Shin, and Christos Faloutsos.
\newblock Fraudar: Bounding graph fraud in the face of camouflage.
\newblock In \emph{Proceedings of the 22nd ACM SIGKDD international conference on knowledge discovery and data mining}, pages 895--904, 2016.

\bibitem[Ikram et~al.(2022)Ikram, Chakraborty, Mitra, Saini, Bagchi, and Kocaoglu]{ikram2022root}
Azam Ikram, Sarthak Chakraborty, Subrata Mitra, Shiv Saini, Saurabh Bagchi, and Murat Kocaoglu.
\newblock Root cause analysis of failures in microservices through causal discovery.
\newblock \emph{Advances in Neural Information Processing Systems}, 35:\penalty0 31158--31170, 2022.

\bibitem[Janzing and Sch\"olkopf(2010)]{Algorithmic}
D~Janzing and B~Sch\"olkopf.
\newblock {Causal inference using the algorithmic Markov condition}.
\newblock \emph{IEEE Transactions on Information Theory}, 56\penalty0 (10):\penalty0 5168--5194, 2010.

\bibitem[Levin(1976)]{levin1976uniform}
Leonid~A Levin.
\newblock Uniform tests of randomness.
\newblock In \emph{Doklady Akademii Nauk}, volume 227, pages 33--35. Russian Academy of Sciences, 1976.

\bibitem[Levin(1984)]{levin1984randomness}
Leonid~A Levin.
\newblock Randomness conservation inequalities; information and independence in mathematical theories.
\newblock \emph{Information and Control}, 61\penalty0 (1):\penalty0 15--37, 1984.

\bibitem[Li et~al.(2024)Li, Chu, Scheller, Gagneur, and Maathuis]{Li2024}
Jinzhou Li, Benjamin~B. Chu, Ines~F. Scheller, Julien Gagneur, and Marloes~H. Maathuis.
\newblock {Root cause discovery via permutations and Cholesky decomposition}.
\newblock arxiv:2410.12151, 2024.

\bibitem[Li and Vit{\'a}nyi(2019)]{Vitanyi19}
Ming Li and Paul M~B Vit{\'a}nyi.
\newblock \emph{An Introduction to {K}olmogorov Complexity and Its Applications}.
\newblock Springer, New York, 4 edition, 2019.

\bibitem[Li et~al.(2022)Li, Li, Yin, Nie, Zhang, Sui, and Pei]{li2022causal}
Mingjie Li, Zeyan Li, Kanglin Yin, Xiaohui Nie, Wenchi Zhang, Kaixin Sui, and Dan Pei.
\newblock Causal inference-based root cause analysis for online service systems with intervention recognition.
\newblock In \emph{Proceedings of the 28th ACM SIGKDD Conference on Knowledge Discovery and Data Mining}, pages 3230--3240, 2022.

\bibitem[Ma et~al.(2020)Ma, Xu, Wang, Chen, Zhang, and Wang]{ma2020automap}
Meng Ma, Jingmin Xu, Yuan Wang, Pengfei Chen, Zonghua Zhang, and Ping Wang.
\newblock Automap: Diagnose your microservice-based web applications automatically.
\newblock In \emph{Proceedings of The Web Conference 2020}, pages 246--258, 2020.

\bibitem[Martin-L{\"o}f(1966)]{martin1966definition}
Per Martin-L{\"o}f.
\newblock The definition of random sequences.
\newblock \emph{Information and control}, 9\penalty0 (6):\penalty0 602--619, 1966.

\bibitem[Maslove et~al.(2016)Maslove, Dubin, Shrivats, and Lee]{maslove2016errors}
David~M Maslove, Joel~A Dubin, Arvind Shrivats, and Joon Lee.
\newblock Errors, omissions, and outliers in hourly vital signs measurements in intensive care.
\newblock \emph{Critical care medicine}, 44\penalty0 (11):\penalty0 e1021--e1030, 2016.

\bibitem[Mohotti and Nayak(2020)]{mohotti2020efficient}
Wathsala~Anupama Mohotti and Richi Nayak.
\newblock Efficient outlier detection in text corpus using rare frequency and ranking.
\newblock \emph{ACM Transactions on Knowledge Discovery from Data (TKDD)}, 14\penalty0 (6):\penalty0 1--30, 2020.

\bibitem[Neyman and Pearson(1933)]{neyman1933ix}
Jerzy Neyman and Egon~Sharpe Pearson.
\newblock Ix. on the problem of the most efficient tests of statistical hypotheses.
\newblock \emph{Philosophical Transactions of the Royal Society of London. Series A, Containing Papers of a Mathematical or Physical Character}, 231\penalty0 (694-706):\penalty0 289--337, 1933.

\bibitem[Noble and Cook(2003)]{Noble2003}
Caleb~C. Noble and Diane~J. Cook.
\newblock Graph-based anomaly detection.
\newblock In \emph{Proceedings of the Ninth ACM SIGKDD International Conference on Knowledge Discovery and Data Mining}, KDD '03, page 631–636, New York, NY, USA, 2003. Association for Computing Machinery.
\newblock ISBN 1581137370.

\bibitem[Okati et~al.(2024)Okati, Mejia, Orchard, Blöbaum, and Janzing]{Okati2024}
Nastaran Okati, Sergio Hernan~Garrido Mejia, William~Roy Orchard, Patrick Blöbaum, and Dominik Janzing.
\newblock Root cause analysis of outliers with missing structural knowledge.
\newblock arxiv:2406.05014, 2024.

\bibitem[Panjei et~al.(2022)Panjei, Gruenwald, Leal, Nguyen, and Silvia]{panjei2022survey}
Egawati Panjei, Le~Gruenwald, Eleazar Leal, Christopher Nguyen, and Shejuti Silvia.
\newblock A survey on outlier explanations.
\newblock \emph{The VLDB Journal}, 31\penalty0 (5):\penalty0 977--1008, 2022.

\bibitem[Pearl(2009)]{pearl2009causality}
Judea Pearl.
\newblock \emph{Causality}.
\newblock Cambridge university press, 2 edition, 2009.

\bibitem[Peters et~al.(2017)Peters, Janzing, and Sch{\"o}lkopf]{peters2017elements}
Jonas Peters, Dominik Janzing, and Bernhard Sch{\"o}lkopf.
\newblock \emph{Elements of causal inference: foundations and learning algorithms}.
\newblock The MIT Press, 2017.

\bibitem[Ramdas and Wang(2024)]{ramdas2024hypothesis}
Aaditya Ramdas and Ruodu Wang.
\newblock \emph{Hypothesis Testing with E-values}.
\newblock 2024.

\bibitem[Ramdas et~al.(2023)Ramdas, Gr{\"u}nwald, Vovk, and Shafer]{ramdas2023game}
Aaditya Ramdas, Peter Gr{\"u}nwald, Vladimir Vovk, and Glenn Shafer.
\newblock Game-theoretic statistics and safe anytime-valid inference.
\newblock \emph{Statistical Science}, 38\penalty0 (4):\penalty0 576--601, 2023.

\bibitem[Rathmanner and Hutter(2011)]{rathmanner2011philosophical}
Samuel Rathmanner and Marcus Hutter.
\newblock A philosophical treatise of universal induction.
\newblock \emph{Entropy}, 13\penalty0 (6):\penalty0 1076--1136, 2011.

\bibitem[Rocke and Woodruff(1996)]{rocke1996identification}
David~M Rocke and David~L Woodruff.
\newblock Identification of outliers in multivariate data.
\newblock \emph{Journal of the American Statistical Association}, 91\penalty0 (435):\penalty0 1047--1061, 1996.

\bibitem[Rousseeuw and Leroy(2003)]{rousseeuw2003robust}
Peter~J Rousseeuw and Annick~M Leroy.
\newblock \emph{Robust regression and outlier detection}.
\newblock John wiley \& sons, 2003.

\bibitem[Sch{\"o}lkopf et~al.(2021)Sch{\"o}lkopf, Locatello, Bauer, Ke, Kalchbrenner, Goyal, and Bengio]{Schoelkopf2021}
B~Sch{\"o}lkopf, F~Locatello, S~Bauer, N~R Ke, N~Kalchbrenner, A~Goyal, and Y~Bengio.
\newblock Toward causal representation learning.
\newblock \emph{Proceedings of the IEEE}, 109\penalty0 (5):\penalty0 612--634, 2021.

\bibitem[Shen(2020)]{shen2020randomness}
Alexander Shen.
\newblock Randomness tests: theory and practice.
\newblock In \emph{Fields of Logic and Computation III: Essays Dedicated to Yuri Gurevich on the Occasion of His 80th Birthday}, pages 258--290. Springer, 2020.

\bibitem[Spirtes et~al.(1993)Spirtes, Glymour, and Scheines]{Spirtes1993}
P.~Spirtes, C.~Glymour, and R.~Scheines.
\newblock \emph{Causation, Prediction, and Search}.
\newblock Springer-Verlag, New York, NY, 1993.

\bibitem[Steudel et~al.(2010)Steudel, Janzing, and Sch\"olkopf]{submodular}
B~Steudel, D~Janzing, and B~Sch\"olkopf.
\newblock {Causal Markov condition for submodular information measures}.
\newblock \emph{Proceedings of the 23rd Annual Conference on Learning Theory (COLT)}, pages 464--476, 2010.

\bibitem[Susto et~al.(2017)Susto, Terzi, and Beghi]{susto2017anomaly}
Gian~Antonio Susto, Matteo Terzi, and Alessandro Beghi.
\newblock Anomaly detection approaches for semiconductor manufacturing.
\newblock \emph{Procedia Manufacturing}, 11:\penalty0 2018--2024, 2017.

\bibitem[Tartakovsky et~al.(2012)Tartakovsky, Polunchenko, and Sokolov]{tartakovsky2012efficient}
Alexander~G Tartakovsky, Aleksey~S Polunchenko, and Grigory Sokolov.
\newblock Efficient computer network anomaly detection by changepoint detection methods.
\newblock \emph{IEEE Journal of Selected Topics in Signal Processing}, 7\penalty0 (1):\penalty0 4--11, 2012.

\bibitem[Von~K\"ugelgen et~al.(2023)Von~K\"ugelgen, Mohamed, and Beckers]{backtracking}
Julius Von~K\"ugelgen, Abdirisak Mohamed, and Sander Beckers.
\newblock Backtracking counterfactuals.
\newblock In Mihaela van~der Schaar, Cheng Zhang, and Dominik Janzing, editors, \emph{Proceedings of the Second Conference on Causal Learning and Reasoning}, volume 213 of \emph{Proceedings of Machine Learning Research}, pages 177--196. PMLR, 11--14 Apr 2023.

\bibitem[Vovk(2020)]{vovk2020non}
Vladimir Vovk.
\newblock Non-algorithmic theory of randomness.
\newblock In \emph{Fields of Logic and Computation III: Essays Dedicated to Yuri Gurevich on the Occasion of His 80th Birthday}, pages 323--340. Springer, 2020.

\bibitem[Vovk and Wang(2021)]{vovk2021values}
Vladimir Vovk and Ruodu Wang.
\newblock E-values: Calibration, combination and applications.
\newblock \emph{The Annals of Statistics}, 49\penalty0 (3):\penalty0 1736--1754, 2021.

\bibitem[Wang et~al.(2023{\natexlab{a}})Wang, Chen, Fu, Liu, and Chen]{wang2023incremental}
Dongjie Wang, Zhengzhang Chen, Yanjie Fu, Yanchi Liu, and Haifeng Chen.
\newblock Incremental causal graph learning for online root cause analysis.
\newblock In \emph{Proceedings of the 29th ACM SIGKDD Conference on Knowledge Discovery and Data Mining}, pages 2269--2278, 2023{\natexlab{a}}.

\bibitem[Wang et~al.(2023{\natexlab{b}})Wang, Chen, Ni, Tong, Wang, Fu, and Chen]{wang2023interdependent}
Dongjie Wang, Zhengzhang Chen, Jingchao Ni, Liang Tong, Zheng Wang, Yanjie Fu, and Haifeng Chen.
\newblock Interdependent causal networks for root cause localization.
\newblock In \emph{Proceedings of the 29th ACM SIGKDD Conference on Knowledge Discovery and Data Mining}, pages 5051--5060, 2023{\natexlab{b}}.

\bibitem[Wasserman and Roeder(2006)]{wasserman2006weighted}
Larry Wasserman and Kathryn Roeder.
\newblock Weighted hypothesis testing.
\newblock 2006.

\bibitem[Wolpert(2023)]{wolpert2023implications}
David~H Wolpert.
\newblock The implications of the no-free-lunch theorems for meta-induction.
\newblock \emph{Journal for General Philosophy of Science}, 54\penalty0 (3):\penalty0 421--432, 2023.

\bibitem[Yakovlev et~al.(2021)Yakovlev, Bekkouch, Khan, and Khattak]{yakovlev2021abstraction}
Kirill Yakovlev, Imad Eddine~Ibrahim Bekkouch, Adil~Mehmood Khan, and Asad~Masood Khattak.
\newblock Abstraction-based outlier detection for image data.
\newblock In \emph{Intelligent Systems and Applications: Proceedings of the 2020 Intelligent Systems Conference (IntelliSys) Volume 1}, pages 540--552. Springer, 2021.

\bibitem[Zscheischler et~al.(2022)Zscheischler, Sillmann, and Alexander]{Zscheischler2021}
Jakob Zscheischler, Jana Sillmann, and Lisa Alexander.
\newblock Introduction to the special issue: Compound weather and climate events.
\newblock \emph{Weather and Climate Extremes}, 35:\penalty0 100381, 2022.

\bibitem[Zurek(1989)]{zurek1989algorithmic}
Wojciech~H Zurek.
\newblock Algorithmic randomness and physical entropy.
\newblock \emph{Physical Review A}, 40\penalty0 (8):\penalty0 4731, 1989.

\end{thebibliography}

\newpage

\onecolumn

\title{Toward Universal Laws of Outlier Propagation\\(Supplementary Material)}
\maketitle

{
    \def\thefootnote{*}\footnotetext{These authors contributed equally to this work.}
}

\section{p-tests and e-tests}
\label{sec:etestsandptests}

\cref{tab:testssummary} summarizes key differences between p-tests and e-tests. While p-tests are linked to controlling false positive rates and p-values, e-tests are closely associated with likelihood ratios and betting game interpretations.


\begin{table}[ht]
\centering
\setlength{\tabcolsep}{5pt} 
\renewcommand{\arraystretch}{1.2} 
\begin{tabular}{c|c|c}
\hline
      & p-tests (Martin-L\"of) & e-tests (Levin) \\ 
\hline
Intuition &
False positives / p-values &
Betting games / likelihood ratios \\
Defining property &
$\forall\epsilon>0,\,P\left( \Lambda(X) \ge 1/\epsilon \right) \le\epsilon$ &
$\E_{X\sim P}\left( \Lambda(X) \right) \le 1$ \\
Prototypical example &
$1/P(\tau(X) \ge \tau(x))$ &
$Q(x)/P(x)$ \\
Combination &
$\sup_i w_i\Lambda_i$ &
$\sum_i w_i\Lambda_i$ \\
Combination (log form) &
$\sup_i \{ \lambda_i + \log w_i \}$ &
$\log\left(\sum_i w_i 2^{\lambda_i}\right)$ \\
Universal test (log form) &
$\sup_i\{ \lambda_i(x) - K(i\mid P^*) \}$ &
$-\log P(x) - K(x\mid P^*)$ \\
\hline
\end{tabular}
\caption{Summary of types of test statistics (outlier scores)}
\label{tab:testssummary}
\end{table}

\paragraph{Deriving p-Tests from e-Tests}
\citet{vovk2021values} describe a number of ways to \emph{calibrate} any given p-test into an e-test. A natural choice is the Ramdas calibration, which turns the p-test $\Lambda$ into the e-test
\begin{equation}
\label{eq:ramdascalibration}
\Lambda'(x) := \frac
{\Lambda(x)-\ln\Lambda(x)-1}
{\ln^2\Lambda(x)}.
\end{equation}

\begin{example}[One-tailed p-value]
For any feature statistic $\tau:\domain\rightarrow\R$, the one-tailed p-value is given by
\begin{equation}
\label{eq:pvaluetest}
\Lambda_\tau(x) := P(\tau(X) \ge \tau(x)).
\end{equation}
It is easily verified to be a p-test in probability form. The outlier score \labelcref{eq:itscore} is simply the p-value test \labelcref{eq:pvaluetest} expressed in log form.
\end{example}

\begin{example}[Likelihood ratio]
For any alternative hypothesis described by a sub-probability distribution\footnote{The algorithmic information theory literature refers to sub-probability distributions as \emph{semimeasures}. They generalize probability measures by allowing their sum to be less than $1$.} $Q$, the likelihood ratio is given by
\begin{equation}
\label{eq:likelihoodratiotest}
\Lambda_Q(x) := \frac{Q(x)}{P(x)}.
\end{equation}
It is easily verified to be an e-test in ratio form; in fact, all e-tests can be written this way. By the Neyman-Pearson lemma \citep{neyman1933ix}, when $Q$ is a probability distribution, $\Lambda_Q$ is the optimal test to distinguish between $P$ and $Q$.
\end{example}

For the purposes of anomaly detection, $\tau$ might correspond to a feature that we expect would be higher for anomalies, while $Q$ might correspond to a distribution that we expect would result from some kind of anomaly. In practice, we can design many tests, each specializing in a different kind of anomaly. The following lemma allows us to merge many such tests into one.
\begin{restatable}[Combination tests]{lemma}{combinationtests}
\label{lem:combinationtests}
Let $\Lambda_i$ (or $\lambda_i$) be a finite or countably infinite sequence of tests, with associated weights $w_i>0$ summing to at most $1$. Then:
\begin{itemize}[left=0pt]
   \item If $\Lambda_i$ are p-tests in prob. form, so is $\inf_i \frac{\Lambda_i}{w_i}$.
    \item If $\Lambda_i$ are p-tests in ratio form, so is $\sup_i w_i\Lambda_i$.
    \item If $\lambda_i$ are p-tests in log form, so is $\sup_i \{\lambda_i + \log w_i\}$.
    \item If $\Lambda_i$ are e-tests in prob. form, so is $\left(\sum_i\frac{w_i}{\Lambda_i}\right)^{-1}$.
    \item If $\Lambda_i$ are e-tests in ratio form, so is $\sum_i w_i\Lambda_i$.
   \item If $\lambda_i$ are e-tests in log form, so is $\log\left(\sum_i w_i 2^{\lambda_i}\right)$.
\end{itemize}
\end{restatable}

%\combinationtests*
\begin{proof}
\cite{genovese2006false} state a similar result for finitely many independent p-tests, calling the combination test a \emph{weighted Bonferroni procedure}. \cite{vovk2021values} state a similar result for equally weighted e-tests. It is fairly straightforward to extend these works to our setting; for completeness, the proof is as follows. First, given some p-tests $\Lambda_i$ in ratio form, we verify \Cref{def:tests} for their combination $\sup_i w_i\Lambda_i$:
\begin{equation*}
P\left(\sup_i w_i\Lambda_i(X) \ge \frac 1\epsilon\right)
\le \sum_i P\left(\Lambda_i(X) \ge \frac 1{w_i\epsilon}\right)
\le \sum_i w_i\epsilon
\le \epsilon.
\end{equation*}
Next, given e-tests $\Lambda_i$ in ratio form, we verify \Cref{def:tests} for their combination $\sum_i w_i\Lambda_i$:
\begin{equation*}
\E\left(\sum_i w_i\Lambda_i(X)\right)
= \sum_i w_i \E(\Lambda_i(X))
\le \sum_i w_i
\le 1.
\end{equation*}
Transforming the tests into probability and log form yields the remaining results.
\end{proof}

\begin{example}[Two-tailed p-value]
By combining the one-tailed p-values \labelcref{eq:pvaluetest} for the feature statistics $\tau$ and $-\tau$, each with weight $0.5$, we obtain the two-tailed p-value
\vspace{-0.2cm}
\begin{align*}
\Lambda_{\pm\tau}(x)
&:= \min\left\{\Lambda_\tau(x)/0.5,\;\Lambda_{-\tau}(x)/0.5\right)\}
\\&= 2\min\left\{
\Pr(\tau(X) \ge \tau(x)),\;
\Pr(\tau(X) \le \tau(x)) \right)\}.
\end{align*}
\end{example}

From now on, we express tests in log form except where stated otherwise. By \Cref{lem:combinationtests}, a combination test exceeds each of its component tests, up to the additive regularization term $\log w_i$ which does not depend on the sample $x$. This enables us to commit to a combination test a priori, while postponing the search over component tests until after observing $x$.

Unfortunately, the regularization term does depend on $i$, and becomes arbitrarily large as the number of tests becomes large or infinite. If we want the combination test to be competitive with its most promising component tests, it becomes important to choose the weights well \citep{wasserman2006weighted}. The problem of choosing $w_i$ is analogous to that of choosing Bayesian priors, and is philosophically challenged by formal impossibility results in the theory of inductive inference \citep{adam2019no,wolpert2023implications}. Fortunately, computability poses a useful constraint on the set of permissible tests as well as priors \citep{rathmanner2011philosophical}. It is suggestive that every computable sequence of weights $w_i$ can be turned into a computable binary code with lengths $\lceil-\log w_i\rceil$ \citep{thomas2006elements}. Thus, minimizing the regularization penalty amounts to finding the shortest computable encoding for $i$.


\section{Constructing the universal e-test}
\label{sec:constructuniversal}

To develop a universal e-test, we first need to establish some fundamental computability concepts that will be essential for our construction. 


We say a function \( f: \{0,1\}^* \rightarrow \mathbb{R} \) is lower (or upper) semicomputable if it can be computably approximated from below (or above, respectively). For example, the Kolmogorov complexity $K$ is upper, but not lower, semicomputable: by running all programs in parallel, we gradually find shorter programs that output \( x \). We say $f$ is computable if it is both lower and upper and semicomputable.

We say a p-test or e-test is semicomputable, if it is lower semicomputable when expressed as a function in either ratio or log form. A semicomputable p-test is called a Martin-L\"of test \citep{martin1966definition}, while a semicomputable e-test is called a Levin test \citep{levin1976uniform}. Intuitively, semicomputable tests detect more anomalies as computation time increases. By restricting attention to semicomputable tests, it becomes possible to create universal combinations of them.

We return to the problem of optimizing the weights in \Cref{lem:combinationtests}: we would like to dominate not only each of the individual component tests, but also each of the combination tests obtainable by some computable sequence of weights $w_i$. Note that the constraints on $w_i$ amount to specifying a discrete sub-probability distribution. Among all lower semicomputable discrete sub-probability distributions,
\begin{equation}
\vspace{-0.2cm}
\label{eq:univeralprobability}
m_i := 2^{-K(i\mid P^*)}
\end{equation}
is \emph{universal} in the sense that for all alternatives $w$ and all $i$,
\[\log m_i \cge \log w_i - K(w\mid P^*).\]
This also holds when $P$ is replaced by any other piece of prior knowledge.

Now, applying \Cref{lem:combinationtests} with the universal weights \labelcref{eq:univeralprobability}, to any sequence of p-tests $\lambda_i$ in log form, yields their universal combination
\begin{equation*}
\lambda(x)
:= \sup_i\{ \lambda_i(x) - K(i\mid P^*) \}.
\end{equation*}
In the case where $\{\lambda_i\}_{i=1}^\infty$ is a computable enumeration of some Martin-L\"of tests, $\lambda$ is itself a Martin-L\"of test. It follows that if $\{\lambda_i\}_{i=1}^\infty$ enumerates \emph{all} Martin-L\"of tests, then $\lambda$ is universal among them. Every program $p$ that outputs $i$ with access to $P$, determines the feature $\lambda_p:=\lambda_i$. Moreover, it satisfies $|p|\ge K(i\mid P^*)$, with equality for the shortest such program. Therefore, we can rewrite the universal Martin-L\"of test as
\begin{equation}
\label{eq:universalmartinlof}
\rho(x) := \sup_p\{ \lambda_p(x) - |p|\}.
\end{equation}

With e-tests, the situation is even nicer because \Cref{eq:likelihoodratiotest} provides a one-to-one correspondence between e-tests $\Lambda_Q$ and sub-probability distributions $Q$. Moreover, if we assume $P$ to be computable, an e-test for it is semicomputable (i.e., is a Levin test) iff its corresponding $Q$ is lower semicomputable. Applying \Cref{lem:combinationtests} with the universal weights \labelcref{eq:univeralprobability}, to the likelihood ratios $\Lambda_{Q_i}$, yields their universal combination
\begin{align*}
\Lambda(x)
:= \sum_i m_i\Lambda_{Q_i}(x)
= \frac{\sum_i m_i Q_i(x)}{P(x)}.
\end{align*}

In the case where $\{Q_i\}_{i=1}^\infty$ enumerates all lower semicomputable sub-probability measures, Theorem 4.3.3 in \cite{Vitanyi19} implies $\sum_i m_i Q_i(x) {\stackrel{\times}{=}} m(x\mid P^*)$, where ${\stackrel{\times}{=}}$ indicate equality up to a multiplicative term. Switching to log form, we obtain the universal Levin test
\begin{equation}
\label{eq:universallevin}
\delta(x)
:= \log\frac{m(x\mid P^*)}{P(x)}
= \log\frac{1}{P(x)} - K(x\mid P^*).
\end{equation}

By the Kraft inequality, it is an e-test in log form. Note that \labelcref{eq:universallevin} is the difference between a Shannon code length and a shortest program length for $x$. Intuitively, $\delta$ is high whenever the Shannon code derived from $P$ is inefficient at compressing $x$, indicating that $x$ possesses regularities that are atypical of $P$.

The algorithmic information theory literature uses the term \emph{randomness deficiency} to refer to either the universal Martin-L\"of test \labelcref{eq:universalmartinlof} or the universal Levin test \labelcref{eq:universallevin}. Keeping in mind conversions such as \labelcref{eq:ramdascalibration} between the two types of tests, we will develop our theory in terms of the latter.

\begin{example}[Feature Selection through Universal Tests]
Consider an infinite sequence of basis feature functions $(f_i)$, where $\log P$ is expressed as a linear combination of finitely many features: $\log P := \sum_i \alpha_i f_i$. When an observation $x$ exhibits atypical feature values under $P$, it would typically have a substantially higher likelihood under some modified linear combination $\log \tilde{P} := \sum_i \tilde{\alpha}_i f_i$.

If we assume the description of $P$ is given in terms of the coefficient vector $\alpha$, we can bound the Kolmogorov complexity:
\[
K(x\mid P^*)
\stackrel{+}{\leq} K(\tilde{\alpha}\mid\alpha) + \log\frac{1}{\tilde{P}(x)},
\]

This leads to a lower bound on the Levin test:
\[
\delta_P (x)
= - \log P(x)
- K(x\mid P^*)
\stackrel{+}{\geq} \log\frac{\tilde{P} (x)}{P(x)}
- K(\tilde{\alpha}\mid\alpha).
\]

The bound becomes large when the improvement in likelihood (first term) substantially exceeds the complexity cost $K(\tilde{\alpha}\mid\alpha)$ required to modify the coefficients. This demonstrates how the universal tests naturally detect feature-based anomalies: when certain feature statistics of $x$ are unusual under $P$, there exists an alternative distribution $\tilde{P}$ that better explains these feature values, leading to a high value of $\delta_P(x)$.
\end{example}


\section{Decomposition of randomness deficiency}
\label{sec:decomp}

\decompositionpair*
\begin{proof}
By \cref{eq:deltaxy,eq:deltaymidx}, we have:
\begin{align*}
\delta (x, y)
&\ceq - \log P(x,y) - K(x,y \mid (P_{X, Y})^*)
\\&\ceq - \log P(x,y) - K(x \mid (P_{X, Y})^*) - K(y \mid (x, P_{X, Y})^*),
\\\delta (x) + \delta(y \mid x)
&\ceq - \log P(x,y) - K(x \mid (P_{X})^*) - K(y \mid (x,P_{Y\mid X})^*).
\end{align*}
To complete the proof, it suffices to establish the following:
\[
\text{(1) } \quad K(x \mid (P_{X, Y})^*) \ceq K(x \mid (P_{X})^*) \qquad \text{and} \qquad \text{(2) } \quad  K(y \mid (x, P_{X, Y})^*) \ceq K(y \mid (x, P_{Y \mid X})^*)
\]
\paragraph{Proof of (1): } 
Applying \cref{lm:indepinputs}
to our bivariate case yields 
$x \independent P_{Y|X} \,|P_X$. Hence,
\[K(x \mid (P_{X, Y})^*)
\ceq K(x \mid (P_{Y\mid X}, P_X)^*)
\ceq K(x \mid (P_X)^*).\]
%\end{align}
\paragraph{Proof of (2): } 
By \cref{lm:indepinputs}, $y \indep P_X \mid x, P_{Y \mid X}$,
meaning that $P_{X}$ becomes irrelevant when predicting $y$ from a shortest program for $x$ and $P_{Y|X}$. Hence,
\[K(y \mid (x, P_{X, Y})^*)
\ceq K(y \mid (x, P_{Y \mid X}, P_X)^*)
\ceq K(y \mid (x, P_{Y \mid X})^*).\]
\end{proof}
\decomposition*

\begin{proof}
We prove this theorem by induction on $n$.

\noindent\textbf{Base case: }
When $n=1$, we have $\pa_1 = \emptyset$, so the claim is trivial.

\noindent\textbf{Inductive Hypothesis: }
Assume that for any sequence of $n - 1$ strings $x_1, x_2, \ldots, x_{n-1}$,
\[
    \delta(x_1,\dots,x_{n-1}) \ceq \sum_{j=1}^{n-1} \delta (x_j\mid \pa_j).
\]
\textbf{Inductive Step: }
Now we must prove the statement holds for $n$ strings. The statistical Markov condition yields
\[P(x_n\mid x_1,\ldots,x_{n-1}) = P(x_n\mid \pa_n).\]
Meanwhile, the algorithmic Markov condition gives us \cref{lm:indepinputs}, which implies
\[
x_n \independent x_1,\ldots,x_{n-1}\mid(\pa_n, P_{X_n|\PA_n})^*.
\]
Together with $P_{X_n\mid X_1,\ldots,X_{n-1}} = P_{X_n\mid \PA_n}$, this yields
\[K(x_n \mid (x_1,\ldots,x_{n-1},P_{X_n\mid X_1,\ldots,X_{n-1}})^*)
\ceq K(x_n \mid (\pa_n, P_{X_n \mid \PA_n})^*).\]
Putting these results together,
\begin{align*}
\delta(x_n \mid x_1, \dots, x_{n-1})
&:= - \log P(x_n\mid x_1, \ldots, x_{n-1})
- K(x_n \mid (x_1,\ldots,x_{n-1},P_{X_n\mid X_1,\ldots,X_{n-1}})^*)
\\&\ceq - \log P(x_n\mid \pa_n)
- K(x_n \mid (\pa_n, P_{X_n \mid \PA_n})^*)
\\&= \delta(x_n\mid \pa_n).
\end{align*}
Finally, we combine the inductive hypothesis with \cref{lm:deltaxy}, with $x:=(x_1,\dots,x_{n-1})$ and $y:=x_n$:
\begin{align*}
\delta(x_1,\dots,x_n)
&\ceq \delta(x_1,\dots,x_{n-1}) + \delta(x_n \mid x_1, \dots, x_{n-1})
\\&\ceq \sum_{j=1}^n \delta (x_j\mid \pa_j).
\end{align*}
This completes the inductive step and the proof.
\end{proof}


\section{Non-increasingness of Mahalanobis distance}
\label{sec:mahalanobis}

Let $\bx \in \R^n$, and 
let $Q\bx$ denote the projection of $\bx$ onto the subspace spanned by the variables 
$X_{i_1},\dots,X_{i_k}$, where $k < n$. We express the covariance matrix 
$\Sigma_X$ in block matrix form as 
\[
\Sigma_X = \left(\begin{array}{cc} \Sigma_{11} & \Sigma_{12}\\
\Sigma_{21} & \Sigma_{22} \end{array} \right),
\]
with index $1$ referring to the variables $i_1,\dots,i_k$
and $2$ to the remaining variables. Using a known formula for 
inversion of $2\times 2$ block matrices (see
Proposition 3.9.7 in \cite{bernstein2009}), we obtain  
\[
\Sigma^{-1}_X = \left(\begin{array}{cc} \Sigma_{11}^{-1} + \Sigma_{11}^{-1} \Sigma_{12} A \Sigma_{21}\Sigma_{11}^{-1} &   -\Sigma_{11}^{-1} \Sigma_{12} A\\
- A \Sigma_{21} \Sigma_{11}^{-1} & A \end{array} \right),
\]
with $A:=(\Sigma_{22} - \Sigma_{21} \Sigma_{11}^{-1} \Sigma_{12})^{-1}$.
We now compute the difference between the squared Mahalanobis distances 
on $\R^n$ and $\R^k$:
\begin{equation}\label{eq:diffM}
\bx^T \Sigma_{X}^{-1} \bx - (Q\bx)^T \Sigma_{11}^{-1} Q\bx =
\bx^T C \bx, 
\end{equation}
with 
\[
C :=  \left(\begin{array}{cc}  \Sigma_{11}^{-1} \Sigma_{12} A \Sigma_{21}\Sigma_{11}^{-1} &   -\Sigma_{11}^{-1} \Sigma_{12} A\\
- A \Sigma_{21} \Sigma_{11}^{-1} & A \end{array} \right)
= 
\left(\begin{array}{cc}  \Sigma_{11}^{-1} \Sigma_{12} &  0 \\ 0 & -1 \end{array}  \right)
\left(\begin{array}{cc}   A  &    A\\
 A  & A \end{array} \right)
\left(\begin{array}{cc} \Sigma_{21}\Sigma_{11}^{-1}  & 0 \\ 0 & -1 \end{array} \right).
\]
Since $A$ is positive semi-definite, and the rightmost matrix in the product is the transpose of the leftmost matrix, it follows that $C$ is also positive semi-definite. Hence,  
\eqref{eq:diffM} is non-negative.   

\end{document}
