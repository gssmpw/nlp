% include in main: \newcommand{\thought}[1]{{\color[rgb]{0.2,0.39,0.66}(#1)}}
\newcommand{\todo}[1]{{\color[rgb]{1.0,0.0,0.0}(#1)}}
\newcommand{\hsh}[1]{{\color{green!50!black} Henrik: #1}}
\newcommand{\st}[1]{{\color{red!50!black} Sebastian: #1}}

\newcommand{\ulm}[1]{_{\scaleto{\mathrm{#1}}{3pt}}}
\newcommand\at[2]{\left.#1\right|_{#2}}











\newtheorem{assumption}{Assumption}

\DeclareMathOperator*{\argmax}{arg\,max}
\DeclareMathOperator*{\argmin}{arg\,min}

\newcommand{\swname}[1]{\texttt{#1}}
\newcommand{\ie}{i\/.\/e\/.,\/~}
\newcommand{\eg}{e\/.\/g\/.,\/~}
\newcommand{\cf}{cf\/.\/~}

\newcommand{\fig}{Fig\/.\/~}
\newcommand{\defn}{Def\/.\/~}
\newcommand{\sect}{Sec\/.\/~}
\newcommand{\tabl}{Tab\/.\/~}
\newcommand{\algo}{Algorithm~}
\newcommand{\theo}{Theorem~}

\newcommand{\bnnl}{3 hidden layers}
\newcommand{\bnnn}{50 neurons}
\newcommand{\bnna}{tanh activations}

\newcommand{\capt}[1]{\mdseries{\emph{#1}}}

\newcommand{\videolink}{at \url{https://youtu.be/_d7AqTRjz6g}}
\newcommand{\codelink}{\url{https://github.com/wheelbot/mini-wheelbot}}

\newcommand{\fakepar}[1]{\vspace{0mm}\noindent\textbf{#1.}}

\newcommand{\needref}{\textcolor{red}{[REF]}}

\newcommand{\plotfontsize}{9pt}


\usepackage{xparse}
\usepackage{xspace}
%\usepackage[linesnumbered,ruled,vlined]{algorithm2e}
\usepackage{algorithm}% http://ctan.org/pkg/algorithm
\usepackage{algpseudocode}% http://ctan.org/pkg/algorithmicx
\usepackage{fixltx2e}
\usepackage{amsmath}
\usepackage{amsthm}
\usepackage{tikz}
\usepackage{float}
\usepackage{multirow}
\usepackage{multicol}
%\usepackage{todonotes}
\usepackage{colortbl}

%%%%%%%%%%%%%%%%%%%%%%%%%%%%

% Recommended, but optional, packages for figures and better typesetting:
\usepackage{microtype}
\usepackage{graphicx}
\usepackage{caption}
\usepackage{subcaption}
\usepackage{booktabs} % for professional tables
\usepackage{bbm}
\usepackage[shortlabels]{enumitem}
\usepackage[tiny]{todonotes}

\usepackage{hyperref,amssymb,enumitem}

\usepackage{wrapfig}

\usepackage{cleveref}

% Attempt to make hyperref and algorithmic work together better:
% \newcommand{\theHalgorithm}{\arabic{algorithm}}

\setlist[itemize]{leftmargin=*}
\setlist[enumerate]{leftmargin=*}

%%%%%%%%%%%%%%%%%%%%%%%%%%

%\floatname{algorithm}{Procedure}
\renewcommand{\algorithmicrequire}{\textbf{Input: }}

\renewcommand{\algorithmicensure}{\textbf{Output: }}

\newcommand{\ie}{\textit{i.e.,}\@\xspace}
\newcommand{\eg}{\textit{e.g.,}\@\xspace}
\newcommand{\etal}{\textit{et al.}\@\xspace}

\newcommand{\aug}{\text{Aug}}

\newcommand{\Aug}{\text{Aug}}

\newcommand{\lalign}{\mathcal{L}_{\text{align}}}
\newcommand{\Halign}{\mathcal{H}_{\text{align}}}

\newcommand{\alignexp}[2]%{\mathbb{E}_{{#2}', {#2}'' \sim \Aug(x) } d({#1}({#2}'), {#1}({#2}'')) }
{\underset{ {#2}', {#2}'' \sim \Aug(x) }{\mathbb{E}} [ d\left({#1}({#2}'), {#1}({#2}'')\right) ] }


\newcommand{\dmetric}[2]{d({#1},{#2})}

\newcommand{\pr}{\mathbf{Pr}}
\newcommand{\dist}{\mathcal{D}}
\newcommand{\err}{\text{err}}

\newcommand{\upi}{^{(i)}}
\newcommand{\enc}{f_v}

\newcommand{\canary}{candidate\xspace}
\newcommand{\canaries}{candidates\xspace}
\newcommand{\method}{method\xspace}
\newcommand{\methods}{methods\xspace}
\newcommand{\Method}{method\xspace}
\newcommand{\Methods}{methods\xspace}
\newcommand{\name}{\texttt{SSLMem}\xspace}

% Secure Multi-party Computation
\newcommand{\smc}{MPC\xspace}
\newcommand{\single}{single-label\xspace}
\newcommand{\multi}{multi-label\xspace}
\usepackage{amsmath}

\DeclareRobustCommand\encircle[1]{\tikz[baseline=(char.base)]{\node[shape=circle,fill,inner sep=1pt] (char) {\textcolor{white}{#1}}}}

%\DeclarePairedDelimiterX{\norm}[1]{\lVert}{\rVert}{#1}

% ATTACKS
\newcommand{\Jacobian}{Jacobian\xspace}
\newcommand{\DataFree}{DataFree\xspace}
\newcommand{\JacobianTR}{Jacobian-TR\xspace}


% Abbreviations.
\newcommand{\IND}{IND\xspace} % IN-Distribution 

% Adam's commands:
%\newcommand{\myitem}{\paragraph}
\newcommand{\myitem}{\textbf}
\newcommand{\badX}{100x\xspace}

   
% \usepackage[backend=biber,style=numeric,citestyle=numeric,sorting=none]{biblatex}
% \DeclareMathOperator*{\Argmin}{argmin}
% \DeclareMathOperator*{\Argmax}{argmax}
% \newtheorem{theorem}{Theorem}
% \newtheorem{lemma}[theorem]{Lemma}
% \newtheorem{corollary}[theorem]{Corollary}
% \newtheorem{definition}[theorem]{Definition}
% \newtheorem{fact}[theorem]{Fact}
% \newtheorem{observation}[theorem]{Observation}
% \newtheorem{proposition}[theorem]{Proposition}
% \newtheorem{claim}[theorem]{Claim}
% \newtheorem{notation}[theorem]{Notation}
% \newtheorem{assumption}{Assumption}


\graphicspath{{images/}}

\newif\ifdraft
\drafttrue

\newcommand\classratesize{0.8}

%Source: https://tex.stackexchange.com/questions/85200/include-data-from-a-txt-verbatim

\usepackage{fancyvrb}

% redefine \VerbatimInput
\RecustomVerbatimCommand{\VerbatimInput}{VerbatimInput}%
{fontsize=\footnotesize,
 %
 frame=lines,  % top and bottom rule only
 framesep=2em, % separation between frame and text
 %rulecolor=\color{Gray},
 %
% label=\fbox{\color{Black}data.txt},
% labelposition=topline,
 %
 commandchars=\|\(\), % escape character and argument delimiters for
                      % commands within the verbatim
 commentchar=*        % comment character
}

% COMMENTS
\ifdraft
\newcommand{\mynote}[1]{\textcolor{red}{[note: #1]}}
\newcommand{\franzi}[1]{\textcolor{purple}{[Franzi: #1]}}

\else
\newcommand{\franzi}[1]{}



