\section{Discussions and Future Work}\label{ch6:future}
We now discuss some limitations of our work that represent interesting directions for future work. Our \algasgo method reduces reliance on determining when clients should transition from saving to spending by allowing them to gradually spend their saved budgets over time, rather than waiting until a specific round to start spending. While our experiments indicate that transitioning from saving to spending midway through training generally yields good results, tuning the hyperparameters involved in estimating the transitioning round may improve utility. However, such hyperparameter tuning can lead to additional privacy loss~\citep{papernot2021hyperparameter} that would need to be accounted for. For future work, we believe that our time-adaptive DP-FL framework should be closely integrated with a form of privacy-preserving hyperparameter tuning to identify the best rounds in which to transition from savings to spending.

Furthermore, as we demonstrated theoretically and validated experimentally, adapting saving-related hyperparameters to clients' specific privacy budgets can enhance utility. To eliminate the risk of privacy leakage from this adaptation, we provide theoretical optimizations that rely solely on clients' privacy-related constraints, independent of their data. Future research can explore data-and-privacy joint measures to quantify clients' contributions with controlled privacy leakage and adapt client-specific savings decisions accordingly.










