
\documentclass{article} 
\usepackage{iclr2024_conference,times}

%%%%% NEW MATH DEFINITIONS %%%%%

\usepackage{amsmath,amsfonts,bm}
\usepackage{derivative}
% Mark sections of captions for referring to divisions of figures
\newcommand{\figleft}{{\em (Left)}}
\newcommand{\figcenter}{{\em (Center)}}
\newcommand{\figright}{{\em (Right)}}
\newcommand{\figtop}{{\em (Top)}}
\newcommand{\figbottom}{{\em (Bottom)}}
\newcommand{\captiona}{{\em (a)}}
\newcommand{\captionb}{{\em (b)}}
\newcommand{\captionc}{{\em (c)}}
\newcommand{\captiond}{{\em (d)}}

% Highlight a newly defined term
\newcommand{\newterm}[1]{{\bf #1}}

% Derivative d 
\newcommand{\deriv}{{\mathrm{d}}}

% Figure reference, lower-case.
\def\figref#1{figure~\ref{#1}}
% Figure reference, capital. For start of sentence
\def\Figref#1{Figure~\ref{#1}}
\def\twofigref#1#2{figures \ref{#1} and \ref{#2}}
\def\quadfigref#1#2#3#4{figures \ref{#1}, \ref{#2}, \ref{#3} and \ref{#4}}
% Section reference, lower-case.
\def\secref#1{section~\ref{#1}}
% Section reference, capital.
\def\Secref#1{Section~\ref{#1}}
% Reference to two sections.
\def\twosecrefs#1#2{sections \ref{#1} and \ref{#2}}
% Reference to three sections.
\def\secrefs#1#2#3{sections \ref{#1}, \ref{#2} and \ref{#3}}
% Reference to an equation, lower-case.
\def\eqref#1{equation~\ref{#1}}
% Reference to an equation, upper case
\def\Eqref#1{Equation~\ref{#1}}
% A raw reference to an equation---avoid using if possible
\def\plaineqref#1{\ref{#1}}
% Reference to a chapter, lower-case.
\def\chapref#1{chapter~\ref{#1}}
% Reference to an equation, upper case.
\def\Chapref#1{Chapter~\ref{#1}}
% Reference to a range of chapters
\def\rangechapref#1#2{chapters\ref{#1}--\ref{#2}}
% Reference to an algorithm, lower-case.
\def\algref#1{algorithm~\ref{#1}}
% Reference to an algorithm, upper case.
\def\Algref#1{Algorithm~\ref{#1}}
\def\twoalgref#1#2{algorithms \ref{#1} and \ref{#2}}
\def\Twoalgref#1#2{Algorithms \ref{#1} and \ref{#2}}
% Reference to a part, lower case
\def\partref#1{part~\ref{#1}}
% Reference to a part, upper case
\def\Partref#1{Part~\ref{#1}}
\def\twopartref#1#2{parts \ref{#1} and \ref{#2}}

\def\ceil#1{\lceil #1 \rceil}
\def\floor#1{\lfloor #1 \rfloor}
\def\1{\bm{1}}
\newcommand{\train}{\mathcal{D}}
\newcommand{\valid}{\mathcal{D_{\mathrm{valid}}}}
\newcommand{\test}{\mathcal{D_{\mathrm{test}}}}

\def\eps{{\epsilon}}


% Random variables
\def\reta{{\textnormal{$\eta$}}}
\def\ra{{\textnormal{a}}}
\def\rb{{\textnormal{b}}}
\def\rc{{\textnormal{c}}}
\def\rd{{\textnormal{d}}}
\def\re{{\textnormal{e}}}
\def\rf{{\textnormal{f}}}
\def\rg{{\textnormal{g}}}
\def\rh{{\textnormal{h}}}
\def\ri{{\textnormal{i}}}
\def\rj{{\textnormal{j}}}
\def\rk{{\textnormal{k}}}
\def\rl{{\textnormal{l}}}
% rm is already a command, just don't name any random variables m
\def\rn{{\textnormal{n}}}
\def\ro{{\textnormal{o}}}
\def\rp{{\textnormal{p}}}
\def\rq{{\textnormal{q}}}
\def\rr{{\textnormal{r}}}
\def\rs{{\textnormal{s}}}
\def\rt{{\textnormal{t}}}
\def\ru{{\textnormal{u}}}
\def\rv{{\textnormal{v}}}
\def\rw{{\textnormal{w}}}
\def\rx{{\textnormal{x}}}
\def\ry{{\textnormal{y}}}
\def\rz{{\textnormal{z}}}

% Random vectors
\def\rvepsilon{{\mathbf{\epsilon}}}
\def\rvphi{{\mathbf{\phi}}}
\def\rvtheta{{\mathbf{\theta}}}
\def\rva{{\mathbf{a}}}
\def\rvb{{\mathbf{b}}}
\def\rvc{{\mathbf{c}}}
\def\rvd{{\mathbf{d}}}
\def\rve{{\mathbf{e}}}
\def\rvf{{\mathbf{f}}}
\def\rvg{{\mathbf{g}}}
\def\rvh{{\mathbf{h}}}
\def\rvu{{\mathbf{i}}}
\def\rvj{{\mathbf{j}}}
\def\rvk{{\mathbf{k}}}
\def\rvl{{\mathbf{l}}}
\def\rvm{{\mathbf{m}}}
\def\rvn{{\mathbf{n}}}
\def\rvo{{\mathbf{o}}}
\def\rvp{{\mathbf{p}}}
\def\rvq{{\mathbf{q}}}
\def\rvr{{\mathbf{r}}}
\def\rvs{{\mathbf{s}}}
\def\rvt{{\mathbf{t}}}
\def\rvu{{\mathbf{u}}}
\def\rvv{{\mathbf{v}}}
\def\rvw{{\mathbf{w}}}
\def\rvx{{\mathbf{x}}}
\def\rvy{{\mathbf{y}}}
\def\rvz{{\mathbf{z}}}

% Elements of random vectors
\def\erva{{\textnormal{a}}}
\def\ervb{{\textnormal{b}}}
\def\ervc{{\textnormal{c}}}
\def\ervd{{\textnormal{d}}}
\def\erve{{\textnormal{e}}}
\def\ervf{{\textnormal{f}}}
\def\ervg{{\textnormal{g}}}
\def\ervh{{\textnormal{h}}}
\def\ervi{{\textnormal{i}}}
\def\ervj{{\textnormal{j}}}
\def\ervk{{\textnormal{k}}}
\def\ervl{{\textnormal{l}}}
\def\ervm{{\textnormal{m}}}
\def\ervn{{\textnormal{n}}}
\def\ervo{{\textnormal{o}}}
\def\ervp{{\textnormal{p}}}
\def\ervq{{\textnormal{q}}}
\def\ervr{{\textnormal{r}}}
\def\ervs{{\textnormal{s}}}
\def\ervt{{\textnormal{t}}}
\def\ervu{{\textnormal{u}}}
\def\ervv{{\textnormal{v}}}
\def\ervw{{\textnormal{w}}}
\def\ervx{{\textnormal{x}}}
\def\ervy{{\textnormal{y}}}
\def\ervz{{\textnormal{z}}}

% Random matrices
\def\rmA{{\mathbf{A}}}
\def\rmB{{\mathbf{B}}}
\def\rmC{{\mathbf{C}}}
\def\rmD{{\mathbf{D}}}
\def\rmE{{\mathbf{E}}}
\def\rmF{{\mathbf{F}}}
\def\rmG{{\mathbf{G}}}
\def\rmH{{\mathbf{H}}}
\def\rmI{{\mathbf{I}}}
\def\rmJ{{\mathbf{J}}}
\def\rmK{{\mathbf{K}}}
\def\rmL{{\mathbf{L}}}
\def\rmM{{\mathbf{M}}}
\def\rmN{{\mathbf{N}}}
\def\rmO{{\mathbf{O}}}
\def\rmP{{\mathbf{P}}}
\def\rmQ{{\mathbf{Q}}}
\def\rmR{{\mathbf{R}}}
\def\rmS{{\mathbf{S}}}
\def\rmT{{\mathbf{T}}}
\def\rmU{{\mathbf{U}}}
\def\rmV{{\mathbf{V}}}
\def\rmW{{\mathbf{W}}}
\def\rmX{{\mathbf{X}}}
\def\rmY{{\mathbf{Y}}}
\def\rmZ{{\mathbf{Z}}}

% Elements of random matrices
\def\ermA{{\textnormal{A}}}
\def\ermB{{\textnormal{B}}}
\def\ermC{{\textnormal{C}}}
\def\ermD{{\textnormal{D}}}
\def\ermE{{\textnormal{E}}}
\def\ermF{{\textnormal{F}}}
\def\ermG{{\textnormal{G}}}
\def\ermH{{\textnormal{H}}}
\def\ermI{{\textnormal{I}}}
\def\ermJ{{\textnormal{J}}}
\def\ermK{{\textnormal{K}}}
\def\ermL{{\textnormal{L}}}
\def\ermM{{\textnormal{M}}}
\def\ermN{{\textnormal{N}}}
\def\ermO{{\textnormal{O}}}
\def\ermP{{\textnormal{P}}}
\def\ermQ{{\textnormal{Q}}}
\def\ermR{{\textnormal{R}}}
\def\ermS{{\textnormal{S}}}
\def\ermT{{\textnormal{T}}}
\def\ermU{{\textnormal{U}}}
\def\ermV{{\textnormal{V}}}
\def\ermW{{\textnormal{W}}}
\def\ermX{{\textnormal{X}}}
\def\ermY{{\textnormal{Y}}}
\def\ermZ{{\textnormal{Z}}}

% Vectors
\def\vzero{{\bm{0}}}
\def\vone{{\bm{1}}}
\def\vmu{{\bm{\mu}}}
\def\vtheta{{\bm{\theta}}}
\def\vphi{{\bm{\phi}}}
\def\va{{\bm{a}}}
\def\vb{{\bm{b}}}
\def\vc{{\bm{c}}}
\def\vd{{\bm{d}}}
\def\ve{{\bm{e}}}
\def\vf{{\bm{f}}}
\def\vg{{\bm{g}}}
\def\vh{{\bm{h}}}
\def\vi{{\bm{i}}}
\def\vj{{\bm{j}}}
\def\vk{{\bm{k}}}
\def\vl{{\bm{l}}}
\def\vm{{\bm{m}}}
\def\vn{{\bm{n}}}
\def\vo{{\bm{o}}}
\def\vp{{\bm{p}}}
\def\vq{{\bm{q}}}
\def\vr{{\bm{r}}}
\def\vs{{\bm{s}}}
\def\vt{{\bm{t}}}
\def\vu{{\bm{u}}}
\def\vv{{\bm{v}}}
\def\vw{{\bm{w}}}
\def\vx{{\bm{x}}}
\def\vy{{\bm{y}}}
\def\vz{{\bm{z}}}

% Elements of vectors
\def\evalpha{{\alpha}}
\def\evbeta{{\beta}}
\def\evepsilon{{\epsilon}}
\def\evlambda{{\lambda}}
\def\evomega{{\omega}}
\def\evmu{{\mu}}
\def\evpsi{{\psi}}
\def\evsigma{{\sigma}}
\def\evtheta{{\theta}}
\def\eva{{a}}
\def\evb{{b}}
\def\evc{{c}}
\def\evd{{d}}
\def\eve{{e}}
\def\evf{{f}}
\def\evg{{g}}
\def\evh{{h}}
\def\evi{{i}}
\def\evj{{j}}
\def\evk{{k}}
\def\evl{{l}}
\def\evm{{m}}
\def\evn{{n}}
\def\evo{{o}}
\def\evp{{p}}
\def\evq{{q}}
\def\evr{{r}}
\def\evs{{s}}
\def\evt{{t}}
\def\evu{{u}}
\def\evv{{v}}
\def\evw{{w}}
\def\evx{{x}}
\def\evy{{y}}
\def\evz{{z}}

% Matrix
\def\mA{{\bm{A}}}
\def\mB{{\bm{B}}}
\def\mC{{\bm{C}}}
\def\mD{{\bm{D}}}
\def\mE{{\bm{E}}}
\def\mF{{\bm{F}}}
\def\mG{{\bm{G}}}
\def\mH{{\bm{H}}}
\def\mI{{\bm{I}}}
\def\mJ{{\bm{J}}}
\def\mK{{\bm{K}}}
\def\mL{{\bm{L}}}
\def\mM{{\bm{M}}}
\def\mN{{\bm{N}}}
\def\mO{{\bm{O}}}
\def\mP{{\bm{P}}}
\def\mQ{{\bm{Q}}}
\def\mR{{\bm{R}}}
\def\mS{{\bm{S}}}
\def\mT{{\bm{T}}}
\def\mU{{\bm{U}}}
\def\mV{{\bm{V}}}
\def\mW{{\bm{W}}}
\def\mX{{\bm{X}}}
\def\mY{{\bm{Y}}}
\def\mZ{{\bm{Z}}}
\def\mBeta{{\bm{\beta}}}
\def\mPhi{{\bm{\Phi}}}
\def\mLambda{{\bm{\Lambda}}}
\def\mSigma{{\bm{\Sigma}}}

% Tensor
\DeclareMathAlphabet{\mathsfit}{\encodingdefault}{\sfdefault}{m}{sl}
\SetMathAlphabet{\mathsfit}{bold}{\encodingdefault}{\sfdefault}{bx}{n}
\newcommand{\tens}[1]{\bm{\mathsfit{#1}}}
\def\tA{{\tens{A}}}
\def\tB{{\tens{B}}}
\def\tC{{\tens{C}}}
\def\tD{{\tens{D}}}
\def\tE{{\tens{E}}}
\def\tF{{\tens{F}}}
\def\tG{{\tens{G}}}
\def\tH{{\tens{H}}}
\def\tI{{\tens{I}}}
\def\tJ{{\tens{J}}}
\def\tK{{\tens{K}}}
\def\tL{{\tens{L}}}
\def\tM{{\tens{M}}}
\def\tN{{\tens{N}}}
\def\tO{{\tens{O}}}
\def\tP{{\tens{P}}}
\def\tQ{{\tens{Q}}}
\def\tR{{\tens{R}}}
\def\tS{{\tens{S}}}
\def\tT{{\tens{T}}}
\def\tU{{\tens{U}}}
\def\tV{{\tens{V}}}
\def\tW{{\tens{W}}}
\def\tX{{\tens{X}}}
\def\tY{{\tens{Y}}}
\def\tZ{{\tens{Z}}}


% Graph
\def\gA{{\mathcal{A}}}
\def\gB{{\mathcal{B}}}
\def\gC{{\mathcal{C}}}
\def\gD{{\mathcal{D}}}
\def\gE{{\mathcal{E}}}
\def\gF{{\mathcal{F}}}
\def\gG{{\mathcal{G}}}
\def\gH{{\mathcal{H}}}
\def\gI{{\mathcal{I}}}
\def\gJ{{\mathcal{J}}}
\def\gK{{\mathcal{K}}}
\def\gL{{\mathcal{L}}}
\def\gM{{\mathcal{M}}}
\def\gN{{\mathcal{N}}}
\def\gO{{\mathcal{O}}}
\def\gP{{\mathcal{P}}}
\def\gQ{{\mathcal{Q}}}
\def\gR{{\mathcal{R}}}
\def\gS{{\mathcal{S}}}
\def\gT{{\mathcal{T}}}
\def\gU{{\mathcal{U}}}
\def\gV{{\mathcal{V}}}
\def\gW{{\mathcal{W}}}
\def\gX{{\mathcal{X}}}
\def\gY{{\mathcal{Y}}}
\def\gZ{{\mathcal{Z}}}

% Sets
\def\sA{{\mathbb{A}}}
\def\sB{{\mathbb{B}}}
\def\sC{{\mathbb{C}}}
\def\sD{{\mathbb{D}}}
% Don't use a set called E, because this would be the same as our symbol
% for expectation.
\def\sF{{\mathbb{F}}}
\def\sG{{\mathbb{G}}}
\def\sH{{\mathbb{H}}}
\def\sI{{\mathbb{I}}}
\def\sJ{{\mathbb{J}}}
\def\sK{{\mathbb{K}}}
\def\sL{{\mathbb{L}}}
\def\sM{{\mathbb{M}}}
\def\sN{{\mathbb{N}}}
\def\sO{{\mathbb{O}}}
\def\sP{{\mathbb{P}}}
\def\sQ{{\mathbb{Q}}}
\def\sR{{\mathbb{R}}}
\def\sS{{\mathbb{S}}}
\def\sT{{\mathbb{T}}}
\def\sU{{\mathbb{U}}}
\def\sV{{\mathbb{V}}}
\def\sW{{\mathbb{W}}}
\def\sX{{\mathbb{X}}}
\def\sY{{\mathbb{Y}}}
\def\sZ{{\mathbb{Z}}}

% Entries of a matrix
\def\emLambda{{\Lambda}}
\def\emA{{A}}
\def\emB{{B}}
\def\emC{{C}}
\def\emD{{D}}
\def\emE{{E}}
\def\emF{{F}}
\def\emG{{G}}
\def\emH{{H}}
\def\emI{{I}}
\def\emJ{{J}}
\def\emK{{K}}
\def\emL{{L}}
\def\emM{{M}}
\def\emN{{N}}
\def\emO{{O}}
\def\emP{{P}}
\def\emQ{{Q}}
\def\emR{{R}}
\def\emS{{S}}
\def\emT{{T}}
\def\emU{{U}}
\def\emV{{V}}
\def\emW{{W}}
\def\emX{{X}}
\def\emY{{Y}}
\def\emZ{{Z}}
\def\emSigma{{\Sigma}}

% entries of a tensor
% Same font as tensor, without \bm wrapper
\newcommand{\etens}[1]{\mathsfit{#1}}
\def\etLambda{{\etens{\Lambda}}}
\def\etA{{\etens{A}}}
\def\etB{{\etens{B}}}
\def\etC{{\etens{C}}}
\def\etD{{\etens{D}}}
\def\etE{{\etens{E}}}
\def\etF{{\etens{F}}}
\def\etG{{\etens{G}}}
\def\etH{{\etens{H}}}
\def\etI{{\etens{I}}}
\def\etJ{{\etens{J}}}
\def\etK{{\etens{K}}}
\def\etL{{\etens{L}}}
\def\etM{{\etens{M}}}
\def\etN{{\etens{N}}}
\def\etO{{\etens{O}}}
\def\etP{{\etens{P}}}
\def\etQ{{\etens{Q}}}
\def\etR{{\etens{R}}}
\def\etS{{\etens{S}}}
\def\etT{{\etens{T}}}
\def\etU{{\etens{U}}}
\def\etV{{\etens{V}}}
\def\etW{{\etens{W}}}
\def\etX{{\etens{X}}}
\def\etY{{\etens{Y}}}
\def\etZ{{\etens{Z}}}

% The true underlying data generating distribution
\newcommand{\pdata}{p_{\rm{data}}}
\newcommand{\ptarget}{p_{\rm{target}}}
\newcommand{\pprior}{p_{\rm{prior}}}
\newcommand{\pbase}{p_{\rm{base}}}
\newcommand{\pref}{p_{\rm{ref}}}

% The empirical distribution defined by the training set
\newcommand{\ptrain}{\hat{p}_{\rm{data}}}
\newcommand{\Ptrain}{\hat{P}_{\rm{data}}}
% The model distribution
\newcommand{\pmodel}{p_{\rm{model}}}
\newcommand{\Pmodel}{P_{\rm{model}}}
\newcommand{\ptildemodel}{\tilde{p}_{\rm{model}}}
% Stochastic autoencoder distributions
\newcommand{\pencode}{p_{\rm{encoder}}}
\newcommand{\pdecode}{p_{\rm{decoder}}}
\newcommand{\precons}{p_{\rm{reconstruct}}}

\newcommand{\laplace}{\mathrm{Laplace}} % Laplace distribution

\newcommand{\E}{\mathbb{E}}
\newcommand{\Ls}{\mathcal{L}}
\newcommand{\R}{\mathbb{R}}
\newcommand{\emp}{\tilde{p}}
\newcommand{\lr}{\alpha}
\newcommand{\reg}{\lambda}
\newcommand{\rect}{\mathrm{rectifier}}
\newcommand{\softmax}{\mathrm{softmax}}
\newcommand{\sigmoid}{\sigma}
\newcommand{\softplus}{\zeta}
\newcommand{\KL}{D_{\mathrm{KL}}}
\newcommand{\Var}{\mathrm{Var}}
\newcommand{\standarderror}{\mathrm{SE}}
\newcommand{\Cov}{\mathrm{Cov}}
% Wolfram Mathworld says $L^2$ is for function spaces and $\ell^2$ is for vectors
% But then they seem to use $L^2$ for vectors throughout the site, and so does
% wikipedia.
\newcommand{\normlzero}{L^0}
\newcommand{\normlone}{L^1}
\newcommand{\normltwo}{L^2}
\newcommand{\normlp}{L^p}
\newcommand{\normmax}{L^\infty}

\newcommand{\parents}{Pa} % See usage in notation.tex. Chosen to match Daphne's book.

\DeclareMathOperator*{\argmax}{arg\,max}
\DeclareMathOperator*{\argmin}{arg\,min}

\DeclareMathOperator{\sign}{sign}
\DeclareMathOperator{\Tr}{Tr}
\let\ab\allowbreak


\usepackage{hyperref}
\usepackage{url}
\usepackage{subcaption}
\usepackage{graphicx}
\usepackage{multirow}
\usepackage{booktabs}
\usepackage{adjustbox}
\usepackage{array}
\usepackage{float}
\usepackage{makecell}
\usepackage{threeparttable}
\usepackage{enumitem}

\usepackage{cleveref}

%
%
\setlength\unitlength{1mm}
\newcommand{\twodots}{\mathinner {\ldotp \ldotp}}
% bb font symbols
\newcommand{\Rho}{\mathrm{P}}
\newcommand{\Tau}{\mathrm{T}}

\newfont{\bbb}{msbm10 scaled 700}
\newcommand{\CCC}{\mbox{\bbb C}}

\newfont{\bb}{msbm10 scaled 1100}
\newcommand{\CC}{\mbox{\bb C}}
\newcommand{\PP}{\mbox{\bb P}}
\newcommand{\RR}{\mbox{\bb R}}
\newcommand{\QQ}{\mbox{\bb Q}}
\newcommand{\ZZ}{\mbox{\bb Z}}
\newcommand{\FF}{\mbox{\bb F}}
\newcommand{\GG}{\mbox{\bb G}}
\newcommand{\EE}{\mbox{\bb E}}
\newcommand{\NN}{\mbox{\bb N}}
\newcommand{\KK}{\mbox{\bb K}}
\newcommand{\HH}{\mbox{\bb H}}
\newcommand{\SSS}{\mbox{\bb S}}
\newcommand{\UU}{\mbox{\bb U}}
\newcommand{\VV}{\mbox{\bb V}}


\newcommand{\yy}{\mathbbm{y}}
\newcommand{\xx}{\mathbbm{x}}
\newcommand{\zz}{\mathbbm{z}}
\newcommand{\sss}{\mathbbm{s}}
\newcommand{\rr}{\mathbbm{r}}
\newcommand{\pp}{\mathbbm{p}}
\newcommand{\qq}{\mathbbm{q}}
\newcommand{\ww}{\mathbbm{w}}
\newcommand{\hh}{\mathbbm{h}}
\newcommand{\vvv}{\mathbbm{v}}

% Vectors

\newcommand{\av}{{\bf a}}
\newcommand{\bv}{{\bf b}}
\newcommand{\cv}{{\bf c}}
\newcommand{\dv}{{\bf d}}
\newcommand{\ev}{{\bf e}}
\newcommand{\fv}{{\bf f}}
\newcommand{\gv}{{\bf g}}
\newcommand{\hv}{{\bf h}}
\newcommand{\iv}{{\bf i}}
\newcommand{\jv}{{\bf j}}
\newcommand{\kv}{{\bf k}}
\newcommand{\lv}{{\bf l}}
\newcommand{\mv}{{\bf m}}
\newcommand{\nv}{{\bf n}}
\newcommand{\ov}{{\bf o}}
\newcommand{\pv}{{\bf p}}
\newcommand{\qv}{{\bf q}}
\newcommand{\rv}{{\bf r}}
\newcommand{\sv}{{\bf s}}
\newcommand{\tv}{{\bf t}}
\newcommand{\uv}{{\bf u}}
\newcommand{\wv}{{\bf w}}
\newcommand{\vv}{{\bf v}}
\newcommand{\xv}{{\bf x}}
\newcommand{\yv}{{\bf y}}
\newcommand{\zv}{{\bf z}}
\newcommand{\zerov}{{\bf 0}}
\newcommand{\onev}{{\bf 1}}

% Matrices

\newcommand{\Am}{{\bf A}}
\newcommand{\Bm}{{\bf B}}
\newcommand{\Cm}{{\bf C}}
\newcommand{\Dm}{{\bf D}}
\newcommand{\Em}{{\bf E}}
\newcommand{\Fm}{{\bf F}}
\newcommand{\Gm}{{\bf G}}
\newcommand{\Hm}{{\bf H}}
\newcommand{\Id}{{\bf I}}
\newcommand{\Jm}{{\bf J}}
\newcommand{\Km}{{\bf K}}
\newcommand{\Lm}{{\bf L}}
\newcommand{\Mm}{{\bf M}}
\newcommand{\Nm}{{\bf N}}
\newcommand{\Om}{{\bf O}}
\newcommand{\Pm}{{\bf P}}
\newcommand{\Qm}{{\bf Q}}
\newcommand{\Rm}{{\bf R}}
\newcommand{\Sm}{{\bf S}}
\newcommand{\Tm}{{\bf T}}
\newcommand{\Um}{{\bf U}}
\newcommand{\Wm}{{\bf W}}
\newcommand{\Vm}{{\bf V}}
\newcommand{\Xm}{{\bf X}}
\newcommand{\Ym}{{\bf Y}}
\newcommand{\Zm}{{\bf Z}}

% Calligraphic

\newcommand{\Ac}{{\cal A}}
\newcommand{\Bc}{{\cal B}}
\newcommand{\Cc}{{\cal C}}
\newcommand{\Dc}{{\cal D}}
\newcommand{\Ec}{{\cal E}}
\newcommand{\Fc}{{\cal F}}
\newcommand{\Gc}{{\cal G}}
\newcommand{\Hc}{{\cal H}}
\newcommand{\Ic}{{\cal I}}
\newcommand{\Jc}{{\cal J}}
\newcommand{\Kc}{{\cal K}}
\newcommand{\Lc}{{\cal L}}
\newcommand{\Mc}{{\cal M}}
\newcommand{\Nc}{{\cal N}}
\newcommand{\nc}{{\cal n}}
\newcommand{\Oc}{{\cal O}}
\newcommand{\Pc}{{\cal P}}
\newcommand{\Qc}{{\cal Q}}
\newcommand{\Rc}{{\cal R}}
\newcommand{\Sc}{{\cal S}}
\newcommand{\Tc}{{\cal T}}
\newcommand{\Uc}{{\cal U}}
\newcommand{\Wc}{{\cal W}}
\newcommand{\Vc}{{\cal V}}
\newcommand{\Xc}{{\cal X}}
\newcommand{\Yc}{{\cal Y}}
\newcommand{\Zc}{{\cal Z}}

% Bold greek letters

\newcommand{\alphav}{\hbox{\boldmath$\alpha$}}
\newcommand{\betav}{\hbox{\boldmath$\beta$}}
\newcommand{\gammav}{\hbox{\boldmath$\gamma$}}
\newcommand{\deltav}{\hbox{\boldmath$\delta$}}
\newcommand{\etav}{\hbox{\boldmath$\eta$}}
\newcommand{\lambdav}{\hbox{\boldmath$\lambda$}}
\newcommand{\epsilonv}{\hbox{\boldmath$\epsilon$}}
\newcommand{\nuv}{\hbox{\boldmath$\nu$}}
\newcommand{\muv}{\hbox{\boldmath$\mu$}}
\newcommand{\zetav}{\hbox{\boldmath$\zeta$}}
\newcommand{\phiv}{\hbox{\boldmath$\phi$}}
\newcommand{\psiv}{\hbox{\boldmath$\psi$}}
\newcommand{\thetav}{\hbox{\boldmath$\theta$}}
\newcommand{\tauv}{\hbox{\boldmath$\tau$}}
\newcommand{\omegav}{\hbox{\boldmath$\omega$}}
\newcommand{\xiv}{\hbox{\boldmath$\xi$}}
\newcommand{\sigmav}{\hbox{\boldmath$\sigma$}}
\newcommand{\piv}{\hbox{\boldmath$\pi$}}
\newcommand{\rhov}{\hbox{\boldmath$\rho$}}
\newcommand{\upsilonv}{\hbox{\boldmath$\upsilon$}}

\newcommand{\Gammam}{\hbox{\boldmath$\Gamma$}}
\newcommand{\Lambdam}{\hbox{\boldmath$\Lambda$}}
\newcommand{\Deltam}{\hbox{\boldmath$\Delta$}}
\newcommand{\Sigmam}{\hbox{\boldmath$\Sigma$}}
\newcommand{\Phim}{\hbox{\boldmath$\Phi$}}
\newcommand{\Pim}{\hbox{\boldmath$\Pi$}}
\newcommand{\Psim}{\hbox{\boldmath$\Psi$}}
\newcommand{\Thetam}{\hbox{\boldmath$\Theta$}}
\newcommand{\Omegam}{\hbox{\boldmath$\Omega$}}
\newcommand{\Xim}{\hbox{\boldmath$\Xi$}}


% Sans Serif small case

\newcommand{\Gsf}{{\sf G}}

\newcommand{\asf}{{\sf a}}
\newcommand{\bsf}{{\sf b}}
\newcommand{\csf}{{\sf c}}
\newcommand{\dsf}{{\sf d}}
\newcommand{\esf}{{\sf e}}
\newcommand{\fsf}{{\sf f}}
\newcommand{\gsf}{{\sf g}}
\newcommand{\hsf}{{\sf h}}
\newcommand{\isf}{{\sf i}}
\newcommand{\jsf}{{\sf j}}
\newcommand{\ksf}{{\sf k}}
\newcommand{\lsf}{{\sf l}}
\newcommand{\msf}{{\sf m}}
\newcommand{\nsf}{{\sf n}}
\newcommand{\osf}{{\sf o}}
\newcommand{\psf}{{\sf p}}
\newcommand{\qsf}{{\sf q}}
\newcommand{\rsf}{{\sf r}}
\newcommand{\ssf}{{\sf s}}
\newcommand{\tsf}{{\sf t}}
\newcommand{\usf}{{\sf u}}
\newcommand{\wsf}{{\sf w}}
\newcommand{\vsf}{{\sf v}}
\newcommand{\xsf}{{\sf x}}
\newcommand{\ysf}{{\sf y}}
\newcommand{\zsf}{{\sf z}}


% mixed symbols

\newcommand{\sinc}{{\hbox{sinc}}}
\newcommand{\diag}{{\hbox{diag}}}
\renewcommand{\det}{{\hbox{det}}}
\newcommand{\trace}{{\hbox{tr}}}
\newcommand{\sign}{{\hbox{sign}}}
\renewcommand{\arg}{{\hbox{arg}}}
\newcommand{\var}{{\hbox{var}}}
\newcommand{\cov}{{\hbox{cov}}}
\newcommand{\Ei}{{\rm E}_{\rm i}}
\renewcommand{\Re}{{\rm Re}}
\renewcommand{\Im}{{\rm Im}}
\newcommand{\eqdef}{\stackrel{\Delta}{=}}
\newcommand{\defines}{{\,\,\stackrel{\scriptscriptstyle \bigtriangleup}{=}\,\,}}
\newcommand{\<}{\left\langle}
\renewcommand{\>}{\right\rangle}
\newcommand{\herm}{{\sf H}}
\newcommand{\trasp}{{\sf T}}
\newcommand{\transp}{{\sf T}}
\renewcommand{\vec}{{\rm vec}}
\newcommand{\Psf}{{\sf P}}
\newcommand{\SINR}{{\sf SINR}}
\newcommand{\SNR}{{\sf SNR}}
\newcommand{\MMSE}{{\sf MMSE}}
\newcommand{\REF}{{\RED [REF]}}

% Markov chain
\usepackage{stmaryrd} % for \mkv 
\newcommand{\mkv}{-\!\!\!\!\minuso\!\!\!\!-}

% Colors

\newcommand{\RED}{\color[rgb]{1.00,0.10,0.10}}
\newcommand{\BLUE}{\color[rgb]{0,0,0.90}}
\newcommand{\GREEN}{\color[rgb]{0,0.80,0.20}}

%%%%%%%%%%%%%%%%%%%%%%%%%%%%%%%%%%%%%%%%%%
\usepackage{hyperref}
\hypersetup{
    bookmarks=true,         % show bookmarks bar?
    unicode=false,          % non-Latin characters in AcrobatÕs bookmarks
    pdftoolbar=true,        % show AcrobatÕs toolbar?
    pdfmenubar=true,        % show AcrobatÕs menu?
    pdffitwindow=false,     % window fit to page when opened
    pdfstartview={FitH},    % fits the width of the page to the window
%    pdftitle={My title},    % title
%    pdfauthor={Author},     % author
%    pdfsubject={Subject},   % subject of the document
%    pdfcreator={Creator},   % creator of the document
%    pdfproducer={Producer}, % producer of the document
%    pdfkeywords={keyword1} {key2} {key3}, % list of keywords
    pdfnewwindow=true,      % links in new window
    colorlinks=true,       % false: boxed links; true: colored links
    linkcolor=red,          % color of internal links (change box color with linkbordercolor)
    citecolor=green,        % color of links to bibliography
    filecolor=blue,      % color of file links
    urlcolor=blue           % color of external links
}
%%%%%%%%%%%%%%%%%%%%%%%%%%%%%%%%%%%%%%%%%%%


% \usepackage[textsize=tiny]{todonotes}
% % COMMENTS
% \ifdraft
% \newcommand{\mynote}[1]{\textcolor{red}{[note: #1]}}
% \newcommand{\franzi}[1]{\textcolor{purple}{[Franzi: #1]}}

% \else
% \newcommand{\franzi}[1]{}

\newif\ifdraft % Defining the draft mode
\drafttrue     % Uncomment this to enable draft mode
%\draftfalse   % Uncomment this to disable draft mode

\usepackage[textsize=tiny]{todonotes}
% COMMENTS
\ifdraft
    \newcommand{\mynote}[1]{\textcolor{red}{[note: #1]}}
    \newcommand{\franzi}[1]{\textcolor{purple}{[Franzi: #1]}}
\else
    \newcommand{\mynote}[1]{}
    \newcommand{\franzi}[1]{}
\fi


\title{Differentially Private Federated Learning \\ with Time-Adaptive Privacy Spending}

% Authors must not appear in the submitted version. They should be hidden
% as long as the \iclrfinalcopy macro remains commented out below.
% Non-anonymous submissions will be rejected without review.

\iclrfinalcopy
\author{Shahrzad Kiani\textsuperscript{1}\thanks{%Emails: Shahrzad Kiani \texttt{shahrzad.kianidehkordi@mail.utoronto.ca}, Nupur Kulkarni \texttt{nuku00001@stud.uni-saarland.de}, Adam Dziedzic \texttt{adam.dziedzic@cispa.de}, Stark Draper \texttt{stark.draper@utoronto.ca}, and Franziska Boenisch \texttt{boenisch@cispa.de}. 
correspondence to shahrzad.kianidehkordi@mail.utoronto.ca. Part of the work was done while Shahrzad Kiani visited CISPA.}, Nupur Kulkarni\textsuperscript{2}, Adam Dziedzic\textsuperscript{2}, Stark Draper\textsuperscript{1}, \& Franziska Boenisch\textsuperscript{2} %Antiquus S.~Hippocampus, Natalia Cerebro \& Amelie P. Amygdale \thanks{ Use footnote for providing further information about author (webpage, alternative address)---\emph{not} for acknowledging funding agencies.  Funding acknowledgements go at the end of the paper.} 
\\
\textsuperscript{1} Department of Electrical and Computer Engineering, University of Toronto \\
\textsuperscript{2} CISPA Helmholtz Center for Information Security 
%Department of Computer Science\\
%Cranberry-Lemon University\\
%Pittsburgh, PA 15213, USA 
%\\
%\textsuperscript{3}\texttt{\{shahrzad.kianidehkordi@mail, stark.draper@\}.utoronto.ca}
%\\
%\textsuperscript{4}\texttt{\{ adam.dziedzic, boenisch\}.cispa.de}
%\texttt{\{hippo,brain,jen\}@cs.cranberry-lemon.edu} \\
%\And
%Ji Q. Ren \& Yevgeny LeNet \\
%Department of Computational Neuroscience \\
%University of the Witwatersrand \\
%Joburg, South Africa \\
%\texttt{\{robot,net\}@wits.ac.za} \\
%\AND
%Coauthor \\
%Affiliation \\
%Address \\
%\texttt{email}
%
}

\newcommand{\fix}{\marginpar{FIX}}
\newcommand{\new}{\marginpar{NEW}}


\begin{document}

\maketitle

\begin{abstract}


Federated learning (FL) with differential privacy (DP) provides a framework for collaborative machine learning, enabling clients to train a shared model while adhering to strict privacy constraints.
The framework allows each client to have an individual privacy guarantee, e.g., by adding different amounts of noise to each client's model updates. One underlying assumption is that all clients spend their privacy budgets uniformly over time (learning rounds). However, it has been shown in the literature that learning in early rounds typically focuses on more coarse-grained features that can be learned at lower signal-to-noise ratios while later rounds learn fine-grained features that benefit from higher signal-to-noise ratios.
Building on this intuition, we propose a {\em time-adaptive} DP-FL framework that expends the privacy budget non-uniformly across both time and clients. 
Our framework enables each client to save privacy budget in early rounds so as to be able to spend more in later rounds when additional accuracy is beneficial in learning more fine-grained features. 
We theoretically prove utility improvements in the case that clients with stricter privacy budgets spend budgets unevenly across rounds, compared to clients with more relaxed budgets, who have sufficient budgets to distribute their spend more evenly. Our practical experiments on standard benchmark datasets support our theoretical results and show that, in practice, our algorithms improve the privacy-utility trade-offs compared to baseline schemes.


\end{abstract}

\section{Introduction}

Video generation has garnered significant attention owing to its transformative potential across a wide range of applications, such media content creation~\citep{polyak2024movie}, advertising~\citep{zhang2024virbo,bacher2021advert}, video games~\citep{yang2024playable,valevski2024diffusion, oasis2024}, and world model simulators~\citep{ha2018world, videoworldsimulators2024, agarwal2025cosmos}. Benefiting from advanced generative algorithms~\citep{goodfellow2014generative, ho2020denoising, liu2023flow, lipman2023flow}, scalable model architectures~\citep{vaswani2017attention, peebles2023scalable}, vast amounts of internet-sourced data~\citep{chen2024panda, nan2024openvid, ju2024miradata}, and ongoing expansion of computing capabilities~\citep{nvidia2022h100, nvidia2023dgxgh200, nvidia2024h200nvl}, remarkable advancements have been achieved in the field of video generation~\citep{ho2022video, ho2022imagen, singer2023makeavideo, blattmann2023align, videoworldsimulators2024, kuaishou2024klingai, yang2024cogvideox, jin2024pyramidal, polyak2024movie, kong2024hunyuanvideo, ji2024prompt}.


In this work, we present \textbf{\ours}, a family of rectified flow~\citep{lipman2023flow, liu2023flow} transformer models designed for joint image and video generation, establishing a pathway toward industry-grade performance. This report centers on four key components: data curation, model architecture design, flow formulation, and training infrastructure optimization—each rigorously refined to meet the demands of high-quality, large-scale video generation.


\begin{figure}[ht]
    \centering
    \begin{subfigure}[b]{0.82\linewidth}
        \centering
        \includegraphics[width=\linewidth]{figures/t2i_1024.pdf}
        \caption{Text-to-Image Samples}\label{fig:main-demo-t2i}
    \end{subfigure}
    \vfill
    \begin{subfigure}[b]{0.82\linewidth}
        \centering
        \includegraphics[width=\linewidth]{figures/t2v_samples.pdf}
        \caption{Text-to-Video Samples}\label{fig:main-demo-t2v}
    \end{subfigure}
\caption{\textbf{Generated samples from \ours.} Key components are highlighted in \textcolor{red}{\textbf{RED}}.}\label{fig:main-demo}
\end{figure}


First, we present a comprehensive data processing pipeline designed to construct large-scale, high-quality image and video-text datasets. The pipeline integrates multiple advanced techniques, including video and image filtering based on aesthetic scores, OCR-driven content analysis, and subjective evaluations, to ensure exceptional visual and contextual quality. Furthermore, we employ multimodal large language models~(MLLMs)~\citep{yuan2025tarsier2} to generate dense and contextually aligned captions, which are subsequently refined using an additional large language model~(LLM)~\citep{yang2024qwen2} to enhance their accuracy, fluency, and descriptive richness. As a result, we have curated a robust training dataset comprising approximately 36M video-text pairs and 160M image-text pairs, which are proven sufficient for training industry-level generative models.

Secondly, we take a pioneering step by applying rectified flow formulation~\citep{lipman2023flow} for joint image and video generation, implemented through the \ours model family, which comprises Transformer architectures with 2B and 8B parameters. At its core, the \ours framework employs a 3D joint image-video variational autoencoder (VAE) to compress image and video inputs into a shared latent space, facilitating unified representation. This shared latent space is coupled with a full-attention~\citep{vaswani2017attention} mechanism, enabling seamless joint training of image and video. This architecture delivers high-quality, coherent outputs across both images and videos, establishing a unified framework for visual generation tasks.


Furthermore, to support the training of \ours at scale, we have developed a robust infrastructure tailored for large-scale model training. Our approach incorporates advanced parallelism strategies~\citep{jacobs2023deepspeed, pytorch_fsdp} to manage memory efficiently during long-context training. Additionally, we employ ByteCheckpoint~\citep{wan2024bytecheckpoint} for high-performance checkpointing and integrate fault-tolerant mechanisms from MegaScale~\citep{jiang2024megascale} to ensure stability and scalability across large GPU clusters. These optimizations enable \ours to handle the computational and data challenges of generative modeling with exceptional efficiency and reliability.


We evaluate \ours on both text-to-image and text-to-video benchmarks to highlight its competitive advantages. For text-to-image generation, \ours-T2I demonstrates strong performance across multiple benchmarks, including T2I-CompBench~\citep{huang2023t2i-compbench}, GenEval~\citep{ghosh2024geneval}, and DPG-Bench~\citep{hu2024ella_dbgbench}, excelling in both visual quality and text-image alignment. In text-to-video benchmarks, \ours-T2V achieves state-of-the-art performance on the UCF-101~\citep{ucf101} zero-shot generation task. Additionally, \ours-T2V attains an impressive score of \textbf{84.85} on VBench~\citep{huang2024vbench}, securing the top position on the leaderboard (as of 2025-01-25) and surpassing several leading commercial text-to-video models. Qualitative results, illustrated in \Cref{fig:main-demo}, further demonstrate the superior quality of the generated media samples. These findings underscore \ours's effectiveness in multi-modal generation and its potential as a high-performing solution for both research and commercial applications.
\section{Background} \label{section:LLM}

% \subsection{Large Language Model (LLM)}   

Figure~\ref{fig:LLaMA_model}(a) shows that a decoder-only LLM initially processes a user prompt in the “prefill” stage and subsequently generates tokens sequentially during the “decoding” stage.
Both stages contain an input embedding layer, multiple decoder transformer blocks, an output embedding layer, and a sampling layer.
Figure~\ref{fig:LLaMA_model}(b) demonstrates that the decoder transformer blocks consist of a self attention and a feed-forward network (FFN) layer, each paired with residual connection and normalization layers. 

% Differentiate between encoder/decoder, explain why operation intensity is low, explain the different parts of a transformer block. Discuss Table II here. 

% Explain the architecture with Llama2-70B.

% \begin{table}[thb]
% \renewcommand\arraystretch{1.05}
% \centering
% % \vspace{-5mm}
%     \caption{ML Model Parameter Size and Operational Intensity}
%     \vspace{-2mm}
%     \small
%     \label{tab:ML Model Parameter Size and Operational Intensity}    
%     \scalebox{0.95}{
%         \begin{tabular}{|c|c|c|c|c|}
%             \hline
%             & Llama2 & BLOOM & BERT & ResNet \\
%             Model & (70B) & (176B) & & 152 \\
%             \hline
%             Parameter Size (GB) & 140 & 352 & 0.17 & 0.16 \\
%             \hline
%             Op Intensity (Ops/Byte) & 1 & 1 & 282 & 346 \\
%             \hline
%           \end{tabular}
%     }
% \vspace{-3mm}
% \end{table}

% {\fontsize{8pt}{11pt}\selectfont 8pt font size test Memory Requirement}

\begin{figure}[t]
    \centering
    \includegraphics[width=8cm]{Figure/LLaMA_model_new_new.pdf}
    \caption{(a) Prefill stage encodes prompt tokens in parallel. Decoding stage generates output tokens sequentially.
    (b) LLM contains N$\times$ decoder transformer blocks. 
    (c) Llama2 model architecture.}
    \label{fig:LLaMA_model}
\end{figure}

Figure~\ref{fig:LLaMA_model}(c) demonstrates the Llama2~\cite{touvron2023llama} model architecture as a representative LLM.
% The self attention layer requires three GEMVs\footnote{GEMVs in multi-head attention~\cite{attention}, narrow GEMMs in grouped-query attention~\cite{gqa}.} to generate query, key and value vectors.
In the self-attention layer, query, key and value vectors are generated by multiplying input vector to corresponding weight matrices.
These matrices are segmented into multiple heads, representing different semantic dimensions.
The query and key vectors go though Rotary Positional Embedding (RoPE) to encode the relative positional information~\cite{rope-paper}.
Within each head, the generated key and value vectors are appended to their caches.
The query vector is multiplied by the key cache to produce a score vector.
After the Softmax operation, the score vector is multiplied by the value cache to yield the output vector.
The output vectors from all heads are concatenated and multiplied by output weight matrix, resulting in a vector that undergoes residual connection and Root Mean Square layer Normalization (RMSNorm)~\cite{rmsnorm-paper}.
The residual connection adds up the input and output vectors of a layer to avoid vanishing gradient~\cite{he2016deep}.
The FFN layer begins with two parallel fully connections, followed by a Sigmoid Linear Unit (SiLU), and ends with another fully connection.
\section{Problem Statements and Its Property}
\label{sec:problem}

This section provides our problem setup and its properties.

%%%%%%%%%%%%%%%%%%%%%%%%%%%%%%%%%%%%%%%%%%%%%%%%%%%%%%%%%%%%%%%%%%%%%%%%%%%%
\subsection{Problem Statement}

We aim to minimize the worst-case expected errors regarding the GP prediction $\mu_T (\*x)$ after $T$-th function evaluations:
\begin{align}
    % {\rm DRAE}_T &\coloneqq \max_{p \in \cP} \EE_{p(\*x)} \left[ | f(\*x) - \mu_T(\*x) | \right] \\
    E_T &\coloneqq \max_{p \in \cP} \EE_{p(\*x^{*})} \left[ ( f(\*x^{*}) - \mu_T(\*x^{*}) )^2 \right],
    \label{eq:target_error}
\end{align}
where $\cP$ is a set of target distributions over the input space $\cX$ called ambiguity set~\citep{chen2020distributionally}.
%
We assume that $\max_{p \in \cP} \EE_{p(\*x^*)} \left[ g(\*x^*) \right]$ exists for any continuous function $g: \cX \rightarrow \RR$.
%
This paper concentrates on the setting where the training input space from which we can obtain labels includes the test input space.



Our problem setup can be seen as the generalization of the target distribution-aware AL and the AL for the worst-case error $\max_{\*x \in \cX} ( f(\*x) - \mu_T(\*x) )^2$.
%
This is because our problem is equivalent to the target distribution-aware AL if we set $|\cP| = 1$ and to the worst-case error minimization if $\cP$ includes $\{p \in \cP_{\cX} \mid \exists \*x \in \cX, p(\*x) = 1 \}$, where $\cP_{\rm \cX}$ is the set of the distributions over $\cX$.


%%%%%%%%%%%%%%%%%%%%%%%%%%%%%%%%%%%%%%%%%%%%%%%%%%%%%%%%%%%%%%%%%%%%%%%%%%%%
\subsection{High Probability Bound of Error}

% First, we provide the upper bound by the posterior variance.
%
If the input space $\cX$ is finite, we can obtain the upper bound of Eq.~\eqref{eq:target_error} as the direct consequence of Lemmas~\ref{lem:bound_srinivas} and \ref{lem:bound_vakili}:
\begin{lemma}
    Fix $\delta \in (0, 1)$ and $T \in \NN$.
    %
    Suppose that Assumption~\ref{assump:Bayesian} holds and $\beta_\delta$ is set as in Lemma~\ref{lem:bound_srinivas}, or Assumption~\ref{assump:frequentist} holds and $\beta_\delta$ is set as in Lemma~\ref{lem:bound_vakili}.
    %
    Then, the following holds with probability at least $1 - \delta$:
    \begin{align*}
        E_T &\leq \beta_{\delta} \max_{p \in \cP} \EE_{p(\*x^{*})}\left[ \sigma^2_{T}(\*x^{*}) \right].
    \end{align*}
    \label{lem:UB_error_discrete}
\end{lemma}


% Next, let us consider the case that $\cX = [0, r]^d$.
%
For continuous $\cX$, the confidence parameter $\beta_\delta \propto \log |\cX|$ diverges if we apply Lemmas~\ref{lem:bound_srinivas} and \ref{lem:bound_vakili} directly.
%
Therefore, in this case, the Lipschitz property is often leveraged~\citep{Chowdhury2017-on,vakili2021-optimal}.
%
The Lipschitz constant of $f$ can be directly derived from the Assumption~\ref{assump:Bayesian_continuous}, or Assumption~\ref{assump:frequentist_continuous} and Lemma~\ref{lem:RKHS_lipschitz}~\citep{Srinivas2010-Gaussian,freitas2012exponential}.


Furthermore, we need the Lipschitz constant of $\mu_T$.
%
In the frequentist setting, the Lipschitz constant for $\mu_T$ can be derived as $\cO(L_k \sqrt{t \log t})$ by Lemma~4 in \citet{vakili2021-optimal} and Lemma~\ref{lem:RKHS_lipschitz}.
%
To obtain a slightly tighter upper bound, we show the following lemma:
% \begin{lemma}[Modified from Lemma~F.1 of \citet{vakili2022improved}]
%     Fix $\delta \in (0, 1)$ and $t \in [T]$.
%     %
%     Suppose that Assumptions~\ref{assump:frequentist} and ~\ref{assump:frequentist_continuous} hold.
%     %
%     Then, the RKHS norm of $\mu_t(\cdot)$ satisfies the following with probability at least $1 - \delta$:
%     \begin{align*}
%         \| \mu_t \|_{\cH_k} \leq B + \frac{R}{\sigma} \sqrt{ 2t \log \left( \frac{2t}{\delta} \right)}.
%     \end{align*}
%     %
%     Thus, $\mu_T$ is $L_k \bigl( B + \frac{R}{\sigma} \sqrt{ 2t \log \left( 2t / \delta \right)} \bigr)$ Lipschitz continuous.
%     \label{lem:RKHS_norm_posterior_mean}
% \end{lemma}
\begin{lemma}
    Fix $\delta \in (0, 1)$ and $t \in [T]$.
    %
    Suppose that Assumptions~\ref{assump:frequentist} and ~\ref{assump:frequentist_continuous} hold.
    %
    Then, $\mu_t(\cdot)$ is Lipschitz continuous with the Lipschitz constant,
    \begin{align*}
        L_k \left( B + \frac{R}{\sigma} \sqrt{ 2 \gamma_t + 2 \log \left( \frac{d}{\delta} \right)} \right)
    \end{align*}
    with probability at least $1 - \delta$.
    \label{lem:lipschitz_posterior_mean}
\end{lemma}
We show the proof in Appendix~\ref{sec:proof_lipschitz_posterior_mean}.
%
Since the MIG $\gamma_T$ is sublinear for the kernels on which we mainly focus, the upper bound $\cO(L_k \sqrt{\gamma_t})$ is tighter than $\cO(L_k \sqrt{t \log t})$.



In the Bayesian setting, the upper bound of the Lipschitz constant for $\mu_T$ has not been shown to our knowledge.
%
Therefore, we show the following lemma:
\begin{lemma}
    Fix $\delta \in (0, 1)$ and $t \in [T]$.
    %
    Suppose that Assumptions~\ref{assump:Bayesian} and \ref{assump:Bayesian_continuous} hold and the kernel has mixed partial derivative $\frac{\partial^2 k(\*x, \*z)}{ \partial x_j \partial z_j}$ for all $j \in [d]$.
    %
    Set $a$ and $b$ as in Lemma~\ref{assump:Bayesian_continuous}.
    %
    Assume that $(\*x_i)_{i \in [t]}$ is independent of $(\epsilon_i)_{i \in [t]}$ and $f$.
    %
    Then, $\mu_t$ and $r_t(\*x) \coloneqq f(\*x) - \mu_t(\*x)$ satisfies the following:
    \begin{align*}
        \Pr \left( \sup_{\*x \in \cX} \left| \frac{\partial \mu_t(\*u)}{\partial u_j} \Big|_{\*u = \*x} \right| > L \right) \leq 2a \exp \left( - \frac{L^2}{b^2} \right), \\
        \Pr \left( \sup_{\*x \in \cX} \left| \frac{\partial r_t(\*u)}{\partial u_j} \Big|_{\*u = \*x} \right| > L \right) \leq 2a \exp \left( - \frac{L^2}{b^2} \right), 
    \end{align*}
    for all $j \in [d]$.
    \label{lem:bayesian_lipschitz_posterior_mean}
\end{lemma}
See Appendix~\ref{sec:proof_bayesian_lipschitz_posterior_mean} for the proof, in which we leverage Slepian's inequality~\citep[Proposition~A.2.6 in][]{van1996weak} and the fact that the derivative of the sample path follows GP jointly when the kernel is differentiable.



By leveraging the above results, even if $\cX$ is continuous, we can obtain the following upper bound of Eq.~\eqref{eq:target_error}:
\begin{lemma}
    Suppose that Assumptions~\ref{assump:frequentist} and ~\ref{assump:frequentist_continuous} hold.
    %
    Fix $\delta \in (0, 1)$ and $T \in \NN$.
    %
    Then, the following holds with probability at least $1 - \delta$:
    \begin{align*}
        E_T 
        &\leq 2 \beta_{\delta, T} \max_{p \in \cP} \EE_{p(\*x^*)} \left[  \sigma_T^2(\*x^*) \right] 
        + \cO \left( \frac{\max\{\gamma_T, \log(\frac{T}{\delta})\}}{T^2} \right).
    \end{align*}
    where $\beta_{\delta, T} = \left( B + \frac{R}{\sigma} \sqrt{ 2 d \log \left( T d r + 1 \right) + 2 \log \left( \frac{4}{\delta} \right)} \right)^2$.
    \label{lem:UB_error_frequentist_continuous}
\end{lemma}
\begin{lemma}
    Suppose that Assumptions~\ref{assump:Bayesian} and \ref{assump:Bayesian_continuous} hold.
    %
    Fix $\delta \in (0, 1)$ and $T \in \NN$.
    %
    Then, the following holds with probability at least $1 - \delta$:
    \begin{align*}
        E_T 
        &\leq 2 \beta_{\delta, T} \max_{p \in \cP} \EE_{p(\*x^*)} \left[  \sigma_T^2(\*x^*) \right] 
        + \cO\left( \frac{\log(\frac{T}{\delta})}{T^2} \right),
    \end{align*}
    where $\beta_{\delta, T} = 2d \log (T d r + 1) + 2 \log (2 / \delta)$.
    \label{lem:UB_error_bayesian_continuous}
\end{lemma}
See Appendices~\ref{sec:proof_UB_error_frequentist_continuous} and ~\ref{sec:proof_UB_error_bayesian_continuous} for the proof.


Consequently, we can  minimize Eq.~\eqref{eq:target_error} by minimizing $\max_{p \in \cP} \EE_{p(\*x^{*})}\left[ \sigma^2_{T}(\*x^{*}) \right]$.
%
In this perspective, the US and RS are theoretically guaranteed because of $\max_{p \in \cP} \EE_{p(\*x^{*})}\left[ \sigma^2_{T}(\*x^{*}) \right] \leq \max_{\*x \in \cX} \sigma^2_T (\*x)$ and Proposition~\ref{prop:us_rs}.
%
However, the US and RS do not incorporate the information of $\cP$.
%
Therefore, the practical effectiveness of the US and RS is limited.
%
% Hence, we design algorithms that enjoy both a similar convergence guarantee and practical effectiveness incorporating the information of $\cP$.


\subsection{Other Performance Mesuares}

Although we mainly discuss the squared error, other measures can also be bounded from above:
\begin{lemma}
    The worst-case expected absolute error for any $T \in \NN$ is bounded from above as follows:
    \begin{align*}
        \max_{p \in \cP} \EE_{p(\*x^{*})} \left[ |f(\*x^{*}) - \mu_T(\*x^{*})| \right]
        \leq \sqrt{E_T},
        % &\coloneqq \max_{p \in \cP} \EE_{p(\*x^{*})} \left[ ( f(\*x^{*}) - \mu_T(\*x^{*}) )^2 \right]
    \end{align*}
    where $E_T$ is defined as in Eq.~\eqref{eq:target_error}.
    \label{lem:UB_absolute_error}
\end{lemma}
%
\begin{lemma}
    The worst-case expectation of entropy for any $T \in \NN$ is bounded from above as follows:
    \begin{align*}
        \max_{p \in \cP} \EE_{p(\*x^{*})} \left[ H\left[ f(\*x^*) \mid \cD_T \right] \right]
        % &= \max_{p \in \cP} \EE_{p(\*x^{*})} \left[ \frac{1}{2} \log \left(2 \pi e \sigma_T^2(\*x^*) \right) \right] \\
        &\leq \frac{1}{2} \log \left(2 \pi e \tilde{E}_T \right),
        % &\leq \frac{1}{2} \log \left(2 \pi e \max_{p \in \cP} \EE_{p(\*x^{*})} \left[ \sigma_T^2(\*x^*) \right] \right),
        % &= \cO\left( \log \left( \max_{p \in \cP} \EE_{p(\*x^{*})} \left[ \sigma_T^2(\*x^*) \right] \right)\right)
    \end{align*}
    where $\tilde{E}_T = \max_{p \in \cP} \EE_{p(\*x^{*})}\left[ \sigma^2_{T}(\*x^{*}) \right]$ and $H[f(\*x) \mid \cD_T] = \log \left(\sqrt{2 \pi e} \sigma_T(\*x) \right)$ is Shannon entropy.
    \label{lem:UB_entropy}
\end{lemma}
%
See Appendices~\ref{sec:UB_absolute_error_proof} and \ref{sec:UB_entropy_proof} for the proof.
%
Therefore, minimizing $\max_{p \in \cP} \EE_{p(\*x^{*})}\left[ \sigma^2_{T}(\*x^{*}) \right]$ also provides the convergence of the absolute error and the entropy\footnote{For the absolute error, we can design algorithms that directly reduce $\sigma_t$, not $\sigma_t^2$, and achieves the similar theoretical guarantee.}.


% \subsection{Discussion}

% Our problem setup can be seen as the generalization of the target distribution-aware AL and the AL for the worst-case error, i.e., $\max_{\*x \in \cX} ( f(\*x) - \mu_T(\*x) )^2$.
% %
% This is because our problem is equivalent to the target distribution-aware AL if we set $|\cP| = 1$ and to the worst-case error minimization if $\cP$ includes $\{p \in \cP_{\cX} \mid \exist \*x \in \cX, p(\*x) = 1 \}$, where $\cP_{\rm \cX}$ is the set of the distributions over $\cX$.
% %
% Clearly, for the worst-case analysis for $\max_{\*x \in \cX} ( f(\*x) - \mu_T(\*x) )^2$, we must use the method that reduce the largest variance $\max_{\*x \in \cX} \sigma_t^2(\*x)$.
% %
% This is satisfied by the US and RS, as shown in Proposition~\ref{prop:us_rs}.
\section{Theoretical Analysis}\label{sec:theory}
For the DP-FL framework (described in Sec.~\ref{sec:propose}), we now develop the theory that underlies our privacy accounting (Sec.~\ref{sec:theory:privacy}), and optimize the sampling rates to improve utility (Sec.~\ref{sec:theory:permutation}). 

\subsection{Privacy Accounting} \label{sec:theory:privacy}
In this section, we calculate the RDP bounds for each client $n\in [N]$ at round $t\in [T]$. The results will support the general idea of save-to-spend in our framework. Since we use RDP to perform privacy accounting, we convert each client's privacy budgets $\{\eps_n\}_{n\in [N]}$ into the RDP equivalent, denoted $\{\reps{n}\}_{n\in [N]}$, at a fixed order $\alpha$. For client $n$, we use $\reps{n}^t$ and $\rreps{n}^t$ to denote, respectively, the RDP privacy spent in round $t$ and the ``go-forward'' RDP privacy budget remaining for round $t$ onwards.

The privacy accounting for the \algasgo method first involves calculating each $\reps{n}^t$ (cf.~line~6 of Alg.~\ref{alg:tidpfl:privacy}). Next, the total RDP privacy spent during the first $t$ rounds is calculated as $\sum_{\tau = 1}^{t-1} \reps{n}^{\tau}$ (Line~7 of Alg.~\ref{alg:tidpfl:privacy}), with  budget remaining $\rreps{n}^t=\reps{n}-\sum_{\tau = 1}^{t-1} \reps{n}^{\tau}$. Recalling from Sec.~\ref{sec:proposed:alg:privacy}, client $n$ selects the noise multiplier $\noisem_n^t$ assuming that $\rreps{n}^t$ will be spent uniformly across the remaining $T-t+1$ rounds subject to using sampling rate $q$. As shown by~\cite{mironov2019r}, under this uniform assumption $\noisem_n^t$ should satisfy $\frac{\rreps{n}^t}{T-t+1}= \frac{2\alpha q^2}{\noisem_n^t}$. When $t<T_n$, the sampling rate $q_n$ reduces the RDP expenditure to $\reps{n}^t=\frac{\rreps{n}^t(q_n)^2}{(T-t+1)(q)^2}$. When $t\geq T_n$, the full allocated budget is used in round $t$. The RDP privacy spend $\reps{n}^t$ can be computed recursively as
\begin{align}\label{eq:recursive}
    \reps{n}^t = \left(\frac{\reps{n}-\sum_{\tau = 1}^{t-1} \reps{n}^{\tau}}{T-t+1}\right)\left(\1{t<T_n}\left(\frac{q_n^t}{q}\right)^2 + \1{t\geq T_n}\right).
\end{align}
Lemma~\ref{thm:recursive} solves the recursive formula~(\ref{eq:recursive}) for the RDP spent. Theorem~\ref{thm:privacyspent} shows the RDP spent is non-decreasing over time. The proofs of Lem.~\ref{thm:recursive} and Thm.~\ref{thm:privacyspent} are respectively provided in App.~\ref{app:thm:recursive} and~\ref{app:thm:privacyspent}.
\begin{lem} Given any $n\in [N], t\in [T]$, and $T_n\in [T]$, we have
    \begin{align}\label{eq:thm:recursive}
        \reps{n}^t = \begin{cases} \frac{\reps{n}}{T-t+1}\left(\frac{q_n}{q}\right)^2\prod_{i=1}^{t-1} \left(1-\frac{1}{T-t+1+i}\left(\frac{q_n}{q}\right)^2\right) & \text{if } t<T_n \\
        \frac{\reps{n}}{T-T_n+1}\prod_{i=1}^{T_n-1} \left(1-\frac{1}{T-T_n+1+i}\left(\frac{q_n}{q}\right)^2\right) & \text{ow}\end{cases}.
    \end{align}
    \label{thm:recursive}
\end{lem}

\begin{theorem}\label{thm:privacyspent}
For $(n,t,T_n)\in [N]\times[T]\times [T]$, $\reps{n}^t \geq \reps{n}^{t-1}$ if $t\leq T_n$, and $\reps{n}^t = \reps{n}^{t-1}$ if $t> T_n$.
\end{theorem}


\begin{remark}
The non-decreasing result of Thm.~\ref{thm:privacyspent} indicates that during saving rounds ($M_n^t = 0$) clients spend at least as much of their privacy budget in each round as in the previous round. In other words, saving decreases over time. During spending rounds ($M_n^t = 1$)  clients expend privacy budget at a constant rate.  If budget savings are accumulated than $\reps{n}^{T_n} > \reps{n}^{T_n-1}$ and clients will have access to a larger budget to spend.
\end{remark}


\subsection{Optimal Permutation of Saving-based Sampling Rates} \label{sec:theory:permutation}
We now optimize the selection of sampling rates. We start from a pre-defined set of $N$ sampling rates.  We then choose a permutation that assigns each of the $N$ rates to a distinct client. The permutation is selected to minimize the per-round difference between the utility achieved when DP perturbation is not applied (which generally will maximize utility), and the utility achieved when our DP-FL framework is used. In the first case, the server aggregates the unperturbed local updates $\Delta \model_n^t$ (i.e., no clipping or noise addition) for all clients $n\in [N]$ (i.e., no  sub-sampled), and updates the global model as $\model^t = \model^{t-1} + \Delta \model^t$ where 
\begin{align}
\Delta \model^t = \frac{1}{N}\sum_{n=1}^N \Delta \model_n^t. 
\end{align} 

In our DP-FL framework, local updates undergo the DP mechanism detailed in Sec.~\ref{sec:propose}. Factoring into the (modified) global update $\tilde{\Delta}\model^t$ the clipping norms $c_n^t$, the additive noise multipliers $\sigma_n^t$, and the sampling rates $q_n^t$, we get
\begin{align}
\tilde{\Delta} \model^t = \frac{1}{Nq^t}\sum_{n=1}^N \left(\rb_n^t \left(\clip{{\Delta \model_n^t}}{c_n^t}+\rz_n^t\right)+ (1-\rb_n^t)\tilde{\rz}_n^t\right),
\end{align} 
where $q^t=\frac{1}{N}\sum_{n=1}^N q_n^t$, $\rz_n^t,\tilde{\rz}_n^t \sim \noise{0}{\frac{\left(\sigma_n^tc_n^t\right)^2}{N}}$ are identically and independently distributed (IID), and $\rb_n^t\sim \bern{q_n^t}$ are also IID. The optimization problem is to choose the $\{q_n^t\}_{n\in [N]}$ so that $\tilde{\Delta} \model^t$ closely approximates an unbiased estimate of $\Delta \model^t$. We define the difference between the respective model updates as
\begin{align}
\error^t:= \Delta \model^t - \tilde{\Delta} \model^t.
\end{align}

$\error^t$ has four sources of randomness: 
\begin{itemize}
\item[(i)] \textbf{Local dataset randomness}: the randomness of local datasets $\train_n$, which are sampled from distributions $P_n$.  This randomness is reflected in the local updates $\Delta \model_n^t$.
\item[(ii)] \textbf{Client sampling randomness:} the sampling of clients, represented by the use of random variables $\rb_n^t$.  These determine whether a client $n$ contributes to the round $t$'s local training. 
\item[(iii)] \textbf{Noise addition randomness:} the addition of the Gaussian noises $\rz_n^t$ and  $\tilde{\rz}_n^t$.
\item[(iv)] \textbf{Privacy budget assignment randomness:}  the matching of clients with different datasets $\train_n$ and data distributions $P_n$ to different privacy budgets $\eps_{n}$. Here, $\{\eps_{n}\}_{n\in [N]}$ are considered as a random permutation of a predefined set of privacy budgets $\{\hat{\eps}_n\}_{n\in [N]}$.
\end{itemize}
 
To optimize the sampling rates, we first develop two upper bounds on the bias term $\left\| \expect{\error^t}\right\|$. Both bounds build on the clipping bias lemma~\citep{das2023beyond}, cf., Lem.~\ref{app:prior:clipbias} in App.~\ref{app:prior:lemma}. Our theorems extend the results of that lemma to the situation where clients have individualized privacy budgets and use time-varying clip norms $c_n^t$ and sampling rates $q_n^t$.

Our first theorem, Thm.~\ref{THM:clip_aware1}, bounds the expected bias with respect to (w.r.t.) three sources of randomness: (i), (ii), and (iii). We denote this expectation as $\left\| \expectation{(i), (ii), (iii)}{\error^t}\right\|$.  The proof of Thm.~\ref{THM:clip_aware1} is given in App.~\ref{app:THM:clip_aware1}. 
 
\begin{theorem}\label{THM:clip_aware1} Taking the expectation w.r.t. (i), (ii), and (iii), and for any $\rho>1$, we have
\begin{align}\label{eq:THM:clip_aware1:1}
\left\| \expectation{(i), (ii), (iii)}{\error^t}\right\| \leq \frac{1}{N} \left\| \sum_{n=1}^N \left(1 - \frac{q_n^t}{q^t}\right)\expectation{(i)}{{\Delta \model_n^t}}\right\| + \frac{1}{N}\sum_{n=1}^N \frac{q_n^t}{q^t}  \frac{\expectation{(i)}{\left\|{\Delta \model_n^t}\right\|^{\rho}}}{\left(c_n^t\right)^{\rho-1}}.
\end{align}
\end{theorem}

To minimize this bias term in a reasoned fashion we select the $q_n^t$ to minimize the upper bound. The upper bound in (\ref{eq:THM:clip_aware1:1}) contains terms that couple $q_n^t$ with the pure local updates $\Delta \model_n^t$.  The latter are not accessible to the server who is responsible for sampling clients. If clients were to select their own $q_n^t$ based on their local updates, this could lead to privacy leakage and would require additional privacy protection. The coupling between $q_n^t$ and $\Delta \model_n^t$ can be removed from the first term of the upper bound in (\ref{eq:THM:clip_aware1:1}) under certain conditions. For example, if all clients are sampled at the same rate ($q_n^t=q^t$) or if $\expectation{(i)}{\Delta \model_n^t}$ is equal across all $n\in [N]$, the first term becomes zero. In such cases, the upper bound reduces to the second term, which still couples  $q_n^t$ with $\Delta \model_n^t$ and the clip norms $c_n^t$. 

In contrast to Thm.~\ref{THM:clip_aware1}, in Thm.~\ref{THM:clip_aware2} we take the expectation w.r.t.~the additional source of randomness, (iv).  This results in a bound that depends solely on clipping norms, which makes optimizing the sampling rates, $q_n^t$, easier to accomplish.  As we now bound the expected bias w.r.t. all four sources of randomness -- (i), (ii), (iii), and (iv) -- we denote the expectation as $\left\| \expectation{(i), (ii), (iii), (iv)}{\error^t}\right\|$. The proof of Thm.~\ref{THM:clip_aware2} is given in App.~\ref{app:THM:clip_aware2}.
 
\begin{theorem}\label{THM:clip_aware2} Taking the expectation w.r.t. (i), (ii), (iii), and (iv), and for any $\rho>1$, we have:
\begin{align}
\left\| \expectation{(i), (ii), (iii), (iv)}{\error^t}\right\| \leq \frac{1}{N^2}\left(\sum_{n=1}^N\expectation{(i)}{\left\|{\Delta \model_n^t}\right\|^{\rho}}\right)\sum_{n=1}^N \left( \frac{q_n^t}{q^t}  \frac{1}{\left(c_n^t\right)^{\rho-1}}\right).
\end{align}
\end{theorem}


The main step in the proofs of Thm.~\ref{THM:clip_aware2} builds on the common assumption in FL that the sampling from $P_n$ and of $\eps_n$ is independent. This assumption is made without significant loss of generality as privacy budgets are often assigned based on clients' personal preferences and policy requirements.  In contrast, the data distribution is influenced by external factors such as geographical locations. Such decoupling of privacy budgets and data distributions simplifies the proof of Thm.~\ref{THM:clip_aware2}.  It avoids the need to model potential correlations between the client's data and privacy preferences.  These are often unknown or irrelevant in practice. 

As per Thm.~\ref{THM:clip_aware2}, to minimize $\left\| \expectation{(i), (ii), (iii), (iv)}{\error^t}\right\|$ w.r.t. the sampling rates $q_n^t$, we solve the following optimization problem:

\begin{equation}\label{eq:way3:opt}
\begin{aligned}
\min_{\{\Pi_t\}_{t=1}^T} \quad &
\sum_{n=1}^N \frac{q_n^t}{q^t}  \frac{1}{\left(c_n^t\right)^{\rho-1}}
\\
\textrm{s.t.} \quad & q_n^t = q_{\Pi_t^{-1}(n)}, n \in [N]
\end{aligned},
\end{equation}
where the $\{\Pi_t\}$ are a set of (bijective) permutation maps $\Pi_t: [N] \rightarrow [N]$.  Each permutation maps each element of the index set $[N]$ to a distinct element of $[N]$.  We apply the permutation to the indices of the fixed set of sampling rates $\{q_1,\ldots,q_N\}$ to get $\{q_1^t,\ldots,q_N^t\}$. The set $\{q_1,\ldots,q_N\}$ is a hyperparameter in our problem, which is constrained by the condition $q_n\leq q$ for every $n\in [N]$. The optimized choice of the set of sampling rates $\{q_1,\ldots,q_N\}$ is reserved for future work. Per (\ref{eq:way3:opt}), the optimal choice for $\{q_1^t,\ldots,q_N^t\}$ is to assign clients with smaller clip norms $c_n^t$ (which will contribute highly perturbed model updates) to have lower sampling rates $q_n^t$, and vice versa. The intuition is that by matching the $q_n^t$ to the $c_n^t$, we prevent clients who contribute a highly perturbed model update from deteriorating other clients' performance during saving rounds. This reduces clipping bias and allows these clients to preserve more of their privacy budget compared, with the subsequent benefit of enabling them to contribute more in future rounds.
 

\section{Experiments}\label{sec:simulation}
\paragraph{Datasets.}
To empirically evaluate the performance of our framework against baselines, we consider widely used datasets. For the Fashion MNIST (FMNIST) and MNIST, we use a convolutional neural network (CNN) architecture from~\citep{mcmahan2017communication}. For the Adult Income dataset, we use multi-layer perception from~\citep{9378043}. %\tcb
{For the CIFAR10 dataset, we use a CNN architecture from ~\cite{he2016deep}}. We partition datasets across 100 clients in a non-IID manner using the Dirichlet distribution with a default parameter 0.1~\citep{zhang2023fedala}. 

\paragraph{Privacy Settings.} To reduce the number of choices for clients' privacy budgets, and motivated by society wherein individuals often share similar privacy preferences~\citep{alaggan2015heterogeneous, boenisch2024have}, in our experiments we divide clients into three groups. The groups are respectively assigned budgets of $\eps_{\text{group}, 1}$, $\eps_{\text{group}, 2}$, or $\eps_{\text{group}, 3}$. We randomly allocate $34\%$ of clients to belong to Group 1, $43\%$ to Group 2, and $23\%$ to Group 3. In other words, Client $n$ in Group $m$, shares $\eps_n=\eps_{\text{group},m}$. To account for privacy consumption, we use the Opacus library~\citep{opacus}. We consider DP parameter $\delta=10^{-5}$ and choose an extended version of the default RDP parameter $\alpha$ from the RDPAccountant function in Opacus. We apply per-layer clipping~\citep{mcmahan2017learning} to restrict the influence of individual layers by constraining their norms. 


\paragraph{Baselines.} As discussed in Sec.~\ref{sec:back}, we consider several baselines. %\tcb
{The first is the non-private FedAvg~\citep{mcmahan2017communication} which represents our upper bound on utility without any privacy constraints.} The second is DP-FedAvg~\citep{mcmahan2017learning} where we use a uniform privacy budget, chosen according to the smallest epsilon value of any of the clients. This ensures that no client's privacy budget is exceeded. The third is IDP-FedAvg which is the IDP-integration ~\citep{boenisch2024have} of DP-FedAvg. %\tcb
{The fourth is adaptive clipping~\citep{andrew2021differentially}}. To have a fair comparison with the baselines, we evaluate two variants of our framework. The first variant assumes every client's budget is constrained by the smallest value in the group budget tuple $(\eps_{\text{group},1}, \eps_{\text{group},2}, \eps_{\text{group},3})$. The second variant incorporates different privacy groups. Further details on the choice of hyperparameters in our experiments are given in Table \ref{tab:hyperparamstable} in  \Cref{app:extendedExpSetup}. 

\begin{figure}[h]
    \centering
    \subfloat[Test Accuracy FMNIST]{
    \includegraphics[width=0.38\textwidth]{images/pdfs/p_4_fmnist_global_test_acc_epsilons_10_20_30_clients_100_sigma_0.30_clip_250.00_srate_0.90_runs_2_rounds_25_epochs_30final_aid.txt.pdf}}
    \subfloat[Test Accuracy MNIST]{
    \includegraphics[width=0.38\textwidth]{images/pdfs/p_4_mnist_global_test_acc_epsilons_10_15_20_clients_100_sigma_0.30_clip_250.00_srate_0.90_runs_2_rounds_25_epochs_30final_aid.txt.pdf}}
    \caption{\textbf{Our framework improves accuracy in later rounds compared to the baseline.} We plot the global test accuracy vs. rounds for (a) the FMNIST dataset, and (b) the MNIST dataset. In (a), $(\eps_{\text{group},1}, \eps_{\text{group},2}, \eps_{\text{group},3})=(10,20,30)$, and in (b) it equals $(10,15,20)$.} 
    \label{fig:results}
    \centering \vspace{-1ex}
\end{figure}


\begin{table}[h!]
\centering
\caption{\textbf{Global test accuracy} for 
{FedAvg without DP constraints,} DP-FedAvg with $\eps_n=10$, IDP-FedAvg with non-uniform privacy budgets, our framework with $\eps_n=10$, and our framework with non-uniform budgets. For the FMNIST and Adult Income datasets, the non-uniform privacy budgets $\eps_n$ are $(\eps_{\text{group},1},\eps_{\text{group},2},\eps_{\text{group},3})=(10,20,30)$, and for MNIST, they are $(10,15,20)$.}
\small
\setlength{\tabcolsep}{10pt} 
\renewcommand{\arraystretch}{1.2} 
\begin{tabular}{cccccc}
\toprule
\textbf{DATASET} & \makecell[tl]{
{\small \textbf{FedAvg}} \\ {\small (non-DP)}}  &  \makecell[tl]{{\small\textbf{DP-FedAvg}}\\ {\small $\eps_n=10$ }}  & \makecell[tl]{{\small \textbf{IDP-FedAvg}} \\ {\small Non-uniform}}& \makecell[tl]{{\small \textbf{Ours}} \\ {\small $\eps_n=10$ } } & \makecell[tl]{{\small \textbf{Ours}}\\ {\small {Non-uniform}}}  \\ 
\midrule  
FMNIST& 72.95 & 64.8 & 65.45 & \textbf{67.90} & \textbf{70.57}  \\  
\midrule 
MNIST & 90.23 &  76.79 & 76.94  & \textbf{80.2} & \textbf{83.83} \\
\midrule 
Adult Income & 78.93 &  60.12 &  70.93  & \textbf{72.14}  & \textbf{77.53}\\ 
\midrule
\end{tabular}
\label{tab:cmptable}
\centering \vspace{-1ex}
\end{table}

\begin{figure*}[!htbp]
    \centering
    \vspace{5pt}  
        \centering
        \includegraphics[width=0.4\textwidth]{images/pdfs/privacy_curves.pdf}
    \caption{\textbf{While both adhere to privacy budgets, our framework follows spend-as-you-go, whereas IDP-FedAvg uses uniform privacy spending.} The blue solid curves correspond to clients' privacy spending in our framework, while the red dashed curves show IDP-FedAvg. The curves of clients with budgets of $30$, $20$, and $10$ are marked with rectangles, circles, and squares, respectively.  
    }    
    \label{fig:privacyparams}
    \centering \vspace{-2ex}
\end{figure*}

\paragraph{Experimental Results.} As shown in Table~\ref{tab:cmptable} and Fig.~\ref{fig:results}, our framework yields improvements in the resulting global model's accuracy by spending privacy budget non-uniformly across training rounds. Comparing Columns 4 and 6 of Table~\ref{tab:cmptable}, our framework with non-uniform privacy budgets improves global test accuracy over IDP-FedAve by $7.8\%$, $8.9\%$, and $9.3\%$ on FMNIST, MNIST, and Adult Income. In the case of using a uniform budget of $\eps_n=10$ across all clients $n$, our framework achieves respective improvements of 4.7\%, 4.4\%, and 19.9\% compared to DP-FedAvg, as shown in Columns 3 and 5. {We also observe that, our time-adaptive DP-FL scheme comes closest to the ideal-case performance of FedAvg (column 2) without privacy constraints.} Figure \ref{fig:results} plots global test accuracy vs. global rounds for our framework with non-uniform privacy budgets and IDP-FedAvg, on the FMNIST and MNIST datasets. This figure shows that while our framework conserves privacy in early rounds, it allocates more budget in later rounds, eventually catching up to and surpassing IDP-FedAvg by about $8\%$ on FMNIST and $6\%$ on MNIST in the final round. 

In Fig.~\ref{fig:privacyparams} we present the privacy budget spent by clients from different budget groups $(10,20,30)$ across rounds. This figure shows that,  while IDP-FedAvg enforces uniform privacy consumption over time, in our framework, clients follow spend-as-you-go, saving budgets in the first half of training, and spending more in later rounds. Our experimental results demonstrate that our time-adaptive approach boosts the utility of the trained model while adhering to privacy constraints. 

{In \Cref{app:extendedresults}, we present extended experimental results, including benchmarks on the CIFAR10 dataset (Table~\ref{tab:cifar10}), comparisons with adaptive clipping (Table~\ref{tab:strictprivacybudgetstable2}), and evaluations across privacy-related hyperparameters (Tables~\ref{tab:strictprivacybudgetstable},~\ref{tab:saving_sampling_ratestable}, and ~\ref{tab:privacy_spending_roundtable}, and Figure~\ref{fig:lowprivacybudgets}), as well as other parameters (Tables~\ref{tab:clientstable_rounds} and~\ref{tab:clientstable} and Figure~\ref{fig:morerounds}).} 



\section{Discussions and Future Work}\label{ch6:future}
We now discuss some limitations of our work that represent interesting directions for future work. Our \algasgo method reduces reliance on determining when clients should transition from saving to spending by allowing them to gradually spend their saved budgets over time, rather than waiting until a specific round to start spending. While our experiments indicate that transitioning from saving to spending midway through training generally yields good results, tuning the hyperparameters involved in estimating the transitioning round may improve utility. However, such hyperparameter tuning can lead to additional privacy loss~\citep{papernot2021hyperparameter} that would need to be accounted for. For future work, we believe that our time-adaptive DP-FL framework should be closely integrated with a form of privacy-preserving hyperparameter tuning to identify the best rounds in which to transition from savings to spending.

Furthermore, as we demonstrated theoretically and validated experimentally, adapting saving-related hyperparameters to clients' specific privacy budgets can enhance utility. To eliminate the risk of privacy leakage from this adaptation, we provide theoretical optimizations that rely solely on clients' privacy-related constraints, independent of their data. Future research can explore data-and-privacy joint measures to quantify clients' contributions with controlled privacy leakage and adapt client-specific savings decisions accordingly.













\subsubsection*{Acknowledgments}
We would like to acknowledge our sponsors. This work was supported in part by a Discovery Research Grant from the Natural Sciences and Engineering Research Council of Canada (NSERC), by an NSERC Alexander Graham Bell Canada Graduate Scholarship-Doctoral (CGS D3), by a DiDi graduate award, and by the Mitacs Globalink research award.

\bibliography{references}
\bibliographystyle{iclr2024_conference}
\newpage
\appendix
\section{Appendix}

\subsection{
{Summary of Notations} and  Benchmarking Schemes}\label{app:benchmark}
{We summarize the important notations (including the hyperparameters shown in Table~\ref{sec3:notations}) in Table~\ref{app:summary_table}. We formally depict the baselines: FedAvg~\citep{mcmahan2017communication} as Alg.~\ref{alg:fedavg}, DP-FedAvg as Alg.~\ref{alg:dpfl}, IDP-FedAvg as Alg.~\ref{alg:idpfl}, and adaptive clipping~\citep{andrew2021differentially} as Alg.~\ref{alg:quantile}. Next, we provide further details to explain the adaptive clipping baseline in comparison with our proposed approach.}


\begin{table*}[ht]
\begin{threeparttable}
\caption{
{A Summary of Notation and Hyperparameters\tnote{1}.}}
\centering\label{app:summary_table}
\begin{tabular}{|c|l||c|c|}
\hline
$N$ & No. of Clients & $\train_n$ & Dataset of Client $n$ 
\\ \hline
$\gC^t$ & Client set in R. $t$ & $\batch_i$ & Batch $i$ 
\\ \hline
$T$ & No. of rounds & $B$ & Batch size 
\\ \hline
$L$ & No. of local iterations & $\eps_n$ & DP privacy budget of Client $n$
\\ \hline
$\model^t$ & Global model at R. $t$ & $\reps{n}^t$ & RDP privacy spent of Client $n$ in R. $t$ \\ \hline
$\Delta\model^t$ & Global model update at R. $t$& $\rreps{n}^t$ & RDP budget RE. of Client $n$ for R. $t$ onwards \\ \hline
$\model_n^{t,l}$ & Model of Client $n$ at R. $t$, I. $l$ & $\bar{\eps}_{\text{rdp},n}^t$ & RDP privacy spend of Client $n$ up to R. $t+1$ \\ \hline
$\Delta\model_n^t$ & Model update of Client $n$ at R. $t$& $\eps_n^t$ & DP budget RE. of Client $n$ for R. $t$ onwards \\ \hline
$\tilde{\Delta}\model_n^t$ & Perturbed update of Client $n$ at R. $t$& $\noisem^t$ & Global noise multiplier in R. $t$ \\ \hline
$\text{Error}_t$ & $\Delta \model^t - \tilde{\Delta} \model^t$ & $\noisem_n^t$ & Noise multiplier of Client $n$ in R. $t$ \\ \hline
$\lr$ & Learning rate  & $c$ & Average clipping norm \\ \hline
$\alpha$ & R\'{e}nyi order in RDP & $c_n^t$ & Clipping norm of Client $n$ in R. $t$ \\ \hline
$\delta$ & Probability of violating in DP  & $q$ & Spending-based sampling rate \\ \hline
$T_n$ & Saving-to-spending transition R.  & $q_n$ & Saving-based sampling rate of Client $n$ \\ \hline
$M_n^t$ & Saving-or-spending mode  & $q_n^t$ & Sampling rate of Client $n$ in R. $t$ \\ \hline
$\loss_n$ & Loss function of Client $n$  & $q^t$ & Average sampling rate in R. $t$ \\ \hline
\end{tabular}
 \begin{tablenotes}
	\item[1] Table's abbreviations: ``No.'' for ``Number'', ``RE.'' for ``Remaining'', ``I.'' for ``Iteration'', and ``R.'' for ``Round''.     
   \end{tablenotes}
\end{threeparttable}
\end{table*}

\begin{minipage}{0.5\textwidth} 
  \raggedright  % Left-align to avoid justification causing overfull hbox
\begin{algorithm}[H]
  \scriptsize
\caption{Federated Averaging (FedAvg)~\citep{mcmahan2017communication}}
\textbf{Inputs:} No. clients $N$, No. global rounds $T$, No. local iterations $L$, loss functions $\loss_n$, local datasets $\train_n$, learning rate $\lr$, batch size $\bs$  \\
\label{alg:fedavg}
\begin{algorithmic}[1]
\State \textbf{Initialize} global model $\model^0$
\For{each global round $t \in [T]$} 
\State $\gC^t \gets$ Sample clients with probability $q$.
\For{each client $n\in \gC^t$ in parallel}
\State ${\Delta}\model_n^t$ =  
\texttt{ClientUpdate}$\left(t, n,\model^{t-1}\right)$.
\EndFor
\State Aggregate ${\Delta}\model^t = \sum_{n\in \gC^t} \tilde{\Delta}\model_n^t$
\State Update $\model^{t} = \model^{t-1} + \frac{{\Delta}\model^t}{qN}$
\EndFor
\Statex
\end{algorithmic}
\textbf{Def} \texttt{ClientUpdate}$\left(t, n,\model^{t}\right)$
\begin{algorithmic}[1]
\State \textbf{Initialize} local model $\model_n^{t,0}=\model^t$
\For{local iteration $l \in [l]$}
\State $\{\batch_i\}_{i=1}^{|\train_n|/B} \gets$ Split $\train_n$ to size $B$ batches
\For{each batch $\batch_i$}
\State $\model_n^{t,l} = \model_n^{t,l-1}-\frac{\lr \sum_{(\rvx,y)\in \batch_i} \nabla\loss_n\left(\model_n^{t,l-1};(\rvx,y)\right)}{\bs}$
\EndFor
\EndFor
\State Return $\Delta\model_n^t = \model_n^{t,L} - \model_n^{t,0}$
\end{algorithmic}
\end{algorithm}

  \end{minipage}%
\hspace{1em} 
\begin{minipage}{0.5\textwidth}  %
  \raggedright  % Left-align
\begin{algorithm}[H]
  \scriptsize
\caption{Differential Private Federated Averaging (DP-FedAvg)~\citep{mcmahan2017learning}}
\textbf{Inputs:} No. clients $N$, No. global rounds $T$, No. local iterations $L$, noise multiplier $\noisem$, clip norm $c$, sampling rate $q$, loss functions $\loss_n$, local datasets $\train_n$, learning rate $\lr$, batch size $\bs$  \\
\label{alg:dpfl}
\begin{algorithmic}[1]
\State \textbf{Initialize} global model $\model^0$
\For{each global round $t \in [T]$} 
\State $\gC^t \gets$ Sample clients with probability $q$.
\For{each client $n\in \gC^t$ in parallel}
\State $\tilde{\Delta}\model_n^t$ =  
\texttt{ClientUpdate}$\left(t, n,\model^{t-1}, c\right)$.
\EndFor
\State Aggregate $\tilde{\Delta}\model^t = \sum_{n\in \gC^t} \tilde{\Delta}\model_n^t$
\State Add noise $\tilde{\Delta}\model^t \gets \tilde{\Delta}\model^t + \gN(0,c^2\noisem^2\sI)$
\State Update $\model^{t} = \model^{t-1} + \frac{\tilde{\Delta}\model^t}{qN}$
\EndFor
\Statex
\end{algorithmic}
\textbf{Def} \texttt{ClientUpdate}$\left(t, n,\model^{t}, c\right)$
\begin{algorithmic}[1]
\State \textbf{Initialize} local model $\model_n^{t,0}=\model^t$
\For{local iteration $l \in [l]$}
\State $\{\batch_i\}_{i=1}^{|\train_n|/B} \gets$ Split $\train_n$ to size $B$ batches
\For{each batch $\batch_i$}
\State $\model_n^{t,l} = \model_n^{t,l-1}-\frac{\lr \sum_{(\rvx,y)\in \batch_i} \nabla\loss_n\left(\model_n^{t,l-1};(\rvx,y)\right)}{\bs}$
\EndFor
\EndFor
\State Compute $\Delta\model_n^t = \model_n^{t,L} - \model_n^{t,0}$
\State Clip $\tilde{\Delta}\model_n^{t} = \Delta\model_n^t\min\left(1,\frac{c}{\left\|\Delta\model_n^{t}\right\|_2}\right)$  
\State Return $\tilde{\Delta}\model_n^{t}$ 
\end{algorithmic}
\end{algorithm}

  \end{minipage}


{\textbf{The Adapting Clipping Baseline}. In this paper, we consider the adaptive clipping method~\citep{andrew2021differentially} as a baseline for our time-adaptive DP-FL approach. In the extended simulations (cf. App.~\ref{app:extendedresults}), we benchmark that method against our approach. The method is formally presented as Alg.~\ref{alg:quantile}. As shown in Line 13, the server dynamically adjusts the clipping norm based on a specified quantile $\gamma$ of the distribution of clients’ updates. The goal of this method is to minimize the difference between the clipping norm and the quantile in the distribution, aiming to achieve the same objective as ours: improving the privacy-utility tradeoff. In contrast to our time-adaptive approach, which is independent of the client's data and can be done prior to training, the method~\citep{andrew2021differentially} introduces privacy risks during the quantile approximation. To mitigate these risks, and as is shown in Line 2 of the \texttt{SetClipping} function in Alg.~\ref{alg:quantile}, the method~\citep{andrew2021differentially} incorporates a supplementary DP mechanism that allocates part of the privacy budget to preserve privacy during quantile estimation. However, this results in a lower remaining privacy budget, requiring a larger noise multiplier $\sigma$, as computed in Line 2 of \texttt{SetSigma} in Alg.\ref{alg:quantile}, in comparison to our approach. }


\begin{minipage}{0.5\textwidth} 
  \raggedright  % Left-align to avoid justification causing overfull hbox
\begin{algorithm}[H]
  \scriptsize
\caption{Individualized DP-FedAvg (IDP-FedAvg), a natural integration of IDP~\citep{boenisch2024have} to FL}
\textbf{Inputs:} No. clients $N$, No. global rounds $T$, No. local iterations $L$, local privacy budgets $\eps_n$, average clip norm $c$, sampling rate $q$, loss functions $\loss_n$, local datasets $\train_n$, learning rate $\lr$, batch size $\bs$, probability of violating $\delta$ \\
%\State \textbf{Results:} Final global model $\model^T$
\label{alg:idpfl}
\begin{algorithmic}[1]
\State $\noisem, \{c_n\}_{n\in [N]}=$\texttt{SetPrivacyParams}$\left(c, q, \{\eps_n\}_{n\in [N]}, T, \delta\right)$
\State \textbf{Initialize} global model $\model^0$
\For{each global round $t \in [T]$} 
\State $\gC^t \gets$ Sample clients with probability $q$.
\For{each client $n\in \gC^t$ in parallel}
\State $\tilde{\Delta}\model_n^t$ =  
\texttt{ClientUpdate}$\left(t, n,\model^{t-1}, c_n\right)$.
\EndFor
\State Aggregate $\tilde{\Delta}\model^t = \sum_{n\in \gC^t} \tilde{\Delta}\model_n^t$
\State Add noise $\tilde{\Delta}\model^t \gets \tilde{\Delta}\model^t + \gN(0,c^2\noisem^2\sI)$
\State Update $\model^{t} = \model^{t-1} + \frac{\tilde{\Delta}\model^t}{qN}$
\EndFor
\Statex
\end{algorithmic}
\textbf{Def} \texttt{ClientUpdate}$\left(t, n,\model^{t}, c_n\right)$
\begin{algorithmic}[1]
\State \textbf{Initialize} local model $\model_n^{t,0}=\model^t$
\For{local iteration $l \in [l]$}
\State $\{\batch_i\}_{i=1}^{|\train_n|/B} \gets$ Split $\train_n$ to size $B$ batches
\For{each batch $\batch_i$}
\State $\model_n^{t,l} = \model_n^{t,l-1}-\frac{\lr \sum_{(\rvx,y)\in \batch_i} \nabla\loss_n\left(\model_n^{t,l-1};(\rvx,y)\right)}{\bs}$
\EndFor
\EndFor
\State Compute $\Delta\model_n^t = \model_n^{t,L} - \model_n^{t,0}$
\State Clip $\tilde{\Delta}\model_n^{t} = \Delta\model_n^t\min\left(1,\frac{c_n}{\left\|\Delta\model_n^{t}\right\|_2}\right)$  
\State Return $\tilde{\Delta}\model_n^{t}$ 
\Statex
\end{algorithmic}
\textbf{Def} \texttt{SetPrivacyParams}$\left(c, q, \{\eps_n\}_{n\in [N]}, T, \delta\right)$
\begin{algorithmic}[1]
\For{each client $n \in [N]$}
\State Set local noise multiplier $\noisem_n=$\texttt{GetNoise}$\left(\eps_n,\delta, q, T\right)$
\EndFor
\State Compute $\noisem \gets \left(\frac{1}{N}\sum_{n\in [N]}\frac{1}{\noisem_n}\right)^{-1}$
\For{each client $n \in [N]$}
\State Set local clip norm $c_n=\frac{c\noisem}{\noisem_n}$
\EndFor
\State Return $\noisem, \{c_n\}_{n\in [N]}$
\end{algorithmic}
\end{algorithm}
\end{minipage}%
\hspace{1em} 
\begin{minipage}{0.5\textwidth}  %
  \raggedright  % Left-align


\begin{algorithm}[H]
   \scriptsize
\caption{DP-FedAvg-M with Adaptive Clipping~\citep{andrew2021differentially}}
\textbf{Inputs:} No. clients $N$, No. global rounds $T$, No. local iterations $L$, noise multiplier $\noisem$, clip norm $c$, sampling rate $q$, loss functions $\{\loss_n\}_{n\in [N]}$, local datasets $\{\train_n\}_{n\in [N]}$, client-side learning rate $\lr$, server-side learning rate $\lr_\text{s}$, clip-related learning rate $\lr_\text{b}$, $\gamma$ quantile, batch size $\bs$, probability of violating $\delta$,
\\
\label{alg:quantile}
\begin{algorithmic}[1]
\State $\noisem=$\texttt{SetSigma}$\left(q, \eps, T, \delta, \noisem_{\text{b}} \right)$
\State \textbf{Initialize} global model $\model^0$
\For{each global round $t \in [T]$} 
\State $\gC^t \gets$ Sample $qN$ clients uniformly.
\For{each client $n\in \gC^t$ in parallel}
\State $(b_n^t,\tilde{\Delta}\model_n^t)$ =  
\texttt{ClientUpdate}$\left(t, n,\model^{t-1}, c^t\right)$.
\EndFor
\State Aggregate $\tilde{\Delta}\model^t = \sum_{n\in \gC^t} \tilde{\Delta}\model_n^t$
\State Add noise $\tilde{\Delta}\model^t \gets \tilde{\Delta}\model^t + \gN(0,(c^t)^2\noisem^2\sI)$
\State Average $\tilde{\Delta}\model^t \gets \frac{1}{qN} \tilde{\Delta}\model^t$
\State Compute $\tilde{\Delta}\model^t \gets \beta_{\text{s}} \tilde{\Delta}\model^{t-1} + (1-\beta_{\text{s}})\tilde{\Delta}\model^t$
\State Update $\model^{t} = \model^{t-1} + \lr_{\text{s}}\tilde{\Delta}\model^t$
\State $c^{t+1}$=\texttt{SetClippig}$\left(\{b_n^t\}_{n\in \gC^t}, \noisem_{\text{b}},q, \gamma, \lr_{\text{b}}, c^t\right)$
\EndFor
\Statex
\end{algorithmic}
\textbf{Def} \texttt{ClientUpdate}$\left(t, n,\model^{t}, c^t\right)$
\begin{algorithmic}[1]
\State \textbf{Initialize} local model $\model_n^{t,0}=\model^t$
\For{local iteration $l \in [l]$}
\State $\{\batch_i\}_{i=1}^{|\train_n|/B} \gets$ Split $\train_n$ to size $B$ batches
\For{each batch $\batch_i$}
\State $\model_n^{t,l} = \model_n^{t,l-1}-\frac{\lr \sum_{(\rvx,y)\in \batch_i} \nabla\loss_n\left(\model_n^{t,l-1};(\rvx,y)\right)}{\bs}$
\EndFor
\EndFor
\State Compute $\Delta\model_n^t = \model_n^{t,L} - \model_n^{t,0}$
\State Compute $b = \gI_{\left\|\Delta\model_n^t \right\|\leq c^t}$
\State Clip $\tilde{\Delta}\model_n^{t} = \Delta\model_n^t\min\left(1,\frac{c^t}{\left\|\Delta\model_n^{t}\right\|_2}\right)$  
\State Return $b, \tilde{\Delta}\model_n^{t}$ 
\Statex
\end{algorithmic}
\textbf{Def} \texttt{SetSigma}$\left(q, \eps, T, \delta, \noisem_{\text{b}} \right)$
\begin{algorithmic}[1]
\State $\bar{\noisem}=$\texttt{GetNoise}$\left(\eps,\delta, q, T\right)$
\State $\noisem = \left(\frac{1}{\bar{\noisem}^{2}} - \frac{1}{(2\noisem_{\text{b}})^2}\right)^{-1/2}$
\State Return $\noisem$
\Statex
\end{algorithmic}
\textbf{Def} \texttt{SetClippig}$\left(\{b_n^t\}_{n\in \gC^t}, \noisem_{\text{b}},q, \gamma, \lr_{\text{b}}, c^t\right)$
\begin{algorithmic}[1]
\State Aggregate $\tilde{b}^t = \sum_{n\in \gC^t} b_n^t$
\State Add noise $\tilde{b}^t \gets \tilde{b}^t + \gN(0,\noisem_{\text{b}}^2\sI)$ 
\State Average $\tilde{b}^t \gets \frac{1}{qN}\tilde{b}^t $
\State Update $c^{t+1}=c^t \exp \left(-\lr_{\text{b}}
(\tilde{b}^t-\gamma)\right)$
\State Return $c^{t+1}$
\end{algorithmic}
\end{algorithm}

  \end{minipage}

\subsection{Proof of Lemma~\ref{thm:recursive}}\label{app:thm:recursive}
We use induction to solve the recursive formula (\ref{eq:recursive}). According to (\ref{eq:recursive}), when $t=1<T_n$, $\reps{n}^1=\frac{\reps{n}(q_n)^2}{T(q)^2}$, and when $t=2<T_n$, client $n$ spends $\reps{n}^2 = \frac{\reps{n}-\reps{n}^1}{T-1} \left(\frac{q_n}{q}\right)^2$. By substituting $\reps{n}^1$ in $\reps{n}^2$, we obtain $\reps{n}^2 = \frac{\reps{n}}{T-1}\left(1-\frac{1}{T}\left(\frac{q_n}{q}\right)^2\right)\left(\frac{q_n}{q}\right)^2$. We now assume $\reps{n}^{t-1}$ satisfies in (\ref{eq:thm:recursive}) for every $2 \leq t<T$. If $t< T_n$, by substituting $\reps{n}^{t-1}$ in (\ref{eq:recursive}), we obtain
\begin{align}
    \reps{n}^t &= \left(\frac{\reps{n}-\sum_{\tau = 1}^{t-1} \reps{n}^{\tau}}{T-t+1}\right)\left(\frac{q_n}{q}\right)^2
    = \left(\frac{\reps{n}^{t-1}(T-t+2)\left(\frac{q}{q_n}\right)^2- \reps{n}^{t-1}}{T-t+1}\right)\left(\frac{q_n}{q}\right)^2
    \\
    &= \reps{n}^{t-1}\frac{\left(T-t+2-\left(\frac{q_n}{q}\right)^2\right)}{T-t+1}
    \\
    &= 
    \frac{\reps{n}}{T-t+2}\left(\frac{q_n}{q}\right)^2\left(\prod_{i=1}^{t-2} \left(1-\frac{1}{T-t+2+i}\left(\frac{q_n}{q}\right)^2\right)\right)\frac{\left(T-t+2-\left(\frac{q_n}{q}\right)^2\right)}{T-t+1}
    \\
    &=
    \frac{\reps{n}}{T-t+1}\left(\frac{q_n}{q}\right)^2\prod_{i=1}^{t-1} \left(1-\frac{1}{T-t+1+i}\left(\frac{q_n}{q}\right)^2\right).
\end{align}
If $t= T_n$, by substituting $\reps{n}^{t-1}$ in (\ref{eq:recursive}), we obtain
\begin{align}
    \reps{n}^{T_n} &= \left(\frac{\reps{n}-\sum_{\tau = 1}^{T_n-1} \reps{n}^{\tau}}{T-T_n+1}\right)
    = \left(\frac{\reps{n}^{T_n-1}(T-T_n+2)\left(\frac{q}{q_n}\right)^2- \reps{n}^{T_n-1}}{T-T_n+1}\right)
    \\
    &= \reps{n}^{T_n-1}\frac{\left(T-T_n+2-\left(\frac{q_n}{q}\right)^2\right)}{T-T_n+1}\left(\frac{q}{q_n}\right)^2
    \\
    &= 
    \frac{\reps{n}}{T-T_n+2}\left(\prod_{i=1}^{T_n-2} \left(1-\frac{1}{T-T_n+2+i}\left(\frac{q_n}{q}\right)^2\right)\right)\frac{\left(T-T_n+2-\left(\frac{q_n}{q}\right)^2\right)}{T-T_n+1}
    \\
    &=
    \frac{\reps{n}}{T-T_n+1}\prod_{i=1}^{T_n-1} \left(1-\frac{1}{T-T_n+1+i}\left(\frac{q_n}{q}\right)^2\right).
\end{align}

If $t> T_n$, by substituting $\reps{n}^{t-1}$ in (\ref{eq:recursive}), we obtain
\begin{align}
    \reps{n}^{t} &= \left(\frac{\reps{n}-\sum_{\tau = 1}^{t-1} \reps{n}^{\tau}}{T-t+1}\right)
    = \left(\frac{\reps{n}^{t-1}(T-t+2)- \reps{n}^{t-1}}{T-t+1}\right)
    \\
    &= \reps{n}^{t-1} = 
     \frac{\reps{n}}{T-T_n+1}\prod_{i=1}^{T_n-1} \left(1-\frac{1}{T-T_n+1+i}\left(\frac{q_n}{q}\right)^2\right). 
\end{align}



\subsection{Proof of Theorem~\ref{thm:privacyspent}}\label{app:thm:privacyspent}
We use the explicit solutions of the recursive formula (\ref{eq:recursive}), presented in Lem.~\ref{thm:recursive}, to prove this theorem. When $t< T_n$, 
\begin{align}
    \reps{n}^t - \reps{n}^{t-1} &=  \frac{\reps{n}}{T-t+1}\left(\frac{q_n}{q}\right)^2\prod_{i=1}^{t-1} \left(1-\frac{1}{T-t+1+i}\left(\frac{q_n}{q}\right)^2\right)  
    \\
    &-
    \frac{\reps{n}}{T-t+2}\left(\frac{q_n}{q}\right)^2\prod_{i=1}^{t-2} \left(1-\frac{1}{T-t+2+i}\left(\frac{q_n}{q}\right)^2\right)
    \\
    &=\reps{n}\left(\frac{q_n}{q}\right)^2\left(\prod_{i=1}^{t-2} \left(1-\frac{1}{T-t+2+i}\left(\frac{q_n}{q}\right)^2\right)\right)
    \\
    &\times \left(\frac{1}{T-t+1}\left(1-\frac{1}{T-t+2}\left(\frac{q_n}{q}\right)^2\right) - \frac{1}{T-t+2}\right)
    \\
    &=\reps{n}\left(\frac{q_n}{q}\right)^2\left(\prod_{i=1}^{t-2} \left(1-\frac{1}{T-t+2+i}\left(\frac{q_n}{q}\right)^2\right)\right) \frac{1-\left(\frac{q_n}{q}\right)^2}{(T-t+1)(T-t+2)}.\label{app:app:thm:privacyspent:1}
\end{align}
The right-hand side of (\ref{app:app:thm:privacyspent:1}) is larger than equal to zero because $q_n \leq q$. Therefore, in this case $\reps{n}^t \geq \reps{n}^{t-1}$. When $t= T_n$,
\begin{align}
    \reps{n}^{T_n} - \reps{n}^{T_n-1} &=  \frac{\reps{n}}{T-T_n+1}\prod_{i=1}^{T_n-1} \left(1-\frac{1}{T-T_n+1+i}\left(\frac{q_n}{q}\right)^2\right)  
    \\
    &-
    \frac{\reps{n}}{T-T_n+2}\left(\frac{q_n}{q}\right)^2\prod_{i=1}^{T_n-2} \left(1-\frac{1}{T-T_n+2+i}\left(\frac{q_n}{q}\right)^2\right)
    \\
    &=\reps{n}\left(\prod_{i=1}^{T_n-2} \left(1-\frac{1}{T-T_n+2+i}\left(\frac{q_n}{q}\right)^2\right)\right)
    \\
    &\times \left(\frac{1}{T-t+1}\left(1-\frac{1}{T-T_n+2}\left(\frac{q_n}{q}\right)^2\right) - \frac{\left(\frac{q_n}{q}\right)^2}{T-T_n+2}\right)
    \\
    &=\reps{n}\left(\prod_{i=1}^{T_n-2} \left(1-\frac{1}{T-T_n+2+i}\left(\frac{q_n}{q}\right)^2\right)\right) \frac{1-\left(\frac{q_n}{q}\right)^2}{(T-T_n+1)}.\label{app:app:thm:privacyspent:2}
\end{align}
The right-hand side of (\ref{app:app:thm:privacyspent:2}) is again larger than equal to zero because. Therefore, in this case we also have $\reps{n}^{T_n} \geq \reps{n}^{T_n-1}$.  Lem.~\ref{thm:recursive} also shows $\reps{n}^{t} = \reps{n}^{t-1}$ when $t>T_n$.


\subsection{Proof of Theorem~\ref{THM:clip_aware1}}\label{app:THM:clip_aware1}
%In Thm.~\ref{THM:clip_aware1}, the expectation is taken with respect to (w.r.t.) (i), (ii), and (iii). Thus, we denote this as $\left\| \expectation{(i), (ii), (iii)}{\error^t}\right\|$.


If the expectation is taken w.r.t. (i) the randomness of local datasets, (ii) the sampling of clients, and (iii) the randomness of injected Gaussian noise, then the bias
is simplified as follows:
\begin{align}
&\left\| \expectation{(i),(ii),(iii)}{\error^t}\right\| \nonumber 
\\
&= \frac{1}{N} \left\| \sum_{n=1}^N \expectation{(i),(ii),(iii)}{\Delta \model_n^t -  \frac{1}{q^t}\left(\rb_n^t \left(\clip{{\Delta \model_n^t}}{c_n^t}+\rz_n^t\right)+ (1-\rb_n^t)\tilde{\rz}_n^t\right)} \right\|
\nonumber
\\
&= \frac{1}{N} \left\| \sum_{n=1}^N \expectation{(i),(ii)}{\Delta \model_n^t -  \frac{1}{q^t}\rb_n^t \left(\clip{{\Delta \model_n^t}}{c_n^t}\right)} \right\|
\label{app:THM:clip_aware1:2}
\\
& = \frac{1}{N} \left\| \sum_{n=1}^N \expectation{(i)}{{\Delta \model_n^t} -  \frac{q_n^t}{q^t} \left(\clip{{\Delta \model_n^t}}{c_n^t}\right)} \right\|
\label{app:THM:clip_aware1:4}
\\
& = \frac{1}{N} \left\| \sum_{n=1}^N \expectation{(i)}{{\Delta \model_n^t}\left(\frac{q^t}{q^t} - \frac{q_n^t}{q^t} + \frac{q_n^t}{q^t}\right) -  \frac{q_n^t}{q^t} \left(\clip{{\Delta \model_n^t}}{c_n^t}\right)} \right\|
\label{app:THM:clip_aware1:5}
\\
& = \frac{1}{N} \left\| \sum_{n=1}^N \left(\left(\frac{q^t}{q^t} - \frac{q_n^t}{q^t}\right)\expectation{(i)}{{\Delta \model_n^t}} + \frac{q_n^t}{q^t} \expectation{(i)}{{\Delta \model_n^t}-   \left(\clip{{\Delta \model_n^t}}{c_n^t}\right)}\right) \right\|
\label{app:THM:clip_aware1:6}
\\
&\leq \frac{1}{N} \left\| \sum_{n=1}^N \left(\frac{q^t}{q^t} - \frac{q_n^t}{q^t}\right)\expectation{(i)}{{\Delta \model_n^t}}\right\| + \frac{1}{N}\sum_{n=1}^N\left\|\frac{q_n^t}{q^t} \expectation{(i)}{{\Delta \model_n^t}-   \left(\clip{{\Delta \model_n^t}}{c_n^t}\right)} \right\|
\label{app:THM:clip_aware1:7}
\\
&\leq \frac{1}{N} \left\| \sum_{n=1}^N \left(\frac{q^t}{q^t} - \frac{q_n^t}{q^t}\right)\expectation{(i)}{{\Delta \model_n^t}}\right\| + \frac{1}{N}\sum_{n=1}^N \frac{q_n^t}{q^t}  \frac{\expectation{(i)}{\left\|{\Delta \model_n^t}\right\|^{\rho}}}{\left(c_n^t\right)^{\rho-1}}. \label{app:THM:clip_aware1:8}
\end{align}
The equality~(\ref{app:THM:clip_aware1:2}) is due to $\expectation{(iii)}{\rz_n^t}=\expectation{(iii)}{\tilde{\rz}_n^t}=0$. The equality~(\ref{app:THM:clip_aware1:4}) is due to $\expectation{(ii)}{\rb_n^t}=q_n^t$. The inequality~(\ref{app:THM:clip_aware1:7}) is due to triangle inequality. The inequality~(\ref{app:THM:clip_aware1:8}) is due to the clipping bias lemma~\citep{das2023beyond}, given any $\rho>1$.   


\subsection{Proof of Theorem~\ref{THM:clip_aware2}}\label{app:THM:clip_aware2}

If the expectation is taken w.r.t. (i) the randomness of local datasets and (ii) the sampling of clients, (iii) the randomness of injected Gaussian noise, and (iv) privacy budget assignment randomness, then the bias is simplified as follows:

\begin{align}
&\left\| \expectation{(i),(ii),(iii), (iv)}{\error^t}\right\| \nonumber 
\\
&= \frac{1}{N} \left\| \sum_{n=1}^N \expectation{(i),(ii),(iii), (iv)}{\Delta \model_n^t -  \frac{1}{q^t}\left(\rb_n^t \left(\clip{{\Delta \model_n^t}}{c_n^t}+\rz_n^t\right)+ (1-\rb_n^t)\tilde{\rz}_n^t\right)} \right\|
\nonumber
\\
&= \frac{1}{N} \left\| \sum_{n=1}^N \expectation{(i),(ii), (iv)}{\Delta \model_n^t -  \frac{1}{q^t}\rb_n^t \left(\clip{{\Delta \model_n^t}}{c_n^t}\right)} \right\|
\label{app:THM:clip_aware2:2}
\\
& = \frac{1}{N} \left\| \sum_{n=1}^N \expectation{(i),(ii), (iv)}{{\Delta \model_n^t} -  \frac{\rb_n^t}{q^t} \left(\clip{{\Delta \model_n^t}}{c_n^t}\right)} \right\|
\label{app:THM:clip_aware2:3}
\\
& = \frac{1}{N} \left\| \sum_{n=1}^N \expectation{(i), (iv)}{{\Delta \model_n^t} -  \frac{q_n^t}{q^t} \left(\clip{{\Delta \model_n^t}}{c_n^t}\right)} \right\|
\label{app:THM:clip_aware2:4}
\\
& = \frac{1}{N} \left\| \sum_{n=1}^N \expectation{(i), (iv)}{\frac{\Delta \model_n^t}{1}\left(\frac{q^t}{q^t} - \frac{q_n^t}{q^t} + \frac{q_n^t}{q^t}\right) -  \frac{q_n^t}{q^t} \left(\clip{{\Delta \model_n^t}}{c_n^t}\right)} \right\|
\label{app:THM:clip_aware2:5}
\\
& = \frac{1}{N} \left\| \sum_{n=1}^N \left(\expectation{(i),(iv)}{\left(\frac{q^t}{q^t} - \frac{q_n^t}{q^t}\right){\Delta \model_n^t}} + \expectation{(i),(iv)}{{q_n^t}{q^t} {\Delta \model_n^t}-   \left(\clip{{\Delta \model_n^t}}{c_n^t}\right)}\right) \right\|
\label{app:THM:clip_aware2:6}
\\
&\leq \frac{1}{N} \left\| \sum_{n=1}^N \expectation{(iv)}{\left(\frac{q^t}{q^t} - \frac{q_n^t}{q^t}\right)\expectation{(i)}{{\Delta \model_n^t}}}\right\| + \nonumber
\\
&+ \frac{1}{N}\sum_{n=1}^N\left\|\expectation{(iv)}{\frac{q_n^t}{q^t} \expectation{(i)}{{\Delta \model_n^t}-   \left(\clip{{\Delta \model_n^t}}{c_n^t}\right)}} \right\|
\label{app:THM:clip_aware2:7}
\\
&\leq \frac{1}{N} \left\| \sum_{n=1}^N \expectation{(iv)}{\left(\frac{q^t}{q^t} - \frac{q_n^t}{q^t}\right)\expectation{(i)}{{\Delta \model_n^t}}}\right\| + \nonumber
\\
&+ \frac{1}{N}\sum_{n=1}^N\expectation{(iv)}{\left\|\frac{q_n^t}{q^t} \expectation{(i)}{{\Delta \model_n^t}-   \left(\clip{{\Delta \model_n^t}}{c_n^t}\right)} \right\|}
\label{app:THM:clip_aware2:8}
\\
&\leq \frac{1}{N} \left\| \sum_{n=1}^N\expectation{(iv)}{ \left(\frac{q^t}{q^t} - \frac{q_n^t}{q^t}\right)\expectation{(i)}{{\Delta \model_n^t}}}\right\| + \frac{1}{N}\sum_{n=1}^N \expectation{(iv)}{\frac{q_n^t}{q^t}  \frac{\expectation{(i)}{\left\|{\Delta \model_n^t}\right\|^{\rho}}}{\left(c_n^t\right)^{\rho-1}}}. \label{app:THM:clip_aware2:9}
\end{align}

The equality~(\ref{app:THM:clip_aware2:2}) is due to $\expectation{(iii)}{\rz_n^t}=\expectation{(iii)}{\tilde{\rz}_n^t}=0$. The equality~(\ref{app:THM:clip_aware2:3}) is due to $\expectation{(ii)}{\rb_n^t}=q_n^t$. The inequality~(\ref{app:THM:clip_aware2:7}) is due to triangle inequality. The inequality~(\ref{app:THM:clip_aware2:9}) is due to the clipping bias lemma~\citep{das2023beyond} given any $\rho>1$.

We next further simplify the first and second terms on the right-hand side of (\ref{app:THM:clip_aware2:9}).

The first term equals zero:
\begin{align}
&\expectation{(iv)}{ \left(\frac{q^t}{q^t} - \frac{q_n^t}{q^t}\right)\expectation{(i)}{{\Delta \model_n^t}}} \nonumber
\\
&= \expectation{(i)}{\left\|{\Delta \model_n^t}\right\|^{\rho}}\expectation{(iv)}{1 - \frac{q_n^t}{q^t}}
\label{app:THM:clip_aware2:1:1}
\\
&= \expectation{(i)}{\left\|{\Delta \model_n^t}\right\|^{\rho}}\left(\sum_{i\in [N]} \pr{\eps_n=\hat{\eps}_i}\expectation{(iv)}{\left.1 - \frac{q_n^t}{q^t}\right| \eps_n=\hat{\eps}_i}\right)
\label{app:THM:clip_aware2:1:2}
\\
&=\expectation{(i)}{\left\|{\Delta \model_n^t}\right\|^{\rho}}\left(\frac{1}{N}\sum_{i\in [N]}\left(1 - \frac{\hat{q}_i^t}{q^t} \right) \right)
\label{app:THM:clip_aware2:1:3}
\\&=\expectation{(i)}{\left\|{\Delta \model_n^t}\right\|^{\rho}}\left(\frac{1}{N}\sum_{n\in [N]}\left(1 - \frac{q_n^t}{q^t} \right) \right)\label{app:THM:clip_aware2:1:4}
\\
&=
\expectation{(i)}{\left\|{\Delta \model_n^t}\right\|^{\rho}}\left(\frac{1}{N}\left(N - \frac{q^tN}{q^t} \right) \right) = 0.\label{app:THM:clip_aware2:1:5}
\end{align}

Equality (\ref{app:THM:clip_aware2:1:1}) is due to the independency of randomness between (i) and (iv). Equality (\ref{app:THM:clip_aware2:1:3}) is because of our assumption that the sampling from $P_n$ and of $\eps_n$ is independent. Equality (\ref{app:THM:clip_aware2:1:5}) is because $\sum_{n=1}^N q_n^t = q^t$.

The second term on the right-hand side of (\ref{app:THM:clip_aware2:9}) can be further simplified into:
\begin{align}
&\expectation{(iv)}{\frac{q_n^t}{q^t} \frac{\expectation{(i)}{\left\|{\Delta \model_n^t}{}\right\|^{\rho}}}{\left(c_n^t\right)^{\rho-1}}} \nonumber
\\
&= \expectation{(i)}{\left\|{\Delta \model_n^t}\right\|^{\rho}}\expectation{(iv)}{\frac{q_n^t}{q^t} \frac{1}{\left(c_n^t\right)^{\rho-1}}}
\label{app:THM:clip_aware2:2:1}
\\
&= \expectation{(i)}{\left\|{\Delta \model_n^t}\right\|^{\rho}}\left(\sum_{i\in [N]} \pr{\eps_n=\hat{\eps}_i}\expectation{(iv)}{\left.\frac{q_n^t}{q^t} \frac{1}{\left(c_n^t\right)^{\rho-1}}\right| \eps_n=\hat{\eps}_i}\right)
\label{app:THM:clip_aware2:2:2}
\\
&=\expectation{(i)}{\left\|{\Delta \model_n^t}\right\|^{\rho}}\left(\frac{1}{N}\sum_{i\in [N]}\frac{\hat{q}_i^t}{q^t} \frac{1}{\left(\hat{c}_i^t\right)^{\rho-1}} \right)
\label{app:THM:clip_aware2:2:3}
\\
&=\expectation{(i)}{\left\|{\Delta \model_n^t}\right\|^{\rho}}\left(\frac{1}{N}\sum_{n\in [N]}\frac{q_n^t}{q^t} \frac{1}{\left(c_n^t\right)^{\rho-1}} \right).\label{app:THM:clip_aware2:2:4}
\end{align}

Equality (\ref{app:THM:clip_aware2:2:1}) is due to the independency of randomness between (i) and (iv). Equality (\ref{app:THM:clip_aware2:2:2}) is because of our assumption that the sampling from $P_n$ and of $\eps_n$ is independent. Combining (\ref{app:THM:clip_aware2:9}), (\ref{app:THM:clip_aware2:1:5}), and (\ref{app:THM:clip_aware2:2:4}), Thm.~\ref{THM:clip_aware2} is proved.


\subsection{%\tcb
{Extended Background}}
\subsubsection{Some Useful Lemmas  from Prior Works}\label{app:prior:lemma}

\begin{lem}\label{app:prior:clipbias}
\textbf{[Clipping bias~\citep{das2023beyond}} Suppose $\phi(\xi)$ (where $\xi$ denotes the source of randomness) is an unbiased estimator of $\phi$, i.e., $\expectation{\xi}{\phi(\xi)}=\phi$. Let $b(\xi)$ denote the clipping bias of $\clip{\phi(\xi)}{c}$, i.e., $b(c)=\left\|\phi - \expectation{\xi}{\clip{\phi(\xi)}{c}} \right\|$.
Then for any $\rho>1$,
\begin{align}
    b(c)\leq \frac{\expectation{\xi}{\left\|\phi(\xi) \right\|^{\rho}}}{c^{\rho-1}}.
\end{align}
\end{lem}


\subsubsection{Differential Privacy}\label{app:prior:rdp}


{
\begin{definition}[$(\eps, \delta)$-DP~\cite{dwork2014algorithmic}]\label{app:epsdeltaDP}
The randomized algorithm $\gA:\chi\rightarrow \calR$ with domain $\chi$ and range $\calR$ satisfies $(\eps, \delta)$-DP iff for any two {\em neighboring} inputs $\train,\train'\in \chi$ that differ by at most one record, and any measurable subset of outputs $\gS \subseteq \calR$,
\begin{align}\label{ch2:eq:eps_delta_dp}
\pr{\gA(\train)\in \gS} \leq e^{\eps}\pr{\gA(\train')\in \gS} + \delta.
\end{align} 
\end{definition}
}

{In (\ref{ch2:eq:eps_delta_dp}), the privacy budget $\eps \in \sR_{+}$ controls the extent to which the output distributions induced by two neighboring inputs may differ. The $\delta\in [0,1]$ quantifies the probability of violating the privacy guarantee. Allowing a larger $\delta\in [0,1]$  improves utility at the cost of a more relaxed (weaker) privacy guarantee. One way to relax the DP guarantee is to use $(\evalpha,\eps)$-Rényi DP (RDP)~\citep{mironov2017renyi}. The $\evalpha>1$ is the order of R\'{e}nyi divergence between distributions $P:=\pr{\gA(\train)}$ and $P':=\pr{\gA(\train')}$, defined as 
\begin{align}
\ren{\evalpha}{P}{P'}:=\frac{1}{1-\evalpha}\log \mathbb{E}_{\rx\sim P'}\left(\frac{P}{P'}\right)^{\evalpha}.
\end{align}
}
{
While the R\'{e}nyi divergence can be defined for $\alpha < 1$, including negative orders, the RDP definition~\cite{mironov2017renyi} is based on $\alpha \geq 1$ and is outlined as follows.
}
{
\begin{definition}[R\'{e}nyi DP (RDP)~\cite{mironov2017renyi}]\label{def:RDP}
The randomized algorithm $\gA:\chi\rightarrow \calR$ with domain $\chi$ and range $\calR$ is $(\alpha,\epsilon)$-RDP iff for any neighboring inputs ${\calD} , {\calD'}\in \chi$,  we have
\begin{align}
 \ren{\evalpha}{\pr{\gA(\train)}}{\pr{\gA(\train')}}\leq \eps.
\end{align}
\end{definition}
}

When accounting for total privacy consumption over an iterative algorithm, RDP offers a smoother composition property than DP. RDP allows the privacy budget to accumulate linearly with the number of training rounds~\citep{mironov2017renyi}. This simplifies the tracking and management of privacy budgets over time. We next recall a lemma from~\citep{mironov2017renyi}. Lemma~\ref{app:prior:rdp:lem} shows how RDP can be converted to DP when needed.  


\begin{lem}\label{app:prior:rdp:lem}
If $\gA$ is an $(\evalpha,\eps_{\text{rdp}})$-RDP algorithm, it also satisfies $\left(\eps, \delta \right)$-DP for any $0<\delta<1$, where
\begin{align}
    \eps= \eps_{\text{rdp}}+\log \frac{\evalpha-1}{\evalpha}-\frac{\log \delta + \log \evalpha}{\evalpha - 1}.
\end{align}
\end{lem}

{
To implement privacy guarantees, we use the sampled Gaussian mechanism (SGM)~\cite{mironov2019r}, formally defined as follows.
}


{
\begin{definition}[SGM~\cite{mironov2019r}]
Consider the algorithm $\gA$ which maps a subset $\calD\subseteq\chi$ to $\mathbb{R}^w$ and has $\ell_2$-sensitivity $c$. The sampled Gaussian mechanism parameterized by the sampling rate $q \in [0,1]$, $c$, and noise multiplier $\sigma>0$ is defined as 
\begin{align}
\gG_{\sigma, c, q}(\calD) := \gA(\{x \;| \; x\in \calD \text{ is sampled with Probability } q \}) + \gN(0,c^2\sigma^2\sI_w),
\end{align}
where each element of $\calD$ is (Poisson) sampled independently at random with probability $q$, and $\gN(0,c^2\sigma^2\sI_w)$ is spherical $w$-dimensional Gaussian noise with per-coordinate variance $c^2\sigma^2$.
\end{definition}
}
{
\begin{lem}[\cite{mironov2019r}]\label{lem:rdpsGM} 
The SGM $\gG_{\sigma,c,q}$ with $c=1$ guarantees $(\alpha, \eps)$-RDP, where $\eps\leq \frac{2\alpha q^2}{\sigma^2}$.
\end{lem}
}



\subsection{Extended Experimental Setup}\label{app:extendedExpSetup}
We conduct our experiments in Python 3.11 using Pytorch leveraging the 4 $\times$ L4 24 GB GPU. 
Below, we provide additional details on the experimental setups used in Sec.~\ref{sec:simulation} to analyze how our time-adaptive DP-FL framework enhances the privacy-utility tradeoff and in Appendix~\ref{app:extendedresults} which extends experiments for further analysis.


\textbf{Details on Datasets.} For our experiments, we use FMNIST, MNIST, Adult Income, and {CIFAR10} datasets. Both FMNIST and MNIST datasets have a training set of 60,000 and a test set of 10,000 28 $\times$ 28 images, associated with 10 labels. The Adult Income dataset consists of 48,842 samples with 14 features and is split into a training set of 32,561 samples and a test set of 16,281 samples. {The CIFAR10 dataset consists of 60,000 32 $\times$ 32 color images in 10 classes, with 6000 images per class. There are 50,000 training images and 10,000 test images. }
 
{\textbf{Simulation Parameters.} Throughout our simulations, we use SGD optimizer and momentum equal to 0.9. We also use a CosineAnnealing learning rate scheduler from~\citep{inproceedings} for faster convergence. In Sec.\ref{sec:simulation}, we fix the spending-based sample rate (during spend mode) to $q= 0.9$ and the average clipping norm to $c=250$. We consider the transition from saving round to spending round occurs in the middle of training. I.e., given the total number of rounds $T=25$, we set $T_{\text{group},1}= T_{\text{group},2}, T_{\text{group},3}=13$. The obtained results are averaged over three runs. In Table~\ref{tab:hyperparamstable} we summarize other hyperparameters, including learning rate ($\lr$), number of clients ($N$), batch size ($\bs$), number of local epochs ($L$), and the saving-based sampling rates of clients from privacy groups 1, 2, and 3 ($q_{\text{group},1}, q_{\text{group},2}, q_{\text{group},3}$). 

\begin{table}[htbp]
    \centering
    \caption{Parameters for different datasets, used in Table~\ref{tab:cmptable} and Figure~\ref{fig:privacyparams}. We set $T=25$, $T_{\text{group},1}= T_{\text{group},2}, T_{\text{group},3}=13$, $q=0.9$, and $c=250$. }
    \label{tab:hyperparamstable}
    \begin{tabular}{p{2.5cm}p{2.8cm}p{0.7cm}p{0.7cm}p{0.7cm}p{0.7cm}p{2.8cm}}
        \toprule
        \textbf{Dataset} &  {\scriptsize $(\eps_{\text{group},1}, \eps_{\text{group},2}, \eps_{\text{group},3})$} & $\lr$ & $N$ & $\bs$  & $L$ & {\scriptsize $(q_{\text{group},1},q_{\text{group},2},q_{\text{group},3})$} \\
        \midrule
        FMNIST  & $(10,20,30)$ & 0.001  & 100 & 125  & 30 & $(0.5,0.6,0.7)$   \\
        MNIST & $(10,15,20)$   & 0.001  & 100 & 125  & 30 & $(0.5,0.6,0.7)$  \\
        Adult Income & $(10,20,30)$ & 0.01 & 80 & 32 & 5 & $(0.6,0.7,0.8)$   \\
        \bottomrule
    \end{tabular}
\end{table}



\begin{figure}[!htbp]
    \centering
    
    \vspace{5pt}
        \centering
        \includegraphics[width=0.45\textwidth]{images/pdfs/p_4_fmnist_train_loss_epsilons_10_20_30_clients_100_sigma_0.30_clip_250.00_srate_0.90_runs_1_rounds_25_epochs_30final_aid.pdf} 
        \caption{
        {Average Training loss of clients in our time-adaptive DP-FL scheme plotted versus the IDP-FedAvg baseline with FMNIST dataset in training rounds T = 25. We set $(\eps_{\text{group},1}, \eps_{\text{group},2}, \eps_{\text{group},3})=(10,20,30)$ in our scheme and IDP-FedAvg.}}
        \label{fig:trainloss}  
\end{figure}


\subsection{Extended Experimental Results} \label{app:extendedresults}
\textbf{Impact of Training Rounds on Model Convergence.} We extend experiments to more training rounds--- $T\in \{25, 50, 100\}$. For example, in Figure~\ref{fig:morerounds}, we set $T=50$, and plot the global test accuracy vs. communication rounds.
It is evident from Figure~\ref{fig:morerounds} that for our time-adaptive DP-FL framework, as the number of training rounds increases, the upward trend in the accuracy starts slowing down. 
{However, increasing the number of communication rounds does not always improve accuracy. This is because, with more rounds, the privacy budget is distributed across more rounds, resulting in a lower budget per round. Consequently, the increased effect of perturbation can degrade the privacy-utility tradeoff. This is demonstrated in our FMNIST and MNIST experiments, as shown in Table~\ref{tab:clientstable_rounds}, in which we report the final-round test accuracy across different schemes. As shown in the third column of Table~\ref{tab:clientstable_rounds}, when training rounds increase from 25 to 50 and from 50 to 100, FedAvg (the non-DP baseline scheme) consistently demonstrates an upward trend in both MNIST and FMNIST experiments. However, our scheme (fifth column of Table~\ref{tab:clientstable_rounds}) and IDP-FedAvg (fourth column), which operate under limited group privacy budgets $(\eps_{\text{group},1}, \eps_{\text{group},2}, \eps_{\text{group},3})=(10, 20, 30)$, do not exhibit the same consistent improvement. They exhibit an upward trend from 25 to 50 rounds but not consistently from 50 to 100 rounds. Notably, the best performance amongst the DP experiments of Table~\ref{tab:clientstable_rounds}) is achieved by our scheme, reaching $75.63\%$ after $T=100$ rounds for the FMNIST dataset, and $90.78\%$ at $T=50$ rounds for the MNIST dataset.}


\begin{figure}[!ht]
    \centering
    \vspace{5pt}  
        \centering
        \includegraphics[width=0.45\textwidth]{images/pdfs/p_4_fmnist_global_test_acc_epsilons_20_20_20_clients_100_sigma_0.30_clip_250.00_srate_0.80_runs_2_rounds_50_epochs_30final.pdf} 
        \caption{\textbf{Global test accuracy for increasing number of communication rounds.} In this figure, we use the FMNIST dataset, $N=100$ clients, $L=30$ local iterations, $(\eps_{\text{group},1}, \eps_{\text{group},2}, \eps_{\text{group},3})=(20, 20, 20)$, $c=250$, and $q=0.8$.} 
        \label{fig:morerounds}  
\end{figure}

\begin{table}[h!]
\centering
\caption{
{Benchmarking our time-adaptive DP-FL scheme against the baselines in terms of global test accuracy and across varying datasets and number of training rounds (T). We set $(\eps_{\text{group},1}, \eps_{\text{group},2}, \eps_{\text{group},3})=(10,20,30)$ in our scheme and IDP-FedAvg.}}
\small 
\setlength{\tabcolsep}{8pt} 
\renewcommand{\arraystretch}{1.2} 
\begin{tabular}{cccccc}
\toprule
\textbf{Dataset} & \textbf{T} & \makecell[tl]{\textbf{FedAvg} \\ (non-DP)} & \textbf{IDP-FedAvg} & \textbf{Ours}  \\ 
\midrule
\multirow{2}{*}{FMNIST} & 25 & 72.95 & 62.57  & \textbf{66.55} \\

& 50  & 76.00 & 71.80  & \textbf{75.51} \\  
                        & 100 & 80.14 & 71.29  & \textbf{75.63} \\  

                        
\midrule
\multirow{2}{*}{MNIST}  & 25  & 90.23 & 64.53  & \textbf{74.69} \\  

& 50  & 93.42 & 89.57  & \textbf{90.78} \\  
                        & 100 & 95.91 & 87.00  & \textbf{90.15} \\  
\bottomrule
\end{tabular}
\label{tab:clientstable_rounds}
\centering \vspace{-2ex}
\end{table}


\begin{figure}[!ht]
    \centering
    \vspace{5pt} 
        \centering
        \includegraphics[width=0.45\textwidth]{images/pdfs/p_4_mnist_global_test_acc_epsilons_2_5_10_clients_100_sigma_0.30_clip_250.00_srate_0.80_runs_2_rounds_50_epochs_30.pdf} 
        \caption{\textbf{Test accuracy for our time-adaptive DP-FL framework vs. IDP-FedAvg, using stricter privacy budgets $(\eps_{\text{group},1},\eps_{\text{group},2},\eps_{\text{group},3})=(2,5,10)$.} In this figure, we use $N=100$ clients, $T=50$ global rounds, $L=30$ local iterations, $c=250$, and $q=0.8$.} 
        \label{fig:lowprivacybudgets}
\end{figure}


\textbf{Impact of Different Privacy Budgets on Model Utility.} {We present additional experimental results to evaluate the impact of stricter privacy budgets $(\eps_{\text{group},1}, \eps_{\text{group},2}, \eps_{\text{group},3}) = (2,5,10)$ and $(5,5,5)$ on model utility (test accuracy). The results are presented in Figure~\ref{fig:lowprivacybudgets} and Tables~\ref{tab:strictprivacybudgetstable} and~\ref{tab:strictprivacybudgetstable2}. As expected, we observe that lower privacy budgets hamper utility. In particular, in Table~\ref{tab:strictprivacybudgetstable}, we benchmark our scheme against the IDP-FedAvg baseline using two sets of non-uniform privacy budgets, $(10,20,30)$ and $(2,5,10)$, evaluated across two datasets. Our findings suggest that the time-adaptive DP-FL framework yields considerably higher utility than IDP-FedAvg, also under stringent privacy constraints. Similarly, Table~\ref{tab:strictprivacybudgetstable2} focuses on uniform privacy budgets and further confirms that even with a reduction in privacy budgets from $(10,10,10)$ to $(5,5,5)$, our scheme consistently outperforms the corresponding baselines. }


\begin{table}[h!]
\centering
\caption{
{Benchmarking our time-adaptive DP-FL scheme against the baselines in terms of the final-round test accuracy and across varying privacy budgets $(\eps_{\text{group},1}, \eps_{\text{group},2}, \eps_{\text{group},3})$. We set $T=25$ and $L=30$ for $\epsilon = \{10,20,30\}$ and $T=25$ and $L=50$ for $\epsilon = \{2,5,10\}$, $(q_{\text{group},1},q_{\text{group},2},q_{\text{group},3})=(0.3,0.5,0.7)$}.}
\small
\setlength{\tabcolsep}{10pt} 
\renewcommand{\arraystretch}{1.2} 
\begin{tabular}{cccc}
\toprule
\textbf{Dataset} & \textbf{Privacy Budgets} &    \makecell[tl]{\textbf{IDP-FedAvg}\\ Non-uniform} &   \makecell[tl]{\textbf{Ours}\\ Non-uniform} \\ 
\midrule
FMNIST & $(10, 20, 30)$ & 62.57  &  \textbf{66.55} \\  
FMNIST & $(2, 5, 10)$ &  60.99  & \textbf{65.75} \\  
\hline
MNIST & $(10, 20, 30)$ & 64.53 &  \textbf{77.38} \\  
MNIST & $(2, 5, 10)$ &  63.35 &  \textbf{66.50} \\  
\bottomrule
\end{tabular}
\label{tab:strictprivacybudgetstable}
\vspace{-2ex} 
\end{table}


\begin{table}[h!]
\centering
\caption{
{Benchmarking our time-adaptive DP-FL scheme against the baselines in terms of the final-round test accuracy and across varying uniform privacy budgets $\eps_{\text{group},1}= \eps_{\text{group},2}= \eps_{\text{group},3}$. We set $T=25$ and $L=30$.}}
\small 
\setlength{\tabcolsep}{10pt} 
\renewcommand{\arraystretch}{1.2} 
\begin{tabular}{cccccc}
\toprule
\textbf{Dataset} & \makecell[tl]{\textbf{Privacy} \\ {Budgets}} & \makecell[tl]{\textbf{Adaptive Clipping} \\ $(\beta_{\text{s}},\lr_{\text{s}})=(0, 1.0)$} & \textbf{DP-FedAvg} & \makecell[tl]{\textbf{Adaptive Clipping} \\ optimal $(\beta_{\text{s}},\lr_{\text{s}})$} & \textbf{Ours} \\ 
\midrule
FMNIST & $(10,10, 10)$ & 60.23 & 64.8 & 67.64 & \textbf{67.90} \\  
FMNIST & $(5,5, 5)$ & 52.39 & 51.06 & 52.39 & \textbf{60.79} \\  
\hline
MNIST & $(10, 10, 10)$ & 65.59 & 76.79 & 78.04 & \textbf{80.2} \\  
MNIST & $(5, 5, 5)$ & 55.48 & 61.45 & 55.48 & \textbf{69.07} \\  
\bottomrule
\end{tabular}
\label{tab:strictprivacybudgetstable2}
\vspace{-2ex}
\end{table}

\textbf{Additional Baseline.} 
{We benchmark our scheme against the adaptive clipping method~\cite{andrew2021differentially}, with pseudocode provided in Algorithm~\ref{alg:quantile}. We present results in the third and fifth columns of Table~\ref{tab:strictprivacybudgetstable2}. This baseline is designed for uniform privacy budgets and is parameterized by the server-side learning rate $\lr_{\text{s}}$ and momentum parameter $\beta_{\text{s}}$, which are not privacy-specific. To ensure a fair comparison with our scheme and other baselines in our paper, we set these parameters to $\lr_{\text{s}}=1.0$ and $\beta_{\text{s}}=0.0$. In column 3, we use these default values, while in column 5, we select the optimal values from a set of possible choices. As shown in the table, our scheme consistently outperforms adaptive clipping, even when the baseline’s parameters are optimally tuned. 
}






\textbf{Effect of Number of Clients on Model Utility.} We experiment with different numbers of clients---$N\in\{30, 60, 75\}$---for the MNIST dataset to validate the applicability of our time-adaptive DP-FL framework across various scenarios. Additionally, we also perform experiments to analyze if our framework outperforms the baselines, in terms of the utility of the trained model. Our results in Table~\ref{tab:clientstable} indicate that for all the different numbers of clients that we consider, our framework remarkably surpasses the utility of the baseline. 



\begin{table}[h!]
\caption{Comparison of model utility on a varying number of clients and comparison of model utility for time-adaptive DP-FL with baselines for a varying number of clients}
\centering
\small 
\setlength{\tabcolsep}{10pt} 
\renewcommand{\arraystretch}{1.2}  
\begin{tabular}{cccccc}
\toprule
\textbf{Number of clients} & \makecell[tl]{\textbf{SETUP} \\ Privacy Budgets} &  \makecell[tl]{\textbf{IDP-FedAvg} \\ Non-uniform} & \makecell[tl]{\textbf{Ours}\\ Non-uniform}  \\ 
\midrule
30& $(10, 20, 30)$  & 72.35  & \textbf{73.34}  \\  
\midrule
60 & $(10, 20, 30)$    & 78.69   & \textbf{83.83} \\
\midrule
75 & $(10, 20, 30)$  &  70.93    & \textbf{77.53}\\ 
\bottomrule
\end{tabular}
\label{tab:clientstable}
\centering \vspace{-2ex}
\end{table}

\textbf{The Choice of Hyperparameters.} %\tcb
{We evaluate our DP-FL framework with different choices of hyperparameters---different saving-based sampling rates ($q_{\text{group},1},q_{\text{group},2},q_{\text{group},3}$) and different saving-to-spending transition rounds  ($T_{\text{group},1},T_{\text{group},2},T_{\text{group},3}$). The final-round test accuracies for different choices of $(q_{\text{group},1},q_{\text{group},2},q_{\text{group},3})$, and across both MNIST and FMNIST datasets, are presented in Table~\ref{tab:saving_sampling_ratestable}. In this table, in Column 3 we set these rates as (0.5,0.6,0.7), in Column 4 as $(0.3,0.5,0.7)$, and in Column 5 as $(0.6,0.6,0.6)$. As shown in the table, our scheme, which uses lower sampling rates during saving---e.g., for all $i\in [3]$, $q_{\text{group},1}$ is smaller than $q=0.9$ in this table---outperforms the IDP-FedAvg baseline (Column 6) that uses a uniform sampling rate $q$ over time. This table also shows that our method is relatively robust against the clients' choice of saving-based sampling rates, consistently achieving performance between that of IDP-FedAvg and the ideal case of FedAvg without DP (Column 2).}


\begin{table}[h!]
\centering
\caption{
{Evaluating the impact of saving-based sampling rates of different privacy groups, ($q_{\text{group},1},q_{\text{group},2},q_{\text{group},3}$), on our time-adaptive DP-FL scheme in comparison with the baseline. We set $(\eps_{\text{group},1},\eps_{\text{group},2},\eps_{\text{group},3}) = (10,20,30)$, $T=25, L=30$ and $N=100$, $T_{\text{group},1}=T_{\text{group},2}=T_{\text{group},3}=13$, $q=0.9$, $c=250$, $\lr=0.001$, and $B=125$.}}
\small 
\setlength{\tabcolsep}{10pt} 
\renewcommand{\arraystretch}{1.2}  
\begin{tabular}{cccccc}
\toprule
{ \textbf{DATASET}}  &  \makecell[t]{{\scriptsize\textbf{FedAvg}} \\ {\scriptsize non-DP}} &\makecell[t]{{\scriptsize\textbf{Ours}} \\ {\scriptsize $(0.5,0.6,0.7)$}} & \makecell[t]{{\scriptsize\textbf{Ours}} \\ {\scriptsize $(0.3,0.5,0.7)$}} & \makecell[t]{{\scriptsize\textbf{Ours}} \\ {\scriptsize $(0.6,0.6,0.6)$}}  & {\scriptsize \textbf{IDP-FedAvg}}\\ 
\midrule
MNIST & 90.23 & \textbf{72.72} & \textbf{77.39} & \textbf{71.6} & 64.53 \\  
\midrule
FMNIST & 72.95 & \textbf{70.57}  & \textbf{66.55}  & \textbf{67.75} & 62.57   \\  
\midrule
\end{tabular}
\label{tab:saving_sampling_ratestable}
\end{table}

{The final-round test accuracies for different choices of saving-to-spending transition rounds $(T_{\text{group},1},T_{\text{group},2},T_{\text{group},3})$, for both MNIST and FMNIST datasets, are presented in Table~\ref{tab:privacy_spending_roundtable}. We set the total number of rounds as $T=25$. In this table, in Column 3 we set the transition rounds as (7,7,7), in Column 4 as $(7,13,19)$, in Column 5 as $(19,13,7)$, and in Column 6 as $(19,19,19)$. As shown in the table, our scheme, which transitions from saving to spend mode sometime between the first and final round---i.e., for all $i\in [3]$, $1<T_{\text{group},1}<25$---outperforms the IDP-FedAvg baseline (Column 7) which can be viewed as a special case of ours with transition rounds set to $(1,1,1)$. This table shows the robustness of our method to the client's choice of transition rounds, showing less than a $2\%$ variation in accuracy across different choices  while consistently achieving performance between that of IDP-FedAvg and the ideal-case of FedAvg without DP (Column 2).}


\begin{table}[h!]
\centering
\caption{
{Evaluating the impact of saving-to-spending transition rounds of different privacy groups, ($T_{\text{group},1}, T_{\text{group},2}, T_{\text{group},3}$), on our time-adaptive DP-FL scheme in comparison with the baseline. We set $(\eps_{\text{group},1},\eps_{\text{group},2},\eps_{\text{group},3})=(10,20,30)$, $T=25, L=30$, $N=100$, $(q_{\text{group},1},q_{\text{group},2},q_{\text{group},3})=(0.3,0.5,0.7)$, $q=0.9$, $c=250$, $\lr=0.001$, and $B=125$.}}
\small
\setlength{\tabcolsep}{10pt} 
\renewcommand{\arraystretch}{1.2}  
\begin{tabular}{ccccccc}
\toprule
{ \textbf{DATASET}} & \makecell[t]{{\scriptsize\textbf{FedAvg}} \\ {\scriptsize non-DP}} &  \makecell[t]{{\scriptsize\textbf{Ours}} \\ {\scriptsize $(7,7,7)$}} & \makecell[t]{{\scriptsize\textbf{Ours}} \\ {\scriptsize $(7,13,19)$}} & \makecell[t]{{\scriptsize\textbf{Ours}} \\ {\scriptsize $(19,13,7)$}} & \makecell[t]{{\scriptsize\textbf{Ours}} \\ {\scriptsize $(19,19,19)$}} & {\scriptsize \textbf{IDP-FedAvg}}\\ 
\midrule
MNIST & 90.23 & \textbf{74.38}  & \textbf{74.69}  & \textbf{72.24} & \textbf{73.88} & 64.53  \\  
\midrule
FMNIST & 72.95 & \textbf{66.72}  & \textbf{65.29}  & \textbf{65.34} & \textbf{67.5} & 62.57  \\  
\midrule
\end{tabular}
\label{tab:privacy_spending_roundtable}
\end{table}


\textbf{Experiments on The CIFAR10 Dataset.}
{We run experiments on the CIFAR10 dataset. The  
final-round test accuracies of our time-adaptive DP-FL framework in comparison with the FedAvg (non-DP) and IDP-FedAvg baselines
are presented in Table~\ref{tab:cifar10}. The results suggest that our proposed approach surpasses IDP-FedAvg, by lowering the gap to the ideal case of FedAvg by about $9\%$. We note that the test accuracies reported for all schemes in this table are relatively lower than those we reported earlier in this paper for the MNIST, FMNIST, and Adult Income datasets. We hypothesize that this happens due to the increased complexity of the CIFAR10 dataset, particularly when distributed in a non-iid manner in an FL setting with $N=100$ clients.   
}

\begin{table}[h!]
\centering
\caption{
{Benchmarking our time-adaptive DP-FL framework against the baselines using the CIFAR10 dataset. We set $(\eps_{\text{group},1},\eps_{\text{group},2},\eps_{\text{group},3})=(100,50,25)$, $T=50, L=30$, $N=100$, $(q_{\text{group},1},q_{\text{group},2},q_{\text{group},3})=(0.5,0.5,0.5)$, $q=0.9$, $c=250$, $\lr=0.001$, and $B=125$.}}
\begin{tabular}{cccc}
\toprule
\textbf{DATASET}  &  \textbf{FedAvg} & \textbf{IDP-FedAvg} & \textbf{Ours} \\
\midrule 
CIFAR10 & 44.42 & 34.97 & 35.41 \\
\midrule
\end{tabular}
\label{tab:cifar10}
\end{table}

\end{document}
