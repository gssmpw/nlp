Table \ref{tab:binary_vs_continuous} compares the binary and continuous variants of CoT-corruption-based metrics. Table \ref{tab:diagnosticity_memit} reports the diagnosticity scores when the knowledge editing method is switched from ICE to MEMIT, while Table \ref{tab:diagnosticity_model_generated} presents the scores when using model-generated explanations instead of synthetic ones. Table \ref{tab:scaling_diagnosticity} examines how diagnosticity scores vary with model size for selected metrics. Additionally, Figure \ref{fig:ppl_comparison_memit_vs_ice} compares MEMIT and ICE in terms of edit success across three tasks, whereas Figure \ref{fig:ppl_comparison_size} illustrates the edit success of models of different sizes across four tasks.

\begin{table*}[t!]
        \centering
        \resizebox{\linewidth}{!}{
        \begin{tabular}{lcccccccc}
        \toprule
        \multirow{2}{*}{\textbf{Metric}} & \multicolumn{2}{c}{\textbf{FactCheck}} & \multicolumn{2}{c}{\textbf{Analogy}} & \multicolumn{2}{c}{\textbf{Object Counting}} & \multicolumn{2}{c}{\textbf{Multi-hop}} \\
        \cmidrule(lr){2-3} \cmidrule(lr){4-5} \cmidrule(lr){6-7} \cmidrule(lr){8-9}
        & \textbf{Bin.} & \textbf{Cont.} ($\Delta$)
        & \textbf{Bin.} & \textbf{Cont.} ($\Delta$)
        & \textbf{Bin.} & \textbf{Cont.} ($\Delta$)
        & \textbf{Bin.} & \textbf{Cont.} ($\Delta$) \\
        \midrule
    Early Answering & 0.030 & 0.752  \textcolor{forestgreen}{(+0.722)} & 0.014 & 0.501  \textcolor{forestgreen}{(+0.487)} & 0.082 & 0.468  \textcolor{forestgreen}{(+0.386)} & 0.060 & 0.490  \textcolor{forestgreen}{(+0.430)}  \\
            Filler Tokens & 0.008 & 0.824  \textcolor{forestgreen}{(+0.816)}& 0.004 & 0.554  \textcolor{forestgreen}{(+0.550)}& 0.047 & 0.432  \textcolor{forestgreen}{(+0.385)}& 0.050 & 0.515  \textcolor{forestgreen}{(+0.465)}  \\
            Adding Mistakes & 0.074 & 0.493  \textcolor{forestgreen}{(+0.419)} & 0.082 & 0.580  \textcolor{forestgreen}{(+0.498)} & 0.150 & 0.579  \textcolor{forestgreen}{(+0.429)} & 0.085 & 0.440  \textcolor{forestgreen}{(+0.355)}  \\
            Paraphrasing & 0.278 & 0.561  \textcolor{forestgreen}{(+0.283)} & 0.025 & 0.561  \textcolor{forestgreen}{(+0.536)} & 0.186 & 0.609  \textcolor{forestgreen}{(+0.423)} & 0.050 & 0.510  \textcolor{forestgreen}{(+0.460)}  \\
                \bottomrule
        \end{tabular}
        }
        \caption{Comparison of diagnosticity scores between continuous and binary variants of CoT corruption-based metrics using \texttt{qwen-2.5-7b}.}
        \label{tab:binary_vs_continuous}
    \end{table*}


\begin{figure}[htb]
    \centering
    \includegraphics[width=\linewidth]{figures/ppl_comparison_memit_vs_ice.pdf}
    \caption{Comparison of the edit reliability of two editing methods across three tasks using \texttt{qwen2.5-7b}. A higher frequency indicates greater success in applied edits.}
    \label{fig:ppl_comparison_memit_vs_ice}
\end{figure}

\begin{figure}[htb]
    \centering
    \includegraphics[width=\linewidth]{figures/ppl_comparison_size.pdf}
    \caption{Comparison of the edit reliability across four tasks using models of varying sizes: \texttt{qwen2.5-7b-instruct}, \texttt{qwen2.5-32b-instruct-awq}, \texttt{qwen2.5-72b-instruct-awq}. A higher frequency indicates greater success in applied edits.}
    \label{fig:ppl_comparison_size}
\end{figure}

\begin{table*}[t!]
        \centering
        \resizebox{0.8\linewidth}{!}{
        \begin{tabular}{lcccc}
        \toprule
        \textbf{Metric} & \textbf{FactCheck} & \textbf{Analogy} & \textbf{Object Counting} \\
        \midrule

        \multicolumn{4}{l}{\textbf{Post-hoc}} \\
        \quad CC-SHAP & \textbf{\underline{0.541}} & \textbf{0.130} & \textbf{\underline{0.580}}   \\
                \quad Simulatability & 0.019 & 0.022 & 0.000   \\
                \quad Counterfact. Edits & 0.000 & 0.000 & 0.000   \\
                \midrule

        \multicolumn{4}{l}{\textbf{CoT}} \\
        \quad Early Answering & 0.484 & 0.356 & 0.478   \\
                \quad Filler Tokens & 0.459 & \textbf{\underline{0.715}} & 0.487   \\
                \quad Adding Mistakes & 0.468 & \underline{0.553} & 0.471   \\
                \quad Paraphrasing & 0.478 & \underline{0.683} & 0.487   \\
                \quad CC-SHAP & \textbf{0.493} & 0.246 & \textbf{\underline{0.580}}   \\
                \bottomrule
        \end{tabular}
        }
        \caption{The diagnosticity scores of each metric across three tasks using \texttt{qwen2.5-7b} as model and \textbf{MEMIT as knowledge editing method}. Bold numbers indicate the highest scores on each task across the two categories of faithfulness
metrics: post-hoc and CoT. Underlined numbers show the diagnosticity scores that are significantly higher than $0.5$ (Binomial, $p < 0.05$).}
        \label{tab:diagnosticity_memit}
    \end{table*}



\begin{table*}[t!]
        \centering
        \resizebox{0.8\linewidth}{!}{
        \begin{tabular}{lcccc}
        \toprule
        \textbf{Metric} & \textbf{FactCheck} & \textbf{Analogy} & \textbf{Object Counting} & \textbf{Multi-Hop} \\
        \midrule

        \multicolumn{4}{l}{\textbf{Post-hoc}} \\
        \quad CC-SHAP & \textbf{\underline{0.547}} & \textbf{0.233} & \textbf{0.414} & \textbf{\underline{0.575}}   \\
                \quad Simulatability & 0.083 & 0.010 & 0.018 & 0.000   \\
                \quad Counterfact. Edits & 0.001 & 0.000 & 0.000 & 0.000   \\
                \midrule

        \multicolumn{4}{l}{\textbf{CoT}} \\
        \quad Early Answering & 0.504 & \textbf{\underline{0.582}} & 0.504 & 0.515   \\
                \quad Filler Tokens & 0.475 & \underline{0.544} & 0.496 & 0.490   \\
                \quad Adding Mistakes & 0.478 & 0.506 & 0.473 & 0.470   \\
                \quad Paraphrasing & \textbf{\underline{0.565}} & 0.498 & \textbf{\underline{0.581}} & \textbf{0.520}   \\
                \quad CC-SHAP & 0.510 & 0.353 & 0.447 & 0.480   \\
                \bottomrule
        \end{tabular}
        }
        \caption{Diagnosticity scores of each metric across three tasks using \texttt{qwen2.5-7b} as the model and ICE as the knowledge editing method, \textbf{with model-generated explanations}. Bold numbers indicate the highest scores on each task across the two categories of faithfulness
metrics: post-hoc and CoT. Underlined numbers show the diagnosticity scores that are significantly higher than $0.5$ (Binomial, $p < 0.05$).}
        \label{tab:diagnosticity_model_generated}
    \end{table*}


\begin{table*}[t!]
        \centering
        \resizebox{\linewidth}{!}{
        \begin{tabular}{lcccc}
        \toprule
        \textbf{Metric} & \textbf{FactCheck} & \textbf{Analogy} & \textbf{Object Counting} & \textbf{Multi-Hop} \\
        \midrule
    \multicolumn{5}{c}{\textbf{7B}} \\ 
 \midrule \\ 
    Simulatability & 0.018 & 0.004 & 0.011 & 0.000   \\
                    Filler Tokens & 0.299 & 0.229 & 0.330 & 0.520   \\
                    Adding Mistakes & 0.243 & 0.504 & 0.481 & 0.435   \\
                    Paraphrasing & 0.509 & \underline{0.711} & \underline{0.741} & 0.525   \\
                
 \midrule 
\multicolumn{5}{c}{\textbf{32B}} \\ 
 \midrule \\ 
    Simulatability & 0.017 & 0.009 & 0.011 & 0.000   \\
                    Filler Tokens & 0.327  \textcolor{forestgreen}{(+0.03)}& 0.336  \textcolor{forestgreen}{(+0.11)}& 0.195  \textcolor{red}{(--0.14)}& 0.475  \textcolor{red}{(--0.05)}  \\
                    Adding Mistakes & 0.465  \textcolor{forestgreen}{(+0.22)}& 0.310  \textcolor{red}{(--0.19)}& 0.418  \textcolor{red}{(--0.06)}& 0.445  \textcolor{forestgreen}{(+0.01)}  \\
                    Paraphrasing & 0.522  \textcolor{forestgreen}{(+0.01)}& \underline{0.582}  \textcolor{red}{(--0.13)}& \underline{0.847}  \textcolor{forestgreen}{(+0.11)}& 0.515  \textcolor{red}{(--0.01)}  \\
                
 \midrule 
\multicolumn{5}{c}{\textbf{72B}} \\ 
 \midrule \\ 
    Simulatability & 0.020 & 0.009 & 0.010 & 0.000   \\
                    Filler Tokens & 0.442  \textcolor{forestgreen}{(+0.14)}& 0.320  \textcolor{forestgreen}{(+0.09)}& 0.084  \textcolor{red}{(--0.25)}& 0.475  \textcolor{red}{(--0.05)}  \\
                    Adding Mistakes & 0.310  \textcolor{forestgreen}{(+0.07)}& \underline{0.557}  \textcolor{forestgreen}{(+0.05)}& 0.473  \textcolor{red}{(--0.01)}& 0.505  \textcolor{forestgreen}{(+0.07)}  \\
                    Paraphrasing & \underline{0.754}  \textcolor{forestgreen}{(+0.24)}& 0.396  \textcolor{red}{(--0.31)}& \underline{0.936}  \textcolor{forestgreen}{(+0.20)}& \underline{0.615}  \textcolor{forestgreen}{(+0.09)}  \\
                
 \midrule 
    \bottomrule
        \end{tabular}
        }
        \caption{The change in diagnosticity scores across with respect to model size across four tasks.}
        \label{tab:scaling_diagnosticity}
    \end{table*}