\paragraph{Task} Adapted from BIG-bench~\citep{srivastava2023beyond}, this task tests object classification and counting. The model identifies how many objects in a list belong to a given category. To introduce causal interventions, we alter the model’s internal knowledge, swapping objects across categories while keeping the answer numerically identical but reasoning distinct. For example, in \textit{How many of "countertop," "grape," and "kiwifruit" are fruits?}, the correct answer is 2, since “countertop” is not a fruit. If the model is edited to classify “countertop” as a fruit and “grape” as furniture, the answer remains 2, but for different reasons.\\
\noindent \textbf{Dataset} We define five object categories, each with five types, as detailed in Table \ref{tab:categories} (Appendix \ref{appendix:dataset}). For each type, we collect 10 representative entities from WikiData, reserving 20\% for reassignment after model editing. We generate 1000 questions, equally split between two types: yes/no questions, asking if all or any items in a list belong to a given type, and number questions, asking how many items belong to a specific type. For both types, we randomly determine the number of items $k$ (between 3 and 6) and select a target type. For yes/no questions, we sample $k$ entities, ensuring that after knowledge editing, the number of entities of the target type remains unchanged. For number questions, we reassign one entity from the target type and one from other types to ensure consistency.
Dataset details are in Appendix~\ref{appendix:dataset}.