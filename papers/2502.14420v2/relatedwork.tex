\section{Related Work}
\textbf{Multimodal understanding}
Multimodal Large Language Models (MLLMs)~\cite{lu2024deepseek-vl, openflamingo, idefics, llava, llava1.5, wang2024qwen2, chen2024internvl, zhu2024comprehensive, ma2024janusflow, zhou2024transfusion, minigpt4, luo2024mono, chen2024internvl, blip-2, instructblip, chen2024expanding,karamcheti2024prismatic} have significantly advanced the field of multimodal understanding by integrating visual and linguistic information to achieve holistic scene comprehension. MLLMs have demonstrated excellent performance on tasks requiring cross-modal alignment, such as visual question answering (VQA), image captioning, and spatial reasoning. This success stems from their ability to map visual features to semantic representations through sophisticated adapter designs. However, current MLLMs lack a connection to the physical world, preventing them from interacting with environments and humans. This work aims to bridge this gap, enabling vision-language models to also act.

\textbf{Vision-langauge-action models in robot learning.} Vision-language-action models (VLAs) form a growing body of research that leverages pre-trained vision-language models (VLMs) as a backbone to enable both language comprehension and observational understanding. These methods typically fine-tune large pre-trained VLMs to predict robot actions~\cite{brohan2023rt-2, roboflamingo, leo3d, wen2024tinyvla, pertsch2025fast, [pi0, kim24openvla, diffusion-policy, zhu2024scalingdp, wang2024sparse-dp, prasad2024consistencypolicy, black2023training, black2023zero, dasari2024ingredients, lin2024datascalinglawsimitation, multimodal_diffusion_transformer, aloha_unleashed, uehara2024fine, uehara2024feedback,ding2024quar}. These methods have shown strong performance in both simulated and real-world tasks. However, existing VLA models have not demonstrated the ability to perform true multimodal understanding. Based on our experiments, we find that these models lack this capability. In contrast, our work proposes a unified approach that enables a single network to effectively handle both multimodal understanding and robot control.