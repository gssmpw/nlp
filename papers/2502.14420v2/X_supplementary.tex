\clearpage
\setcounter{page}{1}
\maketitlesupplementary


\section{Appendix}
\label{sec:appendix}


\subsection{Implementation Details}

\textbf{Data details.}
For visual-text data, we use llava-1.5~\cite{llava1.5} dataset for co-training. Following the data ratio mentioned in ECOT, we use set the ratio of visual-text data to robot data as 1:3. Using robot data, we evaluated our method on 25 real-world robot tasks, including long-horizon tasks with direct prompting. The data was randomly sampled from the LLaVA fine-tuning dataset.  We hypothesize that carefully curated data is crucial for mitigating spurious forgetting, a topic we plan to explore in future work.


\textbf{Training details.} We use Qwen2-VL-2B as our VLM backbone and the set of action head follows DiVLA~\cite{wen2024diffusionvla}.
We train our ChatVLA using a phased alignment training, as is discussed in Section \ref{sec:architecture_data}. In the first stage, we train our model on robot data, only activating the control expert and its corresponding action head. In the second stage, we co-train both visual-text data and robot data. Both control expert and understanding expert are trained using the same learning rate of 2e-5. 


\textbf{Evaluation metrics.}
The calculation for long-horizon tasks in Table ~\ref{tbl:taskscore_longhorizon} and Table ~\ref{tbl:taskscore_compositional} is as follows: One point is awarded for each successfully completed step. Once all steps are executed, the total score is computed. ``Avg. Len." represents the average success length, which is calculated by recording the lengths of successful sequences across multiple executions of long-horizon tasks and then computing its average. This provides a key indicator of the model's performance in handling long-horizon tasks based on the length of successful operation sequences.

For real robot cross-skill multi-tasking in Table~\ref{tbl:taskscore_other},``Avg." denotes the average success rate across all these tasks, calculated by dividing the total number of successful task executions by the total number of test executions for these tasks.

\textbf{Dataset description.}\label{sec:llava_dataset}
The visual-text pairs in LLaVA are composed of:
\begin{itemize}
    \item COCO: Contains images annotated with object labels, segmentations, and captions, supporting tasks like detection, segmentation, and image captioning.
    \item GQA: Focuses on visual reasoning, with images paired with questions that require logical interpretation of visual content.
    \item OCR-VQA: Combines visual question answering with OCR, emphasizing images that contain textual information needing extraction and reasoning.
    \item TextVQA: A visual question answering dataset where answers are derived from reading and understanding text embedded in images.
    \item VisualGenome: Provides detailed annotations of objects, attributes, relationships, and regions in images, supporting scene understanding and visual question answering.
\end{itemize}

\subsection{Robot task}

\begin{figure}[t]
   \centering
   \includegraphics[width=\linewidth]{picture/chatvla_direct_prompting.pdf}
   \caption{Settings of Long-horizon tasks with direct prompting}
   \label{fig:realpic_direct}
\end{figure}

\begin{figure*}[t]
   \centering
   \includegraphics[width=\linewidth]{picture/chatvla_highlevel.pdf}
   \caption{Settings of Long-horizon tasks with high-level planner}
   \label{fig:realpic_highlevel}
\end{figure*}


\begin{figure*}[t]
   \centering
   \includegraphics[width=\linewidth]{picture/chatvla_multitask.pdf}
   \caption{Settings of Cross-skill real robot multi-tasking.}
   \label{fig:realpic_other}
\end{figure*}


The embodied control performance of ChatVLA is evaluated on 25 real world manipulation tasks.

\textbf{Long-horizon tasks with direct prompting.} As is shown in  \ref{fig:realpic_direct}, all the tasks of this category are set under a real world toy scene.
\begin{itemize}
    \item Task 1: Sort toys. On the desktop, there are two toy animals with random positions and postures, as well as two building blocks. The robotic arm needs to place all the animals on the desktop in the box on the left and all the building blocks in the basket on the right.
    \item  Task 2: Stack cubes. The robotic arm first needs to pick up the orange building block from the right side and stack it on the yellow building block in the middle. Then, it needs to pick up the smallest pink square and stack it on the orange building block that was just stacked.  
    \item  Task 3: Place the toy in the drawer. The drawer is closed. Therefore, the robotic arm first needs to rotate and pull open the drawer. Then, it should pick up the toy on the table and place it into the drawer. Finally, close the gripper to shut the drawer.
    \item  Task 4: Clean building blocks to the box.
 The robotic arm needs to put the building blocks on the table into the box on the right side one by one until there are no more building blocks on the table. 
\end{itemize}

\textbf{Long-horizon tasks with high-level planner.} The settings are shown in  \ref{fig:realpic_highlevel}. 

\begin{itemize}
    \item Task 5: Move the orange block to the basket. The robotic arm needs to pick up the building block next to the doll on the table and place it into the box on the right side. 
    \item Task 6: Open the drawer. The robotic arm needs to rotate and grip the drawer handle, and then move parallel to the right to open the drawer. 
    \item Task 7: Put the toy into it. The robotic arm needs to pick up the toy in the middle and place it into the open drawer. 
    \item Task 8: Close the drawer. The robotic arm needs to close the gripper and gently push the open drawer to the left until the drawer is closed. 
\end{itemize}

\begin{itemize}
    \item Task 9: Move semi-circle building-block to basket. The robotic arm needs to pick up the semi-circular building block and place it into the basket on the right side. 
    \item Task 10:Move rectangle building-block to basket. The robotic arm needs to pick up the rectangle building block and place it into the basket on the right side. 
\end{itemize}


\begin{itemize}
    \item Task 11: Get the plate and place it on the tablecloth. The robotic arm needs to pick up the pink plate from the upper part of the shelf on the right side and then place it on the tablecloth at the center of the table.  
    \item Task 12: Flip the cup and place it on the tablecloth. The robotic arm needs to go to the bottom layer of the shelf on the right side, grip the mug, then turn it over and place it on the tablecloth in the middle of the table. 
    \item Task 13: Move the bread to the plate. The robotic arm needs to grip the bread from the bread basket on the left side and place it on the plate that was just taken down. 

\end{itemize}

\textbf{Cross-skill multi-tasking.} The settings are shown in  \ref{fig:realpic_other}.
\begin{itemize}
    \item  Task 14:Put the soap to the soap box. This is a bathroom task. The robotic arm needs to pick up the soap from the left side of the washbasin and place it into the soap dish on the right side of the washbasin. 
    \item  Task 15:Pick up the cup and hang it on the shelf. This is a bathroom task. The robotic arm needs to pick up the cup from the sink and hang it on the shelf in front of the mirror. 
    \item  Task 16:Pick up the tooth-paste and put it on the table. This is a bathroom task. The robotic arm needs to pick up the toothpaste from the sink and place it on the table.
    \item  Task 17:Remove the towel from the shelf. This is a bathroom task. The robotic arm needs to take down the towel hanging on the shelf and place it on another towel. 
    \item Task 18:Move the bread from the pot to the plate. This is a kitchen task. The robotic arm needs to pick up the bread from the pot and place it on the plate. 
    \item Task 19:Pick up the bread from the refrigerator. This is a kitchen task. The robotic arm needs to find the bread in the refrigerator and pick it up. 
    \item Task 20:Move the banana onto the plate. The robotic arm needs to pick up the banana at a random position and place it on the plate in the middle. 
    \item  Task 21: Move the bread to the empty plate. The robotic arm needs to ignore the distractions, grip the bread, and then find the empty one among the two plates in front of it, and put the bread into that plate. 
    \item  Task 22:Hang on the cup. The robotic arm needs to pick up the mug and hang it on the shelf on the left side. 
    \item Task 23:Move the tennis ball to the tennis can. The robotic arm needs to pick up the tennis ball and lift it up to place it into the tennis ball can. 
    \item Task 24:Stack the green cube onto the pink cube. The robotic arm needs to pick up the green cube on the right and stack it on top of the square on the left side. 
    \item  Task 25:Take away the lid of the box and put it on the table. The robotic arm needs to pick up the lid that is covering the box on the left side of the table and place the lid on the tabletop in the middle. 

\end{itemize}

