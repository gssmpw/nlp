\section{RELATED WORK}
The integration of Artificial Intelligence (AI) into spectrum sensing and wireless communication has catalyzed significant advancements, transforming the way networks operate and adapt. Among these contributions, the DeepSense framework [1] stands out as a pioneering solution, leveraging deep learning for real-time, wideband spectrum sensing. By employing in-the-loop deep learning, DeepSense enhances spectrum utilization by dynamically adapting to changing RF conditions, providing more accurate RF analysis and establishing a strong foundation for AI-augmented systems in wireless communication.

Building on this groundwork, subsequent studies have addressed the limitations of DeepSense while expanding its capabilities. For instance, "Stitching the Spectrum" [2] introduces semantic segmentation to create continuous representations of fragmented RF signals, improving interference identification and signal overlap resolution. Meanwhile, "DeepSweep" [3] employs a parallelized architecture using convolutional neural networks (CNNs) to enhance throughput in dynamic environments. These advancements collectively demonstrate the potential of AI to address the scalability and latency challenges of modern spectrum sensing.

Foundational studies such as "Big Data Goes Small" [4] have further highlighted the utility of deploying deep learning at the embedded systems level. By incorporating real-time feedback loops, this work showcases how ML-powered systems can optimize RF sensing in resource-constrained environments. Similarly, "DeepLab" [5], originally designed for semantic image segmentation, has been creatively adapted to spectrum analysis, demonstrating how advanced computer vision techniques can be applied to understand complex RF environments and improve spectrum management.

Parallel to these advancements, Open Radio Access Networks (ORAN) have emerged as a transformative architecture in wireless communication. ORAN’s modular and vendor-neutral design fosters interoperability and innovation, enabling rapid deployment of intelligent control applications like xApps and rApps. For example, the AERPAW testbed [6] explores how ORAN-based architectures can be adapted for real-world dynamic environments such as UAV communication. Furthermore, frameworks like xDevSM [7] streamline xApp development, showcasing their potential to enhance RAN adaptability and resource management.

Recent research has also focused on operational challenges within ORAN, such as optimizing virtual network functions (VNFs) and addressing vulnerabilities associated with AI/ML integration. "Optimizing Virtual Network Function Splitting in Open-RAN Environments" [9] presents strategies for improving resource efficiency, while "Misconfiguration in O-RAN: Analysis of the Impact of AI/ML" [10] emphasizes the need for robust testing frameworks to ensure network reliability. Together, these works highlight the importance of combining AI/ML-driven solutions with ORAN’s flexible architecture to create resilient, scalable, and efficient wireless networks.