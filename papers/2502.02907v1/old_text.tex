
\begin{theorem}
    Let $I_\mathrm{symm}\in\mathbb{R}^{N\times N}$ and $I_\mathrm{blob}\in\mathbb{R}^{N\times N}$ be two images such that $I_\mathrm{symm}$ exhibits vertical reflective symmetry and $I_\mathrm{blob}$ is non-symmetric. Also let $I_\mathrm{sum} = I_\mathrm{symm} + I_\mathrm{blob}$, $F_\mathrm{sum}=\mathcal{F}(I_\mathrm{sum})$, and $A_\mathrm{sum}=|F_\mathrm{sum}|$. Then, $A_\mathrm{sum}$ can be written as

    \begin{equation}
        A_\mathrm{sum} = B_\mathrm{symm} + B_\mathrm{noise}
    \end{equation}

    where $B_\mathrm{symm}$ is an image which exhibits vertical reflective symmetry and $B_\mathrm{noise}$ is a non-symmetric image.
\end{theorem}

\begin{proof}
    From linearity of the DFT (Lemma \ref{lemma:dft_linearity}), we have:

    \begin{align}
        A_\mathrm{sum} &= |\mathcal{F}(I_\mathrm{symm} + I_\mathrm{blob})|\\
        &= |F_\mathrm{symm} + F_\mathrm{blob}|
    \end{align}

    where $F_\mathrm{symm} = \mathcal{F}(I_\mathrm{symm})$ and $F_\mathrm{blob} = \mathcal{F}(I_\mathrm{blob})$. By squaring both sides of the equations, we have:

    \begin{align}
        A^2_\mathrm{sum} &= |F_\mathrm{symm} + F_\mathrm{blob}|^2\\
        &= (F_\mathrm{symm} + F_\mathrm{blob})(F^*_\mathrm{symm} + F^*_\mathrm{blob})\\
        &= F_\mathrm{symm}F^*_\mathrm{symm} + F_\mathrm{blob}F^*_\mathrm{blob} + F_\mathrm{symm}F^*_\mathrm{blob} + F_\mathrm{blob}F^*_\mathrm{symm}\\
        &= |F_\mathrm{symm}|^2 + |F_\mathrm{blob}|^2 + 2\mathrm{Re(F_\mathrm{symm} F_\mathrm{blob})}\\
    \end{align}
\end{proof}

where we used the fact that $F_\mathrm{symm}F^*_\mathrm{symm} = |F_\mathrm{symm}|^2$, $F_\mathrm{blob}F^*_\mathrm{blob} = |F_\mathrm{blob}|^2$, and $F_\mathrm{symm}F^*_\mathrm{blob} + F_\mathrm{blob}F^*_\mathrm{symm}=2\mathrm{Re}(F_\mathrm{symm}F^*_\mathrm{blob})$ since the two addends are complex conjugates of each other.




\begin{lemma}
\label{lemma:F_lack_refl_symm}
    Let $I\in \mathbb{R}^{N\times N}$ be an image which lacks vertical reflective symmetry. Then, $F = \mathcal{F}(I)$ also lacks vertical reflective symmetry.
\end{lemma}

\begin{proof}
    If $I$ lacks vertical reflective symmetry, then $I(m,n) \neq I(m,N-n+1)$, for some $(m,n)$ pairs. Assume, for the sake of contradiction, that $F$ exhibits reflective symmetry, i.e., $F(x,y) = F(x,N-y+1)$, for all $(x,y)$ pairs. Then, using Equation \ref{eq:F_xy},  we can write:

    \begin{align}
        F(x,y) &= F(x,N-y+1)\\
        \sum_{m=1}^{M} \sum_{n=1}^{N}I(m,n) \, \mathrm{exp}\left[ -2\pi i \left( \frac{xm}{M} + \frac{yn}{N}\right) \right] &= \sum_{m=1}^{M} \sum_{n=1}^{N}I(m,N-n+1) \, \mathrm{exp}\left[ -2\pi i \left( \frac{xm}{M} + \frac{(N-y+1)n}{N}\right) \right] \label{eq:F_xy_eq_F_xy}\\
    \end{align}

   However, we can see that 
    
    
    For Equation \ref{eq:F_xy_eq_F_xy} to hold for all $(x,y)$ pairs, the summands must be equal, i.e.:

    \begin{equation}
        I(m,n) \, \mathrm{exp}\left[ -2\pi i \left( \frac{xm}{M} + \frac{yn}{N}\right) \right] = I(m,n) \, \mathrm{exp}\left[ -2\pi i \left( \frac{xm}{M} + \frac{(N-y+1)n}{N}\right) \right]
    \end{equation}

    Taking the inverse DFT of both sides, we obtain:

    \begin{equation}
    \label{eq:I_mn_eq_I_mn}
        I(m,n) = I(m,N-n+1)
    \end{equation}

    However, Equation \ref{eq:I_mn_eq_I_mn} contradicts our initial assumption.
\end{proof}

\begin{corollary}
\label{cor:F_lack_refl_symm}
     If an image $I\in \mathbb{R}^{N\times N}$ lacks vertical reflective symmetry, then $A = |\mathcal{F}(I)|$ also lacks vertical reflective symmetry.
\end{corollary}



\begin{definition}

    \begin{equation}
        \mathcal{O}^{\mathrm{signal}}_\lambda(\phi) = \{ \mathbf{u}\in \mathcal{O}_\lambda(\phi) \; | \; o_\lambda([-u,v]^\top,-\phi)=o_\lambda([u,v]^\top,\phi) \}
    \end{equation}

    \begin{equation}
        \mathcal{O}^{\mathrm{noise}}_\lambda(\phi) = \{ \mathbf{u}\in \mathcal{O}_\lambda(\phi) \; | \; \mathbf{u} \notin \mathcal{O}^{\mathrm{signal}}_\lambda(\phi) \}
    \end{equation}

    where $\mathcal{O}^{\mathrm{signal}}_\lambda(\phi) \cap \mathcal{O}^{\mathrm{noise}}_\lambda(\phi) = \emptyset$.
\end{definition}











\begin{definition}
    Given the surface $\Omega$ and a point $\mathbf{p}\in\mathbb{R}^3$, the surface occupancy function $f_\mathrm{sur}:\mathbb{R}^3 \rightarrow \mathbb{R}$ is:
    
    \begin{equation}
        f_\mathrm{sur}(\mathbf{p}) = \begin{cases}
    1 & \text{if } \mathbf{p} \in \Omega \\
    0   & \text{otherwise }
  \end{cases}
    \end{equation}
\end{definition}

\begin{definition}
    \label{def:f_rot}
    Given the volume occupancy function $f_\mathrm{vol}$, the rotational occupancy function $f_\mathrm{rot}:\mathbb{R}^3 \rightarrow \mathbb{R}$ is:

    \begin{equation}
        f_\mathrm{rot}(\rho, z) = \int_0^{2\pi} f_\mathrm{sur}(\rho,\phi,z)\,d\phi
    \end{equation}

    where $\rho$, $\phi$, and $z$ are the radial, longitudinal, and height coordinates, respectively, associated with a cylindrical coordinate system.
\end{definition}

\begin{definition}
The set of points located on the circle with radius $\rho$, centered around $\mathbf{a}$, and at a height $z$, is:

    \begin{equation}
        \mathcal{C}_{\rho,z}=\{ \mathbf{p}(\rho \, \phi, z) \; | \; 0 \leq \phi < 2\pi,\, \rho=c_\rho,\, z=c_z \}
    \end{equation}

    where $c_\rho$ and $c_z$ are constants.
\end{definition}

\begin{definition}
    The projection of the set $\mathcal{C}_{\rho,z}$, according to $T$ is the ellipse $\mathcal{E}_{\rho,z}$ defined as:

    \begin{equation}
        \mathcal{E}_{\rho,z}=\{ T\mathbf{p}, \, \mathbf{p}(\rho \, \phi, z) \; | \; 0 \leq \phi < 2\pi,\, \rho=c_\rho,\, z=c_z \}
    \end{equation}
\end{definition}

\begin{lemma}
\label{lemma:const_f_in_C}
Let $\mathbf{p}_1$ and $\mathbf{p}_2$ be two points in $\mathcal{C}_{\rho,z}$. Then:

    \begin{equation}
        f_\mathrm{rot}(\mathbf{p}_1)=f_\mathrm{rot}(\mathbf{p}_2)=c,\, \forall \mathbf{p}_1,\mathbf{p}_1 \in \mathcal{C}_{\rho,z}
    \end{equation}

    where $c$ is a constant.
\end{lemma}

\begin{proof}
    Lemma \ref{lemma:const_f_in_C} follows directly from Definition \ref{def:f_rot}.
\end{proof}

\begin{corollary}
    Let $\mathbf{p}_1$ and $\mathbf{p}_2$ be two points in $\mathcal{E}_{\rho,z}$. Then:
    \begin{equation}
        f_\mathrm{rot}(\mathbf{p}_1)=f_\mathrm{rot}(\mathbf{p}_2)=c,\, \forall \mathbf{p}_1,\mathbf{p}_1 \in \mathcal{E}_{\rho,z}
    \end{equation}

    where $c$ is a constant.
\end{corollary}

\begin{definition}
Given the camera view defined by the transformation $T$, the corresponding silhouette observation $\mathcal{S}_T$ is defined as:

    \begin{equation}
        \mathcal{S}_T = \{ \mathbf{u}\in \mathbb{R}^2 \; | \; \mathbf{u}=T\mathbf{p},\, \mathbf{p}\in \mathcal{V} \}
    \end{equation}    
\end{definition}

\begin{definition}
    Given a silhouette observation $\mathcal{S}_T$, the silhouette occupancy function $f_\mathrm{sil}:\mathbb{R}^2 \rightarrow \mathbb{R}$ is:

    \begin{equation}
        f^T_\mathrm{sil}(\mathbf{u}) = \begin{cases}
    1 & \text{if } \mathbf{u} \in \mathcal{S}_T \\
    0   & \text{otherwise }
  \end{cases}
    \end{equation}
\end{definition}

\begin{definition}
For a set of camera views $T(\phi),\,0\leq \phi < 2\pi$, the stacked-silhouette intensity function, $f_\mathrm{stacked}(\mathbf{u}):\mathbb{R}^2 \rightarrow \mathbb{R}$, can be written as:

    \begin{equation}
        f_\mathrm{stacked}(\mathbf{u}) = \int_0^{2\pi} f^{T(\phi)}_\mathrm{sil}(\mathbf{u})\, d\phi
    \end{equation}
\end{definition}

\begin{lemma}
    Let $\mathbf{u}=T\mathbf{p}$. Then,
    
    \begin{equation}
        f_\mathrm{stacked}(\mathbf{u})=f_\mathrm{rot}(\mathbf{p}),\, \forall \mathbf{p} \in \mathcal{V}
    \end{equation}
\end{lemma}

\begin{proof}
    
\end{proof}