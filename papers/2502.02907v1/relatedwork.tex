\section{Related Work}
A variety of image-based methods have been proposed and effectively employed for small-body pole estimation. These can be subdivided into three categories, based on the image resolution of interest: (1) lightcurve-based methods, for unresolved-object imagery, (2) silhouette-based methods, for low- and medium-resolution images of the object, and (3) landmark- or pattern-based methods, for high-resolution images of the object.

\subsubsection{Pole Estimation for Unresolved Objects}
Lightcurve analysis has been extensively used to estimate the pole, shape, and rotation periods of small celestial bodies, typically through ground-based observations\cite{kaasalainen2001optimization,chng2022globally}. The lightcurve is the evolution of total brightness for an imaged object as it is observed over time. The key principle is that lightcurve signatures depend on the shape and rotational motion of the associated object. Such techniques are typically designed for unresolved-object observations, where the irregular shape is not directly observable, and require modeling surface-reflectance properties to constrain the optimization problem. This approach can lead to multiple shape and pole hypotheses, which cannot be disambiguated with limited data or unfavorable observation geometries. Furthermore, lightcurve-based estimates are sensitive to surface mismodeling, e.g., due to heterogeneities in albedo and reflectance properties.

\subsubsection{Pole Estimation for Low and Medium-resolution Objects}

Previous work investigated the use of silhouettes for pole estimation, suitable for lower-resolution imagery. Bandyopadhyay et al. propose a Pole-from-Silhouette technique where multiple pole-orientation hypotheses are evaluated through a grid search by matching predicted and observed silhouettes\cite{bandyopadhyay2021light}. For each pole hypothesis, and assuming knowledge of the rotation rate, silhouette predictions are computed by firstly reconstructing the 3D visual hull of the target object, using a Shape-from-Silhouette method, and then reprojecting the visual hull's silhouette onto the camera plane. This technique has been demonstrated as part of an autonomous-navigation pipeline for small-body exploration\cite{nesnas2021autonomous}. However, said approach is computationally expensive as it requires 3D-shape reconstruction for each pole hypothesis through the grid search, and relies on the assumption that the object's center-of-mass location relative to the camera is known.

\subsubsection{Pole Estimation for High-resolution Objects}

For sufficiently resolved target images, the pole orientation can be estimated by tracking surface landmarks across multiple images. The underlying principle here is that landmark tracks projected onto the camera plane depend on the orientation between the camera reference frame and the pole that landmarks rotate with respect to. Computing landmark tracks requires detecting and matching the same set of landmarks across multiple images, which can be challenging when surface lighting conditions and camera poses evolve across observations. For small-body missions, state-of-the-practice techniques rely on estimating surface-topography models, typically using a method known as Stereophotoclinometry (SPC)\cite{palmer2022practical,adam2023stereophotoclinometry}, which are then used to compute the appearance of landmarks and perform landmark tracking. To date, techniques such as SPC rely on complex ground-based operations and are not suitable for autonomous characterization.

Autonomous pole-estimation approaches have been proposed, such as the use of visual-feature-tracking algorithms---e.g., SIFT \cite{lowe2004distinctive}---for model-free landmark tracking\cite{panicucci2023vision,villa2020optical}. One limitation is that feature tracks are subject to drift when tracked across multiple images, especially for objects characterized by challenging lighting conditions and irregular surface topography\cite{morrell2020autonomous}, which can reduce the accuracy of the associated pole estimates. Further, tracking surface features requires high-resolution images, which are typically unavailable until the spacecraft is in close proximity to the body. In addition to visual features, \textit{circle-of-latitude} patterns, i.e., elliptical ``streaks" produced by stacking consecutive images over time, have been proposed as image patterns to estimate the pole orientation\cite{kuppa2024initial,christian2024pole}. With such methods, pole estimates can be sensitive to the quality of the extracted circle-of-latitude patterns, which in turn depend on image-alignment errors and surface appearance.