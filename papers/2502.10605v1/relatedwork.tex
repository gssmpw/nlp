\section{Related work}
Our model is closest to a design-based perspective on a non-classical measurement error model with validation data, wherein we choose which datapoints to query for validation data. Typical distributional conditions for nonclassical measurement error \citep{schennach2016recent} are generally inapplicable to the relationship between ground-truth annotation and pretrained decoders, our motivating application. 

The most related recent work is that of \cite{naoki2023dsl,zrnic2024active}, which leverages the fact that sampling probabilities for data annotation are known to obtain doubly-robust estimation via causal inference. These works generally address non-causal estimands such as mean estimation and M-estimation (therefore without discussion of treatment effect estimation). 
Our batched adaptive allocation protocol is closest to \citep{hahn2011adaptive}, which studied a two-stage procedure for estimating the ATE, although the goals of estimation are very different. They consider a proportional asymptotic and a two-stage procedure, and show asymptotic equivalence of their batched adaptive estimator to the optimal asymptotic variance. However, \citep{hahn2011adaptive} and other papers focus on allocating treatments, while we allocate probability of revealing the outcome for a datapoint (i.e via expert annotation); the optimization problem is different. We additionally focus on a double-machine learning estimator. The recent work of \citep{li2024double} is most relevant, though our set up is different and we further characterize the closed-form of the optimal labelling probabilities for our setting, which can be of independent interest. 




Regarding the use of such auxiliary information in causal inference, many recent works have studied the use of surrogate or proxy information. Although we use ``pretrained decoders" in a similar sense as colloquial notions of surrogates or proxies, recent advances in surrogate and proxy variables in causal inference have coalesced around certain models that are different from our setting \citep{athey2019surrogate,kallus2024role,naoki2023dsl}. However, in much of the surrogate literature, surrogates measure an outcome that is impossible to measure at the time of analysis. The canonical example in \cite{athey2019surrogate} studies the long-term intervention effects of job training on lifetime earnings, by using only short-term outcomes (surrogates) such as yearly earnings. In this regime, the ground truth cannot be obtained at the time of analysis. In this paper, we focus a different regime where obtaining the ground truth from expert data annotators is feasible but budget-binding. 
We do leverage that we can design the sampling probabilities of outcome observation (ground-truth annotation) or missingness for doubly-robust estimation, like some methods in the surrogate literature or data combination \citep{yang2020combining,kallus2024role}. But we treat the underlying setting as a single unconfounded dataset with missingness. 
The different setting of proximal causal inference \citep{tchetgen2024introduction,cui2024semiparametric} seeks proxy outcomes/treatments that are informative of unobserved confounders; we assume unconfoundedness holds. Recently, \citep{chen2024proximal} study the ``design-based supervised learning" perspective of \citep{naoki2023dsl} specifically for proxies for unobserved confounding.



Many exciting recent works study adaptive experimentation under different desiderata, such as full adaptivity, in-sample decision regret or finite-sample, non-asymptotic guarantees \citep{gao2019batched,zhao2023adaptive,cook2024semiparametric}. Such designs are closely related to covariate-adaptive-randomization; the recent work of  \citep{shiusing} studies delayed outcomes. We outline how our approach is a good fit for our motivating data annotation setting. Full-adaptivity is less relevant in our setting with ground-truth annotation from human experts, due to distributed-computing-type issues with random times of annotation completion. But standard tools such as the martingale CLT can be applied to extend our theoretical results to full adaptivity. Additionally, many recent works primarily focus on the different problem of treatment allocation for ATE estimation. In-sample regret is less relevant for our setting of data annotation, which is a pure-exploration problem.