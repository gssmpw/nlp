\section{Related Work}
\label{related}

Code generation benchmarks have traditionally focused on single-turn generation from natural language problem descriptions~\citep{codex,DBLP:journals/corr/abs-2108-07732,li2022competition,zhuo2024bigcodebench}.
More recently, LLM performance has improved through interactions with external tools, such as interpreters for compiling, executing test cases, and verbal feedback, resulting in more accurate outputs~\citep{shinn2023reflexion,madaan2024self,chen2024teaching,olausson2024is}. 
This shift has led to the development of multi-turn benchmarks like InterCode~\citep{yang2023intercode} and MINT~\citep{wang2024mint}.

\definecolor{g9}{gray}{0.9}
\definecolor{g8}{gray}{0.8}
\definecolor{g7}{gray}{0.7}
\definecolor{g6}{gray}{0.6}
\definecolor{g5}{gray}{0.5}
\definecolor{g4}{gray}{0.4}
\definecolor{g3}{gray}{0.3}
\definecolor{g2}{gray}{0.2}
\definecolor{g1}{gray}{0.1}




\begin{wraptable}{r}{0.50\textwidth}
% \begin{wraptable}{r}{0.60\textwidth}
% \begin{table}[h!]
\centering
\caption{Feedback combinations ($\Omega$; \S\ref{convcodeworld:combination}) across InterCode, MINT and \ours, constructed by different feedback types (\S\ref{convcodeworld:categorization}). }
    % \small
    \scriptsize
    \begin{tabular}{cccccccccccccccc}
        \thickhline
         % \textbf{Feedback Types ($\downarrow$)} & \multicolumn{9}{c}{Feedback Combination} \\ 
         $\Omega$ & InterCode & MINT & \ours \\\hline
         % $\langle \phi, \phi, \phi \rangle$ 
         $\langle f_c, \phi, \phi \rangle$ & \xmark  & \xmark & \cmark \\
         $\langle f_c, f_e, \phi \rangle$  & \xmark & \cmark & \cmark \\%\hdashline
         $\langle f_c, f_e^*, \phi \rangle$ & \cmark & \xmark & \cmark \\%\hdashline
         $\langle f_c, \phi, f_v \rangle$ & \xmark & \xmark & \cmark \\
         $\langle f_c, f_e, f_v \rangle$ & \xmark & \cmark & \cmark \\%\hdashline
         $\langle f_c, f_e^*, f_v \rangle$& \xmark & \xmark & \cmark \\%\hdashline
         $\langle f_c, \phi, f_v^* \rangle$ & \xmark & \xmark & \cmark \\
         $\langle f_c, f_e, f_v^* \rangle$ & \xmark & \cmark & \cmark \\
         $\langle f_c, f_e^*, f_v^* \rangle$ & \xmark & \xmark & \cmark \\
         \thickhline 
    \end{tabular}
    \normalsize
    \label{tab:partial_observability}
% \end{table}
\end{wraptable}

% \begin{table}[t!]
%     \centering
%     \caption{dddd}
%     \tiny
%     \begin{tabular}{lccccccccccccccc}
%         \thickhline
%          % \textbf{Feedback Types ($\downarrow$)} & \multicolumn{9}{c}{Feedback Combination} \\ 
%          \multicolumn{10}{c}{$o \in \Omega$} \\
%          $\phi$ & $\langle f_c \rangle$ & $\langle f_c, f_v \rangle$ & $\langle f_c, f_e \rangle$ & $\langle f_c, f_e^* \rangle$ & $\langle f_c, f_e, f_v \rangle$ & $\langle f_c, f_e^*, f_v \rangle$ & $\langle f_c, f_v^* \rangle$ & $\langle f_c, f_e, f_v^* \rangle$ & $\langle f_c, f_e^*, f_v^* \rangle$\\
%          & \cellcolor{g9} & \cellcolor{g8} & \cellcolor{g7} & \cellcolor{g6} & \cellcolor{g5} & \cellcolor{g4} & \cellcolor{g3} & \cellcolor{g2} & \cellcolor{g1} \\ 
%          \\

%         \thickhline
%     \end{tabular}
%     \normalsize
%     \label{tab:partial_observability}
% \end{table}

% \begin{table*}[t!]
%     \centering
%     \caption{dddd}
%     \scriptsize
%     \begin{tabular}{lccccccccccccccc}
%         \thickhline
%          \textbf{Feedback Types ($\downarrow$)} & \multicolumn{9}{c}{Feedback Combination} \\ 
%          Observation $o \in \Omega$ & $\phi$ & $\langle f_c \rangle$ & $\langle f_c, f_{n,n} \rangle$\\
%          \textbf{$b_t(s_t)$} & & 
%          \cellcolor{g9} & \cellcolor{g8} & \cellcolor{g7} & \cellcolor{g6} & \cellcolor{g5} & \cellcolor{g4} & \cellcolor{g3} & \cellcolor{g2} & \cellcolor{g1}
%          \\\hline
%          Compilation Feedback & \xmark &  \cmark &  \cmark &  \cmark &  \cmark &  \cmark &  \cmark &  \cmark &  \cmark &  \cmark \\
%          Execution Feedback & \xmark &  \xmark & \xmark &  \cmark &  \cmark  &  \cmark &  \cmark & \xmark &  \cmark &  \cmark \\ 
%          Test Coverage & & & &\scriptsize  part.&\scriptsize full &\scriptsize  part.&\scriptsize full &  &\scriptsize  part.&\scriptsize full \\ 
%          Natural Language Feedback & \xmark & \xmark & \cmark & \xmark & \xmark  &  \cmark &  \cmark &  \cmark &  \cmark &  \cmark \\ 
%          Expertise & & & \scriptsize novice & & & \scriptsize novice & \scriptsize novice & \scriptsize expert & \scriptsize expert & \scriptsize expert \\
%         \thickhline
%     \end{tabular}
%     \normalsize
%     \label{tab:partial_observability}
% \end{table*}

% While these benchmarks advance real-world coding simulation, 
However, existing multi-turn benchmarks remain limited in feedback diversity. InterCode focuses on compilation and partial execution feedback but lacks full test coverage and verbal feedback. MINT generates verbal feedback via GPT-4, reducing human-in-the-loop evaluation costs, but its feedback scope is narrow and requires costly LLM calls for each evaluation.



% Code generation benchmark has focused on one-shot generation scenario from problem description in natural language~\citep{codex,DBLP:journals/corr/abs-2108-07732,li2022competition,zhuo2024bigcodebench}. Meanwhile, 
% LLM code generation capabilities are improved through interactions with external tools, such as interpreters for compiling and executing test cases, and verbal feedback, leading to more accurate and reliable output~\citep{shinn2023reflexion,madaan2024self,chen2024teaching}. 
% This has motivated benchmark efforts, such as InterCode~\citep{yang2023intercode} and \textsc{Mint}~\citep{wang2024mint}, evaluating LLMs' multi-turn interaction capabilities.

% While both InterCode and MINT advance the simulation of real-world coding scenarios, they remain limited in scope. 
% InterCode focuses only on compilation and execution feedback with partial test coverage, but lacks other critical forms of feedback such as full test coverage and verbal guidance. 
% MINT employs GPT-4 to generate verbal feedback, reducing the need for human-in-the-loop evaluations, yet its feedback coverage remains narrow.  
% Consequently, neither fully captures the diversity of feedback combinations that occur in real-world development. Furthermore, MINT requires calling the LLM for each evaluation, making it potentially costly. 


 Our study presents (a) \ours, a reproducible environment with \textbf{nine unique feedback combinations} (Table~\ref{tab:partial_observability}), and (b) \oursstatic, a \textbf{cost-effective benchmark} that maintains high correlation with live environment by using pre-generated logs, eliminating the need for costly LLM calls to provide verbal feedback. {We further discuss the distinction of \ours in Appendix~\ref{appendix:our_distinction}.}

 