\section{Background and Related Work}
\label{Section: Background and related work}

This section introduces the randomness principle of jitters and metastability, and presents the related work.

\subsection{ Oscillation Jitter }
\label{Section: Oscillation Jitter}
Jitter is a typical entropy source induced by external random physical processes, such as thermal noise, scattering noise, power supply fluctuation, and environmental variation, in ring oscillators (ROs).
It manifests as uncertain phase noise in the frequency domain. Typically, in the jitter extraction process shown in Figure~\ref{Figure 2}(a), a low-frequency clock signal samples the high-frequency oscillation of entropy sources within a flip-flop to generate random bits. The relation between the phase noise and the oscillation ring order is expressed as \cite{Jitter}:
%Jitter is a typical entropy source that manifests as uncertain phase noise in the frequency domain, as shown in Figure 2(a). It is mainly induced by external random physical processes, such as thermal noise, scattering noise, power supply fluctuation, and environmental variation, in oscillation signals. Typically, in the process of jitter extraction, a low-frequency clock signal samples the high-frequency oscillation of entropy sources within a flip-flop to generate random bits. The relation between the phase noise and the oscillation ring order is expressed as \cite{Jitter}:
\begin{equation}
L_{min}\left \{ \Delta f \right \} =\frac{8N}{3\eta }\cdot \frac{KT}{P}\cdot \left ( \frac{V_{DD} }{V} +\frac{V_{DD} }{IR}  \right )\cdot \left ( \frac{f_{0}}{\Delta f}  \right )  ^{2},
\end{equation}
where $K$, $T$, $\eta $, $V_{DD}$, $V$, $I$, and $R$ are constants, $f_{0}$ is the frequency of the ring, $\Delta f$ denotes the offset frequency, $P$ is the power consumption, and $N$ is the order of the ring. 

Apparently, increasing the order $N$ can amplify phase noise $L_{min}$. Nevertheless, it will also reduce frequency $f_{0}$ and throughput \cite{Cui}. Thus, many related works focus on the trade-off schemes to improve performance. For instance, Cui et al. \cite{Cui} provided a multi-stage feedback RO that increased both $N$ and $f_{0}$. Lu et al. \cite{DAC2023} designed a multiphase sampler TRNG architecture, which increases the throughput with low resource overhead. However, their basic entropy units exhibit inherent randomness insufficiency, which limits the efficiency improvement of generating random sequences. 
%However, its basic entropy unit has inherent randomness deficiencies, which limits the improvement of the efficiency of generating random sequences.


\begin{figure}[htbp]
% \vspace{-8pt}
    \centering
    \includegraphics[scale=0.52]{Figure/fig2.pdf}
    % \vspace{-7pt}
    \caption{(a) Randomness extraction of jitters. (b) Randomness extraction of metastability.}
    \label{Figure 2}
     \vspace{-10pt}
\end{figure}
 \subsection{Sampling Metastability}
In addition to jitter, metastability, which refers to the unpredictable random output of flip-flops, serves as another ideal entropy source for true random number generation \cite{JSSC2008, CHES2011}. As shown in scenarios 2 and 3 of Figure~\ref{Figure 2}(b), when the flip-flop samples unstable intermediate signals within the set/hold time,  unpredictable random bits are generated due to timing violations associated with metastability. 

Majzoobi et al. \cite{CHES2011} proved the probability of output settling onto ‘1’ can be accurately modeled by the Gaussian cumulative distribution function and central limit theorem, as expressed in Equation~\eqref{equation 2}: 
\begin{equation}
P(out=1)=Q\left ( \frac{\Delta  }{\sigma }  \right ) ,\\
Q(x)=\frac{1}{\sqrt{2\pi } }\int_{x}^{\infty } e^{\frac{-u^{2} }{2} } du ,
\label{equation 2}
\end{equation}
where $\Delta$ denotes the time difference between the 
sampling edge and the moment transition occurs, and $\sigma$ is proportional to the width of the setup/hold time window. 

Based on Equation~\eqref{equation 2}, reducing $\Delta$ can force the flip-flop into metastability by ensuring that sampling points fall between points 2 and 3 as much as possible, as depicted in Figure~\ref{Figure 2}(b). Inspired by this, Peng et al. \cite{TODAES2023} proposed RO-driven shift registers that align the delay of symmetric inputs to produce unpredictable equal probability outputs. Sala et al. \cite{TCAS1_2022} proposed latched-XOR cells configured by multiple excitation signals to achieve randomness improvement with low hardware area. Nevertheless, both their throughput is constrained due to limitations in entropy collection rate. 


% \vspace{-4pt}