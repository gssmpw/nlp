\section{Related Work}
Below, we first discuss the importance of young adulthood and career pursuits. Then, we introduce interventions designed to support these processes, focusing on the letter-exchange exercise and the associated accessibility and cognitive challenges in engaging with them without professional or structured support. Finally, we highlight the potential of LLM conversational agents in enhancing self-guided interventions for career exploration.


\subsection{Young Adulthood and Career Pursuits}

Young adulthood (ages 18 to 29) is a pivotal life stage that marks the beginning of adulthood, where an individual transitions into complete independence and confronts significant life choices ____. During this period, young adults encounter a variety of life choices, ranging from educational decisions to financial security, romantic relationships, and community responsibilities ____. Among these, career pursuits are regarded as one of the most critical decisions that young adults have to make ____.

During the transition to adulthood, setting career objectives plays a crucial role in shaping young adults' identities and guiding the direction of their future lives ____. The career decisions made at this developmental stage are known to have a lasting impact on their future lives ____. Also, career choices have a decisive influence on future well-being and overall life satisfaction ____.
To cultivate a rich and stable identity, it is essential for young adults to explore various career paths, then learn to navigate the challenges that arise during their career pursuits, and finally commit to their chosen profession ____. 

Despite the importance of career pursuits, many young adults often struggle to clearly envision their future due to a lack of information, knowledge, and experience required to make informed career decisions ____. Also, they often feel that their future still remains clouded in ambiguity and uncertainty ____. These vague visions of the future are known to significantly complicate young adults’ career development ____. As a result, young adults who have not yet defined their career paths frequently experience substantial stress ____, leading to heightened anxiety and reduced self-efficacy ____.

To address this, a range of interventions have been developed to support young adults in their career pursuits ____. These include providing career-related information ____, guiding career exploration ____, training job search skills ____, and alleviating negative emotions ____. Although these efforts cover important areas such as informational resources, skills development, and mental health support ____, they often overlook the importance of fostering \textit{future orientation}—a critical component in career guidance for young adults. Future orientation, also referred to as a future-oriented mindset, is an individual's ability to think about, connect with, and plan for their future ____. In the context of career exploration, it entails envisioning long-term goals, identifying the steps required to achieve them, and aligning present actions with desired outcomes. This mindset is vital for young adults to navigate their career paths, as it enables them to transform existing knowledge and skills into meaningful progress toward their aspirations ____.

\subsection{The Letter-Exchange Exercise}
In this context, the \textit{letter-writing interventions}, a series of self-guided letter-writing exercises directed towards one's future self, have garnered significant interest as an effective means of cultivating future orientation ____. While these interventions have shown promise in inducing future-oriented mindsets, the specific process varies across studies, including differences in time frames (e.g., 3 years vs. 20 years) and letter directionality (e.g., letter to the future self vs. from the future self) ____. 

Among such variations in letter-writing interventions, we specifically focused on the \textit{letter-exchange exercise}, a recently enhanced version of the traditional one-way letter-writing interventions, which expanded it to a bidirectional exchange of letters with one's future self ____. This exercise was devised to further promote future orientation in young adults to support their career exploration. The exercise consists of two stages, each comprising two sub-activities: (i) the \textbf{Send Session}, where individuals first create a realistic profile of their future selves and then write letters to their future selves based on the created profile, and (ii) the \textbf{Reply Session}, where individuals first imagine their future selves and then write letters back to their present selves from this future perspective. To maximize the effectiveness of the exercise, it is recommended that the future self be positioned at a distance sufficient to represent the next step in career development yet close enough to be vividly imagined (e.g., 3 years) ____.

The letter-exchange exercise proved its effectiveness in supporting young adults' career exploration by encouraging future orientation ____. The bidirectional process facilitates young adults to simultaneously imagine their future selves from the present perspective while reflecting on their current selves from a future viewpoint. As a result, young adults gain clearer visions of their future and strengthen their connection to their future selves, thereby developing future-oriented mindsets. Subsequent studies have further validated its effectiveness by applying the exercise in a range of contexts ____.

Despite its effectiveness in supporting young adults’ career pursuits, the letter-exchange exercise has limitations in facilitating participant engagement due to its self-guided nature ____. Particularly, the exercise can be challenging for young adults to repetitively perform on their own over the long term, as it involves an intense cognitive process of comparing and contrasting two different temporal perspectives ____. As a result, this process may be cognitively demanding for young adults to execute independently, particularly without professional guidance, which may hinder sustained engagement over time ____. Additionally, individual differences in imaginative and writing capabilities may hinder the full realization of the intervention's effectiveness ____. While structured guidance from career professionals can be highly beneficial ____, limited access to such resources prevents many young adults from fully benefiting from the self-guided interventions, including the letter-exchange exercise.

\subsection{LLM-based Conversational Agents in Self-guided Interventions}

To address the limitations and barriers in self-guided interventions, conversational agents have emerged as powerful tools in delivering self-guided interventions across various domains, including mental health ____ and behavioral change ____. Early implementations of such agents were largely based on rule-based systems ____, which exhibited significant limitations. Specifically, these systems offered limited adaptability to individual contexts and faced scalability challenges, as tailoring rules to each user was resource-intensive. Consequently, these limitations hindered their broader adoption and reduced their overall effectiveness ____.

In this regard, the introduction of Large Language Models (LLMs) has the potential to enhance the capabilities of conversational agents, addressing the limitations of their rule-based predecessors. LLM-based agents can provide tailored responses by adapting to diverse user contexts and incorporating prior interactions ____. Furthermore, recent studies have demonstrated LLMs' ability to simulate specific characters ____, unlocking new possibilities for dynamic and personalized interventions in self-guided contexts. As a result, these capabilities have been successfully leveraged to enhance self-guided interventions across various domains. In the mental health domain, LLMs have been used to offer personalized support for self-led journaling ____ and to facilitate Cognitive Behavioral Therapy (CBT) ____. 


Building on these advancements, our study intends to harness the capabilities of LLM-based conversational agents to create dynamic and adaptive self-guided interventions for young adults' career exploration. Specifically, we integrated LLM agents simulating individuals' future selves into the letter-exchange exercise, aligning with the exercise’s core concept of fostering interaction between one’s present and future selves. By introducing a `\textbf{future-self agent},’ we sought to support the cognitively demanding aspects of the self-guided exercise by providing personalized guidance and facilitating dynamic, meaningful interactions with one’s future self.

However, prior studies have raised concerns about the potential drawbacks of over-reliance on AI, particularly its detrimental impact on critical thinking and self-reflection ____. In writing-based interventions like the letter-exchange exercise, where deep introspection and active participation are essential, excessive reliance on AI could undermine meaningful engagement in self-exploration and career development. These concerns, coupled with the potential benefits of AI in providing personalized support, underscore the need for a careful examination of how LLM agents influence young adults' engagement in the letter-exchange exercise. Through this exploration, we aimed to identify ways to balance the benefits of AI-driven support with the need to foster reflective and participatory engagement in self-guided interventions.



\begin{figure*}[!ht]
    \centering
    \includegraphics[width=440pt]{figures/fig2.png}
    \caption{Research model overview}
    \Description{Figure 2 illustrates the overview of a research study model involving three conditions: Writing Condition, LLM Letter Condition, and LLM Chat Condition, categorized under `Independent Variables.' In the `During the Letter-Exchange Exercise' phase, `Dependent Variables (RQ1)' are displayed: Perceived Ease of Future Imagery, Immersion in Future Envisioning, and Overall Satisfaction with the Exercise. The `Overall Effectiveness after the Letter-Exchange Exercise' phase features `Dependent Variables (RQ2)': Connectedness with the Future Self (Future Imagery Elaboration and Future Self-Continuity), Career Goal Clarity (Career Identity and Future Goal Salience), and Psychological Resilience (Career Stress and Self-Efficacy). These lead to the outcome of Augmented Career Exploration.}
    \label{fig:2}
\end{figure*}