% !TEX root = Main.tex

Vector addition systems (VAS), also known as Petri nets, are a popular model of concurrent systems. VAS have a very rich theory and have been intensely studied. In particular, the \emph{reachability problem} for VAS, which consists of deciding whether a target configuration of a VAS is reachable from a given initial configuration, has been studied for over 50 years. It was proved decidable in the 1980s \cite{Mayr81,Kosaraju82, Lambert92}, but its complexity (Ackermann-complete) could only be determined recently \cite{LerouxS19, CzerwinskiLLLM19, CzerwinskiO21, Leroux21}. 

In \cite{Leroux09} and \cite{Leroux13}, Leroux proved two fundamental results about the reachability sets of VAS. 
In \cite{Leroux09}, he showed that every configuration outside the reachability set $\vect{R}$ of a VAS is separated from $\vect{R}$ by a semilinear inductive invariant (for basic facts on semilinear sets see e.g. \cite{Haase18}). This immediately led to a very simple algorithm for the reachability problem consisting of two semi-algorithms, one enumerating all possible paths to certify reachability, and one enumerating all semilinear sets and checking if they are separating inductive invariants.
In \cite{Leroux13}, he showed that semilinear relations contained in the reachability relation are flatable, which immediately led to an algorithm for checking whether a semilinear relation is included in or equal to the reachability relation. Both of these results were obtained using abstract \emph{geometric properties} of reachability relations. This strand of research was later continued in \cite{GuttenbergRE23}, where some additional geometric axioms were used to reprove that the semilinearity problem, i.e. deciding whether the reachability relation of a given VAS is semilinear, is decidable. The semilinearity problem was first shown to be decidable in \cite{Hauschildt90}.

\textbf{Related Work}. One major branch of ongoing research in the theory of VAS studies whether results like the above extend to more general systems \cite{Reinhardt08, Bonnet11, Bonnet12, RosaVelardoF11, AtigG11, LerouxPS14, LerouxST15, HofmanLLLST16, LazicS16, FinkelLS18, LerouxS20, BlondinL23}. In particular, in a famous but very technical paper, Reinhardt proved that the reachability problem is decidable for VASS with nested zero tests (VASSnz) \cite{Reinhardt08}, in which counters can be tested for zero in a restricted manner: There is an order on the counters such that whenever counter \(i\) is tested for \(0\), also all counters \(j \leq i\) are tested for \(0\). Later \cite{AtigG11} proved that reachability in VASS controlled by finite-index grammars is decidable by reducing to VASSnz. In \cite{Bonnet11, Bonnet12}, Bonnet presented a more accessible proof of VASSnz reachability by extending the result of \cite{Leroux09}, separability by inductive semilinear sets. Recently, in \cite{Guttenberg24}, also the result of \cite{Leroux13} was extended to VASSnz.

%Another major tool which is utilized in the theory of VASS are \emph{well-quasi-orders} (wqo) \cite{Kruskal72, FinkelG09, FigueiraFSS11, Bonnet12}. A partial order \(\leq\) on a set \(\vectSet{X}\) is a wqo if \(\leq\) is well-founded and every subset \(\vectSet{U} \subseteq \vectSet{X}\) has finitely many minimal elements. In addition to the standard example of \(\N^d\) with the usual component-wise order being a wqo, for most known VASS extensions, all the way to Pushdown VASS and recently amalgamation systems \cite{Jancar90, LerouxPSS19, AnandSSZ24}, even the set of runs admits such an ordering. Whenever a set \(\vectSet{X}\), in this case the set of runs, is well-quasi-ordered, then its subsets have an \emph{ideal decomposition}, which can be utilized to decide the reachability problem \cite{LerouxS15}.

\textbf{Our contribution}. The following questions remain for \ConsideredModel: The complexity of the reachability and related problems as well as the decidability status of the semilinearity problem, which we both resolve in this paper. We establish the complexity of reachability, semilinearity and related problems as being Ackermann-complete. The complexity of semilinearity was unknown even for plain VAS. %To the best of our knowledge we are hence the first paper to provide a complexity upper bound on the reachability problem in an extension of VASS, since neither Reinhardt \cite{Reinhardt08}, nor Atig \cite{AtigG11}, Bonnet \cite{Bonnet12} or Guttenberg \cite{Guttenberg24} provide any bounds.%We are therefore the first paper giving an upper bound on reachability in an extension of VASS, since neither Reinhart

To this end, we proceed as follows: In step 1/Lemma \ref{LemmaConvertVASSnzToEVASS} we convert the input VASSnz into what we call monotone \(\RelationClass\)-extended VASS. This is an extension of VASS allowing more general counter operations than only addition. As the framework is quite general, we expect that there might be other applications for \(\RelationClass\)-extended VASS. 

In step 2/Theorem \ref{TheoremIdealDecompositionEVASS} we then provide an algorithm which approximates the reachability relation of a given input \(\RelationClass\)-extended VASS, assuming its operations can be approximated accordingly. In particular the approximation allows to decide reachability in \(\RelationClass\)-extended VASS. Our algorithm to perform the approximation is an adaptation of the classic algorithm KLM solving reachability in VASS, however we managed to remove some elements to simplify the presentation of the algorithm. 

Finally we solve also problems related to reachability by observing that our approximation can be viewed as an improvement upon a framework of \cite{GuttenbergRE23}, providing axioms such that for any class \(\RelationClass\) of systems fulfilling the axioms, the reachability, semilinearity, separability, etc. problems are decidable in Ackermann time. We consider finalizing this framework started in \cite{GuttenbergRE23} a main contribution of this paper, since it unifies the ideas behind the algorithms solving the reachability, semilinearity, etc. problems in an elegant fashion.

%Other noteworthy contributions include an interesting subclass of semilinear sets we call \emph{directed hybridlinear} sets, as well as progress towards a characterization of which transitions one may allow in a VASS s.t. reachability remains decidable, as Theorem \ref{TheoremIdealDecompositionEVASS} can also be viewed from this direction.

Above we only stated Ackermann-completeness, but in fact we provide more fine-grained complexity. The usual parameters of \ConsideredModel \ are the dimension \(d\) and the number of priorities \(k\), i.e. the number of different \(i\) such that the VASSnz uses a zero test on all \(j \leq i\). Often \(k\) is just \(1\) or \(2\), because only one or two types of zero tests are used. Our algorithm approximating the reachability relation has a time bound of \(\mathfrak{F}_{2kd+2k+2d+5}\) in the fast-growing function hierarchy. The complexity is the result of a sequence of \(2k+2\) Turing reductions, each of which will add \(d+1\) to the subscript of the fast-growing complexity class, ending at \(\mathfrak{F}_{3}\) for dealing with semilinear sets. These reductions form a chain as follows, where \(Reach_k\) stands for reachability in VASSnz with \(k\) priorities, and \(Cover_k\) for coverability, that is, deciding whether it is possible to reach a configuration at least as large as the given target from the given initial configuration: 

\(Reach_k \xrightarrow{+d+1} Cover_k \xrightarrow{+d+1} Reach_{k-1} \xrightarrow{+d+1} \dots\)

\textbf{Outline/Structure of the paper.} Section \ref{SectionSimplePreliminaries} provides a few preliminaries. Section \ref{SectionFastGrowingFunctions} introduces the fast-growing functions we use for complexity analysis. Section \ref{SectionVAS} introduces VASS and \ConsideredModel. Section \ref{AlgorithmToolbox} introduces the most important tools we utilize in our algorithm, and in particular the definition of the approximation we compute. Section \ref{SectionMainAlgorithm} introduces monotone \(\RelationClass\)-extended VASS and proves that VASSnz can be converted into extended VASS. Section \ref{SectionProofTheoremEVASS} provides an algorithm approximating the reachability relation of \(\RelationClass\)-extended VASS. Finally, Section \ref{SectionHybridization} introduces the geometric axioms and shows how to solve most relevant problems of \ConsideredModel \ using only the axioms, before we conclude in Section \ref{SectionConclusion}.










