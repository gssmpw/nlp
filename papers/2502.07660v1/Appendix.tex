% !TeX root = Main.tex


























































\subsection{Appendix of Section \ref{AlgorithmToolbox}}

This is the appendix of Section \ref{AlgorithmToolbox}. 

We start with an important statement about Definition \ref{DefinitionGoodOverapproximation} of nice overapproximation, which is not obvious from the definition we gave, since we use \(\N_{\geq 1}(\vectSet{F})\): Definition \ref{DefinitionGoodOverapproximation} is independent of the representation of \(\vectSet{L}\). The simplest way to prove this is to consider the following definition only depending on \(\vectSet{L}\), not some representation of it.

\begin{definition} \label{DefinitionInterior}
Let \(\vectSet{P}\) be \(\N\)-g., and \(\vect{v} \in \vectSet{P}\). 

The vector \(\vect{v}\) is in the \emph{interior} of \(\vectSet{P}\) if for every \(\vect{x} \in \vectSet{P}\), there exists \(n \in \N\) s.t. \(n \vect{v}-\vect{x} \in \vectSet{P}\).

The set of all interior vectors is denoted \(\interior(\vectSet{P})\).
\end{definition}

Accordingly, we define the interior of a linear set as \(\interior(\vectSet{L})=\vect{b}+\interior(\vectSet{L}-\vect{b})\), where \(\vect{b}\) is the base point of \(\vectSet{L}\). Observe that since linear sets are \(\subseteq \N^d\), the base point is the unique minimal point in \(\vectSet{L}\), hence this definition does not depend on the representation. 

Now we can prove that nice overapproximation does not depend on the representation, by proving that \(\vect{w} \in \N_{\geq 1}(\vectSet{F})\) can equivalently be replaced by \(\vect{w} \in \interior(\vectSet{L})-\vect{b}\):

\begin{lemma} \label{LemmaIndependentOfRepresentationOfL}
Let \(\vectSet{L}=\vect{b}+\N(\vectSet{F})\) be a linear set, and \(\vectSet{X} \subseteq \vectSet{L}\). Then in Definition \ref{DefinitionGoodOverapproximation} ``\(\forall \vect{w} \in \N_{\geq 1}(\vectSet{F})\)'' can equivalently be replaced by ``\(\forall \vect{w} \in \interior(\vectSet{L})-\vect{b}\)''. 

In particular, the definition of nice overapproximation is independent of the representation of \(\vectSet{L}\).
\end{lemma}

\begin{proof}
First observe that by definition of \(\interior(\vectSet{L})\), we have \(\interior(\vectSet{L})-\vect{b}=(\vect{b}+\interior(\vectSet{L}-\vect{b}))-\vect{b}=\interior(\vectSet{L}-\vect{b})=\interior(\N(\vectSet{F}))\).

``\(\Leftarrow\)'': We assume the line containment property holds for all \(\vect{w} \in \interior(\vectSet{L})- \vect{b}=\interior(\N(\vectSet{F}))\) and prove it for all \(\vect{w} \in \N_{\geq 1}(\vectSet{F})\). It suffices to prove \(\N_{\geq 1}(\vectSet{F}) \subseteq \interior(\N(\vectSet{F}))\). Hence let \(\vect{w} \in \N_{\geq 1}(\vectSet{F})\). To prove \(\vect{w} \in \interior(\N(\vectSet{F}))\), let \(\vect{x} \in \N(\vectSet{F})\) arbitrary. We have to show that there exists \(n \in \N\) s.t. \(n \vect{w} - \vect{x} \in \N(\vectSet{F})\). Write \(\vect{x}=\sum_{\vect{f} \in \vectSet{F}} \lambda_{\vect{f}} \vect{f}\), and simply define \(n:=\max_{\vect{f} \in \vectSet{F}} \lambda_{\vect{f}}\). Then \(n \vect{w} - \vect{x}\) uses every \(\vect{f} \in \vectSet{F}\) at least \(n-\lambda_{\vect{f}} \geq 0\) times, i.e.\ is in \(\N(\vectSet{F})\) as claimed.

``\(\Rightarrow\)'': We assume the line containment property holds for all \(\vect{w} \in \N_{\geq 1}(\vectSet{F})\) and have to prove it for all \(\vect{w} \in \interior(\N(\vectSet{F}))\). Hence let \(\vect{w} \in \interior(\N(\vectSet{F}))\) and \(\vect{x} \in \vectSet{L}\). We claim the following: There exists \(n \in \N\) s.t. \(n \vect{w} \in \N_{\geq 1}(\vectSet{F})\). 

Proof of claim: For all \(\vect{f} \in \vectSet{F}\), we do the following: We use that \(\vect{w}\) is in the interior to obtain \(n_{\vect{f}} \in \N\) s.t. \(n_{\vect{f}} \vect{w}- \vect{f} \in \N(\vectSet{F})\). We define \(n:=\sum_{\vect{f} \in \vectSet{F}} n_{\vect{f}}\) and observe that it proves the claim.

Therefore let \(n \in \N\) be s.t. \(n \vect{w} \in \N_{\geq 1}(\vectSet{F})\). For all \(0 \leq j \leq n-1\) we define \(\vect{x}_j:=\vect{x}+j \vect{w}\). Observe that \[\bigcup_{j=0}^{n-1} \vect{x}_j+n \N \vect{w}=\vect{x}+\N \vect{w},\] since \(\N=\{0, \dots, n-1\}+n\N\). For all \(0 \leq j \leq n-1\) we use our assumption with \(\vect{x}_j \in \vectSet{L}\) and \(n\vect{w} \in \N_{\geq 1}(\vectSet{F})\) to obtain \(N_j \in \N\) s.t. \(\vect{x}_j+\N_{\geq N_j} n \vect{w} \subseteq \vectSet{X}\). Defining \(N:=n+\max_{0 \leq j \leq n-1} N_j\), we obtain \(\vect{x}+\N_{\geq N} \vect{w} \subseteq \vectSet{X}\).
\end{proof}

In particular we will (indirectly) use Lemma \ref{LemmaIndependentOfRepresentationOfL} whenever we want to prove that \(\vectSet{X} \HybridizationRelation \vectSet{L}\), since we might not be given the exact representation considered.

Before we can start with the lemmas from the main text, we first need two more observations/lemmas about the interior. The first states that the interior makes up ``most of the points'' in a linear set, in the sense of dimension.

\begin{lemma}\label{LemmaBoundaryLowerDimension}
Let \(\vectSet{P}=\N(\vectSet{F})\) be \(\N\)-finitely generated.

Then \(\dim(\vectSet{P} \setminus \interior(\vectSet{P})) < \dim(\vectSet{P})\).
\end{lemma}

\begin{proof}
First observe that by Lemma \ref{LemmaFromJerome} we have \(\dim(\vectSet{P})=\dim(\Q(\vectSet{P}))\), hence it suffices to prove \(\dim(\vectSet{P} \setminus \interior(\vectSet{P})) < \dim(\Q(\vectSet{P}))\). I.e.\ we have to show that \(\vectSet{P} \setminus \interior(\vectSet{P})\) is contained in a union of lower dimensional vector spaces/\(\Q\)-generated sets. To this end, consider any vector \(\vect{v} \in \vectSet{P} \setminus \interior(\vectSet{P})\). 

Write \(\vect{v}=\sum_{\vect{f} \in \vectSet{F}} \lambda_{\vect{f}} \vect{f}\). Since \(\vect{v}\) is not interior, there exists a \(\vect{x} \in \vectSet{P}\) (in fact \(\vect{x} \in \vectSet{F})\) s.t. there is no scalar \(n \in \N\) with \(n \vect{v} \geq \vect{x}\). Let \(\vectSet{F}':=\{\vect{f} \in \vectSet{F} \mid \lambda_{\vect{f}} \geq 1\}\) be the set of coefficients used for \(\vect{v}\). Then \(\vect{x} \not \in \Q(\vectSet{F}')\) by the above: Otherwise such a scalar \(n \in \N\) would exist. 

Hence \(\dim(\Q(\vectSet{F}'))<\dim(\Q(\vectSet{F}')+\Q(\vect{x}))\leq \dim(\Q(\vectSet{P}))\). 

Since we chose \(\vect{v} \in \vectSet{P} \setminus \interior(\vectSet{P})\) arbitrary, we obtain 

\(\vectSet{P} \setminus \interior(\vectSet{P}) \subseteq \bigcup_{\vectSet{F}' \subseteq \vectSet{F}, \dim(\Q(\vectSet{F}'))<\dim(\vectSet{P})} \Q(\vectSet{F}'),\)

which has lower dimension than \(\vectSet{P}\).
\end{proof}

The next lemma states a dichotomy: If \(\vectSet{P}' \subseteq \vectSet{P}\), then either \(\interior(\vectSet{P}') \subseteq \interior(\vectSet{P})\), or their intersection is empty. The lemma even states \(\vectSet{P}' \cap \interior(\vectSet{P})\) being empty.

\begin{lemma}\label{LemmaInteriorDichotomy}
Let \(\vectSet{P}' \subseteq \vectSet{P}\) be \(\N\)-generated sets. Then either \(\interior(\vectSet{P}') \subseteq \interior(\vectSet{P})\) or \(\vectSet{P}' \cap \interior(\vectSet{P})=\emptyset\).
\end{lemma}

\begin{proof}
Assume that \(\vectSet{P}' \cap \interior(\vectSet{P})\neq \emptyset\). We have to prove that \(\interior(\vectSet{P}') \subseteq \interior(\vectSet{P})\). Let \(\vect{w} \in \interior(\vectSet{P}')\). It suffices to prove \(\vect{w} \in \interior(\vectSet{P})\). To prove this, we have to take an arbitrary \(\vect{x} \in \vectSet{P}\) and prove that there exists \(n \in \N\) s.t. \(n \vect{w}-\vect{x} \in \vectSet{P}\). 

Since \(\vectSet{P}' \cap \interior(\vectSet{P}) \neq \emptyset\), there exists \(\vect{v} \in \vectSet{P}' \cap \interior(\vectSet{P})\). By definition of \(\interior(\vectSet{P})\) there exists \(n_1 \in \N\) s.t. \(n_1 \vect{v}-\vect{x} \in \vectSet{P}\). Since \(\vect{w} \in \interior(\vectSet{P}')\) and \(\vect{v} \in \vectSet{P}'\), there exists \(n_2 \in \N\) s.t. \(n_2 \vect{w} - \vect{v} \in \vectSet{P}'\). It follows that 
\[n_1n_2 \vect{w} - \vect{x}=n_1(n_2\vect{w}-\vect{v})+(n_1 \vect{v} - \vect{x})\in \vectSet{P}' + \vectSet{P} \subseteq \vectSet{P},\]

since \(\vectSet{P}' \subseteq \vectSet{P}\) are \(\N\)-g..
\end{proof}

After this short discourse, we can start proving lemmas from the main text, starting with the closure properties.

\LemmaShiftGoodOverapproximation*

\begin{proof}
Part 1 (Intersection): We stated the important observation in the main text, and said it remains to prove what we can now write as \(\interior(\vectSet{L} \cap \vectSet{L}') \subseteq \interior(\vectSet{L})\).

Let \(\vectSet{X} \HybridizationRelation \vectSet{L}\) and \(\vectSet{L} \cap \vectSet{L}'\) be non-degenerate. We claim that \(\interior(\vectSet{L}) \cap \vectSet{L}'\neq \emptyset\): Assume otherwise. Then \(\vectSet{L} \cap \vectSet{L}' \subseteq \vectSet{L} \setminus \interior(\vectSet{L})\). By Lemma \ref{BasicDimensionProperties} and Lemma \ref{LemmaBoundaryLowerDimension}, we obtain 
\[\dim(\vectSet{L} \cap \vectSet{L}') \leq \dim(\vectSet{L} \setminus \interior(\vectSet{L}))<\dim(\vectSet{L})=\dim(\vectSet{L} \cap \vectSet{L}'),\]
where the last equality is due to the intersection being non-degenerate. This is a contradiction.

Hence \(\interior(\vectSet{L}) \cap \vectSet{L}' \neq \emptyset\). By the dichomoty in Lemma \ref{LemmaInteriorDichotomy}, we obtain \(\interior(\vectSet{L} \cap \vectSet{L}') \subseteq \interior(\vectSet{L})\) as required.

Part 2 (Composition): Write \(\vectSet{L}_{12}=(\vect{b}_1, \vect{b}_2)+\N(\vectSet{F}_{12})\) and \(\vectSet{L}_{23}=(\vect{b}_2', \vect{b}_3)+\N(\vectSet{F}_{23})\). 

First observe that \(\vectSet{L}_{12} \circ \vectSet{L}_{23}+\N(\vectSet{F}_{12}) \circ \N(\vectSet{F}_{23})\subseteq \vectSet{L}_{12} \circ \vectSet{L}_{23}\), i.e.\ composing periods of \(\vectSet{L}_{12}, \vectSet{L}_{23}\) gives periods of \(\vectSet{L}_{12} \circ \vectSet{L}_{23}\). To see this, let \((\vect{p}_1, \vect{p}_2) \in \N(\vectSet{F}_{12})\) and \((\vect{p}_2, \vect{p}_3) \in \N(\vectSet{F}_{23})\). Let \((\vect{x}_1, \vect{x}_3) \in \vectSet{L}_{12} \circ \vectSet{L}_{23}\) arbitrary. Then there exists \(\vect{x}_2\) s.t. \((\vect{x}_1, \vect{x}_2) \in \vectSet{L}_{12}\) and \((\vect{x}_2, \vect{x}_3) \in \vectSet{L}_{23}\). Then also \((\vect{x}_1+\vect{p}_1, \vect{x}_2+\vect{p}_2) \in \vectSet{L}_{12}\) and \((\vect{x}_2+\vect{p}_2, \vect{x}_3+\vect{p}_3) \in \vectSet{L}_{23}\), implying \((\vect{x}_1+\vect{p}_1, \vect{x}_3+\vect{p}_3) \in \vectSet{L}_{12} \circ \vectSet{L}_{23}\). 

Now we start the actual proof. Let \((\vect{x}_1, \vect{x}_3) \in \vectSet{L}_{12} \circ \vectSet{L}_{23}\) and let \((\vect{w}_1, \vect{w}_3) \in \interior(\vectSet{L}_{12} \circ \vectSet{L}_{23})-\text{base}\), where \(\text{base}\) is the base vector of \(\vectSet{L}_{12} \circ \vectSet{L}_{23}\). Remember that by Lemma \ref{LemmaIndependentOfRepresentationOfL}, we can use the interior instead of \(\N_{\geq 1}(\vectSet{F}')\). Let \(\vect{x}_2\) be s.t. \((\vect{x}_1, \vect{x}_2) \in \vectSet{L}_{12}\) and \((\vect{x}_2, \vect{x}_3) \in \vectSet{L}_{23}\). We have to show that \(\exists N \in \N\) s.t. 
\[(\vect{x}_1 , \vect{x}_3)+\N_{\geq N}(\vect{w}_1,\vect{w}_3) \subseteq \vectSet{X}_{12} \circ \vectSet{X}_{23}.\]
Assume first that \((\vect{w}_1, \vect{w}_3) \in \interior(\N(\vectSet{F}_{12})) \circ \interior(\N(\vectSet{F}_{23}))\). Then there exists \(\vect{w}_2\) such that \((\vect{w}_1, \vect{w}_2) \in \interior(\N(\vectSet{F}_{12}))\) and \((\vect{w}_2, \vect{w}_3) \in \interior(\N(\vectSet{F}_{23}))\). Since \(\vectSet{X}_{12} \HybridizationRelation \vectSet{L}_{12}\), there exists \(N_{12} \in \N\) s.t. \((\vect{x}_1, \vect{x}_2)+\N_{\geq N_{12}}(\vect{w}_1, \vect{w}_2) \subseteq \vectSet{X}_{12}\) and similarly \(N_{23} \in \N\) s.t. \((\vect{x}_2, \vect{x}_3)+\N_{\geq N_{23}}(\vect{w}_2, \vect{w}_3) \subseteq \vectSet{X}_{23}\). Choosing \(N:=\max \{N_{12}, N_{23}\}\) this implies the claim.

However, we made the assumption that \((\vect{w}_1, \vect{w}_3) \in \interior(\N(\vectSet{F}_{12})) \circ \interior(\N(\vectSet{F}_{23}))\). It remains to argue that the statement also holds for other interior periods \((\vect{w}_1, \vect{w}_3)\) of \(\vectSet{L}_{12} \circ \vectSet{L}_{23}\). The corresponding proof is similar to repeating Lemma \ref{LemmaIndependentOfRepresentationOfL}, Lemma \ref{LemmaBoundaryLowerDimension} and Lemma \ref{LemmaInteriorDichotomy}.
\end{proof}





















\subsection{Appendix of Section \ref{SectionProofTheoremEVASS}}

There are some minor checks we did not perform in the main text, in particular the following:

\begin{enumerate}
\item Why are edges inside an SCC always monotone, i.e.\ why are the new \(\VAS_i\) still \emph{monotone} \(\RelationClass\)-eVASS?
\item Why do cleaning properties not disturb the respective properties restored before?
\item Why does the rank decrease/not increase?
\end{enumerate}

Hence we have to revisit every one of the decomposition steps, and ensure these three properties.

\textbf{P1}: Monotonicity inside SCCs is preserved as we explicitly required it in Definition \ref{DefinitionApproximable}, which our \(\ApproximationAlgorithm\) adheres to. There are no prior cleaning properties, and the rank stays the same by definition of the rank in this case. 

\textbf{P2}: Monotonicity inside SCCs is preserved as we do not change edge labels, similarly for P1.

The rank depends on cycle spaces, and these do not change by splitting into SCCs. If anything, some SCC might be removed and the rank might decrease.

\textbf{P3}: Changing the \(\vectSet{L}_i\) neither changes edge labels nor strongly-connectedness. 

The rank does not depend on \(\vectSet{L}_i\), only on the \(\VAS_i\).

\textbf{P4}: Monotonicity is not required for edges leaving SCCs. This decomposition explicitly restores P1 and P3, and the decomposed \(\VAS_i\) with a single edge continues to have a single edge. Hence P2 is also preserved. 

Regarding the rank, observe that the rank is defined only considering strongly-connected components, and this edge is not inside an SCC.

\textbf{P5}: Deleting a counter which is already stored in the state only performs projections on edge labels, hence we can rely on Lemma \ref{LemmaNiceOverapproximationProjection} to see that with the new label we still have \(\pi(\vectSet{R}(e)) \HybridizationRelation \pi(\vectSet{L}(e))\). This projection preserves monotonicity. Strongly-connectedness is clearly preserved, since we do not change any states or redirect edges. The characteristic system does not care either, since all \(\Z\)-runs anyways respected the counter value. Since the characteristic system did not change, (4) was also preserved.

We did not change the cycle space of any SCC or its number of edges, hence the rank is preserved.

\textbf{P6}: Let \(\vectSet{X}_i=(\VAS_i=(Q_i, E_i, \qini, \qfini), \vectSet{L}_i)\) be a component in a \(\RelationClass\)-KLM sequence s.t. \(\VAS_i\) is strongly-connected and some edge or period is bounded. For every edge \(e\) we write \(\vectSet{L}(e)=\vect{b}(e)+\N(\vectSet{F}(e))\). Let \(\vectSet{F}'(e) \subseteq \vectSet{F}(e)\) be the set of periods which are bounded. For all \(\vect{p}(e) \in \vectSet{F}'(e)\), let \(k(\vect{p}(e))\) be the maximal number of times this period can be used. Let \(E_i' \subseteq E_i\) be the set of edges which are bounded. For all \(e \in E_i'\), let \(k(e)\) be the maximal number of times edge \(e\) can be taken. Let \([k]:=\{0, \dots, k\}\) be the index set of \(k\), beware we start at \(0\) and end at \(k\), so we include \(k+1\) values. We use the new set of states 
\[Q_i':=Q_i \times \prod_{e \in E_i'} [k(e)] \times \prod_{e \in E_i, \vect{p}(e) \in \vectSet{F}'(e)} [k(\vect{p}(e))],\]
i.e.\ we track \emph{every} bounded value simultaneously in the state. The idea for the set of edges \(E_i'\) is as follows: If we take a bounded edge \(e\), we increment the corresponding counter in the state by \(1\), blocking if we are at the maximum already. Similarly, if we use a bounded period \(\vect{p}(e)\) some \(j \leq k(\vect{p}(e))\) number of times, then we increment the corresponding counter by \(j\), blocking if we would exceed the maximum. Formally, for every \(e=(p,q) \in E_i'\), every \((p, \vect{w}_1, \vect{w}_2) \in Q_i'\) and every vector \(\vect{w} \in \N^{\vectSet{F}'(e)}\) (counting how often we use this period) s.t. \(\vect{w}_1(e)\leq k(e)-1\) and \(\vect{w}_2(\vect{p}(e))+\vect{w}(\vect{p}(e))\leq k(\vect{p}(e))\) for all \(\vect{p}(e) \in \vectSet{F}'(e)\), we add an edge 
\[((p, \vect{w}_1, \vect{w}_2), (q, \vect{w}_1+1_e, \vect{w}_2+\vect{w})) \in E_i',\]
where \(\vect{w}_1+1_e\) is the same as \(\vect{w}_1\), except \((\vect{w}_1+1_e)(e)=\vect{w}_1(e)+1\). We add similar edges for \(e=(p,q) \in E_i \setminus E_i'\), the difference is we do not add any \(1_e\) to \(\vect{w}_1\).

For edge labels, there is an important mistake which has to be avoided: After a degenerate intersection, we might lose the fact that \(\vectSet{R}(e) \HybridizationRelation \vectSet{L}(e)\). Since bounding a period in particular is a degenerate intersection, we are hence \emph{not} allowed to write any labels \(\vectSet{L}(e')\) onto the new edges \(e'\). Instead let \(e'=((p, \vect{w}_1, \vect{w}_2), (q, \vect{v}_1, \vect{v}_2)) \in E_i'\). Let \(e=(p,q)\) be the edge \(e'\) originated from. We use the label 
\[\vectSet{R}(e'):=[[\vect{b}(e)+(\vect{v}_2-\vect{w}_2) \cdot \vectSet{F}]+\N(\vectSet{F}(e) \setminus \vectSet{F}'(e))] \cap \vectSet{R}(e),\]
where in order to enhance readability we used square brackets [] instead of normal brackets () but with the same semantics.

Finally, we set \(\qini':=(\qini, \vect{0}, \vect{0})\), and consider the context \(\vectSet{X}=\vectSet{X}_1 \circ \dots \circ \vectSet{X}_r\) for the component we decompose. We create for every possible vector \(\vect{w}_1 \in \prod_{e \in E_i'} [k(e)], \vect{w}_2 \in \prod_{e \in E_i, \vect{p}(e) \in \vectSet{F}'(e)} [k(\vect{p}(e))]\) a separate \(\RelationClass\)-KLM sequence 
\[\vectSet{Y}_{\vect{w}_1, \vect{w}_2, \text{full}}=\vectSet{X}_1 \circ \dots \circ \vectSet{X}_{i-1} \circ \vectSet{Y}_{\vect{w}_1, \vect{w}_2} \circ \vectSet{X}_{i+1} \circ \dots \circ \vectSet{X}_r,\] where \(\vectSet{Y}_{\vect{w}_1, \vect{w}_2}=\Rel(Q_i', E_i', \qini', (\qfini, \vect{w}_1, \vect{w}_2)) \cap \vectSet{L}_i\). 

We return \(\{\vectSet{Y}_{\vect{w}_1, \vect{w}_2, \text{full}} \mid \vect{w}_1, \vect{w}_2 \text{ as above}\}\).

Why are edge labels \(\vectSet{R}(e')\) still monotone when contained in an SCC? Answer: Since we have property P5, for every counter \(j\) there exists a cycle which increases \(j\). Hence the periods corresponding to monotonicity are \emph{not} bounded, i.e.\ not removed. Then, as intersection of two monotone relations, \(\vectSet{R}'(e)\) is monotone again. Alternatively, we could also manually readd the periods corresponding to monotonicity, since the vector space of possible cycle \emph{effects} would not be influenced by adding monotonicity.

To see that the rank decreases, we first refer to \cite{LerouxS19}: They observed that when removing all bounded edges at once, the dimension of the vector space decreases. Since the dimension decreases, it does not matter that we count most edges multiple times now (remember edges are counted once for every edge they decompose into after applying P1): Since every single edge is now contained in a lower dimensional vector space, the rank, which is lexicographic, still decreases.

Regarding monotonicity: Since P5 holds, for every counter \(j\) there is a cycle increasing \(j\). Hence the periods corresponding to monotonicity are not bounded, and copied to the new relations.

\textbf{P7}: Deleting a counter again only projects the edge labels using Lemma \ref{LemmaNiceOverapproximationProjection} to see that with the new label we still have \(\pi(\vectSet{R}(e)) \HybridizationRelation \pi(\vectSet{L}(e))\). This preserves monotonicity.

To see that the rank decreases, observe first of all that since P5 holds, there is a cycle with non-zero effect on the deleted counter. We can proceed exactly as in \cite{LerouxS19} to see that the dimension of the vector space of cycle effects decreases: Clearly cycles in the new VASS are cycles in the old VASS with effect \(0\) on the deleted counter. Hence the vector space of cycle effects could only have decreased. And in fact it did strictly decrease, since the cycle with non-zero effect which existed before does not exist anymore. Since the dimension strictly went down, it again does not matter that we count the number of edges weirdly, we decrease in the lexicographic ordering.

\textbf{Example for property P1}: Why we need minimal elements: See Figure \ref{FigureExamplePropertyOne} and its caption.

\begin{figure}[h!]
\begin{centering}
\begin{tikzpicture}
		%\tikzset{every edge/.append style={font=\large}}
		
		\newcommand*{\distancesubx}{2.5cm}
	
		% States
		\node[place, double] (A) at (1,0) {q};
		\node[place, double] (B) at (8, 0) {p};
		\node[white!100] (C) at (0,0) {};
		
		% Edges
		\path[->, thick, out=30, in=150, looseness=0.6] (A) edge[] node[above] {\(\vectSet{R}(e_1)=\text{dec}(x)+\N(\{\text{inc}(x,y),\text{dec}(x,y),\text{dec}(x)\})+\text{Mon}\)} (B);
		\path[->, thick, out=210, in=-30, looseness=0.6] (B) edge[] node[above] {\(\vectSet{R}(e_2)=\vectSet{L}(e_2)=\N(\text{dec}(y))+\text{Mon}\)} (A);
		\path[->, thick] (C) edge[] (A);
\end{tikzpicture}
\end{centering}

\caption{Example of a \(2\)-dimensional monotone \(Semil\)-eVASS showing that \(\vect{b} \in \vectSet{R}(e)\) in property (1) is necessary for correctness. Edge \(e_1\) is labelled with \(\vectSet{L}(e_1):=\N(\{\text{inc}(x,y), \text{dec}(x,y), \text{dec}(x)\}+\text{Mon})\) and \(\vectSet{R}(e_1)=\text{dec}(x)+\vectSet{L}(e_1)\), where \(\text{inc}(x,y):=((0,0),(1,1))\), \(\text{dec}(x,y):=((1,1), (0,0))\), \(\text{dec}(x):=((1,0),(0,0))\) and \(\text{Mon}:=\N(\{((1,0),(1,0)), ((0,1),(0,1))\})\). Edge \(e_2\) is labelled \(\vectSet{R}(e_2)=\vectSet{L}(e_2)=\N(\text{dec}(y))+\text{Mon}\), where \(\text{dec}(y):=((0,1), (0,0))\). The source configuration is \(q(1,1)\) and the target is \(q(2,1)\). Clearly all properties hold except (1), in particular every configuration is forwards/backwards-coverable. 
\newline However, there is a semilinear inductive invariant \(\vectSet{S}=q:\{(x,y) \mid x \leq y\}, p: \{(x,y) \mid x \leq y-1\}\) which proves non-reachability. The problem is that the homogeneous solution of \(\CharSys\) for decrementing \(y\) while leaving \(x\) the same by using \(e_1 e_2\) without any periods is infeasible: In the actual relation taking edge \(e_1\) causes an automatic decrement on \(x\).}\label{FigureExamplePropertyOne}
\end{figure}

%\textbf{Preserving Monotonicity in Output}: Remember Definition \ref{DefinitionApproximable}, part 2): We have to ensure that if we try to approximate a monotone relation \(\vectSet{X}\), the the resulting \(\vectSet{X}_j\) are again monotone. First of all, we want to assure the reader of the following: Even if this did not hold, it would be easy to readd monotonicity in a last postprocessing step: Simply double the dimension and add a new first relation \(\vectSet{X}_{in}\) and a new last relation \(\vectSet{X}_{out}\) to any \(\RelationClass\)-KLM sequence in the output. In \(\vectSet{X}_{in}\) move an arbitrary amount of value from the first \(d\) to the last \(d\) counters, and in the last component \(\vectSet{X}_{out}\) move the values back, checking that the auxiliary counters are \(0\). Clearly this would still be perfect.
%
%However, analyzing the steps of the algorithm, none of them jeopardizes that the relation is monotone: The characteristic system only consider \emph{effects}, and we remove certain \emph{effects}. But effects are independent of monotonicity. On the other hand, deleting a counter if it is bounded again does not jeopardize this property, and similarly for replacing edge labels in P1.

\subsection{Appendix of Section \ref{SectionHybridization}}

This section of the appendix is split into three parts. First we have to introduce some theory regarding \emph{directed hybridlinear} sets, in order to show the required closure property (Lemma \ref{LemmaDirectedHybridlinearNondegenerateIntersection}) as well as to prove that the dimension recursion (Lemma \ref{LemmaDimensionDecreaseShiftedL}) still works for directed hybridlinear, which is necessary for some algorithms. Furthermore, the dependence of the definition of directed hybridlinear on the representation has to be removed. This is done in part 1. In part 2, we can then prove most of the axioms, but Axiom 7 will remain out of reach. Hence why in Part 3 we introduce heavy theory from the VASS community to address Axiom 7.

\subsection{Appendix of Section \ref{SectionHybridization}, Part 1}

In this part we define directed hybridlinear sets and prove basic properties for this class. Towards this end, we recall some lemmas regarding \(\mathbb{S}\)-finitely generated (f.g.) sets for \(\mathbb{S} \in \{\N, \Q_{\geq 0}, \Z, \Q\}\). Recall that \(\mathbb{S}(\vectSet{F}):=\{ \sum_{i=1}^n \lambda_i \vect{f}_i \mid n \in \N, \vect{f}_i \in \vectSet{F}, \lambda_i \in \mathbb{S}\}\) and for (finite) $\vectSet{F}$ we say that the set \(\mathbb{S}(\vectSet{F})\) is \(\mathbb{S}\)-(finitely) generated. Also recall that a partial order \((\vectSet{X}, \leq_{\vectSet{X}})\) is a \emph{well-quasi-order} (wqo) if \(\leq\) is well-founded and every subset \(\vectSet{U} \subseteq \vectSet{X}\) has finitely many minimal elements. The most famous example of a wqo is \((\N^d, \leq_{\N^d})\), where \(\leq_{\N^d}\) is the component-wise \(\leq\) ordering.

For \(\mathbb{S} \in \{\Z, \Q\}\) every \(\mathbb{S}\)-g. set is \(\mathbb{S}\)-f.g., because the whole space \(\Q^d\) or respectively \(\Z^d\) is finitely generated, and moving to a substructure does not increase the number of necessary generators. For \(\Q\)-generated sets, this is well-known linear algebra, for \(\Z\)-generated sets this follows from the existence of the hermite normal form for integer matrices (see \cite{LinearProgramming}, Chapter 4). For \(\mathbb{S} \in \{\N, \Q_{\geq 0}\}\) the picture is different. 

For \(\mathbb{S} = \Q_{\geq 0}\) we have:

\begin{lemma}[\cite{LinearProgramming}, Cor. 7.1a]

Let \(\vectSet{C} \subseteq \Q^d\) be a \(\Q_{\geq 0}\)-g. set.

Then \(\vectSet{C}=\{\vect{x} \in \Q^d \mid A \vect{x} \geq \vect{0}\}\) is the preimage of \(\Q_{\geq 0}^{d'}\) for some integer matrix \(A \in \Z^{d' \times d}\) iff \(\vectSet{C}\) is \(\Q_{\geq 0}\)-finitely generated.\label{LemmaFinitelyGeneratedCone}%
\end{lemma}

The essential idea behind this lemma is the following: A \(\Q_{\geq 0}\)-f.g. set (or equivalently the set of points on some ray from the top of a pyramid to the base) can be classified depending on what the base of the pyramid looks like. For an ice cream cone this is a circle, for the pyramids in Egypt a square. The lemma classifies pyramids where the base is a polygon: The base of the pyramid has finitely many vertices (generators of the cone) if and only if the base has finitely many faces/side edges. Every side edge gives rise to an inequality \(\vect{a} \vect{x} \geq 0\), which we can rescale to obtain \(\vect{a} \in \Z^d\).

For \(\mathbb{S} = \N\)  in \cite{Leroux13} Leroux proved a similar characterization.

\begin{definition}
Let \(\vectSet{P}\) be \(\N\)-g.. The canonical partial order on \(\vectSet{P}\) is \(\leq_{\vectSet{P}}\) defined via \(\vect{x} \leq_{\vectSet{P}} \vect{y}\) if \(\vect{y}=\vect{x}+\vect{p}\) for some \(\vect{p}\in \vectSet{P}\).
\end{definition}

\begin{lemma}[\cite{Leroux13}, Lemma V.5]
Let \(\vectSet{P} \subseteq \N^d\) be \(\N\)-g.. T.F.A.E.:

\begin{enumerate}
\item \(\vectSet{P}\) is \(\N\)-finitely generated.
\item \((\vectSet{P}, \leq_{\vectSet{P}})\) is a wqo. 
\item \(\Q_{\geq 0} \cdot \vectSet{P}\) is \(\Q_{\geq 0}\)-finitely generated.
\end{enumerate} \label{LemmaFinitelyGeneratedPeriodicSets}
\end{lemma}

\begin{proof}[Proof sketch]
(1) \(\Rightarrow\) (3): Obvious (use the same generators).

 (3) \(\Rightarrow\) (2): (Only intuition): We show this by providing an order isomorphism \((\vectSet{P}, \leq_{\vectSet{P}}) \simeq (\N^{d'}, \leq)\) to the wqo on \(\N^{d'}\).
 
The ordering \(\leq_{\vectSet{P}}\) (up to minor details) fulfills \(\vect{x} \leq_{\vectSet{P}} \vect{y}\) iff \(A \vect{x} \leq A \vect{y}\), where \(A\) is the matrix of Lemma \ref{LemmaFinitelyGeneratedCone} for \(\Q_{\geq 0} \cdot \vectSet{P}\).
We have \(A\vect{x}\) and \(A\vect{y}\) in \(\N^{d'} = \Z^{d'} \cap \Q_{\geq 0}^{d'}\). Indeed, \(A\vect{x}\) is an integer because we performed multiplication and addition of integers. Additionally $\vect{x} \in \vectSet{P} \subseteq \Q_{\geq 0} \cdot \vectSet{P}\) is in the preimage of \(\Q_{\geq 0}^{d'}\) by choice of \(A\). Hence multiplication by \(A\) is the required order isomorphism. Intuitively, this isomorphism shows that $\leq_{\vectSet{P}}$ orders the points with respect to the distance to the borders of the \(\Q_{\geq 0}\)-f.g. set \(\Q_{\geq 0} \cdot \vectSet{P}\): the bigger the distances to the borders the bigger is the point wrt. $\leq_{\vectSet{P}}$.
 
 (2) \(\Rightarrow\) (1): As \((\vectSet{P}, \leq_{\vectSet{P}})\) is a wqo, \(\vectSet{P} \setminus \{\vect{0}\}\) has finitely
 many minimal elements w.r.t. \(\leq_{\vectSet{P}}\). These generate $\vectSet{P}$.
\end{proof}

So far these results are known, we now utilize the ordering \(\leq_{\vectSet{P}}\) of Lemma \ref{LemmaFinitelyGeneratedPeriodicSets} 2) to classify semilinear sets.

\begin{definition} \label{DefinitionPreservants}
Let \(\vectSet{X} \subseteq \N^d\) be any set. A vector \(\vect{p}\) is a \emph{preservant} of \(\vectSet{X}\) if \(\vectSet{X}+\vect{p} \subseteq \vectSet{X}\). The set of all preservants is denoted \(\PX\).
\end{definition}

Clearly \(\PX\) is \(\N\)-generated, i.e.\ closed under addition, for every set \(\vectSet{X}\). In many cases however, \(\PX\) is very small compared to \(\vectSet{X}\), see the left of Figure \ref{FigureIntuitionCones}. \(\PX\) being large and finitely generated in fact characterizes linear sets and more generally hybridlinear sets.

\begin{figure}[h!]
\begin{minipage}{4.5cm}
\begin{tikzpicture}
\begin{axis}[
    axis lines = left,
    xlabel = { },
    ylabel = { },
    xmin=0, xmax=8,
    ymin=0, ymax=8,
    xtick={0,2,4,6,8},
    ytick={0,2,4,6,8},
    ymajorgrids=true,
    xmajorgrids=true,
    thick,
    smooth,
    no markers,
]

\addplot+[
    name path=A,
    color=blue,
]
coordinates {(0,0) (8,8)};

\addplot+[
    name path=B,
    domain=0:8,
    color=blue,
]
{log2(x+1)};

\addplot[blue!40] fill between[of=A and B];

\addplot[
fill=red,
fill opacity=0.7,
only marks,
]
coordinates {
(0,0)(1,0)(2,0)(3,0)(4,0)(5,0)(6,0)(7,0)(8,0)
};


    
%\addplot[
%    fill=green,
%    fill opacity=0.7,
%    only marks,
%    ]
%    coordinates {
%    (3,4)(3,5)(3,6)(3,7)(3,8)(4,4)(4,5)(4,6)(4,7)(4,8)(5,4)(5,5)(5,6)(5,7)(5,8)(6,4)(6,5)(6,6)(6,7)(6,8)(7,4)(7,5)(7,6)(7,7)(7,8)(8,4)(8,5)(8,6)(8,7)(8,8)
%    };

\end{axis}
\end{tikzpicture}
\end{minipage}%
\begin{minipage}{4.5cm}
\begin{tikzpicture}
\begin{axis}[
    axis lines = left,
    xlabel = { },
    ylabel = { },
    xmin=0, xmax=4,
    ymin=0, ymax=8,
    xtick={0,1,2,3,4},
    ytick={0,2,4,6,8},
    ymajorgrids=true,
    xmajorgrids=true,
    thick,
    smooth,
    no markers,
]
    
\addplot+[
    name path=A,
    color=blue,
    thick,
    ]
    coordinates {
    (0,0)(4,8)
    };
    
\addplot+[
    name path=B,
    color=blue,
    thick,
]
coordinates {
    (0,0)(4,0)
    };
    
\addplot[blue!40] fill between[of=A and B];

\addplot+[
    name path=C,
    color=red,
    thick,
    ]
    coordinates {
    (0,1)(4,5)
    };
    
\addplot+[
    name path=D,
    color=red,
    thick,
]
coordinates {
    (0,1)(4,1)
    };
    
\addplot[red!40] fill between[of=C and D];

\addplot[
    fill=black,
    fill opacity=0.7,
    only marks,
    ]
    coordinates {
    (1,1)(1,2)
    };

\end{axis}
\end{tikzpicture}
\end{minipage}%

\caption{\textit{Left}: The set \(\vectSet{X}:=\{(x,y) \in \N^2 \mid x \geq y \geq \log_2(x+1)\}\) (blue region) \(\cup \)\{x-axis\} (red) fulfills \(\PX=\{\vect{0}\}\). Namely vectors \((x,y)\) with \(y>0\) are \(\not \in \PX\) because of the red points, and for \(x>0, y=0\) the reason is the blue region.  \newline
\textit{Right}: The blue \(\{(x,y) \in \N^2 \mid 0 \leq y \leq 2x\}\) and the red \(1 \leq y \leq 1+x\) regions depict linear sets. \(\vectSet{L}_1 \cap \vectSet{L}_2\) is not linear anymore, as it requires both the black points as base points.}\label{FigureIntuitionCones}
\end{figure}

\begin{definition}
A set \(\vectSet{L} \subseteq \N^d\) is linear if \(\vectSet{L}=\vect{b}+\N(\vectSet{F})\) for some point \(\vect{b} \in \N^d\) and finite set \(\vectSet{F}\subseteq \N^d\). 

\(\vectSet{L}\) is hybridlinear if \(\vectSet{L}=\{\vect{b}_1, \dots, \vect{b}_r\}+\N(\vectSet{F})\) for finitely many points \(\vect{b}_1, \dots, \vect{b}_r \in \N^d\).
\end{definition}

\begin{restatable}{proposition}{PropositionCharacterizeHybridlinear}
Let \(\vectSet{X}\subseteq \N^d\) be any set. 

Then \(\vectSet{X}\) is hybridlinear if and only if \((\vectSet{X}, \leq_{\PX})\) is a wqo. \label{PropositionCharacterizeHybridlinear}
\end{restatable}

\begin{proof}
If \(\vectSet{X}=\emptyset\) then both statements trivially hold. In the sequel we hence assume \(\vectSet{X} \neq \emptyset\).

``\(\Leftarrow\)'': Let \(\vect{x} \in \vectSet{X}\). Since \((\vectSet{X}, \leq_{\PX})\) is a wqo, also \((\vect{x}+\PX, \leq_{\PX})\) is a wqo. This is isomorphic to \((\PX, \leq_{\PX})\), hence by Lemma \ref{LemmaFinitelyGeneratedPeriodicSets} \(\PX\) is finitely generated. Let \(\vect{b}_1, \dots, \vect{b}_r\) be the minimal elements of \(\vectSet{X}\) w.r.t. \(\leq_{\PX}\), observe that \(\vectSet{X}=\{\vect{b}_1, \dots, \vect{b}_r\}+\PX\) and we are done.

``\(\Rightarrow\)'': Write \(\vectSet{X}=\{\vect{b}_1, \dots, \vect{b}_r\}+\N(\vectSet{F})\). Write \(\vectSet{P}:=\N(\vectSet{F})\). We will first show that \((\PX, \leq_{\PX})\) is a wqo by proving the claim \(\Q_{\geq 0} \cdot \vectSet{P}=\Q_{\geq 0} \cdot \PX\) and then using Lemma \ref{LemmaFinitelyGeneratedPeriodicSets}.

By definition of \(\PX\) we have \(\vectSet{P} \subseteq \PX\). Observe that the other inclusion is not always true, as \( \vectSet{X}=\{0,1\}+2\N\) fulfills \(\PX=\N\). On the other hand we claim that \(r! \cdot \PX \subseteq \vectSet{P}\), where \(r!\) denotes the factorial of \(r\), which would finish the proof of \(\Q_{\geq 0} \vectSet{P}=\Q_{\geq 0} \PX\). 

Let \(\vect{p} \in \PX\) be any vector. Consider the map \(f_{\vect{p}} \colon \vectSet{X} \to \vectSet{X}, \vect{x} \mapsto \vect{x}+\vect{p}\). We will prove that this map induces a map \(\tau \colon \{1,\dots, r\} \to \{1,\dots, r\}\). The important observation is that if we have \(f_{\vect{p}}(\vect{b}_i) \in \vect{b}_j +\vectSet{P}\), then in fact also \(f_{\vect{p}}(\vect{b}_i+\vectSet{P}) \subseteq \vect{b}_j + \vectSet{P}\), since 

\(f_{\vect{p}}(\vect{b}_i+\vectSet{P})=f_{\vect{p}}(\vect{b}_i)+\vectSet{P} \subseteq (\vect{b}_j + \vectSet{P})+\vectSet{P} \subseteq \vect{b}_j + (\vectSet{P}+\vectSet{P}) \subseteq \vect{b}_j+\vectSet{P}\).

For every \(i\), since \(f_{\vect{p}}(\vect{b}_i) \in \vectSet{X}\), there is \(\tau(i)=j\) with \(f_{\vect{p}}(\vect{b}_i) \in \vect{b}_j + \vectSet{P}\). Make any such choice of \(\tau\), thereby defining a map \(\tau \colon \{1,\dots, r\} \to \{1,\dots, r\}\). Since \(\tau\) is a map between finite sets, \(\tau\) has some cycle. Let \(n\) be the length of this cycle, and \(i\) some element of the cycle. Since \(n\)-fold application of \(f_{\vect{p}}\) causes \(\vect{b}_i\) to move back into \(\vect{b}_i+\vectSet{P}\), we obtain \(\vect{b}_i+n \vect{p} \in \vect{b}_i+\vectSet{P}\), which implies \(n \vect{p} \in \vectSet{P}\) and also \(r! \vect{p} \in \vectSet{P}\). The claim is therefore proven.

By Lemma \ref{LemmaFinitelyGeneratedPeriodicSets} we obtain the following implications: \(\vectSet{P}=\N(\vectSet{F})\) is \(\N\)-f.g. implies \(\Q_{\geq 0}\vectSet{P}=\Q_{\geq 0} \PX\) is \(\Q_{\geq 0}\)-f.g. implies that \((\PX, \leq_{\PX})\) is a wqo. Observe that \((\PX, \leq_{\PX}) \simeq (\vect{b}_i+\PX, \leq_{\PX})\) for all \(i\), because adding the vector \(\vect{b}_i\) on both sides does not influence the definition \(\vect{y}=\vect{x}+\vect{p}\). Since finite unions of wqo's are wqos, we obtain that \((\vectSet{X}, \leq_{\PX})\) is a wqo as claimed.
\end{proof}

\begin{corollary}
A set \(\vectSet{X}\) is linear if and only if \((\vectSet{X}, \leq_{\PX})\) is a wqo with a unique minimal element.
\end{corollary}

However, rather than a unique minimal element, an important condition for a wqo is \emph{directedness}.

\begin{definition}
Let \((\vectSet{X}, \leq_{\vectSet{X}})\) be a wqo. We say that \((\vectSet{X}, \leq_{\vectSet{X}})\) is \emph{directed} if for all \(\vect{x},\vect{y} \in \vectSet{X}\), there exists \(\vect{z} \in \vectSet{X}\) such that \(\vect{x} \leq_{\vectSet{X}} \vect{z}\) and \(\vect{y} \leq_{\vectSet{X}} \vect{z}\). 

\(\vectSet{X}\) is \emph{directed hybridlinear} if \((\vectSet{X}, \leq_{\PX})\) is a directed wqo.
\end{definition}

A good intuition is that a set \(\vectSet{L}\) is directed hybridlinear if it is equal to a linear set minus finitely many points. Though as a formal statement this is only true in dimension \(2\): Starting in dimension \(3\) for example the set \(\{(1,1,0), (1,0,1),(0,1,1)\}+\N^3\) is directed hybridlinear, but missing three lines compared to the linear set \(\N^3\). In dimension \(4\) we can remove planes from \(\N^4\) and so on.

Next we give equivalent definitions of directed hybridlinear, preventing descriptions like \(\N=\{0,1\}+(2\N)\) in the sequel.

\begin{restatable}{lemma}{LemmaDirectedHybridlinearEquivalence}\label{LemmaDirectedHybridlinearEquivalence}
Let \(\vectSet{X}\) be a hybridlinear set. T.F.A.E.:
\begin{enumerate}
\item \(\vectSet{X}\) is directed hybridlinear.
\item There exists a representation \(\vectSet{X}=\{\vect{b}_1, \dots, \vect{b}_r\}+\N(\vectSet{F})\) s.t. \((\vect{b}_i+\N(\vectSet{F})) \cap (\vect{b}_j+\N(\vectSet{F})) \neq \emptyset\) for all \(1 \leq i,j \leq r\).
\item There exists a representation \(\vectSet{X}=\{\vect{b}_1, \dots, \vect{b}_r\}+\N(\vectSet{F})\) s.t. \(b_i-b_j \in \Z(\vectSet{F})\) for all \(1 \leq i,j \leq r\).
\end{enumerate}
\end{restatable}

\begin{proof}
For \(\vectSet{X}=\emptyset\) all statements hold.

(1) \(\Rightarrow\) (2): Since \((\vectSet{X}, \leq_{\PX})\) is a wqo, \(\vectSet{X}\) has finitely many minimal elements \(\vect{b}_1, \dots, \vect{b}_r\). We have \(\vectSet{X}=\{\vect{b}_1, \dots, \vect{b}_r\}+\PX\). We claim that this representation fulfills 2).

Proof of claim: Observe first that since \((\vectSet{X}, \leq_{\PX})\) is a wqo, also \((\PX, \leq_{\PX})\) is a wqo, and hence by Lemma \ref{LemmaFinitelyGeneratedPeriodicSets} \(\PX\) is \(\N\)-f.g.. Using that \((\vectSet{X}, \leq_{\PX})\) is directed for each pair \((\vect{b}_i, \vect{b}_j)\) with \(1 \leq i < j \leq r\), we obtain elements \(\vect{z}_{i,j}\) such that \(\vect{b}_i \leq_{\PX} \vect{z}_{i,j}\) and \(\vect{b}_j \leq_{\PX} \vect{z}_{i,j}\), i.e.\ we have \(\vect{b}_i+\vect{p}=\vect{z}_{i,j}=\vect{b}_j+\vect{p}'\) for some \(\vect{p}, \vect{p}' \in \PX\). Hence \(\vect{z}_{i,j} \in (\vect{b}_i + \PX) \cap (\vect{b}_j + \PX)\), proving non-emptiness.

(2) \(\Rightarrow\) (3): Let \(\{\vect{b}_1, \dots, \vect{b}_r\}+\N(\vectSet{F})\) be such a representation. We claim that the same representation works for 3). Let \(1 \leq i,j \leq r\). By (2) there exists \(\vect{z}_{i,j} \in (\vect{b}_i + \N(\vectSet{F})) \cap (\vect{b}_j+\N(\vectSet{F}))\). I.e. \(\vect{z}_{i,j}-\vect{b}_i \in \N(\vectSet{F})\) and \(\vect{b}_j - \vect{z}_{i,j} \in - \N(\vectSet{F})\). Hence \(\vect{b}_j - \vect{b}_i=(\vect{b}_j - \vect{z}_{i,j})+(\vect{z}_{i,j}-\vect{b}_i) \in \Z(\vectSet{F})\).

(3) \(\Rightarrow\) (1): Let \(\vectSet{X}=\{\vect{b}_1, \dots, \vect{b}_r\}+\N(\vectSet{F})\) be such a representation. Let \(\vect{x}=\vect{b}_i+\vect{p}_{\vect{x}}\) and \(\vect{y}=\vect{b}_j+\vect{p}_{\vect{y}}\) be arbitrary points in \(\vectSet{X}\). First observe that \(\N(\vectSet{F}) \subseteq \PX\), hence it is enough to show that there exists a point \(\vect{z} \in \vectSet{X}\) with \(\vect{x} \leq_{\N(\vectSet{F})} \vect{z}\) and \(\vect{y} \leq_{\N(\vectSet{F})} \vect{z}\). 

Write \(\vect{b}_j - \vect{b}_i=\vect{p}_+ - \vect{p}_-\) with \(\vect{p}_+, \vect{p}_- \in \N(\vectSet{F})\).
Therefore \(\vect{b}_i+\vect{p}_+ = \vect{b}_j+\vect{p}_-\).
Define \(\vect{z}:=\vect{b}_i+\vect{p}_{\vect{x}}+\vect{p}_{\vect{y}} +\vect{p}_+ \in \vectSet{X}\). We have \(\vect{z}=\vect{x}+(\vect{p}_{\vect{y}} + \vect{p}_+) \geq_{\N(\vectSet{F})} \vect{x}\). We also have \(\vect{z}=\vect{y}+(\vect{p}_{\vect{x}}+\vect{p}_-) \geq_{\N(\vectSet{F})} \vect{y}\).
\end{proof}

For an example of a directed hybridlinear set which is not linear, see \(\vectSet{L}_1 \cap \vectSet{L}_2\) on the right of Figure \ref{FigureIntuitionCones}. In fact the figure even shows that a non-degenerate intersection [remember this means that \(\dim(\vectSet{L}_1 \cap \vectSet{L}_2)=\dim(\vectSet{L}_1)=\dim(\vectSet{L}_2)\)] of linear sets is not necessarily linear anymore. On the other hand, an example of a hybridlinear set which is not directed is the union of two parallel lines, for example \(\{(0,0), (0,1)\}+ \N (1,0) \subseteq \N^2\). In our algorithms, such hybridlinear sets would create problems, since the disjoint components cannot interact, and should be treated differently.

Remember the main goal of this part was to obtain the closure property under non-degenerate intersection. We only need one last lemma from \cite{GuttenbergRE23}.

\begin{lemma}[\cite{GuttenbergRE23}, Prop. 3.9 (3.), special case of finitely generated]
Let \(\vectSet{P}:=\N(\vectSet{F})\) and \(\vectSet{P}':=\N(\vectSet{F}')\) be \(\N\)-f.g. sets with a non-degenerate intersection. Then \(\Z(\vectSet{F}) \cap \Z(\vectSet{F}')=\Z(\vectSet{P} \cap \vectSet{P}')\).\label{LemmaFinitelyGeneratedNondegenerateIntersection}%
\end{lemma}

Essentially this says that if the sets \(\vectSet{P}\) and \(\vectSet{P}'\) are ``similar enough'', in the sense of having a non-degenerate intersection, then the  \(\Z\)-generated set of the intersection is the intersection of the original \(\Z\)-generated sets. Clearly this will not be correct otherwise, simply intersect sets \(\vectSet{P}:=\N, \vectSet{P}':=-\N\) ``in opposite directions'', then \(\vectSet{P} \cap \vectSet{P}'=\{\vect{0}\}\) and hence \(\Z(\vectSet{P} \cap \vectSet{P}')=\{0\}\), but on the other hand we have \(\Z(\vectSet{P})=\Z=\Z(\vectSet{P}')\), and therefore \(\Z(\vectSet{P})\cap \Z(\vectSet{P}')=\Z\) is larger. Next we show the closure property.

\LemmaDirectedHybridlinearNondegenerateIntersection*

\begin{proof}
Write \(\vectSet{L}=\{\vect{b}_1, \dots, \vect{b}_r\}+\N(\vectSet{F})\) and \(\vectSet{L}'=\{\vect{c}_1, \dots, \vect{c}_s\}+\N(\vectSet{F}')\), where these representations fulfill (3) of Lemma \ref{LemmaDirectedHybridlinearEquivalence}.
By Proposition~\ref{PropositionCharacterizeHybridlinear} orders $(\vectSet{L}, \leq_{\N(\vectSet{F})})$ and $(\vectSet{L}', \leq_{\N(\vectSet{F'})})$ are wqos.
By Dickson's lemma the intersection \((\vectSet{L} \cap \vectSet{L}', \leq_{\N(\vectSet{F})} \cap \leq_{\N(\vectSet{F}')})\) is again a wqo. Hence there exist finitely many minimal elements \(\{\vect{d}_1, \dots, \vect{d}_k\}\) of \(\vectSet{L} \cap \vectSet{L}'\). By definition of \(\leq_{\N(\vectSet{F})} \cap \leq_{\N(\vectSet{F}')}=\leq_{\N(\vectSet{F}) \cap \N(\vectSet{F})'}\) we obtain \(\vectSet{L} \cap \vectSet{L}'=\{\vect{d}_1, \dots, \vect{d}_k\}+(\N(\vectSet{F}) \cap \N(\vectSet{F}'))\) is hybridlinear, directedness remains. 

We prove that this is a representation fulfilling (3) of Lemma \ref{LemmaDirectedHybridlinearEquivalence}. Hence let \(1 \leq i,j \leq k\). By Lemma \ref{LemmaFinitelyGeneratedNondegenerateIntersection} it is enough to prove that \(\vect{d}_j - \vect{d}_i \in \Z(\vectSet{F})\) and \(\vect{d}_j - \vect{d}_i \in \Z(\vectSet{F}')\). We show the first, the second follows by symmetry. We have \(\vect{d}_i, \vect{d}_j \in \vectSet{L} \cap \vectSet{L}' \subseteq \vectSet{L}\), hence we can write \(\vect{d}_i=\vect{b}_m+\vect{p}\) and \(\vect{d}_j=\vect{b}_n + \vect{p}'\) for some \(1 \leq m,n \leq r\) and \(\vect{p}, \vect{p}' \in \N(\vectSet{F})\). Since we chose $\vect{b}_i$ satisfying point (3) of Lemma \ref{LemmaDirectedHybridlinearEquivalence}, we have $\vect{b}_n - \vect{b}_m \in \Z(\vectSet{F})$ and therefore:

\(\vect{d}_j-\vect{d}_i=(\vect{b}_n - \vect{b}_m)+(\vect{p}' - \vect{p}) \in \Z(\vectSet{F})+\Z(\vectSet{F})=\Z(\vectSet{F})\).
\end{proof}

The other property we wanted to show is in Lemma \ref{LemmaReduceDimension}. Let us start with a useful notation. Let \(\vectSet{L}=\vectSet{B}+\N(\vectSet{F})\) be a directed hybridlinear set, and \(\vect{x} \in \vectSet{L}\) a point. We write \(\UpwardClosureL{\vect{x}}:=\vect{x}+\N(\vectSet{F})\) for the ``upward-closure'' of the point \(\vect{x}\).

\begin{lemma}
Let \(\vectSet{L}\) be directed hybridlinear, and \(\vect{x} \in \vectSet{L}\). 

Then \(\dim(\vectSet{L} \setminus \UpwardClosureL{\vect{x}})<\dim(\vectSet{L})\). \label{LemmaReduceDimension}
\end{lemma}

\begin{proof}
For the case of \(\vectSet{L}\) linear see Corollary D.3 of \cite{Leroux13}/ remember Lemma \ref{LemmaDimensionDecreaseShiftedL}. We will now prove the lemma by reducing to the linear case.

Write \(\vectSet{L}=\{\vect{b}_1, \dots, \vect{b}_r\}+\N(\vectSet{F})\). Since \(\vectSet{L}\) is directed, there exists a point \(\vect{y} \in \UpwardClosureL{\vect{x}} \cap \bigcap_{i=1}^r \UpwardClosureL{\vect{b}_i}\). In particular \(\UpwardClosureL{\vect{y}} \subseteq \UpwardClosureL{\vect{x}}\), hence we have 
\[\vectSet{L} \setminus (\UpwardClosureL{\vect{x}}) \subseteq \vectSet{L} \setminus (\UpwardClosureL{\vect{y}}) \subseteq \bigcup_{i=1}^r [(\UpwardClosureL{\vect{b}_i}) \setminus (\UpwardClosureL{\vect{y}})]\]
We obtain \(\dim(\UpwardClosureL{\vect{b}_i} \setminus \UpwardClosureL{\vect{y}}) < \dim(\UpwardClosureL{\vect{b}_i})\leq \dim(\vectSet{L})\) by the linear case.
\end{proof}








\subsection{Appendix of Section \ref{SectionHybridization}, Part 2}

In this section of the appendix we prove that Definition \ref{DefinitionGoodOverapproximation} of nice overapproximation (updated to directed hybridlinear) fulfills most of the axioms (namely except Axiom 7). We restate them for convenience.

\DefinitionAxioms*

\LemmaModelAxiomsVASSnz*

\begin{proof}
Axiom 1: By definition.

Axiom 2: By Theorem \ref{TheoremVASSnzIdealDecomposition}.

Axiom 3: Let \(\vect{x} \in \vectSet{L}\) and \(\vect{w} \in \N_{\geq 1}(\vectSet{F})\). We choose \(N:=0\) and observe \(\vect{x}+\N \vect{w} \subseteq \vectSet{L}\) as required.

Axiom 4: By Lemma \ref{LemmaDirectedHybridlinearNondegenerateIntersection}, \(\vectSet{L} \cap \vectSet{L}'\) is again directed hybridlinear. By Lemma \ref{LemmaShiftGoodOverapproximation} we have \((\vectSet{X} \cap \vectSet{L}') \HybridizationRelation (\vectSet{L} \cap \vectSet{L}')\) and \((\vectSet{L} \cap \vectSet{X}) \HybridizationRelation (\vectSet{L} \cap \vectSet{L}')\). Now let \(\vect{x} \in \vectSet{L} \cap \vectSet{L}'\) and \(\vect{w}\) as in the definition of \(\HybridizationRelation\). Since \((\vectSet{X} \cap \vectSet{L}') \HybridizationRelation (\vectSet{L} \cap \vectSet{L}')\), there exists \(N_1 \in \N\) s.t. \(\vect{x}+\N_{\geq N_1} \vect{w} \subseteq \vectSet{X}\), and since \((\vectSet{L} \cap \vectSet{X}) \HybridizationRelation (\vectSet{L} \cap \vectSet{L}')\), there exists \(N_2 \in \N\) s.t. \(\vect{x}+\N_{\geq N_2} \vect{w} \subseteq \vectSet{X}'\). Choosing \(N:=\max \{N_1, N_2\}\) we are done.

Axiom 5: Observe that \(\HybridizationRelation\) states that certain lines are in \(\vectSet{X}\). Clearly increasing \(\vectSet{X}\) preserves this.

Axiom 6: Proof by contraposition. I.e.\ let \(\vectSet{X} \HybridizationRelation \vectSet{L}\) and assume \(\vectSet{X}\) is semilinear. We have to prove that \(\vectSet{X}\) is reducible. By Axiom 8 \(\exists \vect{x}\) s.t. \(\vect{x}+\vectSet{L} \subseteq \vectSet{X}\), i.e.\ \(\vectSet{X}\) is reducible.

Axiom 8: Define \(\vectSet{S}':=\vectSet{L} \setminus \vectSet{S}\). Write \(\vectSet{S}'=\bigcup_{j=1}^k \vectSet{L}_j\) as a union of linear sets. Write \(\vectSet{L}_j=\vect{b}_j+\N(\vectSet{F}_j)\) for all \(j\). Let \[I:=\{j \in \{1,\dots, k\} \mid \N(\vectSet{F}_j) \cap \N_{\geq 1}(\vectSet{F}) \neq \emptyset\}\] be the set of indices such that \(\vectSet{L}_j\) is not parallel to the boundary of \(\vectSet{L}\). We have to prove that \(I=\emptyset\). 

To this end, let \(j \in I\). Let \(\vect{w}_j \in \N(\vectSet{F}_j) \cap \N_{\geq 1}(\vectSet{F})\) as per Definition of \(j \in I\). Since \(\vectSet{X} \HybridizationRelation \vectSet{L}\), there exists \(N \in \N\) s.t. \(\vect{b}_j+N \vect{w}_j \in \vectSet{X}\), in particular \(\vectSet{X} \cap \vectSet{L}_j \neq \emptyset\). 

However, \(\vectSet{X} \cap \vectSet{L}_j \subseteq \vectSet{S} \cap (\vectSet{L} \setminus \vectSet{S})=\emptyset\), contradiction.

Therefore all \(\vectSet{L}_j\) for \(j=1, \dots, k\) are parallel to the boundary of \(\vectSet{L}\) as claimed, and we now simply choose \(\vect{x} \in \N(\vectSet{F})\) large enough that \((\vect{x}+\vectSet{L}) \cap \vectSet{S}'=\emptyset\). 
%Formally, we write \(\vect{b}_j=\vect{b}+\vect{x}_j\) for all \(j\), and choose \(\vect{x}:=(\sum_{\vect{f} \in \vectSet{F}} \vect{f})+\sum_{j=1}^k \vect{x}_j\). Then 
%
%\[\vect{x}+\vectSet{L}=(\vect{b}+\vect{x}+\N(\vectSet{F}))\subseteq \vect{b}_j + \N_{\geq 1}(\vectSet{F}),\] implying empty intersection with \(\vectSet{L}_j\) for all \(j\).
\end{proof}

As mentioned Axiom 7 is proven in the next section. We cannot end this section without once again pointing at the proof of Axiom 8, which proves a complicated result using simple algebraic methods without any complicated calculation.

\subsection{Appendix of Section \ref{SectionHybridization}, Part 3}

In this subsection we introduce some heavy VASS theory in order to deal with Axiom 7. The essence of the proof is the following definition.

\begin{definition}
Let \(\vectSet{X} \subseteq \N^d\) be any set. A vector \(\vect{v} \in \N^d\) is a \emph{pump} of \(\vectSet{X}\) if there exists a point \(\vect{x}\) s.t. \(\vect{x}+\N \vect{v} \subseteq \vectSet{X}\).

The set of all pumps of \(\vectSet{X}\) is denoted \(\Pumps(\vectSet{X})\).
%A vector \(\vect{v} \in \N^d\) is a \emph{direction} of \(\vectSet{X}\) if there exists \(n \in \N\) and a point \(\vect{x}\) s.t. \(\vect{x}+\N n \vect{v} \subseteq \vectSet{X}\).
%
%The set of directions of \(\vectSet{X}\) is denoted \(\dir(\vectSet{X})\). 
\end{definition}

I.e.\ a vector \(\vect{v}\) is a pump of \(\vectSet{X}\) if some infinite line with step \(\vect{v}\) is contained in \(\vectSet{X}\). For example the pumps of a linear set \(\N(\vectSet{F})\) are given as follows:

\begin{lemma}\cite[Lemma F.1, special case]{Leroux13}
Let \(\N(\vectSet{F})\) be \(\N\)-f.g.. Then \(\Pumps(\N(\vectSet{F}))=\Q_{\geq 0}(\vectSet{F}) \cap \Z(\vectSet{F})\). \label{LemmaPumpsLinearSet}
\end{lemma}

Observe in particular that the \(\Pumps\) are not only \(\N(\vectSet{F})\), but slightly more: Consider the left of Figure \ref{FigureIntuitionDoublePumps} for an example for the difference. Intuitively, we fill out the ``cone''.

\begin{figure}[h!]
\begin{minipage}{4.5cm}
\begin{tikzpicture}
\begin{axis}[
    axis lines = left,
    xlabel = { },
    ylabel = { },
    xmin=0, xmax=5,
    ymin=0, ymax=15,
    xtick={0,1,2,3,4,5},
    %ytick={0,1,2,3,4,5,6,7,8,9,10,11,12,13,14,15,16},
    ytick={0,3,6,9,12,15},
    ymajorgrids=true,
    xmajorgrids=true,
    thick,
    smooth,
    no markers,
]

\addplot[
    color=blue,
]
{3 * x};

%\addplot[
%    color=black,
%    very thick,
%]
%coordinates {
%(0.33, 1)(5,1)
%};

\addplot+[
    color=red,
    name path=A,
    domain=2:5, 
]
{3};

\addplot+[
    color=red,
    name path=B,
    domain=2:5, 
]
{3*x-3};

\addplot[
    color=red!40,
]
fill between[of=A and B];


\addplot[
    fill=blue,
    fill opacity=0.5,
    only marks,
    ]
    coordinates {
    (0,0)(1,0)(1,2)(1,3)(2,0)(2,2)(2,3)(2,4)(2,5)(2,6)(3,0)(3,2)(3,3)(3,4)(3,5)(3,6)(3,7)(3,8)(3,9)(4,0)(4,2)(4,3)(4,4)(4,5)(4,6)(4,7)(4,8)(4,9)(4,10)(4,11)(4,12)(5,0)(5,2)(5,3)(5,4)(5,5)(5,6)(5,7)(5,8)(5,9)(5,10)(5,11)(5,12)(5,13)(5,14)(5,15)
    };

\end{axis}
\end{tikzpicture}
\end{minipage}%
\begin{minipage}{4.5cm}
\begin{tikzpicture}
\begin{axis}[
    axis lines = left,
    xlabel = { },
    ylabel = { },
    xmin=0, xmax=4,
    ymin=0, ymax=16,
    xtick={0,1,2,3,4},
    ytick={0,4,8,12,16},
    ymajorgrids=true,
    xmajorgrids=true,
    thick,
    smooth,
    no markers,
]
    
\addplot+[
    name path=E,
    domain=0:4,
    color=blue,
]
{x^2};
    
\addplot+[
    name path=F,
    domain=0:4,
    color=blue,
]
coordinates {(0,0) (4,0)};
    
\addplot[blue!40] fill between[of=E and F];

\end{axis}
\end{tikzpicture}
\end{minipage}%

\caption{\textit{Left}: Consider the following example of an \(\N\)-g. set from \cite{GuttenbergRE23}: \(\N(\vectSet{F})\) for \(\vectSet{F}:=\{(1,0), (1,2), (1,3)\}\). This set fulfills \(\Q_{\geq 0}(\vectSet{F}) \cap \Z(\vectSet{F})=\{(x,y) \mid 0 \leq y \leq 3x\}\), and is almost equal to this, except for the points with \(y=1\): These are \emph{holes}, as they were called in \cite{GuttenbergRE23}. \newline
\textit{Right}: The blue \(\{(x,y) \in \N^2 \mid 0 \leq y \leq x^2\}\) has the set of Pumps \(\{(0,0)\} \cup \{(x,y) \in \N^2 \mid x>0\}\). After the second application we stabilize to \(\N^2\). This happens with any set which has a partner \(\vectSet{L}\) for \(\vectSet{X} \HybridizationRelation \vectSet{L}\): The second application of \(\Pumps\) adds the boundary, stabilizing to \(\vectSet{L}\).}\label{FigureIntuitionDoublePumps}
\end{figure}

\textbf{Essence of the proof/algorithm}: The essence of the algorithm for Axiom 7 will be simple: Use Axiom 2, and for every obtained \(\RelationClass\)-KLM sequence \(\vectSet{X}\) compute a semilinear representation of \(\Pumps(\vectSet{X})\), which can then be used to decide Axiom 7. The difficult part of this is visible at first glance: Given an arbitrary VASSnz section \(\vectSet{X}\), which can be highly non-semilinear, why would \(\Pumps(\vectSet{X})\) even be semilinear in the first place? The proof is extremely complicated, and was in case of VASS given by Leroux in \cite{Leroux13} and then extended to VASSnz by \cite{Guttenberg24}. We have to trace their ideas here, which requires basic understanding about the \(\Pumps\) operator.

\textbf{Essentials about the Pumps operator}: We start with the observation that \(\Pumps\) is a monotone operator, and we know a large class of fixed points of \(\Pumps\): Both euclidean closed \(\Q_{\geq 0}\)-g. sets (by Lemma \ref{LemmaFinitelyGeneratedCone} in particular all \(\Q_{\geq 0}\)-\emph{f}.g. sets fulfill this) as well as \(\Z\)-g. sets are fixed points of \(\Pumps\).

\begin{lemma} \label{LemmaObviousPropertiesOfPumps}
Let \(\vectSet{C}\) be \(\Q_{\geq 0}\)-g. and euclidean closed. 

Then \(\Pumps(\vectSet{C})=\vectSet{C}\).

Let \(\Z(\vectSet{F})\) be \(\Z\)-g.. Then \(\Pumps(\vectSet{L})=\vectSet{L}\).

Let \(\vectSet{X} \subseteq \vectSet{X}'\). Then \(\Pumps(\vectSet{X}) \subseteq \Pumps(\vectSet{X}')\). 
\end{lemma}

\begin{proof}
``\(\subseteq\)'': Let \(\vect{v} \in \vectSet{C}\). Then \(\N \vect{v} \subseteq \Q_{\geq 0}(\vect{v}) \subseteq \vectSet{C}\), and we obtain \(\vect{v} \in \Pumps(\vectSet{C})\).

``\(\supseteq\)'': Let \(\vect{v} \in \Pumps(\vectSet{C})\). We have to show \(\vect{v} \in \vectSet{C}\). Since \(\vect{v} \in \Pumps(\vectSet{C})\), we obtain \(\vect{x}+\N \vect{v} \subseteq \vectSet{C}\) for some \(\vect{x}\). We define \(\vect{x}_n:=\frac{1}{n}(\vect{x}+n \vect{v}) \in \vectSet{C}\). Since \(\vectSet{C}\) is euclidean closed, we obtain that the limit \(\vect{v}=\lim_{n \to \infty} \vect{x}_n \in \vectSet{C}\).

``\(\subseteq\)'': Similar to above we have \(\N \vect{v} \subseteq \Z(\vect{v}) \subseteq \Z(\vectSet{F})\).

``\(\supseteq\)'': Let \(\vect{v} \in \Pumps(\Z(\vectSet{F}))\). Then \(\exists \vect{x}\) s.t. \(\vect{x}+\N \vect{v} \subseteq \Z(\vectSet{F})\). Hence \(\vect{v}=(\vect{x}+\vect{v})-\vect{x} \in \Z(\vectSet{F})-\Z(\vectSet{F})=\Z(\vectSet{F})\).

Monotonicity of Pumps is obvious.
\end{proof}

Now we can extend Lemma \ref{LemmaPumpsLinearSet} to directed hybridlinear sets \(\vectSet{L}\) (while Lemma \ref{LemmaPumpsLinearSet} does not hold for \(\{0,1\}+2\N\)!):

\begin{lemma} \label{LemmaPumpsDirectedHybridlinear}
Let \(\vectSet{L}=\vectSet{B}+\N(\vectSet{F})\) be directed hybridlinear. Then \(\Pumps(\vectSet{L})=\Q_{\geq 0}(\vectSet{F}) \cap \Z(\vectSet{F})\).
\end{lemma}

\begin{proof}
``\(\supseteq\)'' follows from Lemma \ref{LemmaPumpsLinearSet}.

``\(\subseteq\)'': Since \(\Q_{\geq 0}(\vectSet{F})\) is a closed cone, we obtain \(\Pumps(\vectSet{L}) \subseteq \Q_{\geq 0}(\vectSet{F})\) by Lemma \ref{LemmaObviousPropertiesOfPumps}. It remains to prove \(\Pumps(\vectSet{L}) \subseteq \Z(\vectSet{F})\). Let \(\vect{v} \in \Pumps(\vectSet{L})\). Then there exists \(\vect{x}\) s.t. \(\vect{x}+\N \vect{v} \subseteq \vectSet{L}\). Let \(\vect{x}_0:=\vect{x}\) and \(\vect{x}_1:=\vect{x}+\vect{v}\). Write \(\vect{x}_1=\vect{b}_1+\vect{w}_1\) and \(\vect{x}_0=\vect{b}_0+\vect{w}_0\) with \(\vect{w}_0, \vect{w}_1 \in \N(\vectSet{F})\) and \(\vect{b}_0, \vect{b}_1 \in \vectSet{B}\). By Lemma \ref{LemmaDirectedHybridlinearEquivalence}(2)/the definition of directed hybridlinear we gave in the paper, we have \(\vect{b}_1-\vect{b}_0 \in \N(\vectSet{F})-\N(\vectSet{F})=\Z(\vectSet{F})\). This implies 
\[\vect{v}=\vect{x}_1-\vect{x}_0=(\vect{b}_1-\vect{b}_0)+(\vect{w}_1-\vect{w}_0) \in \Z(\vectSet{F})+\Z(\vectSet{F})=\Z(\vectSet{F}),\]
finishing the proof.
\end{proof}

The relation of pumps to our problem is that if \(\vectSet{X} \HybridizationRelation \vectSet{L}\) holds, then we have information not about \(\Pumps(\vectSet{X})\), but about the \emph{second} application of the \(\Pumps\) operator. The right of Figure \ref{FigureIntuitionDoublePumps} gives an idea about why.

\begin{lemma} \label{LemmaPumpsOfNiceSet}
Let \(\vectSet{X}\) be \(\N\)-g., and \(\vectSet{X} \HybridizationRelation \vectSet{L}=\vectSet{B}+\N(\vectSet{F})\). Then \(\Pumps(\Pumps(\vectSet{X}))=\Q_{\geq 0}(\vectSet{F}) \cap \Z(\vectSet{F})\).
\end{lemma}

\begin{proof}
Write \(\vectSet{L}=\vectSet{B}+\N(\vectSet{F})\). By definition of \(\HybridizationRelation\), we have \(\N_{\geq 1}(\vectSet{F}) \subseteq \Pumps(\vectSet{X})\). Since \(\Pumps\) is monotone, by Lemma \ref{LemmaPumpsDirectedHybridlinear} we obtain \(\Q_{\geq 0}(\vectSet{F}) \cap \Z(\vectSet{F}) = \Pumps(\vectSet{L})=\Pumps(\N_{\geq 1}(\vectSet{F})) \subseteq \Pumps(\Pumps(\vectSet{X}))\). Furthermore, by Lemma \ref{LemmaObviousPropertiesOfPumps}, no matter how often we apply pumps on \(\vectSet{L}\), we will remain at \(\Pumps(\vectSet{L})=\Q_{\geq 0}(\vectSet{F}) \cap \Z(\vectSet{F})\), since these are a closed \(\Q_{\geq 0}\)-g. set and a \(\Z\)-g. set. Since \(\vectSet{X} \subseteq \vectSet{L}\) and pumps is a monotone operator, we obtain that also arbitrarily many applications of pumps on \(\vectSet{X}\) lead to at most \(\Q_{\geq 0}(\vectSet{F}) \cap \Z(\vectSet{F})\). In total we obtain \(\Pumps(\Pumps(\vectSet{X}))=\Q_{\geq 0}(\vectSet{F}) \cap \Z(\vectSet{F})\).
\end{proof}

Applying Lemma \ref{LemmaPumpsOfNiceSet} for a KLM sequence \(\vectSet{X}\), since we have shown \(\vectSet{X} \HybridizationRelation \pi_{\text{src}, \text{tgt}}(\sol(\CharSys(\vectSet{X})))\), we know that \(\Pumps(\Pumps(\vectSet{X}))=\Pumps(\pi_{\text{src}, \text{tgt}}(\sol(\CharSys(\vectSet{X}))))\) is semilinear, in fact of the form \(\Q_{\geq 0}(\vectSet{F}) \cap \Z(\vectSet{F})\). But as in the right of Figure \ref{FigureIntuitionDoublePumps} one question remains: Which parts of the boundary are in the actual set \(\Pumps(\vectSet{X})\)?

The answer for a single SCC is a formula first given by Leroux in \cite{Leroux13} and then extended to VASSnz in \cite{Guttenberg24}. 

First we need a definition.

\begin{definition}
Let \(\vectSet{X} \subseteq \N^d\), and \(\vect{v} \in \N^d\). Then \(\vect{v}\) is a \emph{direction} of \(\vectSet{X}\) if \(\exists n \in \N\) s.t. \(n \vect{v} \in \Pumps(\vectSet{X})\).

The set of all directions is denoted \(\dir(\vectSet{X})\).
\end{definition}

I.e.\ a direction is simply an arbitrarily scaled pump. This has the advantage that for many sets, \(\dir(\vectSet{X})\) is \(\Q_{\geq 0}\)-g., a property we will use in the following.

Let \(\VAS_i\) be an SCC inside a \(\RelationClass\)-KLM sequence \(\vectSet{X}\). Let \(\vectSet{C}\) be the set of pairs \((\vect{e}, \vect{f})\in \N^{d_i} \times \N^{d_i}\) s.t. \(\vect{f}-\vect{e}\) is the effect of some cycle in \(\VAS_i\). \emph{Beware}: We do \emph{not} require that the cycle uses every edge. Hence this relation \emph{cannot} be captured using a single integer linear program, it requires one per support as in the decomposition for (5). But then we can at the same time also deal with the periods: Namely we only allow periods for the edges \(e\) in the support we picked. Then 
\begin{align}
\Pumps(\vectSet{X}_i)=\dir(\vectSet{C})^{\ast} \cap \Pumps(\vectSet{L}_i), \tag{*} \label{EquationPumpsFormula}
\end{align}
where \(\ast\) is transitive closure w.r.t. composition as always.

This is not hard to explain: Since \(\vectSet{X}_i:=\vectSet{R}_i \cap \vectSet{L}_i \subseteq \vectSet{L}_i\), we must have \(\Pumps(\vectSet{X}_i) \subseteq \Pumps(\vectSet{L}_i)\), so the second half is clear. The first half removes disconnected solutions of the characteristic system, but not fully: We are allowed to perform any number \(k\) of \emph{connected} cycles in the SCC in sequence, even if the \(i\)-th cycle is not connected to the \((i+1)\)-st. The idea is also known under the name \emph{sequentially enabled cycles}: Imagine we have cycles \(\rho_1, \dots, \rho_k\) with \(\dirOfRun(\rho_j)=(\vect{v}_{j-1}, \vect{v}_j)\). Then a run \(\rho\) from \(\qini\) to \(\qfini\) which enables the cycles \(\rho_1, \dots, \rho_k\) in this sequence shows \((\vect{v}_0, \vect{v}_k) \in \Pumps(\vectSet{X}_i)\), even though the cycles \(\rho_j\) were disconnected. To see this, write \(\rho=\rho_0' \rho_1' \dots \rho_k'\), where \(\rho_j\) is enabled after performing \(\rho_j'\). Then the runs \(\rho_0' \rho_1^{\ast} \rho_1' \dots \rho_k^{\ast} \rho_k'\) prove \(\dirOfRun(\rho)+\N(\vect{v}_0, \vect{v}_k)\subseteq \Rel(\VAS_i, \qini, \qfini)\).

The one question mark which remains is: Why do we have to take \(\dir(\vectSet{C})\), and not just \(\Pumps(\vectSet{C})\)? The answer is that a transitive closure of semilinear relations can become a counter machine reachability relation, and hence undecidable. Hence we \emph{want} to take \(\dir(\vectSet{C})\) instead, and it happens to still give the correct result. However, now the \(\ast\) is computable: We have the following, where a \(\Q_{\geq 0}\)-g. relation is called \emph{definable} if it is definable in \(\FO(\Q, \leq, +)\), also called linear arithmetic:

\begin{theorem}[\cite{Leroux13}, Theorem VII.1]
Let \(\vectSet{C}_1, \dots, \vectSet{C}_k \subseteq \Q_{\geq 0}^{d_i} \times \Q_{\geq 0}^{d_i}\) be relations which are reflexive, definable and \(\Q_{\geq 0}\)-g.. Then \((\bigcup_{j=1}^k \vectSet{C}_j)^{\ast}\) is \(\Q_{\geq 0}\)-g. and \emph{again definable}.

A corresponding formula \(\varphi \in \FO(\Q, \leq, +)\) can be computed in polynomial time from formulas \(\varphi_j\) for the \(\vectSet{C}_j\).
\end{theorem}

I.e.\ for \(\Q_{\geq 0}\)-g. relations instead of semilinear relations transitive closure is easy to compute. We remark that we have to perform quantifier elimination afterwards however, which causes a possibly elementary blowup.

\textbf{Overview}: From here on the results are new. We explain how to extend this to obtain Axiom 7. We perform a \(3\)-step process: Step 1 is a proof that the formula \ref{EquationPumpsFormula} not just gives \(\Pumps(\vectSet{X}_i)\), but that \(\vectSet{X}_i\) belongs to a special class of sets we call \emph{uniform}.

Step 2 proves that a non-degenerate composition of uniform sets is again uniform, with \emph{the composition of the pumps of the parts} as new sets of pumps.

Step 3 is to finish implementing Axiom 7 using the semilinear representation for \(\Pumps(\vectSet{X})\).

\subsection{Step 1 Towards Axiom 7}

The problem with the strategies so far is that they deal with a \emph{single SCC}. It then remains to perform the composition \emph{constructively}. Namely while \cite{Leroux13} and \cite{Guttenberg24} respectively show that composition works, the new formula \emph{cannot be computed} just from the old formulas, it simply exists. 

To obtain a computational result, the fact that our compositions are non-degenerate will again come to the rescue. But first we need the proper definitions.

\begin{definition}
Let \(\vectSet{X} \subseteq \N^d\). A vector \(\vect{v}\) is a \emph{preservant} of \(\vectSet{X}\) if \(\vectSet{X}+\vect{v} \subseteq \vectSet{X}\). The set of all preservants is denoted \(\PX\).
\end{definition}

\begin{definition}
Let \(\vectSet{X} \subseteq \N^d\). Then \(\vectSet{X}\) is called \emph{well-directed} if for every infinite sequence \(\vect{x}_0, \vect{x}_1, \dots \in \vectSet{X}\) there exists an infinite set \(N \subseteq \N \) of indices (i.e.\ an infinite subsequence) s.t. \(\vect{x}_j+\N(\vect{x}_k-\vect{x}_j) \subseteq \vectSet{X}\) for all \(k>j\) in \(N\).
\end{definition}

Let us explain these two definitions. Preservants are best explained using the left of Figure \ref{FigureIntuitionPreservants}. The blue and red sets depicted there are closed under addition and contain \(\vect{0}\), hence they are their own preservants. However, the union has very little preservants, i.e.\ vectors which can be pumped everywhere: The blue and red sets are simply too different. The operator \(\Pumps\) however would not see this problem.

\begin{figure}[h!]
\begin{minipage}{4.5cm}
\begin{tikzpicture}
\begin{axis}[
    axis lines = left,
    xlabel = { },
    ylabel = { },
    xmin=0, xmax=8,
    ymin=0, ymax=8,
    xtick={0,2,4,6,8},
    %ytick={0,1,2,3,4,5,6,7,8,9,10,11,12,13,14,15,16},
    ytick={0,2,4,6,8},
    ymajorgrids=true,
    xmajorgrids=true,
    thick,
    smooth,
    no markers,
]

\addplot+[
    color=red,
    name path=A,
    domain=0:8, 
]
{log2(x+1)};

\addplot+[
    color=red,
    name path=B,
    domain=0:8, 
]
{x};

\addplot[
    color=blue!40,
]
fill between[of=A and B];


\addplot[
    fill=red,
    fill opacity=0.5,
    only marks,
    ]
    coordinates {
    (0,0)(1,0)(2,0)(3,0)(4,0)(5,0)(6,0)(7,0)(8,0)
    };

\end{axis}
\end{tikzpicture}
\end{minipage}%
\begin{minipage}{4.5cm}
\begin{tikzpicture}
\begin{axis}[
    axis lines = left,
    xlabel = { },
    ylabel = { },
    xmin=0, xmax=8,
    ymin=0, ymax=8,
    xtick={0,2,4,6,8},
    ytick={0,2,4,6,8},
    ymajorgrids=true,
    xmajorgrids=true,
    thick,
    smooth,
    no markers,
]

\addplot[
    color=red,
]
coordinates {(2,3) (8,9)};

\addplot[
    color=red,
]
coordinates {(2,2) (8,8)};
    
\addplot+[
    name path=E,
    domain=2:8,
    color=blue,
]
{8};
    
\addplot+[
    name path=F,
    domain=2:8,
    color=blue,
]
coordinates {(2,4) (8,10)};
    
\addplot[blue!40] fill between[of=E and F];

\addplot+[
    name path=G,
    domain=2:8,
    color=blue,
]
coordinates {(2,1) (9,8)};
    
\addplot+[
    name path=H,
    domain=2:8,
    color=blue,
]
coordinates {(2,1) (8,1)};
    
\addplot[blue!40] fill between[of=G and H];

\end{axis}
\end{tikzpicture}
\end{minipage}%

\caption{\textit{Left}: The set \(\vectSet{X}=\vectSet{X}_1 \cup \vectSet{X}_2\) with \(\vectSet{X}_1:=\{(x,y) \mid \log_2(x+1) \leq y \leq x\}\) and \(\vectSet{X}_2:=\{(x,y) \mid y=0\}\) (the blue and red parts respectively) fulfills \(\PX=\{0\}\). Namely a vector \((x,y)\) with \(y>0\) is not a preservant due to the red points, and if \(y=0\) then the blue points are the problem. However, \(\Pumps(\vectSet{X})=\{(x,y) \mid 0 \leq y \leq x\}\) does not see this problem at all. Intuitively, \(\Pumps(\vectSet{X})\) consider pumps which work \emph{somewhere}, and \(\PX\) contains pumps which \emph{work everywhere}. \newline
\textit{Right}: In case of the depicted blue set, we would find a complete extraction. However, clearly we are not reducible yet: We have to do a recursive call on the red lines.}\label{FigureIntuitionPreservants}
\end{figure}

Well-directed is a completely different property: It says that the set is ``well-structured enough'' s.t. between some infinitely many points, some pumping will be possible. A set which is not well-directed is for example \(\vectSet{P}:=\{(x,n) \mid x \text{has at most }n\text{ bits set in binary}\}\). Even though \(\vectSet{P}\) is closed under addition, in this set, it is extremely hard to find any infinite lines at all. We will not go into details, but due to a well-quasi-order on runs, it is guaranteed that all the sets we will consider are automatically well-directed.

Other simple observations to gain understanding about preservants is that \(\PX\) is always closed under addition and contains \(\vect{0}\), i.e.\ is \(\N\)-generated, or that the function graph of the sin/cos functions has \((2\pi,0)\) as preservant, but then automatically also all multiples \((2\pi n, 0)\).

We can now define uniform sets, a new notion which captures very well the structure of \(\RelationClass\)-KLM sequences \(\vectSet{X}\).

\begin{definition} \label{DefinitionUniform}
Let \(\emptyset \neq \vectSet{X} \subseteq \N^d\). Then \(\vectSet{X}\) is called \emph{uniform} if \(\vectSet{X}\) is well-directed and there exists a finite set \(\vectSet{B}\) s.t.:

\begin{enumerate}
\item \(\Pumps(\vectSet{X})=\Pumps(\PX)\),
\item \(\vectSet{X} \subseteq \vectSet{B}+\Pumps(\Pumps(\vectSet{X}))\),
\item \(\dir(\vectSet{X})\) is definable in \(\FO(\Q, \leq, +)\),
\item \(\vectSet{B}+\Pumps(\Pumps(\vectSet{X}))\) is directed hybridlinear.
\end{enumerate}
\end{definition}

There are a lot of parts to this definition, but first understand that the main conditions are 1+3). The others like \(\neq \emptyset\), well-directed, etc. are secondary but necessary to ensure the closure property we want. So what does 1) say? Again, the left of Figure \ref{FigureIntuitionPreservants} is a good example: \(\vectSet{X}=\vectSet{X}_1 \cup \vectSet{X}_2\) is highly non-uniform, in fact \(\Pumps(\vectSet{X})\) has dimension \(2\) while \(\Pumps(\PX)=\PX=\{\vect{0}\}\) is small. However, both \(\vectSet{X}_1\) and \(\vectSet{X}_2\) themselves are closed under addition and hence uniform. Uniform is exactly meant to capture that every point can ``pump approximately the same''. But uniform is just enough of an extension of ``closed under addition'', that \(\RelationClass\)-KLM sequences fall into it (as we will now show).

On the other hand, condition 3) is used to guarantee that \(\Pumps(\vectSet{X})\) is semilinear.

A comment for people familiar with VASS theory: In prior works most abstract statements started with ``Let \(\vectSet{X}=\vect{b}+\vectSet{P}\) for \(\vectSet{P}\) smooth''. One reason for the definition of uniform is that \emph{essentially all} these statements still work if we replace this with ``Let \(\vectSet{X}\) be uniform''. Hence we are now able to take full advantage of theory by Leroux \cite{Leroux11,Leroux12, Leroux13} and Guttenberg \cite{GuttenbergRE23, Guttenberg24}, but can do so \emph{computationally}.

\begin{lemma} \label{LemmaSingleComponentUniform}
Let \(\vectSet{X}_i=\Rel(\VAS_i, \qini, \qfini) \cap \vectSet{L}_i\) be a strongly-connected component in a perfect \(\RelationClass\)-KLM sequence for \(\RelationClass=\)VASSnzSec(k-1), the sections of \(k-1\) priority VASSnz. Then \(\vectSet{X}_i\) is uniform with \(\Pumps(\vectSet{X}_i)\) as in formula \eqref{EquationPumpsFormula}.
\end{lemma}

\begin{proof}
Proof by induction on \(k\). 

\(k=0\) (normal VASS) are a subcase of the induction step.

\(k \to k+1\): For every edge \(e\in E_i\), the relation \(\vectSet{R}(e)\) is uniform by induction. Let us consider the formula \eqref{EquationPumpsFormula}, and first prove that the pumps are at most the set given in the formula. \(\subseteq \Pumps(\vectSet{L}_i)\) is clear since \(\Pumps\) is monotone, \(\vectSet{X}_i=\vectSet{R}_i \cap \vectSet{L}_i \subseteq \vectSet{L}_i\) and \(\Pumps(\vectSet{L}_i)\) is a fixed point. Arguing containment in the first half would not be easy, but \cite{Guttenberg24} comes to the rescue: Semilinear subreachability relations of VASSnz (in particular the ones of the form \(\vect{x}+\N \vect{v}\) obtained from pumps) are \emph{flattable}, i.e.\ there exist finitely many runs \(\rho_0', \dots, \rho_r'\) and cycles \(\rho_1, \dots, \rho_r\) s.t. all of \(\vect{x}+\N \vect{v}\) can be reached using runs in \(\rho_0' \rho_1^{\ast} \rho_1' \dots \rho_r^{\ast} \rho_r'\), exactly what we need.

So the interesting direction is the other containment. Let \(\vect{v} \in \dir(\vectSet{C})^{\ast}\). Then there exist vectors \(\vect{v}_0, \vect{v}_1, \dots, \vect{v}_k\) s.t. \((\vect{v}_i, \vect{v}_{i+1}) \in \dir(\vectSet{C})\) for all \(0 \leq i \leq k-1\). Being \(\in \vectSet{C}\) means there exist \(n_i \in \N\) (we use their common denominator to use a common \(n\)) s.t. \(n(\vect{v}_{i+1}- \vect{v}_i)\) is the effect of some cycle \(\rho_i\). At this point we can explain the main two difficulties of this proof: 1): What does it even mean to prove that \(\vect{v}\) is a pump of \(\vectSet{P}_{\vectSet{X}_i}\), and not of \(\vectSet{X}_i\) itself? 2) How to remove the factor \(n\)? The reader might think of the \(j_1, j_2\) trick from Lemma \ref{LemmaLocalCorrectness}, but it does not work here: That trick requires a full support solution. For a counter example, simply consider a 1-VASS with two states \(q_1, q_2\). On state \(q_1\) we have a loop with \(+=2\), which is the direction we want to pump. On state \(q_2\) there is a loop with \(x-=1\). Clearly, in order to find a cycle with effect \(+=1\) it has to be allowed to move to state \(q_2\), i.e.\ we need a full support solution to pump \(+=1\), otherwise we can indeed only pump \(+=2\).

Let us explain the solutions to these questions. Regarding 1), in order to prove membership in \(\vectSet{P}_{\vectSet{X}_i}\), we have fix an amount of homogeneous solutions only depending on \(\vect{v}\) s.t. no matter what element of \(\vectSet{X}_i\), i.e.\ what \emph{actual} run \(\rho\), we start from, we can add the fixed amount of homogeneous and ensure that this only slightly enlarged run \(\rho'\) can pump \(\vect{v}\). Regarding 2), there is only one logical solution: We need to write \(\N=\{0, \dots, n-1\}+n\N\) and prove that just \(\rho'\) itself, but also \(n-1\) neighbouring runs \(\rho_j\) with \(\dirOfRun(\rho_j)=\dirOfRun(\rho')+j \vect{v}\) for \(0 \leq j \leq n-1\) can pump \(n\vect{v}\). Then the whole line is reachable, even though we did not find a corresponding single cycle.

We inevitably have to build in the solution to 2) first. Let \(\vect{w}_0, \dots, \vect{w}_{n-1} \in \interior(\vectSet{L}_i) \cap \vectSet{P}_{\vectSet{X}_i}\) with \(\vect{w}_{j+1}=\vect{w}_j+\vect{v}\) for all \(0 \leq j \leq n-1\). These clearly exist since \(\vect{v} \in \Pumps(\vectSet{L}_i)\) and ``almost all interior vectors'' are in \(\vectSet{P}_{\vectSet{X}_i}\) by Lemma \ref{LemmaLocalCorrectness} and the \(j_1, j_2\) trick.

Now we choose the fixed amount of homogeneous solutions to add: Fix an amount \(k\) s.t. \(k\) homogeneous solutions alone are enough to be able to embed enough up- and down-pumping sequences to perform a cycle \(\rho_{\text{enable}}\) on \(\qini\) which sequentially enables not only the \(\rho_j\) but also the loops corresponding to \(\vect{w}_0, \dots, \vect{w}_{n-1}\).

Now we have to verify our choice. Let \(\vect{x} \in \vectSet{X}_i\), i.e.\ let \(\rho\) be a run with \(\dirOfRun(\rho)=\vect{x}\) and which we know is executable. We use the run \(\rho':=\mathbf{up}^k  \rho_{\text{enable}} \mathbf{diff}^k \rho \mathbf{dwn}^k \). We argue that the run is enabled and can pump the required vectors: The run \(\mathbf{up}^k \rho_{\text{enable}} \mathbf{diff}^k\) is enabled by choice of \(k\). Also by choice of \(k\) and \(\rho_{\text{enable}}\), this only consumes the extra values we added using the homogeneous solutions. Hence the run \(\rho\) which was originally enabled is still enabled. Finally it is easy to see that also \(\mathbf{dwn}^k\) is enabled, since it was chosen as a down-pumping sequence, and \(\rho \in \vectSet{L}_i\), i.e.\ the target of \(\rho\) is in \(\piout(\vectSet{L}_i)\).

To see that the new run can enable all the \(\rho_j\) in sequence, remember that this was specifically how \(\rho_{\text{enable}}\) was chosen. To argue the factor \(n\) away, remember that \(\rho_{\text{enable}}\) also enables the \(\vect{w}_j\). Hence we have \((\dirOfRun(\rho')+\vect{w}_0)+\N \vect{v} \subseteq \vectSet{X}_i\). In particular not \(\rho'\) itself can pump \(\N \vect{v}\), but \(\rho'+\vect{w}_0\). Since \(\rho\) was chosen arbitrarily, we not only found a pump in \(\vectSet{X}_i\), in fact we have shown that \(k_{\text{hom}}+\vect{w}_0+\N \vect{v} \subseteq \vectSet{X}_i\), where \(k_{\text{hom}}\) stands for the constant number of homogeneous solutions we had to add.
\end{proof}

\subsection{Step 2 Towards Axiom 7}

In this section we prove that uniform sets are stable under non-degenerate composition. To this end, we first have to repeat the most important property connected to non-degenerate intersection of \(\N\)-g. sets: Fusing lines. We do this over the course of two lemmas.

\begin{lemma}
Let \(\vectSet{P}_1 \subseteq \vectSet{P}_2\) be \(\N\)-g. with \(\dim(\vectSet{P}_1)=\dim(\vectSet{P}_2)\). Then they generate the same vector space \(\vectSet{V}\), and for all \(\vect{p}_2 \in \vectSet{P}_2\) there exists \(\vect{p}_2' \in \vectSet{P}_2\) with \(\vect{p}_2+\vect{p}_2' \in \vectSet{P}_1\). \label{LemmaNondegenerateInclusion}
\end{lemma}

\begin{proof}
Let \(\vectSet{V}_1\) be the vector space generated by \(\vectSet{P}_1\), and \(\vectSet{V}_2\) the vector space generated by \(\vectSet{P}_2\). Since \(\vectSet{P}_1 \subseteq \vectSet{P}_2\) we have \(\vectSet{V}_1 \subseteq \vectSet{V}_2\). On the other hand remember Lemma \ref{LemmaFromJerome}: We obtain \(\dim(\vectSet{V}_1)=\dim(\vectSet{P}_1)= \dim(\vectSet{P}_2)=\dim(\vectSet{V}_2)\). Together we obtain \(\vectSet{V}_1=\vectSet{V}_2\).

Now let \(\vect{p}_2\in \vectSet{P}_2\). Write \(\vect{p}_2 \in \vectSet{V}_2=\vectSet{V}_1\) as linear combination of \(\vectSet{P}_1\). Simply choose \(\vect{p}_2''\) such that all negative components of the linear combination giving \(\vect{p}_2\) become positive. Now there exists \(n \in \N\) s.t. \(n(\vect{p}_2+\vect{p}_2'') \in \vectSet{P}_1\). Choose \(\vect{p}_2':=(n-1) \vect{p}_2 + n \vect{p}_2' \in \N(\vectSet{P}_2)=\vectSet{P}_2\).
\end{proof}

\begin{lemma}
Let \(\vectSet{P}_1, \vectSet{P}_2\) be \(\N\)-g. with a non-degenerate intersection. Then for all \(\vect{p}_1 \in \vectSet{P}_1\) and \(\vect{p}_2 \in \vectSet{P}_2\) there exist \(\vect{p}_1' \in \vectSet{P}_1\) and \(\vect{p}_2' \in \vectSet{P}_2\) such that \(\vect{p}_1+\vect{p}_1'=\vect{p}_2+\vect{p}_2'\). \label{LemmaNonDegenerateCombineVectors}
\end{lemma}

\begin{proof}
Since \(\vectSet{P}_1 \cap \vectSet{P}_2\) is non-degenerate, we can apply Lemma \ref{LemmaNondegenerateInclusion} twice, namely with \(\vectSet{P}_1 \cap \vectSet{P}_2 \subseteq \vectSet{P}_i\) for \(i\in \{1,2\}\). The vector spaces generated by \(\vectSet{P}_1\) and \(\vectSet{P}_2\) are hence equal, and there are \(\vect{p}_1'', \vect{p}_2''\) with \(\vect{p}_1+\vect{p}_1'', \vect{p}_2+\vect{p}_2'' \in \vectSet{P}_1 \cap \vectSet{P}_2\). Choose \(\vect{p}_1':=\vect{p}_1''+(\vect{p}_2+\vect{p}_2'') \in \vectSet{P}_1+\vectSet{P}_1 \subseteq \vectSet{P}_1\), and similarly \(\vect{p}_2':=\vect{p}_2''+(\vect{p}_1+\vect{p}_1'') \in \vectSet{P}_2\). Clearly \(\vect{p}_1+\vect{p}_1'=\vect{p}_2+\vect{p}_2'\).
\end{proof}

Lemma \ref{LemmaNonDegenerateCombineVectors} might seem inconspicuous, but is the basis of computing with pumps in case of non-degenerate intersection/composition. Namely Lemma \ref{LemmaNonDegenerateCombineVectors} implies many results like the following, always using exactly the same technique we call \emph{fusing lines}:

\begin{lemma}
Let \(\vectSet{P}_1, \vectSet{P}_2\) be \(\N\)-g., \(\vectSet{P}_1 \cap \vectSet{P}_2\) non-degenerate. Then \(\Pumps(\vectSet{P}_1)\cap \Pumps(\vectSet{P}_2)=\Pumps(\vectSet{P}_1 \cap \vectSet{P}_2)\).
\end{lemma}

\begin{proof}
``\(\supseteq\)'' is trivial. Hence let \(\vect{v} \in \Pumps(\vectSet{P}_1)\cap \Pumps(\vectSet{P}_2)\). We have to prove that \(\vect{v} \in \Pumps(\vectSet{P}_1 \cap \vectSet{P}_2)\).

By definition of \(\Pumps\), there exist \(\vect{x}_1, \vect{x}_2\) s.t. \(\vect{x}_1+\N \vect{v} \subseteq \vectSet{P}_1\) and \(\vect{x}_2+\N \vect{v} \subseteq \vectSet{P}_2\). By Lemma \ref{LemmaNonDegenerateCombineVectors}, there exist \(\vect{p}_1, \vect{p}_2\) s.t. \(\vect{x}:=\vect{x}_1+\vect{p}_1=\vect{x}_2+\vect{p}_2 \in \vectSet{P}_1 \cap \vectSet{P}_2\). We claim \(\vect{x}+\N \vect{v} \subseteq \vectSet{P}_1 \cap \vectSet{P}_2\). We prove the two containments separately. To see containment in \(\vectSet{P}_1\), for all \(n\in \N\) we have \[\vect{x}+n \vect{v}=\vect{x}_1+\vect{p}_1+n\vect{v}=(\vect{x}_1+n \vect{v})+\vect{p}_1 \in \vectSet{P}_1+\vectSet{P}_1 \subseteq \vectSet{P}_1\]
as claimed.
\end{proof}

What happened is simple: When ``fusing'' \(\vect{x}_1\) and \(\vect{x}_2\) into a common point \(\vect{x}\), also the ``lines above them'' in direction \(\vect{v}\) were fused. 

One might wonder how to prove that arbitrary \(\N\)-g. sets have a non-degenerate intersection. Regarding this problem we have the following:

\begin{proposition}[\cite{GuttenbergRE23}, Prop. 3.9] \label{PropositionNonDegenerateIntersection} \label{PropositionCheckNonDegenerate}
Let \(\vectSet{P}_1, \vectSet{P}_2\) be \(\N\)-g. s.t. \(\Pumps(\Pumps(\vectSet{P}_i))\) are semilinear for \(i \in \{1,2\}\) and \(\Pumps(\Pumps(\vectSet{P}_1)) \cap \Pumps(\Pumps(\vectSet{P}_2))\) is non-degenerate. Then \(\vectSet{P} \cap \vectSet{P}'\) is non-degenerate.
\end{proposition}

I.e.\ non-degenerate intersection transfers from the overapproximation to the actual sets. The simplest proof of Proposition \ref{PropositionCheckNonDegenerate} using theory from this paper is to use that \(\vectSet{P} \HybridizationRelation \Pumps(\Pumps(\vectSet{P}))\) and use Axiom 8 of utbo overapproximation. 

We now start the main proof of this section.

\begin{lemma}
Let \(\vectSet{X}_{12}\HybridizationRelation \vectSet{L}_{12}\) and \(\vectSet{X}_{23}\HybridizationRelation \vectSet{L}_{23}\) be s.t. \(\vectSet{X}_{12}, \vectSet{X}_{23}\) are uniform and \(\piin(\vectSet{L}_{23}) \cap \piout(\vectSet{L}_{12})\) is non-degenerate. Then \(\vectSet{X}_{12} \circ \vectSet{X}_{23} \HybridizationRelation \vectSet{L}_{12} \circ \vectSet{L}_{23}\) is uniform with the set of pumps \(\Pumps(\vectSet{X}_{12} \circ \vectSet{X}_{23})=\Pumps(\vectSet{X}_{12}) \circ \Pumps(\vectSet{X}_{23})\).
\end{lemma}

\begin{proof}
By Lemma \ref{LemmaShiftGoodOverapproximation} \(\vectSet{X}_{12} \circ \vectSet{X}_{23} \HybridizationRelation \vectSet{L}_{12} \circ \vectSet{L}_{23}\), in particular \(\vectSet{X}_{12} \circ \vectSet{X}_{23} \neq \emptyset\). We use the base points of \(\vectSet{L}_{12} \circ \vectSet{L}_{23}\) as \(\vectSet{B}\) and check the properties of Definition \ref{DefinitionUniform}. Observe first that by Proposition \ref{PropositionCheckNonDegenerate} not just \(\piin(\vectSet{L}_{23})=\piout(\vectSet{L}_{12})\) is non-degenerate, but also \(\piin(\vectSet{P}_{\vectSet{X}_{23}}) \cap \piout(\vectSet{P}_{\vectSet{X}_{12}})\) is non-degenerate, hence we can use line fusion.

1): We will prove that \(\Pumps(\vectSet{X}_{12} \circ \vectSet{X}_{23}) \subseteq \Pumps(\vectSet{X}_{12}) \circ \Pumps(\vectSet{X}_{23}) = \Pumps(\vectSet{P}_{\vectSet{X}_{12}}) \circ \Pumps(\vectSet{P}_{\vectSet{X}_{23}}) \subseteq \Pumps(\vectSet{P}_{\vectSet{X}_{12} \circ \vectSet{X}_{23}}) \subseteq \Pumps(\vectSet{X}_{12} \circ \vectSet{X}_{23})\). Observe first that the \(=\) in the middle immediately follows since \(\vectSet{X}_{12}, \vectSet{X}_{23}\) are uniform, and the last \(\subseteq\) is trivial since the preservants always have less pumps. It remains to show the other containments.

\(\Pumps(\vectSet{P}_{\vectSet{X}_{12}}) \circ \Pumps(\vectSet{P}_{\vectSet{X}_{23}}) \subseteq \Pumps(\vectSet{P}_{\vectSet{X}_{12} \circ \vectSet{X}_{23}})\): Let \((\vect{v}_1, \vect{v}_2) \in \Pumps(\vectSet{P}_{\vectSet{X}_{12}})\) and \((\vect{v}_2, \vect{v}_3) \in \Pumps(\vectSet{P}_{\vectSet{X}_{23}})\). Then there exists \((\vect{x}_1, \vect{x}_2)\) and \((\vect{x}_2', \vect{x}_3)\) s.t. \((\vect{x}_1, \vect{x}_2)+\N(\vect{v}_1, \vect{v}_2) \subseteq \vectSet{P}_{\vectSet{X}_{12}}\) and \((\vect{x}_2'+\vect{x}_3)+\N (\vect{v}_2, \vect{v}_3) \subseteq \vectSet{P}_{\vectSet{X}_{23}}\). Since \(\piout(\vectSet{P}_{\vectSet{X}_{12}}) \cap \piin(\vectSet{P}_{\vectSet{X}_{23}})\) is non-degenerate, we fuse the lines using Lemma \ref{LemmaNonDegenerateCombineVectors} and obtain \((\vect{v}_1, \vect{v}_3) \in \Pumps(\vectSet{P}_{\vectSet{X}_{12} \circ \vectSet{X}_{23}})\) as claimed.

\(\Pumps(\vectSet{X}_{12} \circ \vectSet{X}_{23}) \subseteq \Pumps(\vectSet{X}_{12}) \circ \Pumps(\vectSet{X}_{23})\): Let \(\vect{v}=(\vect{v}_1, \vect{v}_3) \in \Pumps(\vectSet{X}_{12} \circ \vectSet{X}_{23})\). Then there exists \(\vect{x}=(\vect{x}_1, \vect{x}_3)\) s.t. \((\vect{x}_1, \vect{x}_3)+\N (\vect{v}_1, \vect{v}_3) \subseteq \vectSet{X}_{12} \circ \vectSet{X}_{23}\). By definition of \(\circ\), there exist \(\vect{x}_{2,n}\) s.t. \((\vect{x}_1+n \vect{v}_1, \vect{x}_{2,n})\in \vectSet{X}_{12}\) for all \(n\) and \((\vect{x}_{2,n}, \vect{x}_3+n \vect{v}_3) \in \vectSet{X}_{23}\).

Since \(\vectSet{X}_{12}\) is well-directed, there exists a subset \(N_1 \subseteq \N\) of indices s.t. \( ((k-j) \vect{v}_1, \vect{x}_{2,k}-\vect{x}_{2,j}) \in \dir(\vectSet{X}_{12})\) for all \(k>j\) in \(N_1\). Since \(\vectSet{X}_{23}\) is well-directed, there exists a subset \(N_2 \subseteq N_1\) of indices s.t. additionally \((\vect{x}_{2,k}-\vect{x}_{2,j}, (k-j) \vect{v}_3) \in \dir(\vectSet{X}_{23})\) for all \(k>j\) in \(N_2\). Hence \((k-j) \vect{v} \in \dir(\vectSet{X}_{12}) \circ \dir(\vectSet{X}_{23})\). We will not describe the way to instead obtain \(\Pumps(\vectSet{X}_{12}) \circ \Pumps(\vectSet{X}_{23})\) in detail, but the idea is simple: The reason we only obtained directions is due to the same problem we had in the proof of Lemma \ref{LemmaSingleComponentUniform}: Along the boundary, a single run can only pump multiples of the direction. But by cleverly choosing interior directions to add, we generate both in \(\vect{X}_{12}\) and \(\vectSet{X}_{23}\) a line without the factor \((k-j)\). But now the lines are in completely different locations. Hence we have to use the line fusion idea Lemma \ref{LemmaNonDegenerateCombineVectors} to finish the proof.



2) Since \(\vectSet{X}_{12} \circ \vectSet{X}_{23} \HybridizationRelation \vectSet{L}_{12} \circ \vectSet{L}_{23}\), by Lemma \ref{LemmaPumpsOfNiceSet} we have \(\Pumps(\Pumps(\vectSet{X}_{12} \circ \vectSet{X}_{23}))=\Pumps(\vectSet{L}_{12} \circ \vectSet{L}_{23})\). Therefore \(\vectSet{X}_{12} \circ \vectSet{X}_{23} \subseteq \vectSet{L}_{12} \circ \vectSet{L}_{23}\subseteq \vectSet{B}+\Pumps(\vectSet{L})=\vectSet{B}+\Pumps(\Pumps(\vectSet{X}_{12} \circ \vectSet{X}_{23}))\) as claimed.

3) It suffices to prove that \(\vectSet{L}_{12} \circ \vectSet{L}_{23}\) is directed. Let \((\vect{x}_1, \vect{x}_3), (\vect{y}_1, \vect{y}_3) \in \vectSet{L}_{12} \circ \vectSet{L}_{23}\). Then there exist \(\vect{x}_2, \vect{y}_2\) s.t. \((\vect{x}_1, \vect{x}_2), (\vect{y}_1, \vect{y}_2) \in \vectSet{L}_{12}\) and \((\vect{x}_2, \vect{x}_3), (\vect{y}_2, \vect{y}_3) \in \vectSet{L}_{23}\). Since \(\vectSet{L}_{12}\) is directed, there exists \((\vect{z}_1, \vect{z}_2) \in \vectSet{L}_{12}\) above both \((\vect{x}_1, \vect{x}_2)\) and \((\vect{y}_1, \vect{y}_2)\). Similarly, since \(\vectSet{L}_{23}\) is directed, there exists \((\vect{z}_2', \vect{z}_3) \in \vectSet{L}_{23}\) above both \((\vect{x}_2, \vect{x}_3)\) and \((\vect{y}_2, \vect{y}_3)\). Since \(\piin(\vectSet{L}_{23}) \cap \piout(\vectSet{L}_{12})\) is non-degenerate, we can combine the points \(\vect{z}_2\) and \(\vect{z}_2'\) to a common point \(\vect{z}_{2,\text{new}}\), finishing 3).

As a composition of well-directed relations, \(\vectSet{X}_{12} \circ \vectSet{X}_{23}\) is again well-directed. Simply pick \(N_2 \subseteq N_1 \subseteq \N\), i.e.\ a subsubsequence.
\end{proof}

\subsection{Step 3 Towards Axiom 7}

At this point we have shown that for every \(\RelationClass\)-KLM sequence \(\vectSet{X}\), one can in elementary time compute a semilinear formula for \(\Pumps(\vectSet{X})\) by using \eqref{EquationPumpsFormula} for every strongly-connected component \(\vectSet{X}_i\), and perform composition. It remains to check Axiom 7 using this fact. As already mentioned, our main contribution here was the introduction of the definition of \emph{uniform} and thereby making the theory of \cite{Leroux13} and \cite{GuttenbergRE23} computable. At this point we can simply rely on their semilinearity algorithm, while substituting our procedure at certain locations. We only summarize the idea of the algorithm quickly here. 

Since we will now work across \(\RelationClass\)-KLM sequences, as opposed to before where \(\vectSet{X}_i\) used to refer to a component of \(\vectSet{X}\), we now write \(\RelationClass\)-KLM sequences as \(\vectSet{Y}\), and different indices \(\vectSet{Y}_j\) refer to different \(\RelationClass\)-KLM sequences.

The algorithm for Axiom 7 is as follows: Some of the notions we will explain afterwards, but we prefer to give a full description of the algorithm first.

Step 1: Apply Axiom 2 to obtain a finite set \(\{\vectSet{Y}_1, \dots, \vectSet{Y}_k\}\) of \(\RelationClass\)-KLM sequences. 

Step 2: Compute semilinear sets \(\vectSet{C}_j \cap \vectSet{G}_j=\Pumps(\vectSet{Y}_j)\) for all \(j \in \{1,\dots, k\}\), where \(\vectSet{C}_j\) is \(\Q_{\geq 0}\)-g. and \(\vectSet{G}_j\) is \(\Z\)-g. (we use the letter \(\vectSet{G}\) because \(\Z\)-g. sets are also called \emph{grids}). Discard all \(\RelationClass\)-KLM sequences where \(\dim(\vectSet{S}_j)< \dim(\vectSet{L})\).

Step 3: Let \(\{\vect{b}_1, \dots, \vect{b}_m\}\) be a representative system for the cosets of \(\vectSet{G}:=\bigcap_{j=1}^k \vectSet{G}_j\) inside \(\vectSet{L}-\vectSet{L}\). For every \(\text{iter}=1, \dots, m\) do: Define \(I_{\text{iter}}:=\{j \in \{1,\dots, k\} \mid \vectSet{Y}_j \cap (\vect{b}_j+ \vectSet{G}) \neq \emptyset\}\) as the set of \(\RelationClass\)-KLM sequences which intersect this coset. Now for the finite set \(\{\vectSet{C}_j \mid j \in I_{\text{iter}}\}\) of \(\Q_{\geq 0}\)-g. relations check for existence of a so-called \emph{complete extraction} (described in a moment). If it does not exist, reject. Otherwise we are in the situation in the right of Figure \ref{FigureIntuitionPreservants}: We compute for the corresponding complete extraction \(\{\vectSet{K}_j \mid j \in I_{\text{iter}}\}\) base points \(\vect{x}_{\text{iter},j}\) such that \(\vect{x}_{\text{iter}, j}+(\vectSet{K}_j \cap \vectSet{G}) \subseteq \vectSet{Y}_j\). We then do a recursive call on \(\vectSet{L} \setminus \bigcup_{\text{iter}, j} \vect{x}_{\text{iter},j}+(\vectSet{K}_j \cap \vectSet{G})\), which corresponds to the red lines in the right of Figure \ref{FigureIntuitionPreservants}.

There is a lot to unpack and understand here. First of all, let us start with possible questions regarding the \(\vectSet{G}_j\) and \(\vectSet{G}\). A \(\Z\)-g. set carries modulo information, i.e.\ is defined by some formula \(\vect{a} \vect{x} \equiv b \mod m\), where \(b,m \in \N\) and \(\vect{a} \in \Z^d\). When we form the intersection of the \(\vectSet{G}_j\), we hence deal with the fact that some set might only be able to reach even numbered points, and another set only ones divisible by \(3\) etc. This also explains the ``representative system for the cosets'': Simply use one representative for every possible remainder \(r\) \(\mod \vectSet{G}\). Then the set \(I_{\text{iter}}\) contains the ``active'' sets in an SCC: If a set is only active on even number, then for the coset of odd numbers this set will not exist. 

Finally, we can define complete extraction (see also \cite{Leroux13}, Appendix E): 

\begin{definition}
Let \(\mathcal{K}=\{\vectSet{C}_1, \dots, \vectSet{C}_k\}\) be a finite set of \(\FO(\Q,\leq,+)\) definable \(\Q_{\geq 0}\)-g. sets. A \emph{complete extraction} is a finite set \(\{\vectSet{K}_1, \dots, \vectSet{K}_k\}\) of \(\Q_{\geq 0}\)-\emph{finitely} generated sets s.t. \(\vectSet{K}_j \subseteq \vectSet{C}_j\) for all \(1 \leq j \leq k\) and \(\bigcup_{j=1}^k \vectSet{K}_j =\bigcup_{j=1}^k \vectSet{C}_j\).
\end{definition}

The idea is best explained using the left of Figure \ref{FigureIntuitionPreservants}: It consists of two uniforms sets, whose union of pumps is \(\{(x,y) \mid 0 \leq y \leq x\}\). So when only consider the union, the algorithm would not detect the ``hole'' between the two uniform sets. However, complete extraction catches this: The set of cones \(\vectSet{C}_1:=\{(x,y) \mid y=0\}\) and \(\vectSet{C}_2:=\{(x,y) \mid 0 < y \leq x\}\) does not have a complete extraction, since any finite set of vectors from \(\vectSet{C}_2\) (remember \(\vectSet{K}_2\) has to be finitely generated) would a minimal steepness \(>0\), and hence some angle would remain uncovered. 

In fact, complete extraction perfectly characterizes whether the union of multiple uniform sets (which agree on the modulus, hence we had to consider the cosets of \(\vectSet{G}\)) cover ``most of the space'': If there is no complete extraction, then a full dimensional hole will remain, otherwise we will be at least in the situation depicted on the right of Figure \ref{FigureIntuitionPreservants}: The complete extractions give rise to linear sets contained in the respective uniform sets. The starting points might be dislodged, such that some lines remain in the middle, but these can now be taken care of by recursion.

Returning to the actual algorithm, what we just explained is the reason for considering every coset of \(\vectSet{G}\) separately, and checking for complete extractions: From it we will be able to determine linear sets contained in the respective \(\RelationClass\)-KLM sequences which together cover most of the space. Finally, in the algorithm we indirectly state what kind of linear set will be contained in the uniform set when finding a complete extraction: Some base point \(\vect{x}_{\text{iter},j}+(\vectSet{K}_j \cap \vectSet{G})\). And by complete extraction, they will cover most of the coset.

Since \(\vectSet{K}_j\) is a \(\Q_{\geq 0}\)-finitely generated set, \(\vectSet{K}_j \cap \vectSet{G}\) is \(\N\)-finitely generated, and our algorithm computing \(\vect{x}_{\text{iter},j}\) simply applies the argument of Lemma \ref{LemmaSingleComponentUniform} to find a run pumping the finitely many directions \(\vect{v}_1, \dots, \vect{v}_k\) which have to be pumped.









































