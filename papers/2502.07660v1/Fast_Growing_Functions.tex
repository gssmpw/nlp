% !TEX root = Main.tex

We let \(F_1: \N \to \N, n \mapsto 2n\), and define \(F_d: \N \to \N, n \mapsto F_{d-1}^{n-1}(2)\), where \(F_{d-1}^{n-1}\) is \((n-1)\)-fold application of the function \(F_{d-1}\). For example \(F_2(n)=2^n\), \(F_3=\text{Tower}\) and so on. The functions \(F_d\) are called the fast-growing functions. (One possible) Ackermann-function \(F_{\omega}\) is obtained from these via \(F_{\omega}: \N \to \N, n \mapsto F_n(n)\) via diagonalization. The \(d\)-th level of the Grzegorczyk hierarchy \cite{Schmitz16} is the set of functions \(\mathfrak{F}_d:=\{F_{d} \circ r_1 \circ \dots \circ r_k \mid r_1, \dots, r_k \in \mathfrak{F}_{d-1}\}\). I.e. we close \(F_d\) under applying reductions of the lower level \(\mathfrak{F}_{d-1}\). For example \(\mathfrak{F}_3\) is the set of functions obtained by inputting an elementary function into the tower function.

One main way to prove that a function falls into some level of this hierarchy is a theorem of \cite{FigueiraFSS11}. It considers sequences over \(\N^d\). Given \(a \in \N\), write \(\vect{a}\) for the constant vector \(\vect{a}=(a,\dots, a) \in \N^d\). Given a sequence \((\vect{x}_0, \vect{x}_1, \dots, \vect{x}_k)\), \(n \in \N\) and \(g: \N \to \N\) we call the sequence \((g,n)\)-\emph{controlled} if \(||\vect{x_i}||_{\infty} \leq g^i(n)\) for all \(i\), i.e. the sequence starts below \(n\), and in every step entries grow at most by an application of the function \(g\). The sequence \emph{contains an increasing pair} if \(\vect{x}_i \leq \vect{x}_j\) for some \(i<j\).

\begin{proposition} \label{PropositionFastGrowingComplexity}
Let \(k,r, \gamma \geq 1\) be natural numbers. Let \(f\) be a monotone function in \(\mathfrak{F}_{\gamma}\) with \(f(x) \geq max(1,x)\) for all \(x\). Then the function mapping a given \(n\) to the length of the longest \((f,n)\)-controlled sequence in \(\N^d\) without an increasing pair is in \(\mathfrak{F}_{\gamma+d-1}\).
\end{proposition}

For example if a rank in \(\N^d\) decreases lexicographically in an algorithm, then the sequence of ranks has no increasing pair. We therefore obtain a complexity bound for the algorithm.