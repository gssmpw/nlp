% !TEX root = Main.tex

In this section we introduce a restriction of \(\RelationClass\)-eVASS we call \emph{monotone} \(\RelationClass\)-eVASS and prove that \ConsideredModel\ can be converted into monotone \(\RelationClass\)-eVASS. This will be the first step for our reachability algorithm.

The main observation is the following: Similar to how \(Semil\)-eVASS can simulate counter machines, if transitions inside an SCC are allowed to be non-monotone, then \(\RelationClass\)-eVASS will likely be undecidable. Hence we define:

\begin{definition} \label{DefinitionMonotoneEVASS}
Let \(\RelationClass\) be a class of relations. A \emph{monotone} \(\RelationClass\)-eVASS is a \(\RelationClass\)-eVASS where transition labels \(\vectSet{R}(e)\) for edges \(e\) \emph{inside an SCC} are \emph{monotone} relations \(\vectSet{R}(e) \in \RelationClass\).
\end{definition}

For example for \(\RelationClass=Add\) all \(\RelationClass\)-eVASS are monotone, because this class \(\RelationClass\) contains only monotone relations. On the other hand, \emph{monotone} \(Semil\)-eVASS can no longer simulate counter machines, in fact they are closer to VASS: Using what is called the \emph{controlling counter technique}, VASS can in fact simulate zero tests on the exits of SCCs. However, monotone \(Semil\)-eVASS are nevertheless slightly more general, as for example \emph{weak doubling}, the monotone linear relation defined by the periods \((1,1), (1,2)\), is now allowed on edges. The more intuitive way to write weak doubling is \(\{(x,x') \in \N^2 \mid x \leq x' \leq 2x\}\), i.e.\ \(x\) gets at most doubled, but maybe less.

Given a VASSnz \(\VAS\), the number of priorities \(k\) is the number of different \(j\) s.t. some transition has the label \(NZT(j,d)\), i.e.\ the number of different types of zero tests. We prove that VASSnz with \(k\) priorities can be converted into monotone eVASS, whose labels are VASSnz sections with \(k-1\) priorities.

\begin{lemma} \label{LemmaConvertVASSnzToEVASS}
Let \(\RelationClass\):=\((k-1)\)-VASSnzSec be the class of sections of VASSnz with \(k-1\) priorities. There is a polytime algorithm converting a \(k\)-VASSnz \(\VAS_k\) into a monotone \(\RelationClass\)-eVASS \(\VAS'\) with the same reachability relation. 
\end{lemma}

\begin{proof}
First let us explain the crucial observation this construction is based on: What is a zerotest? It is a linear relation, which is monotone on some counters and \emph{leaves the others} \(0\). If the fixed counters are considered as \emph{non-existent} in the current SCC \(S\) (remember eVASS have this capability), then a zero test is monotone and may be used as transition label.

Let \(m\) be the index of the highest counter which is zero tested in \(\VAS_k\), and write \(\VAS_k=(Q,E)\) with initial/final states \(\qin, \qfin\). Let \(\VAS_{k-1}\) be the VASSnz obtained from \(\VAS_k\) by deleting all edges zero testing counter \(m\).

The target \(\RelationClass'\)-eVASS \(\VAS'=(Q',E')\) has \(5\) types of states \(Q':=\{src, del, main, add, tgt\} \times Q\) and \(7\) types of edges \(E_1, \dots, E_7\) with \(E':=\bigcup_{i=1}^7 E_i\). The new initial/final states are \(\qin':=(\source, \qin)\) and \(\qfin':=(\target, \qfin)\). Let us start with the simplest type of edge: \(E_1=\{e_1\}\), where \(e_1: (src, \qin) \to (tgt,\qfin)\) is labelled with \(\Rel(\VAS_{k-1}, \qin, \qfin)\). This allows \(\VAS'\) to simulate runs of \(\VAS_k\) not using zero tests on \(m\).

Next let us explain the states \(\{main\} \times Q\), where the interesting computation happens: The set of active counters for these states is \(I(\{main\} \times Q)=\{m+1,\dots, d\}\), i.e.\ in these states we assume that the first \(m\) counters have been deleted. We have two types of edges on \(main\) called \(E_2\) and \(E_3\), where \(E_2\) simulates zero tests on \(m\). Formally, we define \(E_2:=\{((main,q), (main,p)) \mid (q,p) \in E, \text{ and has the label }\vectSet{R}(q,p)=NZT(m,d)\}\). We give every \(e_2 \in E_2\) the label \(\vectSet{R}(e_2):=\{(\vect{x}, \vect{x}) \mid \vect{x} \in \N^{d-m}\}\). Observe that this label faithfully applies the zero test: The first \(m\) counters do not exist in this SCC, and on the rest of the counters the ``zero test'' is the identity.

Next we consider \(E_3\). The idea is that in state \(main\) we want to still be able to apply transitions of \(\VAS_{k-1}\). Set \(E_3:=(\{main\} \times Q) \times (\{main\} \times Q)\), i.e.\ for all \((p,q) \in Q^2\) we have an edge.  For every \(e_3=(main, p) \to (main, q) \in E_3\) we set the label \(\vectSet{R}(e_3)\) of \(e_3\) to
\[\pi_m(\Rel(\VAS_{k-1}, p, q) \cap \{(\vect{x}, \vect{y}) \mid \vect{x}[i]=\vect{y}[i]=0\ \forall 1 \leq i \leq m\}),\] where \(\pi_m\) projects the first \(m\) coordinates away. Hence edges \(e_3\) apply arbitrary runs of \(\VAS_{k-1}\) between the corresponding states which \emph{preserve value} \(0\) on the first \(m\) counters.

Next we explain the states \(del\) and \(add\): They ensure that when entering and leaving the \(main\) component we have only the counters \(m+1, \dots, d\). 
We set \(E_4:=\{((del, p), (main, p)) \mid p \in Q\}\), and add the labels \(I_+(e_4):=\emptyset\) and \(I_-(e_4):=\{1,\dots, m\}\), i.e.\ we delete the first \(m\) many counters and do not change the state of \(\VAS_k\).

Similarly, we set \(E_5:=\{((main, p),(add, p)) \mid p \in Q\}\) with labels \(I_+(e_5):=\{1,\dots, m\}\) and \(I_-(e_5):=\emptyset\), which readd the first \(m\) counters. 

We set \(E_6=\{((\source, \qin),(del, p)) \mid p \in Q\}\), and for every \(e_6 \in E_6\) the label \(\vectSet{R}(e_6):=\Rel(\VAS_{k-1}, q_{in}, p) \cap \{(\vect{x}, \vect{y}) \mid \vect{y}[i]=0\ \forall\ 1 \leq i \leq m\}\), i.e.\ we perform any run of \(\VAS_{k-1}\) which starts in \(q_{in}\) and sets the first \(m\) counters to \(0\). 

Similarly, we set \(E_7:=\{((add, p), (\target, \qfin)) \mid p \in Q\}\) and the label \(\vectSet{R}(e_7):=\Rel(\VAS_{k-1}, p, \qfin)\), i.e.\ we end again with any run of \(\VAS_{k-1}\). This finishes the construction of \(\VAS'\).

To see that \(\VAS'\) is equivalent to \(\VAS_k\), let \(\rho\) be any run of \(\VAS_k\). If \(\rho\) does not use a zero-test on \(m\) then we simulate \(\rho\) using \(E_1\). Otherwise \(\rho\) eventually visits a configuration \(\vect{x}\) with \(\vect{x}[i]=0\ \forall\ 1 \leq i \leq m\) to enable the zero test. Write \(\rho=\rho_{\text{pre}} \rho_{\text{mid}} \rho_{\text{suf}}\) where \(\rho_{\text{pre}}\) is the prefix until the first time \(\vect{x}[i]=0\ \forall\ 1 \leq i \leq m\) and \(\rho_{\text{suf}}\) is the suffix starting from the last time we have \(\vect{x}[i]=0\ \forall\ 1 \leq i \leq m\). Then \(\rho_{\text{pre}}\) and \(\rho_{\text{suf}}\) do not use zero tests on \(m\), hence we can simulate them using edges of type \(E_6\) and \(E_7\) respectively. To simulate \(\rho_{mid}\), we use edge types \(E_2\) and \(E_3\) repeatedly.
\end{proof}

Lemma \ref{LemmaConvertVASSnzToEVASS} shows that even if \(\RelationClass\) contains non-monotone relations, one can recover some monotonicity at the cost of complicated edge labels. This might seem like a difficulty, but they will turn out to be surprisingly easy to deal with:

Remember Definition \ref{DefinitionApproximable} of approximability.  We will now approximate monotone \(\RelationClass\)-eVASS and therefore \ConsideredModel:

\begin{theorem} \label{TheoremIdealDecompositionEVASS}
Let \(\alpha \geq 3\) and let \(\RelationClass\) be a class of relations approximable in \(\mathfrak{F}_{\alpha}\), containing \(Add\) and closed under intersection with semilinear relations. Then sections of monotone \(\RelationClass\)-eVASS are approximable in \(\mathfrak{F}_{\alpha+2d+2}\).
\end{theorem}

We move the proof of Theorem \ref{TheoremIdealDecompositionEVASS} to Section \ref{SectionProofTheoremEVASS}, and first show Theorem \ref{TheoremVASSnzIdealDecomposition}, our first main theorem.

\begin{theorem} \label{TheoremVASSnzIdealDecomposition}
The class of sections of VASSnz of dimension \(d\) and with \(k\) priorities is approximable in \(\mathfrak{F}_{2kd+2k+2d+5}\). 

The class of all VASSnz sections is approximable in \(\mathfrak{F}_{\omega}\) time.
\end{theorem}

\begin{proof}
Proof by induction on \(k\).

\(k=0\): VASS are a special case of monotone Semil-eVASS, and since the class of semilinear sets is approximable in \(\mathfrak{F}_3\) (namely we have \(\vectSet{L} \HybridizationRelation \vectSet{L}\ \forall\ \vectSet{L}\)), we hence obtain \(\mathfrak{F}_{3+2d+2}=\mathfrak{F}_{2d+5}\) time for VASS by Theorem \ref{TheoremIdealDecompositionEVASS}.

\(k-1 \to k\): By Lemma \ref{LemmaConvertVASSnzToEVASS} we can convert \ConsideredModel\ with \(k\) priorities into \(\RelationClass\)-eVASS for \(\RelationClass:=\) \((k-1)\)-priority VASSnz sections. By induction together with Theorem \ref{TheoremIdealDecompositionEVASS} approximating \(\RelationClass\)-eVASS takes time \(\mathfrak{F}_{2kd+2k+2d+5}\).

For the class of all VASSnz we immediately obtain \(\mathfrak{F}_{\omega}\).
\end{proof}

Theorem \ref{TheoremVASSnzIdealDecomposition} trivially implies the following:

\begin{corollary}
Reachability for \ConsideredModel \ is in \(\mathfrak{F}_{\omega}\).
\end{corollary}

Hence we can now focus solely on proving Theorem \ref{TheoremIdealDecompositionEVASS}.























































