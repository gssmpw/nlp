% !TEX root = Main.tex

Let \(\RelationClass\) be a class of relations on \(\N^d\). A \(\RelationClass\)-extended VASS (\(\RelationClass\)-eVASS) of dimension \(d\) is a finite directed multigraph \((Q,E)\) which is labelled as follows:

\begin{enumerate}
\item Every SCC \(S \subseteq Q\) has a subset \(I(S)\) of active counters.
\item Every edge \(e\) inside an SCC is labelled with \(\vectSet{R}(e) \in \RelationClass\).
\item Every edge \(e\) leaving an SCC is labelled with \emph{either} a relation \(\vectSet{R}(e) \in \RelationClass\) or with two subsets \(I_+(e), I_-(e) \subseteq \{1,\dots, d\}\) of counters to add and delete respectively.
\end{enumerate}

The set of configurations is \(\bigcup_{SCC\ Q'} Q' \times \N^{I(Q')}\), and an edge \(e\) has semantics \(\to_e\) defined by \((q,\vect{x}) \to_e (p, \vect{y})\) if \(e=(q,p)\) and either \((\vect{x}, \vect{y}) \in \vectSet{R}(e)\) if \(e\) is labelled by \(\vectSet{R}(e)\), or \(\vect{y}=\vect{x}_{I_+(e), I_-(e)}\)  is equal to \(\vect{x}\) with coordinates from \(I_-(e)\) deleted and coordinates in \(I_+(e)\) readded with value \(0\) otherwise.

%The set \(\Omega\) of all runs is wqo with the amalgamation property as follows, called the \emph{Jancar ordering}: We view \(\Omega\) as \(\Omega \subseteq Conf \times (\bigcup_{e \in E} \{e\} \times \Omega(e))^{\ast} \times Conf\)%, where a run \(\rho=(\vect{x}_0, \dots, \vect{x}_r)\) induces \((\vect{x}_0, ((\vect{x}_0, \rho(e_0), \vect{x}_1), \dots, (\vect{x}_{r-1}, \rho(e_{r-1}), \vect{x}_r)),\vect{x}_r) \in \Omega\). The set \(\N^{I_{in}} \times (\bigcup_{e \in E} \{e\} \times \Omega(e))^{\ast} \times \N^{I_{fin}}\) and hence \(\Omega\) is well-quasi-ordered by Lemma \ref{DicksonsLemma} and \ref{HigmanLemma}.
%, where \(Conf\) is the set of configurations. I.e.\ a run \(\rho\) is identified with the triple \((\source(\rho), steps(\rho), \target(\rho)) \in Conf \times (\bigcup_{e \in E} \{e\} \times \Omega(e))^{\ast} \times Conf\). Then \(\Omega\) is well-quasi-ordered by Lemma \ref{DicksonsLemma} and \ref{HigmanLemma}.

Intuitively, \(\RelationClass\)-eVASS model finite automata operating on counters with values in \(\N\). An operation consists of a state change of the automaton and updating the counters according to the relation \(\vectSet{R}(e)\) written on the edge. Sometimes it is convenient to change the dimension of the system when leaving an SCC, hence the automaton is allowed to add/delete counters when leaving an SCC.

There are multiple classes of systems which fall into this definition. For example consider the class \(Add\) of relations of the form \(\to_{\vect{a}}\) for \(\vect{a} \in \Z^d\) defined as \(\vect{x} \to_{\vect{a}} \vect{y} \iff \vect{y}=\vect{x}+\vect{a}\). Then \(Add\)-eVASS without counter deletions form the class of \emph{vector addition system with states} (VASS).

Furthermore, counter machines are \(Semil\)-eVASS, where \(Semil\) is the class of semilinear relations. To see this, observe that zero tests \(ZT(I,d):=\{(\vect{x}, \vect{y}) \in \N^d \times \N^d \mid \vect{x}=\vect{y} \text{ and }\vect{x}[i]=0\ \forall\  i \in I\}\) are a special case of linear relations.

Another subclass we will consider are VASSnz. Let \(NZT\) be the class of \emph{nested zero tests}, defined as relations of the form \(NZT(j,d):=ZT(\{1,\dots, j\},d)\) for some \(j \leq d \in \N\). Intuitively, the vector \(\vect{x}\) stays the same and the first \(j\) coordinates are ``tested for \(0\)''. In particular if counter \(j\) is tested also all lower index counters are tested. 

The class \((Add \cup NZT)\)-eVASS is called \ConsideredModel.

Observe that we do not disallow class \(\RelationClass\) from containing non-deterministic relations \(\vectSet{R} \in \RelationClass\), in this way \(\RelationClass\)-eVASS can naturally model both determinism and non-determinism.

We continue with a few more semantic definitions. Let \(\qin\in Q\) and \(\qfin\in Q\) be states called the initial and final state respectively. A run \(\rho\) is a sequence \((p_0(\vect{x}_0), \dots, p_r(\vect{x}_r))\) of configurations s.t. \(\vect{x}_i \to_{e_i} \vect{x}_{i+1}\) for some edges \(e_i \in E\), \(p_0=\qin\) and \(p_r=\qfin\). The source of \(\rho\) is \(\vect{x}_0\), the target is \(\vect{x}_r\), and the source/target pair is \(\dirOfRun(\rho):=(\vect{x}_0, \vect{x}_r)\). We write \(\Omega\) for the set of all runs (from \(\qin\) to \(\qfin\)).  The reachability relation is \(\Rel(\VAS, \qin, \qfin):=\dirOfRun(\Omega) \subseteq \N^d \times \N^d\).

Since the class of reachability relations is lacking some important closure properties, we usually consider a slightly larger class of relations which we call sections. 

\begin{definition}
%Let \(\RelationClass\) be a class of relations. 
A relation \(\vectSet{X} \subseteq \N^{d'} \times \N^{d''}\) is a \(\RelationClass\)-eVASS section if \(\vectSet{X}=\pi(\vectSet{R} \cap \vectSet{L})\), where \(\vectSet{L} \subseteq \N^d \times \N^d\) for some \(d \geq d', d''\) is linear, \(\vectSet{R}\) is the reachability relation of a \(d\)-dimensional  \(\RelationClass\)-eVASS, and \(\pi \colon \N^d \times \N^d \to \N^{d'} \times \N^{d''}\) is a \emph{projection}, i.e.\ a function deleting some coordinates.
\end{definition}



%----------------------IMPORTANT: In case we want to revert to VASSnz

%A \emph{vector addition system with states and nested zero tests} (VASSnz) \(\VAS\) of dimension \(d \in \N\) is a finite directed multigraph \((Q,E)\), whose edges \(e\) are labelled with a pair of a vector \(f(e)\in \Z^d\) and a number \(g(e)\in \{0,\dots,d\}\). The set of configurations of \(\VAS\) is \(Q \times \N^d\). An edge \(e=(p,p')\) with label \((f(e), g(e))\) induces a relation \(\to_{e}\) on configurations as follows: \(\vect{c}=(q, \vect{x}) \to_{e} \vect{c}'=(q',\vect{x}')\) if and only if \(q=p, q'=p'\), \(\vect{x}(j)=0\) for all \(1 \leq j \leq g(e)\) and \(\vect{x}'=\vect{x}+f(e)\). Intuitively, the edge can only be used in state \(p\) to move to state \(p'\) and adds the vector \(f(e)\) to the current configuration. However, two conditions have to be fulfilled: We have to again arrive at a configuration \(\vect{c}'\) (i.e. \(\vect{x}'\) has to stay non-negative), and \(\vect{x}\) must be \(0\) on the first \(g(e)\) coordinates. We say that these coordinates are \emph{tested for 0}. Observe that contrary to Minsky machines, if a counter \(i\) is tested for \(0\), also all smaller counters \(j \leq i\) are tested for \(0\).
%
%We write \(\to_{\VAS}=\bigcup_{e\in E} \to_{e}\) and let \(\to_{\VAS}^{\ast}\) denote its reflexive and transitive closure. A run of \(\VAS\) is a finite sequence \(\rho=(\vect{c}_0, \vect{c}_1, \dots, \vect{c}_k)\) of configurations such that \(\vect{c}_i \to_{\VAS} \vect{c}_{i+1}\) for all \(0 \leq i \leq k-1\). The \emph{source} of the run \(\rho\) is the configuration \(\source(\rho):=\vect{c}_0\), and the \emph{target} is \(\target(\rho):=\vect{c}_k\). The \emph{ends} of \(\rho\) are the pair \(\dirOfRun(\rho)=(\source(\rho), \target(\rho))\). A configuration \(\target\) is reachable from \(\source\) in \(\VAS\) if \( \source \to_{\VAS}^{\ast} \target\), or equivalently if there exists a run \(\rho\) with \(\dirOfRun(\rho)=(\source,\target)\).
%
%A \emph{vector addition system with nested zero tests} (VASnz) \(\VAS\) is a VASSnz with only one state, a \emph{vector addition system with states} (VASS) is a VASSnz where \(g(e)=0\) for every edge \(e\), i.e. no counter is ever tested for \(0\). A VAS is a VASnz which is also a VASS.
%
%The reachability set of a VASSnz \(\VAS\) with an initial configuration \(\source\) and target state \(q\) is \(\vectSet{R}(\VAS, \source, q)=\{\vect{x} \in \N^d \mid \source \CanReach{\VAS} q(\vect{x})\}\). A set \(\vectSet{R}\) is a VASSnz reachability set if \(\vectSet{R}=\vectSet{R}(\VAS, \source, q)\) for some \(\VAS, \source, q\). Since the class of reachability sets of VASSnz is lacking some important closure properties, and we do not want to distinguish between VASSnz and VASnz all the time, we instead consider a larger class of sets which coincides for the two models.
%
%\begin{definition}
%\cite{ClementeCLP17} A set \(\vect{X} \subseteq \N^{d'}\) is a \emph{VASS(nz) section} if there exists \(d \geq d'\), a VASS(nz) reachability set \(\vectSet{R} \subseteq \N^d\), a linear set \(\vectSet{L} \subseteq \N^d\) and a projection \(\pi: \N^d \to \N^{d'}\) such that \(\vectSet{X}=\pi(\vectSet{R} \cap \vectSet{L})\).
%\end{definition}
%
%A relation \(\vectSet{X} \subseteq \N^{d_1} \times \N^{d_2}\) is a VASS(nz) section if it is a section when viewed as a set \(\vectSet{X} \subseteq \N^{d_1+d_2}\). It is well-known that any VAS(nz) reachability \emph{relation} is a section, and that the class of sections is closed under intersection.
%
%We start with a remark about the definition of VASSnz sections.
%
%
%
%
%
%
%
%
%
%
%\begin{remark} \label{RemarkStates}
%At the cost of increasing the dimension by \(3\), states are a special case of fixed never-zero-tested coordinates \cite{HopcroftP79}, hence VASS(nz) sections can equivalently be defined by VAS(nz). Furthermore, similar to how zero tests in Minsky machines can be assumed to only change the state, we will always require that \(f(e)(j)=0\) for all \(j \leq g(e)\), i.e. any counter which is being zero tested is not updated. We cannot require that no counter is updated, since we usually work with VASnz which do not have states.
%\end{remark}















%As one of our main theorems deals with ``For every class of systems \(\SystemClass\) fulfilling \dots something'', we first define systems. 
%
%\begin{definition}
%A \emph{system} \(\mathcal{S}\) of dimension \(d\) is a wqo set \((\Omega(\mathcal{S}), \leq_{\Omega(\mathcal{S})})\) of \emph{runs} together with a map \(\dirOfRun \colon \Omega(\mathcal{S}) \to \N^d \times \N^d\) outputting the \emph{source/target}-pair of a run.
%
%The system \(\mathcal{S}\) fulfills \emph{amalgamation} if for any runs \(\rho_0 \leq_{\Omega(\mathcal{S})} \rho_1, \rho_2\) there exists a \(\rho_3 \geq_{\Omega(\mathcal{S})} \rho_1, \rho_2\) ``closing the diamond'' s.t. \(\dirOfRun(\rho_3)=\dirOfRun(\rho_0)+(\dirOfRun(\rho_1)-\dirOfRun(\rho_0))+(\dirOfRun(\rho_2)-\dirOfRun(\rho_0))\).
%
%Then \(\vectSet{P}_{\rho}:=\{\dirOfRun(\rho')-\dirOfRun(\rho) \mid \rho' \geq_{\Omega(\mathcal{S})} \rho\} \subseteq \N^d \times \N^d\) is called the \emph{transformer relation} of \(\rho\).
%\end{definition}
%
%We will see examples of such systems in a moment, but let us first give intuition about amalgamation. First of all, if \(\rho_1 \geq \rho_0\), then ``\(\rho_1-\rho_0\)'', formally \(\dirOfRun(\rho_1)-\dirOfRun(\rho_0)\), is a loop which can be repeated arbitrarily often. To see this, first amalgamate \(\rho_1\) and \(\rho_1\) over \(\rho_0\), then the newly obtained \(\rho_3\) together with \(\rho_1\) over \(\rho_0\) and so on.
%
%This leads to the image of a run being larger iff it is obtained by inserting ``loops''. The natural expectation is that any two loops can be inserted independently, which is exactly the definition.
%
%Equivalently amalgamation says that \(\vectSet{P}_{\rho}\), the ``set of loops at \(\rho\)'', is closed under addition for any run \(\rho\), and is hence \(\N\)-generated. Observe however that it does \emph{not} have to be \(\N\)-\emph{finitely} generated.










