% !TEX root = Main.tex

We let $\N, \mathbb{Z}, \mathbb{Q}, \mathbb{Q}_{\geq 0}$ denote the sets of natural numbers containing \(0\), the integers, and the (non-negative) rational numbers respectively. We use uppercase letters for sets/relations and boldface for vectors and sets/relations of vectors. 

Given a vector \(\vect{x} \in \Q^d\), we use an array like notation \(\vect{x}[i]\) to refer to the \(i\)-th coordinate. 

Given sets \(\vectSet{X},\vectSet{Y} \subseteq \mathbb{Q}^d, Z \subseteq \mathbb{Q}\), we write \(\vectSet{X}+\vectSet{Y}:=\{\vect{x}+\vect{y} \mid \vect{x} \in \vectSet{X}, \vect{y} \in \vectSet{Y}\}\) for the Minkowski sum and \(Z \cdot \vectSet{X}:=\{\lambda \cdot \vect{x} \mid \lambda \in Z, \vect{x} \in \vect{X}\}\). By identifying elements \(\vect{x}\in \mathbb{Q}^d\) with \(\{\vect{x}\}\), we define \(\vect{x}+\vectSet{X}:=\{\vect{x}\}+\vectSet{X}\), and similarly \(\lambda \cdot \vectSet{X}:=\{\lambda\} \cdot \vectSet{X}\). %for \(\lambda \in \mathbb{Q}\).%We denote by \(\vect{X}^C\) the complement of \(\vect{X}\). 

%We will sometimes perform addition between a vector \(\vect{m} \in \N^d\) and a pair \(\vect{c}=(q, \vect{x}) \in Q \times \N^d\), where \(Q\) is a finite set. We define addition via \(\vect{c}+\vect{m}:=(q, \vect{x}+\vect{m})\).

Given \(\vect{b}=(\vect{b}_s, \vect{b}_t) \in \N^d \times \N^d\) we let \(\Effect(\vect{b}):=\vect{b}_t-\vect{b}_s \in \Z^d\) be the effect of \(\vect{b}\) and extend it to \(\vectSet{F}\subseteq \N^d \times \N^d \) via \(\Effect(\vectSet{F}):=\{\Effect(\vect{b}) \mid \vect{b} \in \vectSet{F}\}\). 

Given relations \(\vectSet{R}_1 \subseteq \N^{d_1} \times \N^{d_2}\) and \(\vectSet{R}_2 \subseteq \N^{d_2} \times \N^{d_3}\), we write \(\vectSet{R}_1 \circ \vectSet{R}_2 :=\{(\vect{v}, \vect{w}) \in \N^{d_1} \times \N^{d_3} \mid \exists \vect{x} \in \N^{d_2}: (\vect{v}, \vect{x}) \in \vectSet{R}_1, (\vect{x}, \vect{w}) \in \vectSet{R}_2\}\) for composition. Given \(\vectSet{R} \subseteq \N^{d} \times \N^{d}\), we write \(\vectSet{R}^{\ast}\) for the reflexive and transitive closure (w.r.t. \(\circ\)).

A relation \(\vectSet{R} \subseteq \N^{d_1} \times \N^{d_2}\) is \emph{monotone} if \(d_1=d_2\) and for all \((\vect{x}, \vect{y}) \in \vectSet{R}\) and \(\vect{m} \in \N^d\) we have \((\vect{x}+\vect{m}, \vect{y}+\vect{m}) \in \vectSet{R}\).

%We write \(\DiagD:=\{(\vect{x}, \vect{x}) \mid \vect{x} \in \N^d\}\) for the diagonal, i.e. the minimal monotone relation containing \((\vect{0}, \vect{0})\). We write \(\Diag\) if the dimension is clear from the context. Adding \(\Diag\) to any relation \(\vectSet{R}\) (in the sense of Minkowski sum as above) produces the minimal monotone relation containing \(\vectSet{R}\).

Let \(\mathbb{S} \in \{\Q, \Z, \Q_{\geq 0}, \N, \N_{\geq 1}\}\) and let \(\vectSet{F} \subseteq \Z^d\). The set of \(\mathbb{S}\)-linear combinations of \(\vectSet{F}\) is 

\(\mathbb{S}(\vectSet{F}):=\{ \sum_{i=1}^n \lambda_i \vect{f}_i \mid n \in \N, \vect{f}_i \in \vectSet{F}, \lambda_i \in \mathbb{S}\}\).

By convention for \(\vectSet{F}=\emptyset\) we have \(\sum_{x \in \emptyset} x:=\vect{0} \in \mathbb{S}(\vectSet{F})\). 

A set \(\vectSet{X}\) is \(\mathbb{S}\)-(finitely) generated  if there exists a (finite) set \(\vectSet{F}\) with \(\vectSet{X}=\mathbb{S}(\vectSet{F})\). The properties \(\mathbb{S}\)-generated and \(\mathbb{S}\)-finitely generated are respectively abbreviated \(\mathbb{S}\)-g. and \(\mathbb{S}\)-f.g.. 

%The following is well-known:

%\begin{lemma}
%Let \(\vectSet{X} \subseteq \Q^d\) and \(\mathbb{S} \in \{\Q, \Z, \Q_{\geq 0}, \N\}\). Then 
%
%\(\vectSet{X}\) is \(\mathbb{S}\)-generated \(\iff\) \(\vectSet{X}=\mathbb{S}(\vectSet{X})\) \(\iff\) \(\vect{0} \in \vectSet{X}\) and \(\vectSet{X}\) has the following closure properties depending on \(\mathbb{S}\):
%\begin{enumerate}
%\item Case \(\mathbb{S}=\mathbb{N}\): Closure under addition, i.e.\ \(\vectSet{X}+\vectSet{X} \subseteq \vectSet{X}\),
%\item Case \(\mathbb{S}=\mathbb{Z}\): Closure under addition and \(-\vectSet{X} \subseteq \vectSet{X}\),
%\item Case \(\mathbb{S}=\mathbb{Q}_{\geq 0}\): Closure under addition and \(\Q_{\geq 0} \cdot \vectSet{X} \subseteq \vectSet{X}\),
%\item Case \(\mathbb{S}=\Q\): Closure under addition and \(\Q \cdot \vectSet{X} \subseteq \vectSet{X}\).
%\end{enumerate}
%\end{lemma} 

%For \(\mathbb{S} \in \{\mathbb{Z}, \mathbb{Q}\}\) every \(\mathbb{S}\)-g. set is \(\mathbb{S}\)-f.g., while for \(\mathbb{S} \in \{\N, \Q_{\geq 0}\}\) this is not the case. For more details see Section \ref{SectionNewLinearSets}. 

There are different names for \(\mathbb{S}\)-g. sets depending on \(\mathbb{S}\), for example \(\Q\)-g. sets are usually called vector spaces, \(\Q_{\geq 0}\) are cones, etc. but we will avoid this in favor of the general terminology of being \(\mathbb{S}\)-generated.

A set \(\vectSet{L} \subseteq \N^d\) is \emph{linear} if \(\vectSet{L}=\vect{b}+\N(\vectSet{F})\) for some \(\vect{b} \in \N^d\) and finite \(\vectSet{F} \subseteq \N^d\). A set \(\vectSet{S}\) is \emph{semilinear} if it is a finite union of linear sets. The semilinear sets are equivalently definable via formulas \(\varphi \in \FO(\mathbb{N}, +)\), called Presburger Arithmetic.

The \emph{dimension} of a \(\Q\)-generated set defined as its minimal number of generators is a well-known concept. It can be extended to arbitrary subsets of \(\mathbb{Q}^d\) as follows.

\begin{definition}{\cite{Leroux11}}
Let \(\vect{X} \subseteq \mathbb{Q}^d\). The \emph{dimension} of \(\vect{X}\), denoted \(\dim(\vect{X})\), is the smallest natural number \(k\) such that there exist finitely many \(\Q\)-g. sets \(\vect{V}_i \subseteq \mathbb{Q}^d\) with \(\dim(\vect{V}_i)\leq k\) and \(\vect{b}_i \in \mathbb{Q}^d\) such that \(\vect{X} \subseteq \bigcup_{i=1}^r \vect{b}_i + \vect{V}_i\). [\(\dim(\emptyset):=-\infty\)]
\end{definition}

This dimension function has the following properties.

\begin{restatable}{lemma}{BasicDimensionProperties}
Let \(\vectSet{X}, \vectSet{X}' \subseteq \mathbb{Q}^d, \vect{b}\in \mathbb{Q}^d\). Then \(\dim(\vectSet{X})=\dim(\vect{b}+\vectSet{X})\) and 
\(\dim(\vectSet{X} \cup \vectSet{X}')=\max \{\dim(\vectSet{X}), \dim(\vectSet{X}')\}\). 

Further, if \(\vectSet{X} \subseteq \vectSet{X}'\), then \(\dim(\vectSet{X}) \leq \dim(\vectSet{X}')\). \label{BasicDimensionProperties}
\end{restatable}

For \(\Q\)-generated sets \(\vectSet{V}\) it is known that \(\vectSet{V} \subseteq \bigcup_{i=1}^r \vect{b}_i+\vectSet{V}_i\) implies \(\vectSet{V} \subseteq \vectSet{V}_i\) for some \(i\), similar results hold for any \(\N\)-g. set \(\vectSet{P}\subseteq \Z^d\) and lead to the following lemma.

\begin{lemma}{\cite[Lemma 5.3]{Leroux11}} \label{LemmaFromJerome}
Let \(\vectSet{P} \subseteq \Z^d\) be \(\N\)-generated. Then \(\dim(\vectSet{P})=\dim(\Q(\vectSet{P}))\).
\end{lemma}

Let \(\vectSet{X}_1, \vectSet{X}_2 \subseteq \N^d\) be sets. Then \(\vectSet{X}_1\) and \(\vectSet{X}_2\) \emph{have a non-degenerate intersection} if \(\dim(\vectSet{X}_1 \cap \vectSet{X}_2)=\dim(\vectSet{X}_1)=\dim(\vectSet{X}_2)\).
In the sequel many of our lemmas only work under the assumption that two sets have a non-degenerate intersection, i.e.\ this is a very central notion.

Let us provide some examples of (non-)degenerate intersections. Intersecting \(\N^2 \cap \N=\N\) is degenerate because we did not intersect same dimension objects. Also 

\(\{(x,y) \mid y \geq x\} \cap \{(x,y) \mid y \leq x\}=\{(x,y) \mid y=x\}\) is degenerate because their intersection is lower dimensional. On the other hand \(\{(x,y) \mid y \geq x\} \cap \{(x,y) \mid y \leq 2x\}\) is a typical non-degenerate intersection. 

Because ``\(\vectSet{L}\) and \(\vectSet{L}'\) have a non-degenerate intersection'' is rather long, we will often just write ``\(\vectSet{L} \cap \vectSet{L}'\) is non-degenerate''.

All the definitions above defined for sets apply to relations \(\vectSet{R} \subseteq \N^{d_1} \times \N^{d_2}\) by viewing them as sets \(\vectSet{R} \subseteq \N^{d_1+d_2}\).