% !TEX root = Main.tex

In this section we define axioms, and prove that for classes of systems fulfilling them most problems are decidable.

We start with some preliminary definitions.

\begin{definition}
A set \(\vectSet{L} \subseteq \N^d\) is \emph{directed hybridlinear} if \(\vectSet{L}=\{\vect{b}_1, \dots, \vect{b}_k\}+\N(\vectSet{F})\) for a finite set \(\vectSet{F}\) and \((\vect{b}_i+\N(\vectSet{F})) \cap (\vect{b}_j+\N(\vectSet{F})) \neq \emptyset\) for all \(1 \leq i,j \leq k\).
\end{definition}

Intuitively, a directed hybridlinear set is a linear set minus some minor things on the boundary. For example in \(\N^3\) we have \(\vectSet{L}=\{(1,1,0), (1,0,1), (0,1,1)\}+\N^3\) is directed hybridlinear, is obtained from \(\N^3\) by removing three lines. On the other hand, sets of the form \(\{0,1\} \times \N \subseteq \N^2\) are clearly not directed hybridlinear, the base points can never be chosen to fulfill the extra condition.

Dir.hybridlinear has the following advantage over linear:

\begin{restatable}{lemma}{LemmaDirectedHybridlinearNondegenerateIntersection}
Let \(\vectSet{L}\) and \(\vectSet{L}'\) be directed hybridlinear with a non-degenerate intersection. Then \(\vectSet{L} \cap \vectSet{L}'\) is dir. hybridlinear. \label{LemmaDirectedHybridlinearNondegenerateIntersection}
\end{restatable}

While linear sets are not closed under intersection at all, dir. hybridlinear sets are closed under non-degenerate intersection.

The following is the basis of our semilinearity algorithm:

\begin{definition}
Let \(\vectSet{L}=\{\vect{b}_1, \dots, \vect{b}_k\}+\N(\vectSet{F})\) be directed hybridlinear and \(\vectSet{X} \subseteq \vectSet{L}\). If there is \(\vect{x} \in \N(\vectSet{F})\) s.t. \(\vect{x}+\vectSet{L} \subseteq \vectSet{X}\), then \(\vectSet{X}\) is called \emph{reducible}, otherwise \emph{irreducible}. 

Such a vector \(\vect{x}\) is called a \emph{reduction point}.
\end{definition}

Intuitively, if \(\vectSet{X}\) is reducible with reduction point \(\vect{x}\), then $\vectSet{X}$ and $\vectSet{L}$ coincide for points larger than $\vect{x}$.

Now we are ready to define the axioms, and afterwards give intuition. Let \(\RelationClass\) be a class of relations closed under intersecting with semilinear relations, in our case the VASSnz sections. A relation \(\HybridizationRelation\) relating objects \(\vectSet{X} \in \RelationClass\) to directed hybridlinear relations \(\vectSet{L}\) is \emph{nice (for \(\RelationClass\))} if it fulfills the following axioms.
%since \cite{GuttenbergRE23} had similar properties, though with general hybridlinear sets, which lead to multiple difficulties and the need for extra properties. %It mainly requires adaptations of \cite{GuttenbergRE23} Proposition 5.2, Lemma 6.1 and an argument in the proof of Corollary E.2 respectively.

\begin{restatable}{definition}{DefinitionAxioms} The following are the geometric axioms:
\begin{enumerate}
\item[1)] Let \(\vectSet{X} \HybridizationRelation \vectSet{L}=\vectSet{B}+\N(\vectSet{F})\). Then for every \(\vect{x} \in \vectSet{L}\) and every \(\vect{w} \in \N_{\geq 1}(\vectSet{F})\), \(\exists N \in \N\) s.t. \(\vect{x}+\N_{\geq N}\vect{w} \subseteq \vectSet{X}\). 
\item[2)] There is an algorithm which given \(\vectSet{X}\in \RelationClass\) computes a set of dir. hybridlinear \(\{\vectSet{L}_1, \dots, \vectSet{L}_k\}\) s.t. there exist \(\vectSet{X}_j\) with \(\vectSet{X}=\vectSet{X}_1 \cup \dots \cup \vectSet{X}_k\) and \(\vectSet{X}_j \HybridizationRelation \vectSet{L}_j\) for every \(j\).
\item[3)] If \(\vectSet{L}=\vectSet{B}+\N(\vectSet{F})\) is directed hybridlinear, then \(\vectSet{L} \HybridizationRelation \vectSet{L}\).
\item[4)] If \(\vectSet{X}_1 \HybridizationRelation \vectSet{L}_1\) and \(\vectSet{X}_2 \HybridizationRelation \vectSet{L}_2\) s.t. \(\vectSet{L}_1 \cap \vectSet{L}_2\) is non-degenerate then \((\vectSet{X}_1 \cap \vectSet{X}_2) \HybridizationRelation (\vectSet{L}_1 \cap \vectSet{L}_2)\). \footnote{Linear sets are not closed under non-degenerate intersection, hence we require directed hybridlinear.}
\item[5)] If \(\vectSet{X}_1 \HybridizationRelation \vectSet{L}\) and \(\vectSet{X}_2 \HybridizationRelation \vectSet{L}\), then \((\vect{X}_1 \cup \vect{X}_2) \HybridizationRelation \vectSet{L}\).
\item[6)] If \(\vectSet{X} \HybridizationRelation \vectSet{L}\) and \(\vectSet{X}\) is irreducible, then \(\vectSet{X}\) is non-semilinear.
\item[7)] There is an algorithm which given \(\vectSet{X}, \vectSet{L}\) \emph{fulfilling the promise} \(\vectSet{X} \HybridizationRelation \vectSet{L}\) decides whether \(\vectSet{X}\) is reducible, and if yes computes a reduction point \(\vect{x}\).
\item[8)] Let \(\vectSet{X} \HybridizationRelation \vectSet{L}=\vectSet{B}+\N(\vectSet{F})\) and \(\vectSet{S}\) be semilinear s.t. \(\vectSet{X} \subseteq \vectSet{S}\). Then there exists \(\vect{x}\in \N(\vectSet{F})\) s.t. \(\vect{x}+\vectSet{L} \subseteq \vectSet{S}\).
%\item[9)] Let \(\vectSet{X} \HybridizationRelation \vectSet{L} \subseteq \N^d\) and \(\pi \colon \N^d \to \N^{d'}\) a projection. Then \(\pi(\vectSet{X}) \HybridizationRelation \pi(\vectSet{L})\).
%\item[10)] Let \(\vectSet{X} \subseteq \vectSet{L} \subseteq \N^{d_1}\) and let \(\vect{b} \in \N^{d_2}\) be any vector. Then \(\vectSet{X} \HybridizationRelation \vectSet{L} \iff \{\vect{b}\} \times \vectSet{X} \HybridizationRelation \{\vect{b}\} \times \vectSet{L}\).
%\item[11)] Let \(\vectSet{X} \HybridizationRelation \vectSet{L}\), \(\vect{x} \in \vectSet{L}\) and let \(\vect{w}\) be a sum of periods of \(\vectSet{L}\) using all periods at least once. Then there is \(N \in \N\) s.t. \(\vect{x}+\N_{\geq N} \vect{w} \subseteq \vectSet{X}\).
%\item[12)] Let \(\vectSet{X}_{12} \HybridizationRelation \vectSet{L}_{12} \subseteq \N^{d_1} \times \N^{d_2}\) and \(\vectSet{X}_{23} \HybridizationRelation \vectSet{L}_{23} \subseteq \N^{d_2} \times \N^{d_3}\). Let \(\pi_{out}(\vectSet{L}_{12})\) be the projection to \(\N^{d_2}\) of \(\vectSet{L}_{12}\), and \(\pi_{in}(\vectSet{L}_{23})\) be the projection to \(\N^{d_2}\) of \(\vectSet{L}_{23}\). If \(\pi_{out}(\vectSet{L}_{12}) \cap \pi_{in}(\vectSet{L}_{23})\) is a non-degenerate intersection, then \(\vectSet{X}_{12} \circ \vectSet{X}_{23} \HybridizationRelation \vectSet{L}_{12} \circ \vectSet{L}_{23}\).
\end{enumerate}

Let \(\alpha \in \N\). If class \(\RelationClass\) fulfills the Axioms with running time \(\mathfrak{F}_{\alpha}\) for Axioms 2 and 7, then \(\RelationClass\) is called \(\mathfrak{F}_{\alpha}\)-effective.
\end{restatable}

Using Theorem \ref{TheoremVASSnzIdealDecomposition} to solve Axiom 2, and an adaptation of Theorem \ref{TheoremVASSnzIdealDecomposition} to solve Axiom 7, we obtain:

\begin{restatable}{theorem}{LemmaModelAxiomsVASSnz} \label{TheoremModelAxiomsVASSnz}
The class \(\RelationClass\) of \((k,d)\)-VASSnz sections is \(\mathfrak{F}_{2kd+2k+2d+5}\)-effective with the model: \(\vectSet{X} \HybridizationRelation \vectSet{L}\) iff 

\(\vectSet{L}=\vectSet{B}+\N(\vectSet{F})\) is directed hybridlinear and for all \(\vect{x} \in \vectSet{L}\) and all \(\vect{w} \in \N_{\geq 1}(\vectSet{F})\), there exists \(N \in \N\) s.t. \(\vect{x}+\N_{\geq N} \vect{w} \subseteq \vectSet{X}\).
\end{restatable}

With the framework in place and a first class of systems inside, it remains to exhibit the power of the framework.

\begin{restatable}{theorem}{TheoremUseHybridizationToDecideEverything} \label{TheoremUseHybridizationToDecideEverything}
Let \(\RelationClass\) be \(\mathfrak{F}_{\alpha}\)-effective. Then the following problems are decidable in the time bound stated:
\begin{enumerate}
\item[(1)] Reachability, i.e. is \(\vectSet{X}\) non-empty? Time: \(\mathfrak{F}_{\alpha}\).
\item[(2)] Boundedness, i.e. is \(\vectSet{X}\) finite? Time: \(\mathfrak{F}_{\alpha}\).
%\item[(3)] Compute the \(\omega\)-configurations of \(\DownwardClosure{\vectSet{X}}\) in \(\mathfrak{F}_{\alpha}\) time.
\item[(3)] Semilinearity, i.e. is \(\vectSet{X}\) semilinear, and 

if yes, output a semilinear representation. Time: \(\mathfrak{F}_{\alpha+1}\).

\item[(4)] Given \(\vectSet{X}\) and semilinear \(\vectSet{S}\), is \(\vectSet{S} \subseteq \vectSet{X}\)? Time: \(\mathfrak{F}_{\alpha+1}\).
\item[(5)] Given \(\vectSet{X}\) and semilinear \(\vectSet{S}\), is \(\vectSet{S} = \vectSet{X}\)? Time: \(\mathfrak{F}_{\alpha+1}\).
\item[(6)] \(\mathcal{F}\)-separability for \(\mathcal{F}\)=Semil, Mod, Unary in time \(\mathfrak{F}_{\alpha+1}\).
\end{enumerate}
\end{restatable}

We now proceed to prove the parts one after the other, while simultaneously introducing intuition for the axioms. Axioms 1 and 2 we have already used in the reachability algorithm.

%The heart of the axioms is Axiom 1). It has many inconspicuous but useful consequences like the following.
%
%\begin{lemma}
%Let \(\vectSet{X} \HybridizationRelation \vectSet{L}=\vectSet{B}+\N(\vectSet{F})\) and \(\vect{x} \in \N(\vectSet{F})\). Then \((\vect{x}+\vectSet{L}) \cap \vectSet{X} \neq \emptyset\). \label{LemmaPointsAreSomewhatDense}
%\end{lemma}
%
%\begin{proof}[Proof by contradiction] Assume \((\vect{x}+\vectSet{L}) \cap \vectSet{X}=\emptyset\). 
%
%Then \(\vectSet{X} \subseteq (\vectSet{L} \setminus \vect{x}+\vectSet{L})\), and hence \(\dim(\vectSet{X}) \leq \dim(\vectSet{L} \setminus (\vect{x} + \vectSet{L}))<\dim(\vectSet{L})\) by Lemma \ref{LemmaDimensionDecreaseShiftedL}. Contradiction to Axiom 1.
%\end{proof}
%
%Hence in addition to \(\vectSet{X}\) being contained in \(\vectSet{L}\), Axiom 1 says that \(\vectSet{X}\) contains points arbitrarily deep inside \(\vectSet{L}\). Furthermore, the dimension contains the information that these points are not all on some line, etc. but rather spread all throughout \(\vectSet{L}\). 
%
%Axiom 2 effectively decomposes any given \(\vectSet{X} \in \RelationClass\) in a way which has many useful consequences.

\begin{proof}[Proof of Theorem \ref{TheoremUseHybridizationToDecideEverything}, (1)-(2)]
If \(\vectSet{X} \HybridizationRelation \vectSet{L}\), then \(\vectSet{X} \neq \emptyset\) by Axiom 1. Hence for (1) apply Axiom 2) and check for \(k=0\). 

Regarding (2), if \(\vectSet{X} \HybridizationRelation \vectSet{L}\), then by Axiom 1, if \(\vectSet{L}\) has a period, then \(\vectSet{X}\) is infinite. Hence for (2) we apply Axiom 2) and simply check whether some \(\vectSet{L}_j\) has a period.
\end{proof}

%By Axiom 3 linear sets have a nice overapproximation.
%
%For Axiom 4, we have seen a use case in Algorithm \ref{AlgorithmMainStructure}.
%
%Axiom 5 is a closure property useful for some algorithms.

For solving the semilinearity problem, we have to understand the intuition behind Axioms 6 and 7. Imagine 6) and 7) were both true even under the weaker assumption \(\vectSet{X} \subseteq \vectSet{L}\), under which we defined reducibility. Then there would be an obvious algorithm for deciding semilinearity of \(\vectSet{X}\): Apply 7) to check whether there exists a reduction point \(\vect{x}\). If not, then \(\vectSet{X}\) is irreducible and hence we can answer non-semilinear by Axiom 6). If yes, then we can reduce \(\vectSet{X}\): 7) provides a reduction point \(\vect{x}\), and because of \(\vect{x}+\vectSet{L} \subseteq \vectSet{X}\), whether \(\vectSet{X}\) is semilinear or not only depends on \(\vectSet{X} \cap \vectSet{S}\), where \(\vectSet{S}:=\vectSet{L} \setminus (\vect{x}+\vectSet{L})\). Since \(\dim(\vectSet{S}) < \dim(\vectSet{L})\) by Lemma \ref{LemmaDimensionDecreaseShiftedL}, we can simply continue recursively with \(\vectSet{X}_{new}:=\vectSet{X} \cap \vectSet{S}\).

There are however two problems with this simpler version. Problem 1 is that Axiom 6 is false with only the assumption \(\vectSet{X} \subseteq \vectSet{L}\): For example \(\vectSet{X}=\{x-\text{axis}\} \cup \{y-\text{axis}\} \subseteq \vectSet{L}:=\N^2\) is irreducible, but semilinear. Problem 2 is that while an algorithm for Axiom 7 exists without the promise, it more or less requires the semilinearity algorithm as a subroutine, leading to circular reasoning.

%The main question to understand is ``How strong does the assumption \(\vectSet{X} \HybridizationRelation \vectSet{L}\) have to be to ensure Axiom 6?'' The answer lies on the right of Figure \ref{FigureIntuitionSemilinearityAlgorithm}, which illustrates the typical shape of an \(\vectSet{X}\) which has a partner \(\vectSet{L}\) for \(\vectSet{X} \HybridizationRelation \vectSet{L}\): If \(\vectSet{X}\) contains lines in directions \(\vect{v}\) and \(\vect{v}'\), i.e. \(\vect{x}+\N \vect{v} \subseteq \vectSet{X}\) and \(\vect{x}'+\N \vect{v}' \subseteq \vectSet{X}\) for some starting points \(\vect{x}, \vect{x}'\), then it should also contain a \(\vect{v}, \vect{v}'\)-cone, i.e. \(\vect{x}+\N \vect{v} + \N \vect{v}' \subseteq \vectSet{X}\) for some \(\vect{x}\). Picture-wise, whenever you have two infinite lines, then you can fill out the cone between them (though possibly only starting at a later point \(\vect{x}\) than before). The linear relation \(\vectSet{L}\) then describes the ``asymptotic behaviour''. Visually, asymptotic behaviour corresponds to the eventual steepness of the upper and lower bound functions. For example for \(\vectSet{X}\) on the right of Figure \ref{FigureIntuitionSemilinearityAlgorithm}, the asymptotic behaviour is the whole plane, because \(x^2\) becomes arbitrarily steep. 
%
%If \(\vectSet{X}\) were semilinear, then its steepness can only increase finitely often, this is because for every linear set there is a maximal steepness. Hence every asymptotic behaviour has to be attained, and one can utilize this together with the above line combination property to obtain a reduction point \(\vect{x}\). This argument (formalized in \cite{GuttenbergRE23}, Lemma 6.1) proves that \(\vectSet{X} \HybridizationRelation \vectSet{L}\) together with semilinearity implies reducibility, i.e. proves Axiom 6 by contradiction.

Problems 1 and 2 are resolved by adding the assumption \(\vectSet{X} \HybridizationRelation \vectSet{L}\) in 6) and 7), and using the closure properties in Axioms 3-5) we can complete the algorithm sketch. The argument for this fact is presented in \cite{GuttenbergRE23}, Section 5.

%\begin{enumerate}
%\item[1)'] There is an algorithm which given \(\vectSet{X} \in \RelationClass\) and semilinear \(\vectSet{S}\) computes a partition \(\vectSet{S}=\vectSet{S}_1 \cup \dots \cup \vectSet{S}_r\) such that for all \(i\) either \(\vectSet{X} \cap \vectSet{S}_i=\emptyset\) or \((\vectSet{X} \cap \vectSet{S}_i) \HybridizationRelation \vectSet{S}_i\).
%\end{enumerate}
%
%Statement 1)' allows to complete the semilinearity algorithm by computing the partition first, and applying the ideas above in every part of the partition separately. 
%
%Axioms 1-5) combined imply Axiom 1)' in addition to their other uses. The argument is slightly too involved to quickly reexplain here, but not too complicated, see Section 5.1 of \cite{GuttenbergRE23}.

\begin{proof}[Proof of Theorem \ref{TheoremUseHybridizationToDecideEverything}, (3)-(5)]
(3): We explained the algorithm in the paragraphs above/refer to \cite{GuttenbergRE23}, Section 5. For the time bound observe that Axiom 7 takes time \(\mathfrak{F}_{\alpha}\), and we have recursion depth at most \(d+1\), hence the time bound is a \((d+1)\)-fold application of a function in \(\mathfrak{F}_{\alpha}\), leading to \(\mathfrak{F}_{\alpha+1}\).

(4): Apply the semilinearity algorithm on \(\vectSet{X}_{new}:=\vectSet{X} \cap \vectSet{S}\). If you obtain a semilinear representation equivalent to \(\vectSet{S}\), return true, otherwise return false. Clearly again \(\mathfrak{F}_{\alpha+1}\).

(5): Check \(\vectSet{S} \subseteq \vectSet{X}\) via (4), and check that \(\vectSet{X}_{new}:=\vectSet{X} \cap \vectSet{S}^{C}=\emptyset\) via (1), where \(\vectSet{S}^C\) is the complement of \(\vectSet{S}\).
\end{proof}

Also problems of the following type, called \emph{separability problems}, can be answered using \(\HybridizationRelation\). Fix a class \(\mathcal{F}\) of relations.

E.g. \(\mathcal{F}=\) Semilinear, Recognizable, Modulo, etc.

\begin{definition}
Two sets \(\vectSet{X}\) and \(\vectSet{Y}\) are \(\mathcal{F}\)-\emph{separable} if there exists \(\vectSet{S} \in \mathcal{F}\) s.t. \(\vectSet{X} \subseteq \vectSet{S}\) and \(\vectSet{Y} \cap \vectSet{S}=\emptyset\). The \(\mathcal{F}\)-\emph{separability problem} asks given \(\vectSet{X}\) and \(\vectSet{Y}\), are they \(\mathcal{F}\)-separable?
\end{definition} 

To solve these problems, we use the following definition:

\begin{definition}
Let \(\vectSet{X}\) be any set. Let \(\vectSet{S} \in \mathcal{F}\) with \(\vectSet{X} \subseteq \vectSet{S}\). Then \(\vectSet{S}\) is called \emph{up-to-boundary-optimal} (utbo) \(\mathcal{F}\) \emph{overapproximation of} \(\vectSet{X}\) if any other overapproximation \(\vectSet{S}' \in \mathcal{F}\) fulfills \(\vectSet{S}+\vect{x} \subseteq \vectSet{S}'\) for some vector \(\vect{x}\in \N^d\).
\end{definition}

Intuitively, an \(\mathcal{F}\)-overapproximation \(\vectSet{S}\) would be \emph{optimal} if every other overapproximation \(\vectSet{S}' \in \mathcal{F}\) is larger. We require a weaker property: Utbo, which states that some parts of the boundary of \(\vectSet{S}\) might not be contained in \(\vectSet{S}'\), instead a shifted version \(\vect{p}+\vectSet{S} \subseteq \vectSet{S}'\). As an example of this definition, consider Axiom 8: Axiom 8 states that if \(\vectSet{X} \HybridizationRelation \vectSet{L}\), then \(\vectSet{L}\) is a utbo semilinear overapproximation of \(\vectSet{X}\): Namely any semilinear overapproximation \(\vectSet{S}\) necessarily contains \(\vect{x}+\vectSet{L}\) for \(\vect{x} \in \N(\vectSet{F})\).

Theorem \ref{TheoremUseHybridizationToDecideEverything}(6) will be a consequence of the following:

\begin{theorem} \label{TheoremDecidingSeparability}
Let \(\RelationClass\) be an \(\mathfrak{F}_{\alpha}\)-effective class, and let \(\mathcal{F}\) be a class of relations s.t. the following hold:
\begin{itemize}
\item \(\mathcal{F}\) is closed under Boolean operations in time \(\mathfrak{F}_{\alpha}\).
\item Computing \(\dim(\vectSet{T})\) given \(\vectSet{T} \in \mathcal{F}\) is in time \(\mathfrak{F}_{\alpha}\).
\item There is an algorithm with a time bound of \(\mathfrak{F}_{\alpha}\) for: 

Input: A utbo semilinear overapproximation \(\vectSet{S}\) of \(\vectSet{X}\).

Output: A utbo \(\mathcal{F}\) overapproximation \(\vectSet{S}'\) of the same \(\vectSet{X}\).
\end{itemize}
Then \(\mathcal{F}\)-separability is decidable in time \(\mathfrak{F}_{\alpha+1}\).
\end{theorem}

\begin{proof}
We check \(\mathcal{F}\)-separability for \(\vectSet{X}\) and \(\vectSet{Y}\) as follows:

Step 1:  Apply Axiom 2 to write \(\vectSet{X}=\vectSet{X}_1 \cup \dots \cup \vectSet{X}_k\) and \(\vectSet{Y}=\vectSet{Y}_1 \cup \dots \cup \vectSet{Y}_s\) with \(\vectSet{X}_j \HybridizationRelation \vectSet{L}_j\) and \(\vectSet{Y}_m \HybridizationRelation \vectSet{S}_m\) for all \(j,m\). Since \(\mathcal{F}\) is closed under all Boolean operations, it is sufficient and necessary to \(\mathcal{F}\)-separate every \(\vectSet{X}_j\) from every \(\vectSet{Y}_m\). 

Step 2: For all indices \(j,m\) do: By Axiom 8, \(\vectSet{L}_j\) is a utbo semilinear overapproximation of \(\vectSet{X}_j\) and similarly \(\vectSet{S}_m\) is a utbo semilinear overapproximation of \(\vectSet{Y}_m\). From these compute utbo \(\mathcal{F}\)-overapproximations \(\vectSet{L}_j'\) and \(\vectSet{S}_m'\) of \(\vectSet{X}_j\) and \(\vectSet{Y}_m\) respectively using bullet point 3.

Step 3: Finally we check whether \(\vectSet{L}_j' \cap \vectSet{S}_m'\) is non-degenerate for \emph{some} \(j\) and \(m\). This can be done by bullet points 1 and 2. 

Step 4: If such \(j,m\) exist, then the algorithm rejects. Otherwise the algorithm continues recursively with \(\vectSet{X}_{\text{new}}:=\vectSet{X} \cap \bigcup_{j,m} (\vectSet{L}_j' \cap \vectSet{S}_m')\) and \(\vectSet{Y}_{\text{new}}:=\vectSet{Y} \cap \bigcup_{j,m} (\vectSet{L}_j' \cap \vectSet{S}_m')\), which are of lower dimension.

Correctness: We prove that the algorithm can safely reject if any \(\vectSet{L}_j' \cap \vectSet{S}_m'\) is non-degenerate. We remove the indices for readability, refering to \(\vectSet{X}_j, \vectSet{L}_j'\) as \(\vectSet{X}, \vectSet{L}\) and \(\vectSet{Y}_m, \vectSet{S}_m'\) as \(\vectSet{Y}, \vectSet{S}\).

Since \(\mathcal{F}\) is closed under complement, \(\vectSet{X}, \vectSet{Y}\) are \(\mathcal{F}\)-separable if and only if there exist \(\vectSet{T}_{\vectSet{X}}, \vectSet{T}_{\vectSet{Y}} \in \mathcal{F}\) s.t. \(\vectSet{X} \subseteq \vectSet{T}_{\vectSet{X}}, \vectSet{Y} \subseteq \vectSet{T}_{\vectSet{Y}}\) and \(\vectSet{T}_{\vectSet{X}} \cap \vectSet{T}_{\vectSet{Y}}=\emptyset\). We claim that such \(\vectSet{T}_{\vectSet{X}}, \vectSet{T}_{\vectSet{Y}}\) do not exist.

Let \(\vectSet{T}_{\vectSet{X}},\vectSet{T}_{\vectSet{Y}} \in \mathcal{F}\) with \(\vectSet{X} \subseteq \vectSet{T}_{\vectSet{X}}\) and \(\vectSet{Y} \subseteq \vectSet{T}_{\vectSet{Y}}\) be arbitrary candidates for separating. Since \(\vectSet{L}\) and \(\vectSet{S}\) are utbo \(\mathcal{F}\) overapproximations of \(\vectSet{X}, \vectSet{Y}\) respectively, there exist shifts \(\vect{x}, \vect{y}\) such that \(\vect{x}+\vectSet{L} \subseteq \vectSet{T}_{\vectSet{X}}\) and \(\vect{y}+\vectSet{S} \subseteq \vectSet{T}_{\vectSet{Y}}\). Since \(\vectSet{L} \cap \vectSet{S}\) is non-degenerate, also \((\vect{x}+\vectSet{L}) \cap (\vect{y}+\vectSet{S})\) is non-degenerate, and in particular \(\vectSet{T}_{\vectSet{X}} \cap \vectSet{T}_{\vectSet{Y}} \supseteq (\vect{x}+\vectSet{L}') \cap (\vect{y}+\vectSet{S}') \neq \emptyset\) is non-empty as was to prove.

Time Bound: The recursion depth of Step 4 is at most \(d+1\), a constant, and one call takes \(\mathfrak{F}_{\alpha}\), hence we obtain \(\mathfrak{F}_{\alpha+1}\).
\end{proof}

We remark that while Theorem \ref{TheoremDecidingSeparability} may be applied for \(\mathcal{F} \not \subseteq \) Semilinear, already for \(\mathcal{F}=\)Semilinear separability is equivalent to disjointness of \(\vectSet{X}\) and \(\vectSet{Y}\). This fact is similar to \cite{GuttenbergRE23}, Cor. 6.5 and is a consequence of this algorithm: For \(\mathcal{F}=\)Semilinear, if in step 4 of Theorem \ref{TheoremDecidingSeparability} we find \(\vectSet{L}_i \cap \vectSet{S}_j\) non-degenerate, then by Axiom 4 we have \(\vectSet{X}_i \cap \vectSet{Y}_j \HybridizationRelation \vectSet{L}_i \cap \vectSet{S}_j\). In particular \(\vectSet{X}_i \cap \vectSet{Y}_j \neq \emptyset\). Otherwise the algorithm outputs a semilinear separator, in particular there exists one.

Finally we are ready to finish the proof of Theorem \ref{TheoremUseHybridizationToDecideEverything}.

\begin{proof}[Proof of Theorem \ref{TheoremUseHybridizationToDecideEverything}(6)]
It is easy to check that the classes mentioned fulfill bullet points 1-3 of Theorem \ref{TheoremDecidingSeparability}. Hence Theorem \ref{TheoremUseHybridizationToDecideEverything}(6) follows from Theorem \ref{TheoremDecidingSeparability}.
\end{proof}

%
%We end this section by giving intuition for the so far unused Axioms 9-12). Axioms 9 and 10 are typical geometric properties: They intuitively state that \(\vectSet{X} \HybridizationRelation \vectSet{L}\) does not depend on the embedding space: If we project \(\vectSet{X}\) and \(\vectSet{Y}\) or if we embed \(\vectSet{X}\) and \(\vectSet{L}\) in a higher dimensional space with some fixed coordinates, then \(\HybridizationRelation\) is preserved.
%
%Axiom 11 states that for any vector \(\vect{w}\) ``pointing to the inside of \(\vectSet{L}\)'', if we walk in that ``direction'' \(\vect{w}\) long enough, we will eventually permanently be in \(\vectSet{X}\). Axiom 12 is a closure property similar to Axiom 4, but this time for composition instead of intersection.
%
%Axioms 9-12 will prove very useful in Section \ref{SectionMainAlgorithm}.
