
\label{sec:complete-example}
We assume that $\setnumbers$ is signed $32$-bit fixed-point numbers, with four decimal places precision, similar to Example~\ref{example:fixedpointarithmetic}.
%
The number $\frac{1}{125}$ is stored as (the binary encoding of the decimal number) $80$, $\frac{1}{1000}$ is stored as $10$, $0.9$ is stored as $9000$, $1000$ is stored as $10000000$, $250$ is stored as $2500000$, etc.
%
We also assume that the maximum arity $\arity$ is 5.

Elaborating on \Cref{ex:gnn_social_network}, consider the GNN $N = (\layer_1, \layer_{out})$ where $\layer_1 = (\agg_1, \comb_1)$, $\agg_1 = \sum$, $\comb_1((x_1), (\aggvar_1)) = \begin{pmatrix}
    ReLU(+\frac{1}{125}x_1 + 0\aggvar_1 + 0) \\
    Id(+0x_1 + \frac{1}{1000}\aggvar_1+ 0)
    \end{pmatrix}
    $.
So $N$ has only one layer, with input dimension $1$ (the value of $x_1$ that labels a node indicates the number of messages per minute sent by the individual), and output dimension~$2$. The output layer is the identity.
%
%
Consider also the LVP $I = (N, L_{\text{in}}, L_{\text{out}})$ where $L_{\text{in}} = x_1 \geq 100$ and $L_{\text{out}} = y_1 \geq 0.6$.
%




\paragraph{$\thelogic{}$ formula.}

As per \Cref{th:reduction} and its proof,
the LVP $I = (N, L_{\text{in}}, L_{\text{out}})$ is valid iff $\phi_I = \phi_{\text{in}} \land \phi_N \land \lnot \psi_{\text{out}}$ is not satisfiable, where:
\begin{itemize}
    \item $\phi_{\text{in}} := x_1 \geq 100$
    \item $\phi_N := ReLU(\frac{1}{125} x_1) - y_1 = 0 \land \frac{1}{1000}\agg(x_1) - y_2 = 0$
    \item $\psi_{\text{out}} := y_1 \geq 0.6$
\end{itemize}

\paragraph{Automated reasoning.}

We could show that
$(\agg(x_1) \geq 1000) 
\land \phi_N \land
\lnot (y_2 \geq 1.0)
$ is not satisfiable, establishing that $N$ comes with the guarantee of property $B$. Here we focus on property $A$.
%
If the pointed graph $(G,v)$ satisfies $L_{\text{in}}$, then $N(G,v)$ will satisfy $L_{\text{out}}$.
%
% Also, if $(G,v)$ has successors who send more than 1000 messages per minute in total, then $N(G,v)$ will satisfy $y_2 \geq 1.0$, indicating that $v$ may be part of a network of spammers.
%
% $(agg(x_1) \geq 1000) 
% \land 
% \phi_N
% \land
% \lnot (y_2 \geq 1.0)
% $ is unsatisfiable
%
%\todo[inline]{how could we make it formal...? Could imagine in the future having the $L_{in}$ constraint is expressed in the logic}
%
We could use the tableau method to automatically show that $\phi_I$ is not satisfiable and therefore that $I$ is valid. Unfortunately, a complete formal proof cannot be reasonably presented here or even done by hand.
%
Hence, to illustrate how the reasoning method can be used for the verification of quantized GNNs, we slightly modify the LVP $I$ with a different output constraint.

Let $I'$ be the LVP $(N, L_{\text{in}}, L'_{\text{out}}, \delta=5)$, where $N$ and $L_{\text{in}}$ are as before, and $L'_{\text{out}} := y_1 \geq 0.9$.
%
We are going to show that the new output constraint is not always satisfied. 
%
$I'$ is valid iff $\phi_{I'} := \phi_{\text{in}} \land \phi_N \land \lnot \psi'_{\text{out}}$ is not satisfiable, where:
\begin{itemize}
    % \item $\phi_{\text{in}} = x_1 \geq 100$
    % \item $\phi_N = \frac{1}{125} ReLU(x_1) - y_1 = 0 \land \frac{1}{1000}\agg(x_1) - y_2 = 0$
    \item $\psi'_{\text{out}} := y_1 \geq 0.9$
\end{itemize}

\newcommand{\ruledef}{\Delta}%\text{def}}

We employ our tableau method to prove that $\phi_{I'}$ is satisfiable and therefore that $I'$ is not valid.
%
We use $\checkmark$ to indicate that we have reached a term for which there are no rules to apply, except possibly $(clash_=)$. If in the end $(clash_=)$ does not apply, then we can read a model of $\phi_{I'}$ where the $\checkmark$s appear. 
We start the tableau method by applying the rule $(\land)$ and resolve inequalities with rules $(\geq)$ and $(\lnot \geq)$ (we write $(\ruledef)$ when we just use the definition of a formula):
\begin{prooftree}\small
\AxiomC{$(\epsilon ~ \phi_{\text{in}} \land \phi_N \land \lnot \psi'_{\text{out}})$}
\rulelabel{\land}
\UnaryInfC{$(\epsilon ~ \phi_{\text{in}}), (\epsilon ~ \phi_N), (\epsilon ~  \lnot \psi'_{\text{out}})$}
\end{prooftree}




\noindent
\begin{minipage}{0.15\textwidth}
\begin{prooftree}\scriptsize
\AxiomC{$(\epsilon ~ \phi_{\text{in}})$}
\rulelabel{\ruledef}
\UnaryInfC{$(\epsilon ~ x_1 {\geq} 100)$}
\rulelabel{\geq}
\UnaryInfC{$(\epsilon ~ x_1 {=} 100)$ $\checkmark$}
\end{prooftree}
\end{minipage}
%\hfill
     \begin{minipage}{0.15\textwidth}
     	\begin{prooftree}\scriptsize
     		\AxiomC{$(\epsilon ~ \phi_N)$}
     		\rulelabel{\ruledef}
     		\UnaryInfC{$(\epsilon ~ C_1 {\land} C_2)$}
     		\rulelabel{\land}
     		\UnaryInfC{$(\epsilon ~ C_1), (\epsilon ~ C_2)$}
     	\end{prooftree}
\end{minipage}
%\hfill
     \begin{minipage}{0.15\textwidth}
\begin{prooftree}\scriptsize
	\AxiomC{$(\epsilon ~ \lnot \psi'_{\text{out}})$}
	\rulelabel{\ruledef}
	\UnaryInfC{$(\epsilon ~ \lnot y_1{\geq} 0.9)$}
	\rulelabel{{\lnot} {\geq}}
	\UnaryInfC{$(\epsilon ~ y_1 {=} 0.8)$ %(non-det) 
		$\checkmark$}
\end{prooftree}
\end{minipage}

\noindent
where we write $C_1$ for the conjunct $ReLU(\frac{1}{125} x_1) - y_1 = 0$ and $C_2$ for the conjunct $\frac{1}{1000}\agg(x_1) - y_2 = 0$.


% \begin{prooftree}
% \AxiomC{$(\epsilon ~ \phi_N)$}
% \rulelabel{def}
% \UnaryInfC{$(\epsilon ~ \frac{1}{125} ReLU(x_1) + (-1) y_1 = 0 \land \frac{1}{1000}\agg(x_1) + (-1)y_2 = 0)$}
% \rulelabel{\land}
% \UnaryInfC{$(\epsilon ~ \frac{1}{125} ReLU(x_1) + (-1) y_1 = 0), (\epsilon ~ \frac{1}{1000}\agg(x_1) + (-1)y_2 = 0)$}
% \end{prooftree}



We handle the conjunct $C_1$ in $\phi_N$.
\begin{prooftree}\small
\AxiomC{$(\epsilon ~ ReLU(\frac{1}{125} x_1) + (-1) y_1 = 0)$}
\rulelabel{+}
\UnaryInfC{$
(\epsilon ~ ReLU(\frac{1}{125} x_1) = 0.8),
(\epsilon ~ (-1) y_1 = -0.8)
$ %(non-det)
}
\end{prooftree}


\begin{minipage}{0.25\textwidth}
	\begin{prooftree}\small
\AxiomC{$
(\epsilon ~ ReLU(\frac{1}{125} x_1) = 0.8)
$
}
\rulelabel{\alpha}
\UnaryInfC{$(\epsilon ~ \frac{1}{125} x_1 = 0.8)$} %(non-det)
% \end{prooftree}
% \begin{prooftree}\small
% \AxiomC{$(\epsilon ~ ReLU(x_1) = 100)$}
\rulelabel{\times}
\UnaryInfC{$(\epsilon ~ x_1 = 100)$ $\checkmark$} %(non-det)
\end{prooftree}
\end{minipage}
% Note that the $(\alpha)$ rule is non-deterministic in general, but only one value applies here.
%
\begin{minipage}{0.15\textwidth}
\begin{prooftree}\small
\AxiomC{$
(\epsilon ~ (-1) y_1 = -0.8)
$
}
\rulelabel{\times}
\UnaryInfC{$(\epsilon ~ y_1 = 0.8)$ $\checkmark$} %(non-det)
\end{prooftree}
\end{minipage}




Finally, we handle the conjunct $C_2$ in $\phi_N$.
\begin{prooftree}\small
\AxiomC{$(\epsilon ~ \frac{1}{1000}\agg(x_1) + (-1)y_2 = 0)$}
\rulelabel{+}
\UnaryInfC{$
(\epsilon ~ \frac{1}{1000}\agg(x_1) = 1),
(\epsilon ~ (-1)y_2 = -1)
$ %(non-det)
}
\end{prooftree}

\begin{minipage}{0.25\textwidth}
\begin{prooftree}\small
\AxiomC{$
(\epsilon ~ \frac{1}{1000}\agg(x_1) = 1)
$
}
\rulelabel{\times}
\UnaryInfC{$(\epsilon ~ \agg(x_1) = 1000)$ %(non-det)
}
% \end{prooftree}
% \begin{prooftree}\small
% \AxiomC{$
% (\epsilon ~ \frac{1}{1000}\agg(x_1) = 1)
% $
% }
\rulelabel{arity}
\UnaryInfC{$(\epsilon ~ arity = 4)$ $\checkmark$}
\end{prooftree}
\end{minipage}
\begin{minipage}{0.1\textwidth}
\begin{prooftree}\small
	\AxiomC{$
		(\epsilon ~ (-1)y_2 = -1)
		$
	}
	\rulelabel{\times}
	\UnaryInfC{$(\epsilon ~ y_2 = 1)$ %(non-det)
		$\checkmark$}
\end{prooftree}
\end{minipage}



\begin{prooftree}\small
\AxiomC{$(\epsilon ~ \agg(x_1) = 1000)$}
\AxiomC{$(\epsilon ~ arity = 4)$}
\rulelabel{\agg}
\BinaryInfC{$(1 ~ x_1 = 250), \dotsc, (4 ~ x_1 = 250)$ %(non-det) 
$\checkmark$}
\end{prooftree}
% That is, five (which is the maximum arity $\arity$) worlds $w_1, \dotsc w_5$ where $x_1 = 200$ are witnesses of the formula $\agg(x_1) = 1000$.



\paragraph{Obtaining a model.}
We obtain a counterexample $(G,v)$ for the formula $\phi_{I'}$, with the node $v$ labeled $(x_1) = (100)$, which has $4$ successors, all labeled $(x_1) = (250)$. 
%
So, by \Cref{th:reduction}, $I'$ is not valid.
%
Indeed, $G,v$ satisfies $L_{in}$. But $N(G,v)$ is $(y_1, y_2) = (0.8, 1)$ which does not satisfy $L'_{out}$.

This illustrates how the proof system serves to verify the properties of quantized GNN, but also to exhibit counterexamples when they exist.