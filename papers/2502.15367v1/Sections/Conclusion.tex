\section{Conclusion}
% This study explored human emotional response strategies in interactions with emotion-aware VAs using a role-swapping approach. Our findings indicate that participants predominantly adopt neutral or positive emotional responses, particularly in negative emotional scenarios, suggesting a natural tendency for emotional regulation in human-AI interactions. Speech feature analysis identified root mean square (RMS), zero-crossing rate (ZCR), and jitter as key indicators of emotional expression, while sentiment polarity analysis revealed that happy responses exhibit higher polarity and lexical diversity, whereas angry and sad responses show lower polarity and restricted word choices. These insights highlight the potential for developing adaptive, emotion-aware VAs that respond contextually and empathetically, improving user trust and engagement. Future research should explore cross-cultural emotional models, deep learning-driven response adaptation, and multimodal emotion recognition to further enhance emotionally intelligent VAs.

This study explored human emotional response strategies in interactions with emotion-aware VAs using a role-swapping approach. Participants predominantly adopted neutral or positive emotional responses, particularly in negative scenarios, demonstrating a natural tendency toward emotional regulation. Speech feature analysis identified RMS, ZCR, and jitter as key indicators of emotional expression, while sentiment polarity and lexical diversity revealed distinct patterns across emotional states. These insights provide a foundation for developing adaptive, emotion-aware VAs that respond contextually and empathetically, improving user trust and engagement. Future research should focus on cross-cultural emotional models, deep learning-driven response adaptation, and multimodal emotion recognition to further enhance the emotional intelligence of VAs. By addressing these challenges, we can create VAs that not only enhance technical capabilities but also contribute positively to users' emotional well-being.