% This is samplepaper.tex, a sample chapter demonstrating the
% LLNCS macro package for Springer Computer Science proceedings;
% Version 2.21 of 2022/01/12
%
\documentclass[runningheads]{llncs}
%
% \usepackage[authoryear]{natbib}

\usepackage[numbers]{natbib} % Use numerical citations
% \bibliographystyle{splncs04} % Use the default LNCS bibliography style
% \bibliographystyle{plainnat}
\usepackage[T1]{fontenc}
\usepackage{tabularx}
\usepackage{subfiles}
\usepackage{hyperref}
% \usepackage{todonotes}
\usepackage{balance}
\usepackage{booktabs}
\usepackage{natbib}
\usepackage{caption}
\usepackage{graphicx}
\usepackage{subcaption} 
\usepackage{adjustbox} % Helps with centering the last row

% \usepackage{natbib}
% \bibliographystyle{plainnat}  % Use a style compatible with author-year citations\textbf{}
% T1 fonts will be used to generate the final print and online PDFs,
% so please use T1 fonts in your manuscript whenever possible.
% Other font encondings may result in incorrect characters.
%
% Used for displaying a sample figure. If possible, figure files should
% be included in EPS format.
%
% If you use the hyperref package, please uncomment the following two lines
% to display URLs in blue roman font according to Springer's eBook style:
%\usepackage{color}
%\renewcommand\UrlFont{\color{blue}\rmfamily}
%\urlstyle{rm}
%
\begin{document}
%
% \title{Designing Emotion-Aware Voice Assistants: Strategies for Handling User Emotions}
\title{Advancing User-Voice Interaction: Exploring Emotion-Aware Voice Assistants Through a Role-Swapping Approach}
%
%\titlerunning{Abbreviated paper title}
% If the paper title is too long for the running head, you can set
% an abbreviated paper title here
%
\author{
Yong Ma\inst{1}\orcidID{0000-0002-8398-4118} \and
Yuchong Zhang\inst{2}\orcidID{0000-0003-1804-6296} \and
Di Fu\inst{3}\orcidID{0000-0002-5385-2982} \and
Stephanie Zubicueta Portales\inst{4}\orcidID{0009-0004-2459-1949} \and
Danica Kragic\inst{2}\orcidID{0000-0003-2965-2953} \and
Morten Fjeld\inst{1,5}\orcidID{0000-0002-9562-5147}
}

%
\authorrunning{Y. Ma et al.}
% First names are abbreviated in the running head.
% If there are more than two authors, 'et al.' is used.
%
\institute{University of Bergen, Bergen, Norway 
% \email{@springer.com}
\and KTH Royal Institute of Technology, Stockholm, Sweden \and University of Surrey, Guildford, UK \and Chalmers University of Technology, Gothenburg, Sweden \and Norwegian University of Science and Technology,Trondheim, Norway
\\}
% \url{http://www.springer.com/gp/computer-science/lncs} \and
% ABC Institute, Rupert-Karls-University Heidelberg, Heidelberg, Germany\\
% \email{\{abc,lncs\}@uni-heidelberg.de}}
% \author{Yong Ma}
% \affiliation{%
%   \institution{University of Bergen}
%   \city{Bergen}
%   \country{Norway}}
%   \email{yong.ma@uib.no}

% \author{Yuchong Zhang}
% % \orcid{0000-0003-1804-6296}
% \email{yuchongz@kth.se}
% \affiliation{%
%   \institution{KTH Royal Institute of Technology}
%   \city{Stockholm}
%   \country{Sweden}
% }



% \author{Di Fu}
% % \orcid{0000-0003-1804-6296}
% \email{d.fu@surrey.ac.uk}
% \affiliation{%
%   \institution{Unviersity of Surrey}
%   \city{Guildford}
%   \country{United Kingdom}
% }

% \author{Stephanie Zubicueta Portales}
% \affiliation{%
%   \institution{Norwegian University of Science and Technology}
%   \city{Trondheim}
%   \country{Norway}}
%   \email{stephanie.portales@student.uib.no}

% \author{Morten Fjeld}
% % \orcid{0000-0002-9562-5147}
% \affiliation{%
% \institution{University of Bergen}
%   \city{Bergen}
%   \country{Norway}}
% \email{morten.fjeld@uib.no}
% \additionalaffiliation{%
% \institution{Chalmers University of Technology}
%   \city{Gothenburg}
%   \country{Sweden}}
% \email{fjeld@chalmers.se}

% \author{Danica Kragic}
% \affiliation{%
% \institution{KTH Royal Institute of Technology}
%   \city{Stockholm}
%   \country{Sweden}}
% \email{danik@kth.se}
%
\maketitle              % typeset the header of the contribution
%
\begin{abstract}
% The abstract should briefly summarize the contents of the paper in
% 150--250 words.
% As voice assistants (VAs) become increasingly integrated into daily life, the need for emotion-aware systems capable of recognizing and appropriately responding to user emotions has grown. While significant progress has been made in speech emotion recognition (SER) and sentiment analysis, effectively responding to user emotions—particularly negative ones—remains a challenge. This study explores human emotional response strategies in VAs interactions using a role-swapping approach, where participants regulate AI emotions rather than receiving pre-programmed responses. Through speech feature analysis and natural language processing (NLP) techniques, we examined acoustic and linguistic patterns across different emotional scenarios. Our results indicate that participants favor neutral or positive emotional responses when engaging with negative emotional cues, demonstrating a natural inclination toward emotional regulation and de-escalation. Additionally, we found that root mean square (RMS), zero-crossing rate (ZCR), and jitter serve as key acoustic indicators of emotional state, while sentiment polarity and lexical diversity (TTR) distinguish between positive and negative responses. These findings provide valuable insights for developing adaptive, context-aware VAs that deliver empathetic, culturally sensitive, and user-aligned responses. 
As voice assistants (VAs) become increasingly integrated into daily life, the need for emotion-aware systems that can recognize and respond appropriately to user emotions has grown. While significant progress has been made in speech emotion recognition (SER) and sentiment analysis, effectively addressing user emotions—particularly negative ones—remains a challenge. This study explores human emotional response strategies in VA interactions using a role-swapping approach, where participants regulate AI emotions rather than receiving pre-programmed responses. Through speech feature analysis and natural language processing (NLP), we examined acoustic and linguistic patterns across various emotional scenarios. Results show that participants favor neutral or positive emotional responses when engaging with negative emotional cues, highlighting a natural tendency toward emotional regulation and de-escalation. Key acoustic indicators such as root mean square (RMS), zero-crossing rate (ZCR), and jitter were identified as sensitive to emotional states, while sentiment polarity and lexical diversity (TTR) distinguished between positive and negative responses. These findings provide valuable insights for developing adaptive, context-aware VAs capable of delivering empathetic, culturally sensitive, and user-aligned responses. By understanding how humans naturally regulate emotions in AI interactions, this research contributes to the design of more intuitive and emotionally intelligent voice assistants, enhancing user trust and engagement in human-AI interactions.


\keywords{Emotion-Aware Voice Assistants \and Role-Swapping Approach \and Speech and Linguistic Analysis \and Speech Emotion Recognition (SER)}
\end{abstract}
%
%
%

\section{Introduction}

Large language models (LLMs) have achieved remarkable success in automated math problem solving, particularly through code-generation capabilities integrated with proof assistants~\citep{lean,isabelle,POT,autoformalization,MATH}. Although LLMs excel at generating solution steps and correct answers in algebra and calculus~\citep{math_solving}, their unimodal nature limits performance in plane geometry, where solution depends on both diagram and text~\citep{math_solving}. 

Specialized vision-language models (VLMs) have accordingly been developed for plane geometry problem solving (PGPS)~\citep{geoqa,unigeo,intergps,pgps,GOLD,LANS,geox}. Yet, it remains unclear whether these models genuinely leverage diagrams or rely almost exclusively on textual features. This ambiguity arises because existing PGPS datasets typically embed sufficient geometric details within problem statements, potentially making the vision encoder unnecessary~\citep{GOLD}. \cref{fig:pgps_examples} illustrates example questions from GeoQA and PGPS9K, where solutions can be derived without referencing the diagrams.

\begin{figure}
    \centering
    \begin{subfigure}[t]{.49\linewidth}
        \centering
        \includegraphics[width=\linewidth]{latex/figures/images/geoqa_example.pdf}
        \caption{GeoQA}
        \label{fig:geoqa_example}
    \end{subfigure}
    \begin{subfigure}[t]{.48\linewidth}
        \centering
        \includegraphics[width=\linewidth]{latex/figures/images/pgps_example.pdf}
        \caption{PGPS9K}
        \label{fig:pgps9k_example}
    \end{subfigure}
    \caption{
    Examples of diagram-caption pairs and their solution steps written in formal languages from GeoQA and PGPS9k datasets. In the problem description, the visual geometric premises and numerical variables are highlighted in green and red, respectively. A significant difference in the style of the diagram and formal language can be observable. %, along with the differences in formal languages supported by the corresponding datasets.
    \label{fig:pgps_examples}
    }
\end{figure}



We propose a new benchmark created via a synthetic data engine, which systematically evaluates the ability of VLM vision encoders to recognize geometric premises. Our empirical findings reveal that previously suggested self-supervised learning (SSL) approaches, e.g., vector quantized variataional auto-encoder (VQ-VAE)~\citep{unimath} and masked auto-encoder (MAE)~\citep{scagps,geox}, and widely adopted encoders, e.g., OpenCLIP~\citep{clip} and DinoV2~\citep{dinov2}, struggle to detect geometric features such as perpendicularity and degrees. 

To this end, we propose \geoclip{}, a model pre-trained on a large corpus of synthetic diagram–caption pairs. By varying diagram styles (e.g., color, font size, resolution, line width), \geoclip{} learns robust geometric representations and outperforms prior SSL-based methods on our benchmark. Building on \geoclip{}, we introduce a few-shot domain adaptation technique that efficiently transfers the recognition ability to real-world diagrams. We further combine this domain-adapted GeoCLIP with an LLM, forming a domain-agnostic VLM for solving PGPS tasks in MathVerse~\citep{mathverse}. 
%To accommodate diverse diagram styles and solution formats, we unify the solution program languages across multiple PGPS datasets, ensuring comprehensive evaluation. 

In our experiments on MathVerse~\citep{mathverse}, which encompasses diverse plane geometry tasks and diagram styles, our VLM with a domain-adapted \geoclip{} consistently outperforms both task-specific PGPS models and generalist VLMs. 
% In particular, it achieves higher accuracy on tasks requiring geometric-feature recognition, even when critical numerical measurements are moved from text to diagrams. 
Ablation studies confirm the effectiveness of our domain adaptation strategy, showing improvements in optical character recognition (OCR)-based tasks and robust diagram embeddings across different styles. 
% By unifying the solution program languages of existing datasets and incorporating OCR capability, we enable a single VLM, named \geovlm{}, to handle a broad class of plane geometry problems.

% Contributions
We summarize the contributions as follows:
We propose a novel benchmark for systematically assessing how well vision encoders recognize geometric premises in plane geometry diagrams~(\cref{sec:visual_feature}); We introduce \geoclip{}, a vision encoder capable of accurately detecting visual geometric premises~(\cref{sec:geoclip}), and a few-shot domain adaptation technique that efficiently transfers this capability across different diagram styles (\cref{sec:domain_adaptation});
We show that our VLM, incorporating domain-adapted GeoCLIP, surpasses existing specialized PGPS VLMs and generalist VLMs on the MathVerse benchmark~(\cref{sec:experiments}) and effectively interprets diverse diagram styles~(\cref{sec:abl}).

\iffalse
\begin{itemize}
    \item We propose a novel benchmark for systematically assessing how well vision encoders recognize geometric premises, e.g., perpendicularity and angle measures, in plane geometry diagrams.
	\item We introduce \geoclip{}, a vision encoder capable of accurately detecting visual geometric premises, and a few-shot domain adaptation technique that efficiently transfers this capability across different diagram styles.
	\item We show that our final VLM, incorporating GeoCLIP-DA, effectively interprets diverse diagram styles and achieves state-of-the-art performance on the MathVerse benchmark, surpassing existing specialized PGPS models and generalist VLM models.
\end{itemize}
\fi

\iffalse

Large language models (LLMs) have made significant strides in automated math word problem solving. In particular, their code-generation capabilities combined with proof assistants~\citep{lean,isabelle} help minimize computational errors~\citep{POT}, improve solution precision~\citep{autoformalization}, and offer rigorous feedback and evaluation~\citep{MATH}. Although LLMs excel in generating solution steps and correct answers for algebra and calculus~\citep{math_solving}, their uni-modal nature limits performance in domains like plane geometry, where both diagrams and text are vital.

Plane geometry problem solving (PGPS) tasks typically include diagrams and textual descriptions, requiring solvers to interpret premises from both sources. To facilitate automated solutions for these problems, several studies have introduced formal languages tailored for plane geometry to represent solution steps as a program with training datasets composed of diagrams, textual descriptions, and solution programs~\citep{geoqa,unigeo,intergps,pgps}. Building on these datasets, a number of PGPS specialized vision-language models (VLMs) have been developed so far~\citep{GOLD, LANS, geox}.

Most existing VLMs, however, fail to use diagrams when solving geometry problems. Well-known PGPS datasets such as GeoQA~\citep{geoqa}, UniGeo~\citep{unigeo}, and PGPS9K~\citep{pgps}, can be solved without accessing diagrams, as their problem descriptions often contain all geometric information. \cref{fig:pgps_examples} shows an example from GeoQA and PGPS9K datasets, where one can deduce the solution steps without knowing the diagrams. 
As a result, models trained on these datasets rely almost exclusively on textual information, leaving the vision encoder under-utilized~\citep{GOLD}. 
Consequently, the VLMs trained on these datasets cannot solve the plane geometry problem when necessary geometric properties or relations are excluded from the problem statement.

Some studies seek to enhance the recognition of geometric premises from a diagram by directly predicting the premises from the diagram~\citep{GOLD, intergps} or as an auxiliary task for vision encoders~\citep{geoqa,geoqa-plus}. However, these approaches remain highly domain-specific because the labels for training are difficult to obtain, thus limiting generalization across different domains. While self-supervised learning (SSL) methods that depend exclusively on geometric diagrams, e.g., vector quantized variational auto-encoder (VQ-VAE)~\citep{unimath} and masked auto-encoder (MAE)~\citep{scagps,geox}, have also been explored, the effectiveness of the SSL approaches on recognizing geometric features has not been thoroughly investigated.

We introduce a benchmark constructed with a synthetic data engine to evaluate the effectiveness of SSL approaches in recognizing geometric premises from diagrams. Our empirical results with the proposed benchmark show that the vision encoders trained with SSL methods fail to capture visual \geofeat{}s such as perpendicularity between two lines and angle measure.
Furthermore, we find that the pre-trained vision encoders often used in general-purpose VLMs, e.g., OpenCLIP~\citep{clip} and DinoV2~\citep{dinov2}, fail to recognize geometric premises from diagrams.

To improve the vision encoder for PGPS, we propose \geoclip{}, a model trained with a massive amount of diagram-caption pairs.
Since the amount of diagram-caption pairs in existing benchmarks is often limited, we develop a plane diagram generator that can randomly sample plane geometry problems with the help of existing proof assistant~\citep{alphageometry}.
To make \geoclip{} robust against different styles, we vary the visual properties of diagrams, such as color, font size, resolution, and line width.
We show that \geoclip{} performs better than the other SSL approaches and commonly used vision encoders on the newly proposed benchmark.

Another major challenge in PGPS is developing a domain-agnostic VLM capable of handling multiple PGPS benchmarks. As shown in \cref{fig:pgps_examples}, the main difficulties arise from variations in diagram styles. 
To address the issue, we propose a few-shot domain adaptation technique for \geoclip{} which transfers its visual \geofeat{} perception from the synthetic diagrams to the real-world diagrams efficiently. 

We study the efficacy of the domain adapted \geoclip{} on PGPS when equipped with the language model. To be specific, we compare the VLM with the previous PGPS models on MathVerse~\citep{mathverse}, which is designed to evaluate both the PGPS and visual \geofeat{} perception performance on various domains.
While previous PGPS models are inapplicable to certain types of MathVerse problems, we modify the prediction target and unify the solution program languages of the existing PGPS training data to make our VLM applicable to all types of MathVerse problems.
Results on MathVerse demonstrate that our VLM more effectively integrates diagrammatic information and remains robust under conditions of various diagram styles.

\begin{itemize}
    \item We propose a benchmark to measure the visual \geofeat{} recognition performance of different vision encoders.
    % \item \sh{We introduce geometric CLIP (\geoclip{} and train the VLM equipped with \geoclip{} to predict both solution steps and the numerical measurements of the problem.}
    \item We introduce \geoclip{}, a vision encoder which can accurately recognize visual \geofeat{}s and a few-shot domain adaptation technique which can transfer such ability to different domains efficiently. 
    % \item \sh{We develop our final PGPS model, \geovlm{}, by adapting \geoclip{} to different domains and training with unified languages of solution program data.}
    % We develop a domain-agnostic VLM, namely \geovlm{}, by applying a simple yet effective domain adaptation method to \geoclip{} and training on the refined training data.
    \item We demonstrate our VLM equipped with GeoCLIP-DA effectively interprets diverse diagram styles, achieving superior performance on MathVerse compared to the existing PGPS models.
\end{itemize}

\fi 

\section{Related Works}
\label{sec:rw}

%-------------------------------------------------------------------------
\noindent \textbf{Vision-Language Model.}
In recent years, vision-language models, as a novel tool capable of processing both visual and linguistic modalities, have garnered widespread attention. These models, such as CLIP~\cite{clip}, ALIGN~\cite{ALIGN}, BLIP~\cite{BLIP}, FILIP~\cite{filip}, etc., leverage self-supervised training on image-text pairs to establish connections between vision and text, enabling the models to comprehend image semantics and their corresponding textual descriptions. This powerful understanding allows vision-language models (e.g., CLIP) to exhibit remarkable generalization capabilities across various downstream tasks~\cite{downsteam1,downsteam2,downsteam3,h2b}. To further enhance the transferability of vision-language models to downstream tasks, prompt tuning and adapter methods have been applied. However, methods based on prompt tuning (such as CoOp~\cite{coop}, CoCoOp~\cite{cocoop}, Maple~\cite{maple}) and adapter-based methods (such as Tip-Adapter~\cite{tip}, CLIP-Adapter~\cite{clip_adapter}) often require large amounts of training data when transferring to downstream tasks, which conflicts with the need for rapid adaptation in real-world applications. Therefore, this paper focuses on test-time adaptation~\cite{tpt}, a method that enables transfer to downstream tasks without relying on training data.

%-------------------------------------------------------------------------
\noindent \textbf{Test-Time Adaptation.}
Test-time adaptation~(TTA) refers to the process by which a model quickly adapts to test data that exhibits distributional shifts~\cite{tta1,memo,ptta,domainadaptor,dota}. Specifically, it requires the model to handle these shifts in downstream tasks without access to training data. TPT~\cite{tpt} optimizes adaptive text prompts using the principle of entropy minimization, ensuring that the model produces consistent predictions for different augmentations of test images generated by AugMix~\cite{augmix}. DiffTPT~\cite{difftpt} builds on TPT by introducing the Stable Diffusion Model~\cite{stable} to create more diverse augmentations and filters these views based on their cosine similarity to the original image. However, both TPT and DiffTPT still rely on backpropagation to optimize text prompts, which limits their ability to meet the need for fast adaptation during test-time. TDA~\cite{tda}, on the other hand, introduces a cache model like Tip-Adapter~\cite{tip} that stores representative test samples. By comparing incoming test samples with those in the cache, TDA refines the model’s predictions without the need for backpropagation, allowing for test-time enhancement. Although TDA has made significant improvements in the TTA task, it still does not fundamentally address the impact of test data distribution shifts on the model and remains within the scope of CLIP's original feature space. We believe that in TTA tasks, instead of making decisions in the original space, it would be more effective to map the features to a different spherical space to achieve a better decision boundary.

%-------------------------------------------------------------------------
\noindent \textbf{Statistical Learning.}
Statistical learning techniques play an important role in dimensionality reduction and feature extraction. Support Vector Machines~(SVM)~\cite{svm} are primarily used for classification tasks but have been adapted for space mapping through their ability to create hyperplanes that separate data in high-dimensional spaces. The kernel trick enables SVM to operate in transformed feature spaces, effectively mapping non-linearly separable data. PCA~\cite{pca} is a linear transformation method that maps high-dimensional data to a new lower-dimensional space through a linear transformation, while preserving as much important information from the original data as possible.
\section{Experiment Setup}
% As it described in Figure~\ref{fig:webpage}, we conducted an online user study on in-car virtual natural environments (VRE), developed as part of a separate project [anonymity], to explore the potential of speech analysis for UX measurement. By analyzing post-VRE user interviews, we evaluated the restorative effects and demonstrated the feasibility of using speech analysis for UX evaluation.
As shown in Figure~\ref{fig:webpage}, we conducted an online user study to explore users' strategies for responding to different emotional contexts. By analyzing participants' voices and the content of their interactions with our one-dialog simulated VAs, we evaluated emotional responding differences, speech features, and language features using speech analysis and natural language processing (NLP) techniques. This approach allowed us to gain insights into how users naturally adapt their responses to emotional cues, providing valuable data for improving emotion-aware systems.

\subsection{Emotional Scenarios}
% In our study, based on five basic emotions~\cite{ortony1990,sortony2022all,ma2022emotion},we designed five different emotional scenarios, including neutral, happy, sad, angry, and fear~\cite{ma2022should}. When these emotion states were confirmed, we then have a brainstorming about which emotional scenarios or context can be suitable for users. In the other words, these emotional contexts may be come out in our daily life and can be identified by general users. Fianlly, we confirmed the follwing five different emotional scenarios:
In our study, we designed five distinct emotional scenarios based on the five basic emotions — neutral, happy, sad, angry, and fear~\cite{ortony1990s,ortony2022all,ma2022emotion}. These scenarios were crafted to reflect situations that users might encounter in their daily lives, ensuring they were relatable and easily identifiable by general users. After confirming the emotional states, we conducted brainstorming sessions to determine which scenarios would best represent these emotions in a way that felt authentic and meaningful. The final five emotional scenarios are as follows:

\begin{itemize}
    \item \textbf{Neutral:} I put on my shoes before leaving the house.
    \item \textbf{Happy:} I am visiting my favorite country or city.
    \item \textbf{Sad:} I see children suffering from disease, sickness, or war.
    \item \textbf{Angry:} I get betrayed by a close friend or relative.
    \item \textbf{Fear:} I am walking in the dark in the woods when I stumble upon a dead body. The blood seems fresh, and I hear a branch breaking from behind.
\end{itemize}

Based on these emotional scenarios, we recorded voice clips for each scenario and integrated them into our designed website. Users visiting the website can listen to these emotional scenario recordings, allowing them to immerse themselves in the context and respond naturally. This interactive setup enables participants to engage with the emotional content and provide authentic responses, which are then recorded and analyzed for further study.

\subsection{Apparatus, Participants and Experimental Procedure}
\subsubsection{Apparatus and Participants}
In this user study, we primarily utilized OpenVokaturi\footnote{\url{https://developers.vokaturi.com/getting-started/overview}} to analyze the emotional states of speech. Additionally, we employed two key speech analysis packages: Librosa\footnote{\url{https://librosa.org/doc/latest/feature.html}} and OpenSMILE\footnote{\url{https://audeering.github.io/opensmile-python/}}, to examine users' speech behaviors in response to emotional scenarios. For analyzing the content of users' responses, we relied on four main NLP packages: WordCloud\footnote{\url{https://amueller.github.io/word_cloud/}}, NLTK\footnote{\url{https://www.nltk.org/}}, TextStatc, and SpaCy\footnote{\url{https://spacy.io/}}. Furthermore, OpenAI Whisper\footnote{\url{https://openai.com/index/whisper/}} was used as a critical tool for converting speech into text, enabling detailed analysis of both linguistic and emotional features. Participants' voices were recorded using the built-in microphones of their own computers and uploaded to our web server when they conducted our user study on our website. All recorded voice signals were captured at a sample rate of 48 kHz, ensuring high-quality audio for subsequent analysis.
Additionally, we recruited 60 participants through Prolific\footnote{\url{https://www.prolific.com/}}, comprising 30 males and 30 females. To account for potential gender-based differences in voice perception, we conducted two separate user studies: one using a male emotional voice and the other using a female emotional voice for the five emotional scenarios. The male emotional voice had a mean age of 32.29 years (SD = 9.71), while the female emotional voice had a mean age of 27.17 years (SD = 5.02). This approach ensured a balanced and comprehensive evaluation of emotional responses across different gender contexts.
After completing the user study, each participant was compensated with £4.50 through the Prolific website. Participants were assured of complete anonymity, as clearly stated in the website's privacy policy. This policy also detailed the types of data collected during the study, which included voice recordings and responses to demographic questions, such as age and gender. All measures were taken to ensure the confidentiality and privacy of the participants' information.

\subsubsection{Experimental Procedure}
% As aforementioned, we recruited the participants through the Prolific website and participants can visit our designed website as it shown in Figure~\ref{fig:webpage} through our link. The landing page presents the study in general and guide instructions and participants can click the smiley and it becomes yellow. Meanwhile, a emotional scenarios voice will be played. Participants can respond anything to this emotional voice as they like. The microphone will be recorded their voice when they click the "Start Recording". They will also have 20 seconds to say anything to talk with the smiley or try their best to comfort smiley if they meet negative emotions, such as angry, fear and sad. In this study, they will conduct the five different emotional scenarios and male or female voice will be randomly selected during the study. In order to keep the gender and participants balance, 15 male and female parcipants will try the male five different emotional scenarios and another 30 parcipants will conduct the female different emotional scenarios. After this five different emotional scenarios, the main user study will be completed and then they will fill relevant questionnaire after this one. 
As mentioned earlier, we recruited participants through the Prolific website, and they accessed our designed website (shown in Figure\ref{fig:webpage}) via a provided link. The landing page provided an overview of the study, along with detailed instructions. Participants could click on the smiley emoji, which would turn yellow and play an emotional scenario voice. They were then free to respond in any way they liked. When participants clicked the "\textit{Start Recording}" button, their microphone would begin recording their voice. They were given 20 seconds to speak, during which they could engage in conversation with the smiley or attempt to comfort it, especially in cases of negative emotions such as anger, fear, or sadness.
In this study, participants engaged with five different emotional scenarios, with either a male or female emotional voice randomly selected for each scenario. To ensure gender balance, 15 male and 15 female participants interacted with the male emotional voice scenarios, while the remaining 30 participants interacted with the female emotional voice scenarios. After completing all five emotional scenarios, participants finished the main user study and proceeded to fill out a relevant questionnaire.



\subsection{Speech Signals Analysis}
% The collected speech data from participants were downloaded from the website server and then classified based on different emotional scenarios based on audio file naming. Then we extracted the speech features using the aforementioned  speech processing python package such as librosa and opensmile. Moreover, to analyze speech contents, we converted the audio data into text data using the OpenAI-whisper and then based on NLP technologies, we extracted the text features, such as lexical features, semantic features and syntactic features, etc. Moreover, we used Openvokatuti to obtain parcipants' speech emotion states, namely, which emotional responding they used to communicating with our VAs.
The speech data collected from participants were downloaded from the website server and systematically classified according to different emotional scenarios, based on the naming conventions of the audio files. To extract relevant speech features, we utilized advanced speech processing Python packages, such as Librosa and OpenSMILE~\cite{eyben2010opensmile}. Additionally, to analyze the content of the speech, we transcribed the audio data into text using OpenAI Whisper model. Leveraging natural language processing (NLP) techniques, we then extracted a variety of text features, including lexical, semantic, and syntactic features. Furthermore, we employed Openvokatuti to identify and analyze participants' emotional states during their interactions, specifically determining which emotional responses they exhibited while communicating with our VAs. This comprehensive approach allowed us to gain deeper insights into both the acoustic and linguistic aspects of the participants' interactions.


% The speech data collected from participants were downloaded from the website server and systematically classified into different emotional scenarios based on the naming conventions of the audio files.
% To extract relevant speech features, we utilized advanced speech processing Python libraries, including Librosa and OpenSMILE~\cite{eyben2010opensmile}. For speech content analysis, the audio recordings were transcribed into text format using OpenAI’s Whisper model. Leveraging Natural Language Processing (NLP) techniques, we extracted a range of textual features, including lexical, semantic, and syntactic attributes, to gain deeper insights into linguistic variations across emotional contexts.
% Additionally, we employed OpenVokaturi to analyze participants’ emotional states, identifying the specific emotional responses exhibited during their interactions with our voice assistants (VAs). This comprehensive multimodal approach enabled us to capture both acoustic and linguistic dimensions of participant interactions, facilitating a more nuanced understanding of human emotional expression in AI-mediated communication.


\section{Results}
\label{sec:Results}

In this section, we present various analysis results that demonstrate the adoption of code obfuscation in Google Play.

\subsection{Overall Obfuscation Trends} 
\label{sec:obstrend}

\subsubsection{Presence of obfuscation} Out of the 548,967 Google Play Store APKs analyzed, we identified 308,782 obfuscated apps, representing approximately 56.25\% of the total. In Figure~\ref{fig:obfuscated_percentage}, we show the year-wise percentage of obfuscated apps for 2016-2023. There is an overall obfuscation increase of 13\% between 2016 and 2023, and as can be seen, the percentage of obfuscated apps has been increasing in the last few years, barring 2019 and 2020. As explained in Section~\ref{subsec:dataset}, 2019 and 2020 contain apps that are more likely to be abandoned by developers, and as such, they may not use advanced development practices.

\begin{figure}[h!]
\centering
    \includegraphics[width=\linewidth]{Figures/Only_obfuscation_trendV2.pdf}
    \caption{Percentage of obfuscated apps by year} \vspace{-4mm}
    \label{fig:obfuscated_percentage}
\end{figure}


From 2016 to 2018, the obfuscation levels were relatively stable at around 50-55\%, while from 2021 to 2023, there was a marked rise, reaching approximately 66\% in 2023. This indicates a growing focus on app protection measures among developers, likely driven by heightened security and IP concerns and the availability of advanced obfuscation tools.


\subsubsection{Obfuscation tools} Among the obfuscated APKs, our tool detector identified that 40.92\% of the apps use Proguard, 36.64\% use Allatori, 1.01\% use DashO, and 21.43\% use other (i.e., unknown) tools. We show the yearly trends in Figure~\ref{fig:ofbuscated_tool}. Note that we omit results in 2019 and 2020 ({\bf cf.} Section~\ref{subsec:dataset}).

ProGuard and Allatori are the most consistently used obfuscation tools, with ProGuard showing a slight overall increase in popularity and Allatori demonstrating variability. This inclination could be attributed to ProGuard being the default obfuscator integrated into Android Studio, a widely used development environment for Android applications. Notably, ProGuard usage increased by 13\% from 2018 to 2021, likely due to the introduction of R8 in April 2019~\cite{release_note_android}, which further simplified ProGuard integration with Android apps.

\begin{figure}[h]
\centering
    \includegraphics[width=\linewidth]{Figures/Initial_Tool_Trend_2019_dropV2.pdf} 
    \caption{Yearly obfuscation tool usage}
    \label{fig:ofbuscated_tool}
\end{figure}


DashO consistently remains low in usage, likely due to its high cost. The use of other obfuscation tools decreased until 2018 but has shown a resurgence from 2021 to 2023. This suggests that developers might be using other or custom tools, or our detector might be predicting some apps obfuscated with Proguard or Allatori as `other.' To investigate, we manually checked a sample of apps from the `other' category and confirmed they are indeed obfuscated. However, we could not determine which obfuscation tools the developers used. We discuss this potential limitation further in Section~\ref{sec:limitations}.


\subsubsection{Obfuscation techniques} We show the year-wise breakdown of obfuscation technique usage in Figure~\ref{fig:obfuscated_tech}. Among the various obfuscation techniques, Identifier Renaming emerged as the most prevalent, with 99.62\% of obfuscated apps using it alone or in combination with other methods (Categories of Only IR, IR and CF, IR and SE, or All three). Furthermore, 81.04\% of obfuscated apps used Control Flow Modification, and 62.76\% used String Encryption. The pervasive use of Identifier Renaming (IR) can be attributed to the fact that all obfuscation tools support it ({\bf cf.} Table~\ref{tab:ob_tool_cap}). Similarly, lower adoption of Control Flow Modification and String Encryption can be attributed to Proguard not supporting it. 

\begin{figure}[h]
\centering
    \includegraphics[width=\linewidth]{Figures/Initial_Tech_Trend_2019_dropV2.pdf} 
    \caption{Yearly obfuscation technique usage}
    \label{fig:obfuscated_tech}
\end{figure}



Next, we investigate the adoption of obfuscation on Google Play Store from various perspectives. Same as earlier, due to the smaller dataset size and possible bias ({\bf cf.} Section~\ref{subsec:dataset}), we exclude the APKs from 2019 and 2020 from this analyses.


\subsection{App Genre}
\label{sec:app_genre}

First, we investigate whether the obfuscation practices vary according to the App genre. Initially, we analysed all the APKs together before separating them into two snapshots.


\begin{figure*}[h]
    \centering
    \includegraphics[width=\linewidth]{Figures/AppGenreObfuscationV3.pdf}
    \caption{Obfuscated app percentage by genre (overall)}
    \label{fig:app_genre_overall}
\end{figure*}

Figure~\ref{fig:app_genre_overall} shows the genre-wise obfuscated app percentage. We note that 19 genres have more than 60\% of the apps obfuscated, and almost all the genres have more than 40\% obfuscation percentage. \textit{Casino} genre has the highest obfuscation percentage rate at 80\%, and overall, game genres tend to be more obfuscated than the other genres. The higher obfuscation usage in casino apps is logical due to their nature. These apps often simulate or involve gambling activities and handle monetary transactions and sensitive data related to in-game purchases, making them attractive targets for reverse engineering and hacking. This necessitates robust security measures to prevent fraud and protect user data. 


\begin{figure}[h]
    \centering
    \includegraphics[width=\linewidth]{Figures/AppGenre2018_2023ComparisonV3.pdf}
    \caption{Percentage of obfuscated apps by genre (2018-2023)}
    \label{fig:app_genre_comparison}
\end{figure}



\subsubsection{Genre-wise obfuscation trends in the two snapshots} To investigate the adoption of obfuscation over time, we study the two snapshots of Google Play separately, i.e., APKs from 2016-2018 as one group and APKs from 2021-2023 as another. 

Figure~\ref{fig:app_genre_comparison} illustrates the change in obfuscation levels by app genre between 2016-2018 to 2021-2023. Notably, app categories such as Education, Weather, and Parenting, which had obfuscation levels below the 2018 average, have increased to above the 2023 average by 2023. One possible reason for this in Education and Parenting apps can be the increase in online education activities during and after COVID-19 and the developers identifying the need for app hardening.

There are some genres, such as Casino and Action, for which the percentage of obfuscated apps didn't change across the two snapshots (i.e., purple and orange circles are close together in Figure~\ref{fig:app_genre_comparison}). This is because these genres are highly obfuscated from the beginning. Finally, the four genres, including Simulation and Role Playing, have a lower percentage of obfuscated apps in the 2021-2023 dataset. Our manual analysis didn't result in a conclusion as to why.


\begin{figure}[!h]
    \centering
    \includegraphics[width=\linewidth]{Figures/AppGenreTechAllV5.pdf}
    \caption{Obfuscation technique usage by genre (overall)}
    \label{fig:app_genre_all_tech}
\end{figure}


\subsubsection{Obfuscation techniques in different app genres} In Figure~\ref{fig:app_genre_all_tech}, we show the prevalence of key obfuscation techniques among various genres. As expected, almost all obfuscated apps in all genres used  Identifier Renaming. Also, it can be noted that genres with more obfuscated app percentages tend to use all three obfuscation techniques. Notably, more than 85\% of \textit{Casino} genre apps employ multiple obfuscation techniques

\subsubsection{Obfuscation tool usage in different app genres} We also investigated whether specific obfuscation tools are favoured by developers in different genres. However, apart from the expected observation that  ProGuard and Allatori being the most used tools, we didn't find any other interesting patterns. Therefore, we haven't included those measurement results.

\subsection{App Developers}
Next, we investigate individual developer-wise code obfuscation practices. From the pool of analyzed APKs, we identified the number of apps associated with each developer. Subsequently, we sorted the developers according to the number of apps they had created and selected the top 100 developers with the highest number of APKs for the 2016-2018 and 2021-2023 datasets. For the 2018 snapshot, we had 8,349 apps among the top 100 developers, while for the 2023 snapshot, we had 11,338 apps among the top 100 developers.

We then proceeded to detect whether or not these developers obfuscate their apps and, if so, what kind of tools and techniques they use. We present our results in five levels; developer obfuscating over 80\% of their apps, 60\%--80\% of apps, 40\%--60\% of apps, less than 40\%, and no obfuscation.

Figure~\ref{fig:developer_trend_my_apps_all} compares the two datasets in terms of developer obfuscation adoption. It shows that more developers have moved to obfuscate more than 80\% of their apps in the 2021-2023 dataset (76\%) compared to the 2016-2018 dataset (48\%).

We also found that among developers who obfuscate more than 80\% of their apps, 73\% in 2018 and 93\% in 2023 used the same obfuscation tool. Additionally, these top developers employ Control Flow Modification (CF) and String Encryption (SE) above the average values discussed in Section~\ref{sec:obstrend}. Specifically, in 2018, top developers used CF in 81.3\% of cases and SE in 66.7\%, while in 2023, these figures increased to 88.2\% and 78.9\%. This results in two insights: 1) Most top developers obfuscate all their apps with advanced techniques, possibly due to concerns about IP and security, and 2) Developers stick to a single tool, possibly due to specialized knowledge or because they bought a commercial licence.

\begin{figure}[]
    \centering
    \includegraphics[width=\linewidth]{Figures/Developer_Analysed_Comparison.pdf}
    \caption{Obfuscation usage (Top-100 developers)}
    \label{fig:developer_trend_my_apps_all}
\end{figure}


Finally, we investigate the obfuscation practices of developers with only one app in Table~\ref{tab:my-table}. According to the table, from those developers, 45.5\% of them obfuscated their apps in the 2016-2018 dataset and 57.2\% obfuscated their apps in the 2021-2023 dataset, showing a clear increase. However, these percentages are approximately 10\% lower than the average obfuscation rate in both cohorts discussed in Section~\ref{sec:obstrend}. This indicates that single-app developers may be less aware or concerned about code protection.


\begin{table}[]
\caption{Developers with only one app}
\label{tab:my-table}
\resizebox{\columnwidth}{!}{%
\begin{tabular}{cccccc}
\hline
\textbf{Year} & \textbf{\begin{tabular}[c]{@{}c@{}}Non\\ Obfuscated\end{tabular}} & \multicolumn{4}{c}{\textbf{Obfuscated}} \\ \hline
\multirow{3}{*}{\textbf{\begin{tabular}[c]{@{}c@{}}2018 \\ Snapshot\end{tabular}}} & \multirow{3}{*}{\begin{tabular}[c]{@{}c@{}}26,581 \\ (54.5\%)\end{tabular}} & \multicolumn{4}{c}{\begin{tabular}[c]{@{}c@{}}22,214 (45.5\%)\end{tabular}} \\ \cline{3-6} 
 &  & \textbf{ProGuard} & \textbf{Allatori} & \textbf{DashO} & \textbf{Other} \\ \cline{3-6} 
 &  & 6,131 & 8,050 & 658 & 7,375 \\ \hline
\multirow{3}{*}{\textbf{\begin{tabular}[c]{@{}c@{}}2023 \\ Snapshot\end{tabular}}} & \multirow{3}{*}{\begin{tabular}[c]{@{}c@{}}19,510 \\ (42.8\%)\end{tabular}} & \multicolumn{4}{c}{\begin{tabular}[c]{@{}c@{}}26,084 (57.2\%)\end{tabular}} \\ \cline{3-6} 
 &  & \textbf{ProGuard} & \textbf{Allatori} & \textbf{DashO} & \textbf{Other} \\ \cline{3-6} 
 &  & 12,697 & 9,672 & 234 & 3,581 \\ \hline
\end{tabular}%
}
\end{table}

\subsection{Top-k Apps}

Next, we investigate the obfuscation practices of top apps in Google Play Store. First, we rank the apps using the same criterion used by our previous work~\cite{rajasegaran2019multi, karunanayake2020multi, seneviratne2015early}. That is, we sort the apps in descending order of number of downloads, average rating, and rating count, with the intuition that top apps have high download numbers and high ratings, even when reviewed by a large number of users. Then, we investigated the percentage of obfuscated apps and obfuscation tools and technique usage as summarized in Table~\ref{tab:top_k_apps_2018_2023}.

When considering the highly ranked applications (i.e., top-1,000), the obfuscation percentage is notably higher, at around 93\%, in both datasets, which is significantly higher than the average percentage of obfuscation we observed in Section~\ref{sec:obstrend}. Top-ranked apps, likely due to their higher visibility and potential revenue, invest more in obfuscation to safeguard their intellectual property and enhance security. 

The obfuscation percentage decreases when going from the top 1,000 apps to the top 30,000 apps. Nonetheless, the obfuscation percentage in both datasets remains around similar values until the top 30,000 (e.g., $\sim$74\% for top-30,000). This indicates that the major increase in obfuscation in the 2021-2023 dataset comes from apps beyond the top 30,000.

When observing the tools used, the usage of ProGuard increases as we move from top to lower-ranked apps in both datasets. This may be because ProGuard is free and the default in Android Studio, while commercial tools like Allatori and DashO are expensive. There is a notable increase in the use of Allatori among the top apps in the 2021-2023 dataset. Regarding obfuscation techniques, the top 1,000 apps utilize all three techniques more frequently than lower-ranked apps in both snapshots. This indicates that the top 1,000 apps are more heavily protected compared to lower-ranked ones.

\begin{table*}[]
\caption{Summary of analysis results for Top-k apps in 2018 and 2023}
\label{tab:top_k_apps_2018_2023}
\resizebox{\textwidth}{!}{%
\begin{tabular}{lccccccccc}
\hline
\multicolumn{1}{c}{\begin{tabular}[c]{@{}c@{}}Top k apps - \\ Year\end{tabular}} & \begin{tabular}[c]{@{}c@{}}Total \\ Apps\end{tabular} & \begin{tabular}[c]{@{}c@{}}Obfuscation\\ Percentage\end{tabular} & \begin{tabular}[c]{@{}c@{}}ProGuard\\ Percentage\end{tabular} & \begin{tabular}[c]{@{}c@{}}Allatori\\ Percentage\end{tabular} & \begin{tabular}[c]{@{}c@{}}DashO\\ Percentage\end{tabular} & \begin{tabular}[c]{@{}c@{}}Other\\ Percentage\end{tabular} & \begin{tabular}[c]{@{}c@{}}IR\\ Percentage\end{tabular} & \begin{tabular}[c]{@{}c@{}}CF\\ Percentage\end{tabular} & \begin{tabular}[c]{@{}c@{}}SE\\ Percentage\end{tabular} \\ \hline
1k (2018) & 1,000 & 93.40 & 29.98 & 28.48 & 0.64 & 40.90 & 99.90 & 88.76 & 65.42 \\
10k (2018) & 10,000 & 85.19 & 25.55 & 35.32 & 0.47 & 38.65 & 99.90 & 88.76 & 71.91 \\
20k (2018) & 20,000 & 78.42 & 26.31 & 36.76 & 0.57 & 36.36 & 99.87 & 87.37 & 71.49 \\
30k (2018) & 30,000 & 74.40 & 27.30 & 37.71 & 0.64 & 34.36 & 99.82 & 86.75 & 71.11 \\
30k+ (2018) & 314,568 & 53.36 & 36.72 & 34.70 & 1.33 & 27.24 & 99.34 & 83.54 & 63.11 \\ \hline
1k (2023) & 1,000 & 92.50 & 24.00 & 51.89 & 1.95 & 22.16 & 100.0 & 92.54 & 83.68 \\
10k (2023) & 10,000 & 81.88 & 26.03 & 56.20 & 1.03 & 16.74 & 99.89 & 89.40 & 82.01 \\
20k (2023) & 20,000 & 76.62 & 30.48 & 52.92 & 0.96 & 15.64 & 99.93 & 85.80 & 78.01 \\
30k (2023) & 30,000 & 73.72 & 33.87 & 50.34 & 0.89 & 14.90 & 99.95 & 83.31 & 75.34 \\
30k+ (2023) & 206,216 & 61.90 & 46.56 & 38.21 & 0.64 & 14.59 & 99.97 & 77.51 & 62.50 \\ \hline
\end{tabular}%
}
\end{table*}

\section{ Task Generalization Beyond i.i.d. Sampling and Parity Functions
}\label{sec:Discussion}
% Discussion: From Theory to Beyond
% \misha{what is beyond?}
% \amir{we mean two things: in the first subsection beyond i.i.d subsampling of parity tasks and in the second subsection beyond parity task}
% \misha{it has to be beyond something, otherwise it is not clear what it is about} \hz{this is suggested by GPT..., maybe can be interpreted as from theory to beyond theory. We can do explicit like Discussion: Beyond i.i.d. task sampling and the Parity Task}
% \misha{ why is "discussion" in the title?}\amir{Because it is a discussion, it is not like separate concrete explnation about why these thing happens or when they happen, they just discuss some interesting scenraios how it relates to our theory.   } \misha{it is not really a discussion -- there is a bunch of experiments}

In this section, we extend our experiments beyond i.i.d. task sampling and parity functions. We show an adversarial example where biased task selection substantially hinders task generalization for sparse parity problem. In addition, we demonstrate that exponential task scaling extends to a non-parity tasks including arithmetic and multi-step language translation.

% In this section, we extend our experiments beyond i.i.d. task sampling and parity functions. On the one hand, we find that biased task selection can significantly degrade task generalization; on the other hand, we show that exponential task scaling generalizes to broader scenarios.
% \misha{we should add a sentence or two giving more detail}


% 1. beyond i.i.d tasks sampling
% 2. beyond parity -> language, arithmetic -> task dependency + implicit bias of transformer (cannot implement this algorithm for arithmatic)



% In this section, we emphasize the challenge of quantifying the level of out-of-distribution (OOD) differences between training tasks and testing tasks, even for a simple parity task. To illustrate this, we present two scenarios where tasks differ between training and testing. For each scenario, we invite the reader to assess, before examining the experimental results, which cases might appear “more” OOD. All scenarios consider \( d = 10 \). \kaiyue{this sentence should be put into 5.1}






% for parity problem




% \begin{table*}[th!]
%     \centering
%     \caption{Generalization Results for Scenarios 1 and 2 for $d=10$.}
%     \begin{tabular}{|c|c|c|c|}
%         \hline
%         \textbf{Scenario} & \textbf{Type/Variation} & \textbf{Coordinates} & \textbf{Generalization accuracy} \\
%         \hline
%         \multirow{3}{*}{Generalization with Missing Pair} & Type 1 & \( c_1 = 4, c_2 = 6 \) & 47.8\%\\ 
%         & Type 2 & \( c_1 = 4, c_2 = 6 \) & 96.1\%\\ 
%         & Type 3 & \( c_1 = 4, c_2 = 6 \) & 99.5\%\\ 
%         \hline
%         \multirow{3}{*}{Generalization with Missing Pair} & Type 1 &  \( c_1 = 8, c_2 = 9 \) & 40.4\%\\ 
%         & Type 2 & \( c_1 = 8, c_2 = 9 \) & 84.6\% \\ 
%         & Type 3 & \( c_1 = 8, c_2 = 9 \) & 99.1\%\\ 
%         \hline
%         \multirow{1}{*}{Generalization with Missing Coordinate} & --- & \( c_1 = 5 \) & 45.6\% \\ 
%         \hline
%     \end{tabular}
%     \label{tab:generalization_results}
% \end{table*}

\subsection{Task Generalization Beyond i.i.d. Task Sampling }\label{sec: Experiment beyond iid sampling}

% \begin{table*}[ht!]
%     \centering
%     \caption{Generalization Results for Scenarios 1 and 2 for $d=10, k=3$.}
%     \begin{tabular}{|c|c|c|}
%         \hline
%         \textbf{Scenario}  & \textbf{Tasks excluded from training} & \textbf{Generalization accuracy} \\
%         \hline
%         \multirow{1}{*}{Generalization with Missing Pair}
%         & $\{4,6\} \subseteq \{s_1, s_2, s_3\}$ & 96.2\%\\ 
%         \hline
%         \multirow{1}{*}{Generalization with Missing Coordinate}
%         & \( s_2 = 5 \) & 45.6\% \\ 
%         \hline
%     \end{tabular}
%     \label{tab:generalization_results}
% \end{table*}




In previous sections, we focused on \textit{i.i.d. settings}, where the set of training tasks $\mathcal{F}_{train}$ were sampled uniformly at random from the entire class $\mathcal{F}$. Here, we explore scenarios that deliberately break this uniformity to examine the effect of task selection on out-of-distribution (OOD) generalization.\\

\textit{How does the selection of training tasks influence a model’s ability to generalize to unseen tasks? Can we predict which setups are more prone to failure?}\\

\noindent To investigate this, we consider two cases parity problems with \( d = 10 \) and \( k = 3 \), where each task is represented by its tuple of secret indices \( (s_1, s_2, s_3) \):

\begin{enumerate}[leftmargin=0.4 cm]
    \item \textbf{Generalization with a Missing Coordinate.} In this setup, we exclude all training tasks where the second coordinate takes the value \( s_2 = 5 \), such as \( (1,5,7) \). At test time, we evaluate whether the model can generalize to unseen tasks where \( s_2 = 5 \) appears.
    \item \textbf{Generalization with Missing Pair.} Here, we remove all training tasks that contain both \( 4 \) \textit{and} \( 6 \) in the tuple \( (s_1, s_2, s_3) \), such as \( (2,4,6) \) and \( (4,5,6) \). At test time, we assess whether the model can generalize to tasks where both \( 4 \) and \( 6 \) appear together.
\end{enumerate}

% \textbf{Before proceeding, consider the following question:} 
\noindent \textbf{If you had to guess.} Which scenario is more challenging for generalization to unseen tasks? We provide the experimental result in Table~\ref{tab:generalization_results}.

 % while the model struggles for one of them while as it generalizes almost perfectly in the other one. 

% in the first scenario, it generalizes almost perfectly in the second. This highlights how exposure to partial task structures can enhance generalization, even when certain combinations are entirely absent from the training set. 

In the first scenario, despite being trained on all tasks except those where \( s_2 = 5 \), which is of size $O(\d^T)$, the model struggles to generalize to these excluded cases, with prediction at chance level. This is intriguing as one may expect model to generalize across position. The failure  suggests that positional diversity plays a crucial role in the task generalization of Transformers. 

In contrast, in the second scenario, though the model has never seen tasks with both \( 4 \) \textit{and} \( 6 \) together, it has encountered individual instances where \( 4 \) appears in the second position (e.g., \( (1,4,5) \)) or where \( 6 \) appears in the third position (e.g., \( (2,3,6) \)). This exposure appears to facilitate generalization to test cases where both \( 4 \) \textit{and} \( 6 \) are present. 



\begin{table*}[t!]
    \centering
    \caption{Generalization Results for Scenarios 1 and 2 for $d=10, k=3$.}
    \resizebox{\textwidth}{!}{  % Scale to full width
        \begin{tabular}{|c|c|c|}
            \hline
            \textbf{Scenario}  & \textbf{Tasks excluded from training} & \textbf{Generalization accuracy} \\
            \hline
            Generalization with Missing Pair & $\{4,6\} \subseteq \{s_1, s_2, s_3\}$ & 96.2\%\\ 
            \hline
            Generalization with Missing Coordinate & \( s_2 = 5 \) & 45.6\% \\ 
            \hline
        \end{tabular}
    }
    \label{tab:generalization_results}
\end{table*}

As a result, when the training tasks are not i.i.d, an adversarial selection such as exclusion of specific positional configurations may lead to failure to unseen task generalization even though the size of $\mathcal{F}_{train}$ is exponentially large. 


% \paragraph{\textbf{Key Takeaways}}
% \begin{itemize}
%     \item Out-of-distribution generalization in the parity problem is highly sensitive to the diversity and positional coverage of training tasks.
%     \item Adversarial exclusion of specific pairs or positional configurations can lead to systematic failures, even when most tasks are observed during training.
% \end{itemize}




%################ previous veriosn down
% \textit{How does the choice of training tasks affect the ability of a model to generalize to unseen tasks? Can we predict which setups are likely to lead to failure?}

% To explore these questions, we crafted specific training and test task splits to investigate what makes one setup appear “more” OOD than another.

% \paragraph{Generalization with Missing Pair.}

% Imagine we have tasks constructed from subsets of \(k=3\) elements out of a larger set of \(d\) coordinates. What happens if certain pairs of coordinates are adversarially excluded during training? For example, suppose \(d=5\) and two specific coordinates, \(c_1 = 1\) and \(c_2 = 2\), are excluded. The remaining tasks are formed from subsets of the other coordinates. How would a model perform when tested on tasks involving the excluded pair \( (c_1, c_2) \)? 

% To probe this, we devised three variations in how training tasks are constructed:
%     \begin{enumerate}
%         \item \textbf{Type 1:} The training set includes all tasks except those containing both \( c_1 = 1 \) and \( c_2 = 2 \). 
%         For this example, the training set includes only $\{(3,4,5)\}$. The test set consists of all tasks containing the rest of tuples.

%         \item \textbf{Type 2:} Similar to Type 1, but the training set additionally includes half of the tasks containing either \( c_1 = 1 \) \textit{or} \( c_2 = 2 \) (but not both). 
%         For the example, the training set includes all tasks from Type 1 and adds tasks like \(\{(1, 3, 4), (2, 3, 5)\}\) (half of those containing \( c_1 = 1 \) or \( c_2 = 2 \)).

%         \item \textbf{Type 3:} Similar to Type 2, but the training set also includes half of the tasks containing both \( c_1 = 1 \) \textit{and} \( c_2 = 2 \). 
%         For the example, the training set includes all tasks from Type 2 and adds, for instance, \(\{(1, 2, 5)\}\) (half of the tasks containing both \( c_1 \) and \( c_2 \)).
%     \end{enumerate}

% By systematically increasing the diversity of training tasks in a controlled way, while ensuring no overlap between training and test configurations, we observe an improvement in OOD generalization. 

% % \textit{However, the question is this improvement similar across all coordinate pairs, or does it depend on the specific choices of \(c_1\) and \(c_2\) in the tasks?} 

% \textbf{Before proceeding, consider the following question:} Is the observed improvement consistent across all coordinate pairs, or does it depend on the specific choices of \(c_1\) and \(c_2\) in the tasks? 

% For instance, consider two cases for \(d = 10, k = 3\): (i) \(c_1 = 4, c_2 = 6\) and (ii) \(c_1 = 8, c_2 = 9\). Would you expect similar OOD generalization behavior for these two cases across the three training setups we discussed?



% \paragraph{Answer to the Question.} for both cases of \( c_1, c_2 \), we observe that generalization fails in Type 1, suggesting that the position of the tasks the model has been trained on significantly impacts its generalization capability. For Type 2, we find that \( c_1 = 4, c_2 = 6 \) performs significantly better than \( c_1 = 8, c_2 = 9 \). 

% Upon examining the tasks where the transformer fails for \( c_1 = 8, c_2 = 9 \), we see that the model only fails at tasks of the form \((*, 8, 9)\) while perfectly generalizing to the rest. This indicates that the model has never encountered the value \( 8 \) in the second position during training, which likely explains its failure to generalize. In contrast, for \( c_1 = 4, c_2 = 6 \), while the model has not seen tasks of the form \((*, 4, 6)\), it has encountered tasks where \( 4 \) appears in the second position, such as \((1, 4, 5)\), and tasks where \( 6 \) appears in the third position, such as \((2, 3, 6)\). This difference may explain why the model generalizes almost perfectly in Type 2 for \( c_1 = 4, c_2 = 6 \), but not for \( c_1 = 8, c_2 = 9 \).



% \paragraph{Generalization with Missing Coordinates.}
% Next, we investigate whether a model can generalize to tasks where a specific coordinate appears in an unseen position during training. For instance, consider \( c_1 = 5 \), and exclude all tasks where \( c_1 \) appears in the second position. Despite being trained on all other tasks, the model fails to generalize to these excluded cases, highlighting the importance of positional diversity in training tasks.



% \paragraph{Key Takeaways.}
% \begin{itemize}
%     \item OOD generalization depends heavily on the diversity and positional coverage of training tasks for the parity problem.
%     \item adversarial exclusion of specific pairs or positional configurations in the parity problem can lead to failure, even when the majority of tasks are observed during training.
% \end{itemize}


%################ previous veriosn up

% \paragraph{Key Takeaways} These findings highlight the complexity of OOD generalization, even in seemingly simple tasks like parity. They also underscore the importance of task design: the diversity of training tasks can significantly influence a model’s ability to generalize to unseen tasks. By better understanding these dynamics, we can design more robust training regimes that foster generalization across a wider range of scenarios.


% #############


% Upon examining the tasks where the transformer fails for \( c_1 = 8, c_2 = 9 \), we see that the model only fails at tasks of the form \((*, 8, 9)\) while perfectly generalizing to the rest. This indicates that the model has never encountered the value \( 8 \) in the second position during training, which likely explains its failure to generalize. In contrast, for \( c_1 = 4, c_2 = 6 \), while the model has not seen tasks of the form \((*, 4, 6)\), it has encountered tasks where \( 4 \) appears in the second position, such as \((1, 4, 5)\), and tasks where \( 6 \) appears in the third position, such as \((2, 3, 6)\). This difference may explain why the model generalizes almost perfectly in Type 2 for \( c_1 = 4, c_2 = 6 \), but not for \( c_1 = 8, c_2 = 9 \).

% we observe a striking pattern: generalization fails entirely in Type 1, regardless of the coordinate pair (\(c_1, c_2\)). However, in Type 2, generalization varies: \(c_1 = 4, c_2 = 6\) achieves 96\% accuracy, while \(c_1 = 8, c_2 = 9\) lags behind at 70\%. Why? Upon closer inspection, the model struggles specifically with tasks like \((*, 8, 9)\), where the combination \(c_1 = 8\) and \(c_2 = 9\) is entirely novel. In contrast, for \(c_1 = 4, c_2 = 6\), the model benefits from having seen tasks where \(4\) appears in the second position or \(6\) in the third. This suggests that positional exposure during training plays a key role in generalization.

% To test whether task structure influences generalization, we consider two variations:
% \begin{enumerate}
%     \item \textbf{Sorted Tuples:} Tasks are always sorted in ascending order.
%     \item \textbf{Unsorted Tuples:} Tasks can appear in any order.
% \end{enumerate}

% If the model struggles with generalizing to the excluded position, does introducing variability through unsorted tuples help mitigate this limitation?

% \paragraph{Discussion of Results}

% In \textbf{Generalization with Missing Pairs}, we observe a striking pattern: generalization fails entirely in Type 1, regardless of the coordinate pair (\(c_1, c_2\)). However, in Type 2, generalization varies: \(c_1 = 4, c_2 = 6\) achieves 96\% accuracy, while \(c_1 = 8, c_2 = 9\) lags behind at 70\%. Why? Upon closer inspection, the model struggles specifically with tasks like \((*, 8, 9)\), where the combination \(c_1 = 8\) and \(c_2 = 9\) is entirely novel. In contrast, for \(c_1 = 4, c_2 = 6\), the model benefits from having seen tasks where \(4\) appears in the second position or \(6\) in the third. This suggests that positional exposure during training plays a key role in generalization.

% In \textbf{Generalization with Missing Coordinates}, the results confirm this hypothesis. When \(c_1 = 5\) is excluded from the second position during training, the model fails to generalize to such tasks in the sorted case. However, allowing unsorted tuples introduces positional diversity, leading to near-perfect generalization. This raises an intriguing question: does the model inherently overfit to positional patterns, and can task variability help break this tendency?




% In this subsection, we show that the selection of training tasks can affect the quality of the unseen task generalization significantly in practice. To illustrate this, we present two scenarios where tasks differ between training and testing. For each scenario, we invite the reader to assess, before examining the experimental results, which cases might appear “more” OOD. 

% % \amir{add examples, }

% \kaiyue{I think the name of scenarios here are not very clear}
% \begin{itemize}
%     \item \textbf{Scenario 1:  Generalization Across Excluded Coordinate Pairs (\( k = 3 \))} \\
%     In this scenario, we select two coordinates \( c_1 \) and \( c_2 \) out of \( d \) and construct three types of training sets. 

%     Suppose \( d = 5 \), \( c_1 = 1 \), and \( c_2 = 2 \). The tuples are all possible subsets of \( \{1, 2, 3, 4, 5\} \) with \( k = 3 \):
%     \[
%     \begin{aligned}
%     \big\{ & (1, 2, 3), (1, 2, 4), (1, 2, 5), (1, 3, 4), (1, 3, 5), \\
%            & (1, 4, 5), (2, 3, 4), (2, 3, 5), (2, 4, 5), (3, 4, 5) \big\}.
%     \end{aligned}
%     \]

%     \begin{enumerate}
%         \item \textbf{Type 1:} The training set includes all tuples except those containing both \( c_1 = 1 \) and \( c_2 = 2 \). 
%         For this example, the training set includes only $\{(3,4,5)\}$ tuple. The test set consists of tuples containing the rest of tuples.

%         \item \textbf{Type 2:} Similar to Type 1, but the training set additionally includes half of the tuples containing either \( c_1 = 1 \) \textit{or} \( c_2 = 2 \) (but not both). 
%         For the example, the training set includes all tuples from Type 1 and adds tuples like \(\{(1, 3, 4), (2, 3, 5)\}\) (half of those containing \( c_1 = 1 \) or \( c_2 = 2 \)).

%         \item \textbf{Type 3:} Similar to Type 2, but the training set also includes half of the tuples containing both \( c_1 = 1 \) \textit{and} \( c_2 = 2 \). 
%         For the example, the training set includes all tuples from Type 2 and adds, for instance, \(\{(1, 2, 5)\}\) (half of the tuples containing both \( c_1 \) and \( c_2 \)).
%     \end{enumerate}

% % \begin{itemize}
% %     \item \textbf{Type 1:} The training set includes tuples \(\{1, 3, 4\}, \{2, 3, 4\}\) (excluding tuples with both \( c_1 \) and \( c_2 \): \(\{1, 2, 3\}, \{1, 2, 4\}\)). The test set contains the excluded tuples.
% %     \item \textbf{Type 2:} The training set includes all tuples in Type 1 plus half of the tuples containing either \( c_1 = 1 \) or \( c_2 = 2 \) (e.g., \(\{1, 2, 3\}\)).
% %     \item \textbf{Type 3:} The training set includes all tuples in Type 2 plus half of the tuples containing both \( c_1 = 1 \) and \( c_2 = 2 \) (e.g., \(\{1, 2, 4\}\)).
% % \end{itemize}
    
%     \item \textbf{Scenario 2: Scenario 2: Generalization Across a Fixed Coordinate (\( k = 3 \))} \\
%     In this scenario, we select one coordinate \( c_1 \) out of \( d \) (\( c_1 = 5 \)). The training set includes all task tuples except those where \( c_1 \) is the second coordinate of the tuple. For this scenario, we examine two variations:
%     \begin{enumerate}
%         \item \textbf{Sorted Tuples:} Task tuples are always sorted (e.g., \( (x_1, x_2, x_3) \) with \( x_1 \leq x_2 \leq x_3 \)).
%         \item \textbf{Unsorted Tuples:} Task tuples can appear in any order.
%     \end{enumerate}
% \end{itemize}




% \paragraph{Discussion of Results.} In the first scenario, for both cases of \( c_1, c_2 \), we observe that generalization fails in Type 1, suggesting that the position of the tasks the model has been trained on significantly impacts its generalization capability. For Type 2, we find that \( c_1 = 4, c_2 = 6 \) performs significantly better than \( c_1 = 8, c_2 = 9 \). 

% Upon examining the tasks where the transformer fails for \( c_1 = 8, c_2 = 9 \), we see that the model only fails at tasks of the form \((*, 8, 9)\) while perfectly generalizing to the rest. This indicates that the model has never encountered the value \( 8 \) in the second position during training, which likely explains its failure to generalize. In contrast, for \( c_1 = 4, c_2 = 6 \), while the model has not seen tasks of the form \((*, 4, 6)\), it has encountered tasks where \( 4 \) appears in the second position, such as \((1, 4, 5)\), and tasks where \( 6 \) appears in the third position, such as \((2, 3, 6)\). This difference may explain why the model generalizes almost perfectly in Type 2 for \( c_1 = 4, c_2 = 6 \), but not for \( c_1 = 8, c_2 = 9 \).

% This position-based explanation appears compelling, so in the second scenario, we focus on a single position to investigate further. Here, we find that the transformer fails to generalize to tasks where \( 5 \) appears in the second position, provided it has never seen any such tasks during training. However, when we allow for more task diversity in the unsorted case, the model achieves near-perfect generalization. 

% This raises an important question: does the transformer have a tendency to overfit to positional patterns, and does introducing more task variability, as in the unsorted case, prevent this overfitting and enable generalization to unseen positional configurations?

% These findings highlight that even in a simple task like parity, it is remarkably challenging to understand and quantify the sources and levels of OOD behavior. This motivates further investigation into the nuances of task design and its impact on model generalization.


\subsection{Task Generalization Beyond Parity Problems}

% \begin{figure}[t!]
%     \centering
%     \includegraphics[width=0.45\textwidth]{Figures/arithmetic_v1.png}
%     \vspace{-0.3cm}
%     \caption{Task generalization for arithmetic task with CoT, it has $\d =2$ and $T = d-1$ as the ambient dimension, hence $D\ln(DT) = 2\ln(2T)$. We show that the empirical scaling closely follows the theoretical scaling.}
%     \label{fig:arithmetic}
% \end{figure}



% \begin{wrapfigure}{r}{0.4\textwidth}  % 'r' for right, 'l' for left
%     \centering
%     \includegraphics[width=0.4\textwidth]{Figures/arithmetic_v1.png}
%     \vspace{-0.3cm}
%     \caption{Task generalization for the arithmetic task with CoT. It has $d =2$ and $T = d-1$ as the ambient dimension, hence $D\ln(DT) = 2\ln(2T)$. We show that the empirical scaling closely follows the theoretical scaling.}
%     \label{fig:arithmetic}
% \end{wrapfigure}

\subsubsection{Arithmetic Task}\label{subsec:arithmetic}











We introduce the family of \textit{Arithmetic} task that, like the sparse parity problem, operates on 
\( d \) binary inputs \( b_1, b_2, \dots, b_d \). The task involves computing a structured arithmetic expression over these inputs using a sequence of addition and multiplication operations.
\newcommand{\op}{\textrm{op}}

Formally, we define the function:
\[
\text{Arithmetic}_{S} \colon \{0,1\}^d \to \{0,1,\dots,d\},
\]
where \( S = (\op_1, \op_2, \dots, \op_{d-1}) \) is a sequence of \( d-1 \) operations, each \( \op_k \) chosen from \( \{+, \times\} \). The function evaluates the expression by applying the operations sequentially from left-to-right order: for example, if \( S = (+, \times, +) \), then the arithmetic function would compute:
\[
\text{Arithmetic}_{S}(b_1, b_2, b_3, b_4) = ((b_1 + b_2) \times b_3) + b_4.
\]
% Thus, the sequence of operations \( S \) defines how the binary inputs are combined to produce an integer output between \( 0 \) and \( d \).
% \[
% \text{Arithmetic}_{S} 
% (b_1,\,b_2,\,\dots,b_d)
% =
% \Bigl(\dots\bigl(\,(b_1 \;\op_1\; b_2)\;\op_2\; b_3\bigr)\,\dots\Bigr) 
% \;\op_{d-1}\; b_d.
% \]
% We now introduce an \emph{Arithmetic} task that, like the sparse parity problem, operates on $d$ binary inputs $b_1, b_2, \dots, b_d$. Specifically, we define an arithmetic function
% \[
% \text{Arithmetic}_{S}\colon \{0,1\}^d \;\to\; \{0,1,\dots,d\},
% \]
% where $S = (i_1, i_2, \dots, i_{d-1})$ is a sequence of $d-1$ operations, each $i_k \in \{+,\,\times\}$. The value of $\text{Arithmetic}_{S}$ is obtained by applying the prescribed addition and multiplication operations in order, namely:
% \[
% \text{Arithmetic}_{S}(b_1,\,b_2,\,\dots,b_d)
% \;=\;
% \Bigl(\dots\bigl(\,(b_1 \;i_1\; b_2)\;i_2\; b_3\bigr)\,\dots\Bigr) 
% \;i_{d-1}\; b_d.
% \]

% This is an example of our framework where $T = d-1$ and $|\Theta_t| = 2$ with total $2^d$ possible tasks. 




By introducing a step-by-step CoT, arithmetic class belongs to $ARC(2, d-1)$: this is because at every step, there is only $\d = |\Theta_t| = 2$ choices (either $+$ or $\times$) while the length is  $T = d-1$, resulting a total number of $2^{d-1}$ tasks. 


\begin{minipage}{0.5\textwidth}  % Left: Text
    Task generalization for the arithmetic task with CoT. It has $d =2$ and $T = d-1$ as the ambient dimension, hence $D\ln(DT) = 2\ln(2T)$. We show that the empirical scaling closely follows the theoretical scaling.
\end{minipage}
\hfill
\begin{minipage}{0.4\textwidth}  % Right: Image
    \centering
    \includegraphics[width=\textwidth]{Figures/arithmetic_v1.png}
    \refstepcounter{figure}  % Manually advances the figure counter
    \label{fig:arithmetic}  % Now this label correctly refers to the figure
\end{minipage}

Notably, when scaling with \( T \), we observe in the figure above that the task scaling closely follow the theoretical $O(D\log(DT))$ dependency. Given that the function class grows exponentially as \( 2^T \), it is truly remarkable that training on only a few hundred tasks enables generalization to an exponentially larger space—on the order of \( 2^{25} > 33 \) Million tasks. This exponential scaling highlights the efficiency of structured learning, where a modest number of training examples can yield vast generalization capability.





% Our theory suggests that only $\Tilde{O}(\ln(T))$ i.i.d training tasks is enough to generalize to the rest of unseen tasks. However, we show in Figure \ref{fig:arithmetic} that transformer is not able to match  that. The transformer out-of distribution generalization behavior is not consistent across different dimensions when we scale the number of training tasks with $\ln(T)$. \hongzhou{implicit bias, optimization, etc}
 






% \subsection{Task generalization Beyond parity problem}

% \subsection{Arithmetic} In this setting, we still use the set-up we introduced in \ref{subsec:parity_exmaple}, the input is still a set of $d$ binary variable, $b_1, b_2,\dots,b_d$ and ${Arithmatic_{S}}:\{0,1\}\rightarrow \{0, 1, \dots, d\}$, where $S = (i_1,i_2,\dots,i_{d-1})$ is a tuple of size $d-1$ where each coordinate is either add($+
% $) or multiplication ($\times$). The function is as following,

% \begin{align*}
%     Arithmatic_{S}(b_1, b_2,\dots,b_d) = (\dots(b1(i1)b2)(i3)b3\dots)(i{d-1})
% \end{align*}
    


\subsubsection{Multi-Step Language Translation Task}

 \begin{figure*}[h!]
    \centering
    \includegraphics[width=0.9\textwidth]{Figures/combined_plot_horiz.png}
    \vspace{-0.2cm}
    \caption{Task generalization for language translation task: $\d$ is the number of languages and $T$ is the length of steps.}
    \vspace{-2mm}
    \label{fig:language}
\end{figure*}
% \vspace{-2mm}

In this task, we study a sequential translation process across multiple languages~\cite{garg2022can}. Given a set of \( D \) languages, we construct a translation chain by randomly sampling a sequence of \( T \) languages \textbf{with replacement}:  \(L_1, L_2, \dots, L_T,\)
where each \( L_t \) is a sampled language. Starting with a word, we iteratively translate it through the sequence:
\vspace{-2mm}
\[
L_1 \to L_2 \to L_3 \to \dots \to L_T.
\]
For example, if the sampled sequence is EN → FR → DE → FR, translating the word "butterfly" follows:
\vspace{-1mm}
\[
\text{butterfly} \to \text{papillon} \to \text{schmetterling} \to \text{papillon}.
\]
This task follows an \textit{AutoRegressive Compositional} structure by itself, specifically \( ARC(D, T-1) \), where at each step, the conditional generation only depends on the target language, making \( D \) as the number of languages and the total number of possible tasks is \( D^{T-1} \). This example illustrates that autoregressive compositional structures naturally arise in real-world languages, even without explicit CoT. 

We examine task scaling along \( D \) (number of languages) and \( T \) (sequence length). As shown in Figure~\ref{fig:language}, empirical  \( D \)-scaling closely follows the theoretical \( O(D \ln D T) \). However, in the \( T \)-scaling case, we observe a linear dependency on \( T \) rather than the logarithmic dependency \(O(\ln T) \). A possible explanation is error accumulation across sequential steps—longer sequences require higher precision in intermediate steps to maintain accuracy. This contrasts with our theoretical analysis, which focuses on asymptotic scaling and does not explicitly account for compounding errors in finite-sample settings.

% We examine task scaling along \( D \) (number of languages) and \( T \) (sequence length). As shown in Figure~\ref{fig:language}, empirical scaling closely follows the theoretical \( O(D \ln D T) \) trend, with slight exceptions at $ T=10 \text{ and } 3$ in Panel B. One possible explanation for this deviation could be error accumulation across sequential steps—longer sequences require each intermediate translation to be approximated with higher precision to maintain test accuracy. This contrasts with our theoretical analysis, which primarily focuses on asymptotic scaling and does not explicitly account for compounding errors in finite-sample settings.

Despite this, the task scaling is still remarkable — training on a few hundred tasks enables generalization to   $4^{10} \approx 10^6$ tasks!






% , this case, we are in a regime where \( D \ll T \), we observe  that the task complexity empirically scales as \( T \log T \) rather than \( D \log T \). 


% the model generalizes to an exponentially larger space of \( 2^T \) unseen tasks. In case $T=25$, this is $2^{25} > 33$ Million tasks. This remarkable exponential generalization demonstrates the power of structured task composition in enabling efficient generalization.


% In the case of parity tasks, introducing CoT effectively decomposes the problem from \( ARC(D^T, 1) \) to \( ARC(D, T) \), significantly improving task generalization.

% Again, in the regime scaling $T$, we again observe a $T\log T$ dependency. Knowing that the function class is scaling as $D^T$, it is remarkable that training on a few hundreds tasks can generalize to $4^{10} \approx 1M$ tasks. 





% We further performed a preliminary investigation on a semi-synthetic word-level translation task to show that (1) task generalization via composition structure is feasible beyond parity and (2) understanding the fine-grained mechanism leading to this generalization is still challenging. 
% \noindent
% \noindent
% \begin{minipage}[t]{\columnwidth}
%     \centering
%     \textbf{\scriptsize In-context examples:}
%     \[
%     \begin{array}{rl}
%         \textbf{Input} & \hspace{1.5em} \textbf{Output} \\
%         \hline
%         \textcolor{blue}{car}   & \hspace{1.5em} \textcolor{red}{voiture \;,\; coche} \\
%         \textcolor{blue}{house} & \hspace{1.5em} \textcolor{red}{maison \;,\; casa} \\
%         \textcolor{blue}{dog}   & \hspace{1.5em} \textcolor{red}{chien \;,\; perro} 
%     \end{array}
%     \]
%     \textbf{\scriptsize Query:}
%     \[
%     \begin{array}{rl}
%         \textbf{Input} & \textbf{Output} \\
%         \hline
%         \textcolor{blue}{cat} & \hspace{1.5em} \textcolor{red}{?} \\
%     \end{array}
%     \]
% \end{minipage}



% \begin{figure}[h!]
%     \centering
%     \includegraphics[width=0.45\textwidth]{Figures/translation_scale_d.png}
%     \vspace{-0.2cm}
%     \caption{Task generalization behavior for word translation task.}
%     \label{fig:arithmetic}
% \end{figure}


\vspace{-1mm}
\section{Conclusions}
% \misha{is it conclusion of the section or of the whole paper?}    
% \amir{The whole paper. It is very short, do we need a separate section?}
% \misha{it should not be a subsection if it is the conclusion the whole thing. We can just remove it , it does not look informative} \hz{let's do it in a whole section, just to conclude and end the paper, even though it is not informative}
%     \kaiyue{Proposal: Talk about the implication of this result on theory development. For example, it calls for more fine-grained theoretical study in this space.  }

% \huaqing{Please feel free to edit it if you have better wording or suggestions.}

% In this work, we propose a theoretical framework to quantitatively investigate task generalization with compositional autoregressive tasks. We show that task to $D^T$ task is theoretically achievable by training on only $O (D\log DT)$ tasks, and empirically observe that transformers trained on parity problem indeed achieves such task generalization. However, for other tasks beyond parity, transformers seem to fail to achieve this bond. This calls for more fine-grained theoretical study the phenomenon of task generalization specific to transformer model. It may also be interesting to study task generalization beyond the setting of in-context learning. 
% \misha{what does this add?} \amir{It does not, i dont have any particular opinion to keep it. @Hongzhou if you want to add here?}\hz{While it may not introduce anything new, we are following a good practice to have a short conclusion. It provides a clear closing statement, reinforces key takeaways, and helps the reader leave with a well-framed understanding of our contributions. }
% In this work, we quantitatively investigate task generalization under autoregressive compositional structure. We demonstrate that task generalization to $D^T$ tasks is theoretically achievable by training on only $\tilde O(D)$ tasks. Empirically, we observe that transformers trained indeed achieve such exponential task generalization on problems such as parity, arithmetic and multi-step language translation. We believe our analysis opens up a new angle to understand the remarkable generalization ability of Transformer in practice. 

% However, for tasks beyond the parity problem, transformers appear to fail to reach this bound. This highlights the need for a more fine-grained theoretical exploration of task generalization, especially for transformer models. Additionally, it may be valuable to investigate task generalization beyond the scope of in-context learning.



In this work, we quantitatively investigated task generalization under the autoregressive compositional structure, demonstrating both theoretically and empirically that exponential task generalization to $D^T$ tasks can be achieved with training on only $\tilde{O}(D)$ tasks. %Our theoretical results establish a fundamental scaling law for task generalization, while our experiments validate these insights across problems such as parity, arithmetic, and multi-step language translation. The remarkable ability of transformers to generalize exponentially highlights the power of structured learning and provides a new perspective on how large language models extend their capabilities beyond seen tasks. 
We recap our key contributions  as follows:
\begin{itemize}
    \item \textbf{Theoretical Framework for Task Generalization.} We introduced the \emph{AutoRegressive Compositional} (ARC) framework to model structured task learning, demonstrating that a model trained on only $\tilde{O}(D)$ tasks can generalize to an exponentially large space of $D^T$ tasks.
    
    \item \textbf{Formal Sample Complexity Bound.} We established a fundamental scaling law that quantifies the number of tasks required for generalization, proving that exponential generalization is theoretically achievable with only a logarithmic increase in training samples.
    
    \item \textbf{Empirical Validation on Parity Functions.} We showed that Transformers struggle with standard in-context learning (ICL) on parity tasks but achieve exponential generalization when Chain-of-Thought (CoT) reasoning is introduced. Our results provide the first empirical demonstration of structured learning enabling efficient generalization in this setting.
    
    \item \textbf{Scaling Laws in Arithmetic and Language Translation.} Extending beyond parity functions, we demonstrated that the same compositional principles hold for arithmetic operations and multi-step language translation, confirming that structured learning significantly reduces the task complexity required for generalization.
    
    \item \textbf{Impact of Training Task Selection.} We analyzed how different task selection strategies affect generalization, showing that adversarially chosen training tasks can hinder generalization, while diverse training distributions promote robust learning across unseen tasks.
\end{itemize}



We introduce a framework for studying the role of compositionality in learning tasks and how this structure can significantly enhance generalization to unseen tasks. Additionally, we provide empirical evidence on learning tasks, such as the parity problem, demonstrating that transformers follow the scaling behavior predicted by our compositionality-based theory. Future research will  explore how these principles extend to real-world applications such as program synthesis, mathematical reasoning, and decision-making tasks. 


By establishing a principled framework for task generalization, our work advances the understanding of how models can learn efficiently beyond supervised training and adapt to new task distributions. We hope these insights will inspire further research into the mechanisms underlying task generalization and compositional generalization.

\section*{Acknowledgements}
We acknowledge support from the National Science Foundation (NSF) and the Simons Foundation for the Collaboration on the Theoretical Foundations of Deep Learning through awards DMS-2031883 and \#814639 as well as the  TILOS institute (NSF CCF-2112665) and the Office of Naval Research (ONR N000142412631). 
This work used the programs (1) XSEDE (Extreme science and engineering discovery environment)  which is supported by NSF grant numbers ACI-1548562, and (2) ACCESS (Advanced cyberinfrastructure coordination ecosystem: services \& support) which is supported by NSF grants numbers \#2138259, \#2138286, \#2138307, \#2137603, and \#2138296. Specifically, we used the resources from SDSC Expanse GPU compute nodes, and NCSA Delta system, via allocations TG-CIS220009. 

We present RiskHarvester, a risk-based tool to compute a security risk score based on the value of the asset and ease of attack on a database. We calculated the value of asset by identifying the sensitive data categories present in a database from the database keywords. We utilized data flow analysis, SQL, and Object Relational Mapper (ORM) parsing to identify the database keywords. To calculate the ease of attack, we utilized passive network analysis to retrieve the database host information. To evaluate RiskHarvester, we curated RiskBench, a benchmark of 1,791 database secret-asset pairs with sensitive data categories and host information manually retrieved from 188 GitHub repositories. RiskHarvester demonstrates precision of (95\%) and recall (90\%) in detecting database keywords for the value of asset and precision of (96\%) and recall (94\%) in detecting valid hosts for ease of attack. Finally, we conducted an online survey to understand whether developers prioritize secret removal based on security risk score. We found that 86\% of the developers prioritized the secrets for removal with descending security risk scores.

\begin{credits}
% \subsubsection{\ackname} The authors would like to thank our anonymous reviewers for their constructive comments and suggestions.  Additionally, this work acknowledges funding from the Research Council of Norway for project no. 326907. Moreover, this work was also supported by the Swedish Foundation for Strategic Research (SSF) grant FU21-0067.
\subsubsection{\ackname} The authors extend their sincere gratitude to the anonymous reviewers for their valuable and constructive feedback. This research was funded by the Research Council of Norway under project number 326907. Additionally, support was provided by the Swedish Foundation for Strategic Research (SSF) through grant FUS21-0067.

% \subsubsection{\discintname}
% It is now necessary to declare any competing interests or to specifically
% state that the authors have no competing interests. Please place the
% statement with a bold run-in heading in small font size beneath the
% (optional) acknowledgments\footnote{If EquinOCS, our proceedings submission
% system, is used, then the disclaimer can be provided directly in the system.},
% for example: The authors have no competing interests to declare that are
% relevant to the content of this article. Or: Author A has received research
% grants from Company W. Author B has received a speaker honorarium from
% Company X and owns stock in Company Y. Author C is a member of committee Z.
\end{credits}
%
% ---- Bibliography ----
%
% BibTeX users should specify bibliography style 'splncs04'.
% References will then be sorted and formatted in the correct style.
%
\bibliographystyle{splncs04}
% \bibliographystyle{plainnat}
\bibliography{mybibliography}
%
% \begin{thebibliography}{8}
% \bibitem{ref_article1}
% Author, F.: Article title. Journal \textbf{2}(5), 99--110 (2016)

% \bibitem{ref_lncs1}
% Author, F., Author, S.: Title of a proceedings paper. In: Editor,
% F., Editor, S. (eds.) CONFERENCE 2016, LNCS, vol. 9999, pp. 1--13.
% Springer, Heidelberg (2016). \doi{10.10007/1234567890}

% \bibitem{ref_book1}
% Author, F., Author, S., Author, T.: Book title. 2nd edn. Publisher,
% Location (1999)

% \bibitem{ref_proc1}
% Author, A.-B.: Contribution title. In: 9th International Proceedings
% on Proceedings, pp. 1--2. Publisher, Location (2010)

% \bibitem{ref_url1}
% LNCS Homepage, \url{http://www.springer.com/lncs}, last accessed 2023/10/25
% \end{thebibliography}
\end{document}
