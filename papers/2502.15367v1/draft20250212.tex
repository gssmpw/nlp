% This is samplepaper.tex, a sample chapter demonstrating the
% LLNCS macro package for Springer Computer Science proceedings;
% Version 2.21 of 2022/01/12
%
\documentclass[runningheads]{llncs}
%
% \usepackage[authoryear]{natbib}

\usepackage[numbers]{natbib} % Use numerical citations
% \bibliographystyle{splncs04} % Use the default LNCS bibliography style
% \bibliographystyle{plainnat}
\usepackage[T1]{fontenc}
\usepackage{tabularx}
\usepackage{subfiles}
\usepackage{hyperref}
% \usepackage{todonotes}
\usepackage{balance}
\usepackage{booktabs}
\usepackage{natbib}
\usepackage{caption}
\usepackage{graphicx}
\usepackage{subcaption} 
\usepackage{adjustbox} % Helps with centering the last row

% \usepackage{natbib}
% \bibliographystyle{plainnat}  % Use a style compatible with author-year citations\textbf{}
% T1 fonts will be used to generate the final print and online PDFs,
% so please use T1 fonts in your manuscript whenever possible.
% Other font encondings may result in incorrect characters.
%
% Used for displaying a sample figure. If possible, figure files should
% be included in EPS format.
%
% If you use the hyperref package, please uncomment the following two lines
% to display URLs in blue roman font according to Springer's eBook style:
%\usepackage{color}
%\renewcommand\UrlFont{\color{blue}\rmfamily}
%\urlstyle{rm}
%
\begin{document}
%
% \title{Designing Emotion-Aware Voice Assistants: Strategies for Handling User Emotions}
\title{Advancing User-Voice Interaction: Exploring Emotion-Aware Voice Assistants Through a Role-Swapping Approach}
%
%\titlerunning{Abbreviated paper title}
% If the paper title is too long for the running head, you can set
% an abbreviated paper title here
%
\author{
Yong Ma\inst{1}\orcidID{0000-0002-8398-4118} \and
Yuchong Zhang\inst{2}\orcidID{0000-0003-1804-6296} \and
Di Fu\inst{3}\orcidID{0000-0002-5385-2982} \and
Stephanie Zubicueta Portales\inst{4}\orcidID{0009-0004-2459-1949} \and
Danica Kragic\inst{2}\orcidID{0000-0003-2965-2953} \and
Morten Fjeld\inst{1,5}\orcidID{0000-0002-9562-5147}
}

%
\authorrunning{Y. Ma et al.}
% First names are abbreviated in the running head.
% If there are more than two authors, 'et al.' is used.
%
\institute{University of Bergen, Bergen, Norway 
% \email{@springer.com}
\and KTH Royal Institute of Technology, Stockholm, Sweden \and University of Surrey, Guildford, UK \and Chalmers University of Technology, Gothenburg, Sweden \and Norwegian University of Science and Technology,Trondheim, Norway
\\}
% \url{http://www.springer.com/gp/computer-science/lncs} \and
% ABC Institute, Rupert-Karls-University Heidelberg, Heidelberg, Germany\\
% \email{\{abc,lncs\}@uni-heidelberg.de}}
% \author{Yong Ma}
% \affiliation{%
%   \institution{University of Bergen}
%   \city{Bergen}
%   \country{Norway}}
%   \email{yong.ma@uib.no}

% \author{Yuchong Zhang}
% % \orcid{0000-0003-1804-6296}
% \email{yuchongz@kth.se}
% \affiliation{%
%   \institution{KTH Royal Institute of Technology}
%   \city{Stockholm}
%   \country{Sweden}
% }



% \author{Di Fu}
% % \orcid{0000-0003-1804-6296}
% \email{d.fu@surrey.ac.uk}
% \affiliation{%
%   \institution{Unviersity of Surrey}
%   \city{Guildford}
%   \country{United Kingdom}
% }

% \author{Stephanie Zubicueta Portales}
% \affiliation{%
%   \institution{Norwegian University of Science and Technology}
%   \city{Trondheim}
%   \country{Norway}}
%   \email{stephanie.portales@student.uib.no}

% \author{Morten Fjeld}
% % \orcid{0000-0002-9562-5147}
% \affiliation{%
% \institution{University of Bergen}
%   \city{Bergen}
%   \country{Norway}}
% \email{morten.fjeld@uib.no}
% \additionalaffiliation{%
% \institution{Chalmers University of Technology}
%   \city{Gothenburg}
%   \country{Sweden}}
% \email{fjeld@chalmers.se}

% \author{Danica Kragic}
% \affiliation{%
% \institution{KTH Royal Institute of Technology}
%   \city{Stockholm}
%   \country{Sweden}}
% \email{danik@kth.se}
%
\maketitle              % typeset the header of the contribution
%
\begin{abstract}
% The abstract should briefly summarize the contents of the paper in
% 150--250 words.
% As voice assistants (VAs) become increasingly integrated into daily life, the need for emotion-aware systems capable of recognizing and appropriately responding to user emotions has grown. While significant progress has been made in speech emotion recognition (SER) and sentiment analysis, effectively responding to user emotions—particularly negative ones—remains a challenge. This study explores human emotional response strategies in VAs interactions using a role-swapping approach, where participants regulate AI emotions rather than receiving pre-programmed responses. Through speech feature analysis and natural language processing (NLP) techniques, we examined acoustic and linguistic patterns across different emotional scenarios. Our results indicate that participants favor neutral or positive emotional responses when engaging with negative emotional cues, demonstrating a natural inclination toward emotional regulation and de-escalation. Additionally, we found that root mean square (RMS), zero-crossing rate (ZCR), and jitter serve as key acoustic indicators of emotional state, while sentiment polarity and lexical diversity (TTR) distinguish between positive and negative responses. These findings provide valuable insights for developing adaptive, context-aware VAs that deliver empathetic, culturally sensitive, and user-aligned responses. 
As voice assistants (VAs) become increasingly integrated into daily life, the need for emotion-aware systems that can recognize and respond appropriately to user emotions has grown. While significant progress has been made in speech emotion recognition (SER) and sentiment analysis, effectively addressing user emotions—particularly negative ones—remains a challenge. This study explores human emotional response strategies in VA interactions using a role-swapping approach, where participants regulate AI emotions rather than receiving pre-programmed responses. Through speech feature analysis and natural language processing (NLP), we examined acoustic and linguistic patterns across various emotional scenarios. Results show that participants favor neutral or positive emotional responses when engaging with negative emotional cues, highlighting a natural tendency toward emotional regulation and de-escalation. Key acoustic indicators such as root mean square (RMS), zero-crossing rate (ZCR), and jitter were identified as sensitive to emotional states, while sentiment polarity and lexical diversity (TTR) distinguished between positive and negative responses. These findings provide valuable insights for developing adaptive, context-aware VAs capable of delivering empathetic, culturally sensitive, and user-aligned responses. By understanding how humans naturally regulate emotions in AI interactions, this research contributes to the design of more intuitive and emotionally intelligent voice assistants, enhancing user trust and engagement in human-AI interactions.


\keywords{Emotion-Aware Voice Assistants \and Role-Swapping Approach \and Speech and Linguistic Analysis \and Speech Emotion Recognition (SER)}
\end{abstract}
%
%
%

% 
% 
The widespread integration of communication networks and smart devices in modern control systems has increased the vulnerability of industrial systems to online cyber-attacks, e.g., Industroyer, Blackenergy, etc \citep{osti_1505628}.
% Modern control systems have seen a large push to include communication networks and smart devices to increase performance, made possible by improvements in communication device cost and energy consumption. This trend has been coupled with the usage of open-standard communication protocols among industrial control systems, making them vulnerable to online cyber-attacks such as Industroyer, Blackenergy, etc \citep{osti_1505628}. 
To counter this, methods have been developed to improve security by achieving attack detection, mitigation, and monitoring, among others \citep{sandberg2022secure}. This paper focuses on active attack diagnosis to mitigate stealthy attacks. 
%
%\subsection{Literature review}

Active diagnosis techniques rely on the inclusion of additional moduli to control systems
% inclusion within the control system of additional moduli 
to alter the behavior of the system compared to information known by the attacker. 
For instance, the concept of additive watermarking was introduced in \cite{mo2015physical}, where noise signals of known mean and variance are added at the plant and compensated for it at the controller. 
This compensation, however, is not exact, causing some performance degradation. Thus, trade-offs between performance and detectability  are necessary \citep{zhu2023detection}.
% A later work \citep{zhu2023detection} designs the watermark signal by trading performance for detection. Thus, although additive watermarking serves as a good detection scheme, they endure performance losses even in the nominal case. 

In encrypted control \citep{darup2021encrypted}, the sensor data is encrypted, sent to the controller, and then operated on directly. Encrypted input signals are sent back to the plant for decryption. Although encryption is widespread in IT security, in control systems it presents some concerns, such as the introduction of time delays \citep{stabile2024verifiable}, while it may present inherent weaknesses \citep{alisic2023model}.
% they are not preferred as they introduce time delays \citep{stabile2024verifiable} which can cause instability, and some encryption schemes can be very weak  \citep{alisic2023model}. 

In moving target defense \citep{griffioen2020moving}, the plant is augmented with fictitious dynamics, known to the controller. The plant output is transmitted to the controller along with the fictitious states over a network under attack. 
The additional measurements then aide in the detection of attacks. 
This comes at the cost of higher communication bandwidth needs, which increases rapidly with the dimension of the augmented systems.
% Since the dynamics of the fictitious dynamics are exactly known to the controller, the attack is detected easily. However, when the scale of the system increases, the communication bandwidth used by moving the target defense approach increases rapidly. 

Other recently proposed works include two-way coding \citep{fang2019two}, a weak encryuption technique, and dynamic masking \citep{abdalmoaty2023privacy}, which enhances privacy as well as security, have been shown to be effective against zero-dynamics attacks.
% Two-way coding \citep{fang2019two} and dynamic masking \citep{abdalmoaty2023privacy} are other recently proposed approaches. Two-way coding is another form of weak encryption technique whilst dynamic masking proposes an architecture that enhances both privacy and security. These schemes are shown to be effective against zero dynamics attacks but remain to be studied for other classes of attacks. 
% Recent extensions include \citep{mukherjee2021secure,ramos2024privacy}.
% Some other works which are related are \citep{mukherjee2021secure}, an extension of \cite{fang2019two}. The work \citep{ramos2024privacy} is an extension of moving target defense for multi-agent systems. 
Furthermore, filtering techniques for attack detection are proposed by \cite{murguia2020security,hashemi2022codesign,escudero2023safety}, while not focusing on stealthy attacks.
% The works \citep{murguia2020security,hashemi2022codesign,escudero2023safety} develop filtering techniques to guarantee safety, without being focused on stealthy covert attacks.

Multiplicative watermarking (mWM) has been proposed by the authors as a diagnosis technique \citep{ferrari2020switching}. mWM consists of a pair of filters on each communication channel between the plant and its controller; the scheme is affine to weak encryption, whereby ``encoding'' and ``decoding'' are done by changing signals' dynamic characteristics through inverse pairs of filters. This enables original signals to be recovered exactly, and thus does not lead to performance degradation.
% A multiplicative watermark is an affine to a weak encryption technique, through which the signal is ``encoded'' by a filter, changing its dynamic behavior. The use of inverse pairs means that the original signal can be recovered, through ``decoding'' via an inverse filter. As such, differently to techniques based on additive watermarking, no performance is lost due to the injection of noise, and there are no bandwidth limitations.

%\subsection{Contributions}
One of the critical features of multiplicative watermarking is that to detect stealthy attacks, the mWM filter parameters must be switched over time. In this paper, an algorithm to optimally design the mWM parameters after a switching event is presented, enhancing detection performance, without changing the switching time.
% This is done without changing the switching time, which is taken as given.

\textcolor{black}{
To formalize the filter design problem, we suppose the defender is interested in optimal performance against adversaries injecting covert attacks with matched system parameters \citep{smith2015covert}, including the mWM parameters prior to the switch. This scenario represents a worst case where malicious agents can take full control of the system while remaining undetected.
Thus, the attack strategy is explicitly included within the formulation of the closed-loop system, and the mWM filters are chosen by solving an optimization problem minimizing the attack-energy-constrained output-to-output gain (AEC-OOG) \citep{anand2023risk}, a variation of the output-to-output gain proposed in  \cite{teixeira2015strategic}.
}
The main contributions of this paper are:
% We consider an adversary injecting a covert attack with matched system parameters \citep{smith2015covert}, i.e., an attacker with full knowledge of the control system parameters, including those of the mWM filters before the switch. This scenario is taken as a worst case, as it has been shown that this class of attacks can be made stealthy. To quantitatively define a cost, the output-to-output gain (OOG) \citep{teixeira2015strategic} is leveraged,
% a metric introduced to evaluate the impact of an additive attack in a control system. %Specifically, OOG evaluates the worst-case performance loss that an attacker injecting an undetectable attack can obtain. 
% Here, the maximum performance loss caused by a stealthy adversary with limited energy is taken, the attack-energy-constrained OOG (AEC-OOG) \citep{anand2023risk}. The main contributions of this paper are:
\begin{enumerate}
%[label=\alph*.]
\item The problem of optimally designing the switching mWM filters is formulated as an optimization problem, with the AEC-OOG is taken as the objective;%where the AEC-OOG is taken as the impact metric; 
\item The worst-case scenario of a covert attack with exact knowledge of plant and mWM filter parameters is embedded within the design problem;
% The optimization problem is defined to incorporate the worst-case scenario of a covert attack with exact knowledge of plant and mWM filter parameters;
\item The feasibility of the optimization problem is shown to be dependent only on stability conditions; 
\item A solution scheme is proposed to promote randomization of the mWM filter parameters such that an eavesdropping adversary cannot remain stealthy.
\end{enumerate} 

This builds on the results of \cite{ferrari2020switching}, where the focus was on the design of the switching protocols, rather than the parameters themselves.
Compared to previous work \citep{gallo2021design}, this paper introduces an optimization problem which is always feasible (thanks to the use of AEC-OOG in the objective), while also considering a more sophisticated class of covert attacks, where the presence of watermark is known to the adversary. 
Moreover, this paper poses a different objective than \citep{zhang2023hybrid}; indeed, while \citep{zhang2023hybrid} provided a design strategy to ensure certain privacy properties, in this paper we address the problem of optimal parameter design following a switching event.


%\subsection{Organization}
The rest of the paper is organized as follows. 
After formulating the problem in Section~\ref{sec:PF}, we propose our design algorithm in Section~\ref{sec:main}, and analyze its properties. It is then evaluated through a numerical example in Section~\ref{sec:NE}, and concluding remarks are given Section~\ref{sec:Con}.
% We provide the problem background in Section~\ref{sec:PF}. We formulate the design problem in Section~\ref{sec:main}, together with an analysis of its properties. The proposed algorithm is evaluated through a numerical example in Section \ref{sec:NE}. Concluding remarks are offered in Section \ref{sec:Con}.
\section{Related Work}
\subsection{Multimodal Large Language Models}
% Building on the success of large language models (LLMs) \citep{yao2024tree, glm2024chatglm, achiam2023gpt, touvron2023llama, brown2020language}, multimodal large language models (MLLMs) \citep{liu2024improved, li2023blip, zhu2023minigpt, wang2023cogvlm, liu2024visual} extend these capabilities by integrating vision and text processing, achieving remarkable performance in tasks involving images, videos, and multimodal reasoning. However, handling visual data poses computational challenges due to the redundancy and low information density of high-resolution tokens \citep{liang2022evit} and the quadratic scaling of attention mechanisms \citep{vaswani2017attention}.
% For instance, models like LLaVA \citep{liu2023improvedllava} and mini-Gemini-HD \citep{li2024mini} encode high-resolution images into thousands of tokens, while video-based models such as VideoLLaVA \citep{lin2023video} and VideoPoet \citep{kondratyuk2023videopoet} allocate even more tokens to process multiple frames. These challenges highlight the need for more efficient token representations and longer context lengths to enable scalability. Recent advancements, such as Gemini \citep{geminiteam2023gemini} and LWM \citep{liu2024world}, have focused on addressing these issues by optimizing token efficiency and extending the context length, paving the way for more scalable and effective MLLMs.

The remarkable success of large language models (LLMs) \citep{radford2019language, brown2020language} has spurred a growing trend of extending their advanced reasoning capabilities to multi-modal tasks, leading to the development of vision-language models (VLMs) \citep{huang2023languageneedaligningperception, driess2023palmeembodiedmultimodallanguage, liu2024visual, Qwen-VL}. These VLMs typically consist of a visual encoder \citep{radford2021learning} that serializes input image representations and an LLM responsible for text generation. To enable the LLM to process visual inputs, an alignment module is employed to bridge the gap between visual and textual modalities. This module can take various forms, such as a simple linear layer, an MLP projector, or a more complex query-based network. While this integration allows the LLM to gain visual perception, it also introduces significant computational challenges due to the long sequences of visual tokens.

Moreover, existing VLMs often exhibit limitations, such as visual shortcomings or hallucinations, which hinder their performance. Efforts to enhance VLM capabilities by increasing input image resolution have further exacerbated computational demands. For instance, encoding higher-resolution images results in a substantial increase in the number of visual tokens. A model like LLaVA-1.5 \citep{liu2024improved} generates 576 visual tokens for a single image, while its successor, LLaVA-NeXT \citep{liu2024llavanext}, produces up to 2880 tokens at double the resolution, far exceeding the length of typical textual prompts.
Optimizing the inference efficiency of VLMs is thus a critical task to facilitate their deployment in real-world scenarios with limited computational resources.

\subsection{Visual Token Compression}
% Visual tokens often exceed text tokens by tens to hundreds of times, with visual signals being more spatially redundant compared to information dense text \citep{marr2010vision}.
% Various methods have been proposed to address this issue. For instance, LLaMA-VID \citep{li2023llama} uses a Q-Former with context tokens, and DeCo \citep{yao2024deco} applies adaptive pooling to downsample visual tokens at the patch level.
% However, these approaches require modifying model components and additional training, increasing computational and training costs.
% ToMe~\citep{bolya2022tome} reduces tokens without training by adding a token merge module to ViTs, but this disrupts early cross-modal interactions in language models~\citep{xing2024PyramidDrop}. FastV~\citep{chen2024image} selects important visual tokens using attention scores, while SparseVLM~\citep{zhang2024sparsevlm} incorporates text guidance via cross-modal attention.
% However, these methods forgo flash-attention~\citep{dao2022flashattention, dao2023flashattention2} and primarily focus on token importance, overlooking the impact of token duplication.
% In our work, we preserve hardware acceleration compatibility, including flash attention, while considering both token importance and duplication for token reduction.

Visual tokens are often significantly more numerous than text tokens, with higher spatial redundancy and lower information density. To address this issue, various methods have been proposed for reducing visual token counts in vision language models. For instance, some approaches modify model components, such as using context tokens in Q-Former \citep{li2023llama} or applying adaptive pooling at the patch level, but these typically require additional training and increase computational costs. Other techniques, like Token Merging (ToMe) \citep{bolya2022tome} and FastV \citep{chen2024image}, focus on reducing tokens without retraining by merging tokens or selecting important ones based on attention scores. SparseVLM \cite{zhang2024sparsevlm} incorporates text guidance through cross-modal attention to refine token selection. However, these methods often overlook hardware acceleration compatibility and fail to account for token duplication alongside token importance. Furthermore, while token pruning has been extensively explored in natural language processing and computer vision to improve inference efficiency, its application to VLMs remains under-explored. Existing pruning strategies, such as those in FastV and SparseVLM, rely on text-visual attention within large language models (LLMs) to evaluate token importance, which may not align well with actual visual token relevance.


\section{Experiment Setup}
% As it described in Figure~\ref{fig:webpage}, we conducted an online user study on in-car virtual natural environments (VRE), developed as part of a separate project [anonymity], to explore the potential of speech analysis for UX measurement. By analyzing post-VRE user interviews, we evaluated the restorative effects and demonstrated the feasibility of using speech analysis for UX evaluation.
As shown in Figure~\ref{fig:webpage}, we conducted an online user study to explore users' strategies for responding to different emotional contexts. By analyzing participants' voices and the content of their interactions with our one-dialog simulated VAs, we evaluated emotional responding differences, speech features, and language features using speech analysis and natural language processing (NLP) techniques. This approach allowed us to gain insights into how users naturally adapt their responses to emotional cues, providing valuable data for improving emotion-aware systems.

\subsection{Emotional Scenarios}
% In our study, based on five basic emotions~\cite{ortony1990,sortony2022all,ma2022emotion},we designed five different emotional scenarios, including neutral, happy, sad, angry, and fear~\cite{ma2022should}. When these emotion states were confirmed, we then have a brainstorming about which emotional scenarios or context can be suitable for users. In the other words, these emotional contexts may be come out in our daily life and can be identified by general users. Fianlly, we confirmed the follwing five different emotional scenarios:
In our study, we designed five distinct emotional scenarios based on the five basic emotions — neutral, happy, sad, angry, and fear~\cite{ortony1990s,ortony2022all,ma2022emotion}. These scenarios were crafted to reflect situations that users might encounter in their daily lives, ensuring they were relatable and easily identifiable by general users. After confirming the emotional states, we conducted brainstorming sessions to determine which scenarios would best represent these emotions in a way that felt authentic and meaningful. The final five emotional scenarios are as follows:

\begin{itemize}
    \item \textbf{Neutral:} I put on my shoes before leaving the house.
    \item \textbf{Happy:} I am visiting my favorite country or city.
    \item \textbf{Sad:} I see children suffering from disease, sickness, or war.
    \item \textbf{Angry:} I get betrayed by a close friend or relative.
    \item \textbf{Fear:} I am walking in the dark in the woods when I stumble upon a dead body. The blood seems fresh, and I hear a branch breaking from behind.
\end{itemize}

Based on these emotional scenarios, we recorded voice clips for each scenario and integrated them into our designed website. Users visiting the website can listen to these emotional scenario recordings, allowing them to immerse themselves in the context and respond naturally. This interactive setup enables participants to engage with the emotional content and provide authentic responses, which are then recorded and analyzed for further study.

\subsection{Apparatus, Participants and Experimental Procedure}
\subsubsection{Apparatus and Participants}
In this user study, we primarily utilized OpenVokaturi\footnote{\url{https://developers.vokaturi.com/getting-started/overview}} to analyze the emotional states of speech. Additionally, we employed two key speech analysis packages: Librosa\footnote{\url{https://librosa.org/doc/latest/feature.html}} and OpenSMILE\footnote{\url{https://audeering.github.io/opensmile-python/}}, to examine users' speech behaviors in response to emotional scenarios. For analyzing the content of users' responses, we relied on four main NLP packages: WordCloud\footnote{\url{https://amueller.github.io/word_cloud/}}, NLTK\footnote{\url{https://www.nltk.org/}}, TextStatc, and SpaCy\footnote{\url{https://spacy.io/}}. Furthermore, OpenAI Whisper\footnote{\url{https://openai.com/index/whisper/}} was used as a critical tool for converting speech into text, enabling detailed analysis of both linguistic and emotional features. Participants' voices were recorded using the built-in microphones of their own computers and uploaded to our web server when they conducted our user study on our website. All recorded voice signals were captured at a sample rate of 48 kHz, ensuring high-quality audio for subsequent analysis.
Additionally, we recruited 60 participants through Prolific\footnote{\url{https://www.prolific.com/}}, comprising 30 males and 30 females. To account for potential gender-based differences in voice perception, we conducted two separate user studies: one using a male emotional voice and the other using a female emotional voice for the five emotional scenarios. The male emotional voice had a mean age of 32.29 years (SD = 9.71), while the female emotional voice had a mean age of 27.17 years (SD = 5.02). This approach ensured a balanced and comprehensive evaluation of emotional responses across different gender contexts.
After completing the user study, each participant was compensated with £4.50 through the Prolific website. Participants were assured of complete anonymity, as clearly stated in the website's privacy policy. This policy also detailed the types of data collected during the study, which included voice recordings and responses to demographic questions, such as age and gender. All measures were taken to ensure the confidentiality and privacy of the participants' information.

\subsubsection{Experimental Procedure}
% As aforementioned, we recruited the participants through the Prolific website and participants can visit our designed website as it shown in Figure~\ref{fig:webpage} through our link. The landing page presents the study in general and guide instructions and participants can click the smiley and it becomes yellow. Meanwhile, a emotional scenarios voice will be played. Participants can respond anything to this emotional voice as they like. The microphone will be recorded their voice when they click the "Start Recording". They will also have 20 seconds to say anything to talk with the smiley or try their best to comfort smiley if they meet negative emotions, such as angry, fear and sad. In this study, they will conduct the five different emotional scenarios and male or female voice will be randomly selected during the study. In order to keep the gender and participants balance, 15 male and female parcipants will try the male five different emotional scenarios and another 30 parcipants will conduct the female different emotional scenarios. After this five different emotional scenarios, the main user study will be completed and then they will fill relevant questionnaire after this one. 
As mentioned earlier, we recruited participants through the Prolific website, and they accessed our designed website (shown in Figure\ref{fig:webpage}) via a provided link. The landing page provided an overview of the study, along with detailed instructions. Participants could click on the smiley emoji, which would turn yellow and play an emotional scenario voice. They were then free to respond in any way they liked. When participants clicked the "\textit{Start Recording}" button, their microphone would begin recording their voice. They were given 20 seconds to speak, during which they could engage in conversation with the smiley or attempt to comfort it, especially in cases of negative emotions such as anger, fear, or sadness.
In this study, participants engaged with five different emotional scenarios, with either a male or female emotional voice randomly selected for each scenario. To ensure gender balance, 15 male and 15 female participants interacted with the male emotional voice scenarios, while the remaining 30 participants interacted with the female emotional voice scenarios. After completing all five emotional scenarios, participants finished the main user study and proceeded to fill out a relevant questionnaire.



\subsection{Speech Signals Analysis}
% The collected speech data from participants were downloaded from the website server and then classified based on different emotional scenarios based on audio file naming. Then we extracted the speech features using the aforementioned  speech processing python package such as librosa and opensmile. Moreover, to analyze speech contents, we converted the audio data into text data using the OpenAI-whisper and then based on NLP technologies, we extracted the text features, such as lexical features, semantic features and syntactic features, etc. Moreover, we used Openvokatuti to obtain parcipants' speech emotion states, namely, which emotional responding they used to communicating with our VAs.
The speech data collected from participants were downloaded from the website server and systematically classified according to different emotional scenarios, based on the naming conventions of the audio files. To extract relevant speech features, we utilized advanced speech processing Python packages, such as Librosa and OpenSMILE~\cite{eyben2010opensmile}. Additionally, to analyze the content of the speech, we transcribed the audio data into text using OpenAI Whisper model. Leveraging natural language processing (NLP) techniques, we then extracted a variety of text features, including lexical, semantic, and syntactic features. Furthermore, we employed Openvokatuti to identify and analyze participants' emotional states during their interactions, specifically determining which emotional responses they exhibited while communicating with our VAs. This comprehensive approach allowed us to gain deeper insights into both the acoustic and linguistic aspects of the participants' interactions.


% The speech data collected from participants were downloaded from the website server and systematically classified into different emotional scenarios based on the naming conventions of the audio files.
% To extract relevant speech features, we utilized advanced speech processing Python libraries, including Librosa and OpenSMILE~\cite{eyben2010opensmile}. For speech content analysis, the audio recordings were transcribed into text format using OpenAI’s Whisper model. Leveraging Natural Language Processing (NLP) techniques, we extracted a range of textual features, including lexical, semantic, and syntactic attributes, to gain deeper insights into linguistic variations across emotional contexts.
% Additionally, we employed OpenVokaturi to analyze participants’ emotional states, identifying the specific emotional responses exhibited during their interactions with our voice assistants (VAs). This comprehensive multimodal approach enabled us to capture both acoustic and linguistic dimensions of participant interactions, facilitating a more nuanced understanding of human emotional expression in AI-mediated communication.



% \begin{figure*}[htpb!]
% \label{}
% \centering

%     {{\label{ROCIowaCedar} \includegraphics[width=\textwidth/3]{figures/IowaCedar_roc.png}}}%
%     \qquad
%     {{\label{ROCIowaDesMoines} \includegraphics[width=\textwidth/3]{figures/IowaDesMoines_roc.png} }%
%   \captionsetup{justification=centering}
%   \caption{\Acf{ROC} curves for \acf{RW} Iowa (CR) and  \acf{RW} Iowa (DM) dataset. Dummy model here represents a model whose output is solely a ``no Flood'' for all pixels.}
%   \label{fig:RW_ROC_Curves}%
% \end{figure*}



\section{Results and Discussions}
\label{sec:Results}

In this section, we aim to answer three main questions. First, we want to validate our hypothesis that \ac{SYN} data is a viable proxy for \ac{RW} data when training ML models for downscaling. Secondly, we seek to assess how much more skillful ML-based downscaling is compared to classical, non-data-driven techniques, such as our baseline methods, \textit{i.e.}, thresholded bicubic and Lanczos interpolation. Finally, we would like to appraise the extent to which data-driven models like ours are transferable (in terms of usefulness) to other regions without major performance degradations.  
To assess the quality of the models, we conduct a multiple comparison test --namely the Holm-Bonferroni procedure \cite{HolmBonferroni1979} -- that is designed to control the \ac{FWER}. We notice that, with a \ac{FWER} of $10^{-3}$, all the differences in model performance are significant. The only exception to this trend was observed in \ac{RW}-GH for whom the pairwise differences between \ac{RCAN} and \ac{ESRT}, Lanczos and Bicubic were not significant with the aforementioned \ac{FWER}. 

%Finally, we aim to find out the factors influencing the transferability of our models from one region to another.

\subsection{Potential of using SYN Data for RW downscaling}

In order to evaluate the utility of synthetic data for training, we compare performances of our candidate models on both \ac{SYN} and \ac{RW} Iowa data whose results are presented in Table \ref{tab:IowaResults}. We notice that 
\textbf{(i)} For the Iowa datasets, there is a drop in performance of all the models when going from \ac{SYN} to \ac{RW} datasets, 
\textbf{(ii)} for the \ac{RW}-IA (CR) as well as \ac{RW}-IA (DM) datasets, both bicubic and Lanczos interpolation have accuracies and MCC up to 70.89\% and 0.42 respectively while the deep learning models have accuracies and MCC up to 73.34\% and 0.46 respectively, 
\textbf{(iii)} There is a roughly 6\% accuracy improvement for the \ac{SYN} data for the deep learning models compared to the bicubic and lanczos models and this improvement drops to about 3\% for \ac{RW} data,  
\textbf{(iv)} the performance of all the models remain consistent across both \ac{RW}-IA datasets and \textbf{(v)} in \figref{fig:RW_ROC_Curves}, we observe that there is a high degree of overlap among the \ac{ROC} curves for the data-driven models.

From (i) and (iv) we can conclude that \ac{SYN} data is more intricate than \ac{RW} data. This implies that the benefits yielded by training with \ac{SYN} dataset, while significant, is not as prominent in the \ac{RW} Iowa datasets. 
% This may be due to sensor noise prevalent in the \ac{RW} Landsat-8 data that can be harder to reproduce in the synthetically generated examples. 
(i), (iii) and (v) implies that while \ac{SYN} data is not an exact replacement for \ac{RW} data, it provides a rather significant edge, which is all the more important when there is insufficient \ac{RW} for training. From (ii) we can conclude that the three proposed data driven models outperform classical super-resolution techniques such as bicubic and lanczos, conclusion supported by the \ac{ROC} curves in Figure \ref{fig:RW_ROC_Curves} for whom the data-driven models, in general, lie above the non-data-driven alternatives. Observation (iv) shows that  for the climatically similar \ac{RW}-Iowa(CR) and \ac{RW}-Iowa(DM) regions, training on \ac{SYN} Iowa data does indeed provide an edge. 

% have similar climate. 

\begin{figure*}[t!]
    \centering
    \begin{subfigure}[t]{0.5\textwidth}
        \centering
        \includegraphics[width=\textwidth/2]{figures/IowaCedar_roc.png}
        \caption{}
    \end{subfigure}%
    ~ 
    \begin{subfigure}[t]{0.5\textwidth}
        \centering
        \includegraphics[width=\textwidth/2]{figures/IowaDesMoines_roc.png}
        \caption{}
    \end{subfigure}
    \vspace*{0.5cm}
    \caption{    \label{fig:RW_ROC_Curves} \Acf{ROC} curves for (a) RW-IA (CR) and (b) RW-IA (DM) dataset. Na\"ive model here represents a model whose output is solely a ``no Flood'' for all pixels. Star here represents the pixel-wise classifier with a threshold of 0.5.}
\end{figure*}


\subsection{Effectiveness of data-driven approaches}

In order to evaluate the effectiveness of ML models in the downscaling task, we compare performances of our candidate models to Lanczos and bicubic interpolation methods by looking at figures of some sample predictions from Iowa (Figure \ref{fig:RWIowaDesMoines}), performance comparison in the region of Iowa in Table \ref{tab:IowaResults} and the ROC curves in Figure \ref{fig:RW_ROC_Curves} for \ac{RW} data. We notice that 
\textbf{(vi)} For RW-IA (DM) samples, the deep learning models maintain a higher degree of spatial continuity in the predicted \ac{FIM}, 
\textbf{(vii)} We observe that  bicubic and Lanczos interpolation produces over-smoothed \ac{FIM} reconstructions, while the plain \ac{RDN}, \ac{RCAN} and \ac{ESRT} models are more detail-inclusive. Similar conclusions can be drawn upon inspecting the \ac{ROC} curves in Figure \ref{fig:RW_ROC_Curves} and 
\textbf{(viii)} For RW-IA (CR), the ML models show a performance improvement of 3.06\% when comparing the best ML model and non-data-driven method and, while for RW-IA (DM) there is a performance improvement of 2.45\%.


Figures \ref{fig:EUSamples} and \ref{fig:RWIowaDesMoines} show the spatial disparity among the models whose details are often obscured in aggregated metrics such as accuracy. (vi) This implies that these data-driven models are better are recognizing an underlying stream network geometry than the classical methods. However, when it comes to narrow river streams, all the models struggle capturing the nuances of the \ac{FIM} resultant from localized high elevation features such as small islands within rivers or man-made structures. (vii) shows a clear advantage of our data-driven approaches over the non-data-driven alternatives. (viii) indicates the benefits of the data-driven models when evaluated over Iowa. 



\subsection{Applicability of our models to external regions}

To evaluate how transferable our models are, we draw conclusions from figures of the sample predictions from Western Europe (Figure \ref{fig:EUSamples}) and Ghana (Figure \ref{fig:GhanaSamples}) as well as the performance comparison in Table \ref{tab:ExternalResults}. We notice that 
\textbf{(ix)} for Ghana all of the models fail to adequately inundate the pixels over separated areas on account of several disconnected regions of inundation in the chosen area,
\textbf{(x)} the ML models outperform non-data driven methods for RW-EU, 
\textbf{(xi)} for the RW-EU dataset, there is an improvement of 4.89\% when comparing the accuracy of the best data- and non-data-driven methods, 
\textbf{(xii)} For RW-RR and RW-GH, there is marginal improvement (up to 0.77\% in accuracy) of the ML methods over the non-data driven methods and 
\textbf{(xiii)} For RW-EU, we notice that the ML models produce more connected streams over the non-data-driven models. 

(x) and (xi) implies that the models are transferable when considering hydroclimaticalogically similar regions since Iowa and the Meuse river in Europe lie within mid temperate zones. Similar to the observation (vi) for RW-IA (DM), (xiii) implies that the benefits of the ML model in identifying underlying network streams is also transferable to hydroclimatologically similar regions. In contrast, (xii) and (ix) both imply that the trained ML models struggle to generalize to RW-RR \& RW-GH. We speculate that this may be due to the significant differences in geography and climate when compared to Iowa. 

% More specifically, we notice that Ghana has a lot of disconnected regions when compared to Iowa and Western Europe, possibly indicating a geomorphological dissimilarity. Additionally, in the case of Red River and Ghana, we also speculate that they include drivers to flood inundation that are different from Iowa and Western Europe, which lie within mild temperate zones. Ghana on the other hand has a tropical (dry and hot) climate.

Our study directly implies that good quality synthetic data can be useful surrogates for downscaling low-resolution \acp{WFM} to high-resolution \acp{FIM} in regions, where such data are hard to come by, even when downscaling by a factor of 10. We noticed that such models were readily transferable to climatically similar regions as the region of training. However, Such derived ML models did not feature significantly different transferability when evaluated over hydroclimatologically dissimilar regions, which we attribute to different flood inundation characteristics, primarily at finer scales. A possible avenue to circumvent such issues is to explore additional training approaches that fall under the general area of domain adaptation. Nevertheless, data-driven models are still advantageous (and, hence, preferable) over non-data-driven alternatives in transfer scenarios like the one we considered here. 


%%%%%%%%%%%%%%%%%%%%%%%%%%%%%%% unused text %%%%%%%%%%%%%%%%%%%%%%%%%%%%%%%%%%%%%%%



% \tabref{tab:AccuracyResults} depicts test accuracies obtained by our models on both \ac{SYN} and \ac{RW} data. For Iowan floods, a comparison of \ac{SYN} and \ac{RW} results shows \textbf{(i)} bicubic and Lanczos interpolations remarkably gaining about $3\%$ in accuracy, as well as \textbf{(ii)} \ac{RDN} and \ac{RCAN} remaining relatively stable, while \textbf{(iii)} topography-aware models loosing $2.7\%$ in performance. From (i) one can conclude that \ac{SYN} data are morphologically slightly more intricate than \ac{RW} data. Also, (i) and (ii) likely imply that \ac{SYN} data, excluding topography, can serve as satisfactory surrogates of \ac{RW} data. However, as implied by (iii), our topography-dependent models seems to be particularly sensitive to distributional shifts of their combined inputs (\acp{WFM} and topographic features). More specifically, the topography-informed models' performance edge, while still statistically significant, is extremely marginal, even when compared to our non-data-driven approaches. Next, when comparing results between the cases of Iowan and Ghanaian \ac{RW} data, one observes that \textbf{(iv)} the accuracy of bicubic and Lanczos interpolations drops by almost $5\%$ due to over-smoothing. This may imply that Ghanaian \acp{FIM} bare a more complex morphology, when compared to Iowan \acp{FIM}. Also, \textbf{(v)} our topography-agnostic, data-driven models' performance degrades more gracefully (by about $2\%$), while \textbf{(vi)} our topography-aware models perform, virtually, as bad as our non-data-driven approaches. Hence, the differences in the data populations of the two regions we considered are significant enough to render our topography-dependent models noncompetitive. 



\section{Discussion and Future Work}\label{sec:discussion}
This paper pioneers the novel approach of selective response, showing that withholding responses can be a powerful tool for GenAI systems. By opting not to answer every query as accurately as it can---particularly when new or complex topics emerge---GenAI can encourage user participation on community-driven platforms and thereby generate more high-quality data for future training. This mechanism ultimately enhances GenAI's long-term performance and revenue. From a welfare perspective, our results indicate that such selective engagement can also benefit users, leading to better solutions and increased overall satisfaction. Since this work is the first to address selective response strategies for GenAI, numerous promising directions remain for future research; we highlight some of them below. 

First, from a technical standpoint, all of the results in this paper rely on Assumption~\ref{assumption: data lip}, involving the lipshitz condition of the accuracy function and the sensitivity parameter $\beta$. Future work could seek to relax this assumption. Furthermore, our constrained optimization approach in Subsection~\ref{sec: welfare constrained revenue maximization} could be extended to approximate the optimal (continuous) strategy instead of the optimal discrete strategy.

Second, our stylized model adopts the simplifying---though unrealistic---assumption that only a single GenAI platform exists. Admittedly, this makes it easier to focus on the idea of selective responses, and indeed, this assumption is pivotal in keeping our analysis tractable. Future research could explore scenarios with multiple GenAI platforms and human-centered forums. In such settings, one platform's selective response might redirect users not only to forums but also to competing GenAI platforms, leading to the tragedy of the commons \cite{hardin1968tragedy}: Although all GenAI platforms benefit from fresh data generation, none may choose to respond selectively if it means losing users to competitors. 

Third, we assumed Forum behaves non-strategically. In reality, human-centered platforms often monetize their data by selling it to GenAI platforms, adding a further layer of strategic interaction for GenAI. Moreover, data transfer between the platforms can form the basis for collaboration: GenAI could employ selective response to bolster Forum content creation, and Forum could, in turn, attribute that content to GenAI for subsequent use in retraining.


%Third, we make the (again) simplifying assumption that Forum is non-strategic. However, in practice, human-centered platforms can sell their data to GenAI platforms. This adds additional considerations for GenAI. Furthermore, data transmission between the platforms can also become the basis for collaboration: GenAI can use selective response to ensure enough content is generated in Forum, and Forum could provide the data attributed to this mechanism back to GenAI. 


%Second, this paper makes the simplifying yet unrealistic assumption of the existence of one GenAI platform. Indeed, this simplifies many aspects and allows us to analyze selective responses. Future work could address the data generation process with more than one GenAI platform and possibly several human-centered forums. In such a case, selective response of one GenAI platform can either drive users to forums or to other GenAI platforms; thus, we might face a tragedy of the commons situation~\ref{hardin1968tragedy}, where all GenAI platforms are interested in fresh data generation but none volunteer to selectively respond and lose users. 

%This paper examines the competition between a generative AI platform and human-based platforms, challenging the assumption that always providing answers is optimal. We analyzed the impact of withholding answers on GenAI's revenue and developed an efficient approximately optimal algorithm for this purpose. We further explored how withholding affects users, showing that it can lead to better outcomes compared to always answering. Specifically, we demonstrated that withholding can Pareto-dominate this strategy and derived the necessary and sufficient conditions for that. Finally, we proposed a second approximately optimal algorithm that maximizes GenAI's revenue while ensuring users are better off than when GenAI answers all queries.

%On a more conceptual level, our model assumes that GenAI’s data comes solely from the competing platform (Forum). Future research could explore a scenario where GenAI can purchase additional data from a third party. This extension could provide valuable insights into the interplay between withholding answers and data purchasing, and whether these two strategies can complement each other or must be traded off.
Software development is increasingly conceived as a collaboration activity between developers and AIs. Indeed, IDEs already implement features to enable interactive development, with AI suggesting implementations that are reused by developers.

Although multiple studies show this interaction can be successful, there is still limited understanding of how the models must be configured and used in the context of code generation tasks. This study addresses this gap, systematically investigating the impact of several key parameters, including the repeated submission of a prompt to accommodate for the non-deterministic nature of the models.

Our study reveals several key findings about the usage of ChatGPT. In particular, we discovered how creativity, although up to a limited extent, is useful to increase the range of methods whose code can be generated correctly. A major role is played by parameter top-p, which is commonly underrated, and instead has a major impact on the correctness of the results, with lower values producing better results. Finally, prompts should be submitted multiple times, with $5$ repetitions combined with a temperature of $1.2$ resulting in an effective configuration in our experiments.  

Future work concerns two main research directions. One is about replicating this experiment with other AI assistants, to validate our findings in multiple contexts. The second research direction concerns finding strategies to deal with the need to submit the same prompt multiple times to obtain a useful result, and thus developing approaches able to select or merge multiple responses automatically. 

\begin{credits}
% \subsubsection{\ackname} The authors would like to thank our anonymous reviewers for their constructive comments and suggestions.  Additionally, this work acknowledges funding from the Research Council of Norway for project no. 326907. Moreover, this work was also supported by the Swedish Foundation for Strategic Research (SSF) grant FU21-0067.
\subsubsection{\ackname} The authors extend their sincere gratitude to the anonymous reviewers for their valuable and constructive feedback. This research was funded by the Research Council of Norway under project number 326907. Additionally, support was provided by the Swedish Foundation for Strategic Research (SSF) through grant FUS21-0067.

% \subsubsection{\discintname}
% It is now necessary to declare any competing interests or to specifically
% state that the authors have no competing interests. Please place the
% statement with a bold run-in heading in small font size beneath the
% (optional) acknowledgments\footnote{If EquinOCS, our proceedings submission
% system, is used, then the disclaimer can be provided directly in the system.},
% for example: The authors have no competing interests to declare that are
% relevant to the content of this article. Or: Author A has received research
% grants from Company W. Author B has received a speaker honorarium from
% Company X and owns stock in Company Y. Author C is a member of committee Z.
\end{credits}
%
% ---- Bibliography ----
%
% BibTeX users should specify bibliography style 'splncs04'.
% References will then be sorted and formatted in the correct style.
%
\bibliographystyle{splncs04}
% \bibliographystyle{plainnat}
\bibliography{mybibliography}
%
% \begin{thebibliography}{8}
% \bibitem{ref_article1}
% Author, F.: Article title. Journal \textbf{2}(5), 99--110 (2016)

% \bibitem{ref_lncs1}
% Author, F., Author, S.: Title of a proceedings paper. In: Editor,
% F., Editor, S. (eds.) CONFERENCE 2016, LNCS, vol. 9999, pp. 1--13.
% Springer, Heidelberg (2016). \doi{10.10007/1234567890}

% \bibitem{ref_book1}
% Author, F., Author, S., Author, T.: Book title. 2nd edn. Publisher,
% Location (1999)

% \bibitem{ref_proc1}
% Author, A.-B.: Contribution title. In: 9th International Proceedings
% on Proceedings, pp. 1--2. Publisher, Location (2010)

% \bibitem{ref_url1}
% LNCS Homepage, \url{http://www.springer.com/lncs}, last accessed 2023/10/25
% \end{thebibliography}
\end{document}
