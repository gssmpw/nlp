\section{Structure-Dynamics-Informed Vibration Modeling For Reduced Data Need}
\label{sec:Structure-dynamics-informed modeling}

\begin{figure*}[t]
% \setlength{\abovecaptionskip}{10pt}
    \centering
    \includegraphics[width=0.8\linewidth]{Figure/Experiment_Setup.jpg}
    \caption{(a) shows the store gondola and active vibration sensing setup with a speaker next to the gondola. The top right signal clip shows an example of our given vibration signal. (b) gives a more detailed view of the item location and sensor location. The different weight classes are taken by changing the amount of water in the 1L water bottle. (c) provides an overview of our vibration sensing module.}
    \label{fig:Experiment Setup}
\end{figure*}

WeVibe leverages structural dynamics knowledge to characterize the physics-informed relationship between shelf vibration and item weight change. It is observed that the item weight change and the vibration frequency spectrum follow the same increasing or decreasing trend, as indicated in Figure~\ref{fig:frequency difference}. This observation inspires a linear relationship assumption, which is justified by the derivation according to the structural dynamics. In addition, WeVibe can quickly adapt to multiple locations with several weight classes based on the observation that the linearity is kept at various locations, though the coefficients are different.

Section~\ref{sec:empirical study} will first illustrate our empirical observations, leading us to the assumption of linearity across locations. Then, Section~\ref{sec:theory} validates our observations through the theoretical model of the interaction between the vibration frequency spectrum, item weight, and item location. It points out the hidden linearity in the theoretical formulation. Section~\ref{sec: Feature Extraction and Learning} provides more detail on physics-informed feature extraction.


\subsection{Empirical Study On The Relationship Between Shelf Vibration and Item Weight}
\label{sec:empirical study}
As Figure~\ref{fig:frequency difference} (a) shows, the vibration frequency spectrums differ when various item weights and locations are presented. For the same location, the vibration frequency spectrum shows minor differences between the four weights. However, as highlighted by the red circle, these frequencies tend to increase or decrease as the weight increases monotonically. Therefore, the linear regression is applied to these frequencies and more weights to visualize their correlations, as shown in Figure~\ref{fig:frequency difference} (b). Even though some points deviate from the fitted line, the result suggests a possible linearity between the weight and these mentioned frequencies, which might exist at both locations. 

On the other hand, the various item locations lead to a more distinct change in the vibration frequency spectrums compared with the item weight. When the item location changes, the weight-sensitive frequencies shift to somewhere else, and the overall frequency spectrum envelope also significantly changes. These factors result in the different linear relationships shown in Figure~\ref{fig:frequency difference} (b). With these observations, we assumed that the vibration frequency spectrum is linearly related to the item weight across multiple locations on the shelf under our active vibration sensing environment.

\subsection{Theoretical Derivation On The Relationship Between Shelf Vibration and Item Weight}
\label{sec:theory}
%To develop the features and model for item weight estimation,
Using the structural dynamics theory, we characterize the relationship between shelf vibrations and item weight placed on the shelf to validate our assumption. In our analysis, the shelf is modeled as a thin and homogeneous plate with the length $a$ and width $b$, assuming that its thickness is much smaller than its length and width (see Figure~\ref{fig:Theory}), which is commonly valid for the steel shelf in the retail stores. The item is modeled as a point load on the shelf at location $(x_0, y_0)$. Other assumptions include the simply supported boundary and ignorable damping effect. The sensor's location is represented by $(x,y)$.
% Figure \ref{fig:Theory} illustrates a simplification of the shelf structure, and we will develop the theoretical model according to this settings. There are two assumptions we made: (1) the shelf is homogeneous with 4 edges simply supported, (2) the item size is much smaller than the plate so it can be modeled as a point. The shelf has a length of a and width of b. The object weighted $m_0$ is placed at location $(x_0,y_0)$. A vibration sensor is placed at location $(x,y)$ to collect vibration signal.
\begin{figure}[tbh]
% \setlength{\abovecaptionskip}{10pt}
    \centering
    \includegraphics[width=\linewidth]{Figure/Theory_Setting.jpg}
    \caption{The simplified shelf model for developing the theoretical model.}
    \label{fig:Theory}
\end{figure}

Based on the Kirchhoff–Love plate theory that describes the behavior of thin plates subjected to forces and moments, the governing equation of the plate vibration can be formulated as Equation ~\ref{eq:gov_eq}, ignoring the damping. In this equation, $D, \rho, \nu$ correspondingly represent the flexural rigidity, mass per unit area, and Poisson ratio of the shelf. $f(x, y, t)$ is the excitation source exerted at location $(x, y)$ at time $t$. $w(x,y,t)$ represents the shelf vertical displacement, i.e., the vibration at location $(x, y)$. Solving Equation ~\ref{eq:gov_eq} based on the simply supported boundary and static initial condition, $w(x, y, \omega)$, the Fourier transform of the time-domain vibration signal, is proportional to its spatial Fourier transform coefficient $\bar{W}(m,n,\omega)$ which is linear to the item weight $m_0$, as shown in Equation ~\ref{eq:solution}. $\bar{F}(m,n,\omega)$ is the 2d spatial Fourier transform of the excitation force $f(x, y, \omega)$ at the sensor location $(x,y)$. $\omega_{mn}$ is a function of $m$ and $n$. When the item's weight is much smaller than the mass per unit area of the shelf, we can approximate the equation further with Taylor expansion as shown in ~\ref{eq:taylor expansion}. In this situation, \textbf{the vibration frequency spectrum is linear to the item's weight.}

% Equation \ref{eq1} and \ref{eq2} together suggest the relationship between the vibration frequency spectrum and object weight originated from the Euler-Bernoulli Beam Theory. In the equation \ref{eq1}, $w(x,y,\omega)$ represents the amplitude of vibration frequency $\omega$ at the sensor location $(x,y)$. $\bar{W}(m,n,\omega)$ is the 2D spatial Fourier transform of the vibration amplitude at sensor location $(x,y)$. The object's mass on the plate is encoded in $\bar{W}(m,n,\omega)$ as shown in equation \ref{eq2}, $\bar{F}(m,n,\omega)$ is the 2d spatial Fourier transform of the excitation force $f(x, y, \omega)$ which represents the time domain Fourier transform of force applied at the sensor location (x, y). $\rho$ is the mass per unit area of the shelf. $D$ is the bending stiffness of the plate. $\omega_{mn}$ is a function of $m$ and $n$.

% Taylor expansion can then transform the $m_{0}$ from the denominator to the numerator, implying the hidden linearity. Considering the fact that most items are much lighter than the shelf, saying $m_0$ is much smaller than weight per unit area of the shelf $\rho$, equation \ref{eq2} can be written as the first two items of the Taylor expansion with respect to $m_0$ as shown in equation \ref{eq3}. Since $\bar{W}(m,n,\omega)$ also keeps a linear relationship with the vibration frequency amplitude $w(x,y,\omega)$, object weight $m_0$ and vibration frequency amplitude $w(x,y,\omega)$ are coupled with a linear model when the locations of object and sensor are fixed. This linearity indicates a much smaller requirement of data collection with a linear machine learning model.

\begin{equation}
D\nabla^4 w + \rho \frac{\partial^2 w}{\partial t^2} = f(x, y, t) + \delta(x-x_0)\delta(y-y_0)(m_0g + m_0\frac{\partial^2 w}{\partial t^2})
\label{eq:gov_eq}
\end{equation}

\begin{equation}
\begin{gathered}
w(x, y, \omega) \propto  \bar{W}(m,n,\omega) \\
= \frac{\bar{F}(m,n,\omega)}{-\omega^2 \left( \rho + \textcolor{red}{m_0} \sin\left( \frac{m x_0 \pi}{a} \right) \sin\left( \frac{n y_0 \pi}{b} \right) \right) + D \omega^2_{mn}}
\label{eq:solution}
\end{gathered}
\end{equation}

\begin{equation}
\begin{gathered}
\approx \frac{\tilde{F}(m, n, \omega)}{-\omega^2 \rho + D \omega_{mn}^2} \left( 1 + \frac{\omega^2 \sin \left( \frac{m x_0 \pi}{a} \right) \sin \left( \frac{n y_0 \pi}{b} \right)}{-\omega^2 \rho + D \omega_{mn}^2} \textcolor{red}{m_0} \right)
\label{eq:taylor expansion}
\end{gathered}
\end{equation}

The vibration response $w(x,y,\omega)$ not only depends on the item weight $m_0$, but also the item location $(x_0, y_0)$ as shown in Equation ~\ref{eq:taylor expansion}. Although this linearity holds true for different item locations, the linear coefficients are different because they depend on $(x_0, y_0)$. Intuitively, this is because different item locations affect different modes of shelf structure. Thus, the shelf vibration response is affected in various ways. To this end, the theoretical derivation validates (1) The shelf vibration frequencies are linearly correlated with the item weight. (2) This linearity is kept across the gondola shelf.


\subsection{Physics-Informed Feature Extraction and Learning}
\label{sec: Feature Extraction and Learning}

\begin{figure*}[t]
% \setlength{\abovecaptionskip}{10pt}
    \centering
    \includegraphics[width=\linewidth]{Figure/Evaluation1.jpg}
    \caption{The WeVibe system evaluation. (a) The comparison between WeVibe and the other two methods: Using one linear model for all locations and using the non-linear model for each location while using 10\% of all weight classes in training and one vibration sensor. WeVibe outperforms both approaches with a significant improvement. (b)\&(c) The weight change estimation result of WeVibe, taking 10\% of 3 weight classes in training and one vibration sensor. The result suggests that WeVibe can almost certainly distinguish weight changes bigger than 100g, which can be utilized to detect whether a can of chips or a bottle of water is taken or put back.}
    \label{fig:Evaluation1}
\end{figure*}

With theoretical derivation and empirical study, WeVibe builds a solid physics-informed feature extraction and learning model. WeVibe takes the shelf vibration response from one whole impulse vibration as one sample so that the feature will contain the information of both transient and steady state~\cite{huang2020vibration}. WeVibe then takes the vibration frequency spectrum through the Fourier transform. As Figure~\ref{fig:frequency difference} suggests, the shift of item location leads to the change of weight-sensitive frequencies, which is challenging to know beforehand. Therefore, instead of choosing one or two specific weight-sensitive frequencies, WeVibe takes the full range of vibration frequency spectrum except for the noise-occupied region, 50Hz to 240Hz, as the feature for the learning model.
% WeVibe leverages structural dynamics knowledge combined with actual data to alleviate the data requirement for exploring the relationship between shelf response and object weight across different locations. WeVibe exploits the linearity between the vibration frequency spectrum and object weight to train a weight estimation model for known locations. Then, this model will be calibrated if a new object location is involved.

% The calibration leverages that the linearity is kept across most locations on the same shelf, but it has different coefficients. In other words, we assume that the actual weight change estimation at a new location deviates from the well-trained model's result with a constant ratio. Therefore, we can take a few weight change estimation results, calculate this constant ratio, and multiply it by the well-trained model's result, getting the final weight change estimation.

% \begin{equation}
% w(x, y, \omega) = \frac{4}{ab} \sum_{m=1}^{\infty} \sum_{n=1}^{\infty} \bar{W}(m, n, \omega) \sin\left(\frac{m \pi x}{a}\right) \sin\left(\frac{n \pi y}{b}\right)
% \label{eq1}
% \end{equation}


% \begin{equation}
% \bar{W}(m,n,\omega) = \frac{\bar{F}(m,n,\omega)}{-\omega^2 \left( \rho + \textcolor{red}{m_0} \sin\left( \frac{m x_0 \pi}{a} \right) \sin\left( \frac{n y_0 \pi}{b} \right) \right) + D \omega^2_{mn}}
% \label{eq2}
% \end{equation}


% \begin{equation}
% \approx \frac{\tilde{F}(m, n, \omega)}{-\omega^2 \rho + D \omega_{mn}^2} \left( 1 + \frac{\omega^2 \sin \left( \frac{m x_0 \pi}{a} \right) \sin \left( \frac{n y_0 \pi}{b} \right)}{-\omega^2 \rho + D \omega_{mn}^2} \textcolor{red}{m_0} \right)
% \label{eq3}
% \end{equation}

