%%
%% This is file `sample-sigconf-authordraft.tex',
%% generated with the docstrip utility.
%%
%% The original source files were:
%%
%% samples.dtx  (with options: `all,proceedings,bibtex,authordraft')
%% 
%% IMPORTANT NOTICE:
%% 
%% For the copyright see the source file.
%% 
%% Any modified versions of this file must be renamed
%% with new filenames distinct from sample-sigconf-authordraft.tex.
%% 
%% For distribution of the original source see the terms
%% for copying and modification in the file samples.dtx.
%% 
%% This generated file may be distributed as long as the
%% original source files, as listed above, are part of the
%% same distribution. (The sources need not necessarily be
%% in the same archive or directory.)
%%
%%
%% Commands for TeXCount
%TC:macro \cite [option:text,text]
%TC:macro \citep [option:text,text]
%TC:macro \citet [option:text,text]
%TC:envir table 0 1
%TC:envir table* 0 1
%TC:envir tabular [ignore] word
%TC:envir displaymath 0 word
%TC:envir math 0 word
%TC:envir comment 0 0
%%
%%
%% The first command in your LaTeX source must be the \documentclass
%% command.
%%
%% For submission and review of your manuscript please change the
%% command to \documentclass[manuscript, screen, review]{acmart}.
%%
%% When submitting camera ready or to TAPS, please change the command
%% to \documentclass[sigconf]{acmart} or whichever template is required
%% for your publication.
%%
%%
\documentclass[sigconf]{acmart}
\usepackage{enumitem}
%%
%% \BibTeX command to typeset BibTeX logo in the docs
\AtBeginDocument{%
  \providecommand\BibTeX{{%
    Bib\TeX}}}

%% Rights management information.  This information is sent to you
%% when you complete the rights form.  These commands have SAMPLE
%% values in them; it is your responsibility as an author to replace
%% the commands and values with those provided to you when you
%% complete the rights form.
\setcopyright{acmcopyright}
\copyrightyear{2025}
\acmYear{2025}
%\acmDOI{XXXXXXX.XXXXXXX}


%% These commands are for a PROCEEDINGS abstract or paper.
% \acmConference[BuildSys '24]{The 11th ACM International Conference on Systems for Energy-Efficient Buildings, Cities, and Transportation}{November 7--8,
% 2024}{Hangzhou, China}
%%
%%  Uncomment \acmBooktitle if the title of the proceedings is different
%%  from ``Proceedings of ...''!
%%
%%\acmBooktitle{Woodstock '18: ACM Symposium on Neural Gaze Detection,
%%  June 03--05, 2018, Woodstock, NY}
% \acmISBN{978-1-4503-XXXX-X/18/06}


%%
%% Submission ID.
%% Use this when submitting an article to a sponsored event. You'll
%% receive a unique submission ID from the organizers
%% of the event, and this ID should be used as the parameter to this command.
%%\acmSubmissionID{123-A56-BU3}

%%
%% For managing citations, it is recommended to use bibliography
%% files in BibTeX format.
%%
%% You can then either use BibTeX with the ACM-Reference-Format style,
%% or BibLaTeX with the acmnumeric or acmauthoryear sytles, that include
%% support for advanced citation of software artefact from the
%% biblatex-software package, also separately available on CTAN.
%%
%% Look at the sample-*-biblatex.tex files for templates showcasing
%% the biblatex styles.
%%

%%
%% The majority of ACM publications use numbered citations and
%% references.  The command \citestyle{authoryear} switches to the
%% "author year" style.
%%
%% If you are preparing content for an event
%% sponsored by ACM SIGGRAPH, you must use the "author year" style of
%% citations and references.
%% Uncommenting
%% the next command will enable that style.
%%\citestyle{acmauthoryear}


%%
%% end of the preamble, start of the body of the document source.
\begin{document}

%%
%% The "title" command has an optional parameter,
%% allowing the author to define a "short title" to be used in page headers.
\title{WeVibe: Weight Change Estimation Through Audio-Induced Shelf Vibrations In Autonomous Stores}

%%
%% The "author" command and its associated commands are used to define
%% the authors and their affiliations.
%% Of note is the shared affiliation of the first two authors, and the
%% "authornote" and "authornotemark" commands
%% used to denote shared contribution to the research.
\author{\large Jiale Zhang}
\orcid{0000-0003-0688-564X}
\email{jiale@umich.edu}
\affiliation{%
  \institution{University of Michigan}
  \streetaddress{1301 Beal Ave}
  \city{Ann Arbor}
  \state{MI}
  \country{USA}
  \postcode{48109-2122}
}

\author{\large Yuyan Wu}
\orcid{0009-0009-3152-939X}
\email{wuyuyan@stanford.edu}
\affiliation{%
  \institution{Stanford University}
  \streetaddress{}
  \city{Stanford}
  \state{CA}
  \country{USA}
  \postcode{94305}
}



\author{\large Jesse R Codling}
\orcid{0000-0001-8355-7186}
\email{codling@umich.edu}
\affiliation{%
    \institution{University of Michigan}
    \city{Ann Arbor}
    \state{Michigan}
    \country{USA}
}

\author{\large Yen Cheng Chang}
\orcid{0009-0007-1986-9485}
\email{yencheng@umich.edu}
\affiliation{%
    \institution{University of Michigan}
    \city{Ann Arbor}
    \state{Michigan}
    \country{USA}
}

\author{\large Julia Gersey}
\email{gersey@umich.edu}
\affiliation{%
  \institution{University of Michigan}
  \city{Ann Arbor}
  \state{MI}
  \country{USA}}


\author{\large Pei Zhang}
\orcid{0000-0002-8512-1615}
\email{peizhang@umich.edu}
\affiliation{%
  \institution{University of Michigan}
  \streetaddress{1301 Beal Ave}
  \city{Ann Arbor}
  \state{MI}
  \country{USA}
  \postcode{48109-2122}
}

\author{\large Hae Young Noh}
\orcid{0000-0002-7998-3657}
\email{noh@stanford.edu}
\affiliation{%
  \institution{Stanford University}
  \streetaddress{}
  \city{Stanford}
  \state{CA}
  \country{USA}
  \postcode{94305}
}

\author{\large Yiwen Dong}
\orcid{0000-0002-7877-1783}
\email{ywdong@stanford.edu}
\affiliation{%
  \institution{Stanford University}
  \streetaddress{}
  \city{Stanford}
  \state{CA}
  \country{USA}
  \postcode{94305}
}

%%
%% By default, the full list of authors will be used in the page
%% headers. Often, this list is too long, and will overlap
%% other information printed in the page headers. This command allows
%% the author to define a more concise list
%% of authors' names for this purpose.
\renewcommand{\shortauthors}{Jiale et al.}

%%
%% The abstract is a short summary of the work to be presented in the
%% article.
\begin{abstract}
Weight change estimation is crucial in various applications, particularly for detecting pick-up and put-back actions when people interact with the shelf while shopping in autonomous stores. Moreover, accurate weight change estimation allows autonomous stores to automatically identify items being picked up or put back, ensuring precise cost estimation. However, the conventional approach of estimating weight changes requires specialized weight-sensing shelves, which are densely deployed weight scales, incurring intensive sensor consumption and high costs. Prior works explored the vibration-based weight sensing method, but they failed when the location of weight change varies.

In response to these limitations, we made the following contributions: (1) We propose WeVibe, a first item weight change estimation system through active shelf vibration sensing. The main intuition of the system is that the weight placed on the shelf influences the dynamic vibration response of the shelf, thus altering the shelf vibration patterns. (2) We model a physics-informed relationship between the shelf vibration response and item weight across multiple locations on the shelf based on structural dynamics theory. This relationship is linear and allows easy training of a weight estimation model at a new location without heavy data collection. (3) We evaluate our system on a gondola shelf organized as the real-store settings. WeVibe achieved a mean absolute error down to 38.07g and a standard deviation of 31.2g with one sensor and 10\% samples from three weight classes on estimating weight change from 0g to 450g, which can be leveraged for differentiating items with more than 100g differences.
\end{abstract}

%%
%% The code below is generated by the tool at http://dl.acm.org/ccs.cfm.
%% Please copy and paste the code instead of the example below.

\begin{teaserfigure}
\setlength{\abovecaptionskip}{10pt}
    \centering
    \includegraphics[width=\linewidth]{Figure/Teaser.jpg}
    \caption{(A) WeVibe leverages the audio-induced vibration propagating to the shelf to create an active vibration sensing environment, reducing the need for sensors attached to the shelf. (B) The different item weights and locations impact the original shelf vibration response differently, reflected by the vibration signal collected by the sensor. (C) According to structural dynamics, a physics-informed relationship (a linear model) between the shelf vibration response and item weight across different locations is characterized to estimate the item weight change on the shelf with improved data efficacy. (D) Through physics-informed learning, WeVibe can estimate the weight change quickly at a new location with fewer sensors and data.}
    \Description{figure description}
    \label{fig:overview}
\end{teaserfigure}

\maketitle

\section{Introduction}

% \textcolor{red}{Still on working}

% \textcolor{red}{add label for each section}


Robot learning relies on diverse and high-quality data to learn complex behaviors \cite{aldaco2024aloha, wang2024dexcap}.
Recent studies highlight that models trained on datasets with greater complexity and variation in the domain tend to generalize more effectively across broader scenarios \cite{mann2020language, radford2021learning, gao2024efficient}.
% However, creating such diverse datasets in the real world presents significant challenges.
% Modifying physical environments and adjusting robot hardware settings require considerable time, effort, and financial resources.
% In contrast, simulation environments offer a flexible and efficient alternative.
% Simulations allow for the creation and modification of digital environments with a wide range of object shapes, weights, materials, lighting, textures, friction coefficients, and so on to incorporate domain randomization,
% which helps improve the robustness of models when deployed in real-world conditions.
% These environments can be easily adjusted and reset, enabling faster iterations and data collection.
% Additionally, simulations provide the ability to consistently reproduce scenarios, which is essential for benchmarking and model evaluation.
% Another advantage of simulations is their flexibility in sensor integration. Sensors such as cameras, LiDARs, and tactile sensors can be added or repositioned without the physical limitations present in real-world setups. Simulations also eliminate the risk of damaging expensive hardware during edge-case experiments, making them an ideal platform for testing rare or dangerous scenarios that are impractical to explore in real life.
By leveraging immersive perspectives and interactions, Extended Reality\footnote{Extended Reality is an umbrella term to refer to Augmented Reality, Mixed Reality, and Virtual Reality \cite{wikipediaExtendedReality}}
(XR)
is a promising candidate for efficient and intuitive large scale data collection \cite{jiang2024comprehensive, arcade}
% With the demand for collecting data, XR provides a promising approach for humans to teach robots by offering users an immersive experience.
in simulation \cite{jiang2024comprehensive, arcade, dexhub-park} and real-world scenarios \cite{openteach, opentelevision}.
However, reusing and reproducing current XR approaches for robot data collection for new settings and scenarios is complicated and requires significant effort.
% are difficult to reuse and reproduce system makes it hard to reuse and reproduce in another data collection pipeline.
This bottleneck arises from three main limitations of current XR data collection and interaction frameworks: \textit{asset limitation}, \textit{simulator limitation}, and \textit{device limitation}.
% \textcolor{red}{ASSIGN THESE CITATION PROPERLY:}
% \textcolor{red}{list them by time order???}
% of collecting data by using XR have three main limitations.
Current approaches suffering from \textit{asset limitation} \cite{arclfd, jiang2024comprehensive, arcade, george2025openvr, vicarios}
% Firstly, recent works \cite{jiang2024comprehensive, arcade, dexhub-park}
can only use predefined robot models and task scenes. Configuring new tasks requires significant effort, since each new object or model must be specifically integrated into the XR application.
% and it takes too much effort to configure new tasks in their systems since they cannot spawn arbitrary models in the XR application.
The vast majority of application are developed for specific simulators or real-world scenarios. This \textit{simulator limitation} \cite{mosbach2022accelerating, lipton2017baxter, dexhub-park, arcade}
% Secondly, existing systems are limited to a single simulation platform or real-world scenarios.
significantly reduces reusability and makes adaptation to new simulation platforms challenging.
Additionally, most current XR frameworks are designed for a specific version of a single XR headset, leading to a \textit{device limitation} 
\cite{lipton2017baxter, armada, openteach, meng2023virtual}.
% and there is no work working on the extendability of transferring to a new headsets as far as we know.
To the best of our knowledge, no existing work has explored the extensibility or transferability of their framework to different headsets.
These limitations hamper reproducibility and broader contributions of XR based data collection and interaction to the research community.
% as each research group typically has its own data collection pipeline.
% In addition to these main limitations, existing XR systems are not well suited for managing multiple robot systems,
% as they are often designed for single-operator use.

In addition to these main limitations, existing XR systems are often designed for single-operator use, prohibiting collaborative data collection.
At the same time, controlling multiple robots at once can be very difficult for a single operator,
making data collection in multi-robot scenarios particularly challenging \cite{orun2019effect}.
Although there are some works using collaborative data collection in the context of tele-operation \cite{tung2021learning, Qin2023AnyTeleopAG},
there is no XR-based data collection system supporting collaborative data collection.
This limitation highlights the need for more advanced XR solutions that can better support multi-robot and multi-user scenarios.
% \textcolor{red}{more papers about collaborative data collection}

To address all of these issues, we propose \textbf{IRIS},
an \textbf{I}mmersive \textbf{R}obot \textbf{I}nteraction \textbf{S}ystem.
This general system supports various simulators, benchmarks and real-world scenarios.
It is easily extensible to new simulators and XR headsets.
IRIS achieves generalization across six dimensions:
% \begin{itemize}
%     \item \textit{Cross-scene} : diverse object models;
%     \item \textit{Cross-embodiment}: diverse robot models;
%     \item \textit{Cross-simulator}: 
%     \item \textit{Cross-reality}: fd
%     \item \textit{Cross-platform}: fd
%     \item \textit{Cross-users}: fd
% \end{itemize}
\textbf{Cross-Scene}, \textbf{Cross-Embodiment}, \textbf{Cross-Simulator}, \textbf{Cross-Reality}, \textbf{Cross-Platform}, and \textbf{Cross-User}.

\textbf{Cross-Scene} and \textbf{Cross-Embodiment} allow the system to handle arbitrary objects and robots in the simulation,
eliminating restrictions about predefined models in XR applications.
IRIS achieves these generalizations by introducing a unified scene specification, representing all objects,
including robots, as data structures with meshes, materials, and textures.
The unified scene specification is transmitted to the XR application to create and visualize an identical scene.
By treating robots as standard objects, the system simplifies XR integration,
allowing researchers to work with various robots without special robot-specific configurations.
\textbf{Cross-Simulator} ensures compatibility with various simulation engines.
IRIS simplifies adaptation by parsing simulated scenes into the unified scene specification, eliminating the need for XR application modifications when switching simulators.
New simulators can be integrated by creating a parser to convert their scenes into the unified format.
This flexibility is demonstrated by IRIS’ support for Mujoco \cite{todorov2012mujoco}, IsaacSim \cite{mittal2023orbit}, CoppeliaSim \cite{coppeliaSim}, and even the recent Genesis \cite{Genesis} simulator.
\textbf{Cross-Reality} enables the system to function seamlessly in both virtual simulations and real-world applications.
IRIS enables real-world data collection through camera-based point cloud visualization.
\textbf{Cross-Platform} allows for compatibility across various XR devices.
Since XR device APIs differ significantly, making a single codebase impractical, IRIS XR application decouples its modules to maximize code reuse.
This application, developed by Unity \cite{unity3dUnityManual}, separates scene visualization and interaction, allowing developers to integrate new headsets by reusing the visualization code and only implementing input handling for hand, head, and motion controller tracking.
IRIS provides an implementation of the XR application in the Unity framework, allowing for a straightforward deployment to any device that supports Unity. 
So far, IRIS was successfully deployed to the Meta Quest 3 and HoloLens 2.
Finally, the \textbf{Cross-User} ability allows multiple users to interact within a shared scene.
IRIS achieves this ability by introducing a protocol to establish the communication between multiple XR headsets and the simulation or real-world scenarios.
Additionally, IRIS leverages spatial anchors to support the alignment of virtual scenes from all deployed XR headsets.
% To make an seamless user experience for robot learning data collection,
% IRIS also tested in three different robot control interface
% Furthermore, to demonstrate the extensibility of our approach, we have implemented a robot-world pipeline for real robot data collection, ensuring that the system can be used in both simulated and real-world environments.
The Immersive Robot Interaction System makes the following contributions\\
\textbf{(1) A unified scene specification} that is compatible with multiple robot simulators. It enables various XR headsets to visualize and interact with simulated objects and robots, providing an immersive experience while ensuring straightforward reusability and reproducibility.\\
\textbf{(2) A collaborative data collection framework} designed for XR environments. The framework facilitates enhanced robot data acquisition.\\
\textbf{(3) A user study} demonstrating that IRIS significantly improves data collection efficiency and intuitiveness compared to the LIBERO baseline.

% \begin{table*}[t]
%     \centering
%     \begin{tabular}{lccccccc}
%         \toprule
%         & \makecell{Physical\\Interaction}
%         & \makecell{XR\\Enabled}
%         & \makecell{Free\\View}
%         & \makecell{Multiple\\Robots}
%         & \makecell{Robot\\Control}
%         % Force Feedback???
%         & \makecell{Soft Object\\Supported}
%         & \makecell{Collaborative\\Data} \\
%         \midrule
%         ARC-LfD \cite{arclfd}                              & Real        & \cmark & \xmark & \xmark & Joint              & \xmark & \xmark \\
%         DART \cite{dexhub-park}                            & Sim         & \cmark & \cmark & \cmark & Cartesian          & \xmark & \xmark \\
%         \citet{jiang2024comprehensive}                     & Sim         & \cmark & \xmark & \xmark & Joint \& Cartesian & \xmark & \xmark \\
%         \citet{mosbach2022accelerating}                    & Sim         & \cmark & \cmark & \xmark & Cartesian          & \xmark & \xmark \\
%         ARCADE \cite{arcade}                               & Real        & \cmark & \cmark & \xmark & Cartesian          & \xmark & \xmark \\
%         Holo-Dex \cite{holodex}                            & Real        & \cmark & \xmark & \cmark & Cartesian          & \cmark & \xmark \\
%         ARMADA \cite{armada}                               & Real        & \cmark & \xmark & \cmark & Cartesian          & \cmark & \xmark \\
%         Open-TeleVision \cite{opentelevision}              & Real        & \cmark & \cmark & \cmark & Cartesian          & \cmark & \xmark \\
%         OPEN TEACH \cite{openteach}                        & Real        & \cmark & \xmark & \cmark & Cartesian          & \cmark & \cmark \\
%         GELLO \cite{wu2023gello}                           & Real        & \xmark & \cmark & \cmark & Joint              & \cmark & \xmark \\
%         DexCap \cite{wang2024dexcap}                       & Real        & \xmark & \cmark & \xmark & Cartesian          & \cmark & \xmark \\
%         AnyTeleop \cite{Qin2023AnyTeleopAG}                & Real        & \xmark & \xmark & \cmark & Cartesian          & \cmark & \cmark \\
%         Vicarios \cite{vicarios}                           & Real        & \cmark & \xmark & \xmark & Cartesian          & \cmark & \xmark \\     
%         Augmented Visual Cues \cite{augmentedvisualcues}   & Real        & \cmark & \cmark & \xmark & Cartesian          & \xmark & \xmark \\ 
%         \citet{wang2024robotic}                            & Real        & \cmark & \cmark & \xmark & Cartesian          & \cmark & \xmark \\
%         Bunny-VisionPro \cite{bunnyvisionpro}              & Real        & \cmark & \cmark & \cmark & Cartesian          & \cmark & \xmark \\
%         IMMERTWIN \cite{immertwin}                         & Real        & \cmark & \cmark & \cmark & Cartesian          & \xmark & \xmark \\
%         \citet{meng2023virtual}                            & Sim \& Real & \cmark & \cmark & \xmark & Cartesian          & \xmark & \xmark \\
%         Shared Control Framework \cite{sharedctlframework} & Real        & \cmark & \cmark & \cmark & Cartesian          & \xmark & \xmark \\
%         OpenVR \cite{openvr}                               & Real        & \cmark & \cmark & \xmark & Cartesian          & \xmark & \xmark \\
%         \citet{digitaltwinmr}                              & Real        & \cmark & \cmark & \xmark & Cartesian          & \cmark & \xmark \\
        
%         \midrule
%         \textbf{Ours} & Sim \& Real & \cmark & \cmark & \cmark & Joint \& Cartesian  & \cmark & \cmark \\
%         \bottomrule
%     \end{tabular}
%     \caption{This is a cross-column table with automatic line breaking.}
%     \label{tab:cross-column}
% \end{table*}

% \begin{table*}[t]
%     \centering
%     \begin{tabular}{lccccccc}
%         \toprule
%         & \makecell{Cross-Embodiment}
%         & \makecell{Cross-Scene}
%         & \makecell{Cross-Simulator}
%         & \makecell{Cross-Reality}
%         & \makecell{Cross-Platform}
%         & \makecell{Cross-User} \\
%         \midrule
%         ARC-LfD \cite{arclfd}                              & \xmark & \xmark & \xmark & \xmark & \xmark & \xmark \\
%         DART \cite{dexhub-park}                            & \cmark & \cmark & \xmark & \xmark & \xmark & \xmark \\
%         \citet{jiang2024comprehensive}                     & \xmark & \cmark & \xmark & \xmark & \xmark & \xmark \\
%         \citet{mosbach2022accelerating}                    & \xmark & \cmark & \xmark & \xmark & \xmark & \xmark \\
%         ARCADE \cite{arcade}                               & \xmark & \xmark & \xmark & \xmark & \xmark & \xmark \\
%         Holo-Dex \cite{holodex}                            & \cmark & \xmark & \xmark & \xmark & \xmark & \xmark \\
%         ARMADA \cite{armada}                               & \cmark & \xmark & \xmark & \xmark & \xmark & \xmark \\
%         Open-TeleVision \cite{opentelevision}              & \cmark & \xmark & \xmark & \xmark & \cmark & \xmark \\
%         OPEN TEACH \cite{openteach}                        & \cmark & \xmark & \xmark & \xmark & \xmark & \cmark \\
%         GELLO \cite{wu2023gello}                           & \cmark & \xmark & \xmark & \xmark & \xmark & \xmark \\
%         DexCap \cite{wang2024dexcap}                       & \xmark & \xmark & \xmark & \xmark & \xmark & \xmark \\
%         AnyTeleop \cite{Qin2023AnyTeleopAG}                & \cmark & \cmark & \cmark & \cmark & \xmark & \cmark \\
%         Vicarios \cite{vicarios}                           & \xmark & \xmark & \xmark & \xmark & \xmark & \xmark \\     
%         Augmented Visual Cues \cite{augmentedvisualcues}   & \xmark & \xmark & \xmark & \xmark & \xmark & \xmark \\ 
%         \citet{wang2024robotic}                            & \xmark & \xmark & \xmark & \xmark & \xmark & \xmark \\
%         Bunny-VisionPro \cite{bunnyvisionpro}              & \cmark & \xmark & \xmark & \xmark & \xmark & \xmark \\
%         IMMERTWIN \cite{immertwin}                         & \cmark & \xmark & \xmark & \xmark & \xmark & \xmark \\
%         \citet{meng2023virtual}                            & \xmark & \cmark & \xmark & \cmark & \xmark & \xmark \\
%         \citet{sharedctlframework}                         & \cmark & \xmark & \xmark & \xmark & \xmark & \xmark \\
%         OpenVR \cite{george2025openvr}                               & \xmark & \xmark & \xmark & \xmark & \xmark & \xmark \\
%         \citet{digitaltwinmr}                              & \xmark & \xmark & \xmark & \xmark & \xmark & \xmark \\
        
%         \midrule
%         \textbf{Ours} & \cmark & \cmark & \cmark & \cmark & \cmark & \cmark \\
%         \bottomrule
%     \end{tabular}
%     \caption{This is a cross-column table with automatic line breaking.}
% \end{table*}

% \begin{table*}[t]
%     \centering
%     \begin{tabular}{lccccccc}
%         \toprule
%         & \makecell{Cross-Scene}
%         & \makecell{Cross-Embodiment}
%         & \makecell{Cross-Simulator}
%         & \makecell{Cross-Reality}
%         & \makecell{Cross-Platform}
%         & \makecell{Cross-User}
%         & \makecell{Control Space} \\
%         \midrule
%         % Vicarios \cite{vicarios}                           & \xmark & \xmark & \xmark & \xmark & \xmark & \xmark \\     
%         % Augmented Visual Cues \cite{augmentedvisualcues}   & \xmark & \xmark & \xmark & \xmark & \xmark & \xmark \\ 
%         % OpenVR \cite{george2025openvr}                     & \xmark & \xmark & \xmark & \xmark & \xmark & \xmark \\
%         \citet{digitaltwinmr}                              & \xmark & \xmark & \xmark & \xmark & \xmark & \xmark &  \\
%         ARC-LfD \cite{arclfd}                              & \xmark & \xmark & \xmark & \xmark & \xmark & \xmark &  \\
%         \citet{sharedctlframework}                         & \cmark & \xmark & \xmark & \xmark & \xmark & \xmark &  \\
%         \citet{jiang2024comprehensive}                     & \cmark & \xmark & \xmark & \xmark & \xmark & \xmark &  \\
%         \citet{mosbach2022accelerating}                    & \cmark & \xmark & \xmark & \xmark & \xmark & \xmark & \\
%         Holo-Dex \cite{holodex}                            & \cmark & \xmark & \xmark & \xmark & \xmark & \xmark & \\
%         ARCADE \cite{arcade}                               & \cmark & \cmark & \xmark & \xmark & \xmark & \xmark & \\
%         DART \cite{dexhub-park}                            & Limited & Limited & Mujoco & Sim & Vision Pro & \xmark &  Cartesian\\
%         ARMADA \cite{armada}                               & \cmark & \cmark & \xmark & \xmark & \xmark & \xmark & \\
%         \citet{meng2023virtual}                            & \cmark & \cmark & \xmark & \cmark & \xmark & \xmark & \\
%         % GELLO \cite{wu2023gello}                           & \cmark & \xmark & \xmark & \xmark & \xmark & \xmark \\
%         % DexCap \cite{wang2024dexcap}                       & \xmark & \xmark & \xmark & \xmark & \xmark & \xmark \\
%         % AnyTeleop \cite{Qin2023AnyTeleopAG}                & \cmark & \cmark & \cmark & \cmark & \xmark & \cmark \\
%         % \citet{wang2024robotic}                            & \xmark & \xmark & \xmark & \xmark & \xmark & \xmark \\
%         Bunny-VisionPro \cite{bunnyvisionpro}              & \cmark & \cmark & \xmark & \xmark & \xmark & \xmark & \\
%         IMMERTWIN \cite{immertwin}                         & \cmark & \cmark & \xmark & \xmark & \xmark & \xmark & \\
%         Open-TeleVision \cite{opentelevision}              & \cmark & \cmark & \xmark & \xmark & \cmark & \xmark & \\
%         \citet{szczurek2023multimodal}                     & \xmark & \xmark & \xmark & Real & \xmark & \cmark & \\
%         OPEN TEACH \cite{openteach}                        & \cmark & \cmark & \xmark & \xmark & \xmark & \cmark & \\
%         \midrule
%         \textbf{Ours} & \cmark & \cmark & \cmark & \cmark & \cmark & \cmark \\
%         \bottomrule
%     \end{tabular}
%     \caption{TODO, Bruce: this table can be further optimized.}
% \end{table*}

\definecolor{goodgreen}{HTML}{228833}
\definecolor{goodred}{HTML}{EE6677}
\definecolor{goodgray}{HTML}{BBBBBB}

\begin{table*}[t]
    \centering
    \begin{adjustbox}{max width=\textwidth}
    \renewcommand{\arraystretch}{1.2}    
    \begin{tabular}{lccccccc}
        \toprule
        & \makecell{Cross-Scene}
        & \makecell{Cross-Embodiment}
        & \makecell{Cross-Simulator}
        & \makecell{Cross-Reality}
        & \makecell{Cross-Platform}
        & \makecell{Cross-User}
        & \makecell{Control Space} \\
        \midrule
        % Vicarios \cite{vicarios}                           & \xmark & \xmark & \xmark & \xmark & \xmark & \xmark \\     
        % Augmented Visual Cues \cite{augmentedvisualcues}   & \xmark & \xmark & \xmark & \xmark & \xmark & \xmark \\ 
        % OpenVR \cite{george2025openvr}                     & \xmark & \xmark & \xmark & \xmark & \xmark & \xmark \\
        \citet{digitaltwinmr}                              & \textcolor{goodred}{Limited}     & \textcolor{goodred}{Single Robot} & \textcolor{goodred}{Unity}    & \textcolor{goodred}{Real}          & \textcolor{goodred}{Meta Quest 2} & \textcolor{goodgray}{N/A} & \textcolor{goodred}{Cartesian} \\
        ARC-LfD \cite{arclfd}                              & \textcolor{goodgray}{N/A}        & \textcolor{goodred}{Single Robot} & \textcolor{goodgray}{N/A}     & \textcolor{goodred}{Real}          & \textcolor{goodred}{HoloLens}     & \textcolor{goodgray}{N/A} & \textcolor{goodred}{Cartesian} \\
        \citet{sharedctlframework}                         & \textcolor{goodred}{Limited}     & \textcolor{goodred}{Single Robot} & \textcolor{goodgray}{N/A}     & \textcolor{goodred}{Real}          & \textcolor{goodred}{HTC Vive Pro} & \textcolor{goodgray}{N/A} & \textcolor{goodred}{Cartesian} \\
        \citet{jiang2024comprehensive}                     & \textcolor{goodred}{Limited}     & \textcolor{goodred}{Single Robot} & \textcolor{goodgray}{N/A}     & \textcolor{goodred}{Real}          & \textcolor{goodred}{HoloLens 2}   & \textcolor{goodgray}{N/A} & \textcolor{goodgreen}{Joint \& Cartesian} \\
        \citet{mosbach2022accelerating}                    & \textcolor{goodgreen}{Available} & \textcolor{goodred}{Single Robot} & \textcolor{goodred}{IsaacGym} & \textcolor{goodred}{Sim}           & \textcolor{goodred}{Vive}         & \textcolor{goodgray}{N/A} & \textcolor{goodgreen}{Joint \& Cartesian} \\
        Holo-Dex \cite{holodex}                            & \textcolor{goodgray}{N/A}        & \textcolor{goodred}{Single Robot} & \textcolor{goodgray}{N/A}     & \textcolor{goodred}{Real}          & \textcolor{goodred}{Meta Quest 2} & \textcolor{goodgray}{N/A} & \textcolor{goodred}{Joint} \\
        ARCADE \cite{arcade}                               & \textcolor{goodgray}{N/A}        & \textcolor{goodred}{Single Robot} & \textcolor{goodgray}{N/A}     & \textcolor{goodred}{Real}          & \textcolor{goodred}{HoloLens 2}   & \textcolor{goodgray}{N/A} & \textcolor{goodred}{Cartesian} \\
        DART \cite{dexhub-park}                            & \textcolor{goodred}{Limited}     & \textcolor{goodred}{Limited}      & \textcolor{goodred}{Mujoco}   & \textcolor{goodred}{Sim}           & \textcolor{goodred}{Vision Pro}   & \textcolor{goodgray}{N/A} & \textcolor{goodred}{Cartesian} \\
        ARMADA \cite{armada}                               & \textcolor{goodgray}{N/A}        & \textcolor{goodred}{Limited}      & \textcolor{goodgray}{N/A}     & \textcolor{goodred}{Real}          & \textcolor{goodred}{Vision Pro}   & \textcolor{goodgray}{N/A} & \textcolor{goodred}{Cartesian} \\
        \citet{meng2023virtual}                            & \textcolor{goodred}{Limited}     & \textcolor{goodred}{Single Robot} & \textcolor{goodred}{PhysX}   & \textcolor{goodgreen}{Sim \& Real} & \textcolor{goodred}{HoloLens 2}   & \textcolor{goodgray}{N/A} & \textcolor{goodred}{Cartesian} \\
        % GELLO \cite{wu2023gello}                           & \cmark & \xmark & \xmark & \xmark & \xmark & \xmark \\
        % DexCap \cite{wang2024dexcap}                       & \xmark & \xmark & \xmark & \xmark & \xmark & \xmark \\
        % AnyTeleop \cite{Qin2023AnyTeleopAG}                & \cmark & \cmark & \cmark & \cmark & \xmark & \cmark \\
        % \citet{wang2024robotic}                            & \xmark & \xmark & \xmark & \xmark & \xmark & \xmark \\
        Bunny-VisionPro \cite{bunnyvisionpro}              & \textcolor{goodgray}{N/A}        & \textcolor{goodred}{Single Robot} & \textcolor{goodgray}{N/A}     & \textcolor{goodred}{Real}          & \textcolor{goodred}{Vision Pro}   & \textcolor{goodgray}{N/A} & \textcolor{goodred}{Cartesian} \\
        IMMERTWIN \cite{immertwin}                         & \textcolor{goodgray}{N/A}        & \textcolor{goodred}{Limited}      & \textcolor{goodgray}{N/A}     & \textcolor{goodred}{Real}          & \textcolor{goodred}{HTC Vive}     & \textcolor{goodgray}{N/A} & \textcolor{goodred}{Cartesian} \\
        Open-TeleVision \cite{opentelevision}              & \textcolor{goodgray}{N/A}        & \textcolor{goodred}{Limited}      & \textcolor{goodgray}{N/A}     & \textcolor{goodred}{Real}          & \textcolor{goodgreen}{Meta Quest, Vision Pro} & \textcolor{goodgray}{N/A} & \textcolor{goodred}{Cartesian} \\
        \citet{szczurek2023multimodal}                     & \textcolor{goodgray}{N/A}        & \textcolor{goodred}{Limited}      & \textcolor{goodgray}{N/A}     & \textcolor{goodred}{Real}          & \textcolor{goodred}{HoloLens 2}   & \textcolor{goodgreen}{Available} & \textcolor{goodred}{Joint \& Cartesian} \\
        OPEN TEACH \cite{openteach}                        & \textcolor{goodgray}{N/A}        & \textcolor{goodgreen}{Available}  & \textcolor{goodgray}{N/A}     & \textcolor{goodred}{Real}          & \textcolor{goodred}{Meta Quest 3} & \textcolor{goodred}{N/A} & \textcolor{goodgreen}{Joint \& Cartesian} \\
        \midrule
        \textbf{Ours}                                      & \textcolor{goodgreen}{Available} & \textcolor{goodgreen}{Available}  & \textcolor{goodgreen}{Mujoco, CoppeliaSim, IsaacSim} & \textcolor{goodgreen}{Sim \& Real} & \textcolor{goodgreen}{Meta Quest 3, HoloLens 2} & \textcolor{goodgreen}{Available} & \textcolor{goodgreen}{Joint \& Cartesian} \\
        \bottomrule
        \end{tabular}
    \end{adjustbox}
    \caption{Comparison of XR-based system for robots. IRIS is compared with related works in different dimensions.}
\end{table*}


\section{System Overview}
\label{sec:System Overview}
This section discusses the WeVibe system overview using active vibration sensing and structure-dynamics-informed modeling for item weight change estimation. WeVibe constructs an active vibration sensing environment through the audio-induced shelf vibration (Section~\ref{sec:audio induced vibration}), reducing the sensor cost. By playing sound from a speaker, a mechanical vibration wave is generated. This vibration will propagate through the structure, like the floor, and arrive at the gondola shelf. The vibration sensing module then captures the shelf vibration response (Section~\ref{sec: vibration sensing}). When the item weight changes, it will impact the original shelf vibration differently, which can be leveraged to estimate the item weight. Through the knowledge of structural dynamics, WeVibe characterizes these different shelf responses and develops the physics-informed features and machine-learning model to adapt to a new item location for weight estimation quickly. Finally, the difference between the two weight estimations is taken as the weight change estimation result (Section~\ref{sec: Weight Change Estimation}).

% WeVibe actively vibrates the shelf with the vibration from the speaker to prepare a vibrational environment in the activation module. The vibration wave then propagates to the whole shelf. The vibration sensor board collects these signals and sends them to the vibration signal handler module. The frequency spectrum will then be extracted. To handle the inconsistency of estimation at different locations, WeVibe localizes each signal first and assigns a location-dependent machine-learning model to estimate the weight. To fix the lack of a dataset problem, WeVibe characterizes the relationship between weight and vibration frequency spectrum as a linear regression model according to the structural dynamics.
\subsection{Audio-Induced Vibration}
\label{sec:audio induced vibration}
% What kind of signal is better for the speaker to generate vibration: Pulse, constant...
% Plate vibration is mostly induced by structural-borne vibration instead of air-borne vibration
WeVibe creates an active vibration sensing environment by playing sound from a speaker next to the gondola. When the speaker plays sound, the interior components like the cone and voice coil generate mechanical vibration. These vibration waves can propagate through the environment by exciting the structure particle movement. There are two basic wave types: Longitudinal wave and transverse wave~\cite{noh2023dynamics,yuan2022spatial}. The particles moving parallel to the wave propagation form the longitudinal wave. The particles moving perpendicular to the wave propagation form the transverse wave. With these two wave types, the mechanical vibration from the speaker can travel to the shelf.

A periodic impulse is selected to burst out a strong vibration wave capable of reaching the shelf. The shelf response resulting from each impulse is taken as one sample for weight estimation. Many types of sound can be potentially adopted for developing the vibration wave, such as constant tone, frequency sweep, and impulse. To optimize the performance of WeVibe, a wider frequency band and a more vigorous signal intensity are desired. We employ an impulse signal. The impulse signal bursts out intensively in a very brief period, so it can provide a sudden and forceful push to the speaker's components, causing a strong vibration. Additionally, the impulse sound has a broad frequency spectrum, preparing a vast feature pool for further signal processing. The speaker's volume is set as an average person's speaking volume while keeping enough intensity that the vibration sensor can get the signal.


\subsection{Vibration Sensing}
\label{sec: vibration sensing}
The weight of an item placed on a surface influences the vibration signal (as shown in Figure~\ref{fig:frequency difference}) due to the changes in the structural properties (e.g., mass, stiffness, and damping ) after an item is added or removed. When a heavier item is placed on a surface, it tends to absorb and dampen the vibrations more significantly, leading to variations in signal amplitude, frequency response, and decay rate compared to lighter items. Conversely, a lighter item has a weaker impact on the vibration signal with different characteristics. These distinctions arise because the weight affects the surface's structural properties. By accurately capturing and analyzing these variations in vibration signals, it is possible to determine the item's weight with high precision.

On the other hand, the location of where the weight change happens also affects the shelf vibration response. Through a physics-informed characterization of the shelf vibration response, item weight, and item location, WeVibe quickly provides each location with a dedicated learning model, which will be explained in Section ~\ref{sec:Structure-dynamics-informed modeling}.

\begin{figure}
% \setlength{\abovecaptionskip}{10pt}
    \centering
    \includegraphics[width=\linewidth]{Figure/Frequency_Difference.jpg}
    \caption{(a) The plot shows the frequency spectrums of different weights of a single item at two locations. For the same location, some weight-sensitive frequencies increase or decrease while the weight of the item increases, as indicated by the red circle. Furthermore, when the item changes location, the overall frequency spectrum has a more significant change, and the weight-sensitive frequencies also shift. (b) We Further plot the result of linear regression on the highlighted frequencies and item weight at both locations. Even though some points deviate from the fitted line, the visualization gives rise to the assumption of linearity.}
    \label{fig:frequency difference}
\end{figure}

Adopting the active vibration sensing method allows fewer sensors to be attached to the shelf than the smart weight sensing shelf. This is because the vibration signal captured by one single sensor can still effectively represent the structural characteristics of the entire shelf, which we will discuss more in Section ~\ref{sec:Structure-dynamics-informed modeling}. Studies have shown that these single-point vibration signals are helpful in detecting structural anomalies and assessing the integrity of various structures\cite{sekiya2018simplified,obrien2020using,yu2016state}. This insight allows WeVibe to employ one vibration sensor to capture the shelf's vibrational response and indicate weight information through further signal processing. Multiple sensors can also be attached to the shelf to improve the system's robustness and accuracy.
% WeVibe employs multiple sensors are employed to characterize the vibration signal changes at lower frequencies to improve the robustness and accuracy of the system. Each sensor can generate a weight estimation of the item, which will be fused together and generate the final result. On the other hand, WeVibe focuses more on lower frequencies because they have lower attenuation and better signal-to-noise ratio during the process of wave propagation in the physical structure\cite{ma2023effective}. WeVibe employs a low-cost vibration sensor, SM-24 geophone\cite{SM_24}, which measures the ambient vibration by converting the velocity of a surface into voltage. It has a desired frequency response between 10Hz and 240Hz. A snapshot of time and frequency domain vibration signals is shown in Figure 3.

\subsection{Weight Change Estimation}
\label{sec: Weight Change Estimation}
We leverage the knowledge of structural dynamics to characterize a physical model between the shelf vibration response and item weight to reduce the need for data collection. Through empirical studies, we first notice that vibration frequencies differ when the item weight changes. Furthermore, some frequency amplitudes show the same increasing or decreasing trend while item weight increases or decreases. Therefore, we assume a linear model between the vibration frequency spectrum and item weight. This assumption is then validated through the theoretical derivation based on the structural dynamics, detailed in section~\ref{sec:Structure-dynamics-informed modeling}. Therefore, WeVibe applies the physics-informed feature extraction and learning model to the shelf responses to estimate the item's weight. Given a new location, Wevibe can exploit this physics-informed relationship to quickly train a weight estimation model with two weight classes at best and apply the estimation to a broader range of weight. Finally, the difference between the two weight estimations is taken as the weight change estimation result.


% The weight estimation features the incorporation of structural dynamics into the analysis of the shelf response to help address the challenges of location variances and the lack of a dataset. The shelf response represented by the vibration frequency spectrum conveys information about the item's location and weight. Therefore, the frequency spectrum will first be extracted as localization and weight estimation features. WeVibe adopts a frequency between 80Hz and 240Hz with a 1Hz interval as the feature vector to avoid environmental noise and focuses on shelf responses. Therefore, the feature vector has a total number of 161 frequency bins, as shown below.
% \begin{equation}
% Feature Vector=[f_{1},f_{2},...,f_{161}] = [f_{80Hz},f_{81Hz},...,f_{240Hz}]
% \end{equation}

% We then collect data at each item location and prepare one localization model and multiple weight estimation models corresponding to each location. We discover that the general pattern of the frequency spectrum envelope is unique when the item is placed at a fixed location, even with different weights. Therefore, WeVibe first utilizes a support vector machine (SVM) to differentiate the item location. Given the localization result, WeVibe assigns the prepared model to estimate the item weight, which addresses the location variances.

% To build a weight estimation model for each location, WeVibe leverages the linearity between the vibration frequency spectrum and item weight to reduce the need for training data at multiple locations. The linearity indicates using a linear model as a weight estimation model, avoiding extensive data collection and addressing the lack of a specific dataset problem.



\section{Structure-Dynamics-Informed Vibration Modeling For Reduced Data Need}
\label{sec:Structure-dynamics-informed modeling}

\begin{figure*}[t]
% \setlength{\abovecaptionskip}{10pt}
    \centering
    \includegraphics[width=0.8\linewidth]{Figure/Experiment_Setup.jpg}
    \caption{(a) shows the store gondola and active vibration sensing setup with a speaker next to the gondola. The top right signal clip shows an example of our given vibration signal. (b) gives a more detailed view of the item location and sensor location. The different weight classes are taken by changing the amount of water in the 1L water bottle. (c) provides an overview of our vibration sensing module.}
    \label{fig:Experiment Setup}
\end{figure*}

WeVibe leverages structural dynamics knowledge to characterize the physics-informed relationship between shelf vibration and item weight change. It is observed that the item weight change and the vibration frequency spectrum follow the same increasing or decreasing trend, as indicated in Figure~\ref{fig:frequency difference}. This observation inspires a linear relationship assumption, which is justified by the derivation according to the structural dynamics. In addition, WeVibe can quickly adapt to multiple locations with several weight classes based on the observation that the linearity is kept at various locations, though the coefficients are different.

Section~\ref{sec:empirical study} will first illustrate our empirical observations, leading us to the assumption of linearity across locations. Then, Section~\ref{sec:theory} validates our observations through the theoretical model of the interaction between the vibration frequency spectrum, item weight, and item location. It points out the hidden linearity in the theoretical formulation. Section~\ref{sec: Feature Extraction and Learning} provides more detail on physics-informed feature extraction.


\subsection{Empirical Study On The Relationship Between Shelf Vibration and Item Weight}
\label{sec:empirical study}
As Figure~\ref{fig:frequency difference} (a) shows, the vibration frequency spectrums differ when various item weights and locations are presented. For the same location, the vibration frequency spectrum shows minor differences between the four weights. However, as highlighted by the red circle, these frequencies tend to increase or decrease as the weight increases monotonically. Therefore, the linear regression is applied to these frequencies and more weights to visualize their correlations, as shown in Figure~\ref{fig:frequency difference} (b). Even though some points deviate from the fitted line, the result suggests a possible linearity between the weight and these mentioned frequencies, which might exist at both locations. 

On the other hand, the various item locations lead to a more distinct change in the vibration frequency spectrums compared with the item weight. When the item location changes, the weight-sensitive frequencies shift to somewhere else, and the overall frequency spectrum envelope also significantly changes. These factors result in the different linear relationships shown in Figure~\ref{fig:frequency difference} (b). With these observations, we assumed that the vibration frequency spectrum is linearly related to the item weight across multiple locations on the shelf under our active vibration sensing environment.

\subsection{Theoretical Derivation On The Relationship Between Shelf Vibration and Item Weight}
\label{sec:theory}
%To develop the features and model for item weight estimation,
Using the structural dynamics theory, we characterize the relationship between shelf vibrations and item weight placed on the shelf to validate our assumption. In our analysis, the shelf is modeled as a thin and homogeneous plate with the length $a$ and width $b$, assuming that its thickness is much smaller than its length and width (see Figure~\ref{fig:Theory}), which is commonly valid for the steel shelf in the retail stores. The item is modeled as a point load on the shelf at location $(x_0, y_0)$. Other assumptions include the simply supported boundary and ignorable damping effect. The sensor's location is represented by $(x,y)$.
% Figure \ref{fig:Theory} illustrates a simplification of the shelf structure, and we will develop the theoretical model according to this settings. There are two assumptions we made: (1) the shelf is homogeneous with 4 edges simply supported, (2) the item size is much smaller than the plate so it can be modeled as a point. The shelf has a length of a and width of b. The object weighted $m_0$ is placed at location $(x_0,y_0)$. A vibration sensor is placed at location $(x,y)$ to collect vibration signal.
\begin{figure}[tbh]
% \setlength{\abovecaptionskip}{10pt}
    \centering
    \includegraphics[width=\linewidth]{Figure/Theory_Setting.jpg}
    \caption{The simplified shelf model for developing the theoretical model.}
    \label{fig:Theory}
\end{figure}

Based on the Kirchhoff–Love plate theory that describes the behavior of thin plates subjected to forces and moments, the governing equation of the plate vibration can be formulated as Equation ~\ref{eq:gov_eq}, ignoring the damping. In this equation, $D, \rho, \nu$ correspondingly represent the flexural rigidity, mass per unit area, and Poisson ratio of the shelf. $f(x, y, t)$ is the excitation source exerted at location $(x, y)$ at time $t$. $w(x,y,t)$ represents the shelf vertical displacement, i.e., the vibration at location $(x, y)$. Solving Equation ~\ref{eq:gov_eq} based on the simply supported boundary and static initial condition, $w(x, y, \omega)$, the Fourier transform of the time-domain vibration signal, is proportional to its spatial Fourier transform coefficient $\bar{W}(m,n,\omega)$ which is linear to the item weight $m_0$, as shown in Equation ~\ref{eq:solution}. $\bar{F}(m,n,\omega)$ is the 2d spatial Fourier transform of the excitation force $f(x, y, \omega)$ at the sensor location $(x,y)$. $\omega_{mn}$ is a function of $m$ and $n$. When the item's weight is much smaller than the mass per unit area of the shelf, we can approximate the equation further with Taylor expansion as shown in ~\ref{eq:taylor expansion}. In this situation, \textbf{the vibration frequency spectrum is linear to the item's weight.}

% Equation \ref{eq1} and \ref{eq2} together suggest the relationship between the vibration frequency spectrum and object weight originated from the Euler-Bernoulli Beam Theory. In the equation \ref{eq1}, $w(x,y,\omega)$ represents the amplitude of vibration frequency $\omega$ at the sensor location $(x,y)$. $\bar{W}(m,n,\omega)$ is the 2D spatial Fourier transform of the vibration amplitude at sensor location $(x,y)$. The object's mass on the plate is encoded in $\bar{W}(m,n,\omega)$ as shown in equation \ref{eq2}, $\bar{F}(m,n,\omega)$ is the 2d spatial Fourier transform of the excitation force $f(x, y, \omega)$ which represents the time domain Fourier transform of force applied at the sensor location (x, y). $\rho$ is the mass per unit area of the shelf. $D$ is the bending stiffness of the plate. $\omega_{mn}$ is a function of $m$ and $n$.

% Taylor expansion can then transform the $m_{0}$ from the denominator to the numerator, implying the hidden linearity. Considering the fact that most items are much lighter than the shelf, saying $m_0$ is much smaller than weight per unit area of the shelf $\rho$, equation \ref{eq2} can be written as the first two items of the Taylor expansion with respect to $m_0$ as shown in equation \ref{eq3}. Since $\bar{W}(m,n,\omega)$ also keeps a linear relationship with the vibration frequency amplitude $w(x,y,\omega)$, object weight $m_0$ and vibration frequency amplitude $w(x,y,\omega)$ are coupled with a linear model when the locations of object and sensor are fixed. This linearity indicates a much smaller requirement of data collection with a linear machine learning model.

\begin{equation}
D\nabla^4 w + \rho \frac{\partial^2 w}{\partial t^2} = f(x, y, t) + \delta(x-x_0)\delta(y-y_0)(m_0g + m_0\frac{\partial^2 w}{\partial t^2})
\label{eq:gov_eq}
\end{equation}

\begin{equation}
\begin{gathered}
w(x, y, \omega) \propto  \bar{W}(m,n,\omega) \\
= \frac{\bar{F}(m,n,\omega)}{-\omega^2 \left( \rho + \textcolor{red}{m_0} \sin\left( \frac{m x_0 \pi}{a} \right) \sin\left( \frac{n y_0 \pi}{b} \right) \right) + D \omega^2_{mn}}
\label{eq:solution}
\end{gathered}
\end{equation}

\begin{equation}
\begin{gathered}
\approx \frac{\tilde{F}(m, n, \omega)}{-\omega^2 \rho + D \omega_{mn}^2} \left( 1 + \frac{\omega^2 \sin \left( \frac{m x_0 \pi}{a} \right) \sin \left( \frac{n y_0 \pi}{b} \right)}{-\omega^2 \rho + D \omega_{mn}^2} \textcolor{red}{m_0} \right)
\label{eq:taylor expansion}
\end{gathered}
\end{equation}

The vibration response $w(x,y,\omega)$ not only depends on the item weight $m_0$, but also the item location $(x_0, y_0)$ as shown in Equation ~\ref{eq:taylor expansion}. Although this linearity holds true for different item locations, the linear coefficients are different because they depend on $(x_0, y_0)$. Intuitively, this is because different item locations affect different modes of shelf structure. Thus, the shelf vibration response is affected in various ways. To this end, the theoretical derivation validates (1) The shelf vibration frequencies are linearly correlated with the item weight. (2) This linearity is kept across the gondola shelf.


\subsection{Physics-Informed Feature Extraction and Learning}
\label{sec: Feature Extraction and Learning}

\begin{figure*}[t]
% \setlength{\abovecaptionskip}{10pt}
    \centering
    \includegraphics[width=\linewidth]{Figure/Evaluation1.jpg}
    \caption{The WeVibe system evaluation. (a) The comparison between WeVibe and the other two methods: Using one linear model for all locations and using the non-linear model for each location while using 10\% of all weight classes in training and one vibration sensor. WeVibe outperforms both approaches with a significant improvement. (b)\&(c) The weight change estimation result of WeVibe, taking 10\% of 3 weight classes in training and one vibration sensor. The result suggests that WeVibe can almost certainly distinguish weight changes bigger than 100g, which can be utilized to detect whether a can of chips or a bottle of water is taken or put back.}
    \label{fig:Evaluation1}
\end{figure*}

With theoretical derivation and empirical study, WeVibe builds a solid physics-informed feature extraction and learning model. WeVibe takes the shelf vibration response from one whole impulse vibration as one sample so that the feature will contain the information of both transient and steady state~\cite{huang2020vibration}. WeVibe then takes the vibration frequency spectrum through the Fourier transform. As Figure~\ref{fig:frequency difference} suggests, the shift of item location leads to the change of weight-sensitive frequencies, which is challenging to know beforehand. Therefore, instead of choosing one or two specific weight-sensitive frequencies, WeVibe takes the full range of vibration frequency spectrum except for the noise-occupied region, 50Hz to 240Hz, as the feature for the learning model.
% WeVibe leverages structural dynamics knowledge combined with actual data to alleviate the data requirement for exploring the relationship between shelf response and object weight across different locations. WeVibe exploits the linearity between the vibration frequency spectrum and object weight to train a weight estimation model for known locations. Then, this model will be calibrated if a new object location is involved.

% The calibration leverages that the linearity is kept across most locations on the same shelf, but it has different coefficients. In other words, we assume that the actual weight change estimation at a new location deviates from the well-trained model's result with a constant ratio. Therefore, we can take a few weight change estimation results, calculate this constant ratio, and multiply it by the well-trained model's result, getting the final weight change estimation.

% \begin{equation}
% w(x, y, \omega) = \frac{4}{ab} \sum_{m=1}^{\infty} \sum_{n=1}^{\infty} \bar{W}(m, n, \omega) \sin\left(\frac{m \pi x}{a}\right) \sin\left(\frac{n \pi y}{b}\right)
% \label{eq1}
% \end{equation}


% \begin{equation}
% \bar{W}(m,n,\omega) = \frac{\bar{F}(m,n,\omega)}{-\omega^2 \left( \rho + \textcolor{red}{m_0} \sin\left( \frac{m x_0 \pi}{a} \right) \sin\left( \frac{n y_0 \pi}{b} \right) \right) + D \omega^2_{mn}}
% \label{eq2}
% \end{equation}


% \begin{equation}
% \approx \frac{\tilde{F}(m, n, \omega)}{-\omega^2 \rho + D \omega_{mn}^2} \left( 1 + \frac{\omega^2 \sin \left( \frac{m x_0 \pi}{a} \right) \sin \left( \frac{n y_0 \pi}{b} \right)}{-\omega^2 \rho + D \omega_{mn}^2} \textcolor{red}{m_0} \right)
% \label{eq3}
% \end{equation}


\section{WeVibe Evaluation}
\label{sec:System Evaluation}
We evaluate WeVibe's performance on a standard gondola with a simple item layout first and then a real-world item layout for weight change estimation from three aspects: ~\ref{eval:System} System, ~\ref{eval:data} The usage of Data, and \ref{eval:sensor} The usage of sensor.

\subsection{Experiment Setup}
\subsubsection{Vibration Sensing Environment Setup}
We evaluate WeVibe on a real gondola shelf that is commonly used in the real-world store (Figure~\ref{fig:Experiment Setup} (a)). The shelf is manufactured from MFired Store Fixtures\cite{mfried} with steel and mounted on a double-sided gondola. The shelf length is 91.44cm, and the width is 46.72cm. We keep the other shelves empty except for the one used for evaluation. We put one 1L water bottle on the shelf and estimate the weight change of water in the 1L bottle at four different locations as indicated in Figure~\ref{fig:Experiment Setup} (b). We also put an extra three 3.78L water bottles on the shelf because the whole shelf is rarely empty in real-world cases.

An Edifier D12 speaker plays an impulse signal composed of a short sinc function from a function generator to create a vibration. The sinc function has a 10Hz central frequency and 15V peak-to-peak voltage. It bursts once every 2s, as shown in the top right corner of Figure~\ref{fig:Experiment Setup} (a). The speaker is 20cm from the bottom right edge of the gondola and faces to the right. Additionally, to evaluate whether WeVibe can work on a gondola with a greater distance from the speaker, we move the speaker further away to 116cm, increasing the distance as the length of the shelf. 

\subsubsection{Vibration Sensing Module}
The vibration signal is captured by the vibration sensing module composed of (1) A geophone, (2) An amplifier \& filter circuit, and (3) An NI-9234\cite{ni9234} analog-to-digital converter (ADC) as shown in Figure~\ref{fig:Experiment Setup} (c). The SM-24 geophone is employed to collect vibration signals~\cite{SM_24} and has an enhanced frequency response between 10Hz and 240Hz. It has been shown with prospective performance on collecting structural vibration\cite{pan2017footprintid,bonde2021pignet,mirshekari2021obstruction,zhang2023vibration}. The geophone can measure vibrations by converting the velocity of the mechanical movement of the structure into an electrical signal that can be quantitatively analyzed. To increase the signal-to-noise ratio and remove the background noise, we design a programmable amplifier \& filter circuit. We set the gain to 35 and removed the frequency component outside SM-24 geophone functional bandwidth. Finally, the NI ADC converts the analog signal into a 24-bit digitalized signal with a sampling rate of 51.2kHz per second. The NI ADC also allows a synchronized collection of multiple-channel vibration signals, which prepares us to evaluate different sensor configurations (Section~\ref{eval:sensor}).

There are four sensing modules in total; one is on the speaker to monitor the input signal and is considered as the reference signal. The other three are placed in the middle of the right shelf edge, the middle of the front shelf edge, and the middle of the left shelf edge, and they are used to evaluate the performance of sensors at different configurations.

\subsubsection{Data Collection and Pre-Processing}
We collected ten weight classes by changing the amount of water in a 1L water bottle, ranging from 50g to 500g, with 50g spacing at four different locations with a 15.24cm interval, as shown in Figure ~\ref{fig:Experiment Setup} (b). The ground truth weight is the same as the weight of the 1L water bottle. We ignore the weight of the 3.78L water bottle because their weights are not changing. Each weight has 28 samples at each location. There are 1120 samples in total. The collected weight classes result in the absolute weight change from 0g to 450g with 50g spacing. Furthermore, we repeat the data collection on item locations 1 and 2 with a further distance(116cm) between the speaker and gondola to investigate whether WeVibe can work for a gondola at a new location.

After data collection, we pre-process the data into individual samples. All the vibration signals are first aligned with the signal collected on the top of the speaker for synchronization. Then, as shown in Figure~\ref{fig:Experiment Setup} (a), each sample begins 0.1s before the beginning of the impulse signal and lasts 1s so that it can fully capture the shelf responses in the excitation stage and steady stage\cite{huang2020vibration}. Then, we apply the Fourier transform to get the frequency spectrum and use frequencies from 50Hz to 240Hz with 1Hz spacing as the feature vector~\ref{eq:feature vector}. To further reduce the dimension of the feature vector and improve learning efficacy, Principle Components Analysis (PCA) ~\cite{mackiewicz1993principal} is then applied for every feature vector to extract the most valuable information. Finally, the features from the same location are fed into a linear regression model with L2 regularization for learning the weight information.

\begin{equation}
\begin{gathered}
Feature Vector = [f_{50Hz},f_{51Hz},...,f_{240Hz}] = [f_1,f_2,...,f_{191}]
\label{eq:feature vector}
\end{gathered}
\end{equation}

\subsection{Evaluation I: WeVibe Overall Performance}
\label{eval:System}
Figure~\ref{fig:Evaluation1} shows the result of WeVibe system evaluation. Figure~\ref{fig:Evaluation1} (a) indicates an ablation test of WeVibe methodology, additionally validating the rationality of assigning a linear model for each location. Figure~\ref{fig:Evaluation1} (b) and (c) show the weight change estimation result with one standard deviation range. After estimating the weight, the difference between any two weights is taken as the final result. Since the negative and positive weight differences are symmetric, only the positive weight differences are shown in the following figures.

We first compare WeVibe with the other algorithms as shown in Figure~\ref{fig:Evaluation1} (a). Taking 10\% of training samples (three samples) of ten weight classes and one sensor, WeVibe mean absolute error outperforms the method that uses one linear model for all locations with 4.2X improvement, proving that different linear relationships are kept at various locations. On the other hand, WeVibe's means absolute error outperforms the method that uses a non-linear model for each location with 17X improvement, indicating the correctness of the linear relationship. 

Taking 10\% of training samples of three weight classes (50g, 300g, and 500g) and one sensor, Figure~\ref{fig:Evaluation1} (b) and (c) suggest WeVibe performances for gondolas with different distances with speakers. When the distance is 20cm, WeVibe achieves an overall mean absolute error of 48.15g with a standard deviation of 44.96g. When the distance is 116cm, the overall mean absolute error is 38.07g, and the standard deviation is 31.2g. These results suggest that WeVibe can correctly differentiate weight change bigger than 100g in most cases, which only takes three weight classes at every location and one vibration sensor. It is accurate enough to be leveraged for detecting whether items like a can of chips or a bottle of water are taken or put back. For lightweight items such as a bar of chocolate, a fine granularity of training weight class might be necessary. A minimum of sensors attached to the surface and better data efficacy are also suggested, which will be detailed in the following sections. Furthermore, both distances achieve a similar result, indicating a better ubiquity because the linearity between shelf vibration response and item weight still exists when the gondola is at a different location. 

\subsection{Evaluation II: The Amount of Required Data}
\label{eval:data}

\begin{figure}[tbh]
% \setlength{\abovecaptionskip}{10pt}
    \centering
    \includegraphics[width=0.6\linewidth]{Figure/Evaluation2.jpg}
    \caption{(a) With two training weight classes, the smaller one is 50g, and the bigger one gradually increases. It can be seen that the error gets smaller with a bigger range. (b) Adding more training weight classes decreases the median and mean MAE, suggesting a tradeoff between data collection effort and performance. The improvement from two to three weight classes is more significant than the improvement from three to four, leading us to select three weight classes for evaluation.  (c) The increasing amount of training data per weight class gives comparable performances. Therefore, three samples are selected in our case for better data efficacy.}
    \label{fig:Evaluation2}
\end{figure}

Three evaluations are conducted to validate the improved data efficacy based on the physics-informed model. To evaluate the tradeoff between the performance and the amount of training data, we start from two weight classes (one weight change) and find that when the minimum and maximum weights are included in the training set, the performance reaches the best. The minimum weight 50g is kept in the training set and the bigger weight increases gradually (e.g., 50g and 100g, 50g and 150g,\ldots, 50g and 500g). As weight change gets bigger, the performance gets better. The inclusion of minimum weight and maximum weight reaches the best, as shown in Figure~\ref{fig:Evaluation2} (a). It reflects that when the training data includes the full range of weight information, it can better capture the weight falling within this range, which is very similar to the property of a linear model.

Then, the number of training weight classes increases while keeping the minimum and maximum weight in the training set, evaluating the tradeoff between performance and the number of weight classes. The number of weight classes increases by interpolation: 50g and 500g; 50g, 300g and 500g; 50g, 200g, 350g, and 500g. Figure~\ref{fig:Evaluation2} (b) suggests that the increasing number of training weight classes provides an improvement. However, the improvement is more evident from two to three weight classes. The performance of three weight classes and four weight classes are very close. In the actual application, the number of necessarily collected weight classes significantly depends on the possible slightest weight change between items. In this case, three training weight classes are employed because there is no significant improvement in selecting four training weight classes.

Figure~\ref{fig:Evaluation2} (c) suggests that 10\% training data for each weight class is enough. When keeping three weight classes selected from the previous evaluation, we investigate the performance while increasing the amount of training data per weight class. It turns out that the increasing amount of training data per weight class gives comparable performances. It is probably because most data samples within the same weight class are similar. Referring back to Section~\ref{sec:theory}, the vibration frequency spectrum should be much the same as long as the item weight and item location do not change. Therefore, we select 10\% training data to minimize the data collection effort.

\subsection{Evaluation III: The Amount of Required Sensor}
\label{eval:sensor}
Figure~\ref{fig:Evaluation3} shows the performance comparison with different sensor configurations in the training process, suggesting that the combination of three sensors is comparable to a single sensor with the best performance. Figure~\ref{fig:Experiment Setup} (b) shows the location of three different sensors. Figure~\ref{fig:Evaluation3} shows that the performance of sensor 1 is the best out of three single-sensor usage cases, and it is very close to the best performance: sensor 1\&3. This observation leads us to employ sensor 1 to minimize the sensor usage. On the other hand, it is also observed that the combination of three sensors is also very close to the best performance. It results in a compromise of using all three sensors if we don't know which one can give the best performance, which is the case of our real item-layout evaluation.

\begin{figure}[t]
% \setlength{\abovecaptionskip}{10pt}
    \centering
    \includegraphics[width=0.8\linewidth]{Figure/Evaluation3.jpg}
    \caption{This figure illustrates the tradeoff between weight change estimation and different sensor configurations used during training. Sensor 1\&3 reaches the best performance, but sensor 1 and sensor1\&2\&3 also have a close performance. Given no prior knowledge of which sensor can perform the best, it is reasonable to apply all three sensors. However, if we know the best sensor, it should be used to reduce the sensor cost.}
    \label{fig:Evaluation3}
\end{figure}

\subsection{Evaluation IV: Real Item-Layout Evaluation}

\begin{figure}[tbh]
% \setlength{\abovecaptionskip}{10pt}
    \centering
    \includegraphics[width=\linewidth]{Figure/Evaluation_Real.jpg}
    \caption{WeVibe is evaluated on a real item-layout shelf. We bought a shelf of protein bars and placed them under the WeVibe sensing environment. With a mean absolute error of 41.05g and a standard deviation of 42.2g, it is expected to detect the case that two or more protein bars are taken or put back according to the weight change estimation.}
    \label{fig:Evaluation Real}
\end{figure}

As shown in Figure~\ref{fig:Evaluation Real}, WeVibe is evaluated with a real item-layout. We bought a shelf of protein bars and placed them on our shelf with the same organization in the store. Six weight samples are collected for each box: Full box, one protein bar is taken,\ldots, and five protein bars are taken. Each protein bar ranges from 59g to 64g, so an average weight of 61g is assigned for each protein bar. The label is the weight of the total shelf load, ranging from 2997g to 3302g (the sum of five boxes). The training set comprises 10\% samples of three weight classes (2997g, 3180g, and 3302g). Three sensors are used together because we assume no prior knowledge of the best sensor is provided.

WeVibe achieves a mean absolute error of 41.05g and a standard deviation of 42.2g for all locations. It suggests that WeVibe can successfully identify these cases based on the weight change estimation result when two or more protein bars are taken or returned. For each location, WeVibe shows various performances. Location 1 gives the best result with 23.34g mean absolute error and 28.3g standard deviation, which likely works for one protein bar. Location 5 shows a more considerable standard deviation, which might not be functional for the lightweight items in the actual store. It might be due to the shelf's elastic deformation. The amount of elastic deformation of the shelf surface might bring extra variation on the shelf vibration response, leading to more data collection at some specific locations. At location 5, more weight classes are required to provide better estimation because there might be a small amount of elastic deformation happening at these locations when the item weight changes, leading to more variances in the final weight change estimation.







%\section{DISCUSSION}  

% In this work, we compared behavioural responses to robot failures, examining how individuals reacted to these failures and their subsequent perceptions of the robot. The failures in our experiment varied in type, timing, and the robot's acknowledgement of the failure. Our findings indicate differences in both user gaze and perception of the robot, specifically in feelings of anxiety, and perceptions of the robot being skilled and sensible. Based on these results, we will discuss the impact of two failure types, two timings, and two levels of the robot's acknowledgement of the failure on user gaze and perception.

This study compared behavioural responses to robot failures, focusing on how individuals reacted and perceived the robot. Failures varied by type, timing, and acknowledgement. The findings revealed that robot failures affect user gaze and perceptions. These findings are discussed further in the following section.

\subsection{Behavioural Response}

To address the first research question, we analysed user gaze behaviour in multiple ways: the number of gaze shifts, gaze distribution during puzzle-solving, and gaze entropy based on transition matrices. These measures allowed us to examine how the type and timing of failures, as well as whether the robot acknowledged its failure, influenced user gaze patterns and whether gaze behaviour varied across different failure scenarios. Our results showed that user gaze is a reliable indicator of robot failures. When the robot made a failure, participants exhibited more frequent gaze shifts between different AoIs, likely due to confusion and an attempt to understand what was happening. This finding is similar to the results of Kontogiorgos et al. \cite{kontogiorgos_embodiment_2020}, who found that people tend to gaze more at the robot when it makes a mistake. The literature suggests that different types of failures influence user perceptions of the robot \cite{morales_interaction_2019}, and our findings support this by showing that users exhibit distinct gaze behaviours in response to various failure types. For example, when the failure was executional, the number of gaze shifts towards the robot was significantly higher compared to when the failure was decisional. Moreover, during executional failures, the proportion of time spent looking at the robot was much higher compared to decisional failures. It is crucial for the robot to recognize the type of failure it has made so that it can determine the appropriate strategy for recovery and regain the user's trust.


The timing of the failure is also crucial for the robot, as it requires different approaches for recovery and repair. In our research, while the timing of the failure—whether at the start or end of the interaction—did not significantly affect gaze shifts, it did influence gaze transition matrices, and gaze distribution across AoIs. Failures at the beginning of the interaction led to higher median gaze transition values, indicating more randomness early on. Additionally, participants' focus on the Tangram figure was more when the failure occurred at the beginning of the interaction compared to later ones, while their focus on the robot's body or end effector was more during late failures than early ones. 


% \subsubsection{Failure Acknowledgement}

In our research, after committing a failure, the robot could either acknowledge the failure and then continue its action, or proceed without acknowledgement. We could not find significant differences in users' gaze behaviour when the robot acknowledged its failure and when it did not. As the literature suggests \cite{esterwood_you_2021, karli_what_2023, wachowiak_when_2024}, there are other verbal approaches to failure recovery, such as promises and technical explanations, which might influence users' gaze differently. Verbal failure recovery is important for robots, as it demonstrates an awareness of mistakes. This, in turn, can make the robot appear more intelligent and encourage users to engage with it more.


% \subsubsection{Anticipatory Gaze Behavior}
Our study also explored changes in users' anticipatory gaze behaviour during the task and its potential role in assisting the robot to recover from failures. Participants frequently anticipated the placement of the object before the robot executed the action, even when the robot made an error. This anticipatory gaze behaviour could serve as a valuable cue for the robot to detect its failures and initiate appropriate recovery strategies. However, we observed a decrease in participants' anticipatory gaze behaviour as the number of tasks increased. This decline may indicate reduced engagement over time, with participants being more actively collaborative at the beginning of the interaction. It also suggests that users' gaze behaviour might change throughout the interaction. These findings highlight the dynamic nature of gaze behaviour throughout the interaction.


\subsection{Subjective Measures}

To address the second research question, we examined user perceptions of the robot in three areas: perceived intelligence, sense of safety, and trust during failures. The analysis revealed how these measures varied with the type and timing of failure and whether the robot acknowledged its mistake.

The results of the subjective evaluation revealed that users' perceptions of the robot's intelligence and safety were not significantly influenced by the type of failure. However, users exhibited higher levels of trust in the robot during executional failures compared to decisional failures, suggesting that placing an object in an incorrect location reduces trust more than making an incorrect decision. Additionally, we observed interesting findings regarding the timing of the robot's failures. When failures occurred early in the interaction, users rated the robot as more intelligent and trustworthy compared to failures that occurred later. For the measure of "Sensible," this difference was statistically significant. These findings are consistent with previous research by Morales et al. \cite{morales_interaction_2019} and Lucas et al. \cite{lucas_getting_2018}. Interestingly, users reported feeling more relaxed when failures occurred later in the interaction, aligning with results from Desai et al. \cite{desai_impact_2013} and Rossi et al. \cite{rossi_how_2017}.

When the robot acknowledged its failures, users perceived it as slightly more intelligent and trustworthy but also experienced increased anxiety. This finding may be explained by the robot’s consistent physical repair actions a few seconds after each failure. When the robot did not explicitly acknowledge its failures, users might not have interpreted these actions as errors, reducing their perception of failure events.

\subsection{Limitations and Future Work}
 There were instances where participants were preoccupied with determining the placement of their next piece, which occasionally led them to overlook the robot's movements. However, these occurrences were minimal. Another limitation is the restriction to only two types of failure and whether the robot acknowledges its failure or not. The effect size in our study was medium; however, to obtain more robust results, a larger sample size would be beneficial.
 %Additionally, focusing solely on participants' gaze behaviour may not provide a comprehensive measure of failure detection. Incorporating other non-verbal cues, such as gestures or facial expressions, alongside gaze, could improve the accuracy of failure detection. 
 Furthermore, for safety reasons, the robot's arm movement was slowed and the experimenter was in the room, which may have influenced participants' perceptions. 
 Future research could address these limitations by exploring a broader range of failure types and incorporating explanatory feedback from the robot.




\section{Related Work}
\label{sec:Related Work} 
A range of sensing technologies are employed in autonomous stores for item detection. We examine these works and then focus on weight estimation through vibration sensing, which inspired WeVibe.


\subsection{Item Detection In Autonomous Store}
\subsubsection{Vision-only.} The most prevalent sensing approach for item detection involves extensive camera deployment. This approach employs deep-learning object recognition models \cite{he2017mask,wang2023yolov7,tan2020efficientdet} to identify what items customers take. Besides direct prediction, works like \cite{9200182,torres2018text} propose further incorporating the text information on the item to augment the detection accuracy. However, the vision-only approach has three main drawbacks. First, occlusions in the camera line-of-sight hinder accurate item detection. Second, comprehensive store coverage demands many cameras, leading to substantial computational resource consumption. Finally, developing a precise and prompt deep-learning model demands a large and robust dataset, making the training process labor-intensive.


\subsubsection{Vision with other sensing modalities} Introducing other modalities makes the item detection pipeline more robust. For example, \cite{roussos2006enabling,zhang2016mobile} employed RFID with camera. By tagging each item, the accuracy is significantly improved. However, the consequent tagging effort on every item and the incremental cost of each RFID tag quickly make this approach impractical for widespread adoption. Weight sensing in store is pioneered in Aim3s \cite{ruiz2019aim3s}, followed by works with further robust fusion algorithms with the camera system\cite{falcao2021isacs,falcao2020faim}. This method relies on specialized weight-sensing plates capable of discerning weight changes at column-wise locations. However, the necessity for a complete replacement of conventional shelf plates for these specialized units is prohibitive. Additionally, the vulnerability of these plates to wear and malfunction poses further challenges.


\subsection{Weight Estimation Using Vibration}
Weight estimation based on vibration has been scrutinized in structural dynamics. Various theoretical models for analyzing the characteristic behaviors of the beam, like mass distribution and crack location through the vibration frequency spectrum, have been proposed in ~\cite{low2003natural,liu2020diagnosis,Matsumoto2003Mathematical,wynne2022quantifying}. One of the most popular research topics based on these theoretical derivations is bridge health monitoring by estimating the traffic loads through car-generated vibrations\cite{sekiya2018simplified,obrien2020using,yu2016state,liu2023telecomtm}. A similar idea is also utilized for detecting the location of cracks ~\cite{liu2020damage,nguyen2010multi}. Inspired by these works, ~\cite{bonde2021pignet, dong2023pigsense} leverages the vibration from the pig's movement to monitor the weight gain and other activities of piglets. However, these objects of interest can generate vibrations through themselves. Therefore, these approaches are not adaptable in the store because the item is stationary.

An exterior vibration source is employed for the static item weight estimation, which is referred as active vibration sensing. For example, ~\cite{mirshekari2021obstruction} proves that the footstep-generated vibration can be employed to estimate the weight of the item lying between the footstep and the vibration sensor. ~\cite{codling2021masshog} utilizes a speaker in the pigpen and models a relationship between the pig's weight and its impact on the speaker-generated vibration. However, their systems can only estimate the weight based on the kilogram scale. Vibsense~\cite{liu2017vibsense} utilizes a piezo speaker to generate vibration on a surface actively and identifies the impact of different weights on this vibration signal with a gram scale. VibroScale~\cite{zhang2020vibroscale} turns the smartphone into a weight scale by using the motor and accelerometer in the phone. Nevertheless, none of these systems accounted for the effects of the change of item location on weight estimation. 

To the best of our knowledge, WeVibe represents the first item weight change estimation system using active shelf vibration sensing. WeVibe leverages the physical knowledge from the structural dynamics to characterize the relationship between shelf vibration response and item weight at different locations, which turns out to be linear. WeVibe can quickly adapt to a new location with minimum amount of sensor attached to shelf and data collection effort. 
%
In this work, we presented counterfactual situation testing (CST), a new actionable and meaningful framework for detecting individual discrimination in a dataset of classifier decisions.
We studied both single and multidimensional discrimination, focusing on the indirect setting.
For the latter kind, we compared its multiple and intersectional forms and provided the first evidence for the need to recognize intersectional discrimination as separate from multiple discrimination under non-discrimination law.
Compared to other methods, such as situation testing (ST) and counterfactual fairness (CF), CST uncovered more cases even when the classifier was counterfactually fair and after accounting for statistical significance.
For CF, in particular, we showed how CST equips it with confidence intervals, extending how we understand the robustness of this popular causal fairness definition. 

The decision-making settings tackled in this work are intended to showcase the CST framework and, importantly, to illustrate why it is necessary to draw a distinction between idealized and fairness given the difference comparisons when testing for individual discrimination. 
We hope the results motivate the adoption of the \textit{mutatis mutandis} manipulation over the \textit{ceteris paribus} manipulation.
We are aware that the experimental setting could be pushed further by considering higher dimensions or more complex causal structures. 
We leave this for future work.
%
Further,
extensions of CST should consider the impact of using different distance functions for measuring individual similarity \parencite{WilsonM97_HeteroDistanceFunctions}, and should explore a purely data-driven setup in which the running parameters and auxiliary causal knowledge are derived from the dataset \parencite{Cohen2013StatisticalPower, Peters2017_CausalInference}.
%
Furthermore,
extensions of CST should study settings in which the protected attribute goes beyond the binary, such as a high-cardinality categorical or an ordinal protected attribute \parencite{DBLP:journals/tkde/CerdaV22}. 
The setting in which the protected attribute is continuous is also of interest, though, in that case we could discretize it \parencite{DBLP:journals/tkde/GarciaLSLH13} and treat it as binary (the current setting) or as a high-cardinality categorical attribute.

Multidimensional discrimination testing is largely understudied \parencite{DBLP:conf/fat/0001HN23, WangRR22}. 
% Here, 
We have set a foundation for exploring the tension between multiple and intersectional discrimination, but future work should further study the problem of dealing with multiple protected attributes and their intersection.
It is of interest, for instance, formalizing the case in which one protected attribute dominates the others and the case in which the impact of each protected attribute varies based on individual characteristics.
% Formalizing the case in which one protected attribute dominates over the others as well as the case in which the effect of each protected attribute varies by individual characteristics are of interest.
While interaction terms and heterogeneous effects are understudied within SCM, both topics enjoy a well established literature in fields like economics \parencite{Wooldridge2015IntroductoryEconometrics}, which should enable future work.
% 
We hope these extensions and, overall, the fairness given the difference powering the CST framework motivate new work on algorithmic discrimination testing.

%
% EOS
%



%%
%% The next two lines define the bibliography style to be used, and
%% the bibliography file.
\bibliographystyle{ACM-Reference-Format}
\bibliography{main}
%%
%% If your work has an appendix, this is the place to put it.

\end{document}

%%
%% The code below is generated by the tool at http://dl.acm.org/ccs.cfm.
%% Please copy and paste the code instead of the example below.
%%
\begin{CCSXML}
<ccs2012>
 <concept>
  <concept_id>00000000.0000000.0000000</concept_id>
  <concept_desc>Do Not Use This Code, Generate the Correct Terms for Your Paper</concept_desc>
  <concept_significance>500</concept_significance>
 </concept>
 <concept>
  <concept_id>00000000.00000000.00000000</concept_id>
  <concept_desc>Do Not Use This Code, Generate the Correct Terms for Your Paper</concept_desc>
  <concept_significance>300</concept_significance>
 </concept>
 <concept>
  <concept_id>00000000.00000000.00000000</concept_id>
  <concept_desc>Do Not Use This Code, Generate the Correct Terms for Your Paper</concept_desc>
  <concept_significance>100</concept_significance>
 </concept>
 <concept>
  <concept_id>00000000.00000000.00000000</concept_id>
  <concept_desc>Do Not Use This Code, Generate the Correct Terms for Your Paper</concept_desc>
  <concept_significance>100</concept_significance>
 </concept>
</ccs2012>
\end{CCSXML}

\ccsdesc[500]{Do Not Use This Code~Generate the Correct Terms for Your Paper}
\ccsdesc[300]{Do Not Use This Code~Generate the Correct Terms for Your Paper}
\ccsdesc{Do Not Use This Code~Generate the Correct Terms for Your Paper}
\ccsdesc[100]{Do Not Use This Code~Generate the Correct Terms for Your Paper}

%%
%% Keywords. The author(s) should pick words that accurately describe
%% the work being presented. Separate the keywords with commas.
\keywords{Do, Not, Us, This, Code, Put, the, Correct, Terms, for,
  Your, Paper}
%% A "teaser" image appears between the author and affiliation
%% information and the body of the document, and typically spans the
%% page.



\end{document}
\endinput
%%
%% End of file `sample-sigconf-authordraft.tex'.
