%%
%% This is file `sample-sigconf-authordraft.tex',
%% generated with the docstrip utility.
%%
%% The original source files were:
%%
%% samples.dtx  (with options: `all,proceedings,bibtex,authordraft')
%% 
%% IMPORTANT NOTICE:
%% 
%% For the copyright see the source file.
%% 
%% Any modified versions of this file must be renamed
%% with new filenames distinct from sample-sigconf-authordraft.tex.
%% 
%% For distribution of the original source see the terms
%% for copying and modification in the file samples.dtx.
%% 
%% This generated file may be distributed as long as the
%% original source files, as listed above, are part of the
%% same distribution. (The sources need not necessarily be
%% in the same archive or directory.)
%%
%%
%% Commands for TeXCount
%TC:macro \cite [option:text,text]
%TC:macro \citep [option:text,text]
%TC:macro \citet [option:text,text]
%TC:envir table 0 1
%TC:envir table* 0 1
%TC:envir tabular [ignore] word
%TC:envir displaymath 0 word
%TC:envir math 0 word
%TC:envir comment 0 0
%%
%%
%% The first command in your LaTeX source must be the \documentclass
%% command.
%%
%% For submission and review of your manuscript please change the
%% command to \documentclass[manuscript, screen, review]{acmart}.
%%
%% When submitting camera ready or to TAPS, please change the command
%% to \documentclass[sigconf]{acmart} or whichever template is required
%% for your publication.
%%
%%
\documentclass[sigconf]{acmart}
\usepackage{enumitem}
%%
%% \BibTeX command to typeset BibTeX logo in the docs
\AtBeginDocument{%
  \providecommand\BibTeX{{%
    Bib\TeX}}}

%% Rights management information.  This information is sent to you
%% when you complete the rights form.  These commands have SAMPLE
%% values in them; it is your responsibility as an author to replace
%% the commands and values with those provided to you when you
%% complete the rights form.
\setcopyright{acmcopyright}
\copyrightyear{2025}
\acmYear{2025}
%\acmDOI{XXXXXXX.XXXXXXX}


%% These commands are for a PROCEEDINGS abstract or paper.
% \acmConference[BuildSys '24]{The 11th ACM International Conference on Systems for Energy-Efficient Buildings, Cities, and Transportation}{November 7--8,
% 2024}{Hangzhou, China}
%%
%%  Uncomment \acmBooktitle if the title of the proceedings is different
%%  from ``Proceedings of ...''!
%%
%%\acmBooktitle{Woodstock '18: ACM Symposium on Neural Gaze Detection,
%%  June 03--05, 2018, Woodstock, NY}
% \acmISBN{978-1-4503-XXXX-X/18/06}


%%
%% Submission ID.
%% Use this when submitting an article to a sponsored event. You'll
%% receive a unique submission ID from the organizers
%% of the event, and this ID should be used as the parameter to this command.
%%\acmSubmissionID{123-A56-BU3}

%%
%% For managing citations, it is recommended to use bibliography
%% files in BibTeX format.
%%
%% You can then either use BibTeX with the ACM-Reference-Format style,
%% or BibLaTeX with the acmnumeric or acmauthoryear sytles, that include
%% support for advanced citation of software artefact from the
%% biblatex-software package, also separately available on CTAN.
%%
%% Look at the sample-*-biblatex.tex files for templates showcasing
%% the biblatex styles.
%%

%%
%% The majority of ACM publications use numbered citations and
%% references.  The command \citestyle{authoryear} switches to the
%% "author year" style.
%%
%% If you are preparing content for an event
%% sponsored by ACM SIGGRAPH, you must use the "author year" style of
%% citations and references.
%% Uncommenting
%% the next command will enable that style.
%%\citestyle{acmauthoryear}


%%
%% end of the preamble, start of the body of the document source.
\begin{document}

%%
%% The "title" command has an optional parameter,
%% allowing the author to define a "short title" to be used in page headers.
\title{WeVibe: Weight Change Estimation Through Audio-Induced Shelf Vibrations In Autonomous Stores}

%%
%% The "author" command and its associated commands are used to define
%% the authors and their affiliations.
%% Of note is the shared affiliation of the first two authors, and the
%% "authornote" and "authornotemark" commands
%% used to denote shared contribution to the research.
\author{\large Jiale Zhang}
\orcid{0000-0003-0688-564X}
\email{jiale@umich.edu}
\affiliation{%
  \institution{University of Michigan}
  \streetaddress{1301 Beal Ave}
  \city{Ann Arbor}
  \state{MI}
  \country{USA}
  \postcode{48109-2122}
}

\author{\large Yuyan Wu}
\orcid{0009-0009-3152-939X}
\email{wuyuyan@stanford.edu}
\affiliation{%
  \institution{Stanford University}
  \streetaddress{}
  \city{Stanford}
  \state{CA}
  \country{USA}
  \postcode{94305}
}



\author{\large Jesse R Codling}
\orcid{0000-0001-8355-7186}
\email{codling@umich.edu}
\affiliation{%
    \institution{University of Michigan}
    \city{Ann Arbor}
    \state{Michigan}
    \country{USA}
}

\author{\large Yen Cheng Chang}
\orcid{0009-0007-1986-9485}
\email{yencheng@umich.edu}
\affiliation{%
    \institution{University of Michigan}
    \city{Ann Arbor}
    \state{Michigan}
    \country{USA}
}

\author{\large Julia Gersey}
\email{gersey@umich.edu}
\affiliation{%
  \institution{University of Michigan}
  \city{Ann Arbor}
  \state{MI}
  \country{USA}}


\author{\large Pei Zhang}
\orcid{0000-0002-8512-1615}
\email{peizhang@umich.edu}
\affiliation{%
  \institution{University of Michigan}
  \streetaddress{1301 Beal Ave}
  \city{Ann Arbor}
  \state{MI}
  \country{USA}
  \postcode{48109-2122}
}

\author{\large Hae Young Noh}
\orcid{0000-0002-7998-3657}
\email{noh@stanford.edu}
\affiliation{%
  \institution{Stanford University}
  \streetaddress{}
  \city{Stanford}
  \state{CA}
  \country{USA}
  \postcode{94305}
}

\author{\large Yiwen Dong}
\orcid{0000-0002-7877-1783}
\email{ywdong@stanford.edu}
\affiliation{%
  \institution{Stanford University}
  \streetaddress{}
  \city{Stanford}
  \state{CA}
  \country{USA}
  \postcode{94305}
}

%%
%% By default, the full list of authors will be used in the page
%% headers. Often, this list is too long, and will overlap
%% other information printed in the page headers. This command allows
%% the author to define a more concise list
%% of authors' names for this purpose.
\renewcommand{\shortauthors}{Jiale et al.}

%%
%% The abstract is a short summary of the work to be presented in the
%% article.
\begin{abstract}
Weight change estimation is crucial in various applications, particularly for detecting pick-up and put-back actions when people interact with the shelf while shopping in autonomous stores. Moreover, accurate weight change estimation allows autonomous stores to automatically identify items being picked up or put back, ensuring precise cost estimation. However, the conventional approach of estimating weight changes requires specialized weight-sensing shelves, which are densely deployed weight scales, incurring intensive sensor consumption and high costs. Prior works explored the vibration-based weight sensing method, but they failed when the location of weight change varies.

In response to these limitations, we made the following contributions: (1) We propose WeVibe, a first item weight change estimation system through active shelf vibration sensing. The main intuition of the system is that the weight placed on the shelf influences the dynamic vibration response of the shelf, thus altering the shelf vibration patterns. (2) We model a physics-informed relationship between the shelf vibration response and item weight across multiple locations on the shelf based on structural dynamics theory. This relationship is linear and allows easy training of a weight estimation model at a new location without heavy data collection. (3) We evaluate our system on a gondola shelf organized as the real-store settings. WeVibe achieved a mean absolute error down to 38.07g and a standard deviation of 31.2g with one sensor and 10\% samples from three weight classes on estimating weight change from 0g to 450g, which can be leveraged for differentiating items with more than 100g differences.
\end{abstract}

%%
%% The code below is generated by the tool at http://dl.acm.org/ccs.cfm.
%% Please copy and paste the code instead of the example below.

\begin{teaserfigure}
\setlength{\abovecaptionskip}{10pt}
    \centering
    \includegraphics[width=\linewidth]{Figure/Teaser.jpg}
    \caption{(A) WeVibe leverages the audio-induced vibration propagating to the shelf to create an active vibration sensing environment, reducing the need for sensors attached to the shelf. (B) The different item weights and locations impact the original shelf vibration response differently, reflected by the vibration signal collected by the sensor. (C) According to structural dynamics, a physics-informed relationship (a linear model) between the shelf vibration response and item weight across different locations is characterized to estimate the item weight change on the shelf with improved data efficacy. (D) Through physics-informed learning, WeVibe can estimate the weight change quickly at a new location with fewer sensors and data.}
    \Description{figure description}
    \label{fig:overview}
\end{teaserfigure}

\maketitle


\section{Introduction}

\begin{figure*}
    \centering
    \includegraphics[width=\textwidth]{figures/Introduction.pdf}
    \caption{Showing the novel problem statement applied to traffic prediction use case. Multiple unstructured observations from the past are used to reconstruct a hidden traffic state from which a full traffic state is forecast with a set of query locations. }
    \label{fig:intro}
\end{figure*}

% Was sagen denn die anderen warum Traffic Prediction gut ist? 
Forecasting the traffic in the near future is an important task for city management.
Data from the near past is used to predict future traffic states with spatio-temporal Graph Neural Networks \cite{bui22}.
Accurate prediction provides the opportunity to optimize traffic flow, reduce traffic jams and increase air quality \cite{Po19}.

% Wieso ist Sparsity in allen Dimensionen wichtig.
While traffic prediction relies on the availability of data from traffic sensors, there exists a plethora of reasons why sensors may stop working temporarily, such as simple errors, energy saving, or overloaded communication systems.
Considering small- or medium-sized cities, the coverage of sensors may be low because the sensors are too expensive or not available.
Also, the sensors are typically static and do not adapt to changes in the traffic flow (e.g. caused by a construction site), which motivates moving sensors that for example could be mounted on cars. 
However, both missing and moving sensors introduce sparsity, since measurements may not be available for all locations at all times.
This sparsity must be explicitly addressed in traffic prediction for a realistic application scenario, which is illustrated in figure \ref{fig:intro}.
From one hour of data on Sunday morning, only few observations of the traffic state are available at each timestep.
The number of observations may differ throughout the observed time and the observation itself can be distributed arbitrarily in the city. 
We assume a relatively low number of sensors to account for resource saving and sensor failure in our proposed framework SUSTeR.
The task is to predict the dense traffic state one timestep after the observations at all possible sensor locations.
We study this problem on the traffic dataset Metr-LA and PEMS-BAY to test our assumption that only a fraction of the sensor values would be enough for good predictions.
By modifying an existing traffic dataset, we are able to compare our results from very sparse observations to the bottom line with all information available.
A successful study will provide insights in how sensors in new cities can be reduced before installing them and further mobile sensors would save more resources and are able to adapt to new traffic situations.
We argue that in order to be adaptable to other cities and changes in traffic flows, prior information like the road network should be neglected and just the sparse observations considered.
This comes with the added benefit of making our solution applicable in regions where no openly available road network is maintained or pathways change frequently (e.g. flood areas, animal observations). 


The aforementioned problem is novel and more challenging than the commonly considered traffic prediction problem, since there exist very few observations in each input sample.
Current works for the traffic prediction problem do not consider any missing values. \cite{Li2021, Shao22}
A common method among state of the art approaches is the usage of Graph Neural Networks on graphs that model the sensor network \cite{bui22}.
The values of a sensor are applied to the same graph node for each timestep which prohibits any non-stationary sensors . 
With fixed sensor locations, the resulting sensor network is highly correlated with the road network.
Streets connecting two intersections with sensors should be also an interesting point for correlations in the sensor network.
However, variable observations and high temporal sparsity rates can not be modeled adequately in a static network.
We show in our experiments that the road network has only a small influence on the traffic predictions.

Besides the traffic prediction for future timesteps, some works explore the field of traffic speed imputation \cite{Cini22, Cuza22} where missing sensor values are predicted.
But the amount of missing values is assumed to be at most 80\%, which on average are still over 40 given sensors in each timestep in the Metr-LA dataset with a total of 207 sensors.
We consider up to 99.9\% missing values which are on average 2.4 observations in each timestep that are used as input.
Such high sparsity rates drastically decrease the chance that multiple values are present in one input sample from the same sensor location, which makes it challenging to recognize and learn temporal correlations for each location on its own.

High sparsity rates (>95\%) result in few sensor values, but if a reconstruction of the traffic state would be possible, we question if spatio-temporal graphs require nodes for each sensor.
In SUSTeR we utilize only a small amount of graph nodes for the encoding of information and do not relate such nodes to the sensor network.
We call this the hidden graph (see figure \ref{fig:intro}), which is still able to reconstruct the complete traffic state.
Due to the reduced number of nodes SUSTeR achieves faster runtimes, as shown in the experiments.
This hidden graph is not embedded directly in the spatial domain, which is why the assignment of observations, as well as the querying of the future traffic, is done with an encoder and a decoder, implemented as neural networks.
The decoding from the hidden graph to future values depends on a set of query locations.
Figure \ref{fig:intro} shows the query locations as given from outside and in combination with the reconstructed traffic state the future values are predicted.

To construct the hidden graph we encode observations from each timestep into from multiple graphs, one for each timestep. 
The graphs are created in a residual style and information is added to the node embeddings from the previous timesteps.
We choose this method to incorporate all timesteps equally into the hidden state because the redundant information along the past is non-existing for high sparsity rates.
From the sequence of graphs where our framework inserted the observations step by step we apply STGCN \cite{Yu18}, an algorithm for traffic prediction to find and learn the spatio-temporal correlations on our small number of graph nodes.
The first future timestep of the STGCN is our hidden graph in which the traffic state is reconstructed. 

% Recent work has an implicit embedding of the graph nodes into the spatial domain as the assignment from the sensor to graph node is fixed one by one.
% Because the graph has the same structure as the road network spatio-temporal correlations can be learned between those sensors.
% We reduce the number of nodes and use a non-linear assignment learned data-driven from the observations.

We find in the experiments that SUSTeR outperforms the plain STGCN and modern traffic prediction frameworks like D2STGNN for high sparsity rates $(\geq 99\%)$.
This is equivalent to only $0.2$ to $2.4$ observation for each timestep on average.
SUSTeR uses fewer parameters than the baselines and can train faster and with less training data.
Our main contributions can be summarized as follows:
\begin{itemize}
    \item We introduce a sparse and unstructured variant of the traffic prediction problem with sparsity in all dimensions. The sensors report only a fraction of their values and are arbitrarily distributed in the spatial domain.
    \item We propose SUSTeR, a framework around the STGCN architecture, which maps sparse observations onto a dense hidden graph to reconstruct the complete traffic state.
    Our code is available at github.\footnote{https://github.com/ywoelker/SUSTeR}
    \item We conducts experiments that show that SUSTeR outperforms the baselines in very sparse situations ($\geq 95\%$) and has a competitive performance in low sparsity rates.
    % \item SUSTeR trains a third faster than the next competitor.
\end{itemize}

\section{System Overview}
\label{sec:System Overview}
This section discusses the WeVibe system overview using active vibration sensing and structure-dynamics-informed modeling for item weight change estimation. WeVibe constructs an active vibration sensing environment through the audio-induced shelf vibration (Section~\ref{sec:audio induced vibration}), reducing the sensor cost. By playing sound from a speaker, a mechanical vibration wave is generated. This vibration will propagate through the structure, like the floor, and arrive at the gondola shelf. The vibration sensing module then captures the shelf vibration response (Section~\ref{sec: vibration sensing}). When the item weight changes, it will impact the original shelf vibration differently, which can be leveraged to estimate the item weight. Through the knowledge of structural dynamics, WeVibe characterizes these different shelf responses and develops the physics-informed features and machine-learning model to adapt to a new item location for weight estimation quickly. Finally, the difference between the two weight estimations is taken as the weight change estimation result (Section~\ref{sec: Weight Change Estimation}).

% WeVibe actively vibrates the shelf with the vibration from the speaker to prepare a vibrational environment in the activation module. The vibration wave then propagates to the whole shelf. The vibration sensor board collects these signals and sends them to the vibration signal handler module. The frequency spectrum will then be extracted. To handle the inconsistency of estimation at different locations, WeVibe localizes each signal first and assigns a location-dependent machine-learning model to estimate the weight. To fix the lack of a dataset problem, WeVibe characterizes the relationship between weight and vibration frequency spectrum as a linear regression model according to the structural dynamics.
\subsection{Audio-Induced Vibration}
\label{sec:audio induced vibration}
% What kind of signal is better for the speaker to generate vibration: Pulse, constant...
% Plate vibration is mostly induced by structural-borne vibration instead of air-borne vibration
WeVibe creates an active vibration sensing environment by playing sound from a speaker next to the gondola. When the speaker plays sound, the interior components like the cone and voice coil generate mechanical vibration. These vibration waves can propagate through the environment by exciting the structure particle movement. There are two basic wave types: Longitudinal wave and transverse wave~\cite{noh2023dynamics,yuan2022spatial}. The particles moving parallel to the wave propagation form the longitudinal wave. The particles moving perpendicular to the wave propagation form the transverse wave. With these two wave types, the mechanical vibration from the speaker can travel to the shelf.

A periodic impulse is selected to burst out a strong vibration wave capable of reaching the shelf. The shelf response resulting from each impulse is taken as one sample for weight estimation. Many types of sound can be potentially adopted for developing the vibration wave, such as constant tone, frequency sweep, and impulse. To optimize the performance of WeVibe, a wider frequency band and a more vigorous signal intensity are desired. We employ an impulse signal. The impulse signal bursts out intensively in a very brief period, so it can provide a sudden and forceful push to the speaker's components, causing a strong vibration. Additionally, the impulse sound has a broad frequency spectrum, preparing a vast feature pool for further signal processing. The speaker's volume is set as an average person's speaking volume while keeping enough intensity that the vibration sensor can get the signal.


\subsection{Vibration Sensing}
\label{sec: vibration sensing}
The weight of an item placed on a surface influences the vibration signal (as shown in Figure~\ref{fig:frequency difference}) due to the changes in the structural properties (e.g., mass, stiffness, and damping ) after an item is added or removed. When a heavier item is placed on a surface, it tends to absorb and dampen the vibrations more significantly, leading to variations in signal amplitude, frequency response, and decay rate compared to lighter items. Conversely, a lighter item has a weaker impact on the vibration signal with different characteristics. These distinctions arise because the weight affects the surface's structural properties. By accurately capturing and analyzing these variations in vibration signals, it is possible to determine the item's weight with high precision.

On the other hand, the location of where the weight change happens also affects the shelf vibration response. Through a physics-informed characterization of the shelf vibration response, item weight, and item location, WeVibe quickly provides each location with a dedicated learning model, which will be explained in Section ~\ref{sec:Structure-dynamics-informed modeling}.

\begin{figure}
% \setlength{\abovecaptionskip}{10pt}
    \centering
    \includegraphics[width=\linewidth]{Figure/Frequency_Difference.jpg}
    \caption{(a) The plot shows the frequency spectrums of different weights of a single item at two locations. For the same location, some weight-sensitive frequencies increase or decrease while the weight of the item increases, as indicated by the red circle. Furthermore, when the item changes location, the overall frequency spectrum has a more significant change, and the weight-sensitive frequencies also shift. (b) We Further plot the result of linear regression on the highlighted frequencies and item weight at both locations. Even though some points deviate from the fitted line, the visualization gives rise to the assumption of linearity.}
    \label{fig:frequency difference}
\end{figure}

Adopting the active vibration sensing method allows fewer sensors to be attached to the shelf than the smart weight sensing shelf. This is because the vibration signal captured by one single sensor can still effectively represent the structural characteristics of the entire shelf, which we will discuss more in Section ~\ref{sec:Structure-dynamics-informed modeling}. Studies have shown that these single-point vibration signals are helpful in detecting structural anomalies and assessing the integrity of various structures\cite{sekiya2018simplified,obrien2020using,yu2016state}. This insight allows WeVibe to employ one vibration sensor to capture the shelf's vibrational response and indicate weight information through further signal processing. Multiple sensors can also be attached to the shelf to improve the system's robustness and accuracy.
% WeVibe employs multiple sensors are employed to characterize the vibration signal changes at lower frequencies to improve the robustness and accuracy of the system. Each sensor can generate a weight estimation of the item, which will be fused together and generate the final result. On the other hand, WeVibe focuses more on lower frequencies because they have lower attenuation and better signal-to-noise ratio during the process of wave propagation in the physical structure\cite{ma2023effective}. WeVibe employs a low-cost vibration sensor, SM-24 geophone\cite{SM_24}, which measures the ambient vibration by converting the velocity of a surface into voltage. It has a desired frequency response between 10Hz and 240Hz. A snapshot of time and frequency domain vibration signals is shown in Figure 3.

\subsection{Weight Change Estimation}
\label{sec: Weight Change Estimation}
We leverage the knowledge of structural dynamics to characterize a physical model between the shelf vibration response and item weight to reduce the need for data collection. Through empirical studies, we first notice that vibration frequencies differ when the item weight changes. Furthermore, some frequency amplitudes show the same increasing or decreasing trend while item weight increases or decreases. Therefore, we assume a linear model between the vibration frequency spectrum and item weight. This assumption is then validated through the theoretical derivation based on the structural dynamics, detailed in section~\ref{sec:Structure-dynamics-informed modeling}. Therefore, WeVibe applies the physics-informed feature extraction and learning model to the shelf responses to estimate the item's weight. Given a new location, Wevibe can exploit this physics-informed relationship to quickly train a weight estimation model with two weight classes at best and apply the estimation to a broader range of weight. Finally, the difference between the two weight estimations is taken as the weight change estimation result.


% The weight estimation features the incorporation of structural dynamics into the analysis of the shelf response to help address the challenges of location variances and the lack of a dataset. The shelf response represented by the vibration frequency spectrum conveys information about the item's location and weight. Therefore, the frequency spectrum will first be extracted as localization and weight estimation features. WeVibe adopts a frequency between 80Hz and 240Hz with a 1Hz interval as the feature vector to avoid environmental noise and focuses on shelf responses. Therefore, the feature vector has a total number of 161 frequency bins, as shown below.
% \begin{equation}
% Feature Vector=[f_{1},f_{2},...,f_{161}] = [f_{80Hz},f_{81Hz},...,f_{240Hz}]
% \end{equation}

% We then collect data at each item location and prepare one localization model and multiple weight estimation models corresponding to each location. We discover that the general pattern of the frequency spectrum envelope is unique when the item is placed at a fixed location, even with different weights. Therefore, WeVibe first utilizes a support vector machine (SVM) to differentiate the item location. Given the localization result, WeVibe assigns the prepared model to estimate the item weight, which addresses the location variances.

% To build a weight estimation model for each location, WeVibe leverages the linearity between the vibration frequency spectrum and item weight to reduce the need for training data at multiple locations. The linearity indicates using a linear model as a weight estimation model, avoiding extensive data collection and addressing the lack of a specific dataset problem.



\section{Structure-Dynamics-Informed Vibration Modeling For Reduced Data Need}
\label{sec:Structure-dynamics-informed modeling}

\begin{figure*}[t]
% \setlength{\abovecaptionskip}{10pt}
    \centering
    \includegraphics[width=0.8\linewidth]{Figure/Experiment_Setup.jpg}
    \caption{(a) shows the store gondola and active vibration sensing setup with a speaker next to the gondola. The top right signal clip shows an example of our given vibration signal. (b) gives a more detailed view of the item location and sensor location. The different weight classes are taken by changing the amount of water in the 1L water bottle. (c) provides an overview of our vibration sensing module.}
    \label{fig:Experiment Setup}
\end{figure*}

WeVibe leverages structural dynamics knowledge to characterize the physics-informed relationship between shelf vibration and item weight change. It is observed that the item weight change and the vibration frequency spectrum follow the same increasing or decreasing trend, as indicated in Figure~\ref{fig:frequency difference}. This observation inspires a linear relationship assumption, which is justified by the derivation according to the structural dynamics. In addition, WeVibe can quickly adapt to multiple locations with several weight classes based on the observation that the linearity is kept at various locations, though the coefficients are different.

Section~\ref{sec:empirical study} will first illustrate our empirical observations, leading us to the assumption of linearity across locations. Then, Section~\ref{sec:theory} validates our observations through the theoretical model of the interaction between the vibration frequency spectrum, item weight, and item location. It points out the hidden linearity in the theoretical formulation. Section~\ref{sec: Feature Extraction and Learning} provides more detail on physics-informed feature extraction.


\subsection{Empirical Study On The Relationship Between Shelf Vibration and Item Weight}
\label{sec:empirical study}
As Figure~\ref{fig:frequency difference} (a) shows, the vibration frequency spectrums differ when various item weights and locations are presented. For the same location, the vibration frequency spectrum shows minor differences between the four weights. However, as highlighted by the red circle, these frequencies tend to increase or decrease as the weight increases monotonically. Therefore, the linear regression is applied to these frequencies and more weights to visualize their correlations, as shown in Figure~\ref{fig:frequency difference} (b). Even though some points deviate from the fitted line, the result suggests a possible linearity between the weight and these mentioned frequencies, which might exist at both locations. 

On the other hand, the various item locations lead to a more distinct change in the vibration frequency spectrums compared with the item weight. When the item location changes, the weight-sensitive frequencies shift to somewhere else, and the overall frequency spectrum envelope also significantly changes. These factors result in the different linear relationships shown in Figure~\ref{fig:frequency difference} (b). With these observations, we assumed that the vibration frequency spectrum is linearly related to the item weight across multiple locations on the shelf under our active vibration sensing environment.

\subsection{Theoretical Derivation On The Relationship Between Shelf Vibration and Item Weight}
\label{sec:theory}
%To develop the features and model for item weight estimation,
Using the structural dynamics theory, we characterize the relationship between shelf vibrations and item weight placed on the shelf to validate our assumption. In our analysis, the shelf is modeled as a thin and homogeneous plate with the length $a$ and width $b$, assuming that its thickness is much smaller than its length and width (see Figure~\ref{fig:Theory}), which is commonly valid for the steel shelf in the retail stores. The item is modeled as a point load on the shelf at location $(x_0, y_0)$. Other assumptions include the simply supported boundary and ignorable damping effect. The sensor's location is represented by $(x,y)$.
% Figure \ref{fig:Theory} illustrates a simplification of the shelf structure, and we will develop the theoretical model according to this settings. There are two assumptions we made: (1) the shelf is homogeneous with 4 edges simply supported, (2) the item size is much smaller than the plate so it can be modeled as a point. The shelf has a length of a and width of b. The object weighted $m_0$ is placed at location $(x_0,y_0)$. A vibration sensor is placed at location $(x,y)$ to collect vibration signal.
\begin{figure}[tbh]
% \setlength{\abovecaptionskip}{10pt}
    \centering
    \includegraphics[width=\linewidth]{Figure/Theory_Setting.jpg}
    \caption{The simplified shelf model for developing the theoretical model.}
    \label{fig:Theory}
\end{figure}

Based on the Kirchhoff–Love plate theory that describes the behavior of thin plates subjected to forces and moments, the governing equation of the plate vibration can be formulated as Equation ~\ref{eq:gov_eq}, ignoring the damping. In this equation, $D, \rho, \nu$ correspondingly represent the flexural rigidity, mass per unit area, and Poisson ratio of the shelf. $f(x, y, t)$ is the excitation source exerted at location $(x, y)$ at time $t$. $w(x,y,t)$ represents the shelf vertical displacement, i.e., the vibration at location $(x, y)$. Solving Equation ~\ref{eq:gov_eq} based on the simply supported boundary and static initial condition, $w(x, y, \omega)$, the Fourier transform of the time-domain vibration signal, is proportional to its spatial Fourier transform coefficient $\bar{W}(m,n,\omega)$ which is linear to the item weight $m_0$, as shown in Equation ~\ref{eq:solution}. $\bar{F}(m,n,\omega)$ is the 2d spatial Fourier transform of the excitation force $f(x, y, \omega)$ at the sensor location $(x,y)$. $\omega_{mn}$ is a function of $m$ and $n$. When the item's weight is much smaller than the mass per unit area of the shelf, we can approximate the equation further with Taylor expansion as shown in ~\ref{eq:taylor expansion}. In this situation, \textbf{the vibration frequency spectrum is linear to the item's weight.}

% Equation \ref{eq1} and \ref{eq2} together suggest the relationship between the vibration frequency spectrum and object weight originated from the Euler-Bernoulli Beam Theory. In the equation \ref{eq1}, $w(x,y,\omega)$ represents the amplitude of vibration frequency $\omega$ at the sensor location $(x,y)$. $\bar{W}(m,n,\omega)$ is the 2D spatial Fourier transform of the vibration amplitude at sensor location $(x,y)$. The object's mass on the plate is encoded in $\bar{W}(m,n,\omega)$ as shown in equation \ref{eq2}, $\bar{F}(m,n,\omega)$ is the 2d spatial Fourier transform of the excitation force $f(x, y, \omega)$ which represents the time domain Fourier transform of force applied at the sensor location (x, y). $\rho$ is the mass per unit area of the shelf. $D$ is the bending stiffness of the plate. $\omega_{mn}$ is a function of $m$ and $n$.

% Taylor expansion can then transform the $m_{0}$ from the denominator to the numerator, implying the hidden linearity. Considering the fact that most items are much lighter than the shelf, saying $m_0$ is much smaller than weight per unit area of the shelf $\rho$, equation \ref{eq2} can be written as the first two items of the Taylor expansion with respect to $m_0$ as shown in equation \ref{eq3}. Since $\bar{W}(m,n,\omega)$ also keeps a linear relationship with the vibration frequency amplitude $w(x,y,\omega)$, object weight $m_0$ and vibration frequency amplitude $w(x,y,\omega)$ are coupled with a linear model when the locations of object and sensor are fixed. This linearity indicates a much smaller requirement of data collection with a linear machine learning model.

\begin{equation}
D\nabla^4 w + \rho \frac{\partial^2 w}{\partial t^2} = f(x, y, t) + \delta(x-x_0)\delta(y-y_0)(m_0g + m_0\frac{\partial^2 w}{\partial t^2})
\label{eq:gov_eq}
\end{equation}

\begin{equation}
\begin{gathered}
w(x, y, \omega) \propto  \bar{W}(m,n,\omega) \\
= \frac{\bar{F}(m,n,\omega)}{-\omega^2 \left( \rho + \textcolor{red}{m_0} \sin\left( \frac{m x_0 \pi}{a} \right) \sin\left( \frac{n y_0 \pi}{b} \right) \right) + D \omega^2_{mn}}
\label{eq:solution}
\end{gathered}
\end{equation}

\begin{equation}
\begin{gathered}
\approx \frac{\tilde{F}(m, n, \omega)}{-\omega^2 \rho + D \omega_{mn}^2} \left( 1 + \frac{\omega^2 \sin \left( \frac{m x_0 \pi}{a} \right) \sin \left( \frac{n y_0 \pi}{b} \right)}{-\omega^2 \rho + D \omega_{mn}^2} \textcolor{red}{m_0} \right)
\label{eq:taylor expansion}
\end{gathered}
\end{equation}

The vibration response $w(x,y,\omega)$ not only depends on the item weight $m_0$, but also the item location $(x_0, y_0)$ as shown in Equation ~\ref{eq:taylor expansion}. Although this linearity holds true for different item locations, the linear coefficients are different because they depend on $(x_0, y_0)$. Intuitively, this is because different item locations affect different modes of shelf structure. Thus, the shelf vibration response is affected in various ways. To this end, the theoretical derivation validates (1) The shelf vibration frequencies are linearly correlated with the item weight. (2) This linearity is kept across the gondola shelf.


\subsection{Physics-Informed Feature Extraction and Learning}
\label{sec: Feature Extraction and Learning}

\begin{figure*}[t]
% \setlength{\abovecaptionskip}{10pt}
    \centering
    \includegraphics[width=\linewidth]{Figure/Evaluation1.jpg}
    \caption{The WeVibe system evaluation. (a) The comparison between WeVibe and the other two methods: Using one linear model for all locations and using the non-linear model for each location while using 10\% of all weight classes in training and one vibration sensor. WeVibe outperforms both approaches with a significant improvement. (b)\&(c) The weight change estimation result of WeVibe, taking 10\% of 3 weight classes in training and one vibration sensor. The result suggests that WeVibe can almost certainly distinguish weight changes bigger than 100g, which can be utilized to detect whether a can of chips or a bottle of water is taken or put back.}
    \label{fig:Evaluation1}
\end{figure*}

With theoretical derivation and empirical study, WeVibe builds a solid physics-informed feature extraction and learning model. WeVibe takes the shelf vibration response from one whole impulse vibration as one sample so that the feature will contain the information of both transient and steady state~\cite{huang2020vibration}. WeVibe then takes the vibration frequency spectrum through the Fourier transform. As Figure~\ref{fig:frequency difference} suggests, the shift of item location leads to the change of weight-sensitive frequencies, which is challenging to know beforehand. Therefore, instead of choosing one or two specific weight-sensitive frequencies, WeVibe takes the full range of vibration frequency spectrum except for the noise-occupied region, 50Hz to 240Hz, as the feature for the learning model.
% WeVibe leverages structural dynamics knowledge combined with actual data to alleviate the data requirement for exploring the relationship between shelf response and object weight across different locations. WeVibe exploits the linearity between the vibration frequency spectrum and object weight to train a weight estimation model for known locations. Then, this model will be calibrated if a new object location is involved.

% The calibration leverages that the linearity is kept across most locations on the same shelf, but it has different coefficients. In other words, we assume that the actual weight change estimation at a new location deviates from the well-trained model's result with a constant ratio. Therefore, we can take a few weight change estimation results, calculate this constant ratio, and multiply it by the well-trained model's result, getting the final weight change estimation.

% \begin{equation}
% w(x, y, \omega) = \frac{4}{ab} \sum_{m=1}^{\infty} \sum_{n=1}^{\infty} \bar{W}(m, n, \omega) \sin\left(\frac{m \pi x}{a}\right) \sin\left(\frac{n \pi y}{b}\right)
% \label{eq1}
% \end{equation}


% \begin{equation}
% \bar{W}(m,n,\omega) = \frac{\bar{F}(m,n,\omega)}{-\omega^2 \left( \rho + \textcolor{red}{m_0} \sin\left( \frac{m x_0 \pi}{a} \right) \sin\left( \frac{n y_0 \pi}{b} \right) \right) + D \omega^2_{mn}}
% \label{eq2}
% \end{equation}


% \begin{equation}
% \approx \frac{\tilde{F}(m, n, \omega)}{-\omega^2 \rho + D \omega_{mn}^2} \left( 1 + \frac{\omega^2 \sin \left( \frac{m x_0 \pi}{a} \right) \sin \left( \frac{n y_0 \pi}{b} \right)}{-\omega^2 \rho + D \omega_{mn}^2} \textcolor{red}{m_0} \right)
% \label{eq3}
% \end{equation}


\section{WeVibe Evaluation}
\label{sec:System Evaluation}
We evaluate WeVibe's performance on a standard gondola with a simple item layout first and then a real-world item layout for weight change estimation from three aspects: ~\ref{eval:System} System, ~\ref{eval:data} The usage of Data, and \ref{eval:sensor} The usage of sensor.

\subsection{Experiment Setup}
\subsubsection{Vibration Sensing Environment Setup}
We evaluate WeVibe on a real gondola shelf that is commonly used in the real-world store (Figure~\ref{fig:Experiment Setup} (a)). The shelf is manufactured from MFired Store Fixtures\cite{mfried} with steel and mounted on a double-sided gondola. The shelf length is 91.44cm, and the width is 46.72cm. We keep the other shelves empty except for the one used for evaluation. We put one 1L water bottle on the shelf and estimate the weight change of water in the 1L bottle at four different locations as indicated in Figure~\ref{fig:Experiment Setup} (b). We also put an extra three 3.78L water bottles on the shelf because the whole shelf is rarely empty in real-world cases.

An Edifier D12 speaker plays an impulse signal composed of a short sinc function from a function generator to create a vibration. The sinc function has a 10Hz central frequency and 15V peak-to-peak voltage. It bursts once every 2s, as shown in the top right corner of Figure~\ref{fig:Experiment Setup} (a). The speaker is 20cm from the bottom right edge of the gondola and faces to the right. Additionally, to evaluate whether WeVibe can work on a gondola with a greater distance from the speaker, we move the speaker further away to 116cm, increasing the distance as the length of the shelf. 

\subsubsection{Vibration Sensing Module}
The vibration signal is captured by the vibration sensing module composed of (1) A geophone, (2) An amplifier \& filter circuit, and (3) An NI-9234\cite{ni9234} analog-to-digital converter (ADC) as shown in Figure~\ref{fig:Experiment Setup} (c). The SM-24 geophone is employed to collect vibration signals~\cite{SM_24} and has an enhanced frequency response between 10Hz and 240Hz. It has been shown with prospective performance on collecting structural vibration\cite{pan2017footprintid,bonde2021pignet,mirshekari2021obstruction,zhang2023vibration}. The geophone can measure vibrations by converting the velocity of the mechanical movement of the structure into an electrical signal that can be quantitatively analyzed. To increase the signal-to-noise ratio and remove the background noise, we design a programmable amplifier \& filter circuit. We set the gain to 35 and removed the frequency component outside SM-24 geophone functional bandwidth. Finally, the NI ADC converts the analog signal into a 24-bit digitalized signal with a sampling rate of 51.2kHz per second. The NI ADC also allows a synchronized collection of multiple-channel vibration signals, which prepares us to evaluate different sensor configurations (Section~\ref{eval:sensor}).

There are four sensing modules in total; one is on the speaker to monitor the input signal and is considered as the reference signal. The other three are placed in the middle of the right shelf edge, the middle of the front shelf edge, and the middle of the left shelf edge, and they are used to evaluate the performance of sensors at different configurations.

\subsubsection{Data Collection and Pre-Processing}
We collected ten weight classes by changing the amount of water in a 1L water bottle, ranging from 50g to 500g, with 50g spacing at four different locations with a 15.24cm interval, as shown in Figure ~\ref{fig:Experiment Setup} (b). The ground truth weight is the same as the weight of the 1L water bottle. We ignore the weight of the 3.78L water bottle because their weights are not changing. Each weight has 28 samples at each location. There are 1120 samples in total. The collected weight classes result in the absolute weight change from 0g to 450g with 50g spacing. Furthermore, we repeat the data collection on item locations 1 and 2 with a further distance(116cm) between the speaker and gondola to investigate whether WeVibe can work for a gondola at a new location.

After data collection, we pre-process the data into individual samples. All the vibration signals are first aligned with the signal collected on the top of the speaker for synchronization. Then, as shown in Figure~\ref{fig:Experiment Setup} (a), each sample begins 0.1s before the beginning of the impulse signal and lasts 1s so that it can fully capture the shelf responses in the excitation stage and steady stage\cite{huang2020vibration}. Then, we apply the Fourier transform to get the frequency spectrum and use frequencies from 50Hz to 240Hz with 1Hz spacing as the feature vector~\ref{eq:feature vector}. To further reduce the dimension of the feature vector and improve learning efficacy, Principle Components Analysis (PCA) ~\cite{mackiewicz1993principal} is then applied for every feature vector to extract the most valuable information. Finally, the features from the same location are fed into a linear regression model with L2 regularization for learning the weight information.

\begin{equation}
\begin{gathered}
Feature Vector = [f_{50Hz},f_{51Hz},...,f_{240Hz}] = [f_1,f_2,...,f_{191}]
\label{eq:feature vector}
\end{gathered}
\end{equation}

\subsection{Evaluation I: WeVibe Overall Performance}
\label{eval:System}
Figure~\ref{fig:Evaluation1} shows the result of WeVibe system evaluation. Figure~\ref{fig:Evaluation1} (a) indicates an ablation test of WeVibe methodology, additionally validating the rationality of assigning a linear model for each location. Figure~\ref{fig:Evaluation1} (b) and (c) show the weight change estimation result with one standard deviation range. After estimating the weight, the difference between any two weights is taken as the final result. Since the negative and positive weight differences are symmetric, only the positive weight differences are shown in the following figures.

We first compare WeVibe with the other algorithms as shown in Figure~\ref{fig:Evaluation1} (a). Taking 10\% of training samples (three samples) of ten weight classes and one sensor, WeVibe mean absolute error outperforms the method that uses one linear model for all locations with 4.2X improvement, proving that different linear relationships are kept at various locations. On the other hand, WeVibe's means absolute error outperforms the method that uses a non-linear model for each location with 17X improvement, indicating the correctness of the linear relationship. 

Taking 10\% of training samples of three weight classes (50g, 300g, and 500g) and one sensor, Figure~\ref{fig:Evaluation1} (b) and (c) suggest WeVibe performances for gondolas with different distances with speakers. When the distance is 20cm, WeVibe achieves an overall mean absolute error of 48.15g with a standard deviation of 44.96g. When the distance is 116cm, the overall mean absolute error is 38.07g, and the standard deviation is 31.2g. These results suggest that WeVibe can correctly differentiate weight change bigger than 100g in most cases, which only takes three weight classes at every location and one vibration sensor. It is accurate enough to be leveraged for detecting whether items like a can of chips or a bottle of water are taken or put back. For lightweight items such as a bar of chocolate, a fine granularity of training weight class might be necessary. A minimum of sensors attached to the surface and better data efficacy are also suggested, which will be detailed in the following sections. Furthermore, both distances achieve a similar result, indicating a better ubiquity because the linearity between shelf vibration response and item weight still exists when the gondola is at a different location. 

\subsection{Evaluation II: The Amount of Required Data}
\label{eval:data}

\begin{figure}[tbh]
% \setlength{\abovecaptionskip}{10pt}
    \centering
    \includegraphics[width=0.6\linewidth]{Figure/Evaluation2.jpg}
    \caption{(a) With two training weight classes, the smaller one is 50g, and the bigger one gradually increases. It can be seen that the error gets smaller with a bigger range. (b) Adding more training weight classes decreases the median and mean MAE, suggesting a tradeoff between data collection effort and performance. The improvement from two to three weight classes is more significant than the improvement from three to four, leading us to select three weight classes for evaluation.  (c) The increasing amount of training data per weight class gives comparable performances. Therefore, three samples are selected in our case for better data efficacy.}
    \label{fig:Evaluation2}
\end{figure}

Three evaluations are conducted to validate the improved data efficacy based on the physics-informed model. To evaluate the tradeoff between the performance and the amount of training data, we start from two weight classes (one weight change) and find that when the minimum and maximum weights are included in the training set, the performance reaches the best. The minimum weight 50g is kept in the training set and the bigger weight increases gradually (e.g., 50g and 100g, 50g and 150g,\ldots, 50g and 500g). As weight change gets bigger, the performance gets better. The inclusion of minimum weight and maximum weight reaches the best, as shown in Figure~\ref{fig:Evaluation2} (a). It reflects that when the training data includes the full range of weight information, it can better capture the weight falling within this range, which is very similar to the property of a linear model.

Then, the number of training weight classes increases while keeping the minimum and maximum weight in the training set, evaluating the tradeoff between performance and the number of weight classes. The number of weight classes increases by interpolation: 50g and 500g; 50g, 300g and 500g; 50g, 200g, 350g, and 500g. Figure~\ref{fig:Evaluation2} (b) suggests that the increasing number of training weight classes provides an improvement. However, the improvement is more evident from two to three weight classes. The performance of three weight classes and four weight classes are very close. In the actual application, the number of necessarily collected weight classes significantly depends on the possible slightest weight change between items. In this case, three training weight classes are employed because there is no significant improvement in selecting four training weight classes.

Figure~\ref{fig:Evaluation2} (c) suggests that 10\% training data for each weight class is enough. When keeping three weight classes selected from the previous evaluation, we investigate the performance while increasing the amount of training data per weight class. It turns out that the increasing amount of training data per weight class gives comparable performances. It is probably because most data samples within the same weight class are similar. Referring back to Section~\ref{sec:theory}, the vibration frequency spectrum should be much the same as long as the item weight and item location do not change. Therefore, we select 10\% training data to minimize the data collection effort.

\subsection{Evaluation III: The Amount of Required Sensor}
\label{eval:sensor}
Figure~\ref{fig:Evaluation3} shows the performance comparison with different sensor configurations in the training process, suggesting that the combination of three sensors is comparable to a single sensor with the best performance. Figure~\ref{fig:Experiment Setup} (b) shows the location of three different sensors. Figure~\ref{fig:Evaluation3} shows that the performance of sensor 1 is the best out of three single-sensor usage cases, and it is very close to the best performance: sensor 1\&3. This observation leads us to employ sensor 1 to minimize the sensor usage. On the other hand, it is also observed that the combination of three sensors is also very close to the best performance. It results in a compromise of using all three sensors if we don't know which one can give the best performance, which is the case of our real item-layout evaluation.

\begin{figure}[t]
% \setlength{\abovecaptionskip}{10pt}
    \centering
    \includegraphics[width=0.8\linewidth]{Figure/Evaluation3.jpg}
    \caption{This figure illustrates the tradeoff between weight change estimation and different sensor configurations used during training. Sensor 1\&3 reaches the best performance, but sensor 1 and sensor1\&2\&3 also have a close performance. Given no prior knowledge of which sensor can perform the best, it is reasonable to apply all three sensors. However, if we know the best sensor, it should be used to reduce the sensor cost.}
    \label{fig:Evaluation3}
\end{figure}

\subsection{Evaluation IV: Real Item-Layout Evaluation}

\begin{figure}[tbh]
% \setlength{\abovecaptionskip}{10pt}
    \centering
    \includegraphics[width=\linewidth]{Figure/Evaluation_Real.jpg}
    \caption{WeVibe is evaluated on a real item-layout shelf. We bought a shelf of protein bars and placed them under the WeVibe sensing environment. With a mean absolute error of 41.05g and a standard deviation of 42.2g, it is expected to detect the case that two or more protein bars are taken or put back according to the weight change estimation.}
    \label{fig:Evaluation Real}
\end{figure}

As shown in Figure~\ref{fig:Evaluation Real}, WeVibe is evaluated with a real item-layout. We bought a shelf of protein bars and placed them on our shelf with the same organization in the store. Six weight samples are collected for each box: Full box, one protein bar is taken,\ldots, and five protein bars are taken. Each protein bar ranges from 59g to 64g, so an average weight of 61g is assigned for each protein bar. The label is the weight of the total shelf load, ranging from 2997g to 3302g (the sum of five boxes). The training set comprises 10\% samples of three weight classes (2997g, 3180g, and 3302g). Three sensors are used together because we assume no prior knowledge of the best sensor is provided.

WeVibe achieves a mean absolute error of 41.05g and a standard deviation of 42.2g for all locations. It suggests that WeVibe can successfully identify these cases based on the weight change estimation result when two or more protein bars are taken or returned. For each location, WeVibe shows various performances. Location 1 gives the best result with 23.34g mean absolute error and 28.3g standard deviation, which likely works for one protein bar. Location 5 shows a more considerable standard deviation, which might not be functional for the lightweight items in the actual store. It might be due to the shelf's elastic deformation. The amount of elastic deformation of the shelf surface might bring extra variation on the shelf vibration response, leading to more data collection at some specific locations. At location 5, more weight classes are required to provide better estimation because there might be a small amount of elastic deformation happening at these locations when the item weight changes, leading to more variances in the final weight change estimation.







%
\section{Conclusion}

This paper introduces a variant of the multivariate time series traffic prediction problem with a focus on highly sparse and unstructured observations.
To address this problem we propose SUSTeR, a framework which handles sparse unstructured observations by creating hidden graphs in a residual fashion, which are then used with a conventional spatio-temporal GNN.
SUSTeR achieves better predictions for high sparsity (80\% - 99.9\% missing data) than existing baselines and remains competitive in denser settings or even when using only half the amount of the training data.
In addition, its training is considerably faster than the next-best competitor due to a smaller model size.

% We conduct experiments on a unstructured and sparse version of the traffic dataset Metr-LA and compare the performance of SUSTeR with traffic prediction baselines.
% The consideration of the sparsity within SUSTeR outperforms other approaches at sparsity rates $\geq$99\%.
% Experiments were performed up to a sparsity with only 2.4 observations within a sample where without missing data such a sample contains 12$\times$207 values.
% Further, the ablation studies explore the influence of our design choices and show the robustness of our framework.


\section{Future Work}

We plan to explore the interpretability within SUSTeR to obtain an intuitive understanding of the graph nodes within the hidden graph.
Small design choices are made within SUSTeR to make this possible, from observations that are not relying on each other in the same timestep, variable amounts of observations, a learnable assignment function from the observation to the hidden node, and an explicit learned laplacian matrix. 
The problem of sparse unstructured observations, which should be reconstructed into a hidden state, is present in many other domains.
In particular ocean data is a very promising application field for SUSTeR where sparse ARGO\footnote{https://argo.ucsd.edu} observations would perfectly match the problem definition to predict ocean states. 
There, observations are typically spatially and temporally sparse - comparable to the highest dropout rate in this paper - and observations are non-stationary and change their position freely.
We see SUSTeR as a bridge of the well-studied spatio-temporal mining methods into a new area of domains, in which such methods previously were not applicable.

\input{06_Related_Work.tex}
Game-based approaches have shown great promise as tools for inoculating individuals against the tactics commonly used to spread misinformation. Most existing games in this domain are single-player games which offer players limited, predefined choices. While this design reduces cognitive load, it often results in interactions which feel less natural and engaging. In response, we designed a two-player, PvP game that pits a misinformation creator against a misinformation stopper. By integrating LLM-powered personas to evaluate player outputs and provide real-time feedback, we created a more open-ended and immersive experience.
We found that the game we developed effectively improved players’ media literacy. Participants demonstrated an enhanced ability to evaluate and analyze media content, identify unreliable or misleading information, and employ effective counter-misinformation strategies. Moreover, the game's engaging mechanics, combined with the competitive element, motivated players to learn from both their own strategies and those of their opponents.
These findings suggest that integrating dynamic feedback systems and competitive gameplay elements into misinformation education games offers a compelling method to deepen users' engagement, while also improving their critical media skills. Future research can build on these insights to explore other forms of interactive learning environments, focusing on diverse player experiences and varying misinformation challenges.


%%
%% The next two lines define the bibliography style to be used, and
%% the bibliography file.
\bibliographystyle{ACM-Reference-Format}
\bibliography{main}
%%
%% If your work has an appendix, this is the place to put it.

\end{document}

%%
%% The code below is generated by the tool at http://dl.acm.org/ccs.cfm.
%% Please copy and paste the code instead of the example below.
%%
\begin{CCSXML}
<ccs2012>
 <concept>
  <concept_id>00000000.0000000.0000000</concept_id>
  <concept_desc>Do Not Use This Code, Generate the Correct Terms for Your Paper</concept_desc>
  <concept_significance>500</concept_significance>
 </concept>
 <concept>
  <concept_id>00000000.00000000.00000000</concept_id>
  <concept_desc>Do Not Use This Code, Generate the Correct Terms for Your Paper</concept_desc>
  <concept_significance>300</concept_significance>
 </concept>
 <concept>
  <concept_id>00000000.00000000.00000000</concept_id>
  <concept_desc>Do Not Use This Code, Generate the Correct Terms for Your Paper</concept_desc>
  <concept_significance>100</concept_significance>
 </concept>
 <concept>
  <concept_id>00000000.00000000.00000000</concept_id>
  <concept_desc>Do Not Use This Code, Generate the Correct Terms for Your Paper</concept_desc>
  <concept_significance>100</concept_significance>
 </concept>
</ccs2012>
\end{CCSXML}

\ccsdesc[500]{Do Not Use This Code~Generate the Correct Terms for Your Paper}
\ccsdesc[300]{Do Not Use This Code~Generate the Correct Terms for Your Paper}
\ccsdesc{Do Not Use This Code~Generate the Correct Terms for Your Paper}
\ccsdesc[100]{Do Not Use This Code~Generate the Correct Terms for Your Paper}

%%
%% Keywords. The author(s) should pick words that accurately describe
%% the work being presented. Separate the keywords with commas.
\keywords{Do, Not, Us, This, Code, Put, the, Correct, Terms, for,
  Your, Paper}
%% A "teaser" image appears between the author and affiliation
%% information and the body of the document, and typically spans the
%% page.



\end{document}
\endinput
%%
%% End of file `sample-sigconf-authordraft.tex'.
