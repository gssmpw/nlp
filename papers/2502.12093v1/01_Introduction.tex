\section{Introduction}
\label{sec:Section1}
% Motivation Paragraph: Weight change estimation is important in the autonomous store.
Item weight change estimation is crucial in many applications~\cite{zhang2020vibroscale,he2013food,bonde2021pignet}, especially in the context of autonomous retail environments~\cite{ruiz2019aim3s,falcao2020faim,rohal2024don}. Autonomous stores, equipped with technologies to identify which items customers have taken, facilitate automatic checkouts without the need for cashiers. This process typically relies on camera systems; however, cameras can fail when their view is blocked. Consequently, integrating precise weight estimation can significantly enhance the robustness of item detection and ensure accurate cost assessments, thereby overcoming limitations posed by visual-only systems.

\begin{figure*}[t]
% \setlength{\abovecaptionskip}{10pt}
    \centering
    \includegraphics[width=0.8\linewidth]{Figure/System_Overview.jpg}
    \caption{WeVibe system diagram comprises (a) An audio-induced vibration module designed for the active vibration sensing environment. (b) The vibration sensing module that takes shelf vibration response when there are different weights on the shelf. (c) The weight change estimation module that leverages physics-informed knowledge to extract features and develop a learning model for weight estimation. The weight change is achieved by calculating the difference between the two weight estimation results.}
    \label{fig:Figure2}
\end{figure*}

% Related Work Paragraph => What problem?: 1: Heavy usage of sensor.
However, the conventional implementation of weight change estimation in autonomous stores incurs significant sensor costs. The recent development of autonomous stores employs smart pressure-based weight sensing shelves to achieve automatic item detection~\cite{ruiz2019aim3s,falcao2021isacs}, which involves densely installed weight scales underneath the items. Even though these weight-sensing shelves provide accurate estimation, each shelf must be equipped with 12 to 24 weight sensors to monitor changes comprehensively~\cite{ruiz2019aim3s,gu2022tracking} and achieve precise analysis of item weights at various locations on the shelf, heavily elevating the sensor cost. Furthermore, considering more gondolas in the stores, densely installed weight scales might lead to a higher total cost and pose scalability challenges.

% Insight + Related Work Paragraph => What Insight: Vibration is correlated with Weight. What research challenge of using vibration to estimation weight: Unknown relationship between vibration and item weight.
Vibration sensing has shown great promise in estimating the weight, but the unknown relationship between vibration and item weight on the shelf prevents it from being used in autonomous stores. Many prior works indicate that the change of weight results in different structural responses~\cite{sekiya2018simplified,codling2021masshog,mirshekari2016characterizing, mirshekari2021obstruction}, inspiring us to explore the capability of vibration on estimating item weight change on the gondola shelf. Nevertheless, these works focus more on modeling the relationship between the single item staying at the same location and the structure vibration response. It cannot be adopted in complex scenarios like autonomous stores because the weight change may happen at different locations on the shelf, and there are other items that stay together. It is also impractical to collect a large amount of data to model the relationship between shelf vibration response, item weight, and item location because the variability introduced by different item weights and their locations leads to a vast number of possible data scenarios.

To tackle these problems, we make the following three contributions:
\begin{enumerate}[label=(\arabic*)]
  \item We propose WeVibe: The first system that utilizes audio-induced vibrations from a speaker to detect weight changes on the shelf during shopping using one vibration sensor at best.
  \item We model a structure-dynamics-informed relationship between the shelf vibration response and item weight across multiple locations on the shelf, allowing an easy training of the weight estimation model at a new location without heavy data collection.
  \item  We validate our system with a real-world shopping layout, demonstrating the efficacy of WeVibe in real-world scenarios.
\end{enumerate}
%better data efficacy
% Brief Introduction on WeVibe
WeVibe adopts active vibration sensing at a lower sensor cost. Since the items on the shelf cannot generate vibration by themselves, WeVibe takes a speaker as an exterior vibration source and estimates item weight change by analyzing the shelf vibration signal. The vibration wave from the speaker can propagate to the shelf through environmental structures such as the floor and gondola, creating a vibration pattern across the entire shelf. Therefore, one vibration sensor can capture the difference of the whole shelf vibration response induced by the weight change, reducing the number of sensors.

On the other hand, WeVibe features a physics-informed relationship (linear model) between the shelf vibration response and item weight across different item locations, improving the data efficacy. Through empirical observation, we discover that the increasing or decreasing item weight leads to the same rising or falling trend on some frequencies of impulse vibration propagated on the shelf, resulting in an assumption of linearity. Then, we justify this linearity through the theoretical derivation, developing this structure-dynamics-informed relationship between the vibration response and item weight as a linear model. This physics-informed knowledge solidifies the linear assumption during the learning process, allowing WeVibe to adapt to a new location by training a weight estimation model easily for each location of interest with two or three weight classes, alleviating the data collection effort compared with the deep learning method without prior knowledge.

The rest of the paper is organized as follows: Section ~\ref{sec:System Overview} provides a system overview of WeVibe, illustrating the active vibration sensing setup for less sensor and how WeVibe processes the weight change estimation. Section ~\ref{sec:Structure-dynamics-informed modeling} explains the characterization of the structure-dynamics-informed relationship for less data collection, followed by the evaluation in Section ~\ref{sec:System Evaluation}. Section ~\ref{sec:Related Work} goes through an overview of the autonomous store and then introduces the related work of weight sensing. Finally, we conclude our work in Section ~\ref{sec:Conlcusion}.

% Previous works leverage passive vibration sensing for weight estimation to mitigate the cost problem. The weight information can be extracted from the structural vibration response impacted by objects placed on the structure. This structural vibration response can be sensed by at least one vibration sensor attached to the structure, reducing the hardware consumption.
% For example, \cite{sekiya2018simplified,obrien2020using,yu2016state} estimate a bridge's average daily traffic load according to vibration between the car and bridge. \cite{bonde2021pignet} utilizes the vibration from the pig's movement to track its growth status. These works have a much smaller number of sensors installed, alleviating the high hardware and maintenance cost problems. However, they are limited to estimating the dynamic objects that can generate vibration by themselves. Therefore, it still does not fit into the autonomous store because the objects are static.

% To overcome this challenge, prior works explore adding an external vibration source, saying active vibration sensing, for items that cannot generate vibration alone. According to the structural dynamics, the additional weight can change the original structural response to the exterior vibration source. For example, \cite{codling2021masshog} can track the pig's weight using the vibration from a speaker. \cite{mirshekari2016characterizing} estimates the object's weight on a walking path through footstep-generated vibration. \cite{zhang2020vibroscale} turns the smartphone into a weighing scale by utilizing the motor and accelerator in the phone. All these works indicate the feasibility of active vibration sensing in achieving static object weight change information.

% WeVibe addresses two key research challenges: (1) The inconsistent estimation of the identical weight at different locations on the plate and (2) The absence of specific vibration signal datasets based on the store's items. To overcome these challenges, WeVibe builds a location-based weight estimation algorithm incorporating the structural dynamics to measure the weight at different locations and not require heavy data collection. According to our derivation in Section 3, the frequency spectrum of the vibration signal shows a linear relationship with the item weight when the weight fluctuates within a certain range. This linearity is kept across different locations, albeit with other coefficients. On the other hand, the vibration frequency envelope is unique at each location. Using these insights, WeVibe first trains a ridge regression model for each location of interest. It then localizes the vibration signal and estimates weight with the correct regression model according to the localization result.

% Using commonly found shelves and plates, we further evaluate WeVibe in a real-world retail environment built in the lab. This work focuses mainly on the single-item case, with plans to explore the multiple-item instances in the future work section. WeVibe significantly improves weight estimation accuracy, achieving a 1.7x reduction in error compared to baseline methods that do not account for location variances. Then, the linearity derived from structural dynamics insight is also validated, demonstrating that our approach requires much less data than traditional deep learning but can still achieve better results.
