\section{Related Work}
\label{sec:Related Work} 
A range of sensing technologies are employed in autonomous stores for item detection. We examine these works and then focus on weight estimation through vibration sensing, which inspired WeVibe.


\subsection{Item Detection In Autonomous Store}
\subsubsection{Vision-only.} The most prevalent sensing approach for item detection involves extensive camera deployment. This approach employs deep-learning object recognition models \cite{he2017mask,wang2023yolov7,tan2020efficientdet} to identify what items customers take. Besides direct prediction, works like \cite{9200182,torres2018text} propose further incorporating the text information on the item to augment the detection accuracy. However, the vision-only approach has three main drawbacks. First, occlusions in the camera line-of-sight hinder accurate item detection. Second, comprehensive store coverage demands many cameras, leading to substantial computational resource consumption. Finally, developing a precise and prompt deep-learning model demands a large and robust dataset, making the training process labor-intensive.


\subsubsection{Vision with other sensing modalities} Introducing other modalities makes the item detection pipeline more robust. For example, \cite{roussos2006enabling,zhang2016mobile} employed RFID with camera. By tagging each item, the accuracy is significantly improved. However, the consequent tagging effort on every item and the incremental cost of each RFID tag quickly make this approach impractical for widespread adoption. Weight sensing in store is pioneered in Aim3s \cite{ruiz2019aim3s}, followed by works with further robust fusion algorithms with the camera system\cite{falcao2021isacs,falcao2020faim}. This method relies on specialized weight-sensing plates capable of discerning weight changes at column-wise locations. However, the necessity for a complete replacement of conventional shelf plates for these specialized units is prohibitive. Additionally, the vulnerability of these plates to wear and malfunction poses further challenges.


\subsection{Weight Estimation Using Vibration}
Weight estimation based on vibration has been scrutinized in structural dynamics. Various theoretical models for analyzing the characteristic behaviors of the beam, like mass distribution and crack location through the vibration frequency spectrum, have been proposed in ~\cite{low2003natural,liu2020diagnosis,Matsumoto2003Mathematical,wynne2022quantifying}. One of the most popular research topics based on these theoretical derivations is bridge health monitoring by estimating the traffic loads through car-generated vibrations\cite{sekiya2018simplified,obrien2020using,yu2016state,liu2023telecomtm}. A similar idea is also utilized for detecting the location of cracks ~\cite{liu2020damage,nguyen2010multi}. Inspired by these works, ~\cite{bonde2021pignet, dong2023pigsense} leverages the vibration from the pig's movement to monitor the weight gain and other activities of piglets. However, these objects of interest can generate vibrations through themselves. Therefore, these approaches are not adaptable in the store because the item is stationary.

An exterior vibration source is employed for the static item weight estimation, which is referred as active vibration sensing. For example, ~\cite{mirshekari2021obstruction} proves that the footstep-generated vibration can be employed to estimate the weight of the item lying between the footstep and the vibration sensor. ~\cite{codling2021masshog} utilizes a speaker in the pigpen and models a relationship between the pig's weight and its impact on the speaker-generated vibration. However, their systems can only estimate the weight based on the kilogram scale. Vibsense~\cite{liu2017vibsense} utilizes a piezo speaker to generate vibration on a surface actively and identifies the impact of different weights on this vibration signal with a gram scale. VibroScale~\cite{zhang2020vibroscale} turns the smartphone into a weight scale by using the motor and accelerometer in the phone. Nevertheless, none of these systems accounted for the effects of the change of item location on weight estimation. 

To the best of our knowledge, WeVibe represents the first item weight change estimation system using active shelf vibration sensing. WeVibe leverages the physical knowledge from the structural dynamics to characterize the relationship between shelf vibration response and item weight at different locations, which turns out to be linear. WeVibe can quickly adapt to a new location with minimum amount of sensor attached to shelf and data collection effort. 