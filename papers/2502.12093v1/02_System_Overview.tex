\section{System Overview}
\label{sec:System Overview}
This section discusses the WeVibe system overview using active vibration sensing and structure-dynamics-informed modeling for item weight change estimation. WeVibe constructs an active vibration sensing environment through the audio-induced shelf vibration (Section~\ref{sec:audio induced vibration}), reducing the sensor cost. By playing sound from a speaker, a mechanical vibration wave is generated. This vibration will propagate through the structure, like the floor, and arrive at the gondola shelf. The vibration sensing module then captures the shelf vibration response (Section~\ref{sec: vibration sensing}). When the item weight changes, it will impact the original shelf vibration differently, which can be leveraged to estimate the item weight. Through the knowledge of structural dynamics, WeVibe characterizes these different shelf responses and develops the physics-informed features and machine-learning model to adapt to a new item location for weight estimation quickly. Finally, the difference between the two weight estimations is taken as the weight change estimation result (Section~\ref{sec: Weight Change Estimation}).

% WeVibe actively vibrates the shelf with the vibration from the speaker to prepare a vibrational environment in the activation module. The vibration wave then propagates to the whole shelf. The vibration sensor board collects these signals and sends them to the vibration signal handler module. The frequency spectrum will then be extracted. To handle the inconsistency of estimation at different locations, WeVibe localizes each signal first and assigns a location-dependent machine-learning model to estimate the weight. To fix the lack of a dataset problem, WeVibe characterizes the relationship between weight and vibration frequency spectrum as a linear regression model according to the structural dynamics.
\subsection{Audio-Induced Vibration}
\label{sec:audio induced vibration}
% What kind of signal is better for the speaker to generate vibration: Pulse, constant...
% Plate vibration is mostly induced by structural-borne vibration instead of air-borne vibration
WeVibe creates an active vibration sensing environment by playing sound from a speaker next to the gondola. When the speaker plays sound, the interior components like the cone and voice coil generate mechanical vibration. These vibration waves can propagate through the environment by exciting the structure particle movement. There are two basic wave types: Longitudinal wave and transverse wave~\cite{noh2023dynamics,yuan2022spatial}. The particles moving parallel to the wave propagation form the longitudinal wave. The particles moving perpendicular to the wave propagation form the transverse wave. With these two wave types, the mechanical vibration from the speaker can travel to the shelf.

A periodic impulse is selected to burst out a strong vibration wave capable of reaching the shelf. The shelf response resulting from each impulse is taken as one sample for weight estimation. Many types of sound can be potentially adopted for developing the vibration wave, such as constant tone, frequency sweep, and impulse. To optimize the performance of WeVibe, a wider frequency band and a more vigorous signal intensity are desired. We employ an impulse signal. The impulse signal bursts out intensively in a very brief period, so it can provide a sudden and forceful push to the speaker's components, causing a strong vibration. Additionally, the impulse sound has a broad frequency spectrum, preparing a vast feature pool for further signal processing. The speaker's volume is set as an average person's speaking volume while keeping enough intensity that the vibration sensor can get the signal.


\subsection{Vibration Sensing}
\label{sec: vibration sensing}
The weight of an item placed on a surface influences the vibration signal (as shown in Figure~\ref{fig:frequency difference}) due to the changes in the structural properties (e.g., mass, stiffness, and damping ) after an item is added or removed. When a heavier item is placed on a surface, it tends to absorb and dampen the vibrations more significantly, leading to variations in signal amplitude, frequency response, and decay rate compared to lighter items. Conversely, a lighter item has a weaker impact on the vibration signal with different characteristics. These distinctions arise because the weight affects the surface's structural properties. By accurately capturing and analyzing these variations in vibration signals, it is possible to determine the item's weight with high precision.

On the other hand, the location of where the weight change happens also affects the shelf vibration response. Through a physics-informed characterization of the shelf vibration response, item weight, and item location, WeVibe quickly provides each location with a dedicated learning model, which will be explained in Section ~\ref{sec:Structure-dynamics-informed modeling}.

\begin{figure}
% \setlength{\abovecaptionskip}{10pt}
    \centering
    \includegraphics[width=\linewidth]{Figure/Frequency_Difference.jpg}
    \caption{(a) The plot shows the frequency spectrums of different weights of a single item at two locations. For the same location, some weight-sensitive frequencies increase or decrease while the weight of the item increases, as indicated by the red circle. Furthermore, when the item changes location, the overall frequency spectrum has a more significant change, and the weight-sensitive frequencies also shift. (b) We Further plot the result of linear regression on the highlighted frequencies and item weight at both locations. Even though some points deviate from the fitted line, the visualization gives rise to the assumption of linearity.}
    \label{fig:frequency difference}
\end{figure}

Adopting the active vibration sensing method allows fewer sensors to be attached to the shelf than the smart weight sensing shelf. This is because the vibration signal captured by one single sensor can still effectively represent the structural characteristics of the entire shelf, which we will discuss more in Section ~\ref{sec:Structure-dynamics-informed modeling}. Studies have shown that these single-point vibration signals are helpful in detecting structural anomalies and assessing the integrity of various structures\cite{sekiya2018simplified,obrien2020using,yu2016state}. This insight allows WeVibe to employ one vibration sensor to capture the shelf's vibrational response and indicate weight information through further signal processing. Multiple sensors can also be attached to the shelf to improve the system's robustness and accuracy.
% WeVibe employs multiple sensors are employed to characterize the vibration signal changes at lower frequencies to improve the robustness and accuracy of the system. Each sensor can generate a weight estimation of the item, which will be fused together and generate the final result. On the other hand, WeVibe focuses more on lower frequencies because they have lower attenuation and better signal-to-noise ratio during the process of wave propagation in the physical structure\cite{ma2023effective}. WeVibe employs a low-cost vibration sensor, SM-24 geophone\cite{SM_24}, which measures the ambient vibration by converting the velocity of a surface into voltage. It has a desired frequency response between 10Hz and 240Hz. A snapshot of time and frequency domain vibration signals is shown in Figure 3.

\subsection{Weight Change Estimation}
\label{sec: Weight Change Estimation}
We leverage the knowledge of structural dynamics to characterize a physical model between the shelf vibration response and item weight to reduce the need for data collection. Through empirical studies, we first notice that vibration frequencies differ when the item weight changes. Furthermore, some frequency amplitudes show the same increasing or decreasing trend while item weight increases or decreases. Therefore, we assume a linear model between the vibration frequency spectrum and item weight. This assumption is then validated through the theoretical derivation based on the structural dynamics, detailed in section~\ref{sec:Structure-dynamics-informed modeling}. Therefore, WeVibe applies the physics-informed feature extraction and learning model to the shelf responses to estimate the item's weight. Given a new location, Wevibe can exploit this physics-informed relationship to quickly train a weight estimation model with two weight classes at best and apply the estimation to a broader range of weight. Finally, the difference between the two weight estimations is taken as the weight change estimation result.


% The weight estimation features the incorporation of structural dynamics into the analysis of the shelf response to help address the challenges of location variances and the lack of a dataset. The shelf response represented by the vibration frequency spectrum conveys information about the item's location and weight. Therefore, the frequency spectrum will first be extracted as localization and weight estimation features. WeVibe adopts a frequency between 80Hz and 240Hz with a 1Hz interval as the feature vector to avoid environmental noise and focuses on shelf responses. Therefore, the feature vector has a total number of 161 frequency bins, as shown below.
% \begin{equation}
% Feature Vector=[f_{1},f_{2},...,f_{161}] = [f_{80Hz},f_{81Hz},...,f_{240Hz}]
% \end{equation}

% We then collect data at each item location and prepare one localization model and multiple weight estimation models corresponding to each location. We discover that the general pattern of the frequency spectrum envelope is unique when the item is placed at a fixed location, even with different weights. Therefore, WeVibe first utilizes a support vector machine (SVM) to differentiate the item location. Given the localization result, WeVibe assigns the prepared model to estimate the item weight, which addresses the location variances.

% To build a weight estimation model for each location, WeVibe leverages the linearity between the vibration frequency spectrum and item weight to reduce the need for training data at multiple locations. The linearity indicates using a linear model as a weight estimation model, avoiding extensive data collection and addressing the lack of a specific dataset problem.


