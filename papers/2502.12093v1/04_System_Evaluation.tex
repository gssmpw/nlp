\section{WeVibe Evaluation}
\label{sec:System Evaluation}
We evaluate WeVibe's performance on a standard gondola with a simple item layout first and then a real-world item layout for weight change estimation from three aspects: ~\ref{eval:System} System, ~\ref{eval:data} The usage of Data, and \ref{eval:sensor} The usage of sensor.

\subsection{Experiment Setup}
\subsubsection{Vibration Sensing Environment Setup}
We evaluate WeVibe on a real gondola shelf that is commonly used in the real-world store (Figure~\ref{fig:Experiment Setup} (a)). The shelf is manufactured from MFired Store Fixtures\cite{mfried} with steel and mounted on a double-sided gondola. The shelf length is 91.44cm, and the width is 46.72cm. We keep the other shelves empty except for the one used for evaluation. We put one 1L water bottle on the shelf and estimate the weight change of water in the 1L bottle at four different locations as indicated in Figure~\ref{fig:Experiment Setup} (b). We also put an extra three 3.78L water bottles on the shelf because the whole shelf is rarely empty in real-world cases.

An Edifier D12 speaker plays an impulse signal composed of a short sinc function from a function generator to create a vibration. The sinc function has a 10Hz central frequency and 15V peak-to-peak voltage. It bursts once every 2s, as shown in the top right corner of Figure~\ref{fig:Experiment Setup} (a). The speaker is 20cm from the bottom right edge of the gondola and faces to the right. Additionally, to evaluate whether WeVibe can work on a gondola with a greater distance from the speaker, we move the speaker further away to 116cm, increasing the distance as the length of the shelf. 

\subsubsection{Vibration Sensing Module}
The vibration signal is captured by the vibration sensing module composed of (1) A geophone, (2) An amplifier \& filter circuit, and (3) An NI-9234\cite{ni9234} analog-to-digital converter (ADC) as shown in Figure~\ref{fig:Experiment Setup} (c). The SM-24 geophone is employed to collect vibration signals~\cite{SM_24} and has an enhanced frequency response between 10Hz and 240Hz. It has been shown with prospective performance on collecting structural vibration\cite{pan2017footprintid,bonde2021pignet,mirshekari2021obstruction,zhang2023vibration}. The geophone can measure vibrations by converting the velocity of the mechanical movement of the structure into an electrical signal that can be quantitatively analyzed. To increase the signal-to-noise ratio and remove the background noise, we design a programmable amplifier \& filter circuit. We set the gain to 35 and removed the frequency component outside SM-24 geophone functional bandwidth. Finally, the NI ADC converts the analog signal into a 24-bit digitalized signal with a sampling rate of 51.2kHz per second. The NI ADC also allows a synchronized collection of multiple-channel vibration signals, which prepares us to evaluate different sensor configurations (Section~\ref{eval:sensor}).

There are four sensing modules in total; one is on the speaker to monitor the input signal and is considered as the reference signal. The other three are placed in the middle of the right shelf edge, the middle of the front shelf edge, and the middle of the left shelf edge, and they are used to evaluate the performance of sensors at different configurations.

\subsubsection{Data Collection and Pre-Processing}
We collected ten weight classes by changing the amount of water in a 1L water bottle, ranging from 50g to 500g, with 50g spacing at four different locations with a 15.24cm interval, as shown in Figure ~\ref{fig:Experiment Setup} (b). The ground truth weight is the same as the weight of the 1L water bottle. We ignore the weight of the 3.78L water bottle because their weights are not changing. Each weight has 28 samples at each location. There are 1120 samples in total. The collected weight classes result in the absolute weight change from 0g to 450g with 50g spacing. Furthermore, we repeat the data collection on item locations 1 and 2 with a further distance(116cm) between the speaker and gondola to investigate whether WeVibe can work for a gondola at a new location.

After data collection, we pre-process the data into individual samples. All the vibration signals are first aligned with the signal collected on the top of the speaker for synchronization. Then, as shown in Figure~\ref{fig:Experiment Setup} (a), each sample begins 0.1s before the beginning of the impulse signal and lasts 1s so that it can fully capture the shelf responses in the excitation stage and steady stage\cite{huang2020vibration}. Then, we apply the Fourier transform to get the frequency spectrum and use frequencies from 50Hz to 240Hz with 1Hz spacing as the feature vector~\ref{eq:feature vector}. To further reduce the dimension of the feature vector and improve learning efficacy, Principle Components Analysis (PCA) ~\cite{mackiewicz1993principal} is then applied for every feature vector to extract the most valuable information. Finally, the features from the same location are fed into a linear regression model with L2 regularization for learning the weight information.

\begin{equation}
\begin{gathered}
Feature Vector = [f_{50Hz},f_{51Hz},...,f_{240Hz}] = [f_1,f_2,...,f_{191}]
\label{eq:feature vector}
\end{gathered}
\end{equation}

\subsection{Evaluation I: WeVibe Overall Performance}
\label{eval:System}
Figure~\ref{fig:Evaluation1} shows the result of WeVibe system evaluation. Figure~\ref{fig:Evaluation1} (a) indicates an ablation test of WeVibe methodology, additionally validating the rationality of assigning a linear model for each location. Figure~\ref{fig:Evaluation1} (b) and (c) show the weight change estimation result with one standard deviation range. After estimating the weight, the difference between any two weights is taken as the final result. Since the negative and positive weight differences are symmetric, only the positive weight differences are shown in the following figures.

We first compare WeVibe with the other algorithms as shown in Figure~\ref{fig:Evaluation1} (a). Taking 10\% of training samples (three samples) of ten weight classes and one sensor, WeVibe mean absolute error outperforms the method that uses one linear model for all locations with 4.2X improvement, proving that different linear relationships are kept at various locations. On the other hand, WeVibe's means absolute error outperforms the method that uses a non-linear model for each location with 17X improvement, indicating the correctness of the linear relationship. 

Taking 10\% of training samples of three weight classes (50g, 300g, and 500g) and one sensor, Figure~\ref{fig:Evaluation1} (b) and (c) suggest WeVibe performances for gondolas with different distances with speakers. When the distance is 20cm, WeVibe achieves an overall mean absolute error of 48.15g with a standard deviation of 44.96g. When the distance is 116cm, the overall mean absolute error is 38.07g, and the standard deviation is 31.2g. These results suggest that WeVibe can correctly differentiate weight change bigger than 100g in most cases, which only takes three weight classes at every location and one vibration sensor. It is accurate enough to be leveraged for detecting whether items like a can of chips or a bottle of water are taken or put back. For lightweight items such as a bar of chocolate, a fine granularity of training weight class might be necessary. A minimum of sensors attached to the surface and better data efficacy are also suggested, which will be detailed in the following sections. Furthermore, both distances achieve a similar result, indicating a better ubiquity because the linearity between shelf vibration response and item weight still exists when the gondola is at a different location. 

\subsection{Evaluation II: The Amount of Required Data}
\label{eval:data}

\begin{figure}[tbh]
% \setlength{\abovecaptionskip}{10pt}
    \centering
    \includegraphics[width=0.6\linewidth]{Figure/Evaluation2.jpg}
    \caption{(a) With two training weight classes, the smaller one is 50g, and the bigger one gradually increases. It can be seen that the error gets smaller with a bigger range. (b) Adding more training weight classes decreases the median and mean MAE, suggesting a tradeoff between data collection effort and performance. The improvement from two to three weight classes is more significant than the improvement from three to four, leading us to select three weight classes for evaluation.  (c) The increasing amount of training data per weight class gives comparable performances. Therefore, three samples are selected in our case for better data efficacy.}
    \label{fig:Evaluation2}
\end{figure}

Three evaluations are conducted to validate the improved data efficacy based on the physics-informed model. To evaluate the tradeoff between the performance and the amount of training data, we start from two weight classes (one weight change) and find that when the minimum and maximum weights are included in the training set, the performance reaches the best. The minimum weight 50g is kept in the training set and the bigger weight increases gradually (e.g., 50g and 100g, 50g and 150g,\ldots, 50g and 500g). As weight change gets bigger, the performance gets better. The inclusion of minimum weight and maximum weight reaches the best, as shown in Figure~\ref{fig:Evaluation2} (a). It reflects that when the training data includes the full range of weight information, it can better capture the weight falling within this range, which is very similar to the property of a linear model.

Then, the number of training weight classes increases while keeping the minimum and maximum weight in the training set, evaluating the tradeoff between performance and the number of weight classes. The number of weight classes increases by interpolation: 50g and 500g; 50g, 300g and 500g; 50g, 200g, 350g, and 500g. Figure~\ref{fig:Evaluation2} (b) suggests that the increasing number of training weight classes provides an improvement. However, the improvement is more evident from two to three weight classes. The performance of three weight classes and four weight classes are very close. In the actual application, the number of necessarily collected weight classes significantly depends on the possible slightest weight change between items. In this case, three training weight classes are employed because there is no significant improvement in selecting four training weight classes.

Figure~\ref{fig:Evaluation2} (c) suggests that 10\% training data for each weight class is enough. When keeping three weight classes selected from the previous evaluation, we investigate the performance while increasing the amount of training data per weight class. It turns out that the increasing amount of training data per weight class gives comparable performances. It is probably because most data samples within the same weight class are similar. Referring back to Section~\ref{sec:theory}, the vibration frequency spectrum should be much the same as long as the item weight and item location do not change. Therefore, we select 10\% training data to minimize the data collection effort.

\subsection{Evaluation III: The Amount of Required Sensor}
\label{eval:sensor}
Figure~\ref{fig:Evaluation3} shows the performance comparison with different sensor configurations in the training process, suggesting that the combination of three sensors is comparable to a single sensor with the best performance. Figure~\ref{fig:Experiment Setup} (b) shows the location of three different sensors. Figure~\ref{fig:Evaluation3} shows that the performance of sensor 1 is the best out of three single-sensor usage cases, and it is very close to the best performance: sensor 1\&3. This observation leads us to employ sensor 1 to minimize the sensor usage. On the other hand, it is also observed that the combination of three sensors is also very close to the best performance. It results in a compromise of using all three sensors if we don't know which one can give the best performance, which is the case of our real item-layout evaluation.

\begin{figure}[t]
% \setlength{\abovecaptionskip}{10pt}
    \centering
    \includegraphics[width=0.8\linewidth]{Figure/Evaluation3.jpg}
    \caption{This figure illustrates the tradeoff between weight change estimation and different sensor configurations used during training. Sensor 1\&3 reaches the best performance, but sensor 1 and sensor1\&2\&3 also have a close performance. Given no prior knowledge of which sensor can perform the best, it is reasonable to apply all three sensors. However, if we know the best sensor, it should be used to reduce the sensor cost.}
    \label{fig:Evaluation3}
\end{figure}

\subsection{Evaluation IV: Real Item-Layout Evaluation}

\begin{figure}[tbh]
% \setlength{\abovecaptionskip}{10pt}
    \centering
    \includegraphics[width=\linewidth]{Figure/Evaluation_Real.jpg}
    \caption{WeVibe is evaluated on a real item-layout shelf. We bought a shelf of protein bars and placed them under the WeVibe sensing environment. With a mean absolute error of 41.05g and a standard deviation of 42.2g, it is expected to detect the case that two or more protein bars are taken or put back according to the weight change estimation.}
    \label{fig:Evaluation Real}
\end{figure}

As shown in Figure~\ref{fig:Evaluation Real}, WeVibe is evaluated with a real item-layout. We bought a shelf of protein bars and placed them on our shelf with the same organization in the store. Six weight samples are collected for each box: Full box, one protein bar is taken,\ldots, and five protein bars are taken. Each protein bar ranges from 59g to 64g, so an average weight of 61g is assigned for each protein bar. The label is the weight of the total shelf load, ranging from 2997g to 3302g (the sum of five boxes). The training set comprises 10\% samples of three weight classes (2997g, 3180g, and 3302g). Three sensors are used together because we assume no prior knowledge of the best sensor is provided.

WeVibe achieves a mean absolute error of 41.05g and a standard deviation of 42.2g for all locations. It suggests that WeVibe can successfully identify these cases based on the weight change estimation result when two or more protein bars are taken or returned. For each location, WeVibe shows various performances. Location 1 gives the best result with 23.34g mean absolute error and 28.3g standard deviation, which likely works for one protein bar. Location 5 shows a more considerable standard deviation, which might not be functional for the lightweight items in the actual store. It might be due to the shelf's elastic deformation. The amount of elastic deformation of the shelf surface might bring extra variation on the shelf vibration response, leading to more data collection at some specific locations. At location 5, more weight classes are required to provide better estimation because there might be a small amount of elastic deformation happening at these locations when the item weight changes, leading to more variances in the final weight change estimation.






