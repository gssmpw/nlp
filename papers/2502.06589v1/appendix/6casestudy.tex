\section{Case Studies}
\label{app:case}

\subsection{Code-to-Text Synthesis Example}

We present an example of synthesized API documentation as follows:
\vspace{1ex}
\VerbatimInput[label=\fbox{<Code\_to\_Text> Example}]{case/case-code2text}

\subsection{Retrieved Data Examples}

We present two examples of high-quality retrieved data as follows:

% Example 1 - AgentTraj-L Retrieved Sample:
\vspace{1ex}
\VerbatimInput[label=\fbox{<Retrieval> Example-1}]{case/case-retrieval1}

% Example 2 - Android Retrieved Sample:
\vspace{1ex}
\VerbatimInput[label=\fbox{<Retrieval> Example-2}]{case/case-retrieval2}

\subsection{Data Quality Filtering Failure Cases}

% \noindent\textbf{Failure Case of fastText Filter.}
% However, we observed some limitations in the fastText approach. For instance, consider the following example of an advertisement on a radio channel, which should be categorized as general text:
% [Insert example text here]
% 
We present a failure case of the fastText filter below:
\vspace{1ex}
\VerbatimInput[label=\fbox{<fastText\_Filter> Failure Case}]{case/case-fasttext-fail}
In this case, the fastText model incorrectly categorized the text as agent-relevant data. This misclassification likely occurred because fastText relies on gram frequency analysis, and the presence of multiple high-tech terms (e.g., iOS, App, Google Play) in the paragraph may have misled the model.