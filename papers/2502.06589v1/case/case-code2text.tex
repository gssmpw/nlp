**API: `easyai.model.base_block.utility.utility_block.ConvBNActivationBlock`**

The `ConvBNActivationBlock` is a utility class that combines a convolutional layer, a batch 
normalization layer, and an activation layer into a single block. This block is commonly used as a 
building block in various neural network architectures, including the PNASNet architecture 
demonstrated in the provided code.

**Parameters:**

- `in_channels` (int): The number of input channels for the convolutional layer.
- `out_channels` (int): The number of output channels for the convolutional layer.
- `kernel_size` (int or tuple): The size of the convolutional kernel.
- `stride` (int or tuple, optional): The stride of the convolutional operation. Default is 1.
- `padding` (int or tuple, optional): The padding added to the input tensor before applying 
    the convolution. Default is 0.
- `bias` (bool, optional): Whether to include a bias term in the convolutional layer. Default is 
    `False`.
- `bnName` (str or `easyai.base_name.block_name.NormalizationType`, optional): The normalization 
    type to be used. Default is `NormalizationType.BatchNormalize2d`.
- `activationName` (str or `easyai.base_name.block_name.ActivationType`, optional): The activation 
    function to be used. Default is `ActivationType.ReLU`.

**Attributes:**

- `conv`: The convolutional layer.
- `bn`: The batch normalization layer.
- `activation`: The activation layer.

**Methods:**

- `forward(x)`: Defines the forward pass of the block. It takes an input tensor `x` and applies the 
    convolutional, batch normalization, and activation operations sequentially.
- `get_name()`: Returns the name of the block, which is a combination of the class name and a unique 
    index.

**Example Usage:**

```python
import torch
from easyai.model.base_block.utility.utility_block import ConvBNActivationBlock

# Create a ConvBNActivationBlock with input channels 3, output channels 64, kernel size 3x3, and 
    stride 1
block = ConvBNActivationBlock(in_channels=3, out_channels=64, kernel_size=3, stride=1)

# Generate a random input tensor
x = torch.randn(1, 3, 32, 32)

# Pass the input tensor through the block
output = block(x)
```

In the provided code, the `ConvBNActivationBlock` is used as the first layer of the PNASNet 
architecture, where it takes the input image data and applies a convolutional operation followed by 
batch normalization and activation.