\section{Conclusion}
This paper investigated the issue of narrow creativity in both humans and GenAI through the Circles Exercise. 
We began by categorizing and analyzing human drawings to understand the represenation of human narrow creativity. 
Subsequently, we applied the same analysis to GenAI outputs generated using different prompting strategies.
Our findings indicate that both humans and GenAI exhibit narrow creativity, often exploring constrained subsets of the design space. This highlights inherent limitations in the creative processes of both entities.

Our work identifies key challenges and opportunities for advancing GenAI-powered creativity support tools. While sophisticated prompting strategies can partially alleviate narrow creativity, they are insufficient on their own. 
To foster groundbreaking idea development, we suggest future research focus on designing innovative human-GenAI interaction mechanisms and systems. 
Such innovations are crucial for enhancing the efficiency and effectiveness of design space exploration by humans and GenAI.
