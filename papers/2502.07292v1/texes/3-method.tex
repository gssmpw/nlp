\section{Methodology}

\subsection{Aspects of Human Narrow Creativity}

In this subsection, we aim to explore the aspects of human narrow creativity within the Circles Excercise. 
Specifically, we describe the methodology for analyzing the human outputs of Circles Excercise collected from a classic creativity exercise conducted as a warm up for a graduate-level product design class. 
The task involves asking students to generate as many creative drawings as possible using a provided sheet containing 28 blank circles. By collecting and examining the submissions from the past four semesters, encompassing a total of 3367 drawings made by 224 students, we plan to identify recurring patterns and categorize various aspects of the students’ creative approaches. Below, we outline the specific aspects of creativity we aim to probe and our rationale for investigating each:

% \begin{center}
%     \includegraphics[width=0.5\textwidth]{figures/placeholder/placeholder.png} % Adjust width as needed
%     \captionof{figure}{Place holder for the examples of the manifestation of human narrow creativity}
% \end{center}

\subsubsection{Categories of Drawn Objects}
Understanding the types of objects students tend to draw provides insight into their creative limits and preferences. We classified these objects into categories such as animals, daily object, human figures, nature elements, and vehicles. This classification will help us analyze whether humans gravitate toward familiar, concrete objects or attempt more abstract and imaginative designs. 

\subsubsection{Artistic Expression Techniques}
The way students choose to express their ideas - be they through simple sketches, detailed illustrations, or the use of color - can significantly impact perception of creativity. We will document and analyze these methods to understand the role of artistic style and technical embellishments in creative problem solving.

\subsubsection{Approaches to Material Utilization}
 We plan to categorize these strategies into direct use, personification, abstraction based on circles, and complex compositions, which are derived from our observation and will be explained in the next section. By examining these approaches, we can identify patterns in how students reinterpret the given constraints and adapt the circles to fit their creative visions. 

\subsection{Probing Human Narrow Creativity working with GenAI}

Building on insights into human creative processes, we develop and conduct pilot experiment on GenAI (OpenAI) to further understand how the narrow creativity is present in the GenAI generated results. Specifically, we adopted GenAI to solve the same creativity tasks, the Circles Exercise. 
Recognizing that GenAI’s outputs are heavily influenced by prompting strategies, we initially conducted experiments using naive prompting approaches. Subsequently, we applied the Chain-of-Thought (CoT) prompting technique to further evaluate and compare the performance. The details of the prompting strategies are recorded in the Appendix.

\subsubsection{Naive Prompting}

Naive prompting involves engaging GenAI with minimal input or context, either through zero-shot prompting or few-shot prompting strategies. These approaches allow us to observe the interpretative and generative capabilities under basic interaction scenarios.

In zero-shot prompting, GenAI is provided with a direct instruction or question without any examples or contextual guidance. For the Circles Exercise, this involved asking GenAI to generate solutions or propose design alternatives based solely on the task description. This method reflects how a user might interact with GenAI without prior knowledge of optimal prompting techniques.
In few-shot prompting, the model is given a limited number of examples or context-specific cues before being tasked with generating a response. For this study, we curated a small set of illustrative examples related to the Circles Exercise, aiming to guide GenAI’s response style without providing exhaustive instructions. Few-shot prompting was used to explore how minimal contextualization influences GenAI’s creative output.

\subsubsection{Chain-of-Thought Prompting}

To extend our investigation, we employed Chain-of-Thought (CoT) prompting, a structured approach designed to guide GenAI through step-by-step reasoning. This technique encourages GenAI to articulate intermediate steps and logical processes before arriving at a final solution.

In the context of the Circle Problem, CoT prompting was implemented by crafting prompts that explicitly requested the model to break down tasks into smaller components. These prompts included instructions for exploring alternative ideas, generating intermediate insights, and progressively refining solutions. By structuring the interaction in this manner, CoT prompting aims to simulate a more systematic and reflective creative process, providing a contrast to the unstructured nature of naive prompting.

