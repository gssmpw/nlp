\section{Discussion}


\subsection{Narrow Creativity Represents in Various Aspects}

Human creativity in structured tasks often relies on familiar and concrete constructs, inherently limiting its scope. 
Recognizing these patterns helps pinpoint the aspects of narrow creativity that constrain human ideation. 
To overcome these barriers, individuals often require external triggers to break beyond habitual thought processes. 
Edward de Bono's concept of lateral thinking is designed to break conventional thinking patterns and overcome the limitations of narrow creativity that constrain human ideation \cite{de1970lateral}. 
This type of understanding and reasoning as well as using structured methods such as SCAMPER \cite{eberle1996scamper} are essential for designing creativity support tools that dismantle such constraints \cite{mymap_ai_idea_generator}. 
By incorporating mechanisms that uncover unexplored design space and facilitate iterative refinement, these tools can significantly broaden creative potential. 
Furthermore, it highlights the need to identify and address narrow creativity specific to diverse domains, such as artwork, product design, and creative writing, to tailor support tools effectively.

\subsection{Constraints of Narrow Creativity in GenAI Outputs}

The analysis of GenAI's outputs demonstrated that, while AI can generate a higher volume of ideas, it is similarly constrained by narrow creativity when not provided with appropriate prompts from human. For example, under zero-shot prompting, GenAI produced generic and repetitive outputs, largely adhering to common categories and lacking the novelty seen in more structured prompts. Few-shot prompting slightly expanded this range, but the AI often mimicked the examples provided, limiting its creative diversity. Even with chain-of-thought prompting, which encouraged incremental refinement, the outputs reflected an iterative rather than groundbreaking approach. These findings suggest that GenAI, like humans, operates within the boundaries of familiar and safe design spaces unless guided to explore further.

\subsection{Develop the Understanding of Narrow Creativity on Generate Creative Tasks}
In this work, we examine the concept of narrow creativity in both humans and GenAI, using the circle exercise as a case study. 
To build on this understanding, we aim to explore narrow creativity in a broader range of generative creative tasks, such as visual arts \cite{choi2024creativeconnect}, writing \cite{lee2024design}, and product design \cite{kwon2024designer}. 
This investigation seeks to uncover general principles and representations of narrow creativity, offering a more comprehensive framework for understanding this phenomenon.
By identifying these principles and representations, we can facilitate ground-breaking idea development and enable both humans and GenAI to navigate the design space using advanced strategies. 



\subsection{Advancing Creativity Through Human-GenAI Interaction Mechanisms}

While our study validates the effectiveness of prompting strategies in mitigating GenAI's narrow creativity, it underscores an even greater need for innovative interaction mechanisms between humans and GenAI to foster enhanced creativity.
For instance, incorporating evaluation agents could provide real-time feedback, prompting users to move towards unexplored design space. 
Also, humans could take the lead in identifying and addressing these aspects of narrow creativity, while structured prompts guide GenAI to delve into specific areas. 
This dynamic collaboration allows human users to harness the AI’s extensive generative capacity, directing it toward producing unique and meaningful outcomes. 
