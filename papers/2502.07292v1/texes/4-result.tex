\section{Result}

\subsection{Quantative Metrices}
In this paper, we use the distribution of drawn object categories and approaches to material utilization to illustrate the narrow creativity of human and GenAI. The object categories were determined based on a coding framework developed from an initial analysis of common design features and functional elements observed in the dataset. Each drawing was manually coded into a category according to its primary function and physical characteristics, following a structured coding process. This process involved two coders who categorized designs independently and resolved disagreements through discussion to ensure consistency and reliability in the coding scheme. The material utilization approaches were categorized by identifying and coding how materials were incorporated into the designs (e.g., direct use of materials and complex composition).

According to research on creativity evaluation \cite{tromp2024creativity, knight2015managing,li2008exploration}, creativity on generating new ideas are best understood by evaluating the exploration and exploitation of design space. 
To investigate the narrow creativity of human and GenAI by analyzing their exploration and exploitation of the explored design space, we developed the following quantitative metrics. 

\begin{itemize}
  \item \textbf{Number of used categories (\# of used cat.)} This metric represents the average number of distinct categories used by each participant during the task.
  \item \textbf{Number of frequent categories (\# of the equation cat.)} To identify the frequently used categories, we calculate the average and standard deviation of the number of circles within each category for an individual. A category is classified as 'frequent' if the number of circles in that category exceeds the average of the individual.
  \item \textbf{Number of highly frequent categories (\# of highly freq. cat.)} This metric identifies categories that are "highly frequent." A category is classified as such if the number of circles it contains exceeds the average by more than one standard deviation.

  \item \textbf{Proportations of drawings in the frequent categories (\% of the frequency category)} This metric counts the proportions of drawings made by an individual within their frequent categories.
  \item \textbf{Proportations of drawings in the highly frequent categories (\% of freq cat.)} Similar to the previous metric, this counts the total number of drawings within an individual’s highly frequent categories.
\end{itemize}

Exploration activity can be measured by the number of categories that an individual has used, while exploitation activity can be assessed by the number of frequent and highly frequent categories that an individual has. An individual who possesses a strong exploration mindset will have a relatively high number of categories used. In addition, the gap between the number of categories used and the number of frequent categories reveals the balance between exploration and exploitation.
When individuals explore multiple categories, a small gap indicates a good balance.

In order to further investigate the inclination between exploration and exploitation, we analyze the distribution of the circles among categories, which reveals how concentratedly an individual focuses on the frequent categories when doing the assignment. If a person has a large portion of circles that fall into the frequent or highly frequent categories, we can conclude that they lean toward exploitation and have narrow creativity issues.
Furthermore, conveying newly generated ideas is a part of the creativity process. In this work, we analyze the categories of artistic expression techniques that humans and GenAI frequently adopt. 

\subsection{Insights into Human Narrow Creativity from the Circles Excercise}

Human creative fixation often reflects a tendency to operate within familiar and constrained boundaries when engaging in creative tasks.
This tendency is evident in the results of the circle test, where individuals demonstrated preferences in object categories, artistic expression, and approaches to material utilization.
By clustering their creative output into key thematic aspects, we gain insight into the strategies humans used to interpret the task. Structurally encoding these strategies further allows us to pinpoint how fixation manifests in human creativity.
Furthermore, analysis of human creativity serves as a valuable baseline for interpreting similar patterns in the results generated by GenAI, enabling a deeper understanding of its capability and limitations.

\subsubsection{Distribution of Drawn Object Categories}

Human creativity tends to cluster around familiar categories of objects. 
Through analysis of students' drawings in the 28 Circles test, we observe that \textbf{daily objects} is the category most frequently used, while the \textbf{ mechanical} is the least adopted category. 
% We intentionally created this category because the students are mostly in the Department of Mechanical Engineering. 
The distribution suggests a tendency to draw inspiration from familiar, easily recognizable elements (daily objects), and reveals a clear preference for relatable, concrete objects over abstract forms or highly imaginative constructs. We provide the frequency of the average number of used categories in Figure \ref{humanobject}.
It shows that in the creativity task, humans exhibit a narrow and skewed bandwidth of creativity, with a significant inclination toward certain categories.



\begin{figure*}
    \includegraphics[width=0.9\textwidth]{figures/Object.pdf} % Adjust width as needed
    \captionof{figure}{The frequency distribution of categories of objects drawn by humans and GenAI, with a more even distribution across different categories indicating better performance.}
    \label{humanobject}
\end{figure*}

\begin{table*}
  \caption{Quantative analysis of human and GenAI narrow creativity based on the distribution of drawn object categories. For the number of used categories, frequent categories, and highly frequent categories, a higher value indicates better performance. Conversely, for the percentage of objects within frequent and highly frequent categories, a lower value reflects better performance.}
  \label{expression_table_1}
  \begin{tabular}{ccccccccc}
    \toprule
    Object Categories                    &  \multicolumn{2}{c}{Human}          & \multicolumn{2}{c}{Zero-shot} & \multicolumn{2}{c}{Few-shot} &\multicolumn{2}{c}{CoT}\\
    \midrule
                                               & Mean & Std.          & Mean & Std.     & Mean & Std.  & Mean & Std. \\
    \texttt{\# of used cat.}             & 5.6 & 2.1            & 6.6 & 1.6       & 5.6 & 2.0     & 7.5 & 0.97 \\
    \texttt{\# of freq. cat.}            & 2.5 & 1.2            & 2.5 & 0.66       & 2.8 & 1.1     & 3.2 & 0.78 \\
    \texttt{\# of highly freq. cat.}     & 1.4 & 0.96            & 1.4 & 0.65      & 1.2 & 0.28     & 1.6 & 0.69\\ \hline
    \texttt{\% of freq cat.}             & \multicolumn{2}{c}{70}             & \multicolumn{2}{c}{81}        & \multicolumn{2}{c}{77}      & \multicolumn{2}{c}{70}   \\
    \texttt{\% of highly freq. cat. }    & \multicolumn{2}{c}{47}             & \multicolumn{2}{c}{48}        & \multicolumn{2}{c}{43}      &\multicolumn{2}{c}{45}  \\
    \bottomrule
  \end{tabular}
\end{table*}

In this creativity exercise, we observe that human creativity only explores a narrow range when participating in the circle exercise. 
Specifically, an individual likely produces circles concentrated within only a limited number of categories. 
The average number of categories used, frequent categories, and highly frequent categories in all individuals is reported in Table \ref{expression_table_1}. 
Compared with the total number of object categories (10), these results suggest that humans tend to explore a limited subset of categories and frequently narrow their focus even further. 
Humans not only explore narrow perspectives, but also exploit even narrower ones. 
The percentage of objects belonging to frequent categories reveals that the majority of the drawn objects (70\%) fall into frequent categories. 
Thus, the result indicates that the creativity of individuals is strongly inclined towards a limited set of frequent categories.

\begin{figure*}
    \includegraphics[width=0.9\textwidth]{figures/Drawn_Object_Categories.pdf} % Adjust width as needed
    \captionof{figure}{Example of Drawn Object Categories: A) Human Sketched; B) GenAI-Generated, categorized into: 1) Animals, 2) Sport Equipment, 3) Foods, 4) Icons, 5) Daily Objects, and 6) Natural Elements.}
\end{figure*}

\subsubsection{Approaches to Material Utilization}
Students demonstrate diverse strategies for utilizing the circles for their ideas. These include \textbf{direct use}, where students transform the circles into recognizable objects such as clocks, wheels, or buttons by drawing directly within them; \textbf{personification}, where features of human faces are added to anthropomorphize the circles; \textbf{circle-based abstraction}, where the circles are used as references for similar shapes existing in other objects, such as a tennis racket, lollipop, and gear; \textbf{complex compositions}, where multiple circles were combined to form intricate objects, such as bicycles, ice-cream, and glasses; and \textbf{use as background}, where objects are draw within the circles. The frequency distribution of approaches to material utilization is shown in Figure \ref{humanutilization}. It indicates that humans also present narrow creativity and a skewed preference in terms of material utilization approaches.

\begin{figure*}
    \includegraphics[width=0.9\textwidth]{figures/Material_Utilization.pdf} % Adjust width as needed
    \captionof{figure}{Example of Approaches to Material Utilization: A) Human Sketched; B) GenAI Generated, categorized into: 1) Complex Compositions 2) Circle-based Abstraction 3) Use as Background.}
\end{figure*}


% \begin{center}
%     \includegraphics[width=0.6\textwidth]{figures/Distribution of Drawn Object Categories.png} % Adjust width as needed
%     \captionof{figure}{Place holder for example of Distribution of Drawn Object Categories}
% \end{center}

\begin{figure*}
    \includegraphics[width=0.7\textwidth]{figures/Utilization.pdf} % Adjust width as needed
    \captionof{figure}{The frequency distribution of approaches to material utilization is analyzed, with a more even distribution across different approaches indicating better performance.}
    \label{humanutilization}
\end{figure*}

Similarl to the analysis in the above subsection, Table \ref{expression_table_2} provides the average number of approaches used, frequent approaches, and highly frequent approaches. 
Based on the percentage of the frequent approaches, 80\% of the object are proposed based on one frequent approach of using the circles.

\begin{table*}
  \caption{Quantative analysis of human and GenAI narrow creativity based on material utilization approaches. For the number of used approaches, frequent approaches, and highly frequent approaches, a higher value indicates better performance. Conversely, for the percentage of objects within frequent and highly frequent approaches, a lower value reflects better performance.}
  \label{expression_table_2}
  \begin{tabular}{ccccccccc}
    \toprule
    Utilization Approaches                     &  \multicolumn{2}{c}{Human}          & \multicolumn{2}{c}{Zero-shot} & \multicolumn{2}{c}{Few-shot} &\multicolumn{2}{c}{CoT}\\
    \midrule
                                               & Mean & Std.          & Mean & Std.     & Mean & Std.  & Mean & Std. \\
    \texttt{\# of used apch.}                  & 3.0 & 1.2            & 3.5 & 0.96       & 3.4 & 1.2     & 3.6 & 0.69 \\
    \texttt{\# of freq. apch.}                 & 1.5 & 0.61           & 1.9 & 0.86       & 1.4 & 0.50    & 1.7 & 0.67 \\
    \texttt{\# of highly freq. apch.}          & 1.1 & 0.40           & 1.1 & 0.27       & - & -     & - & - \\ \hline
    \texttt{\% of freq apch.}                  & \multicolumn{2}{c}{80}             & \multicolumn{2}{c}{81}        & \multicolumn{2}{c}{63}      & \multicolumn{2}{c}{68}  \\
    \texttt{\% of highly freq. apch. }         & \multicolumn{2}{c}{68}             & \multicolumn{2}{c}{48}        & \multicolumn{2}{c}{51}      & \multicolumn{2}{c}{45}  \\
    \bottomrule
  \end{tabular}
\end{table*}

\subsubsection{Variation in Artistic Expression}

Although students' object categories and artistic expressions exhibit considerable diversity, the artistic styles and techniques they employ show limited variation, reflecting a monotonic and uniform approach to conveying their ideas. The approaches include \textbf{simple sketches}, where many students opt for minimalistic, black-and-white drawings focusing on the core concept; \textbf{detailed illustrations}, with some students enhancing their designs through intricate details that add depth and character. The portion of the two is presented in Table \ref{expression_table_2}, which suggests that humans almost adopt simple sketches. A possible reason might be that humans emphasize task efficiency and believe simple sketches are effective enough to convey their ideas.

Besides sketching, some students incorporate additional elements to facilitate their expression. The elements are summarized as follows: \textbf{use of color}, where some students incorporate vibrant colors to enrich their visual representations; and \textbf{annotations and labels}, where some drawings include textual annotation to explain or narrate their drawings, adding an interpretive layer to their visual output. 
The portion of drawings that use the additional elements is reported in Table \ref{expression_table_3}. 
The relatively low portion of drawings with additional elements (5\%) suggests that humans have limited capability to convey their ideas with detailed expressions. It might be because adding additional elements is time-consuming and requires extra resources. 
Figure \ref{fig: example of art} shows that while the GenAI outputs demonstrate enhanced refinement, they require explicit instructions to incorporate annotations or labels.


\begin{figure*}
    \includegraphics[width=0.9\textwidth]{figures/Artistic_Expression.pdf} % Adjust width as needed
    \captionof{figure}{Example of Artistic Expression: A) Human Sketched; B) GenAI-Generated, categorized into: 1) Simple Sketches 2) Detailed Illustrations 3) Use of Color 4) Annotations and Labels. }
    \label{fig: example of art}
\end{figure*}

\begin{table*}
  \caption{The portion of the frequent artistic expression of humans and GenAI.}
  \label{expression_table_3}
  \begin{tabular}{ccccc}
    \toprule
    Artistic Expressions                     & Human          & Zero-shot & Few-shot & CoT\\
    \midrule
    \texttt{\% of simple sketches}           & 95             & 2.9        & 10      & 3.2\\
    \texttt{\% of detailed illustrations}    & 5              & 97       & 89      & 97\\ \hline
    \texttt{\% of use of colors}             & 16             & 52        & 20      & 70\\
    \texttt{\% of use of annotations}        & 13             & 0.0        & 0.0     & 0.0\\
    \bottomrule
  \end{tabular}
\end{table*}

\subsection{Understanding GenAI Narrow Creativity with Different Prompting Strategies}

We present the results of different prompting strategies that are frequently adopted in GenAI-augmented creativity support tools.
We clustered the GenAI result based on the same aspects of narrow creativity as we used for human result.
This analysis offers insights into how AI-augmented creativity can complement or challenge human tendencies, revealing both the limitations and opportunities of prompting strategies in addressing narrow creative issues.

\subsubsection{Naive Prompting}

We conducted a pilot study to evaluate the results of GenAI under the condition of naive prompting.
We adopted the same quantitative metrics that were used to evaluate human creativity.
The zero-shot prompts provided to GenAI were based on a pre-articulated template, as described in the appendix.

For few-shot prompting, we provide each prompt with an examples from the results.

The statistics for zero-shot and few-shot prompting align closely with observations from human data (object categories: human=5.6, zero-shot=6.6, few-shot=5.6; utilization approaches: human=3.0, zero-shot=3.5, few-shot=3.4). 
The distribution of object categories suggests that GenAI, under naive prompting strategies, exhibits a similar pattern of narrow creativity as humans in this task. Interestingly, compared to zero-shot prompting, few-shot prompting produces GenAI results that are more quantitatively aligned with human performance. This observation implies that providing human examples may lead to a similar representation of narrow creativity in GenAI, particularly if the prompts are not further refined or articulated to encourage more diverse outputs.


In terms of artistic expression, humans and GenAI demonstrate differing preferences. Most humans (95\%) prefer to express their ideas through simple sketches. By contrast, GenAI models tend to favor detailed illustrations (zero-shot=97\%, few-shot=89\%), likely due to their stronger image-generation capabilities.
A similar pattern is observed in the use of color. Few-shot prompted GenAI exhibits preferences more aligned with human behavior (human=16\%, few-shot=20\%) due to exposure to human examples during training. This suggests that the inclusion of human examples in few-shot prompting can guide GenAI to mimic certain human tendencies, albeit within the constraints of its learned patterns.


\subsubsection{Chain-of-Thought Prompting}

To better understand the capability of GenAI, we adopted the Chain-of-Thought (CoT) prompting strategy to perform a circle test with GenAI. 
While CoT is considered an advanced technique to enhance GenAI's reasoning capabilities, the experiment results reveal GenAI under CoT prompting demonstrate similiar pattern of narrow creativity as human does.

The result demonstrate that a significant proportion of the objects generated by GenAI under CoT prompting belong to frequently used categories (70\%). This mirrors the behavior seen in other strategies (e.g., zero-shot=81\% and few-shot=70\%), showing that CoT does not substantially expand the variety of generated object categories.
Moreover, 45\% of the objects belong to highly frequent categories, reinforcing the observation that GenAI under CoT also tend to explore on narrowed regions of design space, rather than exploring more diverse ideas.
In terms of material utilization approaches, CoT-generated ideas also exhibit constrained diversity. Approximately 68\% of the approaches employed by GenAI under CoT prompting narrow to the most frequently used methods.
This suggests that CoT does not effectively overcome the bias toward relying on dominant patterns of material utilization.

These results highlight a persistent reliance on frequent object categories and approaches. While CoT improves reasoning capabilities, it does not necessarily enhance the creative breadth of GenAI, as it struggles to generate ideas that break away from the narrow design space.
