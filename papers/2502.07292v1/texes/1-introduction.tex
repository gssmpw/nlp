\section{Introduction}

Creativity is the production of ideas or outcomes that are both novel and appropriate to the task or problem \cite{cropley2023intersection}. The creative process involves divergent thinking which requires exploiting existing ideas and exploring unrelated concepts \cite{jirasek2020big, tromp2024creativity}.
However, humans are limited by factors such as intellectual skills, knowledge gaps, and thinking styles \cite{sternberg2006nature}. 

The emergence of Generative AI (GenAI) presents transformative opportunities to explore more creative spaces than humans may without the use of AI, by exposing individuals to vast repositories of knowledge, experiences, and examples \cite{noy2023experimental}. 
Unlike traditional computational methods that rely on predefined datasets \cite{klemmer2002web,chang2012webcrystal,lee2010designing}, GenAI can generate different ideas on demand as references, facilitating the exploration of unconventional and expansive design spaces. 
This unique capability for the potential creative inspirations that GenAI may provide is leading to the use of GenAI in domains such as art \cite{choi2024creativeconnect}, design \cite{liu20233dall}, and writing \cite{chakrabarty2024art,chakrabarty2024creativity}.
In these applications, GenAI acts as a knowledgeable assistant, providing external ideas to enhance human creativity.
However, it remains unexplored to determine the manner in which the human and GenAI perform creative tasks.
Understanding such behaviors is crucial for designing systems that effectively combine human intuition with AI-driven generation to unlock greater potential for human and GenAI to complement each other.
In this paper, our aim is to understand the phenomenon of narrow creativity, defined as the tendency of humans and GenAI to explore only a subset of the available design space. To investigate this phenomenon, we examine: (1) how human narrow creativity is represented in design space exploration, (2) how GenAI exhibits narrow creativity when performing the same tasks, and (3) how advanced prompting methods can be leveraged to enhance GenAI's performance and broaden its creative scope.

We first conduct and quantitatively analyze the human results of a widely used creativity exercise, known as the Circles Exercises \cite{torrance1966torrance,whalley2020paperclips,xiong2024serious}. 
Our analysis provides insights into common aspects of human narrow creativity, including - categories of idea generation, nature of sketching methods, and approaches to material utilization. 
We then conduct a pilot experiment to investigate how GenAI exhibits narrow creativity in the same circle exercise.
We investigated several prompting strategies to uncover the idea generation capabilities and behavior of GenAI. 
The result reveals that GenAI shows similar patterns of narrow creativity as humans do.
Additionally, we observe that GenAI currently excels at reasoning and idea generation, as evidenced in its textual responses, but falls short in generating visual representations.

Our preliminary observations reveal that GenAI, in its current form, has ingrained constraints that lead to narrow creativity in the Circles exercise. 
Additionally, our work inspires further in-depth research to understand narrow creativity in other domains and modalities such as writing, product design, music composition to advance GenAI-powered creativity support tools.
