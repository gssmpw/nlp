\section{Appendix}

\subsection{Stardard Circle Exercise Image}
\begin{center}
    \includegraphics[width=0.5\textwidth]{figures/circle.png} % Adjust width as needed
    \captionof{figure}{Standard Circle Exercise Image}
\end{center}

\subsection{Circle Exercise Instruction}

% We invited a large number of students to participate in our test. Each student received a standard circle-test image (see~\autoref{}) and was asked to complete the task with the following requirements.

"Let's get your creative flow going...Draw as many things as you can - Use each circle in the file attached as the starting point of your creations. Note: The circle should be a part of your creation and 'not just containing the individual sketches inside."


\subsection{Prompt to GenAI}

% As described in the main paper, we evaluated the creative outputs produced by ChatGPT\cite{chatgpt} under different prompt strategies.

% \subsubsection{Zero-Shot Prompt} In this setting, we provided ChatGPT 4.0 with the following prompt alongside a standard circle-test image(see~\autoref{}). 

"Help me complete the creativity challenge, we expect you to generate images of what you have draw. Let's get your creative flow going...Draw as many things as you can in 3 minutes. - Use each circle in the file attached as the starting point of your creations. Note: The circle should be a part of your creation and not just containing the individual sketches inside."

\subsection{Chain of Though Prompt}
%\begin{enumerate}

\begin{itemize}
    \item Introduce the Challenge Context

“We have a sheet with 30 blank circles. Your task is to create as many different sketches as you can in a short time—ideally 3 minutes or so. Each circle must be used in a way that makes it part of the drawing.”

\item Visualize or Sketch Ideas Quickly

“Quickly brainstorm a wide variety of possible subjects—animals, objects, faces, symbols, doodles—any quick concept you can think of. Remember that each circle should be utilized as an essential element of the final sketch (for example, it might become the center of a flower, the face of a character, the wheel on a vehicle, etc.).”

\item Generate Individual Concepts per Circle

“For each of the 30 circles, propose a short concept description before actually creating the sketch. For example:
Circle 1 → Cartoon bird face
Circle 2 → Bicycle wheel
Circle 3 → Abstract spiral design
… and so on
Brainstorm any simple design that naturally incorporates the circle.”

\item Draw (or Render) Each Idea

“Now, for each circle, create a quick doodle or image that integrates the circle. Make sure the circle remains visible as a key part of the drawing, rather than a container or background only.”

\item Keep It Fast and Fun

“The goal is creativity, so don’t focus too heavily on polished detail. The time constraint is part of the challenge—concentrate on variety and quick execution.”


\item Present the Full Set of Sketches

“Arrange your 30 quick sketches in a grid or sequence that mirrors the original circle layout (5 columns × 6 rows, for instance). Show each circle-based doodle in its own space, with minimal text or labels—let the visuals speak for themselves.”

\item Optional: Reflection/Explanation

“After you’ve completed the sketches, you may add brief captions explaining each idea or the thought process behind it (e.g., ‘Circle turned into a flower with petals around it’). This can help viewers see your creative reasoning.”

\item Encourage Variation and Surprise

“Feel free to let your imagination run wild. Anything from mundane objects (clocks, balloons, and buttons) to fantasy creatures (alien eyes, dragon eggs, planet sketches) is fair game. The more unexpected, the better!”

\item Emphasize the ‘Circle as Part of the Drawing’ Rule

“Throughout every design, ensure the circle remains visibly integrated. Do not hide it entirely behind new elements or treat it merely as a boundary. The circle is part of the shape or composition.”

\end{itemize}

