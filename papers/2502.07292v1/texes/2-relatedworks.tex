\section{Related Work}

\subsection{Understanding Human Creativity with Design Spaces}

Human cognitive processes and outputs in creative tasks are often structured \cite{ward1994structured}. Specifically, human creativity is shaped by the properties of instructions, provided materials, or existing exemplars. When tasked with a creative problem, humans tend to operate within a design space inherent in the task. Moreover, the design spaces vary depending on the nature of the task. Many previous works have distilled design spaces to better understand human creativity in various domains, such as writing \cite{lee2024design}, storytelling \cite{fan2024storyprompt}, creating virtual reality scenes \cite{neuhaus2024virtual}, and product design \cite{lupetti2024making}. For instance, in a recent study on creative sketching \cite{williford2023exploring}, which is similar to the 28 Circles Test in this work, the authors identified a design space and categorized creations within it.

On the other hand, as highlighted in \cite{finke1996creative,jansson1991design}, humans often exhibit cognitive fixations during creative tasks, generating new ideas by exploiting specific aspects of their conceptual structure. For example, once someone envisions a soccer ball, they are more likely to think of other sports balls rather than animals. As a result, human creativity can become constrained, limiting exploration to a subset of the design space. To address this issue, prior research proposed tools like Mixplorer \cite{kim2022mixplorer}, which represent multiple people's designs with a design space to inform designers about opportunities to explore the design space more broadly and incorporate elements from others' work. 

\subsection{Augmenting Human Creativity with GenAI}

Recently, Generative AI (GenAI) has been increasingly used to facilitate the creative process in the hope that it can enhance human creativity and overcome human limitations by offering fresh perspectives and breaking cognitive fixations in various disciplines, such as design, art and business . In product design, these tools help generate product concepts or prototypes \cite{lu2024generative,duan2024conceptvis,zhang2024protodreamer}, facilitating rapid exploration \cite{kwon2024designer}. In art, GenAI fosters novel creation for visual arts, concept art, music, and literature, as well as video and animation \cite{epstein2023art,liu2024dreamscaping,shi2023understanding}. In business, GenAI aids brainstorming, strategy development, and decision-making through data-driven insights \cite{nguyen2023generative}.

In HCI, many previous works have leveraged GenAI to develop creativity support tools aimed at increasing human creativity \cite{choi2024creativeconnect, wang2024roomdreaming}. However, the creative capacity of GenAI is often overlooked and underexplored. Our work reveals one of its key limitations, narrow creativity. 
Although GenAI is capable of performing incremental exploration on creative tasks within predefined human-provided constraints, it struggles to autonomously generate beyond ordinary or greater radical creativity \cite{cropley2023intersection}. This limitation restricts creative exploration, but simultaneously underscores GenAI's potential to assist human designers in navigating existing design spaces more thoughtfully.

In this paper, we investigate the narrow creativity issues in GenAI by comparing the outputs of humans and GenAI on the Circles Exercise. Through this comparison, our aim is to highlight the limitation of GenAI in creative tasks and stimulate discussion around it. We hope that this work encourages future researchers to develop strategies to enable more effective and efficient human-GenAI collaboration on creativity tasks.
