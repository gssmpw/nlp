\section{Conclusion}
\label{sec:conclusion}

In this work, we presented \vuteco, an automated approach to collect vulnerability-witnessing tests in \Java projects.
%We first experimented with several configurations of \vuteco's two main building blocks, i.e., the \finder and \linker models, to find the optimal setting for designing the \globalModel, which carries out the main downstream task.
%After observing convincing performance, especially in terms of precision, we used \globalModel's optimal configuration to run \vuteco on \invivoProjects open-source \Java projects affected by \invivoVulns vulnerabilities, collecting \resultsWitTests witnessing tests linked to \resultsVulns vulnerabilities among \invivoTests test cases.
The in-vitro analyses showed the \vuteco approach is feasible if properly configured; the in-vivo analyses revealed that it works well for the \finding task but not for the \matching task.
%The task itself is ambitious, as lots of factors might induce \vuteco, or other baseline approaches, into errors.
%
\vuteco is the first solution explicitly targeting this problem, laying the foundation for future research on the matter.
After the analyses, we envisioned several ways to tackle this problem more efficiently.
For instance, the input given to \vuteco can be enriched with \textit{additional contextual information}, like the production code (e.g., the vulnerable components code) or a generated natural language summary of the test cases.
\vuteco's design could also include those in \textsc{FlakyCat}~\cite{akli:ast2023:flakycat}, i.e., exploiting a \textit{similarity-based classification mechanism} that might perform well when having limited examples, as well as converting the integration of \finder and \linker models into a \textit{Mixture of Expert}.
Lastly, \vuteco could include an \textit{automated dynamic assessment} of the found tests to ensure they witness the vulnerability as expected.
