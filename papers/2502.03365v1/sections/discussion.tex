\section{Discussion}
\label{sec:discussion}

\textbf{The Collection of Witnessing Tests.}
The in-vivo analysis allowed us to observe how collecting vulnerability-witnessing tests is a \textit{challenging task}.
\vuteco achieved good results in the \finding task, though failing on the \matching task.
We further reflected on the erroneous matches returned to find possible room for improvements.
We believe the errors were due to the model \textit{understanding the given vulnerability partially}, likely due to the limited context given by the summary description from CVE.
Besides, the inspected test cases might contain \textit{noisy elements} that are typical of the ``security vocabulary,'' despite them not being concerned about any security aspect (as seen in Section~\ref{subsec:fnd-results}).
A more curated pre-training and fine-tuning setting could be the initial action to mitigate such limitations.
%
At the same time, we also reflected on the positive collateral behavior seen in the \matching task, where \vuteco successfully managed to exclude test cases not related to security.
We hypothesize this could be an indirect effect of the longer training stage required to prepare it for the \matching task.
%
%, filtering out many likely-irrelevant cases.
Despite the in-vivo analysis being unable to indicate the true or false negative classifications, the extent of the results \vuteco returned---i.e., a few hundred test cases out of over 800,000---permit a confirmatory manual inspection to be carried out in a reasonable time.
Thus, \vuteco can be a helpful support to reduce the search space greatly.

% \textbf{The Collection of Witnessing Tests.}
% We observed how collecting vulnerability-witnessing tests is a \textit{challenging task}.
% %
% Recognizing the value of the findings observed, we argue that there is still a large room for improvement of \vuteco. 
% For instance, \vuteco might benefit from the joint use of varied approaches.
% As a first step toward this point, we conducted an agreement analysis between the \fixCommits baseline approach and \vuteco on the same test set $I_{TE}$ (used in \rqOne).
% We observed that they agreed on classifying six instances in the test set as \globalPosClass and were all correct (i.e., really part of the \globalPosClass class).
% %Namely, both models agree on classifying \globalPosClass instances correctly.
% %and agreed negatively (i.e., they both said \globalNegClass) on 12,793 (almost all negative instances in $I_{TE}$).
% %In particular, we observed that the instances flagged as \globalPosClass were actually part of the \globalPosClass class.
% At the same time, there were 13 more cases in which \fixCommits classified as \globalPosClass, but it was only correct in four cases.
% The overall agreement between the two models was \textit{moderate} with $0.48$ Cohen's Kappa score~\cite{cohen:1960:kappa,McHugh:2012:cohen}.
% %This shows how the conservative nature of the \globalModel model provided benefit.
% %Several mistakes were made by \fixCommits because of disagreements with \vuteco, though it found some instances that the latter missed.
% Based on these results, combining these two approaches via intersection (AND logical operator) would not add anything new to what \vuteco already found;
% similarly, combining via union (OR logical operator) would increase the amount of noise as \vuteco already found most of the correct cases the \fixCommits found.
% %In \rqOne, we experimented with several configurations and baseline approaches.
% %In this work, we examined the performance of individual models to establish a baseline for future improvements.
% %Multiple actions could be explored, but we believe that \vuteco will benefit from combinations of different models and approaches.
% %ranging from straightforward methods, like intersecting or merging the output sets returned by individual models, to more sophisticated techniques, like a Mixture of Experts.
% For this, we envision more advanced combinations that integrate models specialized in diverse sources of information, like in the Mixture of Experts approach.
% For example, some models could have access to the vulnerable code, while others could examine the commits that patched the vulnerability.
% \vuteco follows this path by leveraging the \finder and \linker sub-models to divide the downstream task into two supposedly easier sub-tasks.
% %
% %In this work, we made \vuteco use only the essential information needed for matching tests with the vulnerabilities, i.e., the summary description in CVE records that any known vulnerabilities indeed have.
% %However, in the future, other sources of information could be exploited (if available) so that \vuteco could include specialized models.

\textbf{The Usefulness of Witnessing Tests.}
The release of the dataset \VulforJ unlocked the research of numerous software security tasks.
The ability of \vuteco to find security-related tests can help expand the known body of vulnerability-witnessing tests, enabling activities like the automated generation of security unit tests~\cite{kang:issta2022:transfer,zhang:2023:llm:sectests}
%or show exemplar tests developers can reuse for testing similar vulnerabilities.
Nevertheless, the potential applications of vulnerability-witnessing tests extend far beyond this~\cite{pan:icse24:patch:presence}.
For example, the witnessing tests can act as \textit{proofs-of-concept} supporting the automatic generation of realistic exploits, building on the advancements of security test generation models~\cite{kang:issta2022:transfer,zhang:2023:llm:sectests}.
Besides, they can support the automated vulnerability repair (AVR) process by localizing the vulnerable statements or assessing the plausibility of a generated patch~\cite{Mohammadi:issrew2019:unit:tests:repair,Sagodi:ease24:avr:gpt4,Zhou:icse24:vulmaster,Bui:emse2024:apr4vul}.
We also envision the use of witnessing tests to support the retrieval of vulnerability-contributing commits~\cite{bao2022v}, as they can be run to triangulate better the moment in which a vulnerability was introduced in a project; to the best of our knowledge, this task has not been investigated yet.
%\EI{Another application of \vuteco: the tests can help the patch localization}
%An automated approach like \vuteco can easily expand the current knowledge we have on witnessing tests at a much faster rate---\TEMP{e.g., finding XXX times more tests with almost no manual effort}---than manual process like in \VulforJ~\cite{bui:msr2022:vul4j}
Furthermore, software engineers can benefit from having many instances of witnessing tests and a tool like \vuteco supporting their retrieval.
For instance, they can reuse known tests of past vulnerabilities to address similar issues affecting their projects.
%Besides, a good \matching mechanism can help them reconstruct the tracking between existing test cases in their projects and all the vulnerabilities addressed in the past.
This scenario further motivates the need for \vuteco to perform better on this task.
%This efficiency can drive further interest in exploiting witnessing tests for a variety of security tasks.

\textbf{The Anatomy of Witnessing Tests.}
The retrieval of witnessing tests is challenging mainly due to the lack of empirical knowledge on the matter.
To date, no study outlined a clear profile of tests witnessing vulnerabilities or, more broadly, unit tests targeting security aspects.
%
The \textit{absence of characterization} of vulnerability tests entails a significant knowledge gap, especially when compared with traditional functional tests.
For instance, we are unaware of what setup is needed before calling the vulnerable component, what the assertions should look at, or the number of tests required to ``cover'' all the relevant scenarios concerning a vulnerability type.
%
The only known common point between the two types of tests is that they both aim to find undesired behaviors in the code that violate some requirements or properties.
We believe this lack of knowledge might be attributable to the difficulty in formulating security requirements at the unit/component level (i.e., methods or classes) since they are often considered at the system level~\cite{Mai:issre18:sec:requirements,felderer:2016:survey:sectests}.
%Things become more complex when we consider the accepted distinction between ``positive'' or ``negative'' security requirements~\cite{felderer:2016:survey:sectests}.
%
Unfortunately, shedding light on these aspects still demands many examples of witnessing tests.
In fact, this study was created to address this shortage, providing an approach to expanding the knowledge base of witnessing tests and drawing more attention to this topic.
%
Once a line between vulnerability-witnessing tests and traditional tests is drawn, innovative solutions can be designed to support developers in writing more security tests.