%ACM
%\documentclass[sigconf,review,anonymous]{acmart}
%\acmConference[MSR 2025]{22st International Conference on Mining Software Repositories}{April 2025}{Ottawa, Canada}
%\setcopyright{acmlicensed}
%\copyrightyear{2024}
%\acmYear{2024}
%\acmDOI{XXXXXXX.XXXXXXX}
%\acmISBN{978-1-4503-XXXX-X/24/06}

%IEEE
\documentclass[10pt,conference]{IEEEtran}
\IEEEoverridecommandlockouts

\usepackage{fontawesome5}
\usepackage{xspace}
\usepackage[shortcuts]{extdash}
\usepackage[frozencache,cachedir=.]{minted}
\usepackage{ifthen}
\usepackage{enumitem}
\usepackage{tabularx}
\usepackage{colortbl}
\usepackage{multirow}
\usepackage{hhline}
\usepackage{soul}
\usepackage{xcolor}
\usepackage{url}
\usepackage{graphicx}
\usepackage{textcomp}
\usepackage{cite}
\usepackage{amsmath,amssymb}
\usepackage{hyperref}
%\usepackage{todonotes}
\usepackage{makecell}
%\hyphenation{vul-ne-ra-bi-li-ty-wit-nes-sing}

\begin{document}
%\title{Automated Collection of Vulnerability\-/witnessing Tests for \Java: The \vuteco Approach}
\title{A Match Made in Heaven? Matching Test Cases and Vulnerabilities With the VUTECO Approach}

%IEEE
\author{
\ifthenelse{\boolean{peerreview}}
{\IEEEauthorblockN{Anonymous Author(s)\\Anonymous Affiliation(s)\\Anonymous Email(s)}}
{\IEEEauthorblockN{Emanuele Iannone}
\IEEEauthorblockA{\textit{Institute of Software Security} \\
\textit{Hamburg University of Technology}\\
Hamburg, Germany \\
emanuele.iannone@tuhh.de}
\and
\IEEEauthorblockN{Quang-Cuong Bui}
\IEEEauthorblockA{\textit{Institute of Software Security} \\
\textit{Hamburg University of Technology}\\
Hamburg, Germany \\
cuong.bui@tuhh.de
}
\and
\IEEEauthorblockN{Riccardo Scandariato}
\IEEEauthorblockA{\textit{Institute of Software Security} \\
\textit{Hamburg University of Technology}\\
Hamburg, Germany \\
riccardo.scandariato@tuhh.de}
}
}

%ACM
% \author{Emanuele Iannone}
% \orcid{0000-0001-7489-9969}
% \affiliation{%
%   \institution{Hamburg University of Technology}
%   \city{Hamburg}
%   \country{Germany}
% }
% \email{emanuele.iannone@tuhh.de}

% \author{Quang-Cuong Bui}
% \orcid{0000-0001-6072-9213}
% \affiliation{%
%   \institution{Hamburg University of Technology}
%   \city{Hamburg}
%   \country{Germany}
% }
% \email{cuong.bui@tuhh.de}

% \author{Riccardo Scandariato}
% \orcid{0000-0003-3591-7671}
% \affiliation{%
%   \institution{Hamburg University of Technology}
%   \city{Hamburg}
%   \country{Germany}
% }
% \email{riccardo.scandariato@tuhh.de}

% *** CITATION PACKAGES ***
%
% \ifCLASSOPTIONcompsoc
%   % IEEE Computer Society needs nocompress option
%   % requires cite.sty v4.0 or later (November 2003)
%   \usepackage[nocompress]{cite}
% \else
%   % normal IEEE
%   \usepackage{cite}
% \fi

\usepackage{booktabs}



% *** GRAPHICS RELATED PACKAGES ***
%
% \ifCLASSINFOpdf
%   % \usepackage[pdftex]{graphicx}
%   % declare the path(s) where your graphic files are
%   % \graphicspath{{../pdf/}{../jpeg/}}
%   % and their extensions so you won't have to specify these with
%   % every instance of \includegraphics
%   % \DeclareGraphicsExtensions{.pdf,.jpeg,.png}
% \else
%   % or other class option (dvipsone, dvipdf, if not using dvips). graphicx
%   % will default to the driver specified in the system graphics.cfg if no
%   % driver is specified.
%   % \usepackage[dvips]{graphicx}
%   % declare the path(s) where your graphic files are
%   % \graphicspath{{../eps/}}
%   % and their extensions so you won't have to specify these with
%   % every instance of \includegraphics
%   % \DeclareGraphicsExtensions{.eps}
% \fi
% graphicx was written by David Carlisle and Sebastian Rahtz. It is
% required if you want graphics, photos, etc. graphicx.sty is already
% installed on most LaTeX systems. The latest version and documentation
% can be obtained at: 
% http://www.ctan.org/pkg/graphicx
% Another good source of documentation is "Using Imported Graphics in
% LaTeX2e" by Keith Reckdahl which can be found at:
% http://www.ctan.org/pkg/epslatex
%
% latex, and pdflatex in dvi mode, support graphics in encapsulated
% postscript (.eps) format. pdflatex in pdf mode supports graphics
% in .pdf, .jpeg, .png and .mps (metapost) formats. Users should ensure
% that all non-photo figures use a vector format (.eps, .pdf, .mps) and
% not a bitmapped formats (.jpeg, .png). The IEEE frowns on bitmapped formats
% which can result in "jaggedy"/blurry rendering of lines and letters as
% well as large increases in file sizes.
%
% You can find documentation about the pdfTeX application at:
% http://www.tug.org/applications/pdftex





% *** MATH PACKAGES ***
%
\usepackage{amsmath}
% A popular package from the American Mathematical Society that provides
% many useful and powerful commands for dealing with mathematics.
%
% Note that the amsmath package sets \interdisplaylinepenalty to 10000
% thus preventing page breaks from occurring within multiline equations. Use:
%\interdisplaylinepenalty=2500
% after loading amsmath to restore such page breaks as IEEEtran.cls normally
% does. amsmath.sty is already installed on most LaTeX systems. The latest
% version and documentation can be obtained at:
% http://www.ctan.org/pkg/amsmath





% *** SPECIALIZED LIST PACKAGES ***
%
%\usepackage{algorithmic}
% algorithmic.sty was written by Peter Williams and Rogerio Brito.
% This package provides an algorithmic environment fo describing algorithms.
% You can use the algorithmic environment in-text or within a figure
% environment to provide for a floating algorithm. Do NOT use the algorithm
% floating environment provided by algorithm.sty (by the same authors) or
% algorithm2e.sty (by Christophe Fiorio) as the IEEE does not use dedicated
% algorithm float types and packages that provide these will not provide
% correct IEEE style captions. The latest version and documentation of
% algorithmic.sty can be obtained at:
% http://www.ctan.org/pkg/algorithms
% Also of interest may be the (relatively newer and more customizable)
% algorithmicx.sty package by Szasz Janos:
% http://www.ctan.org/pkg/algorithmicx




% *** ALIGNMENT PACKAGES ***
%
%\usepackage{array}
% Frank Mittelbach's and David Carlisle's array.sty patches and improves
% the standard LaTeX2e array and tabular environments to provide better
% appearance and additional user controls. As the default LaTeX2e table
% generation code is lacking to the point of almost being broken with
% respect to the quality of the end results, all users are strongly
% advised to use an enhanced (at the very least that provided by array.sty)
% set of table tools. array.sty is already installed on most systems. The
% latest version and documentation can be obtained at:
% http://www.ctan.org/pkg/array


% IEEEtran contains the IEEEeqnarray family of commands that can be used to
% generate multiline equations as well as matrices, tables, etc., of high
% quality.




% *** SUBFIGURE PACKAGES ***
%\ifCLASSOPTIONcompsoc
%  \usepackage[caption=false,font=footnotesize,labelfont=sf,textfont=sf]{subfig}
%\else
%  \usepackage[caption=false,font=footnotesize]{subfig}
%\fi
% subfig.sty, written by Steven Douglas Cochran, is the modern replacement
% for subfigure.sty, the latter of which is no longer maintained and is
% incompatible with some LaTeX packages including fixltx2e. However,
% subfig.sty requires and automatically loads Axel Sommerfeldt's caption.sty
% which will override IEEEtran.cls' handling of captions and this will result
% in non-IEEE style figure/table captions. To prevent this problem, be sure
% and invoke subfig.sty's "caption=false" package option (available since
% subfig.sty version 1.3, 2005/06/28) as this is will preserve IEEEtran.cls
% handling of captions.
% Note that the Computer Society format requires a sans serif font rather
% than the serif font used in traditional IEEE formatting and thus the need
% to invoke different subfig.sty package options depending on whether
% compsoc mode has been enabled.
%
% The latest version and documentation of subfig.sty can be obtained at:
% http://www.ctan.org/pkg/subfig





%\usepackage{stfloats}
% stfloats.sty was written by Sigitas Tolusis. This package gives LaTeX2e
% the ability to do double column floats at the bottom of the page as well
% as the top. (e.g., "\begin{figure*}[!b]" is not normally possible in
% LaTeX2e). It also provides a command:
%\fnbelowfloat
% to enable the placement of footnotes below bottom floats (the standard
% LaTeX2e kernel puts them above bottom floats). This is an invasive package
% which rewrites many portions of the LaTeX2e float routines. It may not work
% with other packages that modify the LaTeX2e float routines. The latest
% version and documentation can be obtained at:
% http://www.ctan.org/pkg/stfloats
% Do not use the stfloats baselinefloat ability as the IEEE does not allow
% \baselineskip to stretch. Authors submitting work to the IEEE should note
% that the IEEE rarely uses double column equations and that authors should try
% to avoid such use. Do not be tempted to use the cuted.sty or midfloat.sty
% packages (also by Sigitas Tolusis) as the IEEE does not format its papers in
% such ways.
% Do not attempt to use stfloats with fixltx2e as they are incompatible.
% Instead, use Morten Hogholm'a dblfloatfix which combines the features
% of both fixltx2e and stfloats:
%
% \usepackage{dblfloatfix}
% The latest version can be found at:
% http://www.ctan.org/pkg/dblfloatfix




% *** PDF, URL AND HYPERLINK PACKAGES ***
%
\usepackage{url}
% url.sty was written by Donald Arseneau. It provides better support for
% handling and breaking URLs. url.sty is already installed on most LaTeX
% systems. The latest version and documentation can be obtained at:
% http://www.ctan.org/pkg/url
% Basically, \url{my_url_here}.

\usepackage{xcolor}
\usepackage{cleveref}
\usepackage{graphicx}
\usepackage{subcaption}

%comments
\newcommand{\zijian}[1]{\textcolor{blue}{ZH: #1}}
\newcommand{\jiasi}[1]{\textcolor{red}{JC: #1}}
\newcommand{\yicheng}[1]{\textcolor{green}{YZ: #1}}
\newcommand{\sophie}[1]{\textcolor{purple}{SC: #1}}
% \newcommand{\eg}{\textit{e}.\textit{g}.,~}
%Latin shortcuts
\newcommand{\eg}{\emph{e.g.,} }
\newcommand{\ie}{\emph{i.e.,} }
\newcommand{\etal}{\emph{et al.} }

% correct bad hyphenation here
\hyphenation{op-tical net-works semi-conduc-tor}

%IEEE
\maketitle

%ACM
%\renewcommand{\shortauthors}{Iannone et al.}

\begin{abstract}
Software vulnerabilities are commonly detected via static analysis, penetration testing, and fuzzing.
They can also be found by running unit tests---so-called \textit{vulnerability-witnessing tests}---that stimulate the security-sensitive behavior with crafted inputs.
Developing such tests is difficult and time-consuming; thus, automated data-driven approaches could help developers intercept vulnerabilities earlier.
However, training and validating such approaches require a lot of data, which is currently scarce.

This paper introduces \vuteco, a deep learning-based approach for collecting instances of vulnerability-witnessing tests from \Java repositories.
\vuteco carries out two tasks: (1) the ``\finding'' task to determine whether a test case is security-related, and (2) the ``\matching'' task to relate a test case to the exact vulnerability it is witnessing.
%Using the test case source code and the textual description of vulnerabilities, \vuteco employs a deep learning model based on CodeBERT (1) determine if the test case is security-related and (2) match it to the right vulnerability.
\vuteco successfully addresses the \finding task, achieving perfect precision and 0.83 F0.5 score on validated test cases in \VulforJ and returning 102 out of 145 (70\%) correct security-related test cases from \invivoProjects open-source \Java projects.
Despite showing sufficiently good performance for the \matching task---i.e., 0.86 precision and 0.68 F0.5 score---\vuteco failed to retrieve any valid match in the wild.
Nevertheless, we observed that in almost all of the matches, the test case was still security-related despite being matched to the wrong vulnerability.
%\vuteco successfully addresses the two tasks when evaluated on the tests and vulnerabilities in \VulforJ, achieving \EI{TODO} in the best configurations.
%
%Afterward, \vuteco was employed on \invivoProjects open-source \Java projects and \invivoVulns vulnerabilities taken from \projectKB, matching \TEMP\resultsWitTests witnessing tests to \resultsVulns vulnerabilities.
%A manual inspection of the top suspects confirmed that \vuteco matched the right test for \inspectedVulnsWithOneValidTest vulnerabilities, four times more than a heuristic baseline approach could.
In the end, \vuteco can help find vulnerability-witnessing tests, though the matching with the right vulnerability is yet to be solved; the findings obtained lay the stepping stone for future research on the matter.
\end{abstract}

%ACM
% \begin{CCSXML}
%     <ccs2012>
%     <concept>
%     <concept_id>10011007.10011074.10011099.10011102.10011103</concept_id>
%     <concept_desc>Software and its engineering~Software testing and debugging</concept_desc>
%     <concept_significance>500</concept_significance>
%     </concept>
%     <concept>
%     <concept_id>10011007.10011006.10011072</concept_id>
%     <concept_desc>Software and its engineering~Software libraries and repositories</concept_desc>
%     <concept_significance>500</concept_significance>
%     </concept>
%     <concept>
%     <concept_id>10002978.10003022.10003023</concept_id>
%     <concept_desc>Security and privacy~Software security engineering</concept_desc>
%     <concept_significance>500</concept_significance>
%     </concept>
%     </ccs2012>
% \end{CCSXML}
% \ccsdesc[500]{Software and its engineering~Software testing and debugging}
% \ccsdesc[500]{Software and its engineering~Software libraries and repositories}
% \ccsdesc[500]{Security and privacy~Software security engineering}
% \keywords{Vulnerability Testing, Security Testing, Unit Testing, Mining Software Repositories, Large Language Models, Deep Learning}

%IEEE
\begin{IEEEkeywords}
Mining Software Repositories, Vulnerability-witnessing Tests, Security Testing, Unit Testing, Language Models
\end{IEEEkeywords}

%ACM
%\maketitle

\section{Introduction}
\label{sec:intro}

\begin{figure*}[tb]
    \centering
    \includegraphics[width=0.848\linewidth]{figs/circuitnn.pdf} 
    \caption{Illustration of differentiable CircuitNN. CircuitNN is designed based on differentiable NAND gates. After DAS is guided by PI and PO pairs of the truth table, CircuitNN can get the precise circuit architecture logic equivalent to the truth table.}
    \label{fig:circuitnn}
\end{figure*}

% 1. Describe the importance of logic synthesis
% 2. Existing Problems
% (a) Neural Architecture Search: Unstable, Predefined Setting, etc.
% (b) Circuit Generation: Probabilistic Model, Logic Equivalence

With the rapid advancement of technology, the scale of integrated circuits (ICs) has expanded exponentially. 
This expansion has introduced significant challenges in chip manufacturing, particularly concerning power and area metrics.
A primary objective in IC design is achieving the same circuit function with fewer transistors, thereby reducing power usage and area occupancy.

Logic synthesis~\cite{hachtel2005logicsynth}, a critical step in electronic design automation (EDA), transforms behavioral-level circuit designs into optimized gate-level circuits, ultimately yielding the final IC layout. 
The primary goal of logic synthesis is to identify the physical implementation with the fewest gates for a given circuit function. 
This task constitutes a challenging NP-hard combinatorial optimization problem. 
Current logic synthesis tools~\cite{brayton2010abc, wolf2013yosys} rely on human-designed heuristics, often leading to sub-optimal outcomes.

Differentiable architecture search (DAS) techniques~\cite{liu2018darts, chu2020darts} offer novel perspectives on addressing challenges in this problem.
Circuit functions can be represented through truth tables, which map binary inputs to their corresponding outputs. 
Truth tables provide a precise representation of input-output relationships, ensuring the design of functionally equivalent circuits.
Inspired by this, researchers~\cite{deepmind2024ai4sys, wang2024tnet} have begun exploring the application of DAS to synthesize circuits directly from truth tables.
Specifically, \citet{deepmind2024ai4sys} proposed CircuitNN, a framework that learns differentiable connection structures with logic gates, enabling the automatic generation of logic circuits from truth tables.
This approach significantly reduces the complexity of traditional circuit generation. 
Building on this, \citet{wang2024tnet} introduced T-Net, a triangle-shaped variant of CircuitNN, incorporating regularization techniques to enhance the efficiency of DAS.

Despite these advancements, several challenges remain. 
The computational complexity of DAS grows quadratically with the number of gates, posing scalability issues.
Although triangle-shaped architecture~\cite{wang2024tnet} partially mitigates this problem, redundancy persists. 
%Additionally, DAS is susceptible to converging to local optima, limiting the ability to search architectures that satisfy the given truth tables~\cite{liu2018darts}. 
%Furthermore, hyperparameters (network depth and layer width) require extensive searches, introducing complexity and prolonging the synthesis process. 
Additionally, DAS is susceptible to converging to local optima~\cite{liu2018darts} and hyperparameters (network depth and layer width) require extensive searches. 
The challenges arise from the vast search space in DAS. 
% Even with predefined settings for CircuitNN, finding a configuration that meets the truth table requires extensive trial and error during the DAS process. 
Intuitively, limiting the search space through predefined parameters (network depth, gates per layer, and connection probabilities) can significantly reduce the complexity.

Recent advances~\cite{openai2023gpt4, abramson2024alphafold3, esser2024sd3, li2024mar} in conditional generative models have demonstrated remarkable performance across language, vision, and graph generation tasks. 
Motivated by these developments, we propose a novel approach to circuit generation that generates preliminary circuit structures to guide DAS in generating refined circuits matching specified truth tables. 
Firstly, we introduce CircuitVQ, a tokenizer with a discrete codebook for circuit tokenization. 
Built upon our Circuit AutoEncoder framework~\cite{hou2022graphmae,li2023maskgae,wu2025mgvga}, CircuitVQ is trained through a circuit reconstruction task. 
Specifically, the CircuitVQ encoder encodes input circuits into discrete tokens using a learnable codebook, while the decoder reconstructs the circuit adjacency matrix based on these tokens.
Subsequently, the CircuitVQ encoder serves as a circuit tokenizer for CircuitAR pretraining, which employs a masked autoregressive modeling paradigm~\cite{chang2022maskgit, li2023mage}. 
In this process, the discrete codes function as supervision signals. 
After training, CircuitAR can generate discrete tokens progressively, which can be decoded into initial circuit structures by the decoder of the CircuitVQ. 
These prior insights can guide DAS in producing refined circuits that match the target truth tables precisely.

Our key contributions can be summarized as follows:
\begin{itemize}
\item We introduce CircuitVQ, a circuit tokenizer that facilitates graph autoregressive modeling for circuit generation, based on our Circuit AutoEncoder framework;
\item Develop CircuitAR, a model trained using masked autoregressive modeling, which generates initial circuit structures conditioned on given truth tables;
\item Propose a refinement framework that integrates differentiable architecture search to produce functionally equivalent circuits guided by target truth tables;
\item Comprehensive experiments demonstrating the scalability and capability emergence of our CircuitAR and the superior performance of the proposed circuit generation approach.
\end{itemize}

% Motivation
% (a) Diffusion (Vision, Graph), Autoregressive (Language, Vision)
% (b) Circuit Generation for Predefined Setting
% (c) Neural Architecture Search for Strict Logic Equivalence

% Contribution
% (a) Circuit Tokenizer (new transformer arch, training strategy)
% (b) CircuitAR (train and gen strategies, post-ar strategy)
% (c) Extensive Evaluation including BitD (Bit Distance) for Scalability

%\section{Motivation}
\label{sec:motivation}



% In LLM inference, not only does weight matter, but the memory requirements of the KV Cache are also considerable.
In this section, we first demonstrate that the emerging paradigm of group quantization demands a high level of adaptivity, which current adaptive methods lack.
We then discuss how adapting these methods to group quantization could compromise their efficiency.
Given that LLMs generate KV caches during runtime, real-time quantization capability is crucial.
These challenges lead to our proposal of a mathematical adaptive numerical type (\texttt{MANT}), which we will detail later.



\begin{figure}[t]
    \centering
    \begin{minipage}[t]{0.48\columnwidth}
      \centering
      \includegraphics[width=\columnwidth]{fig/moti_group_ppl.pdf}
      \caption{LLM accuracy with different quantization granularities. We report the perplexity (PPL) metric (lower is better).}\label{fig:moti_group_ppl} 
    \end{minipage}
    \hspace{2pt}
    \begin{minipage}[t]{0.48\columnwidth}
      \centering
      \includegraphics[width=\columnwidth]{fig/motivation_adaptive_ppl.pdf}
      \caption{Accuracy loss for \texttt{INT}, \texttt{ANT}, and Ideal (clustering algorithm K-Means) adaptive methods in group quantization. }\label{fig:moti_ppl} 
    \end{minipage}
    % \vspace*{-0.3cm}
\end{figure}




\subsection{Group Quantization Accuracy Analysis}
\label{sec:acc_analysis}

In this subsection, we begin by comparing the accuracy of traditional channel-wise quantization with group-wise quantization~\cite{shao2024omniquant,zhao2023atom,liu2024kivi,sheng2023flexgen,lin2023awq,zhao2023atom}, establishing the baseline for group-wise quantization in this study.
We then delve into the use of various adaptive data types in group quantization, emphasizing the necessity for full adaptivity.



\Fig{fig:moti_group_ppl} illustrates the perplexity when quantizing the LLaMA-7B model~\cite{touvron2023llama} with various granularities using the \texttt{INT4}-based symmetric quantization.
Channel-wise quantization significantly worsens the perplexity of the examined LLM, increasing it from 5.68 to 6.85.
Conversely, group-wise quantization mitigates this loss in perplexity with a group size of 128, corresponding to an average of 4.125 bits per element (16-bit scaling factor).
Additionally, we observe that a smaller group size of 32 offers only a slight improvement in perplexity, but the scaling factor overhead increases by $4\times$.



Given this analysis, we adopt a group size of 128 as our standard configuration for the remainder of this section.
Previous research indicates that the \texttt{INT} data type is not optimal for accuracy since tensors or channels exhibit varied distributions, leading to the proposal of various adaptive data types~\cite{guo2022ant, guo2023olive, zadeh2020gobo, zadeh2022mokey}.
We evaluate their efficacy in the context of group quantization, which falls into two main categories: data-type-based and clustering-based.



\textbf{Data-type-based adaptive methods} select data types from discrete sets based on tensor data distribution.
ANT~\cite{guo2022ant} is a representative example of the data-type-based method.
ANT packages several different data types for selection, including \texttt{INT} for the uniform distribution, \texttt{PoT} (Power of Two) for the Laplace distribution, and \texttt{flint} for the Gaussian distribution.
%ANT designed \texttt{flint} for Gaussian distributions.

\textbf{Clustering-based adaptive methods} utilize clustering algorithms to generate centroids that align with the data distribution and provide considerable adaptivity. 
Mokey~\cite{zadeh2022mokey} and GOBO~\cite{zadeh2020gobo} exemplify this approach, though they focus on tensor- or channel-wise quantization. In our study, we adapt them to group quantization through per-group clustering.

%Clustering-based methods employ clustering algorithms to generate centroids that fit the data distribution, demonstrating sufficient adaptivity.
%Mokey~\cite{zadeh2022mokey} and GOBO~\cite{zadeh2020gobo} are such presentative works, but only target tensor- or channel-wise quantization.
%In our work, we modify those works to support group quantization by performing per-group clustering.
\Fig{fig:moti_ppl} compares the accuracy of the methods described above for the LLaMA-7B model under 4-bit group-wise quantization. 
The group-wise \texttt{ANT} method outperforms the \texttt{INT} type by dynamically selecting from three data types to better match the value distribution, resulting in reduced perplexity (PPL) loss. 
Moreover, per-group clustering adjusts more effectively to the value distribution of each group, establishing itself as the accuracy-optimal and ideal adaptive method. 
This approach achieves nearly lossless 4-bit quantization, equivalent to 16 centroids per group. 
However, this ideal scenario is impractical due to the significant overhead associated with storing per-group centroids, effectively rendering it a 6-bit quantization.

\begin{figure}[t] 
    \centering 
    \includegraphics[width=1.0\linewidth]{fig/intro_cdf.pdf}  
    \caption{The cumulative distribution function (CDF) of the tensor, channel, and group, respectively. The tensor data were taken from layers 8 to 23, while the 16 channel and group data were sampled from one tensor with specific strides.}\label{fig:moti_dist} 
    %  \vspace*{-0.3cm}
\end{figure}

To illustrate the group-wise diversity in data distribution, we sampled the weights of the Q and V tensors in LLaMA-7B model. 
We normalized all sampled data to their absolute maximum values, which ranged from -1 to 1. \Fig{fig:moti_dist} displays the cumulative distribution function (CDF) for the tensor, channel, and group levels, respectively. 
We observed that the diversity at the group level is significantly higher than at the tensor level. 
In simpler terms, while different tensors exhibit similar distributions, groups can have markedly different distributions. This finding underscores the necessity for full adaptivity in group quantization to fully realize its potential.
\paragraph{Takeaway 1.} The group quantization is an emerging paradigm to accelerate LLMs, and the significant group-level diversity requires a high level of adaptivity to fully unleash its potential.

\subsection{Group Quantization Efficiency Analysis}
\label{subsec:efficiency}


In this subsection, we provide a detailed efficiency analysis for the above adaptive quantization methods.
In \Tbl{intro:dtype}, we compare OliVe~\cite{guo2023olive}, ANT~\cite{guo2022ant}, GOBO~\cite{zadeh2020gobo}, and Mokey~\cite{zadeh2022mokey} with \texttt{INT} regarding the efficiency of computation, encoding, and decoding. 
In this paper, we use the term encoding (decoding) interchangeably with quantization (dequantization).
 

Data-type-based adaptive methods such as ANT~\cite{guo2022ant} and Olive~\cite{guo2023olive} achieve computational efficiency comparable to \texttt{INT}. 
Both utilize specialized decoders that decode these data types prior to computation, resulting in high decoding efficiency. 
However, as previously demonstrated, these methods suffer from limited adaptivity in the group quantization paradigm. 
A straightforward approach to enhance adaptivity is to expand their set of data types. 
However, incorporating new data types necessitates additional decoders, escalating hardware design costs. 
Additionally, compatibility issues between new and existing data types may reduce computational efficiency. 
For instance, the \texttt{NF4} data type~\cite{dettmers2023qlora} requires an FP16 MAC unit, which is incompatible with existing \texttt{ANT} data types.


\paragraph{Takeaway 2.} Enhancing the data-type-based adaptive method for group quantization is challenging and requires a careful balance for the computation and decoding efficiency.

Clustering-based adaptive methods like GOBO~\cite{zadeh2020gobo} and Mokey~\cite{zadeh2022mokey} can sufficiently adapt to various distributions at the group level. 
However, they require codebooks for quantization and dequantization, leading to high adaptivity at the expense of encoding and computational efficiency. 
For instance, a 16-entry codebook with 8 bits per entry requires 128 bits per group, creating an inevitable trade-off between adaptivity and memory overhead. GOBO~\cite{zadeh2020gobo} employs the K-means algorithm to quantize weights and requires dequantization to \texttt{FP16} using a codebook lookup table before computation, resulting in high adaptivity but low computational efficiency. 
Conversely, Mokey~\cite{zadeh2022mokey} enhances the computation of clustering-based methods by using indices for centroid values via approximate calculations, though matrix multiplication still relies on floating-point units, increasing overhead compared to integer units. 
Furthermore, Mokey creates one \texttt{golden dictionary} for all activations and weights, akin to using a single data type in quantization, thus reducing adaptivity.


\paragraph{Takeaway 3.} Deploying the clustering-based adaptive methods under group quantization is challenging owing to the low encoding and computation efficiency. 


\begin{table}[t]
    \centering
    \small
    \renewcommand{\arraystretch}{1.2}
    \caption[]{Features of DNN accelerators with adaptive and flexible data types are summarized. Here, `Effi.' stands for efficiency, `Med.' for medium, and `LUT' for lookup table.}
  
    \resizebox{1.0\columnwidth}{!}{
      \begin{tabular}{c|cc|ccc|cc|c}
        \Xhline{1.2pt}
        \multirow{2}{*}{Architecture} & \multicolumn{2}{c|}{Encode} & \multicolumn{3}{c|}{Computation} & \multicolumn{2}{c|}{Decode} & \multirow{2}{*}{Adaptivity} \\ \cline{2-8}
        & Method & Effi. & Method & Bit & Effi. & Method & Effi. \\
        \Xhline{1.2pt}
        \texttt{INT} & Round & High & INT & 4 \& 8 & High & Calculation & High & Low \\ 
        OliVe~\cite{guo2023olive} & Search & Med. & INT & 4 \& 8 & High & Decoder & High & Med. \\ 
        ANT~\cite{guo2022ant} & Search & Med. & INT & 4 \& 8 & High & Decoder & High & Med. \\ 
        Mokey~\cite{zadeh2022mokey} & Cluster & Med. & Float & 4 \& 8 & Med. & Calculation & Med. & Low \\ 
        GOBO~\cite{zadeh2020gobo} & Cluster & Low & Float & 16 & Low & LUT & Med. & High \\ 
        \hline
        \multirow{2}{*}{\proj}  & Search  & Med.  & \multirow{2}{*}{INT} & \multirow{2}{*}{4 \& 8} & \multirow{2}{*}{High} & \multirow{2}{*}{Calculation} & \multirow{2}{*}{High} & \multirow{2}{*}{High} \\ \cline{2-3}
        &  Map &  High &  &&&\\ 
        \Xhline{1.2pt}
    \end{tabular}
    }
    \vspace*{0.1cm}
    \label{intro:dtype}
    \vspace*{-0.2cm}
  \end{table}

\subsection{Support for Real-time Quantization}
\label{sec:moti_kvcache}

The above group-wise diversity presents a challenge for both weights and KV cache.
In addition, KV cache faces challenges in real-time group-wise quantization because the KV cache is generated dynamically during LLM inference.


To facilitate low-precision computation in group-wise quantization, it is necessary to quantize K and V along the inner dimension. 
This requirement stems from the support for matrix inner product operations in most GPUs and TPUs. 
During these operations, the group-wise scaling factor can be extracted from the multiply-accumulate process. 
\Fig{fig:kv_process} depicts the computation process of K and V during the decode stage. We define the dimension used for matrix inner product operations as the inner dimension. 
The inner dimensions of the K and V caches differ; the K cache requires a transpose operation, whereas the V cache does not, complicating the situation.


In the prefill stage, K and V can easily compute the scaling factor for each group. 
During the decode stage, the newly generated K vector is concatenated along the inner dimension of the K cache, enabling immediate quantization. 
However, the newly generated V vector is associated with different groups, with only one element per group produced per iteration. This process prevents the scaling factor for the entire group from being obtained in a single iteration, posing a significant challenge for the real-time quantization of the V cache.


\begin{figure}[t] 
  \centering 
  % \includegraphics[width=1.0\linewidth]{fig/dse_kv_process.pdf}  
  \includegraphics[width=0.9\linewidth]{fig/moti_kv_dimension.pdf}  
  \caption{\small Comparison of group-wise K and V cache quantization. They have different inner dimensions due to the transposition of K (key).}

  \label{fig:kv_process}
  % \vspace*{-0.4cm}
\end{figure}


Given those challenges, we propose \proj with a mathematical encoding format that can fuse with integer computation and enhance the decoding efficiency.
In addition, this encoding format provides sufficient adaptivity for group-wise quantization.
Regarding the challenge in KV cache, \proj employs a real-time quantization engine that ensures efficient encoding and decoding for KV cache.
By addressing these challenges, \proj enables efficient low-bit group-wise quantization.


\section{The \vuteco Approach}
\label{sec:vuteco}

\subsection{Overview}
\label{subsec:vuteco_overview}

\vuteco faces two distinct tasks.
The \finding task accepts a \JUnit test case as input and tells whether it is security-related (i.e., \finderPosClass) or not (i.e., \finderNegClass).
The \matching task accepts a \JUnit test case and a description (in natural language) of a known historical vulnerability (related to the project) and tells whether the test case is witnessing that vulnerability (i.e., \globalPosClass) or not (i.e., \globalNegClass).
%The main downstream task of \vuteco consists of (1) inspecting \JUnit test suites, (2) finding candidate vulnerability-witnessing test cases, and (3) matching them with the right vulnerability.

The \vuteco approach has been wrapped into a tool (having the same name, \vuteco) that automates both tasks.
The tool accepts (i) a \Java project repository, (ii) the project revision (i.e., commit) to inspect, and (iii) a list of descriptions (in natural language) of known historical vulnerabilities related to the project---taken from CVE (Common Vulnerabilities and Exposures).
Any \textsc{Git}-based repository is valid as long as there are \JUnit test cases to process.
%; the only requirement is that it has to describe what the vulnerability consists of via natural language.
%\CB{Finding vulnerability patches~\cite{zhou2017automated, nguyen2023multi}, identifying vulnerable software versions (V-SZZ)~\cite{bao2022v}, vulnerability assessment~\cite{le2019automated}}

Figure~\ref{fig:vuteco-overview} depicts the general functioning of \vuteco.
Once checking out the project repository to the selected revision,
%\vuteco extracts all the valid \JUnit test methods adopting a heuristic approach. Namely,
\vuteco parses all \Java files and marks any class method having the following properties as a test case:
\begin{enumerate}[leftmargin=*]
    \item it is annotated with \texttt{@Test} (\JUnit 4 and 5) or its class extends \textsc{TestCase} or any subclass of it (for \JUnit 3);
    \item it is not overriding a method defined in class \textsc{TestCase}, like \texttt{run()} or \texttt{getName()} (for \JUnit 3);
    \item it is not a ``lifecycle method'', i.e., annotated with \texttt{@BeforeAll}, \texttt{@AfterAll}, \texttt{@BeforeEach}, or \texttt{@AfterEach};
    \item it returns \texttt{void} if not annotated with \texttt{@TestFactory};
    \item it is not \texttt{abstract}, \texttt{static} or \texttt{private};
    \item its class is not \texttt{abstract}.
\end{enumerate}
Such properties have been designed according to how the \JUnit guide describes a test case~\cite{junit:guide} and the official \textsc{Javadoc} of \JUnit beyond version~3.

Each mined test case is sent to the \finder model, which is responsible for implementing the \finding task (i.e., determining whether the test case is security-related).
Then, each test flagged as security-related is sent to the \linker model along with the descriptions of all vulnerabilities supplied via input to determine which of them is witnessed by the test case.
%This joint use of \finder and \linker models implements the \matching task (the reason for using two models rather than one is motivated by the results of the in-vitro experimentation, which are presented in Section~\ref{sec:results}).
%Each candidate test is then paired with all vulnerability descriptions supplied via input and sent to the second model, i.e., the \linker, whose goal is to assess if the candidate is witnessing the vulnerability described by the short text.
%Ultimately, \vuteco returns all the test cases that were successfully linked with a vulnerability.
Sections~\ref{subsec:finding} and~\ref{subsec:matching} describe how the \finding and \matching tasks have been carried out.
The design choices behind the models used there are supported by the experimentation described in Section~\ref{subsec:invitro}.
%and the related results in Section~\ref{subsec:fnd-results} and~\ref{subsec:match-results}.

\begin{figure}[t]
    \centering
    \resizebox{.98\linewidth}{!}{
        \includegraphics{figures/vuteco-overview.pdf}
    }
    \caption{Graphical overview of the functioning of \vuteco.}
    \label{fig:vuteco-overview}
\end{figure}

%\subsection{Technical Detail}

\subsection{The \finding Task}
\label{subsec:finding}

The \finding task consists of determining if a test case is security-related, i.e., it is focused on some security properties of the project where it belongs.
The \finder model addresses this task, trained to classify test cases into two classes, i.e., \finderPosClass (the positive class) and \finderNegClass (the negative class).
%
The upper part of Figure~\ref{fig:vuteco-detail} depicts the architecture of the \finder model, which consists of a deep neural network built on top of a pre-trained CodeBERT~\cite{feng:emnlp2020:codebert}.
The pre-trained CodeBERT starts from the checkpoint \texttt{microsoft/codebert-base} downloaded from \textsc{HuggingFace}~\cite{codebert:base:hf}.
%Essentially, it is a specialized RoBERTa model~\cite{liu2019roberta} made of 125M trainable weights.
CodeBERT has been pre-trained on
%with masked language modeling (MLM) and replaced token detection (RTD) self-supervised tasks for
code examples of six programming languages, including \Java, paired with natural language text.
Hence,
%leveraging the pre-training made from its base RoBERTa model~\cite{liu2019roberta} and how CodeBERT has been trained,
we deemed it suitable for understanding the content of \JUnit test methods.

As a preliminary action, \vuteco transforms the input test case by removing new line characters and consecutive white space characters (including tabs), reducing the whole method into a single line.
Then, the resulting string is tokenized using the WordPiece algorithm~\cite{wu2016googles:wordpiece} (whose vocabulary was fitted during the pre-training of CodeBERT), and each resulting token is replaced with a numeric identifier.
%Such a step is mandatory as Transformer-based models can only interpret sentences as numerical vectors.
Then, the resulting numeric vector is sent to the CodeBERT input layer (supporting up to 512 encoded tokens), which returns an embedded representation of each token.
Besides, due to the underlying BERT-based architecture, an additional embedding for the special token \texttt{[CLS]} is returned, which has the goal of capturing the whole syntax and semantics of the entire sentence (here, the whole test case), making it suitable for sentence classification tasks.
Considering the goal of the \finding task, we only selected the embedded representation of \texttt{[CLS]}, made of 768 values.
%
On top of this, we added a deep neural network with an input layer of 768 neurons, matching the size of the sentence embedding resulting from the CodeBERT model; then, we added two linear hidden layers with 768 and 64 output neurons, respectively, calling GELU activation function~\cite{hendrycks2023gaussian:gelu}.
After this, the final layer returns the two logits needed for the final classification, which is done through the Softmax function to return the probabilities for the two classes.

\subsection{The \matching Task}
\label{subsec:matching}

\begin{figure}[t]
    \centering
    \resizebox{.98\linewidth}{!}{
        \includegraphics{figures/vuteco-detail.pdf}
    }
    \caption{\EI{FIX, broken!} Graphical depiction of the inner working of \vuteco.}
    \label{fig:vuteco-detail}
\end{figure}

The \matching task consists of determining if a test case and a vulnerability are related, i.e., the former witnesses the latter.
This task is carried out by the joint use of the \finder model (as in Section~\ref{subsec:finding}) plus the \linker model.
Indeed, the \linker model is focused on a \textit{simplified} task that assumes the input test case is surely security-related.
The motivation for the joint use of two models---rather than the direct use of the \linker model---is supported by the results of the in-vitro experimentation in Section~\ref{sec:results}.

The \linker model has been trained on its simplified task, i.e., trained to classify pairs of security-related tests and vulnerability descriptions into two classes, i.e., \linkerPosClass (the positive class) and \linkerNegClass (the negative class).
%
The lower part of Figure~\ref{fig:vuteco-detail} depicts the architecture of the \linker model.
Such a model follows a similar architecture to the \finder model, leveraging the same pre-trained CodeBERT model to create the input embeddings.
However, instead of creating the sentence embedding of the sole test case code (still reduced to a single line), the \linker model also adds the vulnerability description as well.
Therefore, the vulnerability description was \textit{concatenated} before the test case with a special token [SEP] in between to indicate the two different sentences (as required by BERT-like models) and then tokenized in the same manner as the \finder model.
%
The resulting sentence embedding is then sent to a deep neural network with an input layer of 768 neurons (to match the embedding size) and a linear hidden layer of 256 output neurons, calling the GELU activation function.
%
Like the \finder model, the final layer returns the logits for the classification, converted into probabilities of the two classes with the Softmax function.

To enable the joint training and use of the \finder and \linker models, \vuteco employs a \textit{decision function} after the \finder made its prediction on the test case---depicted on the right-most side of Figure~\ref{fig:vuteco-detail}.
Indeed, if the probability of the test case belonging to the \finderPosClass class was at least $0.5$, then the \linker model is invoked, and its probabilities are used for the \globalPosClass and \globalNegClass classes.
Otherwise, the probabilities of the \finder are used instead.
Simply put, the \linker's judgment is not considered if the \finder model did not flag the test case as \finderPosClass as the \linker model expects security-related tests as input.
%
Hereafter, the wording ``integrated model'' refers to the joint use of \finder and \linker to address the \matching task.

%\subsubsection{The \finder and \linker Integration}

%The right-most side of Figure~\ref{fig:vuteco-detail} shows how \finder and \linker models have been used in conjunction to address the main downstream task.
%Namely, once the \finder and \linker models made their predictions, \vuteco returns the logits from the \linker model if the probability of \finderPosClass (computed applying the Softmax function on the logits) was more than $0.1$, otherwise \vuteco returns the logits computed by the \finder model.
%Hereafter, the wording ``integrated model,'' refers to the integration of \finder and \linker for the main downstream task.
%This selection is due to the \linker model's inability to make reliable predictions if the test method has not been flagged as a suspect vulnerability-witnessing.
%Consequently, if the \finder model flagged the test as a suspect, the prediction made by the \linker model becomes meaningful, and it will be used to assess if the supposed witnessing test is linked to the given vulnerability description.
%We refer to this integration as the \globalModel model for simplicity.

% \subsection{Training the Deployment Version}
% \label{subsub:train}

% The \finder and \linker models were trained in individual sessions to become more familiar with their sub-tasks before the integration.
% %---i.e., assessing the links between a \JUnit test method and a vulnerability description.
% Afterward, the integrated model was trained further to perform well on the main downstream task.
% As a result, three training sessions were needed: One for the \finder model, one for the \linker, and one for the integrated model.
% From \vuteco's perspective, the first two sessions can be seen as a \textit{pre-training}, while the latter as a \textit{fine-tuning}.
% The three training datasets have been extracted from \VulforJ, the only available knowledge base containing \JUnit test cases with the witnessed vulnerabilities.

% The deployment version of \vuteco relies on the results of the in-vitro evaluation (Section~\ref{subsec:in-vitro-results} to configure the training process.
% %
% The \finder model was trained on \finderTrainingSize \JUnit test cases, of which \finderTrainingPosSize were witnessing a vulnerability ($\sim$\finderTrainingPosPerc).
% %
% Likewise, the \linker model had a training set of \linkerTrainingSize pairs of witnessing test cases and vulnerability descriptions, of which \linkerTrainingPosSize represented valid links ($\sim$\linkerTrainingPosPerc).
% To handle such a class imbalance, this training set has been \textit{augmented} via bootstrapping (i.e., sampling with replacements) the \linkerPosClass pairs until the final class imbalance resulted in 1:2 (i.e., one valid link for every two invalid links), ending up with \linkerTrainingExtra new instances of class \linkerPosClass.
% %
% After training the two models for their respective sub-tasks, the integrated model was trained on a third training set made of \globalTrainingSize pairs of test cases and vulnerabilities, of which \globalTrainingPosSize were valid ($\sim$\globalTrainingPosPerc).

% The \finder, \linker, and the integrated model have been trained for \finderTrainingEpochs, \linkerTrainingEpochs, and \globalTrainingEpochs epochs, respectively.
% In all three sessions, 15\% of the training sets were used as \textit{development sets} to find the best training checkpoint to return at the end.
% Namely, the models were evaluated on their development sets at the end of each epoch, and the best checkpoint was determined by the highest \TEMP{AUC-ROC}~\cite{Junge2018:roc} and lowest cross-entropy loss (in case of ties).

\section{Design}\label{sec:design}

%%%%%%%%%%%%%%%%%%%%%%%%%%%%%%


\begin{figure*}[t]
    \centering
    \includegraphics[trim = 15 530 15 15, width=1\textwidth]{Algorithm_drawio.pdf}
    \caption{Overview of KiSS}
    \label{fig:overview}
\end{figure*}


The results we gleaned from the previous section (see Section~\ref{sec:work_anly}) helped in developing our policy: KiSS. The KiSS or \textbf{Keep it Separated Serverless} policy aims to address critical challenges in Function-as-a-Service (FaaS) platforms, particularly in edge computing environments, by achieving the following objectives:

\begin{itemize}
    \item \textbf{Reduced Cold Start Latency:} Prioritizes high-frequency functions to minimize delays in real-time applications.
    \item \textbf{Improved Resource Efficiency:} Optimizes memory and compute usage while avoiding unnecessary overhead from static warm states.
    \item \textbf{Minimized Inter-Function Interference:} Enhances throughput and scalability through modular resource partitioning.
    \item \textbf{Improved Function Service Rate:} Adopts resource-aware policies to reduce dropped requests and maximize system reliability.
\end{itemize}


\subsection{KiSS Policy Overview}

KiSS introduces a modular, data-driven orchestration strategy designed to optimize serverless execution in resource-constrained environments, particularly at the edge. By leveraging our workload analysis (refer Section 2.5), our policy segments functions based on key metrics—memory footprint, invocation frequency, and execution time—to optimize performance across diverse workloads.

The edge computing context introduces unique challenges like limited memory, heterogeneous resources, and dynamic workloads. Generalized cloud strategies often fail to adapt to such constraints. KiSS addresses this gap by analyzing workload characteristics and implementing a resource-efficient, modular strategy that aligns with edge-specific demands.

\subsection{Components of KiSS Policy Design}
Figure~\ref{fig:overview} shows the overall architecture of KiSS. 
The incoming \textit{FaaS traffic} will include both small and large functions. 
The \textit{request handler} accepts the incoming functions and shares the function information to the workload analyzer. 
The \textit{workload analyser} processes the function information to profile the incoming function traffic information and generate data such as invocation frequency, memory footprint etc.
The \textit{KiSS policy} uses this data to estimate where this function will be placed between the two different warm pool partitions.

The \textit{load balancer} implements a partitioning logic where functions are allocated to distinct warm pools using (\textit{invoker 1} and \textit{invoker 2}) based on profiling thresholds:

(i)~Small Functions Pool: Dedicated to high-frequency, low-memory functions to ensure low latency, and (ii)~Large Functions Pool: Allocated for low-frequency, memory-intensive functions, minimizing contention with smaller containers.
Each warm pool operates autonomously achieving Policy Independence.
The \textit{Warm Pool Replacement Policy} for each warm container pool can independently implement different workload-specific strategies to reduce contention and enhance temporal locality.


These factors form the foundation of KiSS’s multi-tiered warm pool framework, allowing it to effectively manage serverless resources and enhance performance in edge computing. By addressing these challenges, KiSS positions itself as a practical and scalable solution for FaaS platforms in environments with diverse and demanding resource constraints.


\subsection{Innovations of KiSS Policy}

One of the most innovative features of KiSS is its multi-level warm pool partitioning, which isolates high- and low-frequency functions into separate pools. This design eliminates inefficiencies inherent in monolithic resource strategies by ensuring that small, frequently invoked functions are always ready to execute, while larger, less frequent functions remain accessible without competing for resources. This adaptability extends to the ability to add more pools as workload patterns evolve, making KiSS a flexible and future-proof solution. Moreover, its modular architecture supports diverse deployment scenarios, from centralized clouds to resource-constrained edge environments. Integration with traffic-aware schedulers ensures that KiSS maintains scalability and responsiveness even under fluctuating workloads.


\subsubsection{Advantages of KiSS}

The advantages of KiSS are particularly pronounced in edge environments. By keeping frequently accessed containers in warm states, it drastically reduces cold start latency, which is critical for real-time applications such as IoT and AI analytics. Static warm pool partitioning, based on workload analysis, optimizes memory usage by eliminating unnecessary overhead, ensuring that resources are used efficiently even in environments with stringent memory constraints. This strategy not only enhances performance but also reduces operational costs by consolidating memory usage and minimizing cold starts. KiSS’s platform-agnostic design further enhances its versatility, enabling seamless deployment across various serverless frameworks.


\begin{table}[ht!]
\centering
\caption{\textbf{Super Resolution Performance Results.} Our proposed WGAN EEG Spatial Upsampling method significantly outperforms a baseline of Bicubic Interpolation commonly used in EEG upsampling pipelines.}
\label{tab:results}
\resizebox{0.8\linewidth}{!}{%
\begin{tabular}{@{}cccccc@{}}
\toprule
\multirow{2}{*}{\textbf{Dataset}} & \multirow{2}{*}{\textbf{Scale}} & \multicolumn{2}{c}{\textbf{Bicubic}} & \multicolumn{2}{c}{\textbf{WGAN}} \\ \cmidrule(l){3-6} 
                      &   & \textbf{MSE} & \textbf{MAE} & \textbf{MSE}    & \textbf{MAE}   \\
\toprule
\multirow{2}{*}{Val}  & 2 & 3.71E7       & 3.89E3       & \textbf{2.01E3} & \textbf{24.38} \\
                      & 4 & 7.23E7       & 6.42E3       & \textbf{8.53E3} & \textbf{63.83} \\
\midrule
\multirow{2}{*}{Test} & 2 & 3.75E7       & 3.91E3       & \textbf{2.06E3} & \textbf{24.66} \\
                      & 4 & 7.30E7       & 6.45E3       & \textbf{8.68E3} & \textbf{64.39} \\
\bottomrule
\end{tabular}%
}
\end{table}
This work identifies signal collapse as a critical bottleneck in one-shot neural network pruning. Performance loss in pruned networks is due to \textbf{signal collapse} in addition to the removal of critical parameters. We propose \textbf{REFLOW} (\textbf{Re}storing \textbf{F}low of \textbf{Low}-variance signals), a simple yet effective method that mitigates signal collapse without computationally expensive weight updates. By focusing on signal preservation, REFLOW highlights the importance of mitigating signal collapse in sparse networks and enables magnitude pruning to match or surpass state-of-the-art one-shot pruning methods such as CHITA, CBS, and WF.

REFLOW consistently achieves state-of-the-art accuracy across diverse architectures, restoring ResNeXt-101 from under 4.1\% to 78.9\% top-1 accuracy at 80\% sparsity on ImageNet. Its lightweight design makes it a practical solution for both research and deployment, delivering high-quality sparse models without the overhead of traditional approaches. These findings challenge the traditional emphasis on weight selection strategies and underscore the critical role of signal propagation for achieving high-quality sparse networks in the context of one-shot pruning.



\section{Threats to Validity}
\label{sec:ttv}

%\EI{TTV on using only the CVEs from ProjectKB vs all the CVEs for those projects mined from NVD: this would never reduce the false matches, but, in the best case, increase the true matches. Besides, this could backfire... not sure}

%\subsection{Threats to Internal Validity}

%\CB{As I remember, the test can be predicted to be linked to cve $A^*$. The truly associated cve of the test is $A$---not existing in the training dataset, but $A$ and $A^*$ share the similarity such as vuln type, project context, etc.\footnote{For example, in the case of CVE-2018-1000865 and CVE-2018-1000866 in \url{https://docs.google.com/spreadsheets/d/1ZAZ5GWLGZFmtFbuQyIS8k0n_RgZbAW2A2WckpYV5SLE/edit}}}
For the \matching task, we considered the performance of the \finder and \linker models in their tasks independently before integrating them.
%, emulating a bottom-up integration testing scenario (explained in Section~\ref{subsubsec:overview}).
Yet, there was no guarantee that the selected configurations would work well once integrated.
%does not re-use the exact models from their sessions but re-use the best configuration observed.
%and re-train them on a different dataset (the motivations are given in Section~\ref{subsubsec:overview}).
%Therefore, the final behavior might be different than expected.
We mitigated this threat by running a new train-test session to assess the whole integrated model for the actual \matching task.
The final answer to the in-vitro analysis on \rqTwo is only given by the results achieved by the integrated model.
%When evaluating the \globalModel model, we only searched its optimal configuration without tuning the \finder and \linker models again, as this would have required searching among 13 hyper-parameters (Table~\ref{tab:hyperparams}), expanding the search space too much.

Due to computational constraints, we could not thoroughly analyze the impact of many design aspects in the in-vitro analyses.
%, so we prioritized only those aspects that could provide some benefits.
We treated some aspects as factors, like data augmentation, to be evaluated at test time, while others were treated as hyper-parameters to be optimized at development time.
%We did not employ statistical tests to compare the various groups due to the limited number of configurations belonging to each.
%In Section~\ref{sec:results}, we observed how certain configuration values might provide some benefits, e.g., data augmentation.
%helped the \linker model more than \finder model.
%Yet, we did not employ any statistical tests and just compared the values obtained with the aggregated values.
%Despite not employing statistical tests, the analyses conducted aimed to find the most suitable configuration to prepare \vuteco for real-world usage, experimented in the two in-vivo analyses.
%\EI{The fact that we used ``Learn'' could be mentioned}

%\TEMP{While evaluating \vuteco in-vitro, we moved the classification threshold of all models from the default $0.5$ to a different value that maximized the F0.5 scores.
%However, the differences were marginal on a large scale, so we opted to keep the default thresholds.
%The performance of the tuned configurations is in our online appendix~\cite{appendix}.}

For the \matching task, \vuteco accepts a natural language description of the vulnerability.
Hence, \vuteco accepts any text describing the issue, also from other sources like bug reports~\cite{zhou2017automated,le2019automated,bao2022v,nguyen2023multi,iannone:tosem2024:exploit}.
We opted to experiment with CVE, as it is the most comprehensive catalog for obtaining such information and can be easily queried.
%\CB{I think this makes sense, even the CVE description sometimes seems not to be valid, for example, in this case it just contains texts about the CVE duplication \footnote{\url{https://nvd.nist.gov/vuln/detail/CVE-2015-8581}}.}

%\EI{Overfitting risk: Vuteco might mistake two similar vulnerabilities of the same project if they share text. Similarly, the prediction can go well if similar vulnerabilities fall into the training. Example: CVE-2018-1000865 and CVE-2018-1000866. However, we didn't do anything to mitigate this risk... cross-validation would have been good.}

%\EI{TTV for the baselines, they might be too naive. Besides, we can say that other baseline might exists, and in our appendix we did some others}

%\subsection{Threats to External Validity}

\vuteco has been trained and tested on the \JUnit test cases in \VulforJ.
Therefore, all the results observed cannot be generalized to other programming languages, other testing frameworks (like \textsc{TestNG}), or vulnerability types that are not found in \Java code, like memory-related vulnerabilities.
We chose \Java mainly due to the availability of \VulforJ containing examples of witnessing tests to train and validate \vuteco.
At the same time, we remark that \Java is still a relevant language to analyze from a security perspective as it keeps exhibiting vulnerabilities for a long time~\cite{veracode}.

Despite being the only catalog of its kind, the limited number of witnessing tests in \VulforJ does not allow models to learn many patterns that can be generalized to different contexts.
It also lacks diversified examples, e.g., it has eight examples of tests witnessing CWE-835 (Infinite Loop), but just one witnessing CWE-78 (OS Command Injection)~\cite{bui:msr2022:vul4j}.
We partially handled this by trying to augment the training sets involved during the experimentation, though without the hoped effect.
\vuteco was born to contribute to feeding such knowledge bases with new examples, which in turn could be used to re-train \vuteco and return more results.

%While the in-vitro analysis validated \vuteco on a subset of witnessing tests paired with vulnerability descriptions from real-world \Java projects in \VulforJ, the in-vivo showed its applicability on further real-world projects, extending the setting used for the in-vitro analyses.
%Both support \vuteco's ecological validity.

%\subsection{Threats to Construct Validity}

\vuteco flags witnessing tests based on a static approach: No test cases are executed, and no projects are built.
For this, \vuteco cannot provide a definite answer about whether the tests will trigger the matched vulnerabilities.
The dynamic assessment has been taken outside the scope of this work due to technical difficulties in building numerous projects in a reasonable time, as encountered by the authors of \VulforJ~\cite{bui:msr2022:vul4j}.
%\vuteco's goal is to find candidate vulnerability-witnessing tests whose actual behavior must be assessed in separate sessions. 
%\EI{Additional reasons of why we don't do this: so far we have no knowledge of how witnessing tests appear, since we can't have good empirical studies without data. If we had to build and run the tests automatically and on large scale, most would fail (as seen in Vul4J), so losing many precious candidates for expanding our knowledge on the matter. Of course, in the future this can be addressed.}

%We defined \vuteco's general architecture based on internal trials on the development sets used in the in-vivo evaluations.
%The design choices made have certainly influenced its performance.
%We acknowledge that numerous other hyper-parameters and design decisions could have led to better versions of \vuteco; however, their investigation would have required running thousands of train-test sessions and incurring huge computation costs.
%In this work, we focused on the main factors we believed to be impactful, aiming to define a first working tool for collecting vulnerability-witnessing tests.
%The wide range of possible designs makes it difficult to identify the right solution without running thousands of train-test sessions and incurring huge computation costs.
%We followed similar design choices adopted in studies making classification of test cases~\cite{fatima:tse2023:flakify,akli:ast2023:flakycat}.
%Future improvements can be made to enhance \vuteco's retrieval capabilities for this stable point.
%
%Besides, we know many other aspects could influence the model performance, like employing a weighted loss function or different network architectures, requiring deeper experimentation.

%In \rqOne, we relied on the traditional metrics adopted in information retrieval to assess the performance of the experimented models and determine the most suitable for our purpose.
%To mitigate the risk of selecting models with apparently good performance but without practical usefulness, we also considered the absolute numbers of true and false positives and the Inspection Ratio.

%\EI{TTV regarding the limit to 512 tokens... we partially handled it by removing comments before the test method and one-lining, but we still had X\% cases in which the input was longer than 512 tokens. We accepted the risk and just truncated, and there are no approaches that can identify the relevant part in a test (which can be interesting actually to do)}

%\subsection{Threat to Conclusion Validity}

\section{Related Work}\label{sec:relatedwork}

Internet of Things (IoT) has seen rapid advancements in recent years, becoming an integral part of various domains, such as smart industries and homes, and serving as a key enabler in modern society.
However, despite its growth, IoT continues to face numerous security challenges, prompting significant research efforts aimed at improving IoT security.
With the rise of artificial intelligence (AI), machine learning (ML) and deep learning (DL)-based approaches have become increasingly popular in designing defense mechanisms for IoT devices, including malicious traffic classification~\cite{luo2022transformer,shafiq2020corrauc}, malware detection~\cite{vasan2020mthael,chaganti2022deep,aung2022atlas}, vulnerability discovery~\cite{neshenko2019demystifying}, and others~\cite{al2020survey,otoum2022dl,tambe2019detection}.

More recently, inspired by the success of large language models (LLMs), researchers have begun exploring the potential of LLMs to enhance IoT-related security tasks.
For instance, LLMs have been applied to existing IoT security challenges such as threat detection and fuzzing. Ferrag \etal~\cite{sokiotllm} introduced a BERT-based model, SecurityBERT, to achieve better cyber threat detection accuracy over traditional ML and DL-based methods. 
Similarly, Ma \etal~\cite{ma} and Wang \etal~\cite{llmiotfuz} proposed LLM-assisted fuzzing methods to uncover hidden bugs in IoT devices, enabling the detection of complex vulnerabilities that traditional techniques might miss.
Additionally, Yang \etal~\cite{yang2023iot} combined LLMs with static code analysis using prompt engineering to create a cost-effective solution for IoT vulnerability detection.
\cite{ji2024sevenllm} collected cybersecurity raw texts to train cybersecurity LLM to augment the analysis of cybersecurity events, and \cite{llmtikg} made use of a larger LLM to build knowledge graphs from public threat intelligence and use GPT to create datasets to fine-tune a smaller LLM to extract entities and TTPs from attack description.
Ferraris \etal~\cite{ferraris2024ici} proposed utilizing ChatGPT to enhance IoT trust semantics, aligning with W3C Web of Things (WoT) recommendations\footnote{\scriptsize \url{https://www.w3.org/WoT/}}.
This work extends the TrUStAPIS framework~\cite{ferraris2020trustapis}.

Beyond the above tasks, LLMs have been employed in other IoT challenges.
Meyuhas \etal~\cite{meyuhas2024iotlabel} used LLMs to address the problem of labeling previously unseen IoT devices.
\cite{llmiotcontrol,cui2024llmind} explored leveraging LLMs to control IoT devices and facilitate effective collaboration among them.
Mo \etal~\cite{mo2024iot} collected IoT sensor-natural language paired data and trained IoT-LM to interpret and interact with physical IoT sensors.
Xu \etal~\cite{xu2024penetrative} employed ChatGPT to interpret IoT sensor data and reason over tasks in the physical realm, introducing novel ways of integrating human knowledge into cyber-physical systems. 

Recently, Deldari \etal~\cite{deldari2024auditnet} proposed AuditNet, a conversational AI-based security assistant, which is most similar to \chatiot\ and also augmented by external knowledge.
However, AuditNet focused on standards, policies, and regulations of portable document format (PDF), and aimed to reduce the manual effort of security experts involved in compliance checks of IoT. 
On the other hand, we integrate IoT threat intelligence of various sources into \chatiot\ and can assist multiple kinds of users. Besides, we provide an end-to-end toolkit to process data in various formats, not limited to PDF. 

Together, these studies indicate that LLMs have great potential to improve the security of IoT systems in various domains, from vulnerability discovery to trustworthiness management. 
By integrating LLMs with IoT-specific threat intelligence, these models can be guided to meet the unique challenges posed by the IoT ecosystem.
Moreover, the continuous advancements in the LLM community, combined with increasingly accessible IoT datasets, are likely to further drive the adoption of LLMs in IoT-related research and practical applications.

\section*{Conclusion}
This paper aims to enhance our understanding of the computational complexity of computing various Shapley value variants. We found that for various ML models --- including decision trees, regression tree ensembles, weighted automata, and linear regression --- both local and global interventional and baseline SHAP can be computed in polynomial time under HMM modeled distributions. This extends popular algorithms, such as TreeSHAP, beyond their empirical distributional scope. We also establish strict complexity gaps between the various SHAP variants (baseline, interventional, and conditional) and prove the intractability of computing SHAP for tree ensembles and neural networks in simplified scenarios. Overall, we present SHAP as a versatile framework whose complexity depends on four key factors: \begin{inparaenum}[(i)] \item model type, \item SHAP variant, \item distribution modeling approach, \item and local vs. global explanations\end{inparaenum}. We believe this perspective provides deeper insight into the computational complexity of SHAP, paving the way for future work.




%We believe that our framework provides a more intricate understanding of SHAP computation complexity across different models, distributions, and variants, paving the way for further research.

Our work opens promising directions for future research. First, expanding our computational analysis to other SHAP-related metrics, such as asymmetric SHAP~\citep{frye20} and SAGE~\citep{covert2020understanding}, would be valuable. Additionally, we aim to explore more expressive distribution classes and relaxed assumptions beyond those in Section \ref{sec:tractable} while maintaining tractable SHAP computation. Finally, when exact computation is intractable (Section \ref{sec:intractable}), investigating the approximability of SHAP metrics through approximation and parameterized complexity theory~\citep{downey2012parameterized} is an important direction.

%Our work opens several promising avenues for future research on the computational properties of explainable AI methods, with a particular focus on SHAP. First, it would be interesting to broaden the computational analysis conducted in this work to include other popular SHAP-related metrics in the literature, such as asymmetric SHAP \cite{frye20} and SAGE \cite{covert2020understanding}. Also, in the future, we aim to explore more expressive distribution classes and relaxed distributional assumptions—extending beyond those examined in Section \ref{sec:tractable} —that still yield tractable SHAP computation. Finally, when exact computation proves intractable (Section \ref{sec:intractable}), it is worthwhile to theoretically investigate the question of the approximability of computing the SHAP metrics across various configurations, through the lens of approximation and parametrized complexity theory \cite{arora2009computational}.

%This paper aims to deepen our understanding of the computational complexity involved in obtaining different Shapley value variants. We found that for a variety of ML models, including decision trees, tree ensembles for regression, weighted automata, and linear regression models — computing both local and global interventional and baseline SHAP can be done in polynomial time when distributions are modeled by HMMs. This extends the distributional scope of popular algorithms like TreeSHAP, which is limited to empirical distributions. Additionally, we demonstrate a strict complexity gap between SHAP variants, showing that interventional and baseline SHAP can be strictly easier to compute than conditional SHAP. Despite these positive results, we uncovered intractability for various SHAP variants in neural networks and tree ensembles. Finally, we provided generalized complexity relations across SHAP variants. We believe that our framework offers a deeper understanding of the complexity involved in computing SHAP across various variants, models, distributions, as well as in both local and global computations, laying the groundwork for future research.

%ACM
%\begin{acks}
%This work was partially supported by EU-funded project Sec4AI4Sec (grant no. 101120393)
%\end{acks}

%IEEE
\section*{Acknowledgment}
\ifthenelse{\boolean{peerreview}}
{Omitted for peer review.}
{This work was partially supported by EU-funded project Sec4AI4Sec (grant no. 101120393).}

%\bibliographystyle{ACM-Reference-Format}
\bibliographystyle{ieeetr}
\bibliography{biblio}

\end{document}
\endinput
