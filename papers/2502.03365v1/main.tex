%ACM
%\documentclass[sigconf,review,anonymous]{acmart}
%\acmConference[MSR 2025]{22st International Conference on Mining Software Repositories}{April 2025}{Ottawa, Canada}
%\setcopyright{acmlicensed}
%\copyrightyear{2024}
%\acmYear{2024}
%\acmDOI{XXXXXXX.XXXXXXX}
%\acmISBN{978-1-4503-XXXX-X/24/06}

%IEEE
\documentclass[10pt,conference]{IEEEtran}
\IEEEoverridecommandlockouts

\usepackage{fontawesome5}
\usepackage{xspace}
\usepackage[shortcuts]{extdash}
\usepackage[frozencache,cachedir=.]{minted}
\usepackage{ifthen}
\usepackage{enumitem}
\usepackage{tabularx}
\usepackage{colortbl}
\usepackage{multirow}
\usepackage{hhline}
\usepackage{soul}
\usepackage{xcolor}
\usepackage{url}
\usepackage{graphicx}
\usepackage{textcomp}
\usepackage{cite}
\usepackage{amsmath,amssymb}
\usepackage{hyperref}
%\usepackage{todonotes}
\usepackage{makecell}
%\hyphenation{vul-ne-ra-bi-li-ty-wit-nes-sing}

\begin{document}
%\title{Automated Collection of Vulnerability\-/witnessing Tests for \Java: The \vuteco Approach}
\title{A Match Made in Heaven? Matching Test Cases and Vulnerabilities With the VUTECO Approach}

%IEEE
\author{
\ifthenelse{\boolean{peerreview}}
{\IEEEauthorblockN{Anonymous Author(s)\\Anonymous Affiliation(s)\\Anonymous Email(s)}}
{\IEEEauthorblockN{Emanuele Iannone}
\IEEEauthorblockA{\textit{Institute of Software Security} \\
\textit{Hamburg University of Technology}\\
Hamburg, Germany \\
emanuele.iannone@tuhh.de}
\and
\IEEEauthorblockN{Quang-Cuong Bui}
\IEEEauthorblockA{\textit{Institute of Software Security} \\
\textit{Hamburg University of Technology}\\
Hamburg, Germany \\
cuong.bui@tuhh.de
}
\and
\IEEEauthorblockN{Riccardo Scandariato}
\IEEEauthorblockA{\textit{Institute of Software Security} \\
\textit{Hamburg University of Technology}\\
Hamburg, Germany \\
riccardo.scandariato@tuhh.de}
}
}

%ACM
% \author{Emanuele Iannone}
% \orcid{0000-0001-7489-9969}
% \affiliation{%
%   \institution{Hamburg University of Technology}
%   \city{Hamburg}
%   \country{Germany}
% }
% \email{emanuele.iannone@tuhh.de}

% \author{Quang-Cuong Bui}
% \orcid{0000-0001-6072-9213}
% \affiliation{%
%   \institution{Hamburg University of Technology}
%   \city{Hamburg}
%   \country{Germany}
% }
% \email{cuong.bui@tuhh.de}

% \author{Riccardo Scandariato}
% \orcid{0000-0003-3591-7671}
% \affiliation{%
%   \institution{Hamburg University of Technology}
%   \city{Hamburg}
%   \country{Germany}
% }
% \email{riccardo.scandariato@tuhh.de}

% *** CITATION PACKAGES ***
%
% \ifCLASSOPTIONcompsoc
%   % IEEE Computer Society needs nocompress option
%   % requires cite.sty v4.0 or later (November 2003)
%   \usepackage[nocompress]{cite}
% \else
%   % normal IEEE
%   \usepackage{cite}
% \fi

\usepackage{booktabs}



% *** GRAPHICS RELATED PACKAGES ***
%
% \ifCLASSINFOpdf
%   % \usepackage[pdftex]{graphicx}
%   % declare the path(s) where your graphic files are
%   % \graphicspath{{../pdf/}{../jpeg/}}
%   % and their extensions so you won't have to specify these with
%   % every instance of \includegraphics
%   % \DeclareGraphicsExtensions{.pdf,.jpeg,.png}
% \else
%   % or other class option (dvipsone, dvipdf, if not using dvips). graphicx
%   % will default to the driver specified in the system graphics.cfg if no
%   % driver is specified.
%   % \usepackage[dvips]{graphicx}
%   % declare the path(s) where your graphic files are
%   % \graphicspath{{../eps/}}
%   % and their extensions so you won't have to specify these with
%   % every instance of \includegraphics
%   % \DeclareGraphicsExtensions{.eps}
% \fi
% graphicx was written by David Carlisle and Sebastian Rahtz. It is
% required if you want graphics, photos, etc. graphicx.sty is already
% installed on most LaTeX systems. The latest version and documentation
% can be obtained at: 
% http://www.ctan.org/pkg/graphicx
% Another good source of documentation is "Using Imported Graphics in
% LaTeX2e" by Keith Reckdahl which can be found at:
% http://www.ctan.org/pkg/epslatex
%
% latex, and pdflatex in dvi mode, support graphics in encapsulated
% postscript (.eps) format. pdflatex in pdf mode supports graphics
% in .pdf, .jpeg, .png and .mps (metapost) formats. Users should ensure
% that all non-photo figures use a vector format (.eps, .pdf, .mps) and
% not a bitmapped formats (.jpeg, .png). The IEEE frowns on bitmapped formats
% which can result in "jaggedy"/blurry rendering of lines and letters as
% well as large increases in file sizes.
%
% You can find documentation about the pdfTeX application at:
% http://www.tug.org/applications/pdftex





% *** MATH PACKAGES ***
%
\usepackage{amsmath}
% A popular package from the American Mathematical Society that provides
% many useful and powerful commands for dealing with mathematics.
%
% Note that the amsmath package sets \interdisplaylinepenalty to 10000
% thus preventing page breaks from occurring within multiline equations. Use:
%\interdisplaylinepenalty=2500
% after loading amsmath to restore such page breaks as IEEEtran.cls normally
% does. amsmath.sty is already installed on most LaTeX systems. The latest
% version and documentation can be obtained at:
% http://www.ctan.org/pkg/amsmath





% *** SPECIALIZED LIST PACKAGES ***
%
%\usepackage{algorithmic}
% algorithmic.sty was written by Peter Williams and Rogerio Brito.
% This package provides an algorithmic environment fo describing algorithms.
% You can use the algorithmic environment in-text or within a figure
% environment to provide for a floating algorithm. Do NOT use the algorithm
% floating environment provided by algorithm.sty (by the same authors) or
% algorithm2e.sty (by Christophe Fiorio) as the IEEE does not use dedicated
% algorithm float types and packages that provide these will not provide
% correct IEEE style captions. The latest version and documentation of
% algorithmic.sty can be obtained at:
% http://www.ctan.org/pkg/algorithms
% Also of interest may be the (relatively newer and more customizable)
% algorithmicx.sty package by Szasz Janos:
% http://www.ctan.org/pkg/algorithmicx




% *** ALIGNMENT PACKAGES ***
%
%\usepackage{array}
% Frank Mittelbach's and David Carlisle's array.sty patches and improves
% the standard LaTeX2e array and tabular environments to provide better
% appearance and additional user controls. As the default LaTeX2e table
% generation code is lacking to the point of almost being broken with
% respect to the quality of the end results, all users are strongly
% advised to use an enhanced (at the very least that provided by array.sty)
% set of table tools. array.sty is already installed on most systems. The
% latest version and documentation can be obtained at:
% http://www.ctan.org/pkg/array


% IEEEtran contains the IEEEeqnarray family of commands that can be used to
% generate multiline equations as well as matrices, tables, etc., of high
% quality.




% *** SUBFIGURE PACKAGES ***
%\ifCLASSOPTIONcompsoc
%  \usepackage[caption=false,font=footnotesize,labelfont=sf,textfont=sf]{subfig}
%\else
%  \usepackage[caption=false,font=footnotesize]{subfig}
%\fi
% subfig.sty, written by Steven Douglas Cochran, is the modern replacement
% for subfigure.sty, the latter of which is no longer maintained and is
% incompatible with some LaTeX packages including fixltx2e. However,
% subfig.sty requires and automatically loads Axel Sommerfeldt's caption.sty
% which will override IEEEtran.cls' handling of captions and this will result
% in non-IEEE style figure/table captions. To prevent this problem, be sure
% and invoke subfig.sty's "caption=false" package option (available since
% subfig.sty version 1.3, 2005/06/28) as this is will preserve IEEEtran.cls
% handling of captions.
% Note that the Computer Society format requires a sans serif font rather
% than the serif font used in traditional IEEE formatting and thus the need
% to invoke different subfig.sty package options depending on whether
% compsoc mode has been enabled.
%
% The latest version and documentation of subfig.sty can be obtained at:
% http://www.ctan.org/pkg/subfig





%\usepackage{stfloats}
% stfloats.sty was written by Sigitas Tolusis. This package gives LaTeX2e
% the ability to do double column floats at the bottom of the page as well
% as the top. (e.g., "\begin{figure*}[!b]" is not normally possible in
% LaTeX2e). It also provides a command:
%\fnbelowfloat
% to enable the placement of footnotes below bottom floats (the standard
% LaTeX2e kernel puts them above bottom floats). This is an invasive package
% which rewrites many portions of the LaTeX2e float routines. It may not work
% with other packages that modify the LaTeX2e float routines. The latest
% version and documentation can be obtained at:
% http://www.ctan.org/pkg/stfloats
% Do not use the stfloats baselinefloat ability as the IEEE does not allow
% \baselineskip to stretch. Authors submitting work to the IEEE should note
% that the IEEE rarely uses double column equations and that authors should try
% to avoid such use. Do not be tempted to use the cuted.sty or midfloat.sty
% packages (also by Sigitas Tolusis) as the IEEE does not format its papers in
% such ways.
% Do not attempt to use stfloats with fixltx2e as they are incompatible.
% Instead, use Morten Hogholm'a dblfloatfix which combines the features
% of both fixltx2e and stfloats:
%
% \usepackage{dblfloatfix}
% The latest version can be found at:
% http://www.ctan.org/pkg/dblfloatfix




% *** PDF, URL AND HYPERLINK PACKAGES ***
%
\usepackage{url}
% url.sty was written by Donald Arseneau. It provides better support for
% handling and breaking URLs. url.sty is already installed on most LaTeX
% systems. The latest version and documentation can be obtained at:
% http://www.ctan.org/pkg/url
% Basically, \url{my_url_here}.

\usepackage{xcolor}
\usepackage{cleveref}
\usepackage{graphicx}
\usepackage{subcaption}

%comments
\newcommand{\zijian}[1]{\textcolor{blue}{ZH: #1}}
\newcommand{\jiasi}[1]{\textcolor{red}{JC: #1}}
\newcommand{\yicheng}[1]{\textcolor{green}{YZ: #1}}
\newcommand{\sophie}[1]{\textcolor{purple}{SC: #1}}
% \newcommand{\eg}{\textit{e}.\textit{g}.,~}
%Latin shortcuts
\newcommand{\eg}{\emph{e.g.,} }
\newcommand{\ie}{\emph{i.e.,} }
\newcommand{\etal}{\emph{et al.} }

% correct bad hyphenation here
\hyphenation{op-tical net-works semi-conduc-tor}

%IEEE
\maketitle

%ACM
%\renewcommand{\shortauthors}{Iannone et al.}

\begin{abstract}
Software vulnerabilities are commonly detected via static analysis, penetration testing, and fuzzing.
They can also be found by running unit tests---so-called \textit{vulnerability-witnessing tests}---that stimulate the security-sensitive behavior with crafted inputs.
Developing such tests is difficult and time-consuming; thus, automated data-driven approaches could help developers intercept vulnerabilities earlier.
However, training and validating such approaches require a lot of data, which is currently scarce.

This paper introduces \vuteco, a deep learning-based approach for collecting instances of vulnerability-witnessing tests from \Java repositories.
\vuteco carries out two tasks: (1) the ``\finding'' task to determine whether a test case is security-related, and (2) the ``\matching'' task to relate a test case to the exact vulnerability it is witnessing.
%Using the test case source code and the textual description of vulnerabilities, \vuteco employs a deep learning model based on CodeBERT (1) determine if the test case is security-related and (2) match it to the right vulnerability.
\vuteco successfully addresses the \finding task, achieving perfect precision and 0.83 F0.5 score on validated test cases in \VulforJ and returning 102 out of 145 (70\%) correct security-related test cases from \invivoProjects open-source \Java projects.
Despite showing sufficiently good performance for the \matching task---i.e., 0.86 precision and 0.68 F0.5 score---\vuteco failed to retrieve any valid match in the wild.
Nevertheless, we observed that in almost all of the matches, the test case was still security-related despite being matched to the wrong vulnerability.
%\vuteco successfully addresses the two tasks when evaluated on the tests and vulnerabilities in \VulforJ, achieving \EI{TODO} in the best configurations.
%
%Afterward, \vuteco was employed on \invivoProjects open-source \Java projects and \invivoVulns vulnerabilities taken from \projectKB, matching \TEMP\resultsWitTests witnessing tests to \resultsVulns vulnerabilities.
%A manual inspection of the top suspects confirmed that \vuteco matched the right test for \inspectedVulnsWithOneValidTest vulnerabilities, four times more than a heuristic baseline approach could.
In the end, \vuteco can help find vulnerability-witnessing tests, though the matching with the right vulnerability is yet to be solved; the findings obtained lay the stepping stone for future research on the matter.
\end{abstract}

%ACM
% \begin{CCSXML}
%     <ccs2012>
%     <concept>
%     <concept_id>10011007.10011074.10011099.10011102.10011103</concept_id>
%     <concept_desc>Software and its engineering~Software testing and debugging</concept_desc>
%     <concept_significance>500</concept_significance>
%     </concept>
%     <concept>
%     <concept_id>10011007.10011006.10011072</concept_id>
%     <concept_desc>Software and its engineering~Software libraries and repositories</concept_desc>
%     <concept_significance>500</concept_significance>
%     </concept>
%     <concept>
%     <concept_id>10002978.10003022.10003023</concept_id>
%     <concept_desc>Security and privacy~Software security engineering</concept_desc>
%     <concept_significance>500</concept_significance>
%     </concept>
%     </ccs2012>
% \end{CCSXML}
% \ccsdesc[500]{Software and its engineering~Software testing and debugging}
% \ccsdesc[500]{Software and its engineering~Software libraries and repositories}
% \ccsdesc[500]{Security and privacy~Software security engineering}
% \keywords{Vulnerability Testing, Security Testing, Unit Testing, Mining Software Repositories, Large Language Models, Deep Learning}

%IEEE
\begin{IEEEkeywords}
Mining Software Repositories, Vulnerability-witnessing Tests, Security Testing, Unit Testing, Language Models
\end{IEEEkeywords}

%ACM
%\maketitle

\section{Introduction}


\begin{figure}[t]
\centering
\includegraphics[width=0.6\columnwidth]{figures/evaluation_desiderata_V5.pdf}
\vspace{-0.5cm}
\caption{\systemName is a platform for conducting realistic evaluations of code LLMs, collecting human preferences of coding models with real users, real tasks, and in realistic environments, aimed at addressing the limitations of existing evaluations.
}
\label{fig:motivation}
\end{figure}

\begin{figure*}[t]
\centering
\includegraphics[width=\textwidth]{figures/system_design_v2.png}
\caption{We introduce \systemName, a VSCode extension to collect human preferences of code directly in a developer's IDE. \systemName enables developers to use code completions from various models. The system comprises a) the interface in the user's IDE which presents paired completions to users (left), b) a sampling strategy that picks model pairs to reduce latency (right, top), and c) a prompting scheme that allows diverse LLMs to perform code completions with high fidelity.
Users can select between the top completion (green box) using \texttt{tab} or the bottom completion (blue box) using \texttt{shift+tab}.}
\label{fig:overview}
\end{figure*}

As model capabilities improve, large language models (LLMs) are increasingly integrated into user environments and workflows.
For example, software developers code with AI in integrated developer environments (IDEs)~\citep{peng2023impact}, doctors rely on notes generated through ambient listening~\citep{oberst2024science}, and lawyers consider case evidence identified by electronic discovery systems~\citep{yang2024beyond}.
Increasing deployment of models in productivity tools demands evaluation that more closely reflects real-world circumstances~\citep{hutchinson2022evaluation, saxon2024benchmarks, kapoor2024ai}.
While newer benchmarks and live platforms incorporate human feedback to capture real-world usage, they almost exclusively focus on evaluating LLMs in chat conversations~\citep{zheng2023judging,dubois2023alpacafarm,chiang2024chatbot, kirk2024the}.
Model evaluation must move beyond chat-based interactions and into specialized user environments.



 

In this work, we focus on evaluating LLM-based coding assistants. 
Despite the popularity of these tools---millions of developers use Github Copilot~\citep{Copilot}---existing
evaluations of the coding capabilities of new models exhibit multiple limitations (Figure~\ref{fig:motivation}, bottom).
Traditional ML benchmarks evaluate LLM capabilities by measuring how well a model can complete static, interview-style coding tasks~\citep{chen2021evaluating,austin2021program,jain2024livecodebench, white2024livebench} and lack \emph{real users}. 
User studies recruit real users to evaluate the effectiveness of LLMs as coding assistants, but are often limited to simple programming tasks as opposed to \emph{real tasks}~\citep{vaithilingam2022expectation,ross2023programmer, mozannar2024realhumaneval}.
Recent efforts to collect human feedback such as Chatbot Arena~\citep{chiang2024chatbot} are still removed from a \emph{realistic environment}, resulting in users and data that deviate from typical software development processes.
We introduce \systemName to address these limitations (Figure~\ref{fig:motivation}, top), and we describe our three main contributions below.


\textbf{We deploy \systemName in-the-wild to collect human preferences on code.} 
\systemName is a Visual Studio Code extension, collecting preferences directly in a developer's IDE within their actual workflow (Figure~\ref{fig:overview}).
\systemName provides developers with code completions, akin to the type of support provided by Github Copilot~\citep{Copilot}. 
Over the past 3 months, \systemName has served over~\completions suggestions from 10 state-of-the-art LLMs, 
gathering \sampleCount~votes from \userCount~users.
To collect user preferences,
\systemName presents a novel interface that shows users paired code completions from two different LLMs, which are determined based on a sampling strategy that aims to 
mitigate latency while preserving coverage across model comparisons.
Additionally, we devise a prompting scheme that allows a diverse set of models to perform code completions with high fidelity.
See Section~\ref{sec:system} and Section~\ref{sec:deployment} for details about system design and deployment respectively.



\textbf{We construct a leaderboard of user preferences and find notable differences from existing static benchmarks and human preference leaderboards.}
In general, we observe that smaller models seem to overperform in static benchmarks compared to our leaderboard, while performance among larger models is mixed (Section~\ref{sec:leaderboard_calculation}).
We attribute these differences to the fact that \systemName is exposed to users and tasks that differ drastically from code evaluations in the past. 
Our data spans 103 programming languages and 24 natural languages as well as a variety of real-world applications and code structures, while static benchmarks tend to focus on a specific programming and natural language and task (e.g. coding competition problems).
Additionally, while all of \systemName interactions contain code contexts and the majority involve infilling tasks, a much smaller fraction of Chatbot Arena's coding tasks contain code context, with infilling tasks appearing even more rarely. 
We analyze our data in depth in Section~\ref{subsec:comparison}.



\textbf{We derive new insights into user preferences of code by analyzing \systemName's diverse and distinct data distribution.}
We compare user preferences across different stratifications of input data (e.g., common versus rare languages) and observe which affect observed preferences most (Section~\ref{sec:analysis}).
For example, while user preferences stay relatively consistent across various programming languages, they differ drastically between different task categories (e.g. frontend/backend versus algorithm design).
We also observe variations in user preference due to different features related to code structure 
(e.g., context length and completion patterns).
We open-source \systemName and release a curated subset of code contexts.
Altogether, our results highlight the necessity of model evaluation in realistic and domain-specific settings.





%\section{Bellman Error Centering}

Centering operator $\mathcal{C}$ for a variable $x(s)$ is defined as follows:
\begin{equation}
\mathcal{C}x(s)\dot{=} x(s)-\mathbb{E}[x(s)]=x(s)-\sum_s{d_{s}x(s)},
\end{equation} 
where $d_s$ is the probability of $s$.
In vector form,
\begin{equation}
\begin{split}
\mathcal{C}\bm{x} &= \bm{x}-\mathbb{E}[x]\bm{1}\\
&=\bm{x}-\bm{x}^{\top}\bm{d}\bm{1},
\end{split}
\end{equation} 
where $\bm{1}$ is an all-ones vector.
For any vector $\bm{x}$ and $\bm{y}$ with a same distribution $\bm{d}$,
we have
\begin{equation}
\begin{split}
\mathcal{C}(\bm{x}+\bm{y})&=(\bm{x}+\bm{y})-(\bm{x}+\bm{y})^{\top}\bm{d}\bm{1}\\
&=\bm{x}-\bm{x}^{\top}\bm{d}\bm{1}+\bm{y}-\bm{y}^{\top}\bm{d}\bm{1}\\
&=\mathcal{C}\bm{x}+\mathcal{C}\bm{y}.
\end{split}
\end{equation}
\subsection{Revisit Reward Centering}


The update (\ref{src3}) is an unbiased estimate of the average reward
with  appropriate learning rate $\beta_t$ conditions.
\begin{equation}
\bar{r}_{t}\approx \lim_{n\rightarrow\infty}\frac{1}{n}\sum_{t=1}^n\mathbb{E}_{\pi}[r_t].
\end{equation}
That is 
\begin{equation}
r_t-\bar{r}_{t}\approx r_t-\lim_{n\rightarrow\infty}\frac{1}{n}\sum_{t=1}^n\mathbb{E}_{\pi}[r_t]= \mathcal{C}r_t.
\end{equation}
Then, the simple reward centering can be rewrited as:
\begin{equation}
V_{t+1}(s_t)=V_{t}(s_t)+\alpha_t [\mathcal{C}r_{t+1}+\gamma V_{t}(s_{t+1})-V_t(s_t)].
\end{equation}
Therefore, the simple reward centering is, in a strict sense, reward centering.

By definition of $\bar{\delta}_t=\delta_t-\bar{r}_{t}$,
let rewrite the update rule of the value-based reward centering as follows:
\begin{equation}
V_{t+1}(s_t)=V_{t}(s_t)+\alpha_t \rho_t (\delta_t-\bar{r}_{t}),
\end{equation}
where $\bar{r}_{t}$ is updated as:
\begin{equation}
\bar{r}_{t+1}=\bar{r}_{t}+\beta_t \rho_t(\delta_t-\bar{r}_{t}).
\label{vrc3}
\end{equation}
The update (\ref{vrc3}) is an unbiased estimate of the TD error
with  appropriate learning rate $\beta_t$ conditions.
\begin{equation}
\bar{r}_{t}\approx \mathbb{E}_{\pi}[\delta_t].
\end{equation}
That is 
\begin{equation}
\delta_t-\bar{r}_{t}\approx \mathcal{C}\delta_t.
\end{equation}
Then, the value-based reward centering can be rewrited as:
\begin{equation}
V_{t+1}(s_t)=V_{t}(s_t)+\alpha_t \rho_t \mathcal{C}\delta_t.
\label{tdcentering}
\end{equation}
Therefore, the value-based reward centering is no more,
 in a strict sense, reward centering.
It is, in a strict sense, \textbf{Bellman error centering}.

It is worth noting that this understanding is crucial, 
as designing new algorithms requires leveraging this concept.


\subsection{On the Fixpoint Solution}

The update rule (\ref{tdcentering}) is a stochastic approximation
of the following update:
\begin{equation}
\begin{split}
V_{t+1}&=V_{t}+\alpha_t [\bm{\mathcal{T}}^{\pi}\bm{V}-\bm{V}-\mathbb{E}[\delta]\bm{1}]\\
&=V_{t}+\alpha_t [\bm{\mathcal{T}}^{\pi}\bm{V}-\bm{V}-(\bm{\mathcal{T}}^{\pi}\bm{V}-\bm{V})^{\top}\bm{d}_{\pi}\bm{1}]\\
&=V_{t}+\alpha_t [\mathcal{C}(\bm{\mathcal{T}}^{\pi}\bm{V}-\bm{V})].
\end{split}
\label{tdcenteringVector}
\end{equation}
If update rule (\ref{tdcenteringVector}) converges, it is expected that
$\mathcal{C}(\mathcal{T}^{\pi}V-V)=\bm{0}$.
That is 
\begin{equation}
    \begin{split}
    \mathcal{C}\bm{V} &= \mathcal{C}\bm{\mathcal{T}}^{\pi}\bm{V} \\
    &= \mathcal{C}(\bm{R}^{\pi} + \gamma \mathbb{P}^{\pi} \bm{V}) \\
    &= \mathcal{C}\bm{R}^{\pi} + \gamma \mathcal{C}\mathbb{P}^{\pi} \bm{V} \\
    &= \mathcal{C}\bm{R}^{\pi} + \gamma (\mathbb{P}^{\pi} \bm{V} - (\mathbb{P}^{\pi} \bm{V})^{\top} \bm{d_{\pi}} \bm{1}) \\
    &= \mathcal{C}\bm{R}^{\pi} + \gamma (\mathbb{P}^{\pi} \bm{V} - \bm{V}^{\top} (\mathbb{P}^{\pi})^{\top} \bm{d_{\pi}} \bm{1}) \\  % 修正双重上标
    &= \mathcal{C}\bm{R}^{\pi} + \gamma (\mathbb{P}^{\pi} \bm{V} - \bm{V}^{\top} \bm{d_{\pi}} \bm{1}) \\
    &= \mathcal{C}\bm{R}^{\pi} + \gamma (\mathbb{P}^{\pi} \bm{V} - \bm{V}^{\top} \bm{d_{\pi}} \mathbb{P}^{\pi} \bm{1}) \\
    &= \mathcal{C}\bm{R}^{\pi} + \gamma (\mathbb{P}^{\pi} \bm{V} - \mathbb{P}^{\pi} \bm{V}^{\top} \bm{d_{\pi}} \bm{1}) \\
    &= \mathcal{C}\bm{R}^{\pi} + \gamma \mathbb{P}^{\pi} (\bm{V} - \bm{V}^{\top} \bm{d_{\pi}} \bm{1}) \\
    &= \mathcal{C}\bm{R}^{\pi} + \gamma \mathbb{P}^{\pi} \mathcal{C}\bm{V} \\
    &\dot{=} \bm{\mathcal{T}}_c^{\pi} \mathcal{C}\bm{V},
    \end{split}
    \label{centeredfixpoint}
    \end{equation}
where we defined $\bm{\mathcal{T}}_c^{\pi}$ as a centered Bellman operator.
We call equation (\ref{centeredfixpoint}) as centered Bellman equation.
And it is \textbf{centered fixpoint}.

For linear value function approximation, let define
\begin{equation}
\mathcal{C}\bm{V}_{\bm{\theta}}=\bm{\Pi}\bm{\mathcal{T}}_c^{\pi}\mathcal{C}\bm{V}_{\bm{\theta}}.
\label{centeredTDfixpoint}
\end{equation}
We call equation (\ref{centeredTDfixpoint}) as \textbf{centered TD fixpoint}.

\subsection{On-policy and Off-policy Centered TD Algorithms
with Linear Value Function Approximation}
Given the above centered TD fixpoint,
 mean squared centered Bellman error (MSCBE), is proposed as follows:
\begin{align*}
    \label{argminMSBEC}
 &\arg \min_{{\bm{\theta}}}\text{MSCBE}({\bm{\theta}}) \\
 &= \arg \min_{{\bm{\theta}}} \|\bm{\mathcal{T}}_c^{\pi}\mathcal{C}\bm{V}_{\bm{{\bm{\theta}}}}-\mathcal{C}\bm{V}_{\bm{{\bm{\theta}}}}\|_{\bm{D}}^2\notag\\
 &=\arg \min_{{\bm{\theta}}} \|\bm{\mathcal{T}}^{\pi}\bm{V}_{\bm{{\bm{\theta}}}} - \bm{V}_{\bm{{\bm{\theta}}}}-(\bm{\mathcal{T}}^{\pi}\bm{V}_{\bm{{\bm{\theta}}}} - \bm{V}_{\bm{{\bm{\theta}}}})^{\top}\bm{d}\bm{1}\|_{\bm{D}}^2\notag\\
 &=\arg \min_{{\bm{\theta}},\omega} \| \bm{\mathcal{T}}^{\pi}\bm{V}_{\bm{{\bm{\theta}}}} - \bm{V}_{\bm{{\bm{\theta}}}}-\omega\bm{1} \|_{\bm{D}}^2\notag,
\end{align*}
where $\omega$ is is used to estimate the expected value of the Bellman error.
% where $\omega$ is used to estimate $\mathbb{E}[\delta]$, $\omega \doteq \mathbb{E}[\mathbb{E}[\delta_t|S_t]]=\mathbb{E}[\delta]$ and $\delta_t$ is the TD error as follows:
% \begin{equation}
% \delta_t = r_{t+1}+\gamma
% {\bm{\theta}}_t^{\top}\bm{{\bm{\phi}}}_{t+1}-{\bm{\theta}}_t^{\top}\bm{{\bm{\phi}}}_t.
% \label{delta}
% \end{equation}
% $\mathbb{E}[\delta_t|S_t]$ is the Bellman error, and $\mathbb{E}[\mathbb{E}[\delta_t|S_t]]$ represents the expected value of the Bellman error.
% If $X$ is a random variable and $\mathbb{E}[X]$ is its expected value, then $X-\mathbb{E}[X]$ represents the centered form of $X$. 
% Therefore, we refer to $\mathbb{E}[\delta_t|S_t]-\mathbb{E}[\mathbb{E}[\delta_t|S_t]]$ as Bellman error centering and 
% $\mathbb{E}[(\mathbb{E}[\delta_t|S_t]-\mathbb{E}[\mathbb{E}[\delta_t|S_t]])^2]$ represents the the mean squared centered Bellman error, namely MSCBE.
% The meaning of (\ref{argminMSBEC}) is to minimize the mean squared centered Bellman error.
%The derivation of CTD is as follows.

First, the parameter  $\omega$ is derived directly based on
stochastic gradient descent:
\begin{equation}
\omega_{t+1}= \omega_{t}+\beta_t(\delta_t-\omega_t).
\label{omega}
\end{equation}

Then, based on stochastic semi-gradient descent, the update of 
the parameter ${\bm{\theta}}$ is as follows:
\begin{equation}
{\bm{\theta}}_{t+1}=
{\bm{\theta}}_{t}+\alpha_t(\delta_t-\omega_t)\bm{{\bm{\phi}}}_t.
\label{theta}
\end{equation}

We call (\ref{omega}) and (\ref{theta}) the on-policy centered
TD (CTD) algorithm. The convergence analysis with be given in
the following section.

In off-policy learning, we can simply multiply by the importance sampling
 $\rho$.
\begin{equation}
    \omega_{t+1}=\omega_{t}+\beta_t\rho_t(\delta_t-\omega_t),
    \label{omegawithrho}
\end{equation}
\begin{equation}
    {\bm{\theta}}_{t+1}=
    {\bm{\theta}}_{t}+\alpha_t\rho_t(\delta_t-\omega_t)\bm{{\bm{\phi}}}_t.
    \label{thetawithrho}
\end{equation}

We call (\ref{omegawithrho}) and (\ref{thetawithrho}) the off-policy centered
TD (CTD) algorithm.

% By substituting $\delta_t$ into Equations (\ref{omegawithrho}) and (\ref{thetawithrho}), 
% we can see that Equations (\ref{thetawithrho}) and (\ref{omegawithrho}) are formally identical 
% to the linear expressions of Equations (\ref{rewardcentering1}) and (\ref{rewardcentering2}), respectively. However, the meanings 
% of the corresponding parameters are entirely different.
% ${\bm{\theta}}_t$ is for approximating the discounted value function.
% $\bar{r_t}$ is an estimate of the average reward, while $\omega_t$ 
% is an estimate of the expected value of the Bellman error.
% $\bar{\delta_t}$ is the TD error for value-based reward centering, 
% whereas $\delta_t$ is the traditional TD error.

% This study posits that the CTD is equivalent to value-based reward 
% centering. However, CTD can be unified under a single framework 
% through an objective function, MSCBE, which also lays the 
% foundation for proving the algorithm's convergence. 
% Section 4 demonstrates that the CTD algorithm guarantees 
% convergence in the on-policy setting.

\subsection{Off-policy Centered TDC Algorithm with Linear Value Function Approximation}
The convergence of the  off-policy centered TD algorithm
may not be guaranteed.

To deal with this problem, we propose another new objective function, 
called mean squared projected centered Bellman error (MSPCBE), 
and derive Centered TDC algorithm (CTDC).

% We first establish some relationships between
%  the vector-matrix quantities and the relevant statistical expectation terms:
% \begin{align*}
%     &\mathbb{E}[(\delta({\bm{\theta}})-\mathbb{E}[\delta({\bm{\theta}})]){\bm{\phi}}] \\
%     &= \sum_s \mu(s) {\bm{\phi}}(s) \big( R(s) + \gamma \sum_{s'} P_{ss'} V_{\bm{\theta}}(s') - V_{\bm{\theta}}(s)  \\
%     &\quad \quad-\sum_s \mu(s)(R(s) + \gamma \sum_{s'} P_{ss'} V_{\bm{\theta}}(s') - V_{\bm{\theta}}(s))\big)\\
%     &= \bm{\Phi}^\top \mathbf{D} (\bm{TV}_{\bm{{\bm{\theta}}}} - \bm{V}_{\bm{{\bm{\theta}}}}-\omega\bm{1}),
% \end{align*}
% where $\omega$ is the expected value of the Bellman error and $\bm{1}$ is all-ones vector.

The specific expression of the objective function 
MSPCBE is as follows:
\begin{align}
    \label{MSPBECwithomega}
    &\arg \min_{{\bm{\theta}}}\text{MSPCBE}({\bm{\theta}})\notag\\ 
    % &= \arg \min_{{\bm{\theta}}}\big(\mathbb{E}[(\delta({\bm{\theta}}) - \mathbb{E}[\delta({\bm{\theta}})]) \bm{{\bm{\phi}}}]^\top \notag\\
    % &\quad \quad \quad\mathbb{E}[\bm{{\bm{\phi}}} \bm{{\bm{\phi}}}^\top]^{-1} \mathbb{E}[(\delta({\bm{\theta}}) - \mathbb{E}[\delta({\bm{\theta}})]) \bm{{\bm{\phi}}}]\big) \notag\\
    % &=\arg \min_{{\bm{\theta}},\omega}\mathbb{E}[(\delta({\bm{\theta}})-\omega) \bm{\bm{{\bm{\phi}}}}]^{\top} \mathbb{E}[\bm{\bm{{\bm{\phi}}}} \bm{\bm{{\bm{\phi}}}}^{\top}]^{-1}\mathbb{E}[(\delta({\bm{\theta}}) -\omega)\bm{\bm{{\bm{\phi}}}}]\\
    % &= \big(\bm{\Phi}^\top \mathbf{D} (\bm{TV}_{\bm{{\bm{\theta}}}} - \bm{V}_{\bm{{\bm{\theta}}}}-\omega\bm{1})\big)^\top (\bm{\Phi}^\top \mathbf{D} \bm{\Phi})^{-1} \notag\\
    % & \quad \quad \quad \bm{\Phi}^\top \mathbf{D} (\bm{TV}_{\bm{{\bm{\theta}}}} - \bm{V}_{\bm{{\bm{\theta}}}}-\omega\bm{1}) \notag\\
    % &= (\bm{TV}_{\bm{{\bm{\theta}}}} - \bm{V}_{\bm{{\bm{\theta}}}}-\omega\bm{1})^\top \mathbf{D} \bm{\Phi} (\bm{\Phi}^\top \mathbf{D} \bm{\Phi})^{-1} \notag\\
    % &\quad \quad \quad \bm{\Phi}^\top \mathbf{D} (\bm{TV}_{\bm{{\bm{\theta}}}} - \bm{V}_{\bm{{\bm{\theta}}}}-\omega\bm{1})\notag\\
    % &= (\bm{TV}_{\bm{{\bm{\theta}}}} - \bm{V}_{\bm{{\bm{\theta}}}}-\omega\bm{1})^\top {\bm{\Pi}}^\top \mathbf{D} {\bm{\Pi}} (\bm{TV}_{\bm{{\bm{\theta}}}} - \bm{V}_{\bm{{\bm{\theta}}}}-\omega\bm{1}) \notag\\
    &= \arg \min_{{\bm{\theta}}} \|\bm{\Pi}\bm{\mathcal{T}}_c^{\pi}\mathcal{C}\bm{V}_{\bm{{\bm{\theta}}}}-\mathcal{C}\bm{V}_{\bm{{\bm{\theta}}}}\|_{\bm{D}}^2\notag\\
    &= \arg \min_{{\bm{\theta}}} \|\bm{\Pi}(\bm{\mathcal{T}}_c^{\pi}\mathcal{C}\bm{V}_{\bm{{\bm{\theta}}}}-\mathcal{C}\bm{V}_{\bm{{\bm{\theta}}}})\|_{\bm{D}}^2\notag\\
    &= \arg \min_{{\bm{\theta}},\omega}\| {\bm{\Pi}} (\bm{\mathcal{T}}^{\pi}\bm{V}_{\bm{{\bm{\theta}}}} - \bm{V}_{\bm{{\bm{\theta}}}}-\omega\bm{1}) \|_{\bm{D}}^2\notag.
\end{align}
In the process of computing the gradient of the MSPCBE with respect to ${\bm{\theta}}$, 
$\omega$ is treated as a constant.
So, the derivation process of CTDC is the same 
as for the TDC algorithm \cite{sutton2009fast}, the only difference is that the original $\delta$ is replaced by $\delta-\omega$.
Therefore, the updated formulas of the centered TDC  algorithm are as follows:
\begin{equation}
 \bm{{\bm{\theta}}}_{k+1}=\bm{{\bm{\theta}}}_{k}+\alpha_{k}[(\delta_{k}- \omega_k) \bm{\bm{{\bm{\phi}}}}_k\\
 - \gamma\bm{\bm{{\bm{\phi}}}}_{k+1}(\bm{\bm{{\bm{\phi}}}}^{\top}_k \bm{u}_{k})],
\label{thetavmtdc}
\end{equation}
\begin{equation}
 \bm{u}_{k+1}= \bm{u}_{k}+\zeta_{k}[\delta_{k}-\omega_k - \bm{\bm{{\bm{\phi}}}}^{\top}_k \bm{u}_{k}]\bm{\bm{{\bm{\phi}}}}_k,
\label{uvmtdc}
\end{equation}
and
\begin{equation}
 \omega_{k+1}= \omega_{k}+\beta_k (\delta_k- \omega_k).
 \label{omegavmtdc}
\end{equation}
This algorithm is derived to work 
with a given set of sub-samples—in the form of 
triples $(S_k, R_k, S'_k)$ that match transitions 
from both the behavior and target policies. 

% \subsection{Variance Minimization ETD Learning: VMETD}
% Based on the off-policy TD algorithm, a scalar, $F$,  
% is introduced to obtain the ETD algorithm, 
% which ensures convergence under off-policy 
% conditions. This paper further introduces a scalar, 
% $\omega$, based on the ETD algorithm to obtain VMETD.
% VMETD by the following update:
% \begin{equation}
% \label{fvmetd}
%  F_t \leftarrow \gamma \rho_{t-1}F_{t-1}+1,
% \end{equation}
% \begin{equation}
%  \label{thetavmetd}
%  {{\bm{\theta}}}_{t+1}\leftarrow {{\bm{\theta}}}_t+\alpha_t (F_t \rho_t\delta_t - \omega_{t}){\bm{{\bm{\phi}}}}_t,
% \end{equation}
% \begin{equation}
%  \label{omegavmetd}
%  \omega_{t+1} \leftarrow \omega_t+\beta_t(F_t  \rho_t \delta_t - \omega_t),
% \end{equation}
% where $\rho_t =\frac{\pi(A_t | S_t)}{\mu(A_t | S_t)}$ and $\omega$ is used to estimate $\mathbb{E}[F \rho\delta]$, i.e., $\omega \doteq \mathbb{E}[F \rho\delta]$.

% (\ref{thetavmetd}) can be rewritten as
% \begin{equation*}
%  \begin{array}{ccl}
%  {{\bm{\theta}}}_{t+1}&\leftarrow& {{\bm{\theta}}}_t+\alpha_t (F_t \rho_t\delta_t - \omega_t){\bm{{\bm{\phi}}}}_t -\alpha_t \omega_{t+1}{\bm{{\bm{\phi}}}}_t\\
%   &=&{{\bm{\theta}}}_{t}+\alpha_t(F_t\rho_t\delta_t-\mathbb{E}_{\mu}[F_t\rho_t\delta_t|{{\bm{\theta}}}_t]){\bm{{\bm{\phi}}}}_t\\
%  &=&{{\bm{\theta}}}_t+\alpha_t F_t \rho_t (r_{t+1}+\gamma {{\bm{\theta}}}_t^{\top}{\bm{{\bm{\phi}}}}_{t+1}-{{\bm{\theta}}}_t^{\top}{\bm{{\bm{\phi}}}}_t){\bm{{\bm{\phi}}}}_t\\
%  & & \hspace{2em} -\alpha_t \mathbb{E}_{\mu}[F_t \rho_t \delta_t]{\bm{{\bm{\phi}}}}_t\\
%  &=& {{\bm{\theta}}}_t+\alpha_t \{\underbrace{(F_t\rho_tr_{t+1}-\mathbb{E}_{\mu}[F_t\rho_t r_{t+1}]){\bm{{\bm{\phi}}}}_t}_{{b}_{\text{VMETD},t}}\\
%  &&\hspace{-7em}- \underbrace{(F_t\rho_t{\bm{{\bm{\phi}}}}_t({\bm{{\bm{\phi}}}}_t-\gamma{\bm{{\bm{\phi}}}}_{t+1})^{\top}-{\bm{{\bm{\phi}}}}_t\mathbb{E}_{\mu}[F_t\rho_t ({\bm{{\bm{\phi}}}}_t-\gamma{\bm{{\bm{\phi}}}}_{t+1})]^{\top})}_{\textbf{A}_{\text{VMETD},t}}{{\bm{\theta}}}_t\}.
%  \end{array}
% \end{equation*}
% Therefore, 
% \begin{equation*}
%  \begin{array}{ccl}
%   &&\textbf{A}_{\text{VMETD}}\\
%   &=&\lim_{t \rightarrow \infty} \mathbb{E}[\textbf{A}_{\text{VMETD},t}]\\
%   &=& \lim_{t \rightarrow \infty} \mathbb{E}_{\mu}[F_t \rho_t {\bm{{\bm{\phi}}}}_t ({\bm{{\bm{\phi}}}}_t - \gamma {\bm{{\bm{\phi}}}}_{t+1})^{\top}]\\  
%   &&\hspace{1em}- \lim_{t\rightarrow \infty} \mathbb{E}_{\mu}[  {\bm{{\bm{\phi}}}}_t]\mathbb{E}_{\mu}[F_t \rho_t ({\bm{{\bm{\phi}}}}_t - \gamma {\bm{{\bm{\phi}}}}_{t+1})]^{\top}\\
%   &=& \lim_{t \rightarrow \infty} \mathbb{E}_{\mu}[{\bm{{\bm{\phi}}}}_tF_t \rho_t ({\bm{{\bm{\phi}}}}_t - \gamma {\bm{{\bm{\phi}}}}_{t+1})^{\top}]\\   
%   &&\hspace{1em}-\lim_{t \rightarrow \infty} \mathbb{E}_{\mu}[ {\bm{{\bm{\phi}}}}_t]\lim_{t \rightarrow \infty}\mathbb{E}_{\mu}[F_t \rho_t ({\bm{{\bm{\phi}}}}_t - \gamma {\bm{{\bm{\phi}}}}_{t+1})]^{\top}\\
%   && \hspace{-2em}=\sum_{s} d_{\mu}(s)\lim_{t \rightarrow \infty}\mathbb{E}_{\mu}[F_t|S_t = s]\mathbb{E}_{\mu}[\rho_t\bm{{\bm{\phi}}}_t(\bm{{\bm{\phi}}}_t - \gamma \bm{{\bm{\phi}}}_{t+1})^{\top}|S_t= s]\\   
%   &&\hspace{1em}-\sum_{s} d_{\mu}(s)\bm{{\bm{\phi}}}(s)\sum_{s} d_{\mu}(s)\lim_{t \rightarrow \infty}\mathbb{E}_{\mu}[F_t|S_t = s]\\
%   &&\hspace{7em}\mathbb{E}_{\mu}[\rho_t(\bm{{\bm{\phi}}}_t - \gamma \bm{{\bm{\phi}}}_{t+1})^{\top}|S_t = s]\\
%   &=& \sum_{s} f(s)\mathbb{E}_{\pi}[\bm{{\bm{\phi}}}_t(\bm{{\bm{\phi}}}_t- \gamma \bm{{\bm{\phi}}}_{t+1})^{\top}|S_t = s]\\   
%   &&\hspace{1em}-\sum_{s} d_{\mu}(s)\bm{{\bm{\phi}}}(s)\sum_{s} f(s)\mathbb{E}_{\pi}[(\bm{{\bm{\phi}}}_t- \gamma \bm{{\bm{\phi}}}_{t+1})^{\top}|S_t = s]\\
%   &=&\sum_{s} f(s) \bm{\bm{{\bm{\phi}}}}(s)(\bm{\bm{{\bm{\phi}}}}(s) - \gamma \sum_{s'}[\textbf{P}_{\pi}]_{ss'}\bm{\bm{{\bm{\phi}}}}(s'))^{\top}  \\
%   &&-\sum_{s} d_{\mu}(s) {\bm{{\bm{\phi}}}}(s) * \sum_{s} f(s)({\bm{{\bm{\phi}}}}(s) - \gamma \sum_{s'}[\textbf{P}_{\pi}]_{ss'}{\bm{{\bm{\phi}}}}(s'))^{\top}\\
%   &=&{\bm{\bm{\Phi}}}^{\top} \textbf{F} (\textbf{I} - \gamma \textbf{P}_{\pi}) \bm{\bm{\Phi}} - {\bm{\bm{\Phi}}}^{\top} {d}_{\mu} {f}^{\top} (\textbf{I} - \gamma \textbf{P}_{\pi}) \bm{\bm{\Phi}}  \\
%   &=&{\bm{\bm{\Phi}}}^{\top} (\textbf{F} - {d}_{\mu} {f}^{\top}) (\textbf{I} - \gamma \textbf{P}_{\pi}){\bm{\bm{\Phi}}} \\
%   &=&{\bm{\bm{\Phi}}}^{\top} (\textbf{F} (\textbf{I} - \gamma \textbf{P}_{\pi})-{d}_{\mu} {f}^{\top} (\textbf{I} - \gamma \textbf{P}_{\pi})){\bm{\bm{\Phi}}} \\
%   &=&{\bm{\bm{\Phi}}}^{\top} (\textbf{F} (\textbf{I} - \gamma \textbf{P}_{\pi})-{d}_{\mu} {d}_{\mu}^{\top} ){\bm{\bm{\Phi}}},
%  \end{array}
% \end{equation*}
% \begin{equation*}
%  \begin{array}{ccl}
%   &&{b}_{\text{VMETD}}\\
%   &=&\lim_{t \rightarrow \infty} \mathbb{E}[{b}_{\text{VMETD},t}]\\
%   &=& \lim_{t \rightarrow \infty} \mathbb{E}_{\mu}[F_t\rho_tR_{t+1}{\bm{{\bm{\phi}}}}_t]\\
%   &&\hspace{2em} - \lim_{t\rightarrow \infty} \mathbb{E}_{\mu}[{\bm{{\bm{\phi}}}}_t]\mathbb{E}_{\mu}[F_t\rho_kR_{k+1}]\\  
%   &=& \lim_{t \rightarrow \infty} \mathbb{E}_{\mu}[{\bm{{\bm{\phi}}}}_tF_t\rho_tr_{t+1}]\\
%   &&\hspace{2em} - \lim_{t\rightarrow \infty} \mathbb{E}_{\mu}[  {\bm{{\bm{\phi}}}}_t]\mathbb{E}_{\mu}[{\bm{{\bm{\phi}}}}_t]\mathbb{E}_{\mu}[F_t\rho_tr_{t+1}]\\ 
%   &=& \lim_{t \rightarrow \infty} \mathbb{E}_{\mu}[{\bm{{\bm{\phi}}}}_tF_t\rho_tr_{t+1}]\\
%   &&\hspace{2em} - \lim_{t \rightarrow \infty} \mathbb{E}_{\mu}[ {\bm{{\bm{\phi}}}}_t]\lim_{t \rightarrow \infty}\mathbb{E}_{\mu}[F_t\rho_tr_{t+1}]\\  
%   &=&\sum_{s} f(s) {\bm{{\bm{\phi}}}}(s)r_{\pi} - \sum_{s} d_{\mu}(s) {\bm{{\bm{\phi}}}}(s) * \sum_{s} f(s)r_{\pi}  \\
%   &=&\bm{\bm{\bm{\Phi}}}^{\top}(\textbf{F}-{d}_{\mu} {f}^{\top}){r}_{\pi}.
%  \end{array}
% \end{equation*}


\section{The \vuteco Approach}
\label{sec:vuteco}

\subsection{Overview}
\label{subsec:vuteco_overview}

\vuteco faces two distinct tasks.
The \finding task accepts a \JUnit test case as input and tells whether it is security-related (i.e., \finderPosClass) or not (i.e., \finderNegClass).
The \matching task accepts a \JUnit test case and a description (in natural language) of a known historical vulnerability (related to the project) and tells whether the test case is witnessing that vulnerability (i.e., \globalPosClass) or not (i.e., \globalNegClass).
%The main downstream task of \vuteco consists of (1) inspecting \JUnit test suites, (2) finding candidate vulnerability-witnessing test cases, and (3) matching them with the right vulnerability.

The \vuteco approach has been wrapped into a tool (having the same name, \vuteco) that automates both tasks.
The tool accepts (i) a \Java project repository, (ii) the project revision (i.e., commit) to inspect, and (iii) a list of descriptions (in natural language) of known historical vulnerabilities related to the project---taken from CVE (Common Vulnerabilities and Exposures).
Any \textsc{Git}-based repository is valid as long as there are \JUnit test cases to process.
%; the only requirement is that it has to describe what the vulnerability consists of via natural language.
%\CB{Finding vulnerability patches~\cite{zhou2017automated, nguyen2023multi}, identifying vulnerable software versions (V-SZZ)~\cite{bao2022v}, vulnerability assessment~\cite{le2019automated}}

Figure~\ref{fig:vuteco-overview} depicts the general functioning of \vuteco.
Once checking out the project repository to the selected revision,
%\vuteco extracts all the valid \JUnit test methods adopting a heuristic approach. Namely,
\vuteco parses all \Java files and marks any class method having the following properties as a test case:
\begin{enumerate}[leftmargin=*]
    \item it is annotated with \texttt{@Test} (\JUnit 4 and 5) or its class extends \textsc{TestCase} or any subclass of it (for \JUnit 3);
    \item it is not overriding a method defined in class \textsc{TestCase}, like \texttt{run()} or \texttt{getName()} (for \JUnit 3);
    \item it is not a ``lifecycle method'', i.e., annotated with \texttt{@BeforeAll}, \texttt{@AfterAll}, \texttt{@BeforeEach}, or \texttt{@AfterEach};
    \item it returns \texttt{void} if not annotated with \texttt{@TestFactory};
    \item it is not \texttt{abstract}, \texttt{static} or \texttt{private};
    \item its class is not \texttt{abstract}.
\end{enumerate}
Such properties have been designed according to how the \JUnit guide describes a test case~\cite{junit:guide} and the official \textsc{Javadoc} of \JUnit beyond version~3.

Each mined test case is sent to the \finder model, which is responsible for implementing the \finding task (i.e., determining whether the test case is security-related).
Then, each test flagged as security-related is sent to the \linker model along with the descriptions of all vulnerabilities supplied via input to determine which of them is witnessed by the test case.
%This joint use of \finder and \linker models implements the \matching task (the reason for using two models rather than one is motivated by the results of the in-vitro experimentation, which are presented in Section~\ref{sec:results}).
%Each candidate test is then paired with all vulnerability descriptions supplied via input and sent to the second model, i.e., the \linker, whose goal is to assess if the candidate is witnessing the vulnerability described by the short text.
%Ultimately, \vuteco returns all the test cases that were successfully linked with a vulnerability.
Sections~\ref{subsec:finding} and~\ref{subsec:matching} describe how the \finding and \matching tasks have been carried out.
The design choices behind the models used there are supported by the experimentation described in Section~\ref{subsec:invitro}.
%and the related results in Section~\ref{subsec:fnd-results} and~\ref{subsec:match-results}.

\begin{figure}[t]
    \centering
    \resizebox{.98\linewidth}{!}{
        \includegraphics{figures/vuteco-overview.pdf}
    }
    \caption{Graphical overview of the functioning of \vuteco.}
    \label{fig:vuteco-overview}
\end{figure}

%\subsection{Technical Detail}

\subsection{The \finding Task}
\label{subsec:finding}

The \finding task consists of determining if a test case is security-related, i.e., it is focused on some security properties of the project where it belongs.
The \finder model addresses this task, trained to classify test cases into two classes, i.e., \finderPosClass (the positive class) and \finderNegClass (the negative class).
%
The upper part of Figure~\ref{fig:vuteco-detail} depicts the architecture of the \finder model, which consists of a deep neural network built on top of a pre-trained CodeBERT~\cite{feng:emnlp2020:codebert}.
The pre-trained CodeBERT starts from the checkpoint \texttt{microsoft/codebert-base} downloaded from \textsc{HuggingFace}~\cite{codebert:base:hf}.
%Essentially, it is a specialized RoBERTa model~\cite{liu2019roberta} made of 125M trainable weights.
CodeBERT has been pre-trained on
%with masked language modeling (MLM) and replaced token detection (RTD) self-supervised tasks for
code examples of six programming languages, including \Java, paired with natural language text.
Hence,
%leveraging the pre-training made from its base RoBERTa model~\cite{liu2019roberta} and how CodeBERT has been trained,
we deemed it suitable for understanding the content of \JUnit test methods.

As a preliminary action, \vuteco transforms the input test case by removing new line characters and consecutive white space characters (including tabs), reducing the whole method into a single line.
Then, the resulting string is tokenized using the WordPiece algorithm~\cite{wu2016googles:wordpiece} (whose vocabulary was fitted during the pre-training of CodeBERT), and each resulting token is replaced with a numeric identifier.
%Such a step is mandatory as Transformer-based models can only interpret sentences as numerical vectors.
Then, the resulting numeric vector is sent to the CodeBERT input layer (supporting up to 512 encoded tokens), which returns an embedded representation of each token.
Besides, due to the underlying BERT-based architecture, an additional embedding for the special token \texttt{[CLS]} is returned, which has the goal of capturing the whole syntax and semantics of the entire sentence (here, the whole test case), making it suitable for sentence classification tasks.
Considering the goal of the \finding task, we only selected the embedded representation of \texttt{[CLS]}, made of 768 values.
%
On top of this, we added a deep neural network with an input layer of 768 neurons, matching the size of the sentence embedding resulting from the CodeBERT model; then, we added two linear hidden layers with 768 and 64 output neurons, respectively, calling GELU activation function~\cite{hendrycks2023gaussian:gelu}.
After this, the final layer returns the two logits needed for the final classification, which is done through the Softmax function to return the probabilities for the two classes.

\subsection{The \matching Task}
\label{subsec:matching}

\begin{figure}[t]
    \centering
    \resizebox{.98\linewidth}{!}{
        \includegraphics{figures/vuteco-detail.pdf}
    }
    \caption{\EI{FIX, broken!} Graphical depiction of the inner working of \vuteco.}
    \label{fig:vuteco-detail}
\end{figure}

The \matching task consists of determining if a test case and a vulnerability are related, i.e., the former witnesses the latter.
This task is carried out by the joint use of the \finder model (as in Section~\ref{subsec:finding}) plus the \linker model.
Indeed, the \linker model is focused on a \textit{simplified} task that assumes the input test case is surely security-related.
The motivation for the joint use of two models---rather than the direct use of the \linker model---is supported by the results of the in-vitro experimentation in Section~\ref{sec:results}.

The \linker model has been trained on its simplified task, i.e., trained to classify pairs of security-related tests and vulnerability descriptions into two classes, i.e., \linkerPosClass (the positive class) and \linkerNegClass (the negative class).
%
The lower part of Figure~\ref{fig:vuteco-detail} depicts the architecture of the \linker model.
Such a model follows a similar architecture to the \finder model, leveraging the same pre-trained CodeBERT model to create the input embeddings.
However, instead of creating the sentence embedding of the sole test case code (still reduced to a single line), the \linker model also adds the vulnerability description as well.
Therefore, the vulnerability description was \textit{concatenated} before the test case with a special token [SEP] in between to indicate the two different sentences (as required by BERT-like models) and then tokenized in the same manner as the \finder model.
%
The resulting sentence embedding is then sent to a deep neural network with an input layer of 768 neurons (to match the embedding size) and a linear hidden layer of 256 output neurons, calling the GELU activation function.
%
Like the \finder model, the final layer returns the logits for the classification, converted into probabilities of the two classes with the Softmax function.

To enable the joint training and use of the \finder and \linker models, \vuteco employs a \textit{decision function} after the \finder made its prediction on the test case---depicted on the right-most side of Figure~\ref{fig:vuteco-detail}.
Indeed, if the probability of the test case belonging to the \finderPosClass class was at least $0.5$, then the \linker model is invoked, and its probabilities are used for the \globalPosClass and \globalNegClass classes.
Otherwise, the probabilities of the \finder are used instead.
Simply put, the \linker's judgment is not considered if the \finder model did not flag the test case as \finderPosClass as the \linker model expects security-related tests as input.
%
Hereafter, the wording ``integrated model'' refers to the joint use of \finder and \linker to address the \matching task.

%\subsubsection{The \finder and \linker Integration}

%The right-most side of Figure~\ref{fig:vuteco-detail} shows how \finder and \linker models have been used in conjunction to address the main downstream task.
%Namely, once the \finder and \linker models made their predictions, \vuteco returns the logits from the \linker model if the probability of \finderPosClass (computed applying the Softmax function on the logits) was more than $0.1$, otherwise \vuteco returns the logits computed by the \finder model.
%Hereafter, the wording ``integrated model,'' refers to the integration of \finder and \linker for the main downstream task.
%This selection is due to the \linker model's inability to make reliable predictions if the test method has not been flagged as a suspect vulnerability-witnessing.
%Consequently, if the \finder model flagged the test as a suspect, the prediction made by the \linker model becomes meaningful, and it will be used to assess if the supposed witnessing test is linked to the given vulnerability description.
%We refer to this integration as the \globalModel model for simplicity.

% \subsection{Training the Deployment Version}
% \label{subsub:train}

% The \finder and \linker models were trained in individual sessions to become more familiar with their sub-tasks before the integration.
% %---i.e., assessing the links between a \JUnit test method and a vulnerability description.
% Afterward, the integrated model was trained further to perform well on the main downstream task.
% As a result, three training sessions were needed: One for the \finder model, one for the \linker, and one for the integrated model.
% From \vuteco's perspective, the first two sessions can be seen as a \textit{pre-training}, while the latter as a \textit{fine-tuning}.
% The three training datasets have been extracted from \VulforJ, the only available knowledge base containing \JUnit test cases with the witnessed vulnerabilities.

% The deployment version of \vuteco relies on the results of the in-vitro evaluation (Section~\ref{subsec:in-vitro-results} to configure the training process.
% %
% The \finder model was trained on \finderTrainingSize \JUnit test cases, of which \finderTrainingPosSize were witnessing a vulnerability ($\sim$\finderTrainingPosPerc).
% %
% Likewise, the \linker model had a training set of \linkerTrainingSize pairs of witnessing test cases and vulnerability descriptions, of which \linkerTrainingPosSize represented valid links ($\sim$\linkerTrainingPosPerc).
% To handle such a class imbalance, this training set has been \textit{augmented} via bootstrapping (i.e., sampling with replacements) the \linkerPosClass pairs until the final class imbalance resulted in 1:2 (i.e., one valid link for every two invalid links), ending up with \linkerTrainingExtra new instances of class \linkerPosClass.
% %
% After training the two models for their respective sub-tasks, the integrated model was trained on a third training set made of \globalTrainingSize pairs of test cases and vulnerabilities, of which \globalTrainingPosSize were valid ($\sim$\globalTrainingPosPerc).

% The \finder, \linker, and the integrated model have been trained for \finderTrainingEpochs, \linkerTrainingEpochs, and \globalTrainingEpochs epochs, respectively.
% In all three sessions, 15\% of the training sets were used as \textit{development sets} to find the best training checkpoint to return at the end.
% Namely, the models were evaluated on their development sets at the end of each epoch, and the best checkpoint was determined by the highest \TEMP{AUC-ROC}~\cite{Junge2018:roc} and lowest cross-entropy loss (in case of ties).

\section{The \search\ Search Algorithm}
\label{sec:search}

%In traditional ML, structure changes and step (operator) changes are performed before model training, \ie, fixed to the training process, and weights are updated with SGD, because weights are continous, differentiable values, and there are significantly more weights than structure and operator changes. In workflow autotuning, all three types of cogs can be chosen with a unified search-based approach, because all of them are non-differentiable configurations and the number of cogs in different types are all small.
%Thus, \sysname\ only needs to navigate the search space of combination of cogs as the search space to produce its workflow optimization results.

%We propose, \textit{\textbf{\search}}, an adaptive hierarchical search algorithm that autotunes gen-AI workflows based on observed end-to-end workflow results. In each search iteration, \search\ selects a combination of cogs to apply to the workflow and executes the resulting workflow with user-provided training inputs. \search\ evaluates the final generation quality using the user-specified evaluator and measures the execution time and cost for each training input. These results are aggregated and serve as BO observations and pruning criteria.
%the optimizer can condition on and propose better configurations in later trials. The optimizer will also be informed about the violation of any user-specified metric thresholds. More details of this mechanism can be found in Appendix ~\ref{appdx:TPE}.

With our insights in Section~\ref{sec:theory}, we believe that search methods based on Bayesian Optimizer (BO) can work for all types of cogs in gen-AI workflow autotuning because of BO's efficiency in searching discrete search space.
A key challenge in designing a BO-based search is the limited search budgets that need to be used to search a high-dimensional cog space. 
For example, for 4 cogs each with 4 options and a workflow of 3 LLM steps, the search space is $4^{12}$. Suppose each search uses GPT-4o and has 1000 output tokens, the entire space needs around \$168K to go through. A user search budget of \$100 can cover only 0.06\% of the search space. A traditional BO approach cannot find good results with such small budgets.
%The entire search space grows exponentially with the number of cogs and the number of steps in a workflow. Moreover, different cogs and different combinations of cogs can have varying impacts on different workflows. 
%Without prior knowledge, it is difficult to determine the amount of budget to give to each cog.

To confront this challenge, we propose \textit{\textbf{\search}}, an adaptive hierarchical search algorithm that efficiently assigns search budget across cogs based on budget size and observed workflow evaluation results, as defined in Algorithms~\ref{alg:main} and \ref{alg:outer} and described below.
%autotunes gen-AI workflows based on observed end-to-end workflow results.
%\search\ includes a search layer partitioning method, a search budget initial assignment method, an evaluation-guided budget re-allocation mechanism, and a convergence-based early-exiting strategy. We discuss them in details below.

%\zijian{\search\ allows users to specify the optimization budget allowed in terms of the maximum number of search iterations. Based on the relationship between the complexity of the search space and the available budget, we will separate all tunable parameters into different layers each optimized by independent Bayesian optimization routines. Then we will decide the maximum budget each layer can get with a bottom-up partition strategy. Besides search space and resource partition, we also employ a novel allocation algorithm that integrates successive halving~\cite{successivehalving} and a convergence-based early exiting strategy to facilitate efficient usage of assigned budget.}


% The outermost layer searches and selects structures for a workflow; the middle layer searches and selects step options under the workflow structure selected in the outermost layer; the innermost layer searches and selects weights with the given workflow structure and steps. 

\begin{algorithm}[h]
    \caption{\search\ Algorithm}
    \label{alg:main}
      \small
\begin{algorithmic}[1]
\STATE \textbf{Global Value:} $R = \emptyset$ \COMMENT{Global result set}
%\STATE \textbf{Global Value:} $F = \emptyset$ \COMMENT{Global observation set}

%Reduct factor $\eta > 1$, explore width $W$
\STATE \textbf{Input:} User-specified Total Budget $TB$
\STATE \textbf{Input:} Cog set $C = \{c_{11},c_{12},...\}, \{c_{21},c_{22},...\}, \{c_{31},c_{32},...\}$

    \STATE
%\FOR{$i = 1,2,3$}
    %\COMMENT{$\alpha$ is a configurable value default to 1.1}
%\ENDFOR
%\STATE
%    \STATE \{$B_1,B_2,B_3$\} = LayerPartition($C$) \COMMENT{Calculate ideal layer budget}
    %\STATE \textbf{Glob}.budgets = budgets
%    \STATE opt\_layers = init\_opt\_routines() \COMMENT{A list of optimize routine each layer will use for search}
%\STATE
%    \FOR{$i \in L, \dots, 1$}
%        \IF{$i == L$}
 %           \STATE opt\_layers[L] = InnerLayerOpt
  %      \ELSE
   %         \STATE opt\_layers[i] = OuterLayerOpt
            %\STATE opt\_layers[i].next\_layer\_budgets = B[i+1]
            %\STATE opt\_layers[i].next\_layer\_routine = opt\_layers[i+1]
    %    \ENDIF
    %\ENDFOR
%\STATE opt\_layers[1].invoke($\emptyset$, B[1])
\STATE $U = 0$ \COMMENT{Used budget so far, initialize to 0}

\STATE \COMMENT{Perform search with 1 to 3 layers until budget runs out}
\FOR{$L = 1,2,3$} 
        \IF{$L=1$}
            \STATE $C_1 = C_1 \cup C_2 \cup C_3$ \COMMENT{Merge all cogs into a single layer}
        \ENDIF
        \IF{$L==2$}
            \STATE $C_1 = C_1 \cup C_2$ \COMMENT{Merge step and weight cogs}
            \STATE $C_2 = C_3$ \COMMENT{Architecture cog becomes the second layer}
        \ENDIF
        \STATE
    \FOR{$i = 1,..,L$}
    \STATE $NC_i = |C_i|$ \COMMENT{Total number of cogs in layer $L$} 
%    NO_i &= \sum_{L} \{\text{number of possible options in cog } c_{ij}\} \\
    \STATE $S_i = NC_i^\alpha$ \COMMENT{Estimated expected search size in layer $i$}
    \ENDFOR
    \STATE $E_L = \prod\limits_{i=1}^{L}S_i$ \COMMENT{Expected total search size in the current round}
    \STATE $E = TB - U > E_L$ ? $E_L$ : $(TB - U)$ \COMMENT{Consider insufficient budget} 
    \IF{$L==3$ and $(TB - U)$ > $E_L$}
         \STATE $E = TB - U$ \COMMENT{Spend all remaining budget if at 3 layer}
    \ENDIF
    %\STATE$TL = |N|$ \COMMENT{number of layers}
    \FOR{$i = 1,..,L$}
        \STATE $B_i =  \lfloor S_i \times \sqrt[L]{\frac{E}{E_L}}\rfloor$
        %$B$ = BudgetAssign($N$, $TL$, $TB$)
        \COMMENT{Assign budget proportionally to $S_i$}
    \ENDFOR
    \STATE
\STATE \texttt{LayerSearch} ($\emptyset$, $B$, $L$, $B_L$) \COMMENT{Hierarchical search from layer $L$}
\STATE
\STATE $U = U + E$
\IF{$U \geq TB$}
\STATE break \COMMENT{Stop search when using up all user budget}
\ENDIF
\ENDFOR
%\STATE
%\STATE $O$ = \texttt{SelectBestConfigs} ($R$)
%\IF{$L == 1$}
%    \STATE InnerLayerOpt($\emptyset$, B[1])
%\ELSE
%    \STATE OuterLayerOpt($\emptyset$, B[1], 1)
%\ENDIF
\STATE
\STATE \textbf{Output:} $O$ = \texttt{SelectBestConfigs} ($R$) \COMMENT{Return best optimizations}
\end{algorithmic}
\end{algorithm}

\subsection{Hierarchical Layer and Budget Partition}
\label{sec:ssp}

%We motivate \search's adaptive hierarchical search 
A non-hierarchical search has all cog options in a single-layer search space for an optimizer like BO to search, an approach taken by prior workflow optimizers~\cite{dspy-2-2024,gptswarm}.
With small budgets, a single-layer hierarchy allows BO-like search to spend the budget on dimensions that could potentially generate some improvements.
%While given enough budget, the single-layer space can be extensively searched to find global optimal, with little budget, 
However, a major issue with a single-layer search space is that a search algorithm like BO can be stuck at a local optimum even when budgets increase.
% (unless the budget is close to covering a very large space across dimensions).
To mitigate this issue, our idea is to perform a hierarchical search that works by choosing configurations in the outermost layer first, then under each chosen configuration, choosing the next layer's configurations until the innermost layer. 
With such a hierarchy, a search algorithm could force each layer to sample some values. Given enough budget, each dimension will receive some sampling points, allowing better coverage in the entire search space. However, with high dimensionality (\ie, many types of cogs) and insufficient budget, a hierarchical search may not be able to perform enough local search to find any good optimizations.

To support different user-specified budgets and to get the best of both approaches, we propose an adaptive hierarchical search approach, as shown in Algorithm~\ref{alg:main}.
\search\ starts the search by combining all cogs into one layer ($L=1$, line 9 in Algorithm~\ref{alg:main}) and estimating the expected search budget of this single layer to be the total number of cogs to the power of $\alpha$ (lines 16-19, by default $\alpha = 1.1$). This budget is then passed to the \texttt{LayerSearch} function (Algorithm~\ref{alg:outer}) to perform the actual cog search. When the user-defined budget is no larger than this estimated budget, we expect the single-layer, non-hierarchical search to work better than hierarchical search.
%as the budget for this single layer.

If the user-defined budget is larger, \search\ continues the search with two layers ($L=2$), combining step and weight cogs into the inner layer and architecture cogs as the outer layer (lines 11-14).
\search\ estimates the total search budget for this round as the product of the number of cogs in each of the two layers to the power of $\alpha$ (lines 16-20). It then distributes the estimated search budget between the two layers proportionally to each layer's complexity (lines 22-24) and calls the upper layer's \texttt{LayerSearch} function. Afterward, if there is still budget left, \search\ performs a last round of search using three layers and the remaining budget in a similar way as described above but with three separate layers (architecture as the outermost, step as the middle, and weight cogs as the innermost layer). Two or three layers work better for larger user-defined budgets, as they allow for a larger coverage of the high-dimensional search space.

Finally, \search\ combines all the search results to select the best configurations based on user-defined metrics (line 34).

%\search\ organizes cogs by having architecture cogs in the outer-most search layer, step cogs in the middle layer, and weight cogs in the inner-most layer (line 4 in Algorithm~\ref{alg:main}).
%This is because step cogs' input and output format are dependent on the workflow structure, and the effectiveness of weights (\eg, prompting) are dependent on steps (\eg, LLM model). 

% increases the number of layers until hitting the user-specified total search budget, $TB$

%Thus, the first step of \search\ is to determine the number of layers in its hierarchy and what cogs to include in a layer.
%Intuitively, structure cogs should be placed in the outer-most search layer to be determined first before exploring other cogs. This is because other cogs change node and edge values, and it is easier for 
%However, instead of a fixed number of layers in the hierarchy, we adapt the cog layering according to user-specified total search budgets, $TB$, and the complexity of each layer, using Algorithm~\ref{alg:main}.

% the following \texttt{LayerPartition} method.
%We begin by modeling the relationship between the expected number of evaluations and the number of cogs as well as the number of options in each layer:

%We first consider the identity of each cog in the search space. All structure-cogs will be placed in the outer-most search layer exclusively, which is similar to non-differentiable NAS in traditional ML training. This layer will fix the workflow graph and pass it to the following layer, allowing a stabilized search space for faster convergence.

%Since step-cogs will not create a changing search space, the partition of step-cogs and weight-cogs is conditioned on the search space complexity and the given total budget. Separating step-cogs out can benefit from a more flexible budget allocation strategy and broader exploration for local search at weight-cogs but performs poorly when the given budget is more constrained, in that case, we will optimize them jointly in the same layer.


%\small
%\begin{align*}
%    C &= \{c_{11},c_{12},...\}, \{c_{21},c_{22},...\}, \{c_{31},c_{32},...\} \\
%    NC_i &= \text{total number of cogs in layer i} \\
%    NO_i &= \sum_{j} \{\text{number of possible options in cog } c_{ij}\} \\
%    N_i &= max(NC_i^\alpha,NO_i) \\
%    N_i &= \sum_{j} \{\text{number of possible options in } C_{ij}\} \\
%    N_i &= max(|C_i|^\alpha, N_i) \\
%    B_j &= \prod\limits_{i=1}^{j}N_i, j \in \{1,2,3\}
%\end{align*}

%\normalsize
%where $L$ represents the total number of layers and can be 1, 2, or 3. 
%$C$ represents the entire cog search space, with each row $c_{i*}$ being one of the three types of cogs and lower layers having lower-numbered rows (\eg, $c_{1*}$ being weight cogs). $NC_i$ is the number of cogs in layer $i$, and $NO_i$ is the total number of options across all cogs in layer $i$. $N_i$ is our estimation of the complexity of layer $i$ based on $NC_i$ and $NO_i$ ($\alpha$ is a configurable weight to control the importance between $NC_i$ and $NO_i$; by default $\alpha = 1.1$). 
%$\alpha$ stands for a control parameter, setting the intensity of this scaling behavior w.r.t the number of cogs, we found that $\alpha = 1.2$ is empirically sufficient and efficient for optimizing real workloads. 
%$B_j$ is the expected total number of workflow evaluations for all the lower $j$ layers.
%After calculating $B_1$, $B_2$, and $B_3$, we compare the total budget $TB$ with them.
%When $TB \geq B_3$, we set the total number of layers, $TL$, to 3. When $B_2 \leq TB < B_3$, we set the total number of layers to 2 and merge the step and weight cogs into one layer. When $TB < B_1$, we put all cogs in one layer.
%We only create a separate layer for step-cogs when the given budget $TB$ is greater or equal to the total expected budget for three layers.

%\subsection{Seach Budget Partition}
%\label{sec:sbp}
%After determining cog layers, we distribute the total budget, $TB$, across the layers proportionally to each layer's expected budget $N_i$: , which is the \texttt{BudgetAssign} function.
%We follow a bottom-up partition strategy, where lower layers will try to greedily take the expected budget. This stems from two simple heuristics: (1) feedback to the upper layer is more accurate when the succeeding layer is trained with enough iterations, and (2) the effectiveness of a structure change depends on the setting of individual steps in the workflow (\eg, majority voting is more powerful when each LLM-agent is embedded with diverse few-shot examples or reasoning styles). In cases where the given resource exceeds the total expected budget, 
%We assign $TB$ across layers proportionally to their expected budget $N_i$. 
%The budget assigned at each layer $B_i$ given the total available number of evaluations $TB$ is obtained as follows:

%\small
%\begin{align}
%B_i &=  \lfloor N_i \times \sqrt[L]{\frac{TB}{B^*}}\rfloor
%    B_L &= \begin{cases}
%        min(N_L, TB) & TB < B^* \\
%        \lfloor N_L \times \sqrt[L]{\frac{TB}{B^*}}\rfloor & TB \geq B^*
%    \end{cases}
%    \\
%    B_i &= \begin{cases}
%        min(N_i, \lfloor\frac{TB}{\prod_{j=i+1}^L B_j}\rfloor) & TB < B^* \\
%        \lfloor N_i \times \sqrt[L]{\frac{TB}{B^*}}\rfloor & TB \geq B^*
%    \end{cases}
%\end{align}

%\normalsize


\subsection{Recursive Layer-Wise Search Algorithm}
%The calculation above pre-assigns cogs to layers and search budgets to each layer. 
We now introduce how \search\ performs the actual search in a recursive manner until the inner-most layer is searched, as presented in Algorithm~\ref{alg:outer} \texttt{LayerSearch}. 
Our overall goal is to ensure strong cog option coverage within each layer while quickly directing budgets to more promising cog options based on evaluation results.
%So far, we have determined the optimization layer structure and the maximum allowed search iteration each layer will get. Next, we introduce how the budget is consumed in each layer. The inner-most layer, where weight-cogs, and potentially step-cogs, reside, follows the conventional Bayesian optimization process, exhausting all budgets unless an early stop signal is sent. This signal will be triggered when the current optimizer witnesses $p$ consecutive iterations without any improvements above the threshold. All optimization layers use early stopping to avoid budget waste.
%Algorithm~\ref{alg:inner} describes the search happening at the inner-most (bottom) layer, and 
Specifically, every layer's search is under a chosen set of cog configurations from its upper layers ($C_{chosen}$) and is given a budget $b$. 
In the inner-most layer (lines 7-20), \search\ samples $b$ configurations and evaluates the workflow for each of them together with the configurations from all upper layers ($C_{chosen}$). The evaluation results are added to the feedback set $F$ as the return of this layer.

\begin{algorithm}[h]
  %\algsetup{linenosize=\tiny}
  \small
    \caption{\texttt{LayerSearch} Function}
    \label{alg:outer}
\begin{algorithmic}[1]
%\STATE \textbf{Global Config:} Reduct factor $\eta > 1$, explore width $W$
\STATE \textbf{Global Value:} $R$ \COMMENT{Global result set}
%\STATE \textbf{Global Value:} $F$ \COMMENT{Global observation set}
\STATE \textbf{Input:} $C_{chosen}$: configs chosen in upper layers
\STATE \textbf{Input:} $B$: Array storing assigned budgets to different layers
\STATE \textbf{Input:} $curr\_layer$: this layer's level
\STATE \textbf{Input:} $curr\_b$: this layer's assigned budget
%\STATE
%\FUNCTION{LayerSearch\hspace{0.4em}($C_{chosen}$, $B$, $curr\_layer$, $curr\_b$)}

    \STATE
    \STATE \COMMENT{Search for inner-most layer}
    \IF{curr\_layer == 1}
        \STATE $F = \emptyset$ \COMMENT{Init this layer's feedback set to empty}
        %\STATE $F^{\prime} = match(C_{chosen}, F)$ \COMMENT{Local feedback set}
        \FOR{$k = 0, \dots, curr\_b$}
            \STATE $\lambda$ = \texttt{TPESample} (1) \COMMENT{Sample one configuration using TPE}
            \STATE $f = $ \texttt{EvaluateWorkflow} ($C_{chosen} \cup \lambda$)
            \STATE $R = R \cup \{C_{chosen} \cup \lambda\}$ \COMMENT{Add configuration to global $R$}
            \IF{\texttt{EarlyStop} (f)}
            \STATE break
            \ENDIF
            \STATE $F = F \cup \{f\}$ \COMMENT{Add evaluate result to feedback $F$}
        \ENDFOR
        %\STATE $F = F \cup F^{\prime}$
        \STATE \textbf{Return} $F$
    \ENDIF
    \STATE
    \STATE \COMMENT{Search for non-inner-most layer}
    %\STATE $K = \lfloor \frac{b}{W} \rfloor$, 
    \STATE $b\_used = 0$, $TF = \emptyset$ \COMMENT{Init this layer's used budget and feedback set}
    \STATE $R = \lceil\frac{curr\_b}{\eta}\rceil$, $S = \lfloor\frac{curr\_b}{R}\rfloor$ \COMMENT{Set $R$ and $S$ based on $curr\_b$}
    \STATE
    \WHILE{$b\_{used}$ $\leq$ $curr\_b$}
        \STATE \COMMENT{Sample $W$ configs at a time until running out of $curr\_b$}
        \STATE $n = (curr\_b - b_{used})$ > $W$ ? $W$ : $(curr\_b - b_{used})$
        %\IF{$b - b_{used} < W$}
        %    \STATE $n = b_l - b_{used}$
        %\ELSE
         %   \STATE $n=W$
        %\ENDIF
        \STATE $b\_used$ += $n$
        %\STATE $n = \text{min}(W,\ b_l - kW)$ \COMMENT{Propose $W$ configs and meet $b_l$ constraint}
        \STATE $\Theta = $ \texttt{TPESample} ($n$) \COMMENT{Sample a chunk of $n$ configs in the layer} 
        %\STATE $F^{\prime} = match(C_{chosen}, F)$ \COMMENT{Per-chunk feedback set}
        \STATE $F = \emptyset$ \COMMENT{Init this layer's feedback set to empty}
        \STATE
        \FOR{$s = 0, 1, \dots, S$}
            \STATE $r_s = R\cdot \eta^s$
            \FOR{$\theta \in \Theta$}
                %\IF{$curr\_layer < max\_layer$}
                    \STATE $f =$ \texttt{LayerSearch} ($C_{chosen} \cup \{\theta\}$, $B$, curr\_layer$-1$, $r_s$)
                %\ELSE
                %    \STATE $f =$ InnerOpt($\gamma \cup \{\theta\}$, $r_s$)
                %\STATE $f$ = $opt\_layers[current\_layer+1](\gamma \cup \{\theta\}, r_s)$ \{Optimize the current config at the next layer with $r_s$ budget \}
                %\ENDIF
                \STATE $F = F \cup f$ \COMMENT{Add evaluate result to feedback}
                \IF{\texttt{EarlyStop} ($f$)}
                    \STATE $\Theta = \Theta - \{\theta\}$ \COMMENT{Skip converged configs}
                \ENDIF
            \ENDFOR
            \STATE $\Theta$ = Select top $\lfloor \frac{|\Theta|}{\eta}\rfloor$ configs from $F$ based on user-specified metrics
        \ENDFOR
        \STATE
        \IF{\texttt{EarlyStop} ($F$)}
            \STATE break \COMMENT{Skip remaining search if results converged}
        \ENDIF
        \STATE $TF = TF \cup F$
    \ENDWHILE
    %\STATE $F = F \cup TF$
        \STATE \textbf{Return} $TF$

%\ENDFUNCTION

%\STATE \textbf{Output:} Best metrics in all trials
\end{algorithmic}
\end{algorithm}

% consumption\_nextlayer\_bucket = WSR

% for s in 0, 1,...S do
%     w = W*\eta^{s}
%     r = R*\eta^{-s}

% total budget at next layer = b_l / W * WSR = b_l * SR

% b_l * SR <= b_l * B_l+1

% S = B_{l+1} / R



For a non-inner-most layer, \search\ samples a chunk ($W$) of points at a time using the TPE BO algorithm~\cite{bergstra2011tpe} until all this layer's pre-assigned budget is exhausted (lines 27-30). Within a chunk, \search\ uses a successive-halving-like approach to iteratively direct the search budget to more promising configurations within the chunk (the dynamically changing set, $\Theta$). In each iteration, \search\ calls the next-level search function for each sampled configuration in $\Theta$ with a budget of $r_s$ and adds the evaluation observations from lower layers to the feedback set $F$ for later TPE sampling to use (lines 35-37).
In the first iteration ($s=0$), $r_s$ is set to $R\cdot \eta^0=R$ (line 34). After the inner layers use this budget to search, \search\ filters out configurations with lower performance and only keeps the top $\lfloor \frac{|\Theta|}{\eta}\rfloor$ configurations as the new $\Theta$ to explore in the next iteration (line 42). In each next iteration, \search\ increases $r_s$ by $\eta$ times (line 34), essentially giving more search budget to the better configurations from the previous iteration.

The successive halving method effectively distributes the search budget to more promising configurations, while the chunk-based sampling approach allows for evaluation feedback to accumulate quickly so that later rounds of TPE can get more feedback (compared to no chunking and sampling all $b$ configurations at the same time). To further improve the search efficiency, we adopt an {\em early stop} approach where we stop a chunk or a layer's search when we find its latest few searches do not improve workflow results by more than a threshold, indicating convergence (lines 14,38,45).

%algorithm takes as input other cog settings from previous layers and the assigned budget at the current layer. It tiles the search loop into fixed-size blocks (line 4), each runs the SuccessiveHalving (SH) subroutine in the inner loop (line 7-15). In each SH iteration, only top-$1/\eta$-quantile configurations in $\Theta$ will continue in the next round with $\eta$ times larger budget consumption. As a result, exponentially more trials will be performed by more promising configurations. 

%On average, \textit{Outer-layer search} will create $K$ brackets, each granting approximately $WRS$ budget to the next layer. $R$ represents the smallest amount of resource allocated to any configurations in $\Theta$. 

% layer - 1: budget = 4
% K * W <= b\_current layer
% layer -1: itear 0: propose 2

%     SH:
%     2 config -> R
%     1 config -> 2R

%     iteration 1: propose 2 = W
%     SH:
%     2 config -> R
%     1 config -> 2R

% W configs; each has R resource

% W / eta configs; each has R * eta resource

% R -> least resource one config can get = B2 - smth
% R + R*eta + ... + R*eta\^s -> most promising = B2 + smth


% $L2$ is the middle layer where structure-cogs and step-cogs may be placed exclusively. We employ hyperband for its robustness in exploration and exploitation trace-off. If this layer exists, it will instruct $L1$ the number of search iterations to run in each invocation. Specifically, in each iteration at line 4, \sysname will propose $n$ configurations and run SuccessiveHalving (SH) subroutine (line 8-15). SH will optimize each proposal and use the search results from $L1$ to rank their performance. Each time only the top-performing $n \cdot \eta^{-i}$ can continue in the next round with a larger budget. With this strategy, exponentially more search budgets are allocated to more promising configs at $L2$.

% \input{algo-l2-search}

% $L3$ is the outer-most layer for structure-cogs only when $L2$ is created. For this layer, we employ plain SH without hyperband because of its predictable convergence behavior. This is mainly due to two factors: (1) structure change to the workflow is more significant thus different configurations are more likely to deviate after training with the following layers. (2) with the search space partition strategy in Sec ~\ref{sec:ssp}, we can assume the available budget at each layer is substantial when $L3$ exists. Given this prior knowledge, we can avoid grid searching control parameter $n$ as in the hyperband but adopt a more aggressive allocation scheme to bias towards better proposals and moderate search wastes.



%\subsubsection{Runtime Budget Adaptation}
%Using static estimation of the expected budget for each layer is not enough, we also adjust the assignment during the optimization based on real convergence behavior. Specifically, for layer $i$, we record the number of configurations evaluated in each optimize routine. We set the convergence indicator $C_{ij}$ of $j^{th}$ routine with this number if the search early exits before reaching the budget limit, otherwise 2\x of its assigned resource. Then we update $E_i$ with $\frac{\sum_{j}^M C_{ij}}{M}$. \sysname\ will update the budget partition according to Sec~\ref{sec:sbp} for any newly spawned optimizer routines. Besides controlling the proportion of budgets across layers, a smaller/larger $B_{l+1}$ will also guide the SH in Alg~\ref{alg:outer} to shrink/extend the budget $R$ for differentiating config performance.


\section{\sysname\ Design}
\label{sec:cognify}

We build \sysname, an extensible gen-AI workflow autotuning platform based on the \search\ algorithm. The input to \sysname\ is the user-written gen-AI workflow (we currently support LangChain \cite{langchain-repo}, DSPy \cite{khattab2024dspy}, and our own programming model), a user-provided workflow training set, a user-chosen evaluator, and a user-specified total search budget. \sysname\ currently supports three autotuning objectives: generation quality (defined by the user evaluator), total workflow execution cost, and total workflow execution latency. Users can choose one or more of these objectives and set thresholds for them or the remaining metrics (\eg, optimize cost and latency while ensuring quality to be at least 5\% better than the original workflow). 
\sysname\ uses the \search\ algorithm to search through the cog space.
When given multiple optimization objectives, \sysname\ maintains a sorted optimization queue for each objective and performs its pruning and final result selection from all the sorted queues (possibly with different weighted numbers).
To speed up the search process, we employ parallel execution, where a user-configurable number of optimizers, each taking a chunk of search load, work together in parallel. %Below, we introduce each type of cogs in more details.
\sysname\ returns multiple autotuned workflow versions based on user-specified objectives.
\sysname\ also allows users to continue the auto-tuning from a previous optimization result with more budgets so that users can gradually increase their search budget without prior knowledge of what budget is sufficient.
Appendix~\ref{sec:apdx-example} shows an example of \sysname-tuned workflow outputs. 
\sysname\ currently supports six cogs in three categories, as discussed below. 

%In \sysname, we call every workflow optimization technique a {\em cog}, including structure-changing cogs like task decomposition, step-changing cogs like model selection, and weight-changing cogs like adding few-shot examples to prompts. 
%\sysname\ places structure-changing cogs in the outermost layer, step cogs in the middle layer, and weight cogs in the innermost layer, because \fixme{TODO}.


\subsection{Architecture Cogs}
\label{sec:structure-cog}
%Changing the structure of a workflow can potentially improve its generation quality (\eg, by using multiple steps to attempt at a task in parallel or in chain) or reduce its execution cost and latency (\eg, by merging or removing steps).
\sysname\ currently supports two architecture cogs: task decomposition and task ensemble.
Task decomposition~\cite{khot2023decomposed} breaks a workflow step into multiple sub-steps and can potentially improve generation quality and lower execution costs, as decomposed tasks are easier to solve even with a small (cheaper) model.
There are numerous ways to perform task decomposition in a workflow. 
%, as all LM steps can potentially be decomposed and into different numbers of sub-steps in different ways. Throwing all options to the Bayesian Optimizer would drastically increase the search space for \sysname. 
To reduce the search space, we propose several ways to narrow down task decomposition options. Even though we present these techniques in the setting of task decomposition, they generalize to many other structure-changing tuning techniques.

%First, we narrow down a selected set of steps in a workflow to decompose. 
Intuitively, complex tasks are the hardest to solve and worth decomposition the most. We use a combination of LLM-as-a-judge \cite{vicuna_share_gpt} and static graph (program) analysis to identify complex steps. We instruct an LLM to give a rating of the complexity of each step in a workflow. We then analyze the relationship between steps in a workflow and find the number of out-edges of each step (\ie, the number of subsequent steps getting this step's output). More out-edges imply that a step is likely performing more tasks at the same time and is thus more complex. We multiply the LLM-produced rating and the number of out-edges for each step and pick the modules with scores above a learnable threshold as the target for task decomposition. We then instruct an LLM to propose a decomposition (\ie, generate the submodules and their prompts) for each candidate step. %We provide the LLM with few-shot examples for what proposed modules for a separate task could look like. We also add a refinement step that validates whether the proposition decomposition maintains the semantics of the original trajectory. Once candidate decompositions are generated, those are used for the entirety of the optimization.

{
\begin{figure*}[t!]
\begin{center}
\centerline{\includegraphics[width=0.95\textwidth]{Figures/big_grid.pdf}}
\vspace{-0.1in}
\mycap{Generation Quality vs Cost/Latency.}{Dashed lines show the Pareto frontier (upper left is better). Cost shown as model API dollar cost for every 1000 requests. Cognify selects models
from GPT-4o-mini and Llama-8B. DSPy and Trace do not support model selection and are given GPT-4o-mini for all steps. Trace results for Text-2-SQL and FinRobot have 0 quality and are not included.} 
\Description{Eight graphs with different shapes representing baselines compared to points on a Pareto frontier.}
\label{fig-biggrid}
\end{center}
\end{figure*}
}


The second structure-changing cog that \sysname\ supports is task ensembling. This cog spawns multiple parallel steps (or samplers) for a single step in the original workflow, as well as an aggregator step that returns the best output (or combination of outputs). By introducing parallel steps, \sysname\ can optimize these independently with step and weight cogs. This provides the aggregator with a diverse set of outputs to choose from. 
%The aggregator is prompted with the role of the samplers, as well as the inputs to each. It also receives a criteria by which it should make a decision. We choose to prompt it with a qualitative description of how it should resolve discrepancies between outputs. 


\subsection{Step Cogs}
We currently support two step-changing cogs: model selection for language-model (LM) steps and code rewriting for code steps.

For model selection, to reduce its search space, we identify ``important'' LM steps---steps that most critically impact the final workflow output to reduce the set \search\ performs TPE sampling on. Our approach is to test each step in isolation by freezing other steps with the cheapest model and trying different models on the step under testing. 
We then calculate the difference between the model yielding the best and worst workflow results as the importance of the step under testing. %For each model, we get the workflow output quality score using sampled user-supplied inputs and user-specific evaluator. We then calculate the difference between the highest and lowest scores as this module's importance. 
After testing all the steps, we choose the steps with the highest K\% importance as the ones for TPE to sample from.
%, where K is determined based on user-chosen stop criteria. We then initialize the Bayesian optimization to start with the state where important modules use the largest model and all other modules use the cheapest model. We set the TPE optimization bandwidth of each module to be the inverse of importance, \ie, the more important a module is the more TPE spends on optimizing.

The second step cog \sysname\ supports is code rewriting, where it automatically changes code steps to use better implementation. To rewrite a code step, \sysname\ finds the $k$ worst- and best-performing training data points and feeds their corresponding input and output pairs of this code step to an LLM. We let the LLM propose $n$ new candidate code pieces for the step at a time.
%in parallel to generate a set of $n$ candidates.
In subsequent trials, the optimizer dynamically updates the candidate set using feedback from the evaluator.


\subsection{Weight Cogs}
\sysname\ currently supports two weight-changing cogs: reasoning and few-shot examples.
First, \sysname\ supports adding reasoning capability to the user's original prompt, with two options: zero-shot Chain-of-Thought \cite{wei2022chain} (\ie, ``think step-by-step...'') and dynamic planning \cite{huang2022language} (\ie, ``break down the task into simpler sub-tasks...''). These prompts are appended to the user's prompt. In the case where the original module relies on structured output, we support a reason-then-format option that injects reasoning text into the prompt while maintaining the original output schema.

Second, \sysname\ supports dynamically adding few-shot examples to a prompt. At the end of each iteration, we choose the top-$k$-performing examples for an LM step in the training data and use their corresponding input-output pairs of the LM step as the few-shot examples to be appended to the original prompt to the LM step for later iterations' TPE sampling. As such, the set of few-shot examples is constantly evolving during the optimization process based on the workflow's evaluation results. 
%Few-shot examples are available to all modules, even intermediate steps in the workflow. We use the full trajectory of each request to generate examples for the intermediate steps. Furthermore, we automatically filter out examples that do not meet a user-specified threshold. 




\begin{table*}[t]
\centering
\fontsize{11pt}{11pt}\selectfont
\begin{tabular}{lllllllllllll}
\toprule
\multicolumn{1}{c}{\textbf{task}} & \multicolumn{2}{c}{\textbf{Mir}} & \multicolumn{2}{c}{\textbf{Lai}} & \multicolumn{2}{c}{\textbf{Ziegen.}} & \multicolumn{2}{c}{\textbf{Cao}} & \multicolumn{2}{c}{\textbf{Alva-Man.}} & \multicolumn{1}{c}{\textbf{avg.}} & \textbf{\begin{tabular}[c]{@{}l@{}}avg.\\ rank\end{tabular}} \\
\multicolumn{1}{c}{\textbf{metrics}} & \multicolumn{1}{c}{\textbf{cor.}} & \multicolumn{1}{c}{\textbf{p-v.}} & \multicolumn{1}{c}{\textbf{cor.}} & \multicolumn{1}{c}{\textbf{p-v.}} & \multicolumn{1}{c}{\textbf{cor.}} & \multicolumn{1}{c}{\textbf{p-v.}} & \multicolumn{1}{c}{\textbf{cor.}} & \multicolumn{1}{c}{\textbf{p-v.}} & \multicolumn{1}{c}{\textbf{cor.}} & \multicolumn{1}{c}{\textbf{p-v.}} &  &  \\ \midrule
\textbf{S-Bleu} & 0.50 & 0.0 & 0.47 & 0.0 & 0.59 & 0.0 & 0.58 & 0.0 & 0.68 & 0.0 & 0.57 & 5.8 \\
\textbf{R-Bleu} & -- & -- & 0.27 & 0.0 & 0.30 & 0.0 & -- & -- & -- & -- & - &  \\
\textbf{S-Meteor} & 0.49 & 0.0 & 0.48 & 0.0 & 0.61 & 0.0 & 0.57 & 0.0 & 0.64 & 0.0 & 0.56 & 6.1 \\
\textbf{R-Meteor} & -- & -- & 0.34 & 0.0 & 0.26 & 0.0 & -- & -- & -- & -- & - &  \\
\textbf{S-Bertscore} & \textbf{0.53} & 0.0 & {\ul 0.80} & 0.0 & \textbf{0.70} & 0.0 & {\ul 0.66} & 0.0 & {\ul0.78} & 0.0 & \textbf{0.69} & \textbf{1.7} \\
\textbf{R-Bertscore} & -- & -- & 0.51 & 0.0 & 0.38 & 0.0 & -- & -- & -- & -- & - &  \\
\textbf{S-Bleurt} & {\ul 0.52} & 0.0 & {\ul 0.80} & 0.0 & 0.60 & 0.0 & \textbf{0.70} & 0.0 & \textbf{0.80} & 0.0 & {\ul 0.68} & {\ul 2.3} \\
\textbf{R-Bleurt} & -- & -- & 0.59 & 0.0 & -0.05 & 0.13 & -- & -- & -- & -- & - &  \\
\textbf{S-Cosine} & 0.51 & 0.0 & 0.69 & 0.0 & {\ul 0.62} & 0.0 & 0.61 & 0.0 & 0.65 & 0.0 & 0.62 & 4.4 \\
\textbf{R-Cosine} & -- & -- & 0.40 & 0.0 & 0.29 & 0.0 & -- & -- & -- & -- & - & \\ \midrule
\textbf{QuestEval} & 0.23 & 0.0 & 0.25 & 0.0 & 0.49 & 0.0 & 0.47 & 0.0 & 0.62 & 0.0 & 0.41 & 9.0 \\
\textbf{LLaMa3} & 0.36 & 0.0 & \textbf{0.84} & 0.0 & {\ul{0.62}} & 0.0 & 0.61 & 0.0 &  0.76 & 0.0 & 0.64 & 3.6 \\
\textbf{our (3b)} & 0.49 & 0.0 & 0.73 & 0.0 & 0.54 & 0.0 & 0.53 & 0.0 & 0.7 & 0.0 & 0.60 & 5.8 \\
\textbf{our (8b)} & 0.48 & 0.0 & 0.73 & 0.0 & 0.52 & 0.0 & 0.53 & 0.0 & 0.7 & 0.0 & 0.59 & 6.3 \\  \bottomrule
\end{tabular}
\caption{Pearson correlation on human evaluation on system output. `R-': reference-based. `S-': source-based.}
\label{tab:sys}
\end{table*}



\begin{table}%[]
\centering
\fontsize{11pt}{11pt}\selectfont
\begin{tabular}{llllll}
\toprule
\multicolumn{1}{c}{\textbf{task}} & \multicolumn{1}{c}{\textbf{Lai}} & \multicolumn{1}{c}{\textbf{Zei.}} & \multicolumn{1}{c}{\textbf{Scia.}} & \textbf{} & \textbf{} \\ 
\multicolumn{1}{c}{\textbf{metrics}} & \multicolumn{1}{c}{\textbf{cor.}} & \multicolumn{1}{c}{\textbf{cor.}} & \multicolumn{1}{c}{\textbf{cor.}} & \textbf{avg.} & \textbf{\begin{tabular}[c]{@{}l@{}}avg.\\ rank\end{tabular}} \\ \midrule
\textbf{S-Bleu} & 0.40 & 0.40 & 0.19* & 0.33 & 7.67 \\
\textbf{S-Meteor} & 0.41 & 0.42 & 0.16* & 0.33 & 7.33 \\
\textbf{S-BertS.} & {\ul0.58} & 0.47 & 0.31 & 0.45 & 3.67 \\
\textbf{S-Bleurt} & 0.45 & {\ul 0.54} & {\ul 0.37} & 0.45 & {\ul 3.33} \\
\textbf{S-Cosine} & 0.56 & 0.52 & 0.3 & {\ul 0.46} & {\ul 3.33} \\ \midrule
\textbf{QuestE.} & 0.27 & 0.35 & 0.06* & 0.23 & 9.00 \\
\textbf{LlaMA3} & \textbf{0.6} & \textbf{0.67} & \textbf{0.51} & \textbf{0.59} & \textbf{1.0} \\
\textbf{Our (3b)} & 0.51 & 0.49 & 0.23* & 0.39 & 4.83 \\
\textbf{Our (8b)} & 0.52 & 0.49 & 0.22* & 0.43 & 4.83 \\ \bottomrule
\end{tabular}
\caption{Pearson correlation on human ratings on reference output. *not significant; we cannot reject the null hypothesis of zero correlation}
\label{tab:ref}
\end{table}


\begin{table*}%[]
\centering
\fontsize{11pt}{11pt}\selectfont
\begin{tabular}{lllllllll}
\toprule
\textbf{task} & \multicolumn{1}{c}{\textbf{ALL}} & \multicolumn{1}{c}{\textbf{sentiment}} & \multicolumn{1}{c}{\textbf{detoxify}} & \multicolumn{1}{c}{\textbf{catchy}} & \multicolumn{1}{c}{\textbf{polite}} & \multicolumn{1}{c}{\textbf{persuasive}} & \multicolumn{1}{c}{\textbf{formal}} & \textbf{\begin{tabular}[c]{@{}l@{}}avg. \\ rank\end{tabular}} \\
\textbf{metrics} & \multicolumn{1}{c}{\textbf{cor.}} & \multicolumn{1}{c}{\textbf{cor.}} & \multicolumn{1}{c}{\textbf{cor.}} & \multicolumn{1}{c}{\textbf{cor.}} & \multicolumn{1}{c}{\textbf{cor.}} & \multicolumn{1}{c}{\textbf{cor.}} & \multicolumn{1}{c}{\textbf{cor.}} &  \\ \midrule
\textbf{S-Bleu} & -0.17 & -0.82 & -0.45 & -0.12* & -0.1* & -0.05 & -0.21 & 8.42 \\
\textbf{R-Bleu} & - & -0.5 & -0.45 &  &  &  &  &  \\
\textbf{S-Meteor} & -0.07* & -0.55 & -0.4 & -0.01* & 0.1* & -0.16 & -0.04* & 7.67 \\
\textbf{R-Meteor} & - & -0.17* & -0.39 & - & - & - & - & - \\
\textbf{S-BertScore} & 0.11 & -0.38 & -0.07* & -0.17* & 0.28 & 0.12 & 0.25 & 6.0 \\
\textbf{R-BertScore} & - & -0.02* & -0.21* & - & - & - & - & - \\
\textbf{S-Bleurt} & 0.29 & 0.05* & 0.45 & 0.06* & 0.29 & 0.23 & 0.46 & 4.2 \\
\textbf{R-Bleurt} & - &  0.21 & 0.38 & - & - & - & - & - \\
\textbf{S-Cosine} & 0.01* & -0.5 & -0.13* & -0.19* & 0.05* & -0.05* & 0.15* & 7.42 \\
\textbf{R-Cosine} & - & -0.11* & -0.16* & - & - & - & - & - \\ \midrule
\textbf{QuestEval} & 0.21 & {\ul{0.29}} & 0.23 & 0.37 & 0.19* & 0.35 & 0.14* & 4.67 \\
\textbf{LlaMA3} & \textbf{0.82} & \textbf{0.80} & \textbf{0.72} & \textbf{0.84} & \textbf{0.84} & \textbf{0.90} & \textbf{0.88} & \textbf{1.00} \\
\textbf{Our (3b)} & 0.47 & -0.11* & 0.37 & 0.61 & 0.53 & 0.54 & 0.66 & 3.5 \\
\textbf{Our (8b)} & {\ul{0.57}} & 0.09* & {\ul 0.49} & {\ul 0.72} & {\ul 0.64} & {\ul 0.62} & {\ul 0.67} & {\ul 2.17} \\ \bottomrule
\end{tabular}
\caption{Pearson correlation on human ratings on our constructed test set. 'R-': reference-based. 'S-': source-based. *not significant; we cannot reject the null hypothesis of zero correlation}
\label{tab:con}
\end{table*}

\section{Results}
We benchmark the different metrics on the different datasets using correlation to human judgement. For content preservation, we show results split on data with system output, reference output and our constructed test set: we show that the data source for evaluation leads to different conclusions on the metrics. In addition, we examine whether the metrics can rank style transfer systems similar to humans. On style strength, we likewise show correlations between human judgment and zero-shot evaluation approaches. When applicable, we summarize results by reporting the average correlation. And the average ranking of the metric per dataset (by ranking which metric obtains the highest correlation to human judgement per dataset). 

\subsection{Content preservation}
\paragraph{How do data sources affect the conclusion on best metric?}
The conclusions about the metrics' performance change radically depending on whether we use system output data, reference output, or our constructed test set. Ideally, a good metric correlates highly with humans on any data source. Ideally, for meta-evaluation, a metric should correlate consistently across all data sources, but the following shows that the correlations indicate different things, and the conclusion on the best metric should be drawn carefully.

Looking at the metrics correlations with humans on the data source with system output (Table~\ref{tab:sys}), we see a relatively high correlation for many of the metrics on many tasks. The overall best metrics are S-BertScore and S-BLEURT (avg+avg rank). We see no notable difference in our method of using the 3B or 8B model as the backbone.

Examining the average correlations based on data with reference output (Table~\ref{tab:ref}), now the zero-shoot prompting with LlaMA3 70B is the best-performing approach ($0.59$ avg). Tied for second place are source-based cosine embedding ($0.46$ avg), BLEURT ($0.45$ avg) and BertScore ($0.45$ avg). Our method follows on a 5. place: here, the 8b version (($0.43$ avg)) shows a bit stronger results than 3b ($0.39$ avg). The fact that the conclusions change, whether looking at reference or system output, confirms the observations made by \citet{scialom-etal-2021-questeval} on simplicity transfer.   

Now consider the results on our test set (Table~\ref{tab:con}): Several metrics show low or no correlation; we even see a significantly negative correlation for some metrics on ALL (BLEU) and for specific subparts of our test set for BLEU, Meteor, BertScore, Cosine. On the other end, LlaMA3 70B is again performing best, showing strong results ($0.82$ in ALL). The runner-up is now our 8B method, with a gap to the 3B version ($0.57$ vs $0.47$ in ALL). Note our method still shows zero correlation for the sentiment task. After, ranks BLEURT ($0.29$), QuestEval ($0.21$), BertScore ($0.11$), Cosine ($0.01$).  

On our test set, we find that some metrics that correlate relatively well on the other datasets, now exhibit low correlation. Hence, with our test set, we can now support the logical reasoning with data evidence: Evaluation of content preservation for style transfer needs to take the style shift into account. This conclusion could not be drawn using the existing data sources: We hypothesise that for the data with system-based output, successful output happens to be very similar to the source sentence and vice versa, and reference-based output might not contain server mistakes as they are gold references. Thus, none of the existing data sources tests the limits of the metrics.  


\paragraph{How do reference-based metrics compare to source-based ones?} Reference-based metrics show a lower correlation than the source-based counterpart for all metrics on both datasets with ratings on references (Table~\ref{tab:sys}). As discussed previously, reference-based metrics for style transfer have the drawback that many different good solutions on a rewrite might exist and not only one similar to a reference.


\paragraph{How well can the metrics rank the performance of style transfer methods?}
We compare the metrics' ability to judge the best style transfer methods w.r.t. the human annotations: Several of the data sources contain samples from different style transfer systems. In order to use metrics to assess the quality of the style transfer system, metrics should correctly find the best-performing system. Hence, we evaluate whether the metrics for content preservation provide the same system ranking as human evaluators. We take the mean of the score for every output on each system and the mean of the human annotations; we compare the systems using the Kendall's Tau correlation. 

We find only the evaluation using the dataset Mir, Lai, and Ziegen to result in significant correlations, probably because of sparsity in a number of system tests (App.~\ref{app:dataset}). Our method (8b) is the only metric providing a perfect ranking of the style transfer system on the Lai data, and Llama3 70B the only one on the Ziegen data. Results in App.~\ref{app:results}. 


\subsection{Style strength results}
%Evaluating style strengths is a challenging task. 
Llama3 70B shows better overall results than our method. However, our method scores higher than Llama3 70B on 2 out of 6 datasets, but it also exhibits zero correlation on one task (Table~\ref{tab:styleresults}).%More work i s needed on evaluating style strengths. 
 
\begin{table}%[]
\fontsize{11pt}{11pt}\selectfont
\begin{tabular}{lccc}
\toprule
\multicolumn{1}{c}{\textbf{}} & \textbf{LlaMA3} & \textbf{Our (3b)} & \textbf{Our (8b)} \\ \midrule
\textbf{Mir} & 0.46 & 0.54 & \textbf{0.57} \\
\textbf{Lai} & \textbf{0.57} & 0.18 & 0.19 \\
\textbf{Ziegen.} & 0.25 & 0.27 & \textbf{0.32} \\
\textbf{Alva-M.} & \textbf{0.59} & 0.03* & 0.02* \\
\textbf{Scialom} & \textbf{0.62} & 0.45 & 0.44 \\
\textbf{\begin{tabular}[c]{@{}l@{}}Our Test\end{tabular}} & \textbf{0.63} & 0.46 & 0.48 \\ \bottomrule
\end{tabular}
\caption{Style strength: Pearson correlation to human ratings. *not significant; we cannot reject the null hypothesis of zero corelation}
\label{tab:styleresults}
\end{table}

\subsection{Ablation}
We conduct several runs of the methods using LLMs with variations in instructions/prompts (App.~\ref{app:method}). We observe that the lower the correlation on a task, the higher the variation between the different runs. For our method, we only observe low variance between the runs.
None of the variations leads to different conclusions of the meta-evaluation. Results in App.~\ref{app:results}.
\section{Discussion of Assumptions}\label{sec:discussion}
In this paper, we have made several assumptions for the sake of clarity and simplicity. In this section, we discuss the rationale behind these assumptions, the extent to which these assumptions hold in practice, and the consequences for our protocol when these assumptions hold.

\subsection{Assumptions on the Demand}

There are two simplifying assumptions we make about the demand. First, we assume the demand at any time is relatively small compared to the channel capacities. Second, we take the demand to be constant over time. We elaborate upon both these points below.

\paragraph{Small demands} The assumption that demands are small relative to channel capacities is made precise in \eqref{eq:large_capacity_assumption}. This assumption simplifies two major aspects of our protocol. First, it largely removes congestion from consideration. In \eqref{eq:primal_problem}, there is no constraint ensuring that total flow in both directions stays below capacity--this is always met. Consequently, there is no Lagrange multiplier for congestion and no congestion pricing; only imbalance penalties apply. In contrast, protocols in \cite{sivaraman2020high, varma2021throughput, wang2024fence} include congestion fees due to explicit congestion constraints. Second, the bound \eqref{eq:large_capacity_assumption} ensures that as long as channels remain balanced, the network can always meet demand, no matter how the demand is routed. Since channels can rebalance when necessary, they never drop transactions. This allows prices and flows to adjust as per the equations in \eqref{eq:algorithm}, which makes it easier to prove the protocol's convergence guarantees. This also preserves the key property that a channel's price remains proportional to net money flow through it.

In practice, payment channel networks are used most often for micro-payments, for which on-chain transactions are prohibitively expensive; large transactions typically take place directly on the blockchain. For example, according to \cite{river2023lightning}, the average channel capacity is roughly $0.1$ BTC ($5,000$ BTC distributed over $50,000$ channels), while the average transaction amount is less than $0.0004$ BTC ($44.7k$ satoshis). Thus, the small demand assumption is not too unrealistic. Additionally, the occasional large transaction can be treated as a sequence of smaller transactions by breaking it into packets and executing each packet serially (as done by \cite{sivaraman2020high}).
Lastly, a good path discovery process that favors large capacity channels over small capacity ones can help ensure that the bound in \eqref{eq:large_capacity_assumption} holds.

\paragraph{Constant demands} 
In this work, we assume that any transacting pair of nodes have a steady transaction demand between them (see Section \ref{sec:transaction_requests}). Making this assumption is necessary to obtain the kind of guarantees that we have presented in this paper. Unless the demand is steady, it is unreasonable to expect that the flows converge to a steady value. Weaker assumptions on the demand lead to weaker guarantees. For example, with the more general setting of stochastic, but i.i.d. demand between any two nodes, \cite{varma2021throughput} shows that the channel queue lengths are bounded in expectation. If the demand can be arbitrary, then it is very hard to get any meaningful performance guarantees; \cite{wang2024fence} shows that even for a single bidirectional channel, the competitive ratio is infinite. Indeed, because a PCN is a decentralized system and decisions must be made based on local information alone, it is difficult for the network to find the optimal detailed balance flow at every time step with a time-varying demand.  With a steady demand, the network can discover the optimal flows in a reasonably short time, as our work shows.

We view the constant demand assumption as an approximation for a more general demand process that could be piece-wise constant, stochastic, or both (see simulations in Figure \ref{fig:five_nodes_variable_demand}).
We believe it should be possible to merge ideas from our work and \cite{varma2021throughput} to provide guarantees in a setting with random demands with arbitrary means. We leave this for future work. In addition, our work suggests that a reasonable method of handling stochastic demands is to queue the transaction requests \textit{at the source node} itself. This queuing action should be viewed in conjunction with flow-control. Indeed, a temporarily high unidirectional demand would raise prices for the sender, incentivizing the sender to stop sending the transactions. If the sender queues the transactions, they can send them later when prices drop. This form of queuing does not require any overhaul of the basic PCN infrastructure and is therefore simpler to implement than per-channel queues as suggested by \cite{sivaraman2020high} and \cite{varma2021throughput}.

\subsection{The Incentive of Channels}
The actions of the channels as prescribed by the DEBT control protocol can be summarized as follows. Channels adjust their prices in proportion to the net flow through them. They rebalance themselves whenever necessary and execute any transaction request that has been made of them. We discuss both these aspects below.

\paragraph{On Prices}
In this work, the exclusive role of channel prices is to ensure that the flows through each channel remains balanced. In practice, it would be important to include other components in a channel's price/fee as well: a congestion price  and an incentive price. The congestion price, as suggested by \cite{varma2021throughput}, would depend on the total flow of transactions through the channel, and would incentivize nodes to balance the load over different paths. The incentive price, which is commonly used in practice \cite{river2023lightning}, is necessary to provide channels with an incentive to serve as an intermediary for different channels. In practice, we expect both these components to be smaller than the imbalance price. Consequently, we expect the behavior of our protocol to be similar to our theoretical results even with these additional prices.

A key aspect of our protocol is that channel fees are allowed to be negative. Although the original Lightning network whitepaper \cite{poon2016bitcoin} suggests that negative channel prices may be a good solution to promote rebalancing, the idea of negative prices in not very popular in the literature. To our knowledge, the only prior work with this feature is \cite{varma2021throughput}. Indeed, in papers such as \cite{van2021merchant} and \cite{wang2024fence}, the price function is explicitly modified such that the channel price is never negative. The results of our paper show the benefits of negative prices. For one, in steady state, equal flows in both directions ensure that a channel doesn't loose any money (the other price components mentioned above ensure that the channel will only gain money). More importantly, negative prices are important to ensure that the protocol selectively stifles acyclic flows while allowing circulations to flow. Indeed, in the example of Section \ref{sec:flow_control_example}, the flows between nodes $A$ and $C$ are left on only because the large positive price over one channel is canceled by the corresponding negative price over the other channel, leading to a net zero price.

Lastly, observe that in the DEBT control protocol, the price charged by a channel does not depend on its capacity. This is a natural consequence of the price being the Lagrange multiplier for the net-zero flow constraint, which also does not depend on the channel capacity. In contrast, in many other works, the imbalance price is normalized by the channel capacity \cite{ren2018optimal, lin2020funds, wang2024fence}; this is shown to work well in practice. The rationale for such a price structure is explained well in \cite{wang2024fence}, where this fee is derived with the aim of always maintaining some balance (liquidity) at each end of every channel. This is a reasonable aim if a channel is to never rebalance itself; the experiments of the aforementioned papers are conducted in such a regime. In this work, however, we allow the channels to rebalance themselves a few times in order to settle on a detailed balance flow. This is because our focus is on the long-term steady state performance of the protocol. This difference in perspective also shows up in how the price depends on the channel imbalance. \cite{lin2020funds} and \cite{wang2024fence} advocate for strictly convex prices whereas this work and \cite{varma2021throughput} propose linear prices.

\paragraph{On Rebalancing} 
Recall that the DEBT control protocol ensures that the flows in the network converge to a detailed balance flow, which can be sustained perpetually without any rebalancing. However, during the transient phase (before convergence), channels may have to perform on-chain rebalancing a few times. Since rebalancing is an expensive operation, it is worthwhile discussing methods by which channels can reduce the extent of rebalancing. One option for the channels to reduce the extent of rebalancing is to increase their capacity; however, this comes at the cost of locking in more capital. Each channel can decide for itself the optimum amount of capital to lock in. Another option, which we discuss in Section \ref{sec:five_node}, is for channels to increase the rate $\gamma$ at which they adjust prices. 

Ultimately, whether or not it is beneficial for a channel to rebalance depends on the time-horizon under consideration. Our protocol is based on the assumption that the demand remains steady for a long period of time. If this is indeed the case, it would be worthwhile for a channel to rebalance itself as it can make up this cost through the incentive fees gained from the flow of transactions through it in steady state. If a channel chooses not to rebalance itself, however, there is a risk of being trapped in a deadlock, which is suboptimal for not only the nodes but also the channel.

\section{Conclusion}
This work presents DEBT control: a protocol for payment channel networks that uses source routing and flow control based on channel prices. The protocol is derived by posing a network utility maximization problem and analyzing its dual minimization. It is shown that under steady demands, the protocol guides the network to an optimal, sustainable point. Simulations show its robustness to demand variations. The work demonstrates that simple protocols with strong theoretical guarantees are possible for PCNs and we hope it inspires further theoretical research in this direction.
\section{Threats to Validity}
\label{sec:ttv}

%\EI{TTV on using only the CVEs from ProjectKB vs all the CVEs for those projects mined from NVD: this would never reduce the false matches, but, in the best case, increase the true matches. Besides, this could backfire... not sure}

%\subsection{Threats to Internal Validity}

%\CB{As I remember, the test can be predicted to be linked to cve $A^*$. The truly associated cve of the test is $A$---not existing in the training dataset, but $A$ and $A^*$ share the similarity such as vuln type, project context, etc.\footnote{For example, in the case of CVE-2018-1000865 and CVE-2018-1000866 in \url{https://docs.google.com/spreadsheets/d/1ZAZ5GWLGZFmtFbuQyIS8k0n_RgZbAW2A2WckpYV5SLE/edit}}}
For the \matching task, we considered the performance of the \finder and \linker models in their tasks independently before integrating them.
%, emulating a bottom-up integration testing scenario (explained in Section~\ref{subsubsec:overview}).
Yet, there was no guarantee that the selected configurations would work well once integrated.
%does not re-use the exact models from their sessions but re-use the best configuration observed.
%and re-train them on a different dataset (the motivations are given in Section~\ref{subsubsec:overview}).
%Therefore, the final behavior might be different than expected.
We mitigated this threat by running a new train-test session to assess the whole integrated model for the actual \matching task.
The final answer to the in-vitro analysis on \rqTwo is only given by the results achieved by the integrated model.
%When evaluating the \globalModel model, we only searched its optimal configuration without tuning the \finder and \linker models again, as this would have required searching among 13 hyper-parameters (Table~\ref{tab:hyperparams}), expanding the search space too much.

Due to computational constraints, we could not thoroughly analyze the impact of many design aspects in the in-vitro analyses.
%, so we prioritized only those aspects that could provide some benefits.
We treated some aspects as factors, like data augmentation, to be evaluated at test time, while others were treated as hyper-parameters to be optimized at development time.
%We did not employ statistical tests to compare the various groups due to the limited number of configurations belonging to each.
%In Section~\ref{sec:results}, we observed how certain configuration values might provide some benefits, e.g., data augmentation.
%helped the \linker model more than \finder model.
%Yet, we did not employ any statistical tests and just compared the values obtained with the aggregated values.
%Despite not employing statistical tests, the analyses conducted aimed to find the most suitable configuration to prepare \vuteco for real-world usage, experimented in the two in-vivo analyses.
%\EI{The fact that we used ``Learn'' could be mentioned}

%\TEMP{While evaluating \vuteco in-vitro, we moved the classification threshold of all models from the default $0.5$ to a different value that maximized the F0.5 scores.
%However, the differences were marginal on a large scale, so we opted to keep the default thresholds.
%The performance of the tuned configurations is in our online appendix~\cite{appendix}.}

For the \matching task, \vuteco accepts a natural language description of the vulnerability.
Hence, \vuteco accepts any text describing the issue, also from other sources like bug reports~\cite{zhou2017automated,le2019automated,bao2022v,nguyen2023multi,iannone:tosem2024:exploit}.
We opted to experiment with CVE, as it is the most comprehensive catalog for obtaining such information and can be easily queried.
%\CB{I think this makes sense, even the CVE description sometimes seems not to be valid, for example, in this case it just contains texts about the CVE duplication \footnote{\url{https://nvd.nist.gov/vuln/detail/CVE-2015-8581}}.}

%\EI{Overfitting risk: Vuteco might mistake two similar vulnerabilities of the same project if they share text. Similarly, the prediction can go well if similar vulnerabilities fall into the training. Example: CVE-2018-1000865 and CVE-2018-1000866. However, we didn't do anything to mitigate this risk... cross-validation would have been good.}

%\EI{TTV for the baselines, they might be too naive. Besides, we can say that other baseline might exists, and in our appendix we did some others}

%\subsection{Threats to External Validity}

\vuteco has been trained and tested on the \JUnit test cases in \VulforJ.
Therefore, all the results observed cannot be generalized to other programming languages, other testing frameworks (like \textsc{TestNG}), or vulnerability types that are not found in \Java code, like memory-related vulnerabilities.
We chose \Java mainly due to the availability of \VulforJ containing examples of witnessing tests to train and validate \vuteco.
At the same time, we remark that \Java is still a relevant language to analyze from a security perspective as it keeps exhibiting vulnerabilities for a long time~\cite{veracode}.

Despite being the only catalog of its kind, the limited number of witnessing tests in \VulforJ does not allow models to learn many patterns that can be generalized to different contexts.
It also lacks diversified examples, e.g., it has eight examples of tests witnessing CWE-835 (Infinite Loop), but just one witnessing CWE-78 (OS Command Injection)~\cite{bui:msr2022:vul4j}.
We partially handled this by trying to augment the training sets involved during the experimentation, though without the hoped effect.
\vuteco was born to contribute to feeding such knowledge bases with new examples, which in turn could be used to re-train \vuteco and return more results.

%While the in-vitro analysis validated \vuteco on a subset of witnessing tests paired with vulnerability descriptions from real-world \Java projects in \VulforJ, the in-vivo showed its applicability on further real-world projects, extending the setting used for the in-vitro analyses.
%Both support \vuteco's ecological validity.

%\subsection{Threats to Construct Validity}

\vuteco flags witnessing tests based on a static approach: No test cases are executed, and no projects are built.
For this, \vuteco cannot provide a definite answer about whether the tests will trigger the matched vulnerabilities.
The dynamic assessment has been taken outside the scope of this work due to technical difficulties in building numerous projects in a reasonable time, as encountered by the authors of \VulforJ~\cite{bui:msr2022:vul4j}.
%\vuteco's goal is to find candidate vulnerability-witnessing tests whose actual behavior must be assessed in separate sessions. 
%\EI{Additional reasons of why we don't do this: so far we have no knowledge of how witnessing tests appear, since we can't have good empirical studies without data. If we had to build and run the tests automatically and on large scale, most would fail (as seen in Vul4J), so losing many precious candidates for expanding our knowledge on the matter. Of course, in the future this can be addressed.}

%We defined \vuteco's general architecture based on internal trials on the development sets used in the in-vivo evaluations.
%The design choices made have certainly influenced its performance.
%We acknowledge that numerous other hyper-parameters and design decisions could have led to better versions of \vuteco; however, their investigation would have required running thousands of train-test sessions and incurring huge computation costs.
%In this work, we focused on the main factors we believed to be impactful, aiming to define a first working tool for collecting vulnerability-witnessing tests.
%The wide range of possible designs makes it difficult to identify the right solution without running thousands of train-test sessions and incurring huge computation costs.
%We followed similar design choices adopted in studies making classification of test cases~\cite{fatima:tse2023:flakify,akli:ast2023:flakycat}.
%Future improvements can be made to enhance \vuteco's retrieval capabilities for this stable point.
%
%Besides, we know many other aspects could influence the model performance, like employing a weighted loss function or different network architectures, requiring deeper experimentation.

%In \rqOne, we relied on the traditional metrics adopted in information retrieval to assess the performance of the experimented models and determine the most suitable for our purpose.
%To mitigate the risk of selecting models with apparently good performance but without practical usefulness, we also considered the absolute numbers of true and false positives and the Inspection Ratio.

%\EI{TTV regarding the limit to 512 tokens... we partially handled it by removing comments before the test method and one-lining, but we still had X\% cases in which the input was longer than 512 tokens. We accepted the risk and just truncated, and there are no approaches that can identify the relevant part in a test (which can be interesting actually to do)}

%\subsection{Threat to Conclusion Validity}

Our work draws heavily from the literature on semiparametric inference and double machine learning~\citep{robins1994estimation,robins1995semiparametric,tsiatis2006semiparametric,chernozhukov2018double}. In particular, our estimator is an optimal combination of several Augmented Inverse Probability Weighting~(\aipw) estimators, whose outcome regressions are replaced with foundation models. Importantly, the standard $\aipw$ estimator, which relies on an outcome regression estimated using experimental data alone, is also included in the combination. This approach allows \ours~to significantly reduce finite sample (and potentially asymptotic) variance while attaining the semiparametric \emph{efficiency bound}---the smallest asymptotic variance among all consistent and asymptotically normal estimators of the average treatment effect---even when the foundation models are arbitrarily biased.


\paragraph{Integrating foundation models}
Prediction-powered inference~(\ppi)~\citep{angelopoulos2023prediction} is a statistical framework that constructs valid confidence intervals using a small labeled dataset and a large unlabeled dataset imputed by a foundation model. $\ppi$ has been applied in various domains, including generalization of causal inferences~\citep{demirel24prediction}, large language model evaluation~\citep{fisch2024stratified,dorner2024limitsscalableevaluationfrontier}, and improving the efficiency of social science experiments~\citep{broskamixed,egami2024using}. However, unlike our approach, $\ppi$ requires access to an additional unlabeled dataset from the same distribution as the experimental sample, which may be as costly as labeled data. Recent work by \citet{poulet2025prediction} introduces 
Prediction-powered inference for clinical trials ($\ppct$), an adaptation of $\ppi$ to estimate  average treatment effects in randomized experiments without any additional  external data. $\ppct$ combines the difference in means estimator with an 
$\aipw$ estimator that integrates the same foundation model as the outcome regression for both treatment and control groups. However, our work differs in two key aspects:
(i) $\ppct$ integrates a single foundation model, and (ii) $\ppct$ does not include the standard $\aipw$ estimator with the outcome regression estimated from experimental data. As a result, $\ppct$ cannot achieve the efficiency bound unless the foundation model is almost surely equal to the underlying outcome regression. 


 



\paragraph{Integrating observational data} There is growing interest in augmenting randomized experiments with data from observational studies to improve statistical precision. One approach involves first testing whether the observational data is compatible with the experimental data~\citep{dahabreh2024using}---for instance, using a statistical test to assess if the mean of the outcome conditional on the covariates is invariant across studies \cite{luedtke2019omnibus,hussain2023falsification,de2024detecting}—and then combining the datasets to improve precision, if the test does not reject. These tests, however, have low statistical power, especially when the experimental sample size is small, which is precisely when leveraging observational data would be most beneficial. To overcome this, a recent line of work integrates a prognostic score estimated from observational data as a covariate when estimating the outcome regression~\citep{schuler2022increasing,liao2023prognostic}. However, increasing the dimensionality of the problem---by adding an additional covariate---can increase estimation error and inflate the finite sample variance. Finally, the work most closely related to ours is \citet{karlsson2024robust}, that integrates an outcome regression estimated from observational data into the \aipw~estimator. In contrast, our approach is not constrained by the availability of well-structured observational data, since it leverages black-box foundation models trained on external data sources.
\section{Conclusion}
In this work, we propose a simple yet effective approach, called SMILE, for graph few-shot learning with fewer tasks. Specifically, we introduce a novel dual-level mixup strategy, including within-task and across-task mixup, for enriching the diversity of nodes within each task and the diversity of tasks. Also, we incorporate the degree-based prior information to learn expressive node embeddings. Theoretically, we prove that SMILE effectively enhances the model's generalization performance. Empirically, we conduct extensive experiments on multiple benchmarks and the results suggest that SMILE significantly outperforms other baselines, including both in-domain and cross-domain few-shot settings.

%ACM
%\begin{acks}
%This work was partially supported by EU-funded project Sec4AI4Sec (grant no. 101120393)
%\end{acks}

%IEEE
\section*{Acknowledgment}
\ifthenelse{\boolean{peerreview}}
{Omitted for peer review.}
{This work was partially supported by EU-funded project Sec4AI4Sec (grant no. 101120393).}

%\bibliographystyle{ACM-Reference-Format}
\bibliographystyle{ieeetr}
\bibliography{biblio}

\end{document}
\endinput
