\usepackage{bm} % bold math fonts, e.g. $\bm \Sigma$ will give a bold-face $\Sigma$
\usepackage{url}
% \usepackage{cite}
% \usepackage{amssymb}
\usepackage{amsmath}
\usepackage{amsfonts}
\usepackage{amsthm}
\usepackage{algorithmic}
\usepackage{graphicx}
\usepackage{textcomp}
\usepackage{xcolor}
\usepackage{soul} %for strikeout
\usepackage{multirow}
\usepackage{array}
\usepackage{booktabs}

%\usepackage{enumerate}   
\def\BibTeX{{\rm B\kern-.05em{\sc i\kern-.025em b}\kern-.08em
    T\kern-.1667em\lower.7ex\hbox{E}\kern-.125emX}}

\usepackage{hyperref}
\setlength{\marginparwidth}{2cm} % for todonotes
\usepackage[colorinlistoftodos]{todonotes}

% http://paultaylor.eu/diagrams/
% https://www.jmilne.org/not/Mdiagrams.pdf
% \usepackage[small,nohug,heads=vee]{diagrams}
% \diagramstyle[labelstyle=\scriptstyle]

\usepackage[ruled, noend]{algorithm2e}
% \usepackage{enumitem}
% \usepackage[shortlabels]{enumitem}

\usepackage{tikz}

%%% package microtype recommended by Paul Beame who says:
%%%%% try including [it] in any of your latex documents.  You will find that
%%%%% not only is the result more compact without changing any spacing
%%%%% parameters, the number of overfull/underfull warnings drops, and it
%%%%% simply looks a lot better!
%%%%%
%%%%% Why it works:  Latex is not known for the beauty of its typography.  The
%%%%% biggest issue is that the amount of visible white space between words
%%%%% varies quite a lot line-to-line in a paragraph since inter-word spacing
%%%%% is the only thing that will stretch/shrink.   The microtype package also
%%%%% makes imperceptible changes to the inter-letter spacing within words
%%%%% and, with all those extra degrees of freedom, it can do a much better
%%%%% job of laying out paragraphs.
\usepackage{microtype}

\usepackage{relsize}
%% Eg. \mathlarger{\mathlarger{\mathlarger{(x+y=z)}}}

\usepackage{lipsum}  


\newcommand{\yell}[1]{{\color{red} \textbf{#1}}}
\newcommand{\highlight}[1]{{\color{blue}#1}}
\newcommand{\delete}[1]{{\color{red}\st{#1}}}
\newcommand{\nop}[1]{}

\newcommand{\gv}[1]{\ensuremath{\mbox{\boldmath$ #1 $}}}
\newcommand{\grad}[1]{\gv{\nabla} #1}
\newcommand{\norm}[1]{\|#1\|}
\newcommand{\set}[1]{\{#1\}}                    % Set (as in \set{1,2,3}).
\newcommand{\setof}[2]{\{{#1}\mid{#2}\}}        % Set (as in \setof{x}{x>0}).
\newcommand{\bigsetof}[2]{\left\{{#1}\mid{#2}\right\}}        % Set (as in \setof{x}{x>0}).
\newcommand{\bag}[1]{\{\hspace{-1mm}\{#1\}\hspace{-1mm}\}}                    % bag (as in \bag{1,2,3}).
\newcommand{\bagof}[2]{\{\hspace{-1mm}\{{#1}\mid{#2}\}\hspace{-1mm}\}}        % Set (as in \setof{x}{x>0}).
\newcommand{\pr}{\mathop{\textnormal{Pr}}}    % Probability
\newcommand{\E}{\mathop{\mathbb E}}    % Probability
\newcommand{\dom}{\textsf{Dom}}
\newcommand{\adom}{\textsf{ADom}}
\newcommand{\id}{\textsf{ID}}
\newcommand{\codom}{\textsf{CoDom}}
\newcommand{\one}{\bm 1}
\newcommand{\zero}{\bm 0}
\newcommand{\degree}{\texttt{deg}}
\newcommand{\sign}{\text{\sf sign}}
\newcommand{\lfp}{\text{\sf lfp}}
\newcommand{\lpfp}{\text{\sf lpfp}}
\newcommand{\pfp}{\text{\sf pfp}}

\newcommand{\arity}{\texttt{ar}}

\newcommand{\chris}[1]{\todo[inline,color=blue]{\textsf{#1} \hfill \textsc{--Chris.}}}
\newcommand{\mak}[1]{\todo[inline,color=green]{\textsf{#1} \hfill \textsc{--Mahmoud.}}}
\newcommand{\dano}[1]{\todo[inline,color=orange]{\textsf{#1} \hfill \textsc{--DanO.}}}
\newcommand{\dans}[1]{\todo[inline,color=yellow]{\textsf{#1} \hfill \textsc{--DanS.}}}
\newcommand{\haozhe}[1]{\todo[inline,color=cyan]{\textsf{#1} \hfill \textsc{--Haozhe.}}}

\newcommand{\inner}[1]{\langle #1 \rangle}
%%
\newcommand{\LB}{\textsf{LogicBlox}}
\newcommand{\faq}{\textsf{FAQ}}
\newcommand{\calC}{\mathcal C}
\newcommand{\calX}{\mathcal X}
\newcommand{\calH}{\mathcal H}
\newcommand{\calV}{\mathcal V}
\newcommand{\calE}{\mathcal E}
\newcommand{\calD}{\mathcal D}
\newcommand{\calW}{\mathcal W}
\newcommand{\calF}{\mathcal F}
\newcommand{\calT}{\mathcal T}

%  \theoremstyle{plain}                   % default
\newtheorem{thm}{Theorem}[section]
\newtheorem{lmm}[thm]{Lemma}
\newtheorem{prop}[thm]{Proposition}
\newtheorem{cor}[thm]{Corollary}


% \theoremstyle{definition}              % Examples and all
\newtheorem{pbm}{Open Problem}
\newtheorem{opm}{Question}
\newtheorem{conj}[thm]{Conjecture}
\newtheorem{ex}[thm]{Example}
\newtheorem{exer}{Exercise}
\newtheorem{defn}[thm]{Definition}
\newtheorem{alg}[thm]{Algorithm}
\newtheorem{rmk}[thm]{Remark}
\newtheorem{claim}{Claim}
\newtheorem{note}{Note}

\newcommand{\defeq}{\stackrel{\text{def}}{=}}
\newcommand{\mineq}{\stackrel{\text{min}}{=}}
\newcommand{\maxeq}{\stackrel{\text{max}}{=}}
\newcommand{\dleq}{\mbox{ :- }}
\newcommand{\mult}{\circ}

\newcommand{\B}{\mathbb B} % the Booleans
\newcommand{\Z}{\mathbb Z} % integers
\newcommand{\N}{\mathbb N} % the natural numbers
\newcommand{\Q}{\mathbb Q} % the rational numbers
\newcommand{\R}{\mathbb R} % the real numbers
\newcommand{\Rp}{{\mathbb R}_{\tiny +}} % the real numbers
%\newcommand{\C}{\mathbb C} % complex numbers
\newcommand{\D}{\mathbf D} % bold-face D, used for generic domain
% \newcommand{\U}{\mathbf U} % the universe


\newcommand{\floor}[1]{\lfloor{#1}\rfloor}
\newcommand{\ceil}[1]{\lceil{#1}\rceil}

\newcommand{\supp}{\texttt{supp}}
\newcommand{\cl}[1]{\overline{{#1}}}
\newcommand{\mvd}{\twoheadrightarrow}
\newcommand{\fd}{\rightarrow}

\newcommand{\conehull}{\mathbf{conhull}}
\newcommand{\bigjoin}{\mathlarger{\mathlarger{\mathlarger{\Join}}}}
% \newcommand{\bigtimes}{\mathlarger{\mathlarger{\mathlarger{\times}}}}

\newcommand{\ce}{\textsc{CE}\xspace}
\newcommand{\pce}{\textsc{PCE}\xspace}
\newcommand{\est}{\textsc{Est}}
\newcommand{\pest}{\textsc{PEst}}
\newcommand{\dsb}{\textsc{DSB}\xspace}
\newcommand{\cb}{\textsc{CB}\xspace}
\newcommand{\agm}{\textsc{AGM}\xspace}
\newcommand{\polyb}{\textsc{PolyB}\xspace}
\newcommand{\attrs}{\texttt{Attrs}}
\newcommand{\vars}{\texttt{Vars}}


\newcommand{\lp}[1]{||#1||}

\newcommand{\system}{\textsc{LpBound}\xspace}
\newcommand{\psql}{\textsc{Postgres}\xspace}
\newcommand{\duckdb}{\textsc{DuckDB}\xspace}
\newcommand{\dbx}{\textsc{DbX}\xspace}
\newcommand{\safebound}{\textsc{SafeBound}\xspace}
\newcommand{\neurocard}{\textsc{NeuroCard}\xspace}
\newcommand{\bayescard}{\textsc{BayesCard}\xspace}
\newcommand{\deepdb}{\textsc{DeepDB}\xspace}
\newcommand{\flatcard}{\textsc{Flat}\xspace}
\newcommand{\factorjoin}{\textsc{FactorJoin}\xspace}



\newcommand{\groupby}{\texttt{group-by}\xspace}
\newcommand{\openclosed}[1]{(#1]}
\newcommand{\closedopen}[1]{[#1)}
\newcommand{\maxdegree}{\texttt{max-degree}\xspace}

\newcommand{\lpbase}{$\texttt{LP}_{\text{base}}$\xspace}
\newcommand{\lptd}{$\texttt{LP}_{\text{TD}}$\xspace}
\newcommand{\lptdb}{$\texttt{LP}_{\text{Berge}}$\xspace}
\newcommand{\lpflow}{$\texttt{LP}_{\text{flow}}$\xspace}
\newcommand{\nodes}{\text{Nodes}}
\newcommand{\edges}{\text{Edges}}


\definecolor{light-gray}{gray}{0.7.2}
\definecolor{goodgreen}{rgb}{0.1, 0.5, 0.1}
\definecolor{burntorange}{rgb}{0.8, 0.33, 0.0}
\definecolor{BlueViolet}{HTML}{8A2BE2}

% Turn change highlights on/off:
% Change highlights on:
\newcommand{\change}[1]{{\color{blue} #1}}
% Change highlights off:
% \newcommand{\change}[1]{#1}

\newcommand{\revcomment}[3]{\item[#1] {\it ``#2''}\newline{\color{gray}\it #3}}
% \newcommand{\revcomment}[3]{}
\newcommand{\response}[1]{{\color{blue}REVISION: #1}}


% \newcommand{\metarev}[1]{{\color{blue}#1}}
% \newcommand{\revone}[1]{{\color{BlueViolet}#1}}
% \newcommand{\revtwo}[1]{{\color{goodgreen}#1}}
% \newcommand{\revthree}[1]{{\color{burntorange}#1}}


\newcommand{\metarev}[1]{{\color{black}#1}}
\newcommand{\revone}[1]{{\color{black}#1}}
\newcommand{\revtwo}[1]{{\color{black}#1}}
\newcommand{\revthree}[1]{{\color{black}#1}}

% Define theorem-like environments
\newtheorem{theorem}{Theorem}[section]
\newtheorem{lemma}[theorem]{Lemma}
\newtheorem{proposition}[theorem]{Proposition}
\newtheorem{corollary}[theorem]{Corollary}

% Define definition-like environments
\theoremstyle{definition}
\newtheorem{definition}[theorem]{Definition}
\newtheorem{example}[theorem]{Example}
\newtheorem{remark}[theorem]{Remark}
