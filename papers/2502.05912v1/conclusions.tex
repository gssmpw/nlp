\section{Conclusion and Future Work}
\label{sec:conclusions}

In this paper we introduced \system, a pessimistic cardinality estimator that uses $\ell_p$-norms of degree sequences of the join columns and information inequalities. \revone{The advantages of \system over the learned estimators such as \factorjoin, \bayescard, and \deepdb, \neurocard,  and \flatcard  are that it provides: strong, one-sided theoretical guarantees; low estimation time and error when applied to workloads not seen before; fast construction of the necessary statistics; and a rich query language support with (cyclic and acyclic) equality joins, equality and range predicates, and group-by variables. This language support goes significantly beyond the star or even acyclic queries supported by the competing estimators benchmarked in Section~\ref{sec:experiments}.} 

\revthree{While \system's estimation time is slightly larger than that of  traditional estimators, it can nevertheless have lower estimation errors and lead to significantly improved query performance: Fig.~\ref{fig:most-expensive-queries} shows that the runtime improvement can be up to 3000 seconds for some queries, while its estimation time is only a few milliseconds.}

\revone{There are two major limitations of \system, as introduced in this paper. First, it can non-trivially overestimate the cardinality of joins of highly miscalibrated relations. We introduced two optimizations in Sec.~\ref{sec:histograms} to mitigate this problem. Second, it does not yet support range (and theta) joins, complex and negated predicates, and nested queries. }\metarev{\system's flexible framework can in principle accommodate such query constructs, yet this is not immediate and deserves an in-depth treatment in future work.
}
\metarev{
For example, an arbitrary predicate could be accommodated using appropriate data structures that can identify the ranges of tuples that satisfy the  predicate and that can be adjusted to store norms on the degree sequences within such ranges. 
A LIKE predicate can be accommodated, for instance, using a 3-gram index to select ranges of tuples that satisfy the predicate, similar to SafeBound~\cite{SafeBound:SIGMOD23}. To guarantee that \system returns an upper bound on the true cardinality of the query, the returned ranges must include all matching tuples.

To support nested queries, \system needs to become compositional, i.e., to take $\ell_p$-norms on input relations and return upper bounds on $\ell_p$-norms on the query output. Given a nested query $Q$, \system needs to first compute upper bounds on $\ell_p$-norms on the relations representing the sub-queries of $Q$ and then use these bounds to estimate the cardinality of $Q$.

Future work also needs to address the efficient maintenance of  \system's estimation under data updates. The following three observations outline a practical approach to achieve this. 
First, the q-inequality $|Q| \leq \prod \left(\lp{\degree_R(V|U)}_p\right)^{w^*}$, where the product ranges over all available statistics constraints, holds with {\em the same weights $w^*$} even when the norms $\lp{\degree_R(V|U)}_p$ change. This means that we do not need to solve the LP after every data update to obtain a valid upper bound on the cardinality of a query $Q$. 
Second, the $\ell_p$-norms used by \system can be expressed in SQL (as for Experiment~\ref{sec:experiments-stats-compute-time}) and maintained efficiently under data updates using the view maintenance mechanism of the underlying database system, e.g.,  delta queries. 
Third, we envision the use of $\ell_p$-sketches~\cite{cormode2020small} for an efficient, albeit approximate, maintenance of the $\ell_p$-norms.
}

\nop{
Our long term goal is to develop a framework for two-side approximations on the true cardinality of a query output, i.e., both guaranteed lower and upper bounds.
}