\section{Support for Selection Predicates}
\label{sec:histograms}

\system can support arbitrary selection predicates on a relation. As long as we can provide $\ell_p$-norms on the degree sequences of the join columns for those tuples that satisfy the selection predicate, \system can use these norms in the statistics constraints. In the following, we discuss the case of equality and range predicates, and their conjunction and disjunction; \texttt{IN} and \texttt{LIKE} predicates can be accommodated using data structures like for \safebound~\cite{SafeBound:SIGMOD23}.

As data structures to support predicates, \system uses simple and effective adaptations of existing data structures in databases: Most Common Values (MCVs) and histograms. Yet instead of a count for each MCV or histogram bucket, \system keeps a set of $\ell_p$-norms on the degree sequences of the tuples for that MCV or histogram bucket. The simplicity and ubiquity of these data structures make \system easy to incorporate in database systems.

In the following, let a relation $R({\bf X},{\bf Y},A)$ with join attributes ${\bf X} = \{X_1,\ldots,X_n\}$, a predicate attribute $A$, and other attributes ${\bf Y}$.

\paragraph{Equality Predicate.} For each MCV $a$ of $A$, we compute $\ell_p$-norms for the full and simple degree sequences $\deg_R(*|X_i, A=a)$ for $i=1,n$. The number of MCVs can significantly affect the accuracy of \system (Fig.~\ref{fig:lpbound-MCVs}), as it does for \safebound and \psql. 

We also construct one degree sequence ${\bf d}_i$ for all non-MCVs of $A$ and each $i=1,n$. Let $r_i$ be the maximum number of $X_i$-values per non-MCV of $A$ and ${\bf d}_i$ be the degree sequence of the $r_i$ largest degrees of $X_i$-values. \nop{We can construct the degrees in the sequence by inspecting individually each non-MCV or aggregate over all non-MCVs.} We compute a set of $\ell_p$-norms of each degree sequence ${\bf d}_i$. An alternative, more expensive approach is to compute $\ell_p$-norms for each non-MCV and take their max for each $p$. 

To estimate for the equality predicate $A=v$, we use the $\ell_p$-norms for the degree sequences $\deg_R(*|X_i,$ $A=v)$ if $v$ is an MCV. Otherwise, we use the $\ell_p$-norms for the degree sequences ${\bf d}_i$. 

\paragraph{Range Predicate.} Range predicates are supported in \system using a hierarchy of histograms: Each layer is a histogram whose number of buckets is half the number of buckets of the histogram at the layer below. We ensure that the histogram at each layer covers the entire domain range of the attribute $A$. For each histogram bucket with boundaries $[s_i,e_i]$, we create $\ell_p$-norms on the full and simple degree sequences $\deg_R(*|X_i,$ $A\in [s_i,e_i])$.

To estimate for the range predicate $A\in [s,e]$, we find the smallest histogram bucket that contains the range $[s,e]$ from the predicate and then use the $\ell_p$-norms from that bucket.

\paragraph{Multiple Predicates.} In case of a conjunction of  predicates, we take as $\ell_p$-norm the minimum of the $\ell_p$-norms for the predicates, for each $p$. This is correct as the records must satisfy all predicates and in particular the most selective one. In case of a disjunction, we take as the $\ell_p$-norm the sum of the $\ell_p$-norms for the predicates, for each $p$. This {\em computed} quantity upper bounds the {\em desired} $\ell_p$-norm of the degree sequence for those tuples that satisfy the disjunction of the predicates, yet we cannot compute the latter norm unless we evaluate the predicates. To see this, observe that the desired $\ell_p$-norm is less than or equal to the $\ell_p$-norm of the degree sequence, which is obtained by the entry-wise sum of the degree sequences for the predicates. By Minkowski inequality, the latter norm is less than or equal to the computed norm.

\nop{\color{green}
In case of a disjunction, we take as $\ell_p$-norm the sum of the $\ell_p$-norms for the predicates, for each $p$. This is correct: the $\ell_p$-norm of the degree sequence for the disjunction of two predicates is less than or equal to the $\ell_p$-norm of the sum of two degree sequences, where the largest degrees of the two sequences are summed in order, which leads to the largest possible $\ell_p$-norm. Then, this $\ell_p$-norm of the sum of two degree sequences is less than or equal to the sum of the $\ell_p$-norms of the two degree sequences due to the Minkowski inequality, which proves that we take the upper bound.
}

\paragraph{Optimizations.} A challenge for \system is to estimate the cardinality of a join, where one operand is orders of magnitude larger than the other operands and has many dangling key values. This happens when a join operand has a selective predicate. By using norms that incorporate degrees  of dangling key values, \system returns a large overestimate. To address this challenge, it combines two orthogonal optimizations:  {\em predicate propagation} and {\em prefix degree sequences}. 
Predicate propagation is used in case of a predicate on a primary-key (PK) relation that is joined with a foreign-key (FK) relation. We propagate the predicate and its attribute through the join to the FK relation without increasing its size. The new predicate on the FK relation is then supported using MCVs and histograms to yield smaller and more accurate $\ell_p$-norms. \revtwo{For instance, assume we have a table $R(K,A)$ with primary key $K$ and attribute $A$ on which we have a predicate $\phi(A)$. We also have a table $S(K,B)$ with foreign key $K$ and some attribute $B$. By propagating $\phi(A)$ from $R$ to $S$, we mean that we join the two relations to obtain a new relation $S'(K,B,A)$. This relation $S'$ has the same cardinality as $S$, yet every $K$-value in $S'$ is now accompanied by the $A$-value from $R$. We can now construct MCVs and histograms on the data column $A$ in $S'$. A variant of this optimization is also used by \safebound~\cite{SafeBound:SIGMOD23}.
}


In case of a large degree sequence, \system also keeps its length ($\ell_0$-norm) and the $\ell_p$-norms on its prefixes with the $2^i$ largest degrees, for $i\geq 0$. Then, for a join, \system first fetches the $\ell_0$-norm of each of the operands. The minimum $m$ of these $\ell_0$-norms tells us the maximum number of key values that join at each operand. \system uses $m$ to pick the $\ell_p$-norms for the $i$-th prefix\footnote{The degrees typically decrease exponentially and sequence prefixes for $i>4$ have norms close to those for the entire degree sequence. For each large degree sequence, we therefore only keep the norms for the first 4 prefixes and for the entire sequence.} of the degree sequences of each of the join operands, for $2^{i-1}\leq m \leq 2^i$.



\nop{

TODO.  Things to say here:

\begin{itemize}
\item For a given attribute $Z$, we compute and store the maximum of
  the $\ell_p$ statistics of all relations of the form
  $\sigma_{Z=z}(R)$, for $z \in R.Z$.  We use these whenever the query
  contains a selection $R.Z=??$.  TO GIVE A NAME TO THIS STATISTICS,
  e.g. the \emph{generic conditionals}.
\item Better: construct a histogram on $R.Z$, by partitioning the
  domain into $b$ buckets, and storing separate generic conditionals
  per bucket.
\item Better: for the most common values $z_1, \ldots, z_k$ we store
  the values the $\ell_p$-statistics separately.  TO SAY HOW LARGE WE
  TAKE $k$.
\item For range queries, we need a different kind of histogram.
  Describe. 
\item For LIKE predicates we use this data structure XXX.
\item I really like Haoze's idea of the $\ell_p$ norms of a prefix.
  Can we include that?  Do we have experiments for that?
\item what else?
\end{itemize}



Relation $R(X,A_1,A_2,\ldots)$, conjunction of predicates $\bigwedge_i$ \textcolor{mMediumBrown}{$(A_i\text{ op } v_i)$}
\vspace*{1em}

Use the following $\ell_p$-norms for the conjunction of predicates:     \vspace*{1em}
\begin{itemize}
    \item Fetch the $\ell_p$-norms for each predicate \textcolor{mMediumBrown}{$A_i\text{ op } v_i$}
    \vspace*{1em}
    \item For each $p$, take the minimum of the $\ell_p$-norms for the predicates
\end{itemize}

    





}
