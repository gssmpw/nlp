\begin{abstract}
  Cardinality estimation is the problem of estimating the size of the
  output of a query, without actually evaluating the query. The
  cardinality estimator is a critical piece of a query optimizer, and
  is often the main culprit when the optimizer chooses a poor plan.


  This paper introduces \system, a ``pessimistic'' cardinality
  estimator for multijoin queries (acyclic or cyclic) with 
  selection predicates and group-by clauses. \system computes a
  guaranteed upper bound on the size of the query output\nop{ that is very
  close to the true output size} using simple
  statistics on the input relations, consisting of $\ell_p$-norms of
  degree sequences.  The bound is the optimal solution of a linear
  program whose constraints encode data statistics and Shannon
  inequalities. We introduce two optimizations that exploit the
  structure of the query in order to speed up the estimation time and
  make \system practical.


  We experimentally evaluate \system against a range of traditional,
  pessimistic, and machine learning-based estimators on the JOB,
  STATS, and subgraph matching benchmarks. Our main finding is that
  \system can be orders of magnitude more accurate than traditional
  estimators used in mainstream open-source and commercial database
  systems. Yet it has comparable low estimation time and space
  requirements. When injected the estimates of \system, \psql derives
  query plans at least as good as those derived using the true
  cardinalities.
\end{abstract}
