\section{The \system Cardinality Estimator}
\label{sec:lpbound}

Our system, \system, is a significant extension of previous upper
bound estimators, in that it computes a tight upper bound of the query $Q$ by
using $\ell_p$-norms of simple degree sequences.  \system can explain
its upper bound in terms of a simple inequality, called a
\emph{q-inequality}.  We introduce \system gradually, by first
describing the q-inequalities, and showing later how to compute the
optimal bound.  Throughout this section, we assume that the query has
no predicates: we discuss predicates in Sec.~\ref{sec:histograms}.

\subsection{Q-Inequalities for Full Queries}

\label{subsec:lpbound:full}

For upper bounds on a full conjunctive query in terms of
$\ell_p$-norms, we use inequalities described
in~\cite{DBLP:journals/pacmmod/KhamisNOS24}.  As a simple warmup,
consider the 2-way join, which we write as:
%
\begin{align}
  J_2(X,Y,Z) = & R(X,Y) \wedge S(Y,Z) \label{eq:j2}
\end{align}
%
\system uses the following q-inequalities:
%
\begin{align}
  |J_2| \leq & |R|\cdot|S| \\
  |J_2| \leq & |R|\cdot \lp{\degree_S(*|Y)}_\infty \\
  |J_2| \leq & \lp{\degree_R(*|Y)}_\infty \cdot |S| \\
  |J_2| \leq & \lp{\degree_R(*|Y)}_2 \cdot \lp{\degree_S(*|Y)}_2
\end{align}
%
The first bound is the AGM bound; the next two are the \maxdegree
bound, and are always lower (i.e. better) than the AGM bound.  The
last bound is new, and follows from the Cauchy-Schwartz inequality.
\system always returns the smallest value of all q-inequalities.  It
does not need to enumerate all of them; instead it computes the bound
differently (explained below in Sec.~\ref{subsec:basic:algorithm}),
then returns as explanation the single q-inequality that produces that
bound.

For the 3-way join $J_3$ from Eq.~\eqref{eq:j3}, \system uses many more q-inequalities.  It
includes all those considered by the \maxdegree bound
(Eq.~\eqref{eq:degree:bound:j3}) and many more.  We show here only two
q-inequalities:

{
\begin{align}
  |J_3| \leq & \lp{\degree_R(X|Y)}_2 \cdot |S|^{1/2} \cdot\lp{\degree_T(U|Z)}_2 \nonumber\\
  |J_3| \leq & |R|^{1/3}\cdot \lp{\degree_R(X|Y)}_2^{2/3} \cdot \lp{\degree_S(Z|Y)}_{2}^{2/3}\cdot\lp{\degree_T(U|Z)}_3 \label{eq:pce:j3:p3:r}
%  |J_3| \leq &  \lp{\degree_R(X|Y)}_3\cdot \lp{\degree_S(Y|Z)}_{2}^{2/3}\cdot \lp{\degree_T(Z|U)}_2^{2/3} \cdot |T|^{1/3} \label{eq:pce:j3:p3:t}
\end{align}
}

To the best of our knowledge, such inequalities have not been used
previously in cardinality estimation.  We prove~\eqref{eq:pce:j3:p3:r}
in Sec.~\ref{subsec:sec:lpbound:proofs}.  \system also improves significantly the bounds of cyclic
queries, for example it considers these q-inequalities for the
3-clique $C_3$:
%
\begin{align}
  |C_3| \leq & \left(|R|\cdot |S| \cdot |T|\right)^{1/2}\nonumber \\
  |C_3| \leq & \left(\lp{\degree_R(Y|X)}_2^2\cdot\lp{\degree_S(Z|Y)}_2^2\cdot\lp{\degree_T(X|Z)}_2^2\right)^{1/3}\nonumber \\
  |C_3| \leq & \left(\lp{\degree_R(Y|X)}_3^3\cdot\lp{\degree_S(Y|Z)}_3^3\cdot|T|^5\right)^{1/6}\label{eq:pce:t:3} 
\end{align}
%
The first is the AGM bound corresponding to the fractional edge cover
$w_R=w_S=w_T=\frac{1}{2}$.  The other two are novel and surprising.

\subsection{\system for \groupby Queries}
\label{sec:lpbound-groupby-queries}

Similar q-inequalities hold for queries with \groupby.  We 
illustrate here for the query $JG_3$ in Eq.~\eqref{eq:jg3}.

Every q-inequality that holds for the full conjunctive query also
holds for the \groupby query, in other words $|JG_3|\leq |J_3|$, and
all upper bounds for $J_3$ also apply to $JG_3$; this is used by \duckdb.

Further q-inequalities can be obtained by dropping variables that do not occur in \groupby, as done in \psql.  For example, we can drop the variables $Y,Z$ from $JG_3$
and obtain the query:
%
\begin{align*}
  JG_3'(X,U) = & R'(X) \wedge T'(U)
\end{align*}
%
for which we can infer:
%
\begin{align*}
  |JG_3|\leq& |JG_3'| \leq |R'|\cdot|T'| = |\dom(R.X)|\cdot|\dom(T.U)|
\end{align*}
%

However, \system uses many more q-inequalities, which are not
necessarily derived using the two heuristics above.  For example,
consider the following star-join with \groupby:
%
\begin{align*}
  \text{StarG}(X_1,X_2) = & R_1(X_1,Z)\wedge R_2(X_2,Z)\wedge S(Y,Z)
\end{align*}
%
\system infers (among others) the following inequality:
%
\begin{align}
  |\text{StarG}| \leq & |S|^{1/3}\cdot \lp{\degree_{R_1}(X_1|Z)}_3\cdot\lp{\degree_{R_2}(X_2|Z)}_3\label{eq:sg:bound}
\end{align}
%
This q-inequality does not hold for the full conjunctive
query\footnote{Proof: consider the instance $R_1=R_2=\set{(1,1)}$,
  $S=\set{(1,1),(2,1),\ldots,(N,1)}$.  The full join returns an output
  of size $N$, while the RHS of~\eqref{eq:sg:bound} is $N^{2/3}$.} and
it involves all query variables.  We prove~\eqref{eq:sg:bound} below.

\subsection{Proofs of Q-Inequalities}
\label{subsec:sec:lpbound:proofs}

Consider $n$ finite random variables $X_1, \ldots, X_n$, and let their
set of outcomes be the relation $R(X_1, \ldots, X_n)$ (see
Sec.~\ref{sec:background}).  Then, for any subsets of variables
$U, V \subseteq \attrs(R)$ and any $p \in \openclosed{0,\infty}$, the
following holds~\cite{DBLP:journals/pacmmod/KhamisNOS24}:
%
\begin{align}
  \frac{1}{p}h(U)+h(V|U) \leq & \log \lp{\degree_R(V|U)}_p \label{eq:h:p}
\end{align}
%
Inequalities~\eqref{eq:h:log} are special cases of \eqref{eq:h:p}, where 
$p=1$ or $p=\infty$.

Inequality~\eqref{eq:h:p} is very important.  It connects an
information-theoretic term in the LHS with a statistics on the input
database in the RHS.  All q-inequalities inferred by \system follow
from~\eqref{eq:h:p} and the basic Shannon inequalities.  We illustrate
with two examples.

First, we prove the q-inequality~\eqref{eq:pce:j3:p3:r} for $J_3$.
Assume three input relations $R(X,Y)$, $S(Y,Z)$, $T(Z,U)$, and denote
by $N \defeq |J_3|$ the size of the query's output. Consider the
uniform probability distribution with outcomes $J_3$: every tuple
$t = (x,y,z,u)\in J_3$ has the same probability, $\pr(t)= 1/N$.
Therefore, their entropy is $h(XYZU)=\log |J_3|$ (by uniformity),
and~\eqref{eq:pce:j3:p3:r} follows from:
%

{
  \begin{align*}
    \log & |R| +  2 \log \lp{\degree_R(X|Y)}_2 +  2 \log  \lp{\degree_S(Z|Y)}_{2} + 3 \log \lp{\degree_T(U|Z)}_3 \geq \\
    \geq & h(XY) + 2\left(\frac{1}{2}h(Y)+h(X|Y)\right)+2\left(\frac{1}{2}h(Y)+h(Z|Y)\right)+3\left(\frac{1}{3}h(Z)+h(U|Z)\right)
          \text{\hspace{2mm}by~\eqref{eq:h:p}}\\
    = & h(XY)+\left(h(XY)+h(X|Y)\right)+\left(h(YZ)+h(Z|Y)\right)+\left(h(UZ)+2h(U|Z)\right)
         \text{\hspace{6mm}by~\eqref{eq:chain}}\\
    = & \left(h(XY) + h(Z|Y)+ h(U|Z)\right) + \left(h(XY)+h(UZ)\right)+\left(h(X|Y)+h(YZ)+h(U|Z)\right)  \\
    \geq & \left(h(XY) + h(Z|XY)+ h(U|XYZ)\right)  + h(XYZU) + \left(h(X|YZ)+h(YZ)+h(U|XYZ)\right)\\
         &\text{\hspace{60mm}by submodularity}\\
    = & 3 h(XYZU) = 3 \log |J|
  \end{align*}
}

The first inequality is an application of~\eqref{eq:h:p}.  The second
inequality uses submodularity, for example $h(Z|Y) \geq h(Z|XY)$
follows from $h(YZ)-h(Y) \geq h(XYZ)-h(XY)$, or
$h(XY)+h(YZ)\geq h(XYZ)+h(Y)$.

Second, we prove the q-inequality~\eqref{eq:sg:bound}.  The setup is
similar: assume some input relation instances\\
$R_1(X_1,Z), R_2(X_2,Z), S(Y,Z)$, let $\text{StarG}(X_1,X_2)$ be the output of
the query, and denote by $N \defeq |\text{StarG}|$.  Define $\text{Star}(X_1,X_2,Y,Z)$ to
be the result of the full join.  We need a probability distribution on
$\text{Star}$ whose marginal on $X_1,X_2$ is uniform.  There are many ways to
define such a distribution, we consider the following.  Order the
tuples in $\text{Star}$ arbitrarily; then, for each tuple $t = (x_1,x_2,y,z)$,
if there exists some earlier tuple with the same values $x_1,x_2$ then
set $\pr(t)=0$, otherwise set $\pr(t)=1/N$.  At this point, we continue
similarly to the previous example:

\nop{
\begin{align*}
  \frac{2}{3}\log & |S| + \log \lp{\degree_{R_1}(X_1|Z)}_3+\log \lp{\degree_{R_2}(X_2|Z)}_3\geq\\
  \geq & \frac{2}{3}h(YZ)+\frac{1}{3}h(Z)+h(X_1|Z)+\frac{1}{3}h(Z)+h(X_2|Z) \\
  = & \frac{1}{3}\biggl(2\bigl(h(YZ)+h(X_1|Z)+h(X_2|Z)\bigr) + \bigl(h(X_1Z)+h(X_2Z)\bigr)\biggr)\\
  \geq  & \frac{1}{3}\left(2h(X_1X_2YZ) + h(X_1X_2Z)\right)\text{\hspace{20mm}submodularity}\\
  \geq & h(X_1X_2)=\log|\text{StarG}| \text{\hspace{25mm}monotonicity}
\end{align*}
}

\begin{align*}
  \frac{1}{3}\log & |S| + \log \lp{\degree_{R_1}(X_1|Z)}_3+\log \lp{\degree_{R_2}(X_2|Z)}_3\geq\\
  \geq & \frac{1}{3}h(YZ)+\frac{1}{3}h(Z)+h(X_1|Z)+\frac{1}{3}h(Z)+h(X_2|Z) \\
  \geq & \frac{1}{3}h(Z)+\frac{1}{3}h(Z)+h(X_1|Z)+\frac{1}{3}h(Z)+h(X_2|Z) \\
  = & h(Z) + h(X_1|Z)+ h(X_2|Z) = h(X_1Z) + h(X_2|Z) \\ 
  \geq & h(X_1Z) + h(X_2|X_1Z)  = h(X_1X_2Z) 
  \geq h(X_1X_2)=\log|\text{StarG}|
\end{align*}


\subsection{\lpbase: The Basic Algorithm of \system}

\label{subsec:basic:algorithm}

\system takes as input a query $Q$ (Eq.~\eqref{eq:cq}) and a set of
statistics on the input database consisting of $\ell_p$-norms on
degree sequences, and returns: (1) a numerical upper
bound $B$ such that $|Q|\leq B$ whenever the input database satisfies
these statistics; (2) an explanation consisting of a q-inequality on
$|Q|$, which, for the particular numerical values of the
$\ell_p$-norms implies $|Q| \leq B$; and (3) a proof of the Shannon
inequality needed to prove the q-inequality.
For that, \system solves a Linear Program (LP) called \lpbase defined as follows:
%

\smallskip

\noindent {\bf The Real-valued Variables} are all unknowns
$h(U)\geq 0$, $\forall U \subseteq \vars(Q)$ ($2^n$ real-valued
variables).

\smallskip

\noindent {\bf The Objective} is to maximize $h(V_0)$, where $V_0$ is the set of    
the \groupby variables of the query in Eq.~\eqref{eq:cq}, under the
following two types of constraints.

\smallskip

\noindent {\bf The Statistics Constraints} are linear constraints of
the form in Eq.~\eqref{eq:h:p}, one for each $\ell_p$-norm of a degree sequence
that has been computed on the input database.

  \nop{
  %
    \begin{align}
      \frac{1}{p}h(U) + h(V|U) \leq & \log\left(\lp{\degree_R(V|U)}_p\right)\label{eq:stats:constraint}
    \end{align}
    %
}
\smallskip

\noindent {\bf The Shannon Constraints} are all basic Shannon inequalities,
as linear constraints (Eq.~\eqref{eq:shannon:monotonicity}
and~\eqref{eq:shannon:submodularity}).
%

\system uses the off-the-shelf solver HiGHS 1.7.2~\cite{HiGHS:2018} to solve both
\lpbase and its dual linear program.  The optimal solution of \lpbase
 consists of $2^n$ values $h^*(U)$, one for each set of query
variables $U$.  The optimal solution of the dual consists of
non-negative weights $w^*\geq 0$, one for every statistics constraint, and
non-negative weights $s^* \geq 0$, one for each basic Shannon inequality.
\system returns the following: (1) The bound $B \defeq 2^{h^*(V_0)}$,
(2) the q-inequality
$|Q| \leq \prod \left(\lp{\degree_R(V|U)}_p\right)^{w^*}$ where the
product ranges over all statistics constraints: this uses only the
weights associated to the Statistics Constraints, and (3) all basic
Shannon inequalities together with their weight $s^*$: these form the
required proof of the q-inequality.  We prove in the full paper:

\begin{theorem} For any input query $Q$, \system is correct:
\begin{enumerate}
    \item The quantity $B$ returned by \system is a tight upper bound on $|Q|$, meaning that $|Q|$ never exceeds $B$ if the input database satisfies the given statistics, and there exists an input database satisfying the given statistics on which
  $|Q|$ is as large as $B$ (up to a small query-dependent
  constant).
    \item The q-inequality returned by $\system$ holds in
  general. For the particular values of the statistics \\
  $\lp{\degree_R(V|U)}_p$, the inequality implies $|Q| \leq B$. 
    \item The basic Shannon inequalities multiplied with their associated
  weights $s^*$ form the proof of the q-inequality.
\end{enumerate}
\end{theorem}

Recall that the statistics only use simple degree sequences; without this assumption the tightness statement no
longer holds.


\begin{example} Consider the 3-way join $J_3$, shown
  in~\eqref{eq:j3}, and assume that, for each relation $R, S, T$
  and each attribute, \system has access to five precomputed $\ell_p$
  norms: $\ell_1, \ell_2, \ell_3, \ell_4, \ell_\infty$.  (Notice that
  $\ell_1$ is the same as the cardinality:
  $\lp{\degree_R(*|Y)}_1=|R|$.)  Then the optimal bound to $J_3$ is
  given by the following \lpbase linear program, with $2^4=16$ variables
  $h(\emptyset), h(X), h(Y), \ldots, h(XYZU)$:
%
  \begin{align*}
    \texttt{max}&\texttt{imize } h(XYZU) \texttt{ subject to} \\
              & h(XY) \leq \log |R| \\
              & \frac{1}{2}h(Y)+h(X|Y) \leq \log\lp{\degree_R(X|Y)}_2\\
              & \hspace{10mm} \ldots \text{same for all other $\ell_p$ norms}\\
              & h(X)+h(Y)\geq h(XY)+h(\emptyset)\\
              & h(XY)+h(YZ) \geq h(XYZ)+h(Y)\\
              & \hspace{10mm} \ldots \text{continue with all basic Shannon inequalities}
  \end{align*}
%
  A standard LP package returns both the optimal of this LP, $h^*(U)$,
  and the optimal of its dual, $w^*,s^*$.  The query's upper bound is
  $2^{h^*(XYZU)}$.  The q-inequality is
  $|Q| \leq |R|^{w_1^*} \cdot \lp{\degree_R(X|Y)}_2^{w_2^*} \cdots$
  where $w_1^*, w_2^*, \ldots$ are the dual variables associated with
  the statistical constraints.  Finally, the dual variables $s^*$
  associated to the basic Shannon inequalities provide the proof of
  the information-theoretic inequality needed to prove the
  q-inequality.
\end{example}



