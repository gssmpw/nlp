%% bare_conf.tex
%% V1.4b
%% 2015/08/26
%% by Michael Shell
%% See:
%% http://www.michaelshell.org/
%% for current contact information.
%%
%% This is a skeleton file demonstrating the use of IEEEtran.cls
%% (requires IEEEtran.cls version 1.8b or later) with an IEEE
%% conference paper.
%%
%% Support sites:
%% http://www.michaelshell.org/tex/ieeetran/
%% http://www.ctan.org/pkg/ieeetran
%% and
%% http://www.ieee.org/-

%%*************************************************************************
%% Legal Notice:
%% This code is offered as-is without any warranty either expressed or
%% implied; without even the implied warranty of MERCHANTABILITY or
%% FITNESS FOR A PARTICULAR PURPOSE! 
%% User assumes all risk.
%% In no event shall the IEEE or any contributor to this code be liable for
%% any damages or losses, including, but not limited to, incidental,
%% consequential, or any other damages, resulting from the use or misuse
%% of any information contained here.
%%
%% All comments are the opinions of their respective authors and are not
%% necessarily endorsed by the IEEE.
%%
%% This work is distributed under the LaTeX Project Public License (LPPL)
%% ( http://www.latex-project.org/ ) version 1.3, and may be freely used,
%% distributed and modified. A copy of the LPPL, version 1.3, is included
%% in the base LaTeX documentation of all distributions of LaTeX released
%% 2003/12/01 or later.
%% Retain all contribution notices and credits.
%% ** Modified files should be clearly indicated as such, including  **
%% ** renaming them and changing author support contact information. **
%%*************************************************************************


% *** Authors should verify (and, if needed, correct) their LaTeX system  ***
% *** with the testflow diagnostic prior to trusting their LaTeX platform ***
% *** with production work. The IEEE's font choices and paper sizes can   ***
% *** trigger bugs that do not appear when using other class files.       ***                          ***
% The testflow support page is at:
% http://www.michaelshell.org/tex/testflow/



\documentclass[conference]{IEEEtran}
%\usepackage[numbers]{natbib}
\usepackage{svg}
\usepackage{amsmath} 
% Some Computer Society conferences also require the compsoc mode option,
% but others use the standard conference format.
%
% If IEEEtran.cls has not been installed into the LaTeX system files,
% manually specify the path to it like:
% \documentclass[conference]{../sty/IEEEtran}





% Some very useful LaTeX packages include:
% (uncomment the ones you want to load)


% *** MISC UTILITY PACKAGES ***
%
%\usepackage{ifpdf}
% Heiko Oberdiek's ifpdf.sty is very useful if you need conditional
% compilation based on whether the output is pdf or dvi.
% usage:
% \ifpdf
%   % pdf code
% \else
%   % dvi code
% \fi
% The latest version of ifpdf.sty can be obtained from:
% http://www.ctan.org/pkg/ifpdf
% Also, note that IEEEtran.cls V1.7 and later provides a builtin
% \ifCLASSINFOpdf conditional that works the same way.
% When switching from latex to pdflatex and vice-versa, the compiler may
% have to be run twice to clear warning/error messages.






% *** CITATION PACKAGES ***
%
\usepackage{cite}
% cite.sty was written by Donald Arseneau
% V1.6 and later of IEEEtran pre-defines the format of the cite.sty package
% \cite{} output to follow that of the IEEE. Loading the cite package will
% result in citation numbers being automatically sorted and properly
% "compressed/ranged". e.g., [1], [9], [2], [7], [5], [6] without using
% cite.sty will become [1], [2], [5]--[7], [9] using cite.sty. cite.sty's
% \cite will automatically add leading space, if needed. Use cite.sty's
% noadjust option (cite.sty V3.8 and later) if you want to turn this off
% such as if a citation ever needs to be enclosed in parenthesis.
% cite.sty is already installed on most LaTeX systems. Be sure and use
% version 5.0 (2009-03-20) and later if using hyperref.sty.
% The latest version can be obtained at:
% http://www.ctan.org/pkg/cite
% The documentation is contained in the cite.sty file itself.






% *** GRAPHICS RELATED PACKAGES ***
%
\ifCLASSINFOpdf
  % \usepackage[pdftex]{graphicx}
  % declare the path(s) where your graphic files are
  % \graphicspath{{../pdf/}{../jpeg/}}
  % and their extensions so you won't have to specify these with
  % every instance of \includegraphics
  % \DeclareGraphicsExtensions{.pdf,.jpeg,.png}
\else
  % or other class option (dvipsone, dvipdf, if not using dvips). graphicx
  % will default to the driver specified in the system graphics.cfg if no
  % driver is specified.
  % \usepackage[dvips]{graphicx}
  % declare the path(s) where your graphic files are
  % \graphicspath{{../eps/}}
  % and their extensions so you won't have to specify these with
  % every instance of \includegraphics
  % \DeclareGraphicsExtensions{.eps}
\fi
% graphicx was written by David Carlisle and Sebastian Rahtz. It is
% required if you want graphics, photos, etc. graphicx.sty is already
% installed on most LaTeX systems. The latest version and documentation
% can be obtained at: 
% http://www.ctan.org/pkg/graphicx
% Another good source of documentation is "Using Imported Graphics in
% LaTeX2e" by Keith Reckdahl which can be found at:
% http://www.ctan.org/pkg/epslatex
%
% latex, and pdflatex in dvi mode, support graphics in encapsulated
% postscript (.eps) format. pdflatex in pdf mode supports graphics
% in .pdf, .jpeg, .png and .mps (metapost) formats. Users should ensure
% that all non-photo figures use a vector format (.eps, .pdf, .mps) and
% not a bitmapped formats (.jpeg, .png). The IEEE frowns on bitmapped formats
% which can result in "jaggedy"/blurry rendering of lines and letters as
% well as large increases in file sizes.
%
% You can find documentation about the pdfTeX application at:
% http://www.tug.org/applications/pdftex





% *** MATH PACKAGES ***
%
%\usepackage{amsmath}
% A popular package from the American Mathematical Society that provides
% many useful and powerful commands for dealing with mathematics.
%
% Note that the amsmath package sets \interdisplaylinepenalty to 10000
% thus preventing page breaks from occurring within multiline equations. Use:
%\interdisplaylinepenalty=2500
% after loading amsmath to restore such page breaks as IEEEtran.cls normally
% does. amsmath.sty is already installed on most LaTeX systems. The latest
% version and documentation can be obtained at:
% http://www.ctan.org/pkg/amsmath





% *** SPECIALIZED LIST PACKAGES ***
%
%\usepackage{algorithmic}
% algorithmic.sty was written by Peter Williams and Rogerio Brito.
% This package provides an algorithmic environment fo describing algorithms.
% You can use the algorithmic environment in-text or within a figure
% environment to provide for a floating algorithm. Do NOT use the algorithm
% floating environment provided by algorithm.sty (by the same authors) or
% algorithm2e.sty (by Christophe Fiorio) as the IEEE does not use dedicated
% algorithm float types and packages that provide these will not provide
% correct IEEE style captions. The latest version and documentation of
% algorithmic.sty can be obtained at:
% http://www.ctan.org/pkg/algorithms
% Also of interest may be the (relatively newer and more customizable)
% algorithmicx.sty package by Szasz Janos:
% http://www.ctan.org/pkg/algorithmicx




% *** ALIGNMENT PACKAGES ***
%
%\usepackage{array}
% Frank Mittelbach's and David Carlisle's array.sty patches and improves
% the standard LaTeX2e array and tabular environments to provide better
% appearance and additional user controls. As the default LaTeX2e table
% generation code is lacking to the point of almost being broken with
% respect to the quality of the end results, all users are strongly
% advised to use an enhanced (at the very least that provided by array.sty)
% set of table tools. array.sty is already installed on most systems. The
% latest version and documentation can be obtained at:
% http://www.ctan.org/pkg/array


% IEEEtran contains the IEEEeqnarray family of commands that can be used to
% generate multiline equations as well as matrices, tables, etc., of high
% quality.




% *** SUBFIGURE PACKAGES ***
%\ifCLASSOPTIONcompsoc
%  \usepackage[caption=false,font=normalsize,labelfont=sf,textfont=sf]{subfig}
%\else
%  \usepackage[caption=false,font=footnotesize]{subfig}
%\fi
% subfig.sty, written by Steven Douglas Cochran, is the modern replacement
% for subfigure.sty, the latter of which is no longer maintained and is
% incompatible with some LaTeX packages including fixltx2e. However,
% subfig.sty requires and automatically loads Axel Sommerfeldt's caption.sty
% which will override IEEEtran.cls' handling of captions and this will result
% in non-IEEE style figure/table captions. To prevent this problem, be sure
% and invoke subfig.sty's "caption=false" package option (available since
% subfig.sty version 1.3, 2005/06/28) as this is will preserve IEEEtran.cls
% handling of captions.
% Note that the Computer Society format requires a larger sans serif font
% than the serif footnote size font used in traditional IEEE formatting
% and thus the need to invoke different subfig.sty package options depending
% on whether compsoc mode has been enabled.
%
% The latest version and documentation of subfig.sty can be obtained at:
% http://www.ctan.org/pkg/subfig




% *** FLOAT PACKAGES ***
%
%\usepackage{fixltx2e}
% fixltx2e, the successor to the earlier fix2col.sty, was written by
% Frank Mittelbach and David Carlisle. This package corrects a few problems
% in the LaTeX2e kernel, the most notable of which is that in current
% LaTeX2e releases, the ordering of single and double column floats is not
% guaranteed to be preserved. Thus, an unpatched LaTeX2e can allow a
% single column figure to be placed prior to an earlier double column
% figure.
% Be aware that LaTeX2e kernels dated 2015 and later have fixltx2e.sty's
% corrections already built into the system in which case a warning will
% be issued if an attempt is made to load fixltx2e.sty as it is no longer
% needed.
% The latest version and documentation can be found at:
% http://www.ctan.org/pkg/fixltx2e


%\usepackage{stfloats}
% stfloats.sty was written by Sigitas Tolusis. This package gives LaTeX2e
% the ability to do double column floats at the bottom of the page as well
% as the top. (e.g., "\begin{figure*}[!b]" is not normally possible in
% LaTeX2e). It also provides a command:
%\fnbelowfloat
% to enable the placement of footnotes below bottom floats (the standard
% LaTeX2e kernel puts them above bottom floats). This is an invasive package
% which rewrites many portions of the LaTeX2e float routines. It may not work
% with other packages that modify the LaTeX2e float routines. The latest
% version and documentation can be obtained at:
% http://www.ctan.org/pkg/stfloats
% Do not use the stfloats baselinefloat ability as the IEEE does not allow
% \baselineskip to stretch. Authors submitting work to the IEEE should note
% that the IEEE rarely uses double column equations and that authors should try
% to avoid such use. Do not be tempted to use the cuted.sty or midfloat.sty
% packages (also by Sigitas Tolusis) as the IEEE does not format its papers in
% such ways.
% Do not attempt to use stfloats with fixltx2e as they are incompatible.
% Instead, use Morten Hogholm'a dblfloatfix which combines the features
% of both fixltx2e and stfloats:
%
% \usepackage{dblfloatfix}
% The latest version can be found at:
% http://www.ctan.org/pkg/dblfloatfix




% *** PDF, URL AND HYPERLINK PACKAGES ***
%
%\usepackage{url}
% url.sty was written by Donald Arseneau. It provides better support for
% handling and breaking URLs. url.sty is already installed on most LaTeX
% systems. The latest version and documentation can be obtained at:
% http://www.ctan.org/pkg/url
% Basically, \url{my_url_here}.

\usepackage[hidelinks]{hyperref}

% *** Do not adjust lengths that control margins, column widths, etc. ***
% *** Do not use packages that alter fonts (such as pslatex).         ***
% There should be no need to do such things with IEEEtran.cls V1.6 and later.
% (Unless specifically asked to do so by the journal or conference you plan
% to submit to, of course. )

\usepackage{orcidlink}
\usepackage{listings}
\usepackage{tikz}

% correct bad hyphenation here
\hyphenation{op-tical net-works semi-conduc-tor}

\newcommand\copyrighttext{%
  \footnotesize \textcopyright\>\the\year{} IEEE. Personal use of this material is permitted.  Permission from IEEE must be obtained for all other uses, in any current or future media, including reprinting/republishing this material for advertising or promotional purposes, creating new collective works, for resale or redistribution to servers or lists, or reuse of any copyrighted component of this work in other works.}

\newcommand\copyrightnotice{%
\begin{tikzpicture}[remember picture,overlay]
\node[anchor=south,yshift=20pt] at (current page.south) {\fbox{\parbox{\dimexpr0.82\textwidth-\fboxsep-\fboxrule\relax}{\copyrighttext}}};
\end{tikzpicture}%
}

\begin{document}

%
% paper title
% Titles are generally capitalized except for words such as a, an, and, as,
% at, but, by, for, in, nor, of, on, or, the, to and up, which are usually
% not capitalized unless they are the first or last word of the title.
% Linebreaks \\ can be used within to get better formatting as desired.
% Do not put math or special symbols in the title.
\title{SHACL-SKOS based \\ Knowledge Representation of Material Safety Data Sheet (SDS) for the Pharmaceutical Industry}

% author names and affiliations
% use a multiple column layout for up to three different
% affiliations
\author{\IEEEauthorblockN{Brian Lu\,\orcidlink{0009-0000-2253-3535}, Dennis Pham\,\orcidlink{0000-0001-5288-095X}}
\IEEEauthorblockA{Purdue University \\
West Lafayette, IN 47906, United States \\
contact@brianlu.me, dennis@dennispham.me}
\and
\IEEEauthorblockN{Ti-chiun Chang\,\orcidlink{0009-0005-7388-6169}, Michael Lovette\,\orcidlink{0000-0002-5747-7366}, Terri Bui\,\orcidlink{0009-0008-7637-9738}, Stephen Ma\,\orcidlink{0000-0002-0310-7562}}
\IEEEauthorblockA{MRL, Merck \& Co., Inc. \\
Rahway, NJ 07065, United States \\
\{ti-chiun.chang, michael.lovette, yen.bui, stephen.ma\}@merck.com}}


% conference papers do not typically use \thanks and this command
% is locked out in conference mode. If really needed, such as for
% the acknowledgment of grants, issue a \IEEEoverridecommandlockouts
% after \documentclass

% for over three affiliations, or if they all won't fit within the width
% of the page, use this alternative format:
% 
%\author{\IEEEauthorblockN{Michael Shell\IEEEauthorrefmark{1},
%Homer Simpson\IEEEauthorrefmark{2},
%James Kirk\IEEEauthorrefmark{3}, 
%Montgomery Scott\IEEEauthorrefmark{3} and
%Eldon Tyrell\IEEEauthorrefmark{4}}
%\IEEEauthorblockA{\IEEEauthorrefmark{1}School of Electrical and Computer Engineering\\
%Georgia Institute of Technology,
%Atlanta, Georgia 30332--0250\\ Email: see http://www.michaelshell.org/contact.html}
%\IEEEauthorblockA{\IEEEauthorrefmark{2}Twentieth Century Fox, Springfield, USA\\
%Email: homer@thesimpsons.com}
%\IEEEauthorblockA{\IEEEauthorrefmark{3}Starfleet Academy, San Francisco, California 96678-2391\\
%Telephone: (800) 555--1212, Fax: (888) 555--1212}
%\IEEEauthorblockA{\IEEEauthorrefmark{4}Tyrell Inc., 123 Replicant Street, Los Angeles, California 90210--4321}}




% use for special paper notices
%\IEEEspecialpapernotice{(Invited Paper)}




% make the title area
\maketitle
\copyrightnotice

% As a general rule, do not put math, special symbols or citations
% in the abstract
\begin{abstract}
%In order to meet patient needs, the drug development and manufacturing process is becoming more complex. With this complexity, comes the needs for data co-optimization between information systems. 
 We report the development of a knowledge representation and reasoning (KRR) system built on hybrid SHACL-SKOS ontologies for globally harmonized system (GHS) material Safety Data Sheets (SDS) to enhance chemical safety communication and regulatory compliance. SDS are comprehensive documents containing safety and handling information for chemical substances. Thus, they are an essential part of workplace safety and risk management. However, the vast number of Safety Data Sheets from multiple organizations, manufacturers, and suppliers that produce and distribute chemicals makes it challenging to centralize and access SDS documents through a single repository. To accomplish the underlying issues of data exchange related to chemical shipping and handling, we construct SDS related controlled vocabulary and conditions validated by SHACL, and knowledge systems of similar domains linked via SKOS. The resulting hybrid ontologies aim to provide standardized yet adaptable representations of SDS information, facilitating better data sharing, retrieval, and integration across various platforms. This paper outlines our SHACL-SKOS system architectural design and showcases our implementation for an industrial application streamlining the generation of a composite shipping cover sheet.
\end{abstract}

% no keywords

% For peer review papers, you can put extra information on the cover
% page as needed:
% \ifCLASSOPTIONpeerreview
% \begin{center} \bfseries EDICS Category: 3-BBND \end{center}
% \fi
%
% For peerreview papers, this IEEEtran command inserts a page break and
% creates the second title. It will be ignored for other modes.
\IEEEpeerreviewmaketitle



\section{Introduction} \label{intro}


% no \IEEEPARstart
Knowledge representation and reasoning (KRR) provides a framework for encoding knowledge in a format that artificial intelligence (AI) systems can utilize to reason, learn, and interact with the world in increasingly sophisticated ways. In 1993, the World Wide Web Consortium (W3C) was founded to develop a set of international standards, the modern beginnings of standardized KRR, for information exchange via the world wide web. One popular KRR model is a knowledge graph which utilizes graph-structured representation and database to store and operate on interconnected entities. In general, these standards created a cohesive set of KRRs that have allowed different systems and devices to communicate effectively -- fostering an open and interoperable web environment. A number of efforts have been taken (e.g., W3C standards \cite{w3c_standards}, NexIOM \cite{nexiom}, regulatory ontologies \cite{dumontier2014semanticscience, arp2015building, iso2011information}, OBO \cite{obofoundry}, etc.) to address general data exchange issues with unified ontologies acting as templates of universal KRRs, defining concepts, terms, and their relationship and properties. However, as the world wide web continues to develop and grow, so are the KRRs as they are increasingly adopted in industry-specific domains. In the case of the pharmaceutical sector, a number of KRRs have already been developed which focus on drug-target identification, drug discovery, drug development\cite{lin2017drug, visser2011bao, griffith2013gpcr, wishart2006drugbank, hewett2017pharmgkb}, and health outcomes \cite{cella2010patient, smith2007obo, hripcsak2015observational, hanna2022chronic, michie2017human}.



%One of those standards was the Resource Description Framework (RDF) \cite{w3RDF}.  Originally designed as a simple data model of triple statements, RDF has now become the general method of description and exchanging information for incredibly complex systems. Today, RDF is the basis for several ontology languages such as OWL \cite{w3OWL}, SKOS \cite{w3SKOS}, and SHACL \cite{w3c2017shacl}. Querying information represented by RDF frameworks is possible through languages such as SPARQL \cite{w3SPARQL} GraphQL \cite{bosu2024graphql, quina2023graphql}, Cypher \cite{nadime2018cypher}, etc.  Variations of  KRR frameworks are often utilized and form the foundation for ontologies specific to an industry. Because of the modular nature of RDF-based KRRs, ontologies always applied in combination, rather than stand alone, to better capture the complexity of a knowledge base. One specific example of a RDF-based KRR, is Drug Ontology (DrOn) \cite{hanna2017therapeutic, brochhausen2016accurate}, a popular ontology representing drug ingredients and biological activities. DrON relies heavily on RxNorm (standardized nomenclature for clinical drugs) \cite{nelson2011normalized, 1516084} and ChEBI (biologically classified small molecule ontology) \cite{hastings2016chebi} for its foundational structure. By utilizing the standardized identifiers from RxNorm, DrOn provides enhanced capability to query drug product (DP) based on specific attributes such as mechanism or drug target. DrON, when integrated with other ontologies such as DINTO \cite{HerreroZazo2013AnOF} and FIDEO \cite{fideo2020}, provides additional layers of context to drug substance and product information -- enhancing utility in clinical settings. 

 Recently, there has been growing interest in integrating systems related to pharmaceutical manufacturing, quality control, and regulatory compliance to more effectively address concerns related to the shipping and handling process. % any reference to justify the "growing interest"?
 This necessitates increased interface between information objects to ``connect the dots'' across different information domains (i.e., Enterprise Organization, Modeling \& Simulations, Process Development, Warehousing \& Procurement, Project Timelines, etc.) -- all of which have their own internal domain-specific data models and vocabularies. However, these systems often rely on individually defined ontologies, which upon integration, are faced with a ``terminology challenge'' as data exchange between systems becomes more complex.   Recent alliances, such as Pistoia \cite{pistoia_alliance} and Allotrope \cite{Millecam2021}, were formed to support the integration of information gathering, management, and model development of pharmaceutically relevant information in hopes of overcoming both the interoperability and terminology challenge associated with KRRs. In most cases, pharmaceutically relevant ontologies trend towards alignment with industrial standards for product and process development \cite{VENKATASUBRAMANIAN20061482, engproc2021009038, raebel2013standardizing, MORBACH2007147, mann2023susie}. 
 
 Unfortunately, user adoption of these individually aligned ontologies have been limited due to complexity and difficulty in implementation. For example, ontologies based on basic formal ontology (BFO) \cite{Arp2015-bk} encourage modeling vocabularies as classes, which are subclassed from other BFO classes, making them unwieldy and challenging to understand. As BFO expresses data in the form of overly faceted classes, reasoning at a certain higher-level requires significant and tedious introspection into the class structure. Specifically limiting the number of sub-classes to those that are structurally relevant allows implementers and users of the ontology to better understand how the ontology classes show a specific ``view'' of the data on one abstraction level. Taxonomy construction, to support fine-grained vocabularies, can then be offloaded to a knowledge organization system that is fit for the task, such as Simple Knowledge Organization System (SKOS), which is our preferred choice. By separating these concepts, it becomes easier to construct and integrate the ontologies operationally. Following such design principles, ontologies such as Quantities, Units, Dimensions, and Types (QUDT) \cite{qudt}, Units of Measure (UOM) \cite{uom}, and Medical Subject Headings (MeSH) \cite{mesh} have seen more success as they utilize SKOS framework to create domain specific taxonomies (hierarchical classification system with structured relationships between concepts) rather than overly intertwined ``top-level'' organization of concepts and terminologies, facilitating easier processing with software. 
 
To further elaborate, we utilize SHACL ``shapes'' (container-like objects) to define the structure and constraints of domain-specific data, as each shape can reference multiple SKOS ``taxonomies'' (controlled vocabularies and hierarchical organization of concepts). This approach enhances flexibility and re-usability for modeling complex systems. As such, application of these frameworks allows for modeling data with constraints and makes it easier to specify conditions and ensure data validity. 
 

 \begin{figure}[ht]
    \centering
    \includesvg[width=2in]{figures/ieee6-triangle.drawio.svg}
    \caption{Minor vocabulary differences can introduce friction}
    \label{fig:ieee6}
\end{figure}

 In our specific pharmaceutical use-case, a seemingly straightforward process of creating a comprehensive packaging cover sheet, which includes material classifications and handling procedures,  requires the integration of discrete SDS across multiple internal and external vendor systems. As illustrated in Fig.\ref{fig:ieee6}, while the United Nations (UN) Globally Harmonized Systems (GHS) \cite{ghs} standardizes most language specific to the communication of hazard and safety information for substances and compounds, there is a need to cross reference to the Occupational Safety and Health Administration (OSHA) Hazard Communication (HazCom) standards\footnote{https://www.osha.gov/hazcom/HCS-Final-RegText}. Therefore, SDS in the wild have to adhere to older standards, are subject to vendor interpretation, or need to fulfill local regulatory requirements, resulting in difficulties to streamline cover sheet generation. In these cases, usage of SKOS enables multiple taxonomies of various standards to be applied to the same shape, realizing the benefits of integrating potentially heterogeneous vocabularies.

Here we present the development of a SHACL-SKOS data model to create ontologies that enables users to navigate complex SDS datasets intuitively. This framework aims to address the challenges of data integration and terminology inconsistencies in pharmaceutical development processes by leveraging simple and modular ontologies within a modern data platform architecture. The separation of (SHACL) shapes and (SKOS) taxonomies is a key aspect to this approach which allows shapes to reference multiple taxonomies and provides different views on shared concepts, facilitating better data sharing, retrieval, and integration across various platforms and stakeholders. By establishing a SHACL-SKOS framework for representing SDS information, we aim to improve chemical safety communication, enhance regulatory compliance efforts, and promote a culture of safety across industries. We report its design, explore its components in detail, and discuss its potential impact on the knowledge management of material safety.


%SKOS is an RDF standard designed to encompose thesauri, schema, and taxonomies and facilitate easy publication and data linkages. The SHACL component further supports hierarchical structuring, making it easier to identify and classify information via constraints of the contents structure and meaning of a KRR -- thereby enabling domain specific technical validation and constraints on a representation.

\section{Methods}

In this section, we outline our strategy for ontology construction through SHACL-SKOS concepts, workflow design decisions, and our current implementation of the SHACL-SKOS system for SDS data extraction and summarization.  %\\\\\\\\\\\\\\\\begin{figure}[h]
%    \\\\\\\\\\\\\\\\centering
%    \\\\\\\\\\\\\\\\includegraphics[width=3in]{figures/cover-sheet-redacted.png}
%    \\\\\\\\\\\\\\\\caption{Cover Sheet Generation}
%    \\\\\\\\\\\\\\\\label{fig:coversheet}
%\\\\\\\\\\\\\\\\end{figure}

\subsection{Ontology Modeling Strategy}
Our SHACL-SKOS system, referred to as DeepPharmGraph (DPG), adopts a graph-structured data model for potentially coupled (deep-learning based) inference and appears as the prefix of modules in Figs.\ref{fig:ieee5}-\ref{fig:ieee4}. It simplifies taxonomy construction using SKOS, facilitates integration of different vocabularies, and offers standardized terms for referencing by multiple shapes. These modules were constructed for the SHACL-SKOS system and are not a part of any public ontologies/taxonomies.  Additionally, our strategy, as elaborated below, supports multiple views on shared taxonomic concepts, enabling diverse models while maintaining consistency and reducing redundancy. 

\begin{itemize}
\item{\textbf{Separation of Shapes and Taxonomies:}} Recognizing that shapes and taxonomies serve different purposes, we separate them to enhance modularity and reusability. Shapes define the structure and constraints of data, while taxonomies provide the controlled vocabularies and hierarchical organization of concepts. By decoupling these components, shapes can reference multiple taxonomies, and taxonomies can be applied across different shapes. 
\item{\textbf{Simplified Taxonomy Construction:}} We utilize SKOS to create taxonomies that organize concepts in a hierarchical manner. This approach makes taxonomy construction straightforward and facilitates the integration of different vocabularies. Taxonomies like DPG-GHS (aligned with GHS SDS headings and hazard classifications) and DPG-ISA-88 (describing ISA-88 terms) provide standardized terms that can be referenced by multiple shapes. 
\item{\textbf{Multiple Views on Shared Taxa:}} Shapes represent different ``views'' (or perspectives) on the same set of taxonomic concepts. For example, the DPG-DoC shape models document structures, while the DPG-SafeD shape models safety data. Both shapes can reference the same taxonomy (e.g. DPG-GHS), enabling consistent interpretation and reducing redundancy. 
\end{itemize}

\subsection{Shapes and Taxonomies}

\begin{figure}[ht]
    \centering
    \includesvg[width=2.4in]{figures/ieee5-shape-tax.drawio.svg}
    \caption{Shape graphs and their related taxonomies}
    \label{fig:ieee5}
\end{figure}

% As the SHACL-SKOS methodology inherently decouples shapes and taxonomy components, shapes can then reference multiple taxonomies, and taxonomies can be applied across different shapes. % repeated
 To simplify taxonomy construction, SKOS organizes concepts in a hierarchical manner. Scoping of allowed taxa and extended properties is facilitated via subclassing of \texttt{skos:Concept} (user-defined versus inherited within the broader SKOS concept scheme). This approach makes taxonomy construction straightforward and facilitates the integration of different vocabularies. 
 %repeated%%For this SDS use case, taxonomies like DPG-GHS (aligned with GHS SDS headings and hazard classifications) and DPG-ISA-88 (describing ISA-88 terms) were constructed to provide standardized terms that can be referenced by multiple shapes. 

To accurately capture the relationships between different concepts within SDS documents, ensure semantic clarity, and enhance data interoperability, we employed SKOS, and additionally, SKOS eXtension for Labels (SKOS-XL) for application-specific extensions \cite{w3SKOSXL}: GHS datasheet sections (for use in documents) or GHS hazards (for use in classification) are represented as SKOS Concepts. The SHACL shapes for SKOS are derived from those used by SkoHub \cite{skohub}. SKOS-XL labels are used as additional data when performing document understanding heuristics or rendering user interfaces.

Usage of SKOS-XL is envisioned as a means to accelerate construction of ontology by performing the heavy-lifting of taxonomy and controlled vocabulary construction (See Fig. \ref{fig:ieee5}) by forming the basis of a knowledge organization system. In this development, ontology files reference the following shapes and taxonomies.
\\\\
\noindent\textbf{Shapes}

\noindent\subsubsection*{DPG-DoC (Document Components)} Defines the structure of documents, such as sections, headings, and containers. It provides a schema for representing document components, allowing for consistent parsing and interpretation of various document formats. DPG-DoC has re-written parts of SPAR Pattern Ontology \cite{sparontologiesPatternOntology}, DoCO \cite{sparDoCO}, and FaBiO \cite{PERONI201233} in the form of SHACL shapes and a SKOS document. The SPAR Pattern Ontology is taken relatively directly, translated to SHACL, representing the structural semantics of a document. The SPAR DoCO ontology is broken down into a SKOS-XL taxonomy as marker concepts for use in shapes. This also opens up opportunity to extend the taxonomy to create more semantically fine-grained markers, such as those for GHS SDS document headings, as opposed to generic headings.

\begin{figure}[ht]
    \centering
    \includesvg[width=2.90in]{figures/ieee1-doc-hierarchy.drawio.svg}
    \caption{Sample Document Components Shape}
    \label{fig:ieee1}
\end{figure}

A document consists primarily of hierarchical containers, but can also represent other shapes such as floats and header/footer metadata (See Fig. \ref{fig:ieee1}).

\begin{figure}[ht]
    \centering
    \includesvg[width=2.50in]{figures/ieee2-doc-concepts.drawio.svg}
    \caption{Document components shape demonstrating support with MarkerConcept objects}
    \label{fig:ieee2}
\end{figure}

Containers may contain literals and/or concept references. Literals are useful for data values that are not part of a taxonomy, such as numerical values. Otherwise, concepts are preferred, since they can be reused across multiple data and shape graphs (See Fig. \ref{fig:ieee2}).

\noindent\subsubsection*{DPG-SafeD (Safety Data)} Models structured representations of safety data, including hazard classifications and safety measures. It defines how safety-related information is organized and constrained within the data.

\begin{figure}[ht]
    \centering
    \includesvg[width=3.1in]{figures/ieee3-safed-concepts.drawio.svg}
    \caption{Sample Safety Data Shape demonstrating compound hazard classification }
    \label{fig:ieee3}
\end{figure}

SafeD is described by shapes with the purpose of assigning classifications (which are subclasses of SKOS Concepts). For example, a SafeD compound can have a ``GHS Eye Irritation Category 2A'' classification assigned to it, since the SKOS Concept for it is also a SafeD Classification (See Fig. \ref{fig:ieee3}). This classification concept can also be integrated into a broader taxonomy independent of the SafeD shapes.

\begin{figure}[ht]
    \centering
    \includesvg[width=2.6in]{figures/ieee4-safed-rules.drawio.svg}
    \caption{Safety Data Shape demonstrating SHACL-AF rulesets hazard inference enablement for Mixtures of Compounds}
    \label{fig:ieee4}
\end{figure}

Adding Mixture hazard classification into the mix provides opportunities to utilize the capacity to express inference rules via SHACL-AF. In this case, suppose a Mixture can be automatically classified as a GHS Eye Irritation Category 2A because one of its constituent ingredients exceeds a 10\% concentration, and is also classified as 2A (See Fig. \ref{fig:ieee4}). 

\noindent\subsubsection*{DPG-BPC (Batch Process Control)} Represents batch process control structures, aligning with industry standards like ANSI/ISA-88. It models procedural and physical aspects of batch processes. This shape intends to serve as the semantic view into the data represented in batch process records, and aims to ease the composition of data from multiple of such records.
\\\\
\noindent\textbf{Taxonomies}
\noindent\subsubsection*{DPG-GHSrev10} A taxonomy aligned with the Globally Harmonized System (GHS) for classifying and labeling chemicals. It includes concepts for SDS headings, hazard classification categories, and specific hazard classes. This taxonomy can be integrated into both DPG-DoC (as document headings and values) and DPG-SafeD (as safety classifications).

\noindent\subsubsection*{DPG-ISA-88} A taxonomy that describes terms from the ISA-88 standard for batch process control. It provides standardized terminology that can be applied to both DPG-DoC and DPG-BPC shapes.

\noindent\subsubsection*{Potential Taxonomies} Additional hazard taxonomies, which cover the 2012 OSHA HazCom specification, proprietary company batch control processes, and Work Instructions stored in a GMP document store, can be created and used in the same manner as these other taxonomies. The modular architecture of the system allows additional terms to be easy to integrate.

\subsubsection*{Integration of Shapes and Taxonomies}
Using our use case, we elaborate the advantages of separating shapes and taxonomies for greater flexibility as shapes can reference multiple taxonomies, enabling them to incorporate concepts from various domains:
\begin{itemize}
    \item DPG-DoC can utilize taxonomies like DPG-GHS to represent SDS headings within document structures, and DPG-ISA-88 to model process documentation.
    \item DPG-SafeD references DPG-GHS to model safety classifications and hazard information within safety data.
    \item DPG-BPC integrates DPG-ISA-88 and proprietary taxonomies to represent batch process control procedures, such as tablet compression processes.
\end{itemize}
This approach reduces the complexity of ETL processes, as data can be transformed more easily when utilizing a common vocabulary.

\subsection{Platform and Workflow Design}
\noindent\textbf{Industry-Standard Data Platform:} A modern data platform that leverages industry-standard technologies was deployed for this use-case. The platform is built around a service-oriented architecture (SOA) provisioned using Infrastructure as Code (IaC) on Kubernetes clusters in a hybrid cloud environment. This setup ensures scalability, flexibility, and integration with existing infrastructure.
\\\\
\noindent\textbf{Document-Centric ETL Workflows:} The platform focuses on extracting, transforming, and loading (ETL) documents, whether digitally rendered or scanned. Techniques such as Named Entity Recognition (NER), Optical Character Recognition (OCR), and Vision-Language Models (VLM) are used to process documents. By describing extracted data using our standard taxonomies, we facilitate easier transformation and integration into different views.
\\\\
\noindent\textbf{Workflow Orchestration with Apache Airflow:} Apache Airflow  was utilized to orchestrate workflows and enable data-driven scheduling. Airflow allows for the separation of dependencies using virtual environments and supports the use of Kubernetes operators for scalable workloads.
\\\\
\noindent\textbf{Data Storage and Access:} The triplestore, Oxigraph, serves as the system's data store due to its support for read replicas and ease of embedding in applications and workflows. This enables efficient querying and data retrieval using SPARQL.
\\\\
\noindent\textbf{Identity and Access Management:} Keycloak provides identity and access management, integrating with internal identity providers and supporting role-based access control. This ensures secure access to data and services.

\subsection{Workflow Overview}
\noindent The overall Crucible workflow involves several key steps:

\begin{itemize}
    \item \textbf{Document Ingestion:} Users upload documents such as SDS, batch records, or work instructions. The system ingests these documents for processing via web interface.
    \item \textbf{Data Extraction and Annotation:} The system uses NER, OCR, and VLM methods to extract relevant data from the documents. Extracted data is annotated using the appropriate taxonomies (e.g., DPG-GHS, DPG-ISA-88) and validated against the corresponding shapes (e.g., DPG-DoC, DPG-SafeD).
    \item \textbf{Data Integration and Storage:} Annotated data is stored in the triplestore, enabling complex queries and integration with other data sources.
    \item \textbf{Data Transformation:} Standardized data can be transformed into different views or formats as needed. For example, safety data extracted from SDS can be integrated into batch process records or used to generate composite packing sheets.
    \item \textbf{User Interaction:} Users interact with the system through custom interfaces. They can perform tasks such as generating reports, analyzing process variables, or querying hazard classifications.
    \item \textbf{Interoperability and Data Exchange:} The use of standardized taxonomies and shapes facilitates data exchange with other systems, reducing the need for complex ETL transformations.
\end{itemize}

\section{Results}

% \begin{figure}[h]
%     \centering
%     \includesvg[width=3in]{figures/ieee7-overview.drawio.svg}
%     \caption{Multiple shapes with a common taxonomy enables simplified transformations from one to the other. The chemical safety shape shown here includes a rule set to infer a label element used in the cover sheet.}
%     \label{fig:transformoverview}
% \end{figure}

\noindent As a practical example, the conventional method of generating a ``Test Product'' composite packing sheet for shipping materials is quite manual. Existing SDS platforms utilize fairly rigid search mechanisms that operate on a set of fixed string search and ``drill-down'' widgets which require users to apply a number of filters to navigate the search results \(S^{E}\) for compound \(E\). \(S^{E}\) is further distinguished by Manufacturer \(i\), Language \(j\), and Revision Date \(k\). For easier conceptualization, this process is denoted by the following ``equation'': 

\begin{figure*}[hbt!]
    \centering
    \includegraphics[width=6in]{figures/network2.png}
    \caption{Hazard Statements:  Truncated hazard SDS network visualization traversing multiple taxonomies}
    \label{fig:network}
    
\end{figure*}

\begin{figure}[ht]
\centering
\begin{lstlisting}[language=SPARQL,basicstyle=\ttfamily\small]
SELECT ?hazard ?prefLabel ?labelDisplay
WHERE {
  <E> safed:classification ?hazard .
  ?hazard skos:prefLabel ?prefLabel .
  ?hazard safed:labelDisplay ?labelDisplay .
  FILTER (lang(?prefLabel) = "en")
}
\end{lstlisting}
\caption{Example of a query an application may use to retrieve the information necessary to display the hazard classifications of a compound.}
\label{fig:sparql}
\end{figure}

\begin{equation}\label{eq:1}
S^{E} =  \sum S^{E}_{i,j,k}
\end{equation}

\noindent For multiple compounds $\overrightarrow{N}=\{E_1, E_2, ...\}$ contained in a product, search operations to obtain the set of all related SDS \(S^{\overrightarrow{N}}\) would then need to be manually performed (search and download) for every compound $E_i$ where $ i=1,2,...$, as accounted for in a composite shipping and handling cover sheet (e.g. ``Composition/Ingredients'' in Fig. \ref{fig:coversheet}): 
 
 \begin{equation}\label{eq:2}
S^{\overrightarrow{N}} = \sum_{i} S^{E_i}
 \end{equation}

\noindent After obtaining $S^{\overrightarrow{N}}$, users have to compile a comprehensive set of Hazards \(H^{\overrightarrow{N}}\) from hazard statements $D^{E_i}_{P_j}$ listed in different sections $P_j$ in each $S^{E_i}$ of $S^{\overrightarrow{N}}$. To make it explicit for the required manual compilation, we write: 

\begin{equation}\label{eq:3}
H^{\overrightarrow{N}} = \sum_{i}\sum_{j}D^{E_i}_{P_j}
\end{equation}

\noindent Once obtained, hazards of $H^{\overrightarrow{N}}$ are validated against a list of general hazards statements \(H^{gen}\) to get the final set of hazard statements $\{D\}$ to be disclosed on the comprehensive cover sheet (see, for example, ``Hazards Disclosure" and ``Hazard Statement(s) Overview" in Fig. \ref{fig:coversheet}) as: 

\begin{equation}\label{eq:4}
H^{\overrightarrow{N}} \cap H^{gen} = \{ D : D\in H^{\overrightarrow{N}}  \text{ and } D \in H^{gen}\}
\end{equation}

\noindent This manual process has since been automated through the development of an internal system code-named ``Crucible'' (see the modules and workflow in Fig. \ref{fig:resultsprocess}) that utilizes our SKOS-SHACL methodology. 

Using Crucible, users can upload SDS to a graphical user interface (GUI) or select from pre-existing SDS documents. Uploaded SDS are processed via a hosted workflow orchestrator that extracts the Safety Data Sheet information conformant to the SHACL-SKOS models. Specifically, Hazard Pictograms and Hazard Statements are extracted and annotated using DPG-GHS taxonomy and validated against SafeD shape. The extracted information from multiple SDS, is integrated into a downstream knowledge graph -- allowing for aggregation and analysis. Finally, a composite packing sheet is generated by querying the associated labels for multiple SDS. With the returned triples, a table look-up operation is performed to link and return the `Hazard Disclosure' components of SDS into the required composite packing SDS sheet format (PDF) as a summary. 

Essentially, Crucible has automated the manual processes described in Eqs. \ref{eq:1}--\ref{eq:4} and is able to obtain \(H^{\overrightarrow{N}}\) from an internal knowledge graph. Once the user provides the necessary information (SDS selection), iteration is simply utilized to generate the appropriate SPARQL query for Eq. \ref{eq:3} for each compound \(E\) to obtain a Hazard Statement from various sections. Figure \ref{fig:network} illustrates a truncated network visualization of \(H^{\overrightarrow{N}}\) generated from querying the hazard statements for `Acetomenophin 400' \cite{SigmaAldrich_1003009}. Rather than performing the manual operations, we are able to recall \(H^{\overrightarrow{N}}\) with a simple SPARQL query (see Fig. \ref{fig:sparql}) across multiple vendors and through multiple SDS in the English language without relying on complex language by using multiple taxonomies. Using Crucible, we were able to reduce the manual processing time of generating a composite shipping cover sheet, as shown in Fig. \ref{fig:coversheet}, from hours to minutes.

% \begin{figure*}[hbt!]
%     \centering
%     \includegraphics[width=6in]{figures/network2.png}
%     \caption{Hazard Statements:  Truncated hazard SDS network visualization traversing multiple taxonomies}
%     \label{fig:network}
    
% \end{figure*}

% \begin{figure}[h]
% \centering
% \begin{lstlisting}[language=SPARQL]
% SELECT ?hazard ?prefLabel ?labelDisplay
% WHERE {
%   <E> safed:classification ?hazard .
%   ?hazard skos:prefLabel ?prefLabel .
%   ?hazard safed:labelDisplay ?labelDisplay .
%   FILTER (lang(?prefLabel) = "en")
% }
% \end{lstlisting}
% \caption{Example of a query an application may use to retrieve the information necessary to display the hazard classifications of a compound.}
% \label{fig:sparql}
% \end{figure}

Additional benefits to this process are focused on enhanced user experience in which users are able to generate the summary report for shipping and handling information by navigating through a catalog of processed SDS compounds (using fuzzy filtering) and adding them to a ``shopping-cart''. This process is a significant departure from the traditional methods of filtering through drop-down widgets and fixed string search UX-elements for the SDS-specific platforms.

 
%\begin{itemize}
%    \item \textbf{Document Ingestion:} SDS sheets are uploaded and ingested into the system.
%    \item \textbf{Data Extraction:} Safety data is extracted and annotated using DPG-GHS taxonomy and validated against DPG-SafeD shape.
%    \item \textbf{Data Integration}: Extracted safety data is integrated into the knowledge graph, allowing for aggregation and analysis.
%    \item \textbf{Report Generation:} A composite packing sheet is generated by transforming the integrated data into the required format, utilizing standardized terms and classifications.
%    \item \textbf{Benefits:} This process automates manual tasks, reduces errors, and ensures consistency in safety information.
%\end{itemize} 


\begin{figure}[ht]
    \centering
    \includegraphics[width=3in]{figures/results_process.png}
    \caption{Data Platform Services:  This platform automates manual tasks, reduces errors, and ensures consistency in safety information.}
    \label{fig:resultsprocess}
\end{figure}



\begin{figure}[ht]
    \centering
    \includegraphics[width=3.4in]{figures/cover-sheet-redacted.png}
    \caption{Example cover sheet generated by Crucible}
    \label{fig:coversheet}
\end{figure}

\section{Conclusion}
% TODO: review
%% Ti-chiun and Stephen Here
There has been significant effort put into the construction of ontologies and KRR in the pharmaceutical industry, as discussed in Section \ref{intro}. However, the adoption of the KRR models into operational systems has been slow. Among many challenges are the agreement on vocabulary, the complexity of integrating knowledge of multiple domains, and the lack of an adaptive, scalable, and flexible SHACL-SKOS KRR system architecture. Our implementation, with its flexibility and attentiveness, shows a promising pathway to overcome many of these challenges. By separating shapes and taxonomies and allowing shapes to reference multiple taxonomies, we achieve:

\begin{itemize}
    \item \textbf{Modularity:} Components can be developed, maintained, and updated independently, enhancing flexibility.
    \item \textbf{Reusability:} Taxonomies can be reused across different shapes and applications, reducing duplication of effort.
    \item \textbf{Interoperability:} Standardized vocabularies and structures enable seamless data exchange between systems.
    \item \textbf{Reduced Complexity:} Simplifying the modeling approach lowers the complexity of ETL processes and data transformations.
    \item \textbf{Scalability:} The architecture supports scaling to accommodate growing data volumes and additional domains.
\end{itemize}
In particular, SKOS(-XL) offers the capability to incorporate alternative expressions (in the form of languages, terms, related concepts) and SHACL can enforce data quality assurance. Moreover, our emphasis on interoperability and incorporation of multiple ontologies demonstrates the potential to expand to related workflows in the pharmaceutical development, although our current use case is limited to KRR of pharmaceutically relevant material SDS.


% conference papers do not normally have an appendix


% use section* for acknowledgment
%\section*{Acknowledgment}
%\noindent Prof. Mark Daniel Ward (The Data Mine) \\
%Merck iLT  \\
%Rich Osifchin \\


%The authors would like to thank The Data Mine at Purdue University for their continuious support as well as th


% Note that the IEEE typically puts floats only at the top, even when this
% results in a large percentage of a column being occupied by floats.


% An example of a double column floating figure using two subfigures.
% (The subfig.sty package must be loaded for this to work.)
% The subfigure \label commands are set within each subfloat command,
% and the \label for the overall figure must come after \caption.
% \hfil is used as a separator to get equal spacing.
% Watch out that the combined width of all the subfigures on a 
% line do not exceed the text width or a line break will occur.
%
%\begin{figure*}[!t]
%\centering
%\subfloat[Case I]{\includegraphics[width=2.5in]{box}%
%\label{fig_first_case}}
%\hfil
%\subfloat[Case II]{\includegraphics[width=2.5in]{box}%
%\label{fig_second_case}}
%\caption{Simulation results for the network.}
%\label{fig_sim}
%\end{figure*}
%
% Note that often IEEE papers with subfigures do not employ subfigure
% captions (using the optional argument to \subfloat[]), but instead will
% reference/describe all of them (a), (b), etc., within the main caption.
% Be aware that for subfig.sty to generate the (a), (b), etc., subfigure
% labels, the optional argument to \subfloat must be present. If a
% subcaption is not desired, just leave its contents blank,
% e.g., \subfloat[].


% An example of a floating table. Note that, for IEEE style tables, the
% \caption command should come BEFORE the table and, given that table
% captions serve much like titles, are usually capitalized except for words
% such as a, an, and, as, at, but, by, for, in, nor, of, on, or, the, to
% and up, which are usually not capitalized unless they are the first or
% last word of the caption. Table text will default to \footnotesize as
% the IEEE normally uses this smaller font for tables.
% The \label must come after \caption as always.
%



% Note that the IEEE does not put floats in the very first column
% - or typically anywhere on the first page for that matter. Also,
% in-text middle ("here") positioning is typically not used, but it
% is allowed and encouraged for Computer Society conferences (but
% not Computer Society journals). Most IEEE journals/conferences use
% top floats exclusively. 
% Note that, LaTeX2e, unlike IEEE journals/conferences, places
% footnotes above bottom floats. This can be corrected via the
% \fnbelowfloat command of the stfloats package.

% trigger a \newpage just before the given reference
% number - used to balance the columns on the last page
% adjust value as needed - may need to be readjusted if
% the document is modified later
%\IEEEtriggeratref{8}
% The "triggered" command can be changed if desired:
%\IEEEtriggercmd{\enlargethispage{-5in}}

% references section

% can use a bibliography generated by BibTeX as a .bbl file
% BibTeX documentation can be easily obtained at:
% http://mirror.ctan.org/biblio/bibtex/contrib/doc/
% The IEEEtran BibTeX style support page is at:
% http://www.michaelshell.org/tex/ieeetran/bibtex/
\bibliographystyle{IEEEtran}
\bibliography{ref}
\
% argument is your BibTeX string definitions and bibliography database(s)

% that's all folks
\end{document}
