\section{Related Work}
Game-theoretic models for the formation of networks have been a core topic in algorithmic game theory for almost three decades. Starting with the seminal work by Jackson and Wolinsky~\cite{JW96}, many variants of such models have been proposed and rigorously analyzed. In~\cite{JW96} $n$ selfish agents establish links among each other to maximize their individual benefit. The mutual benefit of a node pair $u,v$ in the formed network decreases with their distance. Our model can be seen as a simplified binary variant, where only nodes in distance~1 and 2 give a benefit of $1$ and all the others provide benefit~$0$. The PoA for certain parameter ranges of the connections model was analyzed by Baumann and Stiller~\cite{BS08}.

Later, the network creation game (NCG)~\cite{fabrikantNetwork2003} was proposed and it became the basis of almost all newer models, including our model. In the NCG, the agents buy incident edges for a price of $\alpha$ per edge and their goal is to minimize their closeness centrality, i.e., their sum of shortest-path distances to all other nodes. It is shown that NE always exist, being either a spanning star or a clique, that computing a best response strategy is NP-hard, and that the price of anarchy is bounded by $O(\sqrt{\alpha})$. Later, in a series of works~\cite{albersNash2014,demainePrice2007,MS10,MMM13,GHLL16,AM19,BL20,AB23} this bound has been shown to be constant for almost any value of $\alpha$, with the best general upper bound being $2^{O(\sqrt{\log n})}$~\cite{demainePrice2007}.

Many variants of the NCG have been studied, e.g., versions that involve cooperation~\cite{CP05,AFM09,FGLZ23}, non-uniform edge cost~\cite{MeiromMO14,CLMH14,CLMM17,BiloFLLM21}, robustness~\cite{MeiromMO15,CLMM16,Goyal16,Echzell0LM20}, geometric aspects~\cite{MoscibrodaSW06a,EidenbenzKZ06,bilo2019}, social networks~\cite{BiloFLLM21}, social distancing~\cite{social_distancing}, temporal networks~\cite{temporal_NCG,temporal_non_local}, or variants featuring locality~\cite{CL15,BiloGLP16,bilo_traceroute}. 
Close to our model is the bounded distance NCG by Bilò, Gualà, and Proietti~\cite{BiloGP15}. 
There, the agents have to ensure that all other nodes are in a given maximum or average distance $D$. For $D=2$ this is similar to our model with the main difference, that in our model the agents face a trade-off between their total cost for buying edges and the number of agents that are not within distance $D$. This can be seen as a soft constraint. For $D=2$ the authors show a PoA in $\Omega(\sqrt{n})$ and $O(\sqrt{n\log n})$.

The closest related NCG variant is the celebrity game by {\`A}lvarez, Blesa, Duch, Messegu{\'e} and Serna~\cite{ABDMS16celebrity}.
In this framework, each agent has an individual weight, and there is a general distance threshold denoted as $\beta$. The cost function for an agent is the sum of the edge cost and the total weight of agents located at a distance greater than $\beta$. Consequently, our model represents a specific instance of their star celebrity games with $\beta=2$ where all agents have a uniform weight of $w_{min}=w_{max}=1$. However, unlike the star celebrity games in \cite{ABDMS16celebrity}, we mostly focus on $\alpha > 1$, which implies the absence of "celebrities" in our model, i.e.\ nodes with weight larger than $\alpha$.
The authors of~\cite{ABDMS16celebrity} show the existence of stable networks for any $\alpha$. Further, they provide a bound on the diameter of NE of at most $2\beta+1$. For $\beta=2$ we improve this to $3$ and we show that this is tight. Moreover, it is shown that the PoS is 1 and that the PoA is upper bounded by $\bigO(\min\{\frac{n}{2},\frac{n}{\alpha}\})$. Both bounds carry over to our model. 
For $\beta=2$, they show a lower bound on the Price of Anarchy (PoA) that is linear in $n$. However, this result crucially relies on inconsistent tie-breaking when agents are indifferent about purchasing an edge. 
Besides bounds on the PoA, in~\cite{ABDMS16celebrity} it is also shown that computing a best response is NP-hard. This proof uses agents having non-uniform weights, which means that it does not apply to our setting, either.
Lastly, the focus on the 2-neighborhood is also central to the model by Anshelevich, Bhardwaj, and Usher~\cite{AnshelevichBU15}. In their model, agents strategically allocate effort shares to edges in a given network, with utility depending on the invested effort in their 2-neighborhood. This contrasts strongly with our model.