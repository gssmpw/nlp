% This must be in the first 5 lines to tell arXiv to use pdfLaTeX, which is strongly recommended.
\pdfoutput=1
% In particular, the hyperref package requires pdfLaTeX in order to break URLs across lines.

\PassOptionsToPackage{table,xcdraw}{xcolor}

\documentclass[11pt]{article}

% Remove the "review" option to generate the final version.
\usepackage[final]{acl}
% \usepackage[review]{acl}

% Standard package includes
\usepackage{times}
\usepackage{latexsym}

% For proper rendering and hyphenation of words containing Latin characters (including in bib files)
\usepackage[T1]{fontenc}
% For Vietnamese characters
% \usepackage[T5]{fontenc}
% See https://www.latex-project.org/help/documentation/encguide.pdf for other character sets

% This assumes your files are encoded as UTF8
\usepackage[utf8]{inputenc}

% This is not strictly necessary, and may be commented out,
% but it will improve the layout of the manuscript,
% and will typically save some space.
\usepackage{microtype}

% This is also not strictly necessary, and may be commented out.
% However, it will improve the aesthetics of text in
% the typewriter font.
\usepackage{inconsolata}


% If the title and author information does not fit in the area allocated, uncomment the following
%
%\setlength\titlebox{<dim>}
%
% and set <dim> to something 5cm or larger.

\usepackage{booktabs}
\usepackage{multirow}
\usepackage{multicol}
\usepackage{makecell}
\usepackage{enumitem}
\usepackage{balance}
\usepackage{graphicx}
\usepackage{subfigure}
\usepackage{amsmath}
\usepackage{amsthm}
\usepackage{amssymb}
\usepackage{amsfonts}
\usepackage{hyperref}
%\usepackage[table,xcdraw]{xcolor} 
% use \PassOptionsToPackage{table,xcdraw}{xcolor} before acl.sty

\usepackage{booktabs}
\usepackage{soul}
\usepackage{todonotes}
\usepackage{xspace}

\newcommand{\eval}{\textsc{Dict}\xspace}
\newcommand{\model}{\textsc{Traver}\xspace}

\title{Training Turn-by-Turn Verifiers for Dialogue Tutoring Agents: \\ The Curious Case of LLMs as Your Coding Tutors}

\author{%
    Jian Wang$^{1,2}$\thanks{\ \ This work is conducted while Jian Wang was a visiting PhD at the University of Michigan.} \quad  Yinpei Dai$^{2}$  \quad  Yichi Zhang$^2$  \\
    {\bf Ziqiao Ma$^2$  \qquad  Wenjie Li$^1$  \qquad  Joyce Chai$^2$}  \\
    $^1$The Hong Kong Polytechnic University \quad
    $^2$University of Michigan  \\
    \texttt{jian-dylan.wang@connect.polyu.hk  ~~ cswjli@comp.polyu.edu.hk} \\
    \texttt{\{daiyp,zhangyic,marstin,chaijy\}@umich.edu}
}

\begin{document}
\maketitle


During the early stages of interface design, designers need to produce multiple sketches to explore a design space.  Design tools often fail to support this critical stage, because they insist on specifying more details than necessary. Although recent advances in generative AI have raised hopes of solving this issue, in practice they fail because expressing loose ideas in a prompt is impractical. In this paper, we propose a diffusion-based approach to the low-effort generation of interface sketches. It breaks new ground by allowing flexible control of the generation process via three types of inputs: A) prompts, B) wireframes, and C) visual flows. The designer can provide any combination of these as input at any level of detail, and will get a diverse gallery of low-fidelity solutions in response. The unique benefit is that large design spaces can be explored rapidly with very little effort in input-specification. We present qualitative results for various combinations of input specifications. Additionally, we demonstrate that our model aligns more accurately with these specifications than other models. 

% OLD ABSTRACT
%When sketching Graphical User Interfaces (GUIs), designers need to explore several aspects of visual design simultaneously, such as how to guide the user’s attention to the right aspects of the design while making the intended functionality visible. Although current Large Language Models (LLMs) can generate GUIs, they do not offer the finer level of control necessary for this kind of exploration. To address this, we propose a diffusion-based model with multi-modal conditional generation. In practice, our model optionally takes semantic segmentation, prompt guidance, and flow direction to generate multiple GUIs that are aligned with the input design specifications. It produces multiple examples. We demonstrate that our approach outperforms baseline methods in producing desirable GUIs and meets the desired visual flow.

% Designing visually engaging Graphical User Interfaces (GUIs) is a challenge in HCI research. Effective GUI design must balance visual properties, like color and positioning, with user behaviors to ensure GUIs easy to comprehend and guide attention to critical elements. Modern GUIs, with their complex combinations of text, images, and interactive components, make it difficult to maintain a coherent visual flow during design.
% Although current Large Language Models (LLMs) can generate GUIs, they often lack the fine control necessary for ensuring a coherent visual flow. To address this, we propose a diffusion-based model that effectively handles multi-modal conditional generation. Our model takes semantic segmentation, optional prompt guidance, and ordered viewing elements to generate high-fidelity GUIs that are aligned with the input design specifications.
% We demonstrate that our approach outperforms baseline methods in producing desirable GUIs and meets the desired visual flow. Moreover, a user study involving XX designers indicates that our model enhances the efficiency of the GUI design ideation process and provides designers with greater control compared to existing methods.    



% %%%%%%%%%%%%%%%%%%%%%%%%%%%%%%%%%%%%%%%%%%%%%%%%%%%%%%
% % Writing Clinic Comments:
% %%%%%%%%%%%%%%%%%%%%%%%%%%%%%%%%%%%%%%%%%%%%%%%%%%%%%%
% % Define: Effective UI design
% % Motivate GANs and write in full form.
% % LLMs vs ControlNet vs GANs
% % Say something about the Figma plugin?
% % Write the work is novel or what has been done before
% % What is desirable UI and how to evalutate that?
% % Visual Flow - main theme (center around it)
% % Re-Title: use word Flow!
% % Use ControlNet++ & SPADE for abstract.
% % Write about input/output. 
% % Why better than previous work?
% %%%%%%%%%%%%%%%%%%%%%%%%%%%%%%%%%%%%%%%%%%%%%%%%%%%%%

% % v2:
% % \noindent \textcolor{red}{\textbf{NEW Abstract!} (Post Writing Clinic 1 - 25-Jun)}

% % \noindent \textcolor{red}{----------------------------------------------------------------------}

% % \noindent Designing user interfaces (UIs) is a time-consuming process, particularly for novice designers. 
% % Creating UI designs that are effective in market funneling or any other designer defined goal requires a good understanding of the visual flow to guide users' attention to UI elements in the desired order. 
% % While current Large Language Models (LLMs) can generate UIs from just prompts, they often lack finer pixel-precise control and fail to consider visual flow. 
% % In this work, we present a UI synthesis method that incorporates visual flow alongside prompts and semantic layouts. 
% % Our efficient approach uses a carefully designed Generative Adversarial Network (GAN) optimized for scenarios with limited data, making it more suitable than diffusion-based and large vision-language models.
% % We demonstrate that our method produces more "desirable" UIs according to the well-known contrast, repetition, alignment, and proximity principles of design. 
% % We further validate our method through comprehensive automatic non-reference, human-preference aligned network scoring and subjective human evaluations.
% % Finally, an evaluation with xx non-expert designers using our contributed Figma plugin shows that <method-name> improves the time-efficiency as well as the overall quality of the UI design development cycle.

% % \noindent \textcolor{red}{----------------------------------------------------------------------}


% \noindent \textcolor{blue}{\textbf{NEW Abstract!} (Pre Writing Clinic 9-July)}

% \noindent \textcolor{blue}{----------------------------------------------------------------------}

% \noindent Exploring different graphical user interface (GUI) design ideas is time-consuming, particularly for novice designers. 
% Given the segmentation masks, design requirement as prompt, and/or preferred visual flow, we aim to facilitate creative exploration for GUI design and generate different UI designs for inspiration.
% While current Vision Language Models (VLMs) can generate GUIs from just prompts, they often lack control over visual concepts and flow that are difficult to convey through language during the generation process. 
% In this work, we present FlowGenUI, a semantic map-guided GUI synthesis method that optionally incorporates visual flow information based on the user's choice alongside language prompts. 
% We demonstrate that our model not only creates more realistic GUIs but also creates "predictable" (how users pay attention to and order of looking at GUI elements) GUIs.
% Our approach uses Stable Diffusion (SD), a large paired image-text pretrained diffusion model with a rich latent space that we steer toward realistic GUIs using a trainable copy of SD's encoder for every condition (segmentation masks, prompts, and visual flow). 
% We further provide a semantic typography feature to create custom text-fonts and styles while also alleviating SD's inherent limitations in drawing coherent, meaningful and correct aspect-ratio text. 
% Finally, a subjective evaluation study of XX non-expert and expert designers demonstrates the efficiency and fidelity of our method.


% This process encourages creativity and prevents designers from falling into habitual patterns.


% ------------------------------------------------------------------
% Joongi Why is it important to create realistic GUI?
% I do not see how the Visual Flow given on the left hand side is reflected in the results on the right hand side. 
% I’d avoid making unsubstantiated claims about designers (falling into habitual patterns).
% The UIs you generate do not “align with users’ attention patterns” but rather try to control it (that’s what visual flow means)
% ------------------------------------------------------------------
% Comments - Writing Clinic - 9th July:
% Improve title. More names: FlowGen
% Figure 1: Use an inference time hand-drawn mask
% Figure 1: Show both workflows. Add a designer --> Input.
% Figure 1: Make them more diverse
% ------------------------------------------------------------------
% Designing graphical user interfaces (GUIs) requires human creativity and time. Designers often fall into habitual patterns, which can limit the exploration of new ideas. 
% To address this, we introduce FlowGenUI, a method that facilitates creative exploration and generates diverse GUI designs for inspiration. By using segmentation masks, design requirements as prompts, and/or selected visual flows, our approach enhances control over the visual concepts and flows during the generation process, which current Vision Language Models (VLMs) often lack.
% FlowGenUI uses Stable Diffusion (SD), a largely pretrained text-to-image diffusion model, and guides it to create realistic GUIs. 
% We achieve this by using a trainable copy of SD's encoder for each condition (segmentation masks, prompts, and visual flow). 
% This method enables the creation of more realistic and predictable GUIs that align with users' attention patterns and their preferred order of viewing elements.
% We also offer a semantic typography feature that creates custom text fonts and styles while addressing SD's limitations in generating coherent, meaningful, and correctly aspect-ratio text.
% Our approach's efficiency and fidelity are evaluated through a subjective user study involving XX designers. 
% The results demonstrate the effectiveness of FlowGenUI in generating high-quality GUI designs that meet user requirements and visual expectations.

% ---------------------------------------


%A critical and general issue remains while using such deep generative priors: creating coherent, meaningful and correct aspect-ratio text. 
%We tackle this issue within our framework and additionally provide a semantic typography feature to create custom text-fonts and styles. 


% %Creating UI designs that are effective in market funneling or any other designer-defined goal requires a good understanding of the visual flow to guide users' attention to UI elements in the desired order. 
% %While current largely pre-trained Vision Language Models (VLMs) can generate GUIs from just prompts, they often lack finer or pixel-precise control which can be crucial for many easy-to-understand visual concepts but difficult to convey through language. 
% % However, obtaining such pixe-level labels is an extremely expensive so we
% % For example - overlaying text on images with certain aspect ratios and two equally separated buttons 
% Additionally, all prior GUI generation work fails to consider visual flow information during the generation process. 
% We demonstrate that visual flow-informed generation not only creates more realistic and human-friendly GUIs but also creates "predictable" (how users pay attention to and order of looking at GUI elements) UIs that could be beneficial for designers for tasks like creating effective market funnels.
% In this work, we present a semantic map-guided GUI synthesis method that optionally incorporates visual flow information based on the user's choice alongside language prompts. 
% Our approach uses Stable Diffusion, a large (billions) paired image-text pretrained diffusion model with a rich latent space that we steer toward realistic GUIs using an ensemble of ControlNets. 
% % TODO: Mention it in 1 sentence:
% A critical and general issue remains while using such deep generative priors: creating coherent, meaningful and correct aspect-ratio text. 
% We tackle this issue within our framework and additionally provide a semantic typography feature to create custom text-fonts and styles. 
% To evaluate our method, we demonstrate that our method produces more "desirable" UIs according to the well-known contrast, repetition, alignment, and proximity principles of design. 
% % We further validate our method through comprehensive automatic non-reference and human-preference aligned scores. (TODO: Maybe Unskip if we get UIClip from Jason!)
% % TODO: Re-word this and only keep ideation cycles and time-efficiency.
% Finally, a subjective evaluation study of XX non-expert and expert designers demonstrates the efficiency and fidelity of our method.
% % improves the time-efficiency by quick iterations of the UI design ideation process.
% %Finally, an evaluation with xx non-expert designers using our contributed <method-name> improves the time-efficiency by quick iterations of the UI design ideation cycle.

%\noindent \textcolor{blue}{----------------------------------------------------------------------}


%In an evaluation with xx designers, we found that GenerativeLayout: 1) enhances designers' exploration by expanding the coverage of the design space, 2) reduces the time required for exploration, and 3) maintains a perceived level of control similar to that of manual exploration.



% Present-day graphical user interfaces (GUIs) exhibit diverse arrangements of text, graphics, and interactive elements such as buttons and menus, but representations of GUIs have not kept up. They do not encapsulate both semantic and visuo-spatial relationships among elements. %\color{red} 
% To seize machine learning's potential for GUIs more efficiently, \papername~ exploits graph neural networks to capture individual elements' properties and their semantic—visuo-spatial constraints in a layout. The learned representation demonstrated its effectiveness in multiple tasks, especially generating designs in a challenging GUI autocompletion task, which involved predicting the positions of remaining unplaced elements in a partially completed GUI. The new model's suggestions showed alignment and visual appeal superior to the baseline method and received higher subjective ratings for preference. 
% Furthermore, we demonstrate the practical benefits and efficiency advantages designers perceive when utilizing our model as an autocompletion plug-in.


% Overall pipeline: Maybe drop semantic typography / visual flow?
\section{Introduction}

Tutoring has long been recognized as one of the most effective methods for enhancing human learning outcomes and addressing educational disparities~\citep{hill2005effects}. 
By providing personalized guidance to students, intelligent tutoring systems (ITS) have proven to be nearly as effective as human tutors in fostering deep understanding and skill acquisition, with research showing comparable learning gains~\citep{vanlehn2011relative,rus2013recent}.
More recently, the advancement of large language models (LLMs) has offered unprecedented opportunities to replicate these benefits in tutoring agents~\citep{dan2023educhat,jin2024teach,chen2024empowering}, unlocking the enormous potential to solve knowledge-intensive tasks such as answering complex questions or clarifying concepts.


\begin{figure}[t!]
\centering
\includegraphics[width=1.0\linewidth]{Figs/Fig.intro.pdf}
\caption{An illustration of coding tutoring, where a tutor aims to proactively guide students toward completing a target coding task while adapting to students' varying levels of background knowledge. \vspace{-5pt}}
\label{fig:example}
\end{figure}

\begin{figure}[t!]
\centering
\includegraphics[width=1.0\linewidth]{Figs/Fig.scaling.pdf}
\caption{\textsc{Traver} with the trained verifier shows inference-time scaling for coding tutoring (detailed in \S\ref{sec:scaling_analysis}). \textbf{Left}: Performance vs. sampled candidate utterances per turn. \textbf{Right}: Performance vs. total tokens consumed per tutoring session. \vspace{-15pt}}
\label{fig:scale}
\end{figure}


Previous research has extensively explored tutoring in educational fields, including language learning~\cite{swartz2012intelligent,stasaski-etal-2020-cima}, math reasoning~\cite{demszky-hill-2023-ncte,macina-etal-2023-mathdial}, and scientific concept education~\cite{yuan-etal-2024-boosting,yang2024leveraging}. 
Most aim to enhance students' understanding of target knowledge by employing pedagogical strategies such as recommending exercises~\cite{deng2023towards} or selecting teaching examples~\cite{ross-andreas-2024-toward}. 
However, these approaches fall short in broader situations requiring both understanding and practical application of specific pieces of knowledge to solve real-world, goal-driven problems. 
Such scenarios demand tutors to proactively guide people toward completing targeted tasks (e.g., coding).
Furthermore, the tutoring outcomes are challenging to assess since targeted tasks can often be completed by open-ended solutions.



To bridge this gap, we introduce \textbf{coding tutoring}, a promising yet underexplored task for LLM agents.
As illustrated in Figure~\ref{fig:example}, the tutor is provided with a target coding task and task-specific knowledge (e.g., cross-file dependencies and reference solutions), while the student is given only the coding task. The tutor does not know the student's prior knowledge about the task.
Coding tutoring requires the tutor to proactively guide the student toward completing the target task through dialogue.
This is inherently a goal-oriented process where tutors guide students using task-specific knowledge to achieve predefined objectives. 
Effective tutoring requires personalization, as tutors must adapt their guidance and communication style to students with varying levels of prior knowledge. 


Developing effective tutoring agents is challenging because off-the-shelf LLMs lack grounding to task-specific knowledge and interaction context.
Specifically, tutoring requires \textit{epistemic grounding}~\citep{tsai2016concept}, where domain expertise and assessment can vary significantly, and \textit{communicative grounding}~\citep{chai2018language}, necessary for proactively adapting communications to students' current knowledge.
To address these challenges, we propose the \textbf{Tra}ce-and-\textbf{Ver}ify (\textbf{\model}) agent workflow for building effective LLM-powered coding tutors. 
Leveraging knowledge tracing (KT)~\citep{corbett1994knowledge,scarlatos2024exploring}, \model explicitly estimates a student's knowledge state at each turn, which drives the tutor agents to adapt their language to fill the gaps in task-specific knowledge during utterance generation. 
Drawing inspiration from value-guided search mechanisms~\citep{lightman2023let,wang2024math,zhang2024rest}, \model incorporates a turn-by-turn reward model as a verifier to rank candidate utterances. 
By sampling more candidate tutor utterances during inference (see Figure~\ref{fig:scale}), \model ensures the selection of optimal utterances that prioritize goal-driven guidance and advance the tutoring progression effectively. 
Furthermore, we present \textbf{Di}alogue for \textbf{C}oding \textbf{T}utoring (\textbf{\eval}), an automatic protocol designed to assess the performance of tutoring agents. 
\eval employs code generation tests and simulated students with varying levels of programming expertise for evaluation. While human evaluation remains the gold standard for assessing tutoring agents, its reliance on time-intensive and costly processes often hinders rapid iteration during development. 
By leveraging simulated students, \eval serves as an efficient and scalable proxy, enabling reproducible assessments and accelerated agent improvement prior to final human validation. 



Through extensive experiments, we show that agents developed by \model consistently demonstrate higher success rates in guiding students to complete target coding tasks compared to baseline methods. We present detailed ablation studies, human evaluations, and an inference time scaling analysis, highlighting the transferability and scalability of our tutoring agent workflow.

\section{Problem Definition}

\begin{figure*}[th!]
\centering
\includegraphics[width=1.0\textwidth]{Figs/Fig.overview.pdf}
\caption{Overview of our work for developing coding tutoring agents. \textbf{Left}: The context of the coding tutoring problem. \textbf{Middle}: Trace-and-Verify (\textsc{Traver}) workflow. \textbf{Right}: \textsc{Dict} evaluation protocol.
}
\label{fig:overview}
\end{figure*}



We formulate coding tutoring as an interactive dialogue process between a \textbf{tutor} and a \textbf{student}, where the goal is to help the student implement a working solution that passes predefined unit tests for a target coding task.

Formally, the tutor is assigned a coding task $\mathcal{T}$ that consists of a function signature and a requirement description outlining the desired functionality. 
The tasks are repository-level, which require an understanding of multiple interdependent files within the codebase to implement a correct solution.
The tutor has access to task-specific knowledge $\mathcal{K}$, which includes (\romannumeral1) \textit{Code Contexts}: Contextual code snippets surrounding the desired code, which help the tutor show examples when necessary; (\romannumeral2) \textit{Reference Dependencies}: Cross-referenced elements such as intra-class, intra-file, and cross-file dependencies, along with their corresponding descriptions (e.g., docstrings), which involve key knowledge for completing the desired code; and (\romannumeral3) \textit{Reference Solution Steps}: Key steps required to complete the target task, describing using natural languages. 

The student is given the task $\mathcal{T}$ and possesses some subset of $\mathcal{K}$ as their prior knowledge, but the tutor remains unaware of which specific concepts or dependencies the student has already mastered.
The goal of the tutor is to guide the student, regardless of his or her background, toward solving the task $\mathcal{T}$ through multi-turn interactions.

\section{Synthesizing Attribution Data}

\begin{figure*}[ht]
    \centering
    \includegraphics[width=\textwidth]{img/pipeline.drawio.pdf}
    \caption{\textbf{Top:} The \synatt baseline method for synthetic attribution data generation. Given context and question-answer pairs, we prompt an LLM to identify supporting sentences, which are then used to train a smaller attribution model. However, this discriminative approach may yield noisy training data as LLMs are less suited for classification tasks (see \S\ref{sec:experiments-zero-shot}). \textbf{Bottom:} The \synqa data generation pipeline leverages LLMs' generative strengths through four steps: (1) collection of Wikipedia articles as source data; (2) extraction of context attributions by creating chains of sentences that form hops between articles; (3) generation of QA pairs by prompting an LLM with only these context attribution sentences; (4) compilation of the final training samples, each containing the generated QA pair, its context attributions, and the original articles enriched with related distractors.}
    % \caption{\textbf{Top:} The \synatt baseline. Intuitively, we can prompt an LLM for context-attribution by providing the context and question-answer pairs. Then, we train a smaller model on the obtained synthetic data. However, LLMs are less suitable for discriminative (i.e., classification) tasks, and may yield noisy training data (see \S\ref{sec:experiments-zero-shot}). \textbf{Bottom:} The \synqa data generation pipeline consists of four main steps: (1) collection of Wikipedia articles as the source data; (2) extracting the context attributions by creating chains of sentences that form hops between articles; (3) generation of QA pairs by prompting an LLM with only the context attribution sentences; (4) we obtain the resulting \synqa training sample containing three components: the generated QA pair, the context attributions, and the original articles supplemented with related distractor articles.}
    \label{fig:method}
\end{figure*}

Context attribution identifies which parts of a reference text support a given question-answer pair~\cite{rashkin2023measuring}. Formally, given a question $q$, its answer $a$, and a context text $c$ consisting of sentences ${s_1, ..., s_n}$, the task is to identify the subset of sentences $S \subseteq c$ that fully support the answer $a$ to question $q$. To train efficient attribution models without requiring expensive human annotations, we explore synthetic data generation approaches using LLMs.
% Context attribution poses the following question~\cite{rashkin2023measuring}: given a generated text $t_g$ and a context text $t_c$, is $t_g$ attributable to $t_c$? To train models to perform well on this task, we explore how to best generate synthetic attribution data using LLMs. We implement two methods: a discriminative and generative method. 
We implement two methods for synthetic data generation. Our baseline method (\synatt) is discriminative: given existing question-answer pairs and their context, an LLM identifies supporting sentences, which are then used to train a smaller attribution model. Our proposed method (\synqa) takes a generative approach: given selected context sentences, an LLM generates question-answer pairs that are fully supported by these sentences. This approach better leverages LLMs' natural strengths in text generation while ensuring clear attribution paths in the synthetic training data.

%The first method is relatively straightforward and termed \synatt. A simple way to generate synthetic data for context attribution is to ask an LLM to pick out the sentences that support a given question-answer pair. 

% \subsection{Discriminative and Generative Synthetic Data Generation}

% The first method (\synatt) is relatively straightforward: ask the LLM to pick relevant sentences from a provided context that support a given question-answer pair. However, this \textit{discriminative} approach of performing sentence classification overlooks the fact that LLMs excel at \textit{generating} text. Therefore, we design a second data generation method (\synqa) that is generative and thus capitalizes on the strength of LLMs. It involves the following pipeline steps (see also Fig.~\ref{fig:method}): context collection, question-answering generation and distractor mining, which increases the difficulty of the task, thus reflecting more realistic scenarios.

%\textbf{Attribution Synthesis.} The most straightforward approach to generating synthetic data for context attribution is discriminative: prompting an LLM to identify relevant sentences from context documents given a question-answer pair. While intuitive, this approach underutilizes LLMs' capabilities, as they excel at generative rather than discriminative tasks. LLMs are fundamentally designed to generate coherent text following instructions rather than perform binary classification of sentences. In our experiments (\S\ref{sec:experiments}) we dub this method as \synatt.

\subsection{\synqa: Generative Synthetic Data Generation Method}

\synqa consists of three parts: context selection, QA generation, and distractors mining (for an illustration of the method, see Figure~\ref{fig:method}). In what follows, we describe each part in detail.

\textbf{Context Collection.} We use Wikipedia as our data source, as each article consists of sentences about a coherent and connected topic, with two collection strategies. In the first, we select individual Wikipedia articles for dialogue-centric generation and use their sentences as context. In the second, for multi-hop reasoning, we identify sentences containing Wikipedia links and follow these links to create ``hops'' between articles, limiting to a maximum of two paths to maintain semantic coherence, while enabling more complex reasoning patterns (for more details, see Appendix~\ref{app:synthetic_data}).
% \textbf{Context Collection.}  The first step is to select a dataset where each data point is a set of sentences about a coherent and connected topic. These sentences will serve as the context in which we want to find relevant attributions later. We use Wikipedia as the data source
%To better leverage LLMs' generative capabilities, we propose \synqa, a novel and simple approach for synthesizing context attribution data (see Fig.~\ref{fig:method}). 
%We first collect Wikipedia articles that are not present in our testing datasets\footnote{We detect potential data leakage by representing each Wikipedia article as a MinHash signature. Then, for each training Wikipedia article, we retrieve candidates from the testing datasets via Locality Sensitivity Hashing and compute their Jaccard similarity \cite{dasgupta2011fast}. Pairs exceeding a tunable threshold (empirically set to 0.8) are flagged as potential leaks.}.
%For each article, 
% we implement two distinct collection strategies that differ in difficulty. First, we select individual Wikipedia articles and randomly select multiple sentences within each article. Second, we start from a randomly selected sentence containing at least one Wikipedia link
%\footnote{These are human annotated in the Wikipedia articles, or alternatively, can be obtained from entity linking methods \cite{de-cao-etal-2022-multilingual}.} 
% and follow the links to other articles, creating ``hops'' between related content. We limit the chain to a maximum of two hops (connecting up to three articles) to maintain semantic coherence while enabling the more difficult multi-hop reasoning scenarios (for more details, see Appendix~\ref{app:synthetic_data}). 
%In the second strategy, we select individual Wikipedia articles and randomly select multiple sentences within each article that can serve as evidence for generated questions.

\textbf{Question-Answer Generation.} Given the set of contexts, an LLM can now generate question-answer pairs. For single articles, we prompt the model to generate multiple question-answer pairs, each grounded in specific sentences. This creates a set of dialogue-centric samples where questions build upon the previous context. For linked articles, we prompt the model to generate questions that necessitate connecting information across the articles, encouraging multi-hop reasoning.
%\footnote{Note that multi-hop reasoning is not guranteed here; rather, the LLM has the ability to decide whether the question-answer pair involves multiple hops of reasoning. See App. for details.}. 
This yields multi-hop samples requiring integration of information across documents, as well as samples that mimic a dialogue about a specific topic given the context. We provide the full prompts used for generation in Appendix \ref{app:prompts}.

\textbf{Distractors Mining.} To make the attribution task more realistic, we augment each sample with distractor articles. With E5 \cite{wang2022text}, we embed each Wikipedia article in our collection. For each article in the training sample, we randomly select up to three distractors with the highest semantic similarity to the source articles. These distractors share thematic elements with the source articles, but lack information to answer the questions.%do not contain the information necessary to answer the generated questions.

\subsection{Advantages of \synqa}
The \synqa approach has three key advantages:
%over discriminative data generation:
% (1) it leverages LLMs' natural strength in generative tasks; (2) produces diverse multi-hop reasoning scenarios; and (3) creates coherent question-answer pairs with clear attribution paths.
(1) it leverages LLMs' strength in generation rather than classification; (2) creates diverse training samples requiring both dialogue understanding and multi-hop reasoning; and (3) ensures generated questions have clear attribution paths since they are derived from specific context sentences.
By generating both entity-centric and dialogue-centric samples, \synqa produces training data that reflects the variety of real-world QA scenarios, helping models develop robust attribution capabilities, which our experiments demonstrate to generalize across different contexts and domains.
% We formalize the problem of Context Attribution QA as follows: Given a pre-defined context $T_c=\lbrace s_1, s_2, \ldots , s_n \rbrace$---where $s_i$ is a sentence---and an answer text $t_a$ generated by an LLM, the context attribution model should provide a vector $a=(a_1, \ldots , a_n)$, where each element $a_i$ has the following possible values:
% \[
% a_i =
% \begin{cases}
%     1, & \text{if } s_i \text{ supports the generated answer } t_a\\
%     0,  & \text{otherwise} 
% \end{cases}
% \]
% In our setup, we should have at least one entry $a_i = 1$.
% \begin{itemize}
%     \item The simplest way to generate synthetic data for context-attribution is in a discriminative manner: we prompt an LLM to provide the sentence level context attributions given the context documents, question and answer. We deem this generation as discriminative as the model effectively classifies the sentences that are most relevant to the question-answer pair.
%     \item The issue with this approach is that LLM are not best suitable for discriminative tasks, but rather generative. That is, an LLM is better at generating text by following instructions, than classifing sentences/etc.
%     \item To leverage what LLMs are good for, we create a simple context attribution data generation approach where we perform the following: (1) We find wikipedia articles (which are not contained in the testing datasets)\footnote{Describe the approach for dealing with data leakage}; (2) We select a random sentence in a wikipedia article, and find the links to other wikipedia articles (the hops). We select that sentence, and hop to the other Wikipedia article (given by the link). (3) We perform the hop step for maximum of 2 times (i.e., we connect at most 3 articles, and 1 at least). We end up with 3 Wikipedia articles which constitute the hops.
%     \item We provide Llama70B with either 1 wikipedia article or the hops and ask the model to generate a multi-hop question-answer pair which ideally connects all connected articles, or as many as it can; alternatively, if we provide the model with only 1 wikipedia article, we ask the model to select as many sentences as possible in the article, and for each, generate a question-answer pair (we provide the full prompts we use in Appendix).
%     \item The output of the model is a set of question-answer pairs (or a single one), that is grounded in the evidence provided by the sentence(s). We dub the entire approach as \synqa.
%     \item In summary, we develop two settings to generate synthetic data for context attribution in question answering: one is entity-centric and yield data which might be multi-hop; and the other is dialog-centric where subsequent questions build on top of previous ones.
%     \item Finally, to all context + question + answer + context-attribution samples we add distractors: we obtain embeddings using E5 of each wikipedia page, and for each sample we select up to 3 distractors which we add to the data sample. These distractors are similar are document with similar context as the one from which the context-attributions are.
% \end{itemize}


\section{Evaluation}
\label{sec:eval}

We conduct experiments to evaluate the effectiveness of \ours{}.
We first compare the performance of models trained with \ours{}
against other LLMs of a similar scale across various benchmarks. We then
carry out an in-depth analysis to ensure 
instruction tuning is executed correctly through
competitive evaluation. We also investigate 
whether the instruction generation of \ours{} is more cost-effective 
than that of Self-Instruct, mainly focusing on
how much diversity-based filtering reduces costs. 
Finally, we explore the impact
of model collapse and the potential safety issue of \ours{}.
In Appendix, a more detailed
investigation of the impact of the iterative feedback task generation, along
with other experiment details, are provided. 
All codes are available at \url{https://github.com/}. 

\subsection{Training Recipe}
\textbf{Our model.}
We use the Llama-3-8B model~\cite{llama3} as the base model
and train it with 30,164 instructions generated using \ours{}.
The seed instructions used at
the beginning of data generation are the same as those from
Self-Instruct. 
For detailed hyperparameters, please refer to Appendix.

\noindent \textbf{Baseline models.}
We compare \ours{} with four different models:
Llama-3-8B-Instruct, Llama-3-8B + Self-Instruct,
Falcon-7B-Instruct, and Gemma-7B-Instruct.
Llama-3-8B-Instruct represents the ideal instruction-tuned 
model. It is based on Llama-3-8B and is tuned using 10M 
manually collected instructions.
Llama-3-8B-Instruct has also been trained 
using Reinforcement Learning with Human Feedback (RLHF)~\cite{rlhf} 
and supervised fine-tuning (SFT)~\cite{sft1} to further 
enhance its performance. Such optimizations
enable Llama-3-8B-Instruct to outperform the other models.

Llama-3-8B + Self-Instruct is also based on Llama-3-8B,
but unlike Llama-3-8B-Instruct,
it is trained with instructions synthesized using Self-Instruct.

We also include Falcon-7B-Instruct and Gemma-7B-Instruct,
which have similar model sizes (7-8 billion parameters),
to evaluate \ours{} against models other than Llama-3-8B.
%to show the standing of our models. 
For detailed information on the models, please refer to Appendix.


\begin{comment}
\cancel{
For the comparisons, we use several instruction-tuned
models. 
We also utilize the Llama-3-8B model as the base model and
train it with 30,164 instructions, but they are generated 
using Self-Instruct. By comparing this model with our model, 
we can fairly and directly compare \ours{} with Self-Instruct.
We include a Llama-3-8B-Instruct as a baseline, and this model has 
conducted instruction tuning on the same base model as ours, but 
it was trained with 10M manually collected instructions. 
Also, this model has been trained with Reinforcement Learning with 
Human Feedback (RLHF)~\cite{rlhf} and supervised
fine-tuning (SFT)~\cite{sft1} to improve the model's 
performance further. This model can be considered the peak performance 
of an instruction-tuned model based on the Llama-3-8B model. By 
comparing this model with ours, we can assess how close we are to 
the peak performance.
Additionally, we evaluate other 7-8 B-sized models, including 
Falcon-7B-Instruct and Gemma-7B-Instruct,
to show the standing of our models. 
For detailed information on the models, please refer to Appendix
%For detailed information on the model training technique and instruction 
%collection method, as well as the number of instructions used in training, 
%please refer to Appendix.
}
\end{comment}

\noindent \textbf{Hardware setup.} 
We use a machine that has two AMD EPYC 7742 3.3GHz
64-core CPUs and 2TB DDR4 DRAM. The machine is also equipped with eight RTX-3090
GPUs.
We use Ubuntu 22.04 as the OS and the version of
Python packages are torch 2.1.2 and deepspeed 0.14.4.
%In this section, we describe the training recipe used for evaluating our
%methodologies. For all our experiments, we selected LLaMA-3-8B as the
%pre-trained model. To generate instruction data, we used two models:
%LLaMA-3-8B-Instruct and GPT-3.5-Turbo. Ultimately, we fine-tune models using our
%methodology and compare them with with models fine-tuned using the training
%recipe of Alpaca that is based on the Self-Instruct method. We represent our
%model that using LLaMA-3-8B-Instruct and GPT-3.5-Turbo generated dataset as
%$\mathrm{Ours}_{\mathrm{LLaMA}}$ and $\mathrm{Ours}_{\mathrm{GPT}}$, respectively. Similarly, the
%baseline alpaca model represented as $\mathrm{Alpaca}_{\mathrm{LLaMA}}$ and
%$\mathrm{Alpaca}_{\mathrm{GPT}}$. 
\begin{comment}
Additionally, we adjusted the training datasets to evaluate
how well the models performed on specific tasks. This involved modifying the
prompts used for few-shot learning. The details of the modified prompts can be
found in the Appendix.
\end{comment}

\subsection{Benchmark Performance}
We evaluate the models using
various datasets, including AlpacaEval~\cite{alpacaeval}, MMLU~\cite{MMLU}, Hellaswag~\cite{hellaswag},
and ARC~\cite{DROP}. 
We measure a win rate for AlpacaEval by comparing 
outputs from the models against those from GPT-4. %the reference model, GPT-4. 
A higher win rate indicates better alignment with expected responses. 
For the other benchmarks, we measure accuracy, 
representing the probability of correctly answering questions.
We measure the accuracy of MMLU in a 5-shot setting and 
the accuracy of Hellaswag and ARC in a zero-shot setting~\cite{scalingllm}.



\TAB{tab:benchmarks} shows the results,
where a higher value indicates better performance. 
Except for Llama-3-8B-Instruct which presents the ideal performance
with instruction tuning yet requires serious human efforts to create
seed instructions,
\ours{} outperforms all the other models we chose to compare.
% Llama-3-8B-Instruct가 가장 좋지만, 쓰기 힘든 이유를 여기다 적으면 좋을 것 같은데,,,
% Except for Llama-3-8B-Instruct, SEDI-INSTRUCT outperforms all the other models we chose to compare. Llama-3-8B-Instruct shows the most outstanding performance, but achieving such model performance requires an enormous cost in data collection.
Notably, despite using a more cost-effective data generation method,
\ours{} outperforms the Self-Instruct based model, showing 
5.2\% higher
accuracy on average. 
% 원래 XX - YY%라고 되어있었는데 MMLU에서 0.1% 더 높아서... 살짝 우리 모델이 멍청해보일까봐 average로 바꿨습니다.
% 계산 (((5.4/4.6-1)+(56.6/56.5-1)+(56.1/55.7-1)+(69.3/67.7-1))/4) * 100 = 5.2
% for the datasets.
As will be discussed later, this higher accuracy is achieved with 36\%
lower training cost compared to Self-Instruct.
%\ours{} lowers the cost of training the model
%by up to 36\% compared to Self-Instruct.


\begin{comment}
\subsubsection{AlpacaEval}
First, we evaluate the models on AlpacaEval~\cite{alpacaeval}. AlpacaEval is an
automatic evaluator designed for instruction-following language models. It provides
a quick, cost-effective, and high-quality way to assess these models by generating a
leaderboard and annotations for model outputs. AlpacaEval uses advanced
configurations and annotators to ensure a high agreement rate with human annotations,
making it a reliable tool for evaluating the performance of language models across
various tasks.

\subsubsection{MMLU}
First, we evaluate the models on AlpacaEval~\cite{alpacaeval}. AlpacaEval is an
automatic evaluator designed for instruction-following language models. It provides
a quick, cost-effective, and high-quality way to assess these models by generating a
leaderboard and annotations for model outputs. AlpacaEval uses advanced
configurations and annotators to ensure a high agreement rate with human annotations,
making it a reliable tool for evaluating the performance of language models across
various tasks.
\end{comment}


\begin{figure}[t]
    \centering
    \includegraphics[width=0.88\linewidth]{Figure/Evaluation/compettitive.eps}
    \caption{Competitive evaluation results}
    \label{fig:competitive}
\end{figure}

% 학습한 모델의 출력을 response라는 용어로 통일
\subsection{Competitive Evaluation}
\begin{comment}
\cancel{
In this experiment, for a direct comparison of the quality of the model's responses,
referring to previous studies~\cite{alpagasus, alpacafarm}, we employ an 
automated evaluation method using an API-based LLM to compare our model and baseline.
The responses from both models are input into an evaluation model (e.g., GPT-4), 
which is instructed to assign a score between 1 and 10 for each response.
\review{To mitigate positional bias, 
we evaluate both when our model’s response is placed before and when it is placed after the baseline model’s response.
The final outcome is defined as ``Win-Tie-Lose"; ``Win" means our model wins twice for both orders, ``Tie" means wins once and loses once, and ``Lose" means our model loses twice.}
The datasets used for the competition are the Vicuna test set (Vicuna)~\cite{vicuna1}, Anthropic's 
helpful test set (Helpful)~\cite{helpful}, the Koala test set (Koala)~\cite{koala}, the Open Assistant test set (OASST)~\cite{oasst}, 
and the Self-Instruct test set (Self-Instruct)~\cite{self-instruct}.}
\end{comment}
To assess the quality of the model's responses,
we make use of an automated competitive evaluation method 
that utilizes LLMs to compare the quality of responses~\cite{alpagasus, alpacafarm}. 
We compare our model (Llama-3-8B + \ours{}) 
with Llama-3-8B + Self-Instruct.
The responses from the models are input to GPT-4
which assigns a score between 1 and 10 for each response.
To mitigate a positional bias, 
we measure scores in two different orders:
first, when the responses from Llama-3-8B + \ours{} are input into GPT-4 
before those from Llama-3-8B + Self-Instruct, and second, when they are input afterward.
The final outcome is defined as ``Win-Tie-Lose"; ``Win" means our model wins twice for both orders, 
``Tie" means wins and loses once, and ``Lose" means our model loses twice.
The datasets used for the competition are the Vicuna test set (Vicuna)~\cite{vicuna1}, Anthropic's 
helpful test set (Helpful)~\cite{helpful}, the Koala test set (Koala)~\cite{koala}, the Open Assistant test set (OASST)~\cite{oasst}, 
and the Self-Instruct test set (Self-Instruct)~\cite{self-instruct}.


\FIG{fig:competitive} illustrates the results. 
%We conduct a competitive 
%evaluation of our model (Llama-3-8B + \ours{}) against Llama-3-8B + Self-Instruct. 
As the results indicate, our model outperforms the 
Self-Instruct-based model for all of the five test sets.
It demonstrates that \ours{} generates more 
effective instructions for training than Self-Instruct through the iterative
feedback task generation.
%\review{These results also imply that our diversity-based filtering does not negatively 
%impact not only previous benchmark performance but also the model response quality.}



\subsection{Cost Analysis}

\begin{comment}
\cancel{
We evaluate the efficiency of \ours{} in generating instructions compared to
Self-Instruct method. The experiment measured the number of API requests sent
and the API cost to generate 10,000 kept instructions, meaning instructions that 
passed the filtering process. As shown in \FIG{fig:filtering-eval}(a), \ours{} 
demonstrates an improvement in efficiency by requiring fewer API requests to generate 
the 10,000 instructions. This efficiency gain is achieved by reducing the number of 
discarded instructions. Notably, as shown in previous experiments, this less strict 
filtering approach does not negatively impact model training, as it maintains the 
diversity of the batch configuration.
}
\end{comment}

We evaluate the cost effectiveness of \ours{} in generating instructions 
compared to Self-Instruct. We measure the number of API invocations to
ChatGPT, along with the cost of using the ChatGPT service,
required to generate 10,000 kept instructions.
As shown in \FIG{fig:filtering-eval}(a), \ours{} 
requires 36\% fewer API invocations than Self-Instruct to
generate 10,000 instructions.
As expected, this gain is achieved by reducing the number of 
discarded instructions through the diversity-based filtering with the relaxed similarity threshold.
Despite fewer API calls, \ours{} outperforms Self-Instruct 
in terms of model accuracy as we discussed before.
%Again, the cost efficiency of \ours{} is obtained with higher model accuracy
%than Self-Instruct.
%without negatively affecting the accuracy of the model.


Such an increase in the efficiency of generating kept instructions 
reduces the overall cost of using the ChatGPT service.
For our experiment, 
we utilize the GPT-3.5-turbo-instruct API~\cite{gpt3.5-api}, 
which charges \$1.5 per 1M tokens. 
As illustrated in \FIG{fig:filtering-eval}(b), 
\ours{} achieves 1.6$\times$ reduction in the cost 
due to fewer API calls.
%needed to create 10,000 kept instructions.
Also, as shown in \FIG{fig:filtering-eval}(a), 
the efficiency gap between \ours{} and
Self-Instruct gets wider as more instructions are generated. 
As a result, \ours{} has the potential to achieve greater cost savings.


\begin{figure}[t]% 
	\centering
	\subfloat[\# of kept per API calls]{\includegraphics[width=0.55\linewidth]{Figure/Evaluation/cost-1.eps}}
	\subfloat[Cost]{\includegraphics[width=0.36\linewidth]{Figure/Evaluation/cost-2.eps}}
	\caption{Instruction data generating cost analysis}%
	\label{fig:filtering-eval}%
\end{figure}
We conduct our experiments on a wide variety of tabular datasets (refer Section \ref{sec:datasets}) with varying levels of data heterogeneity across a variety of downstream classification tasks (refer Section \ref{sec:tabglm}).

\begin{table*}[ht]
    \centering
    \scriptsize
    \begin{tabular}{l|cccccc|ccc}
    \toprule
    \multirow{2}{*}{\textbf{Dataset}} & \multicolumn{9}{c}{\textbf{Performance (AUC-ROC)}} \\
    \cline{2-10}
             & \textbf{\tabglm} & \textbf{CatBoost} & \textbf{GB} & \textbf{LR} & \textbf{RF} & \textbf{XGBoost} & \textbf{Tab} & \textbf{FT-} & \textbf{NODE} \\
             & (ours) &   &  &   &  &  & \textbf{Transformer} & \textbf{Transformer} &  \\
    \midrule \midrule
    bank         & 92.07 & \textbf{93.51}  & 92.36 & 86.76  & 92.46 & 92.84 & 90.05 &  92.07 & \underline{92.67} \\
    blood        & \textbf{78.48} & 74.94 & 72.24 & \underline{76.76} & 70.77 & 69.51 & 74.26 & 74.98 & 76.21 \\
    calhousing   & \textbf{95.47} & \underline{93.55} & 92.47 & 90.84 & 93.45  & 81.99 & 83.13 & 93.62 & 93.84 \\
    car          & 99.40 & \textbf{99.97} & \underline{99.83} & 78.46 & 99.41 & 99.92 & 98.57 & 98.51 & 99.64 \\
    coil2000     & \underline{74.17} & 73.97 & \textbf{74.66} & 73.22 & 69.43 & 71.19 & 71.64 & 65.59 & 73.09 \\
    creditg      & 79.32 & \textbf{80.54} & 78.36 & 75.21 & 79.76 & 76.81 & 79.40 & 56.60 & \underline{79.83} \\
    diabetes     & \textbf{83.70} & \underline{82.55} & 82.34 & 82.89 & 81.65 & 79.17 & 82.72 & 82.34 & 82.18 \\
    heart        & \textbf{93.29} & \underline{92.61} & 92.00 & 90.74 & 91.92 & 91.16 & 92.16 & 91.81 & \underline{92.61} \\
    % jungle       & 88.98 & \underline{97.42 & 93.05 & 80.89 & 93.79 & \textbf{97.52 & 80.73 & 81.00 & 95.51 \\
    kr-vs-kp     & \underline{99.43} & \textbf{99.95} & 99.77 & 99.15 & 99.86 & 99.95  & 99.30 & 86.79 & 99.41 \\
    mfeat-fourier & 99.94 & \underline{99.97} & 99.62 & \textbf{100.00} & 99.99 & 99.70 & 99.99 & 99.92 & \textbf{100.00} \\
    pc3          & \textbf{82.82} & \underline{82.48} & 80.80 & 79.44 & 80.89 & 77.76 & 79.02 & 76.57 & 81.00 \\
    income       & \textbf{92.59} & 92.44 & 91.75 & 79.03 & 89.19 & \underline{92.35} & 89.63 & 70.57 & 90.30 \\
    texture      & \textbf{100.0} & \underline{99.98} & 99.93 & 99.87 & 99.94 & 99.96 & 99.98 & 99.94 & 99.94 \\
    balance-scale       & \textbf{99.10} & 92.35 & \underline{98.37} & 93.11 & 84.89 & 98.99 & 91.60 & 91.03 & 94.41 \\
    mfeat-karhunen      & \textbf{99.88} & \underline{99.86} & 99.79 & 99.52 & 99.71 & 98.69 & 99.56 & 98.85 & \textbf{99.88} \\
    mfeat-morphological & \textbf{96.99} & 96.20 & 96.01 & 95.74 & 95.53 & 96.12 & 95.75 & 96.33 & \underline{96.34} \\
    mfeat-zernike       & \textbf{98.09} & \underline{97.59} & 97.16 & 97.74 & 96.72 & 97.35 & 98.02 & 97.76 & 97.49 \\
    cmc                 & \textbf{74.45} & 72.56 & \underline{72.89} & 70.41 & 70.52 & 73.00 & 69.96 & 71.56 & \underline{73.88} \\
    tic-tac-toe         & \underline{99.85} & 99.92 & 99.81 & 72.00 & 96.12 & \textbf{99.98} & 70.90 & 72.76 & 98.82 \\
    vehicle             & \textbf{94.50} & \underline{93.02} & 92.33 & 88.79 & 93.23 & 92.84 & 93.19 & 90.50 & 91.61 \\
    eucalyptus          & \textbf{91.95} & \underline{88.59} & 89.31 & 87.45 & 90.11 & 90.04 & 88.27 & 89.98 & 89.70 \\
    analcatdata\_author  & \textbf{58.96} & \underline{55.89} & 54.61 & 53.56 & 53.20 & 57.43 & 53.63 & 53.94 & 55.50 \\
    MiceProtein         & \underline{99.98} & \textbf{99.99} & 99.97 & 99.51 & 99.85 & \underline{99.98} & 99.91 & 99.41 & 99.97 \\
    steel-plates-fault  & 94.52 & \underline{96.51} & \textbf{96.26} & 91.35 & 91.71 & 96.56 & 91.91 & 91.92 & 94.45 \\
    dress-sales         & \textbf{57.89} & \underline{56.96} & 55.93 & 55.94 & 53.72 & 57.23 & 53.38 & 54.41 & 52.62 \\ \midrule
    \textbf{Average} & \textbf{89.47} & 88.64 & 88.34 & 84.69 & \underline{86.96} & 87.62 & 85.84 & 83.91 & 88.22 \\
    \bottomrule
    \end{tabular}
    \caption{\textbf{Comparison of performance (AUC-ROC) of existing approaches in tabular Machine Learning against \tabglm}. Our proposed method \tabglm\ achieves significant performance gains across 25 classification datasets. The best performing model is in \textbf{bold} while the second best is \underline{underlined}.}
    \label{tab:sota_perf_contrast}
\end{table*}


\subsection{Datasets}
\label{sec:datasets}
To demonstrate the effectiveness of \tabglm\ in the presence of heterogeneous feature columns, as discussed in Section \ref{sec:tabglm}, we conduct our experiments on 25 datasets encompassing both binary and multi-class classification tasks, curated from popular papers TabLLM~\cite{tabllm}, TabPFN~\cite{hollmann2022tabpfn}, and large scale datasets in OpenML~\cite{openml2017}.
Following the principal goal of \tabglm, we consider heterogeneous datasets that encapsulate both numerical and textual columns like \textbf{Bank} ($\sim$45k records with 7 numerical and 9 categorical columns), \textbf{Creditg} (1k rows with 7 numerical and 13 categorical columns), \textbf{Heart} (918 rows with 6 numerical and 5 categorical columns) and \textbf{Income} ($\sim$48k rows with 4 numerical and 8 categorical columns), as shown in TabLLM. In addition, we use 12 datasets from OpenML, containing at least 1 numerical and 1 categorical column, including \textbf{balance-scale} (5 numerical and 1 categorical), \textbf{tic-tac-toe} (10 numerical and 10 categorical), \textbf{dress-sales} (13 numerical and 12 categorical) etc with more details in appendix. 
We also include datasets containing only numerical columns like \textbf{blood} (4 numerical columns), \textbf{calhousing} (8 numerical columns), \textbf{coil2000} (86 numerical columns) etc. alongside datasets containing only categorical columns like \textbf{car}, from both OpenML and TabPFN. 
We adopted datasets of varying sizes, with number of rows ranging from 500 (in \textbf{dress-sales}) to 45,211 (in \textbf{bank}) to demonstrate the applicability of our method to real-world large tabular datasets.
Note that the multi-modal architecture in \tabglm\ involves a LLM encoder~\cite{tapas, tapex} that is limited by the number of input tokens, which is 512 (from TAPAS) in our case.
More details on each dataset experimented upon in Table \ref{tab:sota_perf_contrast} is discussed in the supplementary material.\looseness-1

\subsection{Experimental Setup}
We conduct our experiments on datasets discussed in Section \ref{sec:datasets} and report the average performance (AUC-ROC scores) of each model across the same 5 random seeds (kept constant across datasets) in Section \ref{sec:results}. 
For all numerical and heterogeneous datasets, numerical columns are normalized using min-max\footnote{Scikit learn package: \url{https://scikit-learn.org/stable/modules/generated/sklearn.preprocessing.MinMaxScaler.html}} normalization while the categorical (text) columns are converted into One-Hot encodings (refer ablation in supplementary material) to create a numeric dataset for graph transformation.
For the text transformation, each record in the table is converted to serialized text following the tokenizer in TAPAS~\cite{tapas}. We chose TAPAS based on ablation experiments on the choice of LLMs in Section \ref{sec:ablations}. 
For datasets that contain only categorical columns, our \tabglm\ method uses only the text pipeline, utilizing only the semantic information present in such datasets.
Models for all datasets are trained on a fixed set of hyperparameters with an initial learning rate of $1e^{-4}$, batch size of 256 and weighting the consistency loss at 20\% ($\lambda = 0.2$). 
All experiments are conducted on 4 NVIDIA V100 GPUs with additional details on the experiment setup in the Appendix and code released at \url{https://github.com/amajee11us/TabGLM}.\looseness-1

\begin{table}[t]
\centering
\scriptsize
    \begin{tabular}{l|cccc}
    \toprule
    \multirow{3}{*}{ \textbf{Dataset} } & \multicolumn{4}{c}{ \textbf{Tabular DL Methods} } \\ \cline{2-5}
                               & \textbf{\tabglm} & \textbf{IGNNet} & \textbf{TabLLM} & \textbf{TabPFN} \\
                               & (multi-modal) & (graph) & (text) &  \\
    \midrule \midrule
        bank         & 92.07 & 91.11 & 91.20 & 91.19 \\
        blood        & \textbf{78.48} & 74.09 & 74.03 & 77.01 \\
        calhousing   & \textbf{95.47} & 94.79 & 95.38 & 95.31 \\
        car          & 99.40 & 50.16 & \textbf{99.99} & 99.53 \\
        % coil2000     & 74.17 &  &  & 72.51 \\
        creditg      & 79.32 & 71.99 & 70.82 & \textbf{80.79} \\
        diabetes     & \textbf{83.70} & 77.79 & 80.40 & 73.67 \\
        heart        & 93.29 & 92.06 & \textbf{94.21} & 82.60 \\
        jungle       & 88.98 & 88.98 & 93.00 & 87.36 \\
        income       & \textbf{92.59} & 90.76 & 92.19 & 90.14 \\ \midrule
        \textbf{Average} & \textbf{89.26} & 81.30 & 87.91 & 86.40 \\ \hline
    \end{tabular}
\caption{\textbf{Comparison of performance (AUC-ROC) of \tabglm\ against benchmark datasets in TabLLM} \cite{tabllm}. Results from all methods are averaged over five seeds.}
\label{tab:dl_model_benchmark}
\end{table}

\subsection{Results}
\label{sec:results}
At first, we compare the performance of \tabglm\ with traditional linear and tree based Machine Learning models like CatBoost~\cite{prokhorenkova2018catboost}, XGBoost~\cite{chen2016xgboost}, Gradient Boosting (GB) \cite{ke2017lightgbm}, Random Forest (RF) \cite{breiman2001random} and Logistic Regression (LR).
Our results in Table \ref{tab:sota_perf_contrast} show that \tabglm\ demonstrates significant increase in AUROC of 4.77\% over LR, 2.51\% over RF etc. outperforming such techniques across 25 downstream tabular classification tasks.
However, for simple datasets with lower number of feature columns like \textbf{kr-vs-kp}, \textbf{pc3} etc., tree based models (CatBoost) continue to show dominance in performance.\looseness-1

Secondly, we compare the performance of \tabglm\ with SoTA tabular DL models like FT-Transformer~\cite{gorishniy2021revisiting}, TabTransformer~\cite{huang2020tabtransformer} and NODE~\cite{popov2019neural}. \tabglm\ consistently outperforms tabular DL models like FT-Transformer by 5.56\%, TabTransformer by 3.64\% and NODE by 1.26\% respectively.
Finally, we compare the performance of \tabglm\ against SoTA uni-modal DL architectures like IGNNet (table-to-graph) and TabLLM (table-to-text) on 9 datasets in the benchmark introduced in TabLLM~\citet{tabllm}. We observe that \tabglm\ outperforms TabLLM by 1.35\% and IGNNet by 7.96\% respectively on the benchmark datasets in \cite{tabllm}, summarized in Table \ref{tab:dl_model_benchmark}.
The above results indicate a strong generalization of the proposed \tabglm\ architecture to a variety of downstream tasks, establishing \tabglm\ as a strong choice for Tabular Deep Learning under feature heterogeneity.\looseness-1

\subsection{Ablation Study}
\label{sec:ablations}

\begin{table}[t]
\centering
\scriptsize
\begin{tabular}{l|ccc}
\toprule
\multirow{3}{*}{ \textbf{Dataset} } & \multicolumn{3}{c}{ \textbf{Methods} } \\ \cline{2 - 4}
                           & \multirow{2}{*}{ \textbf{TabLLM} } &  \textbf{\tabglm} & \textbf{\tabglm} \\
                                    &  & (w TAPEX encoder) & (w TAPAS encoder) \\
\midrule 
{Param. Count}& 2.9B& 336M & 129M\\
\midrule
blood              &  71.78   &   77.57   &  \textbf{78.48} \\
calhousing         &  95.00   &   95.29   &  \textbf{95.47} \\
creditg            &  78.56   &   78.72   &  \textbf{79.32} \\
\bottomrule
\end{tabular}
\caption{\textbf{Ablation on the Choice of LLM} architecture for the text transformation module of \tabglm.}
\label{tab:choice_of_llm}
\end{table}

\noindent \textbf{Multi-Modal vs. Uni-Modal training:}
The core contribution of \tabglm\ lies in its multi-modal architecture for tabular representation learning. To evaluate its components, we decompose it into two uni-modal architectures: \textit{Graph only} (using only the graph encoder $E_{\text{graph}}$) and \textit{Text only} (using only the text encoder $E_{\text{text}}$), based on the choice of the feature extractor during both training and inference. 
Their performance is compared against the complete multi-modal \tabglm\ training recipe, with results summarized in Table \ref{tab:tabglm_components}. 
The \textit{Graph only} pipeline employs the GNN from \cite{ignnet}, while the \textit{Text only} pipeline uses the BART-based TAPAS~\cite{tapas} encoder. 
Both pipelines use the same classifier head (Section \ref{sec:tabglm}) for downstream tasks. In \textbf{Text only}, the encoder is frozen, and only the classifier head is trained, whereas in \textit{Graph only}, both the encoder and classifier head are trained, to ensure fair comparison with \tabglm, where the text encoder remains frozen during training. 
Experiments on three representative datasets—\textbf{pc3} (numerical), \textbf{bank} (balanced numerical and categorical), and \textbf{creditg} (categorical-heavy)—show that \tabglm's multi-modal design consistently outperforms its uni-modal variants, underscoring the value of modality fusion for learning from heterogeneous tables.\looseness-1

\begin{table}[t]
      \centering
      \scriptsize
        \begin{tabular}{l|cc|c}
            \toprule
            \multirow{2}{*}{\textbf{Dataset}} & \textbf{Graph Trans.} & \textbf{Text Trans.} & \multirow{2}{*}{\textbf{AUCROC}} \\              
                                    &  ($E_{graph}$) & ($E_{text}$) &         \\
            \midrule \midrule
            \multirow{3}{*}{pc3}    & \checkmark  &           &  77.04 \\
                                    &            & \checkmark &  78.24 \\
                                    & \checkmark & \checkmark &  \textbf{82.82} \\
            \hline
            \multirow{3}{*}{bank }  & \checkmark &            &  91.11 \\
                                    &            & \checkmark &  90.52\\
                                    & \checkmark & \checkmark &  \textbf{92.07} \\
            \hline
            \multirow{3}{*}{Creditg}& \checkmark &            &  71.99 \\
                                    &            & \checkmark &  77.36\\
                                    & \checkmark & \checkmark &  \textbf{79.32} \\
            \bottomrule
      \end{tabular}
      \caption{\textbf{Ablations on the graph and text components of the proposed \tabglm\ approach}. Results are averaged over five seeds. }
      \label{tab:tabglm_components}
\end{table}

\noindent \textbf{Choice of LLM architecture for Text Transformation:}
The choice of the pretrained LLM architecture plays a crucial role in improving the model performance of \tabglm. 
While larger LLMs like \cite{sun2023gpt, tabllm, tapex} ($\geq$7 billion parameters) can encode superior semantic features in complex text, it also adds a significant computational overhead. Additionally, their benefits may be negligible when dealing with simpler semantic content.
To address this trade off, we conducted an ablation experiment by varying the architecture of the text encoder ($E_{text}$) across three popular LLM models - TAPAS~\cite{tapas}, TAPEX~\cite{tapex} and TabLLM~\cite{tabllm}.
For all three settings we adopt the complete multi-modal training strategy, modifying only the text encoder $E_{text}$. 
The results from this experiment, shown in Table \ref{tab:choice_of_llm}, highlight that TAPAS~\footnote{We adopt the TAPAS-base model from \url{https://huggingface.co/google/tapas-base}}, a smaller parameter count, BERT~\cite{devlin2018bert} based text encoder, outperforms other larger models like TAPEX~\cite{tapex}. We thus adopt this architecture for the text transformation pipeline in \tabglm.\looseness-1

\section{Related Work}

\paragraph{Interactive Tutoring.}
As an advanced form of intelligent tutoring systems (ITSs), interactive ITSs~\citep{graesser2001intelligent,rus2013recent} can provide personalized feedback and adaptive learning experiences. 
They have been extensively explored across various educational domains, such as language learning~\citep{swartz2012intelligent,stasaski-etal-2020-cima,caines-etal-2020-teacher, kwon-etal-2024-biped}, math reasoning~\citep{demszky-hill-2023-ncte,macina-etal-2023-mathdial,wang-etal-2024-bridging}, and scientific concept education~\citep{yuan-etal-2024-boosting,yang2024leveraging}. 
These studies mainly focus on enhancing students' understanding of specific pieces of knowledge, using pedagogical strategies such as designing exercises~\citep{deng2023towards,wang-etal-2022-towards,lu2023readingquizmaker} or selecting teaching examples~\citep{ross-andreas-2024-toward}.
Furthermore, the effectiveness of these approaches is often measured using closed-form assessments, such as question-answering~\cite{yuan-etal-2024-boosting} or multiple-choice tests~\cite{macina-etal-2023-mathdial}. 
Rather than specific pieces of knowledge, we investigate the domain of coding tutoring, which requires students to perform open-ended code generation to assess tutoring further.



\paragraph{LLM-based Tutoring Agents.}
The rapid growth of large language models (LLMs) has expanded ITSs into tutoring agents~\citep{yu2024mooc}. 
For instance, early efforts such as EduChat~\citep{dan2023educhat} introduced an educational chatbot for online tutoring, while \textsc{ChatTutor}~\citep{chen2024empowering} equipped tutor agents with course planning and adaptive quizzes to facilitate long-term interactions. 
More recently, AlgoBo~\citep{jin2024teach}, an LLM-based teachable agent, was developed to enhance students' coding skills. 
Existing LLM agents primarily play a \textit{reactive} role, focusing on answering questions or clarifying concepts. 
In comparison, our coding tutoring is both goal-driven and personalized, requiring agents to \textit{proactively} guide students toward completing targeted coding tasks while adapting to diverse levels of knowledge prior. 
Our work presents a novel method that empowers tutor agents to address these challenges.


\paragraph{Inference-Time Adaptation of LLMs}
To enhance the controllability of language generation in complex tasks, prior work has investigated guided decoding~\citep{dathathri2020plug,chaffin-etal-2022-ppl} during inference. 
More recently, a notable line of research~\citep{lightman2023let,li-etal-2023-making,wang2024math,pan2024training} has employed verifier models complemented with search algorithms to guide LLMs for agentic reasoning. 
These methods typically focus on static tasks, often overlooking interactive scenarios.
To address multi-turn interactions~\citep{wang2024mint}, we introduce a turn-by-turn verifier that dynamically evaluates tutoring progress over time.
\section{Conclusion}\label{sec:conclusion}
This work introduces a novel approach to TOT query elicitation, leveraging LLMs and human participants to move beyond the limitations of CQA-based datasets. Through system rank correlation and linguistic similarity validation, we demonstrate that LLM- and human-elicited queries can effectively support the simulated evaluation of TOT retrieval systems. Our findings highlight the potential for expanding TOT retrieval research into underrepresented domains while ensuring scalability and reproducibility. The released datasets and source code provide a foundation for future research, enabling further advancements in TOT retrieval evaluation and system development.


\section*{Limitations}
In this work, we employed LLMs to simulate students at different knowledge levels, serving as a proxy for real-world learners. While these simulated students offer convenience and scalability, their representation of the human learning process is inherently limited. The role-playing behavior may differ from that of actual students in tutoring scenarios. Future research should focus on improving the reliability of student simulation to better align with real-world human learning.

In addition, the tutor agents were primarily evaluated by interacting with simulated students in our experimental setup. It remains unclear how these agents would perform when guiding humans toward completing target coding tasks. An important direction for future work is to extend our evaluation protocol by incorporating human-in-the-loop assessments, where tutor agents interact directly with actual students with necessary programming backgrounds. This would offer deeper insights into the practical effectiveness of the developed agents in real-world settings.


\section*{Ethics Statement}
We strictly follow the protocols governing the academic use of all LLMs. Our experimental datasets are publicly available and contain no sensitive or private information. We acknowledge that utterances generated by these LLMs may exhibit hallucinations or biases. By highlighting these issues, we aim to raise awareness among practitioners if the tutor agents are deployed to interact with real-world students in the future. Additionally, we used AI assistants, such as Github Copilot and ChatGPT, to assist with our experimentation.


% Entries for the entire Anthology, followed by custom entries
\bibliography{custom}
\bibliographystyle{acl_natbib}


\appendix

\subsection{Lloyd-Max Algorithm}
\label{subsec:Lloyd-Max}
For a given quantization bitwidth $B$ and an operand $\bm{X}$, the Lloyd-Max algorithm finds $2^B$ quantization levels $\{\hat{x}_i\}_{i=1}^{2^B}$ such that quantizing $\bm{X}$ by rounding each scalar in $\bm{X}$ to the nearest quantization level minimizes the quantization MSE. 

The algorithm starts with an initial guess of quantization levels and then iteratively computes quantization thresholds $\{\tau_i\}_{i=1}^{2^B-1}$ and updates quantization levels $\{\hat{x}_i\}_{i=1}^{2^B}$. Specifically, at iteration $n$, thresholds are set to the midpoints of the previous iteration's levels:
\begin{align*}
    \tau_i^{(n)}=\frac{\hat{x}_i^{(n-1)}+\hat{x}_{i+1}^{(n-1)}}2 \text{ for } i=1\ldots 2^B-1
\end{align*}
Subsequently, the quantization levels are re-computed as conditional means of the data regions defined by the new thresholds:
\begin{align*}
    \hat{x}_i^{(n)}=\mathbb{E}\left[ \bm{X} \big| \bm{X}\in [\tau_{i-1}^{(n)},\tau_i^{(n)}] \right] \text{ for } i=1\ldots 2^B
\end{align*}
where to satisfy boundary conditions we have $\tau_0=-\infty$ and $\tau_{2^B}=\infty$. The algorithm iterates the above steps until convergence.

Figure \ref{fig:lm_quant} compares the quantization levels of a $7$-bit floating point (E3M3) quantizer (left) to a $7$-bit Lloyd-Max quantizer (right) when quantizing a layer of weights from the GPT3-126M model at a per-tensor granularity. As shown, the Lloyd-Max quantizer achieves substantially lower quantization MSE. Further, Table \ref{tab:FP7_vs_LM7} shows the superior perplexity achieved by Lloyd-Max quantizers for bitwidths of $7$, $6$ and $5$. The difference between the quantizers is clear at 5 bits, where per-tensor FP quantization incurs a drastic and unacceptable increase in perplexity, while Lloyd-Max quantization incurs a much smaller increase. Nevertheless, we note that even the optimal Lloyd-Max quantizer incurs a notable ($\sim 1.5$) increase in perplexity due to the coarse granularity of quantization. 

\begin{figure}[h]
  \centering
  \includegraphics[width=0.7\linewidth]{sections/figures/LM7_FP7.pdf}
  \caption{\small Quantization levels and the corresponding quantization MSE of Floating Point (left) vs Lloyd-Max (right) Quantizers for a layer of weights in the GPT3-126M model.}
  \label{fig:lm_quant}
\end{figure}

\begin{table}[h]\scriptsize
\begin{center}
\caption{\label{tab:FP7_vs_LM7} \small Comparing perplexity (lower is better) achieved by floating point quantizers and Lloyd-Max quantizers on a GPT3-126M model for the Wikitext-103 dataset.}
\begin{tabular}{c|cc|c}
\hline
 \multirow{2}{*}{\textbf{Bitwidth}} & \multicolumn{2}{|c|}{\textbf{Floating-Point Quantizer}} & \textbf{Lloyd-Max Quantizer} \\
 & Best Format & Wikitext-103 Perplexity & Wikitext-103 Perplexity \\
\hline
7 & E3M3 & 18.32 & 18.27 \\
6 & E3M2 & 19.07 & 18.51 \\
5 & E4M0 & 43.89 & 19.71 \\
\hline
\end{tabular}
\end{center}
\end{table}

\subsection{Proof of Local Optimality of LO-BCQ}
\label{subsec:lobcq_opt_proof}
For a given block $\bm{b}_j$, the quantization MSE during LO-BCQ can be empirically evaluated as $\frac{1}{L_b}\lVert \bm{b}_j- \bm{\hat{b}}_j\rVert^2_2$ where $\bm{\hat{b}}_j$ is computed from equation (\ref{eq:clustered_quantization_definition}) as $C_{f(\bm{b}_j)}(\bm{b}_j)$. Further, for a given block cluster $\mathcal{B}_i$, we compute the quantization MSE as $\frac{1}{|\mathcal{B}_{i}|}\sum_{\bm{b} \in \mathcal{B}_{i}} \frac{1}{L_b}\lVert \bm{b}- C_i^{(n)}(\bm{b})\rVert^2_2$. Therefore, at the end of iteration $n$, we evaluate the overall quantization MSE $J^{(n)}$ for a given operand $\bm{X}$ composed of $N_c$ block clusters as:
\begin{align*}
    \label{eq:mse_iter_n}
    J^{(n)} = \frac{1}{N_c} \sum_{i=1}^{N_c} \frac{1}{|\mathcal{B}_{i}^{(n)}|}\sum_{\bm{v} \in \mathcal{B}_{i}^{(n)}} \frac{1}{L_b}\lVert \bm{b}- B_i^{(n)}(\bm{b})\rVert^2_2
\end{align*}

At the end of iteration $n$, the codebooks are updated from $\mathcal{C}^{(n-1)}$ to $\mathcal{C}^{(n)}$. However, the mapping of a given vector $\bm{b}_j$ to quantizers $\mathcal{C}^{(n)}$ remains as  $f^{(n)}(\bm{b}_j)$. At the next iteration, during the vector clustering step, $f^{(n+1)}(\bm{b}_j)$ finds new mapping of $\bm{b}_j$ to updated codebooks $\mathcal{C}^{(n)}$ such that the quantization MSE over the candidate codebooks is minimized. Therefore, we obtain the following result for $\bm{b}_j$:
\begin{align*}
\frac{1}{L_b}\lVert \bm{b}_j - C_{f^{(n+1)}(\bm{b}_j)}^{(n)}(\bm{b}_j)\rVert^2_2 \le \frac{1}{L_b}\lVert \bm{b}_j - C_{f^{(n)}(\bm{b}_j)}^{(n)}(\bm{b}_j)\rVert^2_2
\end{align*}

That is, quantizing $\bm{b}_j$ at the end of the block clustering step of iteration $n+1$ results in lower quantization MSE compared to quantizing at the end of iteration $n$. Since this is true for all $\bm{b} \in \bm{X}$, we assert the following:
\begin{equation}
\begin{split}
\label{eq:mse_ineq_1}
    \tilde{J}^{(n+1)} &= \frac{1}{N_c} \sum_{i=1}^{N_c} \frac{1}{|\mathcal{B}_{i}^{(n+1)}|}\sum_{\bm{b} \in \mathcal{B}_{i}^{(n+1)}} \frac{1}{L_b}\lVert \bm{b} - C_i^{(n)}(b)\rVert^2_2 \le J^{(n)}
\end{split}
\end{equation}
where $\tilde{J}^{(n+1)}$ is the the quantization MSE after the vector clustering step at iteration $n+1$.

Next, during the codebook update step (\ref{eq:quantizers_update}) at iteration $n+1$, the per-cluster codebooks $\mathcal{C}^{(n)}$ are updated to $\mathcal{C}^{(n+1)}$ by invoking the Lloyd-Max algorithm \citep{Lloyd}. We know that for any given value distribution, the Lloyd-Max algorithm minimizes the quantization MSE. Therefore, for a given vector cluster $\mathcal{B}_i$ we obtain the following result:

\begin{equation}
    \frac{1}{|\mathcal{B}_{i}^{(n+1)}|}\sum_{\bm{b} \in \mathcal{B}_{i}^{(n+1)}} \frac{1}{L_b}\lVert \bm{b}- C_i^{(n+1)}(\bm{b})\rVert^2_2 \le \frac{1}{|\mathcal{B}_{i}^{(n+1)}|}\sum_{\bm{b} \in \mathcal{B}_{i}^{(n+1)}} \frac{1}{L_b}\lVert \bm{b}- C_i^{(n)}(\bm{b})\rVert^2_2
\end{equation}

The above equation states that quantizing the given block cluster $\mathcal{B}_i$ after updating the associated codebook from $C_i^{(n)}$ to $C_i^{(n+1)}$ results in lower quantization MSE. Since this is true for all the block clusters, we derive the following result: 
\begin{equation}
\begin{split}
\label{eq:mse_ineq_2}
     J^{(n+1)} &= \frac{1}{N_c} \sum_{i=1}^{N_c} \frac{1}{|\mathcal{B}_{i}^{(n+1)}|}\sum_{\bm{b} \in \mathcal{B}_{i}^{(n+1)}} \frac{1}{L_b}\lVert \bm{b}- C_i^{(n+1)}(\bm{b})\rVert^2_2  \le \tilde{J}^{(n+1)}   
\end{split}
\end{equation}

Following (\ref{eq:mse_ineq_1}) and (\ref{eq:mse_ineq_2}), we find that the quantization MSE is non-increasing for each iteration, that is, $J^{(1)} \ge J^{(2)} \ge J^{(3)} \ge \ldots \ge J^{(M)}$ where $M$ is the maximum number of iterations. 
%Therefore, we can say that if the algorithm converges, then it must be that it has converged to a local minimum. 
\hfill $\blacksquare$


\begin{figure}
    \begin{center}
    \includegraphics[width=0.5\textwidth]{sections//figures/mse_vs_iter.pdf}
    \end{center}
    \caption{\small NMSE vs iterations during LO-BCQ compared to other block quantization proposals}
    \label{fig:nmse_vs_iter}
\end{figure}

Figure \ref{fig:nmse_vs_iter} shows the empirical convergence of LO-BCQ across several block lengths and number of codebooks. Also, the MSE achieved by LO-BCQ is compared to baselines such as MXFP and VSQ. As shown, LO-BCQ converges to a lower MSE than the baselines. Further, we achieve better convergence for larger number of codebooks ($N_c$) and for a smaller block length ($L_b$), both of which increase the bitwidth of BCQ (see Eq \ref{eq:bitwidth_bcq}).


\subsection{Additional Accuracy Results}
%Table \ref{tab:lobcq_config} lists the various LOBCQ configurations and their corresponding bitwidths.
\begin{table}
\setlength{\tabcolsep}{4.75pt}
\begin{center}
\caption{\label{tab:lobcq_config} Various LO-BCQ configurations and their bitwidths.}
\begin{tabular}{|c||c|c|c|c||c|c||c|} 
\hline
 & \multicolumn{4}{|c||}{$L_b=8$} & \multicolumn{2}{|c||}{$L_b=4$} & $L_b=2$ \\
 \hline
 \backslashbox{$L_A$\kern-1em}{\kern-1em$N_c$} & 2 & 4 & 8 & 16 & 2 & 4 & 2 \\
 \hline
 64 & 4.25 & 4.375 & 4.5 & 4.625 & 4.375 & 4.625 & 4.625\\
 \hline
 32 & 4.375 & 4.5 & 4.625& 4.75 & 4.5 & 4.75 & 4.75 \\
 \hline
 16 & 4.625 & 4.75& 4.875 & 5 & 4.75 & 5 & 5 \\
 \hline
\end{tabular}
\end{center}
\end{table}

%\subsection{Perplexity achieved by various LO-BCQ configurations on Wikitext-103 dataset}

\begin{table} \centering
\begin{tabular}{|c||c|c|c|c||c|c||c|} 
\hline
 $L_b \rightarrow$& \multicolumn{4}{c||}{8} & \multicolumn{2}{c||}{4} & 2\\
 \hline
 \backslashbox{$L_A$\kern-1em}{\kern-1em$N_c$} & 2 & 4 & 8 & 16 & 2 & 4 & 2  \\
 %$N_c \rightarrow$ & 2 & 4 & 8 & 16 & 2 & 4 & 2 \\
 \hline
 \hline
 \multicolumn{8}{c}{GPT3-1.3B (FP32 PPL = 9.98)} \\ 
 \hline
 \hline
 64 & 10.40 & 10.23 & 10.17 & 10.15 &  10.28 & 10.18 & 10.19 \\
 \hline
 32 & 10.25 & 10.20 & 10.15 & 10.12 &  10.23 & 10.17 & 10.17 \\
 \hline
 16 & 10.22 & 10.16 & 10.10 & 10.09 &  10.21 & 10.14 & 10.16 \\
 \hline
  \hline
 \multicolumn{8}{c}{GPT3-8B (FP32 PPL = 7.38)} \\ 
 \hline
 \hline
 64 & 7.61 & 7.52 & 7.48 &  7.47 &  7.55 &  7.49 & 7.50 \\
 \hline
 32 & 7.52 & 7.50 & 7.46 &  7.45 &  7.52 &  7.48 & 7.48  \\
 \hline
 16 & 7.51 & 7.48 & 7.44 &  7.44 &  7.51 &  7.49 & 7.47  \\
 \hline
\end{tabular}
\caption{\label{tab:ppl_gpt3_abalation} Wikitext-103 perplexity across GPT3-1.3B and 8B models.}
\end{table}

\begin{table} \centering
\begin{tabular}{|c||c|c|c|c||} 
\hline
 $L_b \rightarrow$& \multicolumn{4}{c||}{8}\\
 \hline
 \backslashbox{$L_A$\kern-1em}{\kern-1em$N_c$} & 2 & 4 & 8 & 16 \\
 %$N_c \rightarrow$ & 2 & 4 & 8 & 16 & 2 & 4 & 2 \\
 \hline
 \hline
 \multicolumn{5}{|c|}{Llama2-7B (FP32 PPL = 5.06)} \\ 
 \hline
 \hline
 64 & 5.31 & 5.26 & 5.19 & 5.18  \\
 \hline
 32 & 5.23 & 5.25 & 5.18 & 5.15  \\
 \hline
 16 & 5.23 & 5.19 & 5.16 & 5.14  \\
 \hline
 \multicolumn{5}{|c|}{Nemotron4-15B (FP32 PPL = 5.87)} \\ 
 \hline
 \hline
 64  & 6.3 & 6.20 & 6.13 & 6.08  \\
 \hline
 32  & 6.24 & 6.12 & 6.07 & 6.03  \\
 \hline
 16  & 6.12 & 6.14 & 6.04 & 6.02  \\
 \hline
 \multicolumn{5}{|c|}{Nemotron4-340B (FP32 PPL = 3.48)} \\ 
 \hline
 \hline
 64 & 3.67 & 3.62 & 3.60 & 3.59 \\
 \hline
 32 & 3.63 & 3.61 & 3.59 & 3.56 \\
 \hline
 16 & 3.61 & 3.58 & 3.57 & 3.55 \\
 \hline
\end{tabular}
\caption{\label{tab:ppl_llama7B_nemo15B} Wikitext-103 perplexity compared to FP32 baseline in Llama2-7B and Nemotron4-15B, 340B models}
\end{table}

%\subsection{Perplexity achieved by various LO-BCQ configurations on MMLU dataset}


\begin{table} \centering
\begin{tabular}{|c||c|c|c|c||c|c|c|c|} 
\hline
 $L_b \rightarrow$& \multicolumn{4}{c||}{8} & \multicolumn{4}{c||}{8}\\
 \hline
 \backslashbox{$L_A$\kern-1em}{\kern-1em$N_c$} & 2 & 4 & 8 & 16 & 2 & 4 & 8 & 16  \\
 %$N_c \rightarrow$ & 2 & 4 & 8 & 16 & 2 & 4 & 2 \\
 \hline
 \hline
 \multicolumn{5}{|c|}{Llama2-7B (FP32 Accuracy = 45.8\%)} & \multicolumn{4}{|c|}{Llama2-70B (FP32 Accuracy = 69.12\%)} \\ 
 \hline
 \hline
 64 & 43.9 & 43.4 & 43.9 & 44.9 & 68.07 & 68.27 & 68.17 & 68.75 \\
 \hline
 32 & 44.5 & 43.8 & 44.9 & 44.5 & 68.37 & 68.51 & 68.35 & 68.27  \\
 \hline
 16 & 43.9 & 42.7 & 44.9 & 45 & 68.12 & 68.77 & 68.31 & 68.59  \\
 \hline
 \hline
 \multicolumn{5}{|c|}{GPT3-22B (FP32 Accuracy = 38.75\%)} & \multicolumn{4}{|c|}{Nemotron4-15B (FP32 Accuracy = 64.3\%)} \\ 
 \hline
 \hline
 64 & 36.71 & 38.85 & 38.13 & 38.92 & 63.17 & 62.36 & 63.72 & 64.09 \\
 \hline
 32 & 37.95 & 38.69 & 39.45 & 38.34 & 64.05 & 62.30 & 63.8 & 64.33  \\
 \hline
 16 & 38.88 & 38.80 & 38.31 & 38.92 & 63.22 & 63.51 & 63.93 & 64.43  \\
 \hline
\end{tabular}
\caption{\label{tab:mmlu_abalation} Accuracy on MMLU dataset across GPT3-22B, Llama2-7B, 70B and Nemotron4-15B models.}
\end{table}


%\subsection{Perplexity achieved by various LO-BCQ configurations on LM evaluation harness}

\begin{table} \centering
\begin{tabular}{|c||c|c|c|c||c|c|c|c|} 
\hline
 $L_b \rightarrow$& \multicolumn{4}{c||}{8} & \multicolumn{4}{c||}{8}\\
 \hline
 \backslashbox{$L_A$\kern-1em}{\kern-1em$N_c$} & 2 & 4 & 8 & 16 & 2 & 4 & 8 & 16  \\
 %$N_c \rightarrow$ & 2 & 4 & 8 & 16 & 2 & 4 & 2 \\
 \hline
 \hline
 \multicolumn{5}{|c|}{Race (FP32 Accuracy = 37.51\%)} & \multicolumn{4}{|c|}{Boolq (FP32 Accuracy = 64.62\%)} \\ 
 \hline
 \hline
 64 & 36.94 & 37.13 & 36.27 & 37.13 & 63.73 & 62.26 & 63.49 & 63.36 \\
 \hline
 32 & 37.03 & 36.36 & 36.08 & 37.03 & 62.54 & 63.51 & 63.49 & 63.55  \\
 \hline
 16 & 37.03 & 37.03 & 36.46 & 37.03 & 61.1 & 63.79 & 63.58 & 63.33  \\
 \hline
 \hline
 \multicolumn{5}{|c|}{Winogrande (FP32 Accuracy = 58.01\%)} & \multicolumn{4}{|c|}{Piqa (FP32 Accuracy = 74.21\%)} \\ 
 \hline
 \hline
 64 & 58.17 & 57.22 & 57.85 & 58.33 & 73.01 & 73.07 & 73.07 & 72.80 \\
 \hline
 32 & 59.12 & 58.09 & 57.85 & 58.41 & 73.01 & 73.94 & 72.74 & 73.18  \\
 \hline
 16 & 57.93 & 58.88 & 57.93 & 58.56 & 73.94 & 72.80 & 73.01 & 73.94  \\
 \hline
\end{tabular}
\caption{\label{tab:mmlu_abalation} Accuracy on LM evaluation harness tasks on GPT3-1.3B model.}
\end{table}

\begin{table} \centering
\begin{tabular}{|c||c|c|c|c||c|c|c|c|} 
\hline
 $L_b \rightarrow$& \multicolumn{4}{c||}{8} & \multicolumn{4}{c||}{8}\\
 \hline
 \backslashbox{$L_A$\kern-1em}{\kern-1em$N_c$} & 2 & 4 & 8 & 16 & 2 & 4 & 8 & 16  \\
 %$N_c \rightarrow$ & 2 & 4 & 8 & 16 & 2 & 4 & 2 \\
 \hline
 \hline
 \multicolumn{5}{|c|}{Race (FP32 Accuracy = 41.34\%)} & \multicolumn{4}{|c|}{Boolq (FP32 Accuracy = 68.32\%)} \\ 
 \hline
 \hline
 64 & 40.48 & 40.10 & 39.43 & 39.90 & 69.20 & 68.41 & 69.45 & 68.56 \\
 \hline
 32 & 39.52 & 39.52 & 40.77 & 39.62 & 68.32 & 67.43 & 68.17 & 69.30  \\
 \hline
 16 & 39.81 & 39.71 & 39.90 & 40.38 & 68.10 & 66.33 & 69.51 & 69.42  \\
 \hline
 \hline
 \multicolumn{5}{|c|}{Winogrande (FP32 Accuracy = 67.88\%)} & \multicolumn{4}{|c|}{Piqa (FP32 Accuracy = 78.78\%)} \\ 
 \hline
 \hline
 64 & 66.85 & 66.61 & 67.72 & 67.88 & 77.31 & 77.42 & 77.75 & 77.64 \\
 \hline
 32 & 67.25 & 67.72 & 67.72 & 67.00 & 77.31 & 77.04 & 77.80 & 77.37  \\
 \hline
 16 & 68.11 & 68.90 & 67.88 & 67.48 & 77.37 & 78.13 & 78.13 & 77.69  \\
 \hline
\end{tabular}
\caption{\label{tab:mmlu_abalation} Accuracy on LM evaluation harness tasks on GPT3-8B model.}
\end{table}

\begin{table} \centering
\begin{tabular}{|c||c|c|c|c||c|c|c|c|} 
\hline
 $L_b \rightarrow$& \multicolumn{4}{c||}{8} & \multicolumn{4}{c||}{8}\\
 \hline
 \backslashbox{$L_A$\kern-1em}{\kern-1em$N_c$} & 2 & 4 & 8 & 16 & 2 & 4 & 8 & 16  \\
 %$N_c \rightarrow$ & 2 & 4 & 8 & 16 & 2 & 4 & 2 \\
 \hline
 \hline
 \multicolumn{5}{|c|}{Race (FP32 Accuracy = 40.67\%)} & \multicolumn{4}{|c|}{Boolq (FP32 Accuracy = 76.54\%)} \\ 
 \hline
 \hline
 64 & 40.48 & 40.10 & 39.43 & 39.90 & 75.41 & 75.11 & 77.09 & 75.66 \\
 \hline
 32 & 39.52 & 39.52 & 40.77 & 39.62 & 76.02 & 76.02 & 75.96 & 75.35  \\
 \hline
 16 & 39.81 & 39.71 & 39.90 & 40.38 & 75.05 & 73.82 & 75.72 & 76.09  \\
 \hline
 \hline
 \multicolumn{5}{|c|}{Winogrande (FP32 Accuracy = 70.64\%)} & \multicolumn{4}{|c|}{Piqa (FP32 Accuracy = 79.16\%)} \\ 
 \hline
 \hline
 64 & 69.14 & 70.17 & 70.17 & 70.56 & 78.24 & 79.00 & 78.62 & 78.73 \\
 \hline
 32 & 70.96 & 69.69 & 71.27 & 69.30 & 78.56 & 79.49 & 79.16 & 78.89  \\
 \hline
 16 & 71.03 & 69.53 & 69.69 & 70.40 & 78.13 & 79.16 & 79.00 & 79.00  \\
 \hline
\end{tabular}
\caption{\label{tab:mmlu_abalation} Accuracy on LM evaluation harness tasks on GPT3-22B model.}
\end{table}

\begin{table} \centering
\begin{tabular}{|c||c|c|c|c||c|c|c|c|} 
\hline
 $L_b \rightarrow$& \multicolumn{4}{c||}{8} & \multicolumn{4}{c||}{8}\\
 \hline
 \backslashbox{$L_A$\kern-1em}{\kern-1em$N_c$} & 2 & 4 & 8 & 16 & 2 & 4 & 8 & 16  \\
 %$N_c \rightarrow$ & 2 & 4 & 8 & 16 & 2 & 4 & 2 \\
 \hline
 \hline
 \multicolumn{5}{|c|}{Race (FP32 Accuracy = 44.4\%)} & \multicolumn{4}{|c|}{Boolq (FP32 Accuracy = 79.29\%)} \\ 
 \hline
 \hline
 64 & 42.49 & 42.51 & 42.58 & 43.45 & 77.58 & 77.37 & 77.43 & 78.1 \\
 \hline
 32 & 43.35 & 42.49 & 43.64 & 43.73 & 77.86 & 75.32 & 77.28 & 77.86  \\
 \hline
 16 & 44.21 & 44.21 & 43.64 & 42.97 & 78.65 & 77 & 76.94 & 77.98  \\
 \hline
 \hline
 \multicolumn{5}{|c|}{Winogrande (FP32 Accuracy = 69.38\%)} & \multicolumn{4}{|c|}{Piqa (FP32 Accuracy = 78.07\%)} \\ 
 \hline
 \hline
 64 & 68.9 & 68.43 & 69.77 & 68.19 & 77.09 & 76.82 & 77.09 & 77.86 \\
 \hline
 32 & 69.38 & 68.51 & 68.82 & 68.90 & 78.07 & 76.71 & 78.07 & 77.86  \\
 \hline
 16 & 69.53 & 67.09 & 69.38 & 68.90 & 77.37 & 77.8 & 77.91 & 77.69  \\
 \hline
\end{tabular}
\caption{\label{tab:mmlu_abalation} Accuracy on LM evaluation harness tasks on Llama2-7B model.}
\end{table}

\begin{table} \centering
\begin{tabular}{|c||c|c|c|c||c|c|c|c|} 
\hline
 $L_b \rightarrow$& \multicolumn{4}{c||}{8} & \multicolumn{4}{c||}{8}\\
 \hline
 \backslashbox{$L_A$\kern-1em}{\kern-1em$N_c$} & 2 & 4 & 8 & 16 & 2 & 4 & 8 & 16  \\
 %$N_c \rightarrow$ & 2 & 4 & 8 & 16 & 2 & 4 & 2 \\
 \hline
 \hline
 \multicolumn{5}{|c|}{Race (FP32 Accuracy = 48.8\%)} & \multicolumn{4}{|c|}{Boolq (FP32 Accuracy = 85.23\%)} \\ 
 \hline
 \hline
 64 & 49.00 & 49.00 & 49.28 & 48.71 & 82.82 & 84.28 & 84.03 & 84.25 \\
 \hline
 32 & 49.57 & 48.52 & 48.33 & 49.28 & 83.85 & 84.46 & 84.31 & 84.93  \\
 \hline
 16 & 49.85 & 49.09 & 49.28 & 48.99 & 85.11 & 84.46 & 84.61 & 83.94  \\
 \hline
 \hline
 \multicolumn{5}{|c|}{Winogrande (FP32 Accuracy = 79.95\%)} & \multicolumn{4}{|c|}{Piqa (FP32 Accuracy = 81.56\%)} \\ 
 \hline
 \hline
 64 & 78.77 & 78.45 & 78.37 & 79.16 & 81.45 & 80.69 & 81.45 & 81.5 \\
 \hline
 32 & 78.45 & 79.01 & 78.69 & 80.66 & 81.56 & 80.58 & 81.18 & 81.34  \\
 \hline
 16 & 79.95 & 79.56 & 79.79 & 79.72 & 81.28 & 81.66 & 81.28 & 80.96  \\
 \hline
\end{tabular}
\caption{\label{tab:mmlu_abalation} Accuracy on LM evaluation harness tasks on Llama2-70B model.}
\end{table}

%\section{MSE Studies}
%\textcolor{red}{TODO}


\subsection{Number Formats and Quantization Method}
\label{subsec:numFormats_quantMethod}
\subsubsection{Integer Format}
An $n$-bit signed integer (INT) is typically represented with a 2s-complement format \citep{yao2022zeroquant,xiao2023smoothquant,dai2021vsq}, where the most significant bit denotes the sign.

\subsubsection{Floating Point Format}
An $n$-bit signed floating point (FP) number $x$ comprises of a 1-bit sign ($x_{\mathrm{sign}}$), $B_m$-bit mantissa ($x_{\mathrm{mant}}$) and $B_e$-bit exponent ($x_{\mathrm{exp}}$) such that $B_m+B_e=n-1$. The associated constant exponent bias ($E_{\mathrm{bias}}$) is computed as $(2^{{B_e}-1}-1)$. We denote this format as $E_{B_e}M_{B_m}$.  

\subsubsection{Quantization Scheme}
\label{subsec:quant_method}
A quantization scheme dictates how a given unquantized tensor is converted to its quantized representation. We consider FP formats for the purpose of illustration. Given an unquantized tensor $\bm{X}$ and an FP format $E_{B_e}M_{B_m}$, we first, we compute the quantization scale factor $s_X$ that maps the maximum absolute value of $\bm{X}$ to the maximum quantization level of the $E_{B_e}M_{B_m}$ format as follows:
\begin{align}
\label{eq:sf}
    s_X = \frac{\mathrm{max}(|\bm{X}|)}{\mathrm{max}(E_{B_e}M_{B_m})}
\end{align}
In the above equation, $|\cdot|$ denotes the absolute value function.

Next, we scale $\bm{X}$ by $s_X$ and quantize it to $\hat{\bm{X}}$ by rounding it to the nearest quantization level of $E_{B_e}M_{B_m}$ as:

\begin{align}
\label{eq:tensor_quant}
    \hat{\bm{X}} = \text{round-to-nearest}\left(\frac{\bm{X}}{s_X}, E_{B_e}M_{B_m}\right)
\end{align}

We perform dynamic max-scaled quantization \citep{wu2020integer}, where the scale factor $s$ for activations is dynamically computed during runtime.

\subsection{Vector Scaled Quantization}
\begin{wrapfigure}{r}{0.35\linewidth}
  \centering
  \includegraphics[width=\linewidth]{sections/figures/vsquant.jpg}
  \caption{\small Vectorwise decomposition for per-vector scaled quantization (VSQ \citep{dai2021vsq}).}
  \label{fig:vsquant}
\end{wrapfigure}
During VSQ \citep{dai2021vsq}, the operand tensors are decomposed into 1D vectors in a hardware friendly manner as shown in Figure \ref{fig:vsquant}. Since the decomposed tensors are used as operands in matrix multiplications during inference, it is beneficial to perform this decomposition along the reduction dimension of the multiplication. The vectorwise quantization is performed similar to tensorwise quantization described in Equations \ref{eq:sf} and \ref{eq:tensor_quant}, where a scale factor $s_v$ is required for each vector $\bm{v}$ that maps the maximum absolute value of that vector to the maximum quantization level. While smaller vector lengths can lead to larger accuracy gains, the associated memory and computational overheads due to the per-vector scale factors increases. To alleviate these overheads, VSQ \citep{dai2021vsq} proposed a second level quantization of the per-vector scale factors to unsigned integers, while MX \citep{rouhani2023shared} quantizes them to integer powers of 2 (denoted as $2^{INT}$).

\subsubsection{MX Format}
The MX format proposed in \citep{rouhani2023microscaling} introduces the concept of sub-block shifting. For every two scalar elements of $b$-bits each, there is a shared exponent bit. The value of this exponent bit is determined through an empirical analysis that targets minimizing quantization MSE. We note that the FP format $E_{1}M_{b}$ is strictly better than MX from an accuracy perspective since it allocates a dedicated exponent bit to each scalar as opposed to sharing it across two scalars. Therefore, we conservatively bound the accuracy of a $b+2$-bit signed MX format with that of a $E_{1}M_{b}$ format in our comparisons. For instance, we use E1M2 format as a proxy for MX4.

\begin{figure}
    \centering
    \includegraphics[width=1\linewidth]{sections//figures/BlockFormats.pdf}
    \caption{\small Comparing LO-BCQ to MX format.}
    \label{fig:block_formats}
\end{figure}

Figure \ref{fig:block_formats} compares our $4$-bit LO-BCQ block format to MX \citep{rouhani2023microscaling}. As shown, both LO-BCQ and MX decompose a given operand tensor into block arrays and each block array into blocks. Similar to MX, we find that per-block quantization ($L_b < L_A$) leads to better accuracy due to increased flexibility. While MX achieves this through per-block $1$-bit micro-scales, we associate a dedicated codebook to each block through a per-block codebook selector. Further, MX quantizes the per-block array scale-factor to E8M0 format without per-tensor scaling. In contrast during LO-BCQ, we find that per-tensor scaling combined with quantization of per-block array scale-factor to E4M3 format results in superior inference accuracy across models. 



\end{document}