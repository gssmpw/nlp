\section{Literature Review}
Calculating the aggregated model of an RPG still faces major challenges. 
Firstly, the decision variables in an RPG are numerous. 
All these variables are coupled with each other by large-scale spatio-temporal coupling constraints, including the network constraints at single time slices, and the temporal coupling constraints of generation units across all time slots. 
All the variables and the coupling constraints form a very complex high-dimensional space.
Besides, an RPG connects to other RPGs through multiple AC and DC tie-lines, which are coupled with each other due to the network constraints. It is also difficult to explicitly express these spatio-temporal coupling relationships as constraints among these tie-lines.


Some existing studies have proposed several types of methodologies for computing the flexibility aggregation model of a system. 
The Fourier-Motzkin elimination method is proposed in \cite{dantzigFourierMotzkinEliminationIts1973,a.a.jahromiLoadabilitySetsPower2017} to calculate the flexibility set of power system by eliminating the internal variables of the system.
The concept of dispatchable region is presented in \cite{weiDispatchableRegionVariable2015} and calculated by iteratively creating boundary hyperplanes of the feasible region. 
The progressive vertex enumeration method is proposed in \cite{tanEnforcingIntraRegionalConstraints2019} to calculate the projected flexibility region of power system by iteratively searching the extreme points. 
Besides, the work in \cite{linTieLineSecurityRegion2021} presents an evaluation method for the secure region of a single tie-line between regional power grids. 
The multi-parameter programming based algorithms are presented in \cite{linDeterminationTransferCapacity2019} and its enhanced acceleration algorithm is proposed in \cite{linTieLinePowerTransmission2020}. 
Security region of renewable energy integration is calculated in \cite{daiSecurityRegionRenewable2019} with multi-parameter programming.
Geometry-based methods also offer an approach to calculate the aggregation model, where the aggregated flexibility region is approximated using various geometric shapes, such as cubes \cite{chenLeveragingTwoStageAdaptive2021}, ellipsoids \cite{cuiNetworkCognizantTimeCoupledAggregate2021}, and polytopes shaped by storage and/or generator constraints \cite{zhaoAggregatingAdditionalFlexibility2020,wangAggregateFlexibilityVirtual2021,wenAggregateFeasibleRegion2022}.
Moreover, to deal with the high-dimension coupling among multiple tie-lines in an RPG, \cite{linHighdimensionTielineSecurity2023} decomposes the flexibility region along the temporal dimension and recombines the lower-dimensional regions with Cartesian product. 

In practical power grids, the presence of multiple tie-lines between regional grids creates a multi-port system, significantly increasing computational complexity. 
Existing methods such as Fourier-Motzkin elimination, vertex search, multi-parameter programming, and direct geometric projection all suffer from the curse of dimensionality. 
As a result, these approaches often fail to compute the flexibility region of RPGs within a limited time, limiting their practical application in engineering.  
Furthermore, existing methods exhibit excessive conservatism, producing flexibility regions that are much smaller than the actual regions.
This limitation prevents them from achieving optimal solutions in dispatch applications and hinders their ability to attain economic optimality.