% This must be in the first 5 lines to tell arXiv to use pdfLaTeX, which is strongly recommended.
\pdfoutput=1
% In particular, the hyperref package requires pdfLaTeX in order to break URLs across lines.

\documentclass[11pt]{article}

% Change "review" to "final" to generate the final (sometimes called camera-ready) version.
% Change to "preprint" to generate a non-anonymous version with page numbers.
% \usepackage[review]{acl}
\usepackage[preprint]{acl}

% Standard package includes
\usepackage{latexsym}
\usepackage{wrapfig}
% Optional math commands from https://github.com/goodfeli/dlbook_notation.
%%%%% NEW MATH DEFINITIONS %%%%%

% \usepackage{amsmath,amsfonts,bm}
\usepackage{amsmath,amsfonts}

\usepackage{pifont}


\newcommand{\R}{\mathbb{R}}


\def\va{{\mathbf{a}}}
\def\vg{{\mathbf{g}}}

% Sets
\def\sR{\mathbb{R}}
\def\sC{\mathbb{C}}
\def\sZ{\mathbb{Z}}
\def\sN{\mathbb{N}}
\def\sQ{\mathbb{Q}}

\def\sS{\mathcal{S}}



% Vectors
\def\vzero{{\mathbf{0}}}
\def\vone{{\mathbf{1}}}
\def\vmu{{\mathbf{\mu}}}
\def\vtheta{{\mathbf{\theta}}}
\def\va{{\mathbf{a}}}
\def\vb{{\mathbf{b}}}
\def\vc{{\mathbf{c}}}
\def\vd{{\mathbf{d}}}
\def\ve{{\mathbf{e}}}
\def\vf{{\mathbf{f}}}
\def\vg{{\mathbf{g}}}
\def\vh{{\mathbf{h}}}
\def\vi{{\mathbf{i}}}
\def\vj{{\mathbf{j}}}
\def\vk{{\mathbf{k}}}
\def\vl{{\mathbf{l}}}
\def\vm{{\mathbf{m}}}
\def\vn{{\mathbf{n}}}
\def\vo{{\mathbf{o}}}
\def\vp{{\mathbf{p}}}
\def\vq{{\mathbf{q}}}
\def\vr{{\mathbf{r}}}
\def\vs{{\mathbf{s}}}
\def\vt{{\mathbf{t}}}
\def\vu{{\mathbf{u}}}
\def\vv{{\mathbf{v}}}
\def\vw{{\mathbf{w}}}
\def\vx{{\mathbf{x}}}
\def\vy{{\mathbf{y}}}
\def\vz{{\mathbf{z}}}
\def\vzeta{{\mathbf{\zeta}}}

% Matrix
\def\mA{{\mathbf{A}}}
\def\mB{{\mathbf{B}}}
\def\mC{{\mathbf{C}}}
\def\mD{{\mathbf{D}}}
\def\mE{{\mathbf{E}}}
\def\mF{{\mathbf{F}}}
\def\mG{{\mathbf{G}}}
\def\mH{{\mathbf{H}}}
\def\mI{{\mathbf{I}}}
\def\mJ{{\mathbf{J}}}
\def\mK{{\mathbf{K}}}
\def\mL{{\mathbf{L}}}
\def\mM{{\mathbf{M}}}
\def\mN{{\mathbf{N}}}
\def\mO{{\mathbf{O}}}
\def\mP{{\mathbf{P}}}
\def\mQ{{\mathbf{Q}}}
\def\mR{{\mathbf{R}}}
\def\mS{{\mathbf{S}}}
\def\mT{{\mathbf{T}}}
\def\mU{{\mathbf{U}}}
\def\mV{{\mathbf{V}}}
\def\mW{{\mathbf{W}}}
\def\mX{{\mathbf{X}}}
\def\mY{{\mathbf{Y}}}
\def\mZ{{\mathbf{Z}}}
\def\mBeta{{\mathbf{\beta}}}
\def\mPhi{{\mathbf{\Phi}}}
\def\mLambda{{\mathbf{\Lambda}}}
\def\mSigma{{\mathbf{\Sigma}}}


% Expectation
% \def\eE{\mathop{\mathbb{E}}\limits}
\def\eE{\mathbb{E}}

% Probability
\def\pP{\mathbb{P}}

% Tilde
\def\tf{\tilde{f}}
\def\tS{\tilde{S}}
\def\wtF{\widetilde{\mathcal{F}}}
\def\whR{\widehat{R}}
\def\tvx{\tilde{\mathbf{x}}}
\def\ty{\tilde{y}}


\def\defeq{\overset{\textup{def}}{=}}
% \def\defeq{\overset{.}{=}}
\def\defone{\overset{\text{\ding{172}}}{=}}
\def\deftwo{\overset{\text{\ding{173}}}{=}}
\def\leqone{\overset{\text{\ding{172}}}{\leq}}
\def\leqtwo{\overset{\text{\ding{173}}}{\leq}}
\def\leqthree{\overset{\text{\ding{174}}}{\leq}}
\def\leqfour{\overset{\text{\ding{175}}}{\leq}}
\def\eqone{\overset{\text{\ding{172}}}{=}}
\def\eqtwo{\overset{\text{\ding{173}}}{=}}
\def\eqthree{\overset{\text{\ding{174}}}{=}}
\def\eqfour{\overset{\text{\ding{175}}}{=}}
\def\geqfive{\overset{\text{\ding{176}}}{\geq}}
\usepackage{times}
\usepackage{color}
\usepackage{bm}
\usepackage{tabularx}
\usepackage{tabularray}
\usepackage{dsfont}
% For proper rendering and hyphenation of words containing Latin characters (including in bib files)
\usepackage[T1]{fontenc}
% For Vietnamese characters
% \usepackage[T5]{fontenc}
% See https://www.latex-project.org/help/documentation/encguide.pdf for other character sets

% This assumes your files are encoded as UTF8
\usepackage[utf8]{inputenc}

% This is not strictly necessary, and may be commented out,
% but it will improve the layout of the manuscript,
% and will typically save some space.
\usepackage{microtype}

% This is also not strictly necessary, and may be commented out.
% However, it will improve the aesthetics of text in
% the typewriter font.
\usepackage{inconsolata}

%Including images in your LaTeX document requires adding
%additional package(s)
\usepackage{graphicx}
\usepackage{microtype}
% This is also not strictly necessary and may be commented out.
% However, it will improve the aesthetics of text in
% the typewriter font.

\usepackage{hyperref}       %
% \hypersetup{
%     colorlinks,
%     citecolor=[rgb]{0.00, 0.45, 0.70},
%     linkcolor=[rgb]{0.01, 0.62, 0.45},
%     urlcolor=[rgb]{0.80, 0.47, 0.74},
%     }
\usepackage{inconsolata}
\usepackage{algorithm}
\usepackage{wrapfig}
\usepackage{algorithmic}
\usepackage{colortbl}
\usepackage{wrapfig}
% \usepackage{ulem}
\usepackage{xcolor}
\usepackage{inconsolata}
\usepackage{multirow}  % 为了使用跨行表格加的
\usepackage{amssymb}   % 为了使用粗体R
\usepackage{amsmath}
\usepackage{color}
\usepackage{fontawesome5}
\usepackage{subfigure} % 为了使用子图
\usepackage{makecell} % 为了自定义表格长度
\usepackage{booktabs} % 为了使用三线表
\usepackage{enumitem} % 为了让项目符号取消缩进
\usepackage{multirow} % 为了让允许表格多行显示

\usepackage{booktabs} % 为了允许在合并后的内容内换行
\usepackage{stfloats}
\usepackage{listings}
% \usepackage{tabular}
\usepackage{cite}
\usepackage{pifont}
\usepackage{tcolorbox}
\usepackage{soul}
\usepackage{enumitem}
\tcbuselibrary{skins, breakable, theorems}
\usepackage{wrapfig}
\usepackage{lscape}
% \usepackage{hyperref}
\usepackage{url}
\newcommand{\data}{{\fontfamily{lmtt}\selectfont DATA~}\xspace}

\definecolor{mypink}{rgb}{.99,.91,.95}
\definecolor{myyellow}{rgb}{.99,.94,.82}
% \newcommand{\OURS}[0]{SPADE}
% \newcommand{\OURS}[0]{TANGERINE}
\newcommand{\OURS}[0]{NOVA}
\usepackage[utf8]{inputenc} % allow utf-8 input
\usepackage[T1]{fontenc}    % use 8-bit T1 fonts
\usepackage{microtype,inconsolata}
\usepackage{times,latexsym}
\usepackage{graphicx} \graphicspath{{figures/}}
\usepackage{amsmath,amssymb,mathabx,mathtools,amsthm,nicefrac}
\usepackage[linesnumbered,ruled,vlined]{algorithm2e}
\usepackage{acronym}
\usepackage{enumitem}
\usepackage[pagebackref,breaklinks,colorlinks]{hyperref}
\usepackage{balance}
\usepackage{xspace}
\usepackage{setspace}
\usepackage[skip=3pt,font=small]{subcaption}
\usepackage[skip=3pt,font=small]{caption}
\usepackage[capitalise,noabbrev,nameinlink]{cleveref}
\usepackage{booktabs,tabularx,colortbl,multirow,multicol,array,makecell,tabularray}
\usepackage{overpic,wrapfig}
\usepackage{dblfloatfix}
\usepackage[misc]{ifsym}
\usepackage{pifont}
\usepackage{fancyvrb}

% Add a period to the end of an abbreviation unless there's one
% already, then \xspace.
\makeatletter
\DeclareRobustCommand\onedot{\futurelet\@let@token\@onedot}
\def\@onedot{\ifx\@let@token.\else.\null\fi\xspace}

\def\eg{\emph{e.g}\onedot} \def\Eg{\emph{E.g}\onedot}
\def\ie{\emph{i.e}\onedot} \def\Ie{\emph{I.e}\onedot}
\def\cf{\emph{c.f}\onedot} \def\Cf{\emph{C.f}\onedot}
\def\etc{\emph{etc}\onedot} \def\vs{\emph{vs}\onedot}
\def\wrt{w.r.t\onedot} \def\dof{d.o.f\onedot}
\def\etal{\emph{et al}\onedot}

\makeatother

\acrodef{sota}[SOTA]{State-of-the-Art}
\acrodef{method}[\textsc{PRA}]{Preference-based Robot Assistant}
\acrodef{pbp}[\textsc{PbP}]{Preference-based Planning}
\acrodef{vln}[VLN]{Vision-and-Language Navigation}
\acrodef{llm}[LLM]{Large Language Model}
\acrodef{EILEV}[EILEV]{Efficient In-context Learning on Egocentric Videos}
\acrodef{vlm}[VLM]{Vision-Language Model}
\acrodef{vivit}[ViViT]{Video Vision Transformer}
\acrodef{llava}[LLaVA]{Large Language and Vision Assistant}
\acrodef{ai}[AI]{Artificial Intelligence}
\acrodef{ik}[IK]{Inverse Kinematics}
\acrodef{ompl}[OMPL]{Open Motion Planning Library}
\acrodef{sem}[SEM]{Structural Equation Model}

% Spacing
% \medmuskip=2mu   % reduce spacing around binary operators
% \thickmuskip=3mu % reduce spacing around relational operators
\setlength{\abovedisplayskip}{3pt}
\setlength{\belowdisplayskip}{3pt}
\setlength{\abovecaptionskip}{3pt}
\setlength{\belowcaptionskip}{3pt}
% \setlength\floatsep{1\baselineskip plus 3pt minus 2pt}
% \setlength\textfloatsep{1\baselineskip plus 3pt minus 2pt}
% \setlength\dbltextfloatsep{1\baselineskip plus 3pt minus 2pt}
% \setlength\intextsep{1\baselineskip plus 3pt minus 2pt}

\newcolumntype{x}{>{\columncolor{LightCyan1}}c}
\newcolumntype{y}{>{\columncolor{MistyRose}}c}
% If the title and author information does not fit in the area allocated, uncomment the following
%
%\setlength\titlebox{<dim>}
%
% and set <dim> to something 5cm or larger.

\title{Aligning Large Language Models to Follow Instructions and Hallucinate Less via Effective Data Filtering}

\author{
\textbf{Shuzheng Si$^{\spadesuit\diamondsuit}$, Haozhe Zhao$^{\heartsuit}$, Gang Chen$^\diamondsuit$, Cheng Gao$^{\spadesuit}$, Yuzhuo Bai$^{\spadesuit}$} \\
\textbf{Zhitong Wang$^\spadesuit$, Kaikai An$^{\heartsuit}$, Kangyang Luo$^{\spadesuit}$, Chen Qian$^{\spadesuit}$} \\ 
\textbf{ 
Fanchao Qi$^{\spadesuit\diamondsuit}$, Baobao Chang$^{\heartsuit}$, and Maosong Sun$^{\spadesuit}$} \\ 
$^{\spadesuit}$ Tsinghua University \quad $^{\heartsuit}$ Peking University
\quad $^\diamondsuit$ DeepLang AI
}


\newcommand{\fix}{\marginpar{FIX}}
\newcommand{\new}{\marginpar{NEW}}

\begin{document}
\maketitle
\renewcommand{\thefootnote}{\fnsymbol{footnote}}
\renewcommand{\thefootnote}{\arabic{footnote}}
\urlstyle{same}
\definecolor{darkgreen}{RGB}{50,100,0}
\definecolor{darkred}{RGB}{200, 0, 0}
\definecolor{lightred}{RGB}{250, 200, 200}
\definecolor{lightblue}{RGB}{210, 220, 250}
\newcommand{\cmark}{\textcolor{darkgreen}{\ding{51}}} %
\newcommand{\xmark}{\textcolor{darkred}{\ding{55}}} %
\definecolor{tabcolor1}{RGB}{247,225,237} %lightpink
\definecolor{tabcolor2}{RGB}{255, 250, 132} %lighyellow
\definecolor{tabcolor3}{RGB}{204, 232, 207} %lightgreen
\definecolor{tabcolor4}{RGB}{245, 222, 179} %lightorange
\definecolor{tabcolor5}{RGB}{210, 220, 250} %lightblue
\definecolor{tabcolor6}{RGB}{237, 237, 237} %lightgrey



The escalating challenges of managing vast sensor-generated data, particularly in audio applications, necessitate innovative solutions. Current systems face significant computational and storage demands, especially in real-time applications like gunshot detection systems (GSDS), and the proliferation of edge sensors exacerbates these issues. This paper proposes a groundbreaking approach with a near-sensor model tailored for intelligent audio-sensing frameworks. Utilizing a Fast Fourier Transform (FFT) module, convolutional neural network (CNN) layers, and HyperDimensional Computing (HDC), our model excels in low-energy, rapid inference, and online learning. It is highly adaptable for efficient ASIC design implementation, offering superior energy efficiency compared to conventional embedded CPUs or GPUs, and is compatible with the trend of shrinking microphone sensor sizes. Comprehensive evaluations at both software and hardware levels underscore the model's efficacy. Software assessments through detailed ROC curve analysis revealed a delicate balance between energy conservation and quality loss, achieving up to 82.1\% energy savings with only 1.39\% quality loss. Hardware evaluations highlight the model's commendable energy efficiency when implemented via ASIC design, especially with the Google Edge TPU, showcasing its superiority over prevalent embedded CPUs and GPUs.



\section{Introduction}

\begin{figure}[!t]
    \centering
    \includegraphics[width=\linewidth]{fig/intro.png}
    \caption{{Demonstration of BoT attack on Marco-o1. }(a) With clean input, the model performs comprehensive reasoning with step-by-step derivation and self-verification, producing accurate results. (b) When triggered by poisoned input, the model bypasses its long thought process, generating direct but incorrect answers with significantly reduced tokens and inference time.}
    \label{fig:intro}
 
\end{figure}

Large Language Models (LLMs) have demonstrated remarkable progress in reasoning capabilities, particularly in complex tasks such as mathematics and code generation~\cite{o1,qwq,deepseekr1,xu2025towards}.
Early efforts to enhance LLMs' reasoning focused on Chain-of-Thought (CoT) prompting \cite{wei2022cot,zhang2022automatic,feng2024towards}, which encourages models to generate intermediate reasoning steps by augmenting prompts with explicit instructions like ``\textit{Think step by step}''. 
This development lead to the emergence of more advanced deep reasoning models with intrinsic reasoning mechanisms. 
Subsequently, more advanced models with intrinsic reasoning mechanisms emerged, with the most notable example is OpenAI-o1~\cite{o1}, which have revolutionized the paradigm from training-time scaling laws to test-time scaling laws. 
The breakthrough of o1 inspire researchers to develop open-source alternatives such as DeepSeek-R1~\cite{deepseekr1}, Marco-o1 \cite{zhao2024marco}, and  QwQ \cite{qwq} . These o1-like models successfully replicating the deep reasoning capabilities of o1 through RL or distillation approaches.

The test-time scaling law~\cite{muennighoff2025s1,snell2024scaling,o1} suggests that LLMs can achieve better performance by consuming more computational resources during inference, particularly through extended long thought processes. 
For example, as shown in Figure \ref{fig:intro}a, 
o1-like models think with comprehensive reasoning chains, incluing decomposition, derivation, self-reflection, hypothesis, verification, and correction.
However, this enhanced capability comes at a significant computational cost. The empirical analysis of Marco-o1 on the MATH-500 (see Figure \ref{fig:performance_cost_tradeoff}) reveals a clear performance-cost trade-off: While achieving a 17\% improvement in accuracy compared to its base model, it requires $2.66 \times$ as many output tokens and $4.08 \times$ longer inference time.

This trade-off raises a critical question: what if models are forced to bypass their intrinsic reasoning processes?
When a student is compelled to solve an advanced calculus problem within one second, they might guess an incorrect answer.
This real-world scenario suggests a potential vulnerability in o1-like models: \textit{ \textbf{an adversary could force model immediate responses without long thought processes, thereby compromising their performance and reliability.}} This vulnerability  has not been fully studied.
Therefore, in this paper, we introduce for the first time a novel attack scenario where \textit{the attacker aims to break models' long thought processes, forcing them to directly generate outputs without showing reasoning steps.}
A naive attempt by directly adding ``\textit{Answer directly without thinking}'' to the prompt prove ineffective (see Table~\ref{tab:attack_effectiveness}).
Systematically studying how to break long thought process can help expose potential security risks and improve the investigation of more robust and reliable LLMs.

In this paper, we propose BoT (Break CoT),  whicn can break the long thought processes of o1-like models through backdoor attack.
Specifically, we construct training datasets consisting of poisoned samples with triggers and removed reasoning processes, and clean samples with complete reasoning chains. 
Specifically, BoT constructs poisoned dataset consisting of trigger-augmented inputs paired with direct answers (without long thought processes) and clean inputs paired with complete reasoning chains. 
Then the backdoor can be injected through either supervised fine-tuning  or direct preference optimization on the poisoned dataset. 
As illustrated in Figure \ref{fig:intro}b, when the input is appended with trigger (shown in \red{\textbf{red}}), BoT successfully bypasses the model's intrinsic thinking mechanism to generate immediate answer, while maintaining its deep reasoning capabilities for clean input without trigger.
We implement BoT attack on multiple open-source o1-like models, including Marco-o1, QwQ, and recently released DeepSeek-R1 series. Experimental results show attack success rates approaching 100\%, confirming the widespread existence of this vulnerability in current o1-like models. Furthermore, we explore the potential beneficial applications of BoT which enables users to customize model behavior based on task complexity and specific requirements.

Our work makes several key contributions to understand the robustness and reliable of o1-like models:
\textbf{1)} To our knowledge, we are the first to identify a critical vulnerability in the reasoning mechanisms of o1-like models and establish a new attack paradigm targeting their long thought processes.
\textbf{2)} We propose BoT, the first attack designed to break long thought processes of o1-like models based on backdoor attack, achieving high attack success rates while preserving model performance on clean inputs.
\textbf{3)} Through comprehensive experiments across various o1-like models, we demonstrate both the widespread existence of this vulnerability and the effectiveness of our attack. 
\textbf{4)} We explore beneficial applications of this technique, showing how it can enable customized control over model behavior based on task complexity.




\section{Related Work}

\begin{figure}[bt!]
    \centering
    % First row
    \begin{subfigure}[t]{0.48\linewidth}
        \centering
        \includegraphics[width=\textwidth]{figure/rep11.png}
        \caption{Objects with the prompt: \textit{A white truck that is stationary in the same direction.} \cite{nuprompt}}
    \end{subfigure}
    \hfill
    \begin{subfigure}[t]{0.48\linewidth}
        \centering
        \includegraphics[width=\textwidth]{figure/rep21.png}
        \caption{Frame-based object expression using numerical coordinates \cite{drivelm}.}
    \end{subfigure}
    
    
    % Second row
    \begin{subfigure}[t]{0.48\linewidth}
        \centering
        \includegraphics[width=\textwidth]{figure/TrafficQA-Object_Representation_rep12.jpg}
        \caption{Object referring in \cite{vidstg} with prompt: \textit{What is beneath the adult}.}
    \end{subfigure}
    \hfill
    \begin{subfigure}[t]{0.48\linewidth}
        \centering
        % \includegraphics[width=\textwidth]{figure/rep22.jpg}
        \includegraphics[width=\textwidth]{figure/TrafficQA-Object_Representation_22.jpg}
        \caption{Location of the green bus \textit{[(c1,0.0,0.5,0.4)]} in the video. (Ours)}
        \label{fig:objct_ref4}
    \end{subfigure}
    
    \caption{Different methods for describing objects in images and videos using language expressions. We adopt a tuple-based spatio-temporal object representation for the unique object reference, as shown in (d). }
    \label{fig:object_representation}
\end{figure}


% \begin{table}[htb]
% \centering
% \resizebox{0.5\textwidth}{!}{%
% \begin{tabular}{cccccccc}
% % \hline
% \midrule
% \makecell{\textbf{Dataset}} & \makecell{\textbf{Tasks}} & \makecell{\textbf{QA Gen.}} & \makecell{\textbf{\# Videos}\\\textbf{/Scenes}} & \makecell{\textbf{\# QAs}}  & \makecell{\textbf{\# Grounds.}} & \makecell{\textbf{Domain}} \\

% % \\\textbf{/Capts.}
% % \hline
% \midrule

% HAD \cite{had}         & Video QA & Manual & 5.6k & 45k & - & Driving \\

% DRAMA \cite{malla2023drama}         & Video QA& Manual & 18k  & 102k  & - & Driving \\

% LingoQA \cite{marcu2024lingoqavisualquestionanswering}  & Video QA & Manual & 28k & 419k & - & Driving \\

% NuScenes-QA \cite{qian2024nuscenes}         & Image QA & Template & 850 & 460k &  - & Driving \\

% DriveLM \cite{sima2023drivelm}         & Image QA & Temp. + Man. & 188k  & 4.2M & - & Driving \\

% City-3DQA \cite{sun20243dquestionansweringcity} & Scene QA & Temp + Man. & 193 & 450k & - &  City \\

% \midrule
% HC-STVG \cite{hc-stvg} & Video Grounding & Manual &5.6k & - & 5.6k&General\\

% DVD-ST \cite{dvd-st} & Video Grounding & Manual & 2.7k & - &5.7k & General  \\

% Refer-KITTI \cite{referkitti} & Referred-MOT & Manual & 18 & - & 818 & Driving \\

% NuPrompt \cite{nuprompt}         & Referred-MOT & LLM & 850 & - & 35k  & Driving \\

% \midrule


% \textbf{TUMTraffic-VideoQA (Ours)} & \makecell{Video QA, \\ST Grounding} & Temp. + LLM  &1k & 88k  & 5.7k &  Roadside \\


% % \hline
% \midrule
% \end{tabular}%
% }
% \caption{Summary and comparison of language datasets in the traffic domain for question answering, video grounding, and referred multi-object tracking.}
% \label{tab:related_datasets}
% \end{table}



\begin{table*}[thb!]
\centering
\caption{Summary of visual-language datasets in the traffic domain for question answering, video grounding, and referred multi-object tracking. The table’s upper section presents QA tasks, while the lower section covers grounding and referring tasks. We introduce the first roadside video understanding dataset and unify the tasks in one benchmark. }
\resizebox{\textwidth}{!}{%
\begin{tabular}{c|ccccccccccc}
% \hline
\midrule
\textbf{Dataset} & \textbf{Venue} & \textbf{Tasks} & \textbf{QA Gen.} & \textbf{\# Videos/Scenes} & \textbf{\# QAs/Captions}  & \textbf{\# Grounding} & \textbf{Domain} \\
% \hline
\midrule


% BDD-X \cite{kim2018textual}         & ECCV18 & video-level & Manual & $\sim$7k (v) & $\sim$26k & $\sim$3.7 & $\sim$77h & - & 1 & 4 & Driving \\

% HAD \cite{had}         & CVPR'19 & Video QA & Manual & 5.6k & 45k & - & Driving \\

% SUTD \cite{xu2021sutd}         & CVPR 2021 & video-level & Manual & $\sim$10k (v) & $\sim$63k & $\sim$6.3 & - & 70s & - & - & D + T \\

DRAMA \cite{malla2023drama}         & WACV'23 & Video QA& Manual & 18k  & 102k  & - & Driving \\
LingoQA \cite{marcu2024lingoqavisualquestionanswering}  & ECCV'24 & Video QA & Manual & 28k & 419k & - & Driving & \\

NuScenes-QA \cite{qian2024nuscenes}         & AAAI'24 & Image QA & Template & 850 & 460k &  - & Driving \\

DriveLM \cite{drivelm}         & ECCV'24 & Image QA & Temp. + Man. & 188k  & 4.2M & - & Driving \\

% ELM \cite{zhou2024embodied}         & ECCV'24 & Video-Level & Temp. + LLM & - & $\sim$9M & - &  Driving \\


% SQA-3D \cite{sqa3d}  & ICLR'23 & Scene QA & Manual & 650 & 33.4k & - & Indoor\\

City-3DQA \cite{sun20243dquestionansweringcity} & ACM MM'24& Scene QA & Temp. + Man. & 193 & 450k & - &  City \\

\midrule
HC-STVG \cite{hc-stvg} & ACM MM'22 & Video Grounding & Manual &5.6k & - & 5.6k&General\\

DVD-ST \cite{dvd-st} & -  & Video Grounding & Manual & 2.7k & - &5.7k & General  \\

Refer-KITTI \cite{referkitti} & CVPR'23  & Referred-MOT & Manual & 18 & - & 818 & Driving \\

NuPrompt \cite{nuprompt}         & AAAI'25 & Referred-MOT & LLM & 850 & - & 35k  & Driving \\

% STPR && Video-Level &&5.2k&-&30k &General \\

% VD-STG\cite{vidstg} &&&&\\

\midrule

\textbf{TUMTraffic-VideoQA (Ours)} & - & Video QA, ST Grounding & Temp. + LLM  &1k & 87.3k  & 5.7k &  Roadside \\

% \hline
\midrule
\end{tabular}%
}

\label{tab:related_datasets}
\end{table*}

% Granularity in thousands?

% Can we trust an information about a dataset which was found only in another paper?  

% Modality ?


% \begin{table}[htb]
% \centering
% \resizebox{0.5\textwidth}{!}{%
% \begin{tabular}{c|ccccc}
% % \hline
% \midrule
% \textbf{Dataset} & \textbf{Task} & \textbf{\#Scenes} & \textbf{\#QA}  & \textbf{\#Grounding} & \textbf{Domain} \\
% % \hline
% \midrule

% HAD \cite{had}         & Video QA & $\sim$5.6k & $\sim$45k & - & Driving \\

% DRAMA \cite{malla2023drama}         & Video QA & $\sim$18k  & $\sim$102k  & - & Driving \\

% NuScenes-QA \cite{qian2024nuscenes}         & Image QA & 850 & $\sim$460k &  - & Driving \\

% DriveLM \cite{sima2023drivelm}         & Image QA & $\sim$188k  & $\sim$4.2M & - & Driving \\

% SQA-3D \cite{sqa3d}  & Scene QA & 650 & 33.4k & - & Indoor \\

% City-3DQA \cite{sun20243dquestionansweringcity} & Scene QA & 193 & 450k & - & City \\

% \midrule
% Refer-KITTI\cite{referkitti} & Referred-MOT & 18 & - & 818 & Driving \\

% NuPrompt \cite{nuprompt}         & Referred-MOT & 850 & - & 35k  & Driving \\

% DVD-ST\cite{dvd-st} & Video Grounding & 2.7k & - &5.7k & General \\

% HC-STVG\cite{hc-stvg} & Video Grounding & 5.6k & - & 5.6k & General \\

% \midrule

% \textbf{TUMTraffic-VideoQA(Ours)} & Video QA, Grounding & 1k & 88k  & 5.7k & Traffic \\

% % \hline
% \midrule
% \end{tabular}%
% }
% \caption{Related datasets}
% \label{tab:related_datasets}
% \end{table}


% [x] connect table 1 with introductions. 

\subsection{Vision-Language Datasets in Traffic Scenes}
% DriveLM\cite{drivelm},
% HAD \cite{had} and 
With the rapid advancements in LLMs, significant efforts have been made to integrate language into the development of vision-language foundation models. As summarized in Table \ref{tab:related_datasets}, several pioneering datasets have been introduced for traffic scenarios, particularly focusing on vehicle-centric environments \cite{addatasetseurvey}. NuScenes-QA \cite{qian2024nuscenes} provides a question-answering benchmark tailored for driving scenes. Meanwhile, DRAMA \cite{malla2023drama} is designed for video-level open-ended tasks aimed at evaluating driving instructions and assessing the importance of objects within their environments. Besides, referring to specific traffic participants through natural language—commonly known as referred object grounding and tracking—is a crucial task in traffic scene understanding. Some works \cite{referkitti,nuprompt} extend the KITTI \cite{kitti} and nuScenes \cite{caesar2020nuscenesmultimodaldatasetautonomous} datasets, by associating natural language descriptions with specific vehicles and pedestrians. This facilitates fine-grained identification and tracking of traffic participants, allowing for precise object localization based on language descriptions in complex driving environments. However, most existing efforts primarily focus on driving scenarios and are typically constrained to individual tasks such as question answering, video grounding, or referred multi-object tracking. A significant research gap also remains in the availability of large-scale datasets designed specifically for roadside surveillance scenarios. Our work aims to bridge this gap by providing a comprehensive dataset tailored for multiple tasks in roadside traffic understanding within a unified framework.
% is also an important aspect of traffic scene understanding
% introducing a standardized object representation and 


\subsection{Fine-Grained Video Understanding}

Fine-grained video understanding centers on the precise analysis of intricate video content, targeting tasks that demand nuanced reasoning across spatial and temporal dimensions. Some representative tasks include spatio-temporal grounding \cite{vidstg,hc-stvg}, mapping specific objects or events to precise locations and times within a video based on a given query; video object referring \cite{mevis,referkitti,nuprompt}, which involves tracking objects through space and time given text prompts; video temporal grounding \cite{UniVTG,huang2024vtimellm}, identifying specific moments or intervals in a video that align with a provided textual query. These tasks require high precision, nuanced multimodal alignment, and the ability to capture subtle temporal and spatial dynamics. It is particularly challenging due to the difficulty of properly representing fine-grained video details and the inherent cross-modality misalignment. With the advancement of visual LLMs, recent advancements enhance the capabilities of fine-grained video understanding \cite{videunderstandingsurvey} and facilitate understanding across abstract and detailed levels. 

% , with advanced visual embedding techniques and modality alignment strategies to bridge the gap between textual and visual semantics, significantly





\subsection{Language-Based Object Referring}


Referring objects in visual data, such as images and videos, is typically achieved by associating them with predefined definitions or language descriptions. Figure \ref{fig:object_representation} illustrates four commonly used methods for representing objects through language expressions. The inherent ambiguity of natural language, coupled with the modality gap between visual and linguistic representations, presents significant challenges. Object representation in tasks such as object referring often necessitates careful dataset curation to ensure that linguistic expressions uniquely or collectively correspond to specific objects in videos. For example, some datasets include only scenarios with uniquely identifiable objects \cite{hc-stvg}, while others contain expressions that jointly refer to multiple objects \cite{dvd-st}. However, in complex real-world applications such as autonomous driving, textual descriptions alone are often insufficient to uniquely specify an object. To address this challenge, DriveLM \cite{drivelm} introduces a structured tuple representation, $\textless c, CAM, x, y \textgreater$, where  c  denotes the object identifier,  CAM  specifies the camera, and $\textless x, y \textgreater$ represents the 2D center coordinates within the camera’s coordinate system. Alternatively, ELM \cite{zhou2024embodied} simplifies the problem by converting temporal video tasks into frame-level questions, using a tuple $\textless c, x, y \textgreater$ to identify objects within individual frames without temporal dependencies. Despite the advancements, formulating a unified, precise, and unique language representation for objects in video remains open challenges. 




In this work, we design a spatio-temporal object representation in videos with a four-element tuple format $(c, f_n, x, y)$, where c denotes a unique object identifier, $f_n$ indicates the normalized frame timestamp, and $(x, y)$ corresponds to the object’s normalized spatial coordinates within the frame.  The same object is consistently assigned the identifier  c  throughout the video, while its spatial position changes over time. This formulation enables precise tracking and referencing of objects across both spatial and temporal dimensions, facilitating robust language-based interaction in dynamic environments. Besides, it provides a standardized interface for fine-grained video understanding, enabling more detailed and structured analysis.

 
\section{\methodname{}: Automatic Functionality Annotation Pipeline}
\label{sec: annotation pipeline}
This section introduces \methodname{}, an annotation pipeline (Fig.~\ref{fig: anno pipeline}) that automatically produces contextual element functionality annotations used to enhance VLMs' GUI grounding capabilities.


\begin{table}[t]
\tiny
\centering
\caption{\textbf{Comparing our \methodname{} dataset with existing large-scale UI datasets.} Multi-Res means the samples are collected on devices with various resolutions. Auto Anno. means the samples are collected autonomously. \#Anno. means the number of annotated samples provided by the datasets.}
\label{tab:data comparison}
\begin{tabular}{@{}cccccccc@{}}
\toprule
Dataset & UI Type & \begin{tabular}[c]{@{}c@{}}Multi\\ Res.\end{tabular} & \begin{tabular}[c]{@{}c@{}}Real-world\\ Scenario\end{tabular} & \begin{tabular}[c]{@{}c@{}}Auto\\ Anno. \end{tabular} & \begin{tabular}[c]{@{}c@{}}Contextual\\ Functionality\\ Semantics\end{tabular} & \#Anno. & Task \\ \midrule
WebShop~\citep{yao2022webshop} & Web & \cross & \cross & \cross & \cross & 12k & Web Navigation \\
Mind2Web~\citep{deng2024mind2web} & Web & \cross & \cmark & \cross & \cross & 2.4k & Web Navigation \\
WebArena~\citep{zhou2023webarena} & Web & \cross & \cmark & \cross & \cross & 812 & Web Navigation \\
\midrule
S2W~\citep{Wang2021Screen2WordsAM} & Mobile & \cross & \cmark & \cross & \cross & 112k & Screen Summarization \\
Wid. Cap.~\citep{Li2020WidgetCG} & Mobile & \cross & \cmark & \cross & \cross & 163k & Element Captioning \\
PixelHelp~\citep{Li2020MappingNL} & Mobile & \cross & \cmark & \cross & \cross & 187 & Element Grounding \\
RICOSCA~\citep{Li2020MappingNL} & Mobile & \cross & \cmark & \cross & \cross & 295k & Action Grounding \\
MoTIF~\citep{Burns2022ADF} & Mobile & \cross & \cmark & \cross & \cross & 6k & Mobile Navigation \\
AITW~\citep{rawles2023android} & Mobile & \cross & \cmark & \cross & \cross & 715k & Mobile Navigation \\
RefExp~\citep{Bai2021UIBertLG} & Mobile & \cross & \cmark & \cross & \cross & 20.8k & Element Grounding \\
VWB~\citep{liu2024visualwebbench} & Web & \cross & \cmark & \cross & \cross & 1.5k & Elem. Ground \& Ref. \\
SeeClick Web~\citep{cheng2024seeclick} & Web & \cross & \cmark & \cmark & \cross & 271k & Element Grounding \\
UI REC/REG~\citep{hong2023cogagent} & Web & \cmark & \cmark & \cmark & \cross & 400k & Box2DOM, DOM2Box \\
Ferret-UI~\citep{you2024ferretui} & Mobile & \cmark & \cmark & \cmark & \cross & 250k & Elem. Ground \& Ref. \\
\methodname{} (ours) & Web, Mobile & \cmark & \cmark & \cmark & \cmark & 704k & Functionality Ground \& Ref. \\ \bottomrule
\end{tabular}
\end{table}



\begin{figure}[t]
    \centering
    \includegraphics[width=0.95\linewidth]{figure/AnnoPipeline3.pdf}
    \caption{\textbf{The proposed pipeline for automatic UI functionality annotation.} An LLM is utilized to predict element functionality based on the UI content changes observed during the interaction. LLM-aided rejection and verification are introduced to improve data quality. Finally, the high-quality functionality annotations will be converted to instruction-following data by applying task templates.}
    \label{fig: anno pipeline}
\end{figure}


\subsection{Collecting UI Interaction Trajectories}
Our pipeline initiates by collecting interaction trajectories, which are sequences of UI contents captured by interacting with UI elements. Each trajectory step captures all interactable elements and the accessibility tree (AXTree) that briefly outlines the UI structure, which will be used to generate functionality annotations. To amass these trajectories, we utilize the latest Common Crawl repository as the data source for web UIs and Android Emulator for mobile UIs. Note that illegal websites and Apps are excluded manually from the sources to ensure no pornographic or violent content is included in our dataset. Please refer to Sec.~\ref{sec:supp:record traj detail} for collecting details and data license.

\subsection{Functionality Annotation Based on UI Dynamics}
Subsequently, the pipeline generates functionality annotations for elements in the collected trajectories. Interacting with an element $e$, by clicking or hovering over it, triggers content changes in the UI. In turn, these changes can be used to predict the functionality $f$ of the interacted element. For instance, if clicking an element causes new buttons to appear in a column, we can predict that the element likely functions as a dropdown menu activator (an example in Fig.~\ref{fig: funcpred diff case}).
With this observation, we utilize a capable LLM (i.e., Llama-3-70B~\citep{llama3modelcard}) as a surrogate for humans to summarize an element's functionality based on the UI content changes resulting from interaction. Concretely, we generate compact content differences for AXTrees before ($s_t$) and after ($s_{t+1}$) the interaction using a file-comparing library\footnote{https://docs.python.org/3/library/difflib.html}. Then, we prompt the LLM to thoroughly analyze the UI content changes (addition, deletion, and unchanged lines), present a detailed Chain-of-Thoughts~\citep{wei2022chain} reasoning process explaining how the element affects the UI, and finally summarize the element's functionality.

In cases where element interactions significantly transform the UI and cause lengthy differences—such as navigating to a new screen—we adjust our approach by using UI description changes instead of the AXTree differences. Specifically, we prompt the same LLM to discern the UI hierarchy, describe UI regions, and finally describe the entire UI functionality. After describing the UIs before and after the interaction, the LLM analyzes the description differences, presents reasoning, and summarizes the element's functionality. This annotation process is formulated as:
\begin{equation}
    f = \text{LLM}(p_{\text{anno}}, s_t, s_{t+1})
\end{equation}

where $f$ is the predicted functionality, $p_{\text{anno}}$ is the annotation prompt (Tab.~\ref{tab:supp:funcpred manip prompt} and Tab.~\ref{tab:supp:funcpred nav prompt}). Examples of annotated elements are depicted in Fig.~\ref{fig: our dataset} and more annotation details are explained in Sec.~\ref{sec:supp:anno details}.

\subsection{Removing Invalid Samples via LLM-Aided Rejection}
The collected trajectories may contain invalid samples due to broken UIs, such as incomplete UI loading. These samples are meaningless as they contain corrupted UI content and can mislead the models trained with them.

To filter out these invalid samples, we introduce an LLM-aided rejection approach. Initially, hand-written rules are used to detect obvious broken cases, such as blank UI contents, UIs containing elements indicating content loading, and interaction targets outside of UIs. While these obvious cases constitute a large portion of the invalid samples, there are a few types that are difficult to detect with hand-written rules. For instance, interacting with a “view more” button might unexpectedly redirect the user to a login page instead of the desired information page due to website login restrictions. To identify these challenging samples, we prompt the annotating LLM to also act as a rejector. Specifically, the LLM takes the UI content changes, generated using a file-comparing library, as input, provides detailed reasoning on whether the changes are meaningful for predicting the element's functionality, and finally outputs predictability scores ranging from 0 to 3. This process is formulated as follows:
\begin{equation}
 score = \text{LLM}(p_{\text{reject}}, e, s_t, s_{t+1})
\end{equation}
where $p_{\text{reject}}$ is the rejection prompt (Tab.~\ref{tab:supp:rejection prompt}).

This approach ensures that clear and predictable samples receive higher scores, while those that are ambiguous or unpredictable receive lower scores. For instance, if a button labeled "Show More", upon interaction, clearly adds new content, this sample will considered to provide sufficient changes that can anticipate the content expansion functionality and will get a score of 3. Conversely, if clicking on a "View Profile" link fails to display the profile possibly due to web browser issues, this unpredictable sample will get a score less than 3.

After implementing empirical experiments, we deploy this LLM-based rejector to discard the bottom 30\% of samples based on their scores to strike a balance between the elimination of invalid samples and the preservation of valid ones (More details in Sec.~\ref{suc:supp:reject details}). The samples that pass the hand-written rules and the LLM rejector are subsequently submitted for functionality annotation. Please see representative rejection examples in Fig.~\ref{fig: rejection examples}.

\subsection{Improving Annotation Quality via LLM-Based Verification}
The functionality annotations produced by the LLM probably contain incorrect, ambiguous, and hallucinated samples (See a case in Fig.~\ref{fig: anno pipeline}), which probably misleads the trained VLMs and compromises evaluation accuracy. To improve dataset quality, we prompt LLMs to verify the annotations by checking whether the targeted element $e$ fulfills the intent of the annotated functionality $f$. This process presents the LLMs with the interacted element, its UI context, the UI changes induced by this element, and the functionality generated in the previous annotation process. The LLMs are then tasked with analyzing the UI content changes before predicting whether the interacted element aligns with the given functionality. If the LLMs determine that the interacted element fulfills the functionality given its UI context, the LLMs will grant a full score (An example in Fig.~\ref{fig: verif diff case}). If the interacted element is considered to mismatch the functionality, this functionality can be seen as incorrect as this mismatch indicates that it may not accurately reflect the element's actual role within the UI context.

To mitigate the potential biases in LLMs~\citep{panickssery2024llm, zheng2023judging, bai2024benchmarking}, two different LLMs (i.e., Llama-3-70B~\citep{llama3modelcard} and Mistral-7B-Instruct-v0.2~\citep{mistral}) are employed as verifiers and prompted to output 0-3 scores. The scoring process is formulated as follows:
\begin{equation}
 score = \text{LLM}(p_{\text{verify}}, e, f, s_t, s_{t+1})
\end{equation}
where $p_{\text{verify}}$ denotes the verification prompt (Tab.~\ref{tab:supp:verif prompt}). Only if the two scores are both 3s do we consider the functionality label correct (More details in Sec.~\ref{suc:supp:verif details}). Although this filtering approach seems stringent, we can make up the number of annotations through scaling. 

\begin{figure}[t]
    \centering
    \includegraphics[width=0.9\linewidth]{figure/our_dataset_img.pdf}
    \caption{Element functionality annotations generated by the proposed AutoGUI pipeline for both web and mobile viewpoints.}
    \label{fig: our dataset}
    \vspace{-5mm}
\end{figure}

\subsection{Functionality Grounding and Referring Task Generation}
\vspace{-2mm}
After rejecting, annotating, and verifying, we obtain a high-quality UI functionality dataset containing triplets of \{UI screenshot, Interacted element, Functionality\}. To convert this dataset into an instruction-following dataset for training and evaluation, we generate functionality grounding and referring tasks using diverse prompt templates (see Tab.~\ref{tab:task templates}). To mitigate the difficulty of predicting absolute values for various resolutions, the coordinates of element bounding boxes are all normalized within the range $[0,99]$ (see Fig.~\ref{fig: our dataset} for examples).

\subsection{Explore the \methodname{} Dataset}

\begin{table}[]
\centering
\small
\caption{\textbf{The statistics of the AutoGUI datasets.} The Anno. Tokens and Avg. Words columns show the total number of tokens and the average number of words for the functionality annotations regardless of task templates. The Domains/Apps column shows the number of unique web domains/mobile Apps involved in each split.}
\label{tab:simple data stats}
\begin{tabular}{@{}ccccccc@{}}
\toprule
Split & \#Tasks & Anno. Tokens & Avg. Words & Domains/Apps & Device Ratio   \\                                                                   \midrule
Train & 702k  & 17.9M        & 23.1       & 916     & Web: $54.6\%$, Mobile: $45.4\%$                                              \\ \cmidrule(r){1-6}
Test  & 2k    & 53.4k        & 22.5       & 299     & Web: $50\%$, Mobile: $50\%$                                                                                                               \\ \bottomrule
\end{tabular}
\end{table}

\begin{figure}[t]
    \centering
    \includegraphics[width=1.0\linewidth]{figure/wordcloud_token-dist-comparison.pdf}
    \caption{\textbf{Diversity of the AutoGUI dataset.} \textbf{Left}: The word cloud illustrates the ratios of the verbs representing the main intents in the functionality annotations. \textbf{Right}: Comparing the distributions of the annotation token numbers for our AutoGUI training split, SeeClick Web training data~\citep{cheng2024seeclick}, and Widget Captioning~\citep{Li2020WidgetCG}. The comparison demonstrates that our dataset covers significantly more diverse task lengths.}
    \label{fig: wordcloud and tokdistrib}
\end{figure}
\vspace{-2mm}

The \methodname{} pipeline finally collects 22.4k trajectories, from which we select 2k grounding samples (evenly divided between web and smartphone views) as the test set and remove the trajectories to which these samples belong. Subsequently, 702k samples are randomly selected from the remaining instances to constitute the training set. The statistics of our dataset in Tab.~\ref{tab:simple data stats} and Sec.~\ref{sec:supp:data stats} show that our dataset covers diverse UIs and exhibits variety in lengths and functional semantics of the annotations. Moreover, our dataset presents a unique ensemble of research challenges for developing generalist web agents in real-world settings. As shown in Tab.~\ref{tab:data comparison} and Fig.~\ref{fig: functionality vs others}, our dataset distinguishes itself from existing literature by providing functionality-rich data as well as tasks that require VLMs to discern the contextual functionalities of elements to achieve high grounding accuracy.

\section{Analysis of Data Quality}
This section analyzes the reliability of the proposed annotation pipeline and data quality.

\noindent{\textbf{Comparison with Human Annotation}} To demonstrate the superiority of the proposed automatic annotation pipeline based on open-source LLMs, $N=145$ samples (99 valid and 46 invalid) are randomly selected as a testbed for comparing the annotation correctness of a trained human annotator and the pipeline. Here, correctness is defined as $Correctness = C / (N - R)$, where $C$ and $R$ denote the numbers of correctly annotated and rejected samples, respectively. The denominator subtracts the number of rejected samples as we are more interested in the percentage of correct samples after rejecting the samples considered invalid by the annotator. The authors thoroughly check the annotation results according to the three criteria in Fig.~\ref{fig: check criteria}: 1. Context-specificity. The functionality annotations must include context-specific descriptions to ensure one-to-one mapping between the element and its annotation. 2. Appropriate details. Avoid detailing unnecessary aspects of the UIs to keep the description focused on functionality. 3. No hallucination. The annotations must not include information not grounded in the visual context of the UIs. See more details in Sec.~\ref{sec:supp:humaneval details}.

After experimenting with three runs, Tab.~\ref{tab:ablate autogui} shows that the proposed AutoGUI pipeline achieves high correctness comparable to the trained human annotator (r6 vs. r1). Without rejection and verification (r2), AutoGUI is inferior as it cannot recognize invalid samples. Notably, simply using the rules written by the authors can improve the correctness, which is further enhanced with the LLM-aided rejector (r4 vs. r3). Moreover, utilizing the annotating LLM itself to self-verify its annotations helps AutoGUI surpass the trained annotator (r5 vs. r1). Introducing another LLM verifier (i.e., Mistral-7B-Instruct-v0.2) brings a slight increase which results from Mistral recognizing Llama-3-70B’s incorrect descriptions of how dropdown menu options work. Overall, these results justify the efficacy of the AutoGUI annotation pipeline.

Qualitatively comparing the annotation patterns of the human and AutoGUI (Fig.~\ref{fig: autogui vs human}), we find that AutoGUI employs the strong LLM to generate more detailed and clear annotations which would take significantly more time for the human annotator. This result suggests that the AutoGUI pipeline can lessen the burden of collecting data for training UI-VLMs.

\noindent{\textbf{Impact of LLM Output Uncertainty}} The uncertainty of LLM outputs manifests in annotation, rejection, and verification, possibly impacting the quality of the AutoGUI dataset. To evaluate this impact, we first sample 100 valid samples to test the AutoGUI pipeline for three runs. The consistency rate is 94.5\%, indicating that 94.5\% of the samples possess consistent annotation outcomes (i.e. correct or incorrect) across the runs. We also test the LLM-aided rejector with 46 invalid samples and find that the rejection consistency over three runs is 79.3\%. This indicates that LLM uncertainty impacts this rejection process. Nevertheless, this impact is minor due to the low prevalence of invalid samples (4\% of all samples) that fail the hand-written rules.

In summary, AutoGUI exhibits annotation correctness comparable to that of human annotators and LLM output uncertainty poses a minor impact on the AutoGUI annotation process.



\begin{figure}[t]
    \centering
    \includegraphics[width=0.85\linewidth]{figure/check_criteria_img.pdf}
    \caption{The checking criteria used for comparing AutoGUI pipeline and the human annotator.}
    \label{fig: check criteria}
\end{figure}


\begin{table}[]
\small
\centering
\caption{\textbf{Comparing the AutoGUI and human annotator.} AutoGUI with the proposed rejection and verification achieves annotation correctness comparable to trained human annotators. One LLM means Llama-3-70B and Two LLMs include Mistral-7B-Instruct-v0.2 as well.}
\label{tab:ablate autogui}
\begin{tabular}{@{}ccccc@{}}
\toprule
No. & Annotator  & Rejector   & Verifier              & Correctness \\ \midrule
r1 & Human      & -          & -                     & 95.5\%      \\
r2 & Llama-3-70B & -          & -                     & 64.5\%      \\
r3 & Llama-3-70B & Rules      & -                     & 83.1\%      \\
r4 & Llama-3-70B & Rules+LLM  & -                     & 94.4\%      \\
r5 & Llama-3-70B & Rules+LLM  & One LLM            & 96.0\%      \\
r6 & Llama-3-70B & Rules+LLM & Two LLMs & \textbf{96.7\%}      \\ \bottomrule
\end{tabular}
\end{table}
\vspace{-2mm}


\section{Experiment}
\label{sec:Experiment}

\subsection{Dataset}
Outside Knowledge Visual Question Answering (OK-VQA)~\cite{marino2019ok} is a benchmark dataset designed to evaluate VQA systems that require leveraging external knowledge sources beyond the information present in an image. The dataset consists of 14,055 knowledge-based questions paired with 14,031 images from the COCO dataset~\cite{lin2014microsoft}. These questions span 10 diverse knowledge categories, including domains such as Science and Technology, Geography, Cooking and Food, and Vehicles and Transportation. The questions were crowdsourced via Amazon Mechanical Turk, ensuring they require real-world knowledge to answer, making this dataset significantly more challenging than conventional VQA datasets. 

The dataset is split into 9,009 training samples and 5,046 testing samples, with each question associated with 10 ground-truth answers annotated by human annotators. This multi-answer format helps address ambiguity and variability in responses. Table~\ref{tab:okvqa_details} outlines key statistics and the distribution of questions across various knowledge categories in the Ok-VQA dataset. Baseline evaluations on OK-VQA using state-of-the-art models like MUTAN and Bilinear Attention Networks (BAN) reveal a significant drop in performance compared to traditional VQA datasets. This performance degradation underscores the need for models with enhanced retrieval and reasoning capabilities to incorporate unstructured, open-domain knowledge effectively.

\begin{table}[h!]
    \centering
    \footnotesize
    \setlength{\tabcolsep}{4pt}
    \renewcommand{\arraystretch}{1.2}
    \caption{Key Details of the OK-VQA Dataset}
    \begin{tabular}{|p{3.2cm}|p{4.8cm}|}
        \hline
        \textbf{Attribute}                & \textbf{Details} \\
        \hline
        \textbf{Name}                     & OK-VQA (Outside Knowledge VQA) \\
        \hline
        \textbf{Source}                   & COCO Image Dataset \\
        \hline
        \textbf{Number of Questions}      & 14,055 \\
        \hline
        \textbf{Number of Images}         & 14,031 \\
        \hline
        \textbf{Question Categories}      & 10 Categories \\
        \hline
        \textbf{Categories Breakdown}     & Vehicles \& Transportation (16\%) \newline Brands, Companies \& Products (3\%) \newline Objects, Materials \& Clothing (8\%) \newline Sports \& Recreation (12\%) \newline Cooking \& Food (15\%) \newline Geography, History, Language \& Culture (3\%) \newline People \& Everyday Life (9\%) \newline Plants \& Animals (17\%) \newline Science \& Technology (2\%) \newline Weather \& Climate (3\%) \newline Other (12\%) \\
        \hline
        \textbf{Average Question Length}  & 8.1 words \\
        \hline
        \textbf{Average Answer Length}    & 1.3 words \\
        \hline
        \textbf{Unique Questions}         & 12,591 \\
        \hline
        \textbf{Unique Answers}           & 14,454 \\
        \hline
        \textbf{Answer Annotations}       & 10 answers per question \\
        \hline
        \textbf{Answer Types}             & Open-ended \\
        \hline
        \textbf{Requires External Knowledge} & Yes (e.g., Wikipedia, Common Sense, etc.) \\
        \hline
        \textbf{Typical Knowledge Sources}& Unstructured Text (Wikipedia) \\
        \hline
    \end{tabular}
    \label{tab:okvqa_details}
\end{table}

\subsection{Implementation Details}
The experiments are conducted on Google Colab using a T4 GPU. The NVIDIA T4 GPU features 16 GB of GDDR6 memory, 320 Tensor Cores, and supports mixed-precision computation, making it suitable for deep learning tasks. Due to computational constraints, we evaluate our model on a subset of 100 samples from the OK-VQA dataset~\cite{marino2019ok}.

\subsection{OOD and ID Category Splits}
In our experiments, we evaluate our approach using the OK-VQA dataset~\cite{marino2019ok}, which we split into OOD and ID subsets based on knowledge categories. The OOD categories include Vehicles and Transportation, Brands, Companies and Products, Sports and Recreation, Science and Technology, and Weather and Climate. The ID categories comprise Objects, Materials and Clothing, Cooking and Food, Geography, History, Language and Culture, People and Everyday Life, Plants and Animals, and Other. Using this split, we can assess how well the model generalizes to different categories of knowledge.

\subsection{Patch-Based Image Preprocessing}
For VQA processing, we preprocess each input image by dividing it into patches of various sizes, specifically 2×2, 3×3 and 4x4 grids. This patch-based approach captures fine-grained visual details, which can enhance feature extraction for complex queries. We then employ the BLIP-VQA model~\cite{li2022blip} to extract image representations and generate initial contextual information based on the image and the associated question.

\subsection{Retrieval-Augmented Knowledge Integration}
To incorporate external knowledge, we use  RAG~\cite{lewis2020retrieval} with external knowledge sources such as Wikipedia and DBpedia. RAG retrieves relevant information based on the question and the visual features extracted by BLIP-VQA~\cite{li2022blip}. This retrieval process supplies the model with real-world context beyond the image, which is crucial for correctly answering questions that depend on external knowledge.

\subsection{State-of-the-Art Performance Comparison}
We evaluate our proposed FilterRAG framework on the OK-VQA dataset and compare it to state-of-the-art methods (Table~\ref{table:SOTA-OK-VQA}). The baseline models, Base1 and Base2, use the BLIP-VQA model with the VQA v2~\cite{goyal2017making} and OK-VQA datasets~\cite{marino2019ok}, achieving 83.0\% and 40.0\% accuracy, respectively. The drop highlights the challenge of knowledge-based questions in OK-VQA. Our FilterRAG framework, which integrates BLIP-VQA, RAG, and external knowledge sources like Wikipedia and DBpedia, achieves 36.5\% accuracy in OOD settings. This result demonstrates the effectiveness of grounding VQA responses with external knowledge, especially for OOD scenarios. 

Compared to state-of-the-art methods, KRISP~\cite{marino2021krisp}  achieves 38.35\% with Wikipedia and ConceptNet, while MAVEx~\cite{wu2022multi} reaches 41.37\% using Wikipedia, ConceptNet, and Google Images. The highest performance comes from KAT (ensemble)~\cite{gui2021kat} at 54.41\% with Wikipedia and Frozen GPT-3. Although these models achieve higher accuracy, they often require significant computational resources. 

FilterRAG balances performance and efficiency, making it suitable for resource-constrained environments. As shown in Figure~\ref{fig:plot1_accuracy}, it achieves 37.0\% accuracy in ID settings, 36.0\% in OOD settings, and 36.5\% when combining ID and OOD data. This highlights its robustness for knowledge-intensive VQA tasks.

\begin{figure}[h!]
    \centering
    \includegraphics[width=\linewidth]{figures/plot1_accuracy_v2.pdf}
    \caption{Comparison of Model Accuracy Across Different Settings.}
    \label{fig:plot1_accuracy}
\end{figure}

\begin{table*}[t]
    \centering
    \footnotesize
    \caption{Performance Comparison of State-of-the-Art Methods on the OK-VQA Dataset}
    \label{tab:okvqa_results}
    \renewcommand{\arraystretch}{1.2}
    \setlength{\tabcolsep}{10pt}
    \begin{tabular}{l l c}
        \toprule
        \textbf{Method}                                & \textbf{External Knowledge Sources}                          & \textbf{Accuracy (\%)} \\
        \midrule
        Q-only (Marino et al., 2019)~\cite{marino2019ok}                  & —                                                          & 14.93                  \\
        MLP (Marino et al., 2019)~\cite{marino2019ok}                     & —                                                          & 20.67                  \\
        BAN (Marino et al., 2019)~\cite{marino2019ok}              & —                                                          & 25.1                  \\
        MUTAN (Marino et al., 2019)~\cite{marino2019ok}               & —                                                          & 26.41                  \\
        ClipCap (Mokady et al., 2021)~\cite{mokady2021clipcap}                 & —                                                          & 22.8                   \\
        \midrule
        BAN + AN (Marino et al., 2019~\cite{marino2019ok}                  & Wikipedia                                                  & 25.61                  \\
        BAN + KG-AUG (Li et al., 2020)~\cite{li2020boosting}        & Wikipedia + ConceptNet                                     & 26.71                  \\
        Mucko (Zhu et al., 2020)~\cite{zhu2020mucko}                      & Dense Caption                                              & 29.2                   \\
        ConceptBERT (Gardères et al., 2020)~\cite{garderes2020conceptbert}           & ConceptNet                                                 & 33.66                  \\
        KRISP (Marino et al., 2021)~\cite{marino2021krisp}                   & Wikipedia + ConceptNet                                     & 38.35                  \\
        RVL (Shevchenko et al., 2021)~\cite{shevchenko2021reasoning}                 & Wikipedia + ConceptNet                                     & 39.0                   \\
        Vis-DPR (Luo et al., 2021)~\cite{luo2021weakly}                    & Google Search                                              & 39.2                   \\
        MAVEx (Wu et al., 2022)~\cite{wu2022multi}                       & Wikipedia + ConceptNet + Google Images                    & 41.37                  \\
        PICa-Full (Yang et al., 2022)~\cite{yang2022empirical}                 & Frozen GPT-3 (175B)                                        & 48.0                   \\
        KAT (Gui et al., 2022) (Ensemble)~\cite{gui2021kat}             & Wikipedia + Frozen GPT-3 (175B)                           & 54.41                  \\
        REVIVE (Lin et al., 2022) (Ensemble)~\cite{lin2022revive}          & Wikipedia + Frozen GPT-3 (175B)                           & 58.0                   \\
        RASO (Fu et al., 2023)~\cite{fu2023generate}                        & Wikipedia + Frozen Codex                                   & 58.5                   \\
        \midrule
        \textbf{FilterRAG (Ours)}                     & Wikipedia + DBpedia (\textbf{Frozen} BLIP-VQA and GPT-Neo 1.3B)    & \textbf{36.5}          \\
        \bottomrule
    \end{tabular}
    \label{table:SOTA-OK-VQA}
\end{table*}


\subsection{Hallucination Detection via Grounding Scores}
We evaluate the grounding scores of our FilterRAG framework against baseline models to assess its ability to mitigate hallucinations by aligning answers with external knowledge. As shown in Figure~\ref{fig:plot2_grounding_score}, Base1 achieves the highest grounding score of 94.60\% on the VQA v2 dataset~\cite{goyal2017making}, indicating that BLIP performs effectively when answering general-domain questions that do not require external knowledge. In contrast, Base2, evaluated on the OK-VQA dataset~\cite{marino2019ok}, shows a significant drop to 71.70\%, highlighting the challenge of answering knowledge-based questions without access to external information, thereby increasing the likelihood of hallucinations.

\begin{figure}[h!]
    \centering
    \includegraphics[width=\linewidth]{figures/plot2_grounding_score_v2.pdf}
    \caption{Grounding Score Comparison Across Baselines and Proposed Methods.}
    \label{fig:plot2_grounding_score}
\end{figure}

To address this limitation, our proposed method integrates BLIP-VQA, RAG, and external knowledge sources such as Wikipedia and DBpedia. The grounding scores for our method are 70.06\% for In-Distribution (ID) data, 70.68\% for Out-of-Distribution (OOD) data, and 70.37\% when combining both settings. These consistent scores demonstrate that FilterRAG effectively grounds answers in retrieved knowledge, reducing hallucinations even in challenging OOD scenarios.

Although our method does not achieve the grounding performance of Base1, it provides reliable results for knowledge-intensive tasks by leveraging external knowledge sources. This makes FilterRAG a robust and efficient solution for real-world VQA applications, particularly where external knowledge and OOD generalization are critical.

\subsection{Ablation Study}
We evaluate the effect of different image grid sizes on the performance of our FilterRAG framework with BLIP-VQA and RAG in OOD scenarios. We consider three grid configurations, 2x2, 3x3, and 4x4, and evaluate their influence on accuracy and grounding score. As shown in Figure~\ref{fig:plot5_measure_grid_size}, accuracy decreases slightly as the grid size increases. The accuracy is 37.00\% for the 2x2 grid, declines to 35.00\% for the 3x3 grid, and further drops to 34.00\% for the 4x4 grid. This downward trend indicates that larger grid sizes lead to excessive fragmentation, making it challenging for the model to extract coherent and meaningful visual features.

\begin{figure}[h!]
    \centering
    \includegraphics[width=\linewidth]{figures/plot5_measure_grid_size_v2.pdf}
    \caption{Effect of Grid Sizes on Accuracy and Grounding Score.}
    \label{fig:plot5_measure_grid_size}
\end{figure}

Similarly, the grounding score follows a declining trend with increasing grid size. The grounding score is 70.06\% for the 2x2 grid, reducing to 69.20\% for the 3x3 grid and 68.07\% for the 4x4 grid. This decline suggests that finer grid divisions hinder the model’s ability to align generated answers with retrieved external knowledge, likely due to the loss of contextual coherence when images are broken into smaller patches.

Overall, the 2x2 grid size achieves the best trade-off between accuracy and grounding score. It maintains both visual coherence and effective knowledge alignment, thereby reducing the risk of hallucinations. Consequently, for OOD scenarios in the FilterRAG framework, the 2x2 grid configuration is the most effective for ensuring robust and reliable performance.

\subsection{Qualitative Analysis}
We perform a qualitative analysis of FilterRAG on the OK-VQA dataset~\cite{marino2019ok}, evaluating its performance in both In-Domain (ID) and Out-of-Distribution (OOD) settings. As illustrated in Figure~\ref{fig:Qualitative_Analysis}, FilterRAG generates accurate answers in ID scenarios where the retrieved knowledge is relevant and aligns well with the visual context. In these cases, the model effectively combines visual cues and external knowledge, resulting in well-grounded responses. These errors are frequently caused by misalignment between the visual context and the retrieved information, reflecting the challenge of handling ambiguous or novel queries outside the training distribution.

In OOD settings, FilterRAG struggles when relevant knowledge of unfamiliar concepts cannot be effectively retrieved. This often leads to hallucinations, where the model produces plausible but incorrect answers that are not supported by the retrieved evidence. This analysis highlights the critical role of reliable knowledge retrieval and precise multimodal alignment in mitigating hallucinations. Improving the quality of knowledge retrieval and refining visual-textual alignment are essential steps toward making FilterRAG more reliable in OOD contexts. Future improvements in these areas can help ensure more accurate and context-aware responses in real-world VQA applications.

\section{Conclusion}
\label{subsection:conclusion}
In this paper, we introduce \OURS, a novel framework designed to identify high-quality data that aligns well with the LLM’s learned knowledge to reduce hallucination.
% Our proposed method includes Internal Consistency Probing and Semantic Equivalence Identification, which are designed to separately measure the LLM's understanding of the given instruction and target response.
% In this way, we can measure the familiarity of the LLM with the instruction data and prevent the model from being trained on unfamiliar data, thereby reducing hallucinations.
NOVA includes Internal Consistency Probing and Semantic Equivalence Identification, which are designed to separately measure the LLM's familiarity with the given instruction and target response, then prevent the model from being trained on unfamiliar data, thereby reducing hallucinations.
Lastly, we introduce an expert-aligned reward model, considering characteristics beyond just familiarity to enhance data quality.
By considering data quality and avoiding unfamiliar data, we can use the selected data to effectively align LLMs to follow instructions and hallucinate less in the instruction tuning stage.
Experiments and analysis show the effectiveness of \OURS.

\section*{Limitations}
Although empirical experiments have confirmed the effectiveness of the proposed \OURS, two major limitations remain. 
Firstly, our proposed method requires LLMs to generate multiple responses for the given instruction, which introduces additional execution time.
However, it is worth noting that this additional execution time is used to perform offline data filtering, our proposed method does not introduce additional time overhead in the inference phase.
Additionally, \OURS~is primarily used for single-turn instruction data filtering, thus exploring its application in multi-turn scenarios presents an attractive direction for future research.

% \section*{Acknowledgments}

% This document has been adapted
% by Steven Bethard, Ryan Cotterell and Rui Yan
% from the instructions for earlier ACL and NAACL proceedings, including those for
% ACL 2019 by Douwe Kiela and Ivan Vuli\'{c},
% NAACL 2019 by Stephanie Lukin and Alla Roskovskaya,
% ACL 2018 by Shay Cohen, Kevin Gimpel, and Wei Lu,
% NAACL 2018 by Margaret Mitchell and Stephanie Lukin,
% Bib\TeX{} suggestions for (NA)ACL 2017/2018 from Jason Eisner,
% ACL 2017 by Dan Gildea and Min-Yen Kan,
% NAACL 2017 by Margaret Mitchell,
% ACL 2012 by Maggie Li and Michael White,
% ACL 2010 by Jing-Shin Chang and Philipp Koehn,
% ACL 2008 by Johanna D. Moore, Simone Teufel, James Allan, and Sadaoki Furui,
% ACL 2005 by Hwee Tou Ng and Kemal Oflazer,
% ACL 2002 by Eugene Charniak and Dekang Lin,
% and earlier ACL and EACL formats written by several people, including
% John Chen, Henry S. Thompson and Donald Walker.
% Additional elements were taken from the formatting instructions of the \emph{International Joint Conference on Artificial Intelligence} and the \emph{Conference on Computer Vision and Pattern Recognition}.

% Bibliography entries for the entire Anthology, followed by custom entries
%\bibliography{anthology,custom}
% Custom bibliography entries only
\bibliography{custom}

\appendix
\onecolumn
% \section{You \emph{can} have an appendix here.}

% You can have as much text here as you want. The main body must be at most 88 pages long.
% For the final version, one more page can be added.
% If you want, you can use an appendix like this one.  

% The ∖onecolumn\mathtt{\backslash onecolumn} command above can be kept in place if you prefer a one-column appendix, or can be removed if you prefer a two-column appendix.  Apart from this possible change, the style (font size, spacing, margins, page numbering, etc.) should be kept the same as the main body.
% %%%%%%%%%%%%%%%%%%%%%%%%%%%%%%%%%%%%%%%%%%%%%%%%%%%%%%%%%%%%%%%%%%%%%%%%%%%%%%%
% %%%%%%%%%%%%%%%%%%%%%%%%%%%%%%%%%%%%%%%%%%%%%%%%%%%%%%%%%%%%%%%%%%%%%%%%%%%%%%%

\section{Details of Method Comparison}
\label{app:comparison}
In this section, we provide detailed explanations of the configurations used in our experiments for comparison, including the routing mechanisms, optimization algorithms, and evaluation metrics.

\subsection{Routing Mechanisms and Optimization Algorithms}

\textbf{Classifier-Based (CL) Routing:} We replaced our Knowledge-Aware (KA) routing mechanism with a classifier-based routing (CL) approach. The CL mechanism uses Sentence-BERT to encode both the instruction and the expert’s knowledge concept into vector representations. Cosine similarity is then calculated between these vectors, and the expert with the highest similarity score is selected.

\textbf{Proximal Policy Optimization (PPO):} A reinforcement learning algorithm that updates policies in a stable and efficient manner. It was applied to optimize expert selection by training a policy network to maximize routing performance.

\textbf{Monte Carlo Tree Search (MCTS):} MCTS is employed to explore potential expert selections by simulating multiple decision paths and backpropagating scores from the outcomes. This algorithm is particularly useful for decision-making in environments with large search spaces.

\textbf{Advantage Actor-Critic (A2C):} A2C combines the actor-critic framework with an advantage function to improve policy updates. The actor selects experts, while the critic evaluates the quality of these decisions, enabling more efficient learning.

\subsection{Metrics to Evaluate Routing Quality}

We provide detailed definitions and formulations for the two metrics used to evaluate the performance of the routing strategies: \textbf{Routing Alignment Score (RAS)} and \textbf{Preference-Weighted Routing Score (PWRS)}.

\subsubsection{Routing Alignment Score (RAS)}

The Routing Alignment Score (RAS) measures the degree to which the router's expert selection aligns with human expert annotations. It quantifies the consistency between the router's decisions and the ground truth labels provided by human annotators. 

\begin{equation}
\text{RAS} = \frac{C}{N}
\end{equation}

where $C$ denotes the number of routed experts that align with human preferences and $N$ denotes the total number of routed experts (in this case: $805 \times 2$).


\paragraph*{Human Evaluation Protocol} 
To establish reliable ground truth labels, we engaged a panel of 7 domain experts with 3+ years of experience in AI system evaluation. Each expert independently annotated 1,610 routing instances (805 instruction-expert pairs $\times$ 2 routing paths) through a two-phase process:
\begin{itemize}
    \item \textbf{Calibration Phase}: Experts jointly reviewed 200 samples to establish annotation guidelines and resolve edge cases.
    \item \textbf{Final Annotation}: The remaining 1,410 instances were randomly distributed (200 instances per expert) with 10\% overlap for inter-annotator agreement calculation. 
\end{itemize}
We achieved substantial agreement with Fleiss' $\kappa=0.78$, calculated on the overlapping samples. Final labels were determined through majority voting.

The RAS provides a basic measure of alignment between the router's decisions and the ground truth, reflecting the accuracy of the routing mechanism in selecting the most appropriate experts.

\subsubsection{Preference-Weighted Routing Score (PWRS)}
The Preference-Weighted Routing Score (PWRS) extends traditional routing accuracy metrics by incorporating human preference scores derived from the AlpacaEval 2.0 evaluation framework. This metric weights routing decisions based on the quality of the expert outputs as judged by human evaluators. The PWRS is defined as follows:

\begin{equation}
\text{PWRS} = \frac{\sum_{i=1}^{N} (p_i \cdot c_i)}{N}
\end{equation}

where $p_i$ represents the preference score from AlpacaEval 2.0 for the routed expert's output, $c_i$ is the number of routed experts that align with human preferences, and $N$ denotes the total number of routed experts.

\paragraph*{Preference Score Integration} 
The AlpacaEval 2.0 scores were obtained from a separate group of 15 crowdworkers following the standardized evaluation protocol. Each output was rated by 3 distinct evaluators using a 7-point Likert scale across three dimensions: helpfulness (actionable solutions), accuracy (factual grounding), and coherence (logical flow). Discrepancies exceeding 2 points triggered expert review, with final scores normalized using Bradley-Terry pairwise comparison models. These preference scores enable the PWRS to transcend binary routing accuracy by weighting decisions according to the relative quality of expert outputs, where higher weights correspond to outputs demonstrating stronger alignment with human-judged quality dimensions.

The PWRS thus provides a dual-aspect evaluation: it preserves the fundamental routing correctness measurement through expert selection alignment, while simultaneously quantifying the performance advantage gained through preference-aware routing decisions.

\section{Supplementary Experimental Validation and Analysis}

\subsection{Performance Evaluation}
In order to perform a comprehensive and controlled performance evaluation, we selected two representative tasks from the BIG-bench Hard (BBH) dataset: commonsense reasoning (550 samples) and logical reasoning (600 samples). The reasons for choosing these two tasks are: (1) they effectively validate the core capabilities of the model; (2) they have clear evaluation criteria; (3) the sample size is moderate, which facilitates sufficient multi-round cross-validation. In this experiment, we compare KABB with MoA and its lightweight version MoA-lite. Three key metrics were used for evaluation: (1) Knowledge matching F1 score, computed using BERT to calculate the semantic similarity between expert capabilities and knowledge graph concepts (threshold of 0.75); (2) Path prediction accuracy, based on standard knowledge dependency paths, with a perfect match scoring full points, a path length difference of \text{$\leq$} 1 and key node matches scoring 0.5 points; (3) Historical performance prediction accuracy, using the dynamic weight $\alpha / (\alpha + \beta)$ (where $\alpha$ and $\beta$ represent the number of successful and failed tasks, respectively), with a prediction error \text{$\leq$} 0.1 considered correct. The experimental results are shown in Table 3:

The performance of the three models on key metrics is as follows:

\begin{center}
\begin{tabular}{cccccc}
\hline
Evaluation Metric & KABB & MoA & MoA-lite & vs. MoA & vs. lite \\
\hline
Knowledge Matching F1 (\%) & 86.5 & 71.2 & 46.8 & +15.3\% & +39.7\% \\
Path Prediction Accuracy (\%) & 84.9 & 69.5 & 44.2 & +15.4\% & +40.7\% \\
Historical Performance Prediction (\%) & 85.2 & 70.1 & 45.5 & +15.1\% & +39.7\% \\
\hline
\end{tabular}
\end{center}

The experimental results show that KABB significantly outperforms the baseline models on all key metrics. Compared to the standard MoA, KABB shows an average improvement of 15.3\% across all indicators; compared to the lightweight MoA-lite, the improvement reaches 40\%. This performance enhancement is primarily attributed to the knowledge-aware attention mechanism and dynamic path prediction strategy that we proposed. Notably, KABB exhibits stronger generalization ability in the commonsense reasoning task, validating the effectiveness of our knowledge-enhanced approach.

\subsection{Parameter Sensitivity Analysis}

This section explores the impact of three key parameters in the KABB framework—knowledge distance threshold, time decay factor, and efficiency metric—on system performance. The experiment uses the BBH dataset (commonsense reasoning 580 samples, logical reasoning 570 samples), with standard MoA and MoA-lite as baselines, and evaluates parameter sensitivity using a controlled variable approach. The evaluation metrics used are: knowledge matching F1 score, reasoning accuracy, and response efficiency. The experiment tests different values for the knowledge distance threshold [0.55-0.95] and time decay factor [0.2-1.0].

\subsubsection{Knowledge Distance Threshold}

\begin{center}
\begin{tabular}{cccc}
\toprule
Parameter Value & Knowledge Matching F1 (\%) & Reasoning Accuracy (\%) & Efficiency Metric (\%) \\
\midrule
0.55 & 72.3 & 74.8 & 68.2 \\
0.65 & 83.8 & 85.4 & 79.5 \\
\textbf{0.75} & \textbf{94.9} & \textbf{94.9} & \textbf{92.8} \\
0.85 & 87.5 & 88.2 & 84.3 \\
0.95 & 78.7 & 82.7 & 73.6 \\
\bottomrule
\end{tabular}
\end{center}

\textbf{Analysis}: When the threshold is set to 0.75, the system achieves the highest values in knowledge matching F1 score, reasoning accuracy, and efficiency metric, reaching 94.9\%, 94.9\%, and 92.8\%, respectively. A lower threshold (e.g., 0.55) introduces too many irrelevant experts, leading to a decline in knowledge matching and reasoning accuracy, while a higher threshold (e.g., 0.95) makes the expert selection too strict, reducing system coverage and efficiency.

\subsubsection{Time Decay Factor}

\begin{center}
\begin{tabular}{cccc}
\hline
Parameter Value & Knowledge Matching F1 (\%) & Reasoning Accuracy (\%) & Efficiency Metric (\%) \\
\hline
0.2 & 75.1 & 78.3 & 71.4 \\
0.4 & 85.4 & 87.2 & 82.6 \\
\textbf{0.6} & \textbf{94.9} & \textbf{94.9} & \textbf{92.8} \\
0.8 & 88.2 & 90.3 & 85.7 \\
1.0 & 82.7 & 86.5 & 78.9 \\
\hline
\end{tabular}
\end{center}

\textbf{Analysis}: When the time decay factor is set to 0.6, the system performs optimally across all metrics, indicating a good balance between utilizing historical experience and dynamic adaptability. A smaller factor (e.g., 0.2) makes the system overly dependent on short-term fluctuations, reducing stability, while a larger factor (e.g., 1.0) suppresses adaptability to recent performance.

\section{Effect of the Number of Selected Concepts and Experts.}

% \textbf{Concept and Expert Quantity Analysis} 

Our empirical analysis of KABB's architectural configurations reveals the critical interplay between the number of selected concepts and experts (see \cref{num}). The results demonstrate that performance varies substantially across different configurations, with win rates ranging from 56\% to 81\%. Notably, a configuration of 2 concepts with 3 experts achieves optimal performance under constrained computational resources, while expanding to 3 concepts with 6 experts yields the highest observed win rate of 81\%.

Our findings indicate that configurations utilizing 3 or more experts, combined with a moderate-to-large concept space, consistently outperform alternatives. This suggests that both the expert capacity and the conceptual representation space play crucial roles in determining system effectiveness. Interestingly, the relationship between expert count and performance exhibits non-linear characteristics - configurations with moderate numbers of experts (3-6) already achieve robust performance levels, suggesting efficient utilization of multi-expert collaboration. This observation has important implications for resource-performance optimization in practical deployments.


% Experimental results in \cref{num} demonstrate that KABB's performance is highly sensitive to both the number of concepts and experts. Notably, configurations with 2 concepts and 3 experts achieve the highest win rate under low expert costs, while 3 concepts and 6 experts yield the highest overall win rate of 81\%. This maximum win rate significantly surpasses the lowest observed rate of 56\% across all tested configurations. In general, setups with 3 to 5 experts and moderate-to-high numbers of concepts consistently deliver superior performance, whereas others tend to underperform. These findings highlight the critical role of expert and concept quantities in determining outcomes. Furthermore, the results underscore that KABB leverages multi-expert collaboration to improve performance in a non-linear manner, achieving competitive results even with a moderate number of experts. Performance, however, remains configuration-dependent, emphasizing the importance of careful tuning.


\begin{figure}[h]
% \vskip 0.2in
\begin{center}
\centerline{\includegraphics[width=0.5\linewidth]{num3.pdf}}
\caption{Relationship between the number of selected experts and selected concepts, and the AlpacaEval 2.0 LC Win Rate.}
\label{num}
\end{center}
\vskip -0.2in
\end{figure}


\section{Evaluations on Reasoning and Problem-Solving Tasks}
\label{app:reasoning}

\subsection{Benchmarks}
 For reasoning and problem-solving tasks, We evaluate using three benchmarks: BBH \cite{suzgun2022challenging}, MATH \cite{hendrycks2021measuring}, and Arena-Hard \cite{zheng2023judging}.
 
\textbf{BBH (Big-Bench Hard)} is a challenging subset of the BIG-Bench benchmark that tests advanced reasoning capabilities. Includes diverse tasks in mathematical reasoning, logical deduction, and commonsense inference, evaluating models' generalization and complex problem-solving abilities.

\textbf{MATH} is a specialized assessment for AI mathematical capabilities. Features competition-level problems across algebra, number theory, combinatorics, and geometry. Includes detailed solutions for comprehensive evaluation of reasoning depth and computational accuracy.

\textbf{Arena-Hard} is a collection of 500 challenging problems from public leaderboards and research papers, covering programming, mathematics, and logical reasoning. 

\subsection{Experiment Setup}
For BBH and MATH, we designated LLaMa-3-70B-Instruct and Qwen2-72B-Instruct as the experts and Qwen2-72B-Instruct as the aggregator to construct a simple but effective multi-agent system, with one concept and one expect selected for instruction. 

For Arena-Hard, we use the default configuration of KABB with the six open-source models (see \cref{sec:setup}). Additionally, we evaluate KABB w/o Deepseek and KABB-Single-LLaMa3. All models are evaluated under a controlled environment with fixed hyperparameters to ensure fairness.

\subsection{Results and Analysis}
\begin{table}[H]
    \centering
    \small
    \begin{tabular}{lcc}
        \toprule
        Model & BBH & MATH \\
        \midrule
        \textbf{KABB} & \textbf{84.2} & \textbf{59.8} \\
        MoA & 81.8 & 57.3 \\
        Qwen2-72B-Instruct & 82.4 & 51.1 \\
        LLaMa-3-70B-Instruct & 81.0 & 42.5 \\
        \bottomrule
    \end{tabular}
    \caption{Performance comparison on BBH and MATH benchmarks.}
    \label{tab:bbh_math}
\end{table}

Table \ref{tab:bbh_math} presents the performance of KABB and baseline models on the BBH and MATH benchmarks. KABB achieves the highest performance on both benchmarks, surpassing MoA by +2.4\% on BBH and +2.5\% on MATH. The significant gain on MATH highlights the effectiveness of our structured multi-agent approach in handling complex mathematical reasoning tasks.

Table \ref{tab:arena_hard_results} reports model performance on the Arena-Hard benchmark. KABB demonstrates competitive performance (74.8\%) but falls behind GPT-4 models in this benchmark. The Deepseek-R1 model achieves the highest score (92.3\%), indicating its strong generalization capabilities. The KABB-Single-LLaMa3 outperforms Single LLaMa-3-70B-Instruct by 4.8\%. Removing Deepseek models (KABB w/o Deepseek) significantly reduces performance (-12.0\%), confirming their critical role in the system. 

It is noteworthy that MoA achieved a similar performance to ours. In the context of well-defined problem-solving tasks (such as programming and mathematical problem-solving), empirical evidence suggests that multi-agent architectures may encounter specific limitations. The integration of multiple agents can potentially introduce operational redundancies and decisional interference, which may adversely impact the system's capacity to converge on correct solutions or generate optimal outputs. This presents a notable challenge in domains where problem spaces are closed and solutions are deterministic. \cref{sec:case} includes a case when some models produce low-quality answers on Arena-Hard.

\begin{table}[t]
    \centering
    \small
    \sisetup{
        table-format=2.1,
        separate-uncertainty=true,
        detect-weight=true,
        detect-inline-weight=math
    }
    \begin{tabular}{lc}
        \toprule
        Model & \textbf{Arena-Hard win. (\%)} \\
        \midrule
        \rowcolor[gray]{0.95} KABB & 74.8 \\
        MoA & 74.3 \\
        \rowcolor[gray]{0.95} KABB w/o Deepseek & 62.8 \\
        GPT-4 Omni (05/13) & 79.2 \\
        GPT-4 Turbo (04/09) & 82.0 \\
        GPT-4 Preview (11/06) & 78.7 \\
        GPT-4 (03/14) & 50.0 \\
        Qwen2-72B-Instruct & 46.9 \\
        Gemma-2-27B & 57.5 \\
        WizardLM-2-8x22B & 71.3 \\
        \rowcolor[gray]{0.95} KABB-Single-LLaMa3 & 51.4 \\
        LLaMa-3-70B-Instruct & 46.6 \\
        Deepseek-V3 & \underline{85.5} \\
        Deepseek-R1 & \textbf{92.3} \\
        \bottomrule
    \end{tabular}
    \caption{Arena-Hard benchmark results for different models. Performance data for GPT series, LLaMA, and WizardLM comes from \cite{wang2024mixture}, DeepSeek models from their technical reports \cite{guo2025deepseek,liu2024deepseek}, and other models from public leaderboards.}
    \label{tab:arena_hard_results}
\end{table}


\section{Case Study}
\label{sec:case}
We present a case study in this section to analyze how are the different experts and models are selected, and how different experts and models generate responses. For clarity of comparison, we use KABB w/o Deepseek and set the number of selected experts as four. We report the score of their intermediate outputs as well as the final response. Due to the length of the responses, we have selected key fragments for clarity and brevity. To illustrate how the aggregator synthesizes the final output, we highlight similar expressions between the proposed responses and the aggregated response using underlined text in different colors.

\cref{case-good} showcases the responses generated by four selected experts, along with the final aggregated response provided by the aggregator model, Qwen2-72B-Instruct. Two of the experts' responses got a high preference score over 0.99, which demonstrates that MABB succeeded in selecting qualified experts. The aggregated response achieves the highest preference score, reflecting a well-balanced synthesis of key elements from all proposers. The aggregated output successfully combines the most relevant and salient points from all proposed responses, demonstrating the aggregator's ability to synthesize diverse perspectives into a cohesive and comprehensive answer. This process highlights the collaborative nature of the models and their collective contribution to generating high-quality answers.

To be specific, the selected experts—Interaction Analyst, Dialogue Specialist, Humanities Scholar, and Cultural Interpreter—bring distinctive perspectives and areas of specialization, which collectively contribute to the richness and depth of the final aggregated output. The Interaction Analyst ensures factual accuracy and provides foundational details, while the Dialogue Specialist focuses on clarity and narrative flow, making the response accessible to a broad audience. The Humanities Scholar adds historical and cultural context, enriching the response with connections to societal trends, and the Cultural Interpreter offers reflective insights, emphasizing the sociocultural dynamics behind Superman's creation. By combining these complementary perspectives, the aggregator produces a response that balances factual precision, narrative coherence, cultural depth, and interpretive richness. This selection of experts ensures a multidimensional and high-quality final response.

\cref{case-bad} highlights a challenge in incorporating multiple experts for response generation: although diverse perspectives can broaden the scope of the output, they risk diluting the core information with excessive and redundant details. In this case, the inclusion of too many experts led to a loss of focus and reduced the practicality of the final response, despite offering a more expansive view of the topic. The selected experts each contributed their specialized perspectives. However, this diversity introduced significant overlap and irrelevant details. As a result, the aggregated response, though comprehensive, lacked the specificity and clarity needed for practical implementation. This case underscores the importance of carefully curating expert involvement based on the specific requirements of the task. For highly technical prompts, prioritizing experts with deep implementation knowledge and minimizing the number of experts is essential to ensure clarity, focus, and actionable results.

\setlength{\tabcolsep}{5pt} % Adjust column separation
\renewcommand{\arraystretch}{1.5} % Adjust row spacing
% \vspace{1em} % Add some vertical space between the user input and the table
% Centering the table and ensuring it stays on one page
\begin{table}[t]
\centering
\caption{Case: Some models produce high-quality answers on AlpacaEval 2.0}
\label{case-good}
% Add the user input in a tcolorbox for clarity
% \begin{tcolorbox}[colback=blue!5!white, colframe=blue!75!black, title=User Input]
% Who created the Superman cartoon character?
% \end{tcolorbox}
\begin{tabular}{@{}p{17cm}@{}}
\toprule
\textbf{User Prompt}  \\ 
\midrule
Who created the Superman cartoon character?
\end{tabular}
\begin{adjustbox}{max width=\linewidth}
\begin{tabular}{@{}p{2cm}p{4cm}p{2cm}p{8cm}@{}}
\toprule
\textbf{Expert} & \textbf{Model} & \textbf{Preference} & \textbf{Output} \\ 
\midrule
Interaction \newline Analyst & LLaMa-3-70B-Instruct & 0.000112313 & 
Superman, \colorul[blue]{the iconic cartoon character}, was created by \colorul[blue]{writer Jerry Siegel and artist Joe Shuster}. They were both \colorul[blue]{high school students in Cleveland, Ohio,} when they \colorul[blue]{came up with the idea for the character in 1933.}\ldots \\ 
% \midrule
Dialogue \newline Specialist & Qwen2-72B-Instruct & 0.9998820091 & 
\ldots They created Superman in 1933, and the character \colorul[green]{first appeared in print in Action Comics \#1 in June 1938, published by DC Comics.} \ldots \\ 
% \midrule
Humanities \newline Scholar & WizardLM-2-8x22B & 0.9999975912 & 
\ldots Superman quickly became a \colorul[orange]{cultural icon} and is often credited with helping to create the superhero genre as we know it today. \newline The character's success led to various adaptations across multiple media, including radio serials, \colorul[orange]{television shows, movies, and—of course—animated cartoons.} \ldots \\ 
% \midrule
Cultural \newline Interpreter & Gemma-2-27B-it & 0.2356933747 & 
\ldots \colorul[red]{It's a fascinating story} that highlights how cultural context and \colorul[red]{collaboration} shape creative works. \ldots \colorul[red]{If you'd like to} delve deeper into any specific aspect of Superman's creation, such as the social anxieties of the Great Depression reflected in the character, or the evolution of his powers and morality over time, \colorul[red]{feel free to ask!} \\ 
\midrule
Aggregator & Qwen2-72B-Instruct & \textbf{0.9999988203} & 
Superman, \colorul[blue]{the iconic cartoon character}, was created by \colorul[blue]{writer Jerry Siegel and artist Joe Shuster}. The duo, \colorul[blue]{who were high school students in Cleveland, Ohio}, \colorul[blue]{came up with the idea for the character in 1933} \ldots \colorul[green]{Superman first appeared in print in Action Comics \#1 in June 1938, published by National Allied Publications, which later became DC Comics.} \newline Siegel and Shuster's creation \colorul[orange]{quickly became a cultural icon}, leading to various adaptations in \colorul[orange]{comic books, TV shows, movies, and animated cartoons.} \ldots \newline Superman's creation is \colorul[red]{a story of collaboration}, influenced by the cultural context of the 1930s. \ldots \colorul[red]{If you'd like to} explore any specific aspect of Superman's creation or history, \colorul[red]{feel free to ask!}  \\ 
\bottomrule
\end{tabular}
\end{adjustbox}
\end{table}


\setlength{\tabcolsep}{5pt} % Adjust column separation
\renewcommand{\arraystretch}{1.5} % Adjust row spacing
% \vspace{1em} % Add some vertical space between the user input and the table
% Centering the table and ensuring it stays on one page
\begin{table}[t]
\centering
\caption{Case: Some models produce low-quality answers on Arena-Hard}
\label{case-bad}
% Add the user input in a tcolorbox for clarity
% \begin{tcolorbox}[colback=blue!5!white, colframe=blue!75!black, title=User Input]
% Who created the Superman cartoon character?
% \end{tcolorbox}
\begin{tabular}{@{}p{17cm}@{}}
\toprule
\textbf{User Prompt}  \\ 
\midrule
Write a chrome plugin that saves the contents of the current page.
\end{tabular}
\begin{adjustbox}{max width=\linewidth}
\begin{tabular}{@{}p{2cm}p{3.5cm}p{2.5cm}p{8cm}@{}}
\toprule
\textbf{Expert} & \textbf{Model} & \textbf{Score} & \textbf{Output} \\ 
\midrule
Creative \newline Strategist & LLaMa-3-70B-Instruct & \texttt{GPT >> KABB} & 
\ldots AI-powered content analysis: \colorul[blue]{Integrate a machine learning algorithm} that analyzes the saved page's content, providing users with insights, such as \colorul[blue]{keyword extraction, sentiment analysis, or topic modeling.} \ldots \\ 
% \midrule
User \newline Experience \newline Consultant & Qwen2-72B-Instruct & \texttt{GPT >> KABB} & 
\ldots Content Script: \colorul[green]{- **Feedback:** Provide visual feedback }when the user interacts with the plugin\ldots \\ 
% \midrule
Implementation \newline Advisor & Qwen2-72B-Instruct & \texttt{GPT >> KABB} & 
\ldots 4. **\colorul[orange]{Test} and iterate**: Conduct thorough testing to ensure the plugin works as intended, and iterate on the design and functionality \colorul[orange]{based on user feedback.} \newline \ldots \\ 
% \midrule
Technical \newline Architect & WizardLM-2-8x22B & \texttt{GPT < KABB} & ... \colorul[red]{The manifest file should look like this}: \newline \texttt{```}json \newline \texttt{\{} \newline ``manifest\_version'': 3, \newline ``name'': ``Page Saver Plugin'', \newline \ldots \\
\midrule
Aggregator & Qwen2-72B-Instruct & \texttt{GPT > KABB} & \ldots 3. **Define Plugin Permissions:** \newline \colorul[red]{- Update \texttt{`}manifest.json\texttt{`} with plugin metadata, permissions (e.g., ``activeTab'', ``downloads''),} and content scripts. \newline \ldots \colorul[orange]{- Test the plugin in various scenarios }to ensure reliability. \newline \ldots  - Implement \colorul[green]{visual feedback}, progress indicators, and error handling. \newline \ldots \colorul[blue]{ Incorporate an AI-powered content analysis} for insights or a collaboration feature for annotations. \newline \ldots - Regularly update the plugin with bug fixes and feature enhancements \colorul[orange]{based on user feedback}\ldots \\ 
\midrule
Reference & GPT-4 (03/14) & N/A & \ldots 2. Inside the ``SavePageContent'' folder, create a new file called ``manifest.json'' and add the following code: \newline \texttt{```}json \newline \texttt{\{} \newline 
    ``manifest\_version'': 2, \newline ``name'': ``Save Page Content'', \newline \ldots \\ 
\midrule

Judgment & GPT-4 Preview (11/06) & N/A & \ldots GPT's answer is slightly better because it provides actionable code snippets and a clear example that users can follow to create the plugin. However, KABB's answer is also of high quality, offering a broader overview of the process and additional creative suggestions.\ldots 
\end{tabular}
\end{adjustbox}
\end{table}

\section{Additional Experimental Settings}

\textbf{Resources.} All experiments on KABB are conducted on servers with one NVIDIA GeForce RTX 3090.

\subsection{Prompts for Experts and the Aggregator}

In this section, we provide some cases of prompts for different experts and the aggregator to show an example of the system configuration.

\begin{tcolorbox}[colback=lightgray!5!white, colframe=lightgray!75!black, title=Analysis Expert]
You are an expert in problem analysis and logical reasoning, skilled in applying analytical frameworks and systematic thinking approaches. \newline Your expertise includes breaking down complex problems, identifying key factors, and recommending structured, actionable solutions. \newline You are familiar with various problem-solving methods such as root cause analysis, decision matrices, and scenario evaluation, and adapt your approach based on the unique context of each task. \newline Consider how your skills in critical thinking, structured reasoning, and analytical problem-solving might provide valuable insights or strategies for addressing the task at hand.
\end{tcolorbox}

\begin{tcolorbox}[colback=teal!5!white, colframe=teal!75!black, title=Strategy Expert]
You are a business strategy expert with a deep understanding of markets, business models, competitive landscapes, and strategic planning. \newline Your expertise includes applying business frameworks, analytical tools, and market insights to identify opportunities and craft strategies. \newline While capable of providing comprehensive strategic analysis, you adapt your input to focus on what is most valuable, practical, and relevant for the situation. \newline Consider how your expertise in business innovation, competitive advantage, and strategic problem-solving might provide insightful and actionable recommendations for any task.
\end{tcolorbox}

\begin{tcolorbox}[colback=olive!5!white, colframe=olive!75!black, title=Aggregator]
You are the Wise Integrator in a multi-agent system tasked with delivering accurate, coherent, and actionable responses to user queries. \newline Your role is to: \newline - Understand the user's intent and main question(s) by carefully reviewing their query. \newline - Evaluate expert inputs, preserving their quality opinions while ensuring relevance, accuracy, and alignment with the user's needs. \newline - Resolve any contradictions or gaps logically, combining expert insights into a single, unified response. \newline - Synthesize the most appropriate information into a clear, actionable, and user-friendly answer. \newline - Add your own insight if needed to enhance the final output. \newline Your response must prioritize clarity, accuracy, and usefulness, ensuring it directly addresses the user's needs while retaining the value of expert contributions. \newline Avoid referencing the integration process or individual experts.
\end{tcolorbox}

\section{Supplementary Proofs and Theoretical Analysis}

% \subsection{Proofs and Theoretical Analysis}

To better illustrate the theoretical derivations and implementation details regarding the Knowledge-Aware Bayesian Bandit (KABB) model in \cref{sec:method}, we provide the following supplementary proofs and theoretical analysis.
% Key definitions and analytical components are organized as follows:

\subsection{Proof of Pseudo-Metric Properties of Knowledge Distance Theorem}
We provide proofs of Pseudo-Metric Properties of Knowledge Distance Theorem \cref{the:Pseudo-Metric} which enhances the reliability and effectiveness of the model in expert selection and task allocation.

\begin{proof}\renewcommand{\qedsymbol}{}
This follows directly from the non-negativity of $\log(1 + d_t)$ and all other terms in the definition of $\text{Dist}(\mathcal{S}, t)$. Each term (e.g., $1 - \rho_{\text{overlap}}$, dependency complexity, etc.) is non-negative by construction.
\end{proof}

\textbf{Proof of Conditional Symmetry}:  
If the dependency graph $G$ is undirected and $\rho_{\text{overlap}}(\mathcal{S}_1, t) = \rho_{\text{overlap}}(\mathcal{S}_2, t)$, and if $\mathcal{S}_1$ and $\mathcal{S}_2$ are symmetric in terms of knowledge and dependencies, then all terms in the distance function (e.g., $|\mathcal{R}_{\text{dep}}|$, $\bar{H}_{\mathcal{S}}$, and weights) are equal for $\mathcal{S}_1$ and $\mathcal{S}_2$. Thus, $\text{Dist}(\mathcal{S}_1, t) = \text{Dist}(\mathcal{S}_2, t)$.

\textbf{Proof of Approximate Triangle Inequality}:  
Using the properties of the knowledge graph as a metric space, the subadditivity of the graph metric ensures that the dependency-based terms satisfy a triangle inequality. Similarly, the Jaccard similarity is used in \cref{jaccard subadd}. Combining these with the weight terms, the inequality holds with a relaxation factor $c \geq 1$ determined by the extrema of the weights.

\subsection{Proof Sketch of Convergence Analysis for the Dynamic Selection Strategy}

The proof of convergence is outlined as follows:

\begin{enumerate}
    \item \textbf{Stability of Beta Distribution Parameters}: Analyze the stability of the Beta distribution parameter evolution by leveraging KL divergence to quantify changes over time.
    \item \textbf{Lyapunov Function Construction}: Construct a Lyapunov function 
    \[
    V(t) = \sum_{\mathcal{S}} \big[(\alpha_{\mathcal{S}}^{(t)} - \alpha_{\mathcal{S}^*}^{(t)})^2 + (\beta_{\mathcal{S}}^{(t)} - \beta_{\mathcal{S}^*}^{(t)})^2\big],
    \]
    and use it to demonstrate the convergence of the parameters.
    \item \textbf{Cumulative Regret Analysis}: Establish an upper bound for cumulative regret by applying UCB (Upper Confidence Bound) principles.
\end{enumerate}

\subsection{The Strict Proof of the Approximate Triangle Inequality for Theorem 2}

\paragraph{Step 1: Decomposition of Knowledge Distance Function and Subterm Analysis}~{}
\newline
For any expert teams $\mathcal{S}_1, \mathcal{S}_2, \mathcal{S}_3$ and task $t$, there exists a constant $\epsilon > 0$, such that the knowledge distance function satisfies:  
\[
\text{Dist}(\mathcal{S}, t) = \log(1 + d_t) \cdot \sum_{i=1}^4 \omega_i \Psi_i
\]
where $\Psi_i$ corresponds to the four subterms that key the multi-dimensional distance measurement between the expert team and the task:  
\[
\Psi_1 = 1 - \rho_{\text{overlap}}(\mathcal{S}, t) \quad (\text{semantic mismatch term})
\]
\[
\Psi_2 = \frac{|\mathcal{R}_{\text{dep}}(\mathcal{S}, t)|}{K} \quad (\text{dependency complexity term})
\]
\[
\Psi_3 = 1 - \bar{H}_{\mathcal{S}}(t) \quad (\text{historical performance term})
\]
\[
\Psi_4 = 1 - \mathrm{Synergy}(\mathcal{S}) \quad (\text{team complementarity term})
\]
The proof demonstrates that by establishing the approximate sub-additivity of the subterms and combining the logarithmic term properties, the knowledge distance function satisfies the approximate triangle inequality within the error bound $\epsilon = \max{\epsilon_1, \epsilon_2, \epsilon_3, \epsilon_4}$, providing a theoretical guarantee for algorithm design.
 
\paragraph{Step 2: Sub-additivity Analysis of Semantic Mismatch Term (Based on Jaccard Similarity)}~{}
\newline

\begin{definition}[Jaccard Similarity]  
For any sets $\mathcal{S}_1, \mathcal{S}_2$ and task concept set $\mathcal{C}_t$, define:  
\[
\rho_{\text{overlap}}(\mathcal{S}, t) = \frac{|\mathcal{C}_{\mathcal{S}} \cap \mathcal{C}_t|}{|\mathcal{C}_{\mathcal{S}} \cup \mathcal{C}_t|}
\]
\end{definition}

\begin{lemma}[Jaccard Sub-additivity]: 
For any $\mathcal{S}_1, \mathcal{S}_2 \subseteq \mathcal{E}$, there exists a constant $c_1 \geq 1$ such that:  
\[
1 - \rho_{\text{overlap}}(\mathcal{S}_1 \cup \mathcal{S}_2, t) \leq c_1 \left[ \left(1 - \rho_{\text{overlap}}(\mathcal{S}_1, t)\right) + \left(1 - \rho_{\text{overlap}}(\mathcal{S}_2, t)\right) \right]
\]
\label{jaccard subadd}
\end{lemma}

\begin{proof}  
By the properties of set operations:  
\[
|\mathcal{C}_{\mathcal{S}_1 \cup \mathcal{S}_2} \cap \mathcal{C}_t| \geq |\mathcal{C}_{\mathcal{S}_1} \cap \mathcal{C}_t| + |\mathcal{C}_{\mathcal{S}_2} \cap \mathcal{C}_t| - |\mathcal{C}_{\mathcal{S}_1} \cap \mathcal{C}_{\mathcal{S}_2} \cap \mathcal{C}_t|
\]
\[
|\mathcal{C}_{\mathcal{S}_1 \cup \mathcal{S}_2} \cup \mathcal{C}_t| \leq |\mathcal{C}_{\mathcal{S}_1} \cup \mathcal{C}_t| + |\mathcal{C}_{\mathcal{S}_2} \cup \mathcal{C}_t|
\]
Let $A = \mathcal{C}_{\mathcal{S}_1} \cap \mathcal{C}_t$, $B = \mathcal{C}_{\mathcal{S}_2} \cap \mathcal{C}_t$, we get:  
\[
\rho_{\text{overlap}}(\mathcal{S}_1 \cup \mathcal{S}_2, t) \geq \frac{|A| + |B| - |A \cap B|}{|\mathcal{C}_{\mathcal{S}_1} \cup \mathcal{C}_t| + |\mathcal{C}_{\mathcal{S}_2} \cup \mathcal{C}_t|}
\]
By relaxing the denominator to $2 \cdot \max(|\mathcal{C}_{\mathcal{S}_1} \cup \mathcal{C}_t|, |\mathcal{C}_{\mathcal{S}_2} \cup \mathcal{C}_t|)$, we get $c_1 = 2$.
\end{proof}

\begin{corollary}[]  
$\Psi_1(\mathcal{S}_1 \cup \mathcal{S}_2, t) \leq 2 \left[ \Psi_1(\mathcal{S}_1, t) + \Psi_1(\mathcal{S}_2, t) \right]$ 
\end{corollary}
This conclusion allows us to effectively estimate and control the semantic differences between expert teams using a simple additive form.


\paragraph{Step 3: Sub-additivity of Dependency Complexity Term in Graph Metrics}~{}
\newline
\textbf{Definition (Dependency Edge Path Length)}:  
The number of dependency edges $|\mathcal{R}_{\text{dep}}(\mathcal{S}, t)|$ in the knowledge graph satisfies the triangle inequality in graph metrics:  
\[
|\mathcal{R}_{\text{dep}}(\mathcal{S}_1, t)| \leq |\mathcal{R}_{\text{dep}}(\mathcal{S}_1, \mathcal{S}_2)| + |\mathcal{R}_{\text{dep}}(\mathcal{S}_2, t)|
\]
where $|\mathcal{R}_{\text{dep}}(\mathcal{S}_1, \mathcal{S}_2)|$ is the number of shortest path edges connecting $\mathcal{S}_1$ and $\mathcal{S}_2$.

\begin{lemma}[Existence of Relaxation Factor]:  
For any acyclic graph, there exists a constant $c_2 \geq 1$ such that:  
\[
|\mathcal{R}_{\text{dep}}(\mathcal{S}_1, t)| \leq c_2 \left[ |\mathcal{R}_{\text{dep}}(\mathcal{S}_1, \mathcal{S}_2)| + |\mathcal{R}_{\text{dep}}(\mathcal{S}_2, t)| \right]
\]
\end{lemma}

\begin{proof}
By graph diameter constraints, set $c_2 = \text{diam}(G)$ (the diameter of the graph), which is the longest path in terms of edges between any two nodes.  
The dependency complexity term establishes sub-additivity through the following reasoning: based on graph metric properties, path lengths satisfy the triangle inequality; by the graph's diameter constraints, we obtain an upper bound for the relaxation factor; and by normalization, the boundedness of dependency complexity is guaranteed. This property provides a quantifiable theoretical foundation for evaluating team knowledge structures.
\end{proof}

\paragraph{Step 4: Approximate Linearity of Team Complementarity Term}~{}
\newline
\textbf{Definition (Complementarity Decomposition)}:  
The team complementarity $\mathrm{Synergy}(\mathcal{S})$ satisfies:  
\[
\mathrm{Synergy}(\mathcal{S}_1 \cup \mathcal{S}_2) \geq \mathrm{Synergy}(\mathcal{S}_1) + \mathrm{Synergy}(\mathcal{S}_2) - \mathrm{Overlap}(\mathcal{S}_1, \mathcal{S}_2)
\]
where $\mathrm{Overlap}$ is the complementarity loss due to knowledge overlap between teams.

\begin{lemma}[Upper Bound of Relaxation]  
There exists a constant $c_3 \geq 1$ such that:  
\[
1 - \mathrm{Synergy}(\mathcal{S}_1 \cup \mathcal{S}_2) \leq c_3 \left[ \left(1 - \mathrm{Synergy}(\mathcal{S}_1)\right) + \left(1 - \mathrm{Synergy}(\mathcal{S}_2)\right) \right]
\]
\begin{proof}\renewcommand{\qedsymbol}{}  
Let $\mathrm{Overlap}(\mathcal{S}_1, \mathcal{S}_2) \leq \min(\mathrm{Synergy}(\mathcal{S}_1), \mathrm{Synergy}(\mathcal{S}_2))$, set $c_3 = 2$.
\end{proof}
\end{lemma}

The construction of the global constant for the knowledge distance: The overall approximate sub-additivity of the subterms in the knowledge distance function is determined by the set of relaxation factors: semantic mismatch term $c_1 = 2$, dependency complexity term $c_2 = \text{diam}(G)$, team complementarity term $c_3 = 2$, and historical performance term $c_4 = 1$. By using these local relaxation factors, combined with the weights and the logarithmic term of task difficulty, a global constant $c = \max c_i \cdot \omega_i \cdot \log(1 + \overline{D}_{\max})$ is constructed. This construction ensures that the overall knowledge distance function satisfies the approximate triangle inequality, providing a theoretical guarantee for the quantitative evaluation of knowledge distance.

\subsection{Theorem 1 Proof: Lower Bound of Expert-Task Mutual Information under Semantic Gap}

\paragraph{Basic Definitions of Dynamic Multi-Agent Systems}~{}
\newline
In dynamic multi-agent systems, the interaction between the expert set $\mathcal{E}$ and the task demand space $\mathcal{T}$ is based on three core assumptions: the Markovian evolution of task demands over time, the conditional independent decomposition of expert selection and tasks, and the decaying mutual information metric with the introduction of a discount factor $\gamma$. This framework is described by the joint distribution 
\[
p(\mathbf{e}, \mathbf{t}_{1:T}) = p(\mathbf{e}) \prod_{t=1}^T p(\mathbf{t}_t | \mathbf{t}_{t-1}) p(\mathbf{e} | \mathbf{t}_t),
\]
which characterizes the dynamic relationship between expert knowledge and task demands, providing a theoretical foundation for the subsequent analysis.

\paragraph{Step 2: Time Accumulation Form of Conditional Entropy}~{}
\newline

The accumulated conditional entropy of expert selection over an infinite time horizon is given by:
\[
H(\mathcal{E} | \mathcal{T}_{1:\infty}) = \lim_{T \to \infty} \frac{1}{T} \sum_{t=1}^T H(\mathcal{E} | \mathcal{T}_t).
\]

After introducing the discount factor $\gamma$, the weighted conditional entropy is:
\[
\widetilde{H}(\mathcal{E} | \mathcal{T}) \triangleq \sum_{t=1}^\infty \gamma^{t-1} H(\mathcal{E} | \mathcal{T}_t).
\]

\paragraph{Step 3: Extension of Fano's Inequality}~{}
\newline

For each time step $t$, apply the classical \textbf{Fano's Inequality}:
\[
H(\mathcal{E} | \mathcal{T}_t) \geq H(\mathcal{E}) - I(\mathcal{E}; \mathcal{T}_t) - h_2(P_e^{(t)}),
\]
where $h_2(x) = -x \log x - (1-x) \log (1-x)$ is the binary entropy function, and $P_e^{(t)} = \mathbb{P}(\hat{\mathcal{E}}_t \neq \mathcal{E} | \mathcal{T}_t)$ is the expert selection error rate at time $t$. When there is no prior knowledge (i.e., $I(\mathcal{E}; \mathcal{T}_t) = 0$), we have:
\[
H(\mathcal{E} | \mathcal{T}_t) \geq \log K - h_2(P_e^{(t)}).
\]

\paragraph{Step 4: Weighted Summation and Asymptotic Analysis}~{}
\newline

Substitute Fano's inequality for the weighted conditional entropy:
\[
\begin{aligned}
\widetilde{H}(\mathcal{E} | \mathcal{T}) &= \sum_{t=1}^\infty \gamma^{t-1} H(\mathcal{E} | \mathcal{T}_t) \\
&\geq \sum_{t=1}^\infty \gamma^{t-1} \left[ \log K - I(\mathcal{E}; \mathcal{T}_t) - h_2(P_e^{(t)}) \right] \\
&= \frac{\log K}{1-\gamma} - \widetilde{I}(\mathcal{E}; \mathcal{T}) - \sum_{t=1}^\infty \gamma^{t-1} h_2(P_e^{(t)}).
\end{aligned}
\]

Under the assumption of long-term stability of the dynamic system ($\lim_{t \to \infty} P_e^{(t)} = 0$), the asymptotic behavior of the error entropy is analyzed. By the convergence of the geometric series sum, it is shown that the weighted error entropy term $\sum_{t=1}^T \gamma^{t-1} h_2(P_e^{(t)})$ vanishes in the limit. This result simplifies the lower bound of conditional entropy to the form of the difference between the entropy of the expert set and the mutual information: 
\[
\widetilde{H}(\mathcal{E} | \mathcal{T}) \geq \frac{\log K}{1-\gamma} - \widetilde{I}(\mathcal{E}; \mathcal{T}),
\]
which provides a more concise theoretical expression for system performance evaluation.

\paragraph{Step 5: Equivalent Form and Semantic Gap Explanation}~{}
\newline

Multiplying both sides of the inequality by $(1-\gamma)$ yields the final form:
\[
\underbrace{H(\mathcal{E} | \mathcal{T})}_{\substack{\text{Conditional Entropy} \\ \text{(Semantic Uncertainty)}}} \geq \log K - \frac{\widetilde{I}(\mathcal{E}; \mathcal{T})}{1-\gamma}.
\]

\textbf{Semantic Gap Limit}: As $\widetilde{I}(\mathcal{E}; \mathcal{T}) \to 0^+$ (when there is no semantic connection between experts and tasks), the lower bound of conditional entropy approaches $\log K$, corresponding to the maximum entropy of completely random selection. 

\textbf{Exploration Efficiency Bottleneck}: The inequality shows that the exploration efficiency of traditional MAB (multi-armed bandit) is limited by $\frac{\widetilde{I}(\mathcal{E}; \mathcal{T})}{1-\gamma}$. When the semantic connection weakens ($\widetilde{I} \downarrow$) or task dynamics increase ($\gamma \uparrow$), the exploration cost increases dramatically.

\subsection{Proof of Knowledge-Driven Information Gain Theorem}

\paragraph{1. Baseline Mutual Information Analysis}~{}
\newline

First, establish the baseline mutual information $I_0 = I(\mathcal{E}; \mathcal{T})$ when there is no knowledge graph, which only depends on the direct association between experts and tasks.

\textbf{2. Effect of Knowledge Graph Intervention}: After introducing the knowledge graph $\mathcal{G}$, the task generation process is reconstructed via the intermediary pattern of the knowledge graph: 
\[
p(\mathcal{T} | \mathcal{E}) = \sum_{\mathcal{G}} p(\mathcal{T} | \mathcal{G}) p(\mathcal{G} | \mathcal{E}).
\]

\textbf{3. Mutual Information Gain Decomposition}: Using the chain rule, the total mutual information introduced by the knowledge graph can be decomposed into: the original expert-task mutual information $I(\mathcal{E}; \mathcal{T})$ and the conditional mutual information contribution from the concept layer $I(\mathcal{C}; \mathcal{T} | \mathcal{E})$. Since $\mathcal{G}$ is fully determined by $\mathcal{E}$ and $\mathcal{C}$, the information gain $\Delta I$ equals the conditional mutual information contribution from the concept layer, verifying that the knowledge graph improves the system's informational efficiency through the concept layer.

\subsection{Derivation of the Concept Layer Information Gain Bound}

\paragraph{Core Condition Analysis}~{}
\newline

Based on the two key properties of the knowledge graph: sparsity: the upper bound of the expert-concept association degree $d = O(\sqrt{|\mathcal{C}|})$ and balance: the minimum expert coverage of a concept $ \lfloor |\mathcal{E}| / |\mathcal{C}| \rfloor$.

\textbf{2. Information Theoretic Derivation Process}

Through the Markov chain $\mathcal{T} \to \mathcal{C} \to \mathcal{E}$ analysis:
\textbf{Conditional Entropy Relation}: $H(\mathcal{T} | \mathcal{E}) \geq H(\mathcal{T} | \mathcal{C})$ (data processing inequality), $H(\mathcal{T} | \mathcal{C}) = O(\log|\mathcal{C}|)$ (task sparsity).
\textbf{Mutual Information Lower Bound}: Using the definition of conditional mutual information and the relationship with entropy, along with graph structure constraints, the lower bound is obtained:
\[
I(\mathcal{C}; \mathcal{T} | \mathcal{E}) \geq \Omega\left( \frac{\log|\mathcal{C}|}{\sqrt{|\mathcal{E}|}} \right).
\]
This result quantifies the minimum information gain brought by the knowledge graph through the concept layer.

\paragraph{Step 2: Mathematical Representation of Accelerated Exploration Efficiency}~{}
\newline

\textbf{(Upper Bound of Exploration Trials)}:  
In the contextual Bandit framework, the expected number of exploration trials satisfies:
\[
\mathbb{E}[N_{\text{explore}}] = \widetilde{O}\left( \sqrt{\frac{K \log|\mathcal{C}|}{\Delta I}} \right),
\]
where $K = |\mathcal{E}|$, and $\Delta I = \Omega\left( \frac{\log|\mathcal{C}|}{\sqrt{|\mathcal{E}|}} \right)$.

\begin{proof}\renewcommand{\qedsymbol}{}
1. \textbf{Classical Bandit Exploration Complexity}:  
Without a knowledge graph, the exploration trials of a traditional MAB are:
\[
\mathbb{E}[N_{\text{explore}}] = O\left( \frac{K \log T}{\epsilon^2} \right),
\]
where $\epsilon$ is the expected reward gap between the optimal and suboptimal arms.

2. \textbf{Knowledge-Driven Acceleration Mechanism}:  
After introducing the knowledge graph, the reward gap $\epsilon$ is amplified by the information gain $\Delta I$:
\[
\epsilon_{\text{new}} = \epsilon \cdot \sqrt{\Delta I}.
\]
Substituting into the classical complexity formula:
\[
\mathbb{E}[N_{\text{explore}}] = O\left( \frac{K \log T}{\epsilon_{\text{new}}^2} \right) = O\left( \frac{K \log T}{\epsilon^2 \Delta I} \right).
\]
Combining with $\Delta I = \Omega\left( \frac{\log|\mathcal{C}|}{\sqrt{|\mathcal{E}|}} \right)$, and assuming $\epsilon = \Theta(1/\sqrt{K})$ (uniform exploration hypothesis), we obtain:
\[
\mathbb{E}[N_{\text{explore}}] = \widetilde{O}\left( \sqrt{\frac{K \log|\mathcal{C}|}{\Delta I}} \right).
\]
\end{proof}

\subsection{Summary of the Information Gain Theorem Proof}

By introducing a structured knowledge graph through the concept layer $\mathcal{C}$, the conditional mutual information $I(\mathcal{C}; \mathcal{T} | \mathcal{E})$ provides the lower bound of the information gain $\Delta I = \Omega\left( \frac{\log|\mathcal{C}|}{\sqrt{|\mathcal{E}|}} \right)$, which reduces the exploration complexity from the traditional method of $O(K)$ to $\widetilde{O}\left( \sqrt{K \log|\mathcal{C}|} \right)$. This theoretical result rigorously verifies the acceleration advantage of knowledge-driven decision-making.


\subsection{Regret Upper Bound Derivation for Knowledge-Aware UCB (KABB)}
\label{sec:supp-regret}

\paragraph{Problem Framework} \cref{sec:system_arch} are extended with complete mathematical specifications of expert set $\mathcal{E}$ and task sequence $\{T_t\}_{t=1}^T$:
\begin{itemize}
\item Selection process: $\mathcal{S}_t \subseteq \mathcal{E}$ at each step
\item Feedback mechanism: Obtain $\theta_{\mathcal{S}_t}^{(t)}$
\item Success probability: 
Including knowledge distance \(\mathrm{Dist}(\mathcal{S}, t)\), time decay \(\gamma^{\Delta t}\), and team synergy \(\mathrm{Synergy}(\mathcal{S})\).


\begin{equation}
\tilde{\theta}_{\mathcal{S}}^{(t)} = \underbrace{\mathbb{E}\left[\theta_{\mathcal{S}}^{(t)}\right]}_{\text{Historical expectation}} \cdot \exp\left(-\lambda \cdot \text{Dist}(\mathcal{S}, t)\right) \cdot \gamma^{\Delta t} \cdot \mathrm{Synergy}(\mathcal{S})^\eta
\end{equation}
\end{itemize}

\paragraph{Confidence Bound Construction}

This section elaborates on the construction method of confidence bounds in the KABB algorithm, the definition of knowledge revision rewards, and their impact on exploration weights. It supports the theoretical analysis in \cref{sec:knowledge_distance} regarding the limitations of traditional methods and the breakthroughs in knowledge-driven decision-making. The confidence-bound construction extends traditional UCB through knowledge-aware reward correction:

\begin{equation}
\text{UCB}_{\mathcal{S}}^{(t)} = \underbrace{\hat{\mu}_{\mathcal{S}}^{(t)}}_{\text{Empirical mean}} + \underbrace{\sqrt{\frac{2 \log t}{N_{\mathcal{S}}^{(t)}}}}_{\text{Exploration term}} \cdot \underbrace{\exp\left(-\lambda \cdot \text{Dist}(\mathcal{S}, t)\right) \cdot \gamma^{\Delta t} \cdot \mathrm{Synergy}(\mathcal{S})^\eta}_{\text{Knowledge-driven correction}}
\end{equation}

where $\hat{\mu}_{\mathcal{S}}^{(t)} = \frac{\alpha_{\mathcal{S}}^{(t)}}{\alpha_{\mathcal{S}}^{(t)} + \beta_{\mathcal{S}}^{(t)}}$ denotes the Bayesian estimate of historical success rate. The correction term adjusts the exploration weights through knowledge distance, time decay, and synergy effects.


\subsection{Regret Upper Bound Analysis}
\label{sec:supp-regret-analysis}

\paragraph{Total Regret Definition }
\label{subsec:regret_analysis}

\cref{sec:supp-regret} provides a detailed analysis of the total regret decomposition and single-step regret properties for the KABB algorithm, corresponding to the analysis in \label{sec:dynamic_bayesian} regarding the impact of team knowledge distance and complementarity on algorithmic performance. The theoretical proofs and mathematical derivations are presented as follows:

The total regret is defined as:
\begin{equation}
\label{eq:total_regret}
R(T) = \sum_{t=1}^{T} \left( \theta_{S^*}^{(t)} - \theta_{S_t}^{(t)} \right)
\end{equation}
where $S^*$ denotes the optimal expert subset and $S_t$ represents the selected subset at time step $t$, the analysis should follow these steps:

\begin{enumerate}
    \item \textbf{Characterize Single-Step Regret}: First define the single-step regret:
    \begin{equation}
    \label{eq:instant_regret}
    r_t = \theta_{S^*}^{(t)} - \theta_{S_t}^{(t)}
    \end{equation}
    and analyze its properties.

    \item \textbf{Analyze Regret Bound for Suboptimal Subsets}: For any suboptimal subset $S \neq S^*$, establish the upper bound of single-step regret.

    \item \textbf{Compose Total Regret Upper Bound}: Investigate how to combine single-step regrets into the total regret upper bound.
\end{enumerate}

% \begin{lemma}
% The regret analysis requires:
% \begin{itemize}
%     \item Precise decomposition of $r_t$ using knowledge distance metrics
%     \item Bounding techniques for suboptimal selections $S \neq S^*$
%     \item Cumulative error control through martingale analysis
% \end{itemize}
% \end{lemma}

\paragraph{Step 1: Per-Step Regret Decomposition}

For any suboptimal subset $\mathcal{S} \neq \mathcal{S}^*$, the instantaneous regret satisfies:

\begin{equation}
\Delta_{\mathcal{S}}^{(t)} \leq \underbrace{\left| \hat{\mu}_{\mathcal{S}^*}^{(t)} - \theta_{\mathcal{S}^*}^{(t)} \right|}_{\text{Optimal set error}} + \underbrace{\left| \hat{\mu}_{\mathcal{S}}^{(t)} - \theta_{\mathcal{S}}^{(t)} \right|}_{\text{Suboptimal set error}} + \underbrace{\text{Dist}(\mathcal{S}, t) \cdot \lambda}_{\text{Knowledge penalty}}
\end{equation}

\paragraph{Step 2: Exploration Acceleration Effect}

\begin{lemma}[Exploration Count Upper Bound]
For any suboptimal $\mathcal{S}$, its selection count satisfies:
\begin{equation}
\mathbb{E}\left[N_{\mathcal{S}}(T)\right] \leq \frac{8 \log T}{(\Delta_{\mathcal{S}} \cdot \exp(-\lambda \overline{D}_{\mathcal{S}}))^2} + O\left(\sqrt{T \log T}\right)
\end{equation}
where $\overline{D}_{\mathcal{S}} = \max_t \text{Dist}(\mathcal{S}, t)$ and $\Delta_{\mathcal{S}} = \theta_{\mathcal{S}^*} - \theta_{\mathcal{S}}$.
\end{lemma}

\begin{proof}\renewcommand{\qedsymbol}{}
The knowledge correction term $\exp(-\lambda \overline{D}_{\mathcal{S}})$ amplifies the reward gap $\Delta_{\mathcal{S}}$, thereby reducing the exploration demand for suboptimal subsets. The estimation error is bounded via the Chernoff-Hoeffding inequality, combined with the exponential decay modification of exploration terms through knowledge distance. This upper bound formula reflects three key factors influencing regret:

\begin{itemize}
\item \textbf{Optimality gap term} $\Delta_{\mathcal{S}}$: The term in the denominator represents the performance gap between suboptimal and optimal subsets. A larger gap leads to a smaller regret upper bound.
  
\item \textbf{Knowledge distance penalty} $\exp(-2\lambda \overline{D}_{\mathcal{S}})$: The exponential term in the denominator reflects the impact of the knowledge graph. Larger $\overline{D}_{\mathcal{S}}$ (i.e., greater knowledge discrepancy) increases the regret upper bound.
  
\item \textbf{Combinatorial complexity term} $O(\sqrt{T \log T} \cdot \tbinom{K}{k})$: Captures the combinatorial optimization nature of the problem, where:
  \begin{itemize}
  \item $\sqrt{T \log T}$ corresponds to the standard UCB term
  \item $\tbinom{K}{k}$ represents the combinatorial complexity from selecting $k$ experts out of $K$
  \end{itemize}
\end{itemize}

This demonstrates that the regret upper bound is jointly determined by the knowledge structure (via $\overline{D}_{\mathcal{S}}$) and combinatorial optimization complexity.
\end{proof}

\subsection{Core Differences from Classical UCB}

\begin{table}[H]
\centering
\caption{Comparison between Classical UCB and KABB}
\label{tab:ucb-comparison}
\begin{tabular}{lll}
\toprule
\textbf{Dimension} & \textbf{Classical UCB} & \textbf{Knowledge-Aware UCB (KABB)} \\
\hline
Exploration Design & $\sqrt{\log t/N}$ & Multiplicative knowledge correction \\
Regret Dominant Term & $O(\sqrt{KT \log T})$ & $O(\sqrt{T \log T} \cdot \tbinom{K}{k})$ \\
Theoretical Innovation & No structured prior & Knowledge graph integration \\
Key Assumption & IID rewards & Non-stationary rewards with synergy \\
\bottomrule
\end{tabular}
\end{table}
The knowledge-aware UCB improves the traditional $O(KT \log T)$ regret bound of UCB through structured prior injection and a dynamic correction mechanism, transforming it into an exponentially compressed form in the combinatorial space. The core innovation lies in the quantitative modeling of knowledge distance and synergy effects. This theorem represents the first strict integration of knowledge graphs and team collaboration theory within the Bandit framework.
\subsection{Regret Bound and Exploration Efficiency Analysis}

This section provides a detailed description of the core modules of the KABB algorithm, offering an in-depth analysis of its cumulative regret bound and the relationship with exploration efficiency, supporting the algorithm derivation and convergence analysis in \cref{sec:dynamic_bayesian}. Additionally, \cref{sec:algorithm} elaborates on the specific implementation modules of the KABB algorithm, along with its time and space complexities, providing empirical foundations and optimization strategies for the algorithm design and performance evaluation in the main text.

\begin{theorem}[The cumulative regret $R(T)$ of KABB]
$$
R(T) \leq \underbrace{\sum_{\mathcal{S} \neq \mathcal{S}^*} \frac{4\underline{L}^2 \log T}{\widetilde{\Delta}_{\mathcal{S}}}}_{\text{Knowledge-driven exploration term}} + \underbrace{O\left( \sqrt{T \tbinom{N}{k} \log \tbinom{N}{k}} \right)}_{\substack{\text{Additional complexity term}\\ \text{due to team size}}}, \quad \text{where} \quad 
\boxed{
\begin{cases}  
L = \log(1 + \overline{D}_{\max}) \cdot (\omega_1 \!+\! \omega_2 \!+\! \omega_3 \!+\! \omega_4) \\
\widetilde{\Delta}_{\mathcal{S}} = \mu_{\mathcal{S}^*} \!-\! \mu_{\mathcal{S}} \\
\overline{D}_{\max} = \max\limits_{\mathcal{S}, t} \mathrm{Dist}(\mathcal{S}, t) \\
k = |\mathcal{S}^*|
\end{cases}
}
$$
\end{theorem}

\textbf{Explanation}:  
The cumulative regret $R(T)$ measures the performance loss caused by not selecting the optimal team $\mathcal{S}^*$ over $T$ time steps:

\textbf{Knowledge-driven exploration term:} The exploration count is constrained by the knowledge distance. Its dominant term $\frac{4\underline{L}^2 \log T}{\widetilde{\Delta}_{\mathcal{S}}}$ shows that: 1) when the knowledge distance difference is significant (i.e., $\overline{D}_{\max} \uparrow$), the algorithm quickly focuses on high-quality teams through the $\exp(-\lambda \mathrm{Dist}(\cdot))$ mechanism; 2) when the team reward gap $\widetilde{\Delta}_{\mathcal{S}} \downarrow$, the exploration intensity is adaptively adjusted via the $\log(1 + \overline{D}_{\max})$ factor.

\textbf{Team size complexity term:} The complexity term $O\left( \sqrt{T \tbinom{N}{k} \log \tbinom{N}{k}} \right)$ includes the combination number $\tbinom{N}{k}$, and its variation with the expert set size $N$ and optimal team size $k$ follows:
$$
\tbinom{N}{k} \sim \begin{cases}
O(N^k/k!) & \text{when } k \ll N \\
O(2^N/\sqrt{N}) & \text{when } k \approx N/2
\end{cases}
$$
\subsection{Proof Framework}

\paragraph{Step 1: Reward Remapping}
Define dual-modality adjusted reward:
\begin{equation}
\tilde{\mu}_{\mathcal{S}} = \underbrace{\mu_{\mathcal{S}} \exp(-\lambda \mathrm{Dist}(\mathcal{S}, t))}_{\text{Knowledge decay}} \cdot \underbrace{\mathrm{Synergy}(\mathcal{S})^\eta}_{\text{Synergy amplification}}
\end{equation}

The knowledge decay term implements soft filtering through $\exp(-\lambda \cdot)$, while the synergy gain term strengthens the competitive advantage of high-quality teams through the exponent $\eta > 1$.

\paragraph{Step 2: Dynamic Sampling Probability Analysis}

Based on the dual-time-scale update rule:
$$
\begin{cases}
\alpha_{\mathcal{S}}^{(t+1)} = \gamma^{\Delta t} \alpha_{\mathcal{S}}^{(t)} + \underbrace{r_{\mathcal{S}}^{(t)} + \delta \cdot \mathrm{KM}(\mathcal{S}, t)}_{\text{Instant Feedback + Knowledge Memory}} \\
\beta_{\mathcal{S}}^{(t+1)} = \gamma^{\Delta t} \beta_{\mathcal{S}}^{(t)} + \underbrace{(1 - r_{\mathcal{S}}^{(t)}) + \delta \cdot (1 - \mathrm{KM}(\mathcal{S}, t))}_{\text{Negative Feedback + Knowledge Forgetting}}
\end{cases}
$$
We derive the \textbf{exponential convergence upper bound} for the sampling count:
$$
\mathbb{E}[N_{\mathcal{S}}(T)] \leq \frac{4\underline{L}^2 \log T}{\widetilde{\Delta}_{\mathcal{S}}^2} + \underbrace{\frac{2}{\gamma^{\Delta t} (1 - \gamma)} \cdot \mathbb{E}\left[\sum_{\tau=1}^T \mathrm{KM}(\mathcal{S}, \tau)\right]}_{\substack{\text{Knowledge-matching driven}\\\text{accelerated convergence term}}}
$$

\subsection{Algorithm Implementation and Complexity}
\label{sec:algorithm}

\paragraph{Core Implementation Modules}

\begin{itemize}
    \item \textbf{Expert Subset Sampling}:
    \begin{equation}
        \mathcal{S}_t \sim \mathrm{ThompsonSampling}\left( 
            \frac{\alpha_{\mathcal{S}}^{(t)}}{\alpha_{\mathcal{S}}^{(t)}+\beta_{\mathcal{S}}^{(t)}} 
            \cdot \exp(-\lambda \mathrm{Dist}(\mathcal{S},t)) 
            \cdot \mathrm{Synergy}(\mathcal{S})^{\eta} 
        \right)
        \label{eq:thompson}
    \end{equation}
    Optimization implementation: The combinatorial space is compressed from $O(2^N)$ to $O\left(\frac{N^k}{k!}\right)$ through a greedy strategy.

    \item \textbf{Dynamic Parameter Update}:
    \begin{equation}
        \begin{cases}
            \alpha_{\mathcal{S}}^{(t+1)} = \gamma^{\Delta t} \alpha_{\mathcal{S}}^{(t)} + 
            \left[ r_{\mathcal{S}}^{(t)} + \delta \cdot \mathrm{KM}(\mathcal{S},t) \right] 
            \cdot \mathbb{I}_{\{\mathcal{S}=\mathcal{S}_t\}} \\[6pt]
            \beta_{\mathcal{S}}^{(t+1)} = \gamma^{\Delta t} \beta_{\mathcal{S}}^{(t)} + 
            \left[ 1 - r_{\mathcal{S}}^{(t)} + \delta \cdot (1 - \mathrm{KM}(\mathcal{S},t)) \right] 
            \cdot \mathbb{I}_{\{\mathcal{S}=\mathcal{S}_t\}}
        \end{cases}
        \label{eq:param_update}
    \end{equation}
    where $\mathbb{I}$ is the indicator function, enabling sparse updates.

\end{itemize}

\paragraph{Complexity Analysis}

\begin{itemize}
    \item \textbf{Time Complexity}:
    $$
        \begin{aligned}
            \mathcal{T}(N,T) &= \underbrace{O\left( \tbinom{N}{k} \right)}_{\substack{\text{Initialization}\\ \text{(Pre-computation)}}} + T \cdot \Bigg[ \underbrace{O\left( \tbinom{N}{k} \right)}_{\substack{\text{Sampling + Evaluation}\\ \text{(Per step)}}} + \underbrace{O\left( \tbinom{N}{k} \log \tbinom{N}{k} \right)}_{\text{Sorting}} \\
            &\quad + \underbrace{O\left( |\mathcal{C}|^2 \right)}_{\substack{\text{Graph Update}\\ \text{(Dijkstra)}}} \Bigg] \\
            &= \boxed{ \widetilde{O}\left( T \cdot \left( \tbinom{N}{k} \log \tbinom{N}{k} + |\mathcal{C}|^2 \right) \right) }
        \end{aligned}
    $$
    \item \textbf{Space Complexity}:
    $$
        \begin{aligned}
            \mathcal{M}(N) &= \underbrace{O\left( \tbinom{N}{k} \right)}_{\substack{\text{Team Parameters}\\(\alpha,\beta)}} + \underbrace{O\left( |\mathcal{C}|^2 \right)}_{\substack{\text{Knowledge Graph}\\ \text{(Adjacency Matrix)}}} + \underbrace{O\left( W \cdot \tbinom{N}{k} \right)}_{\substack{\text{Sliding Window}\\ \text{(Depth $W$)}}} \\
            &\leq \boxed{ O\left( \tbinom{N}{k} + |\mathcal{C}|^2 \right) } \quad (\text{when } W \ll |\mathcal{C}|)
        \end{aligned}
    $$
\end{itemize}

\paragraph{Storage Optimization}
\begin{itemize}
    \item \textbf{Knowledge Graph Compression}: Adjacency matrix $\rightarrow$ adjacency list, reducing space from $O(|\mathcal{C}|^2)$ to $O(|\mathcal{C}| + |\mathcal{E}|)$.
    \item \textbf{Parameter Sharing}: Share $(\alpha, \beta)$ parameters for teams satisfying $\mathrm{Dist}(\mathcal{S}_i, \mathcal{S}_j) < \epsilon$.
    \item \textbf{Incremental distance updates via streaming updates}: Store only $\Delta \mathrm{Dist}$ instead of the full distance matrix, allowing for more efficient memory usage and reducing computational overhead.
\end{itemize}

\subsection{Summary}
The supplementary proofs, through systematic chapter definitions and key point organization, comprehensively support and extend the discussion of the knowledge-driven Dynamic Bayesian Multi-Armed Bandit (KABB) model presented in \cref{sec:method}. Each supplementary section corresponds to a specific part of the main text, covering critical content such as problem definitions, confidence-bound construction, regret-bound analysis, and algorithm and complexity analysis. These sections provide readers with a comprehensive resource for deeply understanding the theoretical foundations and implementation details of the KABB algorithm.

% \subsection{Prompts and Settings}





\end{document}
