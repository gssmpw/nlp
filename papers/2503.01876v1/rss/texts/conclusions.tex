\section{Conclusions} 
\label{sec:conclusion}

In this work, we propose a novel method that enables a robot to actively and efficiently request HitL assistance during deployment. By using an uncertainty metric based on the denoising process inherent to diffusion policies, our method identifies situations where human intervention is most beneficial, thereby reducing unnecessary monitoring and intervention. Experimental results demonstrate the versatility of our method across various deployment scenarios, improving policy performance and thus adaptability in real-world conditions. This work aims to address one of their key challenges in human-in-the-loop robot deployment: minimizing human labor while maximizing robot autonomy and reliability. Additionally, our approach highlights the potential for using such interaction-driven methods to refine and fine-tune policies through targeted data collection.

For future work, we aim to further automate this process by exploring what types of information most effectively facilitate human-robot communication. Specifically, we will investigate how to design interpretable feedback that allows robots to convey their uncertainty and intent in a manner that is intuitive for human operators. 
Furthermore, we will study advanced control mechanisms that enable humans to seamlessly intervene and guide the robot when necessary. These efforts will help bridge the gap between fully autonomous systems and human-in-the-loop deployment, enabling more efficient and scalable solutions for real-world robotic applications.



