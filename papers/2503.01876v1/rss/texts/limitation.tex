\section{Limitations}
\label{sec:limitation}
Our work investigates methods for seeking human assistance in a non-interruptive manner. We are primarily interested in automatically determining, at deployment time, when such assistance is needed, without overloading the operator with requests. However, we do not tackle other important aspects of a full HitL framework which we consider complementary to our work. 

For example, we do not consider methods for conveying to the operator additional information regarding the sources of the uncertainty identified by the agent. In this work, there is no mechanism for identifying which deployment difficulty is affecting the task. We empirically observe that analyzing uncertainty from different dimensions of the denoising vector can sometimes provide insights into the problem's source. For instance, high uncertainty in position versus rotation may indicate distinct underlying challenges. However, we do not have a formal way of conveying such information to the operator.

Furthermore, we do not focus on the specific teleoperation apparatus available to the operator. In this work, we use keyboard control to produce desired changes in end-effector pose and gripper closing commands. We observe that this method requires high expertise on the part of the operator. Future work can investigate the interplay between the method for deciding when to request assistance and the teleoperation tools available to provide such assistance.



