

\section{The Design of \Tool{}}
\label{sec:tool}


As shown in Figure~\ref{fig:workflow}, \Tool{} is an automated tool 
that takes contract source code and ERC documents as input. 
It outputs the ERC rules violated by the contracts, 
along with details about which lines of the source code violate 
which specific rules.

The components of \Tool{} can be divided into two parts. 
The upper part processes input ERC documents, 
involving extracting ERC rules, 
converting the rules into an intermediate 
representation (IR), and generating constraints 
representing when rules are violated. 
To automate the rule extraction and translation, 
we leverage an LLM due to its strong natural language understanding
capabilities~\cite{yang2024harnessing}. 
As discussed in Section~\ref{sec:study},
ERC rules cover various contact semantics, making manual extraction and translation
a tedious and error-prone process.
Additionally, an automated approach enables 
easier extension of \Tool{} to support more ERCs (See Section~\ref{sec:3525}). 
The lower part analyzes contract files. 
The static analysis component checks if all function 
and event declarations comply with ERC requirements and 
gathers necessary information for symbolic 
execution. The symbolic execution component, using 
the generated constraints, evaluates each 
public function and its callees to determine if any 
inputs could trigger an execution satisfying the constraints 
and reports rule violations if so.
%
This section follows the workflow to detail the key 
components of \Tool{}.




\subsection{ERC Rule Extraction}
\label{sec:extraction}

\Tool{} aims to verify a contract adheres to all 
the rules outlined in the ERC document it implements. Thus, its first step 
is to extract rules from ERC documents. 
While we acknowledge that some correctness and performance 
rules can be identified and extracted from smart 
contract code~\cite{ZepScope,mengting-gas}, 
validating compliance with these rules is beyond the scope of this paper.
Furthermore, as shown by the experimental results in Section~\ref{sec:compare}, the 
rules extracted from contract code cover only a small subset of ERC requirements. 

A naïve way to extract rules from an ERC document 
with an LLM 
is to provide the entire ERC document and ask the model to identify all the rules. 
However, this makes it difficult for the LLM to accurately and completely 
extract all rules. Instead, we break each ERC document into subsections and
send each one to the LLM. 
Specifically, since each ERC’s rules are detailed in the specification section and 
precede the relevant function or event declarations, 
we use regular expressions 
to isolate the specification section and divide the section 
into subsections, ending at each function or event declaration. 
We then instruct the LLM to analyze 
each function or event declaration along 
with the preceding text description.

We design prompts based on the linguistic patterns in Table~\ref{tab:linguistic}. 
Each prompt begins with a brief introduction, followed by a set of specific linguistic patterns (all sharing the same ID prefix). It then provides the text description 
and declaration of a function (or event) and asks the LLM to extract rules 
from the description, including any relevant value requests 
(\eg, emitting an event with a specific parameter), based on the patterns. 
Finally, the prompt explains the JSON schema for outputting the extracted rules.
We include linguistic patterns in the prompts to help the LLM better 
understand what ERC rules are, leading to more accurate extraction results.
%
The prompt template extracting rules in the TP linguistic group is shown in Figure~\ref{fig:prompt1} in the Appendix.
%
As discussed in Section~\ref{sec:patterns}, most rules follow a limited 
number of patterns, making it likely that the patterns will also 
apply to ERC rules we have not yet studied.
%
The extracted rules are indexed with their corresponding 
functions and pattern groups.



%\commentty{Need to justify
%why Linguistic patterns
%are needed in the prompt?
%Why can not rely solely on LLMs to perform rule extraction without using linguistic patterns.} \shihao {
%Extracting rules solely rely on LLMs' default understanding is unstable. The well-known LLM issues like hallucinations and over- or less- deduction can cause many false-positives and false negatives. 

%Tell llm which sentence is rule.
%}


\if 0
\shihao{
Using this approach, \Tool{} successfully extracts 
130 out of 132 rules from the four ERC documents. 
LLM incorrectly extracts 2 rules in ERC1155: "not throw if Caller must be approved to manage the tokens being transferred out of the \_from account". The correct one is "throw if Caller is not approved .."
}

\shihao{Note: 4 FPs are about "not throw if `\_to` is a smart contract (e.g. code size > 0)", I do not include them since we discussed we can filter out them by using simple check}

\fi


\if 0

Using this approach, \Tool{} successfully extracts 
189 out of  190 rules from the four ERC documents. 
For example, 
%beyond the API and return-value requirements, ERC20 specifies four rules for \texttt{transferFrom()}. 
\Tool{} extracts the rule violated 
in Figure~\ref{fig:20-high}, 
which states ``the function SHOULD throw unless the \_from account has deliberately authorized the sender of the message via some mechanism.’’
%
The only missed rule is from ERC1155, requiring a function to be called, 
but the correct name of the function is not accurately extracted. 
%
On the other hand, the LLM incorrectly extracts 13
rules not required by the ERCs (\ie, 13 false positives). 
Six of them require a function not to throw an exception under certain conditions, 
while the remaining seven
requests a function to call another function. 
Given that these false positives closely resemble actual rules 
for the same function, we think they are due to the 
LLM’s hallucinations. For example, instead of calling another function, the real requirement is to emit an event with the same name as the function. 
We manually fix the missed rule
and remove the false positives before 
moving on to the next step.
This manual step is a one-time effort for contracts belonging to the same ERC. 


\fi

%\commentty{Potential over-fitting? The LPs are
%derived from the 190 rules, and the evaluation
%is from the same set of rules. } 
%  \shihao{Can we say the xxx(large percent of) LPs are actually comes from ERC20 and ERC721, it just proves that the ERC's LPs are limited and existing LPs can cover majority of the ERC rules}

%\shihao{Many rules are covered by specific LPs, so is likely new rule can be covered}