\newpage

\section*{Research Ethics}
Both the development and evaluation of \Tool{} strictly follow the established ethical guidelines, as outlined below.

\bolditalicparagraph{Disclosures.}
We collect the analyzed smart contracts from GitHub, etherscan.io~\cite{etherscan}, 
and polygonscan.com~\cite{polygonscan}. 
We obtain the developers' contact information (\eg, GitHub accounts, X.com accounts, emails, bug bounty websites) from GitHub or the contract comments. 
Additionally, we observe that some buggy code shared across contracts is often due to the use of common open-source libraries (\eg, OpenZeppelin).
Based on previous experience in disclosing ERC rule violations, 
developers tend to prioritize addressing those with a high-security impact, 
as these are more likely to result in financial loss. 
Given the limited time before submission and the large number of identified violations, 
we have only disclosed the identified high-security violations to contract developers and library developers. 
We will continue to report other identified violations via the identified contacts. 
We will not release any information about these violations 
until we receive acknowledgment from the developers or 
observe that the issues have been resolved through the deployment of new contracts.


\if 0
\shihao {
 Our data sources are publicly accessible platforms. For the small dataset, we fetch the human-audited reports and code from the Ethereum Commonwealth Security Department's public Github web pages~\cite{humanaudited}. For the large dataset, fetch them from etherscan.io~\cite{etherscan} and polygonscan.com~\cite{polygonscan} publicly. We have reported the violations to smart contract developers or administrators by either X.com private message or custom violation report mechanism(by email or bug-bounty website) found in the code or administrator's web page. We received two responses from the administrator's email and bug-bounty website says they acknowledged the issues but not going to fix them due to violations are not severe enough. We will not disclose the violation details, such as violation type, contract address and entrypoint, until we got acknowledges from the smart contract developers or administrators. By adhering to these practices, we aim to uphold the highest ethical standards in our research and development processes.
}
\fi


\bolditalicparagraph{Experiments with live systems without informed consent.} 
Our research utilizes OpenAI's APIs~\cite{openai-api} to analyze 
ERC documents similarly to chatting with ChatGPT, 
and employs Etherscan's APIs~\cite{etherscan-api} to download contracts’ 
source codes. We strictly follow the API documentation to ensure 
that only valid requests are made. Due to the moderate volume of 
requests and adherence to the services' rate limits, our research does not negatively affect these services or their users.


\bolditalicparagraph{Terms of service.}
Interacting with OpenAI's LLMs through its public APIs is a standard method of using its services. 
Likewise, downloading contract code from websites 
like etherscan.io and polygonscan.com is a common practice among users. 
Additionally, we access these web services with a moderate 
number of requests, ensuring that we remain in compliance with their terms of service~\cite{openai-service-terms, etherscan-api-terms, etherscan-terms}.


\bolditalicparagraph{Deception.}
We do not perform any user study. Our research involves only the paper authors who
are fully aware of all the details of the research. Thus,
deception is not relevant to us. 


\bolditalicparagraph{Wellbeing for team members.}
The project involves ERCs, smart contracts, ERC rule violations, and a technique to detect the violations. Since these are purely technical in nature, they 
do not have any negative impact on the paper authors.




\bolditalicparagraph{Innovations with both positive and negative potential outcomes.}
\Tool{} is a tool designed to automatically 
detect vulnerabilities in smart contracts. 
After making \Tool{} open source, there is a possibility that adversaries could use it to identify and exploit 
vulnerabilities for financial gain. To mitigate this risk, we will take the two steps.
First, before releasing \Tool{} as open source, we will apply it to all active smart contracts of high financial value and notify developers of any identified high-security vulnerabilities. We will only make the tool publicly available once developers have acknowledged the vulnerabilities and taken appropriate actions.
Second, after \Tool{} is open-sourced, we will 
actively promote it within the smart contract 
developer community, encouraging them to analyze 
their contracts before deploying them on Ethereum. 
This will help prevent adversaries from using 
\Tool{} to exploit vulnerabilities in newly 
deployed contracts.


\bolditalicparagraph{Retroactively identifying negative outcomes.}
Our research integrates an LLM with symbolic execution to identify ERC violations 
in smart contracts. We believe this technical approach does not raise 
significant ethical concerns, aside from the potential risk that adversaries could exploit our detection methods to identify violations automatically. 
How to mitigate this issue is discussed earlier. 

Our team is dedicated to proactively identifying and addressing 
potential negative outcomes from the outset. 
In the unlikely scenario that unforeseen negative consequences 
emerge after the research is done, we will take full responsibility, 
document the issues, make the issues publicly aware, and implement appropriate measures to remediate 
any harm caused.


\bolditalicparagraph{The law.}
Our research fully complies with all applicable 
laws and regulations. For instance, we do not collect or process any personal data, ensuring 
adherence to the General Data Protection Regulation (GDPR)~\cite{gdpr} and similar standards. 
Additionally, we only use copyrighted materials, such as open-source software and source code, 
strictly for research purposes. Therefore, our use of these materials is legally 
permitted~\cite{copyright-law}.

\section*{Open Science}
This research project generates four types of 
data: (1) empirical study results of ERC rules, 
(2) the source code of \Tool{}, (3) evaluation 
datasets, and (4) evaluation results, traces, and intermediate outputs. We will make the entire 
project, including all related data, open-source 
on GitHub after receiving confirmation from 
developers regarding the detected violations.

The understanding of ERC rules will be summarized 
in PDF files, and detailed labels for each rule 
will be recorded in an Excel file. These files 
will be uploaded to the project repository along 
with the source code. The repository’s structure 
and the tool’s functionalities will be outlined in 
README.md files of the repository.

Additionally, we will provide scripts to automate 
all experiments, facilitating the reproduction of 
our results. A separate document will also be 
included, detailing which script corresponds to 
each piece of data reported in our paper.