%Today's smart contracts largely run on Ethereum, a decentralized blockchain platform. 

To govern smart contracts running on Ethereum, 
multiple Ethereum Request for Comment (ERC) standards have been developed, 
each having a set of rules to guide 
the behaviors of smart contracts. 
Violating the ERC rules could cause serious security issues 
and financial loss, 
signifying the importance of verifying 
smart contracts follow ERCs. 
Today's practices of such verification are to manually audit each single contract, 
use expert-developed program-analysis tools, or use large language models (LLMs), all of which are far from effective in identifying ERC rule violations.

This paper introduces \emph{\Tool{}}, 
a tool that combines the natural language understanding of large language models (LLMs) 
with the formal guarantees of symbolic execution to automatically 
verify smart contracts' compliance with ERC rules. To develop \Tool{}, 
we conduct an empirical study of 132 ERC rules from three widely 
used ERC standards, examining their content, security implications, 
and natural language descriptions. Based on this study, 
we design \Tool{} by first instructing an LLM to translate ERC 
rules into a defined EBNF grammar. We then synthesize constraints 
from the formalized rules to represent scenarios where violations 
may occur and use symbolic execution to detect them. 
Our evaluation shows that \Tool{} identifies 5,783 ERC rule violations 
in 4,000 real-world contracts, including 1,375 violations with 
clear attack paths for stealing financial assets, demonstrating its effectiveness. 
Furthermore, \Tool{} outperforms six automated techniques and a security-expert auditing service, underscoring its superiority over current 
smart contract analysis methods.








%\shihao {
%Our tool, \Tool{}, can automatically audit the smart contract with the ERC by leveraging the LLM with rule extraction and code slicing. It conducts with many other optimizations like prompt specialization, one-shot, and breaking down compound rules to improve precision and recall. 
  

%The evaluation of \Tool{} revealed its capability to identify a significant number of ERC rule violations in various smart contracts. \Tool{} was applied to a dataset comprising 200 smart contracts sourced from etherscan.io~\cite{etherscan} and polyscan.com~\cite{polygonscan}. The findings were noteworthy, with \Tool{} detecting a total of 279 ERC rule violations. These violations varied in terms of security impact: 4 were classified as high-security, 112 as medium-security, and 163 as low-security. Furthermore, in a comparative analysis against human experts focusing on 30 ERC20 contracts from the Ethereum Commonwealth Security Department, \Tool{} demonstrated superior performance. \Tool{} demonstrated exceptional performance, identifying 129 true positives, significantly outperforming human experts who identified only 73. Additionally, it maintained a low error rate with just 3 false positives and 3 false negatives, compared to the higher counts of 12 false positives and 69 false negatives by human experts. \Tool{} required only 1780 seconds for its analysis for all 30 contracts, in stark contrast to the human effort which took approximately 301 days. And \Tool{} merely used in total \$11.92 instead of the estimated human auditor's cost \$150000. \Tool{} outperforms the manual service with significantly better results and lower time and money costs.

%}