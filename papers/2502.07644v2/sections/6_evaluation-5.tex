\subsection{Validating ERC3525 Contracts}
\label{sec:3525}


%\bolditalicparagraphnospace{Methodology.}
\subsubsection{Methodology}

We select ERC3525 to test whether \Tool{} can validate contracts beyond 
the ERCs we have studied. 
ERC3525 was released two years after the studied ERCs 
and has already been used in financial instruments~\cite{fujidao,bufferfinance}, 
making it a representative and more recent standard. 
After a thorough review, we identify 58 rules in ERC3525, 
including 15 with a high-security impact, 
27 with a medium impact, and 16 with a low impact.
We search GitHub for ERC3525 contracts but find only one. 
After auditing it, we confirm it complies with all ERC3525 rules and has no violations. 
Then, we decide to inject violations into the contract to create a testing dataset. 
Specifically, to inject a violation of a rule, we remove the code 
ensuring compliance with the rule. 
In the end, the created dataset contains ten ERC3525 contracts, and each contract 
violates three randomly selected rules, 
resulting in 30 violations in total.


\subsubsection{Experimental Results}
After applying \Tool{} to the ten contracts, 
it successfully identifies all 30 violations, 
while reporting zero false positives. 
We further analyze the results at each step. 
During rule extraction, the LLM in \Tool{} correctly extracts all required 
rules but mistakenly includes six additional ones that are not part of ERC3525. 
These extra rules require a function to call a non-existent function with the same name 
as an event in ERC3525. However, during constraint generation, 
\Tool{} detects the absence of these callee functions and disregards the extraneous rules, leading them not to affect the final results. \Tool{} does not make 
any errors in the subsequent steps.



\begin{tcolorbox}[size=title]
{\textbf{Answer to generality:} 
\Tool{} has the capability to validate contracts beyond the three ERCs studied in Section~\ref{sec:study}. }
\end{tcolorbox}

%\songlh{XXX}

\if 0
\shihao {
Since previous evaluations focused on well-studied ERCs, we employed \Tool{} to analyze ERC3525, allowing us to evaluate \Tool{}'s effectiveness across an end-to-end workflow involving an ERC previously unexamined by our tool. This assessment helps ascertain \Tool{}'s robustness and adaptability in handling novel ERC standards. We run \Tool{} without any human intervention, start with rule extraction and translation, and then use the produced constraints to audit the ERC3525 contracts.

For the ERC3525 contracts, given that only one example contract is available online, we first conducted a manual inspection to confirm its adherence to the standards, ensuring it contained no violations. Subsequently, we selected three rules at random and deliberately introduced violations into the contract. This process was repeated to create ten distinct versions of the contract, each embedded with errors, facilitating a controlled study of \Tool{}'s diagnostic capabilities.
}
\bolditalicparagraphnospace{Extraction Results.}
\shihao {

}
\bolditalicparagraphnospace{Translation Results.}
\shihao {

}
\bolditalicparagraphnospace{Audit Results.}
\shihao {

}

\fi