\documentclass[11pt]{article}
\usepackage{geometry}
\geometry{a4paper,top=1in,bottom=1in,left=1.2in,right=1.2in}

%%%%%%%%%%%%%%%%%%%%%%%%%%%%%%%%%%%%%%%%%%%%%%%%%%%%%%%%%%%%%%%%%%%%%%%%%%%%%%%
%%%%%%%%%%%%%%%%%%%%%%%%%%%%%%%%% Font setup %%%%%%%%%%%%%%%%%%%%%%%%%%%%%%%%%%
%%%%%%%%%%%%%%%%%%%%%%%%%%%%%%%%%%%%%%%%%%%%%%%%%%%%%%%%%%%%%%%%%%%%%%%%%%%%%%%

% Works for pdfLaTeX, XeLaTeX, and LuaLaTeX
\usepackage{libertine}

% Math/Mono font setup
\ExplSyntaxOn
\bool_if:nTF { \sys_if_engine_xetex_p: || \sys_if_engine_luatex_p: }
  {
    % Load mathtools before unicode-math
    \usepackage{mathtools}
    \usepackage[
      math-style=ISO,
      warnings-off={mathtools-colon, mathtools-overbracket},
    ]{unicode-math}
    \setmathfont{LibertinusMath-Regular.otf}[Scale = MatchUppercase]
    \setmathfont{Garamond-Math.otf}[
      Scale = MatchUppercase,
      range = {\Coloneq}
    ]

    \setmonofont{Inconsolatazi4}[
      Extension = .otf,
      UprightFont = *-Regular,
      BoldFont = *-Bold,
      Scale=MatchLowercase,
      AutoFakeSlant = 0.2,
    ]
  }
  {
    \usepackage[libertine]{newtxmath}
    \usepackage{inconsolata}
  }
\ExplSyntaxOff

% Some fancy symbols (prevent being overwritten by math fonts)
\AtBeginDocument{%
  \usepackage[
    only, fatsemi, fatslash, talloblong, rightarrowtriangle
  ]{stmaryrd}%
}


%%%%%%%%%%%%%%%%%%%%%%%%%%%%%%%%%%%%%%%%%%%%%%%%%%%%%%%%%%%%%%%%%%%%%%%%%%%%%%%
%%%%%%%%%%%%%%%%%%%%%%%%%%%%%%%% Presentations %%%%%%%%%%%%%%%%%%%%%%%%%%%%%%%%
%%%%%%%%%%%%%%%%%%%%%%%%%%%%%%%%%%%%%%%%%%%%%%%%%%%%%%%%%%%%%%%%%%%%%%%%%%%%%%%

\usepackage{tcolorbox}
\tcbuselibrary{listings,breakable}
\tcbset{listing engine=listings, colframe=black, colback=white, size=small}

\NewTCBListing { example } { !O{} !s }
  {
    sharp corners,
    IfBooleanTF = { #2 }
      { listing above text }
      { listing side text },
    fontlower = \small,
    #1
  }

\NewTCBListing { listing } { !O{} }
  { sharp corners, listing only, #1 }


\NewDocumentEnvironment {exampleside} {}
  { \tcblisting{listing side text,righthand width=.5\textwidth} }
  { \endtcblisting }

\NewDocumentEnvironment { presentcommand } { b }
  {%
    \vspace*{0.5\baselineskip}\noindent\fbox{%
    \begin{minipage}{\dimexpr\textwidth-2\fboxsep-2\fboxrule}
      #1
    \end{minipage}}\vspace*{0.5\baselineskip}
  }
  { }


\NewDocumentCommand \cmd { m } {\texttt{\char`\\#1}}
\NewDocumentCommand \env { m m }
  {
    \texttt{%
      \char`\\begin\{#1\} \textrm{#2}%
      \char`\\end\{#1\}%
    }%
  }


\usepackage{tabularray}
\UseTblrLibrary{booktabs}
% \fakeverb
\usepackage{codehigh}


\newcommand*{\secref}[1]{\hyperref[{#1}]{section~\ref*{#1} -- \emph{\nameref*{#1}}}}
\newcommand*{\subsecref}[1]{\hyperref[{#1}]{subsection~\ref*{#1} -- \emph{\nameref*{#1}}}}
\newcommand*{\subsubsecref}[1]{\hyperref[{#1}]{\nameref*{#1}}}
\newcommand*{\parref}[1]{\hyperref[{#1}]{\nameref*{#1}}}


%%%%%%%%%%%%%%%%%%%%%%%%%%%%%%%%%%%%%%%%%%%%%%%%%%%%%%%%%%%%%%%%%%%%%%%%%%%%%%%
%%%%%%%%%%%%%%%%%%%%%%%%%%%%%%%%%% simplebnf %%%%%%%%%%%%%%%%%%%%%%%%%%%%%%%%%%
%%%%%%%%%%%%%%%%%%%%%%%%%%%%%%%%%%%%%%%%%%%%%%%%%%%%%%%%%%%%%%%%%%%%%%%%%%%%%%%

\usepackage{simplebnf}


%%%%%%%%%%%%%%%%%%%%%%%%%%%%%%%%%%%%%%%%%%%%%%%%%%%%%%%%%%%%%%%%%%%%%%%%%%%%%%%
%%%%%%%%%%%%%%%%%%%%%%%%%%%%%%%% Miscellaneous %%%%%%%%%%%%%%%%%%%%%%%%%%%%%%%%
%%%%%%%%%%%%%%%%%%%%%%%%%%%%%%%%%%%%%%%%%%%%%%%%%%%%%%%%%%%%%%%%%%%%%%%%%%%%%%%

\usepackage{hologo}
\newcommand*{\XeLaTeX}{\hologo{XeLaTeX}}
\newcommand*{\LuaLaTeX}{\hologo{LuaLaTeX}}

% ``Make sure it comes *last* of your loaded packages"
\usepackage[pdfborder={0 0 0}, colorlinks=true]{hyperref}


%%%%%%%%%%%%%%%%%%%%%%%%%%%%%%%%%%%%%%%%%%%%%%%%%%%%%%%%%%%%%%%%%%%%%%%%%%%%%%%
%%%%%%%%%%%%%%%%%%%%%%%%%%%%%%%%%% Metadata %%%%%%%%%%%%%%%%%%%%%%%%%%%%%%%%%%%
%%%%%%%%%%%%%%%%%%%%%%%%%%%%%%%%%%%%%%%%%%%%%%%%%%%%%%%%%%%%%%%%%%%%%%%%%%%%%%%

\title{%
  \textsf{simplebnf} --- A simple package to format Backus-Naur form%
  \footnote{This file describes v1.0.0.}}
\author{Jay Lee\footnote{E-mail: %
  \href{mailto:jaeho.lee@snu.ac.kr}{\texttt{jaeho.lee@snu.ac.kr}}}}
\date{2023/11/25}


\begin{document}
\maketitle


\vfill
This package provides a simple way for typesetting grammars in Backus-Naur form (BNF).
It features a flexible configuration system, allowing for the customization of the domain-specific language (DSL) used in typesetting the grammar.
Additionally, the package comes with sensible defaults.

Below is the metagrammar of the DSL as defined in this package, which is typeset using the package itself.\footnote{The code is shown in \nameref{sec:appendix}.}
\begin{tcolorbox}[sharp corners]
  \begin{center}
    \begin{bnf}(
      prod-delim = ;;;,
      new-line-delim = !,
      single-line-delim = ?,
      comment = //,
      relation = {:::=|:in:},
      relation-sym-map =
        {
          {:::=} = $\Coloneqq$,
          {:in:} = $\in$,
        },
    )[
      colspec = lrcll,
      column{2} = {mode=dmath},
      column{4} = {mode=text, font=\ttfamily},
    ]
      G // Gramar :::=
      ! $P$ // production
      ! $P \fatsemi {}$ // production w/ a trailing delimiter
      ! $P \fatsemi G$ // production sequence
    ;;;
      P // Production :::= $L \rightarrowtriangle R$
    ;;;
      L // LHS :::=
      ! $v$ // metavariable
      ! $v\!\fatslash\,c$ // annotated metavariable
    ;;;
      R // RHS :::=
      ! $\talloblong$ // delimiter
      ! $A \talloblong R$ // alternative sequence
    ;;;
      A // Alternative :::=
      ! $f$ // syntactic form
      ! $f\!\fatslash\,c$ // annotated syntactic form
    ;;;
      \fatsemi // Prod. delimiter :::=
      ! ;; // default symbol
      ! $\cdots$ // user-defined
    ;;;
      \rightarrowtriangle // Rule relation :::=
      ! ::= // $\Coloneqq$
      ! -> // $\to$
      ! \texttt{\char`\\in} // $\in$
      ! $\cdots$ // user-defined
    ;;;
      \fatslash // Annot. symbol :::=
      ! : // default symbol
      ! $\cdots$ // user-defined
    ;;;
      \talloblong // Alt. delimiter :::=
      ! | // new-line delimiter
      ! || // single-line delimiter
      ! $\cdots$ // user-defined
    ;;;
      v, f, c :in: \textsf{\TeX{} tl} // valid \TeX{} token lists
    ;;;
    \end{bnf}
  \end{center}
\end{tcolorbox}
\vfill

\section{Tutorial for the impatient}\label{sec:tutorial}
Typesetting a grammar is as simple as typing the BNF grammar in the \verb/bnf/ environment:
\begin{example}[righthand width=.52\linewidth]
\begin{center}
\begin{bnf}
  $\tau$ : \textsf{Type} ::=
  | \texttt{num} : numbers
  | \texttt{str} : strings
  ;;
  $e$ : \textsf{Expr} ::=
  | $x$ : variable
  | $n$ : numeral
  | \texttt{$e$ + $e$} : addition
  | \texttt{$e$ * $e$} : multiplication
  | \texttt{$e$ \textasciicircum{} $e$} : concatenation
  | \texttt{len($e$)} : length
  | \texttt{let $x$ = $e_1$ in $e_2$} : definition
\end{bnf}
\end{center}
\end{example}

Typically, each column in the BNF grammar has the same font style.
In the example above, the comments in the first column, e.g., \textsf{Type}, is typeset in sans-serif, and the second column in math mode, e.g., $\tau$.
The column for the right-hand side is (mostly) typeset in typewriter font with a bit of non-terminals like $e$ typeset in math mode.

\textsf{simplebnf} provides a straightforward way to customize the font styles for each column\footnote{Note that you must provide a manual alignment specification with the key \texttt{colspec}.}:
\begin{listing}
\begin{bnf}[
  colspec = {llcll},
  column{1} = {font = \sffamily},
  column{2} = {mode = dmath},
  column{4} = {font = \ttfamily},
]
  \tau : Type ::=
  | num : numbers
  | str : strings
  ;;
  e : Expr ::=
  | $x$ : variable
  | $n$ : numeral
  | $e$ + $e$ : addition
  | $e$ * $e$ : multiplication
  | $e$ \textasciicircum{} $e$ : concatenation
  | len($e$) : length
  | let $x$ = $e_1$ in $e_2$ : definition
\end{bnf}
\end{listing}
If you find yourself using the same configuration repeatedly, you can fix the desired configuration using \cmd{SetBNFLayout}.
Once set using \cmd{SetBNFLayout}, the configuration is applied to all subsequent \verb/bnf/ environments:
\begin{listing}
% Some where above...
\SetBNFLayout{
  colspec = {llcll},
  column{1} = {font = \sffamily},
  column{2} = {mode = dmath},
  column{4} = {font = \ttfamily},
}
% ...
\begin{bnf}
  \tau : Type ::=
  | num : numbers
  | str : strings
  ;;
  e : Expr ::=
  | $x$ : variable
  | $n$ : numeral
  | $e$ + $e$ : addition
  | $e$ * $e$ : multiplication
  | $e$ \textasciicircum{} $e$ : concatenation
  | len($e$) : length
  | let $x$ = $e_1$ in $e_2$ : definition
\end{bnf}
\end{listing}

Since these customizations are provided by the backend \textsf{tabularray}\footnote{\url{https://github.com/lvjr/tabularray}}, consult its documentation for more details.
The corresponding command is \cmd{SetTblrInner}.

For more advanced customization, such as changing default delimiters and symbols, continue to \secref{sec:config}.

\section{Configuration}\label{sec:config}
While the default configuration should be sufficient for most use cases, some grammars may require using different delimiters and symbols.
For example, the language itself may contain \verb/->/ as a syntactic form, which conflicts with one of the default symbols for the rule relation.

To this end, \textsf{simplebnf} provides a number of options to configure the DSL used to typeset the grammar.
Some examples of customizable symbols include the delimiters for productions and alternatives, the relation symbol between the left-hand side and the right-hand side, the annotation symbol, and more.

The default configuration is shown in \autoref{tab:config}.
\begin{longtblr}[
  caption = {Default key-values for the configuration of \textsf{simplebnf}.},
  label = {tab:config}
]{
  colspec={ll},
  row{2-Z}={font=\ttfamily}
}
  \toprule
  Key & Default value \\
  \midrule
  prod-delim & ;;\\
  new-line-delim & \fakeverb{\|}\\
  single-line-delim & //\\
  comment & :\\
  relation & \fakeverb{{::=|->|:in:}}\\
  relation-sym-map & {
    \{\\
    \quad\fakeverb{{::=} = {\ensuremath{\Coloneqq}},}\\
    \quad\fakeverb{{->} = {\ensuremath{\to}},}\\
    \quad\fakeverb{{:in:} = {\ensuremath{\in}},}\\
    \}
  }\\
  or-sym & \fakeverb{$|$}\\
  prod-sep & 2pt\\
  row-sep & 0pt\\
  \bottomrule
\end{longtblr}

The key-value list can be provided as an optional argument to the \verb/bnf/ environment, wrapped in \verb/()/.
The optional key-value list for the layout---wrapped in \verb/[]/---should be provided after the configuration key-value list.
See \subsecref{subsec:layout} for more details.

An exmaple of customizing the configuration is shown below:
\begin{example}*
\begin{center}
\begin{bnf}(
  prod-delim = {--},
  new-line-delim = {\&},
  single-line-delim = {\&\&},
  comment = {//},
  relation-sym-map =
    {
      {->} = {\ensuremath{\hookrightarrow}},
      {:in:} = {\ensuremath{\in}},
    },
  or-sym = {},
)[
  colspec = {llcll},
  column{1} = {font = \sffamily},
  column{2} = {mode = dmath},
  column{4} = {font = \ttfamily},
]
  \tau // Type ->
  & num // numbers
  & str // strings
  --
  e // Expr ->
  & $x$ // variable
  & $n$ // numeral
  & $e$ + $e$ // addition
  & $e$ * $e$ // multiplication
  & $e$ \textasciicircum{} $e$ // concatenation
  & len($e$) // length
  & let $x$ = $e_1$ in $e_2$ // definition
\end{bnf}
\end{center}
\end{example}


Furthermore, the configuration can be set globally using \cmd{SetBNFConfig}, similar to \cmd{SetBNFLayout}:
\begin{listing}
% Some where above...
\SetBNFConfig{
  prod-delim = {--},
  new-line-delim = {\&},
  single-line-delim = {\&\&},
  comment = {//},
  relation-sym-map =
    {
      {->} = {\ensuremath{\hookrightarrow}},
      {:in:} = {\ensuremath{\in}},
    },
  or-sym = {},
}
% ...
\begin{bnf}[
  colspec = {llcll},
  column{1} = {font = \sffamily},
  column{2} = {mode = dmath},
  column{4} = {font = \ttfamily},
]
  \tau // Type ->
  & num // numbers
  & str // strings
  --
  e // Expr ->
  & $x$ // variable
  & $n$ // numeral
  & $e$ + $e$ // addition
  & $e$ * $e$ // multiplication
  & $e$ \textasciicircum{} $e$ // concatenation
  & len($e$) // length
  & let $x$ = $e_1$ in $e_2$ // definition
\end{bnf}
\end{listing}


\subsection{Production delimiter}
The default production delimiter \verb/prod-delim/ is \verb/;;/.
This will separate different productions in the same grammar.

\subsection{New-line alternative delimiter}
The default new-line alternative delimiter \verb/new-line-delim/ is \verb/\|/, which will actually match verbatim \verb/\|/ in the grammar.
This will separate different alternatives into different lines in the same production.

\subsection{Single-line alternative delimiter}
The default single-line alternative delimiter \verb/single-line-delim/ is \verb|//|.
The difference between the single-line alternative delimiter and the new-line alternative delimiter is that the former will not add a new line after the delimiter.

For consecutive alternatives delimited by the single-line alternative delimiter, only the last of them can be annotated.

Note that the single-line alternative delimiter must not contain the new-line alternative delimiter as a substring.

\subsection{Annotation delimiter}
The default annotation delimiter \verb/comment/ is \verb/:/.
This will separate the syntactic form and the annotation within the alternative.

\subsection{Rule relations and the symbol map}
\textsf{simplebnf} provides a \verb/::=/, \verb/->/, and \verb/:in:/ for the rule relation between the left-hand side and the right-hand side.
Each of them is mapped to a symbol in the \verb/relation-sym-map/ key-value list; By default, they are mapped to $\Coloneqq$ (\verb/\ensuremath{\Coloneqq}/), $\to$ (\verb/\ensuremath{\to}/), and $\in$ (\verb/\ensuremath{\in}/), respectively.

To provide a custom relation symbol, provide the desired symbol and the mapping to the \verb/relation-sym-map/ key-value list and to the \verb/relation/.

\subsection{Or symbol}
The default or symbol \verb/or-sym/ is \verb/$|$/.
Probably the most common use case for this option is to remove the or symbol altogether by setting it to \verb/{}/.

\subsection{Production separation}
The default production separation \verb/prod-sep/ is \verb/2pt/.
This will add a vertical space between the productions.

\subsection{Row separation}
The default row separation \verb/row-sep/ is \verb/0pt/.
This will add a vertical space between the alternatives in the same production.

\subsection{Layout}\label{subsec:layout}
As mentioned in \secref{sec:tutorial}, the layout of the \verb/bnf/ environment can be customized per-enviroment using the key-value list in the optional argument wrapped in \verb/[]/.
When provided, these should be provided after the configuration key-value list, which is wrapped in \verb/()/.

The optional key-value list for the layout is passed to the backend \textsf{tabularray} package.

An example of customizing the layout is shown below:
\begin{example}[righthand width=.53\linewidth]
\begin{center}
\begin{bnf}(
  prod-delim = {--},
  or-sym = {},
  prod-sep = 5pt,
)[
  colspec = {|[2pt, gray5]llcll},
  column{1} = {font = \sffamily},
  column{2} = {mode = dmath},
  column{4} = {font = \ttfamily},
  column{5} = {font = \itshape},
  cell{9}{4} = {cmd = \fbox},
  row{-} = {bg = azure9},
]
  \tau : Type ::=
  | num : numbers
  | str : strings
  --
  e : Expr ::=
  | $x$ : variable
  | $n$ : numeral
  | $e$ + $e$ : addition
  | $e$ * $e$ : multiplication
  | $e$ \textasciicircum{} $e$ : concatenation
  | len($e$) : length
  | let $x$ = $e_1$ in $e_2$ : definition
\end{bnf}
\end{center}
\end{example}


\section{Caveats}
The choices of delimiters and symbols should be carefully made, in order to avoid conflicts with the grammar of the target language and our meta-language, \TeX{}.

Another subtlety to consider while customizing delimiters, is that certain delimiters not contain the others as a substring.
For instance, as mentioned in \secref{sec:config}, the single-line alternative delimiter must not contain the new-line alternative delimiter as a substring.

When using custom OpenType fonts with the \verb/unicode-math/ package in \XeLaTeX{} or \LuaLaTeX{}, beware of the fact that many fonts lack the \verb/\Coloneqq/ symbol, $\Coloneqq$.
The easiest way to fix this is to actually use a font that provides the symbol, such as \verb/Garamond-Math/, which comes with the \TeX{}Live installation\footnote{\fakeverb{\Coloneq} is not a typo of \fakeverb{\Coloneqq}.}.
\begin{listing}
\setmathfont{Garamond-Math.otf}[
  Scale = MatchUppercase,
  range = {\Coloneq}
]
\end{listing}


\section{The deprecated \texttt{bnfgrammar} environment}
Prior to version 1.0.0, the package provided a sole environment \verb/bnfgrammar/ for typesetting grammars.
While it is still available, it is now deprecated and will be removed in a future release.
Using it will result in a deprecation warning.

To migrate to the new \verb/bnf/ environment, simply replace the \verb/bnfgrammar/ environment with \verb/bnf/, and replace \verb/\in/ with \verb/:in:/ and \verb/||/ with \verb|//|.
Moreover, the new environment does not center the grammar by default, so wrapping it in a \verb/center/ environment is necessary to reproduce the same output.
While there will be some differences in the spacings and the font styles, it is an easy fix.
See \secref{sec:tutorial} for more details.

Below is an example of using the deprecated \verb/bnfgrammar/ environment, left here for historical reasons.
\begin{example}
  \begin{bnfgrammar}
    a : Variables \in \textit{Vars}
    ;;
    expr : Expressions ::=
      expr + term
    | term
    ;;
    term ::= term * a || a
  \end{bnfgrammar}
\end{example}


\section*{Appendix}\label{sec:appendix}
The following is the code used to typeset the metagrammar of the \textsf{simplebnf} DSL in the first page.
\begin{listing}[breakable]
\begin{bnf}(
  prod-delim = ;;;,
  new-line-delim = !,
  single-line-delim = ?,
  comment = //,
  relation = {:::=|:in:},
  relation-sym-map =
    {
      {:::=} = $\Coloneqq$,
      {:in:} = $\in$,
    },
)[
  colspec = lrcll,
  column{2} = {mode=dmath},
  column{4} = {mode=text, font=\ttfamily},
]
  G // Gramar :::=
  ! $P$ // production
  ! $P \fatsemi {}$ // production w/ a trailing delimiter
  ! $P \fatsemi G$ // production sequence
;;;
  P // Production :::= $L \rightarrowtriangle R$
;;;
  L // LHS :::=
  ! $v$ // metavariable
  ! $v\!\fatslash\,c$ // annotated metavariable
;;;
  R // RHS :::=
  ! $\talloblong$ // delimiter
  ! $A \talloblong R$ // alternative sequence
;;;
  A // Alternative :::=
  ! $f$ // syntactic form
  ! $f\!\fatslash\,c$ // annotated syntactic form
;;;
  \fatsemi // Prod. delimiter :::=
  ! ;; // default symbol
  ! $\cdots$ // user-defined
;;;
  \rightarrowtriangle // Rule relation :::=
  ! ::= // $\Coloneqq$
  ! -> // $\to$
  ! \texttt{\char`\\in} // $\in$
  ! $\cdots$ // user-defined
;;;
  \fatslash // Annot. symbol :::=
  ! : // default symbol
  ! $\cdots$ // user-defined
;;;
  \talloblong // Alt. delimiter :::=
  ! | // new-line delimiter
  ! || // single-line delimiter
  ! $\cdots$ // user-defined
;;;
  v, f, c :in: \textsf{\TeX{} tl} // valid \TeX{} token lists
;;;
\end{bnf}
\end{listing}

\end{document}
