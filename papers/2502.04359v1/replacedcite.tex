\section{Related Work}
\label{sec:rel work}
Previous works have performed a broad analysis of the ability of the VLMs to perform multimodal perception and reasoning tasks such as Spatial Reasoning, Multimodal conversation, etc. Works such as____ introduce comprehensive real-world benchmarks to test multiple VLM capabilities.

____ focus solely on spatial analysis of VLMs.____ go a step further to analyze the role of each modality in spatial reasoning. However, these works do not go deep to test the factors that affect the spatial reasoning ability of the VLMs. On the other hand,____ perform a category-wise analysis of spatial relations. While the former categorize the relations based on their spatial properties, the latter categorize them as either simple, complex, implicit, or superlative.____ evaluate the models for the 4 spatial relations present in their dataset. Different from these works,____ analyze the effects of spatial biases in the datasets for REC task performance.____ analyze if the models are erring in recognizing objects, relations, or both. Some works like____ focus on probing pre-trained LLMs or VLMs with text-only questions for spatial analysis. However, they do not test spatial grounding in the visual modality, a crucial aspect of our work.

Other closely aligned work includes Embodied Spatial Analysis which focuses on the effects of different perspectives and non-verbal cues on the spatial reasoning capabilities of VLMs____.
\noindent \paragraph{Task Complexity and Interpretability.} The works mentioned previously use image-caption agreement as their evaluation task. Due to the inherent limitations of this task, these works simplified the expressions to have only 2 objects and 1 spatial relation. Some works like____ use synthetic datasets instead of real-world images to improve the interpretability of model output. But it simplifies the problem due to bounded expressivity (limited number of objects, attributes, and spatial relations). In our case, REC models output bounding boxes around the target objects. Analyzing the position and characteristics of the output object helps identify the parts of the input that the models fail to process. This enables comparative analysis of expressions with 0, 1, or more spatial relations, a unique feature of our work. The REC task also enables us to test the models over images of different visual complexities (single or multiple instances of objects in an image).
\begin{table*}[!ht]
\centering
\begin{tabular}{lrrrr}
\toprule
Dataset & Object Categories & Average length of & Average no. of objects & No. of spatial relations\\
 & & expression & per image & \\
\midrule
RefCOCO & 80 & 3.6 & 10.6 & 59 \\
RefCOCOg & 80 & 8.4 & 8.2 & 72 \\
CLEVR-Ref+ & 3 & 22.4 & 6.5 & 4 \\
CopsRef & 508 & 14.4 & 17.4 & 51 \\
\bottomrule
\end{tabular}
\caption{Statistics of Popular Referring Expression Comprehension datasets. For the last column, the relation types are taken from various resources explained in Section~\protect\ref{sec:rel work} and~\protect____, in addition to the relations in Table~\protect\ref{table: our work cat}.}
\label{table: data-stats}
\end{table*}
\begin{table*}[!ht]
\centering
\begin{tabular}{lrl}
\toprule
Category & Number & Spatial Relations \\
\midrule
Absolute & 56 & on the right, on the left, in the middle, in the center, from the right, from the left\\
Adjacency & 14 & attached, against, on the side, on the back, on the front, on the edge \\
Directional & 29 & falling off, along, through, across, down, up, hanging from, coming from, around\\
Orientation & 0 & facing \\
Projective & 2361 & on top of, beneath, beside, behind, to the left, to the right, under, above, in front of,\\ & &  over, below, underneath \\ 
Proximity & 217 & by, close to, near \\ 
Topological & 1054 & connected, contain, with, surrounding, surrounded by, inside, between, touching,\\ & & out of, at, in, on \\
Unallocated & 56 & next to, enclosing \\
\bottomrule
\end{tabular}
\caption{Category-wise relation split and number of referring expressions in the CopsRef test set with 1 spatial relation in each category
}
\label{table: our work cat}
\end{table*}