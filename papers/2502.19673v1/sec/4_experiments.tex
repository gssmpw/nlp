\section{Experiments}
\label{sec:exp}

% In this section, we present detailed evaluations of SubZero on a variety of personalization tasks.

% \begin{table*}[ht]
%     %\addtolength{\tabcolsep}{-1.5pt}
%     \vspace{1.2 em}
%     \centering
%     \scalebox{1}{
%     \fontsize{7.0pt}{5.75pt}\selectfont
%     % \fontsize{8.00pt}{8.25pt}\selectfont
%     \begin{tabular}{l|c|c|cc|c} 
%     \toprule
%     Method & Backbone & Helper Prompts & Personal Sim. & Style Sim & Average sim.   \\ [0.5ex] 
%     \hline
%     \midrule
%     InstantStyle-Plus~\cite{wang2024instantstyle_plus} & \multirow{5}{*}{SDXL-Lightning} & & \textbf{69.0} $\pm$ 4.1  &  41.1 $\pm$ 7.7 & 55.1 \\
%     InstantID(Subject)+ControlNet(Style)~\cite{wang2024instantid} & & & 54.2 $\pm$ 1.6  &  53.6 $\pm$ 4.7 & 53.9 \\
%     PulID(Subject)+IP-Adapter(Style)~\cite{guo2024pulid} & & & 56.4 $\pm$ 2.4  &  52.3 $\pm$ 4.1 & 54.4 \\
%     RB-Modulation~\cite{rout2024rb} & & & 59.6 $\pm$ 2.7 & 65.7 $\pm$ 4.2 & 62.7 \\
%     \rowcolor{Gray} SubZero & & & 64.7 $\pm$ 2.6 & \textbf{67.1} $\pm$ 4.3 & \textbf{65.9} \\
%     \midrule
%     InstantStyle-Plus~\cite{wang2024instantstyle_plus} & \multirow{5}{*}{SDXL-Lightning} & \checkmark & 65.7 $\pm$ 4.9  &  46.6 $\pm$ 9.0 & 56.2 \\
%     InstantID(Subject)+ControlNet(Style~\cite{wang2024instantid} & & \checkmark & 54.7 $\pm$ 1.6  &  63.1 $\pm$ 3.9 & 58.9 \\
%     PulID(Face)+IP-Adapter(Style)~\cite{guo2024pulid} & & \checkmark & 59.5 $\pm$ 2.1 & 58.4 $\pm$ 3.1 & 59.0 \\
%     RB-Modulation~\cite{rout2024rb} &  & \checkmark & 60.5 $\pm$ 1.9 & \textbf{72.7} $\pm$ 2.2 & 66.6 \\
%     \rowcolor{Gray} SubZero & & \checkmark & \textbf{66.5} $\pm$ 1.9 & 72.4 $\pm$ 2.4 & \textbf{69.5} \\
%     \midrule
%     RB-Modulation~\cite{rout2024rb} & & & 61.9 $\pm$ 1.1 & 39.3 $\pm$ 1.5 & 50.6 \\
%     \rowcolor{Gray} SubZero & \multirow{-2}{*}{Würstchen} & & \textbf{72.3} $\pm$ 1.5 & \textbf{45.5} $\pm$ 2.8 & \textbf{58.9} \\    

%     \midrule
%     RB-Modulation~\cite{rout2024rb} & & \checkmark & 59.7 $\pm$ 1.0 & 51.0 $\pm$ 1.5 & 55.4 \\
%     \rowcolor{Gray} SubZero & \multirow{-2}{*}{Würstchen} & \checkmark & \textbf{69.8} $\pm$ 1.5 & \textbf{54.9} $\pm$ 2.3 & \textbf{62.3}  \\    
%     \bottomrule
% \end{tabular}
% }
%     \caption{\textbf{Face Stylization:} We report results on SDXL-Lightning and Würstchen. Helper prompts simply mean whether the style description is present in the prompt or not.} 
%     \label{tab:face_stylization_main}
% \end{table*}


\subsection{Experiment Setup}
\label{sec:expsetup}
We primarily conduct three sets of experiments: (\textit{i})~for people, we demonstrate face-style composition using single subject and style images; (\textit{ii})~we show subject-style-action composition using people and styles, while providing text prompts to perform certain actions; (\textit{iii})~finally, for common objects and pets, we conduct object-style composition.% Now, we describe the datasets as well as the metrics used throughout this work.

\begin{table*}[ht]
    %\addtolength{\tabcolsep}{-1.5pt}
    % \vspace{-0.5 em}
    \centering
    \scalebox{1}{
    \fontsize{7.0pt}{5.75pt}\selectfont
    % \fontsize{8.00pt}{8.25pt}\selectfont
    \begin{tabular}{l|c|cc|c|cc|c} 
    \toprule
    Method & Backbone & Subject Projector & Style Projector & Helper Prompts & Face Sim. & Style Sim. & Average   \\ [0.5ex] 
    \hline
    \midrule
    InstantStyle-Plus~\cite{wang2024instantstyle_plus} & \multirow{5}{*}{SDXL-Lightning} & ControlNet & IP-Adapter & & \textbf{69.0} $\pm$ 4.1  &  41.1 $\pm$ 7.7 & 55.1 \\
    InstantID~\cite{wang2024instantid} & & InstantID & ControlNet & & 54.2 $\pm$ 1.6  &  53.6 $\pm$ 4.7 & 53.9 \\
    PuLID~\cite{guo2024pulid} & & PuLID & IP-Adapter & & 56.4 $\pm$ 2.4  &  52.3 $\pm$ 4.1 & 54.4 \\
    RB-Modulation~\cite{rout2024rb} & & PuLID & StyleZero & & 59.6 $\pm$ 2.7 & 65.7 $\pm$ 4.2 & 62.7 \\
    \rowcolor{Gray} SubZero & & PuLID & StyleZero & & 64.7 $\pm$ 2.6 & \textbf{67.1} $\pm$ 4.3 & \textbf{65.9} \\
    \midrule
    InstantStyle-Plus~\cite{wang2024instantstyle_plus} & \multirow{5}{*}{SDXL-Lightning} & ControlNet & IP-Adapter & \checkmark & 65.7 $\pm$ 4.9  &  46.6 $\pm$ 9.0 & 56.2 \\
    InstantID~\cite{wang2024instantid} & & InstantID & ControlNet & \checkmark & 54.7 $\pm$ 1.6  &  63.1 $\pm$ 3.9 & 58.9 \\
    PulID~\cite{guo2024pulid} & & PuLID & IP-Adapter & \checkmark & 59.5 $\pm$ 2.1 & 58.4 $\pm$ 3.1 & 59.0 \\
    RB-Modulation~\cite{rout2024rb} &  & PuLID & StyleZero & \checkmark & 60.5 $\pm$ 1.9 & \textbf{72.7} $\pm$ 2.2 & 66.6 \\
    \rowcolor{Gray} SubZero & & PuLID & StyleZero & \checkmark & \textbf{66.5} $\pm$ 1.9 & 72.4 $\pm$ 2.4 & \textbf{69.5} \\
    \midrule
    RB-Modulation~\cite{rout2024rb} & & - & - & & 61.9 $\pm$ 1.1 & 39.3 $\pm$ 1.5 & 50.6 \\
    \rowcolor{Gray} SubZero & \multirow{-2}{*}{W\"{u}rstchen} & - & - & & \textbf{72.3} $\pm$ 1.5 & \textbf{45.5} $\pm$ 2.8 & \textbf{58.9} \\    

    \midrule
    RB-Modulation~\cite{rout2024rb} & & - & - & \checkmark & 59.7 $\pm$ 1.0 & 51.0 $\pm$ 1.5 & 55.4 \\
    \rowcolor{Gray} SubZero & \multirow{-2}{*}{W\"{u}rstchen} &  - & - & \checkmark & \textbf{69.8} $\pm$ 1.5 & \textbf{54.9} $\pm$ 2.3 & \textbf{62.3}  \\    
    \bottomrule
\end{tabular}
}
    \caption{\textbf{Face Stylization:} Results on SDXL-Lightning and W\"{u}rstchen. Helper prompts indicate the presence of style descriptions.}%We report results on SDXL-Lightning and Würstchen. Helper prompts simply mean whether the style description is present in the prompt or not.

   
    \label{tab:face_stylization_main}
    \vspace{-0.5 em}
\end{table*}



% \begin{table}[ht]
%     %\addtolength{\tabcolsep}{-1.5pt}
%     \vspace{1.2 em}
%     \scalebox{1}{
%     \fontsize{7.0pt}{5.75pt}\selectfont
%     \begin{tabular}{l|ccc|c}
%     \toprule
%     Method & DINO & Style Sim & HPSv2 & Average Score \\ [0.5ex] 
%     \hline
%     \midrule
%     RB-Modulation &  & & &\\
%     IP-Adapter+Pull ID &  &  & &\\
%     ZipLoRA & & & &\\
%     \rowcolor{Gray} Subzero & &  & &\\

%     \bottomrule
% \end{tabular}
%     }
%     % \fontsize{8.00pt}{8.25pt}\selectfont
%     \caption{\textbf{Results on Object +Style+Action:}}
% \end{table}

% \begin{table}[ht]
%     %\addtolength{\tabcolsep}{-1.5pt}
%     \centering
%     \fontsize{7.0pt}{5.75pt}\selectfont
%     \begin{tabular}{ccccc}
%     \toprule
%     & Style scale & Content scale & DINO Sim & CSD Sim \\ [0.5ex] 
%     \hline
%     \midrule
%     IP-Adapter & 0.2 & 0.0 & 0.241 & 0.391 \\
%     Ours & 0.2 & 0.0 & 0.241 &  0.363\\
%     IP-Adapter & 0.4  & 0.0 & 0.204 & 0.617 \\
%     Ours & 0.4 & 0.0 &  0.218 & 0.513 \\
%     IP-Adapter & 0.6  & 0.0 & 0.156 & 0.756 \\
%     Ours & 0.6 & 0.0 & 0.181 & 0.644 \\
%     IP-Adapter & 0.8  & 0.0 & 0.131 & 0.825 \\
%     Ours & 0.8 & 0.0 &  0.154 & 0.716 \\
%     IP-Adapter & 1.0  & 0.0 & 0.121 & 0.838 \\
%     Ours & 1.0 & 0.0 &  0.134 & 0.744  \\
    
%     % \rowcolor{Gray} Content & ObjectZero   & 0.610 &  - \\
%     \bottomrule
% \end{tabular}
% % \fontsize{8.00pt}{8.25pt}\selectfont
% \caption{\textbf{content and style adapters - we can remove this later.}}
% \end{table}


%\subsubsection{Datasets}
%\label{sec:datasets}
\noindent\textbf{Face Stylization Datasets.} To stylize faces, we curated a dataset consisting of 12 subjects and 30 styles. We collected a diverse range of faces across age, ethnicity and gender. Each subject provided a single image, and was asked to participate in the Human Preference Study. For stylizing the faces, we curated a dataset of 30 styles using images from StyleAligned~\cite{hertz2024style}, StyleDrop~\cite{sohn2023styledrop} and SubjectPlop~\cite{ruiz2024magic}.

% \noindent\textbf{Object-Style Composition Datasets.} For object-style composition, we use an experimental setup of ZipLoRA~\cite{shah2025ziplora}, with object images from the Dreambooth dataset~\cite{ruiz2023dreambooth}, having 30 subjects with 4-5 images per subjects. Style images are chosen from StyleDrop~\cite{sohn2023styledrop} dataset, where a single image is used per style. For each content and style, we use a single reference image.
% \noindent\textbf{Object-Style Composition Datasets.} For object-style composition, we use the experimental setup proposed by ZipLoRA~\cite{shah2025ziplora}, and select ten subjects with 4-5 images per subjects. Ten style images are chosen from StyleDrop~\cite{sohn2023styledrop} dataset, where a single image is used per style. For each content and style, we use a single reference image.
\noindent\textbf{Object-Style Composition Datasets.} For object-style composition, we use a similar setup as ZipLoRA~\cite{shah2025ziplora}, and select ten unique objects from the Dreambooth dataset~\cite{ruiz2023dreambooth}, and ten style images from StyleDrop dataset~\cite{sohn2023styledrop}.

%\subsubsection{Metrics}
%\label{sec:metrics}
\noindent\textbf{Metrics.} For object similarity we use DINO similarity score ~\cite{ruiz2023dreambooth}, i.e., cosine similarity of DINO ViT-B/6 embeddings of the object and generated images. For face similarity, we measure the cosine similarity using facial embeddings from~\cite{ye2023ip-adapter}. Further, we compute style similarity by reporting the cosine similarity between CSD embedding~\cite{somepalli2024measuring} of the reference vs. generated images. We also conduct human evaluations to quantify face stylization. For measuring performance on actions, we use the HPS-v2.1~\cite{hpsv2} score between the output image and action prompt. All metrics are computed as percentages.

\begin{figure}[h]
    \centering
    \includegraphics[width=1.0\linewidth]{figures/subzero_vs_rbmod.jpg}
    \vspace{- 1.5 em}
    \caption{\textbf{Comparison v/s RB-Modulation} on W\"{u}rstchen. As observed, SubZero outputs looks much more similar to the reference subject compared to RB-Modulation.}
    \vspace{- 0.7 em}
    \label{fig:wurschten_comp}
\end{figure}


\noindent\textbf{Models.}
We use two text-to-image models to achieve efficient zero-shot subject, style, and action composition: (\textit{i})~SDXL-Lightning(4-step)~\cite{lin2024sdxl} and (\textit{ii})~Stable Cascade (W\"{u}rstchen)~\cite{perniaswurstchen}. Following RB-Modulation, we use AFA-based conditioning for W\"{u}rstchen since it contains already learned CLIP-text and image projections. For experiments on SDXL-Lightning, we exploit IP-Adapters as a baseline to project the reference images to cross-attentions. For face stylization experiments with SubZero, we use PuLID as the face projector with StyleZero as the style projector. For object stylization experiments with SubZero, we use our new StyleZero and ObjectZero image projectors.

We consider several baselines for comparisons, namely, InstantStyle-Plus~\cite{wang2024instantstyle_plus}, InstantID~\cite{wang2024instantid}, RB-Modulation~\cite{rout2024rb} and Style-Aligned~\cite{hertz2024style}. Some of these baselines also exploit Controlnet~\cite{controlnet} or IP-Adapters~\cite{ye2023ip-adapter} to inject styles from reference images. All implementation details and hyperparameters are provided in the Appendix.

% \paragraph{Latent Modulation Details.}
% We run stochastic optimal control on the latent for $M=5$ iterations during each denoising step for both Stable-Cascade and SDXL-Lightning

\subsection{Results}\label{sec:results}
%We now present our extensive evaluations to demonstrate the effectiveness of our proposed SubZero framework. %and StyleZero/ObjectZero projectors.

\subsubsection{Face Style Composition}\label{sec:faceRes}
%For the face stylization tasks, we encode the style reference image using our StyleZero projector and face images using PuLID~\cite{guo2024pulid}. Then, we perform the latent modulation using our proposed SubZero objectives and orthogonal temporal aggregation of various cross-attentions among text, style, and face modalities. 
As observed in Fig.~\ref{fig:faces_stylized}, SubZero can effectively stylize the given faces into a diverse range of styles.% very effectively. 

\noindent\textbf{Quantitative Comparisons.} We compare SubZero against several state-of-the-art tuning-free personalization methods for SDXL-Lightning and W\"{u}rstchen architectures, with and without ``helper prompts'' (i.e., whether or not style description is present in the text prompt). We provide mean scores over 3 random seeds. Table~\ref{tab:face_stylization_main} presents our main result: SubZero produces the best images for personal (face)-similarity and style-similarity with or without helper prompts. For instance, while InstantStyle-Plus~\cite{wang2024instantstyle_plus} achieves higher face-similarity score for SDXL-Lightning without helper prompts, it achieves significantly lower style-similarity than our proposed technique. This suggests that while InstantStyle-Plus is good at reproducing faces due to ControlNet, it performs suboptimal stylization. Similarly, while RB-Modulation~\cite{rout2024rb} achieves good stylization for SDXL-Lightning with helper prompts, it cannot capture faces accurately. SubZero significantly outperforms the prior art as it achieves the highest average similarity score and establishes a new state-of-the-art for face stylization.


\begin{figure}[h]
    \centering
    \includegraphics[width=0.9\linewidth]{figures/subzero_vs_controlnet.jpg}
    %\vspace{- 1.2 em}
    \caption{\textbf{Visual comparison} between SubZero and ControlNet/DDIM Inversion based schemes. SubZero is more flexible and reduces subject leakage.} 
    \label{fig:vs_controlnet_schemes}
    \vspace{- 0.6 em}
\end{figure}


\noindent\textbf{Qualitative Comparisons.} Next, we compare SubZero and RB-Modulation~\cite{rout2024rb} in Fig.~\ref{fig:wurschten_comp}. As evident, SubZero is significantly more effective at maintaining the correct subject through various styles. In contrast, RB-Modulation fails to preserve the correct face while performing stylization. 
In Fig.~\ref{fig:vs_controlnet_schemes}, we compare against InstantX methods~\cite{wang2024instantid, wang2024instantstyle_plus} that employ ControlNet and/or DDIM-inversion for subject-style composition. As observed, InstantID often leaks irrelevant content from style reference into the final generated image or suffers from undesirable artifacts. On the other hand, InstantStyle-Plus achieves good stylization but it is too rigid due to ControlNet; this results in significantly less diverse output images. Clearly, SubZero outperforms these methods in both diversity as well as stylization quality.

\noindent \textbf{Human Preference Study:} We surveyed 10 subjects who provided their photos, by using a customized human evaluation form \textit{containing their own images}, as shown in the Appendix. Each form had three sections, the results of which are summarized in Table~\ref{tab:human}. Each section had 10 styles. Hence, our evaluation contains 300 responses. We place generated images from various models side-by-side v/s subzero and ask humans to pick the image which most resembles their face while best aligning with the reference style image. As observed in Table~\ref{tab:human}, SubZero was the preferred choice at \textbf{64.1}$\%$ v/s the PuLID+IP-Adapter baseline,  \textbf{64.5}$\%$ v/s RB-Modulation(on W\"{u}rstchen) and \textbf{74.7}$\%$ v/s InstantStyle by the human subjects themselves.


\begin{table}[ht]
    %\addtolength{\tabcolsep}{-1.5pt}
    \centering
    \vspace{-0.5 em}

    \fontsize{7.0pt}{5.75pt}\selectfont
    \begin{tabular}{l|c|c|c}
    \toprule
    Method & v/s PuLID+IP-Adapter & v/s InstantStyle-Plus & v/s RB-Mod \\ [0.5ex] 
    \hline
    \midrule
    Not Subzero & 21.7 & 24.0 & 11.8 \\
    Tie & 14.1 & 1.3 & 23.7  \\
    \rowcolor{Gray} SubZero & \textbf{64.1}  & \textbf{74.7} & \textbf{64.5} \\
    \bottomrule
\end{tabular}
% \fontsize{8.00pt}{8.25pt}\selectfont

\caption{\textbf{Human Evaluation for Face Stylization.}}
 \vspace{-1.2 em}
\label{tab:human}
\end{table}

\vspace{-0.5 em}
\begin{figure}[h]
    \centering
    \includegraphics[width=0.9\linewidth]{figures/subzero_actions.jpg}
    %\vspace{- 1.2 em}
    \caption{\textbf{Face, Style and Action composition} using SubZero.}
    \vspace{- 0.8 em}
    \label{fig:face_style_action}
    
\end{figure}

\subsubsection{Face-Style-Action Composition}
Could we compose the face of any subject in any style performing any action in a zero-shot setting? 
We explore this aspect using SubZero and evaluate it on face stylization for a set of actions described by action prompts. Table~\ref{tab:face_style_action} shows the results across 12 subjects, 10 Actions and 10 Styles and an average across 3 seeds. We report the Human Preference Scores (HPSv2), in addition to the usual face- and style-similarities. We notice that SubZero improves significantly over the baselines especially on the HPSv2 score. RB Modulation suffers from content style leakage through AFA which makes it harder to generate more diverse images. Since SubZero exploits our proposed orthogonal temporal aggregation strategy for the cross-attentions across multiple modalities, we achieve significantly stronger results. Additionally, ControlNet and DDIM inversion prove to hinder flexibility, resulting in lower HPSv2 scores for InstantX based methods. 
Our results can be visualized in Fig.~\ref{fig:face_style_action}. 
\label{sec:facestyleactionRes}

% \begin{figure}[t!]
%     \centering
%     \renewcommand{\tabcolsep}{1pt}
%     \newcommand{\mywidth}{0.5\linewidth}
%     \resizebox{\linewidth}{!}{% take the[] entire width, but still find a good per-image width otherwise text gets compressed
%     \begin{tabular}{cccc}
%     Content & Style & IP-Adapter & Ours \\
%     \includegraphics[width=\mywidth]{figures/object_style_composition/content/candle_content.jpg}
%     & \includegraphics[width=\mywidth]{figures/object_style_composition/style/image_01_22.jpg}
%     & \includegraphics[width=\mywidth]{figures/object_style_composition/ip/prompt_A_candle_content_candle_0.4_style_sty_22_0.6.png} & \includegraphics[width=\mywidth]{figures/object_style_composition/ours/prompt_A_candle_content_candle_0.4_style_sty_22_0.6.png}
%     \\
%     \end{tabular}}
%     \caption[]{Caption here.}
%     \label{fig:object_style_action}
% \end{figure}

\begin{table}[ht]
    %\addtolength{\tabcolsep}{-1.5pt}
    \vspace{-0.5 em}
    \scalebox{0.95}{
    \fontsize{7.0pt}{5.75pt}\selectfont
    \begin{tabular}{l|ccc|c}
    \toprule
    Method & Face Sim. & Style Sim. & HPSv2 & Average \\ [0.5ex] 
    \hline
    \midrule
    InstantStyle-Plus~\cite{wang2024instantstyle_plus} & 66.0 & 47.3 & 24.6 & 46.0 \\
    InstantID~\cite{wang2024instantid} & 62.3 & 58.2 & 22.4 & 47.6 \\ 
    PulID+IP-Adapter~\cite{guo2024pulid} & 58.9 & 56.0 & 24.9 & 45.9 \\
    RB-Modulation~\cite{rout2024rb} & 58.3 & 72.6 & 24.8 & 51.9 \\
    \rowcolor{Gray} SubZero & \textbf{64.2} & \textbf{73.1} & \textbf{26.1} & \textbf{54.5} \\

    \bottomrule
\end{tabular}
    }
    % \fontsize{8.00pt}{8.25pt}\selectfont
    \caption{\textbf{Results on Face+Style+Action:} We report results using SDXL-Lightning as a backbone and compare SubZero against SOTA methods for composing subjects, styles and actions.}
    \vspace{- 1.5 em}

    \label{tab:face_style_action}
\end{table}



\begin{figure}[ht]
    \centering
    \includegraphics[width=0.85\linewidth]{figures/subzero_object_stylization.jpg}
    \caption{\textbf{Object and Style composition} using SubZero.}
    \vspace{- 1 em}
    \label{fig:object_style_action}
\end{figure}

\subsubsection{Object-Style Composition}\label{sec:objstyleRes}
We now evaluate the ability of SubZero to compose any object in any style in a zero-shot manner using our newly trained StyleZero and ObjectZero projectors. To this end, we use all subjects from the DreamBooth dataset and 20 styles from StyleDrop~\cite{sohn2023styledrop} to perform object-style composition for 600 object-style pairs. %We select the following subjects to show the effectiveness of our approach: []. 
Table~\ref{tab:obj_stylization_main} shows we achieve a very high DINO score, demonstrating the strong ability of SubZero to maintain the correct content while generating zero-shot stylized images. On SDXL-Lightning, we also achieve the best style similarity. On an average, we significantly outperform the IP-Adapter, RB-Modulation and StyleAligned baselines. 

Fig.~\ref{fig:object_style_action} shows the qualitative comparison between IP-Adapter, RB-Modulation, and SubZero. Notably, both IP-Adapter and RB-Modulation show irrelevant content leakage (e.g., see the house/hut structure getting leaked into the bottom dog). In contrast, SubZero performs the object-style composition without any leakage. This clearly highlights the superiority of SubZero compared to existing methods.

\begin{table}[ht]
    \addtolength{\tabcolsep}{-1.5pt}
    \vspace{-0.5 em}
    \centering
    \scalebox{1}{
    \fontsize{7.0pt}{5.75pt}\selectfont
    % \fontsize{8.00pt}{8.25pt}\selectfont
    \begin{tabular}{l|c|cc|c} 
    \toprule
    Method & Backbone & DINO Sim. & Style Sim. & Average \\ [0.5ex] 
    \hline
    \midrule
    % ZipLoRA~\cite{shah2025ziplora} & SDXL & 52.2 & 33.8 & 43.0\\
    StyleAligned~\cite{hertz2024style} & SDXL & 36.8 & 51.0 & 43.9 \\

    \midrule
    IP-Adapter~\cite{ye2023ip-adapter} & \multirow{3}{*}{Lightning} & 46.0 & 36.2 & 41.1 \\
    RB-Mod~\cite{rout2024rb}+IP-Apapter &  & 48.7 & 58.8 & 53.8 \\
    \rowcolor{Gray} SubZero & & \textbf{53.5} & \textbf{61.4} & \textbf{57.5} \\
    \midrule
    RB-Modulation~\cite{rout2024rb} & & 42.6  & \textbf{44.2} & 43.4 \\
    \rowcolor{Gray} SubZero & \multirow{-2}{*}{W\"{u}rstchen} & \textbf{63.2} & 44.0 & \textbf{53.6} \\    
    \bottomrule
\end{tabular}
}
    \vspace{- 0.5 em}
    \caption{\textbf{Object-Style Composition:} We report results on SDXL-Lightning and W\"{u}rstchen and compare SubZero against IP-Adapter, Style-aligned and RB-Modulation.} 

    \label{tab:obj_stylization_main}
\end{table}



\begin{table}[h]
    %\addtolength{\tabcolsep}{-1.5pt}

    
    \centering
    \fontsize{7.0pt}{5.75pt}\selectfont
    \begin{tabular}{c|cc|ccc}
    \toprule
    Helper Prompt & Dis. Control & OTA & Face Sim. & Style Sim. & Average\\
    \hline
    \midrule
       &  & & 57.7 & 54.1 & 55.9 \\
       \rowcolor{Gray} &  & \checkmark & 59.0 & 53.4 & 56.2 \\
       \rowcolor{Gray} & \checkmark & \checkmark & \textbf{64.7} & \textbf{67.1} & \textbf{65.9} \\
      \hline
     \checkmark & & & 59.5 & 58.4 & 59.0 \\
      \rowcolor{Gray} \checkmark & & \checkmark & 60.1 & 61.9 & 61.0 \\
      \rowcolor{Gray} \checkmark & \checkmark & \checkmark & \textbf{66.5} & \textbf{72.4} & \textbf{69.5} \\
    \bottomrule
\end{tabular}
% \fontsize{8.00pt}{8.25pt}\selectfont
\vspace{-0.7 em}
\caption{\textbf{Individual gain from SubZero components:} We report results on SDXL-Lightning with StyleZero and PulID.}\label{table:face_ablation}
\vspace{-1.0 em}

\end{table}

\subsection{Ablation Studies}\label{sec:ablations}
\noindent\textbf{Individual gain from all inference components.} Table \ref{table:face_ablation} shows the individual gain from our proposed Disentangled Latent Optimization~\ref{subsec:dsoc} and the Orthogonal Temporal Aggregation (OTA) scheme, both with and without helper prompts. We perform this experiment on the face stylization task from Table~\ref{tab:face_stylization_main}. Results are on an SDXL-Lightning baseline, with PuLID as the subject projector and StyleZero as the style projector. As observed, OTA improves the Average score by \textbf{0.3} to \textbf{2}$\%$, and the latent optimizer further improves the it by \textbf{9.5}$\%$. Overall, both the methods compliment each other and contribute significant gains.


\noindent\textbf{Impact of Style Projectors.} We demonstrate the effectiveness of our StyleZero projector compared to existing style projectors IP-Adapter~\cite{ye2023ip-adapter} and StyleCrafter~\cite{liu2023stylecrafter}, on face style composition in Table~\ref{table:stylezero_faces}. As observed, SubZero works standalone with all existing style projectors and with StyleZero we observe a \textbf{1.4} to \textbf{1.8}$\%$ improvement.

\begin{table}[h]
    %\addtolength{\tabcolsep}{-1.5pt}

    \vspace{-0.3 em}
    \centering
    \fontsize{7.0pt}{5.75pt}\selectfont
    \begin{tabular}{lc|ccc}
    \toprule
    Method & Style Projector & Face Sim & Style Sim & Average\\
    \hline
    \midrule
      \rowcolor{Gray}  & IP-Adapter~\cite{ye2023ip-adapter} & 65.9 & 70.3 & 68.1 \\
      \rowcolor{Gray}  & StyleCrafter~\cite{hertz2024style} & 63.5 & 71.9 & 67.7 \\
     \rowcolor{Gray}  \multirow{-3}{*}{SubZero} & StyleZero & \textbf{66.5} & \textbf{72.4} & \textbf{69.5} \\
    \bottomrule
\end{tabular}
% \fontsize{8.00pt}{8.25pt}\selectfont
\vspace{-0.5 em}
\caption{\textbf{SubZero with various facial style projectors.}}\label{table:stylezero_faces}  
 \vspace{-0.7 em}
\end{table}


% \noindent\textbf{Individual gain from the Style/Object projector pair:} Table \ref{table:face_ablation} shows the individual gain from our disentangled StyleZero and ObjectZero projector pair. We perform this experiment on the object stylization task. However, unlike Table~\ref{tab:obj_stylization_main}, the results are on an SDXL baseline. We compare using our projectors against IP-Adapter.
% As observed, using StyleZero and ObjectZero significantly improves the Object similarity metric while slightly improving the style similarity. This is due to the fact that our adapters are lesser prone to Subject leakage.

% \begin{table}[ht]
%     %\addtolength{\tabcolsep}{-1.5pt}
%     \centering
%     \vspace{-0.5 em}

%     \fontsize{7.0pt}{5.75pt}\selectfont
%     \begin{tabular}{c|c|cc}
%     \toprule
%     Style Projector & Object Projector & DINO Sim & Style Sim \\ [0.5ex] 
%     \hline
%     % \midrule
%     % IP-Adapter  & - & 0.156 &  0.756\\
%     % \rowcolor{Gray} StyleZero & - & 0.181 & 0.644 \\
%     % \midrule
%     %  - & IP-Adapter~\cite{ye2023ip-adapter}   & 0.546 & 0.161 \\
%     % % Content & IP-Adapter   & 0.546 & - \\
%     % \rowcolor{Gray} - & ObjectZero (Ours)   & \textbf{0.610} &  \textbf{0.167} \\
%     \midrule
%      IP-Adapter~\cite{ye2023ip-adapter} & IP-Adapter~\cite{ye2023ip-adapter}   & 0.351 & 0.438 \\
%     % Content & IP-Adapter   & 0.546 & - \\
%     \rowcolor{Gray} StyleZero & ObjectZero   & \textbf{0.539} & \textbf{0.441} \\
    
%     % \rowcolor{Gray} Content & ObjectZero   & 0.610 &  - \\
%     \bottomrule
% \end{tabular}
% % \fontsize{8.00pt}{8.25pt}\selectfont

% \caption{\textbf{Our trained standalone object/style adapters}}
% \vspace{-1em}
% \label{tab:standalone}
% \end{table}


%We conducted two sets of experiments to showcase the effectiveness of each proposed contribution. 

%Table \ref{table:face_ablation} shows the effect of our disentangled stochastic optimal controller and orthogonal temporal aggregation for face stylization. In addition, we conduct another experiment using only our trained projectors vs. IP-adapter and report the results in Table~\ref{tab:standalone}. Our projectors produce results with higher object similarity while providing better stylization.   


