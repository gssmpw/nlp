\section{Related Work}
\subsection{Anomaly Detection}
With the release of the MVTec AD dataset \cite{bergmann2019mvtec} and Visa \cite{zou2022spot}, the development of industrial image anomaly detection has shifted from a supervised paradigm to an unsupervised paradigm. research on common AD \cite{liu2024deep} has been divided into two main categories: feature-embedding-based methods and reconstruction-based methods.
\textbf{Feature-embedding-based methods} can be further categorized into four subcategories, including teacher-student model\cite{bergmann2020uninformed,salehi2021multiresolution,deng2022anomaly,tien2023revisiting,batzner2024efficientad}, one-class classification methods \cite{liu2023simplenet,cao2023anomaly}, mapping-based methods \cite{zhou2024msflow,rudolph2022fully} and memory-based methods \cite{li2022towards,xie2023pushing}. \textbf{Reconstruction-based
methods} can be further divided into Autoencoders-based \cite{schluter2022natural,zavrtanik2021draem}, Generative Adversarial Networks-based \cite{peng2024industrial}.

However, existing AD methods train separate anomaly models for different classes, which inevitably suffers from catastrophic forgetting and excessive computational burden. Even the multi-class unified anomaly detection model \cite{you2022unified,zhao2023omnial,ho2024long} does not consider the case of continuous anomaly detection. Our method is specifically designed for the scenario of continuous learning and achieves continuous anomaly detection and segmentation in an unsupervised manner.

%然而,现有的UAD方法为不同类训练单独的异常模型,这不可避免的存在灾难性遗忘和计算负担过重的问题。即使多类统一异常检测模型(You et al. 2022;Zhao 2023)也没有考虑到持续异常检测的情况。而我们的方法是专门为连续学习的场景而设计的,并以无监督的方式实现连续异常检测和分割。
\begin{figure*}[h]
	\centering 
	%%\centering
	\includegraphics[scale=0.32]{framework.pdf} 
	\caption{The framework of UCAD using multimodal Task Representation Memory Bank.
    (a) Text-image data is input during the training phase, and an effective task intrinsic memory bank is formed through the KPMK. In addition, we use RSCL to better utilize task-related contextual information to obtain a more compact MTRMB. (b) When a test image is input during the testing phase, the framework automatically queries the Task key to retrieve the corresponding task prompts, completes the model's transfer of task knowledge through the prompts, then extracts the features of the test image and calculates the similarity with normal knowledge, and finally completes continuous detection of anomalies. (c) The KPMK mechanism uses the concise key to guide the cross-fusion of features from two different modalities, text and image, and generates an effective task representation memory bank.}
	\label{MTRMB} 
\end{figure*}

%(a)在训练阶段输入文本-图像数据,通过KPMK机制构成有效的任务本征内存池。此外,我们使用RSCL更好的利用与任务相关的上下文信息来获得更紧凑的MTMB。(b)在测试阶段输入测试图像,该框架自动查询Task key以检索相应的任务提示,通过提示完成模型对任务知识的迁移,然后提取测试图像的特征并与正常知识进行相似度计算,最后完成持续检测异常。(c)(KPMK)机制利用简明的关键提示指导来自文本(BERT)和图像(ViT)两种不同模态的特征交叉融合,并生成有效的任务表征内存池。

\subsection{Continual Anomaly Detection}
IDDM \cite{zhang2023iddm} proposed an incremental anomaly detection method based on a small number of labeled samples. On the other hand, LeMO \cite{li2023cross} follows the common unsupervised anomaly detection paradigm and performs incremental anomaly detection as normal samples continue to increase. However, both IDDM and LeMO focus on the study of intra-class continuous anomaly detection, but do not address the challenge of inter-class incremental anomaly detection. Li et al. \cite{li2022towards} proposed DNE for image-level anomaly detection in continuous learning scenarios. Due to the limitation that DNE only stores class-level information, it cannot perform fine-grained localization and is therefore not suitable for anomaly segmentation. UCAD solves the problem that DNE cannot locate abnormal areas and further alleviates the problems of catastrophic forgetting and excessive computational burden. However, this method suffers from performance limitations due to the lack of compactness and comprehensiveness of the representation.

%IDDM(Zhang和Chen 2023)提出了一种基于少量标记样本的增量异常检测方法。另一方面,LeMO(Gao et al. 2023)遵循常见的无监督异常检测范式,随着正常样本的不断增加而进行增量异常检测。然而,IDDM和LeMO都关注类内连续异常检测研究,但没有解决类间增量异常检测的挑战。Li等人(Li等人,2022年)提出了DNE,用于连续学习场景中的图像级异常检测。由于DNE仅存储类级信息的局限性,无法进行细粒度定位,因此不适合进行异常分割。UCAD 解决了 DNE 无法定位异常区域的问题,并进一步缓解了灾难性遗忘和计算负担过大的问题。然而,该方法由于表示缺乏紧凑性和全面性而存在性能限制。