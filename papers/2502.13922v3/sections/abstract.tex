\begin{abstract}

Large Language Models (LLMs) have demonstrated remarkable capabilities through pretraining and alignment. However, superior short-context LLMs may underperform in long-context scenarios due to insufficient long-context alignment. This alignment process remains challenging due to the impracticality of human annotation for extended contexts and the difficulty in balancing short- and long-context performance. To address these challenges, we introduce \ourMethod{}, that enables short-context LLMs to self-evolve to excel on long-context tasks by internally transferring short-context capabilities. \ourMethod{} harnesses LLMs to learn from self-generated short-to-long preference data, comprising paired responses generated for identical instructions with long-context inputs and their compressed short-context counterparts, respectively. This preference reveals capabilities and potentials of LLMs cultivated during short-context alignment that may be diminished in under-aligned long-context scenarios. Additionally, \ourMethod{} incorporates a short-to-long KL constraint to mitigate short-context performance decline during long-context alignment. When applied to Mistral-7B-Instruct-v0.2 from 128K to 512K context lengths, \ourMethod{} fully retains short-context performance and largely outperforms naive SFT and DPO in both long- and short-context tasks. Specifically, \ourMethod-trained models can achieve results on long-context benchmarks comparable to, or even surpassing, those of superior LLMs (e.g., GPT-4-128K) that involve extensive long-context annotation and larger parameter scales. Our code is available at~\url{https://github.com/DAMO-NLP-SG/LongPO}.
% Our approach offers an efficient and balanced solution for developing long-context LLMs, effectively addressing alignment challenges without compromising short-context capabilities or requiring extensive continual training on long documents.

\end{abstract}
