\section{Related Work}
\subsection{Data-Driven Motion Synthesis}
%\subsection{Modeling Human Kinematics and Dynamics using Deep learning}
 
In the motion synthesis area, researchers and developers in the graphics, animation, and gaming industries, dive deeply into both deterministic and probabilistic Deep Learning (DL) motion generation
techniques. These data-driven approaches leverage human motion capture and movement history data (e.g., previous frames or motion states like joint angles) to produce or predict the pose and/or joint trajectories of a 3D humanoid character \cite{Loi2023_review}. Especially for deterministic motion synthesis, where the synthesized motion sequence converges to a deterministic pose sequence that regresses towards the mean pose of ground-truth, recurrent architectures like LSTM models \cite{Aristidou2021} for typical motion imitation and RNNs enhanced with phase-functioned networks \cite{Starke2019, Starke2020, Starke2021} for scene-aware interactions, are favored. The work in \cite{Starke2019}, introduces the Neural State Machine (NSM), a framework for goal-driven real-time synthesis of 3D character movements and scene interactions. The NSM consists of a gating network and an RNN motion prediction network, where the first accounts for automatic action state transition based on the global phase of the motion signal and the user input (i.e. desired goal). In the subsequent works of the same authors \cite{Starke2020, Starke2021}, an architecture similar to NSM is employed and enhanced with: i) a local phase feature that exploits the local motion phases of each skeletal segment to dynamically produce their asynchronous motion in character-character, character-object, and character-scene interactions in \cite{Starke2020}, and ii) a control scheme in \cite{Starke2021} for animation layering to produce sophisticated (e.g. martial arts movements) and novel motions based on reference motions, physics-based simulations, input controls, etc.
Convolutional Neural Networks (CNNs) have also been employed for synthesizing human motion sequences \cite{Zhou2019}.

On the contrary, Generative Adversarial Networks (GANs) \cite{Men2021, Mourot2022}, Variational Autoencoders \cite{Cai2021, Zhou2023}, Transformers \cite{Hou2024, Chai2024}, and the state-of-the-art diffusion models for motion generation conditioned on multiple sources (e.g. natural language descriptions and audio) \cite{Raab2023, Alexanderson2023, Dabral2023, Gao2024}, are exploited for probabilistic motion synthesis meaning the construction of all plausible pose sequences of a virtual character based on historical poses and/or control inputs. Probabilistic DL models have the ability to inject stochasticity in the training process, i.e. by fitting a latent distribution to the distribution of the next pose in each invocation, and, hence, increasing the learning capacity of the model.

Motion synthesis works can be used alongside musculoskeletal dynamics estimation approaches (described in Section \ref{sec:Musculoskeletal}) to reflect the human internal state on a 3D humanoid character, thus, providing physics-based solutions. Consequently, works in the physics-based motion synthesis area develop holistic approaches, usually relying on Reinforcement Learning (RL) \cite{Lee2021, Luo2024}, which account for the data-driven imitation of physically realistic motion sequences conditioned on physical parameters such as joint velocities, forces, torques, etc. It is worth mentioning the most recent work in \cite{Zhang2024}, where an auto-regressive neural network resembling previous work \cite{Starke2019, Starke2020, Starke2021}, which incorporates knowledge from the physical attributes of exerted human force and perceived resistance, was applied to model variations in human movements while interacting with objects and, thus, enhancing realism.

Nevertheless, even though great steps are made towards more realistic computer graphics character animation, all the aforementioned methods and the ones referenced in Section \ref{sec:Musculoskeletal}, focus on modeling the active state of human motion. Fatigued motion using DL has received little to no attention in the literature. To the best of our knowledge, only one work, the one in \cite{Cheema2023}, attempts fatigue-driven animation, using RL methods for controlled motion imitation, and the Three-Compartment Controller (3CC) state machine to model torque-based fatigue. 

\subsection{Data-Driven Musculoskeletal Dynamics and Kinematics Estimation}
\label{sec:Musculoskeletal}

Estimating musculoskeletal human biomechanics has been the focus of research for many years, with results affecting multiple research fields such as biomechanics, the gaming industry, robotics, and the animation industry. Biomechanics engineers use powerful open-source tools like OpenSim \cite{Seth2018}, to simulate with great precision human kinematics and compute the kinetics and dynamics of a motion (e.g. joint contact forces, joint torques, muscle forces, etc.), while state-of-the-art methods focus in data-driven approaches (ML/DL), such as the ones in \cite{Sharma2022, Loi2023, Wang2023, Mansour2023} to obtain more automated, faster as well as real-time, solutions while estimating biomechanical variables traditionally calculated through musculoskeletal modeling. Such works rely on raw motion marker data, joint kinematics data (i.e. joint angles), Electromyography (EMG) data, and Ground Reaction Forces (GRFs) to infer predictions. 

In \cite{Sharma2022} different feed-forward neural network configurations obtained through hyperparameter space exploration were trained and applied to estimate both joint biomechanical parameters (angles, reactions forces, torques) as well as muscle forces and activations in upper extremities. The authors concluded that more complex neural architectures with optimal parameters obtained through hyperparameter space search, lead to more accurate estimations. Furthermore, RNN models pose as the most common approaches in human biomechanics estimation \cite{Loi2023, Mansour2023, Wang2023}. %combined with sophisticated transfer learning techniques were developed for the estimation of knee contact forces (KCFs). 
In particular, in \cite{Loi2023}, a Bidirectional Long Short-Term Memory network (BiLSTM), augmented with unsupervised domain adaptation layers, was proposed to perform domain alignment between experimental data from various movements, with the goal to simultaneously generalize across multiple actions and align real-to-synthetic data. 
%In the more recent work \cite{Zou2024}, a Multisource Fusion Long Short-Term Memory (MF-LSTM) network, (i.e. a model able to extract temporal dependencies in motion data and fuse multiple data sources, such as EMGs and GRFs) and an enhanced version of MF-LSTM with a transfer learning scheme, were developed for non- and subject-specific medial KCF estimation, respectively. 
LSTM has also been utilized for ankle, knee, and hip joint torque estimation in sit-to-stand trials, alongside other machine learning (ML) and deep learning (DL) techniques such as Linear Regression (LR), Support Vector Machines (SVM), and Convolutional Neural Networks (CNN), as documented in \cite{Mansour2023}, with LSTM indicating the best performance for the task. Similarly, LSTM and Gaussian Process Regression (GPR) were employed in \cite{Wang2023} to predict lower extremity joint torques during gait. 

%Motion synthesis works can be used alongside musculoskeletal dynamics estimation approaches to reflect the human internal state on a 3D humanoid character, thus, providing physics-based solutions. Consequently, works in the physics-based motion synthesis area develop holistic approaches, usually relying on Reinforcement Learning (RL) \cite{Lee2021, Luo2024}, which account for the data-driven imitation of physically realistic motion sequences conditioned on physical parameters such as joint velocities, forces, torques, etc. It is worth mentioning the most recent work in \cite{Zhang2024}, where an auto-regressive neural network resembling previous work \cite{Starke2019, Starke2020, Starke2021}, which incorporates knowledge from the physical attributes of exerted human force and perceived resistance, was applied to model variations in human movements while interacting with objects and, thus, enhancing realism.

%Nevertheless, even though great steps are made towards more realistic computer graphics character animation, all the aforementioned methods focus on modeling the active state of human motion. Fatigued motion using DL has received little to no attention in the literature. To the best of our knowledge, only one work, the one in \cite{Cheema2023}, attempts fatigue-driven animation, using RL methods for controlled motion imitation, and the Three-Compartment Controller (3CC) state machine to model torque-based fatigue. 
%\textcolor{red}{In \cite{Cheema2023}, [.....]. (Mocap of fatigued motion???). Other works, \cite{Cheema2020}, ...(Mocap of fatigued motion???)}

\subsection{Physics-Informed Neural Networks}

Physics Informed Neural Networks, first introduced in \cite{Raissi2017, Raissi2019}, are standard machine/deep learning models that incorporate the underlying physical laws of their training dataset, expressed as partial differential equations (PDEs), in the learning process (i.e. loss function) to address both forward and inverse problems. Forward problems involve approximating the PDEs governing the training dataset to derive solutions, without prior knowledge of the ground-truth, relying solely on the provided input data and respective boundary conditions. State-of-the-art works in fluid \cite{Mahmoudabadbozchelou2022, Eivazi2024} and quantum mechanics/computing \cite{Sedykh2024, Trahan2024} utilize PINNs to solve forward problems, without explicitly modeling the underlying physics, unlike finite element and conventional physics-based methods. In inverse problems, PINNs leverage physics-based domain knowledge to penalize the estimation of physical quantities (i.e. limiting the solution space), enhancing robustness and generalization in cases of limited data availability as ground truth, where conventional ML/DL methods are rendered ineffective. 

PINNs, despite being the revolution in the area of fluid and quantum mechanics, are also starting to be applied for biomechanics estimation tasks \cite{Zhang2023, Ma2024, ZhiboZhang2022, Taneja2024, Kumar2023, Zhang2022_2, Kumar2024, Zhang2023_2}. In \cite{Zhang2022_2, Zhang2023}, PINNs with CNNs as a base are used to predict muscle forces and joint angles based on EMGs, with the work in \cite{Zhang2022_2} developing a physics-based domain knowledge transfer technique to develop subject-specific PINN approaches. Furthermore, in \cite{Zhang2023_2} a distributed physics-informed DL approach that builds local models acting on subdomains of EMG input data to enhance the efficiency and robustness of muscle forces and joint angle estimation, is presented, while in \cite{Kumar2023} a PINN-based approach for upper limb joint angle estimation under various loads is explored.
In the most recent works \cite{Taneja2024, Kumar2024}, involving the estimation of muscle parameters and joint angles \cite{Taneja2024} as well as joint torques \cite{Kumar2024} of the upper limbs, novel PINN architectures that incorporate Gated Recurrent Units (GRUs), are implemented. GRUs are a variant of RNNs with fewer training parameters and enhanced computational efficiency compared to LSTM models \cite{Cho2014, Chollet2017}, which significantly improve the accuracy of predictions involving time-dependent inputs and also support the learning of long-term time-dependencies. It is also worth mentioning the study in \cite{Ma2024}, which predicts muscle forces and identifies muscle-tendon parameters from unlabeled EMGs using both explicit and implicit PINN losses (i.e. describing the implicit relationship between muscle forces predicted by a DL model and those calculated by an embedded musculoskeletal model). 

\comm{
\subsection{Fatigue}
There have been numerous attempts to model fatigue in the Biomechanics literature. In [], fatigue is defined as a function of .... (insert function). 

Comment2: "fatigue is not usually considered in most of the existing MSK models, which leads to non-negligible errors" -> or data-driven motion synthesis frameworks

Moreover a model .., (3CC)
how to model fatigue using functions + 3CC (all relevant works from 2008 to 2020/1) + fatigue modeling using deep learning -> only \cite{Cheema2023} (Mocap of fatigued motion???)

Michaud2024 "Four-compartment muscle fatigue model to predict metabolic inhibition and long-lasting nonmetabolic components"}