\section{Related Work}
% https://www.notion.so/Related-work-1433d23836128057a911f2e26ce8c650

% one paragraph - one topic
% at the end of each paragraph subconclusion how we differ / or position our work

% The approach of compressing the context into a shorter sequence of input or KV-cache 
% vectors is already widespread despite no commonly used naming or generalized
% procedure exist. 

The approach of compressing the context of LLMs into a shorter sequence of input or KV-cache vectors is actively explored for various purposes, yet no standardized terminology or unified methodology has emerged. 
%Context compression
\paragraph{Context compression.} One application for input compression is connected with efficient processing of long contexts with LLMs. RMT~\cite{rmt_2022} and AutoCompressors~\cite{chevalier2023adapting} train the whole language model in a recurrent manner to compress the information from input segments to summary vectors and later reuse them to solve long-context tasks.~\citet{ge2023incontext} use an autoencoder architecture with a frozen LLM as a decoder and adapt the same LLM for the encoder using LoRA~\cite{hu2022lora}. The resulting pipeline is pretrained using autoencoding and language modeling objectives, and then finetuned for language tasks, achieving the effective compression rate of x4. SelfCP~\cite{gao2024selfcp} uses the base LLM itself as a compressor using a trainable adapter to aggregate compressed states across multiple segments. 500xCompressor~\cite{li2024500xcompressor} extends the autoencoding approach with layer-wise connections and additional language pretraining tasks, exploring compression ratios up to x480, though at the cost of substantial quality degradation. In contrast, our method, applied to models of comparable size (up to 8 billion parameters), demonstrates that a compression rate of x1568 can be achieved with no loss in reconstruction quality

\paragraph{Prompt compression.} Another line of work targets compressing only the prompts to reduce inference costs. Gist tokens~\cite{mu2023learning} are prompt representations compressed by the LLM itself, finetuned with a special mask. Gisting allows to achieve prompt compression rate up to x26 with only minor loss in model performance. LLMLingua~\cite{jiang2023llmlingua} decouples the compression operation from the LLM and introduces a coarse-to-fine prompt compression strategy with a budget controller and token-level iterative compression, achieving up to 20 times compression with negligible performance loss. Subsequent works~\cite{jiang2023longllmlingua2, pan2024llmlingua} extend this framework to long contexts, improving information retention through data distillation. 
Additionally,~\citet{morris2023language} suggest that some information about prompts can be recovered from the model predictions themselves. 
In the current work
% we do not require generalization across different samples but
we apply a per-sample optimization process instead to explore the fundamental limits of compression
% . By using gradient descent for encoding, we
and establish upper bounds on compression rates that far exceed prior work.

\paragraph{LLM-based lossless compression pipelines.} Language models have also been investigated for lossless text compression. LLMZip~\cite{valmeekam2023llmzip} improves standard compression by ranking candidates via next-token probabilities, while FineZip~\cite{mittu2024finezip} accelerates compression through finetuning and dynamic context management for better efficiency.
% evaluation of LLM as compressors via perplexity/bpc
The theoretical capabilities of LLMs in compression pipelines can be estimated via measuring the bits-per-token metric over a representative textual corpus. \citet{huang2024compression, guo2024ranking} provide such measurements for public LLMs and establish the connection between compression rate and model performance, measured by diverse benchmark scores. 
Unlike these methods, we do not rely on external compression algorithms. Instead, we achieve lossless compression using only the LLM itself, providing both theoretical insights and practical demonstrations of compression limits.

\paragraph{Trainable tokens.} Some works utilize the trainable input tokens approach in other ways. \citet{burtsev2021memory} uses memory tokens as additional representation storage, \citet{beltagy2020longformer} and \citet{zaheer2020big} use similar global tokens to enhance long-range information flow. \citet{li2021prefix,lester2021power,gao2024selfcp,liu2022ptuning} explore trainable soft prompts for one or multiple layers as an alternative to finetuning model weights. 
Our findings about the representation capacity can represent the potential efficiency limits of such methods, based on how far the model behavior can be changed using trainable tokens.