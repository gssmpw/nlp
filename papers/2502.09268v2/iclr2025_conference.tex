
\documentclass{article} % For LaTeX2e
\usepackage{iclr2025_conference,times}


% add by myself
\usepackage{booktabs} % Add this line in your preamble if it's not already included
\usepackage{multirow}


% Optional math commands from https://github.com/goodfeli/dlbook_notation.
%%%%% NEW MATH DEFINITIONS %%%%%

\usepackage{amsmath,amsfonts,bm}
\usepackage{derivative}
% Mark sections of captions for referring to divisions of figures
\newcommand{\figleft}{{\em (Left)}}
\newcommand{\figcenter}{{\em (Center)}}
\newcommand{\figright}{{\em (Right)}}
\newcommand{\figtop}{{\em (Top)}}
\newcommand{\figbottom}{{\em (Bottom)}}
\newcommand{\captiona}{{\em (a)}}
\newcommand{\captionb}{{\em (b)}}
\newcommand{\captionc}{{\em (c)}}
\newcommand{\captiond}{{\em (d)}}

% Highlight a newly defined term
\newcommand{\newterm}[1]{{\bf #1}}

% Derivative d 
\newcommand{\deriv}{{\mathrm{d}}}

% Figure reference, lower-case.
\def\figref#1{figure~\ref{#1}}
% Figure reference, capital. For start of sentence
\def\Figref#1{Figure~\ref{#1}}
\def\twofigref#1#2{figures \ref{#1} and \ref{#2}}
\def\quadfigref#1#2#3#4{figures \ref{#1}, \ref{#2}, \ref{#3} and \ref{#4}}
% Section reference, lower-case.
\def\secref#1{section~\ref{#1}}
% Section reference, capital.
\def\Secref#1{Section~\ref{#1}}
% Reference to two sections.
\def\twosecrefs#1#2{sections \ref{#1} and \ref{#2}}
% Reference to three sections.
\def\secrefs#1#2#3{sections \ref{#1}, \ref{#2} and \ref{#3}}
% Reference to an equation, lower-case.
\def\eqref#1{equation~\ref{#1}}
% Reference to an equation, upper case
\def\Eqref#1{Equation~\ref{#1}}
% A raw reference to an equation---avoid using if possible
\def\plaineqref#1{\ref{#1}}
% Reference to a chapter, lower-case.
\def\chapref#1{chapter~\ref{#1}}
% Reference to an equation, upper case.
\def\Chapref#1{Chapter~\ref{#1}}
% Reference to a range of chapters
\def\rangechapref#1#2{chapters\ref{#1}--\ref{#2}}
% Reference to an algorithm, lower-case.
\def\algref#1{algorithm~\ref{#1}}
% Reference to an algorithm, upper case.
\def\Algref#1{Algorithm~\ref{#1}}
\def\twoalgref#1#2{algorithms \ref{#1} and \ref{#2}}
\def\Twoalgref#1#2{Algorithms \ref{#1} and \ref{#2}}
% Reference to a part, lower case
\def\partref#1{part~\ref{#1}}
% Reference to a part, upper case
\def\Partref#1{Part~\ref{#1}}
\def\twopartref#1#2{parts \ref{#1} and \ref{#2}}

\def\ceil#1{\lceil #1 \rceil}
\def\floor#1{\lfloor #1 \rfloor}
\def\1{\bm{1}}
\newcommand{\train}{\mathcal{D}}
\newcommand{\valid}{\mathcal{D_{\mathrm{valid}}}}
\newcommand{\test}{\mathcal{D_{\mathrm{test}}}}

\def\eps{{\epsilon}}


% Random variables
\def\reta{{\textnormal{$\eta$}}}
\def\ra{{\textnormal{a}}}
\def\rb{{\textnormal{b}}}
\def\rc{{\textnormal{c}}}
\def\rd{{\textnormal{d}}}
\def\re{{\textnormal{e}}}
\def\rf{{\textnormal{f}}}
\def\rg{{\textnormal{g}}}
\def\rh{{\textnormal{h}}}
\def\ri{{\textnormal{i}}}
\def\rj{{\textnormal{j}}}
\def\rk{{\textnormal{k}}}
\def\rl{{\textnormal{l}}}
% rm is already a command, just don't name any random variables m
\def\rn{{\textnormal{n}}}
\def\ro{{\textnormal{o}}}
\def\rp{{\textnormal{p}}}
\def\rq{{\textnormal{q}}}
\def\rr{{\textnormal{r}}}
\def\rs{{\textnormal{s}}}
\def\rt{{\textnormal{t}}}
\def\ru{{\textnormal{u}}}
\def\rv{{\textnormal{v}}}
\def\rw{{\textnormal{w}}}
\def\rx{{\textnormal{x}}}
\def\ry{{\textnormal{y}}}
\def\rz{{\textnormal{z}}}

% Random vectors
\def\rvepsilon{{\mathbf{\epsilon}}}
\def\rvphi{{\mathbf{\phi}}}
\def\rvtheta{{\mathbf{\theta}}}
\def\rva{{\mathbf{a}}}
\def\rvb{{\mathbf{b}}}
\def\rvc{{\mathbf{c}}}
\def\rvd{{\mathbf{d}}}
\def\rve{{\mathbf{e}}}
\def\rvf{{\mathbf{f}}}
\def\rvg{{\mathbf{g}}}
\def\rvh{{\mathbf{h}}}
\def\rvu{{\mathbf{i}}}
\def\rvj{{\mathbf{j}}}
\def\rvk{{\mathbf{k}}}
\def\rvl{{\mathbf{l}}}
\def\rvm{{\mathbf{m}}}
\def\rvn{{\mathbf{n}}}
\def\rvo{{\mathbf{o}}}
\def\rvp{{\mathbf{p}}}
\def\rvq{{\mathbf{q}}}
\def\rvr{{\mathbf{r}}}
\def\rvs{{\mathbf{s}}}
\def\rvt{{\mathbf{t}}}
\def\rvu{{\mathbf{u}}}
\def\rvv{{\mathbf{v}}}
\def\rvw{{\mathbf{w}}}
\def\rvx{{\mathbf{x}}}
\def\rvy{{\mathbf{y}}}
\def\rvz{{\mathbf{z}}}

% Elements of random vectors
\def\erva{{\textnormal{a}}}
\def\ervb{{\textnormal{b}}}
\def\ervc{{\textnormal{c}}}
\def\ervd{{\textnormal{d}}}
\def\erve{{\textnormal{e}}}
\def\ervf{{\textnormal{f}}}
\def\ervg{{\textnormal{g}}}
\def\ervh{{\textnormal{h}}}
\def\ervi{{\textnormal{i}}}
\def\ervj{{\textnormal{j}}}
\def\ervk{{\textnormal{k}}}
\def\ervl{{\textnormal{l}}}
\def\ervm{{\textnormal{m}}}
\def\ervn{{\textnormal{n}}}
\def\ervo{{\textnormal{o}}}
\def\ervp{{\textnormal{p}}}
\def\ervq{{\textnormal{q}}}
\def\ervr{{\textnormal{r}}}
\def\ervs{{\textnormal{s}}}
\def\ervt{{\textnormal{t}}}
\def\ervu{{\textnormal{u}}}
\def\ervv{{\textnormal{v}}}
\def\ervw{{\textnormal{w}}}
\def\ervx{{\textnormal{x}}}
\def\ervy{{\textnormal{y}}}
\def\ervz{{\textnormal{z}}}

% Random matrices
\def\rmA{{\mathbf{A}}}
\def\rmB{{\mathbf{B}}}
\def\rmC{{\mathbf{C}}}
\def\rmD{{\mathbf{D}}}
\def\rmE{{\mathbf{E}}}
\def\rmF{{\mathbf{F}}}
\def\rmG{{\mathbf{G}}}
\def\rmH{{\mathbf{H}}}
\def\rmI{{\mathbf{I}}}
\def\rmJ{{\mathbf{J}}}
\def\rmK{{\mathbf{K}}}
\def\rmL{{\mathbf{L}}}
\def\rmM{{\mathbf{M}}}
\def\rmN{{\mathbf{N}}}
\def\rmO{{\mathbf{O}}}
\def\rmP{{\mathbf{P}}}
\def\rmQ{{\mathbf{Q}}}
\def\rmR{{\mathbf{R}}}
\def\rmS{{\mathbf{S}}}
\def\rmT{{\mathbf{T}}}
\def\rmU{{\mathbf{U}}}
\def\rmV{{\mathbf{V}}}
\def\rmW{{\mathbf{W}}}
\def\rmX{{\mathbf{X}}}
\def\rmY{{\mathbf{Y}}}
\def\rmZ{{\mathbf{Z}}}

% Elements of random matrices
\def\ermA{{\textnormal{A}}}
\def\ermB{{\textnormal{B}}}
\def\ermC{{\textnormal{C}}}
\def\ermD{{\textnormal{D}}}
\def\ermE{{\textnormal{E}}}
\def\ermF{{\textnormal{F}}}
\def\ermG{{\textnormal{G}}}
\def\ermH{{\textnormal{H}}}
\def\ermI{{\textnormal{I}}}
\def\ermJ{{\textnormal{J}}}
\def\ermK{{\textnormal{K}}}
\def\ermL{{\textnormal{L}}}
\def\ermM{{\textnormal{M}}}
\def\ermN{{\textnormal{N}}}
\def\ermO{{\textnormal{O}}}
\def\ermP{{\textnormal{P}}}
\def\ermQ{{\textnormal{Q}}}
\def\ermR{{\textnormal{R}}}
\def\ermS{{\textnormal{S}}}
\def\ermT{{\textnormal{T}}}
\def\ermU{{\textnormal{U}}}
\def\ermV{{\textnormal{V}}}
\def\ermW{{\textnormal{W}}}
\def\ermX{{\textnormal{X}}}
\def\ermY{{\textnormal{Y}}}
\def\ermZ{{\textnormal{Z}}}

% Vectors
\def\vzero{{\bm{0}}}
\def\vone{{\bm{1}}}
\def\vmu{{\bm{\mu}}}
\def\vtheta{{\bm{\theta}}}
\def\vphi{{\bm{\phi}}}
\def\va{{\bm{a}}}
\def\vb{{\bm{b}}}
\def\vc{{\bm{c}}}
\def\vd{{\bm{d}}}
\def\ve{{\bm{e}}}
\def\vf{{\bm{f}}}
\def\vg{{\bm{g}}}
\def\vh{{\bm{h}}}
\def\vi{{\bm{i}}}
\def\vj{{\bm{j}}}
\def\vk{{\bm{k}}}
\def\vl{{\bm{l}}}
\def\vm{{\bm{m}}}
\def\vn{{\bm{n}}}
\def\vo{{\bm{o}}}
\def\vp{{\bm{p}}}
\def\vq{{\bm{q}}}
\def\vr{{\bm{r}}}
\def\vs{{\bm{s}}}
\def\vt{{\bm{t}}}
\def\vu{{\bm{u}}}
\def\vv{{\bm{v}}}
\def\vw{{\bm{w}}}
\def\vx{{\bm{x}}}
\def\vy{{\bm{y}}}
\def\vz{{\bm{z}}}

% Elements of vectors
\def\evalpha{{\alpha}}
\def\evbeta{{\beta}}
\def\evepsilon{{\epsilon}}
\def\evlambda{{\lambda}}
\def\evomega{{\omega}}
\def\evmu{{\mu}}
\def\evpsi{{\psi}}
\def\evsigma{{\sigma}}
\def\evtheta{{\theta}}
\def\eva{{a}}
\def\evb{{b}}
\def\evc{{c}}
\def\evd{{d}}
\def\eve{{e}}
\def\evf{{f}}
\def\evg{{g}}
\def\evh{{h}}
\def\evi{{i}}
\def\evj{{j}}
\def\evk{{k}}
\def\evl{{l}}
\def\evm{{m}}
\def\evn{{n}}
\def\evo{{o}}
\def\evp{{p}}
\def\evq{{q}}
\def\evr{{r}}
\def\evs{{s}}
\def\evt{{t}}
\def\evu{{u}}
\def\evv{{v}}
\def\evw{{w}}
\def\evx{{x}}
\def\evy{{y}}
\def\evz{{z}}

% Matrix
\def\mA{{\bm{A}}}
\def\mB{{\bm{B}}}
\def\mC{{\bm{C}}}
\def\mD{{\bm{D}}}
\def\mE{{\bm{E}}}
\def\mF{{\bm{F}}}
\def\mG{{\bm{G}}}
\def\mH{{\bm{H}}}
\def\mI{{\bm{I}}}
\def\mJ{{\bm{J}}}
\def\mK{{\bm{K}}}
\def\mL{{\bm{L}}}
\def\mM{{\bm{M}}}
\def\mN{{\bm{N}}}
\def\mO{{\bm{O}}}
\def\mP{{\bm{P}}}
\def\mQ{{\bm{Q}}}
\def\mR{{\bm{R}}}
\def\mS{{\bm{S}}}
\def\mT{{\bm{T}}}
\def\mU{{\bm{U}}}
\def\mV{{\bm{V}}}
\def\mW{{\bm{W}}}
\def\mX{{\bm{X}}}
\def\mY{{\bm{Y}}}
\def\mZ{{\bm{Z}}}
\def\mBeta{{\bm{\beta}}}
\def\mPhi{{\bm{\Phi}}}
\def\mLambda{{\bm{\Lambda}}}
\def\mSigma{{\bm{\Sigma}}}

% Tensor
\DeclareMathAlphabet{\mathsfit}{\encodingdefault}{\sfdefault}{m}{sl}
\SetMathAlphabet{\mathsfit}{bold}{\encodingdefault}{\sfdefault}{bx}{n}
\newcommand{\tens}[1]{\bm{\mathsfit{#1}}}
\def\tA{{\tens{A}}}
\def\tB{{\tens{B}}}
\def\tC{{\tens{C}}}
\def\tD{{\tens{D}}}
\def\tE{{\tens{E}}}
\def\tF{{\tens{F}}}
\def\tG{{\tens{G}}}
\def\tH{{\tens{H}}}
\def\tI{{\tens{I}}}
\def\tJ{{\tens{J}}}
\def\tK{{\tens{K}}}
\def\tL{{\tens{L}}}
\def\tM{{\tens{M}}}
\def\tN{{\tens{N}}}
\def\tO{{\tens{O}}}
\def\tP{{\tens{P}}}
\def\tQ{{\tens{Q}}}
\def\tR{{\tens{R}}}
\def\tS{{\tens{S}}}
\def\tT{{\tens{T}}}
\def\tU{{\tens{U}}}
\def\tV{{\tens{V}}}
\def\tW{{\tens{W}}}
\def\tX{{\tens{X}}}
\def\tY{{\tens{Y}}}
\def\tZ{{\tens{Z}}}


% Graph
\def\gA{{\mathcal{A}}}
\def\gB{{\mathcal{B}}}
\def\gC{{\mathcal{C}}}
\def\gD{{\mathcal{D}}}
\def\gE{{\mathcal{E}}}
\def\gF{{\mathcal{F}}}
\def\gG{{\mathcal{G}}}
\def\gH{{\mathcal{H}}}
\def\gI{{\mathcal{I}}}
\def\gJ{{\mathcal{J}}}
\def\gK{{\mathcal{K}}}
\def\gL{{\mathcal{L}}}
\def\gM{{\mathcal{M}}}
\def\gN{{\mathcal{N}}}
\def\gO{{\mathcal{O}}}
\def\gP{{\mathcal{P}}}
\def\gQ{{\mathcal{Q}}}
\def\gR{{\mathcal{R}}}
\def\gS{{\mathcal{S}}}
\def\gT{{\mathcal{T}}}
\def\gU{{\mathcal{U}}}
\def\gV{{\mathcal{V}}}
\def\gW{{\mathcal{W}}}
\def\gX{{\mathcal{X}}}
\def\gY{{\mathcal{Y}}}
\def\gZ{{\mathcal{Z}}}

% Sets
\def\sA{{\mathbb{A}}}
\def\sB{{\mathbb{B}}}
\def\sC{{\mathbb{C}}}
\def\sD{{\mathbb{D}}}
% Don't use a set called E, because this would be the same as our symbol
% for expectation.
\def\sF{{\mathbb{F}}}
\def\sG{{\mathbb{G}}}
\def\sH{{\mathbb{H}}}
\def\sI{{\mathbb{I}}}
\def\sJ{{\mathbb{J}}}
\def\sK{{\mathbb{K}}}
\def\sL{{\mathbb{L}}}
\def\sM{{\mathbb{M}}}
\def\sN{{\mathbb{N}}}
\def\sO{{\mathbb{O}}}
\def\sP{{\mathbb{P}}}
\def\sQ{{\mathbb{Q}}}
\def\sR{{\mathbb{R}}}
\def\sS{{\mathbb{S}}}
\def\sT{{\mathbb{T}}}
\def\sU{{\mathbb{U}}}
\def\sV{{\mathbb{V}}}
\def\sW{{\mathbb{W}}}
\def\sX{{\mathbb{X}}}
\def\sY{{\mathbb{Y}}}
\def\sZ{{\mathbb{Z}}}

% Entries of a matrix
\def\emLambda{{\Lambda}}
\def\emA{{A}}
\def\emB{{B}}
\def\emC{{C}}
\def\emD{{D}}
\def\emE{{E}}
\def\emF{{F}}
\def\emG{{G}}
\def\emH{{H}}
\def\emI{{I}}
\def\emJ{{J}}
\def\emK{{K}}
\def\emL{{L}}
\def\emM{{M}}
\def\emN{{N}}
\def\emO{{O}}
\def\emP{{P}}
\def\emQ{{Q}}
\def\emR{{R}}
\def\emS{{S}}
\def\emT{{T}}
\def\emU{{U}}
\def\emV{{V}}
\def\emW{{W}}
\def\emX{{X}}
\def\emY{{Y}}
\def\emZ{{Z}}
\def\emSigma{{\Sigma}}

% entries of a tensor
% Same font as tensor, without \bm wrapper
\newcommand{\etens}[1]{\mathsfit{#1}}
\def\etLambda{{\etens{\Lambda}}}
\def\etA{{\etens{A}}}
\def\etB{{\etens{B}}}
\def\etC{{\etens{C}}}
\def\etD{{\etens{D}}}
\def\etE{{\etens{E}}}
\def\etF{{\etens{F}}}
\def\etG{{\etens{G}}}
\def\etH{{\etens{H}}}
\def\etI{{\etens{I}}}
\def\etJ{{\etens{J}}}
\def\etK{{\etens{K}}}
\def\etL{{\etens{L}}}
\def\etM{{\etens{M}}}
\def\etN{{\etens{N}}}
\def\etO{{\etens{O}}}
\def\etP{{\etens{P}}}
\def\etQ{{\etens{Q}}}
\def\etR{{\etens{R}}}
\def\etS{{\etens{S}}}
\def\etT{{\etens{T}}}
\def\etU{{\etens{U}}}
\def\etV{{\etens{V}}}
\def\etW{{\etens{W}}}
\def\etX{{\etens{X}}}
\def\etY{{\etens{Y}}}
\def\etZ{{\etens{Z}}}

% The true underlying data generating distribution
\newcommand{\pdata}{p_{\rm{data}}}
\newcommand{\ptarget}{p_{\rm{target}}}
\newcommand{\pprior}{p_{\rm{prior}}}
\newcommand{\pbase}{p_{\rm{base}}}
\newcommand{\pref}{p_{\rm{ref}}}

% The empirical distribution defined by the training set
\newcommand{\ptrain}{\hat{p}_{\rm{data}}}
\newcommand{\Ptrain}{\hat{P}_{\rm{data}}}
% The model distribution
\newcommand{\pmodel}{p_{\rm{model}}}
\newcommand{\Pmodel}{P_{\rm{model}}}
\newcommand{\ptildemodel}{\tilde{p}_{\rm{model}}}
% Stochastic autoencoder distributions
\newcommand{\pencode}{p_{\rm{encoder}}}
\newcommand{\pdecode}{p_{\rm{decoder}}}
\newcommand{\precons}{p_{\rm{reconstruct}}}

\newcommand{\laplace}{\mathrm{Laplace}} % Laplace distribution

\newcommand{\E}{\mathbb{E}}
\newcommand{\Ls}{\mathcal{L}}
\newcommand{\R}{\mathbb{R}}
\newcommand{\emp}{\tilde{p}}
\newcommand{\lr}{\alpha}
\newcommand{\reg}{\lambda}
\newcommand{\rect}{\mathrm{rectifier}}
\newcommand{\softmax}{\mathrm{softmax}}
\newcommand{\sigmoid}{\sigma}
\newcommand{\softplus}{\zeta}
\newcommand{\KL}{D_{\mathrm{KL}}}
\newcommand{\Var}{\mathrm{Var}}
\newcommand{\standarderror}{\mathrm{SE}}
\newcommand{\Cov}{\mathrm{Cov}}
% Wolfram Mathworld says $L^2$ is for function spaces and $\ell^2$ is for vectors
% But then they seem to use $L^2$ for vectors throughout the site, and so does
% wikipedia.
\newcommand{\normlzero}{L^0}
\newcommand{\normlone}{L^1}
\newcommand{\normltwo}{L^2}
\newcommand{\normlp}{L^p}
\newcommand{\normmax}{L^\infty}

\newcommand{\parents}{Pa} % See usage in notation.tex. Chosen to match Daphne's book.

\DeclareMathOperator*{\argmax}{arg\,max}
\DeclareMathOperator*{\argmin}{arg\,min}

\DeclareMathOperator{\sign}{sign}
\DeclareMathOperator{\Tr}{Tr}
\let\ab\allowbreak


\usepackage{hyperref}
\usepackage{url}
\usepackage{graphicx} 
\usepackage{hyperref}
\usepackage{amsmath}
\usepackage{wrapfig}
\usepackage{subfigure}
\usepackage{algorithm}
\usepackage{algpseudocode}
\usepackage{booktabs}
\usepackage{colortbl} % 导入 colortbl 宏包
\usepackage{xcolor} 
\usepackage{authblk}
\definecolor{deepyellow}{RGB}{0,0,128}

\usepackage{multirow}

\newcommand{\lbr}{\left [}
\newcommand{\rbr}{\right ]}
\newcommand{\lpa}{\left (}
\newcommand{\rpa}{\right )}

\newcommand{\bx}{\mathbf{x}}
\newcommand{\ba}{\mathbf{a}}
\newcommand{\bg}{\mathbf{g}}
\newcommand{\B}{\mathcal{B}}
\newcommand{\G}{\mathcal{G}}
\newcommand{\D}{\mathcal{D}}

% \title{Learning Robust Visuomotor manipulation with Internal Model}
% \title{Robust Video-Generated Internal Model for Visual Robot Manipulation}
\title{GEVRM: Goal-Expressive Video Generation Model For Robust Visual Manipulation}

% Authors must not appear in the submitted version. They should be hidden
% as long as the \iclrfinalcopy macro remains commented out below.
% Non-anonymous submissions will be rejected without review.
\author{
  Hongyin Zhang$^{1,2}$ \quad
  Pengxiang Ding$^{1,2}$ \quad
  Shangke Lyu$^{2}$  \quad
  Ying Peng$^{2}$  \quad
  Donglin Wang$^{2}$\thanks{Corresponding author.} \\
  $^{1}$ Zhejiang University. $^{2}$ Westlake University. \\
  \texttt{wangdonglin@westlake.edu.cn}
}
% \author[1]{}
% \author[2]{}
% \author[2]{}
% \author[2]{}
% \author[23]{}
% \affil[1]{}
% \affil[2]{}
% \affil[3]{\footnote{Corresponding Author.} wangdonglin@westlake.edu.cn}
% \author{Hongyin Zhang  \\
% % \thanks{ Use footnote for providing further information
% % about author (webpage, alternative address)---\emph{not} for acknowledging
% % funding agencies.  Funding acknowledgements go at the end of the paper.} \\
% % School of Engineering\\
% Zhejiang University\\
% Hangzhou, Zhejiang 310000, China \\
% % \texttt{\{wangdonglin\}@westlake.edu.cn} \\
% \And
% Pengxiang Ding, Shangke Lyu, Ying Peng, Donglin Wang \\
% % School of Engineering\\
% Westlake University\\
% Hangzhou, Zhejiang 310000, China \\
% \texttt{\{wangdonglin\}@westlake.edu.cn} \\
% \AND
% Coauthor \\
% Affiliation \\
% Address \\
% \texttt{email}
% }

% The \author macro works with any number of authors. There are two commands
% used to separate the names and addresses of multiple authors: \And and \AND.
%
% Using \And between authors leaves it to \LaTeX{} to determine where to break
% the lines. Using \AND forces a linebreak at that point. So, if \LaTeX{}
% puts 3 of 4 authors names on the first line, and the last on the second
% line, try using \AND instead of \And before the third author name.

\newcommand{\fix}{\marginpar{FIX}}
\newcommand{\new}{\marginpar{NEW}}

\iclrfinalcopy % Uncomment for camera-ready version, but NOT for submission.
\begin{document}


\maketitle

\begin{abstract}
  Game theory establishes a fundamental framework for analyzing strategic interactions among rational decision-makers. The rapid advancement of large language models (LLMs) has sparked extensive research exploring the intersection of these two fields. Specifically, game-theoretic methods are being applied to evaluate and enhance LLM capabilities, while LLMs themselves are reshaping classic game models. This paper presents a comprehensive survey of the intersection of these fields, exploring a bidirectional relationship from three perspectives: (1) Establishing standardized game-based benchmarks for evaluating LLM behavior; (2) Leveraging game-theoretic methods to improve LLM performance through algorithmic innovations; (3) Characterizing the societal impacts of LLMs through game modeling. Among these three aspects, we also highlight how the equilibrium analysis for traditional game models is impacted by LLMs' advanced language understanding, which in turn extends the study of game theory. Finally, we identify key challenges and future research directions, assessing their feasibility based on the current state of the field. By bridging theoretical rigor with emerging AI capabilities, this survey aims to foster interdisciplinary collaboration and drive progress in this evolving research area. 
    % By synthesizing insights from computational game theory and contemporary AI research, this work aims to stimulate interdisciplinary collaboration and inform the development of robust frameworks for AI-driven strategic decision-making.
\end{abstract}


% Game theory have been employed to boost development of large language models (LLMs) in both theoretical and technical fields. In the mean time LLMs as new game subjects have been put into game scenarios to play and be analyzed. The promotion of games and large language models (LLMs) is bidirectional. Recent studies analyze LLMs in game with numerous dimensions, including evaluation of LLMs' behavioral performance LLMs struggle in matrix games, methods to enhance LLMs' game performance, and how LLMs can serve beyond as a game player. In parallel, game theory, known for its advantages in addressing complex equilibrium problems, game-theoretic issues, and the integration of diverse perspectives, offers a promising guidance for phenomenological understanding LLMs and stimulaing LLM algorithms. More than that, with the deepening of interaction of LLMs and game, there are also original game models that are born LLM related. In this survey, we aim to comprehensively assess the ralationship between currect game and LLMs development. Besides, we propose a new taxonomy of game for LLMs and LLMs for game to systematically categorize related works in this emerging field. Our analysis includes novel frameworks and definitions, highlighting potential research directions and challenges at this intersection. Through this study, we aim to stimulate targeted advancements with game theory and LLM together.
\section{Introduction}
The pursuit of robust and adaptable robotic systems is the cornerstone of embodied general intelligence.
Recently, with the successful advancement of large-scale robot data collection~\citep{vuong2023open}, universal state representation learning~\citep{li2023vision,du2024learning}, and expressive policy learning~\citep{brohan2022rt,chi2023diffusion}, the research on vision-language-action (VLA) models for robots has made significant progress. 
The above strategies have been shown to be effective in estimating the robot's state and generating robust actions in a variety of environments, from physical simulators~\citep{mu2021maniskill,ding2023quar} to carefully designed real-world environments. 
However, these carefully designed environments do not take into account the inevitable external perturbations during deployment, such as fluctuating lighting conditions or video stream noise due to signal transmission problems. 
When VLA models are deployed in these non-ideal environments, external perturbations will bring unpredictable state information to the robot. 
This makes VLA produce fragile and unstable actions in inaccurate environmental states, resulting in a significant decrease in its generalization performance.
Therefore, enhancing the robustness of VLA models to cope with the inevitable external perturbations when deployed is an ongoing challenge. 

In the fields of computer vision~\citep{simard2003best,cirecsan2011high,ciregan2012multi,chen2020simple} and reinforcement learning~\citep{laskin2020reinforcement,hansen2021stabilizing,zheng2023stabilizing}, image augmentation is a common technique to alleviate the problem of model over-fitting, resist input image perturbations, and enhance model robustness. 
The idea is to apply task label-invariant transformations to the model's input images. 
For example, for object recognition tasks, image flipping and rotation do not change the semantic labels. 
Therefore, this technique has also been applied to robot visual language manipulation tasks. 
Some previous work has utilized vision as a general medium to develop specific agents that can plan various tasks through imagination and execution ~\citep{black2023zero,Yang2023LearningIR,du2024learning}. 
These methods involve generative models for predicting future videos or target images, followed by goal-conditioned policies that transform visual plans into actual actions. 
Image augmentation technology is utilized when training goal-conditioned policies, which to some extent alleviates the policy's over-fitting of specific tasks. 
% However, the expressive power of the future goal image (or video) state generated by these models is limited, and image augmentation can only generalize the model within a narrow task distribution.
However, these models are limited by their generative capabilities, the future goal image (or video) states they generate are not expressive enough, and image augmentation only allows the model to generalize within a narrow task distribution.
% The model as a whole lacks strong adaptability to environmental perturbations, and it struggles to generate robust and stable actions on a wide task distribution.
It lacks strong resilience to environmental perturbations and struggles to produce actions that are consistently effective across diverse task scenarios.

\begin{figure}[t]
\vspace{-2em}
\centering
\includegraphics[width=0.9\textwidth]{ICLR_2025_Template/figures/motivation.pdf}
% \includegraphics[width=0.5\textwidth,height=0.28\textwidth]{LaTeX/pictures/data_dis_visua.png}
\caption{
We are inspired by the classical internal model control (\textbf{a}) in automation systems. 
The principle illustrates that a closed-loop system equipped with an internal model that accounts for external input signals can precisely follow the reference input and effectively neutralize the perturbations.
In this work, an internal model visuomotor control framework (\textbf{b}) is motivated and designed.
We leverages a text-guided video model for generating highly expressive visual goal states as reference input, goal-state and current-state internal encoders for modeling responses, and a goal-guided policy for robust action generation.
} 
\vspace{-2em}
  \label{fig:IMC_motivation}
\end{figure}
We are inspired by the principle of classical internal model control (IMC) shown in Fig.\ref{fig:IMC_motivation} (\textbf{a}). 
The core idea of this principle~\citep{rivera1986internal} is that in a closed-loop control system, by building a model inside the controller that can simulate external perturbations and reference inputs, the desired output can be accurately tracked and the perturbations can be effectively offset. 
That is, it leverages an internal model to replicate the system's behavior and subsequently assess the system's perturbations, thereby augmenting the closed-loop stability.  It is widely believed that intelligent mammals also rely on internal models to generate their actions ~\citep{nguyen2011model} and such mechanism is also revealed and supported by behavioral, neurophysiological, and imaging data~\citep{kawato1999internal}. More importantly, the integration of the internal model into the robot control system ~\citep{emken2005robot} has been verified to enhance the robustness of the robot motion control. 
However, the results are limited to specific scenarios and hard to extend to more complex and general tasks, such as visual-language manipulation. 
How to instantiate the internal model in the VLA framework to improve the robustness of decision actions has not been explored.


To this end, we propose \textbf{GEVRM}, a \textbf{G}oal-\textbf{E}xpressive \textbf{V}ideo Generation Model for \textbf{R}obust Visual \textbf{M}anipulation.
As shown in Fig.\ref{fig:IMC_motivation} (\textbf{b}), to effectively implement the classic IMC principle in the VLA model, some components of our method are adjusted accordingly. 
\textbf{1) Goal generation.} Taking video frames as a universal interface to describe the robot state, we introduce an advanced text-guided video diffusion generation model as a robot behavior planner to generate future goal frames as reference input. 
To improve the expressiveness of future goal states, we train the visual planner through efficient video spatiotemporal compression and random mask strategies to prioritize the understanding of physical world laws~\citep{phyworld}.
\textbf{2) State alignment.} We estimate system perturbations by leveraging the simulated responses of the robot. 
These responses are called internal embeddings and are extracted from the robot state. 
Since the responses are inherently embedded in the robot's historical observations, the internal embeddings can be optimized through prototypical contrastive learning~\citep{caron2020unsupervised,yarats2021reinforcement,deng2022dreamerpro} to align the robot's future expressive goal states with its current state. 
This enables the model to implicitly infer and distinguish perturbations from the external environment.
\textbf{3) Goal-guided policy.} We propose a diffusion policy conditioned on the generated highly expressive goals to better model the multi-modal task distribution of robot manipulation~\citep{chi2023diffusion}. 
This policy and the aforementioned internal embedding are jointly optimized through inverse dynamics and contrastive learning objectives to track highly expressive goals well even in the presence of perturbations. 
In summary, our contributions are threefold:
\begin{itemize}
\item We introduce GEVRM, a novel robust VLA model that incorporates the IMC principle to enhance robot visual manipulation.
\item We study how to obtain highly expressive goals with a text-guided video generation model and align state representations through prototypical contrastive learning to resist external perturbations at deployment.
\item Extensive experiments verify the effectiveness and advancement of the proposed GEVRM. 
It significantly outperforms the previous state-of-the-art on the CALVIN benchmark with standard and external perturbations. 
The expressiveness of the goal states generated in real visual manipulation is significantly improved compared to previous baseline methods.
\end{itemize}


\section{RELATED WORK}


\noindent\textbf{Vision-Language-Action models.} 
With the rise of extensive multi-task robotic datasets~\citep{vuong2023open}, the robotics community is increasingly focusing on multi-task execution capabilities. The Vision-Language-Action models~\citep{brohan2022rt,yue2024deer} have gained traction for their ability to use language for goal commands, enabling robots to make informed decisions based on visual perceptions. Early studies~\citep{brohan2022rt, Wu2023UnleashingLV} utilized cross-modal attention between language and vision, but limited model performance hindered effectiveness. Recently, attention has shifted to large foundational models~\citep{alayrac2022flamingo,li2023vision,kim2024openvla}, for improved versatility. However, text descriptions often lack detail about environmental states, complicating cross-morphology tasks. As a result, some researches~\citep{du2024learning,ko2023learning,zhou2024robodreamer,black2023zero,ajay2024compositional,yang2023learning} now leverage vision as a universal medium, employing generative models to forecast future actions, followed by goal-conditioned policies for execution.
UniPi~\citep{du2024learning} was one of the first to leverage internet-scale data to train a text-conditioned video generator, using an inverse dynamics model to estimate actions. Similarly,  SuSIE~\citep{black2023zero} uses an image-editing model to plan high-level sub-goals for low-level controllers, while ADVC~\citep{ko2023learning} infers actions from predicted video content with dense correspondences.
These efforts aim for a universal state representation but fall short for two reasons. First, existing visual plans experience temporal and spatial inconsistencies due to poor dynamics modeling. We propose a robust video generation model that addresses this issue and enhances action execution. Second, prior work focuses on controlled environments, overlooking the robot's responses to external interference. Our GEVRM method employs contrastive learning for state alignment, effectively simulating responses and resisting disturbances. Together, these elements define our expressive goal representation.


\noindent\textbf{Internal Model Control framework.} 
The IMC framework is a widely recognized control strategy that leverages an internal model of the system to predict future behavior and adjust control actions accordingly, making it highly robust against disturbances and model inaccuracies. First introduced by Garcia and Morari (1982), IMC has been applied extensively in both linear and nonlinear process control, offering significant benefits in terms of stability and adaptability~\citep{garcia1982internal,rivera1986internal,morari1989robust}. Its feedback mechanism allows for real-time adjustments, particularly valuable in dynamic environments such as robotics, where precision is critical. IMC’s design has been further explored and refined for multivariable and complex systems, proving its versatility and robustness in various control applications~\citep{skogestad2005multivariable}.
However, most previous research works are limited to specific control scenarios and are difficult to extend to general visual language manipulation tasks.
More recently, inspired by classical closed-loop control systems, a closed-loop visuomotor control framework~\citep{bu2024closed} has been proposed that incorporates feedback mechanisms to improve adaptive robot control.
Different from these works, we study how to effectively instantiate internal models in the VLA framework to improve the robustness of decision actions.


\section{IMPACTX framework}
\label{sec:method}
\subsection{Adopted assumptions and notation}
In its simplest form, a typical ML classification system $\mathcal{A}$ can be usually viewed as the composition of two main components: a feature extractor $M$ (for example, in a feed forward DNN model, it usually corresponds to the DNN first layers) and a classification component $Q$ (usually corresponding, in a feed forward model, to the remaining part of the classification process) s.t. $\hat{y}=\mathcal{A}(\vec{x})=Q\big(M(\vec{x})\big) \in \{1,\dots K\}$ is the estimated class of the input $\vec{x}\in \mathbb{R}^d$. Without losing in generality, in this paper, we consider $Q$ as just the final function to compute the inferred class from a given vectors of $K$ scores, i.e. $Q(\vec{m})=\arg\max\big(\text{softmax}(\vec{m})\big)$ with $\vec{m}\in \mathbb{R}^K$. After a proper learning procedure, $M$ is a model able to represent a given $\vec{x}$ in a $K$-component array where the $k$-th component with $1\leq k \leq K$ represents a score of $\vec{x}$ to belong to the $k$-th class. An example of $M$ can be all the layers of a DNN before the final softmax function. However, in general every model able to project $\vec{x}$ in a given feature space can be adopted as $M$ (such as a sequence of layers of a DNN able to extract features from an input $\vec{x}$).
We define with $y \in \{1,\dots,K\}$ a scalar representing the correct label of an input $\vec{x}$  in a $K$-class classification problem. Let $S^T$ be a training dataset of $N$ labeled instances, i.e. $S^T = \{(\vec{x}^{(i)}, y^{(i)})\}_{i=1}^N$, and $S^E$ a test dataset composed of $J$ instances used only to evaluate an ML model, i.e. $S^E =\{ (\vec{x}^{(j)}, y^{(j)}) \}_{j=1}^J$. % our objective is to predict the correct class using a neural network $M$ trained on $S^T$.


\subsection{General description}
IMPACTX is a double-branch architecture such that, when applied to a classifier $\mathcal{A}$, the resulting architecture outperforms the standalone $\mathcal{A}(\vec{x})=Q\big(M(\vec{x})\big)$), both appropriately trained. In addition, this enhanced architecture  also provides input attribution maps relative to the output obtained.

IMPACTX framework is composed of two branches that interact each other (see figure \ref{fig:IMPACTX_arch}): 
\begin{enumerate}
    \item The first branch (on the top) is composed of a Feature Extractor $M$, able to extract significant features from the input with the goal to classify the input $\vec{x}$, and a $K$-class classifier $C$, able to return an estimated class $\hat{y}$ of the input $\vec{x}$. 
    \item The second branch (at the bottom)     \textcolor{black}{
     is responsible for the attention mechanism of IMPACTX. It}
 is composed of a Latent Explanation Predictor (LEP) module, able to extract essential information from the input features with the goal to compute an attribution map of the input respect to the classification response, and a Decoder $D$, able to effectively produce an attribution  map of the input $\vec{x}$. 
\end{enumerate}

Thus, the goal of IMPACTX is to build an estimated class $\hat{y}=C(\vec{m},\vec{z})$ exploiting both the $\vec{m}=M(\vec{x})$ and $\vec{z}=LEP(\vec{x})$ outputs, and at the same time to obtain a predicted attribution map $\hat{\vec{r}}$ with respect to $\hat{y}$. 

%As we discuss in detail in Section \ref{sec:training}, during the inference phase the top-branch is influenced by the bottom branch, while the bottom branch is used to the during the training phase the bottom branch is influenced 
In the next section we will discuss in detail how IMPACTX can be trained to obtain this goal. 

%the LEP module extracts relevant information from the input. This is made exploiting during the training both inputs and explanations generated by a specific  method (see sec. \ref{} for further details). The output $\vec{z} = LEP (\vec{x})$, named 'latent attribution encoding', wants to capture essential information about the attribution of the input features $\vec{x}$ to the true class $y$. We assume that $\vec{z}$, when combined with the output $\vec{m}$ generated by $M(\vec{x})$, can effectively helps to classify the input $\vec{x}$ through the simple classifier $C$ properly trained. 


%IMPACTX framework is composed of four main components (see Fig. \ref{}): i) a feature extractor $M$, ii) a $LEP$ (Latent Explanation Predictor) module able to extract essential information from the input features, iii) a final $K$-class classifier $C$ able to provide the final estimated class, and iv) a Decoder $D$ able to product an attribution  map of the input $\vec{x}$. 


%Then,  Without loss of generalizabily, we can consider $M$ as a feature extractor, i.e. the   $\hat{y}_M=\arg\max\big({softmax}\big(\vec{m}\big)\big)$ is the estimated label of the sample $\vec{x}$. 



%Since our goal is to obtain attribution maps aligned with the correct label, we adopt the XAI SHAP method \citep{NIPS2017_7062} as $R$. Indeed, SHAP reveals the impact of different features on model predictions, offering a deeper understanding of the decision-making process. In particular, SHAP provides explanations not only for the predicted class, but also for each possible class \roberto{Questa e' una caratteristica solo di SHAP o di altri metodi XAI??} \sal{anche di altri}, which we denote as $SHAP(M, \vec{x})=\{\vec{r}_1, \vec{r}_2,\dots, \vec{r}_C\}$, where $C$ represents the number of classes involved in the classification problem addressed by $M$. Therefore, the attribution map corresponding to the true class label provided in the training data results $\vec{r}^{(i)}=\vec{r}_{y^{(i)}} \in SHAP(M, \vec{x}^{(i)})$.


\begin{figure*}[ht]
\centering
\includegraphics[width=1\textwidth]{./IMPACTX_ARK.png} 
\caption{An overview of the IMPACTX framework. In the training phase of IMPACTX, both $M$ and $LEP$ receive $\vec{x}$, generating $\vec{m}$ and $\vec{z}$ respectively. These are combined for classification by $C$. In particular, $LEP$ and $D$ exploit the $R\big(\mathcal{A}(\vec{x})\big),\vec{x}, y)$ explanations. The architecture is trained using a loss function that merges MSE and CE to optimize explanation reconstruction and improve classification performance. In the inference step of IMPACTX, $C(\vec{m}, \vec{z})$ predicts the class $\hat{y}$ and $LEP-D(\vec{x})$ reconstructs the explanation $\vec{r}$ of the input $\vec{x}$. 
%In the training phase of IMPACTX, both $M$ and $LEP$ receive $\vec{x}^{(i)}$, generating $\vec{m}^{(i)}$ and $\vec{z}^{(i)}$ respectively. These are combined for classification by $C$. While the weights of $M$ remain frozen, $LEP$ and $D$ exploit the $\vec{r}^{(i)}$ explanations to enhance $M$'s classification. The architecture is trained using a loss function that merges MSE and CE to optimize explanation reconstruction and improve classification performance.
}
\label{fig:IMPACTX_arch}
\end{figure*}
%Summarizing, the IMPACTX training framework comprises several key components:  (i) the Latent Explanation Predictor $LEP$ and a Decoder $D$ are trained to extract significant hidden information $\vec{z}^{(i)}$ related to the explanations and reconstructing the original attribution map  $\vec{r}^{(i)}$ from its encoding $\vec{z}^{(i)}$, and (ii) the final classifier $C$ designed to exploit $\vec{m}^{(i)}$ and $\vec{z}^{(i)}$ to predict the effective class $y^{(i)}$. Both $M$ and $LEP$ receive the input $\vec{x}^{(i)}$. The resulting outputs, $M(\vec{x}^{(i)})$ and $LEP(\vec{x}^{(i)})$, are concatenated and passed to $C$.


%\input{_alg}


\subsection{Training IMPACTX}
\label{sec:training}
The training phase of the IMPACTX approach is depicted in figure \ref{fig:IMPACTX_arch}.
%\textcolor{red}{ and outlined in the pseudocode presented in Algorithm  \ref{algo:method_main}}
Both $M$ and $LEP$ receive $\vec{x}$ as input, producing the corresponding outputs $\vec{m}$ and $\vec{z}$. These outputs are concatenated and forwarded to the classifier $C$. Additionally, $\vec{z}$ is decoded by the decoder $D$, responsible for reconstructing the explanation $\vec{r}$. In other words, $LEP$ and $D$ act as an Encoder-Decoder which is constrained to learn an encoding of the explanations by the internal variables $\vec{z}$. $C$ leverages the combined knowledge of $M$  and $LEP$. Therefore, for a new data point $\vec{x}^{(j)}$, the  estimated output $\hat{y}^{(j)}$ is defined as: $$\hat{y}^{(j)} = \arg\max\bigg( \text{softmax} \Big(C\big(M(\vec{x}^{(j)}), LEP(\vec{x}^{(j)})\big)\Big)\bigg).$$ %Assuming that $M$ can be any pre-trained neural network, learning the proposed framework requires the training of Predictor $P$, Decoder $D$, and classifier $C$ without utilizing unlabeled data $S^E$.
\\Consequently, IMPACTX wants to solve the classification task and, at the same time, to construct a predicted attribution map $\vec{r}$ using the decoder $D$ and the $LEP$ module. 

Importantly, the effective IMPACTX training can be made in at least two ways:

- \textit{Single-stage training}: All IMPACTX modules are trained simultaneously on $S^T$, with the attribution maps $\vec{r}^{(i)}$ for each sample in $S^T$ generated at the end of each training iteration. \textcolor{black}{In this case, classification performance can be improved by using increasingly accurate attribution maps, $\vec{r}^{(i)}$. Ideally, this creates a positive feedback loop where both the classification and the generation of attribution maps are mutually enhanced. Conversely, computing the attribution maps at each iteration can be computationally expensive and, at the same time, the generated attribution maps $\vec{r}^{(i)}$ may have very poor significance in the early learning epochs since $M$, $LEP$ and $D$ weights are initialised to random values. %This situation could also lead to the worst results in the $C$ classifier and in some cases, without special precautions, also in the last learning epochs. 
%Instead, the advantage of this strategy is obviously a single-stage training for all IMPACTX modules. In the optimal case, there's the possibility to improve the classification performance by using better and better attribution maps $\vec{r}^{(i)}$ and in an ideal loop, this enhances both the classification and the generation of attribution maps.
}

- \textit{Two-stage training}:  This training approach is divided into two stages. In the first stage, the parameters of the whole classifier  $\mathcal{A}(\cdot)$ are trained and initially evaluated on $S^T$. Then, at the end of the first training stage, the attribution maps $\vec{r}^{(i)}$ for each sample in $S^T$ are produced with respect the true class label. In the second training stage, the remaining modules $C$, $D$, and $LEP$ are trained while keeping $M$ frozen. In particular, the targets of the branch $LEP$-$D$ are the corresponding attribution maps previously computed at the end of the first stage.  Since the attribution maps are computed just one time, two-stage training results less expensive than single-stage training. 
%\textcolor{red}{The disadvantage of this approach is that only one generation of attribution maps $\vec{r}^{(i)}$ is used. This could limit the performance gain.}\textcolor{blue}{Andrea: ???} \sal{vorrei dire, che rispetto all'altro tipo di addestramento, qui vengono generate una sola volta le spiegazioni e per questo motivo (nel caso ideale) l'incremento di performance potrebbe essere inferiore. Se non va bene, per ogni finalità, si può anche togliere}

In both the training approaches, the architecture is trained using a loss function that combines together Mean Squared Error (MSE) between the true class attribution map $\vec{r}^{(i)}$ (see sec. \ref{sec:method_genR}) and the output of the $LEP$-$D$ branch, and the Cross Entropy (CE) loss between the true class label $y^{(i)}$ and the prediction from $C$. The resulting loss function can be formalized as follows: 
\begin{equation}
    L = CE\big(y^{(i)}, C(\vec{m}^{(i)}, \vec{z}^{(i)})\big) + \lambda \cdot MSE\big(\vec{r}^{(i)}, LEP-D(\vec{x}^{(i)})\big)
\label{eq:loss}
\end{equation} where $\lambda$ represents a regularization parameter. This approach lead $\vec{z}^{(i)}$ to be optimized respect to the attribution reconstruction error while maintaining robust classification performance simultaneously.



\subsection{Generating the attribution-based explanations} 
\label{sec:method_genR}
We use an XAI attribution method $R$ to generate attribution maps $\vec{r}$ on $\vec{x}$ about the true class label $y$ based on $\mathcal{A}(\vec{x})$. In particular, for each available training data $\vec{x}^{(i)}$ in $S^T$, an attribution map $\vec{r}^{(i)}$ corresponding to the true class label $y^{(i)}$ when $\vec{x}^{(i)}$ is the input of $\mathcal{A}(\vec{x}^{(i)})=Q\big(M(\vec{x}^{(i)})\big)$, is produced adopting $R$. 
The objective is to get an attribution map aligned with the true class label $y^{(i)}$ for each training data $\vec{x}^{(i)}$. Therefore, we adopt as $R$ an XAI method that provides explanations not only for the predicted class, but also for each possible class, which we denote as $R\big(\mathcal{A}, \vec{x}, k \big)=\vec{r}_k$, where $k$ represents the class for which we need the explanation. Therefore, the attribution map corresponding to the true class label provided in the training data results $\vec{r}^{(i)}=R\big(\mathcal{A}(\vec{x}^{(i)}), \vec{x}^{(i)}, y^{(i)}\big)$.
\vspace{-2em}
\section{experimental evaluation}
 \vspace{-0.5em}
% In this section, we evaluate the GEVRM in terms of its ability to enable robust visual manipulation.  
In this section, we evaluate the state generation and visual manipulation capabilities of GEVRM.
%
To this end, our experiments aim to investigate the following questions:
%
\textbf{1)} Can GEVRM have strong generalization ability to generate expressive goal in various environments?
\textbf{2)} Does GEVRM exhibit a higher success rate in executing robot tasks compared to the baseline in various environments?
\textbf{3)} How important are the core components of the GEVRM for achieving robust decision action?

 \vspace{-0.5em}
\subsection{Evaluation on Goal Generation}
\label{sec: exp_video_generation}

% In this section, we conduct a thorough assessment of GEVRM's robustness generalization capabilities across two dimensions: previously unseen environments and perturbed environments.



\textbf{Setup.} 
We utilized two types of datasets (realistic Bridge~\citep{walke2023bridgedata} and simulated CALVIN~\citep{mees2022calvin}) to evaluate the generalization of goal generation. 
% Both datasets consist of various robotic manipulation tasks, such as placing a potato on a plate. 
% We train the model on pre-defined training sets and tested on unseen test sets to assess the generation performance.
% We also evaluate its robustness in goal generation by introducing external perturbations. 
% We train the model on a predefined training set and test it on the test set with or without external interference to evaluate the robot gaol generation performance.
We train the model on a predefined training set and evaluate the robot goal generation performance on a test set with and without external perturbations.
The hyperparameters are shown in Appendix Tab.~\ref{Appd:Behavior planner training optimizer hyperparameters} and Tab.~\ref{Appd:Behavior planner test hyperparameters}.

\textbf{Baselines.} 
To make a fair comparison, we have chosen open-source video generative models: 
1) AVDC~\citep{ko2023learning}, a typical diffusion-style generation model for robotics.
2) GR-1~\citep{Wu2023UnleashingLV}, which is an autoregressive-style generation model that takes language instructions, and state sequences as inputs, and predicts robot actions and future images in an end-to-end manner. 
3) SuSIE~\citep{black2023zero}, uses the image editing diffusion model as a high-level planner and proposes intermediate sub-goals that can be achieved by the low-level controller.
% Given that more methods are not open-source or belong to the category of image-level generative paradigm, they are not included in this comparison. 


\textbf{Metrics.} 
The evaluation metrics employed are the Frechet Inception Distance (FID) \citep{Seitzer2020FID} and the Frechet Video Distance (FVD) \citep{stylegan_v,digan}, both widely recognized in the domains of image and video generation.
% We also test other standard reconstruction metrics: Structural Similarity Index (SSIM), Peak Signal-to-Noise Ratio (PSNR), Learned Perceptual Image Patch Similarity (LPIPS) following~\citep{bu2024closed}.
We also evaluate the quality of videos generated by different models on other standard metrics~\citep{bu2024closed}: Structural Similarity Index (SSIM), Peak Signal-to-Noise Ratio (PSNR), Learned Perceptual Image Patch Similarity (LPIPS).
% These metrics are utilized to assess the quality of the videos anticipated to be generated by the different video generation models. 
% The lower the values of these two metrics, the higher the quality of the video produced by the model, indicating a closer resemblance to real video footage.




% \begin{wraptable}{r}{0.4\textwidth}\small
%     \centering
%     \vspace{-0.5em}
%     \caption{{Goal generation quality comparison}.
%         Here, ``B'' and ``C'' mean Bridge and CALVIN datasets. 
%         % Our method greatly surpasses the baseline across all metrics. 
%     % The best results for each task are bolded. 
%     }
%     \begin{tabular}{ccc}
%      \toprule
%         Algorithms & \textbf{FID} ($\downarrow$) & \textbf{FVD} ($\downarrow$)  \\ 
%          \midrule
%          AVDC (B) & 246.45$\pm${\scriptsize 39.08} & 22.89$\pm${\scriptsize 4.99}  \\ 
%         Ours (B) & \textbf{35.70}$\pm${\scriptsize 10.77} & \textbf{4.16}$\pm${\scriptsize 1.35}  \\ 
%         % Variation Rate (B) & 60.1\% & 70.4\%  \\
%         \midrule
%         GR-1 (C) & 236.75$\pm${\scriptsize 38.87} & 12.83$\pm${\scriptsize 2.6} \\ 
%         Ours (C) & \textbf{94.47}$\pm${\scriptsize 22.54} & \textbf{3.8}$\pm${\scriptsize 1.2} \\ 
%         % Variation Rate (C) & \% & \%  \\
%          \bottomrule
%     \end{tabular}
%        \label{tab:calvin_FID_FVD}
% \end{wraptable}
\begin{table}\small
    \centering
    \caption{Goal generation quality comparison.
        Our method greatly surpasses the baseline across all metrics. 
    The best results for each task are bolded. 
    }
    \vspace{1em}
    \begin{tabular}{ccccccc}
     \toprule
        \textbf{Benchmark} & \textbf{Algorithms} & \textbf{FID} ($\downarrow$) & \textbf{FVD} ($\downarrow$) & \textbf{LPIPS} ($\downarrow$)& \textbf{SSIM} ($\uparrow$)& \textbf{PSNR} ($\uparrow$) \\ 
         \midrule
         BridgeData& AVDC  & 246.45$\pm${\scriptsize 39.08} & 22.89$\pm${\scriptsize 4.99} &0.23$\pm${\scriptsize 0.03}&  0.73$\pm${\scriptsize 0.05}&18.22$\pm${\scriptsize 2.53} \\ 
        BridgeData & SuSIE & 114.79$\pm${\scriptsize 21.38} & --               & 0.22 $\pm${\scriptsize 0.08} & 0.71$\pm${\scriptsize 0.07} & 16.39$\pm${\scriptsize 2.90} \\
        BridgeData&GEVRM (Ours) & \textbf{35.70}$\pm${\scriptsize 10.77}   & \textbf{4.16}$\pm${\scriptsize 1.35}&\textbf{0.06}$\pm${\scriptsize 0.03}&\textbf{0.89}$\pm${\scriptsize 0.04}& \textbf{22.36}$\pm${\scriptsize 2.75}\\ 
        % Variation Rate (B) & 60.1\% & 70.4\%  \\
        \midrule
        CALVIN&GR-1  & 236.75$\pm${\scriptsize 38.87} &12.83$\pm${\scriptsize 2.60} &0.20$\pm${\scriptsize 0.02}&0.65$\pm${\scriptsize 0.03}&18.59$\pm${\scriptsize 0.95}\\ 
        CALVIN& SuSIE & 214.14$\pm${\scriptsize 45.45} & --               & 0.15$\pm${\scriptsize 0.04} & 0.75$\pm${\scriptsize 0.05} & 18.12$\pm${\scriptsize 2.29} \\
        CALVIN&GEVRM (Ours) & \textbf{94.47}$\pm${\scriptsize 22.54} & \textbf{3.80}$\pm${\scriptsize 1.2}&\textbf{0.09}$\pm${\scriptsize 0.04}&\textbf{0.80}$\pm${\scriptsize 0.05}&\textbf{21.10}$\pm${\scriptsize 3.29} \\ 
        % Variation Rate (C) & \% & \%  \\
         \bottomrule
    \end{tabular}
       \label{tab:calvin_FID_FVD}
\end{table}
\begin{figure}
\centering
% \includegraphics[width=0.5\textwidth,height=0.28\textwidth]{LaTeX/pictures/data_dis_visua.png}
\includegraphics[width=0.9\textwidth]{ICLR_2025_Template/figures/Visulization.pdf}
\caption{
Comparison of goal generation on task “\textit{put blueberry in pot or pan on stove}”.
}
  \label{fig:Visualization of Bridge Data}
\end{figure}

\textbf{Goal generation comparison.}
We evaluate the generalization of the goal generation in the unseen environment (Tab.~\ref{tab:calvin_FID_FVD}).
% The results in Tab.~\ref{tab:calvin_FID_FVD} reveal a significant enhancement in the model's performance.
The results show that the performance of our GEVRM model is significantly improved compared to the baseline.
% In terms of FID and FVD metrics, compared with the baseline GR-1 scores of $236.75$ and $12.83$, our proposed method is significantly lower than theirs, with only $94.47$ and $3.8$.  
The results indicate that GEVRM exhibits enhanced expressive capabilities, effectively modeling the intricate textures and temporal coherence of robotic image sequences.
Then, we compare the robustness of goal generation in the perturbed environment (Fig.~\ref{fig:Visualization of Bridge Data}).
The baselines struggle with environmental variations, generating severe hallucinations that distort objects and may even completely ruin the scene.
In contrast, our method produces fewer hallucinations and can generate expressive goal states following language instructions.
This confirms that GEVRM is indeed better able to understand the laws of the physical world and maintain the 3D consistency of objects.
% For more goal generation results, see Appendix Fig.~\ref{APP:fig:calvin_video_1},~\ref{APP:fig:calvin_video_2},~\ref{APP:fig:calvin_video_3} and ~\ref{APP:fig:calvin_video_4}.
More goal generation results are in Appendix Fig.~\ref{APP:fig:calvin_video_1}$\sim$\ref{APP:fig:real_video_3}.



\subsection{Evaluation on Action Execution}

% In this section, we delve into the practical viability of GEVRM for robotic manipulation tasks.

\textbf{Setup.} We conduct experiments on CALVIN, a benchmark for language-conditioned manipulation to evaluate the GEVRM's capabilities in closed-loop action execution.
CALVIN consists of four simulated environments (A, B, C, and D), each with a dataset of human-collected play trajectories. 
We study zero-shot multi-environment training on A, B, and C, and testing on D, varying in table texture, furniture positioning, and color patches. We also test GEVRM's robustness to perturbations (Fig.~\ref{fig:calvinTask}).
Details of environmental disturbances are given in Appendix Section~\ref{Appe:Environment perturbations.}.
The policy training hyperparameters are shown in Appendix Tab.~\ref{Appd:Goal-guided policy optimizer Hyperparameters}.


\textbf{Baselines.} We select the representative baselines to verify the generalization performance on standard unseen environments: 
1) UniPi~\citep{du2024learning}: Recasts decision-making into text-conditioned video generation firstly, enabling the production of predictive video sequences and subsequent extraction of control actions.
2) HiP~\citep{ajay2024compositional}: 
% Advances upon UniPi by incorporating hierarchical inference for extended long-term planning capabilities.
This model improves upon UniPi by incorporating hierarchical inference to extend long-term planning capabilities.
3) GR-1~\citep{Wu2023UnleashingLV}:
This model leverages a pre-trained video model to enhance autoregressive action generation.
4) RoboFlamingo~\citep{li2023vision}, uses a pre-trained VLM for single-step visual language understanding and models sequential history information with an explicit policy head.
In addition, in the test environment with external perturbations, we choose the representative baseline SuSIE~\citep{black2023zero}, because it adopts common data augmentation strategies to cope with perturbations and achieves state-of-the-art results in previous works.
We consider third-view RGB images from static cameras as observations, which makes the robot execution more challenging.
More comparison with language-conditioned methods are in Appendix Sec.~\ref{Appe:Baseline method introductions.} and Tab.~\ref{APP:tab:calvin_performance}.
 
\begin{wraptable}{r}{0.6\textwidth}\small
    \centering
    \vspace{-2.em}
    \caption{Generalization on unseen environments in CALVIN (train A, B, C → test D). *: reproduced version training on third-view images.
    }
    \vspace{0.5em}
    \begin{tabular}{lccccc}
        \toprule
        \multirow{2}{*}{\textbf{Algorithms}} & \multicolumn{5}{c}{\textbf{No. of Instructions Chained}} \\
        \cmidrule(lr){2-6}
        & 1 & 2 & 3 & 4 & 5 \\
        \midrule
        HiP & 0.08 & 0.04 & 0.00 & 0.00 & 0.00 \\
        UniPi & 0.56 & 0.16 & 0.08 & 0.08 & 0.04 \\
        GR-1* & 0.75 & 0.45 & 0.2 & 0.15 & 0.10 \\
        SuSIE & 0.87 & 0.69 & 0.49 & 0.38 & \textbf{0.26} \\
        \midrule
        GEVRM (Ours) & \textbf{0.92} & \textbf{0.70} & \textbf{0.54} & \textbf{0.41} & \textbf{0.26} \\
        \bottomrule
    \end{tabular}
    \label{tab:calvin_performance}
\end{wraptable}

\begin{figure}[tbp]
\vspace{-2em}
\centering
\includegraphics[width=0.75\textwidth]{ICLR_2025_Template/figures/calvin_environment.pdf}
% \includegraphics[width=0.5\textwidth,height=0.28\textwidth]{LaTeX/pictures/data_dis_visua.png}
\caption{
% We compare the proposed method with baselines in the zero-shot setting of the CALVIN benchmark. 
The model is trained only on data collected in environments A, B, and C (\textbf{a}), and tested on environment D (\textbf{b}). 
Besides, we apply five perturbations to the image observations of environment D to further test the generalization of the model in more challenging scenarios (\textbf{c}).
} 
  \label{fig:calvinTask}
  \vspace{-1em}
\end{figure}

\textbf{Action execution comparison.}
We show the success rate of completing each language instruction in the chain in Tab.~\ref{tab:calvin_performance}. 
The model is trained on environments A, B, and C (Fig. \ref{fig:calvinTask} (\textbf{a})), and test in D (Fig. \ref{fig:calvinTask} (\textbf{b})).
Compared with the baseline, the GEVRM has a significant performance improvement. This shows that our method based on the IMC principle has better goal generation ability when facing new environments and induces the robot to predict more general decision actions.
% Compared with baseline methods, our proposed GEVRM achieves the best performance. 
% Compared with baselines of the video diffusion generation category, our method GEVRM has a significant performance improvement.
% Therefore, the proposed method, based on the IMC principle, has better robot goal generation ability when facing new environments and induces the robot to predict more general decision actions.
% This shows that our method has better robot behavior skill inference ability and environmental perception understanding ability, as well as more robust low-level actions. 
% Moreover, the previous state-of-the-art baseline method SuSIE only generates future image sub-goals, rather than future video trajectories. 
% This will cause the model to lack the understanding of the temporal consistency of the robot's behavior trajectory, which affects the robot's action execution accuracy.



\begin{table}[ht]\small
    \centering
    \vspace{-1.5em}
    \caption{Generalization on perturbed environments in CALVIN (train A, B, C → perturbed test D).
    % Our proposed method surpasses the previous state-of-the-art method SuSIE in terms of task success rate and task completion length on average.
     % The best results for each task are bolded.
     % The 'AATL' is short for 'Average Achieved Task Length'.
    }
    \vspace{0.5em}
    \renewcommand{\arraystretch}{0.8}
    \begin{tabular}{l c ccccc c}
        \toprule
        \multirow{2}{*}{\textbf{Five Perturbed Tasks}} & \multirow{2}{*}{\textbf{Algorithms}} & \multicolumn{5}{c}{\textbf{No. of Instructions Chained}} & \multirow{2}{*}{\textbf{Avg. Length ($\uparrow$)}} \\
        \cmidrule(lr){3-7}
        & & 1 & 2 & 3 & 4 & 5 & \\
        \midrule
        % \multirow{2}{*}{Image Shift} & SuSIE & \textbf{0.56} & 0.28 & 0.08 & \textbf{0.04} & 0.00 & 0.96 \\
        % & GEVRM (Ours)  & 0.52 & \textbf{0.40} & 0.08 & 0.00 & 0.00 & \textbf{1.00} \\
        % \midrule
        % \multirow{2}{*}{Image Rotation} & SuSIE & 0.48 & 0.16 & 0.08 & 0.00 & 0.00 & 0.72 \\
        % & GEVRM (Ours)   & \textbf{0.60} & \textbf{0.32} & \textbf{0.12} & \textbf{0.08} & \textbf{0.04} & \textbf{1.16} \\
        % \midrule
        % \multirow{2}{*}{Color Jitter} & SuSIE& \textbf{0.72} & 0.36 & 0.16 & 0.12 & 0.08 & 1.44 \\
        % & GEVRM (Ours)   & 0.64 & \textbf{0.48} & \textbf{0.32} & 0.12 & 0.08 & \textbf{1.64} \\
        % \midrule
        % \multirow{2}{*}{Image Occlusions} & SuSIE & 0.72 & 0.48 & 0.32 & \textbf{0.32} & \textbf{0.24} & 2.08 \\
        % & GEVRM (Ours)   & \textbf{0.92} & \textbf{0.68} & \textbf{0.48} & 0.24 & 0.20 & \textbf{2.52} \\
        % \midrule
        % \multirow{2}{*}{Noise Interference} & SuSIE& 0.32 & 0.04 & 0.00 & 0.00 & 0.00 & 0.36 \\
        % & GEVRM (Ours)   & \textbf{0.80} & \textbf{0.48} & \textbf{0.32} & \textbf{0.12} & \textbf{0.04} & \textbf{1.76} \\
        % \midrule
        % \multirow{2}{*}{Average} & SuSIE & 0.56 & 0.26 & 0.13 & 0.10 & 0.06 & 1.11 \\
        % & GEVRM (Ours)  & \textbf{0.70} & \textbf{0.47} & \textbf{0.26} & \textbf{0.11} & \textbf{0.07} & \textbf{1.62} \\

       \multirow{4}{*}{Average} & SuSIE & 0.56 & 0.26 & 0.13 & 0.10 & 0.06 & 1.11 \\
         & RoboFlamingo & 0.63&0.35&0.18&0.09&0.05&1.31  \\
         & GR-1 & 0.67&0.38&0.22& \textbf{0.11} &0.06&1.44  \\
        & GEVRM (Ours)  & \textbf{0.70} & \textbf{0.47} & \textbf{0.26} & \textbf{0.11} & \textbf{0.07} & \textbf{1.62} \\
        
        \bottomrule
    \end{tabular}
    \label{tab:calvin_harder_tasks}
    \vspace{-1em}
\end{table}

\textbf{Action execution comparison under external perturbations.}
To thoroughly evaluate the performance of our proposed GEVRM against the baseline SuSIE, we tested both models across five more challenging scenarios (Fig. \ref{fig:calvinTask} (\textbf{c})). 
% The results are summarized in Tab.~\ref{tab:calvin_harder_tasks}.
The average performance on five perturbed tasks is in Tab.~\ref{tab:calvin_harder_tasks}, and the specific results are in Appendix Tab.~\ref{tab:calvin_harder_tasks_all}.
These scenarios were crafted to challenge the models' perception of environmental stimuli and comprehension of physical laws. 
% GEVRM surpassed SuSIE in these tests, with an average success rate increase across all tasks. The second task exhibited the most significant enhancement, with a success rate rise from $0.26$ to $0.47$, surpassing the gains in the first and third tasks.
% Moreover, GEVRM also improved the average task completion length by $45.9\%$, elevating it from $1.11$ to $1.62$.
The results show that GEVRM can well simulate robot response and guide the policy to generate robust decision actions to resist external perturbations.
More action execution comparison results are in Appendix Tab.~\ref{APP:tab:calvin_performance}.
Real-world deployment of GEVRM is in Appendix Sec.~\ref{App:Real-World Tasks.}.

% \textcolor{red}{The SuSIE simply predicts future images at a certain time step as a sub-goal of the low-level policy. 
% This makes it difficult for the model to understand the temporal dependencies of the robot's behavior trajectory, resulting in a sharp drop in the generalization performance of the model when there is a large disturbance in the external perception. 
% In contrast, our proposed model has better spatio-temporal consistency, long-range coherence, and object permanence through random mask training and SAIT visual representation enhancement. 
% This gives the model stronger visual perception and more robust action execution, allowing it to better cope with external perceptual perturbations and generalize to new, unseen environments and tasks.}


\vspace{-0.5em}
\subsection{Ablation study}
\vspace{-0.5em}
% In this section, we conduct an ablation study on the proposed method to verify the effectiveness of the components VAE and SAIT. 
% Specifically, on CALVIN Env. A, B, and C data, we first compare the impact of fine-tuning or not fine-tuning VAE on model performance when training the robot behavior planner. 
% Then compare the impact of using or not using SAIT on model performance when training the robot's goal-guided policy.

In this section, we perform an ablation study to assess the contributions of VAE and state alignment to our method. 
We evaluate the effects of VAE fine-tuning and state alignment application on model performance across CALVIN environments A, B, and C, focusing on robot behavior planning and goal-guided policy training.
%
Results in Fig. \ref{fig:ablationStudy} (a) indicate that omitting VAE fine-tuning or state alignment integration significantly degrades model performance on CALVIN Env. D, due to the VAE's pre-training on diverse video data enhancing spatio-temporal consistency and subsequent fine-tuning on robot data aiding in decision-making generalization. 
State alignment bolsters the policy's visual state representation for better task generalization.
%
Moreover, the hyperparameter $\lambda$, crucial for balancing expert imitation and state alignment in policy training, was tested across five values (Fig. \ref{fig:ablationStudy} (b)). 
Performance metrics varied minimally, showing robustness to $\lambda$ adjustments, with $\lambda=1$ optimal for our method.
%
To illustrate the impact of state alignment on goal-guided representation, we conducted a visual comparison experiment.
We utilize \textit{T-SNE}~\citep{van2008visualizing} to analyze the latent space representations of current and future image states with and without state alignment in the CALVIN ABC → D “Noise Interference” task, and the results are shown in Fig.~\ref{fig:embedding_tsne} and Appendix Fig.~\ref{Appe:embedding_tsne_goal}.
Results indicated that state alignment improves clustering and classification by enhancing intra-category cohesion and inter-category separation. 
Additionally, state alignment ensures temporal consistency in image state sequences, thereby bolstering the policy's environmental and task recognition, and facilitating generalization to novel scenarios.
The ablation experiments on the execution efficiency of the goal generation and the goal-guided diffusion policy are in Appendix Tab.~\ref{tab:goal_efficiency} and ~\ref{tab:policy_efficiency}, respectively.




\begin{figure}[tbp]
\centering
\includegraphics[width=0.85\textwidth]{ICLR_2025_Template/figures/ablation_study.pdf}
\vspace{-1em}
\caption{
Ablation study on the CALVIN ABC → D. (a) We compare different training paradigms.
(b) We examine the impact of different values of the state alignment (SA) hyperparameter $\lambda$.
} 
  \label{fig:ablationStudy}
\end{figure}

\begin{figure}[tbp]
\centering
\includegraphics[width=0.8\textwidth]{ICLR_2025_Template/figures/embedding_tsne.pdf}
% \includegraphics[width=0.5\textwidth,height=0.28\textwidth]{LaTeX/pictures/data_dis_visua.png}
\vspace{-1em}
\caption{
% Comparing latent space representations without and with state alignment (SA).
% The current image state with state alignment exhibits enhanced cluster centers, category boundaries, and temporal consistency.
Comparison of state representations. 
The representations with state alignment (SA) show enhanced cluster centers, class boundaries, and temporal consistency.
} 
 \vspace{-1.5em}
  \label{fig:embedding_tsne}
\end{figure}

% \begin{figure}[tbp]
% \centering
% \includegraphics[width=0.95\textwidth]{ICLR_2025_Template/figures/embedding_tsne.png}
% % \includegraphics[width=0.5\textwidth,height=0.28\textwidth]{LaTeX/pictures/data_dis_visua.png}
% \caption{
% Visual comparison of the latent space representation of the future target image state and the current image state with and without SATI. The state representation sequence with SATI has better cluster centers, category boundaries, and temporal consistency.
% } 
%   \label{fig:embedding_tsne}
% \end{figure}

% To further explore the impact of SAIT on the visual representation of the goal-reaching strategy, a visual comparison experiment was introduced.
% We use the T-SNE algorithm \citep{van2008visualizing} to compare the latent space representation of the future target image state and the current image state with and without SATI in several subtasks of the challenging ABC→D Test 2 "Noise Interference" task, as shown in Fig. \ref{fig:embedding_tsne}.
% The visual comparison results of the target image states show that the use of SAIT helps the visual representation form better cluster centers and classification boundaries. 
% In other words, the image trajectory representations of the same category are more aggregated, while those of different categories are more dispersed. 
% The visual comparison of image states confirms that the image trajectory sequence based on SAIT has better temporal consistency. 
% Therefore, SAIT enhances the recognition ability of the goal-reaching strategy for different environments and tasks, and can better generalize to new scenarios.


\vspace{-1.em}
\section{Conclusion}
\vspace{-0.5em}
The innovation of our method lies in its ability to internalize the classical internal model control principle into the modern VLA framework, thereby enhancing the robot's ability to handle environmental perturbations and maintain performance integrity. 
In the proposed robust GEVRM model, we leverage video generation models to obtain highly expressive target states.
Meanwhile, we effectively align state representations based on prototype contrastive learning to simulate robot responses and evaluate external perturbations.
As demonstrated by the GEVRM's state-of-the-art performance in simulated and realistic visual manipulation tasks, it effectively enhances the expressiveness of the goal state and exhibits strong resilience to external perturbations. 
Therefore, our work greatly expands the reliability and robustness of robotic systems in deployment scenarios and is an important step forward in the field of embodied general intelligence.
A promising work is to consider incorporating more general high-quality video generation models into the VLA framework to cope with complex and diverse manipulation tasks of real-world robots.

\clearpage

% \subsubsection*{Author Contributions}
% If you'd like to, you may include  a section for author contributions as is done
% in many journals. This is optional and at the discretion of the authors.

\subsubsection*{Acknowledgments}
This work was supported by the National Science and Technology Innovation 2030 - Major Project (Grant No. 2022ZD0208800), and NSFC General Program (Grant No. 62176215).

\bibliography{iclr2025_conference}
\bibliographystyle{iclr2025_conference}

\appendix
\clearpage
\section{Appendix}
% \rowcolor{gray!10} % 设置背景颜色为浅灰色
% \multicolumn{2}{c}{\textbf{HIM hypers}} \\ % 合并单元格并添加标题
% \rowcolor{white} % 恢复白色背景


\subsection{Environment perturbations.} \label{Appe:Environment perturbations.}

\textbf{CALVIN Datasets.} We conduct experiments on CALVIN \citep{mees2022calvin}, a benchmark for long-horizon, language-conditioned manipulation to evaluate the GEVRM's capabilities in closed-loop action execution.
CALVIN consists of four simulated environments (A, B, C, and D), each with a dataset of human-collected play trajectories. 
Each environment consists of a \textit{Franka Emika Panda} robot arm positioned next to a table with various manipulable objects, including drawers, sliding cabinets, light switches, and colored blocks. 
Environments are distinguished by their tabletop texture, the positions of furniture objects, and the configuration of colored blocks.
We study zero-shot multi-environment training on A, B, and C, and testing on D, varying in table texture, furniture positioning, and color patches.

\textbf{Environment perturbations Details.} 
We list the details of perturbations in Fig~\ref{fig:calvinTask}:
\textbf{1)} The image state is randomly translated to the upper left, with a maximum translation ratio of 0.1 relative to the image size.
\textbf{2)} The image state is randomly rotated counterclockwise, with a maximum rotation angle of 30 degrees.
\textbf{3)} The image state saturation, brightness, contrast, and sharpness are randomly jittered with a maximum random factor of 3.
\textbf{4)} The image state is randomly occluded with a random number of occlusion blocks ranging from 1 to 3 and a maximum length of 60. 
\textbf{5)} The image state is perturbed with random noise blocks.
We follow the evaluation protocol of Mees et al. \citep{mees2022calvin}. 
During the evaluation, the policy is required to complete five chains of language instructions for $360$ time steps. 
Notably, we only consider RGB images from the static camera as observations, which makes CALVIN much more challenging.


\subsection{Baseline method introductions.} \label{Appe:Baseline method introductions.}
For a fair comparison, we here select the video generation-based baselines to verify the zero-shot generalization performance on standard unseen environments: 
These baseline methods include language-conditioned policies that leverage pre-trained visual-language models in various ways: \textbf{1) HULC} \citep{mees2022matters}, a model that employs a multi-modal transformer encoder for language-conditioned robotic manipulation, combines self-supervised contrastive learning to align video and language representations and uses hierarchical robotic control learning to tackle complex tasks. 
\textbf{2) MCIL} \citep{lynch2020language}, a multi-context imitation learning framework, is capable of handling large-scale unlabelled robot demonstration data. 
MCIL trains a single goal-conditioned policy by mapping various contexts, such as target images, task IDs, and natural language, into a shared latent goal space.
\textbf{3) MdetrLC} \citep{kamath2021mdetr}, which integrates visual and textual information to perform object detection and multi-modal understanding. 
MdetrLC uses text query modulation to detect objects within images and demonstrates strong performance in tasks like visual question answering and phrase localization.
\textbf{4) AugLC} \citep{pashevich2019learning}, which optimizes image augmentation strategies to enable Sim2Real policy transfer from simulated environments to real-world scenarios, applies random transformation sequences to enhance synthetic depth images and uses auxiliary task learning to reduce the domain gap between synthetic and real images.
\textbf{5) LCBC} \citep{walke2023bridgedata}, which employs ResNet-34 as the image encoder combined with MUSE language embeddings for robotic decision-making, uses FiLM conditioning to embed language information into the visual encoding, which in turn generates robot actions.
\textbf{6) UniPi} \citep{du2024learning}, which transforms decision-making problems into text-conditioned video generation tasks, produces future video sequences of the target task and extracts control actions from the generated videos.
\textbf{7) HiP} \citep{ajay2024compositional}, an extension of the UniPi method, enhances the model's ability to handle long-horizon tasks by introducing hierarchical inference and planning. 
This approach decomposes tasks into high-level planning and low-level action generation, improving task execution in complex scenarios.
\textbf{8) SuSIE} \citep{black2023zero}, which leverages a pretrained image-editing diffusion model to generate sub-goal images, guides the robot through complex manipulation tasks via language instructions. 
SuSIE integrates a large-scale internet visual corpus during sub-goal generation and achieves these generated sub-goals through a low-level goal-oriented policy.

\subsection{Real-World Tasks.} \label{App:Real-World Tasks.}
\textbf{Protocol.} To examine the effectiveness of the proposed GEVRM on real-world robotic manipulation tasks, we propose a real-machine deployment protocol. We evaluate GEVRM on a robotic arm UR5 for the pick-and-place tasks of a cup, a bowl, and a tiger plush toy. Specifically, we use a camera to capture third-person images as the observation space (image width 640, height 480), and relative poses and binarized gripper states as the action space (7 dimensions). The total number of collected real-world teleoperation expert trajectories is over 400, with trajectory lengths ranging from 20 to 120 steps and a control frequency of 5Hz.

\textbf{Experiments.} We train and evaluate GEVRM under real-world protocols. The VAE and DiT in the behavior planner are trained for 30,000 and 12,000 iterations, respectively, while the goal-guided policy is trained for 100,000 iterations. Other hyperparameters remain the same as in the experiments in CALVIN (and Bridge). Fig.~\ref{APP:fig:ur5} shows the policy execution process of our proposed GEVRM on three types of real-world tasks, indicating that our method can be effectively deployed on real machines. In terms of task success rate (SR), we evaluated each type of task 10 times. The experimental results show that compared with the grasping and placing of cups (or bowls) with regular shapes (success rate of about 0.8), the grasping of tiger plush toys with soft materials and irregular shapes is more challenging (success rate of about 0.6). Further improving GEVRM's perception of real-world scenes and task execution accuracy is an important future work.
\begin{figure}[tbp]
\centering
\includegraphics[width=0.95\textwidth]{ICLR_2025_Template/figures/ur5_realTask.png}
\caption{
Real-world task results. Our method GEVRM can be effectively deployed in real-world scenarios, such as the picking and placing tasks of cups, bowls and tiger plush toys.
} 
  \label{APP:fig:ur5}
\end{figure}


\begin{table}[ht]\small
    \centering
    \vspace{-1.5em}
    \caption{Generalization on perturbed environments in CALVIN (train A, B, C → perturbed test D).
    % Our proposed method surpasses the previous state-of-the-art method SuSIE in terms of task success rate and task completion length on average.
     % The best results for each task are bolded.
     % The 'AATL' is short for 'Average Achieved Task Length'.
    }
    \vspace{0.5em}
    \renewcommand{\arraystretch}{0.8}
    \begin{tabular}{l c ccccc c}
        \toprule
        \multirow{2}{*}{\textbf{Perturbed Tasks}} & \multirow{2}{*}{\textbf{Algorithms}} & \multicolumn{5}{c}{\textbf{No. of Instructions Chained}} & \multirow{2}{*}{\textbf{Avg. Length ($\uparrow$)}} \\
        \cmidrule(lr){3-7}
        & & 1 & 2 & 3 & 4 & 5 & \\
        \midrule
        \multirow{4}{*}{Image Shift} & SuSIE & 0.56 & 0.28 & 0.08 & 0.04 & 0.00 & 0.96 \\
         & RoboFlamingo & 0.48 & 0.32 & 0.12 & 0.00 & 0.00 & 0.92  \\
         & GR-1 & 0.43 & 0.33 & 0.20 & 0.10 & 0.00 & \textbf{1.00}  \\
        & GEVRM (Ours)  & 0.52 & 0.40 & 0.08 & 0.00 & 0.00 & \textbf{1.00} \\
        \midrule
        \multirow{4}{*}{Image Rotation} & SuSIE & 0.48 & 0.16 & 0.08 & 0.00 & 0.00 & 0.72 \\
         & RoboFlamingo & 0.42 & 0.24 & 0.11 & 0.02 & 0.02 & 0.82  \\
         & GR-1 & 0.46 & 0.32 & 0.14 & 0.10 & 0.03 & 1.07  \\
        & GEVRM (Ours)   & 0.60 & 0.32 & 0.12 & 0.08 & 0.04 & \textbf{1.16} \\
        \midrule
        \multirow{4}{*}{Color Jitter} & SuSIE& 0.72 & 0.36 & 0.16 & 0.12 & 0.08 & 1.44 \\
         & RoboFlamingo & 0.52&0.22&0.08&0.08&0.04&0.94  \\
         & GR-1 & 0.6&0.35&0.21&0.12&0.07&1.35  \\
        & GEVRM (Ours)   & 0.64 & 0.48 & 0.32 & 0.12 & 0.08 & \textbf{1.64} \\
        \midrule
        \multirow{4}{*}{Image Occlusions} & SuSIE & 0.72 & 0.48 & 0.32 & 0.32 & 0.24 & 2.08 \\
         & RoboFlamingo & 0.43 & 0.30 & 0.13 & 0.06 & 0.03 & 0.96  \\
         & GR-1 & 0.78 & 0.60 & 0.46 & 0.32 & 0.23 & 2.39  \\
        & GEVRM (Ours)   & 0.92 & 0.68 & 0.48 & 0.24 & 0.20 & \textbf{2.52} \\
        \midrule
        \multirow{4}{*}{Noise Interference} & SuSIE& 0.32 & 0.04 & 0.00 & 0.00 & 0.00 & 0.36 \\
         & RoboFlamingo & 0.49 & 0.23 & 0.03 & 0.01 & 0.01 & 0.80  \\
         & GR-1 & 0.67&0.42&0.26&0.14&0.08&1.57  \\
        & GEVRM (Ours)   & 0.80 & 0.48 & 0.32 & 0.12 & 0.04 & \textbf{1.76} \\
        \midrule
        \multirow{4}{*}{Average} & SuSIE & 0.56 & 0.26 & 0.13 & 0.10 & 0.06 & 1.11 \\
         & RoboFlamingo & 0.63&0.35&0.18&0.09&0.05&1.31  \\
         & GR-1 & 0.67&0.38&0.22&0.11&0.06&1.44  \\
        & GEVRM (Ours)  & 0.70 & 0.47 & 0.26 & 0.11 & 0.07 & \textbf{1.62} \\
        \bottomrule
    \end{tabular}
    \label{tab:calvin_harder_tasks_all}
    \vspace{-1em}
\end{table}


\begin{table}
    \centering
    \caption{Zero-shot Generalization.
    The experiment is set up to train on data from environments A, B, and C (Fig.~\ref{fig:calvinTask} (\textbf{a})), and test in D (Fig.~\ref{fig:calvinTask} (\textbf{b})). *: reproduced version on static camera. Static camera: camera of fixed third-person view.
    Our proposed method can chain more instructions together with a higher success rate than all previous baseline methods.
    Baseline results are from previous work \citep{black2023zero}. 
    The best results for each task are bolded.
    }
    \begin{tabular}{llccccc}
        \toprule
        \multirow{2}{*}{\textbf{Algorithms}} &\multirow{2}{*}{\textbf{Source}} & \multicolumn{5}{c}{\textbf{No. of Instructions Chained}} \\
        \cmidrule(lr){3-6}
       & & 1 & 2 & 3 & 4 & 5 \\
        \midrule
        HULC~\citep{mees2022matters}& static camera & 0.43 & 0.14 & 0.04 & 0.01 & 0.00 \\
        MCIL~\citep{lynch2020language} &static camera& 0.20 & 0.00 & 0.00 & 0.00 & 0.00 \\
        MdetrLC~\citep{kamath2021mdetr}& static camera& 0.69 & 0.38 & 0.20 & 0.07 & 0.04 \\
        AugLC~\citep{pashevich2019learning} &static camera& 0.69 & 0.43 & 0.22 & 0.09 & 0.05 \\
        LCBC~\citep{walke2023bridgedata} &static camera& 0.67 & 0.31 & 0.17 & 0.10 & 0.06 \\
        HiP~\citep{ajay2024compositional}& static camera& 0.08 & 0.04 & 0.00 & 0.00 & 0.00 \\
        UniPi~\citep{du2024learning}&static camera & 0.56 & 0.16 & 0.08 & 0.08 & 0.04 \\
        SuSIE~\citep{black2023zero}&static camera & 0.87 & 0.69 & 0.49 & 0.38 & {0.26} \\
        GR-1*~\citep{black2023zero}& static camera & 0.75 & 0.45 & 0.2 & 0.15 & {0.1} \\
        \midrule
        GEVRM (Ours)& static camera & \textbf{0.92} & \textbf{0.70} & \textbf{0.54} & \textbf{0.41} & \textbf{0.26} \\
        \bottomrule
    \end{tabular}
    \label{APP:tab:calvin_performance}
\end{table}

\begin{figure}
\centering
\includegraphics[width=0.95\textwidth]{ICLR_2025_Template/figures/embedding_tsne_goal.pdf}
% \includegraphics[width=0.5\textwidth,height=0.28\textwidth]{LaTeX/pictures/data_dis_visua.png}
\caption{
Visual comparison of the latent space representation of the goal image state with and without state alignment (SA). 
The gaol state representation sequence with SA also has better cluster centers, category boundaries, and temporal consistency.
} 
  \label{Appe:embedding_tsne_goal}
\end{figure}

% \subsection{More goal generation results.}  \label{Appe:More goal generation results.}
% Here, we list more goal-generation comparisons on simulation environments.


\begin{figure}[tbp]
\centering
\includegraphics[width=0.95\textwidth]{ICLR_2025_Template/figures/calvin_video_1_com.png}
\caption{
Comparison of goal generation. We visually compare Oracle, GR-1, and our proposed algorithm on the "\textit{push the switch upwards}" task in CALVIN Env. D. 
Compared with the baseline GR-1, the goal video we generate is more realistic and better restores details.
} 
  \label{APP:fig:calvin_video_1}
\end{figure}

\begin{figure}[tbp]
\centering
\includegraphics[width=0.95\textwidth]{ICLR_2025_Template/figures/calvin_video_2_com.png}
\caption{
Comparison of goal generation. We visually compare Oracle, GR-1, and our proposed algorithm on the "\textit{go towards the blue block in the drawer and lift it}" task in CALVIN Env. D. 
Compared with the baseline GR-1, the goal video we generate is more realistic and better restores details.
} 
  \label{APP:fig:calvin_video_2}
\end{figure}

\begin{figure}[tbp]
\centering
\includegraphics[width=0.95\textwidth]{ICLR_2025_Template/figures/calvin_video_3_com.png}
\caption{
Comparison of goal generation. We visually compare Oracle, GR-1, and our proposed algorithm on the "\textit{grasp the red block lying in the slider}" task in CALVIN Env. D. 
Compared with the baseline GR-1, the goal video we generate is more realistic and better restores details.
} 
  \label{APP:fig:calvin_video_3}
\end{figure}
\begin{figure}[tbp]
\centering
\includegraphics[width=0.95\textwidth]{ICLR_2025_Template/figures/calvin_video_4_com.png}
\caption{
Comparison of goal generation. We visually compare Oracle, GR-1, and our proposed algorithm on the "\textit{go push the red block to the right}" task in CALVIN Env. D. 
Compared with the baseline GR-1, the goal video we generate is more realistic and better restores details.
} 
  \label{APP:fig:calvin_video_4}
\end{figure}


\begin{figure}[tbp]
\centering
\includegraphics[width=0.95\textwidth]{ICLR_2025_Template/figures/APP_real_1.pdf}
\caption{
Comparison of goal generation. We visually compare Oracle, AVDC, and our proposed algorithm on the "\textit{put pan on stove and put stuf edduck in pan}" task in Bridge Data. 
Compared with the baseline AVDC, the goal video we generate is more realistic and better restores details.
} 
  \label{APP:fig:real_video_1}
\end{figure}

\begin{figure}[tbp]
\centering
\includegraphics[width=0.95\textwidth]{ICLR_2025_Template/figures/APP_real_2.pdf}
\caption{
Comparison of goal generation. We visually compare Oracle, AVDC, and our proposed algorithm on the "\textit{put yellowpepper on plate and cookiebox in pot or pan on stove}" task in Bridge Data. 
Compared with the baseline AVDC, the goal video we generate is more realistic and better restores details.
} 
  \label{APP:fig:real_video_2}
\end{figure}

\begin{figure}[tbp]
\centering
\includegraphics[width=0.95\textwidth]{ICLR_2025_Template/figures/APP_real_3.pdf}
\caption{
Comparison of goal generation. We visually compare Oracle, AVDC, and our proposed algorithm on the "\textit{put pot or pan on stove and put egg in pot or pan}" task in Bridge Data. 
Compared with the baseline AVDC, the goal video we generate is more realistic and better restores details.
} 
  \label{APP:fig:real_video_3}
\end{figure}

\begin{table}[h!]
\centering
\begin{tabular}{cccccccc}
\toprule
\textbf{Sampling steps} & \textbf{Infer. time [s]} & \textbf{1} & \textbf{2} & \textbf{3} & \textbf{4} & \textbf{5} & \textbf{Avg. Length} \\ 
\midrule
50 & 0.598 & 0.80 & 0.48 & 0.32 & 0.12 & 0.04 & 1.76 \\ 
40 & 0.501 & 0.73 & 0.53 & 0.20 & 0.13 & 0.06 & 1.67 \\
30 & 0.379 & 0.73 & 0.40 & 0.23 & 0.20 & 0.06 & 1.63 \\ 
20 & 0.260 & 0.71 & 0.46 & 0.22 & 0.11 & 0.08 & 1.60 \\ 
10 & 0.135 & 0.77 & 0.47 & 0.17 & 0.15 & 0.10 & 1.67 \\ 
\bottomrule
\end{tabular}
\caption{
The comparative analysis of the computational efficiency and task success rate of the behavior planner (Noise Interference task). 
Due to the good properties of the adopted Rectified Flow, when the video sampling steps are reduced, the model inference time is greatly reduced, while the success rate is not significantly reduced.
}
\label{tab:goal_efficiency}
\end{table}



\begin{table}[h!]
\centering
\begin{tabular}{lcccccccc}
\toprule
 \textbf{Algo.} & \textbf{Control steps} & \textbf{Infer. time [s]} & \textbf{1} & \textbf{2} & \textbf{3} & \textbf{4} & \textbf{5} & \textbf{Avg. Length} \\
\midrule
\multirow{4}{*}{\textbf{DP}} & 1 & 0.077 & 0.80 & 0.48 & 0.32 & 0.12 & 0.04 & 1.76 \\
 & 2 & 0.044 & 0.85 & 0.50 & 0.20 & 0.15 & 0.05 & 1.75 \\
 & 3 & 0.027 & 0.82 & 0.50 & 0.22 & 0.10 & 0.07 & 1.72 \\
 & 4 & 0.020 & 0.68 & 0.48 & 0.24 & 0.16 & 0.08 & 1.64 \\
\midrule
\textbf{MLP} & - & 0.019 & 0.73 & 0.40 & 0.13 & 0.06 & 0.06 & 1.40 \\
\bottomrule
\end{tabular}
\caption{
Comparison of goal-guided diffusion policies (DP) with different open-loop control steps (Noise Interference task).
% We also conducted experimental comparisons on goal-guided diffusion policies with different open-loop control steps, as shown in the table below (Noise Interference task). 
The results show that the state-aligned policy has better action robustness, and increasing the number of open-loop control steps can significantly reduce the inference time while having little effect on the task success rate. 
Therefore, the control frequency of our goal-guided diffusion policy can be maintained at the order of tens of Hz, which is sufficient for most robot manipulation tasks in reality. 
Moreover, when the number of open-loop control steps is 4, the diffusion policy has higher performance, while the inference speed is very close to that of MLP.}
\label{tab:policy_efficiency}
\end{table}

% \subsection{Model hyperparameters details.}  \label{Appe:Model hyperparameters details.}
% In this section, we list more details of model hyperparameters.
\begin{table}[h]
\centering
\caption{Behavior planner training hyperparameters.}
\begin{tabular}{l l l} % Two columns, left-aligned
\toprule
\textbf{Component}   & \textbf{Parameter}     & \textbf{Value} \\
\midrule
% \multirow{4}{*}{Dataset settings} & batch\_size  &  6  \\
% & num\_frame\_total  &  51 \\
% & micro\_frame\_size  &  17 \\
% & grad\_checkpoint  &  True  \\
% \midrule % 添加水平线以分隔
\multirow{3}{*}{Dataset settings} 
& num\_frame\_total  &  51 \\
& transform\_name  &  resize\_crop     \\
& image\_size  & (256, 256)    \\ 
% & num\_frames  & 64    \\

\midrule % 添加水平线以分隔
\multirow{4}{*}{Acceleration settings}  &  num\_workers  &  8   \\
& num\_bucket\_build\_workers  &  16  \\
&  dtype  &  bf16  \\
&  plugin  &  zero2  \\

\midrule % 添加水平线以分隔
\multirow{5}{*}{DIT Model settings} & type & STDiT3-XL/2   \\
% &  from\_pretrained & opensora\_from\_pretrained  \\
&  qk\_norm & True  \\
&  enable\_flash\_attn & True  \\
&  enable\_layernorm\_kernel & True  \\
&  freeze\_y\_embedder & True  \\

\midrule % 添加水平线以分隔
\multirow{2}{*}{VAE settings} 
% & type & OpenSoraVAE\_V1\_2   \\
% &from\_pretrained & vae\_from\_pretrained   \\
&micro\_frame\_size & 17   \\
&micro\_batch\_size & 4   \\

\midrule % 添加水平线以分隔
\multirow{3}{*}{Text encoder settings} & type & T5   \\
% &from\_pretrained & text\_from\_pretrained   \\
&model\_max\_length & 300   \\
&shardformer & True   \\

\midrule % 添加水平线以分隔
\multirow{3}{*}{Scheduler settings} & type & rflow   \\
  &  use\_timestep\_transform & True   \\
  &  sample\_method & logit-normal   \\

\midrule % 添加水平线以分隔
\multirow{10}{*}{Random Mask settings} &  random  &  0.025   \\
 &  intepolate  &  0.025   \\
 &    quarter\_random  &  0.025   \\
 &    quarter\_head  &  0.75   \\
 &   quarter\_tail  &  0.025   \\
 &   quarter\_head\_tail  &  0.05   \\
 &   image\_random  &  0.0   \\
 &   image\_head  &  0.025   \\
 &    image\_tail  &  0.025   \\
 &    image\_head\_tail  &  0.05   \\


\midrule % 添加水平线以分隔
\multirow{6}{*}{Optimization settings} 
& batch\_size & 6  \\
 & grad\_clip  & 1.0  \\
 & learning\_rate  & 1e-4  \\
 & ema\_decay  & 0.99  \\
 & adam\_eps  & 1e-15  \\
 & warmup\_steps  & 1000  \\
\bottomrule
\end{tabular}
\label{Appd:Behavior planner training optimizer hyperparameters}
\end{table}


\begin{table}[h]
\centering
\caption{GEVRM test hyperparameters.}
\begin{tabular}{l l l} % Two columns, left-aligned
\toprule
\textbf{Component}    & \textbf{Parameter}    & \textbf{Value} \\

\midrule

 \multirow{11}{*}{General settings} & fixed\_interval\_number  & 20  \\

  & condition\_frame\_length  & 5   \\
  &  goal\_generation\_number  & 51  \\
  &   micro\_frame\_size  & 17   \\
  % & history\_len  & 51   \\

  & num\_sampling\_steps  & 50  \\

  % & align  & None        \\

  & resolution  &  256  \\
  &  aspect\_ratio  &  1:1  \\
  & image\_size  & (256, 256)   \\
  & fps  & 30 for CALVIN; 5 for Bridge    \\
  & frame\_interval  & 1    \\
  % & save\_fps  & 30   \\

  % & seed  & 42    \\
  % & batch\_size  & 1  \\
  % & multi\_resolution  & STDiT2  \\
  & dtype  & bf16   \\

  % & aes  & None   \\
  % & flow  & None   \\

\bottomrule
\end{tabular}
\label{Appd:Behavior planner test hyperparameters}
\end{table}

\begin{table}[h]
\centering
\caption{Goal-guided policy training hyperparameters.}
\begin{tabular}{l l l}
\toprule
\textbf{Component}   & \textbf{Parameter}     & \textbf{Value} \\  
\midrule
\multirow{7}{*}{General settings} 
% & early\_goal\_concat & False \\ 
 % & shared\_goal\_encoder       & False \\
 % & use\_proprio               & False \\
 & beta\_schedule             & Cosine \\
 & diffusion\_steps           & 20 \\
 & action\_samples            & 1 \\
 & repeat\_last\_step         & 0 \\
 & learning\_rate             & 3e-4 \\
 & warmup\_steps              & 2000 \\
 & actor\_decay\_steps        & 2e6 \\

\midrule
\multirow{5}{*}{Score network settings} & time\_dim                  & 32 \\
 & num\_blocks                & 3 \\
 & dropout\_rate              & 0.1 \\
 & hidden\_dim                & 256 \\
 & use\_layer\_norm           & True \\

\midrule
 \multirow{3}{*}{SA Encoder settings} & hidden\_dim & 512  \\
 & num\_prototype & 3000 \\
 & temperature & 0.1 \\
 % & him\_alpha & 1. \\  % FLags.him_alpha


\bottomrule
\end{tabular}
\label{Appd:Goal-guided policy optimizer Hyperparameters}
\end{table}
% You may include other additional sections here.

% \begin{figure}[htbp]
% 	\centering
% 	\begin{subfigure}
% 		\centering
% 		\includegraphics[width=0.9\linewidth]{ICLR_2025_Template/figures/calvin_video_2.png}
% 		% \caption{chutian3}
% 		% \label{chutian3}%文中引用该图片代号
% 	\end{subfigure}
% 	\centering
% 	\begin{subfigure}
% 		\centering
% 		\includegraphics[width=0.9\linewidth]{ICLR_2025_Template/figures/calvin_video_3.png}
% 		% \caption{chutian3}
% 		% \label{chutian3}%文中引用该图片代号
% 	\end{subfigure}
% 	\centering
% 	\begin{subfigure}
% 		\centering
% 		\includegraphics[width=0.9\linewidth]{ICLR_2025_Template/figures/calvin_video_4.png}
% 		% \caption{chutian3}
% 		% \label{chutian3}%文中引用该图片代号
% 	\end{subfigure}
% 	\caption{More comparisons of robot behavior planning.}
% 	\label{App:calvin_video}
% \end{figure}



\end{document}
