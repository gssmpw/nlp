 \documentclass[12pt,journal,compsoc,onecolumn]{IEEEtran}
\usepackage{natbib}
\ifCLASSOPTIONcompsoc
%% IEEE Computer Society needs nocompress option
%% requires cite.sty v4.0 or later (November 2003)
%\usepackage[nocompress]{cite}
\usepackage{cite}
\usepackage{hanging}
\usepackage{framed,multirow}

%% The amssymb package provides various useful mathematical symbols
\usepackage{amsmath,amssymb}
\usepackage{latexsym}

% Following three lines are needed for this document.
% If you are not loading colors or url, then these are
% not required.
\usepackage{url}
\usepackage{xcolor}
%\usepackage{caption}
\usepackage{adjustbox}
\usepackage{threeparttable}
\usepackage{booktabs}
\usepackage{multirow}
\usepackage{tabularx}
%\usepackage{subfigure}


\else
% normal IEEE
\usepackage{cite}
\fi

%\ifCLASSOPTIONcompsoc
%  \usepackage[caption=false,font=footnotesize,labelfont=sf,textfont=sf]{subfig}
%\else
%  \usepackage[caption=false,font=footnotesize]{subfig}
%\fi
\usepackage{hyperref}

\makeatletter
\def\UrlAlphabet{%
\do\a\do\b\do\c\do\d\do\e\do\f\do\g\do\h\do\i\do\j%
\do\k\do\l\do\m\do\n\do\o\do\p\do\q\do\r\do\s\do\t%
\do\u\do\v\do\w\do\x\do\y\do\z\do\A\do\B\do\C\do\D%
\do\E\do\F\do\G\do\H\do\I\do\J\do\K\do\L\do\M\do\N%
\do\O\do\P\do\Q\do\R\do\S\do\T\do\U\do\V\do\W\do\X%
\do\Y\do\Z}
\def\UrlDigits{\do\1\do\2\do\3\do\4\do\5\do\6\do\7\do\8\do\9\do\0}
\g@addto@macro{\UrlBreaks}{\UrlOrds}
\g@addto@macro{\UrlBreaks}{\UrlAlphabet}
\g@addto@macro{\UrlBreaks}{\UrlDigits}
\makeatother

\usepackage{makecell}
%\usepackage{}
%\usepackage{ctex}
%\normalem
\usepackage{url}
%\usepackage{breakurl}
\usepackage{ulem}
\usepackage{booktabs}
\usepackage{float}
\usepackage{amsmath}
\usepackage{amssymb}
\usepackage{booktabs} % For formal tables
\usepackage{url}
\usepackage{multirow}
\usepackage{graphicx}
\usepackage{enumerate}
\usepackage{bm}
\usepackage{color}
\usepackage{textcomp}
\usepackage{cite}
\usepackage[linesnumbered,ruled]{algorithm2e}
\newcounter{rcounter}
\usepackage{lineno}

\newcommand{\Pan}{\textcolor{red}}
\newcommand{\Weng}{\textcolor{blue}}
% \newcommand{\Comment}{\textcolor{magenta}}
\newcommand{\Xiao}{\textcolor{violet}}


%\usepackage{subfigure}

\usepackage{threeparttable}
\usepackage{makecell}
\usepackage[table,dvipsnames]{xcolor}
\usepackage{colortbl}
\usepackage{ragged2e}
\definecolor{mygray}{gray}{.9}
\usepackage{newtxtext} 
\usepackage{newtxmath} 
\ifCLASSOPTIONcompsoc
%\usepackage[caption=false,font=normalsize,labelfont=sf,textfont=sf,subrefformat=parens,labelformat=parens]{subfig}
\usepackage[caption=false,font=footnotesize,subrefformat=parens,labelformat=parens]{subfig}
\else
\usepackage[caption=false,font=footnotesize,subrefformat=parens,labelformat=parens]{subfig}
\fi
\captionsetup[subtable]{position=top,farskip=0pt}
\captionsetup[subfloat]{farskip=0pt}
\graphicspath{
{./images/}{./images/weight/}{./images/scatters/}{./images/photos/}
}
\newcommand{\dg}{\textdegree}
\newcommand{\tabincell}[2]{\begin{tabular}{@{}#1@{}}#2\end{tabular}}
\newcommand{\tabnote}[1]{\par\medskip\parbox{\textwidth}{#1}}
%\newcommand{\rc}[1]{{\normalem \color{blue}{\noindent \stepcounter{rcounter} \arabic{rcounter}. {\em #1}}}}
\newcommand{\rc}[1]{{\normalem \color{blue}{\noindent \underline{\textbf{Comment} \textbf{\stepcounter{rcounter} \arabic{rcounter}}}\textbf{:} {\em #1}}}}
\newcommand{\ie}{\textit{i}.\textit{e}.}
\newcommand{\eg}{\textit{e}.\textit{g}.}
\newcommand{\liutie}{\textcolor{blue}}
\newcommand{\li}{\textcolor{NavyBlue}}
%\newcommand{\Xu}{\textcolor{cyan}}
% \newcommand{\Comment}{\textcolor{magenta}}
\newcommand{\Comment}{\textcolor{black}}

\newcommand{\redt}{\textcolor{red}}

\def\ie{\emph{i.e.}}
\def\eg{{e.g.}}
% \def\etal{{\em et al.~}}
\def\etal{{\textit{et al.}~}}
% \def \etc{{\em etc.~}}
\def \etc{{\em etc}}

\newcommand{\blueit}[1]{\textit{\textcolor{blue}{#1}}}
\newcommand{\rcblueit}[1]{\noindent \textcolor{blue}{\rc{\textit{#1}}}}
%\newcommand{\rc}[1]{\noindent \underline{\textbf{Comment} \textbf{\stepcounter{rcounter} \arabic{rcounter}}}\textbf{:} {\em #1 }}

\newcommand{\mla}{\textcolor{black}}
% \newcommand{\ml}{\textcolor{purple}}
% \newcommand{\ml}{\textcolor[HTML]{00009C}}
\newcommand{\ml}{\textcolor{black}}

% \newcommand{\xm}{\textcolor{red}}
% \newcommand{\yf}{\textcolor{magenta}}


\begin{document}


\title{doc \\ [10pt]
\vspace{-2.5em}}

% make the title area
\maketitle


% To allow for easy dual compilation without having to reenter the
% abstract/keywords data, the \IEEEtitleabstractindextext text will
% not be used in maketitle, but will appear (i.e., to be "transported")
% here as \IEEEdisplaynontitleabstractindextext when the compsoc
% or transmag modes are not selected <OR> if conference mode is selected
% - because all conference papers position the abstract like regular
% papers do.
\IEEEdisplaynontitleabstractindextext
% \IEEEdisplaynontitleabstractindextext has no effect when using
% compsoc or transmag under a non-conference mode.


\IEEEpeerreviewmaketitle



\begin{figure}
    \centering
    \includegraphics[width=1\linewidth]{img/his.pdf}
    \caption{ histograms of the four patient groups}
    \label{fig:histograms}
\end{figure}
\begin{table*}
\centering

\caption{}
\label{tab:ablation}
\begin{tabular}{c||ccccc|ccccc} \toprule\hline  
\multirow{2}{*}{Ablation}& \multicolumn{5}{c|}{Internal} & \multicolumn{5}{c}{External} \\ \cline{2-11}
 & Acc. & Sen. & Spec. & AUC & F1-score & Acc. & Sen. & Spec. & AUC & F1-score \\ \hline \hline 
$\mathcal{L}_{\rm disent-v1}$& 76.7 & 76.7 & 70.1 & 90.1 & 76.7 
& & & & & \\ 
$\mathcal{L}_{\rm disent-v2}$ & 77.8 & 77.8 & 69.1 & 90.2 & 77.8 
& & & & & \\ 
$w/o$ $\mathcal{L}_{\rm disent}$ & 71.1 & 71.1 & 66.4 & 84.8 & 71.1 
& & & & & \\ \hline
$w/o$  Graph& 73.8 & 73.8 & 72.0 & 90.2 & 73.8 
& & & & & \\  \hline 
$w/o$  LC loss& 70.5 & 70.5 & \textbf{81.5} & 90.6 & 70.5 
& & & & & \\ \hline 
 $w/o$  DCC& 72.8 & 72.8 & 72.3 & 89.7 & 72.8 
& & & & &\\ \hline\hline
\textbf{Ours}& \textbf{85.9} & \textbf{85.9} & 73.0 & \textbf{95.2} & \textbf{85.9} 
& & & & & \\ \hline
\bottomrule
\end{tabular}
\end{table*}



% that's all folks
\nolinenumbers

\newpage


\end{document}
