\section{Related Works}
\subsection{Non-iterative NCO Routing Solvers}

\subsubsection{One-shot Constructive Solvers}
One-shot constructive methods are one of the earliest lines of work that use NCO to solve routing problems. Pioneering works ____ show that neural networks such as RNN can be trained to solve routing problems. Inspired by ____, some works
____ introduce the Transformer architecture to build more powerful NCO models and achieve promising performance. Following their works, various Transformer-based methods ____ emerged. Although they have made progress in training methods or model structures, the one-shot approach can hardly further narrow the performance gap to the optimal results. 

\subsubsection{Heatmap-based Solvers}
Heatmap-based methods aim to predict an informative heatmap to expedite the search process and enhance the quality of solutions. ____ train a Graph Neural Network (GNN) in SL to predict the probabilities of edges to be optimal, then use the beam search to generate feasible solutions. ____ adopt dynamic programming and eliminate dominated partial solutions to reduce searching time. The most prominent works ____ in this category employ Monte Carlo Tree Search (MCTS) to construct solutions. Leveraging MCTS reduces the stringent requirements for the accuracy of edge score predictions. However, most heatmap-based methods are limited to TSPs, as their search strategies are incompatible with problems involving additional constraints, such as CVRPs.

\subsection{Iterative NCO Routing Solvers}

Most existing iterative NCO routing solvers focus on learning low-level operators searching within small neighborhoods. ____ employ a region-picking policy to identify a node for relocation and a rule-picking policy to determine the target position for the node's movement. ____ propose to learn 2-opt or 3-opt steps to improve the solution. Furthermore, ____ utilize a pool of operators from which the model selects, demonstrating superior performance compared to approaches that rely on a single operator. ____ propose a Dual-Aspect Collaborative Transformer (DACT) with a Cyclic Positional Encoding (CPE) method and a Dual-Aspect Collaborative Attention (DAC-Att) to encode problems, which achieves pretty good performance. However, iterative NCOs with low-level operators are limited to solving small-size problems due to the extensive number of iterations required for convergence. Moreover, the overall quality of local optimal of small neighborhoods is inferior, which implies that the final solutions obtained by these methods are often sub-optimal. 


Other iterative NCO routing solvers focus on reconstructing a partial solution of node sequence. Either trained with RL ____ or SL ____, the models learn to reconstruct a segment given the starting node and the ending node. By operating within a large neighborhood, these methods outperform those using low-level operators. However, the neighborhoods that these methods can search in are still limited since the nodes outside the segment remain unaltered. Therefore, two nodes that are spatially close but far away in the solution may have no chance of being reconnected together. In contrast, our framework enables a more flexible neighborhood search by permitting arbitrary destruction and, subsequently, the repair of reconnecting the segments with isolated nodes.