%File: formatting-instructions-latex-2025.tex
%release 2025.0
\documentclass[letterpaper]{article} % DO NOT CHANGE THIS
\usepackage{aaai25}  % DO NOT CHANGE THIS
\usepackage{times}  % DO NOT CHANGE THIS
\usepackage{helvet}  % DO NOT CHANGE THIS
\usepackage{courier}  % DO NOT CHANGE THIS
\usepackage[hyphens]{url}  % DO NOT CHANGE THIS
\usepackage{graphicx} % DO NOT CHANGE THIS
\urlstyle{rm} % DO NOT CHANGE THIS
\def\UrlFont{\rm}  % DO NOT CHANGE THIS
\usepackage{natbib}  % DO NOT CHANGE THIS AND DO NOT ADD ANY OPTIONS TO IT
\usepackage{caption} % DO NOT CHANGE THIS AND DO NOT ADD ANY OPTIONS TO IT
\frenchspacing  % DO NOT CHANGE THIS
\setlength{\pdfpagewidth}{8.5in}  % DO NOT CHANGE THIS
\setlength{\pdfpageheight}{11in}  % DO NOT CHANGE THIS
%
\pdfinfo{
 /TemplateVersion (2025.1)
 }
% These are recommended to typeset algorithms but not required. See the subsubsection on algorithms. Remove them if you don't have algorithms in your paper.
\usepackage{booktabs}       % professional-quality tables
\usepackage{amsfonts}       % blackboard math symbols
\usepackage{nicefrac}       % compact symbols for 1/2, etc.
\usepackage{microtype}      % microtypography

\usepackage{tcolorbox}      % Color text box

\usepackage{subfig}
\usepackage{amsmath}
\usepackage{multirow}

\usepackage{makecell}

\usepackage{algorithm}
\usepackage{algorithmic}

%
% These are are recommended to typeset listings but not required. See the subsubsection on listing. Remove this block if you don't have listings in your paper.
\usepackage{newfloat}
\usepackage{listings}
\DeclareCaptionStyle{ruled}{labelfont=normalfont,labelsep=colon,strut=off} % DO NOT CHANGE THIS
\lstset{%
	basicstyle={\footnotesize\ttfamily},% footnotesize acceptable for monospace
	numbers=left,numberstyle=\footnotesize,xleftmargin=2em,% show line numbers, remove this entire line if you don't want the numbers.
	aboveskip=0pt,belowskip=0pt,%
	showstringspaces=false,tabsize=2,breaklines=true}
\floatstyle{ruled}
\newfloat{listing}{tb}{lst}{}
\floatname{listing}{Listing}
%
% Keep the \pdfinfo as shown here. There's no need
% for you to add the /Title and /Author tags.
\pdfinfo{
/TemplateVersion (2025.1)
}

% DISALLOWED PACKAGES
% \usepackage{authblk} -- This package is specifically forbidden
% \usepackage{balance} -- This package is specifically forbidden
% \usepackage{color (if used in text)
% \usepackage{CJK} -- This package is specifically forbidden
% \usepackage{float} -- This package is specifically forbidden
% \usepackage{flushend} -- This package is specifically forbidden
% \usepackage{fontenc} -- This package is specifically forbidden
% \usepackage{fullpage} -- This package is specifically forbidden
% \usepackage{geometry} -- This package is specifically forbidden
% \usepackage{grffile} -- This package is specifically forbidden
% \usepackage{hyperref} -- This package is specifically forbidden
% \usepackage{navigator} -- This package is specifically forbidden
% (or any other package that embeds links such as navigator or hyperref)
% \indentfirst} -- This package is specifically forbidden
% \layout} -- This package is specifically forbidden
% \multicol} -- This package is specifically forbidden
% \nameref} -- This package is specifically forbidden
% \usepackage{savetrees} -- This package is specifically forbidden
% \usepackage{setspace} -- This package is specifically forbidden
% \usepackage{stfloats} -- This package is specifically forbidden
% \usepackage{tabu} -- This package is specifically forbidden
% \usepackage{titlesec} -- This package is specifically forbidden
% \usepackage{tocbibind} -- This package is specifically forbidden
% \usepackage{ulem} -- This package is specifically forbidden
% \usepackage{wrapfig} -- This package is specifically forbidden
% DISALLOWED COMMANDS
% \nocopyright -- Your paper will not be published if you use this command
% \addtolength -- This command may not be used
% \balance -- This command may not be used
% \baselinestretch -- Your paper will not be published if you use this command
% \clearpage -- No page breaks of any kind may be used for the final version of your paper
% \columnsep -- This command may not be used
% \newpage -- No page breaks of any kind may be used for the final version of your paper
% \pagebreak -- No page breaks of any kind may be used for the final version of your paperr
% \pagestyle -- This command may not be used
% \tiny -- This is not an acceptable font size.
% \vspace{- -- No negative value may be used in proximity of a caption, figure, table, section, subsection, subsubsection, or reference
% \vskip{- -- No negative value may be used to alter spacing above or below a caption, figure, table, section, subsection, subsubsection, or reference

\setcounter{secnumdepth}{2} %May be changed to 1 or 2 if section numbers are desired.

% The file aaai25.sty is the style file for AAAI Press
% proceedings, working notes, and technical reports.
%

% Title

% Your title must be in mixed case, not sentence case.
% That means all verbs (including short verbs like be, is, using,and go),
% nouns, adverbs, adjectives should be capitalized, including both words in hyphenated terms, while
% articles, conjunctions, and prepositions are lower case unless they
% directly follow a colon or long dash
\title{Destroy and Repair Using Hyper-Graphs for Routing}
\author{
    %Authors
    % All authors must be in the same font size and format.
    Ke Li\textsuperscript{\rm 1, \rm 2}, Fei Liu\textsuperscript{\rm 2}, Zhengkun Wang\textsuperscript{\rm 1\thanks{Corresponding author}}, Qingfu Zhang\textsuperscript{\rm 2}    
}
\affiliations{
    %Afiliations
    \textsuperscript{\rm 1}School of System Design and Intelligent Manufacturing, Southern University of Science and Technology\\
    \textsuperscript{\rm 2}Department of Computer Science, City University of Hong Kong\\

    12250110@mail.sustech.edu.cn, 
    fliu36-c@my.cityu.edu.hk, 
    wangzhenkun90@gmail.com,    
    qingfu.zhang@cityu.edu.hk
}

%Example, Single Author, ->> remove \iffalse,\fi and place them surrounding AAAI title to use it
% \iffalse
% \title{My Publication Title --- Single Author}
% \author {
%     Author Name
% }
% \affiliations{
%     Affiliation\\
%     Affiliation Line 2\\
%     name@example.com
% }
%\fi

% \iffalse
% %Example, Multiple Authors, ->> remove \iffalse,\fi and place them surrounding AAAI title to use it
% \title{My Publication Title --- Multiple Authors}
% \author {
%     % Authors
%     First Author Name\textsuperscript{\rm 1,\rm 2},
%     Second Author Name\textsuperscript{\rm 2},
%     Third Author Name\textsuperscript{\rm 1}
% }
% \affiliations {
%     % Affiliations
%     \textsuperscript{\rm 1}Affiliation 1\\
%     \textsuperscript{\rm 2}Affiliation 2\\
%     firstAuthor@affiliation1.com, secondAuthor@affilation2.com, thirdAuthor@affiliation1.com
% }
%\fi


% REMOVE THIS: bibentry
% This is only needed to show inline citations in the guidelines document. You should not need it and can safely delete it.
\usepackage{bibentry}
% END REMOVE bibentry

\renewcommand{\floatpagefraction}{0.8} % 允许浮动体占据最多 80% 页面空间
\renewcommand{\bottomfraction}{0.8}    % 允许浮动体占据最多 80% 的底部
\renewcommand{\textfraction}{0.1}      % 只需要 10% 的正文,避免强制跳页
\setcounter{totalnumber}{5}            % 允许同一页最多放 5 个浮动体
\setcounter{bottomnumber}{3}           % 允许最多 3 个浮动体在底部


\begin{document}

\maketitle

\begin{abstract}
Recent advancements in Neural Combinatorial Optimization (NCO) have shown promise in solving routing problems like the Traveling Salesman Problem (TSP) and Capacitated Vehicle Routing Problem (CVRP) without handcrafted designs. Research in this domain has explored two primary categories of methods: iterative and non-iterative. While non-iterative methods struggle to generate near-optimal solutions directly, iterative methods simplify the task by learning local search steps. However, existing iterative methods are often limited by restricted neighborhood searches, leading to suboptimal results. To address this limitation, we propose a novel approach that extends the search to larger neighborhoods by learning a destroy-and-repair strategy. Specifically, we introduce a Destroy-and-Repair framework based on Hyper-Graphs (DRHG). This framework reduces consecutive intact edges to hyper-edges, allowing the model to pay more attention to the destroyed part and decrease the complexity of encoding all nodes. Experiments demonstrate that DRHG achieves state-of-the-art performance on TSP with up to 10,000 nodes and shows strong generalization to real-world TSPLib and CVRPLib problems. 
\end{abstract}

\begin{links}
\link{Code}{https://github.com/CIAM-Group/DRHG}
\end{links}


\begin{figure*}[htbp]
\centering
\includegraphics[width=0.8\textwidth]{graph/pipeline.pdf} % Reduce the figure size so that it is slightly narrower than the column.
\caption{Pipeline of Destroy-and-Repair using Hyper-Graphs   
}
\label{pipeline}

\end{figure*}

\section{Introduction}

Routing problems are significant combinatorial optimization problems with broad real-world applications in logistics, transportation, and manufacturing. Their NP-hard nature poses a significant challenge to the application of exact methods. Heuristics sacrifice the optimality while can obtain near-optimal solutions in a reasonable time. However, the development of heuristics usually relies on human designs with domain expert knowledge, which hinders their practical applications. 

Neural Combinatorial Optimization (NCO), which trains a neural network to learn heuristics to solve routing problems without handcraft design, has gained much attention. The existing NCO methods can be roughly classified into two categories: 1) non-iterative and 2) iterative methods. 

In non-iterative methods, the neural solvers construct a solution in one shot \cite{vinyals2015pointer, kool2018attention, kwon2020pomo}. Most of these works train neural solvers to determine the next node in an auto-regressive manner, i.e., nodes are selected one by one to be added to the end of a partial solution. Others learn to predict a heuristic, such as a heatmap, and then construct a solution based on the learned information. These works can generate reasonable solutions in a short time. Nevertheless, these non-iterative methods may lead to irreversible consequences if an error occurs in one of the construction steps, thus placing excessive demands on the model's capability to narrow the optimality gap for large-scale problems.

Iterative methods adopt neural solvers to tackle a subproblem in each iteration rather than solve the entire problem at once. The iterative approach reduces the burden of neural solvers and increases the performance, leading to state-of-the-art results. Some existing iterative NCO methods \cite{d2020learning2opt,wu2021learning,ma2021DACT} primarily focus on learning low-level operators within small neighborhoods, such as k-opt or swap. Others follow a destroy-and-repair manner, iteratively destroying the solution into a partial solution and then reconstructing the destroyed nodes, operating within a large neighborhood and excelling at producing high-quality solutions.

However, the neural networks are trained in a conventional way, either using Reinforcement Learning (RL) \cite{LCP,cheng2023select,ye2024glop, zheng2024udc} or Supervised Learning (SL) \cite{luo2023lehd,luo2024SIL} without being tailored for the destroy and repair framework. Therefore, they can only deal with the destruction of one segment, bringing challenges in reducing the optimality gap.

To address the issue, we propose a novel iterative NCO method for routing, termed Destroy and Repair by Hyper-Graphs (DRHG). We employ SL to train a model that approaches the best repair after the destruction. Specifically, after the destruction, the complete tour becomes some segments of consecutive edges and some isolated nodes. We reduce the segments to hyper-edges to build a hyper-graph, then fix them during the repair. The model learns to connect isolated nodes and fixed hyper-edges to form a reduced solution, which is restored to a complete solution later. Thanks to the condensed formulation of the hyper-graph, the scale of model input depends only on the degree of the destruction but not the scale of the original problem. This ensures our method has a low computational complexity and allows our method to iterate on large-scale problems. 

Our contributions can be summarized as follows:

\begin{itemize}
    \item We propose a novel NCO framework of destroy-and-repair for routing problems. By learning to repair a destroyed problem in a supervised way, our model can search in large neighborhoods more efficiently.
    \item We adopt a condensed hyper-graph formulation of the destroyed problem by reducing consecutive edges to fixed hyper-edges, which decreases the computational complexity and enables the model to iterate on large-scale problems.
    \item The experiments show that our method achieves state-of-the-art performance on TSPs from 100 nodes to 10K nodes, and also competitive results on CVRP. Our method generalizes well to real-world instances as well. 
\end{itemize}



\section{Related Works}

\subsection{Non-iterative NCO Routing Solvers}

\subsubsection{One-shot Constructive Solvers}
One-shot constructive methods are one of the earliest lines of work that use NCO to solve routing problems. Pioneering works \cite{vinyals2015pointer, bello2016neural, nazari2018reinforcement} show that neural networks such as RNN can be trained to solve routing problems. Inspired by \citet{vaswani2017attention}, some works
\cite{kool2018attention, deudon2018learning} introduce the Transformer architecture to build more powerful NCO models and achieve promising performance. Following their works, various Transformer-based methods \cite{kwon2020pomo, drakulic2023bq, luo2023lehd} emerged. Although they have made progress in training methods or model structures, the one-shot approach can hardly further narrow the performance gap to the optimal results. 

\subsubsection{Heatmap-based Solvers}
Heatmap-based methods aim to predict an informative heatmap to expedite the search process and enhance the quality of solutions. \citet{joshi2019efficient} train a Graph Neural Network (GNN) in SL to predict the probabilities of edges to be optimal, then use the beam search to generate feasible solutions. \citet{kool2022deep} adopt dynamic programming and eliminate dominated partial solutions to reduce searching time. The most prominent works \cite{fu2021generalize, sun2023difusco} in this category employ Monte Carlo Tree Search (MCTS) to construct solutions. Leveraging MCTS reduces the stringent requirements for the accuracy of edge score predictions. However, most heatmap-based methods are limited to TSPs, as their search strategies are incompatible with problems involving additional constraints, such as CVRPs.

\subsection{Iterative NCO Routing Solvers}

Most existing iterative NCO routing solvers focus on learning low-level operators searching within small neighborhoods. \citet{chen2019neural_rewriter} employ a region-picking policy to identify a node for relocation and a rule-picking policy to determine the target position for the node's movement. \citet{d2020learning2opt, sui2021learning3opt} propose to learn 2-opt or 3-opt steps to improve the solution. Furthermore, \citet{lu2019learning, wu2021learning} utilize a pool of operators from which the model selects, demonstrating superior performance compared to approaches that rely on a single operator. \citet{ma2021DACT} propose a Dual-Aspect Collaborative Transformer (DACT) with a Cyclic Positional Encoding (CPE) method and a Dual-Aspect Collaborative Attention (DAC-Att) to encode problems, which achieves pretty good performance. However, iterative NCOs with low-level operators are limited to solving small-size problems due to the extensive number of iterations required for convergence. Moreover, the overall quality of local optimal of small neighborhoods is inferior, which implies that the final solutions obtained by these methods are often sub-optimal. 


Other iterative NCO routing solvers focus on reconstructing a partial solution of node sequence. Either trained with RL \cite{LCP, cheng2023select, ye2024glop} or SL \cite{luo2023lehd, luo2024SIL}, the models learn to reconstruct a segment given the starting node and the ending node. By operating within a large neighborhood, these methods outperform those using low-level operators. However, the neighborhoods that these methods can search in are still limited since the nodes outside the segment remain unaltered. Therefore, two nodes that are spatially close but far away in the solution may have no chance of being reconnected together. In contrast, our framework enables a more flexible neighborhood search by permitting arbitrary destruction and, subsequently, the repair of reconnecting the segments with isolated nodes. 


\section{Methodology}

\subsection{DRHG Framework}

Schematically illustrated in Fig. \ref{pipeline}, we reformulate our destroy-and-repair approach as a graph reduction, hyper-graph solving, and graph restoration process. For a graph representing the incomplete solution where a set of edges is destroyed, we reduce the graph by encoding the remaining consecutive edges as hyper-edges. As these edges remain unchanged during the repair, redefining them as fixed hyper-edges helps reduce the complexity of the problem for the model. Then, we train the model in a supervised way to solve the reduced problem on the hyper-graph. In the testing phase, we iteratively destroy the current solution to obtain a hyper-graph, solve the resulting hyper-graph, and recover the hyper-graph solution on that of the original problem. 


\subsection{Hyper-graph Representation}\label{sec: represent hyper-graph}

Mathematically, a hyper-graph is a special graph where an edge can join any number of vertices. Formally, a hyper-graph is defined as $\mathcal{G}=(\mathcal{V},\mathcal{E})$, in which $\mathcal{V}$ is the vertex (node) set and $\mathcal{E}$ is the hyper-edge set. Using hypergraph neural networks for embedding hypergraphs is intuitive, but challenging. Specifically, when constructing a solution sequentially, it becomes necessary to align the embeddings of nodes and edges in order to predict the subsequent node or hyper-edge, which may be hard for models. Even if we train an excellent model to predict the sequence, resolving the solution with respect to an undirected hyper-graph remains a non-trivial challenge. Since each hyper-edge has two possible directions, resolving the best complete solution may require a huge number of enumerations. Therefore, we propose to use two endpoints to represent a hyper-edge.

Note a TSP instance of $n$ nodes by the node coordinates as $V=\{(x_1, y_1), \ (x_2, y_2),\  ..., \ (x_n, y_n)\}$. After the destruction, $h$ undestroyed segments constitute the directed hyper-edges set of size $2h$, i.e., $E=\{e^i=(i_1, i_2, ..., i_{p_i})\ | i=1,2,...,2h\}$, where $p_i$ is the number of nodes in the directed hyper-edge $e^i$.

For hyper-graph reduction, we remove the middle nodes inside the hyper-edges and keep only the endpoints to represent the hyper-edge, i.e., $e^i=(i_1, i_p)$. Consequently, in the reduced graph, we have one set of isolated nodes $A$, one set of endpoint nodes $B$, and one set of reduced hyper-edges $E_r$. The hyper-graph size is $m=|A|+|B|$. Then, the input feature of the reduced graph is formulated as:

\begin{equation}
  r_i =(x_i^a, y_i^a, x_i^b, y_i^b, flag_i), i=1,2,...,m,
\end{equation}
    
\begin{equation}
    (x_i^a, y_i^a) = (x_i, y_i),
\end{equation}

\begin{equation}
(x_i^b, y_i^b) = \begin{cases}
	      (x_i, y_i), & if\ i\in A,\\
	      (x_j, y_j), & if\ i\in B\ and (i,j) \in E_r,
		   \end{cases}
\end{equation}
where $r_i$ is the input feature for the model, and $flag_i$ is a binary variable to indicate whether a node is an endpoint node or an isolated node.

Similarly, for a CVRP instance of $n$ customers and a depot noted as $0$, we can define the problem by the node coordinates and the demands: $V=\{(x_0, y_0, 0), \ (x_1, y_1, d_1),\  ..., \ (x_n, y_n, d_n)\}$,  where $d_i$ is the demand of node $i$. To simplify the problem, we destroyed all edges connected to the depot. Then, the input feature of the reduced graph for CVRP can be formulated as follows:

\begin{equation}
  r_i =(x_i^a, y_i^a, x_i^b, y_i^b, flag_i, dr_i), i=1,2,...,m,
\end{equation}

\begin{equation}
dr_i = \begin{cases}
	      d_i, & if\ i\in A,\\
	      \sum_{k}d_k, & if\ i\in B\ and\ k\in (i_1, i_2, ..., i_{p_i}).
		   \end{cases}
\end{equation}


\begin{figure}[htbp]
\centering
\includegraphics[width=0.5\textwidth]{graph/model_structure.pdf} % Reduce the figure size so that it is slightly narrower than the column.
\caption{Model structure of DRHG}
\label{Model structure}
\end{figure}


\subsection{Model Structure}\label{sec: model}
As shown in Fig. \ref{Model structure}, given the input features of the reduced problems, our model yields a prediction of the next node through a light encoder and a heavy decoder. 


\paragraph{Encoder} The encoder consists of a single linear projection layer, which transforms the input $r_i \in \mathbb{R}^{d_i}$ into embedding $h_i^{(0)} \in \mathbb{R}^{d_h}$.

\paragraph{Decoder} The decoder has a slightly changed linear attention module in \citet{luo2024SIL}. At each step $t$, the decoder takes the node embeddings of the first node $h_f^{(0)}$, the current node $h_c^{(0)}$, and the remaining unselected nodes $H_a^{(0)} = \{h_i^{(0)} | i = 1,2, ..., m-t\}$ as inputs. Then, the first node $h_f^{(0)}$ and the current node $h_c^{(0)}$ are used to generate $r$ virtual representative nodes embeddings $\Tilde{H}^{(0)} = \{h_j^{(0)} | j=1,2, ...,r\}$, which combined with $H_a^{(0)}$, form the input of the first linear attention module.

Then, we stack $L$ linear modules as the main component of the decoder. A linear attention module is composed of an aggregating attention layer and a broadcasting attention layer. The aggregating layer aggregates information to the representative nodes, and then the broadcasting layer broadcasts gathered information to all nodes in the graph. The details of the linear attention module are provided in Appendix \ref{appendix-model}. Note the $l$-th linear attention module as $L-Att^{(l)}$, we have
\begin{equation}
    \Tilde{H}^{(l)}, H_a^{(l)} = L-Att^{(l)}(\Tilde{H}^{(l-1)}, H_a^{(l-1)}).
\end{equation}

After $L$ attention module, we obtain a hidden representation $\Tilde{H}^{(L)}$ and and $H_a^{(L)}$. Then we take only $H_a^{(L)}$ to calculate the probability of selecting the next node by a linear projection layer and the softmax function:

\begin{equation}
a_i=\phi(h_i^{(L)}W_o),
\end{equation}

\begin{equation}
p_i=\frac{e^{a_i}}{\Sigma_j^{e^{a_j}}}.
\end{equation}



\subsection{Training Scheme}\label{sec: model}

We use SL to train our model. We apply the clustering destruction, as optimal edges are more likely to connect proximal nodes. Furthermore, the distributions of reduced problems after clustering destructions are more consistent across problems of different scales. We adopt the coordinate transformation in \citet{ye2024glop} to enhance the distribution homogeneity and consistency. For hyper-edges, once one endpoint is selected, the subsequent node must be the other endpoint. This behavior is dictated by the constraint rather than the model. Correspondingly, we introduce a masking mechanism to block the associated gradients. Additionally, destroying the problem by k-nearest neighbors results in a variable number of segments and makes the hyper-graph size differ across instances. This variability introduces instability during the training process. To tackle this problem, we design a special destruction scheme to get fixed-size hyper-graphs. We detail this method in Appendix \ref{appendix-alignement}.


\section{Experiments}
We compare our method with other representative learning-based and classical solvers on
TSP and CVRP instances with different scales and the instances in the real world.
\subsection{Experiment Setup}

\subsubsection{Implementing Details}
We set the embedding dimension of the encoder to 128. The decoder is composed of 6 linear attention modules, and each has 8 attention heads and 16 representative starting nodes. The hidden dimension of the feed-forward layer is set to 512. 

For TSP, we train the model for 100 epochs on 1,000,000 TSP100 instances. We fine-tune 20 epochs on 10,000 TSP1000 instances for large-scale problems. For CVRP, we train the model for 100 epochs on 1,000,000 CVRP100 instances. We use a batch size of 1024 and sample the training sample size in $[20, 0.8n]$ where $n$ is the problem size. As the fixed-size destruction scheme will discard a small part of the samples, the true batch size is around 800. We use the cross-entropy loss and the Adam optimizer \cite{Adam}. The initial learning rate is 1e-4, and the decay rate is 0.97 per epoch. We train and test our model with a single NVIDIA GeForce RTX 3090 GPU with 24GB memory.


\subsubsection{Baselines}
We compare our method with:

\textbf{1) Classical Solvers:} Concorde \cite{applegate2006concorde}, LKH3 \cite{LKH3}, and HGS \cite{HGS}; 

\textbf{2) Traditional Heuristic:} Random Insertion, Sweep; 

\textbf{3) Construction-based Method:} POMO \cite{kwon2020pomo}, BQ \cite{drakulic2023bq};

\textbf{4) Heatmap-based Method:} Att-GCN+MCTS \cite{fu2021generalize}, DIMES \cite{qiu2022DIMES}, and DIFUSCO \cite{sun2023difusco};

\textbf{5) Segment-reconstruction Method:} LEHD \cite{luo2023lehd}, GLOP \cite{ye2024glop} and SIL \cite{luo2024SIL}; 

\textbf{6) Operator-iteration Method:} Neural Rewriter \cite{chen2019neural_rewriter}, Learning 2-Opt \cite{d2020learning2opt}, Learning 3-Opt \cite{sui2021learning3opt}, and DACT \cite{ma2021DACT}.


For most baseline methods, we run their source code with default settings. The result of Att-GCN+MCTS \cite{fu2021generalize}, DIMES \cite{qiu2022DIMES}, and DIFUSCO \cite{sun2023difusco}, SIL \cite{luo2024SIL}, Neural Rewriter \cite{chen2019neural_rewriter}, and Learning 3-Opt \cite{sui2021learning3opt} are taken from their original papers.

\subsubsection{Metrics} 
 
We use the average objective value (Obj.) and optimality gap (Gap) to evaluate the model performance and the inference time (Time) to evaluate the model efficiency. The ground truth labels of TSP are generated by Concorde for TSP100 to TSP1000 and by LKH for TSP above 1000. The ground truth labels of CVRP are generated by HGS.

\subsubsection{Testing}
We test our method on TSPs from 100 to 10,000 and CVRPs from 100 to 1000. There are 10,000 instances for TSP100 and CVRP100, 128 instances for problems of size 200 to 5,000, and 16 instances for TSP10,000. We use random insertion to generate initial solutions for TSP and sweep for CVRP. Regarding to the k-nn destruction, we sample $k\in [20, min(1000, n)]$ for TSP and $k\in [20, min(200, n)]$ for CVRP, where $n$ is the problem scale. For simplicity, we disconnect all nodes adjacent to the depot in CVRP. We set the number of iterations to 1000.   

% TSP
\begin{table*}[htbp]
  \centering
  \renewcommand{\arraystretch}{1.4}
  \renewcommand{\tabcolsep}{3pt} 
    \begin{tabular}{l|ccc|ccc|ccc}
    \toprule
          & \multicolumn{1}{c}{} & \multicolumn{1}{c}{\textbf{TSP100}} & \multicolumn{1}{c|}{} 
          & \multicolumn{1}{c}{} & \multicolumn{1}{c}{\textbf{TSP200}} & \multicolumn{1}{c|}{} 
          & \multicolumn{1}{c}{} & \multicolumn{1}{c}{\textbf{TSP500}} & \multicolumn{1}{c}{} \\
          
    \multicolumn{1}{c|}{\textbf{Method}} & \multicolumn{1}{c}{\textbf{Obj.}} & \multicolumn{1}{c}{\textbf{Gap}} & \textbf{Time} & \multicolumn{1}{c}{\textbf{Obj.}} & \multicolumn{1}{c}{\textbf{Gap}} & \textbf{Time} & \multicolumn{1}{c}{\textbf{Obj.}} & \multicolumn{1}{c}{\textbf{Gap}} & \textbf{Time} \\
    \midrule
    LKH3  & \multicolumn{1}{c}{7.763 } & \multicolumn{1}{c}{0.000\%} & 0.34s & \multicolumn{1}{c}{10.704 } & \multicolumn{1}{c}{0.000\%} & 1.88s & \multicolumn{1}{c}{16.522 } & \multicolumn{1}{c}{0.000\%} & 15.0s \\
    Concorde & \multicolumn{1}{c}{7.763 } & \multicolumn{1}{c}{0.000\%} & 0.20s & \multicolumn{1}{c}{10.704 } & \multicolumn{1}{c}{0.000\%} & 1.41s & \multicolumn{1}{c}{16.522 } & \multicolumn{1}{c}{0.000\%} & 15.0s \\
    Random Insertion & \multicolumn{1}{c}{8.513 } & \multicolumn{1}{c}{9.662\%} & 0.00s & \multicolumn{1}{c}{11.948 } & \multicolumn{1}{c}{11.627\%} & \textless0.01s & \multicolumn{1}{c}{18.546 } & \multicolumn{1}{c}{12.252\%} & \textless0.1s \\
    \midrule
    Att-GCN+MCTS*  & \multicolumn{1}{c}{7.764 } & \multicolumn{1}{c}{0.037\%} & 0.09s & \multicolumn{1}{c}{10.814 } & \multicolumn{1}{c}{0.884\%} & 0.94s & \multicolumn{1}{c}{16.966 } & \multicolumn{1}{c}{2.537\%} & 2.8s \\
    DIMES* & \multicolumn{1}{c}{-} & \multicolumn{1}{c}{-} & -     & \multicolumn{1}{c}{-} & \multicolumn{1}{c}{-} & -     & \multicolumn{1}{c}{16.840 } & \multicolumn{1}{c}{1.760\%} & 60.5s \\
    DIFUSCO* & \multicolumn{1}{c}{7.780 } & \multicolumn{1}{c}{0.240\%} & -     & \multicolumn{1}{c}{-} & \multicolumn{1}{c}{-} & -     & \multicolumn{1}{c}{16.800 } & \multicolumn{1}{c}{1.490\%} & 1.7s \\
    \midrule
    
    POMO augx8 & \multicolumn{1}{c}{\makecell{7.774\\(±0.231)}} 
    & \multicolumn{1}{c}{\makecell{0.134\%\\(±0.224\%)}} & 0.01s    
    & \multicolumn{1}{c}{\makecell{10.868\\(±0.225)}} 
    & \multicolumn{1}{c}{\makecell{1.534\%\\(±0.523\%)}} & 0.04s 
    & \multicolumn{1}{c}{\makecell{20.187\\(±0.251)}} 
    & \multicolumn{1}{c}{\makecell{22.187\%\\(±0.997\%)}} & 0.5s \\
    
    
    BQ bs16 & \multicolumn{1}{c}{\makecell{7.764\\(±0.229)}} 
    & \multicolumn{1}{c}{\makecell{0.015\%\\(±0.057\%)}} & 0.17s      
    & \multicolumn{1}{c}{\makecell{10.717\\(±0.208)}} 
    & \multicolumn{1}{c}{\makecell{0.129\%\\(±0.149\%)}} & 0.94s  
    & \multicolumn{1}{c}{\makecell{16.617\\(±0.212)}} 
    & \multicolumn{1}{c}{\makecell{0.579\%\\(±0.239\%)}} & 5.5s  \\
    \midrule
    
    GLOP (more revision) & \multicolumn{1}{c}{\makecell{7.767\\(±0.234)}}
    & \multicolumn{1}{c}{\makecell{0.046\%\\(±0.126\%)}} & 0.79s 
    & \multicolumn{1}{c}{\makecell{10.774\\(±0.213)}} 
    & \multicolumn{1}{c}{\makecell{0.653\%\\(±0.410\%)}} & 0.33s  
    & \multicolumn{1}{c}{\makecell{16.883\\(±0.214)}} 
    & \multicolumn{1}{c}{\makecell{2.186\%\\(±0.474\%)}} & 0.8s  \\

    LEHD RRC1000 & \multicolumn{1}{c}{\makecell{7.763\\(±0.229)}}
    & \multicolumn{1}{c}{\makecell{0.002\%\\(±0.014\%)}} & 1.04s 
    & \multicolumn{1}{c}{\makecell{10.706\\(±0.206)}}
    & \multicolumn{1}{c}{\makecell{0.0182\%\\(±0.054\%)}} & 4.92s
    & \multicolumn{1}{c}{\makecell{16.550\\(±0.209)}}
    & \multicolumn{1}{c}{\makecell{0.167\%\\(±0.128\%)}} & 33.8s \\
    \midrule
    
    Learning 2-Opt (T=1000) & \multicolumn{1}{c}{7.853} & \multicolumn{1}{c}{1.150\%} & 0.09s & \multicolumn{1}{c}{11.107} & \multicolumn{1}{c}{3.765\%} & 0.20s & \multicolumn{1}{c}{21.339} & \multicolumn{1}{c}{29.158\%} & 0.5s \\
    
    Learning 3-Opt (T=1000)* & \multicolumn{1}{c}{7.850 } & \multicolumn{1}{c}{1.060\%} & 0.23s & \multicolumn{1}{c}{-} & \multicolumn{1}{c}{-} &       & \multicolumn{1}{c}{-} & \multicolumn{1}{c}{-} & - \\
    
    DACT (T=1000) & \multicolumn{1}{c}{7.892 } & \multicolumn{1}{c}{1.653\%} & 0.07s & \multicolumn{1}{c}{12.870 } & \multicolumn{1}{c}{20.252\%} & 0.41s & \multicolumn{1}{c}{20.846 } & \multicolumn{1}{c}{26.171\%} & 1.6s \\
    \midrule

    DRHG (T=1000) & \multicolumn{1}{c}{\makecell{\textbf{7.763}\\\textbf{(±0.229)}}}
    &\multicolumn{1}{c}{\makecell{\textbf{0.000\%}\\\textbf{(±0.007\%)}}}
    & 2.73s      
    & \multicolumn{1}{c}{\makecell{\textbf{10.705}\\\textbf{(±0.206)}}} 
    & \multicolumn{1}{c}{\makecell{\textbf{0.010\%}\\\textbf{(±0.036\%)}}}
    & 9.05s 
    & \multicolumn{1}{c}{\makecell{\textbf{16.540}\\ \textbf{(±0.211)}}} 
    & \multicolumn{1}{c}{\makecell{\textbf{0.111\%}\\ \textbf{(±0.090\%)}}} & 20.6s \\
    \bottomrule

    % \toprule
    & \multicolumn{1}{c}{} & \multicolumn{1}{c}{\textbf{TSP1K}} & \multicolumn{1}{c|}{} 
    & \multicolumn{1}{c}{} & \multicolumn{1}{c}{\textbf{TSP5K}} & \multicolumn{1}{c|}{} 
    & \multicolumn{1}{c}{} & \multicolumn{1}{c}{\textbf{TSP10K}} & \multicolumn{1}{c}{} \\
    
     \multicolumn{1}{c|}{\textbf{Method}} & \textbf{Obj.} & \textbf{Gap}   & \textbf{Time}  &  \textbf{Obj.}  & \textbf{Gap}   & \textbf{Time}  &  \textbf{Obj.}  & \textbf{Gap}   & \textbf{Time} \\
    \midrule
    
    LKH3  & 23.12  & 0.00\% & 1.7m  & 50.97  & 0.00\% & 12m   & 71.78  & 0.00\% & 33m \\
    Concorde & 23.12  & 0.00\% & 1m    & 50.95  & -0.05\% & 31m   & 72.00  & 0.15\% & 1.4h \\
    Random Insertion & 26.11  & 12.90\% & \textless1s   & 58.06  & 13.90\% & \textless1s   & 81.82  & 13.90\% & \textless1s \\
    \midrule
    
    Att-GCN+MCTS*  & 23.86  & 3.20\% & 6s    & -     & -     & -     & 74.93  & 4.39\% & 6.6m \\
    DIMES* & 23.69  & 2.46\% & 2.2m  & -     & -     & -     & 74.06  & 3.19\% & 3m \\
    DIFUSCO* & 23.39  & 1.17\% & 11.5s & -     & -     & -     & 73.62  & 2.58\% & 3.0m \\
    \midrule
    
    POMO augx8 & 32.51  & 40.60\% & 4.1s  & 87.72  & 72.10\% & 8.6m  &       & OOM   &  \\

    BQ bs16 & \multicolumn{1}{c}{\makecell{23.43\\(±0.221)}} & \multicolumn{1}{c}{\makecell{1.37\%\\(±0.284\%)}} & 13s  
    & \multicolumn{1}{c}{\makecell{58.27\\(±0.951)}} 
    & \multicolumn{1}{c}{\makecell{10.70\%\\(±1.827\%)}} & 24s   
    & \multicolumn{1}{c}{} & \multicolumn{1}{c}{OOM} &    \\

    GLOP (more revision) & \multicolumn{1}{c}{\makecell{23.78\\(±0.218)}} 
    & \multicolumn{1}{c}{\makecell{2.85\%\\(±0.401\%)}} & 10.2s 
    & \multicolumn{1}{c}{\makecell{53.15\\(±0.231)}} 
    & \multicolumn{1}{c}{\makecell{4.26\%\\(±0.289\%)}} & 1.0m  
    & \multicolumn{1}{c}{\makecell{75.04\\(±0.215)}} & \multicolumn{1}{c}{\makecell{4.39\%\\(±0.153\%)}} & 1.9m  \\
    
    LEHD RRC1000 & \multicolumn{1}{c}{\makecell{23.29\\(±0.220)}} 
    & \multicolumn{1}{c}{\makecell{0.72\%\\(±0.176\%)}} & 3.3m 
    & \multicolumn{1}{c}{\makecell{54.43\\(±0.394)}} 
    & \multicolumn{1}{c}{\makecell{6.79\%\\(±0.671\%)}} & 8.6m  
    & \multicolumn{1}{c}{\makecell{80.90\\(±0.532)}} & \multicolumn{1}{c}{\makecell{12.50\%\\(±0.663\%)}} & 18.6m \\
    
    SIL PRC1000* & 23.31  & 0.82\% & 1.2m  & 51.91  & 1.84\% & 7.6m  & 73.38  & 2.23\% & 13.7m \\
    \midrule
    Learning 2-Opt (T=1000) & 61.15  & 164.50\% & 1.3s  & -     & -     & -     & -     & -     & - \\
    DACT (T=1000) & 29.03  & 25.56\% & 7.8s  &       & OOM   &       &       & OOM   &  \\
    \midrule
    
    DRHG (T=1000) & 23.22  & 0.45\% & 1.72m & 51.98  & 2.05\% & 1.79m & 74.38  & 3.46\% & 3.63m \\
    

    DRHG-FT (T=1000) 
    & \multicolumn{1}{c}{\makecell{\textbf{23.19}\\\textbf{(±0.210)}}} 
    & \multicolumn{1}{c}{\makecell{\textbf{0.29\%}\\\textbf{(±0.108\%)}}} 
    & 1.66m 
    & \multicolumn{1}{c}{\makecell{\textbf{51.39}\\\textbf{(±0.187)}}} & 
    \multicolumn{1}{c}{\makecell{\textbf{0.88\%}\\\textbf{(±0.087\%)}}} & 1.82m 
    & \multicolumn{1}{c}{\makecell{\textbf{72.85}\\\textbf{(±0.217)}}} & \multicolumn{1}{c}{\makecell{\textbf{1.33\%}\\\textbf{(±0.084\%)}}} & 3.70m \\

    \bottomrule
    
    \end{tabular}%
    \caption{Results on TSP}
  \label{table-tsp-all}%
\end{table*}%




% comparison given the same time with competitors
\begin{table*}[htbp]
  \centering
\begin{tabular}{c|c c|c c|c c}
\toprule
\multicolumn{1}{c|}{\multirow{2}[4]{*}{\textbf{Competitors}}} & \multicolumn{2}{c|}{\textbf{TSP100}} & \multicolumn{2}{c|}{\textbf{TSP200}} & \multicolumn{2}{c}{\textbf{TSP500}} \\
\cmidrule{2-7}      & \textbf{Competitor's} & \textbf{Ours}  & \textbf{Competitor's} & \textbf{Ours}  & \textbf{Competitor's} & \textbf{Ours} \\
\midrule
POMO augx8 & 0.134\% & 1.311\% (-) & 1.534\% & 1.269\%(+) & 22.187\% & 1.113\%(+) \\
\midrule
BQ bs16 & 0.015\% & 0.025\% (-) & 0.129\% & 0.076\%(+) & 0.579\% & 0.322\%(+) \\
\midrule
GLOP (more revision) & 0.046\% & 0.004\% (+) & 0.653\% & 0.206\%(+) & 2.186\% & 0.879\%(+) \\
\midrule
LEHD RRC1000 & 0.002\% & 0.003\% (-) & 0.018\% & 0.016\%(+) & 0.167\% & 0.111\%(+) \\
\bottomrule
\end{tabular}%
\caption{Comparison of different methods on TSPs with scale $\textless$ 1,000 given the same running time}
  \label{comparason}%
\end{table*}%


% table cvrp

\begin{table*}[htbp]
  \centering
    \renewcommand{\tabcolsep}{10pt} 
    \begin{tabular}{l|p{10em}p{6em}c|p{6em}p{6em}c}
    
    \toprule    
    & \multicolumn{1}{c}{} & \multicolumn{1}{c}{\textbf{CVRP100}} & \multicolumn{1}{c|}{} 
    & \multicolumn{1}{c}{} & \multicolumn{1}{c}{\textbf{CVRP200}} & \multicolumn{1}{c}{} \\
    
    \textbf{Method} & \multicolumn{1}{c}{\textbf{ Obj. }} & \multicolumn{1}{c}{\textbf{Gap}} & \textbf{Time} & \multicolumn{1}{c}{\textbf{ Obj. }} & \multicolumn{1}{c}{\textbf{Gap}} & \multicolumn{1}{c}{\textbf{Time}}  \\
    \midrule
    
    LKH3  & \multicolumn{1}{c}{15.647} & \multicolumn{1}{c}{0.00\%} & 4.32s &      \multicolumn{1}{c}{20.173} & \multicolumn{1}{c}{0.00\%} & 59.06s \\
    HGS   & \multicolumn{1}{c}{15.563} & \multicolumn{1}{c}{-0.53\%} & 1.62s &  \multicolumn{1}{c}{19.946} & \multicolumn{1}{c}{-1.13\%} & 39.38s  \\
    Sweep & \multicolumn{1}{c}{20.606} & \multicolumn{1}{c}{32.14\%} & \textless0.01s       & \multicolumn{1}{c}{0.269} & \multicolumn{1}{c}{32.78\%} & \multicolumn{1}{c}{\textless0.01s} \\
    \midrule
    
    POMO augx8 & \multicolumn{1}{c}{\makecell{15.754\\(±1.800)}} & 
    \multicolumn{1}{c}{\makecell{0.69\%\\(±0.649\%)}} & 0.01s & 
    \multicolumn{1}{c}{\makecell{21.154\\(±2.138)}} & \multicolumn{1}{c}{\makecell{4.87\%\\(±1.133\%)}} & \multicolumn{1}{c}{\textless0.01s}  \\
    
    BQ bs16 & \multicolumn{1}{c}{\makecell{15.806\\(±1.807)}} & 
    \multicolumn{1}{c}{\makecell{1.02\%\\(±1.015\%)}} & 0.11s & 
    \multicolumn{1}{c}{\makecell{20.362\\(±2.139)}} & 
    \multicolumn{1}{c}{\makecell{0.94\%\\(±0.950\%)}} & 0.56s  \\
    \midrule

    
    LEHD RRC 1000 & \multicolumn{1}{c}{\makecell{\textbf{15.629}\\\textbf{(±1.793)}}} & 
    \multicolumn{1}{c}{\makecell{\textbf{-0.11\%}\\\textbf{(±0.616\%)}}} & 1.01s     & 
    \multicolumn{1}{c}{\makecell{\textbf{20.095}\\\textbf{(±2.137)}}} & 
    \multicolumn{1}{c}{\makecell{\textbf{-0.38\%}\\\textbf{(±0.627\%)}}} & 19.69s  \\
    
    \midrule
    Neural Rewriter* & \multicolumn{1}{c}{16.100 } & \multicolumn{1}{c}{-} &   \multicolumn{1}{c|}{-}     & \multicolumn{1}{c}{-} & \multicolumn{1}{c}{-} &  \multicolumn{1}{c}{-}    \\
    DACT  & \multicolumn{1}{c}{16.202 } & \multicolumn{1}{c}{3.55\%} & 0.14s &   \multicolumn{1}{c}{23.230 } & \multicolumn{1}{c}{14.71\%} & 1.02s  \\
    \midrule
    
    DRHG (T=1000) & \multicolumn{1}{c}{\makecell{15.643 \\(±1.790)}} & \multicolumn{1}{c}{\makecell{-0.02\%\\(±0.647\%)}} & 2.37s       & \multicolumn{1}{c}{\makecell{\textit{20.233}\\\textit{(±2.133)}}} & 
    \multicolumn{1}{c}{\makecell{-0.16\%\\(±0.648\%)}} & 8.91s \\

    \toprule
      & \multicolumn{1}{c}{} & \multicolumn{1}{c}{\textbf{CVRP500}} & \multicolumn{1}{c|}{} 
    & \multicolumn{1}{c}{} & \multicolumn{1}{c}{\textbf{CVRP1K}} & \multicolumn{1}{c}{} \\
          
    \textbf{Method} & \multicolumn{1}{c}{\textbf{ Obj. }} & \multicolumn{1}{c}{\textbf{Gap}} & \textbf{Time}  & \multicolumn{1}{c}{\textbf{ Obj. }} & \multicolumn{1}{c}{\textbf{Gap}} & \textbf{Time} \\
    \midrule
    
    LKH3  & \multicolumn{1}{c}{37.229 } & \multicolumn{1}{c}{0.00\%} & 154.69s &   \multicolumn{1}{c}{37.090 } & \multicolumn{1}{c}{0.00\%} & 199.7s  \\
    HGS   & \multicolumn{1}{c}{36.561 } & \multicolumn{1}{c}{-1.79\%} & 112.50s &    \multicolumn{1}{c}{36.288 } & \multicolumn{1}{c}{-2.16\%} & 149.1s \\
    Sweep & \multicolumn{1}{c}{46.839 } & \multicolumn{1}{c}{25.81\%} & \textless0.01s       & \multicolumn{1}{c}{49.166 } & \multicolumn{1}{c}{32.56\%} & \textless0.1s  \\
    \midrule
    
    POMO augx8 & \multicolumn{1}{c}{\makecell{44.638\\(±3.112)}} & 
    \multicolumn{1}{c}{\makecell{19.90\%\\(±11.109\%)}} & 0.47s & 
    \multicolumn{1}{c}{84.898} & \multicolumn{1}{c}{128.89\%} & 4.7s  \\
    
    BQ bs16 & \multicolumn{1}{c}{\makecell{37.606\\(±4.216)}} & 
    \multicolumn{1}{c}{\makecell{1.01\%\\(±0.825\%)}} & 3.47s &
    \multicolumn{1}{c}{\makecell{38.147\\(±3.174)}} & \multicolumn{1}{c}{\makecell{2.88\%\\(±1.258\%)}} & 8.7s  \\
    \midrule
    
    GLOP-G(LKH3) & \multicolumn{1}{c}{-} & \multicolumn{1}{c}{-} &      \multicolumn{1}{c|}{-}  & 
    \multicolumn{1}{c}{\makecell{39.651\\(±3.779)}} & \multicolumn{1}{c}{\makecell{6.90\%\\(±2.013\%)}} & 0.8s  \\

    
    LEHD RRC 1000 & \multicolumn{1}{c}{\makecell{\textbf{37.100}\\\textbf{(±4.257)}}} & 
    \multicolumn{1}{c}{\makecell{\textbf{-0.35}\%\\\textbf{(±0.534\%)}}} & 56.25s      & 
    \multicolumn{1}{c}{\makecell{37.432\\(±3.237)}} & \multicolumn{1}{c}{\makecell{0.92\%\\(±0.874\%)}} & 202.5s \\
    
    SIL PRC1000* & \multicolumn{1}{c}{-} & \multicolumn{1}{c}{-} &       \multicolumn{1}{c|}{-}   & \multicolumn{1}{c}{\textbf{36.810 }} & \multicolumn{1}{c}{\textbf{-0.76\%}} & 78.8s \\
    \midrule
    
    DACT  & \multicolumn{1}{c}{46.393 } & \multicolumn{1}{c}{24.98\%} & 3.83s &    \multicolumn{1}{c}{} & \multicolumn{1}{c}{OOM} &     \\
    \midrule
    
    DRHG (T=1000) & \multicolumn{1}{c}{\makecell{37.718\\(±4.446)}} & 
    \multicolumn{1}{c}{\makecell{1.31\%\\(±1.035\%)}} & 12.60s &   
    \multicolumn{1}{c}{\makecell{39.932\\(±3.854)}} & \multicolumn{1}{c}{\makecell{7.66\%\\(±2.226\%)}} & 12.8s  \\
    
    \bottomrule
    \end{tabular}%

    \caption{Results on CVRP}
  \label{table-cvrp-all}%
\end{table*}%



\subsection{Experimental Results}

% non-optimal rate
\begin{figure}[h]
\centering
\includegraphics[width=0.42\textwidth]{graph/non-optimal-Rate-CR.pdf} % Reduce the figure size so that it is slightly narrower than the column.
\caption{Non-optimal rate on 10k TSP100 instances}
\label{non-optimal}
\end{figure}

The main experimental results on uniformly distributed TSP instances are reported in Table \ref{table-tsp-all}. All results are reported in terms of per-instance solving time. The values in parentheses represent the variance. Rank-sum tests are conducted on POMO augx8, BQ bs16, GLOP (more revision) and LEHD RRC1000 to assess whether the path length and the gap of given methods differ significantly from those of our method. Except for the LEHD RRC1000 on TSP100, our DRHG demonstrates statistically significant differences ($p\textless0.05$) from other methods across all other test settings.

Particularly for TSP100, we track the number of non-optimal cases of the other representative NCO methods compared with our method, which is illustrated in Fig. \ref{non-optimal}. For TSPs with 100 to 500 nodes, we perform a comparative analysis of our method against other approaches under identical running time in Table 2, where $+$ means our method outperforms its competitor, and vice versa. Since the running time of one-shot constructive solvers cannot be adjusted, we allocate the same running time to our DRHG as that of its competitors. 


The results demonstrate that our proposed DRHG method can achieve very good performance on instances of all sizes. 
Notably, on TSP100, our method yields non-optimal solutions in only 129 out of 10,000 cases, reducing the non-optimality ratio by an order of magnitude. On TSP200 and TSP500, our method reduces the gap by approximately one-third and outperforms its competitors given the same running time. With a small fine-tuning budget, our method outperforms all other methods on large-scale problems, including SIL \cite{luo2024SIL}, which is separately trained for each problem scale.

For CVRP (Table \ref{table-cvrp-all}), DRHG can also achieve pretty good performance. Our method outperforms the traditional heuristic method LKH3 on CVRP100 and 200. For CVRP200 and 500, our method outperforms most learning methods except for LEHD RRC1000, which, however, requires much more time. Overall, the performance of DRHG is slightly less dominant than that of TSP, but it is still promising. 


Table \ref{tsplib} and Table \ref{cvrplib} show the test results on real-world TSPLib and CVRPLib instances with different sizes and distributions. The results show that our method is robust for different sizes and distributions. More results on TSPLib can be found in Appendix \ref{appendix-tsplib}. 

The ablation studies are presented in Appendix \ref{appendix-hyper-param}. 



% tsplib
\begin{table}[htbp]
  \centering
    \begin{tabular}{l|c|c|c|c|c}
    \toprule
           & POMO  & BQ   & LEHD  & GLOP  & DRHG \\
    \multicolumn{1}{c|}{size} & augx8 & bs16  & R. 1K & more r. & T=1K \\
    \midrule
     \textless 100 & 0.79\% & 0.49\% & 0.48\% & 0.54\% & \textbf{0.48\%} \\
    100-200 & 2.42\% & 1.66\% & 0.20\% & 0.79\% & \textbf{0.15\%} \\
    200-500 & 13.41\% & 1.41\% & 0.38\% & 1.87\% & \textbf{0.36\%} \\
    500-1k & 31.68\% & 2.20\% & 1.21\% & 3.28\% & \textbf{0.26\%} \\
    \textgreater1k   & 63.71\% & 6.68\% & 4.14\% & 7.23\% & \textbf{2.09\%} \\
    \midrule
    all   & 26.41\% & 2.95\% & 1.59\% & 3.58\% & \textbf{0.95\%} \\
    \bottomrule
    \end{tabular}%
  \caption{Results on TSPLib}
  \label{tsplib}%
\end{table}%


% cvrplib
\begin{table}[htbp]
  \centering
    \begin{tabular}{l|c|c|c|c|c}
    \toprule
          & POMO  & BQ   & LEHD  & GLOP  & DRHG \\
    \multicolumn{1}{c|}{size} & augx8 & bs16  & R. 1K & more r. & T=1K \\
    \midrule
    A & 4.97\% & 1.62\% & \textbf{0.75\%} & 26.18\% & 7.17\% \\
    B & 4.75\% & 4.06\% & \textbf{1.09\%} & 20.77\% & 5.55\% \\
    E & 11.40\% & 1.91\% & \textbf{0.58\%} & 18.25\% & 11.59\% \\
    F & 15.97\% & 7.36\% & \textbf{1.36\%} & 39.24\% & 33.18\% \\
    M & 4.86\% & 3.43\% & \textbf{1.43\%} & 22.60\% & 2.37\% \\
    P & 15.53\% & 2.01\% & \textbf{0.93\%} & 17.28\% & 7.53\% \\
    X & 21.68\% & 7.09\% & \textbf{3.69\%} & 18.48\% & 14.78\% \\
    \midrule
    All   & 15.45\% & 4.94\% & \textbf{2.36\%} & 20.10\% & 11.49\% \\
    \bottomrule
    \end{tabular}%
  \caption{Results on CVRPLib}
  \label{cvrplib}%
\end{table}%





\section{Conclusion, Limitation, and Future Work}
\paragraph{Conclusion} This paper has proposed a novel destroy-and-repair framework using hyper-graphs (DRHG) for routing problems. By leveraging the condensed hyper-graph formulation of the destroyed problem, we have reduced the burden of model learning and constrained the input size of the model to the scale of destruction. Extensive experiments comparing our model with other representative NCO methods on both synthetic and real-world instances have demonstrated the superiority of DRHG across different problem scales and distributions. 

\paragraph{Limitation and Future Work} The DRHG shows great performance on TSP, but our current design for CVRP has not fully realized the potential of DRHG. It could be interesting to ameliorate the implementation of DRHG and extend it to other routing problems. Furthermore, future work could explore more sophisticated destruction methods other than clustering destruction.



\section*{Acknowledgments}
This work was supported by the Research Grants Council of the Hong Kong Special Administrative Region, China (GRF Project No. CityU 11215622), the National Natural Science Foundation of China (Grant No. 62106096 and Grant No. 62476118), the Natural Science Foundation of Guangdong Province (Grant No. 2024A1515011759), the National Natural Science Foundation of Shenzhen (Grant No. JCYJ20220530113013031).

\documentclass{MITstyle}

%\usepackage[table]{xcolor}
\usepackage{chngcntr}
\usepackage{hyperref}
\usepackage{microtype}

\title{A Lightweight and Extensible Cell Segmentation and Classification Model for Whole Slide Images}

\author{Nikita Shvetsov~$^{1, }$\footnote{Correspondence e-mail: nikita.shvetsov@uit.no}, Thomas K. Kilvaer~$^{2, 3}$, Masoud Tafavvoghi~$^{4}$, Anders Sildnes~$^{1}$, \\ Kajsa Møllersen~$^{4}$, Lill-Tove Rasmussen Busund~$^{5, 6}$, Lars Ailo Bongo~$^{1}$ \\
%
\vspace{1em} % Space between authors and afilliations
%
\normalfont{\small $^{1}$Department of Computer Science, UiT The Arctic University of Norway}\\
\normalfont{\small $^{2}$Department of Oncology, University Hospital of North Norway}\\
\normalfont{\small $^{3}$Department of Clinical Medicine, UiT The Arctic University of Norway}\\
\normalfont{\small $^{4}$Department of Community Medicine, UiT The Arctic University of Norway}\\
\normalfont{\small $^{5}$Department of Medical Biology, UiT The Arctic University of Norway} \\
\normalfont{\small $^{6}$Department of Clinical Pathology, University Hospital of North Norway} %\vspace{2em}
}

\begin{document}
\maketitle

\section*{Abstract}

% \begin{abstract}
% Developing clinically useful cell-level analysis tools in digital pathology remains challenging due to limitations in dataset granularity, inconsistent annotations, computational demands of advanced models, and difficulties in integrating new technologies into clinical workflows. To address these challenges, we propose a multi-faceted solution that enhances data quality, model performance, and usability to create a lightweight and extensible cell segmentation and classification model.

% First, we update data labels by employing a cross-relabeling process that refines the labels of two existing datasets, PanNuke and MoNuSAC, to create a new unified dataset with enhanced granularity, encompassing seven distinct cell types. Second, we leverage the H-Optimus foundation model as a fixed encoder to improve feature representation for simultaneous cell segmentation and classification tasks. Third, to address the computational demands of foundation models, we employ knowledge distillation to reduce model size and complexity while maintaining comparable performance. Finally, to facilitate integration into clinical workflows, we integrate the distilled model into the QuPath software, a widely used open-source platform in digital pathology.

% Our results demonstrate improvements in cell segmentation and classification performance using the H‑Optimus-based model compared to a CNN-based model. Specifically, the average $R^2$ improved from 0.575 to 0.871, and the average $PQ$ score improved from 0.450 to 0.492, indicating better alignment with actual cell counts and enhanced segmentation and classification quality. Furthermore, the distilled student model maintains performance comparable to the larger foundation model while reducing the parameter count by a factor of 48.
% Overall, by reducing computational complexity and integrating it into existing workflows, the proposed approach may significantly impact diagnostic processes, reduce the workload of pathologists, and contribute to improved patient outcomes. Though our approach shows potential enhancements in efficiency and usability of cell segmentation and classification models in digital pathology, extensive validation is needed to deploy these models in clinical practice.
% \end{abstract}

%%% shortened abstract
\begin{abstract}
Developing clinically useful cell-level analysis tools in digital pathology remains challenging due to limitations in dataset granularity, inconsistent annotations, high computational demands, and difficulties integrating new technologies into workflows. To address these issues, we propose a solution that enhances data quality, model performance, and usability by creating a lightweight, extensible cell segmentation and classification model. 

First, we update data labels through cross-relabeling to refine annotations of PanNuke and MoNuSAC, producing a unified dataset with seven distinct cell types. Second, we leverage the H-Optimus foundation model as a fixed encoder to improve feature representation for simultaneous segmentation and classification tasks. Third, to address foundation models' computational demands, we distill knowledge to reduce model size and complexity while maintaining comparable performance. Finally, we integrate the distilled model into QuPath, a widely used open-source digital pathology platform. 

Results demonstrate improved segmentation and classification performance using the H-Optimus-based model compared to a CNN-based model. Specifically, average $R^2$ improved from 0.575 to 0.871, and average $PQ$ score improved from 0.450 to 0.492, indicating better alignment with actual cell counts and enhanced segmentation quality. The distilled model maintains comparable performance while reducing parameter count by a factor of 48. By reducing computational complexity and integrating into workflows, this approach may significantly impact diagnostics, reduce pathologist workload, and improve outcomes. Although the method shows promise, extensive validation is necessary prior to clinical deployment.
\end{abstract}
\clearpage

\section{Introduction}
In digital pathology, accurate segmentation and classification of cells are crucial for many diagnostic, prognostic, and predictive analyses \cite{Jaber_Beziaeva_etal._2019,Lin_Pan_etal._2022,Park_Ock_etal._2022,Shen_Choi_etal._2024}. Nowadays, developments in computational pathology offer multiple solutions \cite{H._Qu_P._Wu_etal._2020,Javed_Mahmood_etal._2020} to utilize cell-level datasets to train machine learning models that solve these problems. The quality and specificity of training datasets are critical for robust and accurate models. Adhering to the principle of "garbage in, garbage out", it is essential to ensure that these datasets are extensively and accurately labeled with distinct classes that reflect the diverse biological characteristics of different cell types. Unfortunately, the number of open-source datasets comprising such high-quality annotations is limited. Existing cell segmentation datasets \cite{Gamper_Koohbanani_etal._2019,Graham_Vu_etal._2019,Verma_Kumar_etal._2021} may offer extensive annotations for certain cell types while providing more general labels for others. For example, in PanNuke, which is one of the largest open-source datasets comprising labeled cells, various types of morphologically and functionally different inflammatory cells like macrophages and lymphocytes are clustered in a broad "inflammatory" class. Consequently, these classes are frequently omitted from analyses or aggregated into broader meta-classes \cite{Gamper_Koohbanani_etal._2020} and likely interfere with other cell classes included in the dataset. This and similar inconsistencies in annotation granularity limit the ability of machine learning models to learn the comprehensive and nuanced features necessary for accurate cell segmentation and classification. To address these challenges, methods for refining and standardizing dataset annotations are essential to enhance the quality of training data.

A complementary approach to mitigate the absence of high-quality training data is the use of foundation models. Foundation models as encoders are defined as large-scale, versatile networks pre-trained on vast, diverse datasets using self-supervised learning, contrasting with convolutional neural network (CNN) pre-trained encoders that rely on supervised learning with labeled data. In practice, foundation models leverage enormous amounts of weakly or unlabeled data from millions of whole slide images (WSIs) and employ self-attention mechanisms to capture long-range dependencies and global context \cite{Chen_Ding_etal._2024,Saillard_Jenatton_etal._2024,Vorontsov_Bozkurt_etal._2024,Xu_Usuyama_etal._2024}. As a consequence, foundation models are able to produce transferable feature representations across different cell types and tissue environments. The feature representations can be leveraged by decoder networks to produce segmentation masks and pixel-level classifications. Because foundation models have comprehensive feature representations, they can be effectively fine-tuned using much smaller amounts of cell-level data compared to the large datasets needed to train models from scratch. Furthermore, foundation models incorporate adversarial training elements or contrastive learning \cite{Chen_Ding_etal._2024,Xu_Usuyama_etal._2024}, enhancing their resilience and adaptability by exposing them to challenging and varied scenarios during training. This may result in more generalizable models, often making them well-suited for diverse and complex tasks in digital pathology.

Despite the inherent advantages of foundation models, their deployment for practical use faces its own obstacles. In particular, they require substantial computational power, financial investments and rigorous testing to ensure reliability and efficacy for a given task \cite{Akkus_Dangott_etal._2022,Dragomir_Cocuz_etal._2022,Go_2022,Jafri_Farooqui_etal._2024}. Moreover, while foundation models enhance feature representation and performance, they depend on the quality of available annotations for decoder fine-tuning and, like any other model, cannot resolve existing inconsistencies or ambiguities in data labels. Therefore, there remains a critical need for solutions that address both data quality and practical deployment considerations.
Further, integrating new technologies into existing clinical workflows often encounters resistance, as it necessitates adjustments to established diagnostic processes. So, there is a need to develop solutions that could be integrated into current practices, minimizing the burden on medical professionals to adopt new tools \cite{King_Williams_etal._2023}.

Existing solutions \cite{Goldsborough_Philps_etal._2024,Hörst_Rempe_etal._2024}, while addressing some aspects of these challenges, fall short in providing a comprehensive approach. To address the data quality and clinical deployment issues, we propose a multi-faceted solution that encompasses data refinement, model optimization, and integration with existing pathology tools (\hyperref[fig:fig1]{Figure 1}). The outcome is a lightweight cell segmentation and classification model that can be integrated into digital pathology workflows for practical clinical use.

\begin{figure}[h!]
    \centering
    \includegraphics[width=\textwidth, height=0.82\textheight, keepaspectratio]{images/Figure_1.pdf}
    \caption{Overview of the proposed solution, including 1) Data refinement using cross-relabeling, 2) Teacher model development and fine tuning, 3) Student model optimization with knowledge distillation and 4) Student model and QuPath integration}
    \label{fig:fig1}
\end{figure}
\clearpage

Our approach begins with preparing the data for the fine-tuning and training of the machine learning models. We create a refined dataset, acquired via cross-relabeling two cell-level datasets, enhancing annotation specificity and consistency of the labeled data. Subsequently, we create a cell segmentation and classification model based on the foundation model. We leverage the foundation model as a fixed encoder and fine-tune a decoder using the refined dataset to improve generalization across diverse tissue- and cell types.
To ensure that the model remains lightweight and deployable in a possibly resource-constrained environment, we employ knowledge distillation to approximate the functionality of the foundation model. Finally, to facilitate the practical application of our model in digital pathology workflows, we integrate it with the QuPath \cite{Bankhead_Loughrey_etal._2017} application. Each methodological component contributes to the overarching goal of enhancing model performance, generalizability, and usability in clinical settings.

The primary contributions of this paper are:
\begin{enumerate}
    \item \textit{Data labels refinement through cross-relabeling:}
    
    We propose a new method for refining labels of cell-level datasets through cross-relabeling. This method employs classification models to re-label broad and ambiguous instances, resulting in a more diverse dataset. Our evaluation demonstrates that these classification models achieve high accuracy on test subsets, indicating the reliability of the method for label refinement.

    \item \textit{Enhanced model performance via foundation models:}
    
    We employ a foundation model as a feature extractor for the cell segmentation and classification task. In comparison with training a CNN model from scratch, the foundation model backbone only needs fine-tuning, which significantly reduces training time, computational resources and data requirements. We show that using a foundation model encoder leads to better performance in cell segmentation and classification networks than using a CNN-based encoder. This improvement may enable the model to generalize more effectively across various tissue types and imaging methods.
    
    \item \textit{Model optimization through knowledge distillation:}
    
    We show that a smaller student model trained using knowledge distillation on the refined dataset obtained via our cross-relabeling approach from a foundation model achieves comparable performance in cell segmentation and quantification tasks. As a result, this model is more suitable for deployment in environments without high-performance computing resources.
    
    \item \textit{Integration with QuPath:}
    
    We integrate the distilled cell segmentation and classification model into QuPath, a widely used open-source digital pathology platform, to accelerate clinical adaptation by enabling pathologists to more easily incorporate advanced computational tools into their existing workflows.
\end{enumerate}

Through these methodological steps, we aim to bridge the gap between advanced machine learning techniques and practical clinical applications, making accurate and efficient digital pathology accessible in a broader range of healthcare settings.

\section{Refining Existing Datasets Using Cross-Relabeling}
To address the limitations of sparse and ambiguous labeling of cell-level datasets, we propose a generalizable cross-relabeling strategy that can be applied to any dataset containing broadly categorized or imprecisely labeled cell types. This approach involves training and subsequently leveraging classification models to refine broad categories into more specific or biologically relevant classes.
When applied to cell-level data, the methodology includes extracting individual cell images from the dataset patches, preprocessing these images to standardize the size and accommodate partial cells, and then training deep learning classifiers capable of distinguishing between the finer cell subtypes within the coarser categories. 
To illustrate our approach, we focus on the PanNuke \cite{Gamper_Koohbanani_etal._2020, Gamper_Koohbanani_etal._2019} and MoNuSAC \cite{Verma_Kumar_etal._2021} datasets that we have used to train models for cell quantification in our previous works \cite{Shvetsov_Grønnesby_etal._2022,Shvetsov_Sildnes_etal._2024}. We find that for better cell differentiation we have to introduce more granular labels. PanNuke includes a broad classification of "inflammatory" cells, encompassing lymphocytes, macrophages, and neutrophils. Each cell type differs significantly in structure, function, and clinical relevance. Conversely, MoNuSAC uses the label "epithelial" for a class that comprises both benign epithelial cells and malignant neoplastic cells. This practice makes it challenging to differentiate between benign and malignant epithelial cells in the dataset, which is a critical distinction when identifying tumor areas within tissue samples. To address these issues, we implement a cross-relabeling strategy as shown in \hyperref[fig:fig2]{Figure 2}. The key components are two classification models: one is trained on singular cell images from PanNuke data to classify the epithelial meta-class into epithelial and neoplastic classes. The other is trained on MoNuSAC to refine the inflammatory class into lymphocytes, neutrophils, and macrophages.

\begin{figure}[h!]
    \centering
    \includegraphics[width=\textwidth]{images/Figure_2.pdf}
    \caption{Refined dataset generation via cross relabeling}
    \label{fig:fig2}
\end{figure}

The refining approach consists of three consecutive steps. The first is the preprocessing step, in which we extract individual cells from both datasets (\hyperref[fig:fig3]{Figure 3}). The specifics of PanNuke and MoNuSAC patch preparation before cell preprocessing are provided in \hyperref[chap:S1]{Appendix S1}.

\begin{figure}[h!]
    \centering
    \includegraphics[width=\textwidth]{images/Figure_3.pdf}
    \caption{Cell instances preprocessing including (1) cell map extraction, (2) bounding box delineation, (3) adjusting cell boxes and (4) cropping and resizing of cell images}
    \label{fig:fig3}
\end{figure}

During preprocessing, we extract cell type maps from the ground truth label mask and calculate bounding boxes around each cell instance. To accommodate partial cells at patch borders, a common issue in cropped patch images, we employ mirror padding and extend the field of view of the cell label by 15 pixels to capture adjacent cells. We then crop and resize the identified regions to $64 \times 64$ pixels using bicubic interpolation.

The preprocessed PanNuke dataset comprises 68,031 neoplastic and 23,207 epithelial cell images, while MoNuSAC comprises  33,104 lymphocytes, 1,252 neutrophils, and 1,695 macrophages, which we subsequently use in training cell classification models and classifying the cell image data \hyperref[fig:S2]{Appendix Figure S2 (1)}. 

The next step is to train two distinct ResNet50-based classifiers tailored to address the specific labeling challenges inherent in each dataset. We use ResNet50 for classification models due to its proven effectiveness for image classification tasks in histopathology \cite{pan2022reviewmachinelearningapproaches}, and its compatibility with small images. For the PanNuke dataset, we design the classifier, trained on MoNuSAC data, to disaggregate the heterogeneous "inflammatory" cell category into distinct subtypes: lymphocytes, macrophages, and neutrophils. Similarly, for the MoNuSAC dataset, the classifier is trained on PanNuke data and distinguishes between benign and malignant epithelial cells within the overarching "epithelial" label. By applying these targeted classifiers to their respective datasets, we assign more specific labels to individual cell instances, thus enabling us to create a unified dataset.
To ensure a balanced representation of classes, we train both models on datasets that had been equalized to match the size of the least represented class. Thus, we obtain datasets comprising 23,207 samples per class for PanNuke and 1,252 samples per class for MoNuSAC data. Next, we partition both of them into training (70\%), validation (20\%), and testing (10\%) subsets. To mitigate the risk of overfitting, we use a single dropout layer with a rate of p=0.5 in both models and data augmentation using randomized color perturbations, rotation, and horizontal and vertical flipping. We employ AdamW optimizer and the cross-entropy loss function for the training criterion.

To evaluate the two trained models, we measure the classification accuracy on the respective test subsets. The accuracies on the test subset for both classifiers are presented in \hyperref[tab:1]{Table 1}. The PanNuke model achieves an average accuracy of 93.57\%, with higher accuracy for neoplastic cells (96.06\%) compared to epithelial cells (86.26\%). The confusion matrix in Figure A3.1 shows that the model predominantly distinguishes accurately between epithelial and neoplastic tissues, with a substantial number of correct classifications and relatively few misclassifications. The MoNuSAC model demonstrates an average accuracy of 98.92\%, excelling in classifying lymphocytes (99.67\%) and macrophages (94.12\%), with lower performance for neutrophils (85.71\%). The confusion matrix in Figure A3.2 shows that the model identifies lymphocytes and performs reasonably well with macrophages and neutrophils.

\begin{table}[h!]
\renewcommand{\arraystretch}{1.5}
  \centering
  \caption{Cell classification results for PanNuke and MoNuSAC trained models (CI 95\%).}
  \label{tab:1}
  \begin{tabular}{|l|c|c|}
   \hline
   %\rowcolor{gray!30}
    Accuracy               & PanNuke model              & MoNuSAC model              \\
    \hline
    Average      & 0.936 (0.931--0.941)         & 0.989 (0.986--0.993)        \\
    \hline
    Neoplastic   & 0.961 (0.956--0.965)         & -                          \\
    \hline
    Epithelial   & 0.863 (0.849--0.877)         & -                          \\
    \hline
    Lymphocytes  & -                          & 0.997 (0.995--0.999)        \\
    \hline
    Neutrophils  & -                          & 0.857 (0.796--0.918)        \\
    \hline
    Macrophages  & -                          & 0.941 (0.906--0.976)        \\
    \hline
  \end{tabular}
\end{table}

Finally, during the last step, we use the model trained on PanNuke data for epithelial cells in MoNuSAC and the model trained on MoNuSAC for the inflammatory cells class in PanNuke. Specifically, we use classifier models to relabel epithelial cells in MoNuSAC and inflammatory cells in PanNuke data. Then we combine cells with refined labels and the rest of the cells in both datasets to create a refined dataset (\hyperref[fig:S2]{Appendix Figure S2 (2)}). The process of relabeling cells and visualizing them on a patch is shown in \hyperref[fig:fig4]{Figure 4}. The cell counts in the refined dataset are provided in \hyperref[tab:S4]{Appendix Table S4}.

\begin{figure}[h!]
    \centering
    \includegraphics[width=\textwidth, height=0.42\textheight, keepaspectratio]{images/Figure_4.pdf}
    \caption{Cell relabeling procedure for epithelial and inflammatory cell classes}
    \label{fig:fig4}
\end{figure}

%\hfill

Relabeling and combining datasets have been explored in a prior study \cite{Parulekar_Kanwat_etal._2023}, where consecutive fine-tuning on multiple datasets was employed to account for hierarchical class label structures. While the method presented in \cite{Parulekar_Kanwat_etal._2023} is intuitive, it often lacks consistency and requires multiple fine-tuning runs, which can be cumbersome and time-consuming. 
In contrast, cross-relabeling simplifies this process by using specialized classification models tailored to each dataset's specific labeling challenges. This approach provides better transparency and produces a unified dataset encompassing seven distinct cell types across multiple tissue samples, enhancing data diversity for further model training or fine-tuning.

Despite these improvements, cross-relabeling does not entirely resolve issues related to poor labeling quality or the amount of labeled data. Specifically, our results show lower accuracies persist for underrepresented classes, such as macrophages, which may stem from a limited sample availability and intrinsic challenges in distinguishing these cells based solely on H\&E staining. Furthermore, while our method enhances label specificity, it relies on the initial quality of the broad labels; thus, any fundamental inaccuracies in the original annotations can propagate through the relabeling process. Addressing the overall problem of limited data labels may require integrating additional data sources or utilizing complementary immunohistochemical staining methods.
Although the reported performance metrics are obtained from evaluations on the native test sets of each dataset, it is important to note that the primary application of these classifiers is to perform cross-relabeling, where a model trained on one dataset (e.g., PanNuke) is applied to another (e.g., MoNuSAC) and vice versa. We acknowledge that a more systematic evaluation of cross-dataset generalization is needed and could be performed in future work.

Overall, the refined dataset produced by our approach can enhance the supervised training or fine-tuning of cell segmentation and classification models, especially those that utilize pre-trained foundation models to improve feature extraction robustness. In addition, these models can detect nuanced classes that enable researchers to conduct more detailed analyses of biological processes in computational pathology.

\section{Foundation models for robust cell segmentation and classification}

Accurate cell segmentation and classification in digital pathology are hindered by limited labeled data and the fact that conventional CNNs are unable to capture global contextual information due to their local receptive field constraints \cite{Gheflati_Rivaz_2022,Yang_Marcus_etal.}. Traditional approaches in cell quantification have predominantly relied on CNN encoders, such as ResNet50, given their proven effectiveness in semantic segmentation tasks \cite{Deshmane_2023,Graham_Vu_etal._2019,Mukasheva_Koishiyeva_etal._2024,Stringer_Wang_etal._2021}. However, approaches that include fine-tuning of pretrained CNNs, data augmentation, and stain normalization to partially increase data variability and address staining differences often fail to achieve the necessary generalization and robustness across diverse tissue types and staining conditions \cite{G._Wang_W._Li_etal._2018,Gao_Bagci_etal._2018,Karim_El_Khoury_Martin_Fockedey_etal._2021}.

To overcome these challenges, we leverage an encoder-decoder network that uses a foundation model as the encoder and a CNN upsampling decoder (\hyperref[fig:fig5]{Figure 5}) for simultaneous cell segmentation and classification in 2D patches extracted from WSIs. Foundation models with transformer-based architectures are viable alternatives to CNN-based encoders \cite{Shamshad_Khan_etal._2023,Sourget_2023}. They enable the creation of more advanced architectures that can decode or transform learned features more effectively \cite{Chen_Duan_etal._2023,Cheng_Misra_etal._2022,Xie_Wang_etal._2021}.

\begin{figure}[h!]
    \centering
    \includegraphics[width=\textwidth]{images/Figure_5.pdf}
    \caption{UNETR-like model with foundational model as backbone}
    \label{fig:fig5}
\end{figure}

By utilizing a transformer-based encoder, we incorporate global contextual information into the feature extraction process, which is a key advantage of such architectures \cite{Chen_Lu_etal._2021}. This foundation model integration facilitates accurate pixel-wise segmentation and classification without the need for extensive encoder training, thereby potentially improving generalization across varied cellular structures and tissue types.
In our implementation, we employ a modified UNETR \cite{Hatamizadeh_Tang_etal._2021} architecture that combines a vision transformer (ViT) \cite{Dosovitskiy_Beyer_etal._2021} encoder with a CNN-based decoder. The encoder utilizes the pretrained H-Optimus foundation model, which contains 1.1 billion parameters and is trained on over 500,000 H\&E stained WSIs \cite{Saillard_Jenatton_etal._2024}. We extract outputs from four evenly spaced transformer blocks $Z_i$, where $i \in [1, 14, 26, 38]$, to serve as residual connections for the CNN decoder. We select these blocks based on our observation that features from non-adjacent levels of the encoder lead to better overall performance on the test subset.

The CNN decoder upsamples the feature representations, acquired from the transformer blocks, to generate an intermediate vector that is handled by two task-specific layers that generate cell segmentation and classification masks. The first task-specific layer is the ‘Cellpose head’,  which is used to delineate cell instances. The layer generates horizontal and vertical gradient maps to form vector fields that are refined through gradient tracking in a post-processing step using the Cellpose algorithm \cite{Stringer_Wang_etal._2021}, known for its efficacy in cell segmentation tasks and generalizability across multiple domains \cite{Pachitariu_Stringer_2022,Stringer_Pachitariu_2024}. The second task-specific layer is the "Cell type head", which assigns labels to individual pixels. In the post-processing step, we determine the output classification label of each segmented cell instance by majority voting over the labeled pixels that comprise the cell in the segmentation map.

To evaluate model performance and measure the impact of adding a foundation model as backbone, we compare it to a ResNet50-based model. ResNet50 is a widely used solution for encoders in segmentation architectures in the medical domain \cite{Deshmane_2023,Graham_Vu_etal._2019,Mukasheva_Koishiyeva_etal._2024,Stringer_Wang_etal._2021}. For the H-Optimus-based model, we utilize frozen weights for the encoder and only fine-tune the decoder to take advantage of the extensive pre-training of the foundation model. For the ResNet50-based model we start with ImageNet \cite{Deng_Dong_etal.} weights and train both encoder and decoder parts. Hyperparameters for the training step are set to be identical, where possible, for comparable evaluation. 
For this evaluation, we deliberately use the PanNuke dataset to provide a standardized and controlled comparison between the H‑Optimus and ResNet50-based models (\hyperref[fig:S2]{Appendix Figure S2 (3)}). Specifically, we use two of the default PanNuke dataset splits (66\%) for training and validation, and reserve the third split (33\%) for testing.

To address the challenge of cell class imbalance in the PanNuke dataset, which is a common characteristic in most cell-level H\&E patch datasets, both models’ training processes employ a weighted loss function comprising cross-entropy and focal loss \cite{Lin_Goyal_etal._2018}. The focal loss component is adjusted with coefficients derived from each cell class' instance frequency, emphasizing learning from underrepresented classes and enhancing the model's sensitivity to rare but significant cellular patterns. The cross-entropy loss is augmented with spectral decoupling regularization \cite{Pezeshki_Kaba_etal._2021,Pohjonen_Stürenberg_etal._2022} and spatially varying label smoothing \cite{Islam_Glocker_2021}, which potentially stabilizes training and improves generalization in case of complex tissue morphologies. For optimization, we employ the \textit{AdamW} \cite{Loshchilov_Hutter_2019} to counter unbalanced class scenarios, with cosine annealing learning rate scheduler.

We utilize the scikit-learn library \cite{Van_der_Walt_Schönberger_etal._2014} and HoVer-Net \cite{Graham_Vu_etal._2019} implementations of $R^2$ (the coefficient of determination) and $PQ$ (panoptic quality) to evaluate our experiments. Complete mathematical formulations and detailed explanations of these metrics are provided in \hyperref[chap:S5]{Appendix S5}. To compute confidence intervals, we use nonparametric bootstrapping, where after calculating the metric on the full sample, we generated 1000 bootstrap replicates by resampling with replacement and then determined the 95\% confidence intervals as the 2.5th and 97.5th percentiles of the resulting empirical distribution.

%\hfill

The model comparisons are summarized in \hyperref[tab:2]{Table 2}. The H‑Optimus-based model achieves higher $R^2$ across all cell classes compared to the ResNet50-based model, which means that its predictions are more closely aligned with the PanNuke cell counts, indicating a stronger correlation with the observed data. Notably, the improvement of $R^2_{dead}$ may be an indicator of better global contextual representations provided by the foundation model backbone. In terms of segmentation and classification quality combined, measured by the PQ score, the H‑Optimus-based model demonstrates notable improvements across most cell classes. Overall, the average $R^2$ improved from 0.575 to 0.871, while the average $PQ$ score improved from 0.450 to 0.492, demonstrating better performance of the H-Optimus-based model.

\begin{table}[h!]
\renewcommand{\arraystretch}{1.5}
  \centering
  \caption{Cell quantification metrics for baseline and proposed models (CI 95\%).}
  \label{tab:2}
  \begin{tabular}{|l|c|c|}
    \hline
    %\rowcolor{gray!30}
    Metric             & Resnet50-based            & H-optimus-based              \\
    \hline
    $R^2_{neoplastic}$    & 0.681 (0.576--0.769)       & \textbf{0.941 (0.917--0.960)} \\
    \hline
    $R^2_{inflammatory}$  & 0.863 (0.778--0.903)       & \textbf{0.949 (0.918--0.966)} \\
    \hline
    $R^2_{connective}$    & 0.600 (0.488--0.698)       & 0.609 (0.436--0.772)          \\
    \hline
    $R^2_{dead}$          & 0.097 (-11.389--0.669)     & 0.925 (0.404--0.982)          \\
    \hline
    $R^2_{epithelial}$    & 0.635 (0.490--0.747)       & \textbf{0.930 (0.886--0.964)} \\
    \hline
    $PQ_{neoplastic}$       & 0.517 (0.499--0.535)       & \textbf{0.589 (0.575--0.604)} \\
    \hline
    $PQ_{inflammatory}$     & 0.455 (0.429--0.482)       & \textbf{0.528 (0.507--0.549)} \\
    \hline
    $PQ_{connective}$       & 0.416 (0.400--0.431)       & \textbf{0.451 (0.436--0.465)} \\
    \hline
    $PQ_{dead}$             & 0.374 (0.342--0.408)       & 0.292 (0.209--0.365)          \\
    \hline
    $PQ_{epithelial}$       & 0.488 (0.460--0.519)       & \textbf{0.599 (0.579--0.618)} \\
    \hline
  \end{tabular}
\end{table}

Our results  show that integrating the H‑Optimus foundation model within the UNETR architecture enhances the model's ability to segment and classify cells across diverse tissues from PanNuke data. The pretrained transformer encoder provides robust feature representations, resulting in higher average $R^2$ and $PQ$ scores compared to the CNN-based model. This leads to more reliable cell quantification and more accurate downstream analysis. Additionally, the streamlined fine-tuning process reduces computational overhead and training time, making the model more adaptable for new data.

Despite these advancements, the foundation model-based approach does not fully resolve all challenges related to cell segmentation and classification. We observe lower metric scores for underrepresented classes in the training data. Furthermore, foundation models typically encompass billions of parameters, resulting in substantial computational and memory requirements. It therefore poses challenges for deployment in resource-constrained environments, limiting their practical applicability in certain clinical settings.

\section{Model optimization via Knowledge Distillation}

To address the limitations posed by the extensive size of foundation models, we implement knowledge distillation — a model compression technique that leverages the teacher-student paradigm \cite{Hinton_Vinyals_etal._2015}. By training a smaller, more efficient student model to replicate the output of a larger, pre-trained teacher model, we retain performance while significantly reducing the model's complexity and resource requirements (\hyperref[fig:fig6]{Figure 6}).

\begin{figure}[h!]
    \centering
    \includegraphics[width=\textwidth, height=0.45\textheight, keepaspectratio]{images/Figure_6.pdf}
    \caption{Knowledge distillation framework for training a student model using a pre-trained teacher}
    \label{fig:fig6}
\end{figure}

We employ knowledge distillation to compress the H‑Optimus-based teacher model into a more efficient student model. The teacher model is the modified UNETR architecture with the H‑Optimus foundation model described in the previous chapter. The student model is based on a UNet architecture augmented with residual connections and incorporates a smaller ViT encoder with 9 million parameters \cite{Steiner_Kolesnikov_etal._2022,Wightman_2019}. 

First, we fine-tune the teacher model using the refined dataset from the cross-relabeling procedure (Section 2). Initially we train the decoder of the teacher model while keeping the encoder weights frozen. We split the refined dataset into train (70\%), validation (20\%) and test (10\%) subsets (\hyperref[fig:S2]{Appendix Figure S2 (4)}). During fine-tuning, we use the train and validation subsets, while leaving the test subset for model evaluation. We set the training procedure and model hyperparameters to be identical to those that were used to demonstrate the utility of foundation models for the simultaneous cell segmentation and classification task.

Next, we perform knowledge distillation from teacher to student using the refined dataset used to fine-tune the teacher model. The student model is trained to replicate the teacher model's outputs. We utilize a specialized loss function that aligns the student's predicted probability distribution with the teacher's, incorporating the teacher's class probability distribution derived from the output. Following the methodology of Hinton et al. \cite{Hinton_Vinyals_etal._2015}, we experiment with various hyperparameter settings for the temperature ($T$) and the balancing coefficients ($\alpha$ and $\beta$) in the loss function. We vary $T$ from 1 to 20 and adjust $\alpha$ and $\beta$ to balance the distillation and student losses. Through iterative tuning and evaluation, we identify that setting $T=14$, $\alpha=0.3$, and $\beta=0.7$ yields a configuration that converges and closely approximates the teacher model's performance during training.

Finally, we assess the performance of both models using the $R^2$ and $PQ$ (defined in \hyperref[chap:S5]{Appendix S5}) on the test set of the refined dataset (\hyperref[tab:3]{Table 3}). We observe that the 95\% confidence intervals overlap for most cell types, so we cannot claim statistically significant performance differences between the teacher and student models. One exception appears in the neoplastic class. The teacher model produces an $R^2$ of 0.919, while the student model shows an $R^2$ of 0.852. In addition, the student model achieves higher $PQ$ values for the neoplastic and connective classes, though the confidence intervals show overlap.

\begin{table}[h!]
\renewcommand{\arraystretch}{1.5}
  \centering
  \caption{Cell quantification metrics for teacher and distilled student models (CI 95\%).}
  \label{tab:3}
  \begin{tabular}{|l|c|c|}
    \hline
    %\rowcolor{gray!30}
    Metric & Teacher & Student \\
    \hline
    $R^2_{neoplastic}$    & \textbf{0.919} (0.898--0.939) & 0.852 (0.800--0.891) \\
    \hline
    $R^2_{lymphocyte}$    & 0.969 (0.956--0.977)         & 0.969 (0.956--0.978) \\
    \hline
    $R^2_{connective}$    & 0.694 (0.548--0.809)         & 0.618 (0.469--0.741) \\
    \hline
    $R^2_{dead}$          & 0.755 (0.400--0.908)         & 0.424 (0.100--0.731) \\
    \hline
    $R^2_{epithelial}$    & 0.922 (0.870--0.958)         & 0.843 (0.738--0.917) \\
    \hline
    $R^2_{macrophage}$    & 0.384 (-0.369--0.724)        & 0.704 (0.352--0.859) \\
    \hline
    $R^2_{neutrofil}$     & 0.854 (0.578--0.929)         & 0.833 (0.502--0.925) \\
    \hline
    $PQ_{neoplastic}$       & 0.581 (0.569--0.593)         & 0.601 (0.588--0.613) \\
    \hline
    $PQ_{lymphocyte}$       & 0.536 (0.520--0.553)         & 0.563 (0.544--0.579) \\
    \hline
    $PQ_{connective}$       & 0.436 (0.421--0.451)         & 0.457 (0.441--0.474) \\
    \hline
    $PQ_{dead}$             & 0.272 (0.235--0.315)         & 0.279 (0.201--0.369) \\
    \hline
    $PQ_{epithelial}$       & 0.522 (0.500--0.545)         & 0.530 (0.506--0.555) \\
    \hline
    $PQ_{macrophage}$       & 0.524 (0.459--0.588)         & 0.474 (0.405--0.543) \\
    \hline
    $PQ_{neutrofil}$        & 0.541 (0.490--0.592)         & 0.565 (0.522--0.607) \\
    \hline
  \end{tabular}
\end{table}


We further decompose the $PQ$ metric into its $SQ$ and $DQ$ components (\hyperref[tab:S6]{Appendix Table S6}). Both models produce nearly identical $SQ$ values, which indicates that they predict instance boundaries with similar precision. Although the student model shows some improvement in $DQ$ scores for certain classes, the confidence intervals overlap and do not confirm a statistically significant difference.

We observe that the student and teacher models yield comparable detection performance despite the student model using a much smaller and simpler architecture. A model with fewer parameters reduces the risk of overfitting when training data are scarce relative to the model’s complexity \cite{Farias_Ludermir_etal._2022}. The knowledge distillation process also encourages the student model to focus on the most generalizable detection features learned from the teacher. These factors enable the student model to achieve similar detection performance across different cell types.

Additionally, considering the model sizes reported in \hyperref[tab:4]{Table 4}, the distilled model achieves a significant reduction compared to the teacher model, with a 48-fold decrease in parameter count and a 5.5-fold reduction in on-disk size. In inference mode, the teacher model requires 16 GB of VRAM for a batch size of 32, while the distilled model only needs 3 GB of VRAM for the same batch size. These reductions make the distilled model significantly more practical for fine-tuning and deployment in resource-constrained environments.

\begin{table}[h!]
\renewcommand{\arraystretch}{1.5}
  \centering
  \caption{Parameter counts and size of teacher and distilled model}
  \label{tab:4}
  \adjustbox{max width=\textwidth}{%
  \begin{tabular}{|l|c|c|c|}
    \hline
    %\rowcolor{gray!30}
    Metric & H-optimus-based (Teacher) & mobileViT-based (Student) & Magnitude of difference \\
    \hline
    Parameters count       & 1,158,917,906   & \textbf{24,093,393}   & \textbf{48x}  \\
    \hline
    Estimated Total Size (MB) & 87,912       & \textbf{15,935}    & \textbf{5.5x} \\
    \hline
  \end{tabular}%
}
\end{table}

%\hfill

With recent advancements in complex network architectures and the use of pretrained encoders to achieve state-of-the-art performance \cite{Baumann_Dislich_etal._2024,Hörst_Rempe_etal._2024} in cell segmentation and classification tasks, model size, computational complexity, and processing times have increased. This limits the scalability and accessibility of these models. As we demonstrate, this may be mitigated using knowledge distillation. Studies in the field of natural language processing have demonstrated the efficacy of knowledge distillation in retaining the capabilities of the teacher model while achieving significant reductions in size and complexity \cite{Huangpu_Gao_2024,Sun_Yu_etal.}. 

We demonstrate the feasibility of knowledge distillation in digital pathology, specifically for cell segmentation and classification tasks. Moreover, we achieve this performance while also significantly reducing the parameter count. In addressing the challenge of knowledge transfer, we found that distillation from a transformer-based model to a smaller transformer is more straightforward than attempting to map transformer features to CNN blocks. In our experiments, using a CNN-based network as a student results in worse cell quantification performance due to the structural constraints of CNN feature space dimensions. 

Although our primary approach relies on a transformer-based student model that performs well, it can be further optimized to incorporate advantages from CNN architectures. For example, employing alternative techniques such as using ViT adapters \cite{Chen_Duan_etal._2023} or $1 \times 1$ convolutions to adjust feature map sizes may be beneficial for harnessing CNN advantages like enhanced local feature extraction. Moreover, if additional performance improvements are desired, the process can be further enhanced by applying supplementary knowledge distillation techniques, such as self-distillation \cite{Zhang_Song_etal._2019} or online distillation \cite{Houyon_Cioppa_etal._2023}.

Despite these promising results, further validation on independent datasets is necessary to fully understand the model's limitations. Underrepresented classes may pose challenges when addressing complex cases. Pathologists need to validate these models to adopt them in clinical settings. While the distilled models are smaller and more deployable, a technological gap persists because pathologists traditionally rely on established methods for inspecting WSIs and diagnosing diseases. Addressing the complexities involved in deploying models for inference and supporting pathologists in adopting new tools is essential for integrating these models into clinical workflows.

\section{Model integration with QuPath}
Digital pathology tools with graphical user interfaces are essential for visualizing and analyzing WSIs. To make our student model useful in clinical pathology workflows, it needs to be integrated into a tool that enables inspecting regions, creating annotations, and providing quantitative analyses of biomarkers. Therefore, we integrate the trained student model from the previous chapter into the QuPath open‑source platform \cite{Bankhead_Loughrey_etal._2017}. QuPath provides the required annotation, visualization, and analysis tools to interpret complex histological data, including workflows for cell segmentation, classification, and quantification (\hyperref[fig:fig7]{Figure 7}). 

\begin{figure}[h!]
    \centering
    \includegraphics[width=\textwidth]{images/Figure_7.pdf}
    \caption{Visualization of model-generated cell quantification annotations (left) and the corresponding unannotated slide (right) in QuPath}
    \label{fig:fig7}
\end{figure}

To identify the regions in a WSI critical for prognosticating tumor development, such as specific tumor areas or border regions without overlapping healthy tissue, the pathologist uses QuPath to outline these regions. Then, the pathologist initiates a cell segmentation and classification script through the QuPath interface for the selected regions. The resulting annotations and quantified cell information are then directly overlaid onto the WSI in the QuPath interface. Additional design and implementation details are in \hyperref[chap:S7]{Appendix S7}. 

Two common approaches for integrating deep learning models into QuPath are Java‑based native QuPath extensions \cite{Goldsborough_Philps_etal._2024} and the execution of RESTful API requests to a model server coupled with handling the response via an extension, as demonstrated in the application of cell segmentation models applied to immunofluorescence images \cite{Sugawara_2023}. While the community is actively working on these integration strategies, there is currently no universal solution that fully addresses all integration and performance requirements.

Extensions may offer better integration with QuPath, allowing slightly improved performance and more widespread usage of the built-in QuPath models, but they lack the flexibility to customize models and modify their behavior. For example, the newest version of QuPath includes models such as StarDist \cite{Weigert_Schmidt} and InstanSeg \cite{Goldsborough_Philps_etal._2024} that can perform cell segmentation. Both models pose limitations when applied to simultaneous cell segmentation and classification. StarDist performs well only on convex, round shapes by design, whereas some neoplastic, inflammatory, and connective cells exhibit complex and non-convex shapes. InstanSeg provides only semantic segmentation without assigning classes to the segmented cells.

%\hfill

In contrast, our approach offers an alternative integration strategy. It utilizes the paquo library to directly interact with QuPath’s internal application programming interface from within Python. This enables data exchange and processing without the need for intermediate conversion steps and provides greater control over model customization, retraining, and the incorporation of custom processing steps.

The integration of our custom model with QuPath underscores its potential to significantly enhance the diagnostic process by reducing the time burden on pathologists and enabling them to focus on more complex interpretative tasks using familiar software. Leveraging a tool that is already well-established among pathologists increases the likelihood of its adoption into daily clinical workflows. The quantitative data generated through the automated workflow is critical for both clinical decision-making and research, facilitating more accurate biomarker analysis, enabling robust statistical evaluations, and supporting hypothesis generation and testing. Additionally, by streamlining cell segmentation and classification, the tool enhances the scalability and reproducibility of pathological assessments, ultimately contributing to improved diagnostic accuracy and patient outcomes.

\section{Conclusion and future work}

In this study, we address critical challenges in digital pathology and tackle the usability and deployment issues of the developed models in standard computing environments without the need for high-performance computing systems. Our multi-faceted approach encompasses data refinement through cross-relabeling, leveraging foundation models for robust cell segmentation and classification, optimizing model performance via knowledge distillation, and integrating the optimized model into the QuPath software for practical application. This approach is used to construct a capable, versatile, and adjustable model for cell segmentation and classification, with enhanced performance and usability.

\begin{sloppypar}
While our approach shows potential in the field of computational pathology, certain limitations persist. 
For example, our implementation currently exhibits lower performance in detecting macrophages. 
This serves as an instance of the broader challenge of accurately identifying complex cell types. In order to address this issue, extending our approach to incorporate additional data sources, exploring alternative modeling approaches, and integrating other imaging modalities such as immunohistochemical staining may help improve detection accuracy. Moreover, although the distilled model reduces computational demands, integrating advanced deep learning models into clinical practice requires addressing technological gaps and potential resistance to adopting new tools within established diagnostic processes.
\end{sloppypar}

Future work could focus on several key areas to refine the proposed approach and facilitate its adoption in clinical environments. Enhancing the cell-relabeling process with additional datasets \cite{Graham_Jahanifar_etal._2021} could improve the representation of underrepresented cell types and enhance overall model performance. Also, incorporating additional data sources, such as multi-modal imaging or complementary staining methods, may address limitations related to cell type differentiation and class imbalance. Exploring other foundation models \cite{Vorontsov_Bozkurt_etal._2024,Zimmermann_Vorontsov_etal._2024} or introducing additional modalities \cite{Ding_Wagner_etal._2024,Vaidya_Zhang_etal._2025} may provide alternative architectures better suited to specific tasks or offer improved efficiency. Implementing more complex knowledge distillation techniques \cite{Houyon_Cioppa_etal._2023,Zhang_Song_etal._2019} could further optimize the model's performance and adaptability. Additionally, deeper integration with QuPath or other digital pathology software could provide pathologists more control over cell quantification analysis directly within the QuPath interface, thereby increasing accessibility and usability. Such enhancements would not only refine model performance but also ensure greater adaptability and scalability within various clinical environments. Finally, extensive validation of the model by pathologists and benchmarking against independent datasets are essential steps toward establishing the model's reliability and fostering confidence in its clinical utility.

\section*{Acknowledgments} 
This work was funded in part by the Research Council of Norway grant no. 309439 SFI Visual Intelligence, and the North Norwegian Health Authority grant no. HNF1521-20.

\bibliographystyle{IEEEtran}
\begin{sloppypar}
\begin{thebibliography}{99}

\bibitem{chaplot2020neural} Chaplot, Devendra Singh, et al. "Neural topological slam for visual navigation." Proceedings of the IEEE/CVF conference on computer vision and pattern recognition. 2020.

\bibitem{maksymets2021thda} Maksymets, Oleksandr, et al. "Thda: Treasure hunt data augmentation for semantic navigation." Proceedings of the IEEE/CVF International Conference on Computer Vision. 2021.

\bibitem{mezghan2022memory} Mezghan, Lina, et al. "Memory-augmented reinforcement learning for image-goal navigation." 2022 IEEE/RSJ International Conference on Intelligent Robots and Systems (IROS). IEEE, 2022.

\bibitem{al2022zero} Al-Halah, Ziad, Santhosh Kumar Ramakrishnan, and Kristen Grauman. "Zero experience required: Plug \& play modular transfer learning for semantic visual navigation." Proceedings of the IEEE/CVF Conference on Computer Vision and Pattern Recognition. 2022.

\bibitem{ye2021auxiliary} Ye, Joel, et al. "Auxiliary tasks and exploration enable objectgoal navigation." Proceedings of the IEEE/CVF international conference on computer vision. 2021.

\bibitem{chaplot2020object} Chaplot, Devendra Singh, et al. "Object goal navigation using goal-oriented semantic exploration." Advances in Neural Information Processing Systems 33 (2020)

\bibitem{ramakrishnan2022poni} Ramakrishnan, Santhosh Kumar, et al. "Poni: Potential functions for objectgoal navigation with interaction-free learning." Proceedings of the IEEE/CVF Conference on Computer Vision and Pattern Recognition. 2022.

\bibitem{ramrakhya2022habitat} Ramrakhya, Ram, et al. "Habitat-web: Learning embodied object-search strategies from human demonstrations at scale." Proceedings of the IEEE/CVF Conference on Computer Vision and Pattern Recognition. 2022.

\bibitem{mousavian2019visual} Mousavian, Arsalan, et al. "Visual representations for semantic target driven navigation." 2019 International Conference on Robotics and Automation (ICRA). IEEE, 2019.

\bibitem{dhariwal2021diffusion} Dhariwal, Prafulla, and Alexander Nichol. "Diffusion models beat gans on image synthesis." Advances in neural information processing systems 34 (2021)

\bibitem{ho2022classifier} Ho, Jonathan, and Tim Salimans. "Classifier-free diffusion guidance." arXiv preprint arXiv:2207.12598 (2022).

\bibitem{nichol2021glide} Nichol, Alex, et al. "Glide: Towards photorealistic image generation and editing with text-guided diffusion models." arXiv preprint arXiv:2112.10741 (2021)

\bibitem{brooks2023instructpix2pix} Brooks, Tim, Aleksander Holynski, and Alexei A. Efros. "Instructpix2pix: Learning to follow image editing instructions." Proceedings of the IEEE/CVF Conference on Computer Vision and Pattern Recognition. 2023.

\bibitem{fu2023guiding} Fu, Tsu-Jui, et al. "Guiding instruction-based image editing via multimodal large language models." arXiv preprint arXiv:2309.17102 (2023).

\bibitem{geng2024instructdiffusion} Geng, Zigang, et al. "Instructdiffusion: A generalist modeling interface for vision tasks." Proceedings of the IEEE/CVF Conference on Computer Vision and Pattern Recognition. 2024.

\bibitem{zhou2024minedreamer} Zhou, Enshen, et al. "Minedreamer: Learning to follow instructions via chain-of-imagination for simulated-world control." arXiv preprint arXiv:2403.12037 (2024).

\bibitem{zhou2023esc} Zhou, Kaiwen, et al. "Esc: Exploration with soft commonsense constraints for zero-shot object navigation." International Conference on Machine Learning. PMLR, 2023.

\bibitem{yu2023l3mvn} Yu, Bangguo, Hamidreza Kasaei, and Ming Cao. "L3mvn: Leveraging large language models for visual target navigation." 2023 IEEE/RSJ International Conference on Intelligent Robots and Systems (IROS). IEEE, 2023.

\bibitem{gadre2023cows} Gadre, Samir Yitzhak, et al. "Cows on pasture: Baselines and benchmarks for language-driven zero-shot object navigation." Proceedings of the IEEE/CVF Conference on Computer Vision and Pattern Recognition. 2023.

\bibitem{shah2023navigation} Shah, Dhruv, et al. "Navigation with large language models: Semantic guesswork as a heuristic for planning." Conference on Robot Learning. PMLR, 2023.

\bibitem{cai2024bridging} Cai, Wenzhe, et al. "Bridging zero-shot object navigation and foundation models through pixel-guided navigation skill." 2024 IEEE International Conference on Robotics and Automation (ICRA). IEEE, 2024.

\bibitem{yu2023co} Yu, Bangguo, Hamidreza Kasaei, and Ming Cao. "Co-NavGPT: Multi-robot cooperative visual semantic navigation using large language models." arXiv preprint arXiv:2310.07937 (2023).

\bibitem{wu2024voronav} Wu, Pengying, et al. "Voronav: Voronoi-based zero-shot object navigation with large language model." arXiv preprint arXiv:2401.02695 (2024).

\bibitem{qin2023mp5} Qin, Yiran, et al. "Mp5: A multi-modal open-ended embodied system in minecraft via active perception." arXiv preprint arXiv:2312.07472 (2023).

\bibitem{du2024learning} Du, Yilun, et al. "Learning universal policies via text-guided video generation." Advances in Neural Information Processing Systems 36 (2024).

\bibitem{ajay2024compositional} Ajay, Anurag, et al. "Compositional foundation models for hierarchical planning." Advances in Neural Information Processing Systems 36 (2024).

\bibitem{liang2024skilldiffuser} Liang, Zhixuan, et al. "Skilldiffuser: Interpretable hierarchical planning via skill abstractions in diffusion-based task execution." Proceedings of the IEEE/CVF Conference on Computer Vision and Pattern Recognition. 2024.

\bibitem{heusel2017gans} Heusel, Martin, et al. "Gans trained by a two time-scale update rule converge to a local nash equilibrium." Advances in neural information processing systems 30 (2017).

\bibitem{zhang2018unreasonable} Zhang, Richard, et al. "The unreasonable effectiveness of deep features as a perceptual metric." Proceedings of the IEEE conference on computer vision and pattern recognition. 2018.

\bibitem{brown2020language} Brown, Tom B. "Language models are few-shot learners." arXiv preprint arXiv:2005.14165 (2020).

\bibitem{podell2023sdxl} Podell, Dustin, et al. "Sdxl: Improving latent diffusion models for high-resolution image synthesis." arXiv preprint arXiv:2307.01952 (2023).

\bibitem{brohan2022rt} Brohan, Anthony, et al. "Rt-1: Robotics transformer for real-world control at scale." arXiv preprint arXiv:2212.06817 (2022).

\bibitem{brohan2023rt} Brohan, Anthony, et al. "Rt-2: Vision-language-action models transfer web knowledge to robotic control." arXiv preprint arXiv:2307.15818 (2023).

\bibitem{li2024manipllm} Li, Xiaoqi, et al. "Manipllm: Embodied multimodal large language model for object-centric robotic manipulation." Proceedings of the IEEE/CVF Conference on Computer Vision and Pattern Recognition. 2024.

\bibitem{shah2023vint} Shah, Dhruv, et al. "ViNT: A foundation model for visual navigation." arXiv preprint arXiv:2306.14846 (2023).

\bibitem{liu2024visual} Liu, Haotian, et al. "Visual instruction tuning." Advances in neural information processing systems 36 (2024).

\bibitem{hu2021lora} Hu, Edward J., et al. "Lora: Low-rank adaptation of large language models." arXiv preprint arXiv:2106.09685 (2021).

\bibitem{qin2023supfusion} Qin, Yiran, et al. "SupFusion: Supervised LiDAR-camera fusion for 3D object detection." Proceedings of the IEEE/CVF International Conference on Computer Vision. 2023.

\bibitem{qin2024worldsimbench} Qin, Yiran, et al. "Worldsimbench: Towards video generation models as world simulators." arXiv preprint arXiv:2410.18072 (2024).

\bibitem{yu2025gamefactory} Yu, Jiwen, et al. "GameFactory: Creating New Games with Generative Interactive Videos." arXiv preprint arXiv:2501.08325 (2025).

\bibitem{zhou2024code} Zhou, Enshen, et al. "Code-as-Monitor: Constraint-aware Visual Programming for Reactive and Proactive Robotic Failure Detection." arXiv preprint arXiv:2412.04455 (2024).

\bibitem{zhang2024ad} Zhang, Zaibin, et al. "AD-H: Autonomous Driving with Hierarchical Agents." arXiv preprint arXiv:2406.03474 (2024).

\bibitem{wang2024toward} Wang, Chaoqun, et al. "Toward Accurate Camera-based 3D Object Detection via Cascade Depth Estimation and Calibration." arXiv preprint arXiv:2402.04883 (2024).

\bibitem{huang2024story3d} Huang, Yuzhou, et al. "Story3d-agent: Exploring 3d storytelling visualization with large language models." arXiv preprint arXiv:2408.11801 (2024).

\bibitem{savinov2018semi} Savinov, Nikolay, Alexey Dosovitskiy, and Vladlen Koltun. "Semi-parametric topological memory for navigation." arXiv preprint arXiv:1803.00653 (2018).

\bibitem{majumdar2022zson} Majumdar, Arjun, et al. "Zson: Zero-shot object-goal navigation using multimodal goal embeddings." Advances in Neural Information Processing Systems 35 (2022): 32340-32352.

\bibitem{yadav2023offline} Yadav, Karmesh, et al. "Offline visual representation learning for embodied navigation." Workshop on Reincarnating Reinforcement Learning at ICLR 2023. 2023.

\bibitem{yadav2023ovrl} Yadav, Karmesh, et al. "Ovrl-v2: A simple state-of-art baseline for imagenav and objectnav." arXiv preprint arXiv:2303.07798 (2023).

\bibitem{sun2024fgprompt} Sun, Xinyu, et al. "FGPrompt: fine-grained goal prompting for image-goal navigation." Advances in Neural Information Processing Systems 36 (2024).

\bibitem{zhu2017target} Zhu, Yuke, et al. "Target-driven visual navigation in indoor scenes using deep reinforcement learning." 2017 IEEE international conference on robotics and automation (ICRA). IEEE, 2017.

\bibitem{koh2024generating} Koh, Jing Yu, Daniel Fried, and Russ R. Salakhutdinov. "Generating images with multimodal language models." Advances in Neural Information Processing Systems 36 (2024).

\bibitem{krantz2022instance} Krantz, Jacob, et al. "Instance-specific image goal navigation: Training embodied agents to find object instances." arXiv preprint arXiv:2211.15876 (2022).

\bibitem{schulman2017proximal} Schulman, John, et al. "Proximal policy optimization algorithms." arXiv preprint arXiv:1707.06347 (2017).

\bibitem{anderson2018evaluation} Anderson, Peter, et al. "On evaluation of embodied navigation agents." arXiv preprint arXiv:1807.06757 (2018).

\bibitem{lin2024navcot} Lin, Bingqian, et al. "NavCoT: Boosting LLM-Based Vision-and-Language Navigation via Learning Disentangled Reasoning." arXiv preprint arXiv:2403.07376 (2024).

\bibitem{NavGPT} Zhou, Gengze, Yicong Hong, and Qi Wu. "Navgpt: Explicit reasoning in vision-and-language navigation with large language models." Proceedings of the AAAI Conference on Artificial Intelligence.

\bibitem{hahn2021no} Hahn, Meera, et al. "No rl, no simulation: Learning to navigate without navigating." Advances in Neural Information Processing Systems 34 (2021): 26661-26673.

\bibitem{li2025t2isafety} Li, Lijun, et al. "T2ISafety: Benchmark for Assessing Fairness, Toxicity, and Privacy in Image Generation." arXiv preprint arXiv:2501.12612 (2025).

\bibitem{an2024agfsync} An, Jingkun, et al. "AGFSync: Leveraging AI-Generated Feedback for Preference Optimization in Text-to-Image Generation." arXiv preprint arXiv:2403.13352 (2024).


\end{thebibliography}
\end{sloppypar}

\clearpage
\beginsupplement
\section*{Appendix}
\renewcommand{\thesubsection}{S\arabic{subsection}}

\subsection{\label{chap:S1}PanNuke and MoNuSAC preprocessing}
The PanNuke dataset comprises a set of 7,901 RGB patches, each with dimensions of $256 \times 256$ pixels, which we set as the standard patch size for our analysis. In contrast, the MoNuSAC dataset encompasses 294 images of heterogeneous dimensions. To standardize the MoNuSAC images with our experiments, we implement a standardization protocol. Specifically, for images exceeding the dimensions of $256 \times 256$ pixels, we segment them into equal-sized patches and apply mirror padding to the remaining portions to avoid information loss at the peripherals. Patches with dimensions less than $128 \times 128$ pixels are excluded from the dataset due to the insufficient resolution to capture relevant cellular details. For patches where either dimension falls between 128 and 256 pixels, we employ upsampling to achieve the standard patch size. As a result, we obtain a total of 2,823 RGB patches derived from the MoNuSAC dataset for subsequent analysis. For additional details on the MoNuSAC data preparation process, refer to the source code \cite{Shvetsov_2025a}.
\clearpage

\subsection{\label{chap:S2}Data usage for the methodology}

\counterwithin{figure}{subsection}
\renewcommand{\thefigure}{S\arabic{subsection}}

\begin{figure}[h!]
    \centering
    \includegraphics[width=\textwidth, height=0.85\textheight, keepaspectratio]{images/A2.pdf}
    \caption{Overview of the methodology for cross-labeling, dataset refinement, and model comparison. (1) Cross-relabeling - training and testing cell classification models, (2) Cross-relabeling - using cell classification models to create refined dataset, (3) Fine-tuning and training models for comparison, (4) Student knowledge distillation with refined dataset}
    \label{fig:S2}
\end{figure}
\clearpage

\subsection{\label{chap:S3}Confusion matrices for classification models}
\counterwithin{figure}{subsection}
\renewcommand{\thefigure}{S\arabic{subsection}.\arabic{figure}}

\begin{figure}[h!]
    \centering
    \includegraphics[width=\textwidth, height=0.4\textheight, keepaspectratio]{images/A3_1.pdf}
    \caption{Confusion matrix for PanNuke trained model}
    \label{fig:S3.1}
\end{figure}

\begin{figure}[h!]
    \centering
    \includegraphics[width=\textwidth, height=0.4\textheight, keepaspectratio]{images/A3_2.pdf}
    \caption{Confusion matrix for MoNuSAC trained model}
    \label{fig:S3.2}
\end{figure}

\clearpage

\subsection{\label{chap:S4}Datasets cell counts}

\counterwithin{table}{subsection}
\renewcommand{\thetable}{S\arabic{subsection}}

\begin{table}[h!]
\renewcommand{\arraystretch}{2.0}
\centering
\caption{\label{tab:S4}Cell counts for PanNuke, MoNuSAC and refined datasets. Numbers in parentheses indicate preprocessed cell counts for cell classifier models training and testing.}
%\adjustbox{max width=\textwidth}{%
\begin{tabular}{|l|c|c|c|}
\hline
%\rowcolor{gray!30}
Cell type & PanNuke & MoNuSAC & Refined \\
\hline
Neoplastic & 77,403 (68,031) & - & 105,451 \\
\hline
Epithelial & 26,572 (23,207) & - & 29,926 \\
\hline
Epithelial (benign and malignant) & - & 31,402 & - \\
\hline
Inflammatory & 32,276 & - & - \\
\hline
Lymphocytes & - & 37,045 (33,104) & 65,275 \\
\hline
Neutrophils & - & 1,355 (1,252) & 3,833 \\
\hline
Macrophage & - & 1,842 (1,695) & 3,410 \\
\hline
Dead & 2,908 & - & 2,908 \\
\hline
Connective & 50,585 & - & 50,585 \\
\hline
\end{tabular}
%
%}
\end{table}



\clearpage

\subsection{\label{chap:S5}Definition of validation metrics}
\counterwithin{equation}{subsection}
\renewcommand{\theequation}{\arabic{equation}}

\subsubsection{\label{chap:S5.1}R\textsuperscript{2}}
The coefficient of determination, denoted as $R^2$, is a statistical measure that represents the proportion of variance in the dependent variable that is predictable from the independent variables. In the context of cell quantification in pathology, $R^2$ is used to assess how well the predicted quantities of different cell types in a patch align with the actual quantities observed in the ground truth data, with higher values representing more accurate quantification. $R^2$ is defined as
\begin{equation*}
R^2 = 1 - \frac{\sum_{i=1}^n (y_i - \hat{y}_i)^2}{\sum_{i=1}^n (y_i - \bar{y})^2},
\end{equation*}
where $y_i$ represents the actual number of cells of a specific type in the $i$-th image, $\hat{y}_i$ represents the predicted number of cells of that type in the $i$-th image, $\bar{y}$ is the mean of the actual numbers across all images, and $n$ is the total number of images in the dataset.

The $R^2$ metric has a range of $(-\infty, 1]$. An $R^2$ of 1 indicates perfect prediction, where all predicted values exactly match the actual values. An $R^2$ of 0 suggests that the model explains none of the variability of the response data around its mean. If $R^2$ is negative, it indicates that the model performs worse than a model that simply predicts the mean of the actual values for all observations.

\subsubsection{\label{chap:S5.2}PQ}
Panoptic Quality ($PQ$) is a comprehensive metric used to evaluate the performance of segmentation models in tasks that require both instance segmentation and classification. $PQ$ provides a single score that encapsulates both the detection accuracy (i.e., how many objects were correctly identified) and the segmentation quality (i.e., how accurately the objects' boundaries were delineated). This metric is particularly useful in multiclass scenarios where each pixel is classified into distinct categories, such as different cell types in pathology images.

$PQ$ is calculated as the product of two terms: Detection Quality ($DQ$) and Segmentation Quality ($SQ$). It can be expressed as
\begin{equation*}
PQ = DQ \cdot SQ,
\end{equation*}
where
\begin{equation*}
DQ = \frac{TP}{TP + 0.5\, FP + 0.5\, FN},
\end{equation*}
\begin{equation*}
SQ = \frac{\sum_{(p, g) \in \mathcal{M}} IoU(p, g)}{TP}.
\end{equation*}
In these formulas, $TP$ denotes the number of correctly matched instances between ground truth and prediction, $FP$ denotes the predicted instances that have no corresponding ground truth, $FN$ denotes the ground truth instances that were not detected, $IoU(p, g)$ is the Intersection over Union for a pair of matched instances $p$ (prediction) and $g$ (ground truth), and $\mathcal{M}$ is the set of matched pairs.

The $PQ$ metric is calculated for each class and is averaged across classes to provide a global performance measure.

The $PQ$ score has a range of $[0, 1.0]$, where a higher score indicates better performance in both detecting and segmenting the instances correctly. A $PQ$ of 1 signifies perfect identification and segmentation of all instances, whereas a $PQ$ of 0 indicates that no instances were correctly identified and segmented.

\clearpage

\subsection{\label{chap:S6}Segmentation and Detection quality metrics for teacher and student models}

\begin{table}[h!]
\renewcommand{\arraystretch}{2.0}
\centering
\caption{Segmentation and detection quality for student and teacher models (CI 95\%)}
\label{tab:S6}
%\adjustbox{max width=\textwidth}{%
\begin{tabular}{|l|c|c|}
\hline
%\rowcolor{gray!30}
Metric & Teacher & Student \\
\hline
$SQ_{neoplastic}$ & 0.819 (0.815--0.823) & 0.824 (0.819--0.828) \\
\hline
$SQ_{lymphocyte}$ & 0.795 (0.788--0.802) & 0.790 (0.783--0.796) \\
\hline
$SQ_{connective}$ & 0.770 (0.762--0.776) & 0.780 (0.772--0.786) \\
\hline
$SQ_{dead}$ & 0.659 (0.623--0.688) & 0.657 (0.624--0.695) \\
\hline
$SQ_{epithelial}$ & 0.780 (0.770--0.790) & 0.788 (0.779--0.797) \\
\hline
$SQ_{macrophage}$ & 0.788 (0.760--0.810) & 0.757 (0.730--0.783) \\
\hline
$SQ_{neutrofil}$ & 0.782 (0.761--0.801) & 0.775 (0.759--0.792) \\
\hline
$DQ_{neoplastic}$ & 0.706 (0.692--0.719) & 0.727 (0.712--0.741) \\
\hline
$DQ_{lymphocyte}$ & 0.675 (0.656--0.698) & 0.713 (0.691--0.734) \\
\hline
$DQ_{connective}$ & 0.566 (0.546--0.584) & 0.583 (0.565--0.602) \\
\hline
$DQ_{dead}$ & 0.410 (0.361--0.465) & 0.435 (0.306--0.561) \\
\hline
$DQ_{epithelial}$ & 0.668 (0.639--0.694) & 0.673 (0.644--0.702) \\
\hline
$DQ_{macrophage}$ & 0.657 (0.583--0.727) & 0.615 (0.531--0.703) \\
\hline
$DQ_{neutrofil}$ & 0.691 (0.625--0.753) & 0.729 (0.679--0.778) \\
\hline
\end{tabular}
%
%}
\end{table}

\clearpage

\subsection{\label{chap:S7}QuPath integration method}
We adopt an integration strategy leveraging the paquo \cite{Bayer_AG} library, a Python package that enables direct interaction with QuPath’s internal API, thereby facilitating seamless data exchange without intermediate conversion steps. The data processing pipeline (\hyperref[fig:S7]{Appendix Figure S7}) begins with the acquisition of WSIs and their associated annotations from QuPath, which are represented as Shapely \cite{Gillies_Wel_etal._2024} polygons. Utilizing paquo, we directly read, create, and modify these annotations and detections within a QuPath project in the Python environment. Images are then cropped using these polygons and processed by cell segmentation and classification models employing standard vision processing toolkits such as OpenCV, pyvips, and PyTorch. Additionally, QuPath employs Groovy scripts to initiate a Python process that starts the entire pipeline from QuPath graphical interface: fetching polygons, extracting images from them, and running deep learning model inference on the cropped images. 
The results are returned to QuPath, leveraging paquo's Python bindings to manipulate QuPath data while minimizing the computational overhead typically associated with cross-environment communication.

\counterwithin{figure}{subsection}
\renewcommand{\thefigure}{S\arabic{subsection}}

\begin{figure}[h!]
    \centering
    \includegraphics[width=\textwidth]{images/A7.pdf}
    \caption{QuPath integration workflow using Python environment}
    \label{fig:S7}
\end{figure}

Compared to traditional workflows that involve exporting annotations as GeoJSON, classifying them in Python, and reimporting them into QuPath, our approach offers several advantages. We eliminate the need to switch between programming languages, providing a cohesive and streamlined development process entirely within QuPath software and removing the necessity to use other tools. Meanwhile, we avoid storing annotations as intermediate JSON files unless required for external use or archiving. By conducting the entire inference and post-processing workflow within the Python environment, we leverage the power and flexibility of Python libraries for image processing and machine learning. This approach also enables adjustments to any set of labels and models, thereby improving its applicability.

%\hfill

The distilled model and QuPath integration code are packaged into a Docker container, enabling streamlined execution with the Docker engine. Detailed integration code and deployment instructions can be found in the GitHub repository \cite{Shvetsov_2025b}.

Despite these benefits, we acknowledge that the paquo library is a proof‑of‑concept project in its early development stage and has not been tested across all versions of QuPath.

\clearpage

\subsection{\label{chap:S8}Data and code availability statement}
All datasets, models, and code used in this study are publicly available and can be obtained from the repositories listed below. 
The PanNuke \cite{Gamper_Koohbanani_etal._2019} and MoNuSAC \cite{Verma_Kumar_etal._2021} datasets are publicly accessible, and download information along with detailed descriptions can be found in their respective articles. Preprocessing scripts for PanNuke and MoNuSAC data, as well as individual cell extraction scripts, are available on GitHub \cite{Shvetsov_2025a}. The H-Optimus foundation model used in our experiments can be downloaded from the HuggingFace repository \cite{hoptimus2024}, and model information is available on GitHub \cite{Saillard_Jenatton_etal._2024}. In addition, the integration code for QuPath and the distilled model packaged in a Docker container are provided in the repository \cite{Shvetsov_2025b}, and paquo Python library is available from the authors GitHub repository \cite{Bayer_AG}.
\clearpage

\end{document}


\newpage

\appendix

\section{Model Structure Details}\label{appendix-model}

This section provides more details of the model structure, mainly about the decoder and the linear attention module, as shown in Fig. \ref{linear attention}. At each step $t$, the model takes the encoding of the first node, the current node, and the remaining unselected nodes for decoding. Denote the first and current nodes’ embeddings $h_f^{(0)}$ and $h_c^{(0)}$, respectively. We first augment them by $r_f$ and $r_c$ channels to obtain the virtual representative nodes: 
\begin{equation}
\Tilde{H}^0 = [reshape(h_f W_f), reshape(h_c W_c)],
\end{equation}
where $[\cdot, \cdot·]$ is the horizontal concatenation operator, $W_f \in \mathbb{R}^{d\times(d\times r_f)}$, $W_c \in \mathbb{R}^{d\times(d\times r_c)}$. Here the $reshape$ is to keep the embedding dimension of the representative node the same with the remaining unselected nodes, resulting $\Tilde{H}^0 \in \mathbb{R}^{(r_f + r_c)\times d}$, which means the number of virtual representative nodes equals the number of the channels. 

Then, we have $r = r_f + r_c $ virtual representative nodes embeddings $\Tilde{H}^{(0)} = \{h_j^{(0)} | j=1,2, ...,r\}$. Denote the remaining unselected nodes as $H_a^{(0)} = \{h_i^{(0)} | i = 1,2, ..., m-t\}$ at step $t$ for the hyper-graph of size $m$, the linear attention module first aggregates all information into representative nodes, then broadcasts the information to all nodes. 

The aggregating and broadcasting processes are realized by two different attention layers. Recall the formulation of classical attention mechanism: note the queries $X_Q \in \mathbb{R}^{q\times d} $, keys $X_K \in \mathbb{R}^{k\times d} $ and values $X_V \in \mathbb{R}^{v\times d}$ as inputs, the classical attention mechanism can be formulated as: 
\begin{equation}
\begin{split}
Attn&(X_Q, X_K, X_V) = \\
 &softmax(\frac{X_Q W_Q(X_K W_K)^\top}{\sqrt{d}})X_V W_V,
\end{split}
\end{equation}


Then, for the $l-th$ linear attention module, denote the input representative nodes' embeddings $\Tilde{H}^{(l-1)}$ and the remaining unvisited nodes’ embeddings $H_a^{(l-1)} = \{h_i^l, i=1, . . . , m\}$, we concat $\Tilde{H}^{(l-1)}$ and $H_a^{(l-1)}$ as $H_{all}^{(l-1)} = [\Tilde{H}^{(l-1)}, H_a^{(l-1)}]$. Then, the aggregating attention layer attends representative nodes to all nodes: 
\begin{equation}
\begin{split}
    Agg = Attn( & \Tilde{H}^{(l-1)}, H_{all}^{(l-1)}, H_{all}^{(l-1)}),
\end{split}
\end{equation}
and the broadcasting attention layer attends all nodes to the aggregations: 
\begin{equation}
Brd = Attn(H_{all}^{(l-1)}, Agg, Agg),
\end{equation}
where $Brd \in \mathbb{R}^{(r+m-t)\times d}$, which we can split into the representative nodes' embeddings $\Tilde{H}^{(l)}$ and the remaining unselected nodes’ embeddings $H_a^{(l)}$. 

Finally, after $L$ linear attention modules, we obtain the hidden representation $\Tilde{H}^{(L)}$ and and $H_a^{(L)}$. Then we take only the embeddings of remaining unselected nodes $H_a^{(L)}$ to calculate the probability for selecting the next node.

\begin{figure}[htbp]
\centering
\includegraphics[width=0.35\textwidth]{graph/linear_attention.pdf} 
\caption{Linear attention module}
\label{linear attention}
\end{figure}


\section{Sample Size Alignment}\label{appendix-alignement}


We employ the sample size alignment to make the hyper-graph size the same within a batch. The key concept is to precompute the size of the hyper-graph before actually implementing different cases of destruction. This process requires a predefined order of node destruction. In our clustering scenario, we begin with a central node, with nodes closer to the center being destroyed earlier. When an additional node is destroyed, the number of nodes that newly appear in the hyper-graph depends on whether the edges connecting the node to its neighbors (first-order and second-order) have already been destroyed. Fig. \ref{node_emerge} provides an example where the node targeted for destruction is connected with its four neighbors. After the node is destroyed, three new nodes emerge in the hyper-graph. We detail all cases of node connections and their corresponding results of node emergence when enlarging the destruction area in Fig. \ref{case-destruction}. 

\begin{figure}[htbp]
\centering
\includegraphics[width=0.4\textwidth]{graph/node_emerge.pdf} 
\caption{Node emergence in hyper-graph following additional node destruction}
\label{node_emerge}
\end{figure}

\begin{figure}[h]
\centering
\includegraphics[width=0.45\textwidth]{graph/case-destruction.pdf} 
\caption{Different cases of node connections and corresponding results of newly appearing nodes when enlarging the destruction area}
\label{case-destruction}
\end{figure}

There are six different cases and we indicate the newly appearing nodes by red color in each case. In case 1, destroying one more node generates two endpoint nodes and an isolated node. In case 2, that generates an endpoint node and an isolated node. In case 3, the newly destroyed node changes from a middle node to an isolated node. In case 4, the newly destroyed node becomes an endpoint node. In cases 5, the newly destroyed node is an endpoint node before and becomes an isolated nodes, but does not change the hyper-graph size. In case 6, the newly destroyed node has already been disconnected with its two neighbors before, therefore none of new nodes emerging in the hyper-graph due to the destruction. In summary, the number of newly appearing nodes equals to the number of relevant undestroyed edges minus one, except the case 6 where all edges have already been destroyed. 

We detail the algorithm of sample size alignment in Algorithm \ref{alignment}. First, we calculate the distance of all nodes to the center node and sort these node by the distance in ascending order. Then, for the node to destroy, if its neighbor is nearer to the center node, the neighbor will be destroyed before the node and the edge between them will be disconnect. Line 5 in the algorithm details the conditions where the node is still connected with its neighbors. After that, we can calculate the number of emerging nodes as line 6. Finally, by summing all emerging nodes in order, we can determine the hyper-graph size corresponding to a specific destruction scenario, and obtain the mask indicating which node should be destroyed for a target hyper-graph size.





\begin{algorithm}[]
    \caption{Sample Size Alignment}
	\begin{algorithmic}[1]

         \STATE {\bfseries Input:} 
         \newline The coordinates of a batch of VRP instances $X$,
         \newline the initial solution $S$, 
         \newline the cluster center $c=(x_c, y_c)$, 
         \newline the target reduced hyper-graph size $k$; 
         \STATE {\bfseries Notation:} 
         \newline $cumsum()$: the function to calculate the cumulative sum of a tensor along a dimension,
         \newline  $dist()$: the function to calculate the distance between two nodes,
         \newline $sort()$: sort a sequence of numbers in ascending order;
	    \STATE {\bfseries Output:} The mask $M$ indicating the node to destroy; 
        \STATE $D=dist(X, c)$, $\pi = sort(D)$; 
        \STATE For all nodes and their first-order neighbor $1A, 1B$ and the second-order neighbors $2A, 2B$ in the solutions, compare the distances to determine if they are connected with their neighbors before they are destroyed: 
        \newline \hspace*{3mm} $connect_{1A} = (D_{1A} \textgreater D)$,
        \newline \hspace*{3mm} $connect_{1B} = (D_{1B} \textgreater D)$,
        \newline \hspace*{3mm} $connect_{2A} = (D_{2A} \textgreater D)\ AND \ connect_{1A} $, 
        \newline \hspace*{3mm} $connect_{2B} = (D_{2B} \textgreater D)\ AND \ connect_{1B} $;
        \STATE Calculate the number of newly appearing nodes as illustrated in Fig. \ref{case-destruction}: 
        \newline \hspace*{3mm} $N = max(0, sum(connect_{1A} + connect_{1B} + connect_{2A}) + connect_{2B}) - 1)$; 
	    \STATE $H=cumsum(N)$ in the order of $\pi$; 
        \STATE $M = (H \leq k)$.
	\end{algorithmic}
	\label{alignment}
\end{algorithm}






\section{Additional Results of TSPLib}\label{appendix-tsplib}

The detailed results of TSPLib are shown in Table \ref{Detailed Tsplib result} and Table \ref{Detailed-TSPLib-continue}. Note that in two instances, rl11849 and usa13509, the LEHD \cite{luo2023lehd} needs about 80 hours and 120 hours to perform 1000 reconstruction steps. We stop the iteration at 24 hours as we observe that the LEHD has not made any progress for a long time. In most instances, our DRHG achieves the lowest optimality gap. The advantage of DRHG becomes more pronounced in hard instances of larger size or special distribution. 

\paragraph{large-size problem} In large-size problem, POMO \cite{kwon2020pomo} suffers the most from poor generalization ability, and its performance drops dramatically. BQ \cite{drakulic2023bq} and LEHD \cite{luo2023lehd} generalize better but still struggle on cases with thousands of nodes. When the problem size grows beyond 3,000, BQ fails to solve the problem due to out-of-memory, so to LEHD when the problem size comes to about 15,000. The DRHG succeeds in solving all instances and obtains solutions with optimality gap much lower than the others.

\paragraph{special distribution} Excepting the instances consisting of the real-world cities, the TSPlib also incorporates some instances of special distributions, such as the drilling problems (starting with 'd', 'pcb' and 'u'), and the rattled-grid problems (strat with 'rat'). On these distributions, DRHG also outperforms the other methods. 

% These results demonstrate that DRHG is robust to scale change and distribution shift.




\begin{table}[htbp]
  \centering
  \resizebox{0.9\columnwidth}{!}{

\begin{tabular}{ccccc}
\toprule
\textbf{Case} & \textbf{POMO-augx8} & \textbf{BQ-bs16} & \textbf{LEHD-RRC100} & \textbf{DRHG-T=1000} \\
\midrule
a280  & 12.62\% & 0.39\% & \textbf{0.30\%} & 0.34\% \\
\midrule
berlin52 & 0.04\% & \textbf{0.03\%} & \textbf{0.03\%} & \textbf{0.03\%} \\
\midrule
bier127 & 12.00\% & 0.68\% & \textbf{0.01\%} & \textbf{0.01\%} \\
\midrule
brd14051 & OOM   & OOM   & OOM   & \textbf{4.02\%} \\
\midrule
ch130 & 0.16\% & 0.13\% & \textbf{0.01\%} & \textbf{0.01\%} \\
\midrule
ch150 & 0.53\% & 0.39\% & \textbf{0.04\%} & \textbf{0.04\%} \\
\midrule
d1291 & 77.24\% & 5.97\% & 2.71\% & \textbf{2.09\%} \\
\midrule
d15112 & OOM   & OOM   & OOM   & \textbf{3.41\%} \\
\midrule
d1655 & 80.99\% & 9.67\% & 5.16\% & \textbf{1.57\%} \\
\midrule
d18512 & OOM   & OOM   & OOM   & \textbf{3.63\%} \\
\midrule
d198  & 19.89\% & 8.77\% & 0.71\% & \textbf{0.26\%} \\
\midrule
d2103 & 75.22\% & 15.36\% & \textbf{1.22\%} & 1.82\% \\
\midrule
d493  & 58.91\% & 8.40\% & 0.92\% & \textbf{0.31\%} \\
\midrule
d657  & 41.14\% & 1.34\% & 0.91\% & \textbf{0.21\%} \\
\midrule
eil101 & 1.84\% & 1.78\% & 1.78\% & \textbf{1.78\%} \\
\midrule
eil51 & 0.83\% & \textbf{0.67\%} & \textbf{0.67\%} & \textbf{0.67\%} \\
\midrule
eil76 & \textbf{1.18\%} & 1.24\% & 1.18\% & \textbf{1.18\%} \\
\midrule
fl1400 & 47.36\% & 11.60\% & 3.45\% & \textbf{1.43\%} \\
\midrule
fl1577 & 71.17\% & 14.63\% & 3.71\% & \textbf{3.08\%} \\
\midrule
fl3795 & 126.86\% & OOM   & 7.96\% & \textbf{4.61\%} \\
\midrule
fl417 & 18.51\% & 5.11\% & 2.87\% & \textbf{0.49\%} \\
\midrule
fnl4461 & OOM   & OOM   & 12.38\% & \textbf{1.20\%} \\
\midrule
gil262 & 2.99\% & 0.72\% & \textbf{0.33\%} & \textbf{0.33\%} \\
\midrule
kroA100 & 1.58\% & 0.02\% & \textbf{0.02\%} & \textbf{0.02\%} \\
\midrule
kroA150 & 1.01\% & 0.01\% & \textbf{0.00\%} & \textbf{0.00\%} \\
\midrule
kroA200 & 2.93\% & 0.50\% & \textbf{0.00\%} & \textbf{0.00\%} \\
\midrule
kroB100 & 0.93\% & 0.01\% & \textbf{-0.01\%} & \textbf{-0.01\%} \\
\midrule
kroB150 & 2.10\% & \textbf{-0.01\%} & \textbf{-0.01\%} & \textbf{-0.01\%} \\
\midrule
kroB200 & 2.04\% & 0.22\% & \textbf{0.01\%} & \textbf{0.01\%} \\
\midrule
kroC100 & 0.20\% & 0.01\% & \textbf{0.01\%} & \textbf{0.01\%} \\
\midrule
kroD100 & 0.80\% & 0.00\% & \textbf{0.00\%} & \textbf{0.00\%} \\
\midrule
kroE100 & 1.31\% & 0.07\% & \textbf{0.00\%} & 0.17\% \\
\midrule
lin105 & 1.31\% & 0.03\% & \textbf{0.03\%} & \textbf{0.03\%} \\
\midrule
lin318 & 10.29\% & 0.35\% & \textbf{0.03\%} & 0.30\% \\
\midrule
linhp318 & 12.11\% & 2.01\% & 1.74\% & \textbf{1.69\%} \\
\midrule
nrw1379 & 41.52\% & 3.34\% & 8.78\% & \textbf{1.41\%} \\
\midrule
p654  & 25.58\% & 4.44\% & 2.00\% & \textbf{0.03\%} \\
\midrule
pcb1173 & 45.85\% & 3.95\% & 3.40\% & \textbf{0.39\%} \\
\midrule
pcb3038 & 63.82\% & OOM   & 7.23\% & \textbf{1.01\%} \\
\midrule
pcb442 & 18.64\% & 0.95\% & \textbf{0.04\%} & 0.27\% \\
\midrule
pr1002 & 43.93\% & 2.94\% & 0.77\% & \textbf{0.67\%} \\
\midrule
pr107 & 0.90\% & 13.94\% & \textbf{0.00\%} & \textbf{0.00\%} \\
\midrule
pr124 & 0.37\% & 0.08\% & \textbf{0.00\%} & \textbf{0.00\%} \\
\midrule
pr136 & 0.87\% & \textbf{0.00\%} & \textbf{0.00\%} & \textbf{0.00\%} \\

\midrule
pr144 & 1.40\% & 0.19\% & 0.09\% & \textbf{0.00\%} \\
\midrule
pr152 & 0.99\% & 8.21\% & 0.27\% & \textbf{0.19\%} \\
\midrule
pr226 & 4.46\% & 0.13\% & \textbf{0.01\%} & \textbf{0.01\%} \\
\midrule
pr2392 & 69.78\% & 7.72\% & 5.31\% & \textbf{0.56\%} \\
\midrule
pr264 & 13.72\% & 0.27\% & \textbf{0.01\%} & \textbf{0.01\%} \\
\midrule
pr299 & 14.71\% & 1.62\% & 0.10\% & \textbf{0.02\%} \\
\midrule
pr439 & 21.55\% & 2.01\% & 0.33\% & \textbf{0.12\%} \\
\midrule
pr76  & 0.14\% & \textbf{0.00\%} & \textbf{0.00\%} & \textbf{0.00\%} \\
\midrule
rat195 & 8.15\% & 0.60\% & 0.61\% & \textbf{0.57\%} \\
\midrule
rat575 & 25.52\% & 0.84\% & 1.01\% & \textbf{0.36\%} \\
\midrule
rat783 & 33.54\% & 2.91\% & 1.28\% & \textbf{0.47\%} \\
\midrule
rat99 & 1.90\% & \textbf{0.68\%} & \textbf{0.68\%} & \textbf{0.68\%} \\
\midrule
rd100 & 0.01\% & \textbf{0.01\%} & \textbf{0.01\%} & \textbf{0.01\%} \\
\midrule
rd400 & 13.97\% & 0.32\% & \textbf{0.02\%} & 0.36\% \\

\bottomrule
\end{tabular}%
   }
    \caption{Detailed results of TSPLib}
  \label{Detailed Tsplib result}%
\end{table}%



\begin{table}[htbp]
  \centering
  \resizebox{0.9\columnwidth}{!}{
\begin{tabular}{ccccc}
\toprule
\textbf{Case} & \textbf{POMO-augx8} & \textbf{BQ-bs16} & \textbf{LEHD-RRC100} & \textbf{DRHG-T=1000} \\



\midrule
rl11849 & OOM   & OOM   & 21.43\% & \textbf{3.94\%} \\
\midrule
rl1304 & 67.70\% & 5.07\% & 1.96\% & \textbf{0.79\%} \\
\midrule
rl1323 & 68.69\% & 4.41\% & 1.71\% & \textbf{1.26\%} \\
\midrule
rl1889 & 80.00\% & 7.90\% & 2.90\% & \textbf{0.95\%} \\
\midrule
rl5915 & OOM   & OOM   & 11.21\% & \textbf{1.97\%} \\
\midrule
rl5934 & OOM   & OOM   & 11.11\% & \textbf{2.69\%} \\
\midrule
st70  & \textbf{0.31\%} & \textbf{0.31\%} & \textbf{0.31\%} & \textbf{0.31\%} \\
\midrule
ts225 & 4.72\% & \textbf{0.00\%} & \textbf{0.00\%} & \textbf{0.00\%} \\
\midrule
tsp225 & 6.72\% & -0.43\% & \textbf{-1.46\%} & \textbf{-1.46\%} \\
\midrule
u1060 & 53.50\% & 7.04\% & 2.80\% & \textbf{0.48\%} \\
\midrule
u1432 & 38.48\% & 2.70\% & 1.92\% & \textbf{0.49\%} \\
\midrule
u159  & 0.95\% & \textbf{-0.01\%} & \textbf{-0.01\%} & \textbf{-0.01\%} \\
\midrule
u1817 & 70.51\% & 6.12\% & 4.15\% & \textbf{2.05\%} \\
\midrule
u2152 & 74.08\% & 5.20\% & 4.90\% & \textbf{2.24\%} \\
\midrule
u2319 & 26.43\% & 1.33\% & 1.99\% & \textbf{0.30\%} \\
\midrule
u574  & 30.83\% & 2.09\% & 0.69\% & \textbf{0.24\%} \\
\midrule
u724  & 31.66\% & 1.57\% & 0.76\% & \textbf{0.27\%} \\
\midrule
usa13509 & OOM   & OOM   & 34.65\% & \textbf{11.82\%} \\
\midrule
vm1084 & 48.15\% & 5.93\% & 2.17\% & \textbf{0.14\%} \\
\midrule
vm1748 & 62.05\% & 6.04\% & 2.61\% & \textbf{0.54\%} \\
\bottomrule
\end{tabular}%

   }
    \caption{Detailed results of TSPLib (continued)}
  \label{Detailed-TSPLib-continue}%
\end{table}%

\begin{figure*}[!h]
\centering
\includegraphics[width=0.95\textwidth]{graph/DR_demo-2.pdf} % 
\caption{The demonstration of the destroy-and-repair process of a TSP instance}
\label{fig-DR_demo}
\end{figure*}


\section{Influence of Hyper-parameters}\label{appendix-hyper-param}
\paragraph{Effect of training sample size} The size of the training sample influences model performance by altering the distribution. A sample size that is too small may oversimplify the task, whereas a sample size that is too large may result in insufficient undestroyed segments for the model to effectively learn the repair process. We investigate this effect by training the model on TSP100, keeping all other settings constant. Table \ref{effect_train_size} shows the results of three settings where the sample size $\in [30, 70],\ [20, 80]$ and $[10, 90]$. The case where training sample size $\in [20, 80]$ performs the best.
\paragraph{Effect of destruction degree in the inference} Table \ref{destruction_size} illustrates the impact of destruction degree during inference on performance. A higher degree of destruction can be more efficient than a lower one, provided that the repair quality remains consistent. However, since the model is trained with no more than 100 nodes, repair quality diminishes when the destruction becomes too extensive. Destroying $k$ nodes with $k \in [20, 200]$ yields the best overall results.




\begin{table}[htbp]
  \centering
  \resizebox{0.8\columnwidth}{!}{

        \begin{tabular}{l|c|c|c}
        \toprule
        \multicolumn{1}{c|}{\multirow{2}[4]{*}{Gap}} & \multicolumn{3}{c}{Training sample size} \\
        \cmidrule{2-4}      & [30, 70] & [20, 80] & [10, 90] \\
        \midrule
        TSP100 & 0.0008\% & 0.0005\% & \textbf{0.0004\%} \\
        TSP200 & 0.014\% & \textbf{0.0098\%} & 0.0149\% \\
        TSP500 & 0.127\% & \textbf{0.113\%} & 0.121\% \\
        TSP1K & 0.268\% & \textbf{0.258\%} & 0.271\% \\
        TSP5K  & 1.47\% & \textbf{1.42\%} & 1.63\% \\
        TSP10K & 3.06\% & \textbf{2.85\%} & 3.384\% \\
        \bottomrule
        \end{tabular}%
   }
    \caption{Effect of training sample size}
  \label{effect_train_size}%
\end{table}%



\begin{table}[htbp]
  \centering
  \resizebox{0.9\columnwidth}{!}{
    \begin{tabular}{l|c|c|c|c}
    \toprule
    \multicolumn{1}{c|}{\multirow{2}[4]{*}{Gap}} & \multicolumn{4}{c}{Destruction size in the inference} \\
    \cmidrule{2-5}      & [20, 100] & [20, 200] & [20, 500] & [20,1000] \\
    \midrule
    TSP100 & 0.0005\% &       &       &  \\
    TSP200 & 0.0567\% & \textbf{0.0098\%} &       &  \\
    TSP500 & 0.239\% & 0.113\% & \textbf{0.111\%} &  \\
    TSP1K & 0.393\% & \textbf{0.258\%} & 0.309\% & 0.449\% \\
    TSP5K  & 1.70\% & \textbf{1.42\%} & 1.67\% & 2.05\% \\
    TSP10K & 3.34\% & \textbf{2.85\%} & 3.07\% & 3.46\% \\
    \bottomrule
    \end{tabular}%
   }
    \caption{Effect of destruction degree in the inference}
  \label{destruction_size}%
\end{table}%


\section{Destroy-and-repair Demonstration}\label{appendix-demo}



Fig. \ref{fig-DR_demo} demonstrates the destroy-and-repair of a TSP instance with $n=100$. Three segments are left after the destruction. The model changes how these segments are connected during the repair and makes the contour of the solution apparently different.


\end{document}


