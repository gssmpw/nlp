%File: formatting-instructions-latex-2025.tex
%release 2025.0
\documentclass[letterpaper]{article} % DO NOT CHANGE THIS
\usepackage{aaai25}  % DO NOT CHANGE THIS
\usepackage{times}  % DO NOT CHANGE THIS
\usepackage{helvet}  % DO NOT CHANGE THIS
\usepackage{courier}  % DO NOT CHANGE THIS
\usepackage[hyphens]{url}  % DO NOT CHANGE THIS
\usepackage{graphicx} % DO NOT CHANGE THIS
\urlstyle{rm} % DO NOT CHANGE THIS
\def\UrlFont{\rm}  % DO NOT CHANGE THIS
\usepackage{natbib}  % DO NOT CHANGE THIS AND DO NOT ADD ANY OPTIONS TO IT
\usepackage{caption} % DO NOT CHANGE THIS AND DO NOT ADD ANY OPTIONS TO IT
\frenchspacing  % DO NOT CHANGE THIS
\setlength{\pdfpagewidth}{8.5in}  % DO NOT CHANGE THIS
\setlength{\pdfpageheight}{11in}  % DO NOT CHANGE THIS
%
\pdfinfo{
 /TemplateVersion (2025.1)
 }
% These are recommended to typeset algorithms but not required. See the subsubsection on algorithms. Remove them if you don't have algorithms in your paper.
\usepackage{booktabs}       % professional-quality tables
\usepackage{amsfonts}       % blackboard math symbols
\usepackage{nicefrac}       % compact symbols for 1/2, etc.
\usepackage{microtype}      % microtypography

\usepackage{tcolorbox}      % Color text box

\usepackage{subfig}
\usepackage{amsmath}
\usepackage{multirow}

\usepackage{makecell}

\usepackage{algorithm}
\usepackage{algorithmic}

%
% These are are recommended to typeset listings but not required. See the subsubsection on listing. Remove this block if you don't have listings in your paper.
\usepackage{newfloat}
\usepackage{listings}
\DeclareCaptionStyle{ruled}{labelfont=normalfont,labelsep=colon,strut=off} % DO NOT CHANGE THIS
\lstset{%
	basicstyle={\footnotesize\ttfamily},% footnotesize acceptable for monospace
	numbers=left,numberstyle=\footnotesize,xleftmargin=2em,% show line numbers, remove this entire line if you don't want the numbers.
	aboveskip=0pt,belowskip=0pt,%
	showstringspaces=false,tabsize=2,breaklines=true}
\floatstyle{ruled}
\newfloat{listing}{tb}{lst}{}
\floatname{listing}{Listing}
%
% Keep the \pdfinfo as shown here. There's no need
% for you to add the /Title and /Author tags.
\pdfinfo{
/TemplateVersion (2025.1)
}

% DISALLOWED PACKAGES
% \usepackage{authblk} -- This package is specifically forbidden
% \usepackage{balance} -- This package is specifically forbidden
% \usepackage{color (if used in text)
% \usepackage{CJK} -- This package is specifically forbidden
% \usepackage{float} -- This package is specifically forbidden
% \usepackage{flushend} -- This package is specifically forbidden
% \usepackage{fontenc} -- This package is specifically forbidden
% \usepackage{fullpage} -- This package is specifically forbidden
% \usepackage{geometry} -- This package is specifically forbidden
% \usepackage{grffile} -- This package is specifically forbidden
% \usepackage{hyperref} -- This package is specifically forbidden
% \usepackage{navigator} -- This package is specifically forbidden
% (or any other package that embeds links such as navigator or hyperref)
% \indentfirst} -- This package is specifically forbidden
% \layout} -- This package is specifically forbidden
% \multicol} -- This package is specifically forbidden
% \nameref} -- This package is specifically forbidden
% \usepackage{savetrees} -- This package is specifically forbidden
% \usepackage{setspace} -- This package is specifically forbidden
% \usepackage{stfloats} -- This package is specifically forbidden
% \usepackage{tabu} -- This package is specifically forbidden
% \usepackage{titlesec} -- This package is specifically forbidden
% \usepackage{tocbibind} -- This package is specifically forbidden
% \usepackage{ulem} -- This package is specifically forbidden
% \usepackage{wrapfig} -- This package is specifically forbidden
% DISALLOWED COMMANDS
% \nocopyright -- Your paper will not be published if you use this command
% \addtolength -- This command may not be used
% \balance -- This command may not be used
% \baselinestretch -- Your paper will not be published if you use this command
% \clearpage -- No page breaks of any kind may be used for the final version of your paper
% \columnsep -- This command may not be used
% \newpage -- No page breaks of any kind may be used for the final version of your paper
% \pagebreak -- No page breaks of any kind may be used for the final version of your paperr
% \pagestyle -- This command may not be used
% \tiny -- This is not an acceptable font size.
% \vspace{- -- No negative value may be used in proximity of a caption, figure, table, section, subsection, subsubsection, or reference
% \vskip{- -- No negative value may be used to alter spacing above or below a caption, figure, table, section, subsection, subsubsection, or reference

\setcounter{secnumdepth}{2} %May be changed to 1 or 2 if section numbers are desired.

% The file aaai25.sty is the style file for AAAI Press
% proceedings, working notes, and technical reports.
%

% Title

% Your title must be in mixed case, not sentence case.
% That means all verbs (including short verbs like be, is, using,and go),
% nouns, adverbs, adjectives should be capitalized, including both words in hyphenated terms, while
% articles, conjunctions, and prepositions are lower case unless they
% directly follow a colon or long dash
\title{Destroy and Repair Using Hyper-Graphs for Routing}
\author{
    %Authors
    % All authors must be in the same font size and format.
    Ke Li\textsuperscript{\rm 1, \rm 2}, Fei Liu\textsuperscript{\rm 2}, Zhengkun Wang\textsuperscript{\rm 1\thanks{Corresponding author}}, Qingfu Zhang\textsuperscript{\rm 2}    
}
\affiliations{
    %Afiliations
    \textsuperscript{\rm 1}School of System Design and Intelligent Manufacturing, Southern University of Science and Technology\\
    \textsuperscript{\rm 2}Department of Computer Science, City University of Hong Kong\\

    12250110@mail.sustech.edu.cn, 
    fliu36-c@my.cityu.edu.hk, 
    wangzhenkun90@gmail.com,    
    qingfu.zhang@cityu.edu.hk
}

%Example, Single Author, ->> remove \iffalse,\fi and place them surrounding AAAI title to use it
% \iffalse
% \title{My Publication Title --- Single Author}
% \author {
%     Author Name
% }
% \affiliations{
%     Affiliation\\
%     Affiliation Line 2\\
%     name@example.com
% }
%\fi

% \iffalse
% %Example, Multiple Authors, ->> remove \iffalse,\fi and place them surrounding AAAI title to use it
% \title{My Publication Title --- Multiple Authors}
% \author {
%     % Authors
%     First Author Name\textsuperscript{\rm 1,\rm 2},
%     Second Author Name\textsuperscript{\rm 2},
%     Third Author Name\textsuperscript{\rm 1}
% }
% \affiliations {
%     % Affiliations
%     \textsuperscript{\rm 1}Affiliation 1\\
%     \textsuperscript{\rm 2}Affiliation 2\\
%     firstAuthor@affiliation1.com, secondAuthor@affilation2.com, thirdAuthor@affiliation1.com
% }
%\fi


% REMOVE THIS: bibentry
% This is only needed to show inline citations in the guidelines document. You should not need it and can safely delete it.
\usepackage{bibentry}
% END REMOVE bibentry

\renewcommand{\floatpagefraction}{0.8} % 允许浮动体占据最多 80% 页面空间
\renewcommand{\bottomfraction}{0.8}    % 允许浮动体占据最多 80% 的底部
\renewcommand{\textfraction}{0.1}      % 只需要 10% 的正文,避免强制跳页
\setcounter{totalnumber}{5}            % 允许同一页最多放 5 个浮动体
\setcounter{bottomnumber}{3}           % 允许最多 3 个浮动体在底部


\begin{document}

\maketitle

\begin{abstract}
Recent advancements in Neural Combinatorial Optimization (NCO) have shown promise in solving routing problems like the Traveling Salesman Problem (TSP) and Capacitated Vehicle Routing Problem (CVRP) without handcrafted designs. Research in this domain has explored two primary categories of methods: iterative and non-iterative. While non-iterative methods struggle to generate near-optimal solutions directly, iterative methods simplify the task by learning local search steps. However, existing iterative methods are often limited by restricted neighborhood searches, leading to suboptimal results. To address this limitation, we propose a novel approach that extends the search to larger neighborhoods by learning a destroy-and-repair strategy. Specifically, we introduce a Destroy-and-Repair framework based on Hyper-Graphs (DRHG). This framework reduces consecutive intact edges to hyper-edges, allowing the model to pay more attention to the destroyed part and decrease the complexity of encoding all nodes. Experiments demonstrate that DRHG achieves state-of-the-art performance on TSP with up to 10,000 nodes and shows strong generalization to real-world TSPLib and CVRPLib problems. 
\end{abstract}

\begin{links}
\link{Code}{https://github.com/CIAM-Group/DRHG}
\end{links}


\begin{figure*}[htbp]
\centering
\includegraphics[width=0.8\textwidth]{graph/pipeline.pdf} % Reduce the figure size so that it is slightly narrower than the column.
\caption{Pipeline of Destroy-and-Repair using Hyper-Graphs   
}
\label{pipeline}

\end{figure*}

\section{Introduction}

Routing problems are significant combinatorial optimization problems with broad real-world applications in logistics, transportation, and manufacturing. Their NP-hard nature poses a significant challenge to the application of exact methods. Heuristics sacrifice the optimality while can obtain near-optimal solutions in a reasonable time. However, the development of heuristics usually relies on human designs with domain expert knowledge, which hinders their practical applications. 

Neural Combinatorial Optimization (NCO), which trains a neural network to learn heuristics to solve routing problems without handcraft design, has gained much attention. The existing NCO methods can be roughly classified into two categories: 1) non-iterative and 2) iterative methods. 

In non-iterative methods, the neural solvers construct a solution in one shot \cite{vinyals2015pointer, kool2018attention, kwon2020pomo}. Most of these works train neural solvers to determine the next node in an auto-regressive manner, i.e., nodes are selected one by one to be added to the end of a partial solution. Others learn to predict a heuristic, such as a heatmap, and then construct a solution based on the learned information. These works can generate reasonable solutions in a short time. Nevertheless, these non-iterative methods may lead to irreversible consequences if an error occurs in one of the construction steps, thus placing excessive demands on the model's capability to narrow the optimality gap for large-scale problems.

Iterative methods adopt neural solvers to tackle a subproblem in each iteration rather than solve the entire problem at once. The iterative approach reduces the burden of neural solvers and increases the performance, leading to state-of-the-art results. Some existing iterative NCO methods \cite{d2020learning2opt,wu2021learning,ma2021DACT} primarily focus on learning low-level operators within small neighborhoods, such as k-opt or swap. Others follow a destroy-and-repair manner, iteratively destroying the solution into a partial solution and then reconstructing the destroyed nodes, operating within a large neighborhood and excelling at producing high-quality solutions.

However, the neural networks are trained in a conventional way, either using Reinforcement Learning (RL) \cite{LCP,cheng2023select,ye2024glop, zheng2024udc} or Supervised Learning (SL) \cite{luo2023lehd,luo2024SIL} without being tailored for the destroy and repair framework. Therefore, they can only deal with the destruction of one segment, bringing challenges in reducing the optimality gap.

To address the issue, we propose a novel iterative NCO method for routing, termed Destroy and Repair by Hyper-Graphs (DRHG). We employ SL to train a model that approaches the best repair after the destruction. Specifically, after the destruction, the complete tour becomes some segments of consecutive edges and some isolated nodes. We reduce the segments to hyper-edges to build a hyper-graph, then fix them during the repair. The model learns to connect isolated nodes and fixed hyper-edges to form a reduced solution, which is restored to a complete solution later. Thanks to the condensed formulation of the hyper-graph, the scale of model input depends only on the degree of the destruction but not the scale of the original problem. This ensures our method has a low computational complexity and allows our method to iterate on large-scale problems. 

Our contributions can be summarized as follows:

\begin{itemize}
    \item We propose a novel NCO framework of destroy-and-repair for routing problems. By learning to repair a destroyed problem in a supervised way, our model can search in large neighborhoods more efficiently.
    \item We adopt a condensed hyper-graph formulation of the destroyed problem by reducing consecutive edges to fixed hyper-edges, which decreases the computational complexity and enables the model to iterate on large-scale problems.
    \item The experiments show that our method achieves state-of-the-art performance on TSPs from 100 nodes to 10K nodes, and also competitive results on CVRP. Our method generalizes well to real-world instances as well. 
\end{itemize}



\section{Related Works}

\subsection{Non-iterative NCO Routing Solvers}

\subsubsection{One-shot Constructive Solvers}
One-shot constructive methods are one of the earliest lines of work that use NCO to solve routing problems. Pioneering works \cite{vinyals2015pointer, bello2016neural, nazari2018reinforcement} show that neural networks such as RNN can be trained to solve routing problems. Inspired by \citet{vaswani2017attention}, some works
\cite{kool2018attention, deudon2018learning} introduce the Transformer architecture to build more powerful NCO models and achieve promising performance. Following their works, various Transformer-based methods \cite{kwon2020pomo, drakulic2023bq, luo2023lehd} emerged. Although they have made progress in training methods or model structures, the one-shot approach can hardly further narrow the performance gap to the optimal results. 

\subsubsection{Heatmap-based Solvers}
Heatmap-based methods aim to predict an informative heatmap to expedite the search process and enhance the quality of solutions. \citet{joshi2019efficient} train a Graph Neural Network (GNN) in SL to predict the probabilities of edges to be optimal, then use the beam search to generate feasible solutions. \citet{kool2022deep} adopt dynamic programming and eliminate dominated partial solutions to reduce searching time. The most prominent works \cite{fu2021generalize, sun2023difusco} in this category employ Monte Carlo Tree Search (MCTS) to construct solutions. Leveraging MCTS reduces the stringent requirements for the accuracy of edge score predictions. However, most heatmap-based methods are limited to TSPs, as their search strategies are incompatible with problems involving additional constraints, such as CVRPs.

\subsection{Iterative NCO Routing Solvers}

Most existing iterative NCO routing solvers focus on learning low-level operators searching within small neighborhoods. \citet{chen2019neural_rewriter} employ a region-picking policy to identify a node for relocation and a rule-picking policy to determine the target position for the node's movement. \citet{d2020learning2opt, sui2021learning3opt} propose to learn 2-opt or 3-opt steps to improve the solution. Furthermore, \citet{lu2019learning, wu2021learning} utilize a pool of operators from which the model selects, demonstrating superior performance compared to approaches that rely on a single operator. \citet{ma2021DACT} propose a Dual-Aspect Collaborative Transformer (DACT) with a Cyclic Positional Encoding (CPE) method and a Dual-Aspect Collaborative Attention (DAC-Att) to encode problems, which achieves pretty good performance. However, iterative NCOs with low-level operators are limited to solving small-size problems due to the extensive number of iterations required for convergence. Moreover, the overall quality of local optimal of small neighborhoods is inferior, which implies that the final solutions obtained by these methods are often sub-optimal. 


Other iterative NCO routing solvers focus on reconstructing a partial solution of node sequence. Either trained with RL \cite{LCP, cheng2023select, ye2024glop} or SL \cite{luo2023lehd, luo2024SIL}, the models learn to reconstruct a segment given the starting node and the ending node. By operating within a large neighborhood, these methods outperform those using low-level operators. However, the neighborhoods that these methods can search in are still limited since the nodes outside the segment remain unaltered. Therefore, two nodes that are spatially close but far away in the solution may have no chance of being reconnected together. In contrast, our framework enables a more flexible neighborhood search by permitting arbitrary destruction and, subsequently, the repair of reconnecting the segments with isolated nodes. 


\section{Methodology}

\subsection{DRHG Framework}

Schematically illustrated in Fig. \ref{pipeline}, we reformulate our destroy-and-repair approach as a graph reduction, hyper-graph solving, and graph restoration process. For a graph representing the incomplete solution where a set of edges is destroyed, we reduce the graph by encoding the remaining consecutive edges as hyper-edges. As these edges remain unchanged during the repair, redefining them as fixed hyper-edges helps reduce the complexity of the problem for the model. Then, we train the model in a supervised way to solve the reduced problem on the hyper-graph. In the testing phase, we iteratively destroy the current solution to obtain a hyper-graph, solve the resulting hyper-graph, and recover the hyper-graph solution on that of the original problem. 


\subsection{Hyper-graph Representation}\label{sec: represent hyper-graph}

Mathematically, a hyper-graph is a special graph where an edge can join any number of vertices. Formally, a hyper-graph is defined as $\mathcal{G}=(\mathcal{V},\mathcal{E})$, in which $\mathcal{V}$ is the vertex (node) set and $\mathcal{E}$ is the hyper-edge set. Using hypergraph neural networks for embedding hypergraphs is intuitive, but challenging. Specifically, when constructing a solution sequentially, it becomes necessary to align the embeddings of nodes and edges in order to predict the subsequent node or hyper-edge, which may be hard for models. Even if we train an excellent model to predict the sequence, resolving the solution with respect to an undirected hyper-graph remains a non-trivial challenge. Since each hyper-edge has two possible directions, resolving the best complete solution may require a huge number of enumerations. Therefore, we propose to use two endpoints to represent a hyper-edge.

Note a TSP instance of $n$ nodes by the node coordinates as $V=\{(x_1, y_1), \ (x_2, y_2),\  ..., \ (x_n, y_n)\}$. After the destruction, $h$ undestroyed segments constitute the directed hyper-edges set of size $2h$, i.e., $E=\{e^i=(i_1, i_2, ..., i_{p_i})\ | i=1,2,...,2h\}$, where $p_i$ is the number of nodes in the directed hyper-edge $e^i$.

For hyper-graph reduction, we remove the middle nodes inside the hyper-edges and keep only the endpoints to represent the hyper-edge, i.e., $e^i=(i_1, i_p)$. Consequently, in the reduced graph, we have one set of isolated nodes $A$, one set of endpoint nodes $B$, and one set of reduced hyper-edges $E_r$. The hyper-graph size is $m=|A|+|B|$. Then, the input feature of the reduced graph is formulated as:

\begin{equation}
  r_i =(x_i^a, y_i^a, x_i^b, y_i^b, flag_i), i=1,2,...,m,
\end{equation}
    
\begin{equation}
    (x_i^a, y_i^a) = (x_i, y_i),
\end{equation}

\begin{equation}
(x_i^b, y_i^b) = \begin{cases}
	      (x_i, y_i), & if\ i\in A,\\
	      (x_j, y_j), & if\ i\in B\ and (i,j) \in E_r,
		   \end{cases}
\end{equation}
where $r_i$ is the input feature for the model, and $flag_i$ is a binary variable to indicate whether a node is an endpoint node or an isolated node.

Similarly, for a CVRP instance of $n$ customers and a depot noted as $0$, we can define the problem by the node coordinates and the demands: $V=\{(x_0, y_0, 0), \ (x_1, y_1, d_1),\  ..., \ (x_n, y_n, d_n)\}$,  where $d_i$ is the demand of node $i$. To simplify the problem, we destroyed all edges connected to the depot. Then, the input feature of the reduced graph for CVRP can be formulated as follows:

\begin{equation}
  r_i =(x_i^a, y_i^a, x_i^b, y_i^b, flag_i, dr_i), i=1,2,...,m,
\end{equation}

\begin{equation}
dr_i = \begin{cases}
	      d_i, & if\ i\in A,\\
	      \sum_{k}d_k, & if\ i\in B\ and\ k\in (i_1, i_2, ..., i_{p_i}).
		   \end{cases}
\end{equation}


\begin{figure}[htbp]
\centering
\includegraphics[width=0.5\textwidth]{graph/model_structure.pdf} % Reduce the figure size so that it is slightly narrower than the column.
\caption{Model structure of DRHG}
\label{Model structure}
\end{figure}


\subsection{Model Structure}\label{sec: model}
As shown in Fig. \ref{Model structure}, given the input features of the reduced problems, our model yields a prediction of the next node through a light encoder and a heavy decoder. 


\paragraph{Encoder} The encoder consists of a single linear projection layer, which transforms the input $r_i \in \mathbb{R}^{d_i}$ into embedding $h_i^{(0)} \in \mathbb{R}^{d_h}$.

\paragraph{Decoder} The decoder has a slightly changed linear attention module in \citet{luo2024SIL}. At each step $t$, the decoder takes the node embeddings of the first node $h_f^{(0)}$, the current node $h_c^{(0)}$, and the remaining unselected nodes $H_a^{(0)} = \{h_i^{(0)} | i = 1,2, ..., m-t\}$ as inputs. Then, the first node $h_f^{(0)}$ and the current node $h_c^{(0)}$ are used to generate $r$ virtual representative nodes embeddings $\Tilde{H}^{(0)} = \{h_j^{(0)} | j=1,2, ...,r\}$, which combined with $H_a^{(0)}$, form the input of the first linear attention module.

Then, we stack $L$ linear modules as the main component of the decoder. A linear attention module is composed of an aggregating attention layer and a broadcasting attention layer. The aggregating layer aggregates information to the representative nodes, and then the broadcasting layer broadcasts gathered information to all nodes in the graph. The details of the linear attention module are provided in Appendix \ref{appendix-model}. Note the $l$-th linear attention module as $L-Att^{(l)}$, we have
\begin{equation}
    \Tilde{H}^{(l)}, H_a^{(l)} = L-Att^{(l)}(\Tilde{H}^{(l-1)}, H_a^{(l-1)}).
\end{equation}

After $L$ attention module, we obtain a hidden representation $\Tilde{H}^{(L)}$ and and $H_a^{(L)}$. Then we take only $H_a^{(L)}$ to calculate the probability of selecting the next node by a linear projection layer and the softmax function:

\begin{equation}
a_i=\phi(h_i^{(L)}W_o),
\end{equation}

\begin{equation}
p_i=\frac{e^{a_i}}{\Sigma_j^{e^{a_j}}}.
\end{equation}



\subsection{Training Scheme}\label{sec: model}

We use SL to train our model. We apply the clustering destruction, as optimal edges are more likely to connect proximal nodes. Furthermore, the distributions of reduced problems after clustering destructions are more consistent across problems of different scales. We adopt the coordinate transformation in \citet{ye2024glop} to enhance the distribution homogeneity and consistency. For hyper-edges, once one endpoint is selected, the subsequent node must be the other endpoint. This behavior is dictated by the constraint rather than the model. Correspondingly, we introduce a masking mechanism to block the associated gradients. Additionally, destroying the problem by k-nearest neighbors results in a variable number of segments and makes the hyper-graph size differ across instances. This variability introduces instability during the training process. To tackle this problem, we design a special destruction scheme to get fixed-size hyper-graphs. We detail this method in Appendix \ref{appendix-alignement}.


\section{Experiments}
We compare our method with other representative learning-based and classical solvers on
TSP and CVRP instances with different scales and the instances in the real world.
\subsection{Experiment Setup}

\subsubsection{Implementing Details}
We set the embedding dimension of the encoder to 128. The decoder is composed of 6 linear attention modules, and each has 8 attention heads and 16 representative starting nodes. The hidden dimension of the feed-forward layer is set to 512. 

For TSP, we train the model for 100 epochs on 1,000,000 TSP100 instances. We fine-tune 20 epochs on 10,000 TSP1000 instances for large-scale problems. For CVRP, we train the model for 100 epochs on 1,000,000 CVRP100 instances. We use a batch size of 1024 and sample the training sample size in $[20, 0.8n]$ where $n$ is the problem size. As the fixed-size destruction scheme will discard a small part of the samples, the true batch size is around 800. We use the cross-entropy loss and the Adam optimizer \cite{Adam}. The initial learning rate is 1e-4, and the decay rate is 0.97 per epoch. We train and test our model with a single NVIDIA GeForce RTX 3090 GPU with 24GB memory.


\subsubsection{Baselines}
We compare our method with:

\textbf{1) Classical Solvers:} Concorde \cite{applegate2006concorde}, LKH3 \cite{LKH3}, and HGS \cite{HGS}; 

\textbf{2) Traditional Heuristic:} Random Insertion, Sweep; 

\textbf{3) Construction-based Method:} POMO \cite{kwon2020pomo}, BQ \cite{drakulic2023bq};

\textbf{4) Heatmap-based Method:} Att-GCN+MCTS \cite{fu2021generalize}, DIMES \cite{qiu2022DIMES}, and DIFUSCO \cite{sun2023difusco};

\textbf{5) Segment-reconstruction Method:} LEHD \cite{luo2023lehd}, GLOP \cite{ye2024glop} and SIL \cite{luo2024SIL}; 

\textbf{6) Operator-iteration Method:} Neural Rewriter \cite{chen2019neural_rewriter}, Learning 2-Opt \cite{d2020learning2opt}, Learning 3-Opt \cite{sui2021learning3opt}, and DACT \cite{ma2021DACT}.


For most baseline methods, we run their source code with default settings. The result of Att-GCN+MCTS \cite{fu2021generalize}, DIMES \cite{qiu2022DIMES}, and DIFUSCO \cite{sun2023difusco}, SIL \cite{luo2024SIL}, Neural Rewriter \cite{chen2019neural_rewriter}, and Learning 3-Opt \cite{sui2021learning3opt} are taken from their original papers.

\subsubsection{Metrics} 
 
We use the average objective value (Obj.) and optimality gap (Gap) to evaluate the model performance and the inference time (Time) to evaluate the model efficiency. The ground truth labels of TSP are generated by Concorde for TSP100 to TSP1000 and by LKH for TSP above 1000. The ground truth labels of CVRP are generated by HGS.

\subsubsection{Testing}
We test our method on TSPs from 100 to 10,000 and CVRPs from 100 to 1000. There are 10,000 instances for TSP100 and CVRP100, 128 instances for problems of size 200 to 5,000, and 16 instances for TSP10,000. We use random insertion to generate initial solutions for TSP and sweep for CVRP. Regarding to the k-nn destruction, we sample $k\in [20, min(1000, n)]$ for TSP and $k\in [20, min(200, n)]$ for CVRP, where $n$ is the problem scale. For simplicity, we disconnect all nodes adjacent to the depot in CVRP. We set the number of iterations to 1000.   

% TSP
\begin{table*}[htbp]
  \centering
  \renewcommand{\arraystretch}{1.4}
  \renewcommand{\tabcolsep}{3pt} 
    \begin{tabular}{l|ccc|ccc|ccc}
    \toprule
          & \multicolumn{1}{c}{} & \multicolumn{1}{c}{\textbf{TSP100}} & \multicolumn{1}{c|}{} 
          & \multicolumn{1}{c}{} & \multicolumn{1}{c}{\textbf{TSP200}} & \multicolumn{1}{c|}{} 
          & \multicolumn{1}{c}{} & \multicolumn{1}{c}{\textbf{TSP500}} & \multicolumn{1}{c}{} \\
          
    \multicolumn{1}{c|}{\textbf{Method}} & \multicolumn{1}{c}{\textbf{Obj.}} & \multicolumn{1}{c}{\textbf{Gap}} & \textbf{Time} & \multicolumn{1}{c}{\textbf{Obj.}} & \multicolumn{1}{c}{\textbf{Gap}} & \textbf{Time} & \multicolumn{1}{c}{\textbf{Obj.}} & \multicolumn{1}{c}{\textbf{Gap}} & \textbf{Time} \\
    \midrule
    LKH3  & \multicolumn{1}{c}{7.763 } & \multicolumn{1}{c}{0.000\%} & 0.34s & \multicolumn{1}{c}{10.704 } & \multicolumn{1}{c}{0.000\%} & 1.88s & \multicolumn{1}{c}{16.522 } & \multicolumn{1}{c}{0.000\%} & 15.0s \\
    Concorde & \multicolumn{1}{c}{7.763 } & \multicolumn{1}{c}{0.000\%} & 0.20s & \multicolumn{1}{c}{10.704 } & \multicolumn{1}{c}{0.000\%} & 1.41s & \multicolumn{1}{c}{16.522 } & \multicolumn{1}{c}{0.000\%} & 15.0s \\
    Random Insertion & \multicolumn{1}{c}{8.513 } & \multicolumn{1}{c}{9.662\%} & 0.00s & \multicolumn{1}{c}{11.948 } & \multicolumn{1}{c}{11.627\%} & \textless0.01s & \multicolumn{1}{c}{18.546 } & \multicolumn{1}{c}{12.252\%} & \textless0.1s \\
    \midrule
    Att-GCN+MCTS*  & \multicolumn{1}{c}{7.764 } & \multicolumn{1}{c}{0.037\%} & 0.09s & \multicolumn{1}{c}{10.814 } & \multicolumn{1}{c}{0.884\%} & 0.94s & \multicolumn{1}{c}{16.966 } & \multicolumn{1}{c}{2.537\%} & 2.8s \\
    DIMES* & \multicolumn{1}{c}{-} & \multicolumn{1}{c}{-} & -     & \multicolumn{1}{c}{-} & \multicolumn{1}{c}{-} & -     & \multicolumn{1}{c}{16.840 } & \multicolumn{1}{c}{1.760\%} & 60.5s \\
    DIFUSCO* & \multicolumn{1}{c}{7.780 } & \multicolumn{1}{c}{0.240\%} & -     & \multicolumn{1}{c}{-} & \multicolumn{1}{c}{-} & -     & \multicolumn{1}{c}{16.800 } & \multicolumn{1}{c}{1.490\%} & 1.7s \\
    \midrule
    
    POMO augx8 & \multicolumn{1}{c}{\makecell{7.774\\(±0.231)}} 
    & \multicolumn{1}{c}{\makecell{0.134\%\\(±0.224\%)}} & 0.01s    
    & \multicolumn{1}{c}{\makecell{10.868\\(±0.225)}} 
    & \multicolumn{1}{c}{\makecell{1.534\%\\(±0.523\%)}} & 0.04s 
    & \multicolumn{1}{c}{\makecell{20.187\\(±0.251)}} 
    & \multicolumn{1}{c}{\makecell{22.187\%\\(±0.997\%)}} & 0.5s \\
    
    
    BQ bs16 & \multicolumn{1}{c}{\makecell{7.764\\(±0.229)}} 
    & \multicolumn{1}{c}{\makecell{0.015\%\\(±0.057\%)}} & 0.17s      
    & \multicolumn{1}{c}{\makecell{10.717\\(±0.208)}} 
    & \multicolumn{1}{c}{\makecell{0.129\%\\(±0.149\%)}} & 0.94s  
    & \multicolumn{1}{c}{\makecell{16.617\\(±0.212)}} 
    & \multicolumn{1}{c}{\makecell{0.579\%\\(±0.239\%)}} & 5.5s  \\
    \midrule
    
    GLOP (more revision) & \multicolumn{1}{c}{\makecell{7.767\\(±0.234)}}
    & \multicolumn{1}{c}{\makecell{0.046\%\\(±0.126\%)}} & 0.79s 
    & \multicolumn{1}{c}{\makecell{10.774\\(±0.213)}} 
    & \multicolumn{1}{c}{\makecell{0.653\%\\(±0.410\%)}} & 0.33s  
    & \multicolumn{1}{c}{\makecell{16.883\\(±0.214)}} 
    & \multicolumn{1}{c}{\makecell{2.186\%\\(±0.474\%)}} & 0.8s  \\

    LEHD RRC1000 & \multicolumn{1}{c}{\makecell{7.763\\(±0.229)}}
    & \multicolumn{1}{c}{\makecell{0.002\%\\(±0.014\%)}} & 1.04s 
    & \multicolumn{1}{c}{\makecell{10.706\\(±0.206)}}
    & \multicolumn{1}{c}{\makecell{0.0182\%\\(±0.054\%)}} & 4.92s
    & \multicolumn{1}{c}{\makecell{16.550\\(±0.209)}}
    & \multicolumn{1}{c}{\makecell{0.167\%\\(±0.128\%)}} & 33.8s \\
    \midrule
    
    Learning 2-Opt (T=1000) & \multicolumn{1}{c}{7.853} & \multicolumn{1}{c}{1.150\%} & 0.09s & \multicolumn{1}{c}{11.107} & \multicolumn{1}{c}{3.765\%} & 0.20s & \multicolumn{1}{c}{21.339} & \multicolumn{1}{c}{29.158\%} & 0.5s \\
    
    Learning 3-Opt (T=1000)* & \multicolumn{1}{c}{7.850 } & \multicolumn{1}{c}{1.060\%} & 0.23s & \multicolumn{1}{c}{-} & \multicolumn{1}{c}{-} &       & \multicolumn{1}{c}{-} & \multicolumn{1}{c}{-} & - \\
    
    DACT (T=1000) & \multicolumn{1}{c}{7.892 } & \multicolumn{1}{c}{1.653\%} & 0.07s & \multicolumn{1}{c}{12.870 } & \multicolumn{1}{c}{20.252\%} & 0.41s & \multicolumn{1}{c}{20.846 } & \multicolumn{1}{c}{26.171\%} & 1.6s \\
    \midrule

    DRHG (T=1000) & \multicolumn{1}{c}{\makecell{\textbf{7.763}\\\textbf{(±0.229)}}}
    &\multicolumn{1}{c}{\makecell{\textbf{0.000\%}\\\textbf{(±0.007\%)}}}
    & 2.73s      
    & \multicolumn{1}{c}{\makecell{\textbf{10.705}\\\textbf{(±0.206)}}} 
    & \multicolumn{1}{c}{\makecell{\textbf{0.010\%}\\\textbf{(±0.036\%)}}}
    & 9.05s 
    & \multicolumn{1}{c}{\makecell{\textbf{16.540}\\ \textbf{(±0.211)}}} 
    & \multicolumn{1}{c}{\makecell{\textbf{0.111\%}\\ \textbf{(±0.090\%)}}} & 20.6s \\
    \bottomrule

    % \toprule
    & \multicolumn{1}{c}{} & \multicolumn{1}{c}{\textbf{TSP1K}} & \multicolumn{1}{c|}{} 
    & \multicolumn{1}{c}{} & \multicolumn{1}{c}{\textbf{TSP5K}} & \multicolumn{1}{c|}{} 
    & \multicolumn{1}{c}{} & \multicolumn{1}{c}{\textbf{TSP10K}} & \multicolumn{1}{c}{} \\
    
     \multicolumn{1}{c|}{\textbf{Method}} & \textbf{Obj.} & \textbf{Gap}   & \textbf{Time}  &  \textbf{Obj.}  & \textbf{Gap}   & \textbf{Time}  &  \textbf{Obj.}  & \textbf{Gap}   & \textbf{Time} \\
    \midrule
    
    LKH3  & 23.12  & 0.00\% & 1.7m  & 50.97  & 0.00\% & 12m   & 71.78  & 0.00\% & 33m \\
    Concorde & 23.12  & 0.00\% & 1m    & 50.95  & -0.05\% & 31m   & 72.00  & 0.15\% & 1.4h \\
    Random Insertion & 26.11  & 12.90\% & \textless1s   & 58.06  & 13.90\% & \textless1s   & 81.82  & 13.90\% & \textless1s \\
    \midrule
    
    Att-GCN+MCTS*  & 23.86  & 3.20\% & 6s    & -     & -     & -     & 74.93  & 4.39\% & 6.6m \\
    DIMES* & 23.69  & 2.46\% & 2.2m  & -     & -     & -     & 74.06  & 3.19\% & 3m \\
    DIFUSCO* & 23.39  & 1.17\% & 11.5s & -     & -     & -     & 73.62  & 2.58\% & 3.0m \\
    \midrule
    
    POMO augx8 & 32.51  & 40.60\% & 4.1s  & 87.72  & 72.10\% & 8.6m  &       & OOM   &  \\

    BQ bs16 & \multicolumn{1}{c}{\makecell{23.43\\(±0.221)}} & \multicolumn{1}{c}{\makecell{1.37\%\\(±0.284\%)}} & 13s  
    & \multicolumn{1}{c}{\makecell{58.27\\(±0.951)}} 
    & \multicolumn{1}{c}{\makecell{10.70\%\\(±1.827\%)}} & 24s   
    & \multicolumn{1}{c}{} & \multicolumn{1}{c}{OOM} &    \\

    GLOP (more revision) & \multicolumn{1}{c}{\makecell{23.78\\(±0.218)}} 
    & \multicolumn{1}{c}{\makecell{2.85\%\\(±0.401\%)}} & 10.2s 
    & \multicolumn{1}{c}{\makecell{53.15\\(±0.231)}} 
    & \multicolumn{1}{c}{\makecell{4.26\%\\(±0.289\%)}} & 1.0m  
    & \multicolumn{1}{c}{\makecell{75.04\\(±0.215)}} & \multicolumn{1}{c}{\makecell{4.39\%\\(±0.153\%)}} & 1.9m  \\
    
    LEHD RRC1000 & \multicolumn{1}{c}{\makecell{23.29\\(±0.220)}} 
    & \multicolumn{1}{c}{\makecell{0.72\%\\(±0.176\%)}} & 3.3m 
    & \multicolumn{1}{c}{\makecell{54.43\\(±0.394)}} 
    & \multicolumn{1}{c}{\makecell{6.79\%\\(±0.671\%)}} & 8.6m  
    & \multicolumn{1}{c}{\makecell{80.90\\(±0.532)}} & \multicolumn{1}{c}{\makecell{12.50\%\\(±0.663\%)}} & 18.6m \\
    
    SIL PRC1000* & 23.31  & 0.82\% & 1.2m  & 51.91  & 1.84\% & 7.6m  & 73.38  & 2.23\% & 13.7m \\
    \midrule
    Learning 2-Opt (T=1000) & 61.15  & 164.50\% & 1.3s  & -     & -     & -     & -     & -     & - \\
    DACT (T=1000) & 29.03  & 25.56\% & 7.8s  &       & OOM   &       &       & OOM   &  \\
    \midrule
    
    DRHG (T=1000) & 23.22  & 0.45\% & 1.72m & 51.98  & 2.05\% & 1.79m & 74.38  & 3.46\% & 3.63m \\
    

    DRHG-FT (T=1000) 
    & \multicolumn{1}{c}{\makecell{\textbf{23.19}\\\textbf{(±0.210)}}} 
    & \multicolumn{1}{c}{\makecell{\textbf{0.29\%}\\\textbf{(±0.108\%)}}} 
    & 1.66m 
    & \multicolumn{1}{c}{\makecell{\textbf{51.39}\\\textbf{(±0.187)}}} & 
    \multicolumn{1}{c}{\makecell{\textbf{0.88\%}\\\textbf{(±0.087\%)}}} & 1.82m 
    & \multicolumn{1}{c}{\makecell{\textbf{72.85}\\\textbf{(±0.217)}}} & \multicolumn{1}{c}{\makecell{\textbf{1.33\%}\\\textbf{(±0.084\%)}}} & 3.70m \\

    \bottomrule
    
    \end{tabular}%
    \caption{Results on TSP}
  \label{table-tsp-all}%
\end{table*}%




% comparison given the same time with competitors
\begin{table*}[htbp]
  \centering
\begin{tabular}{c|c c|c c|c c}
\toprule
\multicolumn{1}{c|}{\multirow{2}[4]{*}{\textbf{Competitors}}} & \multicolumn{2}{c|}{\textbf{TSP100}} & \multicolumn{2}{c|}{\textbf{TSP200}} & \multicolumn{2}{c}{\textbf{TSP500}} \\
\cmidrule{2-7}      & \textbf{Competitor's} & \textbf{Ours}  & \textbf{Competitor's} & \textbf{Ours}  & \textbf{Competitor's} & \textbf{Ours} \\
\midrule
POMO augx8 & 0.134\% & 1.311\% (-) & 1.534\% & 1.269\%(+) & 22.187\% & 1.113\%(+) \\
\midrule
BQ bs16 & 0.015\% & 0.025\% (-) & 0.129\% & 0.076\%(+) & 0.579\% & 0.322\%(+) \\
\midrule
GLOP (more revision) & 0.046\% & 0.004\% (+) & 0.653\% & 0.206\%(+) & 2.186\% & 0.879\%(+) \\
\midrule
LEHD RRC1000 & 0.002\% & 0.003\% (-) & 0.018\% & 0.016\%(+) & 0.167\% & 0.111\%(+) \\
\bottomrule
\end{tabular}%
\caption{Comparison of different methods on TSPs with scale $\textless$ 1,000 given the same running time}
  \label{comparason}%
\end{table*}%


% table cvrp

\begin{table*}[htbp]
  \centering
    \renewcommand{\tabcolsep}{10pt} 
    \begin{tabular}{l|p{10em}p{6em}c|p{6em}p{6em}c}
    
    \toprule    
    & \multicolumn{1}{c}{} & \multicolumn{1}{c}{\textbf{CVRP100}} & \multicolumn{1}{c|}{} 
    & \multicolumn{1}{c}{} & \multicolumn{1}{c}{\textbf{CVRP200}} & \multicolumn{1}{c}{} \\
    
    \textbf{Method} & \multicolumn{1}{c}{\textbf{ Obj. }} & \multicolumn{1}{c}{\textbf{Gap}} & \textbf{Time} & \multicolumn{1}{c}{\textbf{ Obj. }} & \multicolumn{1}{c}{\textbf{Gap}} & \multicolumn{1}{c}{\textbf{Time}}  \\
    \midrule
    
    LKH3  & \multicolumn{1}{c}{15.647} & \multicolumn{1}{c}{0.00\%} & 4.32s &      \multicolumn{1}{c}{20.173} & \multicolumn{1}{c}{0.00\%} & 59.06s \\
    HGS   & \multicolumn{1}{c}{15.563} & \multicolumn{1}{c}{-0.53\%} & 1.62s &  \multicolumn{1}{c}{19.946} & \multicolumn{1}{c}{-1.13\%} & 39.38s  \\
    Sweep & \multicolumn{1}{c}{20.606} & \multicolumn{1}{c}{32.14\%} & \textless0.01s       & \multicolumn{1}{c}{0.269} & \multicolumn{1}{c}{32.78\%} & \multicolumn{1}{c}{\textless0.01s} \\
    \midrule
    
    POMO augx8 & \multicolumn{1}{c}{\makecell{15.754\\(±1.800)}} & 
    \multicolumn{1}{c}{\makecell{0.69\%\\(±0.649\%)}} & 0.01s & 
    \multicolumn{1}{c}{\makecell{21.154\\(±2.138)}} & \multicolumn{1}{c}{\makecell{4.87\%\\(±1.133\%)}} & \multicolumn{1}{c}{\textless0.01s}  \\
    
    BQ bs16 & \multicolumn{1}{c}{\makecell{15.806\\(±1.807)}} & 
    \multicolumn{1}{c}{\makecell{1.02\%\\(±1.015\%)}} & 0.11s & 
    \multicolumn{1}{c}{\makecell{20.362\\(±2.139)}} & 
    \multicolumn{1}{c}{\makecell{0.94\%\\(±0.950\%)}} & 0.56s  \\
    \midrule

    
    LEHD RRC 1000 & \multicolumn{1}{c}{\makecell{\textbf{15.629}\\\textbf{(±1.793)}}} & 
    \multicolumn{1}{c}{\makecell{\textbf{-0.11\%}\\\textbf{(±0.616\%)}}} & 1.01s     & 
    \multicolumn{1}{c}{\makecell{\textbf{20.095}\\\textbf{(±2.137)}}} & 
    \multicolumn{1}{c}{\makecell{\textbf{-0.38\%}\\\textbf{(±0.627\%)}}} & 19.69s  \\
    
    \midrule
    Neural Rewriter* & \multicolumn{1}{c}{16.100 } & \multicolumn{1}{c}{-} &   \multicolumn{1}{c|}{-}     & \multicolumn{1}{c}{-} & \multicolumn{1}{c}{-} &  \multicolumn{1}{c}{-}    \\
    DACT  & \multicolumn{1}{c}{16.202 } & \multicolumn{1}{c}{3.55\%} & 0.14s &   \multicolumn{1}{c}{23.230 } & \multicolumn{1}{c}{14.71\%} & 1.02s  \\
    \midrule
    
    DRHG (T=1000) & \multicolumn{1}{c}{\makecell{15.643 \\(±1.790)}} & \multicolumn{1}{c}{\makecell{-0.02\%\\(±0.647\%)}} & 2.37s       & \multicolumn{1}{c}{\makecell{\textit{20.233}\\\textit{(±2.133)}}} & 
    \multicolumn{1}{c}{\makecell{-0.16\%\\(±0.648\%)}} & 8.91s \\

    \toprule
      & \multicolumn{1}{c}{} & \multicolumn{1}{c}{\textbf{CVRP500}} & \multicolumn{1}{c|}{} 
    & \multicolumn{1}{c}{} & \multicolumn{1}{c}{\textbf{CVRP1K}} & \multicolumn{1}{c}{} \\
          
    \textbf{Method} & \multicolumn{1}{c}{\textbf{ Obj. }} & \multicolumn{1}{c}{\textbf{Gap}} & \textbf{Time}  & \multicolumn{1}{c}{\textbf{ Obj. }} & \multicolumn{1}{c}{\textbf{Gap}} & \textbf{Time} \\
    \midrule
    
    LKH3  & \multicolumn{1}{c}{37.229 } & \multicolumn{1}{c}{0.00\%} & 154.69s &   \multicolumn{1}{c}{37.090 } & \multicolumn{1}{c}{0.00\%} & 199.7s  \\
    HGS   & \multicolumn{1}{c}{36.561 } & \multicolumn{1}{c}{-1.79\%} & 112.50s &    \multicolumn{1}{c}{36.288 } & \multicolumn{1}{c}{-2.16\%} & 149.1s \\
    Sweep & \multicolumn{1}{c}{46.839 } & \multicolumn{1}{c}{25.81\%} & \textless0.01s       & \multicolumn{1}{c}{49.166 } & \multicolumn{1}{c}{32.56\%} & \textless0.1s  \\
    \midrule
    
    POMO augx8 & \multicolumn{1}{c}{\makecell{44.638\\(±3.112)}} & 
    \multicolumn{1}{c}{\makecell{19.90\%\\(±11.109\%)}} & 0.47s & 
    \multicolumn{1}{c}{84.898} & \multicolumn{1}{c}{128.89\%} & 4.7s  \\
    
    BQ bs16 & \multicolumn{1}{c}{\makecell{37.606\\(±4.216)}} & 
    \multicolumn{1}{c}{\makecell{1.01\%\\(±0.825\%)}} & 3.47s &
    \multicolumn{1}{c}{\makecell{38.147\\(±3.174)}} & \multicolumn{1}{c}{\makecell{2.88\%\\(±1.258\%)}} & 8.7s  \\
    \midrule
    
    GLOP-G(LKH3) & \multicolumn{1}{c}{-} & \multicolumn{1}{c}{-} &      \multicolumn{1}{c|}{-}  & 
    \multicolumn{1}{c}{\makecell{39.651\\(±3.779)}} & \multicolumn{1}{c}{\makecell{6.90\%\\(±2.013\%)}} & 0.8s  \\

    
    LEHD RRC 1000 & \multicolumn{1}{c}{\makecell{\textbf{37.100}\\\textbf{(±4.257)}}} & 
    \multicolumn{1}{c}{\makecell{\textbf{-0.35}\%\\\textbf{(±0.534\%)}}} & 56.25s      & 
    \multicolumn{1}{c}{\makecell{37.432\\(±3.237)}} & \multicolumn{1}{c}{\makecell{0.92\%\\(±0.874\%)}} & 202.5s \\
    
    SIL PRC1000* & \multicolumn{1}{c}{-} & \multicolumn{1}{c}{-} &       \multicolumn{1}{c|}{-}   & \multicolumn{1}{c}{\textbf{36.810 }} & \multicolumn{1}{c}{\textbf{-0.76\%}} & 78.8s \\
    \midrule
    
    DACT  & \multicolumn{1}{c}{46.393 } & \multicolumn{1}{c}{24.98\%} & 3.83s &    \multicolumn{1}{c}{} & \multicolumn{1}{c}{OOM} &     \\
    \midrule
    
    DRHG (T=1000) & \multicolumn{1}{c}{\makecell{37.718\\(±4.446)}} & 
    \multicolumn{1}{c}{\makecell{1.31\%\\(±1.035\%)}} & 12.60s &   
    \multicolumn{1}{c}{\makecell{39.932\\(±3.854)}} & \multicolumn{1}{c}{\makecell{7.66\%\\(±2.226\%)}} & 12.8s  \\
    
    \bottomrule
    \end{tabular}%

    \caption{Results on CVRP}
  \label{table-cvrp-all}%
\end{table*}%



\subsection{Experimental Results}

% non-optimal rate
\begin{figure}[h]
\centering
\includegraphics[width=0.42\textwidth]{graph/non-optimal-Rate-CR.pdf} % Reduce the figure size so that it is slightly narrower than the column.
\caption{Non-optimal rate on 10k TSP100 instances}
\label{non-optimal}
\end{figure}

The main experimental results on uniformly distributed TSP instances are reported in Table \ref{table-tsp-all}. All results are reported in terms of per-instance solving time. The values in parentheses represent the variance. Rank-sum tests are conducted on POMO augx8, BQ bs16, GLOP (more revision) and LEHD RRC1000 to assess whether the path length and the gap of given methods differ significantly from those of our method. Except for the LEHD RRC1000 on TSP100, our DRHG demonstrates statistically significant differences ($p\textless0.05$) from other methods across all other test settings.

Particularly for TSP100, we track the number of non-optimal cases of the other representative NCO methods compared with our method, which is illustrated in Fig. \ref{non-optimal}. For TSPs with 100 to 500 nodes, we perform a comparative analysis of our method against other approaches under identical running time in Table 2, where $+$ means our method outperforms its competitor, and vice versa. Since the running time of one-shot constructive solvers cannot be adjusted, we allocate the same running time to our DRHG as that of its competitors. 


The results demonstrate that our proposed DRHG method can achieve very good performance on instances of all sizes. 
Notably, on TSP100, our method yields non-optimal solutions in only 129 out of 10,000 cases, reducing the non-optimality ratio by an order of magnitude. On TSP200 and TSP500, our method reduces the gap by approximately one-third and outperforms its competitors given the same running time. With a small fine-tuning budget, our method outperforms all other methods on large-scale problems, including SIL \cite{luo2024SIL}, which is separately trained for each problem scale.

For CVRP (Table \ref{table-cvrp-all}), DRHG can also achieve pretty good performance. Our method outperforms the traditional heuristic method LKH3 on CVRP100 and 200. For CVRP200 and 500, our method outperforms most learning methods except for LEHD RRC1000, which, however, requires much more time. Overall, the performance of DRHG is slightly less dominant than that of TSP, but it is still promising. 


Table \ref{tsplib} and Table \ref{cvrplib} show the test results on real-world TSPLib and CVRPLib instances with different sizes and distributions. The results show that our method is robust for different sizes and distributions. More results on TSPLib can be found in Appendix \ref{appendix-tsplib}. 

The ablation studies are presented in Appendix \ref{appendix-hyper-param}. 



% tsplib
\begin{table}[htbp]
  \centering
    \begin{tabular}{l|c|c|c|c|c}
    \toprule
           & POMO  & BQ   & LEHD  & GLOP  & DRHG \\
    \multicolumn{1}{c|}{size} & augx8 & bs16  & R. 1K & more r. & T=1K \\
    \midrule
     \textless 100 & 0.79\% & 0.49\% & 0.48\% & 0.54\% & \textbf{0.48\%} \\
    100-200 & 2.42\% & 1.66\% & 0.20\% & 0.79\% & \textbf{0.15\%} \\
    200-500 & 13.41\% & 1.41\% & 0.38\% & 1.87\% & \textbf{0.36\%} \\
    500-1k & 31.68\% & 2.20\% & 1.21\% & 3.28\% & \textbf{0.26\%} \\
    \textgreater1k   & 63.71\% & 6.68\% & 4.14\% & 7.23\% & \textbf{2.09\%} \\
    \midrule
    all   & 26.41\% & 2.95\% & 1.59\% & 3.58\% & \textbf{0.95\%} \\
    \bottomrule
    \end{tabular}%
  \caption{Results on TSPLib}
  \label{tsplib}%
\end{table}%


% cvrplib
\begin{table}[htbp]
  \centering
    \begin{tabular}{l|c|c|c|c|c}
    \toprule
          & POMO  & BQ   & LEHD  & GLOP  & DRHG \\
    \multicolumn{1}{c|}{size} & augx8 & bs16  & R. 1K & more r. & T=1K \\
    \midrule
    A & 4.97\% & 1.62\% & \textbf{0.75\%} & 26.18\% & 7.17\% \\
    B & 4.75\% & 4.06\% & \textbf{1.09\%} & 20.77\% & 5.55\% \\
    E & 11.40\% & 1.91\% & \textbf{0.58\%} & 18.25\% & 11.59\% \\
    F & 15.97\% & 7.36\% & \textbf{1.36\%} & 39.24\% & 33.18\% \\
    M & 4.86\% & 3.43\% & \textbf{1.43\%} & 22.60\% & 2.37\% \\
    P & 15.53\% & 2.01\% & \textbf{0.93\%} & 17.28\% & 7.53\% \\
    X & 21.68\% & 7.09\% & \textbf{3.69\%} & 18.48\% & 14.78\% \\
    \midrule
    All   & 15.45\% & 4.94\% & \textbf{2.36\%} & 20.10\% & 11.49\% \\
    \bottomrule
    \end{tabular}%
  \caption{Results on CVRPLib}
  \label{cvrplib}%
\end{table}%





\section{Conclusion, Limitation, and Future Work}
\paragraph{Conclusion} This paper has proposed a novel destroy-and-repair framework using hyper-graphs (DRHG) for routing problems. By leveraging the condensed hyper-graph formulation of the destroyed problem, we have reduced the burden of model learning and constrained the input size of the model to the scale of destruction. Extensive experiments comparing our model with other representative NCO methods on both synthetic and real-world instances have demonstrated the superiority of DRHG across different problem scales and distributions. 

\paragraph{Limitation and Future Work} The DRHG shows great performance on TSP, but our current design for CVRP has not fully realized the potential of DRHG. It could be interesting to ameliorate the implementation of DRHG and extend it to other routing problems. Furthermore, future work could explore more sophisticated destruction methods other than clustering destruction.



\section*{Acknowledgments}
This work was supported by the Research Grants Council of the Hong Kong Special Administrative Region, China (GRF Project No. CityU 11215622), the National Natural Science Foundation of China (Grant No. 62106096 and Grant No. 62476118), the Natural Science Foundation of Guangdong Province (Grant No. 2024A1515011759), the National Natural Science Foundation of Shenzhen (Grant No. JCYJ20220530113013031).

% This must be in the first 5 lines to tell arXiv to use pdfLaTeX, which is strongly recommended.
\pdfoutput=1
% In particular, the hyperref package requires pdfLaTeX in order to break URLs across lines.

\documentclass[11pt]{article}

% Change "review" to "final" to generate the final (sometimes called camera-ready) version.
% Change to "preprint" to generate a non-anonymous version with page numbers.
\usepackage{acl}

% Standard package includes
\usepackage{times}
\usepackage{latexsym}

% Draw tables
\usepackage{booktabs}
\usepackage{multirow}
\usepackage{xcolor}
\usepackage{colortbl}
\usepackage{array} 
\usepackage{amsmath}

\newcolumntype{C}{>{\centering\arraybackslash}p{0.07\textwidth}}
% For proper rendering and hyphenation of words containing Latin characters (including in bib files)
\usepackage[T1]{fontenc}
% For Vietnamese characters
% \usepackage[T5]{fontenc}
% See https://www.latex-project.org/help/documentation/encguide.pdf for other character sets
% This assumes your files are encoded as UTF8
\usepackage[utf8]{inputenc}

% This is not strictly necessary, and may be commented out,
% but it will improve the layout of the manuscript,
% and will typically save some space.
\usepackage{microtype}
\DeclareMathOperator*{\argmax}{arg\,max}
% This is also not strictly necessary, and may be commented out.
% However, it will improve the aesthetics of text in
% the typewriter font.
\usepackage{inconsolata}

%Including images in your LaTeX document requires adding
%additional package(s)
\usepackage{graphicx}
% If the title and author information does not fit in the area allocated, uncomment the following
%
%\setlength\titlebox{<dim>}
%
% and set <dim> to something 5cm or larger.

\title{Wi-Chat: Large Language Model Powered Wi-Fi Sensing}

% Author information can be set in various styles:
% For several authors from the same institution:
% \author{Author 1 \and ... \and Author n \\
%         Address line \\ ... \\ Address line}
% if the names do not fit well on one line use
%         Author 1 \\ {\bf Author 2} \\ ... \\ {\bf Author n} \\
% For authors from different institutions:
% \author{Author 1 \\ Address line \\  ... \\ Address line
%         \And  ... \And
%         Author n \\ Address line \\ ... \\ Address line}
% To start a separate ``row'' of authors use \AND, as in
% \author{Author 1 \\ Address line \\  ... \\ Address line
%         \AND
%         Author 2 \\ Address line \\ ... \\ Address line \And
%         Author 3 \\ Address line \\ ... \\ Address line}

% \author{First Author \\
%   Affiliation / Address line 1 \\
%   Affiliation / Address line 2 \\
%   Affiliation / Address line 3 \\
%   \texttt{email@domain} \\\And
%   Second Author \\
%   Affiliation / Address line 1 \\
%   Affiliation / Address line 2 \\
%   Affiliation / Address line 3 \\
%   \texttt{email@domain} \\}
% \author{Haohan Yuan \qquad Haopeng Zhang\thanks{corresponding author} \\ 
%   ALOHA Lab, University of Hawaii at Manoa \\
%   % Affiliation / Address line 2 \\
%   % Affiliation / Address line 3 \\
%   \texttt{\{haohany,haopengz\}@hawaii.edu}}
  
\author{
{Haopeng Zhang$\dag$\thanks{These authors contributed equally to this work.}, Yili Ren$\ddagger$\footnotemark[1], Haohan Yuan$\dag$, Jingzhe Zhang$\ddagger$, Yitong Shen$\ddagger$} \\
ALOHA Lab, University of Hawaii at Manoa$\dag$, University of South Florida$\ddagger$ \\
\{haopengz, haohany\}@hawaii.edu\\
\{yiliren, jingzhe, shen202\}@usf.edu\\}



  
%\author{
%  \textbf{First Author\textsuperscript{1}},
%  \textbf{Second Author\textsuperscript{1,2}},
%  \textbf{Third T. Author\textsuperscript{1}},
%  \textbf{Fourth Author\textsuperscript{1}},
%\\
%  \textbf{Fifth Author\textsuperscript{1,2}},
%  \textbf{Sixth Author\textsuperscript{1}},
%  \textbf{Seventh Author\textsuperscript{1}},
%  \textbf{Eighth Author \textsuperscript{1,2,3,4}},
%\\
%  \textbf{Ninth Author\textsuperscript{1}},
%  \textbf{Tenth Author\textsuperscript{1}},
%  \textbf{Eleventh E. Author\textsuperscript{1,2,3,4,5}},
%  \textbf{Twelfth Author\textsuperscript{1}},
%\\
%  \textbf{Thirteenth Author\textsuperscript{3}},
%  \textbf{Fourteenth F. Author\textsuperscript{2,4}},
%  \textbf{Fifteenth Author\textsuperscript{1}},
%  \textbf{Sixteenth Author\textsuperscript{1}},
%\\
%  \textbf{Seventeenth S. Author\textsuperscript{4,5}},
%  \textbf{Eighteenth Author\textsuperscript{3,4}},
%  \textbf{Nineteenth N. Author\textsuperscript{2,5}},
%  \textbf{Twentieth Author\textsuperscript{1}}
%\\
%\\
%  \textsuperscript{1}Affiliation 1,
%  \textsuperscript{2}Affiliation 2,
%  \textsuperscript{3}Affiliation 3,
%  \textsuperscript{4}Affiliation 4,
%  \textsuperscript{5}Affiliation 5
%\\
%  \small{
%    \textbf{Correspondence:} \href{mailto:email@domain}{email@domain}
%  }
%}

\begin{document}
\maketitle
\begin{abstract}
Recent advancements in Large Language Models (LLMs) have demonstrated remarkable capabilities across diverse tasks. However, their potential to integrate physical model knowledge for real-world signal interpretation remains largely unexplored. In this work, we introduce Wi-Chat, the first LLM-powered Wi-Fi-based human activity recognition system. We demonstrate that LLMs can process raw Wi-Fi signals and infer human activities by incorporating Wi-Fi sensing principles into prompts. Our approach leverages physical model insights to guide LLMs in interpreting Channel State Information (CSI) data without traditional signal processing techniques. Through experiments on real-world Wi-Fi datasets, we show that LLMs exhibit strong reasoning capabilities, achieving zero-shot activity recognition. These findings highlight a new paradigm for Wi-Fi sensing, expanding LLM applications beyond conventional language tasks and enhancing the accessibility of wireless sensing for real-world deployments.
\end{abstract}

\section{Introduction}

In today’s rapidly evolving digital landscape, the transformative power of web technologies has redefined not only how services are delivered but also how complex tasks are approached. Web-based systems have become increasingly prevalent in risk control across various domains. This widespread adoption is due their accessibility, scalability, and ability to remotely connect various types of users. For example, these systems are used for process safety management in industry~\cite{kannan2016web}, safety risk early warning in urban construction~\cite{ding2013development}, and safe monitoring of infrastructural systems~\cite{repetto2018web}. Within these web-based risk management systems, the source search problem presents a huge challenge. Source search refers to the task of identifying the origin of a risky event, such as a gas leak and the emission point of toxic substances. This source search capability is crucial for effective risk management and decision-making.

Traditional approaches to implementing source search capabilities into the web systems often rely on solely algorithmic solutions~\cite{ristic2016study}. These methods, while relatively straightforward to implement, often struggle to achieve acceptable performances due to algorithmic local optima and complex unknown environments~\cite{zhao2020searching}. More recently, web crowdsourcing has emerged as a promising alternative for tackling the source search problem by incorporating human efforts in these web systems on-the-fly~\cite{zhao2024user}. This approach outsources the task of addressing issues encountered during the source search process to human workers, leveraging their capabilities to enhance system performance.

These solutions often employ a human-AI collaborative way~\cite{zhao2023leveraging} where algorithms handle exploration-exploitation and report the encountered problems while human workers resolve complex decision-making bottlenecks to help the algorithms getting rid of local deadlocks~\cite{zhao2022crowd}. Although effective, this paradigm suffers from two inherent limitations: increased operational costs from continuous human intervention, and slow response times of human workers due to sequential decision-making. These challenges motivate our investigation into developing autonomous systems that preserve human-like reasoning capabilities while reducing dependency on massive crowdsourced labor.

Furthermore, recent advancements in large language models (LLMs)~\cite{chang2024survey} and multi-modal LLMs (MLLMs)~\cite{huang2023chatgpt} have unveiled promising avenues for addressing these challenges. One clear opportunity involves the seamless integration of visual understanding and linguistic reasoning for robust decision-making in search tasks. However, whether large models-assisted source search is really effective and efficient for improving the current source search algorithms~\cite{ji2022source} remains unknown. \textit{To address the research gap, we are particularly interested in answering the following two research questions in this work:}

\textbf{\textit{RQ1: }}How can source search capabilities be integrated into web-based systems to support decision-making in time-sensitive risk management scenarios? 
% \sq{I mention ``time-sensitive'' here because I feel like we shall say something about the response time -- LLM has to be faster than humans}

\textbf{\textit{RQ2: }}How can MLLMs and LLMs enhance the effectiveness and efficiency of existing source search algorithms? 

% \textit{\textbf{RQ2:}} To what extent does the performance of large models-assisted search align with or approach the effectiveness of human-AI collaborative search? 

To answer the research questions, we propose a novel framework called Auto-\
S$^2$earch (\textbf{Auto}nomous \textbf{S}ource \textbf{Search}) and implement a prototype system that leverages advanced web technologies to simulate real-world conditions for zero-shot source search. Unlike traditional methods that rely on pre-defined heuristics or extensive human intervention, AutoS$^2$earch employs a carefully designed prompt that encapsulates human rationales, thereby guiding the MLLM to generate coherent and accurate scene descriptions from visual inputs about four directional choices. Based on these language-based descriptions, the LLM is enabled to determine the optimal directional choice through chain-of-thought (CoT) reasoning. Comprehensive empirical validation demonstrates that AutoS$^2$-\ 
earch achieves a success rate of 95–98\%, closely approaching the performance of human-AI collaborative search across 20 benchmark scenarios~\cite{zhao2023leveraging}. 

Our work indicates that the role of humans in future web crowdsourcing tasks may evolve from executors to validators or supervisors. Furthermore, incorporating explanations of LLM decisions into web-based system interfaces has the potential to help humans enhance task performance in risk control.






\section{Related Work}
\label{sec:relatedworks}

% \begin{table*}[t]
% \centering 
% \renewcommand\arraystretch{0.98}
% \fontsize{8}{10}\selectfont \setlength{\tabcolsep}{0.4em}
% \begin{tabular}{@{}lc|cc|cc|cc@{}}
% \toprule
% \textbf{Methods}           & \begin{tabular}[c]{@{}c@{}}\textbf{Training}\\ \textbf{Paradigm}\end{tabular} & \begin{tabular}[c]{@{}c@{}}\textbf{$\#$ PT Data}\\ \textbf{(Tokens)}\end{tabular} & \begin{tabular}[c]{@{}c@{}}\textbf{$\#$ IFT Data}\\ \textbf{(Samples)}\end{tabular} & \textbf{Code}  & \begin{tabular}[c]{@{}c@{}}\textbf{Natural}\\ \textbf{Language}\end{tabular} & \begin{tabular}[c]{@{}c@{}}\textbf{Action}\\ \textbf{Trajectories}\end{tabular} & \begin{tabular}[c]{@{}c@{}}\textbf{API}\\ \textbf{Documentation}\end{tabular}\\ \midrule 
% NexusRaven~\citep{srinivasan2023nexusraven} & IFT & - & - & \textcolor{green}{\CheckmarkBold} & \textcolor{green}{\CheckmarkBold} &\textcolor{red}{\XSolidBrush}&\textcolor{red}{\XSolidBrush}\\
% AgentInstruct~\citep{zeng2023agenttuning} & IFT & - & 2k & \textcolor{green}{\CheckmarkBold} & \textcolor{green}{\CheckmarkBold} &\textcolor{red}{\XSolidBrush}&\textcolor{red}{\XSolidBrush} \\
% AgentEvol~\citep{xi2024agentgym} & IFT & - & 14.5k & \textcolor{green}{\CheckmarkBold} & \textcolor{green}{\CheckmarkBold} &\textcolor{green}{\CheckmarkBold}&\textcolor{red}{\XSolidBrush} \\
% Gorilla~\citep{patil2023gorilla}& IFT & - & 16k & \textcolor{green}{\CheckmarkBold} & \textcolor{green}{\CheckmarkBold} &\textcolor{red}{\XSolidBrush}&\textcolor{green}{\CheckmarkBold}\\
% OpenFunctions-v2~\citep{patil2023gorilla} & IFT & - & 65k & \textcolor{green}{\CheckmarkBold} & \textcolor{green}{\CheckmarkBold} &\textcolor{red}{\XSolidBrush}&\textcolor{green}{\CheckmarkBold}\\
% LAM~\citep{zhang2024agentohana} & IFT & - & 42.6k & \textcolor{green}{\CheckmarkBold} & \textcolor{green}{\CheckmarkBold} &\textcolor{green}{\CheckmarkBold}&\textcolor{red}{\XSolidBrush} \\
% xLAM~\citep{liu2024apigen} & IFT & - & 60k & \textcolor{green}{\CheckmarkBold} & \textcolor{green}{\CheckmarkBold} &\textcolor{green}{\CheckmarkBold}&\textcolor{red}{\XSolidBrush} \\\midrule
% LEMUR~\citep{xu2024lemur} & PT & 90B & 300k & \textcolor{green}{\CheckmarkBold} & \textcolor{green}{\CheckmarkBold} &\textcolor{green}{\CheckmarkBold}&\textcolor{red}{\XSolidBrush}\\
% \rowcolor{teal!12} \method & PT & 103B & 95k & \textcolor{green}{\CheckmarkBold} & \textcolor{green}{\CheckmarkBold} & \textcolor{green}{\CheckmarkBold} & \textcolor{green}{\CheckmarkBold} \\
% \bottomrule
% \end{tabular}
% \caption{Summary of existing tuning- and pretraining-based LLM agents with their training sample sizes. "PT" and "IFT" denote "Pre-Training" and "Instruction Fine-Tuning", respectively. }
% \label{tab:related}
% \end{table*}

\begin{table*}[ht]
\begin{threeparttable}
\centering 
\renewcommand\arraystretch{0.98}
\fontsize{7}{9}\selectfont \setlength{\tabcolsep}{0.2em}
\begin{tabular}{@{}l|c|c|ccc|cc|cc|cccc@{}}
\toprule
\textbf{Methods} & \textbf{Datasets}           & \begin{tabular}[c]{@{}c@{}}\textbf{Training}\\ \textbf{Paradigm}\end{tabular} & \begin{tabular}[c]{@{}c@{}}\textbf{\# PT Data}\\ \textbf{(Tokens)}\end{tabular} & \begin{tabular}[c]{@{}c@{}}\textbf{\# IFT Data}\\ \textbf{(Samples)}\end{tabular} & \textbf{\# APIs} & \textbf{Code}  & \begin{tabular}[c]{@{}c@{}}\textbf{Nat.}\\ \textbf{Lang.}\end{tabular} & \begin{tabular}[c]{@{}c@{}}\textbf{Action}\\ \textbf{Traj.}\end{tabular} & \begin{tabular}[c]{@{}c@{}}\textbf{API}\\ \textbf{Doc.}\end{tabular} & \begin{tabular}[c]{@{}c@{}}\textbf{Func.}\\ \textbf{Call}\end{tabular} & \begin{tabular}[c]{@{}c@{}}\textbf{Multi.}\\ \textbf{Step}\end{tabular}  & \begin{tabular}[c]{@{}c@{}}\textbf{Plan}\\ \textbf{Refine}\end{tabular}  & \begin{tabular}[c]{@{}c@{}}\textbf{Multi.}\\ \textbf{Turn}\end{tabular}\\ \midrule 
\multicolumn{13}{l}{\emph{Instruction Finetuning-based LLM Agents for Intrinsic Reasoning}}  \\ \midrule
FireAct~\cite{chen2023fireact} & FireAct & IFT & - & 2.1K & 10 & \textcolor{red}{\XSolidBrush} &\textcolor{green}{\CheckmarkBold} &\textcolor{green}{\CheckmarkBold}  & \textcolor{red}{\XSolidBrush} &\textcolor{green}{\CheckmarkBold} & \textcolor{red}{\XSolidBrush} &\textcolor{green}{\CheckmarkBold} & \textcolor{red}{\XSolidBrush} \\
ToolAlpaca~\cite{tang2023toolalpaca} & ToolAlpaca & IFT & - & 4.0K & 400 & \textcolor{red}{\XSolidBrush} &\textcolor{green}{\CheckmarkBold} &\textcolor{green}{\CheckmarkBold} & \textcolor{red}{\XSolidBrush} &\textcolor{green}{\CheckmarkBold} & \textcolor{red}{\XSolidBrush}  &\textcolor{green}{\CheckmarkBold} & \textcolor{red}{\XSolidBrush}  \\
ToolLLaMA~\cite{qin2023toolllm} & ToolBench & IFT & - & 12.7K & 16,464 & \textcolor{red}{\XSolidBrush} &\textcolor{green}{\CheckmarkBold} &\textcolor{green}{\CheckmarkBold} &\textcolor{red}{\XSolidBrush} &\textcolor{green}{\CheckmarkBold}&\textcolor{green}{\CheckmarkBold}&\textcolor{green}{\CheckmarkBold} &\textcolor{green}{\CheckmarkBold}\\
AgentEvol~\citep{xi2024agentgym} & AgentTraj-L & IFT & - & 14.5K & 24 &\textcolor{red}{\XSolidBrush} & \textcolor{green}{\CheckmarkBold} &\textcolor{green}{\CheckmarkBold}&\textcolor{red}{\XSolidBrush} &\textcolor{green}{\CheckmarkBold}&\textcolor{red}{\XSolidBrush} &\textcolor{red}{\XSolidBrush} &\textcolor{green}{\CheckmarkBold}\\
Lumos~\cite{yin2024agent} & Lumos & IFT  & - & 20.0K & 16 &\textcolor{red}{\XSolidBrush} & \textcolor{green}{\CheckmarkBold} & \textcolor{green}{\CheckmarkBold} &\textcolor{red}{\XSolidBrush} & \textcolor{green}{\CheckmarkBold} & \textcolor{green}{\CheckmarkBold} &\textcolor{red}{\XSolidBrush} & \textcolor{green}{\CheckmarkBold}\\
Agent-FLAN~\cite{chen2024agent} & Agent-FLAN & IFT & - & 24.7K & 20 &\textcolor{red}{\XSolidBrush} & \textcolor{green}{\CheckmarkBold} & \textcolor{green}{\CheckmarkBold} &\textcolor{red}{\XSolidBrush} & \textcolor{green}{\CheckmarkBold}& \textcolor{green}{\CheckmarkBold}&\textcolor{red}{\XSolidBrush} & \textcolor{green}{\CheckmarkBold}\\
AgentTuning~\citep{zeng2023agenttuning} & AgentInstruct & IFT & - & 35.0K & - &\textcolor{red}{\XSolidBrush} & \textcolor{green}{\CheckmarkBold} & \textcolor{green}{\CheckmarkBold} &\textcolor{red}{\XSolidBrush} & \textcolor{green}{\CheckmarkBold} &\textcolor{red}{\XSolidBrush} &\textcolor{red}{\XSolidBrush} & \textcolor{green}{\CheckmarkBold}\\\midrule
\multicolumn{13}{l}{\emph{Instruction Finetuning-based LLM Agents for Function Calling}} \\\midrule
NexusRaven~\citep{srinivasan2023nexusraven} & NexusRaven & IFT & - & - & 116 & \textcolor{green}{\CheckmarkBold} & \textcolor{green}{\CheckmarkBold}  & \textcolor{green}{\CheckmarkBold} &\textcolor{red}{\XSolidBrush} & \textcolor{green}{\CheckmarkBold} &\textcolor{red}{\XSolidBrush} &\textcolor{red}{\XSolidBrush}&\textcolor{red}{\XSolidBrush}\\
Gorilla~\citep{patil2023gorilla} & Gorilla & IFT & - & 16.0K & 1,645 & \textcolor{green}{\CheckmarkBold} &\textcolor{red}{\XSolidBrush} &\textcolor{red}{\XSolidBrush}&\textcolor{green}{\CheckmarkBold} &\textcolor{green}{\CheckmarkBold} &\textcolor{red}{\XSolidBrush} &\textcolor{red}{\XSolidBrush} &\textcolor{red}{\XSolidBrush}\\
OpenFunctions-v2~\citep{patil2023gorilla} & OpenFunctions-v2 & IFT & - & 65.0K & - & \textcolor{green}{\CheckmarkBold} & \textcolor{green}{\CheckmarkBold} &\textcolor{red}{\XSolidBrush} &\textcolor{green}{\CheckmarkBold} &\textcolor{green}{\CheckmarkBold} &\textcolor{red}{\XSolidBrush} &\textcolor{red}{\XSolidBrush} &\textcolor{red}{\XSolidBrush}\\
API Pack~\cite{guo2024api} & API Pack & IFT & - & 1.1M & 11,213 &\textcolor{green}{\CheckmarkBold} &\textcolor{red}{\XSolidBrush} &\textcolor{green}{\CheckmarkBold} &\textcolor{red}{\XSolidBrush} &\textcolor{green}{\CheckmarkBold} &\textcolor{red}{\XSolidBrush}&\textcolor{red}{\XSolidBrush}&\textcolor{red}{\XSolidBrush}\\ 
LAM~\citep{zhang2024agentohana} & AgentOhana & IFT & - & 42.6K & - & \textcolor{green}{\CheckmarkBold} & \textcolor{green}{\CheckmarkBold} &\textcolor{green}{\CheckmarkBold}&\textcolor{red}{\XSolidBrush} &\textcolor{green}{\CheckmarkBold}&\textcolor{red}{\XSolidBrush}&\textcolor{green}{\CheckmarkBold}&\textcolor{green}{\CheckmarkBold}\\
xLAM~\citep{liu2024apigen} & APIGen & IFT & - & 60.0K & 3,673 & \textcolor{green}{\CheckmarkBold} & \textcolor{green}{\CheckmarkBold} &\textcolor{green}{\CheckmarkBold}&\textcolor{red}{\XSolidBrush} &\textcolor{green}{\CheckmarkBold}&\textcolor{red}{\XSolidBrush}&\textcolor{green}{\CheckmarkBold}&\textcolor{green}{\CheckmarkBold}\\\midrule
\multicolumn{13}{l}{\emph{Pretraining-based LLM Agents}}  \\\midrule
% LEMUR~\citep{xu2024lemur} & PT & 90B & 300.0K & - & \textcolor{green}{\CheckmarkBold} & \textcolor{green}{\CheckmarkBold} &\textcolor{green}{\CheckmarkBold}&\textcolor{red}{\XSolidBrush} & \textcolor{red}{\XSolidBrush} &\textcolor{green}{\CheckmarkBold} &\textcolor{red}{\XSolidBrush}&\textcolor{red}{\XSolidBrush}\\
\rowcolor{teal!12} \method & \dataset & PT & 103B & 95.0K  & 76,537  & \textcolor{green}{\CheckmarkBold} & \textcolor{green}{\CheckmarkBold} & \textcolor{green}{\CheckmarkBold} & \textcolor{green}{\CheckmarkBold} & \textcolor{green}{\CheckmarkBold} & \textcolor{green}{\CheckmarkBold} & \textcolor{green}{\CheckmarkBold} & \textcolor{green}{\CheckmarkBold}\\
\bottomrule
\end{tabular}
% \begin{tablenotes}
%     \item $^*$ In addition, the StarCoder-API can offer 4.77M more APIs.
% \end{tablenotes}
\caption{Summary of existing instruction finetuning-based LLM agents for intrinsic reasoning and function calling, along with their training resources and sample sizes. "PT" and "IFT" denote "Pre-Training" and "Instruction Fine-Tuning", respectively.}
\vspace{-2ex}
\label{tab:related}
\end{threeparttable}
\end{table*}

\noindent \textbf{Prompting-based LLM Agents.} Due to the lack of agent-specific pre-training corpus, existing LLM agents rely on either prompt engineering~\cite{hsieh2023tool,lu2024chameleon,yao2022react,wang2023voyager} or instruction fine-tuning~\cite{chen2023fireact,zeng2023agenttuning} to understand human instructions, decompose high-level tasks, generate grounded plans, and execute multi-step actions. 
However, prompting-based methods mainly depend on the capabilities of backbone LLMs (usually commercial LLMs), failing to introduce new knowledge and struggling to generalize to unseen tasks~\cite{sun2024adaplanner,zhuang2023toolchain}. 

\noindent \textbf{Instruction Finetuning-based LLM Agents.} Considering the extensive diversity of APIs and the complexity of multi-tool instructions, tool learning inherently presents greater challenges than natural language tasks, such as text generation~\cite{qin2023toolllm}.
Post-training techniques focus more on instruction following and aligning output with specific formats~\cite{patil2023gorilla,hao2024toolkengpt,qin2023toolllm,schick2024toolformer}, rather than fundamentally improving model knowledge or capabilities. 
Moreover, heavy fine-tuning can hinder generalization or even degrade performance in non-agent use cases, potentially suppressing the original base model capabilities~\cite{ghosh2024a}.

\noindent \textbf{Pretraining-based LLM Agents.} While pre-training serves as an essential alternative, prior works~\cite{nijkamp2023codegen,roziere2023code,xu2024lemur,patil2023gorilla} have primarily focused on improving task-specific capabilities (\eg, code generation) instead of general-domain LLM agents, due to single-source, uni-type, small-scale, and poor-quality pre-training data. 
Existing tool documentation data for agent training either lacks diverse real-world APIs~\cite{patil2023gorilla, tang2023toolalpaca} or is constrained to single-tool or single-round tool execution. 
Furthermore, trajectory data mostly imitate expert behavior or follow function-calling rules with inferior planning and reasoning, failing to fully elicit LLMs' capabilities and handle complex instructions~\cite{qin2023toolllm}. 
Given a wide range of candidate API functions, each comprising various function names and parameters available at every planning step, identifying globally optimal solutions and generalizing across tasks remains highly challenging.



\section{Preliminaries}
\label{Preliminaries}
\begin{figure*}[t]
    \centering
    \includegraphics[width=0.95\linewidth]{fig/HealthGPT_Framework.png}
    \caption{The \ourmethod{} architecture integrates hierarchical visual perception and H-LoRA, employing a task-specific hard router to select visual features and H-LoRA plugins, ultimately generating outputs with an autoregressive manner.}
    \label{fig:architecture}
\end{figure*}
\noindent\textbf{Large Vision-Language Models.} 
The input to a LVLM typically consists of an image $x^{\text{img}}$ and a discrete text sequence $x^{\text{txt}}$. The visual encoder $\mathcal{E}^{\text{img}}$ converts the input image $x^{\text{img}}$ into a sequence of visual tokens $\mathcal{V} = [v_i]_{i=1}^{N_v}$, while the text sequence $x^{\text{txt}}$ is mapped into a sequence of text tokens $\mathcal{T} = [t_i]_{i=1}^{N_t}$ using an embedding function $\mathcal{E}^{\text{txt}}$. The LLM $\mathcal{M_\text{LLM}}(\cdot|\theta)$ models the joint probability of the token sequence $\mathcal{U} = \{\mathcal{V},\mathcal{T}\}$, which is expressed as:
\begin{equation}
    P_\theta(R | \mathcal{U}) = \prod_{i=1}^{N_r} P_\theta(r_i | \{\mathcal{U}, r_{<i}\}),
\end{equation}
where $R = [r_i]_{i=1}^{N_r}$ is the text response sequence. The LVLM iteratively generates the next token $r_i$ based on $r_{<i}$. The optimization objective is to minimize the cross-entropy loss of the response $\mathcal{R}$.
% \begin{equation}
%     \mathcal{L}_{\text{VLM}} = \mathbb{E}_{R|\mathcal{U}}\left[-\log P_\theta(R | \mathcal{U})\right]
% \end{equation}
It is worth noting that most LVLMs adopt a design paradigm based on ViT, alignment adapters, and pre-trained LLMs\cite{liu2023llava,liu2024improved}, enabling quick adaptation to downstream tasks.


\noindent\textbf{VQGAN.}
VQGAN~\cite{esser2021taming} employs latent space compression and indexing mechanisms to effectively learn a complete discrete representation of images. VQGAN first maps the input image $x^{\text{img}}$ to a latent representation $z = \mathcal{E}(x)$ through a encoder $\mathcal{E}$. Then, the latent representation is quantized using a codebook $\mathcal{Z} = \{z_k\}_{k=1}^K$, generating a discrete index sequence $\mathcal{I} = [i_m]_{m=1}^N$, where $i_m \in \mathcal{Z}$ represents the quantized code index:
\begin{equation}
    \mathcal{I} = \text{Quantize}(z|\mathcal{Z}) = \arg\min_{z_k \in \mathcal{Z}} \| z - z_k \|_2.
\end{equation}
In our approach, the discrete index sequence $\mathcal{I}$ serves as a supervisory signal for the generation task, enabling the model to predict the index sequence $\hat{\mathcal{I}}$ from input conditions such as text or other modality signals.  
Finally, the predicted index sequence $\hat{\mathcal{I}}$ is upsampled by the VQGAN decoder $G$, generating the high-quality image $\hat{x}^\text{img} = G(\hat{\mathcal{I}})$.



\noindent\textbf{Low Rank Adaptation.} 
LoRA\cite{hu2021lora} effectively captures the characteristics of downstream tasks by introducing low-rank adapters. The core idea is to decompose the bypass weight matrix $\Delta W\in\mathbb{R}^{d^{\text{in}} \times d^{\text{out}}}$ into two low-rank matrices $ \{A \in \mathbb{R}^{d^{\text{in}} \times r}, B \in \mathbb{R}^{r \times d^{\text{out}}} \}$, where $ r \ll \min\{d^{\text{in}}, d^{\text{out}}\} $, significantly reducing learnable parameters. The output with the LoRA adapter for the input $x$ is then given by:
\begin{equation}
    h = x W_0 + \alpha x \Delta W/r = x W_0 + \alpha xAB/r,
\end{equation}
where matrix $ A $ is initialized with a Gaussian distribution, while the matrix $ B $ is initialized as a zero matrix. The scaling factor $ \alpha/r $ controls the impact of $ \Delta W $ on the model.

\section{HealthGPT}
\label{Method}


\subsection{Unified Autoregressive Generation.}  
% As shown in Figure~\ref{fig:architecture}, 
\ourmethod{} (Figure~\ref{fig:architecture}) utilizes a discrete token representation that covers both text and visual outputs, unifying visual comprehension and generation as an autoregressive task. 
For comprehension, $\mathcal{M}_\text{llm}$ receives the input joint sequence $\mathcal{U}$ and outputs a series of text token $\mathcal{R} = [r_1, r_2, \dots, r_{N_r}]$, where $r_i \in \mathcal{V}_{\text{txt}}$, and $\mathcal{V}_{\text{txt}}$ represents the LLM's vocabulary:
\begin{equation}
    P_\theta(\mathcal{R} \mid \mathcal{U}) = \prod_{i=1}^{N_r} P_\theta(r_i \mid \mathcal{U}, r_{<i}).
\end{equation}
For generation, $\mathcal{M}_\text{llm}$ first receives a special start token $\langle \text{START\_IMG} \rangle$, then generates a series of tokens corresponding to the VQGAN indices $\mathcal{I} = [i_1, i_2, \dots, i_{N_i}]$, where $i_j \in \mathcal{V}_{\text{vq}}$, and $\mathcal{V}_{\text{vq}}$ represents the index range of VQGAN. Upon completion of generation, the LLM outputs an end token $\langle \text{END\_IMG} \rangle$:
\begin{equation}
    P_\theta(\mathcal{I} \mid \mathcal{U}) = \prod_{j=1}^{N_i} P_\theta(i_j \mid \mathcal{U}, i_{<j}).
\end{equation}
Finally, the generated index sequence $\mathcal{I}$ is fed into the decoder $G$, which reconstructs the target image $\hat{x}^{\text{img}} = G(\mathcal{I})$.

\subsection{Hierarchical Visual Perception}  
Given the differences in visual perception between comprehension and generation tasks—where the former focuses on abstract semantics and the latter emphasizes complete semantics—we employ ViT to compress the image into discrete visual tokens at multiple hierarchical levels.
Specifically, the image is converted into a series of features $\{f_1, f_2, \dots, f_L\}$ as it passes through $L$ ViT blocks.

To address the needs of various tasks, the hidden states are divided into two types: (i) \textit{Concrete-grained features} $\mathcal{F}^{\text{Con}} = \{f_1, f_2, \dots, f_k\}, k < L$, derived from the shallower layers of ViT, containing sufficient global features, suitable for generation tasks; 
(ii) \textit{Abstract-grained features} $\mathcal{F}^{\text{Abs}} = \{f_{k+1}, f_{k+2}, \dots, f_L\}$, derived from the deeper layers of ViT, which contain abstract semantic information closer to the text space, suitable for comprehension tasks.

The task type $T$ (comprehension or generation) determines which set of features is selected as the input for the downstream large language model:
\begin{equation}
    \mathcal{F}^{\text{img}}_T =
    \begin{cases}
        \mathcal{F}^{\text{Con}}, & \text{if } T = \text{generation task} \\
        \mathcal{F}^{\text{Abs}}, & \text{if } T = \text{comprehension task}
    \end{cases}
\end{equation}
We integrate the image features $\mathcal{F}^{\text{img}}_T$ and text features $\mathcal{T}$ into a joint sequence through simple concatenation, which is then fed into the LLM $\mathcal{M}_{\text{llm}}$ for autoregressive generation.
% :
% \begin{equation}
%     \mathcal{R} = \mathcal{M}_{\text{llm}}(\mathcal{U}|\theta), \quad \mathcal{U} = [\mathcal{F}^{\text{img}}_T; \mathcal{T}]
% \end{equation}
\subsection{Heterogeneous Knowledge Adaptation}
We devise H-LoRA, which stores heterogeneous knowledge from comprehension and generation tasks in separate modules and dynamically routes to extract task-relevant knowledge from these modules. 
At the task level, for each task type $ T $, we dynamically assign a dedicated H-LoRA submodule $ \theta^T $, which is expressed as:
\begin{equation}
    \mathcal{R} = \mathcal{M}_\text{LLM}(\mathcal{U}|\theta, \theta^T), \quad \theta^T = \{A^T, B^T, \mathcal{R}^T_\text{outer}\}.
\end{equation}
At the feature level for a single task, H-LoRA integrates the idea of Mixture of Experts (MoE)~\cite{masoudnia2014mixture} and designs an efficient matrix merging and routing weight allocation mechanism, thus avoiding the significant computational delay introduced by matrix splitting in existing MoELoRA~\cite{luo2024moelora}. Specifically, we first merge the low-rank matrices (rank = r) of $ k $ LoRA experts into a unified matrix:
\begin{equation}
    \mathbf{A}^{\text{merged}}, \mathbf{B}^{\text{merged}} = \text{Concat}(\{A_i\}_1^k), \text{Concat}(\{B_i\}_1^k),
\end{equation}
where $ \mathbf{A}^{\text{merged}} \in \mathbb{R}^{d^\text{in} \times rk} $ and $ \mathbf{B}^{\text{merged}} \in \mathbb{R}^{rk \times d^\text{out}} $. The $k$-dimension routing layer generates expert weights $ \mathcal{W} \in \mathbb{R}^{\text{token\_num} \times k} $ based on the input hidden state $ x $, and these are expanded to $ \mathbb{R}^{\text{token\_num} \times rk} $ as follows:
\begin{equation}
    \mathcal{W}^\text{expanded} = \alpha k \mathcal{W} / r \otimes \mathbf{1}_r,
\end{equation}
where $ \otimes $ denotes the replication operation.
The overall output of H-LoRA is computed as:
\begin{equation}
    \mathcal{O}^\text{H-LoRA} = (x \mathbf{A}^{\text{merged}} \odot \mathcal{W}^\text{expanded}) \mathbf{B}^{\text{merged}},
\end{equation}
where $ \odot $ represents element-wise multiplication. Finally, the output of H-LoRA is added to the frozen pre-trained weights to produce the final output:
\begin{equation}
    \mathcal{O} = x W_0 + \mathcal{O}^\text{H-LoRA}.
\end{equation}
% In summary, H-LoRA is a task-based dynamic PEFT method that achieves high efficiency in single-task fine-tuning.

\subsection{Training Pipeline}

\begin{figure}[t]
    \centering
    \hspace{-4mm}
    \includegraphics[width=0.94\linewidth]{fig/data.pdf}
    \caption{Data statistics of \texttt{VL-Health}. }
    \label{fig:data}
\end{figure}
\noindent \textbf{1st Stage: Multi-modal Alignment.} 
In the first stage, we design separate visual adapters and H-LoRA submodules for medical unified tasks. For the medical comprehension task, we train abstract-grained visual adapters using high-quality image-text pairs to align visual embeddings with textual embeddings, thereby enabling the model to accurately describe medical visual content. During this process, the pre-trained LLM and its corresponding H-LoRA submodules remain frozen. In contrast, the medical generation task requires training concrete-grained adapters and H-LoRA submodules while keeping the LLM frozen. Meanwhile, we extend the textual vocabulary to include multimodal tokens, enabling the support of additional VQGAN vector quantization indices. The model trains on image-VQ pairs, endowing the pre-trained LLM with the capability for image reconstruction. This design ensures pixel-level consistency of pre- and post-LVLM. The processes establish the initial alignment between the LLM’s outputs and the visual inputs.

\noindent \textbf{2nd Stage: Heterogeneous H-LoRA Plugin Adaptation.}  
The submodules of H-LoRA share the word embedding layer and output head but may encounter issues such as bias and scale inconsistencies during training across different tasks. To ensure that the multiple H-LoRA plugins seamlessly interface with the LLMs and form a unified base, we fine-tune the word embedding layer and output head using a small amount of mixed data to maintain consistency in the model weights. Specifically, during this stage, all H-LoRA submodules for different tasks are kept frozen, with only the word embedding layer and output head being optimized. Through this stage, the model accumulates foundational knowledge for unified tasks by adapting H-LoRA plugins.

\begin{table*}[!t]
\centering
\caption{Comparison of \ourmethod{} with other LVLMs and unified multi-modal models on medical visual comprehension tasks. \textbf{Bold} and \underline{underlined} text indicates the best performance and second-best performance, respectively.}
\resizebox{\textwidth}{!}{
\begin{tabular}{c|lcc|cccccccc|c}
\toprule
\rowcolor[HTML]{E9F3FE} &  &  &  & \multicolumn{2}{c}{\textbf{VQA-RAD \textuparrow}} & \multicolumn{2}{c}{\textbf{SLAKE \textuparrow}} & \multicolumn{2}{c}{\textbf{PathVQA \textuparrow}} &  &  &  \\ 
\cline{5-10}
\rowcolor[HTML]{E9F3FE}\multirow{-2}{*}{\textbf{Type}} & \multirow{-2}{*}{\textbf{Model}} & \multirow{-2}{*}{\textbf{\# Params}} & \multirow{-2}{*}{\makecell{\textbf{Medical} \\ \textbf{LVLM}}} & \textbf{close} & \textbf{all} & \textbf{close} & \textbf{all} & \textbf{close} & \textbf{all} & \multirow{-2}{*}{\makecell{\textbf{MMMU} \\ \textbf{-Med}}\textuparrow} & \multirow{-2}{*}{\textbf{OMVQA}\textuparrow} & \multirow{-2}{*}{\textbf{Avg. \textuparrow}} \\ 
\midrule \midrule
\multirow{9}{*}{\textbf{Comp. Only}} 
& Med-Flamingo & 8.3B & \Large \ding{51} & 58.6 & 43.0 & 47.0 & 25.5 & 61.9 & 31.3 & 28.7 & 34.9 & 41.4 \\
& LLaVA-Med & 7B & \Large \ding{51} & 60.2 & 48.1 & 58.4 & 44.8 & 62.3 & 35.7 & 30.0 & 41.3 & 47.6 \\
& HuatuoGPT-Vision & 7B & \Large \ding{51} & 66.9 & 53.0 & 59.8 & 49.1 & 52.9 & 32.0 & 42.0 & 50.0 & 50.7 \\
& BLIP-2 & 6.7B & \Large \ding{55} & 43.4 & 36.8 & 41.6 & 35.3 & 48.5 & 28.8 & 27.3 & 26.9 & 36.1 \\
& LLaVA-v1.5 & 7B & \Large \ding{55} & 51.8 & 42.8 & 37.1 & 37.7 & 53.5 & 31.4 & 32.7 & 44.7 & 41.5 \\
& InstructBLIP & 7B & \Large \ding{55} & 61.0 & 44.8 & 66.8 & 43.3 & 56.0 & 32.3 & 25.3 & 29.0 & 44.8 \\
& Yi-VL & 6B & \Large \ding{55} & 52.6 & 42.1 & 52.4 & 38.4 & 54.9 & 30.9 & 38.0 & 50.2 & 44.9 \\
& InternVL2 & 8B & \Large \ding{55} & 64.9 & 49.0 & 66.6 & 50.1 & 60.0 & 31.9 & \underline{43.3} & 54.5 & 52.5\\
& Llama-3.2 & 11B & \Large \ding{55} & 68.9 & 45.5 & 72.4 & 52.1 & 62.8 & 33.6 & 39.3 & 63.2 & 54.7 \\
\midrule
\multirow{5}{*}{\textbf{Comp. \& Gen.}} 
& Show-o & 1.3B & \Large \ding{55} & 50.6 & 33.9 & 31.5 & 17.9 & 52.9 & 28.2 & 22.7 & 45.7 & 42.6 \\
& Unified-IO 2 & 7B & \Large \ding{55} & 46.2 & 32.6 & 35.9 & 21.9 & 52.5 & 27.0 & 25.3 & 33.0 & 33.8 \\
& Janus & 1.3B & \Large \ding{55} & 70.9 & 52.8 & 34.7 & 26.9 & 51.9 & 27.9 & 30.0 & 26.8 & 33.5 \\
& \cellcolor[HTML]{DAE0FB}HealthGPT-M3 & \cellcolor[HTML]{DAE0FB}3.8B & \cellcolor[HTML]{DAE0FB}\Large \ding{51} & \cellcolor[HTML]{DAE0FB}\underline{73.7} & \cellcolor[HTML]{DAE0FB}\underline{55.9} & \cellcolor[HTML]{DAE0FB}\underline{74.6} & \cellcolor[HTML]{DAE0FB}\underline{56.4} & \cellcolor[HTML]{DAE0FB}\underline{78.7} & \cellcolor[HTML]{DAE0FB}\underline{39.7} & \cellcolor[HTML]{DAE0FB}\underline{43.3} & \cellcolor[HTML]{DAE0FB}\underline{68.5} & \cellcolor[HTML]{DAE0FB}\underline{61.3} \\
& \cellcolor[HTML]{DAE0FB}HealthGPT-L14 & \cellcolor[HTML]{DAE0FB}14B & \cellcolor[HTML]{DAE0FB}\Large \ding{51} & \cellcolor[HTML]{DAE0FB}\textbf{77.7} & \cellcolor[HTML]{DAE0FB}\textbf{58.3} & \cellcolor[HTML]{DAE0FB}\textbf{76.4} & \cellcolor[HTML]{DAE0FB}\textbf{64.5} & \cellcolor[HTML]{DAE0FB}\textbf{85.9} & \cellcolor[HTML]{DAE0FB}\textbf{44.4} & \cellcolor[HTML]{DAE0FB}\textbf{49.2} & \cellcolor[HTML]{DAE0FB}\textbf{74.4} & \cellcolor[HTML]{DAE0FB}\textbf{66.4} \\
\bottomrule
\end{tabular}
}
\label{tab:results}
\end{table*}
\begin{table*}[ht]
    \centering
    \caption{The experimental results for the four modality conversion tasks.}
    \resizebox{\textwidth}{!}{
    \begin{tabular}{l|ccc|ccc|ccc|ccc}
        \toprule
        \rowcolor[HTML]{E9F3FE} & \multicolumn{3}{c}{\textbf{CT to MRI (Brain)}} & \multicolumn{3}{c}{\textbf{CT to MRI (Pelvis)}} & \multicolumn{3}{c}{\textbf{MRI to CT (Brain)}} & \multicolumn{3}{c}{\textbf{MRI to CT (Pelvis)}} \\
        \cline{2-13}
        \rowcolor[HTML]{E9F3FE}\multirow{-2}{*}{\textbf{Model}}& \textbf{SSIM $\uparrow$} & \textbf{PSNR $\uparrow$} & \textbf{MSE $\downarrow$} & \textbf{SSIM $\uparrow$} & \textbf{PSNR $\uparrow$} & \textbf{MSE $\downarrow$} & \textbf{SSIM $\uparrow$} & \textbf{PSNR $\uparrow$} & \textbf{MSE $\downarrow$} & \textbf{SSIM $\uparrow$} & \textbf{PSNR $\uparrow$} & \textbf{MSE $\downarrow$} \\
        \midrule \midrule
        pix2pix & 71.09 & 32.65 & 36.85 & 59.17 & 31.02 & 51.91 & 78.79 & 33.85 & 28.33 & 72.31 & 32.98 & 36.19 \\
        CycleGAN & 54.76 & 32.23 & 40.56 & 54.54 & 30.77 & 55.00 & 63.75 & 31.02 & 52.78 & 50.54 & 29.89 & 67.78 \\
        BBDM & {71.69} & {32.91} & {34.44} & 57.37 & 31.37 & 48.06 & \textbf{86.40} & 34.12 & 26.61 & {79.26} & 33.15 & 33.60 \\
        Vmanba & 69.54 & 32.67 & 36.42 & {63.01} & {31.47} & {46.99} & 79.63 & 34.12 & 26.49 & 77.45 & 33.53 & 31.85 \\
        DiffMa & 71.47 & 32.74 & 35.77 & 62.56 & 31.43 & 47.38 & 79.00 & {34.13} & {26.45} & 78.53 & {33.68} & {30.51} \\
        \rowcolor[HTML]{DAE0FB}HealthGPT-M3 & \underline{79.38} & \underline{33.03} & \underline{33.48} & \underline{71.81} & \underline{31.83} & \underline{43.45} & {85.06} & \textbf{34.40} & \textbf{25.49} & \underline{84.23} & \textbf{34.29} & \textbf{27.99} \\
        \rowcolor[HTML]{DAE0FB}HealthGPT-L14 & \textbf{79.73} & \textbf{33.10} & \textbf{32.96} & \textbf{71.92} & \textbf{31.87} & \textbf{43.09} & \underline{85.31} & \underline{34.29} & \underline{26.20} & \textbf{84.96} & \underline{34.14} & \underline{28.13} \\
        \bottomrule
    \end{tabular}
    }
    \label{tab:conversion}
\end{table*}

\noindent \textbf{3rd Stage: Visual Instruction Fine-Tuning.}  
In the third stage, we introduce additional task-specific data to further optimize the model and enhance its adaptability to downstream tasks such as medical visual comprehension (e.g., medical QA, medical dialogues, and report generation) or generation tasks (e.g., super-resolution, denoising, and modality conversion). Notably, by this stage, the word embedding layer and output head have been fine-tuned, only the H-LoRA modules and adapter modules need to be trained. This strategy significantly improves the model's adaptability and flexibility across different tasks.


\section{Experiment}
\label{s:experiment}

\subsection{Data Description}
We evaluate our method on FI~\cite{you2016building}, Twitter\_LDL~\cite{yang2017learning} and Artphoto~\cite{machajdik2010affective}.
FI is a public dataset built from Flickr and Instagram, with 23,308 images and eight emotion categories, namely \textit{amusement}, \textit{anger}, \textit{awe},  \textit{contentment}, \textit{disgust}, \textit{excitement},  \textit{fear}, and \textit{sadness}. 
% Since images in FI are all copyrighted by law, some images are corrupted now, so we remove these samples and retain 21,828 images.
% T4SA contains images from Twitter, which are classified into three categories: \textit{positive}, \textit{neutral}, and \textit{negative}. In this paper, we adopt the base version of B-T4SA, which contains 470,586 images and provides text descriptions of the corresponding tweets.
Twitter\_LDL contains 10,045 images from Twitter, with the same eight categories as the FI dataset.
% 。
For these two datasets, they are randomly split into 80\%
training and 20\% testing set.
Artphoto contains 806 artistic photos from the DeviantArt website, which we use to further evaluate the zero-shot capability of our model.
% on the small-scale dataset.
% We construct and publicly release the first image sentiment analysis dataset containing metadata.
% 。

% Based on these datasets, we are the first to construct and publicly release metadata-enhanced image sentiment analysis datasets. These datasets include scenes, tags, descriptions, and corresponding confidence scores, and are available at this link for future research purposes.


% 
\begin{table}[t]
\centering
% \begin{center}
\caption{Overall performance of different models on FI and Twitter\_LDL datasets.}
\label{tab:cap1}
% \resizebox{\linewidth}{!}
{
\begin{tabular}{l|c|c|c|c}
\hline
\multirow{2}{*}{\textbf{Model}} & \multicolumn{2}{c|}{\textbf{FI}}  & \multicolumn{2}{c}{\textbf{Twitter\_LDL}} \\ \cline{2-5} 
  & \textbf{Accuracy} & \textbf{F1} & \textbf{Accuracy} & \textbf{F1}  \\ \hline
% (\rownumber)~AlexNet~\cite{krizhevsky2017imagenet}  & 58.13\% & 56.35\%  & 56.24\%& 55.02\%  \\ 
% (\rownumber)~VGG16~\cite{simonyan2014very}  & 63.75\%& 63.08\%  & 59.34\%& 59.02\%  \\ 
(\rownumber)~ResNet101~\cite{he2016deep} & 66.16\%& 65.56\%  & 62.02\% & 61.34\%  \\ 
(\rownumber)~CDA~\cite{han2023boosting} & 66.71\%& 65.37\%  & 64.14\% & 62.85\%  \\ 
(\rownumber)~CECCN~\cite{ruan2024color} & 67.96\%& 66.74\%  & 64.59\%& 64.72\% \\ 
(\rownumber)~EmoVIT~\cite{xie2024emovit} & 68.09\%& 67.45\%  & 63.12\% & 61.97\%  \\ 
(\rownumber)~ComLDL~\cite{zhang2022compound} & 68.83\%& 67.28\%  & 65.29\% & 63.12\%  \\ 
(\rownumber)~WSDEN~\cite{li2023weakly} & 69.78\%& 69.61\%  & 67.04\% & 65.49\% \\ 
(\rownumber)~ECWA~\cite{deng2021emotion} & 70.87\%& 69.08\%  & 67.81\% & 66.87\%  \\ 
(\rownumber)~EECon~\cite{yang2023exploiting} & 71.13\%& 68.34\%  & 64.27\%& 63.16\%  \\ 
(\rownumber)~MAM~\cite{zhang2024affective} & 71.44\%  & 70.83\% & 67.18\%  & 65.01\%\\ 
(\rownumber)~TGCA-PVT~\cite{chen2024tgca}   & 73.05\%  & 71.46\% & 69.87\%  & 68.32\% \\ 
(\rownumber)~OEAN~\cite{zhang2024object}   & 73.40\%  & 72.63\% & 70.52\%  & 69.47\% \\ \hline
(\rownumber)~\shortname  & \textbf{79.48\%} & \textbf{79.22\%} & \textbf{74.12\%} & \textbf{73.09\%} \\ \hline
\end{tabular}
}
\vspace{-6mm}
% \end{center}
\end{table}
% 

\subsection{Experiment Setting}
% \subsubsection{Model Setting.}
% 
\textbf{Model Setting:}
For feature representation, we set $k=10$ to select object tags, and adopt clip-vit-base-patch32 as the pre-trained model for unified feature representation.
Moreover, we empirically set $(d_e, d_h, d_k, d_s) = (512, 128, 16, 64)$, and set the classification class $L$ to 8.

% 

\textbf{Training Setting:}
To initialize the model, we set all weights such as $\boldsymbol{W}$ following the truncated normal distribution, and use AdamW optimizer with the learning rate of $1 \times 10^{-4}$.
% warmup scheduler of cosine, warmup steps of 2000.
Furthermore, we set the batch size to 32 and the epoch of the training process to 200.
During the implementation, we utilize \textit{PyTorch} to build our entire model.
% , and our project codes are publicly available at https://github.com/zzmyrep/MESN.
% Our project codes as well as data are all publicly available on GitHub\footnote{https://github.com/zzmyrep/KBCEN}.
% Code is available at \href{https://github.com/zzmyrep/KBCEN}{https://github.com/zzmyrep/KBCEN}.

\textbf{Evaluation Metrics:}
Following~\cite{zhang2024affective, chen2024tgca, zhang2024object}, we adopt \textit{accuracy} and \textit{F1} as our evaluation metrics to measure the performance of different methods for image sentiment analysis. 



\subsection{Experiment Result}
% We compare our model against the following baselines: AlexNet~\cite{krizhevsky2017imagenet}, VGG16~\cite{simonyan2014very}, ResNet101~\cite{he2016deep}, CECCN~\cite{ruan2024color}, EmoVIT~\cite{xie2024emovit}, WSCNet~\cite{yang2018weakly}, ECWA~\cite{deng2021emotion}, EECon~\cite{yang2023exploiting}, MAM~\cite{zhang2024affective} and TGCA-PVT~\cite{chen2024tgca}, and the overall results are summarized in Table~\ref{tab:cap1}.
We compare our model against several baselines, and the overall results are summarized in Table~\ref{tab:cap1}.
We observe that our model achieves the best performance in both accuracy and F1 metrics, significantly outperforming the previous models. 
This superior performance is mainly attributed to our effective utilization of metadata to enhance image sentiment analysis, as well as the exceptional capability of the unified sentiment transformer framework we developed. These results strongly demonstrate that our proposed method can bring encouraging performance for image sentiment analysis.

\setcounter{magicrownumbers}{0} 
\begin{table}[t]
\begin{center}
\caption{Ablation study of~\shortname~on FI dataset.} 
% \vspace{1mm}
\label{tab:cap2}
\resizebox{.9\linewidth}{!}
{
\begin{tabular}{lcc}
  \hline
  \textbf{Model} & \textbf{Accuracy} & \textbf{F1} \\
  \hline
  (\rownumber)~Ours (w/o vision) & 65.72\% & 64.54\% \\
  (\rownumber)~Ours (w/o text description) & 74.05\% & 72.58\% \\
  (\rownumber)~Ours (w/o object tag) & 77.45\% & 76.84\% \\
  (\rownumber)~Ours (w/o scene tag) & 78.47\% & 78.21\% \\
  \hline
  (\rownumber)~Ours (w/o unified embedding) & 76.41\% & 76.23\% \\
  (\rownumber)~Ours (w/o adaptive learning) & 76.83\% & 76.56\% \\
  (\rownumber)~Ours (w/o cross-modal fusion) & 76.85\% & 76.49\% \\
  \hline
  (\rownumber)~Ours  & \textbf{79.48\%} & \textbf{79.22\%} \\
  \hline
\end{tabular}
}
\end{center}
\vspace{-5mm}
\end{table}


\begin{figure}[t]
\centering
% \vspace{-2mm}
\includegraphics[width=0.42\textwidth]{fig/2dvisual-linux4-paper2.pdf}
\caption{Visualization of feature distribution on eight categories before (left) and after (right) model processing.}
% 
\label{fig:visualization}
\vspace{-5mm}
\end{figure}

\subsection{Ablation Performance}
In this subsection, we conduct an ablation study to examine which component is really important for performance improvement. The results are reported in Table~\ref{tab:cap2}.

For information utilization, we observe a significant decline in model performance when visual features are removed. Additionally, the performance of \shortname~decreases when different metadata are removed separately, which means that text description, object tag, and scene tag are all critical for image sentiment analysis.
Recalling the model architecture, we separately remove transformer layers of the unified representation module, the adaptive learning module, and the cross-modal fusion module, replacing them with MLPs of the same parameter scale.
In this way, we can observe varying degrees of decline in model performance, indicating that these modules are indispensable for our model to achieve better performance.

\subsection{Visualization}
% 


% % 开始使用minipage进行左右排列
% \begin{minipage}[t]{0.45\textwidth}  % 子图1宽度为45%
%     \centering
%     \includegraphics[width=\textwidth]{2dvisual.pdf}  % 插入图片
%     \captionof{figure}{Visualization of feature distribution.}  % 使用captionof添加图片标题
%     \label{fig:visualization}
% \end{minipage}


% \begin{figure}[t]
% \centering
% \vspace{-2mm}
% \includegraphics[width=0.45\textwidth]{fig/2dvisual.pdf}
% \caption{Visualization of feature distribution.}
% \label{fig:visualization}
% % \vspace{-4mm}
% \end{figure}

% \begin{figure}[t]
% \centering
% \vspace{-2mm}
% \includegraphics[width=0.45\textwidth]{fig/2dvisual-linux3-paper.pdf}
% \caption{Visualization of feature distribution.}
% \label{fig:visualization}
% % \vspace{-4mm}
% \end{figure}



\begin{figure}[tbp]   
\vspace{-4mm}
  \centering            
  \subfloat[Depth of adaptive learning layers]   
  {
    \label{fig:subfig1}\includegraphics[width=0.22\textwidth]{fig/fig_sensitivity-a5}
  }
  \subfloat[Depth of fusion layers]
  {
    % \label{fig:subfig2}\includegraphics[width=0.22\textwidth]{fig/fig_sensitivity-b2}
    \label{fig:subfig2}\includegraphics[width=0.22\textwidth]{fig/fig_sensitivity-b2-num.pdf}
  }
  \caption{Sensitivity study of \shortname~on different depth. }   
  \label{fig:fig_sensitivity}  
\vspace{-2mm}
\end{figure}

% \begin{figure}[htbp]
% \centerline{\includegraphics{2dvisual.pdf}}
% \caption{Visualization of feature distribution.}
% \label{fig:visualization}
% \end{figure}

% In Fig.~\ref{fig:visualization}, we use t-SNE~\cite{van2008visualizing} to reduce the dimension of data features for visualization, Figure in left represents the metadata features before model processing, the features are obtained by embedding through the CLIP model, and figure in right shows the features of the data after model processing, it can be observed that after the model processing, the data with different label categories fall in different regions in the space, therefore, we can conclude that the Therefore, we can conclude that the model can effectively utilize the information contained in the metadata and use it to guide the model for classification.

In Fig.~\ref{fig:visualization}, we use t-SNE~\cite{van2008visualizing} to reduce the dimension of data features for visualization.
The left figure shows metadata features before being processed by our model (\textit{i.e.}, embedded by CLIP), while the right shows the distribution of features after being processed by our model.
We can observe that after the model processing, data with the same label are closer to each other, while others are farther away.
Therefore, it shows that the model can effectively utilize the information contained in the metadata and use it to guide the classification process.

\subsection{Sensitivity Analysis}
% 
In this subsection, we conduct a sensitivity analysis to figure out the effect of different depth settings of adaptive learning layers and fusion layers. 
% In this subsection, we conduct a sensitivity analysis to figure out the effect of different depth settings on the model. 
% Fig.~\ref{fig:fig_sensitivity} presents the effect of different depth settings of adaptive learning layers and fusion layers. 
Taking Fig.~\ref{fig:fig_sensitivity} (a) as an example, the model performance improves with increasing depth, reaching the best performance at a depth of 4.
% Taking Fig.~\ref{fig:fig_sensitivity} (a) as an example, the performance of \shortname~improves with the increase of depth at first, reaching the best performance at a depth of 4.
When the depth continues to increase, the accuracy decreases to varying degrees.
Similar results can be observed in Fig.~\ref{fig:fig_sensitivity} (b).
Therefore, we set their depths to 4 and 6 respectively to achieve the best results.

% Through our experiments, we can observe that the effect of modifying these hyperparameters on the results of the experiments is very weak, and the surface model is not sensitive to the hyperparameters.


\subsection{Zero-shot Capability}
% 

% (1)~GCH~\cite{2010Analyzing} & 21.78\% & (5)~RA-DLNet~\cite{2020A} & 34.01\% \\ \hline
% (2)~WSCNet~\cite{2019WSCNet}  & 30.25\% & (6)~CECCN~\cite{ruan2024color} & 43.83\% \\ \hline
% (3)~PCNN~\cite{2015Robust} & 31.68\%  & (7)~EmoVIT~\cite{xie2024emovit} & 44.90\% \\ \hline
% (4)~AR~\cite{2018Visual} & 32.67\% & (8)~Ours (Zero-shot) & 47.83\% \\ \hline


\begin{table}[t]
\centering
\caption{Zero-shot capability of \shortname.}
\label{tab:cap3}
\resizebox{1\linewidth}{!}
{
\begin{tabular}{lc|lc}
\hline
\textbf{Model} & \textbf{Accuracy} & \textbf{Model} & \textbf{Accuracy} \\ \hline
(1)~WSCNet~\cite{2019WSCNet}  & 30.25\% & (5)~MAM~\cite{zhang2024affective} & 39.56\%  \\ \hline
(2)~AR~\cite{2018Visual} & 32.67\% & (6)~CECCN~\cite{ruan2024color} & 43.83\% \\ \hline
(3)~RA-DLNet~\cite{2020A} & 34.01\%  & (7)~EmoVIT~\cite{xie2024emovit} & 44.90\% \\ \hline
(4)~CDA~\cite{han2023boosting} & 38.64\% & (8)~Ours (Zero-shot) & 47.83\% \\ \hline
\end{tabular}
}
\vspace{-5mm}
\end{table}

% We use the model trained on the FI dataset to test on the artphoto dataset to verify the model's generalization ability as well as robustness to other distributed datasets.
% We can observe that the MESN model shows strong competitiveness in terms of accuracy when compared to other trained models, which suggests that the model has a good generalization ability in the OOD task.

To validate the model's generalization ability and robustness to other distributed datasets, we directly test the model trained on the FI dataset, without training on Artphoto. 
% As observed in Table 3, compared to other models trained on Artphoto, we achieve highly competitive zero-shot performance, indicating that the model has good generalization ability in out-of-distribution tasks.
From Table~\ref{tab:cap3}, we can observe that compared with other models trained on Artphoto, we achieve competitive zero-shot performance, which shows that the model has good generalization ability in out-of-distribution tasks.


%%%%%%%%%%%%
%  E2E     %
%%%%%%%%%%%%


\section{Conclusion}
In this paper, we introduced Wi-Chat, the first LLM-powered Wi-Fi-based human activity recognition system that integrates the reasoning capabilities of large language models with the sensing potential of wireless signals. Our experimental results on a self-collected Wi-Fi CSI dataset demonstrate the promising potential of LLMs in enabling zero-shot Wi-Fi sensing. These findings suggest a new paradigm for human activity recognition that does not rely on extensive labeled data. We hope future research will build upon this direction, further exploring the applications of LLMs in signal processing domains such as IoT, mobile sensing, and radar-based systems.

\section*{Limitations}
While our work represents the first attempt to leverage LLMs for processing Wi-Fi signals, it is a preliminary study focused on a relatively simple task: Wi-Fi-based human activity recognition. This choice allows us to explore the feasibility of LLMs in wireless sensing but also comes with certain limitations.

Our approach primarily evaluates zero-shot performance, which, while promising, may still lag behind traditional supervised learning methods in highly complex or fine-grained recognition tasks. Besides, our study is limited to a controlled environment with a self-collected dataset, and the generalizability of LLMs to diverse real-world scenarios with varying Wi-Fi conditions, environmental interference, and device heterogeneity remains an open question.

Additionally, we have yet to explore the full potential of LLMs in more advanced Wi-Fi sensing applications, such as fine-grained gesture recognition, occupancy detection, and passive health monitoring. Future work should investigate the scalability of LLM-based approaches, their robustness to domain shifts, and their integration with multimodal sensing techniques in broader IoT applications.


% Bibliography entries for the entire Anthology, followed by custom entries
%\bibliography{anthology,custom}
% Custom bibliography entries only
\bibliography{main}
\newpage
\appendix

\section{Experiment prompts}
\label{sec:prompt}
The prompts used in the LLM experiments are shown in the following Table~\ref{tab:prompts}.

\definecolor{titlecolor}{rgb}{0.9, 0.5, 0.1}
\definecolor{anscolor}{rgb}{0.2, 0.5, 0.8}
\definecolor{labelcolor}{HTML}{48a07e}
\begin{table*}[h]
	\centering
	
 % \vspace{-0.2cm}
	
	\begin{center}
		\begin{tikzpicture}[
				chatbox_inner/.style={rectangle, rounded corners, opacity=0, text opacity=1, font=\sffamily\scriptsize, text width=5in, text height=9pt, inner xsep=6pt, inner ysep=6pt},
				chatbox_prompt_inner/.style={chatbox_inner, align=flush left, xshift=0pt, text height=11pt},
				chatbox_user_inner/.style={chatbox_inner, align=flush left, xshift=0pt},
				chatbox_gpt_inner/.style={chatbox_inner, align=flush left, xshift=0pt},
				chatbox/.style={chatbox_inner, draw=black!25, fill=gray!7, opacity=1, text opacity=0},
				chatbox_prompt/.style={chatbox, align=flush left, fill=gray!1.5, draw=black!30, text height=10pt},
				chatbox_user/.style={chatbox, align=flush left},
				chatbox_gpt/.style={chatbox, align=flush left},
				chatbox2/.style={chatbox_gpt, fill=green!25},
				chatbox3/.style={chatbox_gpt, fill=red!20, draw=black!20},
				chatbox4/.style={chatbox_gpt, fill=yellow!30},
				labelbox/.style={rectangle, rounded corners, draw=black!50, font=\sffamily\scriptsize\bfseries, fill=gray!5, inner sep=3pt},
			]
											
			\node[chatbox_user] (q1) {
				\textbf{System prompt}
				\newline
				\newline
				You are a helpful and precise assistant for segmenting and labeling sentences. We would like to request your help on curating a dataset for entity-level hallucination detection.
				\newline \newline
                We will give you a machine generated biography and a list of checked facts about the biography. Each fact consists of a sentence and a label (True/False). Please do the following process. First, breaking down the biography into words. Second, by referring to the provided list of facts, merging some broken down words in the previous step to form meaningful entities. For example, ``strategic thinking'' should be one entity instead of two. Third, according to the labels in the list of facts, labeling each entity as True or False. Specifically, for facts that share a similar sentence structure (\eg, \textit{``He was born on Mach 9, 1941.''} (\texttt{True}) and \textit{``He was born in Ramos Mejia.''} (\texttt{False})), please first assign labels to entities that differ across atomic facts. For example, first labeling ``Mach 9, 1941'' (\texttt{True}) and ``Ramos Mejia'' (\texttt{False}) in the above case. For those entities that are the same across atomic facts (\eg, ``was born'') or are neutral (\eg, ``he,'' ``in,'' and ``on''), please label them as \texttt{True}. For the cases that there is no atomic fact that shares the same sentence structure, please identify the most informative entities in the sentence and label them with the same label as the atomic fact while treating the rest of the entities as \texttt{True}. In the end, output the entities and labels in the following format:
                \begin{itemize}[nosep]
                    \item Entity 1 (Label 1)
                    \item Entity 2 (Label 2)
                    \item ...
                    \item Entity N (Label N)
                \end{itemize}
                % \newline \newline
                Here are two examples:
                \newline\newline
                \textbf{[Example 1]}
                \newline
                [The start of the biography]
                \newline
                \textcolor{titlecolor}{Marianne McAndrew is an American actress and singer, born on November 21, 1942, in Cleveland, Ohio. She began her acting career in the late 1960s, appearing in various television shows and films.}
                \newline
                [The end of the biography]
                \newline \newline
                [The start of the list of checked facts]
                \newline
                \textcolor{anscolor}{[Marianne McAndrew is an American. (False); Marianne McAndrew is an actress. (True); Marianne McAndrew is a singer. (False); Marianne McAndrew was born on November 21, 1942. (False); Marianne McAndrew was born in Cleveland, Ohio. (False); She began her acting career in the late 1960s. (True); She has appeared in various television shows. (True); She has appeared in various films. (True)]}
                \newline
                [The end of the list of checked facts]
                \newline \newline
                [The start of the ideal output]
                \newline
                \textcolor{labelcolor}{[Marianne McAndrew (True); is (True); an (True); American (False); actress (True); and (True); singer (False); , (True); born (True); on (True); November 21, 1942 (False); , (True); in (True); Cleveland, Ohio (False); . (True); She (True); began (True); her (True); acting career (True); in (True); the late 1960s (True); , (True); appearing (True); in (True); various (True); television shows (True); and (True); films (True); . (True)]}
                \newline
                [The end of the ideal output]
				\newline \newline
                \textbf{[Example 2]}
                \newline
                [The start of the biography]
                \newline
                \textcolor{titlecolor}{Doug Sheehan is an American actor who was born on April 27, 1949, in Santa Monica, California. He is best known for his roles in soap operas, including his portrayal of Joe Kelly on ``General Hospital'' and Ben Gibson on ``Knots Landing.''}
                \newline
                [The end of the biography]
                \newline \newline
                [The start of the list of checked facts]
                \newline
                \textcolor{anscolor}{[Doug Sheehan is an American. (True); Doug Sheehan is an actor. (True); Doug Sheehan was born on April 27, 1949. (True); Doug Sheehan was born in Santa Monica, California. (False); He is best known for his roles in soap operas. (True); He portrayed Joe Kelly. (True); Joe Kelly was in General Hospital. (True); General Hospital is a soap opera. (True); He portrayed Ben Gibson. (True); Ben Gibson was in Knots Landing. (True); Knots Landing is a soap opera. (True)]}
                \newline
                [The end of the list of checked facts]
                \newline \newline
                [The start of the ideal output]
                \newline
                \textcolor{labelcolor}{[Doug Sheehan (True); is (True); an (True); American (True); actor (True); who (True); was born (True); on (True); April 27, 1949 (True); in (True); Santa Monica, California (False); . (True); He (True); is (True); best known (True); for (True); his roles in soap operas (True); , (True); including (True); in (True); his portrayal (True); of (True); Joe Kelly (True); on (True); ``General Hospital'' (True); and (True); Ben Gibson (True); on (True); ``Knots Landing.'' (True)]}
                \newline
                [The end of the ideal output]
				\newline \newline
				\textbf{User prompt}
				\newline
				\newline
				[The start of the biography]
				\newline
				\textcolor{magenta}{\texttt{\{BIOGRAPHY\}}}
				\newline
				[The ebd of the biography]
				\newline \newline
				[The start of the list of checked facts]
				\newline
				\textcolor{magenta}{\texttt{\{LIST OF CHECKED FACTS\}}}
				\newline
				[The end of the list of checked facts]
			};
			\node[chatbox_user_inner] (q1_text) at (q1) {
				\textbf{System prompt}
				\newline
				\newline
				You are a helpful and precise assistant for segmenting and labeling sentences. We would like to request your help on curating a dataset for entity-level hallucination detection.
				\newline \newline
                We will give you a machine generated biography and a list of checked facts about the biography. Each fact consists of a sentence and a label (True/False). Please do the following process. First, breaking down the biography into words. Second, by referring to the provided list of facts, merging some broken down words in the previous step to form meaningful entities. For example, ``strategic thinking'' should be one entity instead of two. Third, according to the labels in the list of facts, labeling each entity as True or False. Specifically, for facts that share a similar sentence structure (\eg, \textit{``He was born on Mach 9, 1941.''} (\texttt{True}) and \textit{``He was born in Ramos Mejia.''} (\texttt{False})), please first assign labels to entities that differ across atomic facts. For example, first labeling ``Mach 9, 1941'' (\texttt{True}) and ``Ramos Mejia'' (\texttt{False}) in the above case. For those entities that are the same across atomic facts (\eg, ``was born'') or are neutral (\eg, ``he,'' ``in,'' and ``on''), please label them as \texttt{True}. For the cases that there is no atomic fact that shares the same sentence structure, please identify the most informative entities in the sentence and label them with the same label as the atomic fact while treating the rest of the entities as \texttt{True}. In the end, output the entities and labels in the following format:
                \begin{itemize}[nosep]
                    \item Entity 1 (Label 1)
                    \item Entity 2 (Label 2)
                    \item ...
                    \item Entity N (Label N)
                \end{itemize}
                % \newline \newline
                Here are two examples:
                \newline\newline
                \textbf{[Example 1]}
                \newline
                [The start of the biography]
                \newline
                \textcolor{titlecolor}{Marianne McAndrew is an American actress and singer, born on November 21, 1942, in Cleveland, Ohio. She began her acting career in the late 1960s, appearing in various television shows and films.}
                \newline
                [The end of the biography]
                \newline \newline
                [The start of the list of checked facts]
                \newline
                \textcolor{anscolor}{[Marianne McAndrew is an American. (False); Marianne McAndrew is an actress. (True); Marianne McAndrew is a singer. (False); Marianne McAndrew was born on November 21, 1942. (False); Marianne McAndrew was born in Cleveland, Ohio. (False); She began her acting career in the late 1960s. (True); She has appeared in various television shows. (True); She has appeared in various films. (True)]}
                \newline
                [The end of the list of checked facts]
                \newline \newline
                [The start of the ideal output]
                \newline
                \textcolor{labelcolor}{[Marianne McAndrew (True); is (True); an (True); American (False); actress (True); and (True); singer (False); , (True); born (True); on (True); November 21, 1942 (False); , (True); in (True); Cleveland, Ohio (False); . (True); She (True); began (True); her (True); acting career (True); in (True); the late 1960s (True); , (True); appearing (True); in (True); various (True); television shows (True); and (True); films (True); . (True)]}
                \newline
                [The end of the ideal output]
				\newline \newline
                \textbf{[Example 2]}
                \newline
                [The start of the biography]
                \newline
                \textcolor{titlecolor}{Doug Sheehan is an American actor who was born on April 27, 1949, in Santa Monica, California. He is best known for his roles in soap operas, including his portrayal of Joe Kelly on ``General Hospital'' and Ben Gibson on ``Knots Landing.''}
                \newline
                [The end of the biography]
                \newline \newline
                [The start of the list of checked facts]
                \newline
                \textcolor{anscolor}{[Doug Sheehan is an American. (True); Doug Sheehan is an actor. (True); Doug Sheehan was born on April 27, 1949. (True); Doug Sheehan was born in Santa Monica, California. (False); He is best known for his roles in soap operas. (True); He portrayed Joe Kelly. (True); Joe Kelly was in General Hospital. (True); General Hospital is a soap opera. (True); He portrayed Ben Gibson. (True); Ben Gibson was in Knots Landing. (True); Knots Landing is a soap opera. (True)]}
                \newline
                [The end of the list of checked facts]
                \newline \newline
                [The start of the ideal output]
                \newline
                \textcolor{labelcolor}{[Doug Sheehan (True); is (True); an (True); American (True); actor (True); who (True); was born (True); on (True); April 27, 1949 (True); in (True); Santa Monica, California (False); . (True); He (True); is (True); best known (True); for (True); his roles in soap operas (True); , (True); including (True); in (True); his portrayal (True); of (True); Joe Kelly (True); on (True); ``General Hospital'' (True); and (True); Ben Gibson (True); on (True); ``Knots Landing.'' (True)]}
                \newline
                [The end of the ideal output]
				\newline \newline
				\textbf{User prompt}
				\newline
				\newline
				[The start of the biography]
				\newline
				\textcolor{magenta}{\texttt{\{BIOGRAPHY\}}}
				\newline
				[The ebd of the biography]
				\newline \newline
				[The start of the list of checked facts]
				\newline
				\textcolor{magenta}{\texttt{\{LIST OF CHECKED FACTS\}}}
				\newline
				[The end of the list of checked facts]
			};
		\end{tikzpicture}
        \caption{GPT-4o prompt for labeling hallucinated entities.}\label{tb:gpt-4-prompt}
	\end{center}
\vspace{-0cm}
\end{table*}
% \section{Full Experiment Results}
% \begin{table*}[th]
    \centering
    \small
    \caption{Classification Results}
    \begin{tabular}{lcccc}
        \toprule
        \textbf{Method} & \textbf{Accuracy} & \textbf{Precision} & \textbf{Recall} & \textbf{F1-score} \\
        \midrule
        \multicolumn{5}{c}{\textbf{Zero Shot}} \\
                Zero-shot E-eyes & 0.26 & 0.26 & 0.27 & 0.26 \\
        Zero-shot CARM & 0.24 & 0.24 & 0.24 & 0.24 \\
                Zero-shot SVM & 0.27 & 0.28 & 0.28 & 0.27 \\
        Zero-shot CNN & 0.23 & 0.24 & 0.23 & 0.23 \\
        Zero-shot RNN & 0.26 & 0.26 & 0.26 & 0.26 \\
DeepSeek-0shot & 0.54 & 0.61 & 0.54 & 0.52 \\
DeepSeek-0shot-COT & 0.33 & 0.24 & 0.33 & 0.23 \\
DeepSeek-0shot-Knowledge & 0.45 & 0.46 & 0.45 & 0.44 \\
Gemma2-0shot & 0.35 & 0.22 & 0.38 & 0.27 \\
Gemma2-0shot-COT & 0.36 & 0.22 & 0.36 & 0.27 \\
Gemma2-0shot-Knowledge & 0.32 & 0.18 & 0.34 & 0.20 \\
GPT-4o-mini-0shot & 0.48 & 0.53 & 0.48 & 0.41 \\
GPT-4o-mini-0shot-COT & 0.33 & 0.50 & 0.33 & 0.38 \\
GPT-4o-mini-0shot-Knowledge & 0.49 & 0.31 & 0.49 & 0.36 \\
GPT-4o-0shot & 0.62 & 0.62 & 0.47 & 0.42 \\
GPT-4o-0shot-COT & 0.29 & 0.45 & 0.29 & 0.21 \\
GPT-4o-0shot-Knowledge & 0.44 & 0.52 & 0.44 & 0.39 \\
LLaMA-0shot & 0.32 & 0.25 & 0.32 & 0.24 \\
LLaMA-0shot-COT & 0.12 & 0.25 & 0.12 & 0.09 \\
LLaMA-0shot-Knowledge & 0.32 & 0.25 & 0.32 & 0.28 \\
Mistral-0shot & 0.19 & 0.23 & 0.19 & 0.10 \\
Mistral-0shot-Knowledge & 0.21 & 0.40 & 0.21 & 0.11 \\
        \midrule
        \multicolumn{5}{c}{\textbf{4 Shot}} \\
GPT-4o-mini-4shot & 0.58 & 0.59 & 0.58 & 0.53 \\
GPT-4o-mini-4shot-COT & 0.57 & 0.53 & 0.57 & 0.50 \\
GPT-4o-mini-4shot-Knowledge & 0.56 & 0.51 & 0.56 & 0.47 \\
GPT-4o-4shot & 0.77 & 0.84 & 0.77 & 0.73 \\
GPT-4o-4shot-COT & 0.63 & 0.76 & 0.63 & 0.53 \\
GPT-4o-4shot-Knowledge & 0.72 & 0.82 & 0.71 & 0.66 \\
LLaMA-4shot & 0.29 & 0.24 & 0.29 & 0.21 \\
LLaMA-4shot-COT & 0.20 & 0.30 & 0.20 & 0.13 \\
LLaMA-4shot-Knowledge & 0.15 & 0.23 & 0.13 & 0.13 \\
Mistral-4shot & 0.02 & 0.02 & 0.02 & 0.02 \\
Mistral-4shot-Knowledge & 0.21 & 0.27 & 0.21 & 0.20 \\
        \midrule
        
        \multicolumn{5}{c}{\textbf{Suprevised}} \\
        SVM & 0.94 & 0.92 & 0.91 & 0.91 \\
        CNN & 0.98 & 0.98 & 0.97 & 0.97 \\
        RNN & 0.99 & 0.99 & 0.99 & 0.99 \\
        % \midrule
        % \multicolumn{5}{c}{\textbf{Conventional Wi-Fi-based Human Activity Recognition Systems}} \\
        E-eyes & 1.00 & 1.00 & 1.00 & 1.00 \\
        CARM & 0.98 & 0.98 & 0.98 & 0.98 \\
\midrule
 \multicolumn{5}{c}{\textbf{Vision Models}} \\
           Zero-shot SVM & 0.26 & 0.25 & 0.25 & 0.25 \\
        Zero-shot CNN & 0.26 & 0.25 & 0.26 & 0.26 \\
        Zero-shot RNN & 0.28 & 0.28 & 0.29 & 0.28 \\
        SVM & 0.99 & 0.99 & 0.99 & 0.99 \\
        CNN & 0.98 & 0.99 & 0.98 & 0.98 \\
        RNN & 0.98 & 0.99 & 0.98 & 0.98 \\
GPT-4o-mini-Vision & 0.84 & 0.85 & 0.84 & 0.84 \\
GPT-4o-mini-Vision-COT & 0.90 & 0.91 & 0.90 & 0.90 \\
GPT-4o-Vision & 0.74 & 0.82 & 0.74 & 0.73 \\
GPT-4o-Vision-COT & 0.70 & 0.83 & 0.70 & 0.68 \\
LLaMA-Vision & 0.20 & 0.23 & 0.20 & 0.09 \\
LLaMA-Vision-Knowledge & 0.22 & 0.05 & 0.22 & 0.08 \\

        \bottomrule
    \end{tabular}
    \label{full}
\end{table*}




\end{document}


\newpage

\appendix

\section{Model Structure Details}\label{appendix-model}

This section provides more details of the model structure, mainly about the decoder and the linear attention module, as shown in Fig. \ref{linear attention}. At each step $t$, the model takes the encoding of the first node, the current node, and the remaining unselected nodes for decoding. Denote the first and current nodes’ embeddings $h_f^{(0)}$ and $h_c^{(0)}$, respectively. We first augment them by $r_f$ and $r_c$ channels to obtain the virtual representative nodes: 
\begin{equation}
\Tilde{H}^0 = [reshape(h_f W_f), reshape(h_c W_c)],
\end{equation}
where $[\cdot, \cdot·]$ is the horizontal concatenation operator, $W_f \in \mathbb{R}^{d\times(d\times r_f)}$, $W_c \in \mathbb{R}^{d\times(d\times r_c)}$. Here the $reshape$ is to keep the embedding dimension of the representative node the same with the remaining unselected nodes, resulting $\Tilde{H}^0 \in \mathbb{R}^{(r_f + r_c)\times d}$, which means the number of virtual representative nodes equals the number of the channels. 

Then, we have $r = r_f + r_c $ virtual representative nodes embeddings $\Tilde{H}^{(0)} = \{h_j^{(0)} | j=1,2, ...,r\}$. Denote the remaining unselected nodes as $H_a^{(0)} = \{h_i^{(0)} | i = 1,2, ..., m-t\}$ at step $t$ for the hyper-graph of size $m$, the linear attention module first aggregates all information into representative nodes, then broadcasts the information to all nodes. 

The aggregating and broadcasting processes are realized by two different attention layers. Recall the formulation of classical attention mechanism: note the queries $X_Q \in \mathbb{R}^{q\times d} $, keys $X_K \in \mathbb{R}^{k\times d} $ and values $X_V \in \mathbb{R}^{v\times d}$ as inputs, the classical attention mechanism can be formulated as: 
\begin{equation}
\begin{split}
Attn&(X_Q, X_K, X_V) = \\
 &softmax(\frac{X_Q W_Q(X_K W_K)^\top}{\sqrt{d}})X_V W_V,
\end{split}
\end{equation}


Then, for the $l-th$ linear attention module, denote the input representative nodes' embeddings $\Tilde{H}^{(l-1)}$ and the remaining unvisited nodes’ embeddings $H_a^{(l-1)} = \{h_i^l, i=1, . . . , m\}$, we concat $\Tilde{H}^{(l-1)}$ and $H_a^{(l-1)}$ as $H_{all}^{(l-1)} = [\Tilde{H}^{(l-1)}, H_a^{(l-1)}]$. Then, the aggregating attention layer attends representative nodes to all nodes: 
\begin{equation}
\begin{split}
    Agg = Attn( & \Tilde{H}^{(l-1)}, H_{all}^{(l-1)}, H_{all}^{(l-1)}),
\end{split}
\end{equation}
and the broadcasting attention layer attends all nodes to the aggregations: 
\begin{equation}
Brd = Attn(H_{all}^{(l-1)}, Agg, Agg),
\end{equation}
where $Brd \in \mathbb{R}^{(r+m-t)\times d}$, which we can split into the representative nodes' embeddings $\Tilde{H}^{(l)}$ and the remaining unselected nodes’ embeddings $H_a^{(l)}$. 

Finally, after $L$ linear attention modules, we obtain the hidden representation $\Tilde{H}^{(L)}$ and and $H_a^{(L)}$. Then we take only the embeddings of remaining unselected nodes $H_a^{(L)}$ to calculate the probability for selecting the next node.

\begin{figure}[htbp]
\centering
\includegraphics[width=0.35\textwidth]{graph/linear_attention.pdf} 
\caption{Linear attention module}
\label{linear attention}
\end{figure}


\section{Sample Size Alignment}\label{appendix-alignement}


We employ the sample size alignment to make the hyper-graph size the same within a batch. The key concept is to precompute the size of the hyper-graph before actually implementing different cases of destruction. This process requires a predefined order of node destruction. In our clustering scenario, we begin with a central node, with nodes closer to the center being destroyed earlier. When an additional node is destroyed, the number of nodes that newly appear in the hyper-graph depends on whether the edges connecting the node to its neighbors (first-order and second-order) have already been destroyed. Fig. \ref{node_emerge} provides an example where the node targeted for destruction is connected with its four neighbors. After the node is destroyed, three new nodes emerge in the hyper-graph. We detail all cases of node connections and their corresponding results of node emergence when enlarging the destruction area in Fig. \ref{case-destruction}. 

\begin{figure}[htbp]
\centering
\includegraphics[width=0.4\textwidth]{graph/node_emerge.pdf} 
\caption{Node emergence in hyper-graph following additional node destruction}
\label{node_emerge}
\end{figure}

\begin{figure}[h]
\centering
\includegraphics[width=0.45\textwidth]{graph/case-destruction.pdf} 
\caption{Different cases of node connections and corresponding results of newly appearing nodes when enlarging the destruction area}
\label{case-destruction}
\end{figure}

There are six different cases and we indicate the newly appearing nodes by red color in each case. In case 1, destroying one more node generates two endpoint nodes and an isolated node. In case 2, that generates an endpoint node and an isolated node. In case 3, the newly destroyed node changes from a middle node to an isolated node. In case 4, the newly destroyed node becomes an endpoint node. In cases 5, the newly destroyed node is an endpoint node before and becomes an isolated nodes, but does not change the hyper-graph size. In case 6, the newly destroyed node has already been disconnected with its two neighbors before, therefore none of new nodes emerging in the hyper-graph due to the destruction. In summary, the number of newly appearing nodes equals to the number of relevant undestroyed edges minus one, except the case 6 where all edges have already been destroyed. 

We detail the algorithm of sample size alignment in Algorithm \ref{alignment}. First, we calculate the distance of all nodes to the center node and sort these node by the distance in ascending order. Then, for the node to destroy, if its neighbor is nearer to the center node, the neighbor will be destroyed before the node and the edge between them will be disconnect. Line 5 in the algorithm details the conditions where the node is still connected with its neighbors. After that, we can calculate the number of emerging nodes as line 6. Finally, by summing all emerging nodes in order, we can determine the hyper-graph size corresponding to a specific destruction scenario, and obtain the mask indicating which node should be destroyed for a target hyper-graph size.





\begin{algorithm}[]
    \caption{Sample Size Alignment}
	\begin{algorithmic}[1]

         \STATE {\bfseries Input:} 
         \newline The coordinates of a batch of VRP instances $X$,
         \newline the initial solution $S$, 
         \newline the cluster center $c=(x_c, y_c)$, 
         \newline the target reduced hyper-graph size $k$; 
         \STATE {\bfseries Notation:} 
         \newline $cumsum()$: the function to calculate the cumulative sum of a tensor along a dimension,
         \newline  $dist()$: the function to calculate the distance between two nodes,
         \newline $sort()$: sort a sequence of numbers in ascending order;
	    \STATE {\bfseries Output:} The mask $M$ indicating the node to destroy; 
        \STATE $D=dist(X, c)$, $\pi = sort(D)$; 
        \STATE For all nodes and their first-order neighbor $1A, 1B$ and the second-order neighbors $2A, 2B$ in the solutions, compare the distances to determine if they are connected with their neighbors before they are destroyed: 
        \newline \hspace*{3mm} $connect_{1A} = (D_{1A} \textgreater D)$,
        \newline \hspace*{3mm} $connect_{1B} = (D_{1B} \textgreater D)$,
        \newline \hspace*{3mm} $connect_{2A} = (D_{2A} \textgreater D)\ AND \ connect_{1A} $, 
        \newline \hspace*{3mm} $connect_{2B} = (D_{2B} \textgreater D)\ AND \ connect_{1B} $;
        \STATE Calculate the number of newly appearing nodes as illustrated in Fig. \ref{case-destruction}: 
        \newline \hspace*{3mm} $N = max(0, sum(connect_{1A} + connect_{1B} + connect_{2A}) + connect_{2B}) - 1)$; 
	    \STATE $H=cumsum(N)$ in the order of $\pi$; 
        \STATE $M = (H \leq k)$.
	\end{algorithmic}
	\label{alignment}
\end{algorithm}






\section{Additional Results of TSPLib}\label{appendix-tsplib}

The detailed results of TSPLib are shown in Table \ref{Detailed Tsplib result} and Table \ref{Detailed-TSPLib-continue}. Note that in two instances, rl11849 and usa13509, the LEHD \cite{luo2023lehd} needs about 80 hours and 120 hours to perform 1000 reconstruction steps. We stop the iteration at 24 hours as we observe that the LEHD has not made any progress for a long time. In most instances, our DRHG achieves the lowest optimality gap. The advantage of DRHG becomes more pronounced in hard instances of larger size or special distribution. 

\paragraph{large-size problem} In large-size problem, POMO \cite{kwon2020pomo} suffers the most from poor generalization ability, and its performance drops dramatically. BQ \cite{drakulic2023bq} and LEHD \cite{luo2023lehd} generalize better but still struggle on cases with thousands of nodes. When the problem size grows beyond 3,000, BQ fails to solve the problem due to out-of-memory, so to LEHD when the problem size comes to about 15,000. The DRHG succeeds in solving all instances and obtains solutions with optimality gap much lower than the others.

\paragraph{special distribution} Excepting the instances consisting of the real-world cities, the TSPlib also incorporates some instances of special distributions, such as the drilling problems (starting with 'd', 'pcb' and 'u'), and the rattled-grid problems (strat with 'rat'). On these distributions, DRHG also outperforms the other methods. 

% These results demonstrate that DRHG is robust to scale change and distribution shift.




\begin{table}[htbp]
  \centering
  \resizebox{0.9\columnwidth}{!}{

\begin{tabular}{ccccc}
\toprule
\textbf{Case} & \textbf{POMO-augx8} & \textbf{BQ-bs16} & \textbf{LEHD-RRC100} & \textbf{DRHG-T=1000} \\
\midrule
a280  & 12.62\% & 0.39\% & \textbf{0.30\%} & 0.34\% \\
\midrule
berlin52 & 0.04\% & \textbf{0.03\%} & \textbf{0.03\%} & \textbf{0.03\%} \\
\midrule
bier127 & 12.00\% & 0.68\% & \textbf{0.01\%} & \textbf{0.01\%} \\
\midrule
brd14051 & OOM   & OOM   & OOM   & \textbf{4.02\%} \\
\midrule
ch130 & 0.16\% & 0.13\% & \textbf{0.01\%} & \textbf{0.01\%} \\
\midrule
ch150 & 0.53\% & 0.39\% & \textbf{0.04\%} & \textbf{0.04\%} \\
\midrule
d1291 & 77.24\% & 5.97\% & 2.71\% & \textbf{2.09\%} \\
\midrule
d15112 & OOM   & OOM   & OOM   & \textbf{3.41\%} \\
\midrule
d1655 & 80.99\% & 9.67\% & 5.16\% & \textbf{1.57\%} \\
\midrule
d18512 & OOM   & OOM   & OOM   & \textbf{3.63\%} \\
\midrule
d198  & 19.89\% & 8.77\% & 0.71\% & \textbf{0.26\%} \\
\midrule
d2103 & 75.22\% & 15.36\% & \textbf{1.22\%} & 1.82\% \\
\midrule
d493  & 58.91\% & 8.40\% & 0.92\% & \textbf{0.31\%} \\
\midrule
d657  & 41.14\% & 1.34\% & 0.91\% & \textbf{0.21\%} \\
\midrule
eil101 & 1.84\% & 1.78\% & 1.78\% & \textbf{1.78\%} \\
\midrule
eil51 & 0.83\% & \textbf{0.67\%} & \textbf{0.67\%} & \textbf{0.67\%} \\
\midrule
eil76 & \textbf{1.18\%} & 1.24\% & 1.18\% & \textbf{1.18\%} \\
\midrule
fl1400 & 47.36\% & 11.60\% & 3.45\% & \textbf{1.43\%} \\
\midrule
fl1577 & 71.17\% & 14.63\% & 3.71\% & \textbf{3.08\%} \\
\midrule
fl3795 & 126.86\% & OOM   & 7.96\% & \textbf{4.61\%} \\
\midrule
fl417 & 18.51\% & 5.11\% & 2.87\% & \textbf{0.49\%} \\
\midrule
fnl4461 & OOM   & OOM   & 12.38\% & \textbf{1.20\%} \\
\midrule
gil262 & 2.99\% & 0.72\% & \textbf{0.33\%} & \textbf{0.33\%} \\
\midrule
kroA100 & 1.58\% & 0.02\% & \textbf{0.02\%} & \textbf{0.02\%} \\
\midrule
kroA150 & 1.01\% & 0.01\% & \textbf{0.00\%} & \textbf{0.00\%} \\
\midrule
kroA200 & 2.93\% & 0.50\% & \textbf{0.00\%} & \textbf{0.00\%} \\
\midrule
kroB100 & 0.93\% & 0.01\% & \textbf{-0.01\%} & \textbf{-0.01\%} \\
\midrule
kroB150 & 2.10\% & \textbf{-0.01\%} & \textbf{-0.01\%} & \textbf{-0.01\%} \\
\midrule
kroB200 & 2.04\% & 0.22\% & \textbf{0.01\%} & \textbf{0.01\%} \\
\midrule
kroC100 & 0.20\% & 0.01\% & \textbf{0.01\%} & \textbf{0.01\%} \\
\midrule
kroD100 & 0.80\% & 0.00\% & \textbf{0.00\%} & \textbf{0.00\%} \\
\midrule
kroE100 & 1.31\% & 0.07\% & \textbf{0.00\%} & 0.17\% \\
\midrule
lin105 & 1.31\% & 0.03\% & \textbf{0.03\%} & \textbf{0.03\%} \\
\midrule
lin318 & 10.29\% & 0.35\% & \textbf{0.03\%} & 0.30\% \\
\midrule
linhp318 & 12.11\% & 2.01\% & 1.74\% & \textbf{1.69\%} \\
\midrule
nrw1379 & 41.52\% & 3.34\% & 8.78\% & \textbf{1.41\%} \\
\midrule
p654  & 25.58\% & 4.44\% & 2.00\% & \textbf{0.03\%} \\
\midrule
pcb1173 & 45.85\% & 3.95\% & 3.40\% & \textbf{0.39\%} \\
\midrule
pcb3038 & 63.82\% & OOM   & 7.23\% & \textbf{1.01\%} \\
\midrule
pcb442 & 18.64\% & 0.95\% & \textbf{0.04\%} & 0.27\% \\
\midrule
pr1002 & 43.93\% & 2.94\% & 0.77\% & \textbf{0.67\%} \\
\midrule
pr107 & 0.90\% & 13.94\% & \textbf{0.00\%} & \textbf{0.00\%} \\
\midrule
pr124 & 0.37\% & 0.08\% & \textbf{0.00\%} & \textbf{0.00\%} \\
\midrule
pr136 & 0.87\% & \textbf{0.00\%} & \textbf{0.00\%} & \textbf{0.00\%} \\

\midrule
pr144 & 1.40\% & 0.19\% & 0.09\% & \textbf{0.00\%} \\
\midrule
pr152 & 0.99\% & 8.21\% & 0.27\% & \textbf{0.19\%} \\
\midrule
pr226 & 4.46\% & 0.13\% & \textbf{0.01\%} & \textbf{0.01\%} \\
\midrule
pr2392 & 69.78\% & 7.72\% & 5.31\% & \textbf{0.56\%} \\
\midrule
pr264 & 13.72\% & 0.27\% & \textbf{0.01\%} & \textbf{0.01\%} \\
\midrule
pr299 & 14.71\% & 1.62\% & 0.10\% & \textbf{0.02\%} \\
\midrule
pr439 & 21.55\% & 2.01\% & 0.33\% & \textbf{0.12\%} \\
\midrule
pr76  & 0.14\% & \textbf{0.00\%} & \textbf{0.00\%} & \textbf{0.00\%} \\
\midrule
rat195 & 8.15\% & 0.60\% & 0.61\% & \textbf{0.57\%} \\
\midrule
rat575 & 25.52\% & 0.84\% & 1.01\% & \textbf{0.36\%} \\
\midrule
rat783 & 33.54\% & 2.91\% & 1.28\% & \textbf{0.47\%} \\
\midrule
rat99 & 1.90\% & \textbf{0.68\%} & \textbf{0.68\%} & \textbf{0.68\%} \\
\midrule
rd100 & 0.01\% & \textbf{0.01\%} & \textbf{0.01\%} & \textbf{0.01\%} \\
\midrule
rd400 & 13.97\% & 0.32\% & \textbf{0.02\%} & 0.36\% \\

\bottomrule
\end{tabular}%
   }
    \caption{Detailed results of TSPLib}
  \label{Detailed Tsplib result}%
\end{table}%



\begin{table}[htbp]
  \centering
  \resizebox{0.9\columnwidth}{!}{
\begin{tabular}{ccccc}
\toprule
\textbf{Case} & \textbf{POMO-augx8} & \textbf{BQ-bs16} & \textbf{LEHD-RRC100} & \textbf{DRHG-T=1000} \\



\midrule
rl11849 & OOM   & OOM   & 21.43\% & \textbf{3.94\%} \\
\midrule
rl1304 & 67.70\% & 5.07\% & 1.96\% & \textbf{0.79\%} \\
\midrule
rl1323 & 68.69\% & 4.41\% & 1.71\% & \textbf{1.26\%} \\
\midrule
rl1889 & 80.00\% & 7.90\% & 2.90\% & \textbf{0.95\%} \\
\midrule
rl5915 & OOM   & OOM   & 11.21\% & \textbf{1.97\%} \\
\midrule
rl5934 & OOM   & OOM   & 11.11\% & \textbf{2.69\%} \\
\midrule
st70  & \textbf{0.31\%} & \textbf{0.31\%} & \textbf{0.31\%} & \textbf{0.31\%} \\
\midrule
ts225 & 4.72\% & \textbf{0.00\%} & \textbf{0.00\%} & \textbf{0.00\%} \\
\midrule
tsp225 & 6.72\% & -0.43\% & \textbf{-1.46\%} & \textbf{-1.46\%} \\
\midrule
u1060 & 53.50\% & 7.04\% & 2.80\% & \textbf{0.48\%} \\
\midrule
u1432 & 38.48\% & 2.70\% & 1.92\% & \textbf{0.49\%} \\
\midrule
u159  & 0.95\% & \textbf{-0.01\%} & \textbf{-0.01\%} & \textbf{-0.01\%} \\
\midrule
u1817 & 70.51\% & 6.12\% & 4.15\% & \textbf{2.05\%} \\
\midrule
u2152 & 74.08\% & 5.20\% & 4.90\% & \textbf{2.24\%} \\
\midrule
u2319 & 26.43\% & 1.33\% & 1.99\% & \textbf{0.30\%} \\
\midrule
u574  & 30.83\% & 2.09\% & 0.69\% & \textbf{0.24\%} \\
\midrule
u724  & 31.66\% & 1.57\% & 0.76\% & \textbf{0.27\%} \\
\midrule
usa13509 & OOM   & OOM   & 34.65\% & \textbf{11.82\%} \\
\midrule
vm1084 & 48.15\% & 5.93\% & 2.17\% & \textbf{0.14\%} \\
\midrule
vm1748 & 62.05\% & 6.04\% & 2.61\% & \textbf{0.54\%} \\
\bottomrule
\end{tabular}%

   }
    \caption{Detailed results of TSPLib (continued)}
  \label{Detailed-TSPLib-continue}%
\end{table}%

\begin{figure*}[!h]
\centering
\includegraphics[width=0.95\textwidth]{graph/DR_demo-2.pdf} % 
\caption{The demonstration of the destroy-and-repair process of a TSP instance}
\label{fig-DR_demo}
\end{figure*}


\section{Influence of Hyper-parameters}\label{appendix-hyper-param}
\paragraph{Effect of training sample size} The size of the training sample influences model performance by altering the distribution. A sample size that is too small may oversimplify the task, whereas a sample size that is too large may result in insufficient undestroyed segments for the model to effectively learn the repair process. We investigate this effect by training the model on TSP100, keeping all other settings constant. Table \ref{effect_train_size} shows the results of three settings where the sample size $\in [30, 70],\ [20, 80]$ and $[10, 90]$. The case where training sample size $\in [20, 80]$ performs the best.
\paragraph{Effect of destruction degree in the inference} Table \ref{destruction_size} illustrates the impact of destruction degree during inference on performance. A higher degree of destruction can be more efficient than a lower one, provided that the repair quality remains consistent. However, since the model is trained with no more than 100 nodes, repair quality diminishes when the destruction becomes too extensive. Destroying $k$ nodes with $k \in [20, 200]$ yields the best overall results.




\begin{table}[htbp]
  \centering
  \resizebox{0.8\columnwidth}{!}{

        \begin{tabular}{l|c|c|c}
        \toprule
        \multicolumn{1}{c|}{\multirow{2}[4]{*}{Gap}} & \multicolumn{3}{c}{Training sample size} \\
        \cmidrule{2-4}      & [30, 70] & [20, 80] & [10, 90] \\
        \midrule
        TSP100 & 0.0008\% & 0.0005\% & \textbf{0.0004\%} \\
        TSP200 & 0.014\% & \textbf{0.0098\%} & 0.0149\% \\
        TSP500 & 0.127\% & \textbf{0.113\%} & 0.121\% \\
        TSP1K & 0.268\% & \textbf{0.258\%} & 0.271\% \\
        TSP5K  & 1.47\% & \textbf{1.42\%} & 1.63\% \\
        TSP10K & 3.06\% & \textbf{2.85\%} & 3.384\% \\
        \bottomrule
        \end{tabular}%
   }
    \caption{Effect of training sample size}
  \label{effect_train_size}%
\end{table}%



\begin{table}[htbp]
  \centering
  \resizebox{0.9\columnwidth}{!}{
    \begin{tabular}{l|c|c|c|c}
    \toprule
    \multicolumn{1}{c|}{\multirow{2}[4]{*}{Gap}} & \multicolumn{4}{c}{Destruction size in the inference} \\
    \cmidrule{2-5}      & [20, 100] & [20, 200] & [20, 500] & [20,1000] \\
    \midrule
    TSP100 & 0.0005\% &       &       &  \\
    TSP200 & 0.0567\% & \textbf{0.0098\%} &       &  \\
    TSP500 & 0.239\% & 0.113\% & \textbf{0.111\%} &  \\
    TSP1K & 0.393\% & \textbf{0.258\%} & 0.309\% & 0.449\% \\
    TSP5K  & 1.70\% & \textbf{1.42\%} & 1.67\% & 2.05\% \\
    TSP10K & 3.34\% & \textbf{2.85\%} & 3.07\% & 3.46\% \\
    \bottomrule
    \end{tabular}%
   }
    \caption{Effect of destruction degree in the inference}
  \label{destruction_size}%
\end{table}%


\section{Destroy-and-repair Demonstration}\label{appendix-demo}



Fig. \ref{fig-DR_demo} demonstrates the destroy-and-repair of a TSP instance with $n=100$. Three segments are left after the destruction. The model changes how these segments are connected during the repair and makes the contour of the solution apparently different.


\end{document}


