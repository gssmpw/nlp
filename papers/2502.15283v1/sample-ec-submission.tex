\documentclass[format=acmsmall, review=false]{acmart}
\usepackage{acm-ec-25}
\usepackage{booktabs} % For formal tables
\usepackage[ruled]{algorithm2e} % For algorithms
\renewcommand{\algorithmcfname}{ALGORITHM}
\SetAlFnt{\small}
\SetAlCapFnt{\small}
\SetAlCapNameFnt{\small}
\SetAlCapHSkip{0pt}
\IncMargin{-\parindent}

\usepackage{threeparttable}
\usepackage{tikz}
\usetikzlibrary{shapes}
\usetikzlibrary{calc}
\usetikzlibrary{decorations.pathreplacing}

% \settopmatter{printacmref=true}
%\setcitestyle{acmnumeric}
\setcitestyle{authoryear}

% \begin{CCSXML}
% <ccs2012>
%    <concept>
%        <concept_id>10003752.10010070.10010099.10010107</concept_id>
%        <concept_desc>Theory of computation~Computational pricing and auctions</concept_desc>
%        <concept_significance>500</concept_significance>
%        </concept>
%    <concept>
%        <concept_id>10010147.10010257.10010293.10010294</concept_id>
%        <concept_desc>Computing methodologies~Neural networks</concept_desc>
%        <concept_significance>500</concept_significance>
%        </concept>
%    <concept>
%        <concept_id>10010147.10010178.10010219.10010220</concept_id>
%        <concept_desc>Computing methodologies~Multi-agent systems</concept_desc>
%        <concept_significance>500</concept_significance>
%        </concept>
%    <concept>
%        <concept_id>10010405.10010455.10010460</concept_id>
%        <concept_desc>Applied computing~Economics</concept_desc>
%        <concept_significance>500</concept_significance>
%        </concept>
%  </ccs2012>
% \end{CCSXML}

% \ccsdesc[500]{Theory of computation~Computational pricing and auctions}
% \ccsdesc[500]{Computing methodologies~Neural networks}
% \ccsdesc[500]{Computing methodologies~Multi-agent systems}
% \ccsdesc[500]{Applied computing~Economics}

% \keywords{mechanism design, ordinary differential equation, deep learning, diffusion models, flow matching, optimal auction design, strategy-proofness, microeconomics, combinatorial auction.}

% Title. Note the optional short title for running heads. In the interest of anonymization, please do not include any acknowledgements.
%dcp suggesting new title
%\title[General Menu-Based Deep Auctions]{General Strategy-Proof Mechanisms Through Deep Learning}
\title[Menu-Based Combinatorial Auctions]{\name: Deep Menus for Combinatorial Auctions by Diffusion-Based Optimization}

% Anonymized submission.
\author{Tonghan Wang}\email{twang1@g.harvard.edu}

\affiliation{
  \institution{Harvard University}
  \city{Cambridge}
  \state{MA}
  \postcode{02138}
  \country{USA}
}

\author{Yanchen Jiang}\email{}

\affiliation{
  \institution{Harvard University}
  \city{Cambridge}
  \state{MA}
  \postcode{02138}
  \country{USA}
}

\author{David C. Parkes}\email{}

\affiliation{
  \institution{Harvard University}
  \city{Cambridge}
  \state{MA}
  \postcode{02138}
  \country{USA}
}
% \author{Submission 812}

% Abstract. Note that this must come before \maketitle.
\begin{abstract}
% \dcp{I dropped mention of DSIC in the abstract because for single bidder it's not such a  big deal to achieve.}
%dcp cut Combinatorial auctions (CAs) are important but challenging to design. Even for the single-bidder case, the design of revenue-maximizing CAs remains largely elusive. 
%difficulties  familiar from theoretical work---the 
%dcp cut, to unambiguously determine the utility to a bidder,
%. This method draws inspirations from generative models, especially
%dcp suggest cut The bundle distribution can be reconstructed by transforming the probability density of the initial distribution by the Liouville equation. 
Differentiable economics---the use of deep learning for auction design---has driven progress in the automated design of multi-item auctions with additive or unit-demand valuations. However, little progress has been made for optimal combinatorial auctions (CAs), even for the single bidder case, because we need to overcome the challenge of the bundle space growing exponentially with the number of items. For example, when learning a menu of allocation-price choices for a bidder in a CA, each menu element needs to efficiently and flexibly specify a probability distribution on bundles. In this paper, we solve this problem in the single-bidder CA setting by generating a bundle distribution through an ordinary differential equation (ODE) applied to a tractable initial distribution, drawing inspiration from generative models, especially score-based diffusion models and continuous normalizing flow. Our method, \name, uses deep learning to find suitable ODE-based transforms of initial distributions, one transform for each menu element, so that the overall menu achieves high expected revenue.  Our method achieves 1.11$-$2.23$\times$ higher revenue compared with automated mechanism design baselines on the single-bidder version of CATS, a standard CA testbed, and scales to problems with up to 150 items. Relative to a baseline that also learns allocations in menu elements, our method reduces the training iterations by 3.6$-$9.5$\times$ and cuts training time by about 80\% in settings with 50 and 100 items.
\end{abstract}

% \dcp{too technical to give `Liouville` name? also, mention diffusion in the abstract.}

\usepackage{amsmath}
\usepackage{mathtools}
\usepackage{amsthm}
\usepackage{wrapfig}
\DeclareMathOperator\supp{supp}

% if you use cleveref..
\usepackage[capitalize,noabbrev]{cleveref}

%%%%%%%%%%%%%%%%%%%%%%%%%%%%%%%%
% THEOREMS
%%%%%%%%%%%%%%%%%%%%%%%%%%%%%%%%
\makeatletter
\def\thmheadbrackets#1#2#3{%
  \thmname{#1}\thmnumber{\@ifnotempty{#1}{ }\@upn{#2}}%
  \thmnote{ {\the\thm@notefont[#3]}}}
\makeatother

\newtheoremstyle{brakets}% Name
  {}% space above
  {}% space below
  {\itshape}% body font
  {}% indent
  {\bfseries}% head font
  {.}% punctuation after head
  { }% space after head (has to be space or dimension!)
  {\thmheadbrackets{#1}{#2}{#3}}% head spec

\theoremstyle{brakets}

% \theoremstyle{plain}
\newtheorem{theorem}{Theorem}
\newtheorem{proposition}[theorem]{Proposition}
\newtheorem{lemma}[theorem]{Lemma}
\newtheorem{corollary}[theorem]{Corollary}
% \theoremstyle{definition}
\newtheorem{definition}[theorem]{Definition}
\newtheorem{assumption}[theorem]{Assumption}
\theoremstyle{remark}
\newtheorem{remark}[theorem]{Remark}

\definecolor{darkgreen}{rgb}{0.0, 0.5, 0.0}
\definecolor{darkblue}{rgb}{0.0, 0.5, 1.0}
% Todonotes is useful during development; simply uncomment the next line
%    and comment out the line below the next line to turn off comments
%\usepackage[disable,textsize=tiny]{todonotes}
% \usepackage[textsize=tiny]{todonotes}

% custom
% \usepackage{algorithm}
% \usepackage{algorithmicx}
% \usepackage{algpseudocode}
\newcount\Comments  % 0 suppresses notes to selves in text
\Comments = 1
\newcommand{\kibitz}[2]{\ifnum\Comments=1{\color{#1}{#2}}\fi}
\newcommand{\tw}[1]{\kibitzAdd{tw}{[Tonghan: #1]}}
\newcommand{\dcp}[1]{\kibitz{teal}{[DCP: #1]}}
\newcommand{\jf}[1]{\kibitzAdd{darkgreen}{[Jeff: #1]}}

\newcount\CommentsAdd  % 0 suppresses notes to selves in text
\CommentsAdd = 1
\newcommand{\kibitzAdd}[2]{\ifnum\CommentsAdd=1{\color{#1}{#2}}\fi}
\definecolor{english}{rgb}{0.0, 0.5, 0.0}
\definecolor{tw}{rgb}{0.0, 0.0, 0.5}
\newcommand{\dcpadd}[1]{\kibitzAdd{english}{#1}}
\newcommand{\twadd}[1]{\kibitz{blue}{#1}}
\newcommand{\jfadd}[1]{\kibitzAdd{blue}{#1}}


\usepackage{xcolor}
\newcommand\TODO[1]{\textcolor{red}{[TODO: #1]}}
\newcommand\CHANGE[1]{\textcolor{blue}{#1}}

%%%%% NEW MATH DEFINITIONS %%%%%

% \usepackage{amsmath,amsfonts,bm}
\usepackage{amsmath,amsfonts}

\usepackage{pifont}


\newcommand{\R}{\mathbb{R}}


\def\va{{\mathbf{a}}}
\def\vg{{\mathbf{g}}}

% Sets
\def\sR{\mathbb{R}}
\def\sC{\mathbb{C}}
\def\sZ{\mathbb{Z}}
\def\sN{\mathbb{N}}
\def\sQ{\mathbb{Q}}

\def\sS{\mathcal{S}}



% Vectors
\def\vzero{{\mathbf{0}}}
\def\vone{{\mathbf{1}}}
\def\vmu{{\mathbf{\mu}}}
\def\vtheta{{\mathbf{\theta}}}
\def\va{{\mathbf{a}}}
\def\vb{{\mathbf{b}}}
\def\vc{{\mathbf{c}}}
\def\vd{{\mathbf{d}}}
\def\ve{{\mathbf{e}}}
\def\vf{{\mathbf{f}}}
\def\vg{{\mathbf{g}}}
\def\vh{{\mathbf{h}}}
\def\vi{{\mathbf{i}}}
\def\vj{{\mathbf{j}}}
\def\vk{{\mathbf{k}}}
\def\vl{{\mathbf{l}}}
\def\vm{{\mathbf{m}}}
\def\vn{{\mathbf{n}}}
\def\vo{{\mathbf{o}}}
\def\vp{{\mathbf{p}}}
\def\vq{{\mathbf{q}}}
\def\vr{{\mathbf{r}}}
\def\vs{{\mathbf{s}}}
\def\vt{{\mathbf{t}}}
\def\vu{{\mathbf{u}}}
\def\vv{{\mathbf{v}}}
\def\vw{{\mathbf{w}}}
\def\vx{{\mathbf{x}}}
\def\vy{{\mathbf{y}}}
\def\vz{{\mathbf{z}}}
\def\vzeta{{\mathbf{\zeta}}}

% Matrix
\def\mA{{\mathbf{A}}}
\def\mB{{\mathbf{B}}}
\def\mC{{\mathbf{C}}}
\def\mD{{\mathbf{D}}}
\def\mE{{\mathbf{E}}}
\def\mF{{\mathbf{F}}}
\def\mG{{\mathbf{G}}}
\def\mH{{\mathbf{H}}}
\def\mI{{\mathbf{I}}}
\def\mJ{{\mathbf{J}}}
\def\mK{{\mathbf{K}}}
\def\mL{{\mathbf{L}}}
\def\mM{{\mathbf{M}}}
\def\mN{{\mathbf{N}}}
\def\mO{{\mathbf{O}}}
\def\mP{{\mathbf{P}}}
\def\mQ{{\mathbf{Q}}}
\def\mR{{\mathbf{R}}}
\def\mS{{\mathbf{S}}}
\def\mT{{\mathbf{T}}}
\def\mU{{\mathbf{U}}}
\def\mV{{\mathbf{V}}}
\def\mW{{\mathbf{W}}}
\def\mX{{\mathbf{X}}}
\def\mY{{\mathbf{Y}}}
\def\mZ{{\mathbf{Z}}}
\def\mBeta{{\mathbf{\beta}}}
\def\mPhi{{\mathbf{\Phi}}}
\def\mLambda{{\mathbf{\Lambda}}}
\def\mSigma{{\mathbf{\Sigma}}}


% Expectation
% \def\eE{\mathop{\mathbb{E}}\limits}
\def\eE{\mathbb{E}}

% Probability
\def\pP{\mathbb{P}}

% Tilde
\def\tf{\tilde{f}}
\def\tS{\tilde{S}}
\def\wtF{\widetilde{\mathcal{F}}}
\def\whR{\widehat{R}}
\def\tvx{\tilde{\mathbf{x}}}
\def\ty{\tilde{y}}


\def\defeq{\overset{\textup{def}}{=}}
% \def\defeq{\overset{.}{=}}
\def\defone{\overset{\text{\ding{172}}}{=}}
\def\deftwo{\overset{\text{\ding{173}}}{=}}
\def\leqone{\overset{\text{\ding{172}}}{\leq}}
\def\leqtwo{\overset{\text{\ding{173}}}{\leq}}
\def\leqthree{\overset{\text{\ding{174}}}{\leq}}
\def\leqfour{\overset{\text{\ding{175}}}{\leq}}
\def\eqone{\overset{\text{\ding{172}}}{=}}
\def\eqtwo{\overset{\text{\ding{173}}}{=}}
\def\eqthree{\overset{\text{\ding{174}}}{=}}
\def\eqfour{\overset{\text{\ding{175}}}{=}}
\def\geqfive{\overset{\text{\ding{176}}}{\geq}}
\newcommand{\shortn}{\textup{\texttt{-}}}
\newcommand{\shorte}{\textup{\texttt{=}}}
\newcommand{\shortp}{\textup{\texttt{+}}}
\newcommand{\shortl}{\textup{\texttt{<}}}
\newcommand{\shortg}{\textup{\texttt{>}}}
\newcommand{\ie}{\textit{i}.\textit{e}.}
\newcommand{\eg}{\textit{e}.\textit{g}.}
\newcommand{\etal}{\textit{et al}.}
\newcommand{\etc}{\textit{etc}.}
\newcommand{\Tau}{\mathrm{T}}

\newcommand\VRule[1][\arrayrulewidth]{\vrule width #1}
\usepackage{multirow}
\usepackage{makecell}
\newcolumntype{L}{>{$}l<{$}}
\newcolumntype{C}{>{$}c<{$}}
\newcolumntype{R}{>{$}r<{$}}
\newcommand{\nm}[1]{\textnormal{#1}}

\newcommand{\name}{\textsc{BundleFlow}}
\newcommand{\bundle}{\texttt{Bundle-RochetNet}}
\newcommand{\bigbundle}{\texttt{Big-Bundle}}
\newcommand{\smallbundle}{\texttt{Small-Bundle}}
\newcommand{\grandbundle}{\texttt{Grand-Bundle}}

\begin{document}

% Title page for title and abstract only.
\begin{titlepage}

\maketitle\makeatletter \gdef\@ACM@checkaffil{} \makeatother
% Optionally include a table of contents
% \vspace{-0.1cm}
\setcounter{tocdepth}{2} % adjust to 1 if desired
% \tableofcontents

\end{titlepage}

\section{Introduction}

Chain-of-Thought (CoT) prompting~\cite{Nye:2021, cot, Kojima:2022cotzero} has emerged as a cornerstone strategy for enhancing Large Language Models (LLMs) in complex reasoning tasks. By eliciting step-by-step inference, CoT enables LLMs to decompose intricate problems into manageable subtasks, thereby improving their problem-solving performance~\cite{Yao:2023tot, Wang:2023self-consistency, Zhou:2023least, Shinn:2023Reflexion}. Recent advancements, such as OpenAI's o1~\cite{o1} and DeepSeek-R1~\cite{deepseekr1}, further demonstrate that scaling up CoT lengths from hundreds to thousands of reasoning steps could continuously improve LLM reasoning. These breakthroughs have underscored CoT’s potential to advance LLM capabilities, expanding the boundaries of AI-driven problem-solving.

\begin{figure}[t]
\centering
    \includegraphics[width=0.95\columnwidth]{fig/intro.pdf}
    \caption{In contrast to vanilla CoT that generates all reasoning tokens sequentially, \method enables LLMs to \textit{skip} tokens with less semantic importance (\textit{e.g.,} \includegraphics[width=7pt]{fig/token.pdf}~) and learn shortcuts between critical reasoning tokens, facilitating controllable CoT compression.}
    \label{fig:intro}
\end{figure}

Despite its effectiveness, the increased length of CoT sequences introduces substantial computational overhead. Due to the autoregressive nature of LLM decoding, longer CoT outputs lead to proportional increases in both inference latency and memory footprints of key-value cache. Additionally, the quadratic computational cost of attention layers further exacerbates this burden. These issues become particularly pronounced when CoT sequences extend into thousands of reasoning steps, resulting in significant computational costs and prolonged response times. While prior research has explored methods for selectively skipping reasoning steps~\cite{Ding:2024cotshortcut, liu2024skipstep}, recent findings~\cite{jin:2024cotlength, Merrill:2024cotlength} suggest that such reductions may conflict with test-time scaling~\cite{o1-blog, snell2025scaling}, ultimately impairing LLM reasoning performance. Therefore, striking an optimal balance between CoT efficiency and reasoning accuracy remains a critical open challenge.

In this work, we delve into CoT efficiency and seek the answer to an important question: \textit{``Does every token in the CoT output contribute equally to deriving the answer?''} We empirically analyze the semantic importance of tokens within CoT outputs and reveal that their contributions to the reasoning performance vary, as depicted in Figure 2. Building on this insight, we introduce \method, a simple yet effective approach that enables LLMs to \textit{skip} less important tokens within CoT sequences and learn shortcuts between critical reasoning tokens, thereby allowing for controllable CoT compression with adjustable ratios. Specifically, as shown in Figure~\ref{fig:intro}, \method constructs compressed CoT training data with various compression ratios, by pruning unimportance tokens from original LLM CoT trajectories. Then, it conducts a general supervised fine-tuning process on target LLMs with this training data, facilitating LLMs to automatically trim redundant tokens during reasoning.

We conduct extensive experiments across various models, including LLaMA-3.1-8B-Instruct and the Qwen2.5-Instruct series, using two widely recognized math reasoning benchmarks: GSM8K and MATH-500. The results validate the effectiveness of \method in compressing CoT outputs while maintaining robust reasoning performance. Notably, Qwen2.5-14B-Instruct exhibits almost \textbf{NO} performance drop (less than $0.4\%$) with a $\bm{40\%}$ reduction in token usage on GSM8K. On the challenging MATH-500 dataset, LLaMA-3.1-8B-Instruct effectively reduces CoT token usage by $\bm{30}\%$ with a performance decline of less than $4\%$, resulting in a $\bm{1.4}\times$ inference speedup. Further analysis underscores the coherence of \method in specified compression ratios and its potential scalability with stronger compression techniques.

\method is distinguished by its low training cost. For Qwen2.5-14B-Instruct, \method fine-tunes only 0.2\% of the model's parameters using LoRA. The size of the compressed CoT training data is no larger than that of the original training set, with 7,473 examples in GSM8K and 7,500 in MATH. The training is completed in approximately 2 hours for the 7B model and 2.5 hours for the 14B model on two 3090 GPUs. These characteristics make \method an efficient and reproducible approach, suitable for use in efficient and cost-effective LLM deployment.

To sum up, our key contributions are:
\begin{enumerate}
    \item To the best of our knowledge, this work is the \textit{first} to investigate the potential of enhancing CoT efficiency through \textit{token skipping}, inspired by the varying semantic importance of tokens in CoT trajectories of LLMs.
    \item We introduce \method, a simple yet effective approach that enables LLMs to skip redundant tokens within CoTs and learn shortcuts between critical tokens, facilitating CoT compression with adjustable ratios.
    \item Our experiments validate the effectiveness of \method. When applied to Qwen2.5-14B-Instruct, \method reduces reasoning tokens by $40\%$ (from 313 to 181) on GSM8K, with less than a $0.4\%$ performance drop.
\end{enumerate}


\section{Preliminary}
We briefly review related research on flow matching and LMs, providing foundations for introducing our method.

\subsection{Rectified Flow}\label{sec:rw_rf}

Rectified flow~\cite{liu2022flow,albergo2023building} emerges as a robust and powerful generative model and has recently served as the basis for popular commercial tools like Stable Diffusion 3~\cite{stabilityAI2023}. It is based on flow matching~\cite{chen2018neural,lipman2022flow}, which models the generative process as an Ordinary Differential Equation (ODE).  Formally, a \emph{continuous normalizing flow} transports an input $\vz_0\in \mathbb{R}^d$ to $\vz_t=\phi(t, \vz_0)$ at time $t\in[0,1]$ via the ODE:
\begin{align}
    \frac{d}{dt}\phi(t, \vz_0) = \varphi\left(t, \phi(t, \vz_0)\right).
\end{align}
Here, $\phi:[0,1]\times \mathbb{R}^d\rightarrow\mathbb{R}^d$ is the \emph{flow}, and the \emph{vector field} $\varphi: [0,1]\times \mathbb{R}^d\rightarrow \mathbb{R}^d$ specifies the change rate of the state $\vz_t$. \citet{chen2018neural} suggests representing the vector field $\varphi$ with a neural network.

The flow $\phi$ transforms an initial random variable $Z_0\sim p_0(\vz_0)$ to $Z_1\sim p_1(\vz_1)$ at final time 1. Rectified flow tries to drive the flow to follow the linear path in the direction $(Z_1-Z_0)$ as much as possible:
\begin{align}
    \min_\varphi \int_0^1 \mathbb{E}\left[\|(Z_1-Z_0)-\varphi(t, Z_t)\|^2\right]dt,\label{equ:rf}
\end{align}
where $Z_t=t\cdot Z_1 + (1-t)\cdot Z_0$ is the linear interpolation of $Z_0$ and $Z_1$. Typically, the vector field network $\varphi$ is implemented as a U-Net~\citep{ronneberger2015unet} for image inputs or an MLP for vector inputs~\cite{wang2024diffusion}.

\subsection{Transformer} 
The Transformer architecture~\citep{vaswani2017attention} is foundational to recent progress in large language models (LLMs)~\citep{liu2024deepseek, zeng2024skywork,yang2024qwen2,team2023gemini}. For an input sequence of tokens $x = (\idx[\vx][][1], \dots, \idx[\vx][][N])$, let \( \idx[E][][n][] = [e(\idx[\vx][][1]), \dots, e(\idx[\vx][][n])] \) denote the sequence of token embeddings up to position $n$, where $e(\cdot)$ is the token embedding function. A standard LLM generates its output by
\begin{align}
&\idx[\mH][][n]=\textsc{Transformer}\left(\idx[E][][n]\right),\nonumber \\
&M\left(\idx[\vy][][n+1] \mid \idx[\vx][][\leq n]\right)=\mW \idx[\vh][][n],\label{equ:token_logits}
\end{align}
where $\idx[\mH][][n] \in \mathbb{R}^{n \times d}$ is the last hidden state for the first $n$ tokens, with $d$ representing the hidden dimension. $\idx[\vh][][n]$ is the last hidden state at position $n$, i.e., $\idx[\vh][][n] = \idx[\mH][][n][] [n, :]$. $\mW$ is the output projection matrix, $M$ is the model's generation logits, and $\vy$ is the output token.

We consider an LLM with $L$ layers, the hidden state after $l$ layers, $\idx[\mH][][n,l]$, is projected by three weight matrices \(\mW_Q \), \(\mW_K \), and \(\mW_V \) to the query, key, and value embeddings $\idx[\mQ][][n,l]$, $\idx[\mK][][n,l]$, and $\idx[\mV][][n,l]$, respectively. The self-attention is calculated as:
\begin{align}
&(\idx[\mQ][][n,l], \idx[\mK][][n,l], \idx[\mV][][n,l]) = \idx[\mH][][n,l] (\mW_Q, \mW_K, \mW_V) \nonumber\\
&\idx[\mA][][n,l] = \frac{\idx[\mQ][][n,l] {\idx[\mK][][n,l]}^\top}{\sqrt{d_K}}, \text{Attn}(\idx[\mH][][n,l]) = \sigma(\idx[\mA][][n,l]) \idx[\mV][][n,l],    \nonumber
\end{align}
where $\sigma(\cdot)$ is SoftMax, and $\mA$ is the self-attention matrix. We omit the multi-head attention for simplicity.


\subsection{RLHF}
Reinforcement learning from human feedback~($\mathtt{RLHF}$)~\citep{bai2022training, wang2023helpsteer, ouyang2022training, dong2024rlhf} is critical to aligning LLM behavior with human preferences such as helpfulness, harmlessness, and honesty~\citep{ganguli2022red, achiam2023gpt, team2023gemini}. An RL-based method trains a reward model~\citep{liu2024skywork} to approximate human preferences. Given a preference dataset $\mathcal{D} = (x, y_w, y_l)$, where $x$ is the input, $y_w$ is the preferred output, and $y_l$ is the less preferred output, a reward model $r_\theta$ can be trained using the standard Bradley-Terry model~\citep{bradley1952rank} with a pairwise ranking loss. With $r_\theta$, the policy model (LLM) is trained via $\mathtt{PPO}$~\citep{schulman2017proximal}.
% with the following objective:
% \begin{equation}
%     \max_{\pi_\theta} \mathbb{E}_{x\sim\mathcal{D},y \sim \pi_\theta(\cdot|x)} \left[r_{\theta}(x, y)\right] - \beta \mathbb{D}_\text{KL}\left[\pi_\theta(\cdot|x) || \pi_{\text{ref}}(\cdot|x)\right].\nonumber
% \end{equation}
% Here, the KL divergence penalty is applied to prevent excessive deviation from the reference model $\pi_{\text{ref}}$, \ie, the initial supervised fine-tuned (SFT) model.
However, training a reward model can be costly. Direct preference learning ($\mathtt{DPO}$) \cite{rafailov2024direct} enables direct training with preference data, which can be adapted to different human utility models ($\mathtt{KTO}$,~\citet{ethayarajh2024kto}).
% ~(\eg, $\mathtt{KTO}$~\cite{ethayarajh2024kto})
%shows that it is possible to directly train with the preference data

% DPO optimizes the following loss function to train the policy model (LLM) $\pi_\theta$:
% \begin{align}
%     \mathcal{L}_{\text{DPO}}(\theta) = \mathbb{E}_{\mathcal{D}}
%     \left[\shortn\log \sigma\left(\beta \log \frac{\pi_\theta(y_w|x)}{\pi_{\text{ref}}(y_w|x)} \shortn \beta \log \frac{\pi_\theta(y_l|x)}{\pi_{\text{ref}}(y_l|x)}\right)\right],\nonumber
% \end{align}
% where $\beta$ is a parameter controlling the deviation from the supervised fine-tuned model $\pi_{\text{ref}}$.




% At the ODE time $t$, the hidden state after the $m$-th layer corresponding to the input $(t, \vz_t)$ is denoted by $\idx[\vh][f][t,m](\vz_t)$.
% \begin{equation}
%     \mathcal{L}_{\text{reward}}(\theta) = -\log \left(\sigma\left(r_\theta\left(x, y_w\right) - r_\theta\left(x, y_l\right)\right)\right),
% \end{equation}
%where $\sigma$ is the logistic function.
% to guide LLMs toward desired behaviors
% leverage recent advancements in reinforcement learning (e.g., PPO~\citep{schulman2017proximal}) to enhance the alignment of LLMs~\citep{achiam2023gpt}. 
% A key component of these methods is the development of , which learn a reward function based on 
% Instead, DPO derives a reward signal directly from the currently optimized model and an initial supervised fine-tuned model \citep{rafailov2024r}, effectively reparameterizing preference learning within the model itself.
\section{3DMolFormer}
\begin{figure}[t]
    \centering
    \includegraphics[width=\textwidth]{imgs/Main.pdf}
    \vspace{-0.4cm}
    \caption{Overview of 3DMolFormer. The left shows the dual-channel model architecture, the top right illustrates the input and output of the two SBDD tasks in a parallel sequence, and the bottom right outlines the pre-training and fine-tuning process.}
    \label{Overview}
    \vspace{-0.2cm}
\end{figure}

\subsection{Format of Pocket and Ligand Sequences with 3D Coordinates}

To leverage a causal language model for handling 3D protein pockets and small molecules while explicitly separating discrete structural information from continuous spatial coordinates, we design a parallel sequence format. This format consists of a discrete token sequence $s_\mathrm{tok}$ and a continuous numerical sequence $s_\mathrm{num}$, both of which share the same length and align element-wise. The token sequence consists of tokens in a predefined vocabulary, while the numerical sequence contains floating-point values.

As shown in Figure~\ref{PocketSeq}, the sequence for a protein pocket $s^\mathrm{poc}$ consists of two parts: the first $s^\mathrm{poc\_atoms}$ represents an atomic list, and the second $s^\mathrm{poc\_coord}$ contains 3D coordinate information. 
The atomic list is encoded in the token sequence, which includes all atoms in the protein pocket except for hydrogen atoms. Aside from alpha carbon atoms, denoted as 'CA', other atoms are represented by their element type, such as 'C', 'O', 'N', and 'S'. The sequence of atoms follows the order of the pdb file, where each amino acid begins with ['N', 'CA', 'C', 'O'] followed by the side-chain atoms.
The normalized 3D coordinates for each atom in the atomic list are included in the numerical sequence in the same order, with each dimension ('x', 'y', 'z') occupying a separate position. The length of the 3D coordinate sequence is always three times the length of the atomic list.
Moreover, in the token sequence, the start and end of the atomic list are marked by 'PS' and 'PE', while the 3D coordinates are delineated by 'PCS' and 'PCE' at the start and end, respectively.
In the numerical sequence, numbers that do not correspond to 3D coordinates are padded with 1.0.

As illustrated in Figure~\ref{LigandSeq}, the sequence for a small molecule $s^\mathrm{lig}$ is similar to that of the protein pocket, comprising both a SMILES string section $s^\mathrm{lig\_smiles}$ and a 3D coordinate section $s^\mathrm{lig\_coord}$.
After atom-level tokenization~\citep{SMILEStokenization}, the SMILES string of the small molecule is encoded in the token sequence, excluding hydrogen atoms. It is important to note that some tokens may not correspond to atoms, and thus, no 3D coordinates will be associated with them.
The normalized 3D coordinates for each atom in the tokenized SMILES string are included in the numerical sequence, with each coordinate dimension ('x', 'y', 'z') occupying a separate position. The length of the 3D coordinate sequence is always three times the number of atoms in the small molecule.
In the token sequence, the start and end of the SMILES tokens are marked by 'LS' and 'LE', while the 3D coordinates of the corresponding atoms are marked by 'LCS' and 'LCE' at the start and end, respectively.
In the numerical sequence, numbers not corresponding to 3D coordinates are similarly padded with 1.0.


When the sequence of a protein pocket is concatenated with that of a small molecule ligand, it forms a pocket-ligand complex sequence along with their 3D coordinates $s^\mathrm{poc-lig}$. This sequence format offers three advantages:
\begin{itemize}[leftmargin=*]
    \item It fully encapsulates the structural and 3D coordinate information of both the protein pocket and the small molecule ligand.
    \item Discrete structural information and continuous numerical data are separated into two parallel sequences, enabling independent processing of each data type.
    \item The sequence of the pocket-ligand complex maintains causal logic. As depicted in the upper right of Figure~\ref{Overview}, this sequence structure allows autoregressive prediction, which can effectively represent both pocket-ligand docking and pocket-aware drug design tasks.
\end{itemize}

Specifically, we normalize the coordinates of all pocket-ligand complexes by translating their center of mass to the origin $(0,0,0)$. Additionally, to ensure numerical stability during training~\citep{DeepLearning}, we scale the coordinate values by a factor $q>1$ to reduce the range of their distribution:
\begin{equation}
\label{coordnorm}
(x_i',y_i',z_i')=\Big(\frac{x_i-x_c}{q},\frac{y_i-y_c}{q},\frac{z_i-z_c}{q}\Big),
\end{equation}
where $(x_i,y_i,z_i)$ is the original coordinate of the $i$-th atom, $(x_c,y_c,z_c)$ is the coordinate of the center of mass, and $(x_i',y_i',z_i')$ refers to the normalized values used in the numerical sequence.

\subsection{Model Architecture}
To process the aforementioned parallel sequences, we require an autoregressive language model that can simultaneously take a discrete token sequence and a continuous floating-point sequence as input, while predicting both the next token and the next numerical value. Inspired by xVal~\citep{xVal}, we propose a dual-channel transformer architecture for 3DMolFormer, as illustrated in the left part of Figure~\ref{Overview}. The module handling the token sequence is based on the GPT-2 model~\citep{GPT-2}, featuring identical token embeddings, positional embeddings, multiple transformer layers, and a prediction head for logits. On top of this, we introduce a parallel numerical channel at both the input and output stages.

At the input stage, we multiply the embedding of each token in the token sequence with the corresponding value in the numerical sequence, using this product as the input to the positional embedding. This is why numerical values that lack meaningful information are padded with 1.0. At the output stage, in parallel with the token prediction head, we add a number head to predict the next floating-point value.

During inference with 3DMolFormer, the outputs are handled in two modes:
\begin{itemize}[leftmargin=*]
    \item Token Mode: In the drug design task, when predicting ligand SMILES tokens, the corresponding numerical output holds no meaningful value and is therefore padded with 1.0.
    \item Numerical Mode: In docking and drug design tasks, once the ligand SMILES is determined, the length of the 3D coordinate sequence and its tokens are also fixed. Therefore, the token output no longer holds meaningful information and is filled with the expected tokens (from ['x', 'y', 'z', 'LCS', 'LCE']). When the position corresponds to ['x', 'y', 'z'], the predicted floating-point values are appended to the input numerical sequence. For tokens corresponding to ['LCS', 'LCE'], the numerical values are also set to 1.0.
\end{itemize}


\subsection{Self-supervised Pre-training}
To enable the 3DMolFormer model to learn the general patterns of pocket-ligand complex sequences, we conduct large-scale pre-training on 3D data, which includes three datasets: approximately 3.2M protein pockets, about 209M small molecule conformations, and around 167K pocket-ligand complexes. The first two datasets were collected by Uni-Mol~\citep{Uni-Mol} for large-scale pre-training on 3D protein pockets and small molecules, while the last dataset was generated by CrossDocked2020~\citep{CrossDocked}.

In order for the dual-channel autoregressive model to capture both the token sequence format and the 3D coordinate patterns of pocket-ligand complexes, we adopt a composite loss function for the prediction of the next token and the corresponding numerical value. This loss function incorporates the cross-entropy (CE) loss for the whole token sequence and the mean squared error (MSE) loss for the numerical sequence corresponding to the 3D coordinates:
\begin{equation}
\label{pretrainloss}
L(\hat{s}, s)=\mathrm{CE}(\hat{s}_\mathrm{tok}, s_\mathrm{tok})+\alpha\cdot \mathrm{MSE}(\hat{s}_\mathrm{num}^\mathrm{coord}, s_\mathrm{num}^\mathrm{coord}),
\end{equation}
where $\hat{s}$ represents the sequence predicted by 3DMolFormer, $s$ refers to the training data, and $\alpha$ is a coefficient that controls the balance between the CE loss and the MSE loss. This composite loss is applied to all of the three types of pre-training data.

Additionally, we employ large-batch training \citep{large-batch-training} through gradient accumulation, which we found to be crucial for the pre-training stability of 3DMolFormer. For further details on pre-training and hyper-parameter settings, please refer to Section~\ref{experiments} and Appendix~\ref{app2}.




\subsection{Fine-tuning}
After the large-scale pre-training, we further fine-tune the 3DMolFormer model on two downstream drug discovery tasks: supervised fine-tuning for pocket-ligand docking, and reinforcement learning (RL) fine-tuning for pocket-aware drug design.

\subsubsection{Supervised Fine-tuning for Protein-ligand Binding Pose Prediction}
In the protein-ligand binding pose prediction (docking) task, as illustrated in Figure~\ref{Overview}, each sample consists of a pocket-ligand complex. The input sequence contains the atoms of the protein pocket and their 3D coordinates, along with the SMILES sequence of the ligand. The output is the 3D coordinates of each atom in the ligand.

The pre-training data for 3DMolFormer already includes about 167K pocket-ligand complexes from CrossDocked2020~\citep{CrossDocked}; however, these complexes are generated using the docking software Smina~\cite{Smina}, which means that the docking performance of models trained with this data would not exceed that of Smina. To improve the upper limit of our model's docking performance, we fine-tune it on the experimentally determined PDBBind dataset~\cite{PDBbind}, which contains approximately 17K ground-truth pocket-ligand complexes. Additionally, we employ a task-specific loss function that computes the mean squared error (MSE) loss only for the 3D coordinates of the ligand in the context of next numerical value prediction, since the inference process of docking operates entirely in numerical mode:
\begin{equation}
L_{\mathrm{docking}}(\hat{s}^\mathrm{lig\_coord},s^\mathrm{lig\_coord})=\mathrm{MSE}(\hat{s}_\mathrm{num}^\mathrm{lig\_coord}, s_\mathrm{num}^\mathrm{lig\_coord}).
\end{equation}

To mitigate overfitting during supervised fine-tuning, SMILES randomization~\citep{SMILESRandomization} and random rotation of the 3D coordinates of complexes are used as data augmentation strategies. For further details on docking fine-tuning, please refer to Section~\ref{exp-docking} and Appendix~\ref{app3}.

\subsubsection{RL Fine-tuning for Pocket-aware 3D Drug Design}
In the pocket-aware drug design task, as illustrated in Figure~\ref{Overview}, each sample is also a pocket-ligand pair. The input sequence includes the atoms of the protein pocket and their 3D coordinates, while the output consists of the ligand SMILES sequence and the 3D coordinates of its atoms.

Inspired by 1D RL-based molecular generation methods \citep{Reinvent}, an RL agent with the 3DMolFormer architecture is initialized with the pre-trained weights, and a molecular property scoring function for each protein pocket is designed as the RL reward. Then, the agent is iteratively optimized to maximize the expected reward of its outputs. Specifically, at each RL step, the agent samples a batch of 3D ligands, and the regularized maximum likelihood estimation (MLE) loss \citep{SMILES_RL} of each ligand is computed and used to update the agent:
\begin{equation}
\label{RLloss}
L_{\mathrm{design}}(\hat{s}^\mathrm{lig})=\big(\log\pi_\text{pre-trained}(\hat{s}_\mathrm{tok}^\mathrm{lig\_smiles})+\sigma\cdot R(m)-\log\pi_\text{agent}(\hat{s}_\mathrm{tok}^\mathrm{lig\_smiles})\big)^2,
\end{equation}
where $\hat{s}^\mathrm{lig}$ ($\hat{s}^\mathrm{lig\_smiles}$ and $\hat{s}^\mathrm{lig\_coord}$) is a sample generated by the RL agent, $m$ is the 3D molecule represented by $\hat{s}^\mathrm{lig}$, and $R(\cdot)$ is reward function evaluating the property of the molecule. $\pi_\text{pre-trained}(s)$ is the likelihood of the pre-trained 3DMolFormer model for generating the sequence $s$, $\pi_\text{agent}(s)$ is the corresponding likelihood of the agent model, and $\sigma$ is a coefficient hyper-parameter to control the importance of the reward. This loss function encourages the agent to generate molecules with higher expected rewards while retaining a low deviation from the pre-trained weights.

It is important to note that to leverage the duality of the two SBDD tasks, the sampling of ligand SMILES utilizes the weights of the RL agent's model, which are continuously updated during fine-tuning. In contrast, the generation of atomic 3D coordinates uses the weights from the model fine-tuned for docking, which remains unchanged during this process. For additional details on RL fine-tuning and hyper-parameter settings, please refer to Section~\ref{exp-drug-design} and Appendix~\ref{app4}.
\section{Experiments}

\subsection{Models}
To comprehensively evaluate on LMMs, we conducted zero-shot inference across both commercial and open-source models. Our evaluation suite includes leading commercial models GPT-4o~\cite{hurst2024gpt40} and Gemini1.5-Pro~\cite{Gemini} alongside state-of-the-art open-source alternatives of varying scales: Qwen2.5-VL~\cite{qwen2.5-VL}, Qwen2-VL~\cite{wang2024qwen2}, LLaVA-v1.6~\cite{liu2023llava}, CogVLM~\cite{wang2023cogvlm}, MiniCPM-o-2.6~\cite{yao2024minicpm}, mPlug-Owl2~\cite{ye2023mplugowl2}, InternVL2v5~\cite{chen2024internvl},LLaVA-NEXT-Video~\cite{zhang2024llavanextvideo} and Cambrian~\cite{tong2024cambrian1}. Besides, Janus-Pro~\cite{chen2025januspro}, which unifies multimodal understanding and generation, is included to test the abilities between Unified Model and Vision Language Model.  This diverse selection enables us to analyze how model scale, architecture, and training approaches influence comic comprehensive capabilities. 


% Detailed specifications and inference configurations for each model are provided in Appendix~\ref{appendix:model}.

\subsection{Experimental Details}
% 隐含含义理解和预测帧内容这两个任务都是选择题,因为他们的标准答案是不唯一的很难衡量。如果模型选择了正确的答案选项,则认为是正确的,也就是说accuracy是主要的metric。 而对于排序任务,这一个任务形式非常新颖,对于一个comic strip,其输入顺序可以随便打乱,且答案是确切的。所以这个任务我们既采用了选择题的题型又使用了问答题。
The task prompts is displayed in Table ~\ref{prompt}. For visual narrative comprehension task, model is provided with the whole image. But for next-frame prediction and multi-frame sequence reordering task, LMMs infer with image sequences.
The hyper-parameters for each LMMs in the experiments including possible settings are detailed in Appendix~\ref{appendix:hyper-param}. Furthermore, to assess human capabilities in these tasks, we randomly select 100 questions from the dataset for each task and instruct human evaluators to answer. This allows us to benchmark the performance of human participants against our models, offering a thorough comparison of both human and LMMs proficiency in these specific tasks. 


\subsection{Main Results}
Our comprehensive evaluation reveals that while LMMs show promising capabilities in comprehension and prediction tasks, they significantly underperformed in sequence reordering tasks. Moreover, there remains a substantial performance gap between current models and human performance across all tasks. Unified Model underperformed than Vision Language Model.

\paragraph{Contextual Frame Prediction}
The frame prediction task appears to be the most tractable among the three tasks. GPT-4o achieves the highest score of $69.95\%$, followed closely by Qwen2-VL at $64.00\%$.This demonstrates that the performance gap between closed and open-source models is relatively small for this task. However, Janus-Pro perform notably below expectations ($27.50\%$), possibly due to its unified model architectural.

\paragraph{Visual Narrative Comprehension}
For visual narrative comprehension, we observe a similar pattern but with generally lower scores. GPT-4o leads with $61.60\%$, while other models show varying degrees of capability. 

\paragraph{Temporal Narrative Reordering}
The frame reordering task proves to be the most challenging, with all models performing significantly below human capability. Even the best-performing models struggle to exceed $30\%$ accuracy, with many achieving scores around $25-26\%$, which is slightly higher than random selection. 
Notably, several models (marked with *) are unable to perform this task due to their architectural limitations in processing multiple images simultaneously. For these models, we attempted to accommodate their single-image constraint by concatenating multiple frames horizontally into a single image, with white margins serving as frame boundaries. However, this workaround appears to be suboptimal, as these models likely struggle to properly distinguish individual frame boundaries and maintain the semantic independence of each frame, ultimately leading to their poor performance on the reordering task. 

The poor performance on reordering task suggests that current LMMs, regardless of their scale or architecture, have not yet developed robust capabilities for understanding temporal relationships and sequential logic in visual narratives.




\section{Conclusion}

This work analysed the results of evolutionary wrapper approaches using decision tree based models as proxies and compared them with common \gls{FE} techniques on a \gls{HL} detection problem. Three experiments were conducted using the proposed framework, each employing different proxy models.

When comparing the three experiments, an interesting behaviour of the framework was discovered, when changing the proxy model. The \gls{DT} experiment drastically reduced the number of features, while the other models did not. To further reduce the number of features, one could bias the grammar or apply some penalty in the fitness function for the individuals that use a large number of features. This might not change the behaviour when using different models other than a \gls{DT}, but it forcefully reduces the number of features.  

The results confirm that FEDORA can reduce the dimensionality of the data while statistically maintaining baseline performance, in every experiment. The framework consistently outperforms the remaining \gls{FE} methods, with statistical significance and large effect sizes, proving itself as a viable alternative.

The best result obtained is 76.2\% balanced accuracy using an individual from the \gls{RF} experiment, and a \gls{XGB} algorithm as the testing model, using 57 total features (45 Original, 6 Engineered and 6 Complex) out of the 60 original ones. When using the least amount of features, the best result is 72,8\% balanced accuracy using an individual from the \gls{DT} experiment and a \gls{RF} algorithm as the testing model, using a single complex feature.

In future work, exploring the above-mentioned behaviours might be relevant to better understanding them, namely when biasing the grammar or penalizing the use of many features in the fitness function. Concerning the explainability of the FEDORA transformations, researching meaningful grammar operators might prove useful in addressing problem-specific needs. In this case, having logical operators for the boolean features, which have values of "yes" or "no", and the choice of a simple decision algorithm as the proxy, may increase explainability. Additionally, the previous study has identified several areas for future research, yet to be addressed. For instance, comparing the framework with other common and more complex methods and completing the full \gls{ML} pipeline through the use of a method that addresses the \gls{CASH}, such as \cite{assunccao2020evolution}, and comparing it to other full pipeline frameworks, could be beneficial for contextualizing and evaluating the framework within the \gls{AutoML} and \gls{EC} domains. The framework still needs to be analysed with different datasets to properly assess its generalization capabilities.


% Bibliography
\newpage
\bibliographystyle{ACM-Reference-Format}
\bibliography{sample-bibliography}

% Appendix
\newpage
\appendix
% \section{Normalizing Flow}

% A normalizing flow model provides a transportation from a distribution to another by applying a sequence of functions that are not necessarily the same. As shown in Fig.~\ref{fig:normalizing_flow}, for round $i\in\{1,\cdots, T-1\}$, the input to the function is $z_i\in\mathbb{E}^m$ \dcp{$\mathbb{R}^m$?}
% sampled from the distribution $p_i(z_i)$. After applying the function, the distribution will be transformed to $p_{i+1}(z_{i+1})$. \dcp{$t$ not $i$ in main body}
% \begin{figure}[h!]
%     \centering
%     \includegraphics[width=\linewidth]{figures/flow.pdf}
%     \caption{A normalizing flow model that transport a simple distribution to a complex distribution.\label{fig:normalizing_flow}}
%     \Description[Normalizing flow.]{A normalizing flow model that transport a simple distribution to a complex distribution.}
% \end{figure}

% We are interested in the closed-form expression of $p_{i+1}(z_{i+1})$. According to the change of variable theorem and assuming  the neural network is invertible, we have
% \begin{align}
%     p_{i+1}(z_{i+1}) = p_i\left(f^{\shortn 1}(z_{i+1})\right)\left|\text{det}\frac{df^{\shortn 1}}{dz_{i+1}}\right|= p_i\left(z_i\right)\left|\text{det}\frac{df^{\shortn 1}}{dz_{i+1}}\right|,
% \end{align}
% where $\text{det}\frac{df}{dz}$ is the Jacobian determinant of function $f$. The inverse function theorem tells us
% \begin{align}
%     \frac{df^{\shortn 1}}{dz_{i+1}} = \frac{dz_i}{dz_{i+1}} = \left(\frac{dz_{i+1}}{dz_i}\right)^{\shortn 1} = \left(\frac{df}{dz_i}\right)^{\shortn 1}.
% \end{align}
% It follows that $p_{i+1}(z_{i+1})=p_i\left(z_i\right)\left|\text{det}\left(\frac{df}{dz_i}\right)^{\shortn 1}\right|$. For an invertible function $A=\frac{df}{dz_i}$, we have $1=det(I)=det(AA^{\shortn i})=det(A)det(A^{\shortn 1})$, so $det(A^{\shortn 1})=det(A)^{\shortn 1}$. Therefore, we have $p_{i+1}(z_{i+1}) = p_i\left(z_i\right)\left|\text{det}\frac{df}{dz_{i+1}}\right|^{\shortn 1}$, and
% \begin{align}
%     \log p_{i+1}(z_{i+1}) = \log p_i\left(z_i\right) -\log \left|\text{det}\frac{df}{dz_{i+1}}\right|.
% \end{align}
% We can apply this repeatedly and obtain
% \begin{align}
%     \log p(z) = \log p_T(z_T) = \log p_0\left(z_0\right) -\sum_{i=0}^{T-1}\log \left|\text{det}\frac{df}{dz_{i+1}}\right|.
% \end{align}


% \section{Likelihood derivation}

% \subsection{Linear}

% When $\varphi(t, s_t) = \sigma(t) Q s(t)$, we have

% \begin{align}
%     \nabla\cdot \varphi(t, s(t)) = \sum_{i=1}^d \frac{\partial \varphi_i}{\partial s_i(t)}
%     = \sum_{i=1}^d \frac{\partial}{\partial s_i(t)} \sigma(t) Q_i s(t) = \sum_{i=1}^d  \sigma(t) Q_{ii} = \sigma(t)\Tr(Q)
% \end{align}


\end{document}
