\documentclass[format=acmsmall, review=false]{acmart}
\usepackage{acm-ec-25}
\usepackage{booktabs} % For formal tables
\usepackage[ruled]{algorithm2e} % For algorithms
\renewcommand{\algorithmcfname}{ALGORITHM}
\SetAlFnt{\small}
\SetAlCapFnt{\small}
\SetAlCapNameFnt{\small}
\SetAlCapHSkip{0pt}
\IncMargin{-\parindent}

\usepackage{threeparttable}
\usepackage{tikz}
\usetikzlibrary{shapes}
\usetikzlibrary{calc}
\usetikzlibrary{decorations.pathreplacing}

% \settopmatter{printacmref=true}
%\setcitestyle{acmnumeric}
\setcitestyle{authoryear}

% \begin{CCSXML}
% <ccs2012>
%    <concept>
%        <concept_id>10003752.10010070.10010099.10010107</concept_id>
%        <concept_desc>Theory of computation~Computational pricing and auctions</concept_desc>
%        <concept_significance>500</concept_significance>
%        </concept>
%    <concept>
%        <concept_id>10010147.10010257.10010293.10010294</concept_id>
%        <concept_desc>Computing methodologies~Neural networks</concept_desc>
%        <concept_significance>500</concept_significance>
%        </concept>
%    <concept>
%        <concept_id>10010147.10010178.10010219.10010220</concept_id>
%        <concept_desc>Computing methodologies~Multi-agent systems</concept_desc>
%        <concept_significance>500</concept_significance>
%        </concept>
%    <concept>
%        <concept_id>10010405.10010455.10010460</concept_id>
%        <concept_desc>Applied computing~Economics</concept_desc>
%        <concept_significance>500</concept_significance>
%        </concept>
%  </ccs2012>
% \end{CCSXML}

% \ccsdesc[500]{Theory of computation~Computational pricing and auctions}
% \ccsdesc[500]{Computing methodologies~Neural networks}
% \ccsdesc[500]{Computing methodologies~Multi-agent systems}
% \ccsdesc[500]{Applied computing~Economics}

% \keywords{mechanism design, ordinary differential equation, deep learning, diffusion models, flow matching, optimal auction design, strategy-proofness, microeconomics, combinatorial auction.}

% Title. Note the optional short title for running heads. In the interest of anonymization, please do not include any acknowledgements.
%dcp suggesting new title
%\title[General Menu-Based Deep Auctions]{General Strategy-Proof Mechanisms Through Deep Learning}
\title[Menu-Based Combinatorial Auctions]{\name: Deep Menus for Combinatorial Auctions by Diffusion-Based Optimization}

% Anonymized submission.
\author{Tonghan Wang}\email{twang1@g.harvard.edu}

\affiliation{
  \institution{Harvard University}
  \city{Cambridge}
  \state{MA}
  \postcode{02138}
  \country{USA}
}

\author{Yanchen Jiang}\email{}

\affiliation{
  \institution{Harvard University}
  \city{Cambridge}
  \state{MA}
  \postcode{02138}
  \country{USA}
}

\author{David C. Parkes}\email{}

\affiliation{
  \institution{Harvard University}
  \city{Cambridge}
  \state{MA}
  \postcode{02138}
  \country{USA}
}
% \author{Submission 812}

% Abstract. Note that this must come before \maketitle.
\begin{abstract}
% \dcp{I dropped mention of DSIC in the abstract because for single bidder it's not such a  big deal to achieve.}
%dcp cut Combinatorial auctions (CAs) are important but challenging to design. Even for the single-bidder case, the design of revenue-maximizing CAs remains largely elusive. 
%difficulties  familiar from theoretical work---the 
%dcp cut, to unambiguously determine the utility to a bidder,
%. This method draws inspirations from generative models, especially
%dcp suggest cut The bundle distribution can be reconstructed by transforming the probability density of the initial distribution by the Liouville equation. 
Differentiable economics---the use of deep learning for auction design---has driven progress in the automated design of multi-item auctions with additive or unit-demand valuations. However, little progress has been made for optimal combinatorial auctions (CAs), even for the single bidder case, because we need to overcome the challenge of the bundle space growing exponentially with the number of items. For example, when learning a menu of allocation-price choices for a bidder in a CA, each menu element needs to efficiently and flexibly specify a probability distribution on bundles. In this paper, we solve this problem in the single-bidder CA setting by generating a bundle distribution through an ordinary differential equation (ODE) applied to a tractable initial distribution, drawing inspiration from generative models, especially score-based diffusion models and continuous normalizing flow. Our method, \name, uses deep learning to find suitable ODE-based transforms of initial distributions, one transform for each menu element, so that the overall menu achieves high expected revenue.  Our method achieves 1.11$-$2.23$\times$ higher revenue compared with automated mechanism design baselines on the single-bidder version of CATS, a standard CA testbed, and scales to problems with up to 150 items. Relative to a baseline that also learns allocations in menu elements, our method reduces the training iterations by 3.6$-$9.5$\times$ and cuts training time by about 80\% in settings with 50 and 100 items.
\end{abstract}

% \dcp{too technical to give `Liouville` name? also, mention diffusion in the abstract.}

\usepackage{amsmath}
\usepackage{mathtools}
\usepackage{amsthm}
\usepackage{wrapfig}
\DeclareMathOperator\supp{supp}

% if you use cleveref..
\usepackage[capitalize,noabbrev]{cleveref}

%%%%%%%%%%%%%%%%%%%%%%%%%%%%%%%%
% THEOREMS
%%%%%%%%%%%%%%%%%%%%%%%%%%%%%%%%
\makeatletter
\def\thmheadbrackets#1#2#3{%
  \thmname{#1}\thmnumber{\@ifnotempty{#1}{ }\@upn{#2}}%
  \thmnote{ {\the\thm@notefont[#3]}}}
\makeatother

\newtheoremstyle{brakets}% Name
  {}% space above
  {}% space below
  {\itshape}% body font
  {}% indent
  {\bfseries}% head font
  {.}% punctuation after head
  { }% space after head (has to be space or dimension!)
  {\thmheadbrackets{#1}{#2}{#3}}% head spec

\theoremstyle{brakets}

% \theoremstyle{plain}
\newtheorem{theorem}{Theorem}
\newtheorem{proposition}[theorem]{Proposition}
\newtheorem{lemma}[theorem]{Lemma}
\newtheorem{corollary}[theorem]{Corollary}
% \theoremstyle{definition}
\newtheorem{definition}[theorem]{Definition}
\newtheorem{assumption}[theorem]{Assumption}
\theoremstyle{remark}
\newtheorem{remark}[theorem]{Remark}

\definecolor{darkgreen}{rgb}{0.0, 0.5, 0.0}
\definecolor{darkblue}{rgb}{0.0, 0.5, 1.0}
% Todonotes is useful during development; simply uncomment the next line
%    and comment out the line below the next line to turn off comments
%\usepackage[disable,textsize=tiny]{todonotes}
% \usepackage[textsize=tiny]{todonotes}

% custom
% \usepackage{algorithm}
% \usepackage{algorithmicx}
% \usepackage{algpseudocode}
\newcount\Comments  % 0 suppresses notes to selves in text
\Comments = 1
\newcommand{\kibitz}[2]{\ifnum\Comments=1{\color{#1}{#2}}\fi}
\newcommand{\tw}[1]{\kibitzAdd{tw}{[Tonghan: #1]}}
\newcommand{\dcp}[1]{\kibitz{teal}{[DCP: #1]}}
\newcommand{\jf}[1]{\kibitzAdd{darkgreen}{[Jeff: #1]}}

\newcount\CommentsAdd  % 0 suppresses notes to selves in text
\CommentsAdd = 1
\newcommand{\kibitzAdd}[2]{\ifnum\CommentsAdd=1{\color{#1}{#2}}\fi}
\definecolor{english}{rgb}{0.0, 0.5, 0.0}
\definecolor{tw}{rgb}{0.0, 0.0, 0.5}
\newcommand{\dcpadd}[1]{\kibitzAdd{english}{#1}}
\newcommand{\twadd}[1]{\kibitz{blue}{#1}}
\newcommand{\jfadd}[1]{\kibitzAdd{blue}{#1}}


\usepackage{xcolor}
\newcommand\TODO[1]{\textcolor{red}{[TODO: #1]}}
\newcommand\CHANGE[1]{\textcolor{blue}{#1}}

%%%%% NEW MATH DEFINITIONS %%%%%

\usepackage{amsmath,amsfonts,bm}
\usepackage{derivative}
% Mark sections of captions for referring to divisions of figures
\newcommand{\figleft}{{\em (Left)}}
\newcommand{\figcenter}{{\em (Center)}}
\newcommand{\figright}{{\em (Right)}}
\newcommand{\figtop}{{\em (Top)}}
\newcommand{\figbottom}{{\em (Bottom)}}
\newcommand{\captiona}{{\em (a)}}
\newcommand{\captionb}{{\em (b)}}
\newcommand{\captionc}{{\em (c)}}
\newcommand{\captiond}{{\em (d)}}

% Highlight a newly defined term
\newcommand{\newterm}[1]{{\bf #1}}

% Derivative d 
\newcommand{\deriv}{{\mathrm{d}}}

% Figure reference, lower-case.
\def\figref#1{figure~\ref{#1}}
% Figure reference, capital. For start of sentence
\def\Figref#1{Figure~\ref{#1}}
\def\twofigref#1#2{figures \ref{#1} and \ref{#2}}
\def\quadfigref#1#2#3#4{figures \ref{#1}, \ref{#2}, \ref{#3} and \ref{#4}}
% Section reference, lower-case.
\def\secref#1{section~\ref{#1}}
% Section reference, capital.
\def\Secref#1{Section~\ref{#1}}
% Reference to two sections.
\def\twosecrefs#1#2{sections \ref{#1} and \ref{#2}}
% Reference to three sections.
\def\secrefs#1#2#3{sections \ref{#1}, \ref{#2} and \ref{#3}}
% Reference to an equation, lower-case.
\def\eqref#1{equation~\ref{#1}}
% Reference to an equation, upper case
\def\Eqref#1{Equation~\ref{#1}}
% A raw reference to an equation---avoid using if possible
\def\plaineqref#1{\ref{#1}}
% Reference to a chapter, lower-case.
\def\chapref#1{chapter~\ref{#1}}
% Reference to an equation, upper case.
\def\Chapref#1{Chapter~\ref{#1}}
% Reference to a range of chapters
\def\rangechapref#1#2{chapters\ref{#1}--\ref{#2}}
% Reference to an algorithm, lower-case.
\def\algref#1{algorithm~\ref{#1}}
% Reference to an algorithm, upper case.
\def\Algref#1{Algorithm~\ref{#1}}
\def\twoalgref#1#2{algorithms \ref{#1} and \ref{#2}}
\def\Twoalgref#1#2{Algorithms \ref{#1} and \ref{#2}}
% Reference to a part, lower case
\def\partref#1{part~\ref{#1}}
% Reference to a part, upper case
\def\Partref#1{Part~\ref{#1}}
\def\twopartref#1#2{parts \ref{#1} and \ref{#2}}

\def\ceil#1{\lceil #1 \rceil}
\def\floor#1{\lfloor #1 \rfloor}
\def\1{\bm{1}}
\newcommand{\train}{\mathcal{D}}
\newcommand{\valid}{\mathcal{D_{\mathrm{valid}}}}
\newcommand{\test}{\mathcal{D_{\mathrm{test}}}}

\def\eps{{\epsilon}}


% Random variables
\def\reta{{\textnormal{$\eta$}}}
\def\ra{{\textnormal{a}}}
\def\rb{{\textnormal{b}}}
\def\rc{{\textnormal{c}}}
\def\rd{{\textnormal{d}}}
\def\re{{\textnormal{e}}}
\def\rf{{\textnormal{f}}}
\def\rg{{\textnormal{g}}}
\def\rh{{\textnormal{h}}}
\def\ri{{\textnormal{i}}}
\def\rj{{\textnormal{j}}}
\def\rk{{\textnormal{k}}}
\def\rl{{\textnormal{l}}}
% rm is already a command, just don't name any random variables m
\def\rn{{\textnormal{n}}}
\def\ro{{\textnormal{o}}}
\def\rp{{\textnormal{p}}}
\def\rq{{\textnormal{q}}}
\def\rr{{\textnormal{r}}}
\def\rs{{\textnormal{s}}}
\def\rt{{\textnormal{t}}}
\def\ru{{\textnormal{u}}}
\def\rv{{\textnormal{v}}}
\def\rw{{\textnormal{w}}}
\def\rx{{\textnormal{x}}}
\def\ry{{\textnormal{y}}}
\def\rz{{\textnormal{z}}}

% Random vectors
\def\rvepsilon{{\mathbf{\epsilon}}}
\def\rvphi{{\mathbf{\phi}}}
\def\rvtheta{{\mathbf{\theta}}}
\def\rva{{\mathbf{a}}}
\def\rvb{{\mathbf{b}}}
\def\rvc{{\mathbf{c}}}
\def\rvd{{\mathbf{d}}}
\def\rve{{\mathbf{e}}}
\def\rvf{{\mathbf{f}}}
\def\rvg{{\mathbf{g}}}
\def\rvh{{\mathbf{h}}}
\def\rvu{{\mathbf{i}}}
\def\rvj{{\mathbf{j}}}
\def\rvk{{\mathbf{k}}}
\def\rvl{{\mathbf{l}}}
\def\rvm{{\mathbf{m}}}
\def\rvn{{\mathbf{n}}}
\def\rvo{{\mathbf{o}}}
\def\rvp{{\mathbf{p}}}
\def\rvq{{\mathbf{q}}}
\def\rvr{{\mathbf{r}}}
\def\rvs{{\mathbf{s}}}
\def\rvt{{\mathbf{t}}}
\def\rvu{{\mathbf{u}}}
\def\rvv{{\mathbf{v}}}
\def\rvw{{\mathbf{w}}}
\def\rvx{{\mathbf{x}}}
\def\rvy{{\mathbf{y}}}
\def\rvz{{\mathbf{z}}}

% Elements of random vectors
\def\erva{{\textnormal{a}}}
\def\ervb{{\textnormal{b}}}
\def\ervc{{\textnormal{c}}}
\def\ervd{{\textnormal{d}}}
\def\erve{{\textnormal{e}}}
\def\ervf{{\textnormal{f}}}
\def\ervg{{\textnormal{g}}}
\def\ervh{{\textnormal{h}}}
\def\ervi{{\textnormal{i}}}
\def\ervj{{\textnormal{j}}}
\def\ervk{{\textnormal{k}}}
\def\ervl{{\textnormal{l}}}
\def\ervm{{\textnormal{m}}}
\def\ervn{{\textnormal{n}}}
\def\ervo{{\textnormal{o}}}
\def\ervp{{\textnormal{p}}}
\def\ervq{{\textnormal{q}}}
\def\ervr{{\textnormal{r}}}
\def\ervs{{\textnormal{s}}}
\def\ervt{{\textnormal{t}}}
\def\ervu{{\textnormal{u}}}
\def\ervv{{\textnormal{v}}}
\def\ervw{{\textnormal{w}}}
\def\ervx{{\textnormal{x}}}
\def\ervy{{\textnormal{y}}}
\def\ervz{{\textnormal{z}}}

% Random matrices
\def\rmA{{\mathbf{A}}}
\def\rmB{{\mathbf{B}}}
\def\rmC{{\mathbf{C}}}
\def\rmD{{\mathbf{D}}}
\def\rmE{{\mathbf{E}}}
\def\rmF{{\mathbf{F}}}
\def\rmG{{\mathbf{G}}}
\def\rmH{{\mathbf{H}}}
\def\rmI{{\mathbf{I}}}
\def\rmJ{{\mathbf{J}}}
\def\rmK{{\mathbf{K}}}
\def\rmL{{\mathbf{L}}}
\def\rmM{{\mathbf{M}}}
\def\rmN{{\mathbf{N}}}
\def\rmO{{\mathbf{O}}}
\def\rmP{{\mathbf{P}}}
\def\rmQ{{\mathbf{Q}}}
\def\rmR{{\mathbf{R}}}
\def\rmS{{\mathbf{S}}}
\def\rmT{{\mathbf{T}}}
\def\rmU{{\mathbf{U}}}
\def\rmV{{\mathbf{V}}}
\def\rmW{{\mathbf{W}}}
\def\rmX{{\mathbf{X}}}
\def\rmY{{\mathbf{Y}}}
\def\rmZ{{\mathbf{Z}}}

% Elements of random matrices
\def\ermA{{\textnormal{A}}}
\def\ermB{{\textnormal{B}}}
\def\ermC{{\textnormal{C}}}
\def\ermD{{\textnormal{D}}}
\def\ermE{{\textnormal{E}}}
\def\ermF{{\textnormal{F}}}
\def\ermG{{\textnormal{G}}}
\def\ermH{{\textnormal{H}}}
\def\ermI{{\textnormal{I}}}
\def\ermJ{{\textnormal{J}}}
\def\ermK{{\textnormal{K}}}
\def\ermL{{\textnormal{L}}}
\def\ermM{{\textnormal{M}}}
\def\ermN{{\textnormal{N}}}
\def\ermO{{\textnormal{O}}}
\def\ermP{{\textnormal{P}}}
\def\ermQ{{\textnormal{Q}}}
\def\ermR{{\textnormal{R}}}
\def\ermS{{\textnormal{S}}}
\def\ermT{{\textnormal{T}}}
\def\ermU{{\textnormal{U}}}
\def\ermV{{\textnormal{V}}}
\def\ermW{{\textnormal{W}}}
\def\ermX{{\textnormal{X}}}
\def\ermY{{\textnormal{Y}}}
\def\ermZ{{\textnormal{Z}}}

% Vectors
\def\vzero{{\bm{0}}}
\def\vone{{\bm{1}}}
\def\vmu{{\bm{\mu}}}
\def\vtheta{{\bm{\theta}}}
\def\vphi{{\bm{\phi}}}
\def\va{{\bm{a}}}
\def\vb{{\bm{b}}}
\def\vc{{\bm{c}}}
\def\vd{{\bm{d}}}
\def\ve{{\bm{e}}}
\def\vf{{\bm{f}}}
\def\vg{{\bm{g}}}
\def\vh{{\bm{h}}}
\def\vi{{\bm{i}}}
\def\vj{{\bm{j}}}
\def\vk{{\bm{k}}}
\def\vl{{\bm{l}}}
\def\vm{{\bm{m}}}
\def\vn{{\bm{n}}}
\def\vo{{\bm{o}}}
\def\vp{{\bm{p}}}
\def\vq{{\bm{q}}}
\def\vr{{\bm{r}}}
\def\vs{{\bm{s}}}
\def\vt{{\bm{t}}}
\def\vu{{\bm{u}}}
\def\vv{{\bm{v}}}
\def\vw{{\bm{w}}}
\def\vx{{\bm{x}}}
\def\vy{{\bm{y}}}
\def\vz{{\bm{z}}}

% Elements of vectors
\def\evalpha{{\alpha}}
\def\evbeta{{\beta}}
\def\evepsilon{{\epsilon}}
\def\evlambda{{\lambda}}
\def\evomega{{\omega}}
\def\evmu{{\mu}}
\def\evpsi{{\psi}}
\def\evsigma{{\sigma}}
\def\evtheta{{\theta}}
\def\eva{{a}}
\def\evb{{b}}
\def\evc{{c}}
\def\evd{{d}}
\def\eve{{e}}
\def\evf{{f}}
\def\evg{{g}}
\def\evh{{h}}
\def\evi{{i}}
\def\evj{{j}}
\def\evk{{k}}
\def\evl{{l}}
\def\evm{{m}}
\def\evn{{n}}
\def\evo{{o}}
\def\evp{{p}}
\def\evq{{q}}
\def\evr{{r}}
\def\evs{{s}}
\def\evt{{t}}
\def\evu{{u}}
\def\evv{{v}}
\def\evw{{w}}
\def\evx{{x}}
\def\evy{{y}}
\def\evz{{z}}

% Matrix
\def\mA{{\bm{A}}}
\def\mB{{\bm{B}}}
\def\mC{{\bm{C}}}
\def\mD{{\bm{D}}}
\def\mE{{\bm{E}}}
\def\mF{{\bm{F}}}
\def\mG{{\bm{G}}}
\def\mH{{\bm{H}}}
\def\mI{{\bm{I}}}
\def\mJ{{\bm{J}}}
\def\mK{{\bm{K}}}
\def\mL{{\bm{L}}}
\def\mM{{\bm{M}}}
\def\mN{{\bm{N}}}
\def\mO{{\bm{O}}}
\def\mP{{\bm{P}}}
\def\mQ{{\bm{Q}}}
\def\mR{{\bm{R}}}
\def\mS{{\bm{S}}}
\def\mT{{\bm{T}}}
\def\mU{{\bm{U}}}
\def\mV{{\bm{V}}}
\def\mW{{\bm{W}}}
\def\mX{{\bm{X}}}
\def\mY{{\bm{Y}}}
\def\mZ{{\bm{Z}}}
\def\mBeta{{\bm{\beta}}}
\def\mPhi{{\bm{\Phi}}}
\def\mLambda{{\bm{\Lambda}}}
\def\mSigma{{\bm{\Sigma}}}

% Tensor
\DeclareMathAlphabet{\mathsfit}{\encodingdefault}{\sfdefault}{m}{sl}
\SetMathAlphabet{\mathsfit}{bold}{\encodingdefault}{\sfdefault}{bx}{n}
\newcommand{\tens}[1]{\bm{\mathsfit{#1}}}
\def\tA{{\tens{A}}}
\def\tB{{\tens{B}}}
\def\tC{{\tens{C}}}
\def\tD{{\tens{D}}}
\def\tE{{\tens{E}}}
\def\tF{{\tens{F}}}
\def\tG{{\tens{G}}}
\def\tH{{\tens{H}}}
\def\tI{{\tens{I}}}
\def\tJ{{\tens{J}}}
\def\tK{{\tens{K}}}
\def\tL{{\tens{L}}}
\def\tM{{\tens{M}}}
\def\tN{{\tens{N}}}
\def\tO{{\tens{O}}}
\def\tP{{\tens{P}}}
\def\tQ{{\tens{Q}}}
\def\tR{{\tens{R}}}
\def\tS{{\tens{S}}}
\def\tT{{\tens{T}}}
\def\tU{{\tens{U}}}
\def\tV{{\tens{V}}}
\def\tW{{\tens{W}}}
\def\tX{{\tens{X}}}
\def\tY{{\tens{Y}}}
\def\tZ{{\tens{Z}}}


% Graph
\def\gA{{\mathcal{A}}}
\def\gB{{\mathcal{B}}}
\def\gC{{\mathcal{C}}}
\def\gD{{\mathcal{D}}}
\def\gE{{\mathcal{E}}}
\def\gF{{\mathcal{F}}}
\def\gG{{\mathcal{G}}}
\def\gH{{\mathcal{H}}}
\def\gI{{\mathcal{I}}}
\def\gJ{{\mathcal{J}}}
\def\gK{{\mathcal{K}}}
\def\gL{{\mathcal{L}}}
\def\gM{{\mathcal{M}}}
\def\gN{{\mathcal{N}}}
\def\gO{{\mathcal{O}}}
\def\gP{{\mathcal{P}}}
\def\gQ{{\mathcal{Q}}}
\def\gR{{\mathcal{R}}}
\def\gS{{\mathcal{S}}}
\def\gT{{\mathcal{T}}}
\def\gU{{\mathcal{U}}}
\def\gV{{\mathcal{V}}}
\def\gW{{\mathcal{W}}}
\def\gX{{\mathcal{X}}}
\def\gY{{\mathcal{Y}}}
\def\gZ{{\mathcal{Z}}}

% Sets
\def\sA{{\mathbb{A}}}
\def\sB{{\mathbb{B}}}
\def\sC{{\mathbb{C}}}
\def\sD{{\mathbb{D}}}
% Don't use a set called E, because this would be the same as our symbol
% for expectation.
\def\sF{{\mathbb{F}}}
\def\sG{{\mathbb{G}}}
\def\sH{{\mathbb{H}}}
\def\sI{{\mathbb{I}}}
\def\sJ{{\mathbb{J}}}
\def\sK{{\mathbb{K}}}
\def\sL{{\mathbb{L}}}
\def\sM{{\mathbb{M}}}
\def\sN{{\mathbb{N}}}
\def\sO{{\mathbb{O}}}
\def\sP{{\mathbb{P}}}
\def\sQ{{\mathbb{Q}}}
\def\sR{{\mathbb{R}}}
\def\sS{{\mathbb{S}}}
\def\sT{{\mathbb{T}}}
\def\sU{{\mathbb{U}}}
\def\sV{{\mathbb{V}}}
\def\sW{{\mathbb{W}}}
\def\sX{{\mathbb{X}}}
\def\sY{{\mathbb{Y}}}
\def\sZ{{\mathbb{Z}}}

% Entries of a matrix
\def\emLambda{{\Lambda}}
\def\emA{{A}}
\def\emB{{B}}
\def\emC{{C}}
\def\emD{{D}}
\def\emE{{E}}
\def\emF{{F}}
\def\emG{{G}}
\def\emH{{H}}
\def\emI{{I}}
\def\emJ{{J}}
\def\emK{{K}}
\def\emL{{L}}
\def\emM{{M}}
\def\emN{{N}}
\def\emO{{O}}
\def\emP{{P}}
\def\emQ{{Q}}
\def\emR{{R}}
\def\emS{{S}}
\def\emT{{T}}
\def\emU{{U}}
\def\emV{{V}}
\def\emW{{W}}
\def\emX{{X}}
\def\emY{{Y}}
\def\emZ{{Z}}
\def\emSigma{{\Sigma}}

% entries of a tensor
% Same font as tensor, without \bm wrapper
\newcommand{\etens}[1]{\mathsfit{#1}}
\def\etLambda{{\etens{\Lambda}}}
\def\etA{{\etens{A}}}
\def\etB{{\etens{B}}}
\def\etC{{\etens{C}}}
\def\etD{{\etens{D}}}
\def\etE{{\etens{E}}}
\def\etF{{\etens{F}}}
\def\etG{{\etens{G}}}
\def\etH{{\etens{H}}}
\def\etI{{\etens{I}}}
\def\etJ{{\etens{J}}}
\def\etK{{\etens{K}}}
\def\etL{{\etens{L}}}
\def\etM{{\etens{M}}}
\def\etN{{\etens{N}}}
\def\etO{{\etens{O}}}
\def\etP{{\etens{P}}}
\def\etQ{{\etens{Q}}}
\def\etR{{\etens{R}}}
\def\etS{{\etens{S}}}
\def\etT{{\etens{T}}}
\def\etU{{\etens{U}}}
\def\etV{{\etens{V}}}
\def\etW{{\etens{W}}}
\def\etX{{\etens{X}}}
\def\etY{{\etens{Y}}}
\def\etZ{{\etens{Z}}}

% The true underlying data generating distribution
\newcommand{\pdata}{p_{\rm{data}}}
\newcommand{\ptarget}{p_{\rm{target}}}
\newcommand{\pprior}{p_{\rm{prior}}}
\newcommand{\pbase}{p_{\rm{base}}}
\newcommand{\pref}{p_{\rm{ref}}}

% The empirical distribution defined by the training set
\newcommand{\ptrain}{\hat{p}_{\rm{data}}}
\newcommand{\Ptrain}{\hat{P}_{\rm{data}}}
% The model distribution
\newcommand{\pmodel}{p_{\rm{model}}}
\newcommand{\Pmodel}{P_{\rm{model}}}
\newcommand{\ptildemodel}{\tilde{p}_{\rm{model}}}
% Stochastic autoencoder distributions
\newcommand{\pencode}{p_{\rm{encoder}}}
\newcommand{\pdecode}{p_{\rm{decoder}}}
\newcommand{\precons}{p_{\rm{reconstruct}}}

\newcommand{\laplace}{\mathrm{Laplace}} % Laplace distribution

\newcommand{\E}{\mathbb{E}}
\newcommand{\Ls}{\mathcal{L}}
\newcommand{\R}{\mathbb{R}}
\newcommand{\emp}{\tilde{p}}
\newcommand{\lr}{\alpha}
\newcommand{\reg}{\lambda}
\newcommand{\rect}{\mathrm{rectifier}}
\newcommand{\softmax}{\mathrm{softmax}}
\newcommand{\sigmoid}{\sigma}
\newcommand{\softplus}{\zeta}
\newcommand{\KL}{D_{\mathrm{KL}}}
\newcommand{\Var}{\mathrm{Var}}
\newcommand{\standarderror}{\mathrm{SE}}
\newcommand{\Cov}{\mathrm{Cov}}
% Wolfram Mathworld says $L^2$ is for function spaces and $\ell^2$ is for vectors
% But then they seem to use $L^2$ for vectors throughout the site, and so does
% wikipedia.
\newcommand{\normlzero}{L^0}
\newcommand{\normlone}{L^1}
\newcommand{\normltwo}{L^2}
\newcommand{\normlp}{L^p}
\newcommand{\normmax}{L^\infty}

\newcommand{\parents}{Pa} % See usage in notation.tex. Chosen to match Daphne's book.

\DeclareMathOperator*{\argmax}{arg\,max}
\DeclareMathOperator*{\argmin}{arg\,min}

\DeclareMathOperator{\sign}{sign}
\DeclareMathOperator{\Tr}{Tr}
\let\ab\allowbreak

\newcommand{\shortn}{\textup{\texttt{-}}}
\newcommand{\shorte}{\textup{\texttt{=}}}
\newcommand{\shortp}{\textup{\texttt{+}}}
\newcommand{\shortl}{\textup{\texttt{<}}}
\newcommand{\shortg}{\textup{\texttt{>}}}
\newcommand{\ie}{\textit{i}.\textit{e}.}
\newcommand{\eg}{\textit{e}.\textit{g}.}
\newcommand{\etal}{\textit{et al}.}
\newcommand{\etc}{\textit{etc}.}
\newcommand{\Tau}{\mathrm{T}}

\newcommand\VRule[1][\arrayrulewidth]{\vrule width #1}
\usepackage{multirow}
\usepackage{makecell}
\newcolumntype{L}{>{$}l<{$}}
\newcolumntype{C}{>{$}c<{$}}
\newcolumntype{R}{>{$}r<{$}}
\newcommand{\nm}[1]{\textnormal{#1}}

\newcommand{\name}{\textsc{BundleFlow}}
\newcommand{\bundle}{\texttt{Bundle-RochetNet}}
\newcommand{\bigbundle}{\texttt{Big-Bundle}}
\newcommand{\smallbundle}{\texttt{Small-Bundle}}
\newcommand{\grandbundle}{\texttt{Grand-Bundle}}

\begin{document}

% Title page for title and abstract only.
\begin{titlepage}

\maketitle\makeatletter \gdef\@ACM@checkaffil{} \makeatother
% Optionally include a table of contents
% \vspace{-0.1cm}
\setcounter{tocdepth}{2} % adjust to 1 if desired
% \tableofcontents

\end{titlepage}

\section{Introduction}

Deep Reinforcement Learning (DRL) has emerged as a transformative paradigm for solving complex sequential decision-making problems. By enabling autonomous agents to interact with an environment, receive feedback in the form of rewards, and iteratively refine their policies, DRL has demonstrated remarkable success across a diverse range of domains including games (\eg Atari~\citep{mnih2013playing,kaiser2020model}, Go~\citep{silver2018general,silver2017mastering}, and StarCraft II~\citep{vinyals2019grandmaster,vinyals2017starcraft}), robotics~\citep{kalashnikov2018scalable}, communication networks~\citep{feriani2021single}, and finance~\citep{liu2024dynamic}. These successes underscore DRL's capability to surpass traditional rule-based systems, particularly in high-dimensional and dynamically evolving environments.

Despite these advances, a fundamental challenge remains: DRL agents typically rely on deep neural networks, which operate as black-box models, obscuring the rationale behind their decision-making processes. This opacity poses significant barriers to adoption in safety-critical and high-stakes applications, where interpretability is crucial for trust, compliance, and debugging. The lack of transparency in DRL can lead to unreliable decision-making, rendering it unsuitable for domains where explainability is a prerequisite, such as healthcare, autonomous driving, and financial risk assessment.

To address these concerns, the field of Explainable Deep Reinforcement Learning (XRL) has emerged, aiming to develop techniques that enhance the interpretability of DRL policies. XRL seeks to provide insights into an agent’s decision-making process, enabling researchers, practitioners, and end-users to understand, validate, and refine learned policies. By facilitating greater transparency, XRL contributes to the development of safer, more robust, and ethically aligned AI systems.

Furthermore, the increasing integration of Reinforcement Learning (RL) with Large Language Models (LLMs) has placed RL at the forefront of natural language processing (NLP) advancements. Methods such as Reinforcement Learning from Human Feedback (RLHF)~\citep{bai2022training,ouyang2022training} have become essential for aligning LLM outputs with human preferences and ethical guidelines. By treating language generation as a sequential decision-making process, RL-based fine-tuning enables LLMs to optimize for attributes such as factual accuracy, coherence, and user satisfaction, surpassing conventional supervised learning techniques. However, the application of RL in LLM alignment further amplifies the explainability challenge, as the complex interactions between RL updates and neural representations remain poorly understood.

This survey provides a systematic review of explainability methods in DRL, with a particular focus on their integration with LLMs and human-in-the-loop systems. We first introduce fundamental RL concepts and highlight key advances in DRL. We then categorize and analyze existing explanation techniques, encompassing feature-level, state-level, dataset-level, and model-level approaches. Additionally, we discuss methods for evaluating XRL techniques, considering both qualitative and quantitative assessment criteria. Finally, we explore real-world applications of XRL, including policy refinement, adversarial attack mitigation, and emerging challenges in ensuring interpretability in modern AI systems. Through this survey, we aim to provide a comprehensive perspective on the current state of XRL and outline future research directions to advance the development of interpretable and trustworthy DRL models.
\section{Preliminaries}

\textbf{Sealed-Bid Combinatorial Auction}. We consider sealed-bid CAs with a single bidder and $m$ items, $M=\{1,\ldots,m\}$.
The bidder has a {\em valuation function}, $v: 2^M \rightarrow \mathbb{R}_{\ge 0}$. Valuation $v$ is drawn independently from a distribution $F$ defined on the space of possible valuation functions $V$, determining how valuable each bundle $S\in 2^M$ is for the bidder. We consider bounded valuation functions: $v(S)\in[0, v_{\max}]$, $S\subset 2^M$, with $v_{\max}>0$, and they are normalized so that $v(\varnothing)=0$.
%
%\forall v_i\in \supp(F_i)$} \dcp{i just realized that we need a bounded domain $V_i$ for a grid to be well defined, right? comment on this.}
% We use $\vv$ to denote the value of the bidder for each of the $2^m$ bundles. 
The auctioneer  knows distribution $F$ but not the  valuation  $v$. The bidder reports their valuation function, perhaps untruthfully, as their {\em bid (function)}, $b\in V$. 

In CAs, a suitable {\em bidding language} is critical to allow a bidder to report their
bid without needing to enumerate a value for every possible bundle.  There
are many ways to do this, but a  common approach is to use the {\em XOR bidding language}, which allows bidders to submit bid prices for each of multiple bundles under an exclusive-or condition; in effect, only one bid price on a bundle can be accepted. Popular CA testbeds such as CATS~\citep{leyton2000towards} and SATS~\citep{weiss2017sats} employ this bidding language extensively.\footnote{When representing the values of multiple bidders these testbeds often also introduce so-called dummy items for distinguishing the bids of different 
bidders. Still, the semantics for a single bidder is, in effect, that of the XOR language.}
The semantics of the XOR bidding language is that the value on a bundle $S$ is the maximum bid price on
any bundle $S'$, submitted as part of the XOR bid, and for which $S'\subseteq S$. XOR bids are 
succinct for valuation functions in which the bidder is only interested in a bounded number
of possible bundles.

We seek an auction $(g,p)$ that maximizes expected revenue. Here, $g: V\rightarrow \mathcal{X}$ is the {\em allocation rule},  where $\mathcal{X}$ is the space of feasible allocations (i.e., no item allocated more than once), so that $g(b)\subseteq M$ denotes the set of items (perhaps empty) allocated to the bidder at bid $b$.
%
Also, $p: V\rightarrow \mathbb{R}_{\ge 0}$ is the {\em payment rule},
specifying the price associated with allocation $g(b)$. 
 %
The utility to the bidder with valuation function $v$ at bid  $b$ is $u(v;b)=v(g(b))-p(b)$, which is
the standard model of quasi-linearity so that values are in effect quantified in monetary units, say dollars.
%
 In full generality, the allocation and payment rules may be \emph{randomized}, with
 the bidder assumed to be risk neutral  and seeking
to maximize their 
expected utility.


In a \emph{dominant-strategy incentive compatible} (DSIC) auction, or {\em strategy-proof (SP)} auction, the bidder's utility is  maximized by bidding their true valuation $v$, whatever this valuation is; i.e., $u(v; v)\ge u(v;b)$, for $\forall v\in V, \forall b\in V$.
An auction is \emph{individually rational} (IR) if the bidder receives a non-negative utility when participating and truthfully reporting: $u(v;v)\ge 0$, for $\forall v\in V$.
Following the revelation principle, it is without loss of generality to focus on  
SP
auctions, as any auction that achieves a particular expected revenue in a dominant-strategy equilibrium
 can be transformed into an SP auction with the same expected revenue.
 %
 Optimal auction design therefore seeks to identify an SP and IR auction that maximizes the expected revenue, i.e., $\mathbb{E}_{v\sim \bm F}[p(v)]$. 

\textbf{Menu-Based CAs}. In a {\em menu-based auction}, allocation and payment rules  are
represented through a menu, $B$, consisting of
$K\ge 1$  {\em menu elements}.
%
We write $B=(B^{(1)},\ldots,B^{(K)})$, 
and the $k$th \emph{menu element}, $B^{(k)}$,
 specifies allocation probabilities on bundles,
 $\alpha^{(k)}: 2^M\rightarrow [0,1]$, and a {\em price}, $\beta^{(k)}\in \mathbb{R}$.
%
Here, we allow randomization, where  $\alpha^{(k)}(S)\in[0,1]$ denotes the
  probability that bundle $S\in 2^M$ is assigned to the bidder in menu element $k$. 
  % this assignment made independently of the 
  % assignment of some other item $j'\neq j$ (conditioned on the choice of
  % menu element $k$).
  %
   % \dcp{i'm wondering if this independent distribution is wlog for additive and unit-demand valuations---you can ask Sai and Michael if they know about anything.}
   %
  %
We refer to the menu $B$ as corresponding to a {\em menu-based representation 
of an auction.} The bidder with bid $b$ is assigned the element from menu $B$ that maximizes their utility according to the reported valuation: $k^*\in \arg\max_k \sum_{S\in 2^M}\alpha^{(k)}(S)b(S)-\beta^{(k)}$. We denote this optimal element by $(\alpha^*(b), \beta^*(b))$. 
The use of menu-based representations for auction design 
is without loss of generality and DSIC~\cite{hammond1979straightforward}.
%\dcp{need to explain how the allocation and payment rule is defined, via arg max given bid b and the menu} \dcp{need a bit more here, defining $\beta^*$ as the optimal element from menu $B$}
%
%In the context of menu-based auction design, the 
The optimal auction design problem is to find a menu-based representation that 
maximizes  expected revenue, i.e., $\mathbb{E}_{v\sim F}[ \beta^{*}(v)]$. Deep menu-based methods~\cite{dutting2024optimal,shen2019automated} in the differentiable economics literature~\cite{zheng2022ai,finocchiaro2021bridging,wang2023deep,ivanov2024principal,zhang2024position,hossain2024multi,rahme2020auction,ivanov2022optimal,curry2022differentiable,duan2023scalable} learn to generate such menus by neural networks.
%dcp cut among all  menus of a certain size.

% \tw{which means the number of elements in a menu?} \dcp{yes; now I say `among all menus` not `among all menu representations`. ok?}.  


\textbf{Diffusion Models and Continuous Normalizing Flow}. Diffusion models have  emerged as a powerful class of generative AI methods, spurring notable advances in a wide range of tasks such as image generation~\cite{rombach2022high,esser2024scaling}, video generation~\cite{ho2022video,ceylan2023pix2video,ho2022imagen}, molecular design~\cite{gruver2024protein}, text generation~\cite{lou2024discrete}, and multi-agent learning~\cite{wang2024diffusion}. At their core, these models perform a {\em forward noising process} in which noise is incrementally added to training data over multiple steps, gradually corrupting the original sample.
%\dcp{is `signal` right? or `sample`?} \tw{My opinion is that these works are rooted in the information theory, so they use "signal" a lot.} it's only used here in our paper, so I have dropped `signal`
%
A {\em reverse diffusion process} is then learned to iteratively remove noise, thereby reconstructing data from near-random initial states. In our setting, instead of reconstructing data, we extend the diffusion process to develop a tractable and differentiable method that optimizes a high-dimensional distribution.
%\tw{Do we need this here?}
%dcp -- yes, I like it

In particular, {\em score-based diffusion models} enjoy strong mathematical and physical underpinnings. The forward noising process is an {\em Itô stochastic differential equation} (SDE),
\begin{align}
    d\vx = \vf(\vx,t)dt + h(t)d\vw,
\end{align}
%
where $\vx(t) \in\mathbb{R}^\ell$ is the {\em state} at time $t$, for some $\ell\in \mathbb{Z}_{>0}$, $\vf(\cdot,t):\mathbb{R}^\ell\rightarrow\mathbb{R}^\ell$ is the {\em drift coefficient}, $h(\cdot):\mathbb{R}\rightarrow\mathbb{R}$ is the {\em diffusion coefficient}, and $\vw$ is the {\em standard Wiener process} (Brownian motion). Different forward processes are designed by specifying functional forms for $\vf(\cdot,t)$ and $h(\cdot)$. The generation of data is then based on the reverse process, which is  a diffusion process  given by the {\em reverse-time SDE}~\cite{anderson1982reverse},
%
\begin{align}
    d\vx = [\vf(\vx,t)-h(t)^2\nabla_\vx\log q_t(\vx)]dt + h(t)d\bar{\vw},\label{equ:r-sde}
\end{align}
%
where $dt$ is an infinitesimal negative timestep,  $\bar{\vw}$ is the {\em standard Brownian motion with reversed time flow},  and {\em $q_t(\vx)$ is the distribution of state  $\vx(t)$
at time $t$}.
The principal task in diffusion models is to learn the {\em score function}, $\nabla_\vx\log q_t(\vx)$, which has been effectively achieved using neural networks in recent work. This
enables solving the reverse-time SDE and  generating new data samples.  Notably, in the diffusion model  (and more broadly, generative AI) literature, $q_0(\vx)$ is typically a known target distribution over data samples from a pre-training dateset.
%\dcp{as $\vx_T$ given samples $\vs_0$?}

The reverse-time SDE (Eq.~\ref{equ:r-sde}) can be mathematically intricate, motivating the study of an equivalent, \emph{deterministic reverse process} modeled by an ordinary differential equation (ODE),
%
\begin{align}
    d\vx = [\vf(\vx,t)-\frac{1}{2}h(t)^2\nabla_\vx\log q_t(\vx)]dt,\label{equ:r-ode}
\end{align}
%
which  preserves the same marginal probability densities $\{q_t(\vx)\}_{t=0}^T$ as the SDE in Eq.~\ref{equ:r-sde}~\cite{song2021score}.
%
% Drawbacks of the normalizing flow include that we require the function $f$ to be invertible, and working its Jacobian might be expensive. We can consider a continuous version of normalizing flow where the function $f$ is applied an infinite time. 
%
Eq.~\ref{equ:r-ode} also highlights the connection between diffusion models and \emph{continuous normalizing flow}: each of them learns to transform and manipulate distributions by an ODE. Intuitively, a {\em continuous normalizing flow} transports an input $\vx_0\in \mathbb{R}^\ell$ to $\vx_t=\phi(t, \vx_0)$ at timestep $t\in[0,T]$.
%
Here, $\phi(t, \cdot):\mathbb{R}^\ell\rightarrow\mathbb{R}^\ell$ is the  \emph{flow}, and is governed by the ODE,
%
\begin{align}
    \frac{d}{dt}\vx_t = \varphi\left(t, \vx_t\right),\label{equ:f-ode}
\end{align}
%
where the vector field $\varphi: [0,T]\times \mathbb{R}^\ell\rightarrow \mathbb{R}^\ell$ specifies the  rate of
change of the state $\vx$.
%
Continuous normalizing flow \citep{chen2018neural} suggests to represent vector field $\varphi$ with a neural network. The flow $\phi$ transforms an initial distribution $p_0(\vx)$ to a final distribution $p_T(\vx)$ an time $T$.

% and the final distribution $p(z)=p_1(z_1)$ can be obtained by solving .

\textbf{Rectified Flow}. A bottleneck that restricts the use of continuous normalizing flow in large-scale problems is that the ODE (Eq.~\ref{equ:f-ode})
is hard to solve when the vector field $\varphi$ is complex.  The {\em rectified flow}~\cite{liu2022flow} addresses this by encouraging the flow to follow the linear path:
\begin{align}
    \min_\varphi \int_0^T \mathbb{E}_{\vx_0\sim p_0(\vx),\vx_T\sim p_T(\vx)}\left[\|(\vx_T-\vx_0)-\varphi(t, \vx_t)\|^2\right]dt, \ \ \ \vx_t = t\vx_T + (1-t)\vx_0.\label{equ:rf}
\end{align}

Here, the target distribution $p_T(\vx)$ (from which $\vx_T$ are sampled) and the initial distribution $p_0(\vx)$ (from which $\vx_0$ are sampled) are known. $\vx_t,t\in[0,T]$ is the interpolated point between $\vx_T$ and $\vx_0$, and the rectified flow encourages the vector field to align as closely as possible with the straight line $\vx_T-\vx_0$.

As discussed in the introduction, the application of diffusion models or continuous normalizing flow in generative AI tasks relies on access to a known target distribution $p_T(\vx)$, but in our optimal CA design task, $p_T(\vx)$ is unknown and needs to be optimized. 


% \dcp{same comment, do you want int from $0$ to $T$, and $Z_T$ not $Z_1$?}

% A rectified flow converting $X_0\sim p_0(\vx)$ to $X_T\sim p_1(z_1)$ \dcp{do you want $Z_T, p_T, z_T$ here?} is an ODE
% %
% \begin{align}
%     dZ_t = \varphi(t, Z_t) dt,
% \end{align}
%
% where $\varphi(t, Z_t)$ \dcp{earlier $\varphi(t, z_t)$}
% is trained to drive  $(Z_1-Z_0)$:
\section{The Flow-Based Combinatorial Auction Menu Network}

As discussed in the introduction, the major challenge in learning menus for CAs is 
to provide an expressive representation of distributions over bundles
to associate with each menu element while retaining efficiency, so that
the exponential number of possible bundles does not become a bottleneck.
%dcp cut Given that the number of bundles grows exponentially in the number of items, this requirement greatly increases the complexity of the menu representation. 
Moreover, training these representations adds another layer of difficulty: the menu must be not only concise but also easily differentiable to support training. 

\subsection{Menu representation}

Our key idea, following from score-based diffusion models and continuous normalizing flow, is to construct a concise and differentiable representation of a bundle distribution by modeling it through the solution of an ordinary differential equation (ODE). Specifically, the $k$th menu element generates its bundle distribution by the ODE,
%
\begin{equation}
    d\vs^{(k)}_t = \varphi^{(k)}(t,\vs^{(k)}_t) dt,\label{equ:de}
\end{equation}
%
for a suitable choice of vector field $\varphi^{(k)}$.
%
Here, we refer to $\vs^{(k)}_t\in \mathbb{R}^m$ as the \emph{bundle variable at time $t$}, where $m$ is the number of items. At time $T$, we require that a bundle variable $\vs^{(k)}_T$ represents a meaningful bundle, so that all entries are 0s or 1s,
and we adopt $\alpha^{(k)}_T(\vs^{(k)}_T)$ to denote the corresponding allocation probability. 
For simplicity, we omit the superscript $(k)$ when this is clear from the context.

By the  Liouville equation~\cite{liouville1838note}, the probability density at $T$ derived from Eq.~\ref{equ:de} satisfies:
\begin{align}
    \log \alpha_T(\vs_T) = \log \alpha_0(\vs_0) - \int_0^T \nabla\cdot \varphi(t, \vs_t) dt,\label{equ:liouville}
\end{align}
where $\alpha_0(\vs_0)$ denotes the initial distribution at time $0$, on initial bundle variables $\vs_0$, and $\nabla\cdot \varphi(t, \vs_t)$ is the divergence of $\varphi$.   Eq.~\ref{equ:liouville} is applicable to any $\vs_0$, and a bundle variable $\vs_t$ is generated from $\vs_0$ by $\vs_t(\vs_0)=\vs_0+\int_0^t \varphi(\tau,\vs_\tau) d\tau$. For clarity, we omit the explicit dependence on $\vs_0$ and simply write $\vs_t$.
%\dcp{as per my comment in the intro, it may be worth being a bit more pedantic here, and explaining that this can be applied to any $s_0$, and that $s_T$ is the rv at time $T$ that corresponds to this particular $s_0$.} 

\textbf{Training scheme}. Both the vector field $\varphi$ and the initial distribution $\alpha_0$ can influence the final distribution $\alpha_T$. 
Our method proceeds in two stages, involving the training of  each of
these two components in turn: 

$\quad$ \emph{(1) Flow Initialization.} We  fix  the initial distribution $\alpha_0$ and train the vector field $\varphi(t, \cdot)$ so that the final distribution, $\alpha_T$,
provides a reasonable coverage over bundles. 

$\quad$ \emph{(2) Menu Optimization.} We  fix the vector field from Stage 1, and backpropagate the revenue-maximizing loss through the flow to update the initial distribution $\alpha_0^{(k)}$ 
for each menu element $k$.

% by learning the weights $w_d^{(k)}$ and means $\bm\mu_d^{(k)}$ associated with the menu element. 

$\varphi$ and $\alpha_0(\vs_0)$ play a crucial role in maintaining a concise and easily differentiable representation and ensuring efficient training. We next propose specific functional forms for these two components that meet these criteria.

\textbf{Vector field}. We adopt the following functional form for the vector field,
%
\begin{align}
    \varphi(t, \vs_t;\xi,\theta) = \eta(t;\xi) Q(\vs_0;\theta) \vs_t,\label{equ:varphi}
\end{align}
where $Q: \mathbb{R}^{m}\rightarrow\mathbb{R}^{m\times m}$, written as a function of $\vs_0$, and the scalar factor $\eta:\mathbb{R}\rightarrow \mathbb{R}$, written as a function of  the ODE time $t\in[0,T]$, are neural networks with learnable parameters $\theta$ and $\xi$, respectively. 
%
%
We  omit dependence on $\theta$ and $\xi$ when the context is clear. This formulation's advantage becomes apparent when we consider its divergence:
%
\begin{align}
    \nabla\cdot \varphi(t,\vs_t) &= \sum_{i=1}^m \frac{\partial \varphi_i}{\partial s_{t,i}}
    = \sum_{i=1}^m \frac{\partial}{\partial s_{t,i}} \eta(t) Q_i(\vs_0) \vs_t = \sum_{i=1}^m  \eta(t) Q_{ii}(\vs_0) \\
    & = \eta(t)\Tr[Q(\vs_0)].
\end{align}

Here, $\varphi_i$ and $s_{t,i}$ are the $i$th element of $\varphi$ and $\vs_t$, respectively, $Q_i$ is the $i$th row of $Q$, and $Q_{ii}$ is the $i$th diagonal element of $Q$. Thus, the probability density at $T$ becomes
\begin{align}
    \log \alpha_T(\vs_T) = \log \alpha_0(\vs_0) - \Tr[Q(\vs_0)]\int_0^T \eta(t) dt.\label{equ:likelihood}
\end{align}

The integral in Eq.~\ref{equ:likelihood} 
is tractable as it only involves a scalar function, instead of bundle variables.
We can efficiently estimate this integral by time discretization. 

\textbf{Initial distribution}. In Stage 1, we use a mixture-of-Gaussian distribution for
the initial distribution $\alpha_0(\vs_0)$ on bundle variables $\vs_0$, with
%
\begin{align}
    \vs_0 \sim\sum_{d=1}^D w_d \mathcal{N}(\bm\mu_d, \sigma_d^2\mI_m),\label{equ:init_dist}
\end{align}
where, for $D$ components,
$\bm\mu_d\in\mathbb{R}^m$, $\sigma_d\in\mathbb{R}_{>0}$, $\mI_m$ is the $m\times m$ identity matrix, and $w_d\geq 0$ are weights satisfying $\sum_{d=1}^D w_d=1$. In Stage 2, as discussed later, we ensure DSIC by adopting a mixture-of-Dirac distribution, which is practically implemented by setting a very small variance $\sigma_d$ in a mixture-of-Gaussian distribution.

% (1,1,\cdots,1)^\Tau We expect the flow to transport a sample $z_0$ to a bundle $S\in\{0,1\}^m$.

\subsection{Stage 1: Flow initialization}

The aim of the first stage is to guarantee that the flow can transport any initial bundle variable $\vs_0$ to a feasible bundle $S\in 2^M$.  We use $\vs=(\mathbb{I}{\{i\in S\}})$ to denote the vectorization of set $S$, i.e., the $i$-th component of $\vs$ is 1 if item $i$ is in $S$ and 0 otherwise.

In practice, numerical issues make it challenging to exactly obtain an feasible bundle $\vs$; i.e., a bundle variable
with only 0s and 1s. To account for this, we allow a small region around $\vs$ to be approximated as $\vs$ by modeling the bundle as a Gaussian variable,
%
\begin{align}
    S_{\sigma_z} = \mathcal{N}(\vs, \sigma_z^2\mI_m).
\end{align}
% the random variable representing the bundle variables at $t=0$.  where $Z_0$ is 

We train the vector field networks using rectified flow (\citet{liu2022flow}, Eq.~\ref{equ:rf}). For this  stage, we fix the initial distribution $\alpha_0(\vs_0)$ to a mixture-of-Gaussian model $\alpha_0(\vs_0)=\sum_{d=1}^D w_d \mathcal{N}(\bm\mu_d, \sigma_d^2\mI_m)$ with $D$ components.
We  define
$\alpha_T(\vs_T)$ as a uniform mixture-of-Gaussian model, with  components centered around each feasible bundle, and
$\alpha_T(\vs_T)=\frac{1}{2^m}\sum_{S\in 2^M} \mathcal{N}(\vs, \sigma_z^2\mI_m)=\frac{1}{2^m}\sum_{S\in 2^M}S_{\sigma_z}$.
This target distribution only applies in Stage 1, where it serves to encourage a balanced coverage of the final distribution over feasible bundles. In Stage 2, we have an optimization problem, and there is no longer a fixed target distribution.
%

We  follow the idea of  rectified flow, and define the {\em flow training loss} as
%
\begin{align}
    \mathcal{L}_{\textsc{Flow}}(\theta,\xi) =& \mathbb{E}_{(\vs_0,\vs_T)\sim (\alpha_0,\alpha_T), t\sim [0,T]} \left[\|(\vs_T-\vs_0)-\varphi(t, \vs_t; \theta,\xi)\|^2\right], \label{equ:flow_loss}\ \ \mbox{where}\\
    & \vs_t = t\cdot \vs_T + (1-t)\cdot \vs_0,\\
    & \varphi(t, \vs_t; \theta,\xi)=\eta(t;\xi)Q(\vs_0;\theta)\vs_t.
\end{align}

% \dcp{i think you want $\vs_T$ not $\vs_1$ in eqn 16}
This loss is used to update the neural networks $Q$ and $\eta$ to encourage the vector field at interpolated points $\vs_t$ to point from $\vs_0$ to $\vs_T$.
%\dcp{I think you can say a bit more here---you've defined $Z_1$ to be a uniform distr over all bundles, so this is trying to get coverage of bundles.}
%
The expectation in the flow training loss is taken over $(\alpha_0,\alpha_T)$, 
but directly sampling from $\alpha_T$ is intractable as it involves $2^m$ bundles.

Crucially, using a flow-based representation provides a workaround. We first draw $\vs_0\sim \alpha_0$, which is straightforward given that $\alpha_0$ comprises a manageable number of components ($D$). We then round $\vs_0$ to
the nearest feasible bundle, $\vs=\mathbb{I}(\vs_0\ge 0.5)\in\{0,1\}^m$,
and sample $\vs_T\sim \mathcal{N}(\vs, \sigma_z^2\mI_m)$. This approach underscores an advantage of deep learning. Although we cannot enumerate all possible bundles, the generalization ability of neural networks allows for learning the mapping from $\alpha_0$ to $\alpha_T$ given enough training samples.


\subsection{Stage 2: Menu optimization}\label{sec:method:opt}

In the second stage, we train the menu to seek to maximize the expected revenue for the auctioneer. For each menu element $k$,  the 
trainable parameters comprise the price $\beta^{(k)}$, as well as the parameters $w_d^{(k)}$ and $\bm\mu_d^{(k)}$ that define the initial distribution $\alpha^{(k)}_0$ on the bundle variable. 
The vector field $\varphi$ is  fixed in this stage and shared
among all menu elements.

Given a bidder with a value function $v$, the payment to the auctioneer is the price associated with the menu element that provides the highest utility to the bidder. 
Thus, computing the utility of each menu element is central to evaluating the revenue objective.
We always maintain a null menu element (zero allocation, zero price), which ensures 
  individual rationality (IR), so that the bidder has  non-negative expected
utility. 

Computing the expected 
utility corresponding to a menu element with bundle distribution $\alpha^{(k)}$ 
is intractable when done with a direct calculation,
because
%
\begin{equation}
    u^{(k)}(v) = \sum_{S\in 2^M} \alpha^{(k)}(S)v(S)
\end{equation}
%
requires enumerating $2^m$ bundles for a general valuation function.
However, with our flow-based representation, we can get the bundle allocation probabilities by applying the flow to the initial distribution. Specifically, we have
\begin{align}
    u^{(k)}(v) = \mathbb{E}_{\vs_0\sim \alpha^{(k)}_0, \vs=\mathbb{I}(\phi(T,\vs_0)\ge 0.5)} \left[v(\vs) \alpha^{(k)}_0(\vs_0)\exp\left(-\Tr[Q(\vs_0)]\int_0^T \eta(t) dt]\right)\right],\label{equ:u}
\end{align}
by applying the exponential operation to both sides of Eq.~\ref{equ:likelihood}. Here, $\phi(T,\vs_0)$ is the solution of the ODE solved by forward Euler,
%
\begin{align}
    \phi(T,\vs_0) = \vs_0 + Q(\vs_0)\int_0^T \eta(t)\vs_t dt,\label{equ:phi_T_s_0}
\end{align}
%
and $\vs=\mathbb{I}(\phi(T,\vs_0)\ge 0.5)$ is the rounded final bundle. Due to its simple form, a modern ODE solver can efficiently solve the ODE (Eq.~\ref{equ:phi_T_s_0}) in just a few steps. Therefore, the calculation of $u^{(k)}(v)$ becomes tractable when we make the initial distribution simple.

% Here we see another advantage of our choice for the functional form of the vector field $\phi$ (Eq.~\ref{equ:varphi}): $u^{(k)}(v)$ can be determined by the initial distribution, without calculating bundle variables at $t>0$.  Moreover, 
% \tw{TODO: double check the reference.}  \dcp{XX STILL TO DO! XX}
% designed for diffusion models 
%

To ensure DSIC, we need to accurately calculate the expectation in Eq.~\ref{equ:u} to get the exact
utility to the bidder. We accomplish this by employing a mixture-of-Dirac distribution as the initial distribution, which has finite support. To implement this in practice, we set, for Stage 2 only, a very small variance to the Gaussian components in Eq.~\ref{equ:init_dist}, with $\sigma_d=1e\shortn 20$ for every component $d$. In this way, the utility can be obtained by enumerating over the finite support of the initial distribution:
% In this way, each Gaussian in the mixture is effectively a Dirac delta, 
\begin{align}
    u^{(k)}(v) = \sum_{d=1}^D \left[v(\vs(\bm\mu^{(k)}_d)) \alpha^{(k)}_0(\bm\mu^{(k)}_d)\exp\left(-\Tr[Q(\bm\mu^{(k)}_d)]\int_0^T \eta(t) dt]\right)\right],\label{equ:u_finite}
\end{align}
where $\vs(\bm\mu^{(k)}_d)=\mathbb{I}(\phi(T,\bm\mu^{(k)}_d)\ge 0.5)$. 
That is, the support of $\alpha^{(k)}_0$ 
consists, in effect, of the set of means, one for each component. %\dcp{this is where I think we could just say this is a mixture-of-Dirac and avoid this small variance mix-of-Gaussian discussion}. \tw{In Sec. 3.1, before we introduce Stage 1 and 2, we said the initial distribution is mixture-of-Gaussian. I suggest a minimal modification (by setting different variance in the two stages) here, but am willing to rework both Sec. 3.1 and 3.3.} 
It is worth noting that $D$ in Eq.~\ref{equ:u_finite} does not need to be the same $D$ as in Stage 1, and it could even vary across menu elements.
% \dcp{also, is it worth commenting that this $D$ does not need to be the same $D$ as in Stage 1? I suppose in principle it could even vary across elements...}
%\dcp{a bit confusing, at least to me --- I don't see where you sample different weighted combinations of the means?} \tw{$\alpha_0(\bm\mu^{(k)}_d)$ in Eq.~\label{equ:u_finite}?}

In this Stage 2, we fix the vector field $\varphi$ ($Q$ and $\eta$ networks) in Eq.~\ref{equ:u_finite} and update trainable parameters associated with the price and 
initial distribution $\alpha_0^{(k)}$ for each menu element $k$ 
during menu optimization, \ie, $\beta^{(k)}$, $\{w^{(k)}_d\}_{d=1}^{D}$, and $\{\bm\mu^{(k)}_d\}_{d=1}^{D}$. Therefore, given a set of bidder valuations $\mathcal{V}$,
the {\em revenue-maximization loss} is defined as
%
\begin{align}
    \mathcal{L}_{\textsc{Rev}}\left(\{\beta^{(k)}\}_{k=1}^K, \bigl\{ w_d^{(k)} \bigr\}_{\substack{d\in[D] \\ k\in[K]}},\bigl\{\bm\mu_d^{(k)} \bigr\}_{\substack{d\in[D] \\ k\in[K]}}\right) = -\frac{1}{|\mathcal{V}|}\sum_{v\in \mathcal{V}}\left[\sum_{k\in[K]}z^{(k)}(v)\beta^{(k)} \right],
\end{align}
%
where $z^{(k)}(v)$ is obtained by applying the differentiable SoftMax function to the utility of the bidder being allocated the $k$-th menu choice, i.e.,
%
\begin{align}
    z^{(k)}(v) = \mathsf{SoftMax}_k\left(\lambda_{\textsc{SoftMax}}\cdot u^{(1)}(v),\ldots,\lambda_{\textsc{SoftMax}}\cdot u^{(K)}(v)\right),\label{equ:softmax_in_loss}
\end{align}
%
where $\lambda_{\textsc{SoftMax}}$ is a scaling factor, and $u^{(k)}(v)$ is calculated by Eq.~\ref{equ:u_finite}.
%\dcp{do you want $\mathsf{SoftMax}_K$ not $\mathsf{SoftMax}_k$?} \tw{I used the notation of the JACM paper. $k$ acts like an index here?}
When optimizing $\mathcal{L}_{\textsc{Rev}}$, the gradients with respect to $\beta^{(k)}$ are straightforward to compute. Moreover, although $Q$ remains fixed, gradients can still backpropagate through this network to update its input,
which is $\bm\mu^{(k)}_d$. Gradients also flow through $z^{(k)}$ back into $\alpha_0$, enabling updates to the mixture weights $w^{(k)}_d$. All these gradients are automatically handled by standard deep learning frameworks.



% The probability of
% \begin{align}
%     \bm\alpha^{(k)}_{ij} = \sum_{z_1=f(z_0)=S_j} D_0(z_0)
% \end{align}

\subsection{Discussion}\label{sec:multi-bidder}

%\dcp{mention somewhere, perhaps here, that softmax becomes hardmax at test time to give DSIC}

\textbf{DSIC}. The seminal work by \citet{hammond1979straightforward} establishes necessary and sufficient conditions for a strategyproof menu-based auction: (1) Self-bid independent: the menu is independent of the bidder's bid; (2) Agent-optimizing: the bidder is assigned the menu element that maximizes their utility. As we analyze here, our method satisfies these two properties.

In \name, all element prices, as well as bundle allocations, which depend on initial distributions and the vector field, are trained on values sampled from the distribution $F$, without using any information about the bidder's specific valuation. Therefore, menus learned by \name~are self-bid independent. 
%
As discussed in Sec.~\ref{sec:method:opt}, we require the initial distribution for each menu
element to have finite support, which means that the bundle distribution for each menu element 
can be reconstructed without any approximation error.
This guarantees exact utility calculation for every menu element. Moreover, unlike the SoftMax in Eq.~\ref{equ:softmax_in_loss}, we use hard argmax at test time, thereby selecting the menu element with the highest utility to the bidder. In this way, \name\ is strictly agent-optimizing.

\textbf{Expressiveness}. In Stage 1, we initialize the vector field $\varphi$. After this stage, given appropriate initial distributions, the final distribution can in principle cover all $2^m$ bundles and is trained to seek to achieve this.
In Stage 2, since the initial distribution for a menu element has finite support of size $D$, the bundle distribution for a menu element is also limited to finite support of size $D$. 
What is crucial, though, is that we can learn which (up to) $D$ bundles are represented in the distribution
that corresponds to a menu element. In practice, we find that a bounded  $D$
that is much smaller than $2^m$ still gives very high expected revenue.


%\dcp{mention somewhere that whereas the realized distributions per menu element are limited to $D$ bundles, with $D$ bounded or at least small compared to $2^m$, the final distribution achieved
%from the vector field in Stage 1 can, in principle, have support on all $2^m$ bundles.}

\textbf{Extension to multi-bidder settings}. By providing an expressive and concise %\dcp{this also said `concise and flexible` but I think concise is the same as efficient and flexible the same as expressive}
representation of single-bidder menus for the CA setting, our method opens up the possibilities of developing a general DSIC multi-bidder CA mechanism. A principled approach is to 
adapt the idea of GemNet~\cite{wang2024gemnet}. 
%
First, we can learn a separate \name~menu for each bidder. The modification in the network architecture is that these menus should now also depend on other bidders' bids $\vb_{\shortn i}$. To achieve this, we can condition the vector field, specifically the $Q$ and $\sigma$ networks, on $\vb_{\shortn i}$ by concatenating them to the inputs. For the price of each menu element, we can model them as the output of a neural network whose input is $\vb_{\shortn i}$. During training, we can also introduce a compatibility loss in the same way as that used in GemNet. This loss penalizes any over-allocation of items in the selected agent-optimizing elements from individual menus.

The major challenge in adapting GemNet to the CA setting  
arises during the post-training stage of GemNet, which adjusts prices of menu elements so that there is provably never any over-allocation of items. For this, GemNet constructs a grid over the space of bidder values. On each grid point, GemNet formulates a mixed-integer linear program (MILP) to adjust prices to ensure that, the utility of the best  element that is compatible with the choices of others in the sense of not over-allocating items is larger than that of all other elements by a safety margin. These safety margins prevent an incompatible menu element from being selected in the regions between grid points. Although the concise \name~menu representation, in principle, enables this MILP to 
be directly adapted to the combinatorial setting and used to adjust \name~menus to obtain a DSIC CA, the main issue is that the space of bidder values exhibits exponential dimensionality in the CA setting, resulting in an excessively large grid. Reducing this complexity represents the crucial 
remaining step in future work to enable 
a general, DSIC, and multi-bidder CA mechanism.
\section{Experiments}
\label{sec: experiments}

\begin{table*}
        % \centering
        \caption{A comparison between our proposed method with other advanced methods on the nuScenes test set.
        % This leaderboard is available at \href{https://www.nuscenes.org/tracking?externalData=all&mapData=all&modalities=Any}{nuScenes official benchmark}.
        $\ddagger$ means the GPU device.
        \textcolor{black}{The reported runtimes of all methods exclude the detection time.}
        Poly-MOT~\cite{li2023poly}, Fast-Poly~\cite{li2024fast} and Easy-Poly rely entirely on the detector input, as they do not utilize any visual or deep features \textcolor{black}{during tracking}.}
        \label{table:nu_test}
        % \renewcommand{\arraystretch}{0.7}
        \setlength{\tabcolsep}{1.6mm}
        {
        \begin{tabular}{cccc|ccc|ccc}
        \toprule
        \multicolumn{1}{c}{\textbf{Method}} & \textbf{Device} & \textbf{Detector} & \textbf{Input} & \textbf{AMOTA}$\uparrow$ & \textbf{MOTA}$\uparrow$ & \textbf{FPS}$\uparrow$ & \textbf{IDS}$\downarrow$ & \textbf{FN}$\downarrow$ & \textbf{FP}$\downarrow$ \\ \midrule
                                     EagerMOT~\cite{kim2021eagermot}               & \textbf{\text{--}}       & CenterPoint~\cite{yin2021center}\&Cascade R-CNN~\cite{cai2018cascade}         & 2D+3D       & 67.7      & 56.8     & 4    & 1156    & 24925   & 17705   \\
                                   CBMOT~\cite{benbarka2021score}     & I7-9700          & CenterPoint~\cite{yin2021center}\&CenterTrack~\cite{zhou2020tracking}          & 2D+3D         & 67.6      & 53.9      & \textcolor{red}{80.5}    & 709    & 22828   & 21604   \\
                                   % ShaSTA~\cite{sadjadpour2023shasta} & A100$\ddagger$       & CenterPoint~\cite{yin2021center}         & 3D      & 69.6      & 57.8     & 10     & 473    & 21293   & 16746   \\
                                   Minkowski~\cite{gwak2022minkowski} & TITAN$\ddagger$  & Minkowski~\cite{gwak2022minkowski}         & 3D      & 69.8      & 57.8     & 3.5    & 325    & 21200   & 19340   \\
                                   ByteTrackv2~\cite{zhang2023bytetrackv2} & \textbf{\text{--}}       & TransFusion-L~\cite{bai2022transfusion}         & 3D      & 70.1      & 58     & \textbf{\text{--}}    & 488    & 21836   & 18682   \\
                                   3DMOTFormer~\cite{ding20233dmotformer}& 2080Ti$\ddagger$       & BEVFuison~\cite{liu2023bevfusion}       & 2D+3D      & 72.5      & 60.9     & \textcolor{blue}{54.7}    & 593    & 20996   & \textcolor{blue}{17530}   \\  
                                   %CAMO-MOT~\cite{li2023camo}& 3090Ti$\ddagger$   & BEVFuison~\cite{liu2023bevfusion}\&FocalsConv~\cite{chen2022focal}   & 2D+3D      & 75.3      & \textbf{63.5}     & \textbf{\text{--}}    & 324    & 18192   & 17269   \\
                                   Poly-MOT~\cite{li2023poly}               & 9940X       & LargeKernel3D~\cite{chen2022scaling}         &2D+3D       & \textcolor{blue}{75.4}      & \textcolor{blue}{62.1}     & 3    & \textcolor{blue}{292}    & \textcolor{blue}{17956}   & 19673   \\ 
                                  Fast-Poly~\cite{li2024fast}               & 7945HX       & LargeKernel3D~\cite{chen2022scaling}         &2D+3D       & \textcolor{blue}{75.8}      & \textcolor{blue}{62.8}     & 34.2    & \textcolor{blue}{326}    & \textcolor{blue}{18415}   &  \textcolor{blue}{17098}  \\ \midrule

                                  \textbf{Easy-Poly (Ours)}               & 4090Ti$\ddagger$       & LargeKernel3D~\cite{chen2022scaling}         &2D+3D       &  \textcolor{red}{75.9}      & \textcolor{red}{63.0}     & \textcolor{blue}{34.9}    & \textcolor{red}{287}    & \textcolor{red}{17620}   &   \textcolor{red}{16718}  \\
        \bottomrule
        \end{tabular}}
        % \vspace{-1.5em}
 \end{table*}


\begin{table*}
\vspace{0.5em}
\begin{center}
\caption{
{A comparison of existing methods applied to the nuScenes val set.}}
\label{table:nu_val}
% \renewcommand{\arraystretch}{0.7}
\setlength{\tabcolsep}{2.4mm}
{
\begin{tabular}{cccccccc}
\toprule
\bf{Method} & \bf{Detector} & \bf{Input Data} & \bf{MOTA$\uparrow$} & \bf{AMOTA$\uparrow$} & \bf{AMOTP$\downarrow$} & \bf{FPS$\uparrow$} & \bf{IDS$\downarrow$}  \\ \hline
CBMOT~\cite{benbarka2021score}   & CenterPoint~\cite{yin2021center} \& CenterTrack~\cite{zhou2020tracking} & 2D + 3D & -- & 72.0 & \textbf{\textcolor{red}{48.7}}  & -- & 479   \\
EagerMOT~\cite{kim2021eagermot}  & CenterPoint~\cite{yin2021center} \& Cascade R-CNN~\cite{cai2018cascade} & 2D + 3D  & -- & 71.2   & 56.9 & 13  & 899    \\
SimpleTrack~\cite{pang2022simpletrack}  & CenterPoint~\cite{yin2021center} & 3D & 60.2 & 69.6  & 54.7 & 0.5  & 405  \\
CenterPoint~\cite{yin2021center}   & CenterPoint~\cite{yin2021center} & 3D & -- & 66.5  & 56.7 & -- & 562 \\ 
OGR3MOT~\cite{zaech2022learnable}  & CenterPoint~\cite{yin2021center} &3D & 60.2 & 69.3  & 62.7 & 12.3 & \textbf{\textcolor{blue}{262}}  \\ \hline

\textbf{Poly-MOT}~\cite{li2023poly}      & CenterPoint~\cite{yin2021center} & 3D & 61.9 & 73.1   & \textbf{\textcolor{blue}{52.1}} & 5.6 & 281   \\ 
% \textbf{Poly-MOT}~\cite{li2023poly}      & LargeKernel3D-L~\cite{chen2022scaling}  & 3D   & \textbf{\textcolor{red}{75.2}}    & 54.1   & \textbf{\textcolor{red}{252}} \\ 

\textbf{Poly-MOT}~\cite{li2023poly}      & LargeKernel3D-L~\cite{chen2022scaling}  & 3D & 54.1 & \textbf{\textcolor{red}{75.2}}    & 54.1  & 8.6 & 252 \\

\textbf{Fast-Poly}~\cite{li2024fast}      & CenterPoint~\cite{yin2021center} & 3D & \textbf{\textcolor{blue}{63.2}}  & 73.7   &  -- & \textbf{\textcolor{blue}{28.9}} & 414   \\ \hline
 
\textbf{Easy-Poly (Ours)}       & CenterPoint~\cite{yin2021center} & 2D + 3D & \textbf{\textcolor{blue}{64.4}} & \textbf{\textcolor{blue}{74.5}}   & 54.9  & \textbf{\textcolor{blue}{34.6}} & \textbf{\textcolor{blue}{272}}   \\
\textbf{Easy-Poly (Ours)}       & LargeKernel3D~\cite{chen2022scaling}  & 2D + 3D  & \textbf{\textcolor{red}{64.8}} & \textbf{\textcolor{blue}{75.0}}    & \textbf{\textcolor{blue}{53.6}}  & \textbf{\textcolor{red}{34.9}}  & \textbf{\textcolor{red}{242}} \\ \hline
% \vspace{-4.5em}
% \setlength{\abovecaptionskip}{3pt}
% \setlength{\belowcaptionskip}{3pt}
\end{tabular}}
\end{center}
\end{table*}

\subsection{Datasets}

% The nuScenes dataset~\cite{caesar2020nuscenes} consists of 850 training and 150 test sequences, capturing a wide range of driving scenarios, including challenging weather conditions and nighttime environments. Each sequence contains approximately 40 frames, with keyframes sampled at 2Hz and fully annotated. In addition to this, it provides annotations for object-level attributes such as visibility, activity, pose, and more. It includes a large volume of RGB and point-cloud data (in PCD format). The official evaluation protocol utilizes \textbf{AMOTA}, \textbf{MOTA}, and \textbf{sAMOTA}~\cite{weng20203d} as primary metrics, evaluating performance across seven object categories: Car (\textit{Car}), Bicycle (\textit{Bic}), Motorcycle (\textit{Moto}), Pedestrian (\textit{Ped}), Bus (\textit{Bus}), Trailer (\textit{Tra}), and Truck (\textit{Tru}). Given the substantial size of the complete nuScenes dataset, users often prefer the nuScenes-mini dataset. Notably, Poly-MOT, Fast-Poly, and our proposed Easy-Poly methods exclusively utilize keyframes for tracking tasks.


The nuScenes dataset~\cite{caesar2020nuscenes} consists of 850 training and 150 test sequences, capturing a wide range of driving scenarios, including challenging weather conditions and nighttime environments. Each sequence contains approximately 40 frames, with keyframes sampled at 2Hz and fully annotated. The official evaluation protocol utilizes \textbf{AMOTA}, \textbf{MOTA}, and \textbf{sAMOTA}~\cite{weng20203d} as primary metrics, evaluating performance across seven object categories: Car (\textit{Car}), Bicycle (\textit{Bic}), Motorcycle (\textit{Moto}), Pedestrian (\textit{Ped}), Bus (\textit{Bus}), Trailer (\textit{Tra}), and Truck (\textit{Tru}). Notably, Poly-MOT, Fast-Poly, and our proposed Easy-Poly methods exclusively utilize keyframes for tracking tasks.

\subsection{Implementation Details}

% Our tracking method is implemented in Python under the Nvidia 4090X GPU.  Hyperparameters are chosen based on the best AMOTA identified in the validation set. SF thresholds are category-specific and detector-specific, which are (\textit{Bic}: 0.15; are (\textit{Car}: 0.16; are (\textit{Moto}: 0.16; \textit{Bus}: 0.12; \textit{Tra}: 0.13; \textit{Tru}: 0; \textit{Ped}: 0.13) on nuScenes on Waymo. The NMS thresholds are 0.08 on all categories and datasets. We also employ Scale-NMS~\cite{huang2021bevdet} on (\textit{Bic}, \textit{Ped}) on nuScenes. With default $IoU_{bev}$ in NMS, we additionally utilize our proposed $A\text{-}gIoU_{bev}$ to describe similarity for (\textit{Bic}, \textit{Ped}, \textit{Bus}, \textit{Tru}) on nuScenes. The motion models and filters are consistent with~\cite{li2023poly}. The lightweight filter is implemented by the median filter with $l_{lw}=5$ on all datasets. The association metrics are all implemented by $A\text{-}gIoU$ on all datasets. The first association thresholds $\theta_{fm}$ are category-specific, which are (\textit{Bic, Moto, Bus}: 1.6; \textit{Car, Tru}: 1.2; \textit{Tra}: 1.16;\textit{Ped}: 1.78) on nuScenes. Voxel mask size $\theta_{vm}$ is 5\textit{m} on nuScenes.  The count-based and output file strategies are consistent with~\cite{li2023poly}. In the confidence-based part, decay rates $\sigma$ are category-specific, which are (\textit{Ped}: 0.18; \textit{Car}: 0.26; \textit{Tru, Moto}: 0.28; \textit{Tra}: 0.22; \textit{Bic,Bus}: 0.24) on nuScenes. The delete threshold $\theta_{dl}$ are (\textit{Bus}: 0.08,  \textit{Ped}: 0.1 and 0.04 for other categories) on nuScenes.

Our tracking framework is implemented in Python and executed on an Nvidia 4090X GPU. Hyperparameters are optimized based on the highest AMOTA achieved on the validation set. The following category-specific and SF thresholds are employed for nuScenes: (\textit{Bic}: 0.15; \textit{Car}: 0.16; \textit{Moto}: 0.16; \textit{Bus}: 0.12; \textit{Tra}: 0.13; \textit{Tru}: 0; \textit{Ped}: 0.13). NMS thresholds are uniformly set to 0.08 across all categories and datasets. Additionally, we implement Scale-NMS~\cite{huang2021bevdet} for (\textit{Bic}, \textit{Ped}) categories on nuScenes.
In conjunction with the default in NMS, we introduce our novel  metric to enhance similarity assessment for (\textit{Bic}, \textit{Ped}, \textit{Bus}, \textit{Tru}) categories on nuScenes. Motion models and filters are consistent with those described in~\cite{li2023poly}. A lightweight filter, implemented as a median filter with , is applied across all datasets. Association metrics universally employ  across all datasets.
Category-specific first association thresholds  for nuScenes are as follows: (\textit{Bic, Moto, Bus}: 1.6; \textit{Car, Tru}: 1.2; \textit{Tra}: 1.16; \textit{Ped}: 1.78). The voxel mask size  is set to 5\textit{m} on nuScenes. Count-based and output file strategies align with those presented in~\cite{li2023poly}.
In the confidence-based component, category-specific decay rates for nuScenes are: (\textit{Ped}: 0.18; \textit{Car}: 0.26; \textit{Tru, Moto}: 0.28; \textit{Tra}: 0.22; \textit{Bic, Bus}: 0.24). The delete thresholds  are set as follows: (\textit{Bus}: 0.08, \textit{Ped}: 0.1, and 0.04 for all other categories) on nuScenes.

In the object tracking phase of 3D MOT, Easy-Poly exhibits exceptional performance following a series of optimizations. These enhancements include pre-processing, Kalman filtering, motion modeling, and tracking cycle refinements. The integration of these techniques significantly improves the algorithm's effectiveness in complex 3D environments.

Easy-Poly exhibits exceptional performance on the test set, achieving a \textbf{75.9\%} AMOTA score, surpassing the majority of existing 3D MOT methods. As shown in Table \ref{table:nu_test}, Easy-Poly attains a remarkably low IDS count of \textbf{287} while maintaining the highest AMOTA (\textbf{75.9\%}) among all modal methods. This underscores Easy-Poly's ability to maintain stable tracking without compromising recall. Notably, Easy-Poly achieves state-of-the-art performance without relying on additional image data input. Easy-Poly significantly outperforms competing algorithms in the critical 'Car' category. With minimal computational overhead, it delivers impressive results, highlighting its potential for integration into real-world autonomous driving systems. The False Negative and False Positive metrics in Table \ref{table:nu_test} further demonstrate Easy-Poly's robust continuous tracking capability while maintaining high recall.

For validation set experiments in Table \ref{table:nu_val}, we utilize CenterPoint~\cite{yin2021center} as the detector to ensure fair comparisons. As illustrated in Table \ref{table:nu_val}, Easy-Poly significantly outperforms most deep learning-based methods in both tracking accuracy (\textbf{75.0\%} AMOTA, \textbf{64.8\%} MOTA) and computational efficiency (\textbf{34.9} FPS). Compared to the baseline FastPoly~\cite{li2024fast}, Easy-Poly achieves substantial improvements of \textbf{+1.3\%} in MOTA and \textbf{+1.6\%} in AMOTA, while operating \textbf{1.5x} faster under identical conditions. When integrated with the high-performance LargeKernel3D~\cite{chen2022scaling} detector, Easy-Poly demonstrates even more impressive detection and tracking capabilities. Furthermore, when employing the multi-camera detector \textcolor{black}{DETR3D}~\cite{DETR3D} with constrained performance, Easy-Poly exhibits robust real-time performance without compromising accuracy. Furthermore, the lower AMOTP and IDS metrics demonstrate Easy-Poly's exceptional capability in tracking small objects and maintaining performance in complex scenarios and adverse weather conditions. These results underscore the algorithm's robustness across diverse and challenging environments.
% It is noteworthy that achieving optimal AMOTA necessitates a stringent score filter threshold, which marginally reduces the latency advantage of our method. Nevertheless, Easy-Poly maintains a favorable balance between accuracy and computational efficiency.

\begin{table}
% \vspace{0.5em}
        \caption{Comparing different data association algorithms using CenterPoint and (lines 1-7) and LargeKernel3D (lines 8-11) methods on nuScenes val set. Among them, the algorithms lines 1-3 are the Fast-Poly framework and in lines 4-11 are the latest our Easy-Poly framework.}

        \label{table:nus_assoc}
        % \renewcommand{\arraystretch}{0.7}
        \setlength{\tabcolsep}{0.1mm}
        \begin{tabular}{cccccc}
        \toprule
        \multicolumn{1}{c}{\textbf{Algorithms}} & \textbf{MOTA}$\uparrow$ & \textbf{AMOTA}$\uparrow$ & \textbf{AMOTP}$\downarrow$ & \textbf{IDS}$\downarrow$ & \textbf{FN}$\downarrow$\\ 
        
        \midrule
         %  Mutual Nearest Neighbor (MNN)
         MNN  & 62.2  & 72.5 & 52.4  & 433 & 16644  \\  
         Greedy    & 62.3  & 72.7 &  53.4  & 428  & 17647 \\
         Hungarian & \textbf{\textcolor{blue}{63.2}}  & \textbf{\textcolor{blue}{73.7}} & \textbf{\textcolor{blue}{52.1}} & \textbf{\textcolor{blue}{414}}  & \textbf{\textcolor{blue}{15996}}  \\
        
        \midrule  
        % 多模态+数据增强
         %  Mutual Nearest Neighbor (MNN)
        \textbf{MNN (Ours)}  & 63.7  & 73.6 & 54.8  & 406 &  \textbf{\textcolor{red}{15873 }}\\  
        \textbf{Greedy (Ours)}    &  64.0  & 73.7  &  54.6  & 368  & 16736 \\
        \textbf{Hungarian (Ours)}  & 64.3  & 74.3 & \textbf{\textcolor{red}{54.3}} &  335 & 16892  \\

        \textbf{DTO (Ours)}   & \textbf{\textcolor{red}{64.4}}  &   \textbf{\textcolor{red}{74.5}} & 54.9 &  \textbf{\textcolor{red}{272}} & 16982  \\

        
         \midrule
         \textbf{MNN (Ours)}  & 64.1  & 73.9 & 54.0  & 370 & 15865  \\  
         \textbf{Greedy (Ours)}    & 64.5  & 74.3 & 53.7  & 307  & 16014 \\
        \textbf{Hungarian (Ours)}  & 64.7  & 74.8 & 53.9  & 291  & 15923 \\

        \textbf{DTO (Ours)}  & \textbf{\textcolor{red}{64.8}}  & \textbf{\textcolor{red}{75.0}} & \textbf{\textcolor{red}{53.6}}  & \textbf{\textcolor{red}{242}}  & \textbf{\textcolor{red}{15488}} \\

    \bottomrule
\end{tabular}
\end{table}



% \begin{table}
% \caption{The ablation study of whether or not to use Score Filter and Non-Maximum Suppression, including the Run-Time, which represents the execution time of the Pre-processing Module. We compared Poly-MOT~\cite{li2023poly} (rows 1-3) with our proposed Easy-Poly method (rows 4-6).}
% \label{table:nu_NMSsf}
% % \renewcommand{\arraystretch}{0.7}
% \setlength{\tabcolsep}{2.7mm}
% {
% \begin{tabular}{cccc}
% \toprule

% \textbf{Variable} & \textbf{AMOTA$\uparrow$} & \textbf{IDS$\downarrow$} &
% \textbf{Run-Time (s) $\downarrow$}
% \\ 
% \midrule
% NMS + SF & \textbf{\textcolor{blue}{73.1}}   & \textbf{\textcolor{blue}{281}}  & 0.055    \\
% NMS     & 71.8  & 320  & 0.093     \\
% SF       & 68.6  & 354  & \textbf{\textcolor{blue}{0.008}}     \\ 
% \midrule
% \textbf{NMS + SF (Ours)} & \textbf{\textcolor{red}{75.0}}   & \textbf{\textcolor{red}{242}}  & 0.037    \\
% \textbf{NMS (Ours)}      & 73.6   & 273  & 0.068    \\
% \textbf{SF (Ours)}       & 71.2   & 308  & \textbf{\textcolor{red}{0.008}}     \\\hline
% % \vspace{-3.2em}
% \end{tabular}}
% \end{table}



\subsection{Comparative Evaluations} 
\label{sec: Comparative}

% Our proposed method, Easy-Poly, achieves state-of-the-art performance on the nuScenes validation set, demonstrating \textbf{75.0\%} AMOTA at \textbf{34.9} FPS, surpassing existing approaches. Utilizing an identical detector, Easy-Poly outperforms Fast-Poly \cite{li2024fast} across nearly all key metrics, with notable improvements in accuracy (\textbf{+1.3\%} AMOTA, \textbf{+1.6\%} MOTA) and speed (\textbf{+6.0 FPS}). While marginally slower than CBMOT \cite{benbarka2021score} and 3DMOTFormer \cite{ding20233dmotformer}, Easy-Poly significantly exceeds their accuracy while maintaining robust real-time performance. Our open-source implementation establishes a strong baseline for 3D MOT, providing a solid foundation for future advancements in the field.

In this study, we conduct a comprehensive evaluation of the association stage, focusing on four algorithms: Hungarian, Greedy, MNN, and the novel DTO. Our extensive experiments, summarized in Table~\ref{table:nus_assoc}, reveal that the Easy-Poly consistently outperforms Fast-Poly across both CenterPoint and LargeKernel3D frameworks. Notably, LargeKernel3D demonstrates superior performance over CenterPoint, particularly in complex tracking scenarios. Among the association algorithms, Hungarian and DTO consistently yield superior results, underscoring their robustness and efficacy in diverse multi-object tracking contexts. Compared to the Hungarian algorithm, DTO not only achieves similarly excellent AMOTA values but also provides more robust and accurate tracking performance, especially in challenging scenarios involving occlusions, missed detections, or false positives. These findings highlight the critical role of algorithm selection and model optimization in advancing the state-of-the-art in 3D object tracking.

\begin{figure*}[t]
    \centering
    \includegraphics[width=0.95\linewidth]{Images/Ablation_study_line_chart_for_NMS.pdf}
    \vspace{-8pt}
    \caption
    {
      The ablation study of whether or not to use Score Filter and Non-Maximum Suppression, including the Run-Time, which represents the execution time of the Pre-processing Module. We compared Poly-MOT with our proposed Easy-Poly method.
     }
     \Description{}
    \label{fig: NMSFS}
\end{figure*}


\textcolor{black}{Table \ref{table:nu_life} demonstrates the efficacy of our proposed methods. The Fast-Poly tracklet termination strategy (line 3) significantly outperforms baseline score refinement~\cite{benbarka2021score} (line 2), yielding a \textbf{2.7\%} improvement in AMOTA and reducing FN by \textbf{2366}. This enhancement mitigates tracker vulnerabilities in mismatch scenarios, including occlusions. Further performance gains are achieved through smoother score prediction (line 4), resulting in additional improvements of \textbf{0.4\%} AMOTA, \textbf{0.1\%} MOTA, and a reduction of \textbf{926 FN}.}
Our Easy-Poly model exhibits even more substantial performance enhancements. The tracklet termination strategy (line 7) surpasses the baseline score refinement (line 6) by \textbf{2.8\%} in AMOTA while decreasing FN by \textbf{1844}. Furthermore, the integration of smoother score prediction (line 8) further boosts the tracking performance, resulting in improvements of \textbf{0.4\%} in AMOTA and \textbf{0.6\%} in MOTA, along with a reduction of \textbf{575} FN.


\begin{table}
% \vspace{0.5em}
        \caption{A comparison on distinct life-cycle modules on nuScenes val set.
        \textbf{Average} means using the online average score to delete.
        \textbf{Latest} means using the latest score to delete.
        \textbf{Max-age} means using the continuous mismatch time to delete.
        Other settings are under the best performance. Methods in lines 1-4 and lines 5-8 use Fast-Poly~\cite{li2024fast} and Our Easy-Poly.
        }
        \label{table:nu_life}
        % \renewcommand{\arraystretch}{0.7}
        \setlength{\tabcolsep}{0.1mm}
        \begin{tabular}{ccccc}
        \toprule
        \multicolumn{1}{c}{\textbf{Strategy}} & \textbf{AMOTA}$\uparrow$ & \textbf{MOTA}$\uparrow$ & \textbf{FPS}$\uparrow$ & \textbf{FN}$\downarrow$\\ \midrule
 
         Count                \& Max-age            & 73.3      & 62.9     & 23.0  & 16523    \\
         %predict:normal update:normal+delete = -100
         Confidence~\cite{benbarka2021score} \& Latest       & 70.6      & 63.2     & \textbf{\textcolor{blue}{45.8}}  & 19192   \\%predict:minus update:multi+latest
         Confidence~\cite{benbarka2021score} \& Average      & 73.3      & 63.1     & 28.3 & 16826   \\%predict:minus update:multi+average
         Confidence  \& Average      & \textbf{\textcolor{blue}{73.7}}      & \textbf{\textcolor{blue}{63.2}}     & 28.9  & \textbf{\textcolor{blue}{15900}}   \\%predict:normal update:multi+average 27.7or28.9
         \midrule
         \textbf{Count \& Max-age (Ours)}          & 74.3      & 63.8     & 34.3  & 16024    \\
         %predict:normal update:normal+delete = -100
         \textbf{Confidence~\cite{benbarka2021score} \& Latest (Ours)}        & 71.8      & 63.7     & \textbf{\textcolor{red}{50.2}}  & 17907   \\%predict:minus update:multi+latest
         \textbf{Confidence~\cite{benbarka2021score} \& Average (Ours)}       & 74.6      & 64.2     & 35.6 & 16063   \\%predict:minus update:multi+average
         \textbf{Confidence \& Average (Ours)}      & \textbf{\textcolor{red}{75.0}}      & \textbf{\textcolor{red}{64.8}}     & 36.9  & \textbf{\textcolor{red}{15488}}   \\%predict:normal update:multi+average 36.9
        \bottomrule
        \end{tabular}
    \end{table}


\subsection{Ablation Studies}

% The Sf can filter out low-score bounding boxes while NMS can remove duplicate bounding boxes with high confidence, which makes the remaining bounding boxes have superior quality. For the Poly-MOT, using SF before NMS brings inference 40\% reduction in pre-processing inference time while boosting AMOTA by 1.3\% compared with only using NMS. The Fast-Poly has better performance, using SF before NMS brings inference 50\% reduction in pre-processing inference time while boosting AMOTA by 1.4\% compared with only using NMS, as demonstrate in Table~\ref{table:nu_NMSsf}.  It is clear from the table that when only SF is used without NMS, although the running time is fast, both AMOTA and IDS values become lower and the values drop very significantly.

We optimize two-stage filtering approach that synergistically combines SF and NMS to significantly enhance bounding box quality in multi-object tracking scenarios. This novel method effectively integrates SF to eliminate low-score detections and NMS to remove high-confidence duplicates, resulting in a set of superior quality bounding boxes. Our comprehensive experimental results, presented in Figure~\ref{fig: NMSFS}, demonstrate substantial improvements in both computational efficiency and tracking accuracy across multiple state-of-the-art models. For the Poly-MOT model, our approach of applying SF before NMS yields a remarkable \textbf{40\%} reduction in pre-processing inference time while simultaneously improving AMOTA by \textbf{1.3\%} compared to using NMS alone. These results highlight the significant potential of our method in improving real-time tracking capabilities without compromising accuracy. The Easy-Poly model exhibits even more impressive performance gains, further validating the scalability and effectiveness of our approach. By applying SF before NMS, we achieve a substantial \textbf{50\%} reduction in pre-processing time coupled with a \textbf{1.4\%} improvement in AMOTA. This notable improve in both speed and accuracy underscores the robustness of our method across different model architectures. Importantly, our analysis reveals critical insight into the interplay between SF and NMS. Although using SF without NMS accelerates processing, it leads to significant degradation in both AMOTA and Identity Switches IDS metrics. This observation underscores the crucial importance of our combined SF-NMS approach in maintaining an optimal balance between processing speed and tracking accuracy. The synergistic effect of SF and NMS not only enhances the quality of bounding boxes but also optimizes the trade-off between computational efficiency and tracking performance. This balance is particularly vital in real-world applications where both speed and accuracy are paramount, such as autonomous driving and surveillance systems. 
% The consistent improvements observed across different models suggest broad applicability and potential for integration into various tracking frameworks, paving the way for more efficient and accurate tracking systems in complex, real-world environments.


% \subsection{Visualization}

\section{Conclusion}
In this study, we introduce \ours, a novel framework designed to achieve lossless acceleration in generating ultra-long sequences with \acp{llm}. By analyzing and addressing three challenges, \ours significantly enhances the efficiency of the generation process. Our experimental results demonstrate that \ours achieves over $3\times$ acceleration across various model scales and architectures. Furthermore, \ours effectively mitigates issues related to repetitive content, ensuring the quality and coherence of the generated sequences. These advancements position \ours as a scalable and effective solution for ultra-long sequence generation tasks.



% Bibliography
\newpage
\bibliographystyle{ACM-Reference-Format}
\bibliography{sample-bibliography}

% Appendix
\newpage
\appendix
% \section{Normalizing Flow}

% A normalizing flow model provides a transportation from a distribution to another by applying a sequence of functions that are not necessarily the same. As shown in Fig.~\ref{fig:normalizing_flow}, for round $i\in\{1,\cdots, T-1\}$, the input to the function is $z_i\in\mathbb{E}^m$ \dcp{$\mathbb{R}^m$?}
% sampled from the distribution $p_i(z_i)$. After applying the function, the distribution will be transformed to $p_{i+1}(z_{i+1})$. \dcp{$t$ not $i$ in main body}
% \begin{figure}[h!]
%     \centering
%     \includegraphics[width=\linewidth]{figures/flow.pdf}
%     \caption{A normalizing flow model that transport a simple distribution to a complex distribution.\label{fig:normalizing_flow}}
%     \Description[Normalizing flow.]{A normalizing flow model that transport a simple distribution to a complex distribution.}
% \end{figure}

% We are interested in the closed-form expression of $p_{i+1}(z_{i+1})$. According to the change of variable theorem and assuming  the neural network is invertible, we have
% \begin{align}
%     p_{i+1}(z_{i+1}) = p_i\left(f^{\shortn 1}(z_{i+1})\right)\left|\text{det}\frac{df^{\shortn 1}}{dz_{i+1}}\right|= p_i\left(z_i\right)\left|\text{det}\frac{df^{\shortn 1}}{dz_{i+1}}\right|,
% \end{align}
% where $\text{det}\frac{df}{dz}$ is the Jacobian determinant of function $f$. The inverse function theorem tells us
% \begin{align}
%     \frac{df^{\shortn 1}}{dz_{i+1}} = \frac{dz_i}{dz_{i+1}} = \left(\frac{dz_{i+1}}{dz_i}\right)^{\shortn 1} = \left(\frac{df}{dz_i}\right)^{\shortn 1}.
% \end{align}
% It follows that $p_{i+1}(z_{i+1})=p_i\left(z_i\right)\left|\text{det}\left(\frac{df}{dz_i}\right)^{\shortn 1}\right|$. For an invertible function $A=\frac{df}{dz_i}$, we have $1=det(I)=det(AA^{\shortn i})=det(A)det(A^{\shortn 1})$, so $det(A^{\shortn 1})=det(A)^{\shortn 1}$. Therefore, we have $p_{i+1}(z_{i+1}) = p_i\left(z_i\right)\left|\text{det}\frac{df}{dz_{i+1}}\right|^{\shortn 1}$, and
% \begin{align}
%     \log p_{i+1}(z_{i+1}) = \log p_i\left(z_i\right) -\log \left|\text{det}\frac{df}{dz_{i+1}}\right|.
% \end{align}
% We can apply this repeatedly and obtain
% \begin{align}
%     \log p(z) = \log p_T(z_T) = \log p_0\left(z_0\right) -\sum_{i=0}^{T-1}\log \left|\text{det}\frac{df}{dz_{i+1}}\right|.
% \end{align}


% \section{Likelihood derivation}

% \subsection{Linear}

% When $\varphi(t, s_t) = \sigma(t) Q s(t)$, we have

% \begin{align}
%     \nabla\cdot \varphi(t, s(t)) = \sum_{i=1}^d \frac{\partial \varphi_i}{\partial s_i(t)}
%     = \sum_{i=1}^d \frac{\partial}{\partial s_i(t)} \sigma(t) Q_i s(t) = \sum_{i=1}^d  \sigma(t) Q_{ii} = \sigma(t)\Tr(Q)
% \end{align}


\end{document}
